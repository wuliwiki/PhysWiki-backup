% 氢原子电离计算(一阶微扰)
% keys 氢原子|偶极子|跃迁|微扰

\begin{issues}
\issueDraft
\end{issues}

\pentry{跃迁概率(含时微扰)\upref{HionCr}}

本文使用原子单位制\upref{AU}。

\subsection{长度规范(平面波近似)}
归一化的平面波和归一化的氢原子基态为(用平面波近似末态库仑函数)
\begin{equation}
\ket{\bvec k} = (2\pi)^{-3/2} \E^{\I \bvec k \vdot \bvec r},
\qquad \ket{0} = \frac{1}{\sqrt{\pi}} \E^{-r}~.
\end{equation}
长度规范下的跃迁偶极子, 可以在极坐标系中积分(令 $\uvec k$ 方向为极轴, $\uvec r$ 与其夹角为 $\theta$)
\begin{equation}
\mel{\bvec k}{\bvec r}{0}
=  \frac{\uvec k}{(2\pi)^{3/2}\sqrt{\pi}} \int_0^{+\infty} \int_0^\pi \E^{-r} \E^{-\I k r \cos\theta} r \cos\theta \cdot 2\pi r^2 \sin\theta \dd{\theta} \dd{r}
\end{equation}
换元, 令 $u = \cos\theta$, 得\footnote{最后一步可通过 Wolfram Alpha 获得}
\begin{equation}\label{HyIon2_eq1}\ali{% 已检查多次
\mel{\bvec k}{\bvec r}{0} &= \frac{1}{\sqrt 2 \pi} \uvec k \int_0^{+\infty} r^3 \E^{-r} \int_{-1}^1 \E^{-\I k r u} u  \dd{u} \cdot \dd{r}\\
&=  \I\frac{\sqrt2 \uvec k}{k\pi}  \int_0^{+\infty} r^2 \E^{-r} \qty[\cos(kr) - \frac{1}{kr}\sin(kr)] \dd{r}\\
&= -\I \frac{8 \sqrt2}{\pi} \frac{\bvec k}{(k^2+1)^3}
}\end{equation}


Matlab 代码如下
\begin{lstlisting}[language=matlab, caption=hydrogen\_trans\_dipole0.m]
% hydrogen transition dipole, approximate Coulomb plane wave with plane wave
% since middle and right parts are symmetric,
%     assuming \vec k in z+ direction
% <\vec k|\vec r|n,0,0> = [0, 0, <\vec k|z|n,0,0>]
% output numerical integration and analytical result (HyIon2_eq1)
function [dipole_z, dipole_analy_z] = hydrogen_trans_dipole0(kz, ZZ, n)
% === params ===
rmax = 20; Nr = 200; Nth = 100;
% ==============
k = [0, 0, kz];
r = linspace(0, rmax, Nr); dr = rmax / (Nr-1);
th = linspace(0, pi, Nth); dth = pi / (Nth-1);
ph = 0;
[R, Th] = ndgrid(r, th);
Z = R .* cos(Th);
Psi_n = hydrogen_Psi(ZZ, n, 0, 0, r', th, ph);
% <\vec k|\vec r|0>
Psi_k = 1/(2*pi)^(3/2) * exp(1i*kz.*Z);
dipole_z = sum(sum(conj(Psi_k).*Z.*Psi_n .*R.^2.*sin(Th))) ...
    * dr * dth * 2*pi;
% HyIon2_eq1
if n == 1
    dipole_analy_z = -1i*(8*sqrt(2))/pi * kz / (dot(k,k) + 1)^3;
end
end
\end{lstlisting}



\subsection{速度规范}
注意一阶微扰理论中的初态和末态波函数都是无微扰(无外场)情况下的, 与规范无关。 要计算 $\mel{\bvec k}{\grad}{0}$, 先看积分
% Merzbacher 19.87 上面一条
\begin{equation}
\int \exp(-\I \bvec k \vdot \bvec r) \exp(-r) \dd[3]{r} = \frac{8\pi }{(k^2 + 1)^2}
\end{equation}
使用算符 $\grad$ 的反厄米性得
\begin{equation}
\begin{aligned}
&\int \exp(-\I \bvec k \vdot \bvec r) \grad \exp(-r) \dd[3]{r}
= -\int [\grad \exp(\I \bvec k \vdot \bvec r)]^* \exp(-r) \dd[3]{r}\\
&= \I \bvec k \int \exp(-\I \bvec k \vdot \bvec r) \exp(-r) \dd[3]{r}
= \I \frac{8 \pi  \bvec k}{(k^2 + 1)^2}
\end{aligned}
\end{equation}
乘以归一化系数得
\begin{equation}\label{HyIon2_eq2}
\mel{\bvec k}{\grad}{0} = \I\frac{2\sqrt{2}}{\pi}\frac{\bvec k}{(k^2 + 1)^2}
\end{equation}
该式代入\autoref{SIcros_eq8}~\upref{SIcros}($q = -1$)得微分截面为
\begin{equation}
% dipole approximation 下与第二项 Merzbacher 19.87 相同, 但他的据说与实验吻合, 相差应该不大。
\pdv{\sigma}{\Omega} = \frac{32}{mc\omega} \frac{k(\bvec k \vdot \uvec e)^2}{(k^2 + 1)^4}
= \frac{64}{mc} \frac{k(\bvec k \vdot \uvec e)^2}{(k^2 + 1)^5}
\end{equation}
对于质子数为 $Z$ 类氢原子有
\begin{equation}
\pdv{\sigma}{\Omega} = \frac{32 Z^5}{mc\omega} \frac{k(\bvec k \vdot \uvec e)^2}{(k^2 + Z^2)^4}
\end{equation}

\subsection{两种规范对比}
如果 $\ket{\bvec k}$ 是库仑函数(能量本征态)应该有(\autoref{DipEle_eq3}~\upref{DipEle})
\begin{equation}
\mel{\bvec k}{\bvec r}{0} = -\frac{\mel{\bvec k}{\grad}{0}}{m\omega_{k0}}
\end{equation}
其中 $\omega_{k0} = k^2/2 + 1/2$, 但实际上\autoref{HyIon2_eq1} 和\autoref{HyIon2_eq2} 满足
\begin{equation}
\mel{\bvec k}{\bvec r}{0} = -2\frac{\mel{\bvec k}{\grad}{0}}{m\omega_{k0}}
\end{equation}
这说明在使用平面波近似库仑函数时, 长度规范的 transition amplitude 恰好是速度规范的 2 倍, 截面是四倍(待求证)。

教材中推导微分截面一般使用速度规范, 因为速度规范的结果与实验吻合更好。

\subsection{使用库仑平面波}

理论上若把上面的平面波换成库仑平面波(库仑势能中的精确散射态\upref{CulmWf}), 那么理论上用不同的规范结果是一样的。
\begin{equation}
\begin{aligned}
&\quad\mel*{\psi_{\bvec k}^{(-)}}{z}{\psi_{n,l,m}} = \int \psi_{\bvec k}^{(-)}(\bvec r)^* z \psi_{nlm}(r,\theta,\phi) r^2 \dd{r}\dd{\Omega}\\
&= \sum_{l',m'} \frac{\I^{-l'}}{k} \E^{\I\sigma_{l'}} Y_{l',m'}(\uvec k) \int \frac{1}{r} \sqrt{\frac{2}{\pi}} F_{l'}\qty(\eta, kr) Y_{l',m'}^*(\uvec r) \qty(r \sqrt{\frac{4\pi}{3}} Y_{1,0}(\uvec r))R_{n,l}(r) Y_{l,m}(\bvec r) r \dd{r}\dd{\Omega}\\
&= \sqrt{\frac{8}{3}}\sum_{l',m'} \frac{\I^{-l'}}{k} \E^{\I\sigma_{l'}} Y_{l',m'}(\uvec k) \int F_{l'}\qty(\eta, kr) R_{n,l}(r) r \dd{r} \int Y_{l',m'}^*(\uvec r)Y_{1,0}(\uvec r)Y_{l,m}(\bvec r) \dd{\Omega}\\
&= \sqrt{\frac{2}{\pi}} \sum_{l'} \sqrt{(2l'+1)(2l+1)} \frac{\I^{-l'}}{k} \E^{\I\sigma_{l'}} \int_0^\infty F_{l'}\qty(\eta, kr) R_{n,l}(r) r\dd{r}\times\\
& \qquad \sum_{m'}  (-1)^{m'} Y_{l',m'}(\uvec k) \pmat{l'& 1& l\\ 0 & 0 & 0}\pmat{l' & 1 & l\\  -m' & 0 & m}
\end{aligned}
\end{equation}
其中 $\eta = -Z/k$。 对 $l=0$ 有
\begin{equation}
\mel*{\psi_{\bvec k}^{(-)}}{z}{\psi_{n00}} = \frac{1}{\sqrt{2}\pi k\I} \E^{\I\sigma_1} \cos\theta \int_0^\infty F_1\qty(\eta, kr) rR_{n,0}(r) \dd{r}~.
\end{equation}


笔者没有见过该积分的解析解, 长度规范的 Matlab 数值积分代码如下:

\begin{lstlisting}[language=matlab, caption=hydrogen\_trans\_dipole.m]
% exact hydrogen transition dipole with Coulomb plane wave
% since middle and right parts are symmetric,
%     assuming \vec k in z+ direction
% <C_{\vec k}|\vec r|n,0,0> = [0, 0, <C_{\vec k}|z|n,0,0>]
function dipole_z = hydrogen_trans_dipole(kz, ZZ, n)
% === params ===
rmax = 25; Nr = 300; Nth = 180;
Sign = -1; % plane wave out
% ==============
k = [0, 0, kz];
dr = rmax / Nr; dth = pi / Nth;
r = linspace(dr/2, rmax-dr/2, Nr);
th = linspace(dth/2, pi-dth/2, Nth); 
ph = 0;
Psi_n = hydrogen_Psi(ZZ, n, 0, 0, r', th, ph);
[R, Th] = ndgrid(r, th); Ph = ones(size(R)) * ph;
[X, Y, Z] = Sph2Cart(R, Th, Ph); % Y all 0
Psi_k = coulomb_plane_par(k, X, Y, Z, ZZ, Sign);
% <C_{\vec k}|\vec r|0>
dipole_z = sum(sum(conj(Psi_k).*Z.*Psi_n .*R.^2.*sin(Th))) ...
    * dr * dth * 2*pi;
end
\end{lstlisting}

\addTODO{速度规范数值积分,验证}
