% 南京大学 2015 年考研普通物理
% 南大|南京大学|普物|普通物理

\subsection{力学}
1. 密度均匀的长绳盘在水平面上,拎着长绳的一端,使绳以速度 $v$ 匀速被拎起;\\
(1) 求施加在绳端的力\\
(2) 求拉力的功率与机械能变化率之差.

2. 在轻杆上穿两个相同的小球,起初两个小球距 $O$ 点距离均为 $a$,现在给轻杆施加冲量使杆以角速度 $\omega_{0}$ 转动, 求 $t$ 时刻 $\omega$ 与 $r$.
\subsection{热学}
1. 温度为 $27^{\circ} \mathrm{C}$ 时,$1 \mathrm{~mol}$ 氧气具有多少平均动能,多少转动动能?

2. 已知光子内能 $U=a T^{4}$,在 $T=0$ 时 $S=0$,求任意温度时的熵.
\subsection{电磁学}
1. 将平行板电容器充电至电压为 $V$,断开电源后将平行板距离由 $d$ 拉至 $d'$,求拉力做的功.

2. 带电圆柱体电荷密度为 $\rho$,使圆柱体绕圆柱体中心轴以角速度 $\omega(t)=\beta t$ 旋转\\
(1) 求空间 $\bf{B}$ 分布;\\
(2) 求涡旋电场;\\
(3) 求圆柱体受到的阻力矩.
\subsection{光学}
1. 以波长为 $4300 \mathring{\opn A}$ 和 $6800 \mathring{\opn A}$ 的光垂直入射到光栅上,请问二级,三级光谱是否重迭?

2. 以 $45^{\circ}$ 观察薄膜为绿色 $\lambda=5000 \mathring{\opn A}$,$n=1.33$\\
(1) 求膜的最薄厚度;\\
(2) 如果垂直观察,膜是什么颜色?