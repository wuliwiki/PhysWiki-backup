% 幂的定义

\pentry{极限存在的判据、柯西序列\upref{CauSeq}}

在中学数学中, 我们已经学习过正实数的幂的定义. 按照此定义, 给定实数$x>0$和任意实数$y,z$, 幂$x^y$和$x^z$满足如下性质:

\begin{itemize}
\item 对于任意的$x>0$, 均有$x^0=1$.
\item $x^{y+z}=x^y\cdot x^z$.
\item $(x^y)^z=x^{yz}$.
\item 给定$x_1,x_2>0$, 那么$x_1^y\cdot x_2^y=(x_1x_2)^y$.
\item 如果$x>1$, $y<z$, 那么$x^y<x^z$.
\end{itemize}

特别地, 对于正整数$n$, $x^n$就是将$x$自乘$n$次, $x^{-n}$就是$1/x^n$, 而$x^{1/n}$就是$x$的$n$次算术根. 

一个自然的问题是, 满足以上四条性质的运算是否唯一? 答案是肯定的. 严格的实数理论给出了一个构造幂的方法.

\subsection{幂的构造}

首先当然不难构造出