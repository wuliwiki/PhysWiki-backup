% 埃尔米特型
% 埃尔米特型|Hermitian|正定埃尔米特型
本节半双线性型定义采用物理上习惯的定义,即\autoref{sequil_def1}~\upref{sequil}.
\pentry{半双线性形式\upref{sequil}}
\begin{definition}{埃尔米特型}
称半双线性型是\textbf{埃尔米特型}(Hermitian),若
\begin{equation}\label{HeFor_eq2}
f(\bvec y,\bvec x)=\overline{f(\bvec x,\bvec y)}
\end{equation}
其中横线表共轭复数.
\end{definition}
\begin{example}{埃尔米特型对应矩阵元的性质}
试证明埃尔米特型 $f$ 对应的矩阵 $F$ 的系数满足 $f_{ij}=\overline f_{ji}$.其中 $f_{ij}=f(\bvec e_i,\bvec e_j)$.这就是说 $F^*=F$,其中 $F^*=\overline {F^T}$.

\textbf{证明:}由埃尔米特型定义知
\begin{equation}
\begin{aligned}
\sum_{i,j}\overline{x_i}y_j f_{ij}&=f(\bvec x,\bvec y)=\overline{f(\bvec y,\bvec x)}\\
&=\overline{\sum_{i,j}\overline{y_j} x_i f_{ji}}=\sum_{i,j}y_j\overline{x_i}\overline{f_{ji}}
\end{aligned}
\end{equation}
对比即得 $f_{ij}=\overline{f_{ji}}$.
\end{example}
按照二次型对应的线性型与对应矩阵的命名的惯例(即名为 $name$ 型的线性型对应的矩阵称 $name$ 矩阵),有下面定义
\begin{definition}{埃尔米特矩阵}
称矩阵 $A$ 为\textbf{埃尔米特矩阵},若 $A^*=A$,其中,$A^*=\overline{A^T}$
\end{definition}
\begin{example}{埃尔米特矩阵在不同基底之下的性质}
设 $F,F'$ 分别是埃尔米特型 $f$ 在基底 $\bvec e_i$ 和 $\bvec e'_i$ 下对应的矩阵,试证明 $F'=A^*FA$,其中 $A$ 是基底 $\bvec e_i$ 到 $\bvec e'_i$ 的转换矩阵.

\textbf{证:}对任意双线性型 $g$ ,其满足关系
\begin{equation}
\begin{aligned}
&g(\bvec x,\bvec y)=\sum_{i,j}x_i y_j g_{ij}=X^TGY,\\
&X=(x_1,\cdots,x_n),\quad Y=(y_1,\cdots,y_n)
\end{aligned}
\end{equation}
而对半双线性型 $f$,则是
\begin{equation}\label{HeFor_eq1}
\begin{aligned}
&f(\bvec x,\bvec y)=\sum_{i,j}\overline{x_i} y_j f_{ij}=\overline{X^T}FY=X^*FY,\\
&X=(x_1,\cdots,x_n),\quad Y=(y_1,\cdots,y_n)
\end{aligned}
\end{equation}
于是对半双线性型成立
\begin{equation}
X^*FY=(AX')^*F(AY')={X'}^*(A^*FA)Y'
\end{equation}
又 $X^* F Y=X'^* F'Y'$,对比即得
\begin{equation}\label{HeFor_eq3}
F'=A^*F A
\end{equation}
事实上,由 ${X'}*(A^*FA-F')Y'=0$ 对任意 $X'^*$ 成立,只能是 $(A^*FA-F')Y'=\bvec 0$,而 $Y'$ 是任意的,说明 $(A^*FA-F')$ 是零矩阵,于是便得\autoref{HeFor_eq3} .
\end{example}