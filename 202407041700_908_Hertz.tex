% 海因里希 赫兹
% license CCBYSA3
% type Wiki

(本文根据 CC-BY-SA 协议转载自原搜狗科学百科对英文维基百科的翻译)

\textbf{海因里希·鲁道夫·赫兹}(/hɜːrts/;德语:[ˈhaɪ̯nʁɪç ˈhɛɐ̯ts];[1][2]1857年2月22日 –1894年1月1日)是一个德国物理学家,他最先令人信服地证明了由詹姆斯·克拉克·麦克斯韦的电磁学方程所预测的电磁波的存在。为了纪念他,频率的单位(每秒周数)被命名为“赫兹”。[3]

\subsection{生平}
海因里希·鲁道夫·赫兹于1857年出生在汉堡的一个富有而有教养的汉萨家庭,汉堡当时是德意志邦联的一个主权国家。他的父亲是古斯塔夫·费迪南·赫兹。[4]他的母亲是安娜·伊丽莎白·普费弗科恩。

在汉堡的学校Gelehrtenschule des Johanneums学习的时候,赫兹就显示出了在科学和语言学习方面的天赋,并学习了阿拉伯语和梵文。他在德国城市德累斯顿、慕尼黑和柏林在古斯塔夫·基尔霍夫和赫尔曼·亥姆霍兹的指导下学习科学和工程。1880年,赫兹从柏林大学获得博士学位,在接下来的三年里,他一直在亥姆霍兹手下做博士后研究并担任他的助手。1883年,赫兹在基尔大学担任理论物理学讲师。1885年,赫兹成为卡尔斯鲁厄大学的正教授。

1886年,赫兹与卡尔斯鲁厄大学的几何学讲师马克斯·多尔博士的女儿伊丽莎白·多尔结婚。他们有两个女儿:生于1887年10月20日的约翰娜以及生于1891年1月14日的玛蒂尔德,后者后来成为一位著名的生物学家。在此期间,赫兹对电磁波进行了里程碑式的研究。

赫兹于1889年4月3日开始在波恩物理研究所担任物理学教授和研究所主任,并在这个职位上一直工作到去世。在此期间,他致力于理论力学的研究,他的研究成果发表在了名为《Die Prinzipien der Mechanik in neuem Zusammenhange dargestellt》的著作中(即《新形式下的力学原理》),这本书出版于他死后的1894年。

\subsubsection{1.1 逝世}
1892年,赫兹在一轮严重的偏头痛后被诊断出感染,并接受手术。他最终于1894年因肉芽肿伴多血管炎在德国波恩去世,葬于汉堡的奥尔斯多夫公墓,享年36岁。[5][6][7][8]

赫兹的妻子伊丽莎白·赫兹(1864-1941)没有再婚。赫兹留下了两个女儿,乔安娜(1887-1967)和玛蒂尔德(1891-1975)。赫兹的女儿们从未结婚,因此他也没有后代。[9]

\subsection{科学工作}
\subsubsection{2.1 电磁波}
\begin{figure}[ht]
\centering
\includegraphics[width=10cm]{./figures/324973487fa4c6ed.png}
\caption{赫兹的1887年用于产生和检测无线电波的装置:火花发射器(左)由偶极天线和一个接收器(右)组成,该偶极天线具有由Ruhmkorff线圈(T)的高压脉冲供电的火花隙(S),接收器则由环形天线和火花间隙组成。} \label{fig_Hertz_1}
\end{figure}
1864年,苏格兰数学物理学家詹姆斯·克拉克·麦克斯韦提出了一个全面的电磁学理论,现在被称为麦克斯韦方程组。麦克斯韦的理论预测耦合的电场和磁场可以以“电磁波”的形式在空间中传播。麦克斯韦提出光包括短波长的电磁波,但没有人能够证明这一点,也没有人能够产生或探测到其他波长的电磁波。

当赫兹在1879年做研究的时候,亥姆霍兹建议赫兹将他的博士论文方向定为对麦克斯韦理论的检验。亥姆霍兹在那一年也将普鲁士科学院的“柏林奖”问题设定为通过实验证明绝缘体极化和去极化的一个由麦克斯韦理论所预测的电磁效应。[10][11]亥姆霍兹确信赫兹是最有可能获胜的候选人。[11]赫兹没有看出任何方法来建造一个仪器以进行实验测试,他认为这太困难了,于是转而研究电磁感应。赫兹在基尔期间确实对麦克斯韦方程组进行了分析,表明它们确实比当时流行的“超距作用”理论更有效。[12]
\begin{figure}[ht]
\centering
\includegraphics[width=6cm]{./figures/d8723858e49c1754.png}
\caption{赫兹的无线电波接收器中的一个:一个圈状天线,并带有可调节的微米火花间隙 (底部).[1]} \label{fig_Hertz_2}
\end{figure}
赫兹在卡尔斯鲁厄获得教授职位后,于1886年秋季用一对里斯螺旋进行了实验,当时他注意到将莱顿瓶放电到一个线圈会在另一个线圈中产生火花。赫兹有了一个关于如何建造仪器的想法,现在有了一个方法来解决1879年那个证明麦克斯韦理论的“柏林奖”问题(尽管实际的奖已经于1882年因为没有人能领取而过期)。[13][14]他用了一个Ruhmkorff线圈驱动火花隙和一米长线对作为发射器。电容球位于末端,用于电路谐振调节。他的接收器是一个简单的半波偶极天线,其末端之间有一个微米火花间隙。这个实验产生并接收了现在所谓的在非常高频率范围内的无线电波。
\begin{figure}[ht]
\centering
\includegraphics[width=14.25cm]{./figures/2c4fa40df05afd00.png}
\caption{赫兹的第一个无线电发射器:一个偶极谐振器,由一对一米的铜线组成,它们之间有7.5毫米的火花间隙,最后是30厘米的锌球。[1]当感应线圈在两侧之间施加高电压时,穿过火花间隙的火花在导线中产生射频电流的驻波,并辐射无线电波。波的频率大约为50 MHz,大约是现代电视发射机中使用的频率。} \label{fig_Hertz_3}
\end{figure}
1886年至1889年间,赫兹进行了一系列实验,证明他观察到的效应是麦克斯韦所预测的电磁波的结果。从1887年11月的论文《绝缘体中电扰动产生的电磁效应》开始,赫兹向柏林科学院的亥姆霍兹发送了一系列论文,包括1888年那篇显示横向自由空间电磁波以有限的速度传播一段距离的论文。[14][15]在赫兹使用的设备中,电场和磁场以横波的形式从电线辐射出去。赫兹将振荡器放置在距离锌反射板约12米的位置以产生驻波。每个波约4米长。使用环形探测器,他记录下了波的振幅和分量方向是如何变化的。赫兹测量了麦克斯韦波,并证明了这些波的速度等于光速。赫兹也测量了波的电场强度、极化和反射。这些实验证明光和这些波都是服从麦克斯韦方程组的电磁辐射。
\begin{figure}[ht]
\centering
\includegraphics[width=6cm]{./figures/9675a86697cec035.png}
\caption{赫兹的定向火花发射器(中心),一个半波偶极天线,由两个13厘米黄铜棒制成,中心有火花间隙(特写左侧),由Ruhmkorff线圈驱动,位于1.2 m x 2 m的圆柱形金属抛物面反射器的焦线上。[8] 它辐射出一束66厘米的波,频率约为450 MHz。接收器(右)是类似抛物线偶极天线,具有微米火花间隙.} \label{fig_Hertz_4}
\end{figure}
\begin{figure}[ht]
\centering
\includegraphics[width=6cm]{./figures/bb88f518466efa24.png}
\caption{赫兹对无线电波的极化的证明:接收器在天线垂直时没有响应(如图所示),但随着接收器旋转,接收信号变得更强(如火花长度所示),直到偶极子平行时达到最大值.[8]} \label{fig_Hertz_5}
\end{figure}
\begin{figure}[ht]
\centering
\includegraphics[width=6cm]{./figures/423b3ad9229622dc.png}
\caption{极化的另一种证明:只有当导线垂直于偶极子(A)时,而不是平行时(B),波才通过偏振滤波器到达接收器.[8]} \label{fig_Hertz_6}
\end{figure}
\begin{figure}[ht]
\centering
\includegraphics[width=6cm]{./figures/1eb57a3733b94aae.png}
\caption{折射演示:无线电波在通过由沥青制成的棱镜时发生弯曲,类似于通过玻璃棱镜时的光波.[8]} \label{fig_Hertz_7}
\end{figure}
\begin{figure}[ht]
\centering
\includegraphics[width=6cm]{./figures/1d80af93f157620d.png}
\caption{赫兹所绘的产生于无线电波被从金属片上反射的驻波.} \label{fig_Hertz_8}
\end{figure}
赫兹没有意识到他的无线电波实验的实际重要性。他说,[16][17][18]

" 这一点用都没有[...] 这只是一个证明麦克斯韦大师是对的的实验——我们有这些肉眼看不见的神秘电磁波。但它们确实在那里。"
当被问及他的发现的应用时,赫兹回答说:[16][19]

" 我猜没什么应用。"
赫兹对空气中传播的电磁波的存在性的证明导致了利用这种新形式电磁辐射所进行的实验的爆发,这种新形式电磁辐射被称为“赫兹波”,一直到1910年左右“无线电波”这个新名字流行起来。在10年内,欧里佛·洛兹、费迪南·布劳恩和古列尔莫·马可尼等科学家将无线电波用在了最早的无线电报无线电通信系统中,这导致了无线电广播和后来的电视的出现。1909年,布劳恩和马可尼因他们“对无线电报发展的贡献”而获得了诺贝尔物理学奖。[20]今天,无线电是全球电信网络中的一项重要技术,也是现代无线设备的传输媒介。[21]

\subsubsection{2.2 阴极射线}
1892年,赫兹开始实验并证明阴极射线可以穿透非常薄的金属箔(例如铝)。海因里希·赫兹学生菲利普·勒纳德进一步研究了这种“射线效应”。他开发了一种阴极管,并研究了X射线对各种材料的穿透。然而,菲利普·勒纳德并没有意识到他正在产生X射线。赫尔曼·亥姆霍兹提出了X射线的数学方程。在伦琴做出发现和宣布之前,亥姆霍兹就提出了色散理论。它是在光的电磁理论的基础上形成的(Wiedmann's Annalen, Vol. XLVIII)。然而,他没有使用过真正的$X$射线。

\subsubsection{2.3 光电效应}
赫兹帮助确立了光电效应(这后来被阿尔伯特·爱因斯坦所解释),当时他注意到带电的物体在紫外辐射(UV)照射下更容易失去电荷。1887年,他对光电效应以及电磁波的产生和接收进行了观察,研究成果发表在《物理年鉴》上。他的接收器由一个线圈和一个火花间隙组成,在检测到电磁波时会出现火花。他把仪器放在一个黑暗的盒子里,以便更好地观察到火花。他观察到在盒子里时最大火花长度减小了。一块被放置在电磁波源和接收器之间的玻璃面板吸收紫外线,这些紫外线有助于电子跨越间隙。移除玻璃面板后,火花长度会增加。当他用石英代替玻璃时,他没有观察到火花长度的减少,因为石英不吸收紫外辐射。赫兹结束了他几个月的研究,并报告了获得的结果。他没有进一步研究这种效应,也没有试图解释观察到的现象是如何产生的。

\subsection{2.4 接触力学}
\begin{figure}[ht]
\centering
\includegraphics[width=6cm]{./figures/6d84f9dee85b7bcc.png}
\caption{卡尔斯鲁厄理工学院校园海因里希·赫兹纪念碑,上面的文字翻译为“在这个地方,海因里希·赫兹在1885-1889年发现了电磁波”。} \label{fig_Hertz_9}
\end{figure}
1886-1889年,赫兹发表了两篇关于后来被称为接触力学的领域的文章,它们被证明是该领域后来的理论的重要基础。约瑟夫·瓦伦丁·布辛尼斯克发表了一些关于赫兹的工作的具有批判性的重要观察,尽管如此,他仍然将赫兹这项关于接触力学的工作定位为极其重要的成果。赫兹的工作基本上总结了两个互相接触的轴对称物体在载荷下的行为,他的成果是基于经典的弹性理论和连续介质力学理论上的。他的理论最大的失败是忽略了两种固体之间的任何性质的粘附,后者被证明在组成固体的材料开始呈现高弹性时是重要的。在那个时代忽视粘附是很自然的做法,因为没有实验方法可以用来检测它。

为了发展他的理论,赫兹使用他对将玻璃球放置在透镜上所形成的椭圆牛顿环的观察作为假设球体施加的压力遵循椭圆分布的基础。他再次利用牛顿环的形成通过实验验证他的理论,在他的理论中赫兹计算了球体进入透镜的位移。K. L. Johnson、K. Kendall和A. D. Roberts (JKR)于1971年使用赫兹的理论作为基础计算了存在粘附的情况下的理论位移或压痕深度。[22]如果假设材料的粘附为零,他们的表述便退化为赫兹的理论。类似于这个理论,然而使用不同的假设, B. V. Derjaguin 、V. M. Muller和Y. P. Toporov在1975年发表了另一个理论,这个理论在研究界被称为DMT理论,它也在零粘附的假设下退化为赫兹的理论。这个DMT理论被证明是相当不成熟的,它在被接受为除JKR理论之外的另一种材料接触理论之前还需要进行几次修改。DMT理论和JKR理论都是接触力学的基础,所有过渡接触模型都是以其为基础的,并被用于纳米压痕和原子力显微镜中的材料参数预测。因此,赫兹的研究从他做讲师的日子开始已经走到了纳米技术时代,这些研究发生在他关于电磁学的伟大工作之前,对于这些工作,向来持重的赫兹确认为是微不足道的。

赫兹还描述了“赫兹锥”,这是一种由应力波的传播所引起的脆性固体中的断裂模式。

\subsubsection{2.5 气象学}
赫兹对气象学一直很感兴趣,这可能源于他与威廉·冯·贝佐德的接触(后者是1878年夏天他在慕尼黑理工学院的一门实验室课程的教授)。作为亥姆霍兹在柏林的助手,赫兹在气象学领域贡献了几篇小文章,包括对液体的蒸发的研究、一种新型的湿度计以及测定潮湿空气在经历绝热变化时的特性的图形方法。[23]

\subsection{纳粹迫害}
海因里希·赫兹一生都是路德教徒,他并不认为自己是犹太人,因为1834年他父亲还陪伴着童年的赫兹(7岁))时[24]他父亲的家人就都皈依了路德教派。[25]

然而,当纳粹政权在赫兹去世几十年后掌权时,他的肖像被他们从汉堡市政厅(市议会厅)中显眼的荣誉位置上移走了,这是因为他有部分犹太血统。(那幅画后来又重新公开展出了)[26]

赫兹的遗孀和女儿在20世纪30年代离开德国去了英国。

\subsection{遗产和荣誉}
\begin{figure}[ht]
\centering
\includegraphics[width=6cm]{./figures/ed5f9bb8e213db7d.png}
\caption{海因里希·赫兹} \label{fig_Hertz_10}
\end{figure}
海因里希·赫兹的侄子古斯塔夫·路德维希·赫兹是诺贝尔奖获得者,古斯塔夫的儿子卡尔·赫尔穆特·赫兹发明了医学超声检查。他的女儿玛蒂尔德·卡门·赫兹是一位著名的生物学家和比较心理学家。赫兹的侄孙赫尔曼·格哈德·赫兹是卡尔斯鲁厄大学的教授,他也是核磁共振波谱法的先驱,并于1995年发表了赫兹的实验室笔记。[27]

国际单位制中的频率的单位赫兹(Hz)是由国际电工委员会在1930年为纪念他而设立的,表示每秒重复事件发生的次数。它于1960年被国际度量衡大会采用,正式取代了以前的名称“每秒循环数”(CPS)。

1928年,海因里希-赫兹振荡研究所在柏林成立。今天被称为Fraunhofer Institute for Telecommunications, Heinrich Hertz Institute, HHI。

1969年(东德),海因里希·赫兹纪念奖章被设立。[28]成立于1987年的IEEE海因里希·赫兹奖章是“为了对赫兹波的杰出成就 [...]每年授予一个人,基于其理论性或实验性的成就“。

1980年,在意大利,一所名为“海因里希·赫兹工业学院”的高中在罗马的奇尼奇塔东部附近成立。

一个位于月球背面的撞击坑(就在东部边缘后面)就是以赫兹的名字命名的。位于俄罗斯下诺夫哥罗德的赫兹无线电电子产品市场也是以他的名字命名的。汉堡的海因里希-赫兹无线电通信塔就是以该市著名的儿子命名的。

赫兹被日本授予了瑞宝章,这是日本为包括科学家在内的杰出人士所提供的多层次荣誉。[29]

世界上许多国家以发行纪念邮票的形式纪念赫兹,二战后的德国邮票上也出现了赫兹的形象。

2012年赫兹诞辰那天,谷歌在其主页上用一个 Google涂鸦纪念赫兹,这个 Google涂鸦是受赫兹一生工作的启发而创作的。[30][31]