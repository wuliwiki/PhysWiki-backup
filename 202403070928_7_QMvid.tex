% 量子力学科普视频脚本
% license Usr
% type Art

\begin{issues}
\issueDraft
\end{issues}


\subsubsection{}




\pentry{量子力学的基本原理(科普)\nref{nod_QM0}}{nod_5f89}

解说:现代的量子力学如何描述微观粒子的运动? 为了简单我们先看直线运动。 经典力学中,根据牛顿第一定律,不受力的小球保持静止或匀速直线运动。

动画:画出一个数轴(动画拉长方式出现,有刻度线,配 click 音效),小球静止在原点(渐变出现), 一个火柴人出现,踢一脚(画出受力箭头突然变长然后变短,速度箭头突然变长然后固定不变),匀速向右运动, 镜头跟随。

画面暂停: 解说:如果我们不考虑小球的形状和旋转,可以把它抽象成一个运动的有质量的点(画面把球缩成一点)。 经典力学告诉我们, 每个时刻质点的状态由位置和速度完全确定。

解说:如果我们想要描述电子、质子等基本粒子的运动,我们就需要使用量子力学。 在量子力学中,微观粒子不再适合用确定的位置和速度描述, 而是需要把它看成一个波动。

解说:生活中处处可见波动现象,使用一根橡皮绳,就可以探索波动的各种规律。

动画:天上掉下一根绳子,弯曲躺在地上,右边超出屏幕,小人捡起绳子在地上的左端,拉进绳子,抖动几下停下来,绳子出现向右运动的波,消失在屏幕右边。 又胡乱抖了几下,出现了各种形状各异的波消失在屏幕右边。

解说:在橡皮绳的一头振动几次,就可以产生向右传递的波。 这些波看起来像一个个包,我们把它叫做波包。波包的特点是长度有限。 如果我们假设绳子右边无限长,那么这些波包会一直保持固定的形状向右匀速运动。

解说:如果在橡皮绳的一头以固定的频率振动,那么经过一段时间绳子就形成了一列完美的简谐波。
(动画根据解说)

解说:

解说:如果把绳的另一端固定,那么这些波包会发生反射。(动画根据解说)




波函数描述。 而波函数通常以波包的形状出现。

动画:渐变成开始的坐标系,高斯波包的实部渐变出现,不画虚部和模长。

解说:波函数通常记为 $\psi(x)$, 把每个坐标 $x$ 对应到函数值 $y = \psi(x)$。

动画:画出一个从数轴向上指向曲线的箭头, 快速向右移动, 坐标轴下面跟随显示坐标, 箭头上方画水平线到 $y$ 轴,显示 $y$ 坐标。

解说:事实上波函数的值是一个复数, 刚才画出的只是函数值的实数部分,我们还可以用另一条曲线表示虚数部分,再用一条曲线表示复数的模长。 模长曲线就是某个位置振动幅度的大小。

动画:根据解说

解说:当时间开始流逝,波包的整体向右移动,同时它的形状也会发生一些改变。

动画:根据解说

解说:这样一个匀速运动的波包对应的是经典力学中向右匀速运动的质点,它不受任何外力作用。

动画:上下画两个数轴,上面演示波包向右运动,下面演示小球向右运动。

解说:
