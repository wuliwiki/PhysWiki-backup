% 恩里科·费米(综述)
% license CCBYSA3
% type Wiki

本文根据 CC-BY-SA 协议转载翻译自维基百科 \href{https://en.wikipedia.org/wiki/Enrico_Fermi}{相关文章}。

恩里科·费米(意大利语:[enˈriːko ˈfermi],1901年9月29日-1954年11月28日)是一位意大利裔、后归化为美国公民的物理学家,以建造世界上第一座人工核反应堆——芝加哥一号堆而闻名,并曾是曼哈顿计划的重要成员。他被誉为“核时代的建筑师”以及“原子弹之父”。\(^\text{[1]}\)他是极少数在理论物理和实验物理两个领域都卓有成就的物理学家之一。费米因其在中子轰击引发放射性方面的研究以及对超铀元素的发现而获得1938年诺贝尔物理学奖。他与同事们共同申请了多项与核能应用相关的专利,所有这些专利最终都被美国政府接管。他在统计力学、量子理论、核物理和粒子物理的发展中都作出了重要贡献。

费米的第一个重大贡献是在统计力学领域。1925年,沃尔夫冈·泡利提出了著名的泡利不相容原理,随后费米发表了一篇论文,将该原理应用于理想气体,发展出一种统计方法,如今被称为费米–狄拉克统计。今天,那些遵守不相容原理的粒子被称为“费米子”。后来,泡利为了解释β衰变中能量守恒的问题,提出了在电子发射的同时还会发射一种不带电的不可见粒子这一假设。费米接纳了这个想法,并构建了一个理论模型,纳入了这一假想粒子,并将其命名为“中微子”。他的这一理论后来被称为“费米相互作用”,现今称为“弱相互作用”,是自然界四种基本相互作用之一。在用新发现的中子进行诱导放射性实验时,费米发现慢中子比快中子更容易被原子核俘获,并据此发展出描述该过程的“费米年龄方程”。在用慢中子轰击钍和铀的实验中,费米认为自己合成了新的元素。尽管他因这一发现获得了诺贝尔奖,但后来证实这些“新元素”其实是核裂变的产物。1938年,为了躲避影响其犹太妻子劳拉·卡蓬的意大利新种族法,费米离开意大利,移民美国。在第二次世界大战期间,他参与了“曼哈顿计划”。在芝加哥大学,费米领导的团队设计并建造了“芝加哥堆-1”,该堆于1942年12月2日首次实现了人类制造的、自持的核链式反应。他还在田纳西州橡树岭的X-10石墨反应堆于1943年达到临界状态时在场,次年又见证了华盛顿州汉福德基地的B反应堆启动。在洛斯阿拉莫斯国家实验室,他领导F部门,其中一部分致力于爱德华·泰勒的热核“超级炸弹”项目。他还亲历了1945年7月16日的“特立尼蒂试验”,即首次核弹爆炸测试,并使用著名的“费米估算法”评估了炸弹的当量。

战后,费米协助创建了芝加哥的核研究所,并在J·罗伯特·奥本海默担任主席的总顾问委员会中任职,为美国原子能委员会提供核事务建议。1949年8月苏联成功引爆第一颗裂变原子弹后,费米从道德和技术两方面都强烈反对研制氢弹。1954年,在导致奥本海默失去安全许可的听证会上,费米也是为奥本海默作证的科学家之一。

费米在粒子物理领域也做出了重要贡献,尤其是在与介子(如π介子和μ子)相关的研究方面。他还推测宇宙射线的产生是由于星际空间中的磁场加速物质所致。许多奖项、概念和机构都以费米的名字命名,包括费米一号(快中子增殖反应堆)、恩里科·费米核发电站、恩里科·费米奖、恩里科·费米研究所、费米国家加速器实验室、费米伽马射线太空望远镜、“费米悖论”,以及人造元素“镄”,这使他成为仅有的十六位拥有化学元素以自己命名的科学家之一。
\subsection{早年生活}
\begin{figure}[ht]
\centering
\includegraphics[width=6cm]{./figures/635b80f64a6090ee.png}
\caption{费米出生于罗马盖塔街19号。} \label{fig_ELK_1}
\end{figure}
恩里科·费米于1901年9月29日出生在意大利罗马。\(^\text{[3]}\)他是铁路部司局长阿尔贝托·费米与小学教师伊达·德·加蒂斯的第三个孩子。\(^\text{[3][4][5]}\)他的姐姐玛丽亚比他大两岁,哥哥朱利奥大他一岁。两位男孩幼年时被送到乡下由乳母抚养,直到恩里科两岁半时才返回罗马与家人团聚。\(^\text{[6]}\)尽管按照祖父母的意愿他接受了天主教洗礼,但他的家庭并不虔诚;恩里科成年后一直是无神论者。\(^\text{[7]}\)童年时期,他和哥哥朱利奥有着相同的兴趣爱好,比如制作电动机、玩电动和机械玩具。\(^\text{[8]}\)1915年,朱利奥因喉部脓肿手术不幸去世;玛丽亚则于1959年在米兰附近的一场空难中遇难。\(^\text{[9][10]}\)

在罗马的鲜花广场集市上,费米发现了一本物理书,名为《Elementorum physicae mathematicae》,全书900页,由耶稣会士、罗马学院教授安德烈亚·卡拉法神父用拉丁文编写。书中介绍了当时(1840年出版)对数学、经典力学、天文学、光学和声学的理解。\(^\text{[11][12]}\)在一位同样对科学感兴趣的朋友恩里科·佩尔西科的陪伴下,\(^\text{[13]}\)费米开始了诸如制作陀螺仪、测量地球重力加速度等实验项目。\(^\text{[14]}\)

1914年,费米常常在父亲下班后到办公室门口与其会合。有一次,他遇见了父亲的一位同事阿道夫·阿米代伊。恩里科得知阿道夫对数学和物理感兴趣,便趁机向他请教几何问题。阿道夫意识到小费米问的是射影几何,随后送给他一本由特奥多尔·雷耶所著的相关书籍。两个月后,费米将书还给了阿道夫,声称自己已完成了书后所有的习题,其中一些题目连阿道夫都认为很难。阿道夫核实后惊叹道费米“至少在几何方面是个天才”,并开始更加系统地指导他,给他提供更多关于物理与数学的书籍。阿道夫还指出,费米的记忆力极好,读完书后就能记住全部内容,因此通常读完便归还书籍。\(^\text{[15]}\)
\subsection{比萨高等师范学院}
\begin{figure}[ht]
\centering
\includegraphics[width=6cm]{./figures/7f8ade509d759b3a.png}
\caption{} \label{fig_ELK_2}
\end{figure}
费米于1918年7月高中毕业,他跳过了第三学年。在阿米代伊的敦促下,费米学习了德语,以便阅读当时大量以德语发表的科学论文,并申请了比萨高等师范学院。阿米代伊认为,这所学校能为费米的发展提供比当时的罗马萨皮恩扎大学更好的条件。费米的父母在失去一个儿子后,勉强同意他在学校的住宿区生活四年,远离罗马。\(^\text{[16][17]}\)费米在难度极高的入学考试中获得第一名,其中包括一篇题为“声音的特征”的作文;17岁的费米选择使用傅里叶分析来导出并求解振动棒的偏微分方程。考官在面试他之后断言,费米将成为一位杰出的物理学家。\(^\text{[16][18]}\)

在比萨高等师范学院,费米与同学弗兰科·拉塞蒂一起恶作剧,两人成为了亲密的朋友和合作伙伴。物理实验室主任路易吉·普恰蒂是费米的导师,他曾表示自己几乎教不了费米什么,反而常常请费米教他一些东西。\(^\text{[19]}\)费米在量子物理方面的知识如此扎实,以至于普恰蒂请他主持有关该主题的研讨会。在此期间,费米学习了张量计算法,这是一种在广义相对论中至关重要的数学工具。\(^\text{[20]}\)费米最初选择数学作为主修专业,但很快转为物理学。他在很大程度上是自学成才,系统学习了广义相对论、量子力学和原子物理学。\(^\text{[21]}\)

1920年9月,费米被正式录取进入物理系。由于系里只有三名学生——费米、拉塞蒂和内洛·卡拉拉——导师普恰蒂便允许他们自由使用实验室,进行任何他们感兴趣的研究。\(^\text{[22]}\)费米决定他们应该研究X射线晶体学,于是三人开始合作拍摄劳厄照片——即晶体的X射线照片。1921年,费米在大学三年级期间,在意大利期刊《新物理杂志》上发表了他的第一批科学论文。第一篇题为《平移运动中电荷刚体系统的动力学》(意大利语原文:Sulla dinamica di un sistema rigido di cariche elettriche in moto traslatorio)。这篇论文展现了费米未来研究方向的端倪:他将质量表达为张量——这是一个常用于描述三维空间中运动和变化的数学结构。在经典力学中,质量是一个标量,但在相对论中,它随着速度变化。第二篇论文是《在均匀引力场中电磁电荷的静电学以及电磁电荷的重量》。在这篇论文中,费米使用广义相对论证明,一个电荷的重量等于其系统的静电能量$U$除以光速$c$的平方,即 $U/c^2$。\(^\text{[21]}\)

第一篇论文似乎指出了电动力学理论与相对论之间在电磁质量计算方面的矛盾:电动力学预测的电磁质量为 $\frac{4}{3} \frac{U}{c^2}$,而相对论则给出 $\frac{U}{c^2}$。次年,费米在论文《论电动力学与相对论在电磁质量问题上的一个矛盾》中解决了这个问题,他指出这种表面上的矛盾实际上是相对论自身的一个推论。这篇论文得到了高度评价,并于1922年被翻译成德文,发表在德国科学期刊《物理学杂志》上。\(^\text{[23]}\)同年,费米向意大利期刊《林琴学会会报》提交了题为《在时线附近发生的现象》(意大利语原文:Sopra i fenomeni che avvengono in vicinanza di una linea oraria)的论文。在这篇论文中,他研究了等效原理,并引入了后被称为“费米坐标”的概念。他证明,在接近某条时线的世界线上,空间的行为可以被视为欧几里得空间。\(^\text{[24][25]}\)
\begin{figure}[ht]
\centering
\includegraphics[width=8cm]{./figures/b24ae5308fe0c01a.png}
\caption{光锥是时空中从某一点出发或到达该点的所有可能光线所构成的三维曲面。在图示中,压缩了一个空间维度。时间线为垂直轴。} \label{fig_ELK_3}
\end{figure}
费米于1922年7月向比萨高等师范学校提交了他的论文《一个概率定理及其若干应用》(意大利语:Un teorema di calcolo delle probabilità ed alcune sue applicazioni),并在异常年轻的20岁时获得学位。论文主题是X射线衍射图像。当时理论物理在意大利尚未被视为一个正式学科,唯一能被接受的毕业论文是实验物理类。因此,意大利物理学界在接受来自德国的新思想(如相对论)方面进展缓慢。由于费米在实验室工作中如鱼得水,这一限制对他而言并非不可逾越的障碍。\(^\text{[25]}\)

1923年,在为奥古斯特·科普夫所著《爱因斯坦相对论基础》意大利语版撰写附录时,费米是第一个指出爱因斯坦公式($E = mc^2$)中隐藏着巨大的核能可供开发的人。\(^\text{[26]}\)他写道:“至少在可预见的将来,似乎不可能找到释放这些可怕能量的方法——这也好,因为如此惊人能量一旦爆炸,首先被炸得粉碎的就是那个不幸找到释放方法的物理学家。”\(^\text{[25]}\)

1924年,费米加入了意大利东方大会下属的“阿德里亚诺·莱米”共济会支部。\(^\text{[27]}\)

在1923至1924年间,费米在哥廷根大学跟随马克斯·玻恩学习了一个学期,并在那里结识了维尔纳·海森堡和帕斯夸尔·约尔当。随后,他在1924年9月至12月期间,凭借数学家维托·沃尔特拉的推荐,获得洛克菲勒基金会的奖学金,在莱顿大学与保罗·埃伦费斯特一同学习。在那里,费米遇到了亨德里克·洛伦兹和阿尔伯特·爱因斯坦,并与塞缪尔·古兹密特和扬·廷贝亨成为朋友。

从1925年1月到1926年末,费米在佛罗伦萨大学教授数学物理和理论力学课程,并与拉塞蒂合作,开展了一系列关于磁场对汞蒸气影响的实验。他还在罗马大学参与学术研讨会,讲授量子力学和固体物理。\(^\text{[28]}\)在讲授基于薛定谔方程的、能极为准确地进行预测的新量子力学时,费米常半带惊讶地说:“它居然吻合得这么好,真是不可思议!”\(^\text{[29]}\)

1925年,沃尔夫冈·泡利提出泡利不相容原理后,费米迅速作出回应,发表论文《关于理想单原子气体的量子化》(意大利语:Sulla quantizzazione del gas perfetto monoatomico),将该原理应用于理想气体。这篇论文最具特色之处在于费米提出了一种统计表达形式,描述了大量服从不相容原理的同类粒子在系统中的分布。英国物理学家保罗·狄拉克随后也独立发展了这一统计理论,并指出它与玻色–爱因斯坦统计的关系。因而,该统计方法如今被称为费米–狄拉克统计。\(^\text{[30]}\)自狄拉克之后,遵循不相容原理的粒子被称为“费米子”,而不遵循该原理的粒子则被称为“玻色子”。\(^\text{[31]}\)
\subsection{罗马的教授}
\begin{figure}[ht]
\centering
\includegraphics[width=6cm]{./figures/904220e3d5812f55.png}
\caption{费米和他的研究小组(称为“潘尼斯佩尔纳小组”)在罗马大学物理学研究所的院子里,约1934年。从左到右:奥斯卡·达戈斯蒂诺、埃米利奥·塞格雷、埃多瓦尔多·阿马尔迪、弗朗科·拉塞蒂和费米。} \label{fig_ELK_4}
\end{figure}
意大利的教授职位是通过竞争来授予的,申请人根据其发表的作品,由一委员会的教授进行评定。费米曾申请了萨丁岛卡利亚里大学的数学物理学教授职位,但由于吉奥万尼·乔尔吉的竞争,他未能成功当选。\(^\text{[32]}\)1926年,24岁的费米申请了罗马大学的一个教授职位。这个职位是新设立的,是意大利首批三个位于理论物理学领域的教授职位之一,该职位是应实验物理学教授、物理学研究所所长以及贝尼托·墨索里尼内阁成员奥尔索·马里奥·科尔比诺教授的倡议,由教育部长创建的。科尔比诺教授同时也主持了评选委员会,他希望这个新职位能够提升意大利物理学的标准和声誉。\(^\text{[33]}\)评选委员会选择了费米,而非恩里科·佩尔西科和阿尔多·庞特雷莫利。\(^\text{[34]}\)科尔比诺帮助费米招募了他的团队,团队很快吸引了埃多瓦尔多·阿马尔迪、布鲁诺·庞特科沃、埃托雷·马约拉纳、埃米利奥·塞格雷等杰出学生的加入,并且费米任命了弗朗科·拉塞蒂为他的助手。\(^\text{[35]}\)他们很快就被昵称为“潘尼斯佩尔纳小组”,以纪念物理学研究所所在的街道——潘尼斯佩尔纳街。\(^\text{[36]}\)

费米于1928年7月19日与大学的科学专业学生劳拉·卡彭结婚。\(^\text{[37]}\)他们有两个孩子:内拉,生于1931年1月,朱利奥,生于1936年2月。\(^\text{[38]}\)1929年3月18日,费米被墨索里尼任命为意大利皇家学会成员,4月27日加入法西斯党。当1938年墨索里尼出台种族法案,使意大利法西斯主义在意识形态上更加接近德国纳粹主义时,费米反对法西斯主义。这些法律威胁到了犹太人身份的劳拉,并使费米的许多研究助手失业。\(^\text{[39][40][41][42][43]}\)

在罗马期间,费米和他的团队对物理学的许多实际和理论方面做出了重要贡献。1928年,他出版了《原子物理学导论》,为意大利大学生提供了一本最新的、易于理解的教材。费米还举办了公开讲座,并为科学家和教师撰写了普及性文章,以尽可能广泛地传播新物理学的知识。\(^\text{[44]}\)他的一部分教学方法是,在一天的工作结束后,召集同事和研究生们一起讨论问题,通常是来自他自己研究中的问题。\(^\text{[44][45]}\)成功的标志之一是,外国学生开始来意大利学习。最著名的学生之一是德国物理学家汉斯·贝特(Hans Bethe),他作为洛克菲勒基金会的研究员来到罗马,并与费米合作撰写了1932年的论文《两电子之间的相互作用》(德语:Über die Wechselwirkung von Zwei Elektronen)。\(^\text{[47][44]}\)

在这个时候,物理学家们对β衰变感到困惑,在β衰变中,一个电子从原子核中发射出来。为了满足能量守恒定律,保利假设存在一个无电荷且质量极小或没有质量的不可见粒子,它在同一时间也被发射出来。费米接受了这一想法,并在1933年写下了初步的论文,接着在第二年写了一篇更长的论文,纳入了这一假设的粒子,费米称之为“中微子”。\(^\text{[48][49][50]}\)他的理论,后来被称为费米相互作用,进一步被称为弱相互作用理论,描述了自然界的四种基本相互作用之一。中微子在费米去世后被发现,而他的相互作用理论则揭示了为什么中微子如此难以被探测到。当费米将他的论文提交给英国期刊《自然》时,期刊的编辑拒绝了它,因为它包含了“与物理现实相距太远的推测,不足以引起读者的兴趣”。\(^\text{[49]}\)根据费米的传记作者David N. Schwartz的说法,至少可以说,费米认真要求该期刊出版他的论文是非常奇怪的,因为当时《自然》期刊只会发表这种类型文章的简短笔记,并不适合发表任何新的物理理论。如果要找更合适的期刊,可能《伦敦皇家学会会刊》会更合适。他同意一些学者的假设,即英国期刊的拒绝让费米的年轻同事们(其中一些是犹太人和左翼分子)放弃了对德国科学期刊的抵制,尤其是在希特勒于1933年1月上台后。\(^\text{[51]}\)因此,费米看到了该理论首先在意大利语和德语中出版,然后才在英语中发表。\(^\text{[35]}\)

在1968年英文版的序言中,物理学家Fred L. Wilson指出:

费米的理论,除了支持保利提出的中微子假设外,在现代物理学史上具有特殊意义。必须记住,在理论提出时,仅知道天然发生的$\beta$发射源。后来,当发现正电子衰变时,这一过程很容易被纳入费米的原始框架中。根据他的理论,预测了轨道电子被原子核捕获的现象,并最终得到了观测。随着时间的推移,实验数据显著积累。尽管在$\beta$衰变中多次观察到一些异常现象,费米的理论始终能够应对挑战。

费米理论的影响是广泛的。例如,$\beta$光谱学被确立为研究核结构的强有力工具。但或许费米工作的最具影响力的方面是,他特定形式的$\beta$相互作用确立了一个模式,这一模式适用于研究其他类型的相互作用。这是第一个成功的物质粒子创造与湮灭的理论。在此之前,只有光子被知晓是可以创造和消灭的。\(^\text{[50]}\)

1934年1月,伊雷娜·约里奥-居里和弗雷德里克·约里奥宣布,他们用α粒子轰击元素并使其产生了放射性。\(^\text{[52][53]}\)到3月,费米的助手吉安-卡洛·威克用费米的$\beta$衰变理论提供了理论解释。费米决定转向实验物理,使用1932年詹姆斯·查德威克发现的中子。\(^\text{[54]}\)1934年3月,费米希望通过拉塞蒂的钋铍中子源来看看是否能引发放射性。中子没有电荷,因此不会被带正电的原子核偏转。这意味着它们穿透原子核所需的能量远低于带电粒子,因此不需要粒子加速器,而“Via Panisperna boys”并没有粒子加速器。\(^\text{[55][56]}\)
\begin{figure}[ht]
\centering
\includegraphics[width=8cm]{./figures/b5362410b9f2f7c9.png}
\caption{恩里科·费米(Enrico Fermi)与弗朗科·拉塞蒂(Franco Rasetti,左)和埃米利奥·塞格雷(Emilio Segrè)穿着学术服装合影。} \label{fig_ELK_5}
\end{figure}
费米想到用氡-铍中子源替代钋-铍中子源,他通过将铍粉末填充到玻璃瓶中,抽出空气,然后加入50毫居里的氡气(由朱利奥·切萨雷·特拉巴基提供)来制造这种源。\(^\text{[57][58]}\)这创造了一个更强的中子源,但其有效性随着氡的3.8天半衰期而衰减。他知道这个源还会发射伽马射线,但基于他的理论,他认为这不会影响实验结果。他开始用铂金属进行轰击,铂是一种具有较高原子序数且易于获得的元素,但没有成功。接着他转向铝,铝发射出一个$\alpha$粒子并生成钠,钠通过$\beta$粒子发射衰变为镁。他又尝试了铅,但没有成功,然后尝试了氟,形式是氟化钙,氟发射一个$\alpha$粒子并生成氮,氮通过$\beta$粒子发射衰变为氧。总的来说,他在22种不同的元素中诱发了放射性。\(^\text{[59]}\)费米迅速在1934年3月25日的意大利期刊《La Ricerca Scientifica》上报告了中子诱发放射性的发现。\(^\text{[58][60][61]}\)

由于钍和铀的天然放射性,使得很难确定当这些元素被中子轰击时发生了什么。但在正确排除了比铀轻但比铅重的元素的存在后,费米得出结论,他们已经创造了新元素,他称之为奥斯尼姆和赫斯佩里姆。\(^\text{[62][56]}\)化学家伊达·诺达克建议,某些实验可能产生了比铅更轻的元素,而不是新产生的、更重的元素。当时,她的建议并未受到重视,因为她的团队既没有进行铀的实验,也没有为这一可能性建立理论基础。当时,裂变被认为在理论上是不可行的。如果不是不可能的话。尽管物理学家们预计,通过中子轰击轻元素会形成原子序数更高的元素,但没有人预料到中子会有足够的能量以诺达克所建议的方式,将较重的原子分裂成两个轻元素碎片。\(^\text{[63][62]}\)
\begin{figure}[ht]
\centering
\includegraphics[width=6cm]{./figures/0d154bb9cd18a6d5.png}
\caption{$\beta$衰变。一颗中子衰变为一个质子,并发射出一个电子。为了保持系统的总能量不变,保利和费米假设也会发射一个反电子中微子(${\displaystyle {\bar {\nu }}_{e}}$)。} \label{fig_ELK_6}
\end{figure}
“Via Panisperna的男孩们还注意到一些无法解释的现象。实验似乎在木桌上比在大理石桌面上效果更好。费米记得Joliot-Curie和Chadwick曾指出石蜡在减速中子方面非常有效,所以他决定试一试。当中子通过石蜡时,它们在银上诱发的放射性是没有石蜡时的100倍。费米猜测这是因为石蜡中的氢原子,木头中的氢原子也解释了木桌面与大理石桌面之间的差异。通过重复用水做实验,证明了这一点。他得出结论,氢原子与中子的碰撞使得中子减速。\(^\text{[64][56]}\)中子与其碰撞的原子核的原子序数越低,每次碰撞损失的能量越多,因此,减少中子的速度所需的碰撞次数就越少。\(^\text{[65]}\)费米意识到,这会引发更多的放射性,因为慢中子比快中子更容易被捕获。他开发了一个扩散方程来描述这一现象,这个方程后来被称为费米年龄方程。\(^\text{[64][56]}\)

1938年,费米因其“通过中子辐照证明了新放射性元素的存在,并因此发现了由慢中子引起的核反应”而获得了诺贝尔物理学奖,时年37岁。\(^\text{[66]}\)在斯德哥尔摩获得诺贝尔奖后,费米没有回到意大利,而是与家人一起于1938年12月继续前往纽约市,并在那里申请了永久居留权。决定移居美国并成为美国公民,主要是由于意大利的种族法。\(^\text{[39][67]}\)”
\subsection{曼哈顿计划}
\begin{figure}[ht]
\centering
\includegraphics[width=8cm]{./figures/e2134171ebe1af60.png}
\caption{芝加哥堆-1的插图,这是第一个实现自持链式反应的核反应堆。由费米设计,堆芯由铀和铀氧化物组成,形成立方体晶格,嵌入石墨中。} \label{fig_ELK_7}
\end{figure}
费米于1939年1月2日抵达纽约市。\(^\text{[68]}\)他立即收到了五所大学的职位邀请,并接受了哥伦比亚大学的职位,\(^\text{[69]}\)他曾在1936年夏天在那里讲授过课程。\(^\text{[70]}\)他得知,在1938年12月,德国化学家奥托·哈恩和弗里茨·斯特拉斯曼通过用中子轰击铀元素后,发现了铋元素,\(^\text{[71]}\)这被莉泽·迈特纳和她的侄子奥托·弗里施正确地解释为核裂变的结果。弗里施于1939年1月13日进行了实验验证。\(^\text{[72][73]}\)迈特纳和弗里施对哈恩和斯特拉斯曼发现的解释的消息通过尼尔斯·玻尔传到了大西洋彼岸,玻尔要在普林斯顿大学讲学。伊西多尔·艾萨克·拉比和威利斯·兰姆这两位在普林斯顿工作的哥伦比亚大学物理学家得知了这一消息,并将其带回了哥伦比亚大学。拉比表示,他告诉了恩里科·费米,但费米后来将功劳归于兰姆。\(^\text{[74]}\)
\begin{figure}[ht]
\centering
\includegraphics[width=6cm]{./figures/6366ab5e40a0d17a.png}
\caption{费米在洛斯阿拉莫斯的身份证照片} \label{fig_ELK_8}
\end{figure}
我记得非常清楚,1939年1月,我开始在普平实验室工作的第一个月,因为事情进展得非常迅速。在那段时间,尼尔斯·玻尔正在普林斯顿大学进行讲座,我记得有一个下午,威利斯·兰姆非常兴奋地回来了,说玻尔泄露了一个重磅消息。这个消息就是裂变的发现,至少有了其解释的初步轮廓。然后,稍后在同一个月,华盛顿举行了一个会议,首次以半开玩笑的认真态度讨论了裂变这一新发现现象可能作为核能来源的重要性。\(^\text{[75]}\)

诺达克最终被证明是对的。费米曾根据自己的计算排除了裂变的可能性,但他没有考虑到当一个中子数为奇数的核素吸收一个额外中子时,所产生的结合能。\(^\text{[63]}\)对于费米来说,这个消息是一个深刻的尴尬,因为他部分因发现的超铀元素被授予了诺贝尔奖,而这些所谓的超铀元素根本不是超铀元素,而是裂变产物。他在诺贝尔奖获奖演讲中加入了这一点的脚注。\(^\text{[74][76]}\)
\begin{figure}[ht]
\centering
\includegraphics[width=6cm]{./figures/41d7c72dc08e530f.png}
\caption{欧内斯特·O·劳伦斯、费米和伊西多尔·艾萨克·拉比} \label{fig_ELK_9}
\end{figure}
哥伦比亚大学的科学家们决定尝试检测铀在中子轰击下发生核裂变时释放的能量。1939年1月25日,在哥伦比亚大学普平大厅的地下室,费米所在的实验团队进行了美国第一次核裂变实验。团队的其他成员包括赫伯特·L·安德森、尤金·T·布斯、约翰·R·邓宁、G·诺里斯·格拉索和弗朗西斯·G·斯莱克。\(^\text{[77]}\)第二天,第五届华盛顿理论物理会议在华盛顿特区开幕,由乔治·华盛顿大学和华盛顿卡内基学会共同主办。在那里,关于核裂变的消息进一步传播,推动了更多实验演示的开展。\(^\text{[78]}\)
\begin{figure}[ht]
\centering
\includegraphics[width=6cm]{./figures/9cdea6e7bd3be4ac.png}
\caption{FERMIAC 是费米发明的一种模拟计算机,用于研究中子传输。} \label{fig_ELK_10}
\end{figure}
法国科学家汉斯·冯·哈尔班、刘·科瓦尔斯基和弗雷德里克·居里证明了中子轰击铀时,铀释放的中子比吸收的更多,这暗示了链式反应的可能性。\(^\text{[79]}\)费米和安德森在几周后也做了类似的实验。\(^\text{[80][81]}\)莱欧·西拉德从加拿大的铀生产商埃尔多拉多黄金矿业公司获得了200公斤(440磅)铀氧化物,这使得费米和安德森能够进行大规模的裂变实验。\(^\text{[82]}\)费米和西拉德合作设计了一种装置,旨在实现自持的核反应——即核反应堆。由于水中的氢吸收中子的速率,自然铀和水作为中子慢化剂不太可能实现自持反应。费米基于他与中子的工作,建议可以用铀氧化物块和石墨代替水作为中子慢化剂,这样可以减少中子的捕获率,理论上使得自持链式反应成为可能。西拉德提出了一个可行的设计:将铀氧化物块与石墨砖交替堆积。\(^\text{[83]}\)西拉德、安德森和费米发布了关于“铀中子生产”的论文。\(^\text{[82]}\)但他们的工作习惯和个性不同,费米与西拉德合作时遇到了困难。\(^\text{[84]}\)

费米是最早警告军事领导人核能潜在影响的人之一,他于1939年3月18日在海军部就这一主题发表了演讲。尽管海军同意提供1500美元用于哥伦比亚大学的进一步研究,但回应未能达到他预期的效果。\(^\text{[85]}\)同年晚些时候,西拉德、尤金·维格纳和爱德华·泰勒将爱因斯坦签署的信件发送给美国总统富兰克林·D·罗斯福,警告纳粹德国可能会制造原子弹。作为回应,罗斯福成立了铀咨询委员会,调查此事。\(^\text{[86]}\)

铀咨询委员会提供资金帮助费米购买石墨,\(^\text{[87]}\)他在普平大厅实验室的七楼建造了一个石墨砖堆。\(^\text{[88]}\)到1941年8月,他已经拥有了六吨铀氧化物和三十吨石墨,并用它们在哥伦比亚大学的谢尔梅霍恩大厅建造了一个更大的堆。\(^\text{[89]}\)

科学研究与发展办公室的S-1小组(原名铀咨询委员会)于1941年12月18日召开会议,此时美国已卷入第二次世界大战,迫使其工作变得紧急。该委员会资助的大部分工作都集中在生产浓缩铀上,但委员会成员阿瑟·康普顿确定,钚是一个可行的替代品,可以在1944年底之前通过核反应堆大量生产。\(^\text{[90]}\)他决定将钚的工作集中在芝加哥大学。费米勉强同意搬迁,他的团队成为了新成立的冶金实验室的一部分。\(^\text{[91]}\)

自持核反应的可能结果尚不清楚,因此在芝加哥市中心的芝加哥大学校园内建造第一座核反应堆似乎不太可行。康普顿找到了位于芝加哥约20英里(32公里)外的阿尔贡森林保护区的一个位置。石头与韦伯公司被委托开发该地点,但由于一场工业争议,工作被暂停。费米随后说服康普顿,表示他可以在芝加哥大学Stagg球场看台下的壁球场内建造反应堆。反应堆的建设于1942年11月6日开始,芝加哥堆1号于12月2日达到了临界状态。\(^\text{[92]}\)该堆的形状原计划为大致球形,但随着工作进展,费米计算出无需按照原计划完成整个堆就能实现临界。\(^\text{[93]}\)

这个实验是能源探索中的一个里程碑,典型地体现了费米的做事方式。每一步都经过精心规划,每一个计算都非常严谨。\(^\text{[92]}\)当第一个自持核链式反应实现时,康普顿打了一个加密的电话给国家国防研究委员会主席詹姆斯·B·康纳特。

“我接起电话,打给康纳特。当时他在哈佛大学校长办公室接电话。我说,‘吉姆,你会很感兴趣的,意大利航海家刚刚抵达新世界。’然后,我有些歉意地补充道,因为我曾告诉S-1小组,堆芯需要一周或更长时间才能完成,我加了一句,‘地球不像他估计的那么大,他比预期更快到达了新世界。’”

“是吗,”康纳特兴奋地回应,“土著人友好吗?”

“每个人都安全愉快地登陆了。”\(^\text{[94]}\)

为了继续研究而不对公众健康构成威胁,反应堆被拆解并移至阿贡森林地点。在那里,费米主持了核反应实验,并享受着反应堆大量生产自由中子的机会。\(^\text{[95]}\)实验室很快从物理学和工程学扩展到利用反应堆进行生物学和医学研究。最初,阿贡由费米作为芝加哥大学的一部分来管理,但它在1944年5月成为一个独立的实体,费米成为其主任。\(^\text{[96]}\)

当位于橡树岭的空气冷却X-10石墨反应堆在1943年11月4日实现临界时,费米在场以防万一出现问题。技术人员早早把他叫醒,以便他能够看到这一时刻。\(^\text{[97]}\)使X-10正常运行是钚项目的另一个里程碑。它提供了关于反应堆设计的数据,为杜邦公司员工提供了反应堆操作的培训,并生产了首批小量的反应堆产生的钚。\(^\text{[98]}\)费米于1944年7月成为美国公民,这是法律允许的最早日期。\(^\text{[99]}\)

1944年9月,费米将第一块铀燃料插入位于汉福德地点的B反应堆,这是一个旨在大量生产钚的生产型反应堆。与X-10一样,它是由费米的团队在冶金实验室设计的,并由杜邦公司建造,但它要大得多并且是水冷的。在接下来的几天里,共加载了838根管子,反应堆实现了临界。9月27日午夜后不久,操作员开始抽出控制棒以启动生产。最初一切似乎正常,但大约在03:00时,功率开始下降,到06:30时,反应堆完全停机。军方和杜邦公司向费米的团队寻求答案。调查了冷却水,看看是否有泄漏或污染。第二天,反应堆突然重新启动,但几个小时后再次停机。问题最终追溯到来自氙-135(Xe-135)的中子毒化,氙-135是一个有9.1至9.4小时半衰期的裂变产物。费米和约翰·惠勒都推断出,Xe-135负责吸收反应堆中的中子,从而破坏了裂变过程。费米的同事埃米利奥·塞格雷建议他请吴健雄,因为她正在准备关于这一主题的打印稿,并计划发表在《物理评论》上。\(^\text{[100]}\)阅读完草稿后,费米和科学家们确认了他们的怀疑:Xe-135确实吸收了中子,实际上它具有巨大的中子截面。\(^\text{[101][102][103]}\)杜邦公司偏离了冶金实验室的原始设计,在该设计中,反应堆有1,500根管子排列成圆形,并且在角落处增加了504根管子。科学家们原本认为这种过度设计浪费了时间和金钱,但费米意识到,如果加载所有2,004根管子,反应堆就能达到所需的功率水平,并高效地生产钚。\(^\text{[104][105]}\)
\begin{figure}[ht]
\centering
\includegraphics[width=8cm]{./figures/be41e0544bc065ac.png}
\caption{芝加哥大学团队的一部分成员,他们参与了世界上首次由人类引发的自持核反应的研究,其中前排是恩里科·费米,第二排是莱奥·西拉德。} \label{fig_ELK_11}
\end{figure}
1943年4月,费米向罗伯特·奥本海默提出了利用浓缩过程中产生的放射性副产品污染德国食品供应的可能性。背景是当时担心德国的原子弹项目已经进入了一个较为先进的阶段,费米当时也对原子弹是否能够迅速开发成功持怀疑态度。奥本海默与爱德华·泰勒讨论了这个“有前景”的提案,泰勒建议使用锶-90。詹姆斯·B·康南特和莱斯利·格罗夫斯也被简要通报了情况,但奥本海默表示,只有在足够的食品能够被这种武器污染,足以杀死50万人时,他才愿意推进这个计划。\(^\text{[106]}\)

1944年中,奥本海默说服费米加入他位于新墨西哥州洛斯阿拉莫斯的Y计划。\(^\text{[107]}\)费米于9月到达后,被任命为实验室的副主任,负责核物理和理论物理,并负责领导以他名字命名的F部门。F部门下设四个分支:F-1超级和一般理论,由泰勒领导,研究“超级” (热核)弹;F-2水锅炉,由L.D.P.金领导,负责“水锅炉”水溶性均匀研究反应堆;F-3超级实验,由埃贡·布雷特谢尔领导;F-4裂变研究,由安德森领导。\(^\text{[108]}\)费米在1945年7月16日观察了三位一体试验,并通过将纸条放入爆炸波中进行实验,以估算炸弹的威力。他根据纸条被爆炸波吹动的距离计算出爆炸威力为10千吨TNT;实际威力约为18.6千吨。\(^\text{[109]}\)

费米与奥本海默、康普顿和欧内斯特·劳伦斯一起,成为了为过渡委员会提供目标选择建议的科学小组成员。该小组同委员会一致认为,原子弹应无预警地对工业目标使用。\(^\text{[110]}\)和洛斯阿拉莫斯实验室的其他人一样,费米是通过技术区的公共广播系统得知广岛和长崎被原子弹轰炸的消息。费米不相信原子弹能够阻止国家发动战争,也不认为此时已经适合建立世界政府。因此,他没有加入洛斯阿拉莫斯科学家协会。\(^\text{[111]}\)
\subsection{战后工作}
费米于1945年7月1日成为芝加哥大学的查尔斯·H·斯威夫特杰出物理学教授,\(^\text{[112]}\)尽管他直到1945年12月31日才与家人一起离开洛斯阿拉莫斯实验室。\(^\text{[113]}\)他于1945年当选为美国国家科学院院士。\(^\text{[114]}\)冶金实验室于1946年7月1日更名为阿贡国家实验室,这是曼哈顿计划设立的第一个国家实验室。\(^\text{[115]}\)芝加哥与阿贡之间的短距离使费米能够在两个地方工作。在阿贡,他继续从事实验物理工作,与莱奥纳·马歇尔共同研究中子散射。\(^\text{[116]}\)他还与玛丽亚·迈耶讨论理论物理,帮助她发展关于自旋-轨道耦合的见解,这些见解最终帮助她获得了诺贝尔奖。\(^\text{[117]}\)

曼哈顿计划于1947年1月1日被原子能委员会(AEC)取代。\(^\text{[118]}\)费米成为了原子能委员会总顾问委员会的成员,这是一个由罗伯特·奥本海默主持的有影响力的科学委员会。\(^\text{[119]}\)他还喜欢每年花几周时间在洛斯阿拉莫斯国家实验室\(^\text{[120]}\)在那里他与尼古拉斯·梅特罗波利斯\(^\text{[121]}\)以及约翰·冯·诺依曼合作研究雷利-泰勒不稳定性,这是一种研究不同密度流体之间边界现象的科学。\(^\text{[122]}\)
\begin{figure}[ht]
\centering
\includegraphics[width=8cm]{./figures/42a942e95cdd0bdb.png}
\caption{1954年,劳拉与恩里科·费米在洛斯阿拉莫斯的核研究所。} \label{fig_ELK_12}
\end{figure}
在1949年8月苏联首次引爆裂变炸弹后,费米与伊西多尔·拉比一起为委员会撰写了一份措辞强烈的报告,反对基于道德和技术原因发展氢弹。\(^\text{[123]}\)尽管如此,费米仍继续作为顾问参与洛斯阿拉莫斯的氢弹研究工作。与斯坦尼斯瓦夫·乌拉姆一起,他计算出,不仅特勒模型所需的氚的量将是巨大的,而且即使拥有如此大量的氚,聚变反应仍然无法确保传播。\(^\text{[124]}\)费米是为奥本海默在1954年奥本海默安全听证会上作证的科学家之一,该听证会导致奥本海默的安全许可被拒绝。\(^\text{[125]}\)

在晚年,费米继续在芝加哥大学教授工作,并且是后来的恩里科·费米研究所的创始人之一。战后期间,他的博士生包括欧文·钱伯兰、杰弗里·丘、杰罗姆·弗里德曼、马文·戈尔德贝格、李政道、亚瑟·罗森费尔德和萨姆·特里曼。\(^\text{[126][76]}\)杰克·斯坦伯格是他的研究生,米尔德里德·德雷塞尔豪斯在与费米共同作为博士生的那一年受到了费米的极大影响。\(^\text{[127][128]}\)费米在粒子物理学方面进行了重要研究,特别是与介子和缪子相关的研究。他首次预测了介子-核子共振,\(^\text{[121]}\)依赖于统计方法,因为他认为,即使理论本身错误,也不需要精确的答案。\(^\text{[129]}\)在与杨振宁合著的一篇论文中,他推测介子实际上可能是复合粒子。\(^\text{[130]}\)这个观点由坂田祥一进一步阐述。后来,这一理论被夸克模型取代,在该模型中,介子由夸克组成,这也完成了费米的模型,并证明了他的研究方法是正确的。\(^\text{[131]}\)

费米写了一篇论文《论宇宙辐射的起源》,在其中他提出宇宙射线是通过物质在星际空间中的磁场加速而产生的,这导致了他与泰勒之间的意见分歧。\(^\text{[129]}\)费米研究了螺旋星系臂中磁场相关的问题。\(^\text{[132]}\)他还思考了现在被称为“费米悖论”的问题:外星生命存在的可能性和至今没有与其接触之间的矛盾。\(^\text{[133]}\)
\begin{figure}[ht]
\centering
\includegraphics[width=6cm]{./figures/c85a7ed9f16f406c.png}
\caption{费米的墓地位于芝加哥的橡树树林公墓,靠近大学。} \label{fig_ELK_13}
\end{figure}
在生命的最后阶段,费米开始质疑社会在核技术方面做出明智选择的能力。他说:

“你们中的一些人可能会问,为什么要如此辛苦地工作,仅仅是为了收集一些事实,而这些事实除了让一些长发的教授喜欢收集之外,根本不会带来任何乐趣,而且对任何人都没有用处,因为只有少数专家才能理解它们?对于这样的问题,我可以大胆地做出一个相对安全的预测。

科学和技术的历史一贯告诉我们,基础理解的科学进展迟早会导致技术和工业应用,彻底改变我们的生活方式。在我看来,物质结构的探索不可能是这一规律的例外。较不确定的是,我们都热切希望的是,人类能尽快成长为足够成熟,以便善用他所获得的自然力量。”\(^\text{[134]}\)
\subsection{去世}
费米于1954年10月在比林斯纪念医院进行了一次所谓的“探查性”手术,手术后他返回家中。五十天后,他因无法手术的胃癌在芝加哥的家中去世,享年53岁。\(^\text{[2]}\)费米曾怀疑在核堆附近工作存在极大的风险,但他仍然坚持工作,因为他认为收益远大于对个人安全的风险。他的两位研究生助理也在核堆附近工作时因癌症去世。\(^\text{[135]}\)

在芝加哥大学礼拜堂举行了纪念服务,同行们塞缪尔·K·艾利森、埃米里奥·塞格雷和赫伯特·L·安德森发表了悼词,悼念这位世界上“最杰出和最富有成效的物理学家”之一。\(^\text{[136]}\)他的遗体安葬在橡树树林公墓,家属在墓地举行了由路德宗牧师主持的私人葬礼。\(^\text{[137]}\)
\subsection{影响与遗产}
\subsubsection{遗产}
费米因其成就获得了众多奖项,包括1926年的马泰乌奇奖章、1938年的诺贝尔物理学奖、1942年的休斯奖章、1947年的富兰克林奖章以及1953年的鲁姆福德奖。他还因对曼哈顿计划的贡献于1946年获得了功勋奖章。\(^\text{[138]}\)费米于1939年当选为美国哲学学会会员,并于1950年成为英国皇家学会外籍会员。\(^\text{[139][140]}\)佛罗伦萨的圣十字大教堂,以其埋葬了许多艺术家、科学家和意大利历史上杰出人物的墓地而闻名,被称为意大利荣耀的圣殿,其中有纪念费米的牌匾。\(^\text{[141]}\)1999年,《时代》杂志将费米列入20世纪100位最具影响力人物榜单。\(^\text{[142]}\)费米被广泛认为是20世纪物理学家的一个特殊案例,他在理论和实验方面都表现出色。化学家兼小说家C·P·斯诺写道:“如果费米出生的时间早几岁,可以想象他会发现卢瑟福的原子核,然后发展出玻尔的氢原子理论。如果这听起来像是夸大其词,那么关于费米的一切可能都像夸张一样。”\(^\text{[143]}\)

费米被誉为一位鼓舞人心的教师,以其对细节的关注、简洁性和精心准备的讲座而闻名。\(^\text{[144]}\)后来,他的讲义被整理成书。\(^\text{[145]}\)他的论文和笔记本如今存放在芝加哥大学。\(^\text{[146]}\)维克托·魏斯科普夫曾指出,费米“总能找到最简单、最直接的方法,尽量避免复杂和精致的处理。”\(^\text{[147]}\)他不喜欢复杂的理论,尽管他有很强的数学能力,但在可以通过更简单的方法完成任务时,他绝不会使用复杂的数学。他因能迅速且准确地解决那些让其他人困惑的问题而著名。后来,他通过简化计算迅速得出近似答案的方法,被非正式地称为“费米法”,并且在各地广泛教授。\(^\text{[148]}\)

费米喜欢指出,当亚历山德罗·伏打在他的实验室工作时,伏打根本不知道电学研究将会带来什么结果。\(^\text{[149]}\)费米通常因其在核能和核武器方面的工作而被人们铭记,尤其是他创造了世界上第一个核反应堆,并参与了第一个原子弹和氢弹的开发。他的科学成就经得起时间的考验。这包括他对β衰变的理论、他在非线性系统方面的工作、他对慢中子的研究、他对π介子与核子的碰撞的研究以及他提出的费米-狄拉克统计学。他关于π介子不是基本粒子的猜想,为研究夸克和轻子指明了方向。\(^\text{[150]}\)

作为一个人,费米似乎极为简单朴素。他精力充沛,热爱运动和游戏。在这些场合,他雄心勃勃的性格显露无疑。他打网球时非常激烈,爬山时更像是一名向导。可以说,他是一个仁慈的独裁者。我记得有一次在山顶,费米站起来说:“好,现在是1点58分,我们2点钟出发。”当然,大家都信守承诺,准时起身。正是这种领导力和自信使得费米被称为“教皇”,他的物理学宣言在大家眼中是无误的。他曾说过:“我可以在几张纸上计算出物理中的任何问题,误差范围在2倍以内;而得到公式前面的数值因子可能需要一个物理学家花上一年时间计算,但我对这个不感兴趣。”他的领导力强大到足以影响与他合作的人的独立性。我记得有一次,在他家举办的聚会上,我妻子切面包时,费米走过来说他有一种不同的切面包的哲学,拿过我妻子手中的刀,开始自己切,因为他坚信自己的方法更好。但这些都不会让人反感,反而让大家更喜欢费米。他对物理之外的兴趣非常少,记得有一次我在泰勒的钢琴上弹奏时,费米坦言,他对音乐的兴趣仅限于简单的旋律。

——埃贡·布雷彻\(^\text{[140]}\)
\subsubsection{以费米命名的事物}
\begin{figure}[ht]
\centering
\includegraphics[width=8cm]{./figures/275255082548f997.png}
\caption{} \label{fig_ELK_14}
\end{figure}
许多事物都以费米的名字命名。其中包括位于伊利诺伊州巴塔维亚的费米国家加速器实验室,该粒子加速器和物理实验室于1974年以他的名字命名以示纪念。[151] 此外,还有2008年为表彰他在宇宙射线方面的研究成果而命名的费米伽马射线空间望远镜。[152]共有三座核反应堆设施以他命名:位于密歇根州纽波特的费米1号和费米2号核电站,意大利特里诺-韦尔切莱塞的恩里科·费米核电站,[153] 以及阿根廷的RA-1恩里科·费米研究堆。[154]在1952年“常春藤迈克”核试验的残留物中分离出的一种人工合成元素被命名为“锎”(Fermium),以纪念费米对科学界的杰出贡献。[155][156] 这使他成为全世界仅有的16位以元素命名的科学家之一。[157]
\begin{figure}[ht]
\centering
\includegraphics[width=6cm]{./figures/544eb3c52e0f1994.png}
\caption{意大利佛罗伦萨圣十字大教堂内的纪念碑牌} \label{fig_ELK_15}
\end{figure}
自1956年起,美国原子能委员会,以及自1977年起的美国能源部,将其最高荣誉命名为“费米奖”以纪念他。[158] 该奖项的获奖者包括奥托·哈恩、罗伯特·奥本海默、爱德华·泰勒和汉斯·贝特。
\subsection{著作出版物}
\begin{itemize}
\item Introduzione alla Fisica Atomica(《原子物理学导论》,意大利语),博洛尼亚:N. Zanichelli出版社,1928年,OCLC 9653646。
\item Fisica per i Licei(《高中物理》,意大利语),博洛尼亚:N. Zanichelli出版社,1929年,OCLC 9653646。
\item Molecole e cristalli(《分子与晶体》,意大利语),博洛尼亚:N. Zanichelli出版社,1934年,OCLC 19918218。
\item Thermodynamics(《热力学》),纽约:Prentice Hall出版社,1937年,OCLC 2379038。
\item Fisica per Istituti Tecnici(《技术学院物理》,意大利语),博洛尼亚:N. Zanichelli出版社,1938年。
\item Fisica per Licei Scientifici(《理科高中物理》,意大利语,与爱德华多·阿马尔迪合著),博洛尼亚:N. Zanichelli出版社,1938年。
\item Elementary Particles(《基本粒子》),纽黑文:耶鲁大学出版社,1951年,OCLC 362513。
\item Notes on Quantum Mechanics(《量子力学笔记》),芝加哥:芝加哥大学出版社,1961年,OCLC 1448078。
\end{itemize}
\subsection{专利}
\begin{itemize}
\item 美国专利 US 2206634:《放射性物质的生产方法》,颁发日期:1940年7月1日
\item 美国专利 US 2836554:《风冷中子反应堆》,颁发日期:1950年4月1日
\item 美国专利 US 2524379:《中子速度选择器》,颁发日期:1950年10月1日
\item 美国专利 US 2852461:《中子反应堆》,颁发日期:1953年9月1日
\item 美国专利 US 2708656:《中子反应堆》,颁发日期:1955年5月1日
\item 美国专利 US 2768134:《在中子反应堆中测试材料的方法》,颁发日期:1956年10月1日
\item 美国专利 US 2780595:《指数试验堆》,颁发日期:1957年2月1日
\item 美国专利 US 2798847:《中子反应堆的运行方法》,颁发日期:1957年7月1日
\item 美国专利 US 2807581:《中子反应堆》,颁发日期:1957年9月1日
\item 美国专利 US 2807727:《中子反应堆屏蔽结构》,颁发日期:1957年9月1日
\item 美国专利 US 2813070:《维持中子链式反应系统的方法》,颁发日期:1957年11月1日
\item 美国专利 US 2837477:《链式反应系统》,颁发日期:1958年6月1日
\item 美国专利 US 2931762:《中子反应堆》,颁发日期:1960年4月1日
\item 美国专利 US 2969307:《检测热中子裂变材料纯度的方法》,颁发日期:1961年1月1日
\end{itemize}
\subsection{参考文献}
\begin{enumerate}
\item “核时代的建筑师恩里科·费米去世”,1954年秋,原始内容已于2015年11月17日存档,检索时间:2015年11月2日。
\item “恩里科·费米逝世,终年53岁;原子弹的设计师”,《纽约时报》,1954年11月29日,原始内容已于2019年3月14日存档,检索时间:2013年1月21日。
\item “祖谱门户网站”,意大利语,检索时间:2023年5月23日。
\item 塞格雷,1970年,第3–4页,第8页。
\item 阿马尔迪,2001年,第23页。
\item 库珀,1999年,第19页。
\item 劳拉·费米(,《家庭中的原子:我与恩里科·费米的生活》,芝加哥大学出版社,2014年10月24日,第52页,ISBN 9780226149653。
\item 塞格雷,1970年,第5–6页。
\item 费米(,1954年,第15–16页。
\item “玛丽亚·费米·萨凯蒂(1899–1959)”,[www.OlgiateOlona26giugno1959.org(意大利语),原始内容已于2017年8月30日存档,检索时间:2017年5月6日。](http://www.OlgiateOlona26giugno1959.org(意大利语),原始内容已于2017年8月30日存档,检索时间:2017年5月6日。)
\item 塞格雷(Segrè),1970年,第7页。
\item 博诺利斯(Bonolis),2001年,第315页。
\item 阿马尔迪(Amaldi),2001年,第24页。
\item 塞格雷(Segrè),1970年,第11–12页。
\item 塞格雷(Segrè),1970年,第8–10页。
\item 塞格雷(Segrè),1970年,第11–13页。
\item 费米(Fermi),1954年,第20–21页。
\item “意大利国家数学文集——朱利奥·皮塔雷利”(意大利语),比萨高等师范学校,原始内容存档于2017年12月17日,检索于2017年5月6日。
\item 塞格雷(Segrè),1970年,第15–18页。
\item 博诺利斯(Bonolis),2001年,第320页。
\item 博诺利斯(Bonolis),2001年,第317–319页。
\item 塞格雷(Segrè),1970年,第20页。
\item “关于电磁质量的电动力学与相对论理论之间的矛盾”\[*Über einen Widerspruch zwischen der elektrodynamischen und relativistischen Theorie der elektromagnetischen Masse*],《物理学杂志》(*Physikalische Zeitschrift*,德文),第23卷,第340–344页,原始内容已于2021年2月3日存档,检索时间:2013年1月17日。
\item 贝尔托蒂(Bertotti),2001年,第115页。
\end{enumerate}
