% 中国科学院 2018 年考研普通物理
% 中国科学院|考研|普通物理

\subsection{选择题}

\begin{enumerate}

\item 某时刻的弦波如图所示,在此弦段中,振动动能最大的部位为\\

A. A处$\quad$
B. B处$\quad$
C. C处$\quad$
C. A和C处$\quad$

\item 以下关于质点系描述正确的是\\

A. 质点系质心运动只与只与质点系所受合内力有关,质点系的内力不可能改变质点系的总动能.\\
B. 质点系质点运动只与质点系所受合外力有关,质点系的内力可以改变质点系的总动能.\\
C. 质点系质点运动只与质点所受合内力有关,质点系的内力可以改变质点系的总动能.\\
D. 质点系质点运动只与质点系所受合外力有关,质点系的内力不可以改变质点系的总动能.\\

\item 

\item 

\item 

\item 关于磁化电流与传导电流,下面说法不正确的是\\

A. 磁化电流是大量分子电流统计平均的宏观效果,传导电流是电荷迁移的结果\\
B. 磁化电流和传导电流都能产生磁场\\
C. 磁化电流和传导电流都能产生焦耳热\\
C. 磁化电流产生的磁场服从安培环路定理\\

\item 一电路如下图所示,两电阻大小均为 $R$ ,电感 $L$ ,电源电动势 $\epsilon$ 开关 $S$ 闭合后电感 $L$ 上的电流 $i$ 随时间的变化关系为\\

A. $i = \frac{\epsilon}{2R}$\\
B. $i = \frac{\epsilon}{R}e^{-\frac{2R}{L}t}$\\
C. $i = \frac{\epsilon}{2R}e^{-\frac{R}{L}t}$\\
C. $i = \frac{\epsilon}{2R}(1-e^{-\frac{R}{L}t})$\\

\item 原子态 $^{1}D_{1}$ 的能级在磁感应强度 $\vec{B}$ 的弱磁场中分裂成多少子能级?\\

A. 3个 $\quad$
B. 2个 $\quad$
C. 5个 $\quad$
C. 4个 $\quad$
\end{enumerate}

\subsection{简答题}

\begin{enumerate}

\item 所谓的二体问题,是指两个指点只有相互作用力,不受外力,对于二体问题,试推导一个质点相对于另一个质点的运动学方程.\\
对比 $\vec{F} = \mu \vec{a}$ 形式,请写出 $\mu$ 的表达式( $\mu$ 代表二体约化质量)

\item 圆盘上有一圈带正电的带点球,问中间的螺线管断电时,圆盘是否会转动,如何转,为什么.

\item 牛顿环,凸透镜和平面介质的折射率分别为 $n_1$ $n_3$ ,中间为空气,此时牛顿环中心为暗纹.若在中间注入一种液体,折射率在凸透镜和平面介质中间.

\end{enumerate}