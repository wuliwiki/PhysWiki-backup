% 莫尔斯引理

\begin{issues}
\issueTODO
\issueDraft
\end{issues}

\pentry{隐函数定理的不动点证明 \upref{IFTFix}}

莫尔斯引理在微分几何学中占有基本地位。 粗略地说, 它表示: 一个光滑函数在其非退化临界点附近的表现, 完全由其在这一点处的二阶导数决定。

\subsection{表述与直观}
\begin{lemma}{莫尔斯引理}
设$n$元光滑函数$f$定义在坐标原点附近. 如果 Hessian 矩阵
$$
f''(0)=\left(\frac{\partial^2f}{\partial x^i\partial x^j}(0)\right)
$$
是可逆的, 则在坐标原点的某个小邻域上, 存在微分同胚$\varphi(x)$, 使得
$$
(f\circ\varphi)(x)=f(0)+\frac{1}{2}\langle f''(0)x,x\rangle.
$$
\end{lemma}
