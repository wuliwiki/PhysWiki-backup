% 辗转相除法
% 互素|最大公因式|算法|辗转相除

\pentry{多项式的整除\upref{ExDiv}}
本节将证明最大公因式的存在性,并介绍最大公因式的算法程序——辗转相除法.简单来说,所谓的\textbf{辗转相除法}是指多次利用带余除法计算两个多项式的最大公因式的算法.

为证明最大公因式的存在,我们先来证明下面一个定理.
\begin{theorem}{}
设 $f(x),g(x),q(x),r(x)\in\mathbb{F}[x]$,并且
\begin{equation}\label{SucDiv_eq1}
f(x)=q(x)g(x)+r(x)
\end{equation}
则 $f(x),g(x)$ 与 $g(x),r(x)$ 有相同的公因式.
\end{theorem}
\textbf{证明:}1.$\phi(x)|g(x),\phi(x)|f(x)\Rightarrow \phi(x)|g(x),\phi(x)|r(x)$ 的证明:

因为$\phi(x)|g(x),\phi(x)|f(x)$ ,由整除的性质3多项式的整除\upref{ExDiv} ,$\phi(x)$ 必整除 $f(x),g(x)$ 的组合
\begin{equation}
\phi(x)|f(x)-q(x)g(x)\Rightarrow \phi(x)|r(x)
\end{equation}

2.$\phi(x)|g(x),\phi(x)|r(x)\Rightarrow\phi(x)|g(x),\phi(x)|f(x)$ 的证明:同样有整除的性质3,立即证得.

下面利用\autoref{SucDiv_eq1} 来证明最大公因式的存在性,该证明方法便是辗转相除法.
\begin{theorem}{最大公因式存在定理}
设 $f(x),g(x)$ 是 $\mathbb{F}[x]$ 中的两个多项式,则在 $\mathbb{F}[x]$ 中, $f(x)$ 与 $g(x)$ 的最大公因式 $d(x)$ 存在,且有 $\mathbb{F}[x]$ 中的两个多项式 $u(x),v(x)$ 使得
\begin{equation}
d(x)=u(x)f(x)+v(x)g(x)
\end{equation}

\end{theorem}