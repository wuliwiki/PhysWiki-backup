% Floyd 判圈算法
% keys Floyd|判环|判圈
% license Usr
% type Tutor

\subsection{Floyd 判圈算法}
首先需要注意的一点,Floyd 判圈算法\textbf{不是} Floyd 全源最短路算法。Floyd 判圈算法是用来判断链表中是否有圈(或环)的存在的。

Floyd 判圈算法(又称为 Floyd 判环算法或龟兔赛跑算法)的思想类似于快慢指针。原理是“龟兔赛跑”,慢指针每次向前移动 $1$ 步、快指针每次向前移动两步。如果两者在遍历链表的\textbf{过程中}相遇,则说明链表存在一个圈;如果快指针达到了链表的结尾(有尾则一定无环),说明链表无环。

下面给出一个常见的实现方法:
\begin{lstlisting}[language=cpp]
bool hasCyc(ListNod* head) {
    if(head == nullptr)  return false;
    ListNod *slow = head, *fast = head;
    fast = fast -> next;

    while(fast != nullptr && fast -> next != nullptr) {
        if(slow == fast)  return true;
        slow = slow -> next;
        fast = fast -> next -> next;
    }
    
    return false;
}
\end{lstlisting}

\subsection{Floyd 判圈算法的拓展}
\subsubsection{求环的长度}
在找到环后可以很容易地求出环的长度,固定 \verb`fast` 快指针不动,然后让 \verb`slow` 慢指针一直移动直到再与 \verb`fast` 快指针相遇,这移动的距离恰好就是环的长度。
\subsubsection{求环的起点}
假设链表是类似于如下形式:
\begin{figure}[ht]
\centering
\includegraphics[width=6cm]{./figures/d33edec540251cb5.png}
\caption{链表示意图} \label{fig_FloydC_2}
\end{figure}
不失一般性地,考虑这个环的顺时针走的。

设环长 $L$(可以通过刚才的方法求出),非环部分(也就是头节点到环)长 $a$,环的起点到快慢指针第一次相遇点的距离为 $b$。设快指针走了 $f$ 圈、慢指针走了 $s$ 圈。