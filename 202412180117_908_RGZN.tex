% 人工智能史(综述)
% license CCBYSA3
% type Wiki

本文根据 CC-BY-SA 协议转载翻译自维基百科\href{https://en.wikipedia.org/wiki/History_of_artificial_intelligence}{相关文章}。

人工智能(AI)的历史可以追溯到古代,那个时候有关于由工匠们赋予智慧或意识的人工生命体的神话、故事和传闻。从古代到现代,逻辑学和形式推理的研究直接促成了1940年代可编程数字计算机的发明,这是一种基于抽象数学推理的机器。这个设备及其背后的理念启发了科学家们开始讨论构建电子大脑的可能性。

人工智能研究领域是在1956年于达特茅斯学院举行的一次研讨会上创立的。[1] 参加该研讨会的人成为了人工智能研究的领导者,并且在几十年里引领着这一领域的发展。许多人预测,在一代人之内,像人类一样智能的机器将会问世。美国政府也提供了数百万美元,希望能够将这一愿景变为现实。[2]

最终,研究人员明显低估了这一壮举的难度。[3] 1974年,詹姆斯·莱特希尔的批评以及美国国会的压力导致美国和英国政府停止资助无目标的人工智能研究。七年后,日本政府的远见性倡议和专家系统的成功重新激发了对人工智能的投资,到了1980年代末,人工智能产业已经成长为一个价值十亿美元的行业。然而,到了1990年代,投资者的热情减退,人工智能在媒体中受到批评,行业也开始回避这一领域(这一时期被称为“人工智能寒冬”)。尽管如此,研究和资金在其他名称下依然持续增长。

进入2000年代,机器学习被应用于学术和工业中的广泛问题。这一成功归功于强大计算机硬件的可用性、大规模数据集的收集以及扎实的数学方法的应用。很快,深度学习证明是一项突破性的技术,超越了所有其他方法。2017年,变换器架构的首次亮相带来了令人印象深刻的生成型人工智能应用,及其他多个应用场景。到2020年代,人工智能的投资呈现爆发式增长。
\subsection{前驱}
\subsubsection{神话、小说和推测性的前驱}
\textbf{神话与传说}

在希腊神话中,塔罗斯(Talos)是一个由青铜铸成的巨人,担任克里特岛的守护者。他会向入侵者的船只投掷大石块,并每天绕岛的周围完成三次巡逻。[4] 根据伪阿波罗多罗斯的《博物志》(Bibliotheke),赫淮斯托斯(Hephaestus)在一位独眼巨人的帮助下锻造了塔罗斯,并将这个自动装置作为礼物献给米诺斯(Minos)。[5] 在《阿尔戈英雄传》(Argonautica)中,杰森(Jason)和阿尔戈英雄们通过拔出塔罗斯脚旁的塞子,导致其体内的生命之液流出,从而使塔罗斯丧命。[6]

皮格马利翁(Pygmalion)是希腊神话中的一位传奇国王与雕刻家,著名的故事出自奥维德的《变形记》。在奥维德的叙事诗《变形记》第十卷中,皮格马利翁因目睹普罗波埃提德斯(Propoetides)自愿卖淫的行为而对女性感到厌恶。尽管如此,他还是在维纳斯(Venus)的神庙中献上祭品,请求女神赐予他一位像他雕刻的雕像一样的女子。[7]

\textbf{中世纪关于人工生命体的传说}
\begin{figure}[ht]
\centering
\includegraphics[width=6cm]{./figures/7e497a845b48fea1.png}
\caption{歌德《浮士德》中的人造小人描绘} \label{fig_RGZN_1}
\end{figure}
在《事物的本质》中,瑞士炼金术士帕拉塞尔苏斯(Paracelsus)描述了一种他声称能制造“人造人”的方法。他将“一名男子的精液”放入马粪中,并在40天后喂入“人血的秘方”,这种混合物将变成一个活生生的婴儿。[8]

关于制造“哥雷姆”(Golem)的最早书面记载出现在13世纪初沃尔姆斯的以利以撒·本·犹大(Eleazar ben Judah)的著作中。[9] 在中世纪,人们认为,通过将写有上帝名字的纸条放入泥人(哥雷姆)口中,可以使其复活。[10] 与像青铜头这样的传奇自动装置不同,哥雷姆是无法说话的。[11][12]

在伊斯兰教的炼金术手稿中,\textbf{塔克温}(Takwin,即人工生命的创造)是一个常见的主题,尤其是在那些归属于贾比尔·伊本·海扬(Jabir ibn Hayyan)的作品中。伊斯兰炼金术士尝试通过炼金术创造各种生命形式,从植物到动物不等。[13]

在约翰·沃尔夫冈·冯·歌德(Johann Wolfgang von Goethe)的《浮士德:悲剧的第二部分》中,炼金术制造的“人造小人”(homunculus)注定要永远生活在他被制造出来的瓶子里,但他努力想要变成一个完整的人类身体。然而,在这一转变开始时,瓶子破裂,人造小人也随之死去。[14]

\textbf{现代小说}

到19世纪,关于人工人和思考机器的理念成为小说中的一个流行主题。像玛丽·雪莱的《弗兰肯斯坦》和卡雷尔·恰佩克的《R.U.R.》(罗索姆的万能机器人)[15]等著名作品探讨了人工生命的概念。像塞缪尔·巴特勒的《机器中的达尔文》[16]和爱德加·爱伦·坡的《梅尔策尔的棋手》[17]等推测性文章反映了社会对具有人工智能的机器日益增长的兴趣。人工智能至今仍是科幻小说中的常见主题。[18]

\textbf{自动装置}
\begin{figure}[ht]
\centering
\includegraphics[width=8cm]{./figures/dcdee03784e1a89f.png}
\caption{阿尔·贾扎里的可编程自动装置(公元1206年)} \label{fig_RGZN_2}
\end{figure}
许多文明的工匠建造了现实中的类人自动装置,包括燕师[19]、亚历山大英雄[20]、阿尔·贾扎里[21]、哈鲁恩·拉希德[22]、雅克·德·沃康松[23][24]、莱昂纳多·托雷斯·伊·凯韦多[25]、皮埃尔·贾凯-德罗兹和沃尔夫冈·冯·肯普伦[26][27]。

已知最古老的自动装置是古埃及和古希腊的神像[28][29]。信徒们相信,工匠们赋予这些雕像非常真实的心智,能够表现出智慧和情感——赫尔墨斯·特里斯梅吉斯图斯曾写道:“通过发现神的真正本质,人类已能够复制它”[30]。英国学者亚历山大·内克哈姆(Alexander Neckham)主张,古罗马诗人维吉尔曾建造了一座拥有自动雕像的宫殿[31]。

在早期现代时期,这些传奇的自动装置据说具有回答提问的神奇能力。晚期中世纪的炼金术士和原始新教徒罗杰·培根据说曾制造过一颗青铜头,并发展出自己是巫师的传说[32][33]。这些传说与北欧神话中的米米尔之头相似。传说中,米米尔以智力和智慧著称,在 Æsir-Vanir 战争中被斩首。奥丁被认为“用草药”保存了米米尔的头,并对其念咒语,使米米尔的头仍能向奥丁传授智慧。奥丁随后将头放在身边,作为咨询之用[34]。
\subsubsection{形式推理}
人工智能基于这样一种假设:人类的思维过程可以被机械化。机械或“形式”推理的研究有着悠久的历史。中国、印度和希腊的哲学家们在公元前一千年左右就已经发展出了结构化的形式推理方法。这些思想经过几百年的发展,得到了像亚里士多德(他对三段论进行了形式分析)、欧几里得(他的《几何原本》是形式推理的典范)、阿尔·花拉子米(他发展了代数,并将自己的名字赋予了“算法”一词)以及威廉·奥卡姆和邓斯·司各图等欧洲经院哲学家的深化和扩展[35][36]。

西班牙哲学家拉蒙·柳尔(1232–1315)发展了几种逻辑机器,致力于通过逻辑手段生产知识[37][38];柳尔将他的机器描述为机械实体,能够通过简单的逻辑操作将基本且不容否认的真理组合在一起,机器通过机械方式产生这些操作,从而生成所有可能的知识[39]。柳尔的工作对戈特弗里德·莱布尼茨产生了巨大影响,后者重新发展了他的思想[40]。
\begin{figure}[ht]
\centering
\includegraphics[width=6cm]{./figures/d3b8a79f2fbc4764.png}
\caption{戈特弗里德·莱布尼茨,他推测人类的理性可以归结为机械计算。} \label{fig_RGZN_3}
\end{figure}
17世纪,莱布尼茨、托马斯·霍布斯和勒内·笛卡尔探索了将所有理性思维系统化的可能性,使其如代数或几何一样具有体系性[41]。霍布斯在《利维坦》中著名地写道:“理性……无非是计算,就是加法和减法”[42]。莱布尼茨设想了一种普遍的推理语言——通用符号系统(characteristica universalis),这种语言将论证简化为计算,令“两个哲学家之间不再需要争论,正如两个会计师之间也不必争论一样。因为只需拿起他们的铅笔和黑板,就可以彼此对话(如果愿意,可以有朋友作为见证者):让我们计算吧”[43]。这些哲学家开始阐述物理符号系统假设,这一假设最终成为人工智能研究的指导信条。

数学逻辑的研究提供了使人工智能看似可行的关键突破。布尔的《思想的法则》和弗雷格的《概念文字》为此奠定了基础[44]。基于弗雷格的系统,罗素和怀特海德在1913年出版的《数学原理》中对数学基础进行了形式化处理。受罗素成功的启发,戴维·希尔伯特在1920年代和1930年代挑战数学家们回答一个根本性的问题:“所有的数学推理能否形式化?”[36]这个问题最终得到了哥德尔不完备定理[45]、图灵机[45]和丘奇的λ演算[注]的回答。
\begin{figure}[ht]
\centering
\includegraphics[width=8cm]{./figures/f837aba8957e9db3.png}
\caption{美国陆军拍摄的ENIAC照片,拍摄地点为摩尔电气工程学院[47]} \label{fig_RGZN_4}
\end{figure}
他们的答案在两个方面令人惊讶。首先,他们证明了数学逻辑实际上是有局限的。其次(对人工智能更为重要的是),他们的工作表明,在这些局限内,任何形式的数学推理都可以被机械化。丘奇-图灵论题意味着,一台机械设备,只需按简单的符号(如0和1)进行排列,就可以模仿任何可以想象的数学推理过程。关键的洞察是图灵机——一个简单的理论构造,它捕捉了抽象符号操作的本质。[48] 这一发明激发了一些科学家开始讨论思维机器的可能性。
\subsubsection{计算机科学}  
计算机器在古代和历史上由许多人设计或制造,包括戈特弗里德·莱布尼茨、约瑟夫·玛丽·雅卡尔、查尔斯·巴贝奇、珀西·卢德盖特、莱昂纳多·托雷斯·凯维多、范尼瓦·布什等。艾达·洛夫莱斯曾推测巴贝奇的机器是“一台思考或...推理机器”,但她警告说:“有必要防止对机器的能力产生夸大的想法。”  

第一台现代计算机是第二次世界大战期间的大型机器(如康拉德·楚泽的Z3、艾伦·图灵的希思·罗宾逊和巨人机、阿塔纳索夫与贝里的ABC以及宾夕法尼亚大学的ENIAC)。ENIAC基于艾伦·图灵奠定的理论基础,并由约翰·冯·诺依曼发展而成,证明它是最具影响力的计算机。[57]
\subsection{人工智能的诞生(1941-1956)}
\begin{figure}[ht]
\centering
\includegraphics[width=8cm]{./figures/dc69a2c824689d03.png}
\caption{IBM 702:第一代人工智能研究人员使用的计算机。} \label{fig_RGZN_5}
\end{figure}
早期的思维机器研究受到了20世纪30年代末、40年代和50年代初流行的思想汇聚的启发。神经学的最新研究表明,大脑是一个由神经元构成的电气网络,神经元以“全或无”的脉冲方式发射。诺伯特·维纳的控制论描述了电气网络中的控制与稳定性。克劳德·香农的信息理论描述了数字信号(即“全或无”的信号)。艾伦·图灵的计算理论表明,任何形式的计算都可以通过数字化的方式进行描述。这些思想之间的紧密关系暗示,构建一个“电子大脑”可能是可行的。

在40年代和50年代,来自各个领域(数学、心理学、工程学、经济学和政治学)的少数科学家探索了几个对后来的人工智能研究至关重要的研究方向。艾伦·图灵是最早认真研究“机器智能”理论可能性的人之一。“人工智能研究”作为一门学科在1956年成立。[59][60][61]
\subsubsection{图灵测试} 
\begin{figure}[ht]
\centering
\includegraphics[width=8cm]{./figures/3f5360e4685a515c.png}
\caption{图灵测试[62]} \label{fig_RGZN_6}
\end{figure}
1950年,图灵发表了具有里程碑意义的论文《计算机机械与智能》,在其中他推测了创造能够思考的机器的可能性。在论文中,他指出,“思考”是一个难以定义的概念,并提出了著名的图灵测试:如果一台机器能够进行一场(通过电传机进行的)对话,且这场对话与与人类的对话无法区分,那么就可以合理地说这台机器是在“思考”。这种简化版的问题让图灵能够有力地论证“思考机器”至少是可行的,这篇论文回答了所有对这一命题的常见反对意见。图灵测试是人工智能哲学中第一个严肃的提案。
\subsubsection{人工神经网络}  
沃尔特·皮茨(Walter Pitts)和沃伦·麦卡洛克(Warren McCulloch)于1943年分析了理想化的人工神经元网络,并展示了它们如何执行简单的逻辑功能。他们是首个描述后来被称为神经网络的学者[66]。该论文受到了图灵1936年《可计算数的论述》一文的影响,采用了类似的两状态布尔“神经元”,但首次将其应用于神经功能[60]。受皮茨和麦卡洛克启发的学生之一是马文·敏斯基(Marvin Minsky),当时他是一个24岁的研究生。1951年,敏斯基和迪恩·埃德蒙兹(Dean Edmonds)建立了第一个神经网络机器——SNARC[67]。敏斯基后来成为人工智能领域最重要的领导者和创新者之一。
\subsubsection{控制论机器人}  
20世纪50年代,W·格雷·沃尔特(W. Grey Walter)的海龟机器人和约翰霍普金斯大学的野兽机器人等实验性机器人相继问世。这些机器人没有使用计算机、数字电子学或符号推理,而是完全由模拟电路控制[68]。
\subsubsection{游戏人工智能}  
1951年,克里斯托弗·斯特雷奇(Christopher Strachey)利用曼彻斯特大学的费兰提Mark 1计算机编写了一个跳棋程序[69],而迪特里希·普林茨(Dietrich Prinz)则为国际象棋编写了一个程序[70]。阿瑟·塞缪尔(Arthur Samuel)的跳棋程序是他1959年论文《机器学习的若干研究:以跳棋为例》中的研究成果,该程序最终达到了足以挑战一位相当水平的业余玩家的水平[71]。塞缪尔的程序是后来被称为机器学习的早期应用之一[72]。游戏人工智能将继续作为人工智能发展的衡量标准,贯穿其历史。
\subsubsection{符号推理与《逻辑理论家》}
\begin{figure}[ht]
\centering
\includegraphics[width=6cm]{./figures/7245eef62d056505.png}
\caption{符号推理与《逻辑理论家》} \label{fig_RGZN_7}
\end{figure}
当50年代中期开始可以访问数字计算机时,一些科学家本能地认识到,一台能够操控数字的机器同样也能操控符号,而符号的操控可能正是人类思维的本质。这是一种创造思维机器的新方法。[73][74]

1955年,艾伦·纽厄尔(Allen Newell)和未来的诺贝尔奖得主赫伯特·A·西蒙(Herbert A. Simon)在J·C·肖(J. C. Shaw)的帮助下创造了“逻辑理论家”(Logic Theorist)。该程序最终证明了拉塞尔和怀特海德《数学原理》(Principia Mathematica)中的前52个定理中的38个,并为其中一些定理找到了新的、更优雅的证明。[75] 西蒙表示,他们“解决了久远的心灵/身体问题,解释了一个由物质组成的系统如何具备心灵的特性。”[76][c] 他们所提出的符号推理范式将主导人工智能的研究和资金支持,直到90年代中期,并且启发了认知革命。
\subsubsection{达特茅斯研讨会}  
1956年的达特茅斯研讨会是一个关键事件,标志着人工智能作为一门学科的正式诞生。它由马文·明斯基和约翰·麦卡锡组织,并得到了IBM的两位资深科学家克劳德·香农和内森·罗切斯特的支持。会议提案中指出,他们旨在验证这一断言:“学习的每一个方面或智能的任何其他特征都可以被如此精确地描述,以至于可以制造一台机器来模拟它。”  

“人工智能”这一术语由约翰·麦卡锡在研讨会上提出。与会者包括雷·所罗门诺夫、奥利弗·塞尔弗里奇、特伦查德·摩尔、阿瑟·塞缪尔、艾伦·纽厄尔和赫伯特·A·西蒙,他们都将在人工智能研究的初期几十年里创建重要的程序。在研讨会上,纽厄尔和西蒙首次展示了《逻辑理论家》程序。  

这次研讨会是人工智能获得名称、使命、首个重大成功和关键人物的时刻,被广泛认为是人工智能的诞生。
\subsubsection{认知革命}  
1956年秋天,纽厄尔和西蒙在麻省理工学院(MIT)信息理论特别兴趣小组会议上展示了《逻辑理论家》。在同一会议上,诺姆·乔姆斯基讨论了他的生成语法,乔治·米勒描述了他的开创性论文《神奇的数字七,加或减二》。米勒写道:“我带着一种比理性更为直观的信念离开了研讨会,那就是实验心理学、理论语言学和认知过程的计算机模拟都是一个更大整体中的一部分。”  

这次会议标志着“认知革命”的开始——一个跨学科的范式转变,涉及心理学、哲学、计算机科学和神经科学。它启发了符号人工智能、生成语言学、认知科学、认知心理学、认知神经科学以及计算主义和功能主义哲学学派的创建。所有这些领域都使用相关工具来建模心智,并且在一个领域中发现的结果对其他领域也具有相关性。  

认知方法使研究人员能够考虑“心理对象”,如思想、计划、目标、事实或记忆,通常使用高级符号在功能网络中进行分析。这些对象在早期的行为主义等范式中被视为“不可观察的”,因此不被允许作为研究对象。[h] 符号心理对象将成为接下来几十年人工智能研究和资金投入的主要焦点。
\subsection{早期的成功(1956-1974)}  
在达特茅斯研讨会之后开发的程序,对于大多数人来说,简直是“令人震惊的”[i]:计算机开始解决代数应用题,证明几何定理,并学习说英语。当时几乎没有人相信机器能够表现出如此“智能”的行为。[90][91][89] 研究人员在私下和公开场合表达了强烈的乐观情绪,预测不到20年内将建成完全智能的机器。[92] 像国防高级研究计划局(DARPA,时称“ARPA”)这样的政府机构向该领域注入了大量资金。[93] 到了1950年代末和1960年代初,多个英国和美国的大学设立了人工智能实验室。[60]  
\subsubsection{方法}  
在50年代末和60年代,出现了许多成功的程序和新的研究方向。其中最具影响力的有:

\textbf{推理、规划和问题解决作为搜索}  

许多早期的人工智能程序使用了相同的基本算法。为了实现某个目标(如赢得游戏或证明定理),它们一步步朝着目标前进(通过进行一次移动或推理),就像在迷宫中搜索一样,每当到达死胡同时便回溯。[94] 主要的困难在于,对于许多问题,"迷宫"中可能的路径数量是天文数字(这种情况被称为“组合爆炸”)。研究人员通过使用启发式方法来减少搜索空间,排除那些不太可能通向解决方案的路径。[95]

纽厄尔和西蒙试图在一个名为“通用问题求解器”的程序中捕捉这种算法的一般版本。[96][97] 其他“搜索”程序也能够完成令人印象深刻的任务,比如解决几何和代数问题,例如赫伯特·格尔恩特的几何定理证明器(1958年)[98] 和由敏斯基的学生詹姆斯·斯莱格尔在1961年编写的符号自动积分器(SAINT)[99][100]。其他程序则通过搜索目标和子目标来规划行动,例如斯坦福大学开发的STRIPS系统,用于控制机器人Shakey的行为。[101]