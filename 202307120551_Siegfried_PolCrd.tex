% 极坐标中的曲线方程
% 极坐标|曲线方程|切线|求导

\begin{issues}
\issueDraft
\end{issues}

\pentry{极坐标\upref{Polar}}

极坐标系中, 我们可以用一元函数
\begin{equation}\label{eq_PolCrd_1}
r = f(\theta)~
\end{equation}
表示一条曲线。 简单的例子如阿基米德螺线\upref{ArcSpl} 或圆锥曲线\upref{Cone}。 以下我们讨论如何从函数中计算曲线的一些特征。

\subsection{计算某点切线的方向}
\pentry{导数\upref{Der}}

% 图未完成

在 $x$-$y$ 平面直角坐标系中, 我们可以通过求导($\dv*{y}{x}$)计算曲线某点切线, 那么\autoref{eq_PolCrd_1} 表示的极坐标中如何求某点切线的方向呢? 可以证明, 切线与 $\uvec\theta$ 方向的夹角为
\begin{equation}
\alpha = \frac{1}{r} \dv{r}{\theta} = \frac{f'(\theta)}{f(\theta)}~,
\end{equation}
或者说与 $\uvec r$ 方向的夹角为 $\pi/2 - \alpha$。

\begin{figure}[ht]
\centering
\includegraphics[width=8cm]{./figures/b714fe71ebcd2028.png}
\caption{极坐标切线示例} \label{fig_PolCrd_1}
\end{figure}
证明:
\begin{align}
\text{对极坐标中的位置矢量}\qquad\bvec r &= r \uvec r,\qquad\text{两边取全微分,得到:}\\
\dd{\bvec r} &= \dd r \uvec r + r\dd{\uvec r},\\
\text{由于:}\qquad\dd{\uvec r} &= \pdv {\uvec r}{r} \dd r + \pdv {\uvec r}{\theta} \dd\theta~,
\end{align}
其中,$\pdv {\uvec r}{r}$ 和 $\pdv {\uvec r}{\theta}$ 可由极坐标系中单位矢量的偏导\upref{DPol1}处证得,结果带入上式,得到:
\begin{equation}\label{eq_PolCrd_6}
\dd{\bvec r} = \dd r \uvec r+ r \dd \theta\uvec{\theta}~.
\end{equation}
上式中,$\dd{\bvec r}$的方向,即是切线的方向。切线与 $\uvec\theta$ 方向的夹角为:
\begin{align}
\tan \alpha &= \pdv{\bvec r}{\uvec r}\left/ \pdv{\bvec r}{\uvec\theta}\right.
=\frac{1}{r} \dv{r}{\theta} = \frac{f'(\theta)}{f(\theta)}~.
\end{align}

\subsection{曲线长度}
\pentry{定积分\upref{DefInt}}

若用 $\theta \in [\theta_1, \theta_2]$, 来表示曲线的一段, 那么其长度为
\begin{equation}
l = \int_{\theta_1}^{\theta_2} \sqrt{f(\theta)^2 + f'(\theta)^2} \dd{\theta}~.
\end{equation}

\addTODO{推导}
证明:继\autoref{eq_PolCrd_6}:
\begin{align}
\dd{l} = \abs{\dd{\bvec r}} &=\sqrt{(\dd r)^2 + (r \dd \theta)^2} \\
&=\sqrt{(\dv {r}{\theta} )^2 + r^2} \dd \theta \\
&=\sqrt{f(\theta)^2 + f'(\theta)^2} \dd{\theta}~.
\end{align}

\begin{align}
\text{对极坐标中的位置矢量}\qquad\bvec r &= r \uvec r,\qquad\text{两边取全微分,得到:}\\
\dd{\bvec r} &= \dd r \uvec r + r\dd{\uvec r},\\
\text{由于:}\qquad\dd{\uvec r} &= \pdv {\uvec r}{r} \dd r + \pdv {\uvec r}{\theta} \dd\theta~,
\end{align}