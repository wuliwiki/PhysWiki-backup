% 大气密度和压强
% keys 大气|积分方程|理想气体|气压|摩尔质量

\pentry{一阶线性微分方程\upref{ODE1}, 理想气体分压定律\upref{PartiP}, 积分方程} % 未完成

\begin{figure}[ht]
\centering
\includegraphics[width=10cm]{./figures/067b9f12027dc45f.png}
\caption{根据\autoref{eq_atmDen_4}  以及分压定律得到的大气压强随高度变化图, 假设大气中没有水蒸气且温度恒定(来自维基百科)}\label{fig_atmDen_1}
\end{figure}
现实中观测到大气温度随高度的增加而会发生变化,因此不能将大气简单地视为处于平衡态的热力学系统,这其中涉及到非平衡态的热力学机制。下面我们将构建两个较好的理论模型:等温大气模型和干绝热大气模型。湿绝热大气模型能够更好地解释大气温度随高度变化的一些现象(例如一座高山的顺风面和背风面可能存在温差),这在气象学中是很重要的一个理论:
\href{https://t-inp.readthedocs.io/zh_CN/latest/热力学基础.html}{湿绝热方程}。

\subsection{等温大气模型}
\footnote{参考 Wikipedia \href{https://en.wikipedia.org/wiki/Atmospheric_pressure}{相关页面}以及\href{https://en.wikipedia.org/wiki/International_Standard_Atmosphere}{另一个页面}。}以下介绍一个理想模型。 假设大气是理想气体, 密度随高度变化为 $\rho(z)$。 所以高度 $z$ 处压强为
\begin{equation}\label{eq_atmDen_1}
P(z) = \int_{z}^\infty \rho(z') g \dd{z'}
\end{equation}
其中由于大气厚度远小于地球半径, 我们取 $g$ 为常数。 根据理想气体状态方程\upref{PVnRT},
\begin{equation}
PV = n R T
\end{equation}
先假设大气只是由一种分子构成, 摩尔质量为 $\mu$, 即 $m = n\mu$, 代入有
\begin{equation}\label{eq_atmDen_3}
P = \frac{m}{\mu V} RT = \frac{R}{\mu} \rho T~.
\end{equation}
其中 $P, T, \rho$ 都是高度的函数。 代入\autoref{eq_atmDen_1} 得关于 $\rho(z)$ 的积分方程 % 链接未完成
\begin{equation}
\frac{R}{\mu} \rho(z) T(z) = \int_{z}^\infty \rho(z') g \dd{z'}
\end{equation}
通常来说海拔越高的地方气温越低, 如果 $T(z)$ 是已知的, 就可以解出 $\rho(z)$。 方程两边对 $z$ 求导, 整理得
\begin{equation}\label{eq_atmDen_5}
\rho'(z)  +  \frac{1}{T(z)}\qty[T'(z) + \frac{\mu g}{R}]\rho(z) = 0
\end{equation}
这是一个一阶线性微分方程, 可以直接用公式求解\upref{ODE1}。 把解出的 $\rho(z)$ 代入\autoref{eq_atmDen_3} 即可求出对应的压强 $P(z)$。

作为一种简单情况, 假设温度不随高度变化(实际上,空气的热导率很小,考虑成绝热过程能得到更加精确的结果\autoref{eq_Adiab_6}~\upref{Adiab}), 那么方程变为常系数的
\begin{equation}
\rho'(z)  +  \frac{\mu g}{RT}\rho(z) = 0
\end{equation}
容易解得
\begin{equation}\label{eq_atmDen_2}
\rho(z) = \rho_0\exp(-\frac{\mu g}{RT} z)
\end{equation}
或者
\begin{equation}\label{eq_atmDen_4}
P(z) = P_0\exp(-\frac{\mu g}{RT} z)
\end{equation}
其中 $\rho_0, P_0$ 是某个高度 $z_0$ 处的大气密度和气压。 这说明恒温条件下气压随海拔升高呈指数下降, 且温度越低下降越快。

当大气中有多种气体时, 可以对每种气体分别求解, 把 $P_0$ 替换为改气体在 $z_0$ 处的分压\upref{PartiP}。 总密度就是每种气体的密度之和。 大气中的水蒸气同样也可能随着高度变化。

\subsection{干绝热大气模型}

假设大气是理想气体,其热导率很小,所以大气的对流过程可以近似考虑成绝热过程(实验表明随着高度的增加大气温度下降,这说明不宜用等温大气模型),即
\begin{equation}
\begin{aligned}
&\begin{cases}
&PV_m^\gamma=C\\
&V_m=\frac{RT}{P},P=\frac{\rho R T}{\mu}
\end{cases}
\\
&\Rightarrow \rho^{1-\gamma}T=C'\\
&\Rightarrow (1-\gamma)T\dd \rho+\rho\dd T=0  
\end{aligned}
\end{equation}

那么\autoref{eq_atmDen_5} 变为
\begin{equation}
\frac{\gamma}{\gamma-1}T'(z)=-\frac{\mu g}{R}
\end{equation}
积分得\footnote{
并且可以验证,当 $\gamma$ 趋近于 $1$ 时,下面的方程就变为等温模型的大气压强公式。}
\begin{equation}
\begin{aligned}
T&=T_0-\int_{z_0}\frac{\gamma-1}{\gamma}\frac{\mu g}{R} \dd z\\&\approx T_0\left[1-\frac{\gamma-1}{\gamma}\frac{\mu g}{R T_0}z\right]\\ 
P&=P_0\left[1-\frac{\gamma-1}{\gamma}\frac{\mu g}{R T_0}z\right]^{\gamma/(\gamma-1)}
\end{aligned}
\end{equation}
温度 $T$ 总是 $>0$ 的,且大气的绝热指数 $\gamma>1$,这意味着温度随高度的增加而降低。可以利用气体热容量公式\autoref{eq_ThCapa_4}~\upref{ThCapa}约去 $\gamma$,积分得:
\begin{equation}
T=T_0-\int_{z_0}\frac{\mu g}{c_{p,m}} \dd z 
\end{equation}

$\mu,c_{p,m}$ 可近似看成常数。大气的摩尔质量为 $29 \rm{g\cdot mol^{-1}}$,摩尔定压热容约为 $29 \rm{J\cdot mol^{-1}K^{-1}}$,因此计算得
\begin{equation}
\begin{aligned}
&T=T_0-\frac{\mu g}{c_{p,m}} z\\
&T\approx T_0-z\cdot 10 \rm{K/km}
\end{aligned}
\end{equation}
即每升高一千米,温度降低约 $10$ 摄氏度。该数值称为干绝热递减率。