% 高斯消元法求逆矩阵

\pentry{高斯消元法解线性方程组\upref{GAUSS}, 逆矩阵\upref{InvMat}}

对可逆矩阵 $\mat M$ 做行变换相当于左乘一个矩阵 $\mat R$。 假设某种行变换能使 $\mat M$ 变为单位矩阵, 即
\begin{equation}
\mat R \mat M = \mat I~.
\end{equation}
那么根据定义 $\mat R$ 就是 $\mat M$ 的逆矩阵即 $\mat R = \mat M^{-1}$。 利用这个性质, 我们可以同时对 $\mat M$ 和 $\mat I$ 做相同的行变换, 当 $\mat M$ 变为单位矩阵后, $\mat I$ 就变为 $\mat M^{-1}$:
\begin{equation}
\mat R\mat M = \mat I
\qquad
\mat R \mat I = \mat M^{-1}
\end{equation}
注意 $\mat M$ 可逆当且仅当它是一个满秩矩阵, 即每行都线性无关的方阵。

\begin{example}{求逆矩阵}\label{ex_InvMGs_1}
求矩阵 $\mat M = \pmat{1 & 4 \\ 2 & 9}$ 的逆矩阵。

解: 先并列写出 $\mat M$ 和单位矩阵 $\mat I$, 以下所有行变换都对两个矩阵同时进行
\begin{equation}
\pmat{1 & 4 \\ 2 & 9} \qquad \pmat{1 & 0\\ 0 & 1}
\end{equation}
第一行乘 $-2$ 加到第二行得
\begin{equation}
\pmat{1 & 4 \\ 0 & 1} \qquad \pmat{1 & 0\\ -2 & 1}
\end{equation}
第二行乘 $-4$ 加到第一行得
\begin{equation}
\pmat{1 & 0 \\ 0 & 1} \qquad \pmat{9 & -4\\ -2 & 1}
\end{equation}
所以 $\mat M^{-1} = \pmat{9 & -4\\ -2 & 1}$。
\end{example}

同理, 我们也可以把上文中所有行变换改为列变换, 列变换相当于把矩阵 $\mat M$ 右乘一个矩阵 $\mat C$:
\begin{equation}
\mat M\mat C = \mat I
\qquad
\mat I\mat C = \mat M^{-1}
\end{equation}
\begin{exercise}{}
用列变换的方法求\autoref{ex_InvMGs_1} 中的逆矩阵。
\end{exercise}
