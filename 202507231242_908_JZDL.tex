% 介值定理(综述)
% license CCBYSA3
% type Wiki

本文根据 CC-BY-SA 协议转载翻译自维基百科\href{https://en.wikipedia.org/wiki/Intermediate_value_theorem}{相关文章}。

\begin{figure}[ht]
\centering
\includegraphics[width=8cm]{./figures/17b13a0f3803ea4f.png}
\caption{} \label{fig_JZDL_1}
\end{figure}
在数学分析中,中值定理指出:如果函数 $f$ 是一个连续函数,且其定义域包含区间 $[a, b]$,那么对于任意介于 $f(a)$ 与 $f(b)$ 之间的值,函数 $f$ 在该区间内至少有一个点取到这个值。

该定理有两个重要的推论:
\begin{enumerate}
\item 如果一个连续函数在某个区间内的两个端点处函数值符号相反,那么它在这个区间内至少有一个零点(即有解)——这被称为波尔查诺定理[1][2]。
\item 一个连续函数在一个区间上的值域本身也是一个区间。
\end{enumerate}
\subsection{动机}
\begin{figure}[ht]
\centering
\includegraphics[width=6cm]{./figures/277ed962e4079ea4.png}
\caption{介值定理} \label{fig_JZDL_2}
\end{figure}
这揭示了实数上连续函数的一个直观性质:设 $f$ 是定义在区间 $[1, 2]$ 上的连续函数,且已知 $f(1) = 3$、$f(2) = 5$,那么函数图像 $y = f(x)$ 在 $x$ 从 1 变到 2 的过程中,必定会穿过水平直线 $y = 4$。这表达了一个概念:在闭区间上的连续函数的图像可以不抬笔地画出来。
\subsection{定理}
介值定理表述如下:

设实数集合 $\mathbb{R}$ 上的闭区间 $I = [a, b]$,函数 $f: I \to \mathbb{R}$ 连续,则有:

\begin{itemize}
\item 版本一:如果某个数 $u$ 介于 $f(a)$ 和 $f(b)$ 之间,即
$$
\min(f(a), f(b)) < u < \max(f(a), f(b)),~
$$
那么存在某个点 $c \in (a, b)$,使得$f(c) = u$.
\item 版本二:函数值的集合 $f(I)$ 也是一个闭区间,且包含区间
$$
[\min(f(a), f(b)), \max(f(a), f(b))].~
$$
\end{itemize}
备注:版本二说明函数值的集合中不存在间断。对于任意两个函数值 $c, d \in f(I)$,若 $c < d$,则所有处于区间 $[c, d]$ 中的数也都是函数值,即
$$
[c, d] \subseteq f(I).~
$$
在实数集合中,没有内部间隙的子集就是一个区间。版本一是版本二的一个特殊情况。
\subsection{与完备性的关系}
介值定理依赖于实数的完备性,并且与之等价。该定理不适用于有理数集 $\mathbb{Q}$,因为有理数之间存在“空隙”;这些空隙正是由无理数填补的。

例如,函数$f(x) = x^2$在有理数域 $x \in \mathbb{Q}$ 上满足$f(0) = 0, \quad f(2) = 4$但不存在一个有理数 $x$,使得$f(x) = 2$因为 $\sqrt{2}$ 是一个无理数。

尽管如此,在实闭域上,仍然存在一种适用于多项式的介值定理形式,可参见魏尔斯特拉斯零点定理。
\subsection{证明}
\subsubsection{证明版本 A}
该定理可以作为实数完备性的一个推论来证明,如下所示:[3]

我们将证明第一种情况:$f(a) < u < f(b)$第二种情况的证明是类似的。

设
$$
S = \{x \in [a, b] \mid f(x) < u\}~
$$
由于 $f(a) < u$,所以 $a \in S$,因此 $S$ 非空。又因为 $S \subseteq [a, b]$,所以它有上界 $b$。根据实数的完备性,集合 $S$ 的上确界$c = \sup S$
存在。也就是说,$c$ 是大于或等于 $S$ 中所有元素的最小实数。

注意,由于 $f$ 在 $a$ 处连续,我们可以通过让 $x$ 足够接近 $a$,使得 $f(x)$ 与 $f(a)$ 的差小于任意指定的 $\varepsilon > 0$。既然 $f(a) < u$ 是一个严格不等式,我们可以取
$$
\varepsilon = u - f(a) > 0~
$$
根据连续性的定义,存在 $\delta > 0$,使得对所有 $x \in [a, b]$,只要
$$
|x - a| < \delta~
$$
就有
$$
|f(x) - f(a)| < u - f(a) \Rightarrow f(x) < u~
$$
考虑区间
$$
I_1 = [a, \min(a + \delta, b))~
$$
注意到 $I_1 \subseteq [a, b]$,并且对所有 $x \in I_1$,都有 $|x - a| < \delta$,因此 $f(x) < u$。这说明在 $a$ 附近存在比 $a$ 更大的属于 $S$ 的点,所以 $c \ne a$。

(
