% 电极化强度
% keys 电解质|极化强度|电偶极矩|极化电荷

\pentry{电介质\upref{Dielec}}

(未完成: 从宏观上, 可以直接把极化强度看作正负电荷密度 $\rho$ 乘以位移, 而不需要有分子极化的微观概念.)

在电介质\upref{Dielec}一节中,我们从分子的电结构出发,说明了两类电介质极化的微观过程虽然不同,但宏观的效果却是相同的,都是在电介质的两个相对表面上出现了异号的极化电荷,在电介质内部有沿电场方向的电偶极矩.因此下面从宏观上描述电介质的极化现象时,就不分两类电介质来讨论了.

在电介质内任取一物理无限小的体积元$\Delta V$(但其中仍有大量的分子),当没有外电场时,这体积元中所有分子的电偶极矩的矢量和$\sum \mathbf p$等于零.但是,在外电场的影响下,由于电介质的极化,$\sum \mathbf p$将不等于零.外电场愈强,被极化的程度愈大,$\sum \mathbf p$的值也愈大.因此我们取单位体积内分子电偶极矩的矢量和,即
\begin{equation}
\mathbf P=\frac{\sum \mathbf p}{\Delta V} 
\end{equation}
作为量度电介质极化程度的基本物理量,称为该点($\Delta V$所包围的一点)的电极化强度(electric polarization)或$\mathbf P$矢量.在国际单位制中,电极化强度的单位是$\rm C/m^2$.

