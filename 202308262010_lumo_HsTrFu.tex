% 三角函数(高中)
% keys 高中|三角函数
% license Xiao
% type Tutor
\pentry{角的概念(高中)\upref{HsAngl}}
\begin{issues}
\issueDraft
\end{issues}

\subsection{定义}
我们取单位圆上一点 $P(u,v)$,令 $OP$ 与 $x$ 轴夹角为 $\alpha$,则 
\begin{equation}
\begin{aligned}
\cos\alpha &= u,\\
\sin\alpha &= v,\\
\tan\alpha &= \frac{v}{u}~.
\end{aligned}
\end{equation}
\begin{figure}[ht]
\centering
\includegraphics[width=10cm]{./figures/82d1d7f3babb1847.png}
\caption{如图所示} \label{fig_HsTrFu_3}
\end{figure}
易得,正弦函数和余弦函数的\textbf{定义域为全体实数},正切函数的定义域为 $\begin{Bmatrix}\alpha|\alpha \neq \frac{\pi}{2}+k\pi,k\in Z\end{Bmatrix}~.$

\subsection{性质}
我们把随自变量的变化呈周期性变化的函数叫作\textbf{周期函数},正周期中最小的一个称为\textbf{最小正周期}。

对于函数 $f(x)$ 为,如果存在非零实数 $T$,对定义域内的任意一个 $x$ 值,都有
\begin{equation}
f(x+T) = f(x)~.
\end{equation}
我们把 $f(x)$ 称为周期函数,$T$ 称为这个函数的周期。

正弦函数、余弦函数、正切函数的都是周期函数,根据定义易得,正弦函数和余弦函数,周期为 $2k\pi(k\in Z,k\neq0)$,正切函数的周期为 $k\pi(k\in Z,k\neq0)$.

\subsection{图像}
我们可以通过计算机绘制出函数图像
\begin{figure}[ht]
\centering
\includegraphics[width=14.25cm]{./figures/bb986656153a547d.png}
\caption{正弦函数} \label{fig_HsTrFu_1}
\end{figure}
\begin{figure}[ht]
\centering
\includegraphics[width=14.25cm]{./figures/d0ae1e0d9c167f1e.png}
\caption{余弦函数} \label{fig_HsTrFu_2}
\end{figure}
可以看出正弦函数和余弦函数是定义域为 $R$ 值域为 $[-1,1]$ 最小正周期 $T = 2\pi$ 的周期函数。

\subsection{同角三角函数的基本关系}
三角函数除以上介绍的三种,还有三种:
\begin{enumerate}
\item 正割函数 \textbf{\textsl{sec α}} 
\begin{equation}
\sec \alpha = \frac{r}{u}~.
\end{equation}
(这里的 “r” 指上文OP的距离)
\item 余割函数 \textbf{\textsl{csc α}} 
\begin{equation}
\csc \alpha = \frac{r}{v}~.
\end{equation}
\item 余切函数 \textbf{\textsl{cot α}} 
\begin{equation}
\cot \alpha = \frac{u}{v}~.
\end{equation}
\end{enumerate}
了解三种函数后,同角三角函数的基本关系符合以下图示:
\begin{figure}[ht]
\centering
\includegraphics[width=12cm]{./figures/6390d1e662067a9b.png}
\caption{同角三角函数的基本关系} \label{fig_HsTrFu_4}
\end{figure}
\begin{itemize}
\item 三角函数倒数关系:
\begin{equation}
\tan \alpha  \cot \alpha = 1~,
\end{equation}
\begin{equation}
\sin \alpha  \csc \alpha = 1~,
\end{equation}
\begin{equation}
\sec \alpha  \cos \alpha = 1~.
\end{equation}
聪明的你可以发现这个关系是六边形的对角线。
\item 三角函数商数关系:
\begin{equation}
\tan \alpha = \frac{\sin \alpha}{\cos \alpha}~,
\end{equation}
\begin{equation}
\cot \alpha = \frac{\cos \alpha}{\sin \alpha}~.
\end{equation}
剩下的不做列举
聪明的你可以发现这个关系是:顶点函数值等于相邻两顶点乘积
\item 三角函数平方关系:
\begin{equation}
\sin ^{2} \alpha + \cos ^{2}\alpha =1~,
\end{equation}
\begin{equation}
\tan  ^{2} \alpha + 1 =\sec ^{2}\alpha~,
\end{equation}
\begin{equation}
1 + \cot ^{2}\alpha =\csc ^{2}\alpha~.
\end{equation}
聪明的你可以发现这个关系按照图示颜色三角形进行。
\end{itemize}
