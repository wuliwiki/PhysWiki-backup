% Berkeley-ECS 方法

\footnote{参考 Renate Pazourek Thesis Eq 3.36}要把最后的波函数 $\ket{\Psi(t_0)}$ 投影到精确散射态 $\ket{\varphi_{\alpha,E}}$ 上, 而避免计算精确散射态, 可以先解
\begin{equation}\label{BerECS_eq1}
(E-H)\ket{\Psi_{SC}(E)} = \ket{\Psi(t_0)}
\end{equation}
根据\autoref{LipSch_eq2}~\upref{LipSch}, 形式上这相当于计算
\begin{equation}
\ket{\Psi_{SC}(E)} = G^+(E)\ket{\Psi(t_0)}
\end{equation}
但这并不重要, 数值上仍然直接解非齐次线性方程组\autoref{BerECS_eq1} 即可.

下一步, 把末态投影到精确散射态上得
\begin{equation}
\braket{\varphi_{\alpha,E}}{\Psi(t_0)} = \mel{\varphi_{\alpha,E}}{E-H}{\Psi_{SC}(E)}
= \int \frac{\laplacian}{2}\Psi_{SC}(E) + (E-V)\dd{V}
\end{equation}

