% 戈特洛布·弗雷格(综述)
% license CCBYSA3
% type Wiki

本文根据 CC-BY-SA 协议转载翻译自维基百科\href{https://en.wikipedia.org/wiki/Gottlob_Frege}{相关文章}。

\begin{figure}[ht]
\centering
\includegraphics[width=6cm]{./figures/49ab91a025d7e7b7.png}
\caption{弗雷格,约1879年} \label{fig_Gottlo_1}
\end{figure}
弗里德里希·路德维希·戈特洛布·弗雷格[7] (Friedrich Ludwig Gottlob Frege,1848年11月8日-1925年7月26日)是德国哲学家、逻辑学家和数学家。他曾担任耶拿大学的数学教授,被许多人视为分析哲学的奠基人,专注于语言哲学、逻辑学和数学哲学。尽管他在生前几乎未受到关注,但朱塞佩·皮亚诺(Giuseppe Peano,1858–1932)、伯特兰·罗素(Bertrand Russell,1872–1970)以及在某种程度上路德维希·维特根斯坦(Ludwig Wittgenstein,1889–1951)将他的工作介绍给了后来的哲学家。弗雷格被广泛认为是自亚里士多德以来最伟大的逻辑学家,也是有史以来最深刻的数学哲学家之一。[8] 

他的贡献包括在《概念文字》(Begriffsschrift)中发展了现代逻辑,以及在数学基础方面的工作。他的著作《算术基础》是逻辑主义项目的开创性文本,迈克尔·杜梅特(Michael Dummett)将其视为语言学转向的标志。弗雷格的哲学论文《论意义与指称》和《思想》也被广泛引用。前者论证了两种不同的意义类型和描述主义。在《算术基础》和《思想》中,弗雷格分别在关于数字和命题的问题上主张与心理主义或形式主义对立的柏拉图主义。
\subsection{生活}  
\subsubsection{童年(1848–1869)}  
弗雷格于1848年出生在维斯马,梅克伦堡-什未林(今天属于梅克伦堡-前波美拉尼亚)。他的父亲卡尔(卡尔)·亚历山大·弗雷格(1809–1866)是女子中学的共同创始人和校长,直至去世。卡尔去世后,学校由弗雷格的母亲奥古斯特·威尔赫尔米娜·索菲·弗雷格(原姓比亚洛布洛茨基,1815年1月12日–1898年10月14日)领导;她的母亲是奥古斯特·阿玛利亚·玛丽亚·巴尔霍恩,菲利普·梅兰希通的后裔;她的父亲是约翰·海因里希·西格弗里德·比亚洛布洛茨基,来自一个17世纪离开波兰的波兰贵族家庭。[9][10] 弗雷格是路德宗信徒。[11]

在童年时期,弗雷格接触到了一些将指导他未来科学事业的哲学思想。例如,他的父亲编写了一本面向9至13岁儿童的德语教材,名为《Hülfsbuch zum Unterrichte in der deutschen Sprache für Kinder von 9 bis 13 Jahren》(第二版,维斯马,1850年;第三版,维斯马和路德维希斯卢斯特:辛斯托夫出版社,1862年),该书的第一部分讨论了语言的结构和逻辑。

弗雷格在维斯马的大城市学校(Große Stadtschule Wismar)学习,并于1869年毕业。[12] 数学和自然科学教师古斯塔夫·阿道夫·利奥·萨克斯(Gustav Adolf Leo Sachse,1843–1909),他同时也是一位诗人,在决定弗雷格未来的科学事业方面发挥了重要作用,鼓励他继续在自己的母校耶拿大学深造。[13]
\subsubsection{大学学习(1869–1874)}  
弗雷格于1869年春季以北德意志联邦公民身份入学耶拿大学。在四个学期的学习中,他参加了大约二十门讲座课程,其中大多数是关于数学和物理学的。他最重要的老师是恩斯特·卡尔·阿贝(Ernst Karl Abbe,1840–1905;物理学家、数学家和发明家)。阿贝讲授了引力理论、电流学和电动力学、复分析的复变函数理论、物理应用、机械学的各个分支以及固体力学。阿贝不仅是弗雷格的老师,还是他值得信赖的朋友,并且作为卡尔·蔡司光学制造公司(Carl Zeiss AG)的总监,他有能力推动弗雷格的职业生涯。弗雷格毕业后,他们开始了更紧密的通信。[citation needed]

他其他值得注意的大学老师包括克里斯蒂安·菲利普·卡尔·斯内尔(Christian Philipp Karl Snell,1806–1886;授课科目:几何学中的微积分分析应用、平面解析几何、解析力学、光学、力学的物理基础);赫尔曼·卡尔·尤利乌斯·特劳戈特·谢费尔(Hermann Karl Julius Traugott Schaeffer,1824–1900;授课科目:解析几何、应用物理学、代数分析、电报及其他电子机器);以及哲学家库诺·费舍尔(Kuno Fischer,1824–1907;康德主义和批判哲学)。[citation needed]

从1871年开始,弗雷格在哥廷根继续他的学业,哥廷根是德语地区数学领域的领先大学,他参加了鲁道夫·弗里德里希·阿尔弗雷德·克莱布施(Rudolf Friedrich Alfred Clebsch,1833–1872;解析几何)、恩斯特·克里斯蒂安·尤利乌斯·舍林(Ernst Christian Julius Schering,1824–1897;函数理论)、威廉·爱德华·韦伯(Wilhelm Eduard Weber,1804–1891;物理学研究、应用物理)、爱德华·里克(Eduard Riecke,1845–1915;电学理论)和赫尔曼·洛策(Hermann Lotze,1817–1881;宗教哲学)的讲座。成熟弗雷格的许多哲学理论与洛策相似;是否弗雷格的观点直接受到洛策讲座的影响,一直是学术讨论的主题。[citation needed]

1873年,弗雷格在恩斯特·克里斯蒂安·尤利乌斯·舍林的指导下获得了博士学位,论文题目为《Über eine geometrische Darstellung der imaginären Gebilde in der Ebene》(《关于平面中虚构形态的几何表示》),他在其中旨在解决几何学中的一些基本问题,如射影几何中无限远(虚构)点的数学解释。[citation needed]

弗雷格于1887年3月14日娶了玛格丽特·卡塔琳娜·索菲亚·安娜·丽泽贝格(Margarete Katharina Sophia Anna Lieseberg,1856年2月15日–1904年6月25日)。[12] 夫妻二人至少有两个孩子,不幸的是,他们都在幼年时去世。多年后,他们收养了一个儿子,阿尔弗雷德。然而,关于弗雷格的家庭生活,知之甚少。[14]
\subsection{作为逻辑学家的工作}  
尽管他的教育背景和早期的数学工作主要集中在几何学上,弗雷格的研究很快转向了逻辑学。他的《Begriffsschrift, eine der arithmetischen nachgebildete Formelsprache des reinen Denkens》(《概念文字:一种模仿算术的纯思维形式语言》),哈雷a/S:路易斯·内贝特出版社,1879年,标志着逻辑学历史的一个转折点。《概念文字》开辟了新的领域,其中包括对函数和变量概念的严格处理。弗雷格的目标是证明数学来源于逻辑,为此,他设计了与亚里士多德的三段论相区分的方法,而这些方法又使他与斯多亚学派的命题逻辑有着相当紧密的联系。[15]
\begin{figure}[ht]
\centering
\includegraphics[width=6cm]{./figures/10c46d21c77003b9.png}
\caption{《概念文字》(1879) 封面页} \label{fig_Gottlo_2}
\end{figure}
事实上,弗雷格发明了公理化谓词逻辑,这在很大程度上归功于他发明了量化变量,这些变量最终在数学和逻辑中变得无处不在,并解决了多重普遍性的问题。以前的逻辑处理的是逻辑常量,如“和”、“或”、“如果...那么...”、“非”和“某些与所有”,但这些运算的迭代,特别是“某些”和“所有”,理解得不多:即使是像“每个男孩都爱某个女孩”和“某个女孩被每个男孩爱”这样的句子之间的区别,也只能以非常人为的方式表示,而弗雷格的形式主义在表达“每个男孩都爱某个女孩,那个女孩爱某个男孩,那个男孩又爱某个女孩”以及类似句子的不同理解时,没有任何困难,这与他对“每个男孩都是傻瓜”的处理完全平行。

一个经常被提到的例子是,亚里士多德的逻辑无法表示像欧几里得定理这样的数学陈述,这是一条关于数论的基本命题,即质数的数量是无限的。然而,弗雷格的“概念符号”能够表示这样的推理。[16] 逻辑概念的分析以及形式化机制,这对于《数学原理》(《Principia Mathematica》,三卷本,1910–1913年,由伯特兰·罗素(Bertrand Russell,1872–1970)和阿尔弗雷德·诺斯·怀特黑德(Alfred North Whitehead,1861–1947)编写)、罗素的描述理论、库尔特·哥德尔(Kurt Gödel,1906–1978)的不完备性定理,以及阿尔弗雷德·塔尔斯基(Alfred Tarski,1901–1983)的真理理论至关重要,这一切最终归功于弗雷格。

弗雷格所声明的一个主要目的,是要孤立出真正的逻辑推理原则,以便在数学证明的恰当表示中,任何时候都不依赖于“直觉”。如果存在直觉元素,它应该被孤立出来,并单独作为公理表示:从那时起,证明应当完全是逻辑性的,没有任何漏洞。在展示了这一可能性后,弗雷格的更大目的就是捍卫算术是逻辑的一个分支的观点,这一观点被称为逻辑主义:与几何学不同,算术应当被证明没有“直觉”基础,也不需要非逻辑的公理。早在1879年的《概念文字》中,弗雷格就在他所理解的纯粹逻辑框架内,推导出了重要的初步定理,例如三分法则的一般化形式。

这个观点在弗雷格的《算术基础》(Die Grundlagen der Arithmetik,1884)一书中以非符号化的形式表述。后来,在他的《算术基本法则》(Grundgesetze der Arithmetik,第一卷,1893年;第二卷,1903年;第二卷由他自费出版)中,弗雷格试图通过使用他的符号体系,从他所主张的逻辑公理中推导出所有算术法则。这些公理大多数从他的《概念文字》(Begriffsschrift)中延续而来,尽管有一些显著的变化。唯一真正的新原则是他称之为“基本法则V”的原理:当且仅当 ∀x[f(x) = g(x)] 时,函数 f(x) 的“值域”与函数 g(x) 的“值域”相同。

这个法则的关键情况可以用现代符号表示如下。设 {x|Fx} 表示谓词 Fx 的扩展,也就是所有满足 Fx 的元素的集合,Gx 同理。那么基本法则\(\lor\)表示,如果且仅如果 ∀x[Fx ↔ Gx],则谓词 Fx 和 Gx 具有相同的扩展。即,Fs 的集合与 Gs 的集合相同,当且仅当每一个 F 都是 G,每一个 G 都是 F。(这种情况是特殊的,因为这里所称的谓词的扩展,或者集合,仅是函数的“值域”的一种类型。)

在一个著名的事件中,伯特兰·罗素在1903年《算术基本法则》第二卷即将付印时写信给弗雷格,指出罗素悖论可以从弗雷格的基本法则\(\lor\)中推导出来。在弗雷格的体系中,集合或扩展的成员关系是很容易定义的;然后罗素提到“所有 x 使得 x 不是 x 的成员的集合”。《算术基本法则》的体系意味着,这样定义的集合既是又不是它自己的成员,从而导致了自相矛盾。弗雷格匆忙在第二卷中写了一个附录,推导出这一矛盾,并提议通过修改基本法则\(\lor\)来消除它。弗雷格在附录开头写下了异常诚实的评论:“没有什么比在工作完成后,科学作者的基本理论被动摇更不幸的事情了。这就是我在伯特兰·罗素先生的来信中所面临的情况,当时这卷书的印刷工作已接近完成。”(这封信和弗雷格的回复被收录在Jean van Heijenoort 1967年出版的版本中。)

在一个著名的事件中,伯特兰·罗素在1903年《算术基本法则》第二卷即将付印时写信给弗雷格,指出罗素悖论可以从弗雷格的基本法则\(\lor\)中推导出来。在弗雷格的体系中,集合或扩展的成员关系是很容易定义的;然后罗素提到“所有 x 使得 x 不是 x 的成员的集合”。《算术基本法则》的体系意味着,这样定义的集合既是又不是它自己的成员,从而导致了自相矛盾。弗雷格匆忙在第二卷中写了一个附录,推导出这一矛盾,并提议通过修改基本法则V来消除它。弗雷格在附录开头写下了异常诚实的评论:“没有什么比在工作完成后,科学作者的基本理论被动摇更不幸的事情了。这就是我在伯特兰·罗素先生的来信中所面临的情况,当时这卷书的印刷工作已接近完成。”(这封信和弗雷格的回复被收录在Jean van Heijenoort 1967年出版的版本中。)

弗雷格提议的补救措施后来被证明意味着在其话语宇宙中只有一个对象,因此是无价值的(事实上,如果弗雷格将“真”和“假”是两个不同对象的这个思想——这一思想是他讨论的基础——做为公理化,那么这将导致他系统中的矛盾;参见例如Dummett 1973),但最近的研究表明,《算术基本法则》的大部分程序可以通过其他方式得以挽回:
\begin{itemize}
\item 基本法则V可以以其他方式进行弱化。最著名的一种方式是由哲学家和数学逻辑学家乔治·布洛斯(George Boolos,1940–1996)提出的,他是弗雷格研究的专家。一个“概念”F是“微小的”,如果落在F下的对象无法与话语宇宙中的对象建立一对一的对应关系,即,除非:∃R[R是1对1的且∀x∃y(xRy & Fy)]。现在将V弱化为V*:如果且仅如果F和G都不是微小的,或者∀x(Fx ↔ Gx),则“概念”F和“概念”G具有相同的“扩展”。如果第二阶算术是一致的,那么V*是一致的,并且足以证明第二阶算术的公理。
\item 基本法则V可以简单地用休谟原则(Hume's principle)替代,休谟原则说,如果且仅如果F的集合可以与G的集合建立一对一的对应关系,那么F的数量与G的数量相同。这个原则,如果第二阶算术是一致的,也是一致的,并且足以证明第二阶算术的公理。这个结果被称为弗雷格定理,因为人们注意到,在发展算术的过程中,弗雷格对基本法则V的使用仅限于休谟原则的证明;而从这一原则中,算术原则被推导出来。关于休谟原则和弗雷格定理,见“弗雷格的逻辑、定理与算术基础”[17]。
\item 弗雷格的逻辑,现在被称为第二阶逻辑,可以弱化为所谓的“可预测的第二阶逻辑”。可预测的第二阶逻辑加上基本法则V,通过有限方法或构造方法可以证明一致性,但它只能解释算术的非常弱的片段。[18]
\end{itemize}
弗雷格在逻辑方面的工作直到1903年才引起国际关注,当时罗素在《数学原理》中写了一篇附录,阐述了他与弗雷格的不同观点。弗雷格使用的图示符号法没有前例(并且此后也没有模仿者)。此外,在1910年至1913年罗素和怀特海德的《数学原理》三卷本问世之前,数学逻辑的主导方法仍然是乔治·布尔(George Boole,1815–1864)及其知识继承人的方法,特别是恩斯特·施罗德(Ernst Schröder,1841–1902)。尽管如此,弗雷格的逻辑思想通过他的学生鲁道夫·卡尔纳普(Rudolf Carnap,1891–1970)和其他崇拜者,特别是伯特兰·罗素和路德维希·维特根斯坦(Ludwig Wittgenstein,1889–1951)的著作传播开来。
\subsection{哲学家}
\begin{figure}[ht]
\centering
\includegraphics[width=6cm]{./figures/44f72f78b9a77e4b.png}
\caption{} \label{fig_Gottlo_3}
\end{figure}
弗雷格是分析哲学的奠基人之一,他在逻辑和语言方面的工作促成了哲学中的语言转向。他对语言哲学的贡献包括:
\begin{itemize}
\item 命题的函数与论元分析;
\item 概念与对象的区分(Begriff und Gegenstand);
\item 组合性原理;
\item 语境原理;
\item 名词和其他表达的意义与指称的区分(Sinn und Bedeutung),有时被认为涉及一种媒介指称理论。
\end{itemize}
作为一位数学哲学家,弗雷格反对心理主义对判断内容的心理学解释,尤其是对句子意义的心理学解释。他最初的目的是远远超出回答关于意义的一般性问题;相反,他设计了自己的逻辑体系来探索算术的基础,致力于回答诸如“什么是数字?”或“数字词(‘一’,‘二’等)指向什么对象?”之类的问题。但在追寻这些问题的过程中,他最终发现自己在分析和解释“意义是什么”,因此得出了几个对后来的分析哲学和语言哲学发展具有深远影响的结论。
\subsection{意义与指称}  
弗雷格在1892年的论文《论意义与指称》("Über Sinn und Bedeutung")中提出了他具有影响力的意义与指称的区分:意义("Sinn")与指称("Bedeutung",也有翻译为“意义”或“指称”)。传统的意义观念认为表达式只有一个特征(指称),而弗雷格提出了表达式有两个不同的意义方面:它们的意义和指称。

指称(或“Bedeutung”)适用于专有名词,其中一个给定的表达式(例如“汤姆”这个表达式)只是指向承载该名字的实体(即名为汤姆的人)。弗雷格还认为命题与其真值之间有指称关系(换句话说,一个陈述“指向”它所代表的真值)。相对而言,完整句子的意义(或“Bedeutung”)是它所表达的思想。一个表达式的意义被认为是其所指称项的“呈现方式”,而同一个指称项可以有多种呈现方式。

这种区分可以通过以下例子来说明:在日常使用中,名字“查尔斯·菲利普·阿瑟·乔治·蒙巴顿-温莎”,在逻辑上是一个无法分析的整体,和功能性表达式“联合王国的国王”,其中包含有意义的部分“ξ的国王”和“联合王国”,它们有相同的指称,即人们最熟知的查尔斯三世国王。但“联合王国”这个词的意义是后者表达式意义的一部分,但不属于查尔斯国王“全名”的意义的一部分。

这些区分曾受到伯特兰·罗素的争议,尤其是在他的论文《论指称》中;这一争论一直持续至今,尤其是由索尔·克里普基的著名讲座《命名与必然性》所推动。
\subsection{1924年日记}
弗雷格发表的哲学著作具有很强的技术性,远离实际问题,以至于弗雷格学者邓梅特在阅读弗雷格的日记时感到震惊,发现他的英雄竟是一个反犹主义者。[19] 在1918-1919年的德国革命之后,他的政治观点变得更加激进。在他生命的最后一年,76岁时,他的日记中充斥着反对议会制度、民主党人、自由主义者、天主教徒、法国人和犹太人的政治观点,他认为犹太人应该被剥夺政治权利,最好是被驱逐出德国。[20] 弗雷格曾向人透露,“他曾经认为自己是一个自由主义者,还是俾斯麦的崇拜者”,但后来他对鲁登多夫将军产生了同情。1924年5月5日的日记中,弗雷格表示同意休斯顿·斯图尔特·张伯伦的《德国的重生》一书中的一篇文章,该文称赞了阿道夫·希特勒。[21] 弗雷格在日记中记录了他认为德国的犹太人“最好消失,或者更好的是希望他们从德国消失”的观点。[21] 关于这一时期的日记,有一些不同的解读。[22] 日记中还包含了对普选制和社会主义的批评。尽管弗雷格在现实生活中与犹太人保持友好关系:他的学生中有格尔肖姆·舍勒姆,[23][24] 后者高度评价弗雷格的教学,正是舍勒姆鼓励路德维希·维特根斯坦前往英格兰,去和伯特兰·罗素学习。[25] 1924年日记在1994年死后出版。[26]
\subsection{个性}
弗雷格被他的学生描述为一个极为内向的人,几乎不与他人进行对话,上课时大多面向黑板。然而,他偶尔会在课堂上展现出机智,甚至带有辛辣的讽刺。[27]
\subsection{重要日期}  
\begin{itemize}
\item 1848年11月8日,出生于维斯马,梅克伦堡-什未林。  
\item 1869年 — 就读于耶拿大学。  
\item 1871年 — 就读于哥廷根大学。  
\item 1873年 — 获得数学博士学位(几何学),在哥廷根大学取得。  
\item 1874年 — 在耶拿大学获得资格认证(Habilitation);成为私人教师。  
\item 1879年 — 在耶拿大学担任非常任教授(Ausserordentlicher Professor)。  
\item 1896年 — 在耶拿大学担任正式荣誉教授(Ordentlicher Honorarprofessor)。  
\item 1918年 — 退休。[28]  
\item 1925年7月26日,逝世于巴德·克莱宁(现为梅克伦堡-前波美拉尼亚的一部分)。
\end{itemize}
\subsection{重要著作}  
\subsubsection{逻辑与算术基础} 
《 Begriffsschrift: eine der arithmetischen nachgebildete Formelsprache des reinen Denkens 》(1879),哈雷(萨尔):路易斯·内贝尔出版社(在线版)。
\begin{itemize}
\item 英文版:*Begriffsschrift, a Formula Language, Modeled Upon That of Arithmetic, for Pure Thought*,收录于 J. van Heijenoort(编),*From Frege to Gödel: A Source Book in Mathematical Logic, 1879–1931*,哈佛大学,马萨诸塞州:哈佛大学出版社,1967年,第5-82页。  
\item 英文版(选定章节修订为现代形式符号):R. L. Mendelsohn, *The Philosophy of Gottlob Frege*,剑桥:剑桥大学出版社,2005年:“附录A. Begriffsschrift的现代符号表示:(1)至(51)”与“附录B. Begriffsschrift的现代符号表示:(52)至(68)”。[a]
\end{itemize}
《Die Grundlagen der Arithmetik: Eine logisch-mathematische Untersuchung über den Begriff der Zahl》(1884),布雷斯劳:威廉·科布纳出版社(在线版)。
\begin{itemize}
\item 英文版:*The Foundations of Arithmetic: A Logico-Mathematical Enquiry into the Concept of Number*,由J. L. Austin翻译,牛津:巴兹尔·布莱克韦尔出版社,1950年。
\end{itemize}
《Grundgesetze der Arithmetik, Band I》(1893);《Band II》(1903),耶拿:赫尔曼·波勒出版社(在线版)。
\begin{itemize}
\item 英文版(选定章节翻译):“*Translation of Part of Frege's Grundgesetze der Arithmetik*”,由彼得·吉奇(Peter Geach)和马克斯·布莱克(Max Black)翻译并编辑,收录于《*Translations from the Philosophical Writings of Gottlob Frege*》,纽约:哲学图书馆出版社,1952年,第137–158页。  
\item 德文版(现代符号修订版):*Grundgesetze der Arithmetik, Korpora*(杜伊斯堡-埃森大学门户),2006年:第一卷(2016年10月21日归档)和第二卷(2017年8月29日归档)。  
\item 德文版(现代符号修订版):“*Grundgesetze der Arithmetik – Begriffsschriftlich abgeleitet. Band I und II: In moderne Formelnotation transkribiert und mit einem ausführlichen Sachregister versehen*”,由T. Müller、B. Schröder和R. Stuhlmann-Laeisz编辑,帕德博恩:mentis出版社,2009年。  
\item 英文版:*Basic Laws of Arithmetic*,由菲利普·A·埃伯特(Philip A. Ebert)和马库斯·罗斯伯格(Marcus Rossberg)翻译并编辑,带有导言,牛津:牛津大学出版社,2013年。ISBN 978-0-19-928174-9。
\end{itemize}
\subsubsection{哲学研究}  
《函数与概念》(1891)
\begin{itemize}
\item 原文:“Funktion und Begriff”,发表于耶拿医学与自然科学学会演讲,1891年1月9日。  
\item 英文版:“Function and Concept”。
\end{itemize}
《论意义与指称》(1892)
\begin{itemize}
\item 原文:“Über Sinn und Bedeutung”,发表于《哲学与哲学批评期刊》第C卷(1892年):第25–50页。  
\item 英文版:“On Sense and Reference”,另有翻译版本(在后期版本中)为“*On Sense and Meaning*”。
\end{itemize}
《概念与对象》(1892)
\begin{itemize}
\item 原文:“Ueber Begriff und Gegenstand”,发表于《科学哲学季刊》第XVI卷(1892年):第192–205页。  
\item 英文版:“Concept and Object”。
\end{itemize}
《什么是函数?》(1904)
\begin{itemize}
\item 原文:“Was ist eine Funktion?”,发表于《为路德维希·玻尔兹曼六十岁生日献上的纪念文集》,由S. Meyer编,1904年2月20日,莱比锡,1904年,第656–666页。[29]  
\item 英文版:“What is a Function?”。
\end{itemize}
《逻辑研究》(1918–1923)。弗雷格原计划将以下三篇论文合编成一本名为《Logische Untersuchungen》(逻辑研究)的书。尽管这本德文书籍从未出版,但这些论文于1966年在《Logische Untersuchungen》中由G. Patzig编辑出版,英文翻译则在1975年由彼得·吉奇(Peter Geach)编辑,黑威尔出版社出版。
\begin{itemize}
\item 1918–19年。《Der Gedanke: Eine logische Untersuchung”(《思想:一项逻辑探究》),发表于《德国唯心主义哲学贡献》第I卷:[b] 第58–77页。  
\item 1918–19年。《Die Verneinung”(《否定》),发表于《德国唯心主义哲学贡献》第I卷,第143–157页。  
\item 1923年。《Gedankengefüge”(《复合思想》),发表于《德国唯心主义哲学贡献》第III卷,第36–51页。
\end{itemize}
\subsubsection{几何学文章}  
\begin{itemize}
\item 1903年:“Über die Grundlagen der Geometrie”。《德国数学家协会年报》第II卷,第XII期(1903年),第368–375页。  
\item 英文版:“On the Foundations of Geometry”。
\item 1967年:Kleine Schriften(I. Angelelli 编)。达姆施塔特:科学出版社,1967年;希尔德斯海姆,G. Olms出版社,1967年。  
《小论文》,这本书收录了弗雷格的大部分著作(例如前述文章),是弗雷格去世后出版的合集。
\end{itemize}
\subsection{另见}
\begin{itemize}
\item 弗雷格体系  
\item 计算机科学先驱名单  
\item 新弗雷格主义  
\end{itemize}
\subsection{注释}\\
a. 仅《Begriffsschrift》第二部分的证明在本书中被重新写成现代符号。第三部分的部分证明已在乔治·布洛斯(George Boolos)所著《Reading the Begriffsschrift》中重新编写,见《Mind》期刊第94卷第375期,第331–344页(1985年)。\\  
b. 《德国唯心主义哲学贡献》期刊是德国哲学学会(Deutsche Philosophische Gesellschaft)的刊物。
\subsection{参考文献}  
\begin{enumerate}
\item Balaguer, Mark (2016年7月25日)。Zalta, Edward N.(编)。Platonism in Metaphysics。斯坦福大学形而上学研究实验室 – 通过《斯坦福哲学百科全书》。  
\item Hans Sluga, “Frege's alleged realism,” Inquiry 20 (1–4):227–242 (1977)。  
\item Michael Resnik, “Frege as Idealist and then Realist,” Inquiry 22 (1–4):350–357 (1979)。  
\item Tom Rockmore, On Foundationalism: A Strategy for Metaphysical Realism,罗曼与小菲尔德出版社,2004年,第111页。  
\item 弗雷格在《Über Sinn und Bedeutung》中批评了直接实在论(参见Samuel Lebens,Bertrand Russell and the Nature of Propositions: A History and Defence of the Multiple Relation Theory of Judgement,劳特利奇出版社,2017年,第34页)。  
\item Truth – 《互联网哲学百科全书》;The Deflationary Theory of Truth(《斯坦福哲学百科全书》)。  
\item “Frege”。《随机出版社韦氏全大词典》。  
\item Wehmeier, Kai F.(2006年)。“Frege, Gottlob”。收录于Borchert, Donald M.(编),Encyclopedia of Philosophy,第3卷(第二版)。麦克米伦参考出版社,美国,ISBN 0-02-866072-2。  
\item Lothar Kreiser, Gottlob Frege: Leben – Werk – Zeit,Felix Meiner Verlag出版社,2013年,第11页。  
\item Arndt Richter, “Ahnenliste des Mathematikers Gottlob Frege, 1848–1925”。
\item Jacquette, Dale(2019年4月4日)。*Frege: A Philosophical Biography*。剑桥大学出版社。ISBN 9780521863278。  
\item Jacquette, Dale,Frege: A Philosophical Biography,剑桥大学出版社,2019年,第xiii页。  
\item Jacquette, Dale(2019年4月4日)。“第2章 - 大学时期的教育(1854–1874)”。Frege: A Philosophical Biography(第一版)。剑桥大学出版社,第37、42页。doi:10.1017/9781139033725.005。ISBN 978-1-139-03372-5。  
\item “Frege, Gottlob”,《互联网哲学百科全书》。  
\item Susanne Bobzien于2021年发表了一篇标题具有挑衅意味的论文《Frege抄袭了斯多亚派》:Bobzien, S., 收录于*Themes in Plato, Aristotle, and Hellenistic Philosophy, Keeling Lectures 2011–2018*,第149-206页;Zalta, Ed,《斯坦福哲学百科全书》中的“Frege”条目。
\item Horsten, Leon 和 Pettigrew, Richard,“引言”,收录于 The Continuum Companion to Philosophical Logic(Continuum International Publishing Group,2011年),第7页。  
\item Frege's Logic, Theorem, and Foundations for Arithmetic,《斯坦福哲学百科全书》,可在 [plato.stanford.edu](https://plato.stanford.edu) 查阅。  
\item Burgess, John(2005年)。Fixing Frege,普林斯顿大学出版社。ISBN 978-0-691-12231-1。  
\item Hersh, Reuben,What Is Mathematics, Really?(牛津大学出版社,1997年),第241页。  
\item Dummett, Michael A. E.(1973年)。Frege: Philosophy of Language,纽约:Harper & Row,第xii页。ISBN 978-0-06-011132-8 – 通过互联网档案馆(Internet Archive)获取。  
\item Yvonne Sherratt(2013年5月21日)。Hitler's Philosophers,耶鲁大学出版社,第60页。ISBN 978-0-300-15193-0。OCLC 1017997313。  
\item Hans Sluga:Heidegger's Crisis: Philosophy and Politics in Nazi Germany,第99页及后续。Sluga 的信息来源为 Eckart Menzler-Trott 的文章:《“我希望得到真相,且只有真相”:德国数学家和逻辑学家戈特洛布·弗雷格的政治遗言》。收录于 *Forvm*,第36卷,第432期,1989年12月20日,第68–79页。[在线链接](http://forvm.contextxxi.org/-no-432-.html)。
\item “弗雷格传记”。  
\item “Frege, Gottlob – 互联网哲学百科全书”。  
\item Juliet Floyd,《The Frege-Wittgenstein Correspondence: Interpretive Themes》(PDF)。PDF版本于2013年5月21日存档。  
\item Gottfried Gabriel, Wolfgang Kienzler(编辑):“Gottlob Freges politisches Tagebuch”。收录于《德国哲学杂志》第42卷,1994年,第1057–1098页。编辑序言见第1057–1066页。该文章已被翻译成英文,见《Inquiry》第39卷,1996年,第303–342页。  
\item 《弗雷格的逻辑讲义》,Erich H. Reck 和 Steve Awodey 编,Open Court Publishing,2004年,第18–26页。  
\item Jacquette, Dale(编辑)(2019年),“弗雷格生平大事件年表”,收录于 Frege: A Philosophical Biography,剑桥:剑桥大学出版社,第xiii–xiv页,doi:10.1017/9781139033725.001,ISBN 978-1-139-03372-5,S2CID 242262152。  
\item 《献给路德维希·玻尔兹曼六十岁生日的纪念文集》,1904年2月20日出版,附肖像、101幅文本插图及2张图版,莱比锡,J.A. Barth出版社,1904年。
\end{enumerate}
\subsection{来源}  
\subsubsection{原始文献}  
\begin{itemize}
\item 弗雷格作品及其英文译本的在线书目(由 Edward N. Zalta 编撰,斯坦福哲学百科全书)。  
\item 1879年:《*Begriffsschrift, eine der arithmetischen nachgebildete Formelsprache des reinen Denkens*》,哈雷(萨勒):路易斯·内贝尔出版社。  
  **英文译本**:*Concept Script, a formal language of pure thought modelled upon that of arithmetic*,由 S. Bauer-Mengelberg 翻译,收录于 Jean Van Heijenoort(编),*From Frege to Gödel: A Source Book in Mathematical Logic, 1879–1931*,哈佛大学出版社,1967年。  
- **1884年**:《*Die Grundlagen der Arithmetik: Eine logisch-mathematische Untersuchung über den Begriff der Zahl*》,布雷斯劳:W. Koebner出版社。  
  **英文译本**:J. L. Austin 译,1974年,《*The Foundations of Arithmetic: A Logico-Mathematical Enquiry into the Concept of Number*》(第2版),Blackwell出版社。  
- **1891年**:《*Funktion und Begriff*》。  
  **英文译本**:“*Function and Concept*”,收录于 Geach 和 Black(1980年)。  
- **1892a年**:《*Über Sinn und Bedeutung*》,发表于《哲学与哲学批评期刊》第100卷,第25–50页。  
  **英文译本**:“*On Sense and Reference*”,收录于 Geach 和 Black(1980年)。  
- **1892b年**:《*Ueber Begriff und Gegenstand*》,发表于《科学哲学季刊》第16卷,第192–205页。  
  **英文译本**:“*Concept and Object*”,收录于 Geach 和 Black(1980年)。
\end{itemize}