% 南京理工大学 2007 年 研究生入学考试试题 普通物理(B)
% license Usr
% type Note

\textbf{声明}:“该内容来源于网络公开资料,不保证真实性,如有侵权请联系管理员”

\subsection{一、填空题 I(30 分,每空 2 分)}
1. 一质点运动方程为$\vec{r}=4t^2i+(2t^3+3)\vec{j}$ ,则质点运动的轨道方程为___________,质点从 $t=0$ 到 $t=1s$ 的位移为________,质点在 $t=1s$ 时速度为__________,$t=1s$ 时加速度为__________。

2. 一轻质弹簧振子,弹簧的倔强系数 $k=25(N/m)$,初始动能为 $0.02J$,初始势能为 $0.06J$,则其振幅 $A=$_______,当位移 $x=$________时,动能与势能相等。

3. 一频率为 $500Hz$ 的平面谐波,其波速 $u=350m/s$,则其波长为_________$m$;在同一波线上,周相差为 $\pi/3$ 的两点的间距为___________$m$。

4. 若入射波方程为$y_1=A\cos(\omega t+\frac{2\pi x}{\lambda})$ ,在 $x=0$ 处反射,若反射端为固定端,则反射波方程为 $y2=$_____________(假设振幅不变),合成波方程为____________,波节点的位置为_______________。

5. $1mol$ 双原子刚性分子的理想气体,在一等压过程中对外作功 $100J$,则该等压
过程中理想气体内能变化为___________,吸收热量为______________。

6. 均匀带电的半径为 $R$ 的金属球,带电 $Q$,在距球心为 $a(a<R)$的一点 P 处
的电场强度大小 $E$=_________,电势大小 $U=$___________。
\subsection{二、填空题 II(28 分,每空 2 分)}
1. 波长为 $500nm$ 的单色光垂直照射一缝宽为 $0.25mm$ 的单缝,衍射图像中,中
央明纹极大两边第四暗纹极小距离为 $4mm$,则焦距为___________,其中央明
纹极大宽度为___________。

2. 如图,电流在 $O$ 点的磁感应强度的大小为__________,其方向为_________。
\begin{figure}[ht]
\centering
\includegraphics[width=6cm]{./figures/5d271a49aae9f94c.png}
\caption{} \label{fig_NJUB07_1}
\end{figure}
3. 迈克尔逊干涉仪中,当 $M2$ 移动距离$\Delta d=0.322mm$ 时,测得某单色光的干涉条纹移动过$\Delta N=1024$ 条,则该单色光的波长为__________;若在 $M2$ 镜前插入一透明片,观测到 150 条条纹移过,若薄片的折射率为 1.632,所用单色光的波长为 $500nm$,则薄片的厚度为_______________。

4. 一个 50 匝的半径 $R=5.0cm$ 的通电线圈处于 $B=1.5(T)$的均匀磁场中,线圈中电流 $I=0.2A$,则该线圈的磁矩大小为____________;当线圈的磁矩与外磁场方向的夹角从 0 转到 $\pi$ 时,外磁场对线圈所作的功为______________。

5. 涡旋电场(或感生电场)的场方程是___________,其物理意义是_______。

6. 一立方体的静质量为 m0,边长为 l0,若该物体沿其一边的方向作以速度为 u的高速运动 , 则静止的观测者测得其体积为 V=____________ , 密 度ρ=___________。

7. 一粒子静质量为 m0,其动能是静能的 n 倍,则该粒子的运动质量为________
__________,运动速度大小为____________________。