% 南京理工大学 普通物理 B(845)模拟五套卷 第二套
% license Usr
% type Note

\textbf{声明}:“该内容来源于网络公开资料,不保证真实性,如有侵权请联系管理员”

\subsection{填空题(30 分,每空 2 分)}
1. 已知一质点的运动方程为 $r=\left[(5\sin2\pi t)i+(4\cos2\pi t)j\right]$(单位为米),则该质点在 $0.25s$ 末的位置是__________,从 $0.25s$ 末到 $1s$ 末的位移是___
______。

2. 一质量为 $m$的质点在指向圆心的平方反比力的作用下,作半径为 r 的圆
周运动,此质点的速度 $v=$__________。若取距圆心无穷远处为势能零点,
它的机械能 $E$=____________。

3. 一轻绳绕于 $r=0.2m$ 的飞轮边缘,以恒力 $F=98N$ 拉绳,如图所示,已知
飞轮的转动惯量 $J=0.5kg\cdot m2$,轴承无摩擦。则飞轮的角加速度为______
_____;绳子拉下 $5m$ 时,飞轮的角速度为___________,动能为_______
\begin{figure}[ht]
\centering
\includegraphics[width=6cm]{./figures/406c268cd9b48d00.png}
\caption{} \label{fig_NJUD2_1}
\end{figure}
4. 已知一平面简谐波沿 $x$ 轴负向传播,振动周期 $T=0.5s$,波长 $l=10m$,振
幅 $A=0.1m$。当 $t=0$ 时波源振动的位移恰好为正的最大值。若波源处为原
点,则沿波传播方向距离波源为 $l/2$ 处的振动方程为 $y=$____________;
当 $t=T/2$ 时,$x=l/4$ 处质点的振动速度为____________。

5. 一 簧振子作简谐振动 ,其振动曲线图所示 。 则它的周期$T=$________ , 其余弦函数描述时初相位为 _________ 。
\begin{figure}[ht]
\centering
\includegraphics[width=6cm]{./figures/14a5896dfb807c1f.png}
\caption{} \label{fig_NJUD2_2}
\end{figure}
6. 如图,电流在 $O$ 点的磁感应强度的大小___________,方向__________。
\begin{figure}[ht]
\centering
\includegraphics[width=6cm]{./figures/99eb7464e0e920eb.png}
\caption{} \label{fig_NJUD2_3}
\end{figure}
7. 气缸内有 $2mol$ 的氦气,初始温度为 27°C,体积为 $20L$。先将氦气定压膨胀直到体积加倍,然后绝热膨胀直到恢复到初温为止,若视氦气为理想气体,则氦气的内能变化量_________,氦气所作总功为_________。
\subsection{填空题(16 分,每空 2 分)}
1. 用很薄的云母片$(n=1.58)$覆盖在双缝实验中的一条缝上,这时屏幕上的零级明条纹移到原来的第七级明条纹的位置上,如果入射光波长为 $\lambda=550nm$,则问此云母片的厚度为__________。

2. 若电荷以相同的面密度均匀分布在半径分别为 $r_1=10cm$ 和 $r_2=20cm$ 的两个同心球面上,设无穷远处电势为零,已知球心电势为 $300V$,则两球面的电荷面密度为__________。

3. 波长为 $600nm$ 的平行光垂直照射 $12cm$ 长的两玻璃片上,两玻璃片一端相互接触,另一端夹一直径为 $d$ 的金属丝。若测得这 $12cm$ 内由141条明纹,则金属丝的直径为___________。

4. 如图所示,一无限长直电流 $I_0$ 一侧一有其共面的矩形线圈,则通过此线圈的磁通量为____________。
\begin{figure}[ht]
\centering
\includegraphics[width=6cm]{./figures/5bc6573e36410949.png}
\caption{} \label{fig_NJUD2_4}
\end{figure}
5. 一半圆形闭合线圈,半径 $R=0.2m$,通过电流 $I=5A$,放在均匀磁场中,磁场方向与线圈平面平行,如图所示,磁感应强度 $B=0.5T$,则线圈所受到磁力矩为_________。若此线圈受磁力矩的作用从上述位置转到线圈平面与磁场方向成 30°的位置,则此过程中磁力矩做功为________。_______。
\begin{figure}[ht]
\centering
\includegraphics[width=6cm]{./figures/b03f5628a919e081.png}
\caption{} \label{fig_NJUD2_5}
\end{figure}
6. 粒子在加速器中被加速到动能为静止能量的4倍,其质量$m=$_______。($m_0$ 为静止质量)

7. 一体积为 $V_0$,质量为 $m_0$ 的立方体沿某一棱方向相对观察者 $A$ 以速度$v$ 运动,则观察者 $A$ 测得密度为__________。
\subsection{三、(13 分)}
一根细绳跨过一光滑的定滑轮,绳两端分别悬挂着质量为 $m1$ 和 $m2$的物体,$m1>m2$。求物体的加速度及绳对物体的拉力。绳与滑轮间的摩擦力可以略去不计。绳不伸长,滑轮和绳的质量也可略去不计。
\begin{figure}[ht]
\centering
\includegraphics[width=6cm]{./figures/4e19dc30069f2f8c.png}
\caption{} \label{fig_NJUD2_6}
\end{figure}
\subsection{四、(13 分)}
