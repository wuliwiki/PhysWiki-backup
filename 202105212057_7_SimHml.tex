% 复形的单纯同调群
% keys 同调|homology|复形|complex|单形|simplex|单纯形|复合形|边缘算子|链|链群|chain group|单纯同调


\pentry{单纯形与复形\upref{SimCom},自由群\upref{FreGrp}}


\subsection{复形上的链}

\begin{definition}{可定向单形}
一个单形$[a_0, a_1, \cdots, a_n]$,根据下标是奇排列还是偶排列,可以分为两类.称这样划分出来的两个等价类,是单形的\textbf{有向单形(oriented simplex)}.

偶排列的有向单形记为$a_0a_1a_2a_3\cdots a_n=a_1a_2a_0a_3\cdots a_n=\cdots$,奇排列的有向单形为$a_1a_0a_2a_3\cdots a_n=a_0a_2a_1a_3\cdots a_n=\cdots$.也就是说,有向单形的表示就是去掉单形表示的中括号.
\end{definition}

\begin{definition}{链群}
有向单形作为字母,可以构造自由生成阿贝尔群.其中同一个单形的两个有向单形,互为彼此的逆元.

称这样的自由生成阿贝尔群为一个\textbf{链群(chain group)},它的元素被称为\textbf{链(chain)}.

链群的运算用加法表示,于是也可以使用$\sum$符号.$n$个相同的有向单形$s^q$相加,记为$ns^q$.
\end{definition}

链群的名称很直观.考虑$1$-单形$[a_0, a_1]$的有向单形$a_0a_1$和$a_1a_0$,前者可以视为从点$a_0$到点$a_1$的一个箭头,后者可以视为点$a_1$到点$a_0$的一个箭头,而$[a_0, a_1]$就视为两点之间的线段,没有方向区分.这样一来,一维有向单形的链就是箭头的组合,其中首尾相连的部分就像链条一样,所以被称为链.也就是说,链群的群运算,就是链之间的“连接”运算,自然是交换的,因此链群被定义为有向单形的自由生成阿贝尔群.

\begin{definition}{复形上的链群}
给复形$K$的每一个单形都指定一个定向为正定向,则全体正定向的有向单形被称为$K$的\textbf{有向单形基本组}.另一定向的有向单形们,被表示为对应正定向单形的\textbf{相反数},也就是前面加一个负号.

$K$的全体$q$维有向单形的链群,称为$K$的$q$维链群,记为$C_q(K)$,其元素被称为$K$的$q$维链,简称$q$-链.
\end{definition}

以一维单形为例,如果$a_0a_1$被指定为正向单形,那么我们从此都把$a_1a_0$表示为$-a_0a_1$,方便计算和讨论.

根据ding'yi

\subsection{边缘链和闭链}

\begin{definition}{有向单形的顺向面和逆向面}
给定有向单形$s^q=a_0a_1a_2a_3\cdots a_q$,用$\hat{a_i}$表示\textbf{缺失了}$a_i$,即$a_0\hat{a_1}a_2\cdots a_q=a_0a_2\cdots a_q$.

那么称$t^{q-1}=(-1)^ia_0a_1\cdots\hat{a_i}\cdots a_q$为$s^q$的\textbf{顺向面};相应地,$-t^{q-1}$为\textbf{逆向面}.
\end{definition}

我们举一个例子:考虑有向单形$a_0a_1a_2$,如\autoref{SimHml_fig1} 所示,我们可以把它看成一个“顺时针方向的三角形”.它的面都是箭头,其中$a_0a_1$,$a_1a_2$和$a_2a_0$是顺向面,$a_0a_2$,$a_2a_1$和$a_1a_0$是逆向面.

\begin{figure}[ht]
\centering
\includegraphics[width=5cm]{./figures/SimHml_1.pdf}
\caption{顺向面和逆向面的示意图.} \label{SimHml_fig1}
\end{figure}

有了顺向面和逆向面的概念之后,我们就可以比较方便地引入一个重要的映射,边缘算子.

\begin{definition}{边缘算子}




\end{definition}










