% 余维数
% keys 余维数|无限维空间
% license Usr
% type Tutor

\pentry{商空间(线性代数)\nref{nod_QuoSpa}}{nod_9572}

\enref{商空间}{QuoSpa}一文已经介绍了余维数的概念,然而在那里,定理\autoref{the_QuoSpa_3} 的证明依赖于“ $\ker f=0$ 则满射 $f$ 是双射”。而该论断只对有限维空间上的线性映射 $f$ 有效。在无限维空间里,该论断失效了,从而定理\autoref{the_QuoSpa_3} 失效,其得到的商空间的维数公式\autoref{eq_QuoSpa_2} 也失效了。其实,通过 $\dim(V/U) = \dim V - \dim U$ 就能看出,$\dim V=\infty$ 时,该公式是有问题的,因为关于无穷的加减法我们一无所知。特别,当 $\dim U=\infty$ 时,$\infty-\infty$ 本身就不是确定的。

然而,商空间一文的其它内容并不依赖于空间是否为有限,因此在无穷维空间时仍然是成立的。特别,余维数仍有意义。 本文将看到,在余维数为有限时,矢量空间的元素将有唯一的形式。为和有限维空间相区别,我们用 $L$ 来代表(不限于有限维的)一般的线性空间。

\subsection{余维数}

\begin{definition}{余维数}
设 $L$ 是任意线性空间,$L'$ 是它的子空间。则商空间 $L/L'$ 的维数叫着(空间 $L$中)$L'$ 的\textbf{余维数},记作 $\mathrm{codim}_L L'$。 
\end{definition}


\begin{theorem}{有限余维数矢量表述的唯一性}
设子空间 $L'\subset L$ 具有有限余维数 $n$,则在 $L$ 中可选取 $x_1,\\cdots,x_n$ ,使得任一元素 $x\in L$ 具有唯一的形式:
\begin{equation}
x=\sum_{i=1}^n\alpha_i x_i+y,~
\end{equation}
其中 $\alpha_1,\cdots,\alpha_n$ 是 $L$ 的域 $\mathbb F$ 上的数,$y\in L'$。
\end{theorem}

\textbf{证明:}由于 $\dim L/L'=n$,因此可在 $L/L'$ 上选取基 $\xi_1,\cdots,\xi_n$。


\textbf{证毕!}

