% 角动量加法(量子力学)
% 角动量|自旋|轨道角动量|角量子数|磁量子数|叠加

\pentry{轨道角动量\upref{QOrbAM}, 张量积空间\upref{DirPro}}

考虑两个系统的角动量空间, 基底分别为 $\ket{l_1, m_1}$ 和 $\ket{l_2, m_2}$, 其中 $l_1, l_2$ 固定空间的维度分别为 $2l_1+1$ 和 $2l_2+1$. 两空间的角动量算符分别为
\begin{equation}\label{AMAdd_eq1}
L_1^2 \qquad L_{1x} \qquad L_{1y} \qquad L_{1z}
\end{equation}
\begin{equation}
L_2^2 \qquad L_{2x} \qquad L_{2y} \qquad L_{2z}
\end{equation}

我们用这两个空间生成 $(2l_1+1)(2l_2+1)$ 维的张量积空间, 基底为 $\ket{l_1, m_1}\otimes\ket{l_2, m_2}$ (或记为 $\ket{l_1, m_1, l_2, m2}$). 张量积空间中一组显然的 \textbf{Complete Set of Commutable Operators (CSCO)}为 % 引用未完成 (\cite{Sakurai} 1.4.36 有提到, 以及 \cite{Shankar})
\begin{equation}
\{L_{1z}, L_{2z}\}
\end{equation}
也就是说只需要 $m_1, m_2$ 两个量子数就可以唯一确定一个基底.

在张量积空间上定义两个系统总角动量算符为\footnote{注意有时候为了方便我们会直接记为如 $L_x = L_{1x} + L_{2x}$ 的形式, 严格来说这是不对的, 因为等号左边的算符所在的空间是等号右边两个算符所在空间的张量积, 而并非同一空间中的两个算符相加.}
\begin{equation}\ali{
&L_x = L_{1x} \otimes I +  I \otimes L_{2x} \\
&L_y = L_{1y} \otimes I +  I \otimes L_{2y} \\
&L_z = L_{1z} \otimes I +  I \otimes L_{2z}
}\end{equation}
\begin{equation}
L^2 = L_x^2 + L_y^2 + L_z^2
\end{equation}
可以证明新增的对易关系为
\begin{equation}
\comm*{L^2}{L_z} = \comm*{L^2}{L_1^2} = \comm*{L^2}{L_2^2} = 
\comm*{L_z}{L_{1z}} = \comm*{L_z}{L_{2z}} = 0
\end{equation}
由此可以得到另一组 CSCO 为 %未完成:如何得到?
\begin{equation}
\{L^2, L_z\}
\end{equation}
即我们可以想得到张量积空间的另一组基底. 令两个算符的本征值(量子数)分别为 $L$  和 $M$, 这组基底可记为 $\ket{L, M}$.  首先由对易关系, $\ket{l_1, m_1, l_2, m_2}$ 已经是 $L_z$ 的本征矢,每个本征值 $M$ 对应一个本征子空间(因为有简并), 以下称为 “$M$ 子空间”. 子空间的维数 $N_M$ 是满足 $m_1 + m_2 = M$ 的不同 $m_1,m_2$ 组合数(\autoref{AMAdd_fig1} 左). 例如图中当 $M = -1/2$ 时 $N_M = 4$.
\begin{figure}[ht]
\centering
\includegraphics[width=14.25cm]{./figures/AMAdd_1.pdf}
\caption{角量子数分别为 $l_1 = 3/2$ 和 $l_2 = 2$ 的两个粒子进行角动量加法. 图中每个方格代表张量积空间中一个基底. 左图中每个基底是 $\ket{l_1, m_1, l_2, m_2}$ 的形式, 每个条斜线穿过的基底张成一个 $M$ 子空间. 而右图每个基底是 $\ket{L, M}$ 的形式. 基底变换在每个 $M$ 子空间中独立进行, 即左图的一条斜线变成右图的一行.} \label{AMAdd_fig1}
\end{figure}

\begin{equation}
N_M =
\begin{cases}
l_1 + l_2 - \abs{M} + 1 &(\abs{M} > \abs{l_1 - l_2}) \\
2\min\{l_1, l_2\}  + 1   &(\abs{M} \leqslant\abs{l_1 - l_2})
\end{cases}
\end{equation}
$M$ 子空间中每组基底中 $m_1$ 的最大值为
\begin{equation}
\max \qty{m_1} =
\begin{cases}
l_1 \quad &(M \geqslant l_1 - l_2)  \\
l_2 + M &(M < l_1 - l_2)
\end{cases}
\end{equation}
例如图中当 $M = 1/2$ 时 $m_1$ 的最大值是 $3/2$.

线性代数中基底的排列顺序十分重要, 顺序会影响坐标和矩阵的值. 我们通常令 $\ket{l_1, m_1, l_2, m_2}$ ($m_2 = M - m_1$)按 $m_1$ 从大到小的顺序排列, $\ket{L, M}$ 按 $L$ 从大到小排序.

可以证明在每个 $M$ 子空间中, $\ket{L, M}$ 取 $L = l_1 + l_2, l_1 + l_2 - 1,\dots$ (共 $N_M$ 个). % 未完成,不会证明
$L$ 在所有 $M$ 子空间的最小值是 $\abs{l_1 - l_2}$, 当 $\abs{M} \leqslant \abs{l_1 - l_2}$ 时取得(\autoref{AMAdd_fig1} 右边最后一列),所以 $L$ 在整个张量积空间中的范围是
\begin{equation}\label{AMAdd_eq2}
L = l_1 + l_2, \dots, \abs{l_1 - l_2}
\end{equation}

\subsection{CG 系数}
既然每个 $M$ 子空间可以通过 $N_M$ 个正交归一的 $\ket{l_1, m_1, l_2, m_2}$ 基底或者 $\ket{L,M}$ 基底展开, 那么就有必要讨论它们之间的酉变换矩阵 $\mat U_M$. 其矩阵元 $\braket{l_1, m_1, l_2, m_2}{L, M}$ 被称为 \textbf{Clebsch–Gordan 系数}或者简称 \textbf{CG 系数}. 详见 “Clebsch–Gordan 系数\upref{SphCup}”.
\begin{equation}\label{AMAdd_eq3}
\ket{L, M} = \sum_{m_1, m_2}\braket{l_1, m_1, l_2, m_2}{L, M} \cdot \ket{l_1, m_1, l_2, m_2}
\end{equation}
\begin{equation}
\ket{l_1, m_1, l_2, m_2} = \sum_{L}\braket{L, M}{l_1, m_1, l_2, m_2} \cdot \ket{L, M}
\end{equation}
注意\autoref{AMAdd_eq3} 对 $m_1,m_2$ 求和需要满足 $M = m_1 + m_2$, 所以以上两式都是 $N_M$ 项求和. CG 系数也可以用另外两种符号记为
\begin{equation}
\bmat{l_1 & l_2 & L\\ m_1 & m_2 & M} = C_{l_1, m_1, l_2, m_2}^{L, M} = \braket{l_1, m_1, l_2, m_2}{L, M}
\end{equation}

要求出 CG 系数, 我们只需要在每个 $M$ 子空间中把 $\ket{l_1, m_1, l_2, m_2}$ 基底下的 $L^2$ 矩阵对角化
\begin{equation}
L^2 = L_1^2 + L_2^2 + 2(L_{1x} L_{2x} + L_{1y} L_{2y} + L_{1z} L_{2z})
\end{equation}
其中只有 $L_{1x} L_{2x} + L_{1y} L_{2y}$  不是对角矩阵. 可以利用升降算符表示为
\begin{equation}
2 (L_{1x} L_{2x} + L_{1y} L_{2y} ) = L_{1+} L_{2-} + L_{1-} L_{2+}
\end{equation}
于是以 $\ket{l_1, m_1, l_2, m_2}$ 为基底, $L^2$ 矩阵的矩阵元为
\begin{equation}\ali{
&\quad \bra{l_1, m'_1, l_2, m'_2} L^2 \ket{l_1, m_1, l_2, m_2} = \hbar ^2 \times\\
& \qty[ \ali{
&\delta_{m'_1, m_1} \delta_{m'_2, m_2} [l_1(l_1 + 1) + l_2(l_2 + 1) + 2 m_1 m_2]  \\
+ &\delta_{m'_1, m_1 + 1} \delta_{m'_2, m_2 - 1} \sqrt{l_1 (l_1 + 1) - m_1(m_1 + 1)} \sqrt{l_2 (l_2 + 1) - m_2(m_2 - 1)}\\
+ &\delta_{m'_1, m_1 - 1} \delta_{m'_2, m_2 + 1} \sqrt{l_1 (l_1 + 1) - m_1(m_1 - 1)} \sqrt{l_2 (l_2 + 1) - m_2(m_2 + 1)} }]
}\end{equation}
可以看出这是一个三对角矩阵, 其 $N_M$ 个正交归一的本征列矢量就是 $\ket{L, M}$ ($L = l_1 + l_2, l_1 + l_2 - 1, \dots$)在 $\ket{l_1, m_1, l_2, m_2}$ 基底下的坐标. 把这些列矢量按基底顺序从左到右排列就是 $\mat U_M$.

可以证明, $\mat U_M$ 的 $N_M$ 个本征值为 $L(L + 1) \hbar ^2$,  其中 $L = l_1 + l_2, l_1 + l_2 - 1,\dots$ (共 $N_M$ 个). % 未完成,不会证明
$L$ 在所有 $M$ 子空间的最小值是 $\abs{l_1 - l_2}$, 当 $\abs{M} \leqslant \abs{l_1 - l_2}$ 时取得(\autoref{AMAdd_fig1} 右边最后一列).
\addTODO{如何证明?}
