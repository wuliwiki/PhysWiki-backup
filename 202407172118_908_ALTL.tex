% 阿兰·图灵
% license CCBYSA3
% type Wiki

(本文根据 CC-BY-SA 协议转载自原搜狗科学百科对英文维基百科的翻译)

艾伦·麦席森·图灵·奥博·弗莱斯(英语:Alan Mathison Turing,/ˈtjʊərɪŋ/,又译阿兰·图灵,Turing也常翻译成涂林或者杜林,1912年6月23日——1954年6月7日),英国数学家、计算机科学家、逻辑学家、密码分析学家、哲学家和理论生物学家。图灵在理论计算机科学的发展中具有很大的影响力,他为算法和计算的概念提供了一个形式化的图灵机,图灵机可以被认为是一个通用计算机的模型,为现代计算机的逻辑工作方式奠定了基础,[1]同时也被称为理论计算机科学和人工智能之父。[2]

第二次世界大战期间,图灵在破解截获的编码信息方面发挥了关键作用,这些信息使盟军能够在许多关键战役中击败纳粹,包括大西洋战役。[3][4]从长远意义上讲,据估计这项工作缩短了欧洲战争两年多,挽救了1400多万人的生命。[3]1952年因同性恋行为被起诉,拉博彻修正案规定“严重猥亵”在英国是刑事犯罪。他接受了化学阉割治疗,用DES替代监狱。1954年,图灵死于氰化物中毒,享年41岁。[5]2013年12月24日,在英国司法大臣克里斯·格雷灵的要求下,英国女王伊丽莎白二世向图灵颁发了皇家赦免。

\subsection{ 早期生活}
\subsubsection{1.1 家庭}
图灵出生在伦敦迈达谷,[6]而他的父亲朱利叶斯·马西森·图灵(1873-1947年)则离开了他在查特拉布尔(Chatrapur)的印度公务员队伍,当时是马德拉斯(Madras)总统任期,目前在印度奥迪沙邦( Odisha)。[6][7]图灵的父亲是牧师约翰·罗伯特·图灵牧师的儿子,他来自一个位于荷兰的苏格兰商人家庭,其中包括一名准男爵。图灵的母亲,朱利叶斯(Julius)的妻子,是埃塞尔·萨拉·图灵(内·斯通尼,1881-1976),[6]爱马德拉斯铁路公司(Madras Railways)总工程师爱德华·沃勒·斯通尼的女儿。斯通尼一家是蒂珀雷里县和龙福德县的新教盎格鲁爱尔兰贵族家庭,而埃塞尔本人童年大部分时间都在克莱尔县(County Clare)度过。[8]

朱利叶斯与ICS的合作将这个家庭带到了英属印度,他的祖父曾在那里的孟加拉军队中担任将军。然而,朱利叶斯和埃塞尔都想让他们的孩子在英国长大,所以他们搬到了伦敦的迈达山谷,[9]艾伦·图灵于1912年6月23日出生在那里,正如他出生的房子(后来的Colonnade酒店)外面的蓝色匾所记录的那样[10][11] [6][12]。图灵有一个哥哥约翰(约翰·德莫特·图灵爵士的父亲,图灵男爵的第12个男爵)。[13]

图灵父亲的公务员委员会仍然很活跃,在图灵童年时期,图灵的父母在英国黑斯廷斯( Hastings)和印度之间旅行,[14]留下他们的两个儿子和一对退休的陆军夫妇住在一起。在黑斯廷斯,图灵住在上梅兹山的巴斯顿小屋(Baston Lodge, Upper Maze Hill),圣莱昂纳斯(St Leonards-on-Sea),现在有一块蓝色的匾,[15]这块匾于图灵诞生一百周年的2012年6月23日揭幕。[16]

在早年,图灵展示了天才的迹象,后来他又突出地展示了这一点。[17]1927年,他的父母在吉尔福德(Guildford)买了一栋房子,图灵在学校放假期间住在那里。该位置还标有蓝色斑块。[18]
\subsubsection{1.2 学校}
图灵的父母在他六岁的时候就给他注册了圣迈克尔学校(St Michael's),这是一所位于海边的查尔斯路(Charles Road)20号的日制学校。女校长很早就发现了他的才华,他后来的许多老师也是如此。

1922年1月至1926年1月间,图灵在哈斯赫斯特预备学校(Hazelhurst Preparatory School)接受教育,这所学校是苏塞克斯郡(Sussex)(现为东苏塞克斯郡)弗兰特村的一所独立学校。[19]1926年,13岁时,他继续就读于舍伯恩(Sherborne) 学校,这是一所位于多塞特郡舍伯恩集镇的独立寄宿学校。开学的第一天正好赶上1926年英国总罢工,但他下定决心要参加,以至于他独自骑自行车从南安普敦到谢伯恩60英里(97公里),并在一家旅店过夜。[20]

图灵对数学和科学的天生爱好并没有赢得舍伯恩的一些老师的尊重,舍伯恩的老师们对教育的定义更加强调经典。他的校长写信给他的父母:“我希望他不会两头落空。如果他要留在公立学校,他必须以接受教育为目标。如果他是一名科学专家,那他在公立学校是对时间的一种浪费”。[21]尽管没有得到老师们的认可,但是图灵继续在他热爱的研究中表现出非凡的能力。1927年,他甚至在连初等微积分没接触过的基础上就解决了高级问题;1928年,16岁的图灵读到了阿尔伯特·爱因斯坦的作品;他不仅领会了这一点,而且有可能从一篇从未明确阐述过的文章中推断出爱因斯坦对牛顿运动定律的质疑。[22]

舍伯恩毕业后,图灵从1931年到1934年在剑桥国王学院学习,[6]并在那里获得了一级数学荣誉。1935年,他22岁时,凭借一篇证明中心极限定理的论文,被选为国王学院的研究员。[23]委员会不知道,这个定理已经在1922年被贾尔·瓦尔德马尔·林德伯格(Jarl Waldemar Lindeberg)证明了。[24]2012年6月23日,该学院的一块蓝色牌匾在他诞辰100周年之际揭幕,现已安装在国王阅兵式上的学院凯恩斯大楼(Keynes Building)。[25][26]
\subsubsection{1.3 朋友}
在舍伯恩,图灵和他的同学克里斯托弗·莫克姆(Christopher Morcom,1911-1930)建立了重要的友谊,克里斯托弗·莫克姆被描述为图灵的“初恋”。他们的关系为图灵未来的成就提供了灵感,但由于莫克姆于1930年2月死于牛结核病的并发症,这一关系被打断了。牛结核病是几年前喝了受感染的牛奶后感染的。[27][28][29]

这件事给图灵带来了巨大的悲痛。他通过更加努力地研究他与莫克姆共享的科学和数学主题来应对自己的悲伤。在给莫克姆的母亲的一封信中图灵说:

我相信我在任何地方都找不到另一个如此聪明、如此迷人、如此不可思议的伴侣。我认为我对我的工作和天文学(他向我介绍了天文学)的兴趣是可以和他分享的,我想他对我也有一点相同的感觉...我知道我必须把尽可能多的精力投入到我的工作中,就像他活着一样,因为那是他希望我做的。

莫克姆去世后很久,图灵和莫克姆的母亲一直保持着联系,她给图灵送礼物,而他通常在莫克姆生日那天写信。莫克姆去世三周年的前一天(1933年2月12日),他写信给莫克姆夫人:

我希望你收到这封信时会想起克里斯。我也会的,这封信只是想告诉你,明天我会想到克里斯和你。我确信他现在和曾经在这里时一样快乐。你亲爱的艾伦。

一些人推测,莫克姆的死是图灵无神论和唯物主义的原因。显然,在生命的这一点上,他仍然相信精神(独立于肉体,在死亡中幸存)的概念。图灵在后来的一封信中说:

就我个人而言,我相信精神确实永远与物质联系在一起,但肯定不是由同一种身体联系在一起的...至于精神和身体之间的实际联系,我认为身体可以抓住一个“精神”,当身体活着和清醒时,两者是紧密相连的。当身体睡着的时候,我无法猜测会发生什么,但是当身体死亡的时候,身体的“机制”就消失了,灵魂迟早会找到一个新的身体,也许是马上。

\subsection{研究成就}
图灵在科学、特别在数理逻辑和计算机科学方面,取得了举世瞩目的成就,他的一些科学成果,构成了现代计算机技术的基础。
\subsubsection{2.1 可计算性研究}
\begin{figure}[ht]
\centering
\includegraphics[width=8cm]{./figures/e29f60892c491fbe.png}
\caption{剑桥国王学院。图灵于1931年成为这里的学生,并在1935年成为研究员。微机室是以他的名字命名的。} \label{fig_ALTL_1}
\end{figure}
1936年,图灵发表了他的论文《关于可计算的数字,及其在Entscheidungsproblem中的应用》(1936)[30],在这篇论文中,图灵重新表述了库尔特·哥德尔(Kurt Gödel's)1931年关于证明和计算极限的结果,用后来被称为图灵机的形式和简单的假设装置取代了哥德尔的通用算术形式语言。决策问题最初是由德国数学家戴维·希尔伯特在1928年提出的。图灵证明了他的“通用计算机器”能够执行任何可以想象的数学计算,如果它可以表示为一种算法的话。他继续证明决策问题没有解决方案,首先证明图灵机的停机问题是不可判定的:不可能通过算法决定图灵机是否会停机。

虽然图灵的证明是在阿隆佐·邱奇(Alonzo Church's)用λ演算进行等价证明后不久发表的,[31]但图灵的方法比丘奇的方法更容易理解和直观。[32]它还包含了“通用机器”(现在被称为通用图灵机)的概念,认为这样的机器可以执行任何其他计算机器的任务(实际上教会的λ演算也可以)。根据丘奇-图灵理论,图灵机和λ演算能够计算任何可计算的东西。约翰·冯·诺依曼(John von Neumann)承认现代计算机的核心概念源于图灵的论文。[33]迄今为止,图灵机是计算理论的一个中心研究对象。

从1936年9月到1938年7月,图灵第二年大部分时间在普林斯顿大学教堂学习,[34]除了纯粹的数学工作之外,他还研究密码学,并建立了机电二进制乘法器的四个阶段中的三个[53]。1938年6月,他获得了普林斯顿数学系的博士学位;[35]他的论文《基于序数的逻辑系统》引入了序数逻辑的概念和相对计算的概念,[36][37]在这两个概念中,图灵机被所谓的预言所增强,从而可以研究图灵机无法解决的问题。约翰·冯·诺依曼想雇他做博士后助理,但他回到了英国。[38]

图灵回到剑桥后,他参加了1939年路德维希·维特斯坦根(Ludwig Wittgenstein)关于数学基础的讲座。[39]讲座逐字逐句地进行了重构,包括图灵和其他学生的感叹词,以及学生笔记。[40]图灵和维特根斯坦争论不休,图灵为形式主义辩护,维特根斯坦提出他的观点,即数学不发现任何绝对真理,而是发明了它们。[41]
\subsubsection{2.2 密码分析学}
\begin{figure}[ht]
\centering
\includegraphics[width=8cm]{./figures/66f4e52ad7bf9342.png}
\caption{布莱奇利公园的马场里有两间小屋。图灵1939年和1940年在这里工作,后来搬到8号小屋。} \label{fig_ALTL_2}
\end{figure}