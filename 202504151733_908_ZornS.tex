% 佐恩引理(综述)
% license CCBYSA3
% type Wiki

本文根据 CC-BY-SA 协议转载翻译自维基百科\href{https://en.wikipedia.org/wiki/Zorn\%27s_lemma}{相关文章}。

\begin{figure}[ht]
\centering
\includegraphics[width=6cm]{./figures/d0b4921eea445e97.png}
\caption{佐恩引理可以用来证明每个连通图都有一个生成树。所有是树的子图构成一个由包含关系排序的集合,而链的并集是一个上界。佐恩引理指出,必须存在一个极大树,而这个极大树就是生成树,因为图是连通的。[1] 对于有限图,例如这里所示的图,实际上不需要使用佐恩引理。} \label{fig_ZornS_1}
\end{figure}
佐恩引理,也称为库拉托夫斯基–佐恩引理,是集合论中的一个命题。它表明,对于一个包含每个链(即每个全序子集)上界的偏序集,该集合必定包含至少一个极大元素。

该引理由卡济米日·库拉托夫斯基于1922年证明(假设选择公理),并由马克斯·佐恩于1935年独立证明。[2] 它出现在若干关键性定理的证明中,例如泛函分析中的哈恩–巴拿赫定理、每个向量空间都有一个基的定理,[3] 拓扑学中的泰赫诺夫定理(即每个紧致空间的乘积是紧致的),以及抽象代数中的定理(在带有单位的环中,每个真理想都包含在一个极大理想中,且每个域都有代数闭包)[4]。

佐恩引理与良序定理等价,并且与选择公理等价,意味着在ZF(不含选择公理的泽梅洛–弗兰克尔集合论)中,任何一个引理都足以证明另外两个。[5] 佐恩引理的早期表述是豪斯多夫最大原理,它表明给定偏序集的每个全序子集都包含在该偏序集的一个极大全序子集中。[6]
\subsection{动机}  
为了证明一个数学对象的存在,这个对象可以被视为某个偏序集中某种方式下的极大元素,可以尝试通过假设不存在极大元素,并利用超限归纳法和该情况的假设来得到矛盾,从而证明该对象的存在。佐恩引理整理了一个情境需要满足的条件,以便使这种证明方法有效,并使数学家们无需每次都手动重复超限归纳法的论证,而只需检查佐恩引理的条件。

如果你正在分阶段构建一个数学对象,发现(i)即使经过无限多个阶段,你仍然没有完成,且(ii)似乎没有任何东西能阻止你继续构建,那么佐恩引理可能能够帮助你。

— 威廉·蒂莫西·高尔斯,《如何使用佐恩引理》[7]
\subsection{引理的陈述}  
前提概念:
\begin{itemize}
\item 一个集合\( P \)配备一个二元关系\( \leq \),如果该关系是自反的(即对于每个\( x \),有\( x \leq x \))、反对称的(如果\( x \leq y \)且\( y \leq x \)都成立,则\( x = y \))、传递的(即如果\( x \leq y \) 且 \( y \leq z \)成立,则\( x \leq z \)),那么我们称\( P \) 是通过\( \leq \) 部分有序的。给定\( P \)中的两个元素 \( x \)和\( y \),如果\( x \leq y \),则称\( y \)大于或等于\( x \)。词语“部分”表示并非每一对部分有序集中的元素都需要在该顺序关系下可比较,也就是说,在一个带有顺序关系\( \leq \)的部分有序集\( P \)中,可能存在元素\( x \)和\( y \),使得既不\( x \leq y \) 也不\( y \leq x \)。一个有序集,如果其中每一对元素都可以比较,则称为全序的。
\item 部分有序集\( P \)的每个子集\( S \)可以通过将从\( P \)继承的顺序关系限制到\( S \)上,自己被看作是部分有序的。如果一个部分有序集\( P \)的子集\( S \) 在继承的顺序下是全序的,则称\( S \)是一个链(在\( P \) 中)。
\item 如果部分有序集\( P \)中有一个元素\( m \),使得没有其他元素大于 \( m \),即没有\( P \)中的元素\( s \)满足\( s \neq m \)且\( m \leq s \),则称\( m \)是最大元素(相对于\( \leq \))。根据顺序关系的不同,部分有序集可能有任意数量的最大元素。然而,完全有序集最多只能有一个最大元素。
\item 给定部分有序集\( P \) 的一个子集\( S \),如果\( P \)中的元素\( u \)大于或等于\( S \) 中的每个元素,则称\( u \)是 \( S \)的上界。在这里,\( S \)不要求是链,并且\( u \)必须与\( S \)中的每个元素可比较,但不需要是\( S \)的元素。
\end{itemize}
佐恩引理可以表述为:

\textbf{佐恩引理}—[8][9] 设\( P \)是一个部分有序集,满足以下两个条件:
\begin{enumerate}
\item \( P \) 非空;
\item \( P \) 中的每个链都有一个上界。
\end{enumerate}
则,\( P \)至少有一个最大元素。