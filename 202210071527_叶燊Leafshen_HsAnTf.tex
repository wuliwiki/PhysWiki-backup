% 三角恒等变换(高中)
% 高中|三角恒等变换

\begin{issues}
\issueDraft
\issueOther{和三角恒等式\upref{TriEqv}有部分重复,但这篇更符合高中数学要求}
\issueOther{应该与三角恒等式\upref{TriEqv}合并,并讲超出高中范围的知识写在最后}
\end{issues}

\subsection{两个基本公式}
\begin{equation}
\sin^2\alpha + \cos^2\alpha = 1
\end{equation}
\begin{equation}
\frac{\sin\alpha}{\cos\alpha} = \tan\alpha
\end{equation}

\subsection{两角和与两角差}
\begin{equation}\label{HsAnTf_eq5}
\sin(\alpha + \beta) = \sin\alpha \cos\beta + \cos\alpha \sin\beta
\end{equation}
\begin{equation}\label{HsAnTf_eq6}
\sin(\alpha - \beta) = \sin\alpha \cos\beta - \cos\alpha \sin\beta
\end{equation}
\begin{equation}\label{HsAnTf_eq4}
\cos(\alpha + \beta) = \cos\alpha \cos\beta - \sin\alpha \sin\beta
\end{equation}
\begin{equation}\label{HsAnTf_eq3}
\cos(\alpha - \beta) = \cos\alpha \cos\beta + \sin\alpha \sin\beta
\end{equation}
\begin{equation}\label{HsAnTf_eq7}
\tan(\alpha + \beta) = \frac{\tan\alpha+\tan\beta}{1-\tan\alpha \tan\beta}
\end{equation}
\begin{equation}\label{HsAnTf_eq8}
\tan(\alpha - \beta) = \frac{\tan\alpha - \tan\beta}{1+\tan\alpha \tan\beta}
\end{equation}

\subsection{二倍角公式}
\begin{equation}
\sin2\alpha = 2\sin\alpha \cos\alpha 
\end{equation}
\begin{equation}
\cos2\alpha = \cos^2\alpha - \sin^2\alpha = 1 - 2\sin^2\alpha = 2\cos^2\alpha -1
\end{equation}
\begin{equation}
\tan2\alpha = \frac{2\tan\alpha}{1-\tan^2\alpha}
\end{equation}

\subsection{半角公式}
\begin{equation}
\sin\frac{\alpha}{2} = \pm\sqrt{\frac{1-\cos\alpha}{2}}
\end{equation}
\begin{equation}
\cos\frac{\alpha}{2}= \pm\sqrt{\frac{1+\cos\alpha}{2}}
\end{equation}
\begin{equation}
\tan\frac{\alpha}{2} = \pm\sqrt{\frac{1-\cos\alpha}{1+\cos\alpha}} = \frac{\sin\alpha}{1+\cos\alpha} = \frac{1-\cos\alpha}{\sin\alpha}
\end{equation}
注意正负号的选择需要根据 $\alpha$ 的具体取值判断.

\subsection{升幂公式}
\begin{equation}
\cos2\alpha + 1 = 2\cos^2\alpha
\end{equation}
\begin{equation}
1-\cos2\alpha = 2\sin^2\alpha
\end{equation}

\subsection{降幂公式}
\begin{equation}
\cos\alpha = \pm\sqrt{\frac{1+\cos2\alpha}{2}}
\end{equation}
\begin{equation}
\sin\alpha = \pm\sqrt{\frac{1-\cos2\alpha}{2}}
\end{equation}

\subsection{万能公式}
\begin{equation}
\sin2\alpha = \frac{2\sin\alpha \cos\alpha}{\sin^2\alpha + \cos^2\alpha} = \frac{2\tan\alpha}{1+\tan^2\alpha}
\end{equation}
\begin{equation}
\cos2\alpha = \frac{\cos^2\alpha-\sin^2\alpha}{\sin^2\alpha+\cos^2\alpha} = \frac{1-\tan^2\alpha}{1+\tan^2\alpha}
\end{equation}

\subsection{辅助角公式}
\begin{equation}
a\sin\alpha + b\cos\alpha = \sqrt{a^2+b^2}\sin(\alpha + \phi)
\end{equation}
注: $\tan\phi = \frac{b}{a}$

\subsection{证明}

可参考与本节内容相似的\autoref{TriEqv_sub1}~\upref{TriEqv}.

\subsubsection{两角和与两角差(旋转法)}

这一证法是先通过旋转法求出余弦的加法公式\autoref{HsAnTf_eq3} ,然后进行简单变换得到剩下的加法公式.思路来自П. М. Котельников的学位论文\footnote{据刘培杰工作室出版的《世界著名三角学经典著作钩沉 平面三角卷I》第22节.笔者按此名字搜索,并未找到出处,特此声明.}.

\begin{figure}[ht]
\centering
\includegraphics[width=14cm]{./figures/HsAnTf_2.pdf}
\caption{旋转法示意图.左图和右图表示的是同样的坐标系,圆都是同一个单位圆.右图中所有图形和点都围绕坐标系原点顺时针旋转了$\beta$,从而使得$B$点落在$x$轴上.} \label{HsAnTf_fig2}
\end{figure}


如\autoref{HsAnTf_fig2} 所示,在单位圆上取两个角(不一定是图示的锐角)$\alpha$和$\beta$,与单位元相交得交点$A$和$B$.由于是单位圆,故可知$A$的坐标为$\pmat{x_A, y_A}=\pmat{\cos\alpha, \sin\alpha}$,$B$的坐标为$\pmat{x_B, y_B}=\pmat{\cos\beta, \sin\beta}$.由此可计算线段$AB$的长度,或者准确来说,长度的平方:
\begin{equation}\label{HsAnTf_eq1}
\begin{aligned}
\abs{AB}^2 =& (x_A-x_B)^2+(y_A-y_B)^2\\
=& (\cos^2\alpha+\cos^2\beta-2\cos\alpha\cos\beta)+\\&(\sin^2\alpha+\sin^2\beta-2\sin\alpha\sin\beta)\\
=& 2\qty(1-\cos\alpha\cos\beta-\sin\alpha\sin\beta)
\end{aligned}
\end{equation}

注意这里利用了$\cos^2x+\sin^2x=1$恒等式,下面也一样.

接下来,把所有点和图形都围绕坐标原点,顺时针旋转$\beta$,得到右图.此时$A$的坐标变成了$\pmat{x'_A, y'_A}=\pmat{\cos(\alpha-\beta), \sin(\alpha-\beta)}$,$B$的坐标变成了$\pmat{x'_B, y'_B}=\pmat{1, 0}$.

同样地,计算线段$AB$的长度平方:
\begin{equation}\label{HsAnTf_eq2}
\begin{aligned}
\abs{AB}^2 =& (x'_A-x'_B)^2+(y'_A-y'_B)^2\\
=& \qty(\cos^2(\alpha-\beta)+1-2\cos(\alpha-\beta))+\sin^2(\alpha-\beta)\\
=& 2(1-\cos(\alpha-\beta))
\end{aligned}
\end{equation}

\autoref{HsAnTf_eq1} 和\autoref{HsAnTf_eq2} 应相等,比较它们的最后一步即可得\autoref{HsAnTf_eq3} .

计算$\cos\qty(\alpha-(-\beta))$即可得\autoref{HsAnTf_eq4} .将$\sin x=\cos(x-\pi/2)$代入这两个余弦加法公式,即可得到正弦加法公式\autoref{HsAnTf_eq5} 与\autoref{HsAnTf_eq6} .再代入$\tan x=\sin x/\cos x$即可得正切的加法公式\autoref{HsAnTf_eq7} 与\autoref{HsAnTf_eq8} .



\subsubsection{两角和与两角差(几何矢量证法)}
\begin{figure}[ht]
\centering
\includegraphics[width=7cm]{./figures/HsAnTf_1.png}
\caption{图示} \label{HsAnTf_fig1}
\end{figure}
设 $\alpha$、$\beta$ 对应的单位向量分别为 $a(\cos\alpha,\sin\alpha)$、$b(\cos(\alpha+\beta),\sin(\alpha+\beta))$\\
设 $a$ 与其垂直的单位向量 $m(-\sin\alpha,\cos\alpha)$ 为基向量,则
\begin{equation}
\begin{aligned}
b &= \cos\beta \cdot a + \sin\beta \cdot m \\
&= (\cos\beta \cos\alpha,\cos\beta \sin\alpha) + (-\sin\beta \sin\alpha,\sin\beta \cos\alpha) \\
&= (\cos\alpha \cos\beta-\sin\alpha \sin\beta,\sin\alpha \cos\beta + \cos\alpha \sin\beta)
\end{aligned}
\end{equation}
由此可得
\begin{equation}
\sin(\alpha+\beta) = \sin\alpha \cos\beta + \cos\alpha \sin\beta
\end{equation}
\begin{equation}
\cos(\alpha+\beta) = \cos\alpha \cos\beta - \sin\alpha \sin\beta
\end{equation}

用 $-\beta$ 代换 $\beta$ ,可得
\begin{equation}
\begin{aligned}
\sin(\alpha-\beta) &= \sin\alpha \cos(-\beta) + \cos\alpha \sin(-\beta)\\
&=\sin\alpha \cos\beta - \cos\alpha \sin\beta
\end{aligned}
\end{equation}
\begin{equation}
\begin{aligned}
\cos(\alpha-\beta) &= \cos\alpha \cos(-\beta) + \sin\alpha \sin(-\beta)\\
&=\cos\alpha \cos\beta + \sin\alpha \sin\beta
\end{aligned}
\end{equation}

由 $\tan\alpha = \frac{\sin\alpha}{\cos\alpha}$,得
\begin{equation}
\tan(\alpha+\beta) = \frac{\sin\alpha \cos\beta + \cos\alpha \sin\beta}{\cos\alpha \cos\beta - \sin\alpha \sin\beta}
\end{equation}
上下同除 $\cos\alpha\cos\beta$ ,得
\begin{equation}
\tan(\alpha+\beta) = \frac{\tan\alpha+\tan\beta}{1 - \tan\alpha\tan\beta}
\end{equation}

同理,可得
\begin{equation}
\tan(\alpha-\beta) = \frac{\tan\alpha - \tan\beta}{1 + \tan\alpha\tan\beta}
\end{equation}

\subsubsection{二倍角公式}
在两角和公式中,令 $\alpha = \beta$
\begin{equation}
\sin2\alpha = \sin\alpha \cos\alpha+\cos\alpha \sin\alpha = 2\sin\alpha \cos\alpha
\end{equation}
\begin{equation}
\cos2\alpha = \cos^2\alpha - \sin^2\alpha
\end{equation}
\begin{equation}
\tan2\alpha = \frac{2\tan\alpha}{1 - \tan^2\alpha}
\end{equation}

由 $\sin^2\alpha + \cos^2\alpha = 1$ ,可得
\begin{equation}
\cos^2\alpha = 1 - \sin^2\alpha
\end{equation}
\begin{equation}
\sin^2\alpha = 1 - \cos^2\alpha
\end{equation}

代入,可得
\begin{equation}
\cos2\alpha = 1 - \sin^2 - \sin^2 = 1 - 2\sin^2\alpha
\end{equation}
\begin{equation}
\cos2\alpha = \cos^2\alpha - (1 - cos^2\alpha) = 2\cos^2\alpha - 1
\end{equation}

\subsubsection{半角公式}
由余弦的二倍角公式,可得
\begin{equation}
2\sin^2\alpha = 1 - \cos2\alpha
\end{equation}
\begin{equation}
2\cos^2\alpha = 1 + \cos2\alpha
\end{equation}

用 $\frac{\alpha}{2}$ 代还 $\alpha$ ,可得
\begin{equation}
\begin{aligned}
2\sin^2\frac{\alpha}{2} &= 1 - \cos\alpha\\
\sin\frac{\alpha}{2}&= \pm\sqrt{\frac{1-\cos\alpha}{2}}
\end{aligned}
\end{equation}
\begin{equation}
\begin{aligned}
2\cos^2\frac{\alpha}{2} &= 1 + \cos\alpha\\
\cos\frac{\alpha}{2} &= \pm\sqrt{\frac{1+\cos\alpha}{2}}
\end{aligned}
\end{equation}
