% 深度学习 CNN 入门 2
% keys 卷积
% license Usr
% type Wiki


\pentry{卷积\nref{nod_Conv},神经网络\nref{nod_NN},全连接网络\nref{nod_FCNN},Python 导航\nref{nod_PyFi}}{nod_c889}

本文从更基础的部分来解释CNN神经网络,上文是说的神经网络的基本结构,这篇解释的是CNN的底层逻辑。

首先需要了解一下卷积的概念。
定义:在泛函分析中,卷积是通过两个函数$f$和$g$生成第三个函数的一种数学运算,其本质是一种特殊的积分变换,表征函数$f$与$g$经过翻转和平移的重叠部分函数值乘积对重叠长度的积分。
数学表达式:对于连续函数,卷积的数学表达式通常为
\begin{equation}
(f*g)(t)=\int_{-\infty}^{+\infty}f(\tau)g(t-\tau)~.
\label{juanji}
\end{equation}
在CNN中,卷积操作主要用于特征提取。它通过滑动一个称为“卷积核”(或“滤波器”)的小型矩阵窗口在输入数据(如图像)上,进行元素级别的乘法并求和,从而生成新的特征图(Feature Map)。

假设我们有一个简单的3x3的输入矩阵(图像的一个局部区域)和一个2x2的卷积核:
输入矩阵(Input Matrix):
\begin{table}[ht]
\centering
\caption{输入矩阵F}\label{tab_CNN21}
\begin{tabular}{|c|c|c|}
\hline
1 & 0 & 1 \\
\hline
2 & 1 & 0 \\
\hline
0 & 1 & 1 \\
\hline
\end{tabular}
\end{table}
卷积核(Kernel)(只是方便演示,一般卷积核是奇数,方便确定中心):
\begin{table}[ht]
\centering
\caption{卷积核(Kernel)G}\label{tab_CNN22}
\begin{tabular}{|c|c|}
\hline
1 & 0 \\
\hline
0 & 1 \\
\hline
\end{tabular}
\end{table}
卷积操作的过程如下:

1.定位初始位置:首先,将卷积核放置在输入矩阵的左上角(或指定的起始位置)。
进行元素级乘法并求和:将卷积核中的每个元素与输入矩阵中对应位置的元素相乘,然后将所有乘积相加。在本例中,卷积核与输入矩阵左上角2x2区域的乘积之和为 1∗1+0∗2+0∗1+1∗0=1。这里注意对应的坐标是不同的,是

2.滑动卷积核:按照指定的步长(Stride)将卷积核向右滑动(通常是1个单位),然后重复步骤2,直到卷积核无法再向右滑动为止。之后,将卷积核回到最左边,向下移动一个步长,继续向右滑动,直到遍历完整个输入矩阵。

3.记录输出:每次卷积操作的结果都会被记录下来,形成新的特征图的一个元素。在本例中,由于步长为1且没有填充(Padding),所以输出特征图的大小会比输入矩阵小(具体取决于卷积核大小、步长和填充方式)。

步长(Stride):卷积核在输入矩阵上滑动的距离。步长越大,输出特征图越小。
填充(Padding):在输入矩阵的边界外添加额外的值(通常是0),以便在卷积过程中保持输入和输出的尺寸相同或按预期变化。
卷积核大小:卷积核的维度决定了每次卷积操作覆盖的输入矩阵的区域大小,也影响了特征图的尺寸和提取的特征类型。
​
最后输出是:\begin{table}[ht]
\centering
\caption{输出矩阵}\label{tab_CNN23}
\begin{tabular}{|c|c|c|}
\hline
2 & * & * \\
\hline
* & * & * \\
\hline
* & * & * \\
\hline
\end{tabular}
\end{table}
