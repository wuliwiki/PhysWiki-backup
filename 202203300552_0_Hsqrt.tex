% 手动计算开根号(长除法)

\begin{example}{}
\begin{figure}[ht]
\centering
\includegraphics[width=6cm]{./figures/Hsqrt_1.pdf}
\caption{手动开根号的例子, 计算 $\sqrt{21} = 4.582\dots$} \label{Hsqrt_fig1}
\end{figure}
作为一个例子, 我们来计算 $\sqrt{21}$. 首先试一个最大的个位数, 使它的平方恰好略小于 $21$, 易得 $4$.把 $4$ 分别写到根号左边和上方, 相乘得 $16$ 写到 $21$ 下方. 现在计算 $21-16 = 5$, 写到下方并在后面添加两个零得 $500$. 接下来把最上方的 $4$ 乘以二写到第二个根号左边, 并试一个橙色的最大一位数 $x$ 使得 $8x\times x$ 同样略小于 $500$, 易得 $x = 5$, $85\times 5 = 425$. 继续用 $500-425$ 并在后面加两个零得 $7500$. 把当前最上方的两位数 $45$ 乘以二得 $90$ 写到第三个根号左边, 再试一位绿色的数 $x$, 使 $90 x\times x$ 略小于 $7500$, 易得 $x = 8$, $908\times 8 = 7264$. 再算 $7500-7264$, 添加两个零得 $23600$. 把最上方所得三位数 $458$ 乘以二写到第四个根号左边, 试一位蓝色的数 $x$ 使 $916x\times x$ 略小于 $23600$, 得 $x = 2$. 以此类推, 就可以精确到任意位小数, 即 $\sqrt{21} = 4.582\dots$.
\end{example}

\subsubsection{推导}
若我们要算 $s^2$ 的开根号, 并假设 $s$ 的 $n$ 位有效数字(不做四舍五入)为 $s_n$, 例如 $s = \sqrt{2} = 1.414213562\dots$, 则 $s_1 = 1$, $s_2=1.4$, $s_3=1.41$,…… 那么显然有
\begin{equation}
s^2 = s_1^2 + (s_2+s_1)(s_2-s_1) + (s_3+s_2)(s_3-s_2) + \dots
\end{equation}
现在, 第 $i$ 位有效数字(小数点位置不变)为 $d_i = s_i-s_{i-1}$, 且 $d_1 = s_1$, 易得
\begin{equation}
s^2 = d_1^2 + (2s_1 + d_2)d_2 + (2s_2 + d_3)d_3 + \dots
\end{equation}
现在, 若已知 $s^2$, 我们就可以先试出最大的 $d_1$, 满足 $d_1^2\leqslant s^2$. 然后再试出最大的 $d_2$, 满足 $(2s_1 + d_2)d_2 \leqslant s^2 - d_1^2$, 以此类推就可以求出任意位的小数.
