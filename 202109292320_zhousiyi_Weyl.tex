% Weyl旋量
% Weyl|旋量|费米子

洛仑兹群的狄拉克表示是可约的.我们可以构造两个二维表示
\begin{equation}
\psi = \begin{pmatrix}
\psi_L \\
\psi_R
\end{pmatrix}~.
\end{equation}
$\psi_L$被称为左手的Weyl旋量,$\psi_R$被称为右手的Weyl旋量.在无穷小转动$\mathbf \theta$和boost $\mathbf \beta$下, 它们的变换规则为
\begin{align}
\psi_L \rightarrow (1-i \boldsymbol \theta \cdot \frac{\boldsymbol \sigma}{2} - \boldsymbol \beta \cdot \frac{\boldsymbol \sigma}{2})\psi_L ~, \\
\psi_R \rightarrow (1-i\boldsymbol \theta \cdot \frac{\boldsymbol \sigma}{2}+ \boldsymbol \beta \cdot \frac{\boldsymbol \sigma}{2})\psi_R ~.
\end{align}
下面这个恒等式很有用
\begin{equation}
\sigma^2\boldsymbol \sigma^* = - \boldsymbol \sigma \sigma^2~.
\end{equation}
不难证明$\sigma^2\psi^*_L$像右手旋量一样变换.用$\psi_L$和$\psi_R$,我们可以把狄拉克方程写为如下的形式
\begin{equation}
(i\gamma^\mu\partial_\mu - m)\psi = \begin{pmatrix}
- m & i (\partial_0+\boldsymbol \sigma \cdot \boldsymbol \nabla) \\
i(\partial_0-\boldsymbol\sigma\cdot \boldsymbol\nabla) & -m 
\end{pmatrix} \begin{pmatrix}
\psi_L \\ \psi_R 
\end{pmatrix}=0~.
\end{equation}
从上式可以看出,两个洛仑兹群的表示$\psi_L$和$\psi_R$在狄拉克方程中通过质量项进行混合.如果我们把$m$置成0,那么关于$\psi_L$和$\psi_R$的两个方程就分开了
\begin{align}
i(\partial_0 - \boldsymbol \sigma \cdot \boldsymbol \nabla) \psi_L = 0~, \\
i(\partial_0 + \boldsymbol \sigma \cdot \boldsymbol \nabla) \psi_R = 0~. 
\end{align}
这两个方程被称为Weyl方程.在处理\textbf{中微子物理}和\textbf{弱相互作用的物理}的时候尤为重要.
