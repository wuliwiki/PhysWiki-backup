% 正交函数系 2
% 正交;函数系;贝塞尔不等式
\pentry{定积分(简明微积分)\upref{DefInt}}
\begin{issues}
\issueDraft
\end{issues}

\footnote{$\Gamma$. M. 菲赫金哥尔茨.微积分学教程\;卷三[M].北京:高等教育出版社,2006:345-349}
\footnote{斯米尔诺夫.斯米尔诺夫高等数学卷四第一分册[M].哈尔滨:哈尔滨工业大学出版社,2018:6-12}
若函数系 
\begin{equation}\label{OFS_eq1}
\qty{\varphi_n(x)}\quad(n\in \mathbb{N})
\end{equation}
中各函数及其平方在区间 $\qty[a,b]$ 上皆可积分,且满足
\begin{equation}
\int_a^b\varphi_n(x)\varphi_m(x)\dd x=
\left\{
    \begin{aligned}
    &0\quad &m\neq n\\
   &\lambda_n>0\quad &m=n
    \end{aligned}
\right.
\end{equation}
则称函数系$\qty{\varphi_n(x)}$ 为\textbf{正交函数系}.当 $\lambda_n=1(n\in\mathbb{N})$ 时,该函数系称为\textbf{正交规范系}(或\textbf{正交标准系}).显然,任意的正交函数系都可化为正交规范系 
\begin{equation}
\qty{\frac{\varphi_n(x)}{\sqrt{\lambda_n}}}\quad(n\in \mathbb{N})
\end{equation}

设 $f(x)$ 是任一实函数,在区间 $[a,b]$ 内是连续的,则数值
\begin{equation}
c_k=\frac{1}{\lambda_k}\int_a^b f(x)\varphi_k(x)\dd x\quad (k\in \mathbb{N})
\end{equation}
称为函数 $f(x)$ 关于函数系\autoref{OFS_eq1} 的\textbf{傅里叶系数}.由 $c_k$ 的定义,我们有等式
\begin{equation}\label{OFS_eq2}
\int_a^b\qty[f(x)-\sum_{k=1}^nc_k\varphi_k(x)]^2\dd x=\int_a^b\qty[f(x)]^2\dd x-\sum_{k=1}^nc_k^2\lambda_k
\end{equation}

根据\autoref{OFS_eq2} ,可得
\begin{equation}\label{OFS_eq3}
\sum_{k=1}^\infty c_k^2\lambda_k\leq\int_a^b\qty[f(x)]^2\dd x
\end{equation}
对于正交规范系, $\lambda_k=1$,此时\autoref{OFS_eq3} 便是所谓的\textbf{贝塞尔(bessel)不等式}. 

若对任何在区间 $[a,b]$ 定义的连续函数 $f(x)$,\autoref{OFS_eq3} 中的等号成立,则称函数系 $\qty{\varphi_n(x)}$ 是\textbf{完整的}(或\textbf{完备的}).
%\begin{example}{三角函数系}
%%\begin{equation}
%\qty{1,\cos nx,\sin nx}\quad(n\in \mathbb{Z^{+}})
%%\end{equation}
%是正交函数系.
\subsection{例子}
可以证明:
\begin{enumerate}
\item 在区间 $[-\pi,\pi]$ 上,三角函数系
\begin{equation}
\qty{1,\cos nx,\sin nx}\quad(n\in \mathbb{Z^{+}})
\end{equation}
是正交函数系.
\item 由超越方程
\begin{equation}
\tan\xi=c\xi \quad(c\; \mathrm{is\; a\; constant})
\end{equation}
 的所有正根组成的集合$\qty{\xi_n\vert n\in\mathbb{Z}^+}$
 构造的函数系
 \begin{equation}
 \sin\frac{\xi_1}{l}x,\sin\frac{\xi_2}{l}s,\cdots,\sin\frac{\xi_n}{l}x,\cdots
 \end{equation}
 在区间 $\qty[0,l]$ 上是正交函数系.\\
 
 同样,以超越方程
\begin{equation}
\cot\xi=c\xi \quad(c\; \mathrm{is\; a\; constant})
\end{equation}
 的所有正根组成的集合$\qty{\xi'_n\vert n\in\mathbb{Z}^+}$
 构造的函数系
 \begin{equation}
 \cos\frac{\xi'_1}{l}x,\cos\frac{\xi'_2}{l}s,\cdots,\cos\frac{\xi'_n}{l}x,\cdots
 \end{equation}
 在区间 $\qty[0,l]$ 上也是正交函数系.
 \item 勒让德多项式
\begin{equation}
X_0(x)=1,X_n(x)=\frac{1}{2^nn!}\dv[n]{\qty(x^2-1)^n}{x}\quad(n=1,2,\cdots)
\end{equation}
是在区间 $\qty[-1,1]$ 是正交函数系,且 $\lambda_n=\frac{2}{2n+1}$
\item 以贝塞尔函数 $J_0(x)$ 的所有正根构成的集合$\qty{\xi_n\vert n\in\mathbb{Z}^+}$
构造的函数系$\qty{\sqrt{x}J_0(\xi_n x)} $在区间 $\qty[0,1]$ 上是正交函数系.
\end{enumerate}

\subsection{线性无关函数的正交化}
在线性代数里面,通过斯密特正交化\upref{SmdtOt}手续,可由 $N$ 个线性无关的向量构造出同样多个两两正交且标准的向量,使原来的向量可由新向量线性表出.这一切对函数来说完全适用.

$N$ 个在区间 $[a,b]$ 上定义的连续函数 $\psi_i(x)\quad (i=1,\cdots ,N)$ 称为在区间 $[a,b]$ 线性无关的,若含常系数 $a_i$ 的关系式
\begin{equation}
\sum_i^N a_i\psi_i(x)=0
\end{equation}
仅当 $a_i=0$ 时成立.现在来作在区间 $[a,b]$ 上可由 ${\psi_i(x)}$ 线性表示的正交标准化的函数${\varphi_i(x)}$.

记(显然,这里 $\ev{*,*}$ 相当于矢量空间中的内积)
\begin{equation}
\ev{f,F}=\int_a^b f(x)F(x)\dd x
\end{equation}

函数 ${\psi_i(x)}$ 的正交化过程可按如下方式进行(为简洁起见,将函数 $f(x)$ 简记为 $f$)
\begin{equation}
\begin{aligned}
&&\varphi_1=\frac{\psi_1}{\sqrt{\ev{\psi_1,\psi_1}}}\\
\chi_2&=\psi_2-\ev{\psi_2,\varphi_1}\varphi_1,\quad&\varphi_2=\frac{\chi_2}{\sqrt{\ev{\chi_2,\chi_2}}}\\
\chi_3&=\psi_3-\ev{\psi_3,\varphi_2}\varphi_2-\ev{\psi_3,\varphi_1}\varphi_1,\quad &\varphi_3=\frac{\chi_3}{\sqrt{\ev{\chi_3,\chi_3}}}\\
&\vdots &\vdots
\\
\chi_N&=\psi_N-\sum_{i=1}^N\ev{\psi_N,\psi_i}\varphi_i,\quad&\varphi_N=\frac{\chi_N}{\sqrt{\ev{\chi_N,\chi_N}}}
\end{aligned}
\end{equation}

在函数 $\chi_i$ 中,没有一个可变为恒等于0,即 $\ev{\chi_i,\chi_i}\neq 0$.因为比如说 $\chi_2=0$,则 $\varphi_1$ 与 $\psi_2$ 线性相关
\begin{equation}
\psi_2-\ev{\psi_2,\varphi_1}=0
\end{equation}
这归结为 $\psi_1$ 与 $\psi_2$ 之间线性相关,这与条件 $\psi_1$ 和 $\psi_2$ 的线性无关矛盾.