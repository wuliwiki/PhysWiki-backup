% 天津大学 2014 年考研量子力学
% keys 考研|天津大学|量子力学|2014

\subsection{ }
\begin{enumerate}
\item 已知氢原子的状态为$\varPsi (r,\theta,\varphi)=\frac{\varPsi (310)}{\sqrt{5}}+\frac{2\varPsi (211)}{\sqrt{5}}$,求主量子数、角动量和角动量第三分量的可能值、相应几率及平均值.
\item 写出电子和光的波粒二象性事实.
\item 磁矩为$\bar{\mu}=-g^{\bar{s}}_{n}$的电子位于均匀磁场$\vec{B}$中,求电子的能级.
\end{enumerate}
\subsection{ }
质量为$M$,频率为$\Omega$的二维谐振子,求:\\
\begin{enumerate}
\item 其哈密顿量,其能量本征态和能级;
\item 各能级的简并度;
\item 各能级的宇称.
\end{enumerate}
\subsection{ }
谐振子处于$\varPsi =\sqrt{\frac{1}{3}}\varPsi_{0}+\sqrt{\frac{2}{3}}\varPsi_{1}$态.
\begin{enumerate}
\item 求坐标的平均值(任意时刻的);
\item 求任意时刻的能量平均值,并解释结果.
\end{enumerate}