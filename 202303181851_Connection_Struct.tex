% 数学结构
% 结构


\subsection{引言}
一堆砖块,如果只是散乱的堆在一起,意义并不是很大。
然而,如果根据设计图,将砖块堆砌起来,让它们之间拥有各种关系(比如这块砖隶属于地基区域,为其它砖奠基),这些砖块就可能成为一个有组织的有机整体,一个具有功能与价值的建筑物。
在数学上,也是一样的道理。集合内的零散元素往往缺乏研究价值,需要我们赋予某种“结构”。
法国Bourbaki学派将结构作为数学的根源,并确定了三个母体结构:代数结构、拓扑结构和序结构。

\subsection{基本释义}
在汉语中,结构一词,可以指组成体系的各部分的组织与搭配方式,也可以指拥有这种各部分组织与搭配方式的体系。而数学上的结构也有类似的含义。
对全体实数的集合,我们往往默认实数间的加法与乘法(减法与除法作为它们的逆运算自然诱导出),但这些运算并不是理所当然的,我们可以考虑一个没有任何运算的纯数集。
此时,集合内除了散乱的数,没有别的东西。显然,这样的集合也没有太大的意义。这种情况下,我们无法问两个数加起来等于什么,无法问两个数之间的距离,因为这些东西都尚无定义。 

若对该集合赋予一些满足公理(如加法与乘法的交换律、结合律、分配律等)的运算,这些运算描述了数集中元素的关系,我们可以称赋予了这些运算的集合为一个代数结构。代数结构的例子有群、环、域等。
有时可能我们也会遇到一种说法:我们为集合赋予了一种“代数结构”。此时的”代数结构”与前文中“给定运算的集合”之释义显然不一样。在这种情况下,我们所说的“代数结构”,往往指赋予给集合的集合内元素的关系。
在实践中,数学结构的这两种含义往往可以通过语境与上下文判断。

又比如,对于一个集合A={纽约,莫斯科,巴黎},我们可以为该集合附加一个满足相应条件(比如必须大于0)的“度量”,或者说距离函数,将任意两个城市组成的二元组,例如(纽约,莫斯科)作为自变量,以一个数作为因变量。我们可以将这个数当作是这两个城市间的距离。换言之,度量(结构)使得我们可以询问任意两个元素间的距离。此时,集合A也就成为了一个(数学)结构,即度量空间。

最一般地说,”结构”就是满足一定公理的、附加给集合的\textbf一些数学对象,或者指附加了满足前述条件的数学对象的集合。

比方说,给一个数集
\begin{equation}
A = {0,1,2,3}
\end{equation}
定义一个新运算,称作“自定义映射”:d(x,y)
\begin{equation}
d:A * A  -> N
\end{equation}
此处*指笛卡尔积,N指自然数集。
假设d的具体计算定义为:d(第一个元素,任何元素)=第一个元素。
此时,我们可以认为集合A与结构(此处取“附加给集合的数学对象”的释义)d构成了一个”自定义空间“(A,d)。
一般而言,集合+(数学)结构=空间。常见的例子有\href{https://wuli.wiki/online/LSpace.html}{线性空间}、线性赋范空间、内积空间、n维欧几里得空间、希尔伯特空间、拓扑空间等。

\subsection{常见分类}
我们往往可以看到一些代数结构(Algebraic Structure) ,比如交换代数、结合代数、外代数、李代数等。

也可能会看见一些“几何结构”(Geometric Structure),比如n维欧几里得空间等。

代数结构常指被赋予关系与运算的集合,由这些关系与运算所能得到的结果,往往也就构成了一个“代数”,此时“代数”一词是作为结构而存在的(如上文的李代数)。

几何结构,最广泛地说,可以被视为定义了一些集合子集的分类,以及它们之间运算的规则。可选地,我们可以将一个集合S上的几何结构G看作满足(确定性质的)S的幂集的一个或多个子集。 

举例来说,欧几里得的平面几何,平面是一个点集,点、线、平面图形等都是平面的子集,它们自身也是点集。两点之间只有一连线之类的规则,定义了它们之间的关系与运算。

需要注意的是,这两种结构并不是泾渭分明的,例如,对于一个平面,我们可以将其视为一个欧几里得的几何结构,也可以将其视为一个解析几何的代数结构。

\subsection{举例说明}
对一个实数三元组的集合{(x,y,z)|x,y,z∈R},作为(其中是笛卡尔积)即,可以给其附加一个线性结构,一个欧式内积(结构),由内积诱导出度量(结构)与范数(结构),使得其成为一个实数域R上的3维欧几里得空间。 

又例如,对于一个(实)拓扑流形,赋予其$C^k$类微分结构,使其成为一个$C^k$类(实)微分流形。

\subsection{数学结构间的映射(1)——同构——引言}
本科阶段的数学基本都可以建立在集合论的基础之上,对于它们而言,集合与映射几乎就是最为重要的概念,赋予集合的结构,往往也就是一个映射(如内积,是一个双线性形式)。

当我们得到了数学结构,我们很自然地就会去考虑数学结构之间的映射。
我们先来考虑同构映射,它简称为同构(Isomorphism)。

假设你是一个程序员,你与老板之间存在着一个关系,即他发给你工资。
你无疑在全体程序员以及老板的集合P中。

接着我们构造一个映射f,把程序员集合中的元素映射到公务员集合(加上过国家财务机关)中的元素,要求这个映射是一一对应的,即——每个程序员只能被映射到一个公务员,且所有公务员都要被映射到。

此时,我们称f是一个双射。

假设老板被映射到国家财务机关,那么在映射之前,你与老板存在着一个“他发给你工资”的关系,在映射之后,你与老板的像——国家财务机关,依然存在着一个“他发给你工资”的关系;那么,这个关系——或者说这个结构,在这个映射中是被保持的。
此时,我们称f是一个同构,即保持结构的双射。

假设你去买车,看到一辆车C,我们来考虑另一辆新造的车D。
假设有一个同构f将C映射到D。(我们此时可将映射f视为一个施加给C的变换,它将C车变换为D车)

如果f不是双射,C的零件就无法与D的零件一一对应,这样以来,假如C的两个零件之间有一个关系,结果映射到D后,只有其中一个零件被映射过去了,此时又何谈保持这两个零件之间的关系呢。
如果f不保持结构,仅仅在两辆车的零件之间建立一一对应,那么C也许一切正常,然而D的发动机可能接到了车尾,根本发动不起来,甚至更严重的,完全丧失了C的结构,导致D只是一堆散乱的零件——你怎么开,用气功么。

\subsection{数学结构间的映射(2)——同构}
考虑两个线性空间V和W。

从V到W的同构映射,就是一个线性映射,或者,如果是从V映射到V,也可以叫做线性变换(将V中的元素E变换为另一个元素F)。是的,线性代数里的核心概念之一——线性映射,就是两个代数结构之间同构映射的一个例子。它保持了线性组合的结构(几何上将直线映射为直线,平面映射为平面),即如果V中的几个向量a、b、c有一个线性组合的关系——2a+3b-c,在映射后变成f(2a+3b-c)=2f(a)+3f(b)-f(c)。

f(a)、f(b)、f(c)是a,b,c被映射之后的像,。显然,a,b,c被映射后,其线性组合的形式依然不变,依旧是2d+3e-g的形式[其中d、e、g是代数项,代表任何向量,例如f(a)、f(b)],而不是别的,例如114d+514e+3.1415g。
