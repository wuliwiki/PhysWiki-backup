% 群表示
% 表示|群论|置换群|线性空间|同态

\begin{issues}
\issueOther{定义并不准确,需要修改.群表示是一种特殊的群作用,是线性空间上的线性作用}
\end{issues}


\pentry{群作用\upref{Group3},矢量空间\upref{LSpace},域上的代数\upref{AlgFie}}
\subsection{表示}

字面上讲,\textbf{表示(representation)}是用一个易于理解的表达方法来描述一些数学对象.比如说,对于学龄前的小朋友,想理解$1+1=2$ 的概念可能过于抽象,不利于理解,这个时候可以使用“一个苹果加一个苹果等于两个苹果”来表示相同的概念,会更容易理解.

数学上讲,\textbf{表示论}是研究对称性的学科,它把一个抽象的代数结构的元素映射成一个相对具体线性变换.其中群是表示论所研究的最简单的代数结构,这个学科就被称为\textbf{群表示论}.

\subsection{群的表示}

% Giacomo: 这什么鬼???
% 
% 最直观和易于讨论的群,莫过于置换群.事实上,当伽罗华(Galois)第一次提出群的概念的时候,并没有像我们现代理论那样高度抽象和严格;他主要都在讨论置换群的性质.遗憾的是,当年的数学家们都迷惑于伽罗华研究这种东西的意义何在.
% 
% 我们使用置换群来尝试表示任意的群.
% 
% \begin{definition}{群在置换群上的表示}
% 设有群 $G$ 和一个置换群 $S_n$.如果存在\textbf{同态}$\phi: G\rightarrow S_n$,那么我们称 $\phi$ 是群 $G$ 在 $S_n$ 上的一个\textbf{表示}.
% \end{definition}
% 
% 我们也可以使用线性空间来对群进行表示,利用线性空间的线性变换.

\begin{definition}{群的(线性)表示}
设有群 $G$ 和一个线性空间 $V$,记 $V$ 上的全体可逆线性变换为 $\opn{GL}(V)$\footnote{一般线性群\autoref{GL_def1}~\upref{GL}}.如果存在\textbf{同态}$\phi: G\rightarrow \opn{GL}(V)$,那么我们称 $\phi$ 是群 $G$ 在 $V$ 上的一个\textbf{群表示}.
\end{definition}

群表示是一种特殊的群作用\autoref{Group3_def1}~\upref{Group3},而且一个群作用可以诱导出一个与之相关的群表示.

\begin{definition}{形式代数}
设有域 $\mathbb{F}$ 和集合 $S$, 我们可以定义一个 $\mathbb{F}$-代数,其元素为 $S$ 中元素的形式线性组合,记做
$$
\mathbb{F}[S]: = \left\{ \sum a_i s_i \mid \text{有限个非零} a_i \right\}
$$
\end{definition}
\addTODO{名字不确定}

\begin{definition}{群作用诱导的群表示}
设有群 $G$ 到集合 $S$ 的群作用 $\rho: G \to \opn{Aut}(S)$,
\end{definition}