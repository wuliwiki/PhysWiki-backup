% 椭圆坐标系
\pentry{椭圆的三种定义\upref{Elips3}、正交曲线坐标系\upref{CurCor}}
\begin{issues}
\issueDraft
\end{issues}
\subsection{椭圆坐标系}
椭圆坐标系中点的位置可由 $(\xi,\eta,z)$ 这3个有序实数表示.$\xi$的等值曲面为一组共焦椭圆柱面族,焦距为 $2c$;$\eta$ 的等值曲面为一组共焦的双曲柱面族,其焦点与椭圆柱面族焦点相重.而 $z$轴一般与直角坐标系 $z$ 轴重合.

椭圆坐标与直角坐标的关系为
\begin{equation}\label{EliCor_eq1}
\left\{\begin{aligned}
&x=c\cosh\xi\cdot\cos\eta\\
&y=c\sinh\xi\cdot\sin\eta\\
&z=z
\end{aligned}\right.
\end{equation}
其中 $\xi\geq0,0\leq\eta<2\pi,-\infty<z<+\infty$
\addTODO{图}
\subsection{椭圆坐标系中的单位矢量}
\pentry{过渡矩阵\upref{TransM}}
容易证明椭圆坐标系是一个正交曲线坐标系.由\autoref{EliCor_eq1} 
\begin{equation}\label{EliCor_eq2}
\left\{
    \begin{aligned}
    &\dd x=c\sinh\xi\cdot\cos\eta\dd\xi-c\cosh\xi\sin\eta\dd\eta\\
    &\dd y=c\cosh\xi\cdot\cos\eta\dd\xi+c\sinh\xi\cdot\cos\eta\dd\eta\\
    &\dd z=\dd z
    \end{aligned}\right.
\end{equation}
令3个椭圆坐标 $\xi$、 $\eta$、 $z$ 对应的单位矢量为 $\uvec{\xi}$、 $\uvec{\eta}$、 $\uvec{z}$,则\autoref{EliCor_eq2} 结合\pentry{过渡矩阵\upref{TransM}}的知识,有
\begin{equation}
\left\{
    \begin{aligned}
    &\uvec{\xi}=c\sinh\xi\cdot\cos\eta\uvec{x}+c\cosh\xi\cdot\cos\eta\uvec{y}\\
    &\uvec{\eta }=-c\cosh\xi\sin\eta\uvec{x}+c\sinh\xi\cdot\cos\eta\dd\uvec{y}\\
    &\uvec{z}=\uvec{z}
    \end{aligned}\right.
\end{equation}


参考 wikipedia, 参考抛物线坐标系, 球坐标系等写法

写出梯度散度旋度, 引用 “正交曲线坐标系中的矢量算符\upref{CVecOp}”
