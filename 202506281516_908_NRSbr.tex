% 尼尔斯·玻尔(综述)
% license CCBYSA3
% type Wiki

本文根据 CC-BY-SA 协议转载翻译自维基百科\href{https://en.wikipedia.org/wiki/Niels_Bohr}{相关文章}。

尼尔斯·亨里克·戴维·玻尔(Niels Henrik David Bohr,美国发音:/boʊr/,英国发音:/bɔː/,\(^\text{[2]}\)丹麦语:[ˈne̝ls ˈpoɐˀ];1885年10月7日-1962年11月18日)是一位丹麦理论物理学家,他在原子结构和量子理论的理解方面作出了奠基性贡献,并因此于1922年获得诺贝尔物理学奖。玻尔同时也是一位哲学家和科学研究的推动者。

玻尔提出了著名的玻尔原子模型,他提出电子的能级是离散的,电子围绕原子核在稳定轨道上运行,但可以从一个能级(或轨道)跃迁到另一个能级(或轨道)。尽管玻尔模型已被其他模型取代,但其基本原理仍然有效。他提出了互补性原理:事物可以从相互矛盾的属性中被分别分析,例如表现为波动或粒子流的行为。这一互补性概念贯穿了玻尔在科学和哲学领域的思考。

玻尔在哥本哈根大学创立了理论物理研究所(现称尼尔斯·玻尔研究所),该研究所于1920年开放。玻尔指导并与多位物理学家合作,包括汉斯·克拉默斯、奥斯卡·克莱因、乔治·德·赫维希和沃尔夫冈·海森堡。他预测了一种类似锆的新元素的特性,该元素因在哥本哈根被发现而以哥本哈根的拉丁名称命名为“铪”。后来,合成元素“𬬻”因玻尔在原子结构领域的开创性工作而以他的名字命名。

在20世纪30年代,玻尔帮助了逃离纳粹主义的难民。丹麦被德国占领后,他会见了已成为德国核武器项目负责人的海森堡。1943年9月,玻尔得知德国人即将逮捕他,于是他逃往瑞典。从那里,他被空运到英国,加入了英国的“合金管”核武器项目,并作为英国代表团成员参与了曼哈顿计划。战争结束后,玻尔呼吁在核能领域开展国际合作。他参与了欧洲核子研究中心(CERN)和丹麦原子能委员会下属的里瑟研究机构的建立,并于1957年成为北欧理论物理研究所的首任主席。
\subsection{早年生活}
尼尔斯·亨里克·戴维·玻尔于1885年10月7日出生在丹麦哥本哈根,是克里斯蒂安·玻尔和妻子埃伦(娘家姓阿德勒,Ellen née Adler)的三个孩子中的老二。[3][4] 其父克里斯蒂安是哥本哈根大学的生理学教授,母亲埃伦出身于一个富裕的犹太银行世家。\(^\text{[5]}\)他有一个姐姐珍妮和一个弟弟哈拉尔。\(^\text{[3]}\)珍妮后来成为教师,\(^\text{[4]}\)而哈拉尔成为数学家和足球运动员,曾代表丹麦国家队参加1908年在伦敦举行的夏季奥运会。尼尔斯本人也是一名热情的足球运动员,两兄弟曾一起为位于哥本哈根的学术足球俱乐部效力,尼尔斯担任守门员。\(^\text{[6]}\)

玻尔七岁时进入加梅尔霍姆拉丁学校就读。[7] 1903年,玻尔进入哥本哈根大学本科就读,主修物理学,师从当时该校唯一的物理学教授克里斯蒂安·克里斯蒂安森。此外,他还在托瓦尔·蒂勒教授指导下学习天文学和数学,并在其父的朋友哈拉尔·霍夫丁教授指导下学习哲学。[8][9]
\begin{figure}[ht]
\centering
\includegraphics[width=6cm]{./figures/3736fdffe67b90d2.png}
\caption{} \label{fig_NRSbr_1}
\end{figure}
1905 年,丹麦皇家科学院举办了一项金质奖章竞赛,题目是研究测量液体表面张力的方法,该方法最初由瑞利勋爵于 1879 年提出。这项研究需要测量水射流半径振动的频率。玻尔在大学里利用他父亲的实验室进行了一系列实验;当时大学本身并没有物理实验室。为了完成实验,他不得不自己制作玻璃器皿,吹制出具有所需椭圆形横截面的试管。他不仅完成了原先的任务,还在瑞利的理论和方法上进行了改进,他考虑了水的黏滞性,并使用有限振幅而非仅限于无穷小振幅进行实验。他在最后一刻提交的论文赢得了这项奖章。他随后将改进后的论文提交给伦敦皇家学会,在《皇家学会哲学汇刊》上发表。[10][11][9][12]
