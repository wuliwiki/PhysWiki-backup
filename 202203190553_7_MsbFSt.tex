% 可测函数的结构
% 可测函数|实变函数|简单函数|simple function|Lusin定理

\pentry{Egoroff定理\upref{EgrfTh}}

从\autoref{MsbFun_the4}~\upref{MsbFun}可知可测函数的一个性质:其下方图形是可测集.

本节我们继续讨论可测函数具有哪些结构,来加深对可测函数的理解.

\subsection{简单函数}

\begin{definition}{}\label{MsbFSt_def1}
设$E$是$\mathbb{R}^n$上的一个可测集,$f$是其上一个函数.如果$E$能分解为\textbf{互不相交}的$N$个$E_k$,其中$k$取值为正整数,使得$E=\bigcup_{k=1}^N$,且$f$在各$E_k$上都是\textbf{常数},那么我们称$f$是一个$E$上的\textbf{简单函数(simple function)}.
\end{definition}

显然,简单函数都是可测函数.你甚至可能留意到了,简单函数可以直接定义Lebesgue积分了:如果$f$是\autoref{MsbFSt_def1} 中所说的那种简单函数,那么可以定义其Lebesgue积分为:

\begin{equation}
\int_E f \dd x = \sum_{k=1}^N f(x_k) \opn{m}(E_k)
\end{equation}
其中$x_k\in E_k$.

事实上,简单函数是定义Lebesgue积分的基础,这可以由以下定理得出:

\begin{theorem}{}\label{MsbFSt_the1}
设$E$是$\mathbb{R}^n$上的一个可测集,$f$是其上的\textbf{可测函数}.那么存在一列$E$上的\textbf{简单函数}$\{f_k\}_{k=1}^\infty$,使得$\lim\limits_{k\to\infty}f_k=f$在$E$上处处成立.
\end{theorem}

\textbf{证明}:

定义数列$\{a_{k, i}\}$为$a_{k, i}=\arctan(\frac{\pi}{2}\cdot\frac{i}{k})$,其中$k$是正整数,$i$是$(-k, k)$中的整数.容易发现,固定$k$时,$a_{k, i}$可以将整个$\mathbb{R}$划分为$2k+1$份.

由于$f$是可测函数,故$f^{-1}([a_{k, i}, a_{k, i+1}))$是可测集.定义$E$上的函数$f_k$如下:$f_k(x)=a_{k, i}$,如果$f(x)\in [a_{k, i}, a_{k, i+1})$.

则各$f_k$都是简单函数,并且$\lim_{k\to\infty}f_k=f$在$E$上处处成立.

\textbf{证毕}.

简单函数都是可测函数,加上\autoref{MsbFSt_the1} ,我们就得到可测函数的另一个定义:可测函数都是简单函数列的极限.很多与可测函数有关的问题都可以通过讨论简单函数再取极限得到可测函数来解决.



\subsection{Lusin定理}

\begin{definition}{连续性}
设$E$是$\mathbb{R}^n$上的一个可测集,$f$是其上的\textbf{可测函数}.如果在$x_0\in E$处,对于任何$\e$
\end{definition}





















