% 华东师范大学 2015 年 考研 量子力学
% license Usr
% type Note

\textbf{声明}:“该内容来源于网络公开资料,不保证真实性,如有侵权请联系管理员”

\subsection{选择题}
\begin{enumerate}
\item 对于库仑势形式的中心力场,动能、势能平均值之间应有如下关系
$A) \ \vec{T} = -\vec{V} \quad ; \quad B) \ \vec{T} = -2\vec{V} \quad ; \quad 
C) \ \vec{V} = \vec{T} \quad ; \quad D) \ \vec{V} = -2\vec{T} $
\item 若 $a^{\dagger}$ 和 $a$ 分别是谐振子的升降算符,变换 $e^{\lambda a^{\dagger}} a e^{-\lambda a^{\dagger}}$ 的计算结果是($\lambda$ 为常数):
A) $a + \lambda$ ; B) $a - \lambda$ ; C) $a e^{\lambda}$ ; D) $a e^{-\lambda}$ 。
\item 下列哪种物理实验或现象与自旋无关?\\
A)斯特恩-盖拉赫实验;C)斯塔克效应:B)精细结构:D)反常塞曼效应。
\item 考虑三个自旋为1/2的非全同粒子组成的体系,哈密顿量为
$$\hat{H} = A \vec{S}_1 \cdot \vec{S}_2 + B (\vec{S}_1 + \vec{S}_2) \cdot \vec{S}_3~$$(A 和 B 为两个实常数)
问下列哪个不属于该体系的能量本征值?\\
A) $\left(\frac{A}{4} - \frac{B}{2}\right)\hbar^2 \quad B \left(\frac{A}{4} + \frac{B}{2}\right)\hbar^2$\\
C) $\left(\frac{A}{4} - B\right)\hbar^2 \quad D)  -\frac{3}{4}A \hbar^2$
\item 三个全同电子处于一个圆频率为 $\omega$ 的一维谐振子势场中,忽略它们之间的相互作用。假设任一电子所处的本征波函数为 $\psi_{n\sigma }= \varphi_{n}(x) \chi_{\sigma}$,其中 $\varphi_{n}(x)$ 表示一维谐振子的本征态($n = 0,1,2,3$,...),$\chi_{\sigma}$ 表示电子自旋 $z$ 分量 $s_z$ 的本征态,$\sigma = \pm \frac{1}{2}$。该系统的基态能量是:\\
A) $\frac{3}{2}\hbar\omega \quad B)2\hbar\omega$\\
C) $\frac{5}{2}\hbar\omega \quad D) 3\hbar\omega$
\end{enumerate}
\subsection{(本题 22分)}
质量为 $m$ 的粒子处于一维势场
$$V(x) = \begin{cases} \infty & x \geq a \\\\0 & 0 < x < a \\\\\infty & x = 0 \\\\0&-a < x < 0 \\\\\infty & x \leq -a \\\\\ \end{cases}~$$
中,求其定态本征能量与本征波函数。
\subsection{(本题 20分)}
两个质量同为 $m$ 的一维谐振子,限制在 $x$ 轴上运动,圆频率均为 $\omega$,振子之间的耦合作用势为 $\lambda m \omega^2 (x_1 - x_2)^2$,这里 $\lambda$ 为一正常数。两个振子的平衡位置分别位于 $a$ 和 $-a$ 处。求该体系的本征能量。