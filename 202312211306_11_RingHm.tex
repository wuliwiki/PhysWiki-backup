% 环同态
% keys 环|同态|环同态|环同态基本定理
% license Xiao
% type Tutor

\begin{issues}
\issueDraft 
1.缺定理及证明。2.需要作图插入。
\end{issues}

\pentry{环的理想和商环\upref{Ideal},}

环同态的概念和群同态是类似的,只不过我们需要兼顾两个运算的性质。


\begin{definition}{环同态}
给定环 $R_1$ 和 $R_2$,如果存在映射 $f:R_1\rightarrow R_2$,使得对于任意的 $r, s\in R_1$ 都有 $f(r)+f(s)=f(r+s)$ 和 $f(rs)=f(r)f(s)$,那么我们称 $f$ 是一个从 $R_1$ 到 $R_2$ 的\textbf{环同态映射(ring homomorphism)},而环 $R_1$ 和 $R_2$ 是\textbf{同态(homomorphic)}的环。
\end{definition}

\begin{definition}{环同构}
如果一个环同态是双射,那么我们称它为一个\textbf{环同构(ring isomorphism)},相应的两个环就是\textbf{同构(isomorphic)}的。记“$R_1$ 和 $R_2$ 同构”为 $R_1\cong R_2$。
\end{definition}
\begin{exercise}{验证下列映射是否为环同态}
\begin{enumerate}
\item $f:\mathbb R^{2\times 2}\rightarrow \mathbb R,\quad f(A)=A^i_i,\quad \forall A\in \mathbb R^{2\times 2}$
\item $f:\mathbb R^{2\times 2}\rightarrow \mathbb R,f(A)=\opn {Tr}A,\quad \forall A\in \mathbb R^{2\times 2}$
\item $f:\mathbb R^{2\times 2}\rightarrow \mathbb R,f(A)=\opn {det}A,\quad \forall A\in \mathbb R^{2\times 2}$
\end{enumerate}
\end{exercise}
\begin{exercise}{环同态的基本性质}
\begin{enumerate}
\item 证明:环同态把单位元映射为单位元(乘法单位元\textbf{1}和加法单位元\textbf{0})。
\item 证明:环同态把子环映射为子环。
\item 证明:$f(-r)=f(r)$
\item 证明:若$r$和$f(r)$非零,则有$f(r^{-1})=f(r)^{-1}$
\end{enumerate}
\end{exercise}
同构的两个环具有完全相同的运算结构,而同态的两个环的运算结构似而不同,和群的同态、同构类似。

\subsection{环同态基本定理}\pentry{群同态基本定理:\autoref{exe_Group2_1}~\upref{Group2}}

\begin{definition}{核}
给定环同态 $f:R_1\rightarrow R_2$,记 $\opn{ker}f=\{r\in R_1|f(r)=0\}$,称为同态 $f$ 的\textbf{核}。
\end{definition}

从定义可见,环同态首先是加法群的同态,因此同态映射的核是原环的正规子群。实际上,$\opn{ker} f$还是$R_1$的\textbf{理想},只需验证“左右吸收律”成立即可。
和群同态基本定理类似,我们有如下环同态基本定理:

\begin{theorem}{第一同态定理}
给定\textbf{满射}的环同态:$f:R_1\rightarrow R_2$,记$K= \opn{ker}f$,则有:
\begin{itemize}
\item $R_1/K\cong R_2$,且存在同构映射$g:R_1/K\rightarrow R_2$,使得交换图成立;
\item 取$R_1$中包含$K$的\textbf{子环},则$f(B)$也是子环,且$B$的左陪集与$f(B)$的左陪集一一对应;
\item 取$R_1$中包含$K$的\textbf{理想}$I$,则$R_1/I \cong R_2/f(I)$
\end{itemize}
\end{theorem}
Proof.

首先是第一条,摘去环的乘法结构后这就是群同态基本定理。因此假设同构映射就是$f$,可证$f(r+K)=f(r),\forall r\in R_1$。也就是说,$K$的每个左陪集映射到$R_2$上的一个点。同构映射是保运算结构不变的双射,因此先证明运算保加法和乘法:
\begin{equation}
\begin{aligned}
f(r_1 +K)+f(r_2 +K)&=f(r_1)+f(r_2)=f(r_1+r_2+K).\\
f(r_1 +K)f(r_2 +K)&=f(r_1 r_2+K)~.
\end{aligned}
\end{equation}
由于该环同态是满射,所以相应的同构映射也是满射。下面证明这还是单射。
\begin{equation}
\begin{aligned}
f(r_1+K)=f(r_2+K)&\Leftrightarrow f(r_1)=f(r_2)\\
&\Leftrightarrow  f(r_1-r_2)=0\\
&\Leftrightarrow r_1-r_2\in K\\
&\Leftrightarrow r_1+K=r_2+K~.
\end{aligned}
\end{equation}
然后我们证明第二条。

由于环同态先是群同态,因而要证明$f(B)$是子环,只需证明对乘法封闭。
\begin{theorem}{第二同态定理}

\end{theorem}

\begin{theorem}{第三同态定理}

\end{theorem}



