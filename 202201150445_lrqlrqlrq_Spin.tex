% 自旋角动量
% 自旋|角动量|对易关系

\begin{issues}
\issueDraft
\end{issues}

\pentry{轨道角动量\upref{QOrbAM}, 张量积空间\upref{DirPro}}

刚体在经典力学中有两种角动量:1.轨道角动量 $(\bvec L = \bvec r\cross\bvec p)$,也就是物体的质心围绕原点运动(公转)所产生的.2.自转角动量 $(\bvec S = \bvec I\bvec\omega)$,也就是刚体绕其质心(自转)运动所产生的.量子力学中,除了像氢原子这样由电子围绕原子核运动所产生的由球谐函数所描写的电子轨道角动量之外,电子还有着另一种自旋角动量.不过这一自旋角动量和经典力学中的有着本质的区别,因为这一角动量是和空间无关的,也就是不能被坐标 $r,\theta,\phi$ 所描述,因此我们也称其为基本粒子所固有的内禀角动量 $S$. 

\subsection{斯特恩–格拉赫实验}
\begin{figure}[ht]
\centering
\includegraphics[width=12cm]{./figures/Spin_1.pdf}
\caption{斯特恩–格拉赫实验示意:银原子在一个不均匀的磁场中运动,并根据它们的自旋而上下偏转;(1)热熔Furnace,(2)银原子束,(3)不均匀磁场,(4)经典预期结果,(5)观测结果} \label{Spin_fig1}
\end{figure}

物理学的理论发展往往是由实验和现象所推动的,因此再直接进入抽象的自旋理论之前,我们可以先了解最初斯特恩–格拉赫实验实验观察到的观测结果与经典预期结果“异常”现象,如\autoref{Spin_fig1} 所示\footnote{参考 Wikipedia \href{https://en.wikipedia.org/wiki/Stern–Gerlach_experiment}{相关页面}.}.在\textbf{斯特恩–格拉赫(Stern-Gerlach)}实验中,一束银原子朝一个方向(我们设这个方向为 $\bvec y$ 轴方向)发射并穿过不均匀的磁场.除了有力矩(类比经典电磁理论中的\upref{EBLoop}),还有另外一个力作用在磁偶极子上(类比经典电磁理论中的\upref{EBTorq}):
\begin{equation}\label{Spin_eq1}
\bvec F=\nabla(\bvec \mu\cdot \bvec B)
\end{equation}
这个力可以用来分离具有特定自旋指向的粒子.假设一束较重的中性原子(例如该实验中的银原子)沿 $z$ 轴方向通过一个非均匀磁场区域——比如说
\begin{equation}\label{Spin_eq2}
\bvec B(x,y,z)=-\alpha x \hat{\bvec x}+(B_0+\alpha z)\hat {\bvec z}
\end{equation}
其中 $B_0$ 是一个较强的均匀场,而 $\alpha$ 描述对均匀性的一个小的偏离(实际上我们只需要 $z$ 方向上一个小的偏离,但不幸的是这违背了 $\nabla\cdot \bvec B=0$,所以必须有 $x$ 分量出现).作用在原子上的力为
\begin{equation}
\bvec F=\gamma \alpha (-S_x \hat{\bvec x}+S_z\hat{\bvec z})
\end{equation}
这里的 $S_x$ 表示原子自旋角动量 $x$ 分量的期待值,$S_z$ 表示自旋角动量 $z$ 分量的期待值.由于电子的磁矩和自旋角动量成正比,$\gamma$ 表示这一比例系数(称为磁旋比\upref{BohMag}.不同于经典理论的 $q/2m$,这里的 $\gamma$ 实际上是 $q/m$\upref{BohMag}).

但由于绕 $\bvec B_0$ 的拉莫尔(Larmor)进动,$S_x$ 快速振荡,且平均值为 $0$.所以净力是沿 $z$ 轴方向的.
\begin{equation}
F_z=\gamma\alpha S_z
\end{equation}

原子束穿过磁场受到偏转后打在检测器屏幕上.经典上,我们预期的结果是一条模糊带($S_z$ 没有量子化).但实验结果却是几个分立的模糊的点(在银原子实验中,屏幕上出现两个点).原子束分成了 $2s+1$ 个分立的束,在 Stern-Gerlach 实验中 $s=1/2$.这是因为在银原子中,原子内层的所有电子都是配对的,它们的轨道和自旋角动量都相互抵消.所以净自旋就是最外层一个非配对电子的自旋:$1/2$.

接下来我们将用量子力学自旋角动量的原理来解释这一现象.量子力学中的“力”是没有位置的,因此我们将随着银原子束一同运动的参照系中
来研究这个过程.在这个参照系中,哈密顿的初值为零,当粒子经过磁场的一段时间$T$内不为零,然后再归于零,也就是有:
\begin{equation}
H(t)=-\gamma(B_0+\alpha z)S_z, \quad 0\leq t\leq T
\end{equation}
在其他时间段内,哈密顿量都是零:
\begin{equation}
H(T)=0, \quad t<0,t>0
\end{equation}
我们假定银原子的自旋为$1/2$并且在$t\leq 0$时有初始态:
\begin{equation}
\chi(t)=a\chi_++b\chi_-
\end{equation}
然后当哈密顿算符作用时,$\chi(t)$将随时间演化:
\begin{equation}
\chi(t)=a\chi_+\E^{\I E_+ t/\hbar} + b\chi_-\E^{-\I E_- t/\hbar}
\end{equation}
由
\begin{equation}
H=-\gamma\bvec B\cdot \bvec S
\end{equation}
可得
\begin{equation}
E_\pm = \mp \gamma(B_0+az)\frac{\hbar}{2}
\end{equation}
这样我们就得到了当$t\geq T$时,银原子的态为:
\begin{equation}
\chi(t)=(a\E^{\I\gamma TB_0/2}\chi_+)\E^{\I(\alpha\gamma T/2)z} + (b\E^{\I\gamma TB_0/2}\chi_-)\E^{-\I(\alpha\gamma T/2)z} 
\end{equation}
这两项含有沿$ z $轴方向的动量,其中自旋向上部分的动量为:
\begin{equation}
p_z=\frac{\alpha\gamma T\hbar}{2}
\end{equation}
方向沿$ z $轴正方向;自旋向下部分的动量恰好相反,方向沿$ z $轴负方向.因此,我们就解释了之前实验所观测的那样,原子束经过磁场后分为两束.

\subsubsection{连续多个Stern-Gerlach装置的顺序实验}
\begin{figure}[ht]
\centering
\includegraphics[width=12cm]{./figures/Spin_2.pdf}
\caption{连续两个Stern-Gerlach装置的顺序实验的三种情况} \label{Spin_fig2}
\end{figure}
我们这里考虑连续两个Stern-Gerlach装置的顺序实验:
\begin{enumerate}
\item \textbf{实验1:}当第二个相同的S-G设备放置在第一个设备为向上自旋的出口处时,在第二个设备的输出中只能看到z +.这一结果是可以预期的,因为此时的所有中子都预计具有z +自旋,因为只有来自第一个装置的z +束进入第二个装置.

\item  实验e
\end{enumerate}
\textbf{注意}接下来的内容只是运用比喻方便大家理解和记忆连续多个Stern-Gerlach装置的顺序实验的后果:有趣的是,粒子的自旋似乎有着某种奇怪的“记忆”.如果是连续相同方向的Stern-Gerlach装置测量,那么粒子的自旋会记得上一次的结果.但是,如果其中有一个SG装置的方向和之前的不一样,那么该粒子的“记忆”就会被抹除,直到下一次再出现相同方向的SG转置才能体现重新的“记忆”.

\subsection{自旋角动量算符与泡利矩阵}

自旋是量子力学中的基本粒子特有的性质, 描述粒子的波函数不包含自旋的信息, 自旋处于单独的有限维希尔伯特空间中, 和波函数的空间做张量积以后用于描述粒子的状态.

\begin{enumerate}
\item 自旋角动量三个分量算符 $S_x, S_y, S_z$ 的互相对易关系以及自旋模长平方算符 $S^2$ 的对易关系:
\begin{equation}
[S_x,S_y]=\I\hbar S_z,\ [S_y,S_z]=\I\hbar S_x,\ [S_z,S_x]=\I\hbar S_y
\end{equation}

\item 与轨道角动量同理,存在一组本征态 $\ket{s,m}$ 

( $s = 0, 1/2, 1, 3/2\dots$, $m = -s, -s+1\dots ,s-1, s$ 但是每种粒子都有固有的 $s$ ) 满足
\begin{equation}
S^2\ket{s, m} = \hbar^2 s(s+1)\ket{s, m}  \quad \text{和} \quad
S_z\ket{s, m} = \hbar m\ket{s, m}
\end{equation}

\item 存在升降算符 $S_\pm = S_x \pm \I S_y$, 且(根号项是归一化系数)
\begin{equation}
S_\pm \ket{s,m} = \hbar \sqrt{s(s + 1) - m(m \pm 1)} \ket{s, m+1} 
\end{equation}

\item 对于 $s = 1/2$ 的粒子(这也是最为重要的情况,因为它是构成普通物质的粒子(质子、中子和电子)的自旋,以及
所有夸克和所有轻子的自旋.),$S^2$ 和 $S_z$ 一共有 2 个本征态, 分别是 上自旋态:$\ket{1/2, 1/2}$ 和下自旋态:$\ket{1/2, -1/2}$. 它们的角动量模长平方都是 $3\hbar^2/4$, 角动量 $z$ 分量都是 $\hbar/2$. 利用这两个基矢量,一个自旋为 $1/2$ 粒子的一般态 $\chi$ 可以表示为一个
两元列矩阵或\textbf{旋量儿(spinor)}:
\begin{equation}\label{Spin_eq4}
\chi = \pmat{a\\b}=a\chi_++b\chi_-
\end{equation}
以这两个本征态为基底,令第一个代表上自旋为 $\chi_+ =(1, 0)\Tr$, 第二个代表下自旋为 $\chi_- = (0, 1)\Tr$. 可以得出角动量平方算符的矩阵为
\begin{equation}
\mat S^2 = \frac{3\hbar^2}{4} \pmat{1&0\\0&1} \qquad
\mat S_z = \frac{\hbar}{2} \pmat{1&0\\0&-1}
\end{equation}
根据 $S_+ \chi_- = \hbar \chi_+$ 和 $S_- \chi_+ = \hbar \chi_-$,   得到
\begin{equation}
S_x = \frac{\hbar}{2}\pmat{0&1\\1& 0} \qquad
S_y = \frac{\hbar}{2}\pmat{0&-\I\\ \I& 0}
\end{equation}
然后, 定义泡利矩阵.  
\begin{equation}
\sigma_x = \pmat{0&1\\1& 0} \qquad
\sigma_y = \pmat{0&-\I\\ \I& 0} \qquad
\sigma_z = \pmat{1&0\\ 0&-1}
\end{equation}
其实, 根据对易关系直接就可以得到泡利矩阵.
\end{enumerate}
注意到 $S_x,S_y,S_z$ 都是厄密矩阵,当然它们也都表示可观测量.另外,$S_+$ 和 $S_-$ 不是厄密的,它们显然也不是可观测量.

如果我们要对一个粒子的态 $\chi$\autoref{Spin_eq4} 测量它的 $S_z$,那么我们得到 $+\hbar/2$ 的概率为 $|a|^2$,得到 $-\hbar/2$ 的概率为 $|b|^2$,也就是旋量必须要归一化:$|a|^2+ |b|^2 = 1$.

接下来我们自然就想要问道如果测量 $S_x$,那么其可能值及其概率是多少?根据广义的统计诠释,我们需要知
道 $\bvec S_x$ 的本征值和\textbf{本征旋量(eigenspinors)}.特征方程(characteristic equation)是
\begin{equation}
\rm{det}\pmat{-\lambda \quad \hbar/2\\\hbar/2\quad -\lambda}=0\Rightarrow\lambda=\pm\frac{\hbar}{2}
\end{equation}
$S_x$ 的可能值由此看来和 $S_z$ 是一样的,同样我们可以计算出本征旋量:

\begin{equation}
\frac{\hbar}{2}\pmat{0\quad 1\\1\quad0}\pmat{\alpha\\\beta}=\pm\frac{\hbar}{2}\pmat{\alpha\\\beta} \Rightarrow \pmat{\beta\\\alpha}=\pm\pmat{\alpha\\\beta}
\end{equation}
所以 $\beta=\pm\alpha$,显然 $\bvec S_x$ 的对应本征值为 $+\hbar/2$ 的归一化的本征旋量为:
\begin{equation}\label{Spin_eq5}
\chi_+^{(x)}=\frac{1}{\sqrt{2}}\pmat{1\\1}
\end{equation}
对应本征值为 $-\hbar/2$ 的归一化的本征旋量为:
\begin{equation}\label{Spin_eq6}
\chi_-^{(x)}=\frac{1}{\sqrt{2}}\pmat{1\\-1}
\end{equation}
\begin{equation}
\chi = \pmat{\frac{a+b}{\sqrt{2}}}\chi_+^{(x)}+\pmat{\frac{a-b}{\sqrt{2}}}\chi_-^{(x)}
\end{equation}
\begin{example}{}
假设自旋为 $1/2$ 的粒子处于态:
\begin{equation}
\chi = A\pmat{1-2\I\\2}
\end{equation}
(a). 归一化 $\chi$ 计算出常数 $A$.
\begin{equation}
1=\chi\chi^\dagger=|A|^2(1+4+4)=9|A|^2;\quad A=\frac{1}{3}
\end{equation}
(b). 如果对电子测量 $S_z$,那么可能值及其概率分别是多少?$S_z$ 的期待值是多少?

这里 $a=(1-2\I)/3,b=2/3$,因此对 $S_z$,得到 $+\hbar/2$ 的概率为:$|(1-2\I)/3|^2=5/9$.得到 $-\hbar/2$ 的概率为:$|2/3|^2=4/9$.$S_z$ 的期待值为:
\begin{equation}
\langle S_z \rangle =\frac{5}{9}\frac{\hbar}{2}+\frac{4}{9}\left(-\frac{\hbar}{2}\right)=\frac{\hbar}{18}
\end{equation}
(c). 如果对电子测量 $S_x$,那么可能值及其概率分别是多少?$S_z$ 的期待值是多少?

由\autoref{Spin_eq5} 和\autoref{Spin_eq6} 可得:
\begin{equation}
c_+^{(x)}=\left(\chi_+^{(x)}\right)^\dagger\chi=\frac{1}{3}\frac{1}{\sqrt{2}}\pmat{1\quad 1}\pmat{1-2\I\\2}=\frac{3-2\I}{3\sqrt{2}}\Rightarrow |c_+^{(x)}|^2=\frac{13}{18}
\end{equation}
因此可能值 $+\hbar/2$ 的概率为 $13/18$.接下来类似地我们继续计算可能值 $-\hbar/2$ 的概率为 $5/18$:
\begin{equation}
c_-^{(x)}=\left(\chi_-^{(x)}\right)^\dagger\chi=\frac{1}{3}\frac{1}{\sqrt{2}}\pmat{1\quad -1}\pmat{1-2\I\\2}=\frac{1+2\I}{3\sqrt{2}}\Rightarrow |c_+^{(x)}|^2=\frac{5}{18}
\end{equation}
$S_x$ 的期待值为:
\begin{equation}
\langle S_x \rangle =\frac{13}{18}\frac{\hbar}{2}+\frac{5}{18}\left(-\frac{\hbar}{2}\right)=\frac{2\hbar}{9}
\end{equation}
(d). 如果对电子测量 $S_y$,那么可能值及其概率分别是多少?$S_y$ 的期待值是多少?

上面三个小问可以说是简单繁琐的开胃菜,你可以选择跳过.但是你一定不能错过这个“小问”,尽管这个问题的解答或许有些费时间,请务必独立思考并试图解答.

解:首先我们和之前处理 $S_x$ 一样,先要找到 $S_y$ 的本征值,其特征方程为:
\begin{equation}
\rm{det}\pmat{-\lambda \quad -\I\hbar/2\\\I\hbar/2\quad -\lambda}=0\Rightarrow\lambda=\pm\frac{\hbar}{2}
\end{equation}
$S_x$ 的可能值由此看来和 $S_z$ 是一样的,同样我们可以计算出本征旋量:
\begin{equation}
\frac{\hbar}{2}\pmat{0& -\I\\\I&0}\pmat{\alpha\\\beta}=\pm\frac{\hbar}{2}\pmat{\alpha\\\beta} \Rightarrow -\I\beta=\pm\alpha
\end{equation}
接下来由旋量必须归一化性质可得:
\begin{equation}
|\alpha|^2+|\beta|^2=1\Rightarrow\alpha=\frac{1}{\sqrt{2}}
\end{equation}
显然 $\bvec S_y$ 的对应本征值为 $+\hbar/2$ 的归一化的本征旋量为:
\begin{equation}
\chi_+^{(y)}=\frac{1}{\sqrt{2}}\pmat{1\\-i}
\end{equation}
对应本征值为 $-\hbar/2$ 的归一化的本征旋量为:
\begin{equation}
\chi_-^{(y)}=\frac{1}{\sqrt{2}}\pmat{1\\i}
\end{equation}
接下来由本征旋量可得:
\begin{equation}
c_+^{(y)}=\left(\chi_+^{(y)}\right)^\dagger\chi=\frac{1}{3}\frac{1}{\sqrt{2}}\pmat{1\quad -\I}\pmat{1-2\I\\2}=\frac{1-4\I}{3\sqrt{2}}\Rightarrow |c_+^{(y)}|^2=\frac{17}{18}
\end{equation}
因此可能值 $+\hbar/2$ 的概率为 $17/18$.接下来类似地我们继续计算可能值 $-\hbar/2$ 的概率为 $1/18$:
\begin{equation}
c_-^{(y)}=\left(\chi_-^{(y)}\right)^\dagger\chi=\frac{1}{3}\frac{1}{\sqrt{2}}\pmat{1\quad \I}\pmat{1-2\I\\2}=\frac{1}{3\sqrt{2}}\Rightarrow |c_+^{(y)}|^2=\frac{1}{18}
\end{equation}
$S_y$ 的期待值为:
\begin{equation}
\langle S_y \rangle =\frac{17}{18}\frac{\hbar}{2}+\frac{1}{18}\left(-\frac{\hbar}{2}\right)=\frac{4\hbar}{9}
\end{equation}
当然我们还可以更加直接地用下面的方法计算出来:
\begin{equation}
\langle S_y \rangle = \chi^\dagger\bvec S_y\chi = \frac{1}{3}\pmat{1-2\I\quad 2}\frac{\hbar}{2}\frac{1}{3}\pmat{0&-\I\\ \I& 0}\pmat{1-2\I\\2}=\frac{4\hbar}{9}
\end{equation}

\end{example}

% 未完成

\begin{exercise}{}
假设电子为一个经典的刚性球体,其经典电子半径 $r_0$ 为:
\begin{equation}\label{Spin_eq3}
r_c=\frac{e^2}{8\pi\sigma_0m_e c^2}
\end{equation}
其中 $e$ 为单位电荷,$\sigma_0$ 为真空电容率,$m_e$ 为电子静止质量,$c$ 为光速.试着结合爱因斯坦质能公式 $E=mc^2$ 并且假设电子质量可归因于其电场能量推导出\autoref{Spin_eq3} .电子的角动量为 $1/2\hbar$,求它“赤道”上的一个点的运动速度.这个模型有什么意义?
\end{exercise}