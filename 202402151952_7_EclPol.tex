% 圆锥曲线的配极(高中数学)
% keys 高中数学|圆锥曲线|极点|极线
% license Xiao
% type Art






\begin{definition}{关于圆的极点与极线}

给定一个圆$C$,取$C$外一点$P$。

过点$P$作圆$C$的切线,切点为$A$和$B$,则称直线$AB$是点$P$\textbf{关于圆}$C$的极线。

反之,取圆$C$的一根弦,与圆的交点为$A$和$B$,过这两个点作圆$C$的切线,称其交点$P$为直线$AB$\textbf{关于圆}$C$的极点。

\begin{figure}[ht]
\centering
\includegraphics[width=8cm]{./figures/542dd0f77d5d5e19.pdf}
\caption{关于圆的极点与极线的示意图。点$P$和直线互为关于给定圆的极点和极线。} \label{fig_EclPol_1}
\end{figure}

\end{definition}




极点和极线总是成对出现,因此一定要强调“关于圆的”。比如说,单独给定一个圆和一个点,不能说这个点就是圆的极点,因为没有“圆的极点”这种说法。






















