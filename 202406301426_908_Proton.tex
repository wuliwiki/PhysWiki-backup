% 质子
% license CCBYSA3
% type Wiki

(本文根据 CC-BY-SA 协议转载自原搜狗科学百科对英文维基百科的翻译)

\textbf{质子}属于亚原子粒子,符号为p或者$p^+$,携带为一个基本电荷,其质量略小于中子。质子和中子的质量约为一个原子质量,它们都被称为“核子”。

每个原子的原子核内均存在一个或多个质子,它们是核子的必要组成部分。原子核中质子的数量被称为原子序数(由符号Z表示)。因为每种元素的质子数量不一样,因此每种元素都有自己独特的原子序数。

1917年,欧内斯特·卢瑟福做实验发现,使用$\alpha$粒子撞击氮原子核,可以提取氢原子核。卢瑟福因此推断,氢原子核是氮原子核与所有更重的原子核的基础材料。由于这重要结果,卢瑟福被公认为质子的发现者。

在粒子物理学的现代标准模型里,质子是由两个上夸克与一个下夸克组成的强子。夸克的静质量只贡献出大约1\%质子质量,剩余的质子质量主要源自于夸克的动能与捆绑夸克的胶子场的能量。

因为质子是由三个夸克组成,质子可视为基本粒子,质子具有物理尺寸,但这尺寸并不能完美良好定义,由于质子的表面很模糊,因为这表面是由作用力的影响来定义,而这作用力不会突然终止。质子的半径(更仔细地说,电荷半径)大约为0.84到0.87飞米即$0.84\times10^{-15}$到$0.87\times10^{-15}$ m。

在足够低的温度下,自由质子将和电子结合。然而,这种结合并没有改变质子的性质。当一个质子快速移动通过物质时,它会和电子以及原子核相互作用,进而速度变慢,直至被原子的电子云俘获。因此质子化原子将被产生,它是氢的化合物。当真空中存在自由电子时,足够慢的质子可以和单个自由电子结合,成为中性的氢原子,其在化学上是自由基。在足够低的能量下,这种“自由氢原子”倾向于与许多其他类型的原子发生化学反应。当自由氢原子相互反应时,它们形成中性氢分子$(H_2)$。因此氢分子是星际空间中分子云最常见的分子成分。