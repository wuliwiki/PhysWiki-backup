% C++ 迭代器笔记
% license Usr
% type Note

\begin{itemize}
\item 5 类迭代器的要求越来越严格,前一个的要求是后一个的子集。
\item \verb`LegacyInputIterator`, 只能 single-pass,即遍历一遍(例如 stdin 不能反复从头读)
\item \verb`LegacyForwardIterator`, 可以反复遍历
\item \verb`LegacyBidirectionalIterator`,
\item \verb`LegacyRandomAccessIterator`, 多了 \verb`+, -, +=, -=, >, >=, <, <=, []` 等算符支持
\item \verb`LegacyContiguousIterator`
\item STL 容器最少都是 Forward,Input 仅用于文件或 stdin 等 stream。
\item 超出范围的迭代器最多只能是 \verb`v.end()`,其他的超出范围都是未定义行为。
\item Range-based for 要求至少 \verb`LegacyForwardIterator`。
\item \verb`for (aito &e: v)` 基本相当于 \verb`for (auto it = v.begin(); it != v.end(); ++it) {auto &e = *it; ...}`
\item 为了适用于 \verb`std::sort, std::search` 等标准算法,还需要定义一些属性(不用的话就不强制)
\begin{lstlisting}[language=cpp]
// Required for standard algorithms:
using iterator_category = std::forward_iterator_tag;
using value_type = T;
using difference_type = std::ptrdiff_t;
using reference = T&;
using pointer = T*;
\end{lstlisting}
\item 可以用 \verb`std::iterator_traits<迭代器类>` 获取以上属性。
\item 指针自动适用于 \verb`std::iterator_traits`
\end{itemize}
