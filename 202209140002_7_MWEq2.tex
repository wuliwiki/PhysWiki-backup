% 麦克斯韦方程组(外微分形式)
% Maxwell|外代数|外微分|exterior product|exterior algebra|Grassmann algebra|exterior derivative|微分形式|differential form|外导数|楔积|wedge product|霍奇算子|霍奇星算子|Hodge|Hodge star operator

\addTODO{本词条处于草稿阶段.}

\subsection{前两个方程}
\pentry{外导数\upref{ExtDer}}

\addTODO{百科中尚未创建“规范场论”部分.使用引用\cite{KnotsVol4}.}


在 $\mathbb{R}^3$ 中考虑电磁场,三个空间轴分别为 $x, y, z$ 轴.

考虑麦克斯韦方程组中的两个方程,$\nabla\cdot\bvec{B}=0$ 和 $\nabla\times\bvec{E}=\partial_t\bvec{B}$.为了尝试用外代数来表达这两个式子,我们就要把 $\bvec{B}$
表示成一个2-形式 $B=B_z\dd x\wedge\dd y+B_x\dd y\wedge\dd z+B_y\dd z\wedge\dd x$,把 $\bvec{E}$ 表示成一个1-形式 $E=E_x\dd x+E_y\dd y+E_z\dd z$,这样以上两个方程的左边就都可以写成外导数的形式,从而有:

\begin{equation}\label{MWEq2_eq3}
\dd B=0
\end{equation}
和
\begin{equation}\label{MWEq2_eq4}
\dd E=\partial_tB
\end{equation}

其中 $\mathbb{R}^4$ 可以写成三维空间和一维时间的乘积:$\mathbb{R}^3\times\mathbb{R}$.这个四维欧几里得空间中的时间轴记为 $x^0$ 轴,空间轴则记为 $x^1, x^2, x^3$ 轴.要注意在这种表示下,$\partial_tB$ 就成了 $\partial_0B$.

现在考虑用一个统一的2-形式 $F=B+E\wedge\dd x^0$ 来表示电磁场\footnote{电场外积一个 $\dd x^0$ 是为了凑成合适的2-形式.},也就是

\begin{equation}
\ali{
    F = {}&B_z\dd x^1\wedge\dd x^2+B_x\dd x^2\wedge\dd x^3+B_y\dd x^3\wedge\dd x^1\\
    &+E_x\dd x^1\wedge \dd x^0+E_y\dd x^2\wedge \dd x^0+E_z\dd x^3\wedge\dd x^0
}
\end{equation}
这个电磁场形式的外导数计算如下,我们把结果分成四个部分来方便阅读:
\begin{equation}
\ali{
    \dd F ={}& \partial_0 B_z\dd x^0\wedge \dd x^1\wedge \dd x^2 +\\& \partial_0 B_x \dd x^0\wedge \dd x^2\wedge \dd x^3 +\\& \partial_0 B_y \dd x^0\wedge \dd x^3\wedge \dd x^1\\
    &+\\&(\partial_1 B_x+\partial_2 B_y+\partial_3 B_z)\dd x^1\wedge \dd x^2\wedge \dd x^3\\
    &+\\&(\partial_2 E_z-\partial_3 E_y)\dd x^2\wedge \dd x^3\wedge\dd x^0+\\&(\partial_3 E_x-\partial_1 E_z)\dd x^3\wedge \dd x^1\wedge\dd x^0 +\\&(\partial_1 E_y-\partial_2 E_x)\dd x^1\wedge \dd x^2\wedge\dd x^0 \\
}
\end{equation}

那么\autoref{MWEq2_eq3} 和\autoref{MWEq2_eq4} 就可以统一用一个式子来表达:
\begin{equation}
\dd F = 0
\end{equation}

% 证明过程此处略去,留给规范场论部分中相关章节来处理.





\subsection{后两个方程}

\pentry{霍奇星算子\upref{HodgeO}}









