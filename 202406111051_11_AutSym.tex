% 有限对称群的性质
% license Usr
% type Tutor

\begin{definition}{}
阶数为1或2的置换,称为\textbf{对合变换(involution)}
\end{definition}
\begin{theorem}{}
任意有限置换群都可以表示为两个对合变换的复合。
\end{theorem}
\textbf{证明:}
我们首先证明,循环置换可以分解为两个对合变换的复合。

一个n元循环置换可以看作n边形上的旋转,而我们知道,二维空间上的旋转可以分解为两个反射,只要保证两次反射轴的夹角是旋转角度的$1/2$,对合变换就是这种反射变换的置换表示。以正五边形为例,该过程如下所示:
\begin{figure}[ht]
\centering
\includegraphics[width=14cm]{./figures/f32c9320160af59c.png}
\caption{} \label{fig_AutSym_2}
\end{figure}
该循环的分解写作$(12345)=[(23)(14)[()()]$