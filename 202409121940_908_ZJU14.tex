% 浙江大学 2014 年 考研 量子力学
% license Usr
% type Note


\subsection{第一题: 简答题 (30 分)}

(1) 一维问题的能级的最大简并度最大是多少?

(2) 什么是量子力学中的守恒量,它们有什么性质。

(3) 什么是受激辐射?什么是光电效应?

(4) 试写出非简并微扰论的能级修正公式(到二阶)。

(5) 由正则对易关系 $[\hat{x}, \hat{p}] = i\hbar$ 导出角动量的三个分量

\[L_x = i \hbar \left( y \frac{\partial}{\partial z} - z \frac{\partial}{\partial y} \right)~\]

\[L_y = i \hbar \left( z \frac{\partial}{\partial x} - x \frac{\partial}{\partial z} \right)~\]

\[L_z = i \hbar \left( x \frac{\partial}{\partial y} - y \frac{\partial}{\partial x} \right)~\]

的对易关系。
\subsection{第二题 (20 分)}
原子序数较大的原子的最外层电子感受到的原子核和内层电子的总位势可以表示为
\[V(r) = -\frac{e^2}{r} - \lambda \frac{e^2}{r^2}, \quad \lambda = 1~\]
试求其基态能量。