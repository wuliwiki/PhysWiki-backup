% 黎曼重排定理

\pentry{绝对收敛与条件收敛\upref{Convg}}

相比于绝对收敛级数, 条件收敛级数究竟是哪里有问题呢? 

不妨假定讨论的一直是实数项级数. 沿用证明绝对收敛级数的重排性质时的构造: 
$$
a_n^+=\left\{\begin{array}{cc}
{a_n}\quad &a_n\geq0;\\
0\quad &a_n<0,
\end{array}\right.
\quad\quad
a_n^-=\left\{\begin{array}{cc}
0\quad &a_n\geq0;\\
-a_n\quad &a_n<0.
\end{array}\right.
$$
可以很合理地将$a_n^+$称为正部, $a_n^-$称为负部, 而当然$a_n=a_n^+-a_n^-$. 如果级数$\sum_{n=1}^\infty a_n$是绝对收敛的, 那么正部和负部组成的都是收敛的正项级数. 而如果这级数只是条件收敛的, 那么正部和负部组成的都是发散到无穷的正项级数. 在这种情况下, 式子
$$
\sum_{n=1}^\infty a_n=\sum_{n=1}^\infty a_n^+-\sum_{n=1}^\infty a_n^-
$$
是没有意义的.

而后, 回忆 绝对收敛与条件收敛\upref{Convg}中给出的例子
$$
\sum_{n=1}^\infty\frac{(-1)^{n+1}}{n}
=1-\frac{1}{2}+\frac{1}{3}-\frac{1}{4}+...
$$
这个级数的和等于$\ln2$, 但我们已经通过一种重排技巧将它的和缩减到了本来的和的一半. 这个性质不是偶然的. 条件收敛级数的收敛完全是因为相邻项之间的正负抵消, 而将这件事精确化的正是如下的

\begin{theorem}{黎曼重排定理}
设$\sum_{n=1}^\infty a_n$是条件收敛但不绝对收敛的级数, 一般项都是实数. 则对于任何一个实数$A$, 都存在正整数集$\mathbb{N}$的一个重排$\sigma$, 使得
$$
\sum_{n=1}^\infty a_{\sigma(n)}=A.
$$
\end{theorem}

可以说, 黎曼重排定理将条件收敛级数的正负抵消性质发挥到了极致.

要证明黎曼重排定理, 实际上只是需要重复一些构造而已. 回忆一下级数收敛的必要条件, 知道$a_n^+$和$a_n^-$都收敛到零. 不妨先假定给定的实数$A>0$. 从正部里取出前$N_1$项, 使得恰好有
\[
\sum_{n=1}^{N_1-1}a_n^+\leq A,
\quad
\sum_{n=1}^{N_1}a_n^+> A.
\]
而后从负部里取出前$N_2$项, 使得恰好有
\[
\sum_{n=1}^{N_1}a_n^+-\sum_{n=1}^{N_1}a_n^-> A,
\quad
\sum_{n=1}^{N_1+1}a_n^+-\sum_{n=1}^{N_1}a_n^-\leq A.
\]
重复这个操作, 就得到了级数$\sum_{n=1}^\infty a_n$的一个重排. 首先, 这重排是可以一直继续下去的, 因为正部和负部组成的级数$\sum_{n=1}^\infty a_n^+$和$\sum_{n=1}^\infty a_n^-$都发散至无穷, 所以在每一步操作之后总能找到足够多的项来超过/低于给定数$A$ (这个操作对于绝对收敛级数就做不到, 因为在这种情况下找再多的项都不够). 另外, 这重排给出的是收敛级数, 因为有$a_n\to0$的关系, 每一步操作所给出的部分和将会震荡着逼近$A$.

\begin{example}{黎曼重排定理的例题}
上面的论证可能比较抽象, 来看具体例子. 还是讨论级数
$$
\sum_{n=1}^\infty\frac{(-1)^{n+1}}{n}
=1-\frac{1}{2}+\frac{1}{3}-\frac{1}{4}+...
$$
现在希望将它重排使之收敛到1. 所作的重排过程如下:
$$
1-\frac{1}{2}=0.5<1;
$$ 
$$
1-\frac{1}{2}+\frac{1}{3}+\frac{1}{5}=1.0333...>1;
$$ 
$$
1-\frac{1}{2}+\frac{1}{3}+\frac{1}{5}-\frac{1}{4}=0.7833...<1;
$$ 
$$
1-\frac{1}{2}+\frac{1}{3}+\frac{1}{5}-\frac{1}{4}+\frac{1}{7}+\frac{1}{9}=1.0373...>1;
$$
如此续行.
\end{example}