% 因式分解与一元二次方程(高中)
% license Xiao
% type Tutor
\pentry{函数回顾\nref{nod_HsFunB}}{nod_b9ee}

\subsection{一元二次方程的根}

一元二次方程指形如
\begin{equation}
ax^2+bx+c=0 \qquad (a\neq 0)~.
\end{equation}
的方程,它可以变形得到如下形式:
\begin{equation}
\begin{split}
 & ax^2+bx+c = 0 \\ 
\iff&x^2+2{b\over 2a} x+\left({b\over 2a}\right)^2 = \left({b\over 2a}\right)^2-{c\over a} \\ 
\iff&\left(x+{b\over 2a}\right)^2 = {b^2-4ac\over 4a^2} \\ 
\end{split}~.
\end{equation}

从函数的视角来看,如果设函数$y=ax^2+bx+c$,那么求解方程$ax^2+bx+c=0$即化为寻找所有使$y=0$成立的点。

\subsubsection{存在性判定}

由于左侧及右侧分母是一个平方形式,因此$x$的值只与右侧分子的符号相关,定义其为判别式:
\begin{equation}
\Delta = b^2-4ac~.
\end{equation}

则:
\begin{itemize}
\item $\Delta > 0$时,方程有实数解,为两个不同的实根。
\item $\Delta = 0$时,方程有实数解,为两个相同的实根。\footnote{这里称两个相同实根而非一个实根,是根据\aref{代数基本定理}{the_HsEquN_1}。}
\item $\Delta < 0$时,方程无实数解。
\end{itemize}

\subsubsection{求根公式}

方程有解时,两个根分别为:
\begin{equation}\label{eq_quasol_1}
x_1=\frac{-b+\sqrt{\Delta}}{2a}\qquad x_2=\frac{-b-\sqrt{\Delta}}{2a}~.
\end{equation}
这被称为\textbf{二次方程的求根公式(Quadratic Formula)}。

\autoref{eq_quasol_1} 中,若$\Delta = 0$,则$x_1=x_2=-\frac{b}{2a}$与之前推理的结论相符。

\subsubsection{几何含义}

\begin{figure}[ht]
\centering
\includegraphics[width=14cm]{./figures/652b2f4efe76656f.pdf}
\caption{$f(x)$示意图。从左到右为$\Delta > 0, \Delta = 0, \Delta < 0$} \label{fig_quasol_1}
\end{figure}
可见,$x=-\frac{b}{2a}$ 为函数$f(x)$的对称轴,两个零点(如果存在)关于该轴对称。

\subsection{配方法}
对于一些特定的问题,可以将方程配方并求解,有时这比直接使用求根公式更为简便。

例如,可以将方程配方为如下形式:
$$(x-a)(x-b)=0\Rightarrow x_1=a, x_2=b~.$$
\begin{figure}[ht]
\centering
\includegraphics[width=5cm]{./figures/fe527c666ebfd775.pdf}
\caption{$f(x)=(x-a)(x-b)$示意图} \label{fig_quasol_2}
\end{figure}
或者
$$(x-a)^2=b\Rightarrow x_1=\sqrt{b}+a, x_2=-\sqrt{b}+a~.$$

\subsection{韦达定理}

\textbf{韦达定理(Vieta's formulas)}是一组描述代数方程的根和方程系数的关系的公式,因此也称\textbf{根与系数关系}。

\begin{theorem}{韦达定理(二次情况)}
设方程 $a x^2 + b x + c = 0\quad(a \neq 0)$的两个根为 $x_1$ 和 $x_2$,则它们满足:
$$\begin{aligned}
x_1 + x_2 &= -\frac{b}{a} ~,\\
x_1 x_2 &= \frac{c}{a}~.
\end{aligned}$$
\end{theorem}

下面给出证明:

对方程 $a x^2 + b x + c = 0\quad(a \neq 0)$,设其有两个根 $x_1$ 和 $x_2$,则有:
\begin{equation}\label{eq_quasol_2}
\begin{split}
ax^2+bx+c &= 0 \\ 
&=a(x-x_1)(x-x_2)\\
&=ax^2-a(x_1+x_2)x+ax_1x_2
\end{split}~.
\end{equation}
由于\autoref{eq_quasol_2} 为恒成立的代数式,所以两侧对应变量的系数相等,从而有:
\begin{equation}
\begin{cases}
-a(x_1+x_2)=b\\
ax_1x_2=c
\end{cases}
\implies
\begin{cases}
x_1 + x_2 &= -\frac{b}{a} \\
x_1 x_2 &= \frac{c}{a}
\end{cases}.~
\end{equation}

证毕。

这是一种常用的证明方法,即恒成立的代数方程,两侧对应位置的系数相等。韦达定理使得面对方程时,即使不求解具体的解的形式,也可以描述两个解之间的关系,这在\enref{解析几何}{JXJH}部分会有非常重要的应用。另外,对于任意$n$次代数方程的情况,有:
\begin{theorem}{韦达定理}
对于$n$次代数方程$a_0x^n+a_1x^{n-1}+\cdots+a_n=0\quad(a_0\neq0)$,若存在$n$个根$x_1,x_2,\cdots,x_n$,则有:
\begin{equation}
\begin{split}
x_1+x_2+\cdots +x_n&=-{a_1\over a_0}\\
(x_1x_2+x_1x_3+\cdots+ x_1x_n)+(x_2x_3+\cdots +x_2x_n)+\cdots+x_{n-1}x_n&={a_2\over a_0}\\
\vdots\\
x_1x_2\cdots x_n&=(-1)^n{a_n\over a_0}
\end{split}~.
\end{equation}
也可以统一记作\footnote{此处使用的是\enref{求和符号}{SumSym}和\enref{求积符号}{ProdSy}。如果不理解可以跳过。}:
\begin{equation}
\sum_{1\le i_1 < i_2 < \cdots < i_k\le n} \left(\prod_{j = 1}^k r_{i_j}\right)=(-1)^k\frac{a_{n-k}}{a_n}~.
\end{equation}
\end{theorem}

它的证明方式与之前完全相同,可以自己尝试做一下,在计数原理部分会有更透彻的讲解。