% 用 FFT 计算傅里叶变换(Matlab)

\begin{issues}
\issueDraft
\end{issues}

\pentry{离散傅里叶变换\upref{DFT}}

\begin{lstlisting}[language=matlab]
% fft approximation of the analytical fourier transform from f(x) to g(k)
% x and k are both equally spaced, x starts from x0 equally spaced by dx
% norm(g) = norm(f)
% numel(g) = Nk

function [g, k] = FFT(f, x0, dx, Nk, dim)
x_mid = (2*x0 + (numel(f)-1)*dx)/2; % mid point of x grid
if exist('Nk', 'var')
    f = fftresize(f, Nk);
end
if exist('dim', 'var')
    g = sffts(f, dim)*(dx/sqrt(2*pi));
else
    g = sffts(f)*(dx/sqrt(2*pi));
end

if ~exist('dim', 'var')
    if isvector(f)
        k = fftlinspace(2*pi/dx, numel(f));
    else
        k = fftlinspace(2*pi/dx, size(f,1));
    end
else
    k = fftlinspace(2*pi/dx, size(f,dim));
end

if (abs(x_mid/x0) > 1e-14)
    if (isvector(g))
        k = reshape(k, size(g));
        g = g .* exp(-1i*k*x_mid);
    else
        error('asymmetric x not implemented!');
    end
end
end
\end{lstlisting}

\begin{lstlisting}[language=matlab]
% resize vector/matrix length for ftt by zero padding on both ends
function y = fftresize(x, newN)
% === x is row vector ===
if size(x, 1) == 1 
    N = numel(x);
    Ndiff = abs(newN - N);
    if newN > N % 0-padding
        if mod(Ndiff,2) == 0
            Ndiff = 0.5*Ndiff;
            y = [zeros(1, Ndiff), x, zeros(1, Ndiff)];
        else
            Ndiff = 0.5*(Ndiff-1);
            if mod(N, 2) == 0
                y = [zeros(1, Ndiff), x, zeros(1, Ndiff+1)];
            else
                y = [zeros(1, Ndiff+1), x, zeros(1, Ndiff)];
            end
        end
    elseif newN < N % shrink
        y = shrink(x, N, Ndiff);
    else
        y = x;
    end

% === x is column vector ===
elseif size(x, 2) == 1
    N = numel(x);
    Ndiff = abs(newN - N);
    if newN > N % 0-padding
        if mod(Ndiff,2) == 0
            Ndiff = 0.5*Ndiff;
            y = [zeros(Ndiff, 1); x; zeros(Ndiff, 1)];
        else
            Ndiff = 0.5*(Ndiff-1);
            if mod(N, 2) == 0
                y = [zeros(Ndiff, 1); x; zeros(Ndiff+1, 1)];
            else
                y = [zeros(Ndiff+1, 1); x; zeros(Ndiff, 1)];
            end
        end
    elseif newN < N % shrink
        y = shrink(x, N, Ndiff);
    else
        y = x;
    end

% === x is matrix ===
else
    [N, Ncol] = size(x);
    Ndiff = abs(newN - N);
    if newN > N % 0-padding
        if mod(Ndiff,2) == 0
            Ndiff = 0.5*Ndiff;
            y = [zeros(Ndiff, Ncol); x; zeros(Ndiff, Ncol)];
        else
            Ndiff = 0.5*(Ndiff-1);
            if mod(N, 2) == 0
                y = [zeros(Ndiff, Ncol); x; zeros(Ndiff+1, Ncol)];
            else
                y = [zeros(Ndiff+1, Ncol); x; zeros(Ndiff, Ncol)];
            end
        end
    elseif newN < N % shrink
        if mod(Ndiff,2) == 0
            Ndiff = 0.5*Ndiff;
            y = x(Ndiff+1:end-Ndiff, :);
        else
            Ndiff = 0.5*(Ndiff-1);
            if mod(N, 2) == 0
                y = x(Ndiff+2:end-Ndiff, :);
            else
                y = x(Ndiff+1:end-Ndiff-1, :);
            end
        end
    else
        y = x;
    end
end
end


function y = shrink(x, N, Ndiff)
    if mod(Ndiff,2) == 0
        Ndiff = 0.5*Ndiff;
        y = x(Ndiff+1:end-Ndiff);
    else
        Ndiff = 0.5*(Ndiff-1);
        if mod(N, 2) == 0
            y = x(Ndiff+2:end-Ndiff);
        else
            y = x(Ndiff+1:end-Ndiff-1);
        end
    end
end
\end{lstlisting}

\begin{lstlisting}[language=matlab]
% shifted fft
function y = sffts(x, dim)
    if nargin < 2
        y = fftshift(fft(ifftshift(x)));
    else
        y = fftshift(fft(ifftshift(x, dim),[], dim), dim);
    end
end
\end{lstlisting}

\begin{lstlisting}[language=matlab]
% generate N grid points from bandwidth
% input 2 or 3 arguments
function x = fftlinspace(L, N, x0)
if mod(N, 2) == 0
    Lh = 0.5*L; dx = L/N;
    if nargin == 3
        x = linspace(-Lh+x0, Lh-dx+x0, N);
    else
        x = linspace(-Lh, Lh-dx, N);
    end
else
    a = (N-1)*L/(2*N);
    if nargin == 3
        x = linspace(-a+x0, a+x0, N);
    else
        x = linspace(-a, a, N);
    end
end
end
\end{lstlisting}
