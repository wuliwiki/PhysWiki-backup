% 球极平面投影
% license CCBYSA3
% type Wiki

(本文根据 CC-BY-SA 协议转载自原搜狗科学百科对英文维基百科的翻译)

在几何中,球极平面投影是一种将圆球面投影到平面上的特殊映射(函数)。该映射的定义域为除投影点外的整个球面。在定义域内,该映射具有光滑性、双射性和共形性。共形性意味着其为保持曲线相交角度不变的映射。但是,该映射既不保距离也不保面积:也就是说,其既不能维持映射后图形的距离不变,也不能维持映射后图形的面积不变。

直观地说,球极平面投影是一种用平面来描绘球面的方法,但该方法在图案品质方面,也包含一些不可避免的妥协。因为球面和平面出现于数学及其应用的诸多领域,所以球极平面投影也同样很常见。在各个领域,例如复分析、制图学、地质学、摄影等,球极平面投影都有广泛的应用。在实践中,球极平面投影通常是通过计算机或手工在一种称为立体球面投影网的特殊的方格纸上绘制完成,该立体球面投影网简称立体网,或乌尔夫网。

\subsection{历史}
\begin{figure}[ht]
\centering
\includegraphics[width=8cm]{./figures/cdf5afd3b6eb90c9.png}
\caption{鲁本斯(Rubens)为弗朗索瓦·达吉隆(François d'Aguilon)的著作《光学六册:适用于哲学家和数学家》所作的插图。该图展现了一般透视投影的原理,球极平面投影是其中一种特殊情况。} \label{fig_QJPMTY_2}
\end{figure}
球极平面投影为希帕克斯(Hipparchus)、托勒密(Ptolemy),甚至可能更早期的古埃及人所知。其最初被称为“平面天体投影”。[1] 托勒密的著作《平球论》(Planisphaerium)是现存最古老的描述球极平面投影的文献,其最重要的用途之一就是描绘天体图。[1] “平面天体投影”这个术语仍然用于指代这样的图表。

在16世纪和17世纪,球极平面投影的赤道面通常被用于东半球和西半球地图上。通常认为,1507年瓜尔特里厄斯·路德(Gualterius Lud)绘制的地图[2] 使用的就是球极平面投影,包括后来的让·罗兹(Jean Roze)(1542年)、鲁莫尔德·墨卡托(Rumold Mercator)(1595年),以及其他许多人绘制的地图。[3] 而在星图上,这个赤道面也已经被像托勒密这样的古代天文学家利用了。[4]

弗朗索瓦·德阿吉隆(François d'Aguilon)在他1613年的著作《光学六册:适用于哲学家和数学家》(Opticorum libri sex philosophis juxta ac mathematicis utiles)中给球极平面投影取了现在的名字。[5]

1695年,在兴趣的驱使下,埃德蒙多·哈雷(Edmond Halley)发表了第一篇关于星图共形性的数学证明。[6] 在证明中,他使用了由他的朋友艾萨克·牛顿(Isaac Newton)在当时刚刚建立起来的微积分工具。

\subsection{定义}
\subsubsection{2.1 第一类计算公式}
\begin{figure}[ht]
\centering
\includegraphics[width=6cm]{./figures/82c46f88c891178f.png}
\caption{单位球面上从北极点向z = 0 平面的球极平面投影的截面示意图} \label{fig_QJPMTY_1}
\end{figure}
三维空间 $\mathbb{R}^3$ 中的单位球面是一组点 $(x, y, z)$ 的集合,且满足 $x^2 + y^2 + z^2 = 1$。如果令点 $N = (0, 0, 1)$ 为该球面上的“北极点”,令球面的其余部分为 $M$,则 $z = 0$ 平面穿过该球面所包裹的球体的中心,“赤道”就是球面与 $z = 0$ 平面的交线。

对于 $M$ 上的任意一点 $P$,有且仅有一条直线穿过 $N$ 和 $P$ 点,且这条线与 $z = 0$ 平面相交于 $P'$ 点。定义 $P$ 点在 $z = 0$ 平面上的球极平面投影为 $P'$ 点。

在笛卡尔坐标系中,球面上的坐标 \((x, y, z)\) 与对应的平面上的坐标 \((X, Y)\) 满足以下投影及其逆投影关系式:
$$(X, Y) = \left( \frac{x}{1 - z}, \frac{y}{1 - z} \right),~$$
$$(x, y, z) = \left( \frac{2X}{1 + X^2 + Y^2}, \frac{2Y}{1 + X^2 + Y^2}, \frac{-1 + X^2 + Y^2}{1 + X^2 + Y^2} \right).~$$
在球面坐标系中,球面上的坐标 \((\varphi, \theta)\)(用 \(\varphi\) 表示天顶角, \(0 \leq \varphi \leq \pi\);用 \(\theta\) 表示方位角, \(0 \leq \theta \leq 2\pi\))与对应的平面上的极坐标 \((R, \Theta)\) 满足以下投影及其逆投影关系式:
$$(R, \Theta) = \left( \frac{\sin \varphi}{1 - \cos \varphi}, \theta \right) = \left( \cot \frac{\varphi}{2}, \theta \right),~$$
$$(\varphi, \theta) = \left( 2 \arctan \frac{1}{R}, \Theta \right).~$$
这里,定义当 \(R = 0\) 时 \(\varphi = \pi\)。此外,可以用与上述公式等价的方式将这些公式改写为多种不同的形式。在圆柱坐标中,球面上的坐标 \((r, \theta, z)\) 与对应的平面上的极坐标 \((R, \Theta)\) 满足以下投影及其逆投影关系式:
$$(R, \Theta) = \left( \frac{r}{1 - z}, \theta \right),~$$
$$(r, \theta, z) = \left( \frac{2R}{1 + R^2}, \Theta, \frac{R^2 - 1}{R^2 + 1} \right).~$$
\subsubsection{2.2 其他约定形式}
\begin{figure}[ht]
\centering
\includegraphics[width=6cm]{./figures/e15da8788802fff9.png}
\caption{单位球面上从北极点向z = −1 平面的球极平面投影的截面示意图} \label{fig_QJPMTY_3}
\end{figure}
一些作者[7] 将球极平面投影定义为从单位球面上的北极点 (0, 0, 1) 向z = −1 平面的投影,该平面与单位球面的南极点(0, 0, −1) 相切。这种极点切面所得到的$X$ 和$Y$ 的坐标值恰好是上一节描述的赤道平面投影坐标值的两倍。例如,该投影将赤道映射为以原点为中心,半径为2的圆。可以看出,赤道平面投影在赤道上不会产生无穷小的面积变形,而极点切面投影在南极不会产生无穷小的面积变形。

还有其他一些作者[8] 使用半径为12 的球面和z = − 12 平面来定义球极平面投影。在这种情况下,计算公式为: