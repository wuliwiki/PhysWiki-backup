% 万有引力定律(高中)
% keys 开普勒定律|万有引力|天体|宇宙速度|同步卫星

\begin{issues}
\issueDraft
\issueTODO
\end{issues}

\subsection{开普勒定律}

开普勒定律是开普勒根据对太阳系中行星运动的观测数据总结出来的,是一个普适定律,也适用于其他天体绕中心天体运动的情况,如卫星围绕地球的运动等.

第一定律:所有行星绕太阳运动的轨道都是椭圆,太阳在椭圆的一个焦点上.

第二定律:对任意一个行星来说,它与太阳的连线在相等时间内扫过相等的面积.

第三定律:所有行星的轨道半长轴($a$)的三次方跟它的公转周期($T$)的二次方之比都相等.表达式为
\begin{equation}\label{HSPM06_eq1}
\frac{a^3}{T^2}=k
\end{equation}
要注意的是,\autoref{HSPM06_eq1} 中代表“都相等“的比值$k$,是针对围绕同一中心天体运动的所有天体而言的,因为$k$的大小与中心天体的质量有关.

\subsection{万有引力定律}

内容:任意两个质点都相互吸引,这个引力的大小与两质点的质量的乘积成正比,与两质点的距离的平方成反比.

万有引力的大小为
\begin{equation}
F=G\frac{m_1m_2}{r^2}
\end{equation}
$m_1$、$m_2$分别为两质点的质量,$r$为两质点间的距离,$G$为引力常量,且$G=6.67\times 10^{-11}\mathrm{N\cdot m^2/kg^2}$
