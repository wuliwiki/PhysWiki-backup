% 测度与广义函数
% Lebesgue测度|广义函数|绝对连续|Lebesgue积分

\pentry{Lebesgue 积分\upref{Lebes1},映射\upref{map}}

在\textbf{集合的测度(实变函数)}\upref{SetMet}中,我们了解了Lebesgue外测度.它是将“开集的体积”进行推广而得来的,其定义思路也有明显的物理对应:用容器去衡量被容纳物的大小.具象的概念虽然容易想象,但因其具象也束缚了概念范围,所以我们在得到一个新的概念时,总会想把它的核心特征抽离出来,抽象出一个更一般的概念,看看能不能引申出有趣的理论.

Lebesgue外测度该怎么抽象呢?抛去其定义的方式不谈,它就是用来衡量“集合的体积”的,对不对?体积是一个数字,那我们就把测度看成是给各集合赋予一个数字,也就是“集合函数”\footnote{注意这个术语的意思:集合函数是指把集合映射到数字上的映射.值域是数字的映射,通常又称为函数.}.再考虑一些集合体积所具有的性质,我们可以构造出这样一个定义:

\begin{definition}{测度}
设$S$是一个集合,$\mathcal{A}$是由$S$的子集构成的一个$\sigma$-代数\footnote{即用$\mathcal{A}$中元素进行任意多次的交、并、差、补等运算,结果仍在$\mathcal{A}$中.}.称映射$\mu:\mathcal{A}\to [0, +\infty]$是$(S, \mathcal{A})$上的一个\textbf{测度(measure)},如果它满足:
\begin{enumerate}
\item $\mu(\varnothing)=0$;\\
\item 对于两两不交的至多可数个$A_i\in\mathcal{A}$,有
\begin{equation}
\mu\qty(\bigcup_i A_i) = \sum_i \mu(A_i)
\end{equation}
\end{enumerate}
\end{definition}

注意,定义中测度$\mu$的值域是$[0, +\infty]$而非$[0, +\infty)$,意味着测度值也可以取广义实数$+\infty$.由于测度值非负,因此在不至于混淆的时候,也可以用$\infty$代替$+\infty$.




















