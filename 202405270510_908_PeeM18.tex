% 2018 年考研数学试题(数学一)
% keys 考研|数学
% license Copy
% type Tutor


\textbf{声明}:“该内容来源于网络公开资料,不保证真实性,如有侵权请联系管理员”


\subsection{选择题}
1.下列函数中,在 $x=0$ 处不可导的是 $(\quad)$\\
(A)$f(x)=\abs{x} \sin \abs{x} \qquad$  (B)$f(x)=\abs{x} \sin \sqrt{\abs{x}}$\\
(c)$f(x)=\cos \abs{x} \qquad \quad$  (D)$ f(x)=\cos \sqrt{\abs{x}}$

2.过点$(1,0,0),(0,1,0)$,且与曲面$z=x^2+y^2$相切的平面为 $(\quad)$\\
(A)$z=0$ 与 $x+y-z=1$\\
(B)$z=0$ 与 $2x+2y-z=2$\\
(C)$x=y$ 与 $x+y-z=1$\\
(D)$x=y$ 与 $2x+2y-z=2$

3.$\displaystyle \sum_{n=0}^\infty (-1)^n \frac{2n+3}{(2n+1)!}$ = $(\quad)$ \\
(A)$\sin 1+\cos 1 \qquad$  (B)$2\sin 1+\cos 1$ \\
(C)$2\sin 1+2\cos 1  \quad$(D)$2\sin 1+3\cos 1$

4.设$\displaystyle M=\int_{-\frac{\pi}{2}}^\frac{\pi}{2}\frac{(1+x)^2}{1+x^2}\dd{x},N=\int_{-\frac{\pi}{2}}^\frac{\pi}{2}\frac{1+x}{e^x}\dd{x},K=\int_{-\frac{\pi}{2}}^\frac{\pi}{2}(1+\sqrt{\cos x})\dd{x}$,则 $(\quad)$ 。\\
(A)$M>N>K \quad$ (B)$M>K>N \quad$ (C)$K>M>N \quad$ (D)$K>N>M$

5.下列矩阵中,与矩阵 $\pmat{1&1&0\\0&1&1\\0&0&1}$ 相似的为 $(\quad)$\\
(A)$\pmat{1&1&-1\\0&1&1\\0&0&1} \quad$
(B)$\pmat{1&0&-1\\0&1&1\\0&0&1} \quad$
(C)$\pmat{1&1&-1\\0&1&0\\0&0&1} \quad$
(D)$\pmat{1&0&-1\\0&1&0\\0&0&1}$

6.设$\mat A,\mat B$为$n$阶矩阵,记$r(\mat X)$为矩阵$\mat X$的秩,$(\mat X,\mat Y)$表示分块矩阵,则 $(\quad)$ \\
(A)$r(\mat A,\mat {AB})=r(\mat A) \qquad \qquad$
(B)$r(\mat A,\mat {BA})=r(\mat A) \quad$\\
(C)$r(\mat A,\mat B)$=max $ \{r(\mat A),r(\mat B)\}$
(D)$r(\mat A,\mat B)=r(\mat A \Tr,\mat B \Tr)$

7.设随机变量$X$的概率密度 $f(x)$满足$f(1+x)=f(1-x)$,且 $\int_0^2 f(x)\dd{x}=0.6$, 则 $P\{X<0\}$= $(\quad)$\\
(A)$0.2 \quad$
(B)$0.3 \quad$
(C)$0.4 \quad$
(D)$0.5 \quad$


8.设总体 $X$ 服从正态分布 $N(\mu,\sigma^2).X_1,X_2,\dots,X_n$ 是来自总体 $X$ 的简单随机样本,据此样本检验假设:$H_0:\mu=\mu_0,H_1:\mu\neq \mu_0,$则 $(\quad)$\\
(A)如果在检验水平 $\alpha=0.05$ 下拒绝 $H_0$ ,那么 $\alpha=0.01$ 下必拒绝 $H_0$\\
(B)如果在检验水平 $\alpha=0.05$ 下拒绝 $H_0$ ,那么 $\alpha=0.01$ 下必接受 $H_0$\\
(C)如果在检验水平 $\alpha=0.05$ 下接受 $H_0$ ,那么 $\alpha=0.01$ 下必拒绝 $H_0$\\
(D)如果在检验水平 $\alpha=0.05$ 下接受 $H_0$ ,那么 $\alpha=0.01$ 下必接受 $H_0$

\subsection{填空题}
1.若 $\displaystyle \lim_{x \to  0} (\frac{1-\tan x}{1+\tan x})^\frac{1}{\sin kx}=e$ ,则 $k=(\quad)$。

2.设函数 $f(x)$ 具有2阶连续导数,若曲线 $y=f(x)$ 过点 $(0,0)$ 且与曲线 $y=2^x$ 在点 $(1,2)$ 处相切,则 $\int_0^1 xf''(x) \dd{x}= (\quad)$。

3.设 $F(x,y,z)=xyi-yzj+zxk$ , 则 $rot  F(1,1,0)$ = $(\quad)$。

4.设 $L$ 为球面 $x^2+y^2+z^2=1$ 与平面 $x+y+z=0$ 的交线,则 $\oint_L xy\dd{s}$ = $(\quad)$。

5.设2阶矩阵 $\mat A$ 有两个不同特征值,$\mat a_1,\mat a_2$  是 $\mat A$ 的线性无关的特征向量,且满足 $A^2(a_1+a_2)=a_1+a_2$ ,则 $\abs{A}$ = $(\quad)$。

6.设随机事件 $A$ 与 $B$ 相互独立,$A$ 与 $C$ 相互独立,$BC=\emptyset$ 若 $P(A)=P(B)=\frac{1}{2},P(AC|AB \cup C)=\frac{1}{4}$ ,则 $P(C)= (\quad)$。

\subsection{解答题}
1.求不定积分 $\displaystyle \int e^{2x}\arctan \sqrt{e^x-1}\dd{x}$。

2.将长为2m的铁丝分成三段,依次围成圆,正方形与正三角形。三个图形的面积之和是否存在最小值?若存在,求出最小值。

3.设 $\Sigma$ 是曲面 $x=\sqrt{1-3y^2-3z^2}$ 的前侧,计算曲面积分 $\displaystyle I=\int\int_\Sigma x\dd{y}\dd{z}+(y^3+2)\dd{z}\dd{x}+z^3 \dd{x}\dd{y}$。

4.已知微分方程 $y'+y=f(x)$ ,其中 $f(x)$ 是 $R$ 上的连续函数。\\
(1)若 $f(x)=x$ ,求方程的通解;\\
(2)若 $f(x)$ 是周期为 $T$ 的函数,证明:方程存在唯一的以 $T$ 为周期的解。

5.设数列 $\{x_n\}$ 满足:$x_1>0,x_n e^{x_{n+1}}=e^{x_n} -1\quad (n=1,2,\dots)$  。证明数列 $\{x_n\}$ 收敛,并求 $\displaystyle \lim_{n \to \infty} x_n$。

6.设实二次型 $f(x_1,x_2,x_3)=(x_1-x_2+x_3)^2+(x_2+x_3)^2+(x_1+ax_3)^2$,其中 $a$ 是参数。\\
(1)求 $f(x_1,x_2,x_3)=0$ 的解;\\
(2)求 $f(x_1,x_2,x_3)$ 的规范形。

7.已知 $a$ 是常数,且矩阵 $ \mat A=\pmat{1&2&a\\1&3&0\\2&7&-a}$ 可经初等列变换化为矩阵 $\mat B=\pmat{1&a&2\\0&1&1\\-1&1&1}$ 。\\
(1)求 $a$;\\
(2)求满足 $\mat {AP}=\mat B$ 的可逆矩阵 $\mat P$。

8.设随机变量 $X$ 与 $Y$ 相互独立,$X$ 的概率分布为 $P\{X=1\}=P\{X=-1\}=\frac{1}{2}$,  $Y$服从参数为 $\lambda$ 的泊松分布。令 $Z=XY$。\\
(1)求 $Cov(X,Z)$;\\
(2)求 $Z$ 的概率分布。

9.设总体 $X$ 的概率密度为
$$f(x;\sigma)=\frac{1}{2\sigma}e^{-\frac{\abs{x}}{\sigma}},\quad  -\infty<x<+\infty,~$$
其中 $\sigma \in (0,\infty)$ 为未知参数,$X_1,X_2,\dots,X_n$ 为来自总体 $X$ 的简单随机样本。记 $\sigma$ 的最大似然估计量为 $\uvec \sigma$ 。\\
(1)求  $\uvec \sigma$ ;\\
(2)求 $E(\uvec \sigma),D(\uvec \sigma)$。
