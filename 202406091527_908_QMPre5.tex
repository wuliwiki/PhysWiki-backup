% 波动和粒子(转载)
% license Usr
% type Art

(本文根据 CC-BY 协议转载自季燕江的《量子序曲》, 进行了重新排版和少量修改)

我们给自己设定一个任务,就是努力把事情说清楚,把事情说清楚就是使听众认同,如果没有听众就是使假想的听众——另一个自己——认同。这听起来有点分裂,但确实要求这另一个自己有一点“较真”和“不断置疑”的精神。

说服,或者使用图像,或者使用语言。而使用图像更有说服力。

图像就是图景,是Picture,但要让它动起来,这需要一点想象力。有时候我们会说动起来的图像比静止的图像更真实,这是由于人在大多数情况已经适应了动起来的图像,或赋予动起来的图像以更高的审美价值。

比如拍照,我们在任一瞬间拍下的照片都是“真实”的,但拿来看却往往惹人笑,表情的瞬间也许很丑,我们从来都没有看过小于$0.01$秒的瞬间,我们看的是“时间”的绵延,是处理过的富于动感的图像,我们是借助动感的图像建立起我们的审美,对人神情的,人动态的……审美。我们觉得眨眼很迷人,但我们觉得眼睛完全闭合的瞬间丑死了。

\begin{figure}[ht]
\centering
\includegraphics[width=6cm]{./figures/356af97ab00da9ae.png}
\caption{米洛的维纳斯像:看起来有动感的雕像都有不符合解剖学比例的地方。} \label{fig_QMPre5_1}
\end{figure}

在我们的想象中要让比如一个点动起来,一个三角形变大,旋转……是需要想象力的,而我们确实有这样的能力,有时我们在瞬间看到万丈光芒,极度具有结构特征的几何图形,色彩的斑驳变化,伴随着节奏、音乐的节奏,或我们思绪的内在节奏,在自己的脑海里轮番更替。

在这样的瞬间我们仿佛透彻了宇宙间所有的真理,但却没有合适的语言把它们描述下来,我们只能试一试,努力凭记忆把它们画下来,借着篝火,用颜料涂抹在山洞的顶上。

一些极富想象力的抽象作品。

据说这种抽象风格的绘画和极度写实的绘画同时出现。“写实”是对“可见”事物的模仿,而“抽象”是对“可以想见”事物的模仿。它们都是有力量的存在。

今天的人会设想,也许是出于交流或记录的目的,我们的祖先才开始绘画。但也有人说最早的绘画应该没有任何教育或交流的目的,它纯粹是出于精神的需要,是人灵性的发泄,是与神亲近的方式,所以它们才被创作于黑黢黢的山洞里面。

\begin{figure}[ht]
\centering
\includegraphics[width=6cm]{./figures/38011ba9c33d6a3b.png}
\caption{肖维岩洞(Chauvet Cave)中的绘画已经有三万多年的历史了,几乎与人类本身的历史一样悠久,图中奔跑的野牛被表现为有很多条腿的野牛。} \label{fig_QMPre5_2}
\end{figure}

\subsection{粒子的图像}

我们现在就需要通过想象而非观瞧来建立关于粒子的图像,和关于波动的图像。这是我们谈论波粒二象性的前提。

何谓粒子?

它是一个点,理想的点,只有方位,而无部分。在我的想象里它居于一个三维的空间,我可以让它上下稍微动一动,而完全不会影响其在左右方向上的位置,也不会影响其在前后方向上的位置。

假如世界只有这一个粒子,这是多么的空洞。

它可以任意地上下、左右、前后地改变位置,但对我来说是完全一样的。它都在我想象的空洞空间的中央,位置的改变无从说起。

我们必须有个参照,有了参照,我们就能说出粒子的位置了。

这个参照就是三把想象的尺子,它们互相垂直构成一个参照。或我们需要第二个点,作为原点,粒子是参照原点运动变化的。我们还需要三个箭头,标明三维空间的三个方向。

但“三”,为什么是“三”呢?

这是我们对我们所在空间的合理分类。

分类是理性的活动,是智慧的活动,谁善于分类谁就是聪明人!

“亚当证明自己是世界上第一位且最伟大的哲学家:他能根据物种的真正的本质和差异恰如其分地对它们加以区分。”

合理分类的标准是:既不重复,也不遗漏。

把我们所在的空间分解为三个不同方向是既不重复也不遗漏的。

我们也可以设想二维或一维的世界,这或者是出于限制,比如我们人类的活动就长期被限制在二维的空间上,当然是在球面上。物理学家现在也喜欢谈量子限制效应(quantum confinement effect),所谓限制就是出于某种原因,粒子(比如电子)就仅仅在二维或一维空间里运动。

想象低维空间的好处是想起来比较容易。比如落体运动,其实就是粒子在引力的作用下以越来越快的速度下落,描述这个运动只需要想象一维空间。比如炮弹的运动,就是粒子一方面以初始速度往斜刺里飞,要想飞的最远就需要以$45^o$的角度往斜刺里飞,另一方面粒子仍然受引力的作用在以越来越快的速度往地面落。

所以这是一个想象中的两个运动的叠加,先往斜刺里飞一段,再往下掉一段,然后再往斜刺里飞一段,然后再以更大速度往下掉一段。最后在我们的想象中再让这一段段锯齿缩小,让它看起来圆顺光滑一些,这就是炮弹的运动——抛物线了。

我们还可以想象很多,比如坐过山车,呼啸而下,越来越快,心悸的感觉,然后在极度的空虚中,我们向上,越来越高,在最高处,时间仿佛停止了,其实是此时速度最小,最后向下,加速,新的循环开始。

这里的窍门,是把我们自己想象成粒子,把自己的心替换为粒子的心,让我们进入粒子的世界。我们会感到有风迎面吹来,感到阻力,……

阻力是阻碍粒子运动的。我们喜欢引力,只要速度合适我们能围绕一个引力的中心(比如地球)循环往复地运动起来,一个椭圆:当我们如过山车一般冲向地球的时候,我们的速度最快,因为速度,我们从离地球最近的地方呼啸而过,然后摆脱地球,离它越来越远,向上,弯曲着向上,依靠惯性,或依靠动能($K = \frac{mv^2}{2}$)反抗地球的吸引,直到冲到离地球最远的地方,空虚地失去了太多动能,然后引力又占了上风,拉着我们加速下降,如此循环。

我们在飞,我们努力想控制飞行的轨迹,但很可惜,我们没有办法,就像梦境中的人想努力控制自己的飞行一样,徒劳和无能为力,我们在虚空中飞行,引力是唯一的外部原因,它严格地按$F = \frac{G M m}{r^2}$行为,椭圆轨道由我们的初始冲动决定,即我们在距离地球多远的地方,决定以一个什么样的速度,什么样的角度,开始运动。

(对一个二阶常微分方程$F = m a$来说,只要给定两个初始条件,初始的位置$x_0$和初始的动量$p_0$,粒子的运动就完全决定了。换句话说我们就能求解出粒子运动的轨迹。)

粒子是我们现在思维的基本单元,每个粒子都可以用质量,位置,和动量来描述。

质量就是粒子的质量。我们假设万事万物都有质量,但光子(光的粒子)除外,我们暂时先不讨论它。位置就是一个三维矢量,我们一般把它记为$\vec r$,它分解为三个固定方向上向量的叠加:

\begin{equation}
\vec r = \vec i x + \vec j y + \vec k z~
\end{equation}

动量的定义是质量乘以速度,速度定义为对位置的微分:

\begin{align}
\vec v & = & \frac{d \vec r}{d t} \\
\vec p  & = & m \vec v~
\end{align}

位置,速度,动量都是矢量,还有力,这给我们的想象力带来极大的挑战,为了思维的轻松,我们往往把它们想象为二维的或一维的。

粒子在三维空间里飞来飞去,但它并没有自由意志,它是由力和它的初始状态完全决定的,我们可以把粒子位置随时间变化的关系求出来。

\begin{equation}
 \vec r(t)~
\end{equation}

粒子如$\vec r(t)$般在时间和空间里存在,$\vec r(t)$就是粒子的世界,粒子的一生,它完全由它受到的力,它的初始位置和它的动量决定。这是很宿命的世界。

如果只存在牛顿力学,我们的世界就是这样的一个世界。万事万物不过是粒子的集合,很多很多个粒子,虽然多,但它总数的过来,比如整个宇宙中质子的数目就是$10^{80}$数量级。

只要可数,我们就可对它们列方程:可数个质点,可数个力,可数个初始位置,可数个初始动量,一个非常巨大,但确实是可数个微分方程联立,虽然我们不可能对这$10^{80}$个方程求解,但它们的解是存在的,我们求不出是因为我们人自身的局限。

如此巨大的方程,在我想象的世界里是即刻被求解出来的,$10^{80}$数量级的粒子,它们冲撞,互相缠绕,各自远离,经过无限时间后又相互吸引,重新凝聚……就如一场戏剧在笛卡尔空间这个三维的舞台上出演。

每个粒子都是莎士比亚戏剧中的一个人物,各有各的命运,但这一切在戏剧开演的一瞬就已经决定,我们张大嘴巴好奇地看,假想自己进入到戏剧里,与某个粒子化为一体,或进入环绕某个粒子运行的轨道,我们有我们的自由意志,但我们就如梦境中的人一样,我们根本就无法驾驭粒子的运动,我们徒劳地想,徒劳地扭动思想的身躯,但这场戏只由力,初始位置和动量决定,我们的命运早已被安排,只是我们不知道。而我们所有的意志都是无用的,我们不受我们自己指挥。

\begin{figure}[ht]
\centering
\includegraphics[width=6cm]{./figures/1d686f78ffd8fbf5.png}
\caption{莎⼠比亚时期的舞台提供给我们想象空间和运动的原型。} \label{fig_QMPre5_3}
\end{figure}

所谓粒子的图像就是粒子们在虚空中运动,它们各有各的质量,相互之间存在着万有引力。假如上帝是造物者的话,他在造物的瞬间会用他的大手抛洒出这$10^{80}$数量级的粒子,然后以他的全能使其按各自的方式具有初始位置和初始动量,然后他老人家就休息了,在异度空间翘着脚看质点们成形(pattern formation)演化。

这里虚空很重要,无虚空粒子就没有舞台。而所谓质点则是一些没有大小、形状的几何点,它们可以集中地携带一份能量和一份动量。虚空是古代原子论强调的,但他们认为原子有形状,因为古代原子论者没有发现相互作用(力),它们需要形状来解释物性。

古代原子论者还会强调有粒子的突转(swerve),这是想给它们灰暗的世界图景保留一点人性的努力,突转后来被看做是“自由意志”的体现,被挪用到基督教哲学中。引入突转纯粹是哲学的考虑或审美的原因,这在僵硬或被构建得很死的牛顿力学里是没有地位的。粒子有没有自由意志,会不会突然神经质似的蹦跶一下是微分方程($F = ma$)决定的,它告诉我们不行,就是不行!

牛顿力学关于世界的图景是缺乏解释力的,它归根到底只是一个关于机械运动的理论。原则上说由此出发可以构建一个物性的理论,甚至一个电磁学的理论,但实际上太不可行了以至于几乎没人尝试。

\subsection{波动的图像}

牛顿本人也研究光学,它认为光是由很多细小,运动速度很快的粒子组成的,这其实就是在延续古代原子论者的观点。光的粒子说可以解释光的直线传播,光的反射、折射,光可以被物体遮挡等常见的光学现象。

牛顿的粒子说是很不完备的,仅可看作是为了理解方便而做的一种权宜性假设,比如他并没有告诉我们光的粒子(光子)的质量是多少?它的能量是多少?它的动量又是多少?以及如何由这些基本的物理陈述出发,解释牛顿环的条纹间距。

仅仅说光子砸在玻璃上然后激起涟漪形成了圆环是很有想象力的猜测,但还不构成一个靠谱的理论。

一个靠谱的光学理论必须能够对最突出的光学现象做出定量的解释。就好像我们在氢原子的玻尔理论中体会到的,一个靠谱的关于原子的理论首先要能定量地解释这个领域里最独特而且也是最显著的现象,比如——里德堡公式。

\subsubsection{三棱镜分光实验}

波动图像的兴起和牛顿粒子图像在光学研究中的无能为力有关。但吊诡的是牛顿同时还是近代光学实验的先驱,他做的那些实验恰恰可以用波动说去解释。

这里我们只讨论他的三棱镜分光实验。

\begin{figure}[ht]
\centering
\includegraphics[width=6cm]{./figures/60e47347270f3567.png}
\caption{⽜顿的三棱镜实验。} \label{fig_QMPre5_4}
\end{figure}

彩虹是自然界中常见的现象,又或者我们喝口水对着太阳喷一口,就可看到圆弧形的颜色分布。这些都提示我们,光,看上去很纯,但其实很复杂,也许还可以进一步分类。

牛顿就是那个对光进行分类的聪明人,他是这么做的:

在封闭的屋子里,把厚厚的窗帘稍稍拉开一条缝,使光透进来,然后拿一个玻璃做的三棱镜,玻璃的折射率比较大,它可以使光比较明显地偏离原先的运动方向。但玻璃的折射率对不同波长的光是不一样的,不同波长的光会有不同偏转的角度,不同波长对应不同颜色,一束白光于是就被分成了一系列颜色的光。

假如我们只留下一种颜色,把其他颜色挡掉,继续让光通过三棱镜,我们发现光不会再继续分解了。

在我们的视觉经验里,自然光或白光是纯净的,但现在白光是复杂的,可以被某个操作继续分解,而带颜色的光反而有可能是纯净的了,它不能被这某个操作继续分解。

\begin{figure}[ht]
\centering
\includegraphics[width=6cm]{./figures/0151269f65c30d9c.png}
\caption{可见光的波长(nm)} \label{fig_QMPre5_5}
\end{figure}

初看起来这有点象变戏法,但为什么我们相信科学家(但不相信魔术师)呢?

原因是科学家不隐瞒自己的发现,科学家做演示(实验其实就是一种“公开”的演示),但他会把演示的步骤一步一步告诉你,使大家都能重复。这个说法在今天有点不确切,但主要是因为科学已经发展到极其复杂和昂贵的程度,致使普通人很难重复他们的工作,但在科学家社群内,各个竞争的小组还是可以的,否则科学发现就不会得到确认,科学活动的功利价值——通过首先发现权获取名誉和利益——就无从体现。

而魔术师就不一样了,他也表演,但他不会教你,除非你付钱成为他的徒弟。

科学活动有一套规范能够认定牛顿并非是在变戏法晃点大家,否则这么反常识的结论大家是不会承认的。而今天我们讨论科学的时候也一定要记住,科学之所以可信,并不在科学方法的严谨,也不在科学家多么有良心,而全在有一套科学实作之成规,我们信赖的是制度,而非个人或具体的科学知识。但普通人在谈论科学话题的时候往往因为知识上的欠缺,就先自己矮了一头,这是不对的,我们当然需要一定的知识做基础,但我们考虑问题的焦点应该是拷问科学实作的成规,检讨其在运作中的漏洞等等。

(这里面的道理其实很简单,假想你是个总统,你需要做决策,很多方面的决策,并对这些决策负责,但显然你不可能是每一方面的专家。)

\subsubsection{水波}

牛顿关于光学的系列实验开创了物理光学,而牛顿粒子说的无所作为则给了波动说机会。

波动也是我们日常生活中常见的图像,比如:水波。

假设我们身处一个巨大的游泳池,或者有人造的海浪或者有天然的海浪(以海为池),套上游泳圈,我们会随着波浪周期性地一起一伏,这是很好玩的体验,起来到顶是波峰,而降到低是波谷,在波峰的时候,我们会看见远处某段距离外的波峰向我们移来,当我们降到谷底的时候,我们会感到波峰离我们越来越迫近,像座小山一样扑来,然后我们随着这下一个波峰的到来和救生圈一起飞快的上升,很像坐过山车,到了波峰的顶部会有空虚的感觉,然后又加速向下,如此反复。

没有水面我们就体验不到这美妙的运动,在每个波峰来临的时候,我们都感到它巨大的力量,它狠狠地把我们从谷底掀起来,往上扔,波峰越高,我们越体验到它巨大的力量。我们无时不刻、处处体验到波的能量,当我们冲到波峰的顶部的时候我们有很大的势能,势就是“位置的优越”,因为我居于此位置,我可以把这位置的好处转化为运动的动能去冲,而当我们居于波谷的时候,我们其实是处于另一个波峰,因为水波的弯曲,水波整体的挤压,我们仍然会居“位置的优越”,或换句话说,我们仍有一个大的势能……

波的能量正比于振幅的平方($\propto A^2$),水波的能量分布在整个水面,波传播到哪里,波的能量就到哪里,我们套上救生圈就可以体验到这种能量。

波是整体的运动,当我们研究水波的时候,能量并非集中在某一点,它是能量的分布,“均匀”地分布在整个波动着的水面。此时虚空的概念就多余了,空荡荡的舞台可以让莎士比亚戏剧中的人物一个个登场,但水波必须要有个游泳池,里面装满水,水波存在于水波的表面上。

在水面上,每一点都随着水波在波动,我们如果一个一个位置地去描述水波的运动可要累死了,因为位置是不可数的(innumerable),我们没有办法用$1, 2, 3, ...$的方式遍历水面上的每一个位置。这是和我们研究牛顿粒子世界的一大区别,在那个世界里,有$10^{80}$数量级个粒子,虽然很多,但到底可数,可以用$1, 2, 3, ...$的方式穷尽。

这里我们必须说$1, 2, 3, ...$的方式其实是个很强大的方式,如果你不限定时间的话,我们可以穷尽无穷多个粒子。这个无穷有专门的名字,叫:“阿列夫零”,即最低阶的无穷多,或可数的无穷多,即用$1,2,3...$数数的方式可以穷尽的无穷多。

\subsubsection{无穷多的自由度}

由此我们已经进入了场论的研究领域,所谓场论就是研究无穷多自由度的运动。场就是物理量随时间、空间的分布,比如电场$E(x,t)$,空间上的每个点$x$都有自己的场,不同位置的场如$E(x_1)$和$E(x_2)$的取值是独立的。

$E(x_1)$和$E(x_2)$是独立的自由度,或每个不同的$x$对应的$E(x)$都是一个独立的自由度,考虑到在空间里有无穷多个$x$,我们这里就有无穷多的自由度。

(自由度就是描述一个物理系统所需要的独立变量的个数,比如对自由落体,只需要一个,即位置。)

假设我们讨论一个经典的场论。我们应如何研究波动呢?

一种方法是把它离散化,因为处处皆在的连续太难想象了。我把它们想象成为一个弹簧床垫,在场里面做想象的切割,每一个小方块,或每一个三角形,六角形,收缩成为一个质点,每个质点的质量将等于面积乘以密度,然后我在想象中让每一个质点按照某种结构相连,用假想的弹簧相连,如果你不想让你的场破碎的话,你就必须用弹簧把它们编织起来。

此时上帝变成了一个编织弹簧床垫的工人,弹簧各有各的弹性系数,它们可用弹性模量来表示。离散模型的好处是好想。一个离散的模型又部分地回到了粒子的图像,但粒子是被固定在各自平衡位置附近的,它们被弹簧束缚住,不能自由地在虚空中跑来跑去。

波的能量现在体现为所有质点的能量和,而每一个质点又由动能和势能两部分组成,很多细小的$\frac{1}{2} m v^2$和$\frac{1}{2}k x^2$之和,然后利用密度,弹性模量等概念再重新把这些关系表示为连续的情形。这时我们会得到一个用连续的场$\varphi(x,t)$表示的动能和势能。

我们使用一套由牛顿力学发展出的技巧(相当于是某种数学变换),我们不考虑力,转而考虑拉氏量$L = T - V$,它的定义是动能减去势能,由拉氏量出发我们利用一个变分求极值条件,就可以得到波动方程。波动方程和我们利用偏微分方程技巧求得的形式是相同的。它的解却可以是自由的,即波可以在弹簧床垫里自由地传播,并不衰减。

这是一个很漂亮的结果,如果你把“弹簧床垫”看做是个粒子的世界的话,这里的粒子是“固定”不动的,它们只能在各自位置的附近做微小的振动,但这些振动由于弹簧的耦合,却可把一个局部的振动向各个方向传播出去,传播给其他位置上的粒子。整体看就是一个波动在自由地传播,所谓自由指的是波动在传播的过程中并不损失能量,而且波动确实是可以传播到弹簧床垫的任意部分的。

\subsubsection{波的干涉}

两列波相遇,会发生干涉现象。这也是我们极熟悉的自然现象。

比如在夏日的傍晚,我们来到一个池塘边,有轻风吹过水面,池塘里泛起一阵涟漪,此时追踪波的运动,观察它,假如波撞在一根芦苇上,我们会发现水波似乎以芦苇为中心形成了一个新的波,你如果仔细看,还会发现波每碰到一个细小的障碍物,都会以此为中心形成一个新的波,以同心圆的形状向外扩散。

波动运行中的每一个点都可看做是一个新的波源,而波动整体的效果就是这无数波源扰动的波动的叠加。这就是惠更斯原理。

这么说很有美感,只是具体计算的时候会很麻烦。但假如波动撞到某面墙上,我们在这墙上只开两个小孔,这个运算就会变得简单。我们只需要计算两列波的叠加,这两列波其实是源自同一个波的,是我们把它们的兄弟姐妹们都挡住了。

如前所述,波动可以看做是相位的奔跑,相位是$k x - \omega t$,假如我们考虑某一时刻波动的情况,即$t$是固定的,我们需要看的是$k x$。

现在一列波由A出发按照某个路径(Path A)跑了$x_A$距离,相位是$k x_A$,而另一列波由B出发按照某个路径(Path B)跑了$x_B$距离,相位是$k x_B$,假设这两列波最终在C相遇。

如果相位差$k (x_A - x_B)$正好是$2 \pi$的整数倍,意味着两列波是同步的,振幅会变为$2A$,而波的强度则会正比于$4A^2$。

但假如相位差$k (x_A - x_B)$正好是$\pi$的奇数倍,则意味着当一列波位于波峰时,另一列波必位于波谷,它们总是相消的,振幅会变为$0$,波的强度也只能是0了,这是无法再弱的情况了。

当然还有居于二者之间的情况,但我们不妨把最强和最弱(0)当做标志。

\begin{figure}[ht]
\centering
\includegraphics[width=6cm]{./figures/1972ea5517bdda3b.png}
\caption{双缝实验示意图。} \label{fig_QMPre5_6}
\end{figure}

这就是著名的双缝实验,我们将会在双缝的对面观察到一系列振动“最强、最弱、最强、最弱……”的条纹分布。

对水波而言,我们是能直接看到这种现象的,所谓水波涟漪。就是这种波峰、波谷错综相杂的条纹镶嵌。

光的干涉要难些,首先对光源有要求,我们必须用特定波长的光源,而且光源要稳定,行话叫相干光;另外由于光的波长本来就很短,这需要我们专门设计实验才能观察到干涉现象。

光的干涉现象用波动图像很容易解释,而这仅仅是个开始,还有衍射现象,艾里斑等。衍射光栅因为对分辨不同频率/波长的光具有特别强的分辨本领成为测量光谱现象强大的工具。

物理光学就是波动光学,它很快超越几何光学成为光学研究的主流。比如我们可以用衍射光栅测量氢原子的光谱,类似地还有钠原子光谱,我们只要在炙热的火焰里放进某种元素的金属丝,我们就可以研究这种元素的光谱了。

到19世纪,光的波动说开始全面占上风。等到赫兹做实验证明电磁波也是波动,并且如光波一样有干涉、衍射等现象,物理学家普遍相信光波就是一种电磁波,光的粒子说就正式被扫进了历史的故纸堆。(就像我们现在说到以太或燃素的感觉一样)

但情况在一点点的起变化。

\subsection{黑体辐射}

19世纪末,随着工业的进步,人们开始用衍射光栅去研究黑体辐射。所谓黑体就是一个大空腔,这个空腔包纳着一个大空间,这个空间就是电磁场的海洋,各种电磁波在空腔里碰撞奔腾着,就好像是个电磁波的冲浪游乐场。空腔(把它想象为太上老君炼丹的炉子)的温度是可以调节的,电磁辐射场与空腔达成了热平衡,所谓热平衡是个与材质无关的事情,这很平等,铁的$300^o$和陶瓷的$300^o$是一样的,它们都能和$300^o$的电磁辐射场达到热平衡。

我们还记得热交换的几种途径:传导、对流和辐射。这个辐射就是电磁辐射,太阳离我们很远,中间又没有导热的介质,传导和对流都无从谈起,靠的就是辐射。所以我们说辐射、热辐射,还是电磁辐射讲的都是一个东西。

我们可以在想象中把电磁辐射场的一部分画个范围,其内是个物体,这个物体——其实就是电磁辐射场——与腔体通过辐射达成热平衡,假如腔体的温度是$T$的话,那么电磁辐射场也处在温度$T$。

现在温度进来了。

在统计物理的框架内,温度和无规则的热运动有关,无规则代表着信息的缺失,我们没法知道系统内每一个质点的运动,但我们知道它们作为整体的统计规律,这个规律说,质点具有动能$K$的几率正比于玻尔兹曼因子$e^{-\frac{K}{kT}}$。这个概念我们可以推广运用到黑体辐射中。

所谓黑体辐射就是很多电磁波的振动,很多振动的模式,它们的能量正比于电场强度的平方,也正比于磁场强度的平方,而现在电场和磁场就相当于质点的动能$K$。

我们不进入计算的细节,只是声明统计物理现在就能排上用场了,我们有很多振动的模式,每个振动模式用不同的波矢$\vec k$表示,很多波矢,每个波矢对应的振动都有能量,我们希望把这些能量都加起来,我们现在并不知道每个振动的能量,我们求热力学平均,再把它们加起来,并把能量的密度$u_T(\nu)$用波长或频率表示出来。

\begin{equation}
u_T(\nu) = \frac{ 8 \pi }{ c^3 } \nu^2 k_B T~
\end{equation}

这是一个很漂亮的计算,就一个毛病,计算出来的能量密度对短波长(或高频率,所谓紫外)是发散的,即$\nu \to \infty$时,$u_T(\nu) \to \infty$。这导致了一个荒唐的结果,即每单位空间我们都有无穷多的能量在等着我们。而且理论计算出来的能量密度随波长的分布与实际实验测量的结果也不匹配。

1900年,普朗克利用内插法凑出了一个黑体辐射的能量密度公式,与实验数据符合的很好:

\begin{equation}
u_T (\nu )  = \frac{ 8 \pi h \nu^3  }{ c^3 } \cdot  \frac{1 }{ e^{ \frac{ h \nu }{ k_B T } }  - 1 }~
\end{equation}

这个公式叫普朗克公式,它是凑出来的,我们还需要把它推导出来。但考虑到三维太复杂了,我们先考虑一维,然后由一维的结果推广。

首先对电磁波而言,每个波矢$k$都有两个独立的振动模式,比如一个是$x$线偏振光,另一个模式就是$y$偏振光。其次对大小为$L$的一维空间而言,电磁波存在于这样一个空间里需要满足一定的边界条件,比如在边界处某个方向的分量为0等等。这要求波矢满足这样的关系:

\begin{equation}
k L = n \pi~
\end{equation}

这里$n = 1, 2, ...$,可见并不是一个$k$允许,而是一系列的$k$都满足边界条件和波动方程,所以我们考虑的空间里是很多不同波矢电磁波的叠加。

把这个结果推广到三维:

\begin{equation}
\frac{1}{8} \cdot 2 \cdot \frac{4 \pi k^3}{3} V = N \pi^3~
\end{equation}

针对这个公式做一些说明:

\begin{enumerate}
\item 

因子$\frac{4 \pi k^3}{3}$对应的是波矢小于等于$k$包围起来的波矢空间的体积;

\item

因子$\frac{1}{8}$表示我们这里的求和是只针对$k_x >0, k_y > 0, k_z >0 $进行的,对应是波矢空间里的$\frac{1}{8}$体积;

\item 

因子$2$指的是对每个波矢,有两个独立的振动模式;

\item

V是电磁波所在空间的体积;

\item

$N$代表波矢小于等于$k$时系统允许的独立振动模式的个数;

\item

$\pi^3$,因为现在是三维了,波矢空间的体积乘以空间的体积对应的应该是$\pi^3$。

\end{enumerate}

黑体辐射能量密度$u_T(\nu)$是单位体积内处于频率由$\nu $到$\nu + d\nu $的电磁波的能量,当然是热力学平均意义下的能量。

现在首先要问单位体积内处于频率由$\nu $到$\nu + d\nu $的独立的电磁振荡的数目有多少,我们管它叫“态密度”(Density of states),写成数学的式子就是:

\begin{equation}
\frac{1}{V} \frac{d N (\nu)}{d \nu}~
\end{equation}

首先由$\frac{1}{8} \cdot 2 \cdot \frac{4 \pi k^3}{3} V = N \pi^3$,我们可求出$N(k)$,

\begin{equation}
\frac{N(k )}{ V } = \frac{ k^3 }{3 \pi^2}~
\end{equation}

然后根据$k = \frac{2 \pi }{ \lambda}  = \frac{2 \pi \nu}{c}$,把$N(k)$变换为$N(\nu )$:

\begin{equation}
\frac{N(\nu)  }{ V } = \frac{ 8 \pi }{ 3 } \frac{ \nu^3  } { c^3 }~
\end{equation}

现在态密度$g (\nu)$是:

\begin{equation}
g(\nu) = \frac{1}{V} \frac{d N (\nu)}{d \nu} = \frac{8 \pi \nu^2 }{ c^3 }~
\end{equation}

现在黑体辐射的能量密度可表示为:

\begin{equation}
u_T (\nu) = g (\nu) \left\langle \epsilon (\nu)  \right\rangle_T~
\end{equation}

这里$\left\langle \epsilon (\nu)  \right\rangle_T$表示对振荡频率为$\nu$的电磁波能量的热力学平均。

如果使用经典的统计物理,我们需要做的就是对电磁波的能量$ \frac{\epsilon_0}{2} E^2 + \frac{1}{2 \mu_0} B^2 $进行热力学平均,计算结果为$k_B T$,于是我们得到了先前的那个会发生紫外发散的能量密度。

\begin{equation}
u_T(\nu) = \frac{ 8 \pi }{ c^3 } \nu^2 k_B T~
\end{equation}

这个公式最早是由瑞利和金斯计算出来的,因此也叫瑞利-金斯公式。

普朗克面对的就是这样一个问题。他并没有直接假设电磁波本身是量子化的,他假设的是电磁波与物质发生能量交换时,必须以一份一份的方式进行,一份就是一个量子(quanta),大小是:

\begin{equation}
h \nu~
\end{equation}

这里$h = 6.626 \times 10^{-34 }$焦耳·秒,叫做普朗克常数。现在频率为$\nu$的电磁波按普朗克的假设就是它可以用$n = 0 , 1, 2, ...$的整数份额与物质交换能量。

而以我们现在的观念则说电磁辐射本身是量子化的,对频率为$\nu$的振荡模式,它可能存在1个激发,也可能是2个激发……,当然也可以是0个激发,对应的就是没有一个$h \nu$光子的状态。

现在我们可以按照统计物理的基本法则计算$\left\langle \epsilon (\nu)  \right\rangle_T$:

\begin{align}
\left\langle \epsilon (\nu)  \right\rangle_T &= \frac{ \sum\limits_{n=0}^{\infty} n h \nu e^{- \beta n h \nu } }{ \sum\limits_{n=0}^{\infty} e^{- \beta n h \nu  } } =  - \frac{\partial }{\partial \beta } \ln \sum\limits_0^{\infty} e^{-\beta n h \nu } \\
{} &= - \frac{\partial }{\partial \beta } \ln \frac{1}{ 1- e^{- \beta h \nu} } = \frac{\partial }{\partial \beta} \ln \left( 1 - e^{- \beta h \nu} \right)~
\end{align}

这里$\beta = \frac{1}{k_B T}$,所以:

\begin{equation}
\left\langle \epsilon (\nu)  \right\rangle_T = \frac{ h \nu e^{- \beta h \nu } }{1 - e^{ - \beta h \nu }} = \frac{ h \nu  }{e^{\beta h \nu} - 1 }~
\end{equation}

最终我们得到了普朗克公式:

\begin{equation}
u_T (\nu )  = \frac{ 8 \pi h \nu^3  }{ c^3 } \cdot  \frac{1 }{ e^{ \frac{ h \nu }{ k_B T } }  - 1 }~
\end{equation}

在黑体辐射实验之前光的波动说可谓是牢固地竖立起来了,但普朗克的量子假说打开了一个缺口,这之后的光电效应、康普顿散射等实验都更进一步说明光在这些实验里必须被理解为粒子才可以获得一种简单的解释。

\subsubsection{维恩位移定律}

热辐射即固体、液体或气体由于自身温度($T$)而释放的辐射。用三棱镜使热辐射折射,不同波长的光会被三棱镜折射到不同的方向上,测量不同方向上的光强分布,我们得到连续谱,即一个强度随波长连续变化的分布。

温度低于500摄氏度时,大部分热辐射能量在红外波长的区域。随着温度的升高,辐射能量的主要部分逐渐向短波高频端移动,移动到可见光,紫外光等等。

\begin{figure}[ht]
\centering
\includegraphics[width=6cm]{./figures/a4d544b82261c256.png}
\caption{ 维恩位移定律。} \label{fig_QMPre5_7}
\end{figure}

看来热辐射的主要部分的波长和温度是成反比的,这就是维恩位移定律(Wien's displacement law):

\begin{align}
\text{$\lambda_{max} T = 2.898 \times 10^{-3 }$米·开尔文}~
\end{align}

这里$\lambda_{max}$是使谱分布$u_T(\lambda)$取最大值的波长。

这个现象本身是很常见的,比如我们在看电影《大闹天宫》的时候,太上老君把孙悟空扔进炼丹炉里烧,炼丹炉除一个开口外都是封闭的,而那个开口就是黑体,随着炉子里面的温度逐渐升高,窗口会显现不同的颜色,按照波长由长到短,或频率由低到高的次序依次出现,“红橙黄绿青蓝紫”。

用这个方法我们可以确定远方恒星的温度,比如一颗黄星的表面温度就比一颗红星的表面温度高。

\subsubsection{斯特番-玻尔兹曼定律}

物体的温度越高,发射的总能量就越多,这就是斯特番-玻尔兹曼定律(Stefan and Boltzman's law):

\begin{equation}
R(T) = \sigma T^4~
\end{equation}

这里$R(T)$是辐射本领,被定义为单位表面积物体向外辐射出的功率。

\begin{equation}
\sigma = 5.67 \times 10^{-8} J s^{-1} m^{-2} K^{-4}~
\end{equation}

叫做斯特番-玻尔兹曼常数。

我们现在可以来做一个练习:

假设太阳、地球、火星、金星都可看作是绝对黑体,利用斯特藩-玻尔兹曼定律,估算金星、地球和火星的表面温度,这给出了一个讨论温室效应的定量框架。

太阳半径: $R_S  = 6.96 \times 10^5 km$

太阳表面温度: $T_S  = 5780K$

水星距离太阳的距离是0.39AU,火星距离太阳的距离是1.52AU (1AU就是地球到太阳的距离,$1AU = 1.496 \times 10^8 km$)

根据斯特番-玻尔兹曼定律,太阳的总辐射功率为:

\begin{equation}
P = \sigma T^4 4 \pi R_S^2~
\end{equation}

其中只有一部分会照在“地球”, 比例是: 

\begin{equation}
\frac{\pi R_E^2}{4 \pi D^2}= \frac{R_E^2}{4 D^2}~
\end{equation}

这里$D$是地球到太阳的距离(地心到日心)。

假设地球表面温度是$T_E$,地球向外辐射的功率是: 

\begin{equation}
\sigma T_E^4 4 \pi R_E^2~
\end{equation}

当这个辐射功率和吸收功率相等时,可估算出地球表面的温度。

\begin{equation}
\sigma T^4 4 \pi R_S^2 \cdot \frac{R_E^2}{4 D^2} = \sigma T_E^4 4 \pi R_E^2~
\end{equation}

解出:

\begin{equation}
T_E = T \sqrt{\frac{R_S}{2D}} = 278.77 K~
\end{equation}

这个温度与地球半径($R_E$)无关。更多结果见下表:

\begin{figure}[ht]
\centering
\includegraphics[width=6cm]{./figures/1325cb9ca390200f.png}
\caption \label{fig_QMPre5_8}
\end{figure}
