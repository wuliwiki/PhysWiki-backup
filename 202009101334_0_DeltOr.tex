% 狄拉克 delta 函数的正交归一性

\begin{issues}
\issueTODO
\end{issues}

\pentry{狄拉克 delta 函数\upref{Delta}}

从某种意义上来说, 狄拉克 $\delta$ 函数是正交归一的
\begin{equation}\label{DeltOr_eq1}
\int_{-\infty}^{\infty}\delta(x - x_1) \delta(x - x_2) \dd{x} = \delta(x_2 - x_1)
\end{equation}

这就像傅里叶变换\upref{FTExp}中, 简谐波也是两两归一的(\autoref{FTvec_eq1}~\upref{FTvec})
\begin{equation}
\int \frac{\E^{\I k_1 x}}{\sqrt{2\pi }} \frac{\E^{-\I k_2 x}}{\sqrt{2\pi }} \dd{x} = \delta (k_2 - k_1)
\end{equation}

这里来给出一个不严谨的 “证明”. 我们可以像\autoref{Delta_fig1}~\upref{Delta} 所示的那样, 先假设 $\delta(x)$ 具有某个具体的函数表达式, 计算\autoref{DeltOr_eq1} 的定积分, 结果记为 $I(x_2 - x_1)$, 然后再取极限让这些函数变得无穷窄. 在这个过程中我们会看见 $I(x_2 - x_1)$ 也是关于 $x_1$ 或 $x_2$ 的 $\delta$ 函数, 即变得无穷窄, 且曲线下的面积趋近于 1.

\addTODO{举例}
