% 动量表象的薛定谔方程

本文使用原子单位制\upref{AU}. 一维薛定谔方程为
\begin{equation}
-\frac{1}{2m} \pdv[2]{x}\psi(x, t) + V(x,t)\psi(x,t) = \I \pdv{t} \psi(x,t)
\end{equation}
可以证明, 如果 $V(x,t)$ 关于 $x$ 可以被泰勒展开, 那么动量表象的薛定谔方程为
\begin{equation}
-\frac{k^2}{2m}\varphi(k, t) + V(\I \pdv{k})\varphi(k, t) = \I \pdv{t} \varphi(k,t)
\end{equation}

证明: 把
\begin{equation}
\psi(x,t) = \int_{-\infty}^{+\infty}
\end{equation}
