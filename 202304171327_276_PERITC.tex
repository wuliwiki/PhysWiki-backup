% 包晶相图


\pentry{匀晶相图\upref{ISOMOR}}
\footnote{本文参考了刘智恩的《材料科学基础》}包晶转变:由一个液相与一个固相生成另一种固相。 $L + \alpha \rightarrow \beta$

在热力学中,往往只关心相的变化;但由于动力学因素,实际冷却时,各相往往形成一定的有序组织结构。本文一并简要讨论,以Pt-Ag合金的平衡冷却为例。

\subsection{包晶相图}

\begin{figure}[ht]
\centering
\includegraphics[width=10cm]{./figures/a1cebc5caa072440.pdf}
\caption{典型的包晶系合金$Pt-Ag$相图.数据来源:刘智恩《材料科学基础》。仅供示意,未按实际比例绘制。} \label{fig_PERITC_4}
\end{figure}

\subsubsection{$10.5\%<wAg<42.4\%$的合金} 
\begin{figure}[ht]
\centering
\includegraphics[width=14cm]{./figures/3be45f11094e0354.pdf}
\caption{$10.5\%<wAg<42.4\%$的合金。} \label{fig_PERITC_2}
\end{figure}

\begin{itemize}
\item 在$A$点处(液相线以上),系统中只有液相$L$。
\item 在$B$点处(液相线以下,固相线以上),系统先发生匀晶转变,析出$\alpha$相固体。
\item 在$C$点处(固相线),系统中开始析出$\beta$相。注意\textbf{不同于}共晶转变\upref{EUTECT}中“一个液相生成两个固相$L\to\alpha+\beta$”,此处的转变是原有的固相$\alpha$与液相$L$转变为另一种新的一种固相$\beta$: $\alpha+L\to\beta$,被称为包晶转变。
\item 由于实际原子扩散的需要,$\beta$相往往围绕$\alpha$相生成,并不断消耗液相与$\alpha$相长大,形成独特的包晶组织。
\begin{figure}[ht]
\centering
\includegraphics[width=10cm]{./figures/0feb6ec781419d7c.pdf}
\caption{包晶转变中的原子扩散。} \label{fig_PERITC_6}
\end{figure}
\item 在$D$点处(包晶线以下),液相被完全消耗,但$\alpha$相仍存在。此时系统由$\alpha$与$\beta$组成,但系统的组织结构明显不同于共晶组织。
\item $D$点以下,发生脱熔转变,过饱和溶质开始析出。
\end{itemize}

从相图中我们可以得知:
\begin{itemize}
\item 包晶转变是恒成分转变,即包晶转变中,先后结晶的部分的成分一致
\item 包晶转变是恒温转变: f=2-3+1=0,系统没有自由度
\end{itemize}
包晶转变的这两条性质与共晶转变的类似。

全程的相转变:$L \rightarrow \alpha+\beta~.$

全程的组织转变:$L \rightarrow (\alpha+\beta)_{Peritectic} + \alpha_{II} + \beta_{II}$

% \subsubsection{$wAg=42.4\%$的合金} 

% \begin{figure}[ht]
% \centering
% \includegraphics[width=14cm]{./figures/PERITC_1.pdf}
% \caption{$w_{Ag}=42.4\%$的合金} \label{fig_PERITC_1}
% \end{figure}

% 全程的相转变:$L \rightarrow \alpha+\beta$

% 全程的组织转变:$L \rightarrow \beta + \alpha_{II}$

% D点以上,先发生匀晶转变 $L \rightarrow \alpha$

% D点处,发生包晶转变,由固相α与液相L生成一个固相β, $L + \alpha \rightarrow \beta$。此处相变完成后,L与α均被完全消耗,系统由β组成。

% D点以下,发生脱熔转变。$\beta \rightarrow \alpha_{II}$

\subsubsection{$42.4\%<wAg<66.3\%$的合金}
\begin{figure}[ht]
\centering
\includegraphics[width=14cm]{./figures/3d9ff5dfd604664d.pdf}
\caption{$42.4\%<wAg<66.3\%$的合金} \label{fig_PERITC_5}
\end{figure}

\begin{itemize}
\item $A$、$B$点处系统的状态类似上文所述。
\item 在$C$点处(包晶线处),系统同样发生包晶转变。然而此处转变完成后,被完全消耗的反而是$\alpha$相。
\item 在转变完全后的$D$处(包晶线以下,固相线以上),系统仅由$\beta$相与L组成。随后发生匀晶转变,液体逐渐凝固为$\beta$相。
\item 在$E$点处,液体完全凝固为$\beta$相。
\item 随后系统又进入$\alpha+\beta$的双相区。这是因为$\beta$相过饱和,过饱和的溶质以$\alpha$相点形式析出。
\end{itemize}

全程的相转变:$L \rightarrow \alpha+\beta$

全程的组织转变:$L \rightarrow \beta + \alpha_{II}$

\subsubsection{其余成分的合金}
仅发生简单的匀晶-脱熔转变,不再列举。

