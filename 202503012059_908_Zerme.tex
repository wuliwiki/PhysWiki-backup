% 恩斯特·策梅洛(综述)
% license CCBYSA3
% type Wiki

本文根据 CC-BY-SA 协议转载翻译自维基百科\href{https://en.wikipedia.org/wiki/Ernst_Zermelo}{相关文章}。

\begin{figure}[ht]
\centering
\includegraphics[width=6cm]{./figures/e3826f3f611894f0.png}
\caption{} \label{fig_Zerme_1}
\end{figure}
恩斯特·弗里德里希·费迪南德·策梅洛(Ernst Friedrich Ferdinand Zermelo,发音:/zɜːrˈmɛloʊ/;德语:[tsɛɐ̯ˈmeːlo];1871年7月27日-1953年5月21日)是德国的逻辑学家和数学家,他的工作对数学基础具有重要影响。他因在发展策梅洛-弗伦克尔公理化集合论以及证明良序定理方面的贡献而闻名。此外,他在1929年关于国际象棋棋手排名的研究,首次描述了一种对偶比较模型,这一方法在多个应用领域中继续产生深远的影响。
\subsection{生活}
恩斯特·策梅洛于1889年毕业于柏林的路易森斯特第高等学校(现为海因里希·施利曼中学)。随后,他在柏林大学、哈雷大学和弗赖堡大学学习数学、物理和哲学。他于1894年在柏林大学完成了博士学位,论文题目为变分法(Untersuchungen zur Variationsrechnung)。策梅洛继续留在柏林大学,成为普朗克的助手,并在其指导下开始研究流体动力学。1897年,策梅洛前往哥廷根大学,这时的哥廷根大学是世界领先的数学研究中心,他于1899年完成了博士后资格论文。

1910年,策梅洛离开哥廷根,受聘为苏黎世大学数学系的教授,直到1916年辞职。他于1926年被授予弗赖堡大学的名誉教授职位,但因不满阿道夫·希特勒的政权,於1935年辞去该职位。在第二次世界大战结束后,应策梅洛的请求,他被重新恢复了弗赖堡的名誉教授职务。


