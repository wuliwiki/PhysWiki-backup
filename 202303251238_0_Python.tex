% Python 简介与安装
% Python|科学计算|Jupyter|模块|函数|变量

\begin{itemize}
\item \autoref{Python_sub1} Python 简介
\item \autoref{Python_sub2} Python 安装
\end{itemize}

\subsection{Python 简介}\label{Python_sub1}

\subsubsection{1.1.Python 是什么?}

\begin{itemize}
\item Python 是一门流行的编程语言。由荷兰数学和计算机科学研究学会的吉多·范罗苏姆于1990年代初设计。
\item Python 是一门解释型编程语言: 这意味着开发过程中没有了编译这个环节。类似于PHP和Perl语言。
\item Python 是一门交互式编程语言: 这意味着,您可以在一个 Python 提示符 >>> 后直接执行代码。
\item Python 是一门面向对象的语言: 这意味着Python支持面向对象的风格或代码封装在对象的编程技术。
\end{itemize}
 
\subsubsection{1.2.Python 可以做什么?}
\begin{itemize}
\item Web开发:可以在服务器上使用 Python 来创建 Web 应用程序。
\item 软件开发:可用于快速原型设计,也可用于生产就绪的软件开发。
\item 数学运算:Python 可用于处理大数据并执行复杂的数学运算。
\item 系统脚本:可以与软件一起使用来创建工作流。
\end{itemize}

\subsubsection{1.3.Python 有什么特点?}

\begin{itemize}
\item 易于学习:Python有相对较少的关键字,结构简单和一个明确定义的语法。
\item 易于阅读:Python代码定义的更清晰。
\item 易于维护:Python的成功在于它的源代码是相当容易维护的。
\item 可嵌入: 你可以将Python嵌入到C/C++程序,让你的程序拥有"脚本化"的能力。
\item 广泛的标准库:Python的最大的优势之一是丰富的库,且跨平台在 UNIX,Windows 和 Mac 兼容很好。
\item 可扩展:如果你需要一段运行很快的关键代码,或者是想要编写一些不愿开放的算法,你可以使用C或C++完成那部分程序,然后从你的Python程序中调用。
\end{itemize}

%如果不想安装软件可以直接用浏览器访问 \href{https://jupyter.org/}{Jupyter Notebook} 运行 Python 程序, 要在本地使用 Python 推荐安装 Anoconda 或 miniconda\upref{CondaN}。 以下我们用前者进行讲解。 Jupyter Notebook 的优点是交互式编程, 即每输入一个命令都可以立即执行(快捷键 Shift + Enter), 利于学习和实验。

\subsection{Python 安装}\label{Python_sub2}
Python 已经能够工作在不同平台上。您需要下载适用于您使用平台的二进制代码,然后安装 Python。如果您平台的二进制代码是不可用的,你需要使用C编译器手动编译源代码。编译的源代码,功能上有更多的选择性, 为 Python 安装提供了更多的灵活性。\textbf{不过需要注意:太高版本的Python部分库是无法安装的(推荐3.8.6)}

\subsubsection{2.1.Window 平台安装 Python:}
访问 \href{https://www.python.org/downloads/windows/}{Python 官网关于 Windows 下载},一般就下载  Windows installer,其中“Stable Releases”是指稳定版本,推荐下载。
\begin{figure}[ht]
\centering
\includegraphics[width=14.25cm]{./figures/Python_1.png}
\caption{Python 官网关于 Windows} \label{Python_fig1}
\end{figure}
安装以后在开始菜单中搜索程序如 Python 3.8, 点击后即可打开 Python 命令行。 安装包也会自动安装 \verb|pip3|, 可以在 Powershell 或者 cmd 中输入 \verb|pip3 --version| 查看。

\subsubsection{2.2.MAC 平台安装 Python:}
MAC 系统都自带有 Python2.7 环境,你可以在链接 \href{https://www.python.org/downloads/mac-osx/}{Python 官网关于 mac 下载} 上下载最新版安装 Python 3.x。你也可以参考源码安装的方式来安装。

\subsubsection{2.3.Unix 和 Linux 平台安装 Python:}
你可以访问 \href{https://www.python.org/downloads/source/}{Python 官网关于 Linux 下载},选择适用于 Unix/Linux 的源码压缩包。
\begin{figure}[ht]
\centering
\includegraphics[width=14.25cm]{./figures/Python_2.png}
\caption{Python 官网关于 Linux} \label{Python_fig2}
\end{figure}

也可以直接用命令行下载
\begin{lstlisting}[language=bash]
wget https://www.python.org/ftp/python/3.7.6/Python-3.7.6.tgz
\end{lstlisting}

创建安装目录(你想放哪就放哪)
\begin{lstlisting}[language=bash]
mkdir -p /usr/local/python3
\end{lstlisting}

解压
\begin{lstlisting}[language=bash]
tar -zxvf Python-3.7.6.tgz
\end{lstlisting}

编译安装
\begin{lstlisting}[language=bash]
# gcc 环境、zlib 库和 readline-devel 包
yum -y install gcc
yum -y install zlib*
yum install readline-devel
# 配置
cd Python-3.7.6
./configure --prefix=/usr/local/python3
# 编译安装
make && make install
\end{lstlisting}

建立软链接
\begin{lstlisting}[language=bash]
ln -s /usr/local/python3/bin/python3.7 /usr/bin/python3
ln -s /usr/local/python3/bin/pip3.7 /usr/bin/pip3
\end{lstlisting}

测试安装
\begin{lstlisting}[language=bash]
# 返回 Python 3.7.6(版本)
python3 --version
# 命令行输出
python3
......
print("你好")
\end{lstlisting}

\textbf{百科下一个词条—Python - IDE安装-\upref{PyIDE}}

\textbf{教程下一个词条——Python 解释器的使用\upref{PyItpt}}
