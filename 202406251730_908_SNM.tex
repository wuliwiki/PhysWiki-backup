% 势能面
% license CCBYSA3
% type Wiki

(本文根据 CC-BY-SA 协议转载自原搜狗科学百科对英文维基百科的翻译)

势能面用某些参数,通常是原子的位置来描述一个系统(如原子集合)的能量。表面可以将能量定义为一个或多个坐标的函数;如果只有一个坐标,该表面称为势能曲线或能量剖面。莫尔斯/长程潜力就是一个例子。

使用地形图的类比是有帮助的:对于具有两个自由度(例如两个键长)的系统,能量的值(类比:陆地的高度)是两个键长的函数(类比:地面上位置的坐标)。[1]

势能面的概念适用于化学和物理等领域,特别是这些学科的理论分支。它可以用于理论上探索由原子组成的结构的性质,例如,找到分子的最小能量形状或计算化学反应的速率。

\subsection{数学定义和计算}
一组原子的几何位置可以用向量r来描述,向量$r$的元素代表原子的位置。向量$r$可以是原子笛卡尔坐标的集合,或者也可以是原子间距离和角度的集合。

给定r,能量作为位置的函数,$E(r)$是所有感兴趣的$r$的$E(r)$值。利用引言中的地形图类比,E给出了“能量地形图”的高度,从而产生了势能面的概念。

为了研究使用势能面作为原子位置函数的化学反应,有必要计算每个感兴趣的原子排列的能量。计算特定原子排列的能量的方法在计算化学文章中有很好的描述,这里的重点将是找到$E(r)$的近似值,以产生精细的能量位置信息。

对于非常简单的化学系统,或者当对原子间相互作用进行简化近似时,有时可以使用作为原子位置函数的能量的解析导出表达式。一个例子是伦敦-艾林-波兰尼-佐藤潜力[2][3][4] 作为三个氢-氢距离的函数的系统氢+ H2。

对于更复杂的系统来说,计算特定原子排列的能量对于表面的大规模表示来说通常太过昂贵而不可行。对于这些系统,一种可能的方法是仅计算势能面上的一组简化的点,然后使用计算成本更低的插值方法,例如谢泼德插值,来填充间隙。[5]

\subsection{应用}
势能面是一个概念工具,有助于分析分子几何和化学反应动力学。一旦在势能面上评估了必要的点,就可以根据能量相对于位置的一阶导数和二阶导数对这些点进行分类,这两个导数分别是梯度和曲率。静止点(或零梯度点)具有物理意义:能量最小值对应于物理稳定的化学物质,鞍点对应于过渡态,即反应坐标上的最高能量点(它是连接化学反应物和化学产物的最低能量路径)。