% 黎曼积分
% 积分|黎曼积分|微积分

\pentry{定积分\upref{DInt}}
“定积分\upref{DInt}"里给出了定积分的定义,并且强调了该定义只适用于有解函数的情况。也就是说定积分的存在是有条件的,所以自然而然就出现这样的疑问:是否有一般的办法断定积分是否存在?这就是本词条要完成的任务。
\subsection{达布和}
定积分的定义(\autoref{def_DInt_3}~\upref{DInt})中,积分和(黎曼和)的极限要求 $\xi_i$ 在区间 $[x_i,x_{i+1}]$ 上的选择是任意的。为简化研究,除积分和外,按照达布(Darboux)的方法,引进一类更为简单的和。
\begin{definition}{达布和}
在将区间 $[a,b]$ 插入分点进行划分时,如果在每一部分区间 $[x_i,x_{i+1}]$ 上都选取使得函数 $f(x)$ 在该区间上为最大(最小)的 $x\in[x_i,x_{i+1}]$,则得到的积分和称\textbf{达布和},或\textbf{上(下)积分和}。
\end{definition}
显然,达布和仅仅是给出了 $\xi_i$ 在区间 $[x_i,x_{i+1}]$ 的两种选择方式:使 $f(x)$ 在该区间上最大或最小。所以它只是某种特殊的积分和。

为方便起见,设
\begin{equation}
\mathcal P = \qty{a = x_0 < x_1 < \dots < x_n = b}~,
\end{equation}
是给定的分点序列(或区间 $[a,b]$ 的分划)。用 $m_i,M_i$ 分别代表函数 $f(x)$ 在第 $i$ \footnote{从0开始计数}个部分区间 $[x_i,x_{i+1}]$ 的最大与最小值,而下、上积分和记为
\begin{equation}
s= \sum_{k = 1}^n m_k\Delta k, \qquad  S = \sum_{k = 1}^n M_k \Delta_k~,
\end{equation}

显然,$s,S$ 分别是给定分划时积分和中的\textbf{下确界}和\textbf{上确界}(\autoref{def_SupInf_2}~\upref{SupInf})。因为如果数集有最大值和最小值,则最大值和最小值就是数集的上确界和下确界。这个证明很简单,因为上确界是上界中的最小者,而如果数集有最大值 $a$,那么它的上界 $A$ 必须满足 $A\geq a$,由这不等式容易知道,上界中的最小值就是 $a$,同样下确界可以类似证明。
\subsubsection{达布和的性质}
达布和的两个基本性质可用如下两个定理说明。
\begin{theorem}{}\label{the_Rieman_1}
若在原来分划中加入新的分点,则达布下和只能增大,而达布上和只能减小。
\end{theorem}
\textbf{证明:}
我们仅讨论下和,上和可类似讨论。只需讨论在任一部分区间 $[x_i,x_{i+1}]$ 内加入一个新的分点 $x'$ 即可,即
\begin{equation}
x_i<x'<x_{i+1}
\end{equation}
用 $s'$ 表示新的下和,那么,$s,s'$ 仅在这个区间  $[x_i,x_{i+1}]$ 有所不同,两者对应于该区间的项分别是
\begin{equation}
m_i(x_{i+1}-x_i),\qquad m_{1i}(x'-x_i)+m_{2i}(x_{i+1}-x'),
\end{equation}
其中 $m_{1i},m_{2i}$ 分别是函数 $f(x)$ 在区间 $[x_i,x'],[x',x_{i+1}]$ 山的最小值,由于 $m_i$ 是区间 $[x_i,x_{i+1}]$ 上 $f(x)$ 的最小值,所以
\begin{equation}
m_i\leq m_{1i},\qquad m_i\leq m_{2i}.
\end{equation}
于是
\begin{equation}
\begin{aligned}
m_i(x_{i+1}-x_i)&=m_i(x_{i+1}-x')+m_i(x'-x_i)\\
&\leq m_{2i}(x_{i+1}-x')+m_{1i}(x'-x_i)
\end{aligned}
\end{equation}
由此推得 $s'>s$。

\textbf{证毕!}
\begin{theorem}{}\label{the_Rieman_2}
任一达布下和都不大于任一达布上和。
\end{theorem}
\textbf{证明:}
考虑\textbf{任意两个}分划,设它们的达布和分别为 $s_1,S_1$ 和 $s_2,S_2$。所以定理的证明相当于只需证明 $s_1\leq S_2$。把这两分划的分点合在一起构成新的分划,记其达布和为 $s_3,S_3$。根据\autoref{the_Rieman_1} ,
\begin{equation}
s_1\leq s_3,\qquad S_3\leq S_2
\end{equation}
 而 $s_3\leq S_3$,所以 $s_1\leq S_2$。

\textbf{证毕!}

由\autoref{the_Rieman_2} ,下和的整个集合 $\{s\}$ 上有界,上和的整个集合 $\{S\}$ 下有界。由确界定理(\autoref{def_SupInf_3}~\upref{SupInf}),$\{s\},\{S\}$ 分别有有限的上确界和下确界:
\begin{equation}
I_*=\sup\qty{s},\qquad I^*=\inf\qty{S}.
\end{equation}
显然 $I_*\leq I^*$。

上述可总结为
\begin{equation}
s\leq I_*\leq I^*\leq S
\end{equation}
\begin{definition}{达布积分}
称 $I_*,I^*$ 分别为\textbf{达布下积分}和\textbf{达布上积分}。
\end{definition}

\subsection{定积分的存在条件}

借助达布和的知识,现在容易得到定积分存在的充要条件了。
\begin{theorem}{定积分存在的充要条件}
函数 $f(x)$ 从 $a$ 到 $b$ 的区间上定积分存在的充要条件是:其达布上和与下和之差在
\begin{equation}
\lambda=\max \left\{\Delta x_i|i=0,\cdots,n-1 \right\} 
\end{equation}
趋于0时其极限为0,即
\begin{equation}
\lim_{\lambda\rightarrow0}(S-s)=0
\end{equation}
\textbf{证明:}
1.必要性:假定积分为 $I$,于是对

\textbf{证毕!}
\end{theorem}
\footnote{参考\href{https://math.berkeley.edu/~arveson/Dvi/105/note1.pdf}{这个讲义}。}
我们来定义区间 $[a, b]$ 的实函数 $f(x)$ 的黎曼积分。 令有序实数列
\begin{equation}
\mathcal P = \qty{a = x_0 < x_1 < \dots < x_n = b}~,
\end{equation}
令
\begin{equation}
m_k = \inf_{x_{k-1} \le x \le x_k} f(x), \qquad M_k = \sup_{x_{k-1} \le x \le x_k} f(x) \quad (1 \le k \le n)~,
\end{equation}
对应的下和上黎曼求和为
\begin{equation}
\underline I(f, \mathcal P) = \sum_{k = 1}^n m_k\Delta k, \qquad \bar I(f, \mathcal P) = \sum_{k = 1}^n M_k \Delta_k~,
\end{equation}
其中 $\Delta_k = x_k - x_{k-1}$, $1\le k\le n$。 下和上黎曼积分的定义为
\begin{equation}
\underline I(f) = \sup_{\mathcal P} \underline I(f, \mathcal P), \qquad \bar I(f) = \inf_{\mathcal P} \bar I(f, \mathcal P)~,
\end{equation}
显然, $\underline I(f) \le \bar I(f)$。 $f$ 叫做\textbf{黎曼可积(Riemann integrable)} 当且仅当上下黎曼积分相等
\begin{equation}
I(f) = \underline I(f) = \bar I(f)~.
\end{equation}
% 我们以下用 $I(f)$ 表示黎曼积分, $\int$ 符号表示勒贝格积分。

\begin{theorem}{}
任何连续函数 $f \in C[a, b]$ 都是黎曼可积的。
\end{theorem}

\begin{theorem}{}
任何单调函数 $f: [a, b] \to \mathbb{R}$ 都是黎曼可积的。
\end{theorem}

我们可以证明闭区间$[a, b]$上的单调函数最多只有可数个不连续点\autoref{def_confun_1}~\upref{confun}(待添加\footnote{参考 \href{https://en.wikipedia.org/wiki/Discontinuities_of_monotone_functions}{Wikipedia 相关页面}}),实际上我们可以把定理推广为

\begin{theorem}{}
任何函数 $f: [a, b] \to \mathbb{R}$ ,如果它只有有限个不连续点,那么 $f$ 是黎曼可积的。
\end{theorem}

% 移动到勒贝格积分里去
% \begin{theorem}{}
% 任何连续函数 $f \in C[a, b]$, 勒贝格积分和黎曼积分相等。
% \end{theorem}

% \begin{theorem}{}
% 令 $f$ 为 $[a, b]$ 上的有界实函数, 那么 $f$ 是黎曼可积的当且仅当 $f$ 的所有不连续点的勒贝格测度为零。
% \end{theorem}

% 这样的函数可以在勒贝格测度为零的集合上修改, 使其 Borel-measurable, 若这样做, 勒贝格积分和黎曼积分结果相等。

% \subsection{黎曼不可积函数的例子}

不是所有函数都是黎曼可积的,比如

\begin{example}{Dirichlet函数}

Dirichlet函数
\begin{equation}
D(x)=
\leftgroup{
    1, x\in\mathbb{Q}\\
    0, x\not\in\mathbb{Q}
}
\end{equation}
不是Riemann可积的:对于任何一个子区间$[a, b]$,其上最大值为$1$,最小值为$0$,因此上黎曼积分为 $1$,下黎曼积分为 $0$。
\end{example}
