% 线段树
% keys 线段树|数据结构|C++|高级数据结构
% license Xiao
% type Tutor

\textbf{线段树(Segment tree)}是一种二叉树形的数据结构,用以存储区间或线段,并且可以在 $O(\log N)$ 的时间复杂度查询区间最大值、最小值、总和等属性。

\textbf{线段树的存储:}

线段树除了最后一层节点外是一棵满二叉树,因此可以用\enref{堆}{heap}的存储方式来存储线段树。
具体来说就是开一个一维数组,根节点的编号为 $1$,编号为 $x$ 的结点的左子节点的编号为 $x \times 2$,右子节点的编号为:$x \times 2 + 1$,父节点的编号为 $\left\lfloor\dfrac{x}{2}\right\rfloor$。

因此我们可以用一个结构体来存储线段树,线段树除了最后一层结点外是一棵满二叉树,除了最后一层结点外的结点个数为:$N + \dfrac{N}{2} + \dfrac{N}{4} + \cdots + 2 + 1 = 2N - 1$,最后一层的结点个数最坏情况下是 $2N$ 个结点,所以数组大小应不小于 $4N$ 才能保持不越界。

\begin{figure}[ht]
\centering
\includegraphics[width=14.25cm]{./figures/e95858b6f0a2f98b.png}
\caption{二叉树视角} \label{fig_STree_1}
\end{figure}

\begin{figure}[ht]
\centering
\includegraphics[width=10.25cm]{./figures/e39b0dfcfd28d92d.png}
\caption{区间视角} \label{fig_STree_2}
\end{figure}


可以看出,线段树的每个结点都代表一个区间,叶结点的区间长度都为 $1$,对于每个区间结点 $[l, r]$,左子结点为 $[l, mid]$,右子结点为 $[mid + 1, r]$,$mid = \left\lfloor\dfrac{l+r}{2}\right\rfloor$。

\textbf{线段树的建树($\texttt{build}$)操作}:

一般来说,线段树每个结点上存储了很多信息,具体存什么信息得根据具体情况判断,这里以存储区间最大值为例,我们用递归来建树,每个叶结点 $[i, i]$ 保存 $a_i$ 的值,每次递归左子节点和右子节点,最后根据子节点的信息更新当前结点的信息,这一操作称为 $\text{pushup}$ 操作。

\begin{figure}[ht]
\centering
\includegraphics[width=14.25cm]{./figures/9bb3ecc13bcaea50.png}
\caption{$\text{build}$} \label{fig_STree_3}
\end{figure}


\begin{lstlisting}[language=cpp]
struct Node {
    int l, r, v;  // v 代表区间最大值
}tr[4 * N];

void build(int u, int l, int r) 
{
    tr[u] = {l, r};
    
    if (l == r) // 叶节点
    {
        tr[u] = {l, r, a[l]};  // 也可只写 tr[u].v = a[l];
        return;
    }
    
    int mid = l + r >> 1;
    build(u << 1, l, mid);          // 左子节点[l, mid],编号为:u << 1
    build(u << 1 | 1, mid + 1, r);  // 右子节点[mid + 1, r],编号为:u << 1 | 1
    pushup(u);
}

build(1, 1, n);   // 调用建树,从根节点开始
\end{lstlisting}


\textbf{线段树的 $\texttt{pushup}$ 操作:}

线段树可以很容易的把左右两个子结点的信息上传到当前结点,所以在记录每个结点 $i$ 的最大值就可以用左子节点 $\mathtt{i<<1}$ 的最大值和右子节点 $\mathtt{i<<1|1}$ 的最大值两者取一个最大值就是当前结点 $i$ 的最大值。

\begin{lstlisting}[language=cpp]
void pushup(int u)
{
    tr[u].v = max(tr[u << 1].v, tr[u << 1 | 1].v);
}
\end{lstlisting}

线段树的单点修改($\texttt{modify}$)操作:

单点修改操作一般是把 $a[x]$ 的值修改成 $v$,我们从根结点开始,递归左右两个子节点,找到 $[x, x]$ 区间,然后从下往上把对应的父节点保存的最大值进行更新。

\begin{figure}[ht]
\centering
\includegraphics[width=14.25cm]{./figures/481cb3c3da23a73d.png}
\caption{$\text{modify}$} \label{fig_STree_4}
\end{figure}


\begin{lstlisting}[language=cpp]
void modify(int u, int x, int v)    // 把 a[x] 修改为 v
{
    if (tr[u].l == x && tr[u].r == x) tr[u].v = v;  // 叶节点
    else
    {
        int mid = tr[u].l + tr[u].r >> 1;       // mid 为树中区间的 mid
        if (x <= mid) modify(u << 1, x, v);     // x 属于左半区间
        else modify(u << 1 | 1, x, v);          // x 属于右半区间
        pushup(u);  // 记得更新父节点的值
    }
}
\end{lstlisting}

\textbf{线段树的区间查询($\texttt{query}$)操作:}

查询操作一般为查询某个区间的某种属性,例如查询区间 $[l. r]$ 内的最大值。我们只需要从根节点开始递归查询,会出现如下几种情况:

\begin{enumerate}
\item 查询的区间 $[l, r]$ 完全包含了当前结点的代表区间,则直接返回该区间的最大值,因为没必要再递归查询下去了。
\item 查询的区间 $[l, r]$ 与左子节点有交集,则递归查询左子节点
\item 查询的区间 $[l, r]$ 与右子节点有交集,则递归查询右子节点
\end{enumerate}

\begin{figure}[ht]
\centering
\includegraphics[width=14.25cm]{./figures/9c7c305e70536863.png}
\caption{$\text{query}$} \label{fig_STree_5}
\end{figure}

查询操作会把询问的区间 $[l, r]$ 分成 $\log N$ 个区间,来简单的证明一下:
在递归每个树上的结点 $[T_l, T_r]$ 时,$mid = \left\lfloor\dfrac{T_l+T_r}{2}\right\rfloor$ 会出现以下几种情况:

\begin{enumerate}
\item $l \leq T_l \leq T_r \leq r$ 即当前树中结点对应的区间完全在查询区间的内部
\item $T_l \leq l \leq T_r \leq r$ 即只有 $l$ 与当前树中结点对应的区间有交集\\
      (1) $l > mid$,$l$ 只与当前树中结点对应的区间的右半部分 $[mid + 1, r]$ 有交集; \\
      (2) $l \leq mid$,$l$ 与当前树中结点对应的区间的左右两边都有交集,需要递归左右两边,但是递归的右子结点会在递归后直接返回,即触发了情况 $1$.\\
\item $l \leq T_l \leq r \leq T_r$ 即只有 $r$ 与当前树中结点对应的区间有交集,对应情况 $2$。
\item $T_l \leq l \leq r \leq T_r$ 即查询区间完全在当前树中结点对应的区间内部。\\
      (1) $l > mid$ 时只会递归右子结点; \\
      (2) $r < mid$ 时只会递归左子节点; \\
      (3) $l$、$r$ 都与当前树中结点对应的区间有交集,此时需要递归左右子结点。 \\

\end{enumerate}

\begin{figure}[ht]
\centering
\includegraphics[width=10cm]{./figures/7cfed25108c1568b.png}
\caption{情况 $1$} \label{fig_STree_6}
\end{figure}

\begin{figure}[ht]
\centering
\includegraphics[width=9cm]{./figures/83b4c5eb2ac0e2a4.png}
\caption{情况 $2(2)$} \label{fig_STree_7}
\end{figure}

\begin{figure}[ht]
\centering
\includegraphics[width=8.5cm]{./figures/0ba8505642f29f24.png}
\caption{情况 $4(3)$} \label{fig_STree_8}
\end{figure}

只有 $4(3)$ 这种情况会对线段树的左右两棵子树递归,但这种操作至多发生一次,也就是最开始递归根结点,然后就变成了前三种情况。

\begin{lstlisting}[language=cpp]
int query(int u, int l, int r)
{
    // 完全包含
    if (tr[u].l >= l && tr[u].r <= r) return tr[u].v;  
    
    int mid = tr[u].l + tr[u].r >> 1;
    int v = 0;
    if (l <= mid) v = query(u << 1, l, r);
    if (r > mid) v = max(v, query(u << 1 | 1, l, r));
    
    return v;
}
\end{lstlisting}

以上就是线段树的基本操作。

例题:\href{https://www.spoj.com/problems/GSS3/}{GSS 3}

题目大意:

给定长度为 $N$ 的数列 $A$,以及 $M$ 条指令,每条指令可能是以下两种之一:

1.  \verb|1 x y|,查询区间 $[x,y]$ 中的最大连续子段和,即 $\max\limits_{x \le l \le r \le y}${$\sum\limits^r_{i=l} A[i]$}。

2.  \verb|2 x y|,把 $A[x]$ 改成 $y$。

对于每个查询指令,输出一个整数表示答案。


本题因为要求最大连续子段和,首先考虑一下线段树要存什么信息。肯定要存一个最大连续子段和 $\tt tmax$。那父节点的最大连续子段和能否由子节点更新得来呢?答案是不行的,因为当父节点的最大连续子段和是横跨左右子节点的时候,不能由子节点直接更新得来。

所以还需存左子节点的最大后缀和 $\tt rmax$ 以及右子节点的最大前缀和 $\tt lmax$。所以由子节点向父节点更新的最大连续子段和只有三种情况:

\begin{enumerate}
\item 左子节点的最大连续子段和 $\tt l.tmax$。
\item 右子节点的最大连续子段和 $\tt r.tmax$。
\item 左子节点的最大后缀和 $\tt l.rmax$ 加右子节点的最大前缀和 $\tt r.lmax$。
\end{enumerate}

所以父节点 $u$ 的最大连续子段和为:$\texttt{u.tmax = max(l.tmax, r.tmax, l.rmax + r.lmax)}$。

那么父节点的最大前/后缀和能否由子节点的最大前/后缀和更新呢?还是不行,因为当父节点的最大前缀和横跨左右两个子节点的时候,不能只由单个子节点更新。

所以父节点的最大前缀和有两种情况:

\begin{enumerate}
\item 左子节点的最大前缀和 $\tt l.lmax$
\item 左子节点的和 $\sum\limits^{\tt mid}_{i = l}a_i$ 加右子节点的最大前缀和 $\tt r.lmax$。
\end{enumerate}

所以还需维护一个区间和 $\tt sum$ 用于计算当前区间的所有元素之和。

后缀和同理。
\begin{lstlisting}[language=cpp]
const int N = 500010;
struct Node
{
    int l, r, lmax, rmax, tmax, sum;
}tr[N * 4];
int n, m, a[N];

void pushup(Node &u, Node &l, Node &r)
{
    u.sum = l.sum + r.sum;
    u.lmax = max(l.lmax, l.sum + r.lmax);
    u.rmax = max(r.rmax, r.sum + l.rmax);
    u.tmax = max(max(l.tmax, r.tmax), l.rmax + r.lmax);
}

void pushup(int u)
{
    pushup(tr[u], tr[u << 1], tr[u << 1 | 1]);
}
\end{lstlisting}

修改与建树操作较简单,着重分析一下查询操作。

查询操作肯定是要返回一个区间的最大连续子段和,但是如果查询的区间横跨树中左右子节点,就需要先递归求左子节点的最大连续子段和,再递归求一下右子节点的最大连续子段和,再利用 $\tt pushup$ 操作合并答案。因为父节点的 $\tt tmax$ 不只是由左右子节点的 $\tt tmax$ 两者取最大值,还有可能横跨区间计算,所以需要分情况讨论。

\begin{lstlisting}[language=cpp]
Node find(int u, int l, int r)
{
    if (tr[u].l >= l && tr[u].r <= r) return tr[u];
    else 
    {
        int mid = tr[u].l + tr[u].r >> 1;
        // 查询的区间完全在树中节点对应的区间的左半边
        if (r <= mid) return find(u << 1, l, r);    
        // 查询的区间完全在树中节点对应的区间的右半边
        else if (l > mid) return find(u << 1 | 1, l, r);
        // 查询的区间在树中节点对应的区间的左右两边
        else 
        {
            auto left = find(u << 1, l, r);
            auto right = find(u << 1 | 1, l, r);
            Node res;
            pushup(res, left, right);
            return res;
        }
    }
}
\end{lstlisting}

以上讲的线段树的操作只涉及到了单点修改,那如果想要进行区间修改该怎么做呢?如果直接暴力修改,则单次修改的时间复杂度最坏为 $\mathcal{O}(n)$,时间复杂度太高。并且如果修改的区间后续根本没被查询,那么完全没有必要修改这些区间。

可以像区间查询那样,如果树中区间已经被查询区间完全包含了,直接返回,前面证明了查询的时间复杂度为 $\mathcal{O}(\log_2 n)$,但还要在回溯之前在当前区间内加一个\textbf{延迟标记(懒标记)}:“表示当前区间已被修改,但子区间还未被更新”,只有在后续查询或者更新区间的时候,如果涉及到了这个区间,就往下面的两个子结点下传标记,并且清除当前这个区间的标记。

\begin{figure}[ht]
\centering
\includegraphics[width=14cm]{./figures/04d444132941f681.png}
\caption{给 $[1, 3]$ 这个区间都加一} \label{fig_STree_9}
\end{figure}

如上图,若要给 $[1, 3]$ 这个区间都加一,这个树中结点正好被查询区间完全包含,给这个节点打一个延迟标记 $\texttt{add = 1}$,并且更新这个节点的值 $\texttt{v = 14}$。

\begin{figure}[ht]
\centering
\includegraphics[width=14cm]{./figures/2d51502ecf3feab3.png}
\caption{查询区间 $[1, 2]$ 的和} \label{fig_STree_10}
\end{figure}

查询区间 $[1, 2]$ 的和,第一次递归到 $[1, 3]$ 这个区间的时候,发现这个区间有标记,则将标记下传到左右两个子节点(右子节点没画出来),然后更新左右两个子节点的值。

所以可以很容易的写出延迟标记 $\tt pushdown$ 的代码:

\begin{lstlisting}[language=cpp]
#define lson u << 1
#define rson u << 1 | 1

void pushdown(int u)
{
    if (tr[u].add)  // 如果节点 u 有标记
    {
        tr[lson].add += tr[u].add;    // 更新左子节点的延迟标记
        // 更新左子节点的值
        tr[lson].sum += (1ll * tr[lson].r - tr[lson].l + 1) * tr[u].add; 
        
        tr[rson].add += tr[u].add;
        tr[rson].sum += (1ll * tr[rson].r - tr[rson].l + 1) * tr[u].add;
        
        tr[u].add = 0;  // 清除当前结点的延迟标记
    }
}
\end{lstlisting}
