% SQLite C API 笔记

\begin{issues}
\issueDraft
\end{issues}

\pentry{SQL 基础笔记\upref{SQLnt}}

这里介绍官方的 C API, 可以在 C 或者 C++ 中调用。 如果想要更方便的 C++ wrapper, 可以见 \href{https://github.com/SRombauts/SQLiteCpp}{SQLiteCpp}。

\subsection{安装}
ubuntu 上用 \verb|sudo apt install libsqlite3-dev| 安装

另外安装命令行工具 \verb|sudo apt install sqlite3|

\subsection{例子}
参考\href{https://www.tutorialspoint.com/sqlite/sqlite_c_cpp.htm}{这里}。 注意所有的数据必须先转换成字符串才能加到数据库中。
\begin{lstlisting}[language=cpp]
#include <iostream>
#include <sqlite3.h>

void test_sqlite()
{
	using namespace slisc;
#ifdef SLS_USE_SQLITE
	sqlite3* DB;
	int exit;
	file_remove("example.db");
	exit = sqlite3_open("example.db", &DB);
	if (exit) {
		cout << sqlite3_errmsg(DB) << endl;
		SLS_ERR("Error open DB!");
	}
	char* messaggeError;
	string sql = "CREATE TABLE PERSON("
					  "ID       INT     PRIMARY KEY  NOT NULL, "
					  "NAME     TEXT    NOT NULL, "
					  "SURNAME  TEXT    NOT NULL, "
					  "AGE      INT     NOT NULL, "
					  "ADDRESS  CHAR(50), "
					  "SALARY   REAL );";
	exit = sqlite3_exec(DB, sql.c_str(), NULL, 0, &messaggeError);
	if (exit != SQLITE_OK) {
		SLS_ERR("Error Create Table");
		sqlite3_free(messaggeError);
	}

	sql = "INSERT INTO PERSON"
					  "(ID, NAME, SURNAME, AGE, ADDRESS, SALARY)"
					  " VALUES "
					  "(0, 'Addis', 'Chen', 30, 'ABC Rd,
                          Manhattan, NY, 12345', 1500.03);"
		   "INSERT INTO PERSON"
					  " VALUES "
					  "(1, 'Bob', 'Chen', 31, 'DEF Ave,
                          Manhattan, NY, 12345', 4500.03);";
	exit = sqlite3_exec(DB, sql.c_str(), NULL, 0, &messaggeError);
	if (exit != SQLITE_OK) {
		SLS_ERR("Error Inserting Table :" + Str(messaggeError));
		sqlite3_free(messaggeError);
	}

	sqlite3_close(DB);
#else
	cout << "---------- disabled! ----------" << endl;
#endif
}
\end{lstlisting}

另一个挺有用的小程序是(修改自\href{https://www.sqlite.org/quickstart.html}{官方教程}), 可以在命令行对任意数据库文件执行任意 SQL 命令。 编译: \verb|g++ -o sqcmd sqcmd.cpp -l sqlite3|。 使用方法如 \verb|./sqcmd example.db "SELECT * FROM PERSON"|
\begin{lstlisting}[language=cpp]
#include <stdio.h>
#include <sqlite3.h>

static int callback(void *NotUsed, int argc, char **argv, char **col_name){
  int i;
  for(i=0; i<argc; i++){
    printf("%s = %s\n", col_name[i], argv[i] ? argv[i] : "NULL");
  }
  printf("\n");
  return 0;
}

int main(int argc, char **argv){
  sqlite3 *db;
  char *zErrMsg = 0;
  int ret;

  if( argc!=3 ){
    fprintf(stderr, "Usage: %s DATABASE SQL-STATEMENT\n", argv[0]);
    return(1);
  }
  ret = sqlite3_open(argv[1], &db);
  if( ret ){
    fprintf(stderr, "Can't open database: %s\n", sqlite3_errmsg(db));
    sqlite3_close(db);
    return(1);
  }
  ret = sqlite3_exec(db, argv[2], callback, 0, &zErrMsg);
  if( ret!=SQLITE_OK ){
    fprintf(stderr, "SQL error: %s\n", zErrMsg);
    sqlite3_free(zErrMsg);
  }
  sqlite3_close(db);
  return 0;
}
\end{lstlisting}

sqlite3 的数据库文件是二进制的, 可以用 \verb|sqlite3| 命令行 dump 出一个通用的文本文件
\begin{lstlisting}[language=none]
sqlite> .output example.sql
sqlite> .dump
sqlite> .exit
\end{lstlisting}

\subsection{binding}
当然也不需要把所有数据类型都转换成 string 那么蠢。 这就要用到 binding, 参考\href{https://www.sqlite.org/c3ref/bind_blob.html}{这里}。
\begin{itemize}
\item 事实上 \verb|sqlite3_exec()| 命令是调用一系列命令, 如果用 binding, 就必须把它们分开用。
\item 首先用 \verb|sqlite3_stmt stmt;| 创建一个 statement object。
\item 然后把 sql 命令 \verb|str| parse 为 \verb|stmt|: \verb|ret = sqlite3_prepare_v2(db, str.c_str(), str.size()+1, &stmt, NULL);|。 这里 \verb|str| 中的数据可以用 \verb|?| 或者 \verb|?编号| 代替, 以后再 bind 到不同的数据上。 这样就不需要每换一次数据都要重新 parse 一次。
\item 注意只有数据可以用 \verb|?| 代替, \verb|表名| 和 \verb|列名| 不可以。
\item 不可以对字符串用 \verb|'?'|, \verb|sqlite3_prepare| 会自己添加。 如果用了就会错。
\item \verb|int sqlite3_bind_parameter_count(stmt)| 可以返回 \verb|?| 的个数。
\item \verb|ret = sqlite3_bind_text(stmt, 问号的编号, names[i].c_str(), names[i].size(), SQLITE_STATIC);| 用于把字符串 bind 到指定的问号上。 如果问号没有指定编号则从左到右从 1 开始。
\item \verb|SQLITE_STATIC| 的意思是假设字符串写入数据库之前都不会改变, 如果会, 就用 \verb|SQLITE_TRANSIENT|, 这样 \verb|stmt| 会立即把字符串复制下来。
\item \verb|ret = sqlite3_bind_text(stmt, 问号的编号, int);| 用于把整数 bind 到指定的问号上。
\item bind 好之后用 \verb|ret = sqlite3_step(stmt)| 来执行, 当 statement 不是 query 时返回 \verb|SQLITE_DONE|; 是 query 时返回 \verb|SQLITE_ROW|, 每调用一次返回一行。
\item 要查看 query 的结果, \verb|sqlite3_column_int(stmt, 从0开始的列编号)| 获取当前返回行的某个整数类型的列
\item \verb|(char*)sqlite3_column_text(stmt, 从0开始的列编号)| 同理获得字符串类型
\item 最后用 \verb|sqlite3_finalize(stmt);| 释放内存。
\item 每次操作完, \verb|int sqlite3_changes(db)| 可以获取改变的行数。
\end{itemize}


\subsection{sqlite3 命令}
\begin{itemize}
\item \verb|sqlite3| 是一个命令行程序, 可以手动输入 SQL 命令来操作 sqlite 数据库。
\item \verb|sqlite3 文件.db| 可以打开二进制的 sqlite3 数据库文件。
\item \verb|sqlite3| 命令行中的 \verb|VACUUM| 命令可以重建数据库, 让它尺寸更小, 一些操作更快。
\item \verb|Ctrl+D| 或者 \verb|.quit| 可以退出。 \verb|.| 开头的命令是 \verb|sqlite3| 自己的命令而不是 SQL 命令。
\item 更简单地也可以用 \verb|sqlite3 文件.db "命令; 命令;..."| 直接执行命令不进入 sqlite3 命令行。
\item \verb|sqlite3 文件.db .dump > 文件.sql| 可以把二进制数据库 dump 成 SQL 命令的文本文件。 用这些命令可以重建该数据库。
\item 相反, 可以 \verb|sqlite3 文件.db < 文件.sql| 由 \verb|sql| 命令生成数据库(如果 \verb|文件.db| 不存在)。 如果存在, 就打开 \verb|文件.db| 继续运行 \verb|文件.sql| 中的命令。
\end{itemize}

\subsection{GUI 浏览器}
\begin{itemize}
\item DB Browser for SQLite 是一个比较简单的 GUI 工具, 可以直接在 snap store 里面安装。
\item \href{https://sqlitestudio.pl/}{SQLiteStudio} 有更多特性和更漂亮的界面, 无需安装。
\end{itemize}
