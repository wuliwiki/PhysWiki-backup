% 时间演化算符(量子力学)
% keys 薛定谔方程

\pentry{量子力学的基本原理(量子力学)\upref{QMPrcp}}


本节介绍时间演化算符,以此为切入点,引入量子态的演化方程,即薛定谔方程.


“我们应当记住的首要之点是:时间在量子力学中只是一个参量而不是一个算符.特别地,时间不是前一章所说的可观测量.像谈论位置算符一样谈论时间算符是无意义的.”——樱井纯,J. 拿波里塔诺,《现代量子力学》,2.1节.

量子力学中,时间不是一个算符,意味着量子力学认为时间是独立存在的,即采用经典时空观.


\subsection{时间演化算符}

\begin{definition}{时间演化算符}\label{TOprt_def1}
设一个物理系统在时间$t$时的态矢量为$\ket{s, t}$,而$t_0<t$是一个初始时间,那么定义
\begin{equation}
\mathcal{U}(t, t_0)\ket{s, t_0}=\ket{s, t}
\end{equation}
其中$\mathcal{U}(t, t_0)$称为从$t_0$到$t$的\textbf{时间演化算符(time evolution operator)}.
\end{definition}

从















