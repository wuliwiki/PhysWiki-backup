% 线性变换与矩阵的代数关系
% 线性变换|矩阵|代数同构

在本页中,我们将所有 $m$ 行 $n$ 列的实矩阵组成的集合记为 $\mathbb{R}^{m\times n}.$

并记从 $\mathbb{R}^{n}$ 到 $\mathbb{R}^{m}$ 的一切线性变换全体构成的空间为 $L(\mathbb{R}^{n},\mathbb{R}^{m}).$
\begin{itemize}
\item $\mathbb{R}^{m\times n}$ 中的矩阵与 $L(\mathbb{R}^{n},\mathbb{R}^{m})$
中的线性变换是代数同构的关系. 矩阵可以看成线性变换, 线性变换在取定一组基底后, 可以表示为矩阵. 
\end{itemize}

\begin{itemize}
\item \textbf{矩阵的秩}既可以定义为其行向量组的秩 (称为\textbf{行秩}), 也可以定义为其列向量组的秩 (称为\textbf{列秩}),
这是因为矩阵的行秩和列秩被证明是相同的. 注:矩阵行向量组形成的空间称为\textbf{行空间}, 行向量组的秩也正是行空间的维数;
列向量组形成的空间称为\textbf{列空间}, 列向量组的秩也正是列空间的维数.
\end{itemize}

\begin{itemize}
\item \textbf{线性变换的秩}定义为其像空间的维数. 注:如果将矩阵看成线性变换, 那么该变换的像空间的维数, 恰是矩阵的行空间的维数. 
\end{itemize}

证明:设 $A\in\mathbb{R}^{m\times n}$, $A$ 也可以看成从 $\mathbb{R}^{n}$ 到 $\mathbb{R}^{m}$
的线性变换, 那么取定 $\mathbb{R}^{n}$ 的标准基 $(e_{1},e_{2},\ldots,e_{n})$ 后,
$A$ 的像空间就由 $\{Ae_{1},Ae_{2},\ldots,Ae_{n}\}$ 张成, 因此像空间的维数就是 $\{Ae_{1},Ae_{2},\ldots,Ae_{n}\}$
的极大无关组的个数, 也就是 $\{Ae_{1},Ae_{2},\ldots,Ae_{n}\}$ 的秩; 而每个 $Ae_{i}$
正好又是矩阵 $A$ 的第 $i$ 行, 这就是说明了 $A$ 的像空间的维数等于它的行空间的维数. 

\begin{itemize}
\item 矩阵是行满秩的, 当且仅当其线性变换是满射. (像空间等于行空间)
\end{itemize}

\begin{itemize}
\item 矩阵是列满秩的, 当且仅当其线性变换是单射. (维数定理)
\end{itemize}

\begin{itemize}
\item 方阵是满秩的, 当且仅当它是可逆的, 当且仅当其线性变换是双射. \end{itemize}

\begin{itemize}
\item 矩阵是有右逆的, 当且仅当其线性变换是满射. \end{itemize}

\begin{itemize}
\item 矩阵是有左逆的, 当且仅当其线性变换是单射. 
\end{itemize}