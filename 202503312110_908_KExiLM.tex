% 柯西-黎曼方程(综述)
% license CCBYSA3
% type Wiki

本文根据 CC-BY-SA 协议转载翻译自维基百科\href{https://en.wikipedia.org/wiki/Cauchy\%E2\%80\%93Riemann_equations}{相关文章}。

\begin{figure}[ht]
\centering
\includegraphics[width=8cm]{./figures/9b579a327747d777.png}
\caption{一个向量 \( X \) 在一个域中被复数 \( z \) 乘以后再通过函数 \( f \) 映射,和先通过函数 \( f \) 映射再乘以 \( z \) 的情况进行的视觉对比。如果这两种情况对于所有 \( X \) 和 \( z \) 都导致点最终落在相同的位置,那么函数 \( f \) 满足柯西-黎曼条件。} \label{fig_KExiLM_1}
\end{figure}
在数学的复分析领域,柯西–黎曼方程以奥古斯丁·柯西和伯恩哈德·黎曼命名,由一组二阶偏微分方程组成,这些方程为复变量的复函数可复微分的必要与充分条件。

这些方程是:
\[
\frac{\partial u}{\partial x} = \frac{\partial v}{\partial y} \quad (1a)~
\]

和
\[
\frac{\partial u}{\partial y} = -\frac{\partial v}{\partial x} \quad (1b)~
\]
其中,\(u(x, y)\)和\(v(x, y)\)是实值的二元可微函数。

通常,\(u\) 和\(v\)分别是复值函数\(f(x+iy)=f(x,y)=u(x,y)+iv(x,y)\)的实部和虚部,其中\( z = x + iy \)是单一复变量,且\( x \) 和 \( y \)是实变量;\( u \)和\( v \)是实变量的实值可微函数。如果\( f \)在某个复数点可复微分,当且仅当\( u \) 和\( v \)的偏导数在该点满足柯西–黎曼方程时,\( f \)才是复微分的。

全纯函数是指在复平面某个开子集的每一点都可微分的复函数。已经证明,全纯函数是解析的,而解析的复函数是复微分的。特别地,全纯函数是无限次可复微分的。

微分性与解析性之间的等价性是所有复分析的起点。
\subsection{历史}  
柯西–黎曼方程首次出现在让·勒·朗·达朗贝尔的研究中。\(^\text{[1]}\)后来,莱昂哈德·欧拉将这一系统与解析函数联系起来。\(^\text{[2]}\)随后,柯西使用这些方程构建了他的函数理论。\(^\text{[3]}\)黎曼关于函数理论的博士论文于1851年发表。\(^\text{[4]}\)
\subsection{简单示例}  
假设\( z = x + iy \)。复值函数\( f(z) = z^2 \)在复平面上的任意点\( z \)都是可微分的。  
\[
f(z) = (x + iy)^2 = x^2 - y^2 + 2ixy~
\]
实部\( u(x, y) \) 和虚部 \( v(x, y) \)分别为  
\[
u(x, y) = x^2 - y^2, \quad v(x, y) = 2xy~
\]
它们的偏导数是  
\[
u_x = 2x, \quad u_y = -2y, \quad v_x = 2y, \quad v_y = 2x~
\]
我们可以看到,柯西–黎曼方程确实成立,\(u_x = v_y\)和\(u_y = -v_x\).
\subsection{解释与重新表述}  
柯西–黎曼方程是从复分析的角度看待函数可微分条件的一种方式:换句话说,它们通过常规的微分学方法来概括复变量函数的概念。在这一理论中,还有几种主要的方式来理解这一概念,通常需要将该条件转化为其他语言。
\subsubsection{共形映射}   
首先,柯西–黎曼方程可以写成复数形式:
\[
i \frac{\partial f}{\partial x} = \frac{\partial f}{\partial y}.~
\]
在这种形式下,方程在结构上对应于雅可比矩阵的形式:
\[
\begin{pmatrix} a & -b \\ b & a \end{pmatrix},~
\]
其中\(a=\partial u/\partial x=\partial v/\partial y\)和\(b=\partial v/\partial x=-\partial u/\partial y\).这种形式的矩阵是复数的矩阵表示。从几何上讲,这样的矩阵总是旋转与缩放的组合,特别是它保持角度不变。函数\(f(z)\)的雅可比矩阵将两个曲线交点处的无穷小线段旋转到\(f(z)\)中的相应线段。因此,满足柯西–黎曼方程且导数不为零的函数保持平面中曲线之间的角度。这就是说,柯西–黎曼方程是函数保持共形性的条件。

此外,由于共形变换的组合仍然是共形的,柯西–黎曼方程的解与共形映射的组合必须自身满足柯西–黎曼方程。因此,柯西–黎曼方程是共形不变的。
\subsubsection{复微分性}  
设\( f(z) = u(z) + i \cdot v(z) \),其中\( u \)和\( v \)是实值函数,\( f \)是复变量\( z = x + iy \)的复值函数,\(x\)和\(y\)是实变量。
\[f(z) = f(x + iy) = f(x, y)~\]
因此,函数也可以看作是实变量\(x\)和\(y\)的函数。那么,复函数\(f\)在点\( z_0 = x_0 + iy_0 \)处的复导数定义为:
\[
f'(z_0) = \lim_{\underset{h \in \mathbb{C}}{h \to 0}} \frac{f(z_0 + h) - f(z_0)}{h}~
\]
前提是该极限存在(即,沿着每一条趋近于\( z_0 \)的路径该极限存在,并且不依赖于选择的路径)。

复分析的一个基本结果是,函数\( f \)在点\( z_0 \)处是复微分的(即,它具有复导数),当且仅当实值二元函数\( u(x + iy) \)和\( v(x + iy) \)在点\( (x_0, y_0) \)处可微分,并且在该点满足柯西–黎曼方程。\(^\text{[5][6][7]}\)

事实上,如果复导数在\( z_0 \)处存在,那么可以通过沿实轴和虚轴在\( z_0 \)处取极限来计算复导数,并且这两个极限必须相等。沿实轴时,极限为:
\[
\lim_{\underset{h \in \mathbb{R}}{h \to 0}} \frac{f(z_0 + h) - f(z_0)}{h} = \left. \frac{\partial f}{\partial x} \right|_{z_0}~
\]
而沿虚轴时,极限为:
\[
\lim_{\underset{h \in \mathbb{R}}{h \to 0}} \frac{f(z_0 + ih) - f(z_0)}{ih} = \left. \frac{1}{i} \frac{\partial f}{\partial y} \right|_{z_0}~
\]
因此,导数的相等性意味着:
\[
i \left. \frac{\partial f}{\partial x} \right|_{z_0} = \left. \frac{\partial f}{\partial y} \right|_{z_0}~
\]
这就是柯西–黎曼方程在\( z_0 \)处的复数形式。

(注意,如果\( f \) 在 \( z_0 \)处是复微分的,那么它也是实微分的,并且\( f \)在\( z_0 \)处的雅可比矩阵是复数标量\( f'(z_0) \),可以看作是\( \mathbb{C} \)上的实线性映射,因为极限\(|f(z) - f(z_0) - f'(z_0)(z - z_0)|/|z - z_0| \to 0\)和\(z \to z_0\)成立。)

相反,如果 \( f \) 在 \( z_0 \) 处(在实数意义下)是可微分的,并且在该点满足柯西-黎曼方程,那么它在该点是复微分的。假设 \( f \) 作为两个实变量 \( x \) 和 \( y \) 的函数在 \( z_0 \) 处是可微分的(即实微分的)。这等价于存在以下线性近似:

\[
\Delta f(z_0) = f(z_0 + \Delta z) - f(z_0) = f_x \Delta x + f_y \Delta y + \eta(\Delta z)
\]

其中\(f_x = \left. \frac{\partial f}{\partial x} \right|_{z_0}, \quad f_y = \left. \frac{\partial f}{\partial y} \right|_{z_0}, \quad z = x + iy, \quad \eta(\Delta z) / |\Delta z| \to 0\)当\Delta z \to 0


由于  
\[
\Delta z + \Delta \bar{z} = 2 \Delta x \quad \text{和} \quad \Delta z - \Delta \bar{z} = 2i \Delta y,
\]

上述表达式可以重新写为:

\[
\Delta f(z_0) = \frac{f_x - i f_y}{2} \Delta z + \frac{f_x + i f_y}{2} \Delta \bar{z} + \eta(\Delta z)
\]

\[
\frac{\Delta f}{\Delta z} = \frac{f_x - i f_y}{2} + \frac{f_x + i f_y}{2} \cdot \frac{\Delta \bar{z}}{\Delta z} + \frac{\eta(\Delta z)}{\Delta z}, \quad (\Delta z \neq 0).
\]