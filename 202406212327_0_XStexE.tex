% 小时百科常用 LaTeX 公式
% license Usr
% type Tutor

单个参数的命令若只有一个字符, 那么不需要加花括号, 例如 $\mathrm a$ 等效于 $\mathrm{a}$。 如果参数是数字甚至也不需要空格, 例如 $\frac12$。

注意行内公式的排版会和行间公式不太一样,例如 $\frac12$ 中的字体会变小,求和的上下标会跑到右边:$\sum_{i=1}^{N} a_i$。 如果希望行内公式和行间公式效果一样,在前面用 \verb`\displaystyle`,如 $\displaystyle\frac12$ 和 $\displaystyle\sum_{i=1}^{N} a_i$。

公式中的空格从小到大如 $a\, b\; c\quad d\qquad e$。 公式中三点省略号用 $\dots$, 如果要强制在下方, 用 $\ldots$。 实部和虚部如 $\Re[z], \Im[z]$。 双重极限如
\begin{equation}
\lim_{\substack{\Delta x_i\to 0\\ \Delta y_i\to 0}} \sum_{i, j} f(x_i,y_i) \Delta x_i \Delta y_j~.
\end{equation}
注意行内公式的 \verb|\lim| 和 \verb|\sum| 如果要在正上方或正下方写东西, 需要加 \verb|\limits|, 或 \verb`\displaystyle` 如 $\lim\limits_{x\to 0}$ 和 $\displaystyle \lim_{x\to 0}$。 但是 \verb|\limits| 只能给算符使用, 如果要对任意符号使用可以用 \verb|\underset| 或者 \verb`\mathop` 或者 \verb|\overset|, 例如 $\underset{i \ge 0}{\bigcup}$ 或者 $\displaystyle\mathop{\mathrm{ABC}}_{x\to 0}$。

上下方的花括号例如
\begin{equation}
\overbrace{n(n-1)(n-2)\dots}^{\text{共 $m$ 项}} \qquad \underbrace{n(n-1)(n-2)\dots}_{\text{共 $m$ 项}}~
\end{equation}
方括号用 \verb|\overbracket| 和 \verb|\underbracket|。

如果要强制分式正常大小显示, 用 \verb|\dfrac| 命令, 如果 \verb|\frac| 后面只有两个字符的代码,可以写成如 \verb|\frac12|, \verb|\frac ab|, \verb|\frac1a| 等。 斜分数线后面若多于一个变量需要加括号, 如 $ab/(cd)$。

行内分式如 $a/b$, 不允许行内用立体分式。 几何矢量如 $\bvec a$, 不要使用带箭头的矢量如 $\vec a$。 

行间公式换行及对齐用 aligned 环境(可以选中一段公式代码然后按菜单的 “对齐” 按钮), 注意该环境可嵌套。
\begin{equation}
\begin{aligned}
(a-b)^2 &= a^2+b^2 - 2ab \\
& = a^2+b^2+2ab-4ab\\
& = (a+b)^2-4ab~.
\end{aligned}
\end{equation}

用 \verb|\begin{enumerate}[resume]|  来继续上一个 enumerate 的编号。

可变化尺寸的斜分数线如下
\begin{equation}
\left. \pdv[2]{X}{x} \middle/ X + \pdv[2]{Y}{y} \middle/ Y + \pdv[2]{Z}{z} \middle/ Z  = \frac{1}{c^2}  \pdv[2]{T}{t} \middle/ T\right.~.
\end{equation}
左大括号用 \verb|cases| 环境, 如
\begin{equation}
\begin{cases}
d+e+f = \int \frac{a}{b} \dd{x} &(a > 0)\\
a+b = c &(b > 0)~.
\end{cases}
\end{equation}
但是注意 \verb|cases| 环境里面的符号都是小尺寸的(与行内公式相同), 且每行只能用一次 \verb|&|。 若需要全尺寸公式可以用下文定义的 \verb|\leftgroup| 命令。

反斜杠有两种形式:$\backslash$和$\setminus$。前者在一般需要斜杠时用,而集合减法应该用后者。使用实例:$A\backslash B$,$A\setminus B$。

\subsubsection{字母样式}
希腊字母如下
\begin{equation}
\begin{aligned}
&\alpha (a), \beta (b), \chi (c), \Delta\delta (d), \epsilon/\varepsilon (e), \Phi\phi (f), \Gamma\gamma (g),
\eta (h), \iota (i), \varphi (j), \kappa (k), \Lambda\lambda (l), \mu (m),\\
& \nu (n), o (o),\Pi\pi (p), \Theta\theta (q), \rho (r), \Sigma\sigma (s), \tau (t), \upsilon (u), \varpi (v), \Omega\omega (w), \Xi\xi (x), \Psi\psi (y), \zeta (z)~.
\end{aligned}
\end{equation}
普朗克常数为 $\hbar$。 另外我们自定义了 $\lambdabar$, 相当于 $\mkern-0.25mu {\bar{\phantom{a}}} \mkern -10.2mu\lambda$。

双线字母如下
\begin{equation}
\begin{aligned}
&\mathbb{a\, b\, c\, d\, e\, f\, g\, h\, i\, j\, k\, l\, m\, n\, o\, p\, q\, r\, s\, t\, u\, v\, w\, x\, y\, z}~,\\
&\mathbb{A\, B\, C\, D\, E\, F\, G\, H\, I\, J\, K\, L\, M\, N\, O\, P\, Q\, R\, S\, T\, U\, V\, W\, X\, Y\, Z}~,
\end{aligned}
\end{equation}
几种花体字母如下(\verb`{\cal A B C}` 相当于 \verb`\mathcal{A B C}`)
\begin{equation}
\mathcal{A\, B\, C\, D\, E\, F\, G\, H\, I\, J\, K\, L\, M\, N\, O\, P\, Q\, R\, S\, T\, U\, V\, W\, X\, Y\, Z}~,
\end{equation}
\begin{equation}
\begin{aligned}
&\mathscr{a\, b\, c\, d\, e\, f\, g\, h\, i\, j\, k\, l\, m\, n\, o\, p\, q\, r\, s\, t\, u\, v\, w\, x\, y\, z}~,\\
&\mathscr{A\, B\, C\, D\, E\, F\, G\, H\, I\, J\, K\, L\, M\, N\, O\, P\, Q\, R\, S\, T\, U\, V\, W\, X\, Y\, Z}~,
\end{aligned}
\end{equation}
\begin{equation}
\begin{aligned}
&\mathfrak{a\, b\, c\, d\, e\, f\, g\, h\, i\, j\, k\, l\, m\, n\, o\, p\, q\, r\, s\, t\, u\, v\, w\, x\, y\, z}~,\\
&\mathfrak{A\, B\, C\, D\, E\, F\, G\, H\, I\, J\, K\, L\, M\, N\, O\, P\, Q\, R\, S\, T\, U\, V\, W\, X\, Y\, Z}~.
\end{aligned}
\end{equation}

\subsubsection{使用 physics 宏包}
MathJax 本身不支持 physics 宏包(更新:后来支持了), 我们进行了一定的改进\footnote{后台程序在把 tex 文件翻译成 html 页面时, 把 physics 宏包或者我们自定义的命令翻译成了 MathJax 兼容的代码。 如果对 html 的公式点击右键获取公式代码, 得到的仍然是兼容 MathJax 的代码。}。 Physics 宏包的文档见\href{http://mirrors.ibiblio.org/CTAN/macros/latex/contrib/physics/physics.pdf}{这里}, 目前仅支持 Physics 宏包的以下命令。

绝对值如 $\abs{y}$ (自动尺寸,如果不需要自动尺寸用 $\abs*{y}$ 下同),范数如 $\norm*{x}$, 微分符号如 $\dd{x}$, 可变尺寸的小中大括号可以用 \verb|\qty|, 如
\begin{equation}
\qty(\frac ab)^2~, \qquad \qty[\frac ab]^2~, \qquad \qty{ \frac ab}~.
\end{equation}
矢量内积如 $\bvec A \vdot \bvec B$ (\verb|\vdot| 不可省略, 不要用 \verb|\cdot|),矢量叉乘如 $\bvec A\cross\bvec B$。

常用三角函数和对数函数后面的小括号会自动改变尺寸(中括号不可以),若要给他们加幂,用中括号
\begin{equation}
\sin(\frac ab)~, \qquad \sin[2](\frac ab)~.
\end{equation}
自然指数后面的小括号和中括号都可以自动尺寸
\begin{equation}
\exp(\frac12)~, \qquad \exp[\frac12]~.
\end{equation}
支持同样用法的还有 \verb`\arccos`, \verb`arcsin`, \verb`arctan`, \verb`\cos`, \verb`\cosh`, \verb`\cot`, \verb`\csc`, \verb`\log`, \verb`\ln`。

无穷小的阶数如 $\order{x^2}$, 导数和偏导可以用
\begin{equation}
\dv{x}~, \quad \dv{f}{x}~, \quad \dv[2]{f}{x}~, \quad \dv*[2]{f}{x}~, \quad
\pdv{x}~, \quad \pdv{f}{x}~, \quad \pdv[2]{f}{x}~, \quad \pdv{f}{x}{y}~, \quad \pdv*[2]{f}{x}~.
\end{equation}
要写到行内可以加 \verb|*|, 如 $\pdv*[2]{f}{x}$。

定积分求值如 $\eval{x^2}_0^1$ (自动尺寸)。

对易算符如 $\comm{\Q A}{\Q B}$ 或 $\comm*{\Q A}{\Q B}$, 前者自动尺寸, 后者强制小尺寸。 泊松括号同理: $\pb{\frac12}{B}$, $\pb*{\frac12}{B}$。

梯度散度旋度拉普拉斯如 $\grad V$,$\div\bvec A$, $\curl\bvec A$, $\laplacian V$。

狄拉克符号(加 \verb|*| 号强制小尺寸)如 $\bra{a}, \ket{b}, \braket{a}, \braket{a}{b}, \mel{a}{Q}{b}$。 平均值(加 \verb|*| 号强制小尺寸)如 $\ev*{Q}, \ev{Q}{\psi}$。 同理也有 $\ketbra*{a}{b}$, $\ketbra{a}{b}$, $\ketbra{a}$。

\subsubsection{交换图:amsCD包}

amsCD 包只支持方形的交换图;示例
\begin{equation}
\begin{CD}
A @>>> B @= E \\
@VVV @AAA @| \\
C @<<< D @. F
\end{CD}~.
\end{equation}

amsCD包使用\verb|CD|环境,其中使用\verb|\\|换行,支持7种箭头:

3种水平箭头,水平向右\verb|@>>>|、水平向左\verb|@<<<|和水平等于\verb|@=|;

3种竖直箭头,竖直向下\verb|@VVV|、竖直向上\verb|@AAA|和竖直等于\verb!@|!;

以及水平竖直皆可的无箭头\verb|@.|。

上下左右箭头支持标签\verb|A @>{上标签}>{下标签}> B|
\begin{equation}
\begin{CD}
A @>{\text{上标签}}>{\text{下标签}}> B \\
@V{\text{左标签}}V{\text{右标签}}V @A{\text{左标签}}A{\text{右标签}}A \\
C @<{\text{上标签}}<{\text{下标签}}< D
\end{CD}~~~~~~,
\end{equation}

最右边的右标签可能会和公式最后的标点重叠,需要特别注意。

\subsection{小时百科自定义的公式命令}

小时百科和云笔记的模板的自定义的命令如下, 如果不使用, 要保证编译后效果相同。 如果想新添加定义, 请联系管理员。

科学计数法如 $1.2\e3$ 或 $1.2\e{34}$, 单位矢量如 $\uvec a$, 自然对数底如 $\E$, 复数如 $u+\I v$ 虚数单位不能用斜体 $i$,复共轭如 $z\Cj$。

定义 \verb|\opn{}| 和 \verb|\operatorname{}| 等效。 自定义的 operator 或者函数名(如 \verb|\sinc| 相当于 \verb|\operatorname{sinc}| 或者 \verb|\opn{sinc}|)有: \verb|\Arctan|, \verb|\sinc|,\verb|\erf|,\verb|\erfi|, \verb|\Im|, \verb|\Re|。

\verb`\begin{aligned}...\end{aligned}` 环境可以用 \verb`ali{...}` 命令代替。

写量纲用 \verb|\Si{}| 命令, 如 $100\Si{m/s^2}$, 这个命令只能出现在公式环境内。 这相当于 \verb|\mathrm{}|。

矩阵 $\mat A$, 转置 $\mat A\Tr$, 厄米共轭用 $\mat A\Her$。 常见的几种矩阵括号如
\begin{equation}
\pmat{1&2\\3&4}~, \quad
\vmat{1&2\\3&4}~, \quad 
\bmat{1&2\\3&4}~, \quad
\Bmat{1&2\\3&4}~, \quad
\pmat{1&2\\3&4}\Tr~, \quad
\pmat{1&2\\3&4}\Her~.
\end{equation}
行内的列矢量用行矢量的转置表示,如 $\pmat{1& 2& 3& \dots}\Tr$。 张量如 $\ten T$, 效果同 \verb|\mat|。

单独一个粗体的 $\nabla$ 用 $\Nabla$。

自定义的 \verb|\leftgroup{}| 命令, 相当于 \verb|\left\{\begin{aligned}... \end{aligned}\right.|
\begin{equation}
\leftgroup{
&d+e+f = \int \frac{a}{b} \dd{x} &&(a > 0)\\
&a+b = c &&(b > 0)~.
}
\end{equation}
这相当于一个可变尺寸的 \verb|{| 加上 \verb|aligned| 环境
\begin{equation}
\left\{
\begin{aligned}
&d+e+f = \int \frac{a}{b} \dd{x} &&(a > 0)\\
&a+b = c &&(b > 0)~.
\end{aligned}
\right.
\end{equation}
注意百科中我们用圆括号表示条件, 而不是逗号。 另外这种公式的标点加在最后一行末。

表格中若用 \verb|\dfrac|, 需要在行首加上 \verb|\dfracH| 命令。% 未完成: 为什么?

量子力学算符如 $\Q a$(一般可以不加, 只有必要的时候加), 矢量量子算符如 $\Qv p$ (效果同 \verb|\uvec|)。
