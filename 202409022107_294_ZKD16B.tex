% 中国科学技术大学 2016 年考研普通物理 B 考试试题
% keys 中国科学技术大学|考研|物理|2016年
% license Copy
% type Tutor

\textbf{声明}:“该内容来源于网络公开资料,不保证真实性,如有侵权请联系管理员”
\begin{enumerate}
\item 长为$l$、质量为$M$的均质木板,可绕穿过一端的光滑水平轴$O$转动,开始时用细线拉着木板的另一端使其静止于水平位置,然后剪断细线。\\
(1)试求细线刚剪断时作用于轴上的力;\\
(2)试求当木板通过竖直位置时作用于轴上的力。
\item 在光滑水平地板上滑动的弹性立方体撞到竖直墙上,撞击时立方体的一个面与墙平行。墙与立方体之间的摩擦系数为$\mu$。碰撞前立方体速度方向与墙的夹角为$a$,问碰撞后夹角变为多大?
\item 一个粒子在势能$V(r)=\alpha r^p+\beta r^q$所确定的中心力作用下作平面运动,$\alpha,\beta,p$以及$q$均为常数,且$p<q$。已知该粒子可沿着螺旋线$r=c\theta^2$所确定的轨道运动,其中$c$为常数。试确定常数$p$和$q$的数值。
\item 两个相同细金属圆环同轴放置,带电量分别为$+Q$和$-Q(Q>0)$。设圆环半径为a,两圆心$A$和$B$相距$2d$,以轴线为$x$轴,轴线在两圆环之间的中点$O$为原点,\\
(1)求轴线上任一点的电势 $U(x)$,并定性画出 U~x曲线;\\
(2)求轴线上任一点的电场$E(x)$,并定性画出 E~x曲线.
\item 极板尺寸相同的两个平板空气电容器充以相同的电量$Q$。第一个电容器两极板的间距是第二个电容器的两倍。设第一个电容器的电容值为$C_0$,\\
(1)求两个电容器各自的静电能;\\
(2)如果将第二个电容器插入第一个电容器两极板的中央,所有极板保持平行,求整个体系的静电能。
\item 孤立导体球半径为$a$,充电到电势$V$,球绕其一直径以角速度$\omega$旋转。求\\
(1)球心的磁场强度;\\
(2)旋转球的磁矩。
\item 求氢原子中:\\
(1)电子在$n=1$轨道上运动时相应的电流值大小;\\
(2)$n=1$和100时轨道中心处的磁场强度;\\
(3)$n=1$和100时电子分别感受到的原子核的电场大小。已知:$(m_e=9.11*10^{-31}kg=0.511MeV/c^2,h=6.626*10^{-34}J.s=4.136*10^{-15}eV.s,\varepsilon_0=8.854187817*10^{-12 A.s.V^{-1}},\mu_0=4\pi*10^{-7}=12.5663)$

碳原子 2p3s'P→2p2p'D,的跃迁发出波长约为2000埃的谱线,若在弱的外磁场中,该谱线将如何分裂?设外磁场B=0.100T,试求在垂直于磁场方向观察各谱线与原谱线的波长差是多少?
9.(13 分)分别写出单电子2s、2p、3d、4f的原子态符号、总角动量大小和可能的5.Z取值
\end{enumerate}