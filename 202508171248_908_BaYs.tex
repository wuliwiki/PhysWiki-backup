% 八元数(综述)
% license CCBYSA3
% type Wiki

本文根据 CC-BY-SA 协议转载翻译自维基百科\href{https://en.wikipedia.org/wiki/Octonion}{相关文章}。

在数学中,八元数是一种实数域上的赋范除代数,是一种超复数系统。八元数通常用大写字母 $\mathbf{O}$ 表示,也可以写作黑板粗体 $\mathbb{O}$。八元数具有 8 个维度,是四元数的 2 倍维数,而它们正是四元数的扩展。八元数是非交换的、非结合的,但满足一种较弱的结合性,即所谓的可交替性。此外,它们还具有幂结合性。

八元数不像四元数和复数那样广为人知,后者在研究和应用上更为广泛。八元数与数学中的一些例外结构(exceptional structures,需要进一步澄清)有关,其中包括例外李群。八元数在弦理论、狭义相对论和量子逻辑等领域都有应用。将 Cayley–Dickson 构造应用于八元数,可以得到十六元数。
\subsection{历史}
八元数是在 1843 年 12 月由约翰·T·格雷夫斯发现的,他的灵感来自好友威廉·罗恩·哈密顿发现四元数。就在格雷夫斯发现八元数前不久,他在 1843 年 10 月 26 日写给哈密顿的一封信中写道:

“如果凭借你的炼金术,你能炼出三磅黄金,为什么要止步于此呢?”

格雷夫斯把他的发现称为 “octaves”(八度数),并在 1843 年 12 月 26 日写给哈密顿的信中提到这一点。他最早发表研究结果的时间,比阿瑟·凯莱的文章稍晚一些。八元数也被凯莱独立发现,有时被称为 Cayley 数或凯莱代数。哈密顿后来描述过格雷夫斯发现八元数的早期经过。
\subsection{定义}
八元数可以被看作是实数的 八元组(octets 或 8-元组)。
每一个八元数都是单位八元数的实线性组合:
$$
\{e_{0}, e_{1}, e_{2}, e_{3}, e_{4}, e_{5}, e_{6}, e_{7}\},~
$$
其中,$e_{0}$ 是标量或实数元,可以与实数 1 对应。

也就是说,每个八元数 $x$ 都可以写成以下形式:
$$
x = x_{0} e_{0} + x_{1} e_{1} + x_{2} e_{2} + x_{3} e_{3} 
  + x_{4} e_{4} + x_{5} e_{5} + x_{6} e_{6} + x_{7} e_{7},~
$$
其中系数 $x_{i} \in \mathbb{R}$。
\subsubsection{凯莱–迪克森构造}
一种更系统地定义八元数的方法是通过 凯莱–迪克森构造。将凯莱–迪克森构造应用于四元数,就能得到八元数。可以表示为$\mathbb{O} = \mathcal{CD}(\mathbb{H}, 1)$.

就像四元数可以定义为复数的有序对一样,八元数也可以定义为四元数的有序对。加法按分量逐一进行。若 $(a, b)$ 和 $(c, d)$ 是两对四元数,则它们的乘法定义为
$$
(a, b)(c, d) = (ac - d^{*}b,\; da + bc^{*}),~
$$
其中 $z^{\*}$ 表示四元数 $z$ 的共轭。

当将八个单位八元数与以下有序对对应时,这一定义与前面给出的等价:
$$
(1, 0),\; (i, 0),\; (j, 0),\; (k, 0),\; (0, 1),\; (0, i),\; (0, j),\; (0, k).~
$$
\subsection{算术与运算}
\subsubsection{加法与减法}
八元数的加法与减法是逐项进行的,即对应项的系数相加或相减,这与四元数的情况相同。
\subsubsection{乘法}
八元数的乘法要复杂得多。乘法对加法是分配的,因此两个八元数的积可以通过逐项相乘再求和得到,这一点和四元数类似。

每一对项的乘积由系数的乘法以及单位八元数的乘法表共同决定。一个这样的乘法表(由 阿瑟·凯莱(Arthur Cayley, 1845) 和 约翰·T·格雷夫斯(John T. Graves, 1843)分别给出)如下所示:[7]
