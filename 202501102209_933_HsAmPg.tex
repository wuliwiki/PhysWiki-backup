% 等差数列(高中)
% keys 高中|等差数
% license Usr
% type Tutor

\begin{issues}
\issueDraft
\end{issues}

\pentry{数列(高中)\nref{nod_HsSeFu}}{nod_53a2}

在日常生活中,常会遇到一些有规律的现象。例如,每月存固定金额,或者每天增加相同的步行距离。这些“有规律的增加”可以通过数学语言描述,体现为等差数列的应用。






\subsection{等差数列}

\begin{definition}{等差数列}
如果数列 $\{a_n\}$ 满足对于 $n > 1$ 的所有项,每一项与前一项的差为同一个常数 $d$,则称 $\{a_n\}$ 为\textbf{等差数列(arithmetic sequence)},$d$ 称为$\{a_n\}$的\textbf{公差(common difference)},即等差数列满足递推公式
\begin{equation}
a_{n}=a_{n-1}+d\qquad(n>1)~.
\end{equation}
\end{definition}

特别地,之前提到的常数列是 $d = 0$ 的等差数列。如果公差 $d$ 为负,数列的值会逐渐减少,例如 $10, 8, 6, \dots$。

\subsection{通项}
如果等差数列 $\begin{Bmatrix} a_n \end{Bmatrix}$ 的首项是 $a_1$,公差是 $d$,那么根据等差数列的定义可得
\begin{equation}
\begin{aligned}
&a_1 = a_1,\\
&a_2 = a_1 + d,\\
&a_3 = a_2 + d = a_1 + 2d,\\
&\cdots \\
&a_n = a_{n-1} + d = a_1 + (n - 1)d~.
\end{aligned}
\end{equation}

当 $n = 1$ 时
\begin{equation}
a_1 = a_1 + (1 - 1)d = a_1~,
\end{equation}
也就是说这个公式对 $n = 1$ 同样适用。

综上得到等差数列的通项公式。

\begin{corollary}{等差数列通项公式}
对等差数列$\{a_n\}$,其通项公式为:
\begin{equation}
a_n = a_1 + (n - 1)d~.
\end{equation}
其中,$a_n$ 表示数列的第 $n$ 项,$a_1$ 是首项,$d$ 是公差,$n$ 是项数。
\end{corollary}

\textsl{注:这里需要说明一下,带入 $n = 1$ 验算的原因是,我们推算的是 $n > 1$ 时的通项公式,不能说明对首项成立。正如上一节所说,不是所有数列都能写出通项公式,在题目中,经常会出现首项不符合其余项通项公式的情况。}

如果能够证明一个数列满足定义,也就是任意相邻两项的差是定值,那么就能确定他是等差数列,同时也能确定公差。对于任何等差数列,如果确定了是等差数列,那么公差和首项就能唯一确定,也就是说两个数列如果()对应相等,那么他们就是同一个数列。而可以看出,尽管首项对确认等差数列很重要,但更重要的是公差,事实上,知道任意一项都可以通过公差来推出所有其他项,满足:
\begin{equation}
a_n=a_k+(n-k)d~.
\end{equation}
通过这个关系还可以得到另一个关系:
\begin{equation}\label{eq_HsAmPg_1}
d={a_n-a_k\over n-k}~.
\end{equation}
这像极了\aref{平均变化率}{def_HsFunC_3}或者\aref{斜率}{def_HsFunC_5}的定义。其实,如果将等差数列的通项公式变形可以得到:
\begin{equation}
a_n=dn+(a_1-d)~.
\end{equation}
在\enref{数列}{HsSeFu}中曾介绍过,这个形式,其实就是一次函数的离散形式,也就是说,在直角坐标系中,将等差数列的项数$n$作为横坐标,数列的值$a_n$作为纵坐标,点$(n,a_n)$分布在一条直线上。

根据\autoref{eq_HsAmPg_1} 还能得到等差数列的一个重要性质,取四个整数,满足$m+n=p+q$,即$p-n=m-q$,那么根据\autoref{eq_HsAmPg_1} 有:
\begin{equation}
{a_p-a_n\over p-n}=d={a_m-a_q\over m-q}~.
\end{equation}
两侧分母相等,化简得到推论。
\begin{corollary}{}\label{cor_HsAmPg_1}
对于等差数列$\{a_n\}$,若满足$m+n=p+q$,则有:
\begin{equation}
a_m+a_n=a_p+a_q~.
\end{equation}
\end{corollary}

\subsection{等差中项}

如果在 $a$ 和 $b$ 中间插入一个数 $A$,使 $a,A,b$ 成等差数列,那么 $A$ 叫作 $a$ 与 $b$ 的\textbf{等差中项}.

如果 $A$ 是 $a$ 与 $b$ 的等差中项,那么
\begin{equation}
A - a = b - A~,
\end{equation}
\begin{equation}
A = \frac{a+b}{2}~.
\end{equation}



\textsl{注:等差中项是等差数列的重要考点,大部分考察等差数列的题目都会考察等差中项。}

\subsection{等差数列的数列和}

等差数列前 $n$ 项的和可以通过如下方法计算:  
在学习数列时,曾提及过数列和的一个\aref{性质}{eq_HsSeFu_1},据此将数列两次排列后相加有:
\begin{equation}
\begin{split}
2S_n &= (a_1 + a_2 + \cdots + a_n)+(a_n + a_{n-1} + \cdots + a_1)\\
&=(a_1+a_{n})+(a_2+a_{n-1}) +\cdots +(a_n+a_1)~.
\end{split}
\end{equation}

此时,所有括号中的两项的下标之和都是$n+1$,根据\autoref{cor_HsAmPg_1} 这意味着他们的和全部相等,也就是说:
\begin{equation}
2S = n \cdot (a_1+a_n)\implies S = \frac{n\cdot(a_1+a_n)}{2}~.
\end{equation}
这也就是常说的“首项加末项乘以项数除以”

将通项公式 $a_n = a_1 + (n-1)d$ 代入,有:
\begin{equation}
S_n = \frac{n}{2} \times [2a_1 + (n-1)d]~.
\end{equation}




\textbf{等差数列与函数}

求数列和的最小值

若$a_0<0,d>0$,则一定有$a_1<a_2<\dots<a_{k}<0$。

验证$a_{k+1} =0$且$a_{k+m}>0(m\in\mathbb{N}^+)$,或$a_{k+m}>0(m\in\mathbb{N})$。

因此$S_n$的最小値$S_k=S_{k+1}=S$或$S_k=S$