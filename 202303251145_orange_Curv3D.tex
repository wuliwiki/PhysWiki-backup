% 三维空间中的曲线
% 曲线|微分几何|曲率|curvature|扭率|第二曲率|torsion|活动坐标架|活动标架

\pentry{欧几里得空间中的曲线\upref{eucur},张成空间\upref{VecSpn}}

三维欧几里得空间是我们最为熟悉的空间,其性质也非常好,易于研究。本节讨论的是三维空间中的 $C^2$ 参数曲线 $f: I = (a, b) \to \mathbb{R}^3$ 的性质,介绍了Frenet向量组以及该组中向量之间的关系,并引入了曲率、挠率等概念。

\subsection{Frenet向量组}

\begin{figure}[ht]
\centering
\includegraphics[width=4cm]{./figures/Curv3D_1.pdf}
\caption{Frenet向量组示意图。途中三个蓝色向量就是红色曲线在某一点处的Frenet向量组,分别是单位切向量 $T$、单位主法向量 $N$ 和单位副法向量 $B$。} \label{Curv3D_fig1}
\end{figure}

Frenet向量组是\textbf{三维}欧几里得空间 $\mathbb{R}^3$ 中特有的概念,是刻画\textbf{可微曲线}性质的优良工具。

Frenet向量组中的第一个向量,是\textbf{单位切向量}。关于 $t$ 对曲线 $f(t)$ 求导就能得到切向量,如果把 $t$ 看作时间,那么切向量 $\frac{\dd}{\dd t}f(t)$ 也可以看成是曲线上一点的切速度。单位切向量就是方向与切速度相同,但长度为 $1$ 的向量。

为了得到单位切向量,我们可以用切速度除以切速度的大小来获得:$\frac{\dd}{\dd t}f(t)/\abs{\frac{\dd}{\dd t}f(t)}$。但更简单的方法是,令切速度的大小恒为 $1$,或者说只讨论切速度大小恒为 $1$ 的情况,即弧长参数曲线\autoref{eucur_def1}~\upref{eucur};我们将用字母 $s$来表示弧长参数。

\begin{definition}{单位切向量}\label{Curv3D_def1}
使用弧长参数来描述可微曲线 $f(s)$,记 $T(s)=\frac{\dd}{\dd s}f(s)$,称为曲线在参数为 $s$ 处的\textbf{单位切向量(unit tangent vector)}。
\end{definition}

Frenet向量组中的第二个向量,是\textbf{单位主法向量}。对于处处可微的曲线,切向量总是存在的,但是主法向量不一定存在。主法向量的定义如下:

\begin{definition}{主法向量}\label{Curv3D_def2}
使用弧长参数来描述可微曲线 $f(s)$,其单位切向量为 $T(s)$。如果 $\frac{\dd}{\dd s}T(s)$ 不为零,则称 $\kappa(s) N(s)=\frac{\dd}{\dd s}T(s)$ 为曲线在 $s$ 处的\textbf{主法向量(normal vector)}。其中 $N(s)$ 是单位\textbf{向量},$\kappa$ 是一个\textbf{正数},称作曲线在这一点处的\textbf{曲率(curvature)}。
\end{definition}

由于使用弧长参数以后,单位切向量的大小不变,因此 $N(s)$ 恒与 $T(s)$ 垂直。$N(s)$ 就是那第二个成员,单位主法向量。

Frenet向量组中的最后一个成员,是\textbf{单位副法向量},是用前两个向量叉乘而来的。

\begin{definition}{副法向量}
条件设定如\autoref{Curv3D_def1} 和\autoref{Curv3D_def2} ,称 $B(s)=T(s)\times N(s)$ 为曲线在 $s$ 处的\textbf{单位副法向量(unit binormal vector)}。
\end{definition}

以上定义的副法向量必然是单位向量,因为 $T$ 和 $N$ 都是单位向量且正交。

Frenet向量组可以用于构成所谓的\textbf{Frenet坐标架},也称\textbf{活动坐标架}。相比预先规定的自然坐标系,活动坐标架忽略了曲线在空间中的具体位置,更简洁和方便地表示了曲线本身的局部性质。

\begin{definition}{密切平面}
条件设定如\autoref{Curv3D_def1} 和\autoref{Curv3D_def2} ,由 $T(s)$ 和 $N(s)$ 两向量在点 $f(s)$ 处所张成的二维空间,称为曲线在这一点处的\textbf{密切平面(osculating plane)}。
\end{definition}

对于\textbf{平面曲线},即整体都在一个平面上的曲线,其所在的平面就是每个点的密切平面。

\subsection{曲率和扭率}

曲率已经在\autoref{Curv3D_def2} 中定义清楚了,是表征单位切向量旋转速率的数字。要说明的是,如果单位切向量在某点处不变化,那么主法向量的概念就不复存在,也就没有 Frenet 向量组的后两位成员了。在这种情况下,我们说曲线在这一点处的曲率为 $0$。更一致的曲率定义方法如下:
\begin{definition}{曲率}
使用弧长参数来描述可微曲线 $f(s)$,其单位切向量为 $T(s)$。称 $\kappa(s)=\abs{\frac{\dd}{\dd s}T(s)}$ 为曲线在 $s$ 处的\textbf{曲率(curvature)}。
\end{definition}

当Frenet向量组存在时,还有一个重要的概念叫\textbf{扭率}或\textbf{挠率},在有的文献中也会被称为\textbf{第二曲率}。

在定义扭率之前,我们要先讨论副法向量的一个性质:$\frac{\dd}{\dd s}B(s)$ 和 $N(s)$ 平行。首先,因为单位副法向量的长度不变,故有 $\frac{\dd}{\dd s}B(s)$ 和 $B(s)$ 垂直。由于 $B=T\times N$,故 
\begin{equation}
\begin{aligned}
\frac{\dd}{\dd s}B(s)&=[\frac{\dd}{\dd s}T(s)]\times N(s)+T(s)\times\frac{\dd}{\dd s}B(s)\\&=T(s)\times\frac{\dd}{\dd s}B(s)
\end{aligned}
\end{equation}
进而可知 $\frac{\dd}{\dd s}B(s)$ 垂直于 $T(s)$。因此,$\frac{\dd}{\dd s}B(s)$ 垂直于 $T$ 和 $B$,必然和 $N$ 在一条线上了。

\begin{definition}{扭率}
使用弧长参数来描述可微曲线 $f(s)$,令其Frenet向量组为 $T(s), N(s), B(S)$。令实数 $\tau$ 满足 $\frac{\dd}{\dd s}B(s)=-\tau N(s)$,那么称实数 $\tau$ 为曲线在 $s$ 处的\textbf{扭率(torsion)}。
\end{definition}

曲率描述了曲线单位切向量方向变化的速率,越快则曲线弯曲程度越强。扭率则描述了曲线偏离平面曲线的程度,也就是其扭曲程度;也可以理解为,扭率描述了密切平面旋转的速率。

\subsection{Frenet-Serret公式}

三维欧几里得空间中的可微曲线 $f(s)$,如果处处有非零曲率 $\kappa(s)$,且其扭率为 $\tau(s)$(可以为零),则它一定有Frenet向量组。$T$,$N$ 和 $B$ 具有如下关系:

\begin{theorem}{Frenet-Serret公式}
对于曲线的Frenet向量组 $\{T(s), N(s), B(s)\}$,我们有如下关系:
\begin{equation}\label{Curv3D_eq1}
\frac{\dd}{\dd s}T(s)=\kappa N(s)
\end{equation}
\begin{equation}\label{Curv3D_eq2}
\frac{\dd}{\dd s}N(s)=-\kappa(s) T(s)+\tau(s) B(s)
\end{equation}
\begin{equation}\label{Curv3D_eq3}
\frac{\dd}{\dd s}B(s)=-\tau(s) N(s)~.
\end{equation}

用矩阵的形式可以更简洁地表示为

\begin{equation}
\pmat{T'\\N'\\B'}=\pmat{0&\kappa&0\\-\kappa&0&\tau\\0&-\tau&0}\pmat{T\\N\\B}
\end{equation}

矩阵表示也更好记忆。
\end{theorem}

\textbf{证明}:

\begin{enumerate}
\item 根据 $N$ 的定义,直接可得\autoref{Curv3D_eq1}。
\item 根据扭率的定义,直接可得\autoref{Curv3D_eq3}。
\item 考虑到三个向量的方向关系,以及它们都是单位向量的事实,我们可以得到 $N=B\times T$。因此,$\frac{\dd}{\dd s}N=\frac{\dd B}{\dd s}\times T+B\times\frac{\dd T}{\dd s}=-\tau N\times T+B\times\kappa N=\tau B-\kappa T$。
\end{enumerate}

\textbf{证毕}。





