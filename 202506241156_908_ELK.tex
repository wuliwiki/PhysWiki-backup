% 恩里科·费米(综述)
% license CCBYSA3
% type Wiki

本文根据 CC-BY-SA 协议转载翻译自维基百科 \href{https://en.wikipedia.org/wiki/Enrico_Fermi}{相关文章}。

恩里科·费米(意大利语:[enˈriːko ˈfermi],1901年9月29日-1954年11月28日)是一位意大利裔、后归化为美国公民的物理学家,以建造世界上第一座人工核反应堆——芝加哥一号堆而闻名,并曾是曼哈顿计划的重要成员。他被誉为“核时代的建筑师”以及“原子弹之父”。他是极少数在理论物理和实验物理两个领域都卓有成就的物理学家之一。费米因其在中子轰击引发放射性方面的研究以及对超铀元素的发现而获得1938年诺贝尔物理学奖。他与同事们共同申请了多项与核能应用相关的专利,所有这些专利最终都被美国政府接管。他在统计力学、量子理论、核物理和粒子物理的发展中都作出了重要贡献。

费米的第一个重大贡献是在统计力学领域。1925年,沃尔夫冈·泡利提出了著名的泡利不相容原理,随后费米发表了一篇论文,将该原理应用于理想气体,发展出一种统计方法,如今被称为费米–狄拉克统计。今天,那些遵守不相容原理的粒子被称为“费米子”。

后来,泡利为了解释β衰变中能量守恒的问题,提出了在电子发射的同时还会发射一种不带电的不可见粒子这一假设。费米接纳了这个想法,并构建了一个理论模型,纳入了这一假想粒子,并将其命名为“中微子”。他的这一理论后来被称为“费米相互作用”,现今称为“弱相互作用”,是自然界四种基本相互作用之一。

在用新发现的中子进行诱导放射性实验时,费米发现慢中子比快中子更容易被原子核俘获,并据此发展出描述该过程的“费米年龄方程”。在用慢中子轰击钍和铀的实验中,费米认为自己合成了新的元素。尽管他因这一发现获得了诺贝尔奖,但后来证实这些“新元素”其实是核裂变的产物。
