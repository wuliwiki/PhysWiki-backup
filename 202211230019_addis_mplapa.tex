% MPLAPACK 笔记

\begin{issues}
\issueDraft
\end{issues}

\pentry{Lapack 笔记\upref{Lapack}}

\subsection{__float128}
\begin{itemize}
\item 源码和 manual 见 \href{https://github.com/nakatamaho/mplapack}{github}.
\item GCC 里面 \verb|_Float128| 就是 \verb|__float128|
\item 应该只支持列主元\upref{MatSto}, 没有 \verb|LAPACK_ROW_MAJOR| 或者 \verb|LAPACK_COL_MAJOR| 参数.
\item 动态链接库: \verb|libmplapack__Float128.so|
\end{itemize}

\begin{lstlisting}[language=cpp]
// 包含 mplapack_utils__Float128.h
#include <mplapack/mpblas__Float128.h>
// 包含 mplapack__config.h 和 quadmath.h
#include <mplapack/mplapack__Float128.h>

typedef int64_t mplapackint;
typedef mplapackint mplapacklogical;

// 从 LU 分解计算逆矩阵
// 所以 mplapack 中的参数和 LAPACKE 并不完全一致, 需要手动查看.
    // lapacke 的版本
lapack_int LAPACKE_zgetri
    (int matrix_layout , lapack_int n , lapack_complex_double *a,
    lapack_int lda , const lapack_int *ipiv);
    // mplapack 的版本
void Cgetri(mplapackint const n, std::complex<_Float128> *a,
    mplapackint const lda, mplapackint *ipiv,
    std::complex<_Float128> *work, mplapackint const lwork,
    mplapackint &info);

// 解带对角矩阵
    // lapacke 的版本
lapack_int LAPACKE_zgbsv(int matrix_layout, lapack_int n,
    lapack_int kl, lapack_int ku, lapack_int nrhs,
    lapack_complex_double *ab, lapack_int ldab, lapack_int *ipiv,
    lapack_complex_double *b, lapack_int ldb);
    // mplapack 的版本
void Cgbsv(mplapackint const n, mplapackint const kl,
    mplapackint const ku, mplapackint const nrhs,
    std::complex<_Float128> *ab, mplapackint const ldab,
    mplapackint *ipiv, std::complex<_Float128> *b,
    mplapackint const ldb, mplapackint &info);
\end{lstlisting}
