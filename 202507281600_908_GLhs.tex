% 格林函数(综述)
% license CCBYSA3
% type Wiki

本文根据 CC-BY-SA 协议转载翻译自维基百科\href{https://en.wikipedia.org/wiki/Green\%27s_function}{相关文章}。

\begin{figure}[ht]
\centering
\includegraphics[width=8cm]{./figures/36e3d9728324115d.png}
\caption{如果我们已知某个微分方程在点源作用下的解 $G(x, x')$,即满足$\hat{L}(x)G(x,x') = \delta(x - x')$的格林函数,其中 $\hat{L}(x)$ 是一个线性微分算子,那么我们可以通过叠加原理构造出任意源项 $f(x)$ 下的解:$u(x) = \int f(x') G(x, x')\, dx'$从而求得方程$\hat{L}(x) u(x) = f(x)$的解。简而言之,如果你知道格林函数,就可以用它通过积分方式表示出任意源函数 $f(x)$ 下的解 $u(x)$。} \label{fig_GLhs_1}
\end{figure}
在数学中,格林函数(Green's function,亦称 Green 函数)是定义在某一给定初始条件或边界条件下的非齐次线性微分算子的冲激响应。

这意味着,如果 $L$ 是一个线性微分算子,那么:
\begin{itemize}
\item 格林函数$G$ 是满足方程 $LG = \delta$ 的解,其中 $\delta$ 是狄拉克δ函数;
\item 初值问题 $Ly = f$ 的解为 $G * f$(即格林函数与 $f$ 的卷积)。
\end{itemize}
通过叠加原理,对于一个线性常微分方程(ODE)$Ly = f$,可以先对每个源点 $s$ 求解 $LG = \delta_s$,由于源项可以看作多个δ函数的叠加,根据线性算子 $L$ 的线性性,最终的解也是各个格林函数的线性叠加。

格林函数得名于英国数学家乔治·格林,他在 1820 年代首次提出了这一概念。在现代关于线性偏微分方程的研究中,格林函数更多被视为基本解的一种方式来研究。

在多体理论中,这一术语也广泛用于物理学,特别是在量子场论、空气动力学、空气声学、电动力学、地震学和统计场论中,用来表示各种类型的关联函数,即便它们并不总是符合严格的数学定义。在量子场论中,格林函数还扮演着传播子的角色。
\subsection{定义与用途}
格林函数 $G(x, s)$ 是一个线性微分算子 $L = L(x)$ 在欧几里得空间子集 $\mathbb{R}^n$ 上的某个点 $s$ 处作用于分布(广义函数)上的解,它满足以下方程:
$$
L\,G(x,s) = \delta(s - x),~
$$
其中 $\delta$ 是狄拉克 δ 函数。
格林函数的这个性质可以被用来求解如下形式的微分方程:
$$
L\,u(x) = f(x).~
$$
如果算子 $L$ 的核(解空间)非平凡,那么格林函数不唯一。不过在实际中,通过对称性、边界条件或其他外部施加的限制条件,通常可以选定一个唯一的格林函数。格林函数可根据所满足的边界条件类型进行分类,称为“格林函数编号”。格林函数不一定是实变量意义下的函数,通常是在分布意义下理解的。

格林函数也是求解波动方程和扩散方程的重要工具。在量子力学中,哈密顿算子的格林函数是一个核心概念,与态密度的概念有重要联系。

在物理学中,格林函数的定义通常与上述数学定义符号相反,即满足:
$$
L\,G(x, s) = \delta(x - s).~
$$
由于狄拉克 δ 函数是偶函数,这种符号差异不会对格林函数的基本性质产生实质影响。

如果算子是平移不变的,也就是说当 $L$ 对 $x$ 的系数是常数时,格林函数可以写成卷积核的形式:
$$
G(x, s) = G(x - s).~
$$
在这种情形下,格林函数等同于线性时不变系统理论中的脉冲响应。
\subsection{动机}
粗略地说,如果对于算子 $L$ 能找到一个函数 $G$ 满足格林函数方程(公式 1),那么将该方程两边乘以 $f(s)$ 并对 $s$ 积分,就得到:
$$
\int L G(x, s)\, f(s)\, ds = \int \delta(x - s)\, f(s)\, ds = f(x).~
$$
因为算子 $L = L(x)$ 是线性的,且仅作用于变量 $x$(而不作用于积分变量 $s$),我们可以将 $L$ 移出积分号,得到:
$$
L\left(\int G(x, s)\, f(s)\, ds\right) = f(x).~
$$
这意味着:
$$
u(x) = \int G(x, s)\, f(s)\, ds~
$$
是微分方程$Lu(x) = f(x)$的一个解。

因此,只要已知公式 1 中的格林函数 $G(x, s)$ 和公式 2 中右侧的源项 $f(x)$,就可以通过公式 3 求得解 $u(x)$。这个过程依赖于算子 $L$ 的线性性质。

换句话说,公式 2 的解 $u(x)$ 可以通过公式 3 的积分表达式来确定。尽管 $f(x)$ 是已知的,但除非 $G$ 也已知,否则这个积分无法实际计算。因此,问题的关键转变为寻找满足公式 1 的格林函数 $G$。因此,格林函数有时也被称为与算子 $L$ 对应的基本解。

并非每一个算子 $L$ 都允许存在格林函数。从另一个角度看,格林函数也可以被视为算子 $L$ 的一个右逆元。除了找到特定算子的格林函数可能具有挑战之外,公式 3 中的积分本身可能也相当难以计算。但这种方法在理论上给出了一个精确解法。

这个过程还可以被理解为:将 $f$ 分解为狄拉克 δ 函数基底的展开(即将 $f$ 投影到 $\delta(x - s)$ 上),并将每个投影对应的解叠加起来。这样的积分方程被称为Fredholm 积分方程,其研究构成了Fredholm 理论的基础。
