% 旋度
% 多元微积分|矢量场|散度|线积分|通量|环流量|旋度

\pentry{圆周运动的速度\upref{CMVD}, 散度\upref{Divgnc}, 线积分\upref{IntL}}

我们在三维空间的矢量场 $\bvec F(\bvec r)$ 中取一个闭合回路 $\mathcal L$ 并规定一个正方向, 并定义该回路的\textbf{环流量}为矢量场在回路上的线积分
\begin{equation}
\oint_{\mathcal L} \bvec F(\bvec r) \vdot \dd{\bvec r}
\end{equation}
以下假设这个矢量场在某个区域内具有处处具有一阶偏导数.

\begin{figure}[ht]
\centering
\includegraphics[width=4cm]{./figures/Curl_3.pdf}
\caption{面元, 法向量为 $\bvec S$, 面积为 $\abs{\bvec S}$}\label{Curl_fig3}
\end{figure}

现在来定义该矢量场的\textbf{旋度(curl)}. 旋度是一个矢量, 记为 $\curl \bvec F$. 如\autoref{Curl_fig3} , 在空间某点 $(x,y,z)$ 处选取一个小面元 $\bvec S$(模长为面元的面积, 方向为面元的一个法向量), 令面元边界构成的回路为 $\mathcal L$, 正方向由右手定则\upref{RHRul} 判断. 要定义旋度 $\curl \bvec F$ 在 $x$ 方向的分量, 就取 $\bvec S$ 与 $x$ 轴单位矢量 $\uvec x$ 同向, 再计算换流量除以面积的极限, 即
\begin{equation}
\uvec x \vdot (\curl\bvec F) = \lim_{S\to 0} \frac 1S \oint_{\mathcal L} \bvec F(\bvec r) \vdot \dd{\bvec r}
\end{equation}
同理, 要计算旋度的 $y, z$ 分量就把 $\bvec S$ 分别指向单位矢量 $\uvec y, \uvec z$ 的方向, 再讲上式的 $\uvec x$ 分别替换为 $\uvec y, \uvec z$. 这种定义方法叫做 “积分—比—极限”.

可以证明, 旋度处处存在且与回路的形状选取无关\footnote{本书不作证明}. 所以选取任意方向的面元 $\bvec S$, 都有
\begin{equation}\label{Curl_eq3}
(\curl\bvec F) \vdot \uvec S = \lim_{S\to 0} \frac 1S \oint_{\mathcal L} \bvec F(\bvec r) \vdot \dd{\bvec r}
\end{equation}

\subsection{直角坐标系中的旋度}
在直角坐标系中给出矢量场
\begin{equation}
\bvec F(x,y,z) = F_x(x,y,z)\uvec x + F_y(x,y,z)\uvec y + F_z(x,y,z)\uvec z
\end{equation}
在点 $(x,y,z)$ 附近, 我们可以对场使用微分近似(\autoref{TDiff_eq6}~\upref{TDiff})
\begin{equation}
F_i(x+x', y+y', z+z') = F_i(x,y,z) + \pdv{F_i}{x}x' + \pdv{F_i}{y}y' + \pdv{F_i}{z}z'
\end{equation}

\begin{figure}[ht]
\centering
\includegraphics[width=5cm]{./figures/Curl_1.pdf}
\caption{直角坐标系中旋度的 $z$ 分量} \label{Curl_fig1}
\end{figure}

要求 $z$ 方向的旋度, 令闭合回路为\autoref{Curl_fig1} 所示的正方形, 延 $x$ 方向的两条边的线积分仅由 $F_x$ 贡献, 延 $y$ 方向的两条边的线积分仅由 $F_y$ 贡献, 所以整个环路的线积分为
\begin{equation}\ali{
\oint_{\mathcal L} \bvec F \vdot \dd{\bvec r}
&= \int_0^h \qty(F_x + \pdv{F_x}{x}x') \dd{x'} - \int_0^h \qty(F_x + \pdv{F_x}{x}x' + \pdv{F_x}{y}h) \dd{x'}\\
&\quad +\int_0^h \qty(F_y + \pdv{F_y}{x}h + \pdv{F_y}{y}y')\dd{y'} - \int_0^h  \qty(F_y + \pdv{F_y}{y}y') \dd{y'}\\
&= h^2 \qty(\pdv{F_y}{x} - \pdv{F_x}{y})
}\end{equation}
所以旋度的 $z$ 分量为
\begin{equation}\label{Curl_eq7}
G_z = \lim_{h^2\to 0} \frac{1}{h^2} \oint_{\mathcal L} \bvec F \vdot \dd{\bvec r} = \pdv{F_y}{x} - \pdv{F_x}{y}
\end{equation}
类似地, 我们可得 $x, y$ 分量
\begin{equation}
G_x = \pdv{F_z}{y} - \pdv{F_y}{z} \qquad G_y = \pdv{F_x}{z} - \pdv{F_z}{x}
\end{equation}
所以类似叉乘的行列式表示(\autoref{Cross_eq13}~\upref{Cross}), 我们可以将旋度记为
\begin{equation}\label{Curl_eq9}
\curl \bvec F = \vmat{\uvec x & \uvec y & \uvec z\\ \pdv*{x}&\pdv*{y}&\pdv*{z}\\ F_x&F_y&F_z}
\end{equation}
现在我们知道为什么旋度要记为 $\curl \bvec F$ 了, 类比散度, 旋度可以从形式上理解为矢量算符 $\grad$ 与矢量场 $\bvec F$ 的叉乘.

与梯度和散度不同的是, 以上定义的旋度运算只能对三维空间的矢量场作用.

\begin{example}{旋转体速度场的旋度}
一个物体绕 $z$ 轴旋转, 角速度矢量为 $\bvec\omega = \omega\uvec z$, 物体上任意一点的位矢为 $\bvec r$, 则速度关于位置的函数 $\bvec v(\bvec r)$ 构成一个矢量场(\autoref{CMVD_eq5}~\upref{CMVD}) 
\begin{equation}
\bvec v(\bvec r) = \bvec\omega \cross \bvec r = \omega\uvec z \cross (x\uvec x + y\uvec y)
= -\omega y\uvec x + \omega x\uvec y
\end{equation}
使用\autoref{Curl_eq9} 计算 $\bvec v(\bvec r)$ 的散度, 得
\begin{equation}
\curl \bvec v = \vmat{\uvec x&\uvec y&\uvec z\\ \pdv*{x}&\pdv*{y}&\pdv*{z}\\-\omega y&\omega x& 0} = 2\omega \uvec z
\end{equation}
可见该场的旋度是一个 $\uvec \omega$ 方向的常矢量. 从这个例子也可以看出, 如果一个(三维)矢量场在某个方向没有分量(即平面场), 则其旋度必然延该方向(即平面的法向量).
\end{example}

\begin{example}{无旋度的旋转场}
现在我们来看另一个旋转场 $\bvec F(\bvec r) = \uvec z\cross\uvec r /r$, 写成分量的形式就是
\begin{equation}
\bvec F(\bvec r) = \uvec z \cross \qty(\frac{x}{x^2 + y^2}\uvec x + \frac{y}{x^2 + y^2}\uvec y) = -\frac{y}{x^2 + y^2} \uvec x + \frac{x}{x^2 + y^2}\uvec y
\end{equation}
由于这个场也是一个 $xy$ 平面场, 旋度 $\uvec z$ 共线, 可以直接使用\autoref{Curl_eq7} 计算
\begin{equation}
\curl\bvec F = \qty(\pdv{F_y}{x} - \pdv{F_x}{y})\uvec z = \qty(\frac{1}{r^2} - \frac{2x^2}{r^4} + \frac{1}{r^2} - \frac{2y^2}{r^4})\uvec z = \bvec 0
\end{equation}
要注意的是, 在原点处由于矢量场不连续(而是出现了无限大的奇点), 以上计算在原点处并不成立. 
% 未完成:以后会在XXX详细讨论原点处的旋度如何表示.
\end{example}
