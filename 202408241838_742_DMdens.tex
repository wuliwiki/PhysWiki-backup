% 暗物质密度分布
% license Usr
% type Tutor

 

**2.2.1 银河系暗物质密度函数**

我们现在回顾一下文献中通常被认为是最可信的银河系暗物质密度分布 \(\rho(r)\)。所有这些分布都假设球对称性,其中 \(r\) 是从银河中心 (GC) 测量的径向坐标。这些函数在表2.1中列出,并在图2.3中用合理的自由参数值进行绘图。其中一些函数可以方便地重构为具有三个参数 \(\alpha, \beta, \gamma\) 的集体公式[41],也被称为“双幂律”公式:

Hernquist \(\alpha\beta\gamma\): \(\rho_{\alpha\beta\gamma}(r) = \rho_s \left(\frac{r}{r_s}\right)^\gamma \left[1 + \left(\frac{r}{r_s}\right)^\alpha\right]^{\beta-\gamma}\)

参数 \(\alpha, \beta,\) 和 \(\gamma\) 控制从银河中心不同径向距离处暗物质密度分布的形状。具体来说,\(\alpha\) 控制内坡和外坡过渡的锐度,\(\beta\) 影响外坡,而 \(\gamma\) 决定分布的内坡。这些函数的动机如下:

- 纳瓦罗、弗兰克和怀特 (NFW) [42] 配置文件是最常见的基准选择,它是由N体模拟激发的。密度在接近GC时以 \(r^{-1}\) 发散。参数固定为 \(\alpha = 1, \beta = 3\),\(\gamma\) 作为自由参数的版本(在方程 (2.10) 的符号中),有时被称为“广义NFW (gNFW)”配置文件。当 \(\gamma > 1\) 时,中心的斜率比标准NFW更陡:在这种情况下,该配置文件被称为“收缩NFW (cNFW)”。例如,摩尔和他的合作者[43]发现 \(\gamma = 1.16\)。这样的收缩配置文件出现在一些早期的数值模拟中,这些模拟包括了重子。

- Einasto [44] 配置文件作为对更近期数值模拟的更好拟合而出现。广义地说,Einasto配置文件的密度比NFW更“丰满”。更准确地说,Einasto幂律指数随着 \(r\) 的变化而连续变化,由形状参数 \(\alpha_{\text{Ein}}\) 控制。\(\alpha_{\text{Ein}} = 0.17\) 的值代表了不同N体模拟建议的不同值的合理平均值。

- 具有核心的配置文件,如“具有核心”(“截断”)的等温配置文件[45]或Burkert配置文件[46],具有恒定的中心密度。它们是出于一些观测到的银河旋转曲线的动机,这些曲线可能指向核心的存在,参见例如[47],以及数值模拟发现重子反馈降低了中心密度,参见例如[48]。

- Di Cintio等人(2014)[49]提出了一个配置文件,它在方程(2.10)中具有双幂律形式,但参数 \(\alpha, \beta, \gamma\) 并不适用于所有星系,而是依赖于每个星系的恒星-晕质量比。然后,暗物质配置文件有效地从有尖点的变为有核心的,适应每个星系的恒星和暗物质内容。这是基于包括重子的模拟的结果,如第8.5.1节所讨论的。

在考虑的配置文件中,当 \(r \to 0\) 时,\(\rho(r)\) 会发散,然而,在所有情况下 \(r^2\rho(r) \to 0\)(除非 \(\gamma\) 不切实际地大),这样银河中心区域只包含少量的暗物质。

虽然各种密度分布模型在距离银河中心几kpc以上的地方给出相似的结果,包括地球附近,但它们在更小的距离上的差异相当大——相差数个数量级。在银河中心附近,没有关于暗物质分布的观测数据。此外,数值模拟的分辨率不允许我们观测到比大约1kpc更小的尺度。因此,\( \rho(r) \) 的行为在 \( r \to 0 \) 时简单地由假定的渐近函数形式控制。结果,来自银河内部区域的间接暗物质信号,如来自银河中心周围几度区域的伽马射线通量,强烈依赖于不确定的暗物质分布。这与许多其他暗物质信号形成对比,后者取决于远离银河中心的暗物质分布:暗物质直接探测信号取决于地球位置的暗物质密度;一些暗物质信号探测了地球附近的局部银河区域(例如,最高产生于地球几kpc外的高能正电子通量),以及探测远离银河中心区域(例如,来自高纬度的伽马射线)的暗物质信号。

**图2.3** 展示了暗物质分布(图左)和表中列出的暗物质分布模型参数(表右)。确定等温分布参数的过程与其他模型不同,详见文本。表中提供了 \( r_s \)(\( \rho_s \))的2位(3位)有效数字,这对于大多数计算已经足够精确。然而,在特定情况下,例如为了精确重现围绕银河中心的小角度区域的 \( J \) 因子(见第6.2节),需要更精确的输入。

接下来,需要确定进入暗物质分布 \( \rho(r) \) 的参数 \( r_s \)(典型尺度半径)和 \( \rho_s \)(典型尺度密度)。这可以通过不同的方式完成,例如,从银河系类光环的数值模拟中提取它们的值,或者以某种方式从对银河系或类似外部星系的观测中确定它们。一种方法是要求银河系的暗物质分布满足以下一组约束条件:

A) 太阳位置的暗物质密度 \( \rho_\odot \)。许多研究小组使用不同的技术研究了这个量[50],特别是全局方法,它拟合了整个银河系的旋转曲线,如上文所讨论的,或者依赖于研究局部恒星运动学(特别是垂直方向的恒星运动)来确定局部引力作用,从而确定局部暗物质密度的局部方法。全局方法可以提供 \( \rho_\odot \) 的精确测定,但对银河系重子组分的不确定建模很敏感。局部方法不够精确,并且一方面受到由于特殊情况(例如,北南银河半球的不对称性)引起的系统误差的影响,另一方面受到分析中的简化(例如,是否包括所谓的“倾斜项”,它关联了径向和垂直的恒星运动)的影响。

最近的 \( \rho_\odot \) 测定指向

\[ \rho_\odot = \rho(r_\odot) = 0.40 \text{ GeV/cm}^3 \approx 0.0106 M_\odot/\text{pc}^3 ~. \]





