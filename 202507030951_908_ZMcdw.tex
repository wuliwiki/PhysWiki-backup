% 詹姆斯·查德威克(综述)
% license CCBYNCSA3
% type Wiki

本文根据 CC-BY-SA 协议转载翻译自维基百科\href{https://en.wikipedia.org/wiki/James_Chadwick}{相关文章}。

詹姆斯·查德威克爵士(1891年10月20日-1974年7月24日)是英国核物理学家,因发现中子于1935年获得诺贝尔物理学奖。1941年,他撰写了《MAUD报告》的最终稿,促使美国政府开始认真进行原子弹研究工作。二战期间,他担任英国参与曼哈顿计划团队的负责人。由于在核物理领域的成就,他于1945年在英国被授予爵士荣誉。

查德威克于1911年毕业于曼彻斯特维多利亚大学,在那里师从被誉为“核物理之父”的欧内斯特·卢瑟福。在曼彻斯特,他继续在卢瑟福的指导下学习,并于1913年获得硕士学位。同年,查德威克获得了1851年皇家博览会委员会颁发的1851年研究奖学金。他选择前往柏林在汉斯·盖革手下研究β射线。利用盖革新近研发的盖革计数器,查德威克能够证明β射线产生的是连续谱,而非此前认为的离散谱。第一次世界大战在欧洲爆发时,他仍在德国,之后在鲁勒本拘留营度过了四年。

战争结束后,查德威克追随卢瑟福来到剑桥大学卡文迪许实验室,在卢瑟福的指导下于1921年6月在剑桥冈维尔与凯斯学院获得博士学位。他在卡文迪许实验室担任卢瑟福的助理研究主任超过十年,当时该实验室是世界上最重要的物理研究中心之一,吸引了约翰·考克饶夫特、诺曼·费瑟、马克·奥利芬特等学生前来求学。在发现中子后,查德威克测量了中子的质量。他预见到中子将在抗击癌症中成为一项重要武器。1935年,查德威克离开卡文迪许实验室,成为利物浦大学物理学教授,他改造了陈旧的实验室,并通过安装回旋加速器,将其建设成为核物理研究的重要中心。
\subsection{教育与早年生活}
詹姆斯·查德威克于1891年10月20日出生在柴郡博林顿,\(^\text{[4][5]}\)是纺棉工约翰·约瑟夫·查德威克与家庭女佣安妮·玛丽·诺尔斯的长子。他以祖父詹姆斯的名字命名。1895年,他的父母搬到曼彻斯特,将他留在外祖父母家照顾。他曾就读于博林顿十字小学,并曾获得曼彻斯特文法学校的奖学金,但由于家庭仍需支付少量费用而负担不起,不得不放弃这一机会。随后,他进入曼彻斯特男孩中央文法学校就读,并在那里与父母团聚。当时他已有两个弟弟,哈里和休伯特;他还有一个妹妹在婴儿时期便夭折。16岁时,他参加了两场大学奖学金考试,并双双获得。\(^\text{[6][7]}\)

1908年,查德威克选择进入曼彻斯特维多利亚大学就读。他原本打算学习数学,但误选了物理专业。和大多数学生一样,他住在家里,每天步行往返于家和大学之间,单程4英里(6.4公里)。在第一学年结束时,他获得了赫金伯顿奖学金,用于继续学习物理。当时物理系由欧内斯特·卢瑟福领导,他会给毕业班学生分配研究项目,查德威克被指派设计一种方法,用以比较两种不同放射源的放射能量。卢瑟福的想法是以1克(0.035盎司)镭的放射活度作为测量单位,这一单位后来被称为“居里”。卢瑟福提出的方法不可行——查德威克知道这一点,但不敢告诉卢瑟福——于是他继续尝试,最终设计出了所需的方法。这项结果成为查德威克的第一篇论文,由他与卢瑟福合作,于1912年发表。\(^\text{[8]}\)1911年,他以一等荣誉毕业。\(^\text{[9]}\)

在设计出测量伽马射线的方法后,查德威克接着测量了伽马射线在不同气体和液体中的吸收情况。这一次,研究结果以他个人名义单独发表。他于1912年获得理学硕士(MSc)学位,并被任命为拜尔研究员。第二年,他获得了1851年博览会奖学金,使他能够前往欧洲大陆的大学继续学习和研究。他选择于1913年前往柏林的帝国物理技术研究院,在汉斯·盖革的指导下研究β射线。\(^\text{[10]}\)借助盖革新近研发的盖革计数器(其测量精度高于早期的摄影技术),他能够证明β射线并不像此前认为的那样产生离散谱线,而是产生在某些区域有峰值的连续谱。\(^\text{[11][12][13][14]}\)爱因斯坦曾在参观盖革实验室时对查德威克说:“我可以单独解释其中任何一个现象,但我无法同时解释这两个现象。”\(^\text{[13]}\)这一连续谱现象在许多年里仍然是一个未解之谜。\(^\text{[15]}\)
\subsection{研究员}
\subsubsection{剑桥时期}
查德威克的克拉克–麦克斯韦奖学金于1923年到期,由俄罗斯物理学家彼得·卡皮查接替。英国科学与工业研究部顾问委员会主席威廉·麦考密克爵士安排查德威克担任卢瑟福的助理研究主任。在这一职位上,查德威克协助卢瑟福挑选博士研究生。在接下来的几年里,这些学生包括约翰·考克饶夫特、诺曼·费瑟和马克·奥利芬特,他们后来都与查德威克建立了深厚的友谊。由于许多学生入学时并不清楚自己想研究什么课题,卢瑟福和查德威克会为他们推荐研究方向。查德威克还负责编辑实验室产出的所有论文。[23]
\begin{figure}[ht]
\centering
\includegraphics[width=8cm]{./figures/d47fc0098f52cb5e.png}
\caption{卡文迪许实验室的原址曾经见证了物理学上一些重要发现的诞生。该实验室于1874年由德文郡公爵(其家族姓氏为卡文迪许)创建,首任教授是詹姆斯·克拉克·麦克斯韦。如今,实验室已迁至剑桥西区。[24]} \label{fig_ZMcdw_1}
\end{figure}
1925年,查德威克结识了利物浦一位股票经纪人的女儿艾琳·斯图尔特–布朗。两人在同年8月结婚,[23] 卡皮查担任伴郎。这对夫妇于1927年2月迎来了双胞胎女儿乔安娜和朱迪思。[25]

在研究中,查德威克继续探索原子核结构。1925年,自旋概念的提出使物理学家能够解释塞曼效应,但同时也带来了未解的异常现象。当时,人们认为原子核是由质子和电子组成的,因此例如质量数为14的氮核被认为包含14个质子和7个电子。这使其质量和电荷正确,但自旋却不正确。[26]

1928年在剑桥举行的一次关于β粒子和伽马射线的会议上,查德威克再次见到了盖革。盖革带来了由他的博士后学生瓦尔特·穆勒改进的新型盖革计数器。自战争以来,查德威克一直没有使用过盖革计数器,而新型的盖革–穆勒计数器相较于剑桥当时使用的闪烁技术(依赖人眼观测)而言,可能是一个重大进步。其主要缺点在于,它会同时探测α、β和γ射线,而卡文迪许实验室通常用于实验的镭会同时发射这三种射线,因此不适合查德威克的研究需求。然而,钋只发射α粒子,于是莉泽·迈特纳从德国寄给查德威克大约2毫居里(约0.5微克)的钋样品供其研究使用。[27][28]

在德国,瓦尔特·玻特和他的学生赫伯特·贝克尔使用钋以α粒子轰击铍,产生了一种异常形式的射线。查德威克让他的澳大利亚1851年博览会奖学金生休·韦伯斯特重复了他们的实验结果。对查德威克来说,这证明了他和卢瑟福多年来一直假设的东西:中子,即一种没有电荷的理论核粒子。[27]

随后在1932年1月,费瑟向查德威克报告了另一个令人惊讶的实验结果。弗雷德里克和伊雷娜·居里使用钋和铍作为他们认为是伽马射线的射线源,从石蜡中打出了质子。卢瑟福和查德威克对此持不同意见;他们认为质子的质量太大,不可能被伽马射线打出。但中子只需要很少的能量就能产生相同的效果。在罗马,埃托雷·马约拉纳也得出了相同的结论:居里夫妇实际上已经发现了中子,只是他们自己还没有意识到这一点。[29]
\begin{figure}[ht]
\centering
\includegraphics[width=8cm]{./figures/43b1499974483c6a.png}
\caption{} \label{fig_ZMcdw_2}
\end{figure}
查德威克放下了自己所有其他事务,全力投入到证明中子存在的研究中,由费瑟协助[30],经常工作到深夜。他设计了一个简单的装置,由一个圆柱体组成,内置钋射线源和铍靶,产生的射线可以直接照射到诸如石蜡之类的材料上。被击出的粒子是质子,这些质子会进入一个小型电离室,通过示波器进行探测。[29]

1932年2月,在使用中子实验仅约两周后,[16] 查德威克向《自然》杂志投去了一封题为《可能存在的中子》的信件。[31] 同年5月,他向《皇家学会A辑会刊》提交了详细阐述研究结果的论文《中子的存在》。[32][33] 他对中子的发现是理解原子核结构的一个重要里程碑。在阅读查德威克的论文后,罗伯特·巴彻和爱德华·康登意识到,当时理论中的一些异常现象(如氮的自旋问题)如果将中子视为自旋为1/2的粒子,则可以得到解释,同时也意味着一个氮原子核由7个质子和7个中子组成。[34][35]

理论物理学家尼尔斯·玻尔和维尔纳·海森堡曾探讨过中子是否可能像质子和电子一样是一种基本核粒子,而不是质子–电子对。[36][37][38][39] 海森堡证明,中子最好被描述为一种新的核粒子,[38][39] 但其确切本质仍不明确。在1933年的贝克讲座中,查德威克估计中子的质量约为1.0067 Da。由于质子和电子的总质量是1.0078 Da,这意味着如果中子是质子–电子复合体,其结合能约为2 MeV,这在数值上看似合理,[40] 尽管很难理解一个结合能如此之小的粒子为何能保持稳定。[39] 然而,要估算如此微小的质量差,需要极具挑战性的精密测量,而在1933至1934年期间得出了数个相互矛盾的结果。弗雷德里克和伊雷娜·居里使用α粒子轰击硼时,得出中子质量较大的结果,而加州大学欧内斯特·劳伦斯的团队则得出了较小的结果。[41]

随后,从纳粹德国逃亡出来、在卡文迪许实验室读研究生的莫里斯·戈德哈伯向查德威克建议,可以利用208Tl(当时称作钍C")发射的2.6 MeV伽马射线光致解离氘核来测定中子质量:
$$
{}^{2}_{1}\text{D} + \gamma \rightarrow {}^{1}_{1}\text{H} + n~
$$
通过这一过程便可以准确测定中子的质量。查德威克和戈德哈伯尝试了这一方法,发现可行。[42][43][44] 他们测量到产生的质子的动能为1.05 MeV,此时方程中的未知量便是中子的质量。查德威克和戈德哈伯据此计算得出中子质量为1.0084或1.0090原子单位,具体取决于所使用的质子和氘核质量数据。[45][44] (目前公认的中子质量为1.00866 Da。)这一质量数值表明中子的质量太大,不可能是质子–电子对。[45]

由于发现了中子,查德威克于1932年获英国皇家学会休斯奖章,1935年获诺贝尔物理学奖,1950年获科普利奖章,1951年获富兰克林奖章。[7] 他对中子的发现使得通过慢中子俘获并伴随β衰变在实验室中制造比铀更重的元素成为可能。与带正电的α粒子不同,α粒子会被其他原子核中的电力排斥,而中子不需要克服任何库仑势垒,因此可以穿透并进入甚至像铀这样最重元素的原子核中。这激励了恩里科·费米研究慢中子与原子核碰撞引发的核反应,费米因此项工作于1938年获得诺贝尔奖。[46]

为了说明查德威克在1914年报告的β射线连续谱,沃尔夫冈·泡利于1930年12月4日提出了另一种粒子的存在。由于β射线的能量并未完全被测量到,似乎违反了能量守恒定律,但泡利认为,如果有另一种尚未被发现的粒子参与,能量守恒可以得以维持。[47] 泡利也将这种粒子称为“中子”,但它显然与查德威克发现的中子不同。费米将其改名为“中微子”,在意大利语中意为“小中子”。[48] 1934年,费米提出了β衰变理论,解释了从原子核中发射出来的电子是由于中子衰变为质子、电子和中微子时产生的。[49][50] 中微子能够解释丢失的能量,但这种质量极小、无电荷的粒子极难被探测到。鲁道夫·皮尔斯和汉斯·贝特计算得出中微子可以轻易穿过地球,因此探测到它们的几率极小。[51][52] 直到1956年6月14日,弗雷德里克·雷因斯和克莱德·考恩在一个大型核反应堆产生的大量反中微子流中放置探测器,才确认了中微子的存在。[53]

