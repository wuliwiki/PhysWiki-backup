% JavaScript 笔记

\pentry{HTML 基础}
JavaScript 常见于网页中, 一般浏览器都可以运行. 我们以 Chrome 浏览器为例演示 \verb|Hello World| 程序.

\begin{lstlisting}[language=js]
<!DOCTYPE html>
<html>
	<body>
		<div id = a></div>
		<script>
            console.log("hello world 1");
            window.alert("hello world 2");
			document.getElementById("a").innerHTML = "hello world 3";
            document.write("hello world 4");
		</script>
	</body>
</html>
\end{lstlisting}
把这段程序复制到一个文本文件, 并命名为 \verb|test.html|, 用 Chrome 打开即可自动运行. 按 F12 可打开调试窗口和命令行.

\verb|<script>...</script>| 中的 4 行程序就是 JavaScript, 它们这里分别演示了用 4 种不同的方法显示 “hello world”:
\begin{enumerate}
\item 输出到命令行(需要 F12 才能看到)
\item 弹出提示窗
\item 插入到 \verb|<div>...</div>| 元素中
\item 插入到 \verb|<script>| 之前
\end{enumerate}

\subsection{常识}
\begin{itemize}
\item \verb|<body>| 中任意位置可以包含任意多个 \verb|<script>|
\item 用 \verb|<script src="路径/文件名.js 或者 url.js"></script>| 插入代码文件
\item 在第一行插入 \verb|'use strict'| 后使用严格语法. 例如不声明的变量会出错.
\item 命令后的 \verb|;| 不是必须的, 可用于在同一行中分隔两个命令.
\end{itemize}

\subsection{变量}
\begin{itemize}
\item 声明变量 \verb|var a = 1, b = 2|
\item \verb|typeof| 获取变量类型
\item 自带类型有 \verb|number| (不区分整数和浮点数,浮点是双精度), \verb|string| (单引号双引号通用, 没有 char)
\item 所有自定义的类型都叫做 \verb|object|
\item 合并字符串用 \verb|字符串 + 字符串|, 也可以 \verb|字符串 + 数字|
\end{itemize}
