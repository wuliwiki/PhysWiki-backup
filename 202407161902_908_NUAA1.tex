% 南京航空航天大学 2004 量子真题
% license Usr
% type Note

\textbf{声明}:“该内容来源于网络公开资料,不保证真实性,如有侵权请联系管理员”

\subsection{简答题(每小题12分,共60分)}
1. 已知一维运动粒子波函数为
$ \Psi(x, t) = e^{\frac{i}{\hbar}(px - Et)} + e^{-\frac{i}{\hbar}(px + Et)}$ 
其中 $p$ 为动量,$E$ 为能量。说明:(1)它是不是动量算符 $\hat{p}$ 的本征函数?(2)它是不是动量平方算符 $\hat{p}^2$ 的本征函数?

2. 对一维运动粒子,求算符 $\hat{p} +\hat{x}$ 的本征函数和本征值。其中 $\hat{p}$ 为动量算符,$\hat{x}$ 为位置坐标算符。

3. 已知氢原子处在
$$ \psi(r, \theta, \phi) = \frac{1}{\sqrt{\pi a_0^3}} e^{-r/a_0}~$$ 
状态,其中 $a_0$ 为第一玻尔半径。计算其势能 $V(r) = -\frac{1}{4\pi \epsilon_0} \frac{e^2}{r}$ 的平均值。

4. 已知力学量算符 $A, \, B$ 均可表示为二阶矩阵,它们满足关系:
$$A^2 = 0 \quad ; \quad AA^{\dagger} + A^{\dagger}A = 1 \quad ; \quad B = A^{\dagger}A~$$

(1) 证明: $B^2 = B$

(2) 在 $B$ 表象中求出矩阵 $\hat{A}, \, \hat{B}$.

5. 已知在一维无限深势阱
$$
\begin{cases}
V = \infty & (x < 0, \, x > a) \\
V = 0 & (0 \leq x \leq a)
\end{cases}~
$$
中运动粒子处于波函数 $\psi(x) = \frac{4}{\sqrt{a}} \sin{\frac{\pi x}{a}} \cos^2{\frac{\pi x}{a}}$ 所描述的状态中,求:
粒子能量的可能数值及其相应概率。

\subsection{(本题30分)}
在沿$Y$轴方向的恒定磁场 $\overline{B}(0, B, 0)$ 中考察电子的自旋运动。已知电子和磁场相互作用哈密顿算符为 $\hat H = -\hat{\mu} \cdot \hat{B}$,其中 $\hat{\mu} = -\mu_{0} \hat{\sigma}$ 是电子自旋磁矩,$\mu_0$ 是玻尔磁子,$\hat{\sigma}$ 为泡利(Pauli)矩阵。设初始 $t=0$ 时电子处在自旋态
$ \chi(0) = \begin{pmatrix} 1 \\ 0 \end{pmatrix},$
求在以后$t$时刻:

1. 电子的自旋态 $\chi(t)$。

2. 在该态中自旋算符的平均值: $\overline S_x(t)$, $\overline S_y(t) $, $\overline S_z(t)$。

3. 在该时刻测量电子自旋向上和向下的概率各是多少?

4. 经过多少时间,电子自旋反转处在自旋态 $\chi(t) = \begin{pmatrix} 0 \\ 1 \end{pmatrix}$?

\subsection{(本题30分)}
一个质量 $M$,半径 $R$ 的薄圆盘绕通过中心与盘面垂直的轴旋转。

1. 试用量子力学描述该转盘的运动:求出系统哈密顿量和它的本征函数和本征值,讨论能级的简并情况。

2. 如果该转盘在转动过程中受到一个微扰 $H' = F_0 \delta(\varphi - \varphi_0)$,准确定到一级近似,求系统能量和零级波函数。

\subsection{(本题 30 分)}
一原子总能量算符 $H_0$ 的正交归一本征函数为 $\psi_n(x)$,能级 $E_n$ ($n = 1, 2, 3, \ldots$)。已知 $t = 0$ 时系统处于基态 $\psi_1$,$t > 0$ 时原子受到外来微扰 $H(X,t) = F(x) e^{-\frac{t}{\tau}}$ 的作用。试用微扰理论(一级近似)求$t\gg\tau(t \rightarrow \infty)$ 时,原子处于各激发态 $ \psi_n $ 的概率。