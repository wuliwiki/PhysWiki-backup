% 磁介质摘要

完全类比于电介质的极化\upref{DLT},我们可以了解出磁介质的极化性质。

\begin{itemize}
\item 在外加磁场下,介质中产生大量总体有序的磁偶极子(顺磁质的磁化\upref{ParaMa}与抗磁质的磁化\upref{DiaMaM})。这使介质的磁偶密度(磁化强度)$\bvec M$不再为零\upref{MaInte}。

\item 磁偶极子导致了磁化电流 $\bvec j_s = \curl \bvec M$。\upref{MaInte}
\item 同时,由于电偶的旋转,还产生了极化电流 $\bvec j_p = \pdv{P}{t}$ \upref{PolCur}。 %似乎这个词条没有说明极化电流的来历

\item 磁化电流、极化电流产生了额外的磁场。因此磁性介质的存在改变了磁场的分布。\upref{PolCur} $\bvec B = \bvec B_0 + \bvec B'$, $\curl {\bvec B} = \mu_0 (\bvec j_f + \bvec j_s + \bvec j_p) + \mu_0\epsilon_0 \pdv{\bvec E}{t} = \mu_0 \bvec j_f + \mu_0 \curl \bvec M + \mu_0 \pdv{\bvec P}{t} + \mu_0 \epsilon_0 \pdv{\bvec E}{t}$

\item 为了简化极化、磁化电流的影响,引入$\bvec H$矢量:$\bvec H = \frac{\bvec B}{\mu_0} - \bvec M$,并有$\bvec H$的环路定理 $\curl \bvec H = \bvec j_f + \pdv{\bvec D}{t}$\upref{PolCur}

\item 在线性磁介质中,磁偶密度与$\bvec B$存在线性关系  $\bvec M = \frac{1}{\mu_0} \frac{\chi_B}{1+\chi_B}\bvec B$,因此$\bvec H = \frac{\bvec B}{\mu_0\mu_r}= \frac{\bvec B}{\mu}$,其中$\mu_r = 1+\chi_B,  \mu = \mu_0 \mu_r$\upref{PolCur}。
\end{itemize}
