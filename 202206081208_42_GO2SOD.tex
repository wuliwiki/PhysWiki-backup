% 化一般常微分方程组为标准方程组(常微分方程)
% 一般常微分方程|标准方程组

\pentry{基本知识(常微分方程)\upref{ODEPr}}
如\autoref{ODEPr_sub1}~\upref{ODEPr}所说,存在及唯一定理是对标准常微分方程组所证明的,而一般常微分方程组都可化为这一方程组,本节就来给出这一转化程序.

\subsection{一个常微分方程化为标准微分方程}
在仅含一个常微分方程的情形,即将一般常微分方程
\begin{equation}\label{GO2SOD_eq1}
F(x,y,y',\cdots,y^{(n)})=0
\end{equation}
化为具有标准常微分方程组\autoref{ODEPr_def1}~\upref{ODEPr}的形式:
\begin{equation}\label{GO2SOD_eq3}
y_i'=f_i(x,y_i,\cdots,y_m),\quad i=1,\cdots,m
\end{equation}

我们总可以从方程\autoref{GO2SOD_eq1} 的已解出最高阶导数的方程开始,即
\begin{equation}\label{GO2SOD_eq2}
y^{(n)}=f(x,y,y',\cdots,y^{(n-1)})
\end{equation}
这是因为,从\autoref{GO2SOD_eq1} 到\autoref{GO2SOD_eq2} 的过程不属于微分方程领域,而属于函数论领域.这一过程中,某些问题被归结为微分方程领域,比如\autoref{GO2SOD_eq1} 中关于 $y^{(n)}$ 是二次的,因此 $y^{(n)}$ 是其余变量的二值函数(假设这二值不同),实际上得到的是两个形如\autoref{GO2SOD_eq2} 的微分方程 $y_1^{(n)},y_2^{(n)}$.但是,\autoref{GO2SOD_eq3} 将 $y_1,y_2$ 并在一起了,这就需要讨论\autoref{GO2SOD_eq1} .关于这种方程的研究,引向微分方程的\textbf{奇解}概念,而我们并不讨论这一问题.故这里只需记住,从\autoref{GO2SOD_eq1} 总能获得\autoref{GO2SOD_eq2} .
\begin{theorem}{}
方程
\begin{equation}\label{GO2SOD_eq4}
y^{(n)}=f(x,y,y',\cdots,y^{(n-1)})
\end{equation}
与标准方程组
\begin{equation}\label{GO2SOD_eq5}
\leftgroup{
&y_1'=y_2\\
&y_2'=y_3\\
&\cdots\\
&y_{n-1}'=y_n\\
&y_n'=f(x,y_1,\cdots,y_n)
}
\end{equation}
等价.

其中,
\begin{equation}\label{GO2SOD_eq6}
y_1=y,\;y_2=y',\;\cdots,\;y_n=y^{(n-1)}
\end{equation}
\end{theorem}
\textbf{证明:}\begin{enumerate}
\item \autoref{GO2SOD_eq4} $\Rightarrow$ \autoref{GO2SOD_eq5} 

设 $y$ 满足\autoref{GO2SOD_eq4} ,对\autoref{GO2SOD_eq6} 求导,得到
\begin{equation}
y_i'=y^{(i)},\quad i=1,\cdots,n
\end{equation}
将

\end{enumerate}
