% 流体力学方程组
% 流体力学|守恒方程|方程组

\pentry{Navier-Stokes 方程\upref{NSeq}}

描述流体微元的物理量有 $\rho,p,T,e,\bvec u$ 等,而完整描述它们随时间的演化需要流体力学方程组。它们分别是质量守恒方程(\autoref{fluidC_eq6}~\upref{fluidC})、动量守恒方程(\autoref{NSeq_eq2}~\upref{NSeq})、动能守恒方程(\autoref{fluidC_eq5}~\upref{fluidC})、内能守恒方程(\autoref{fluidC_eq7}~\upref{fluidC})、本构方程(\autoref{NSeq_eq1}~\upref{NSeq})和状态方程 $\rho=\rho(p,T)$。
\begin{equation}
\begin{aligned}
&\pdv{\rho}{t}+\nabla\cdot(\rho\bvec u)=0\\
&
\rho \dv{u_j}{t}=-\pdv{p}{x_j}+\rho g_j+\mu\pdv[2]{u_j}{x_i} +\qty(\mu_\nu+\frac{1}{3}\mu)\pdv{x_j}\pdv{u_m}{x_m}\\
&\rho \dv{}{t}\qty(\frac{1}{2}|\bvec u|^2)=\rho g_iu_i+u_j\pdv{T_{ij}}{x_i}=\rho g_iu_i+u_j\qty(-\pdv{p}{x_j}+\pdv{\tau_{ij}}{x_i})\\
&\rho \dv{e}{t}=-p\pdv{u_m}{x_m}+2\mu\qty(S_{ij}-\frac{1}{3}\pdv{u_m}{x_m}\delta_{ij})^2+\mu_\nu\qty(\pdv{u_m}{x_m})^2+\pdv{x_i}\qty(k\pdv{T}{x_i})
\\
&T_{ij}=-p\delta_{ij}+\tau_{ij}=-p\delta_{ij}+2\mu \qty(S_{ij}-\frac{2}{3}S_{mm}\delta_{ij})+\mu_\nu S_{mm} \delta_{ij}\\
&\rho=\rho(p,T)~.
\end{aligned}
\end{equation}
上面的本构方程和状态方程可以代入随时间演化的方程,随时间演化的为前 $4$ 行(其中动量守恒方程是三分量的),共 $6$ 个方程;其中动能守恒方程可以由动量守恒方程推出,所以可以舍去,总共可以得到 $5$ 个分量的独立的方程。现在再来看变量个数,$\rho,p,T,e,\bvec u$ 一共 $7$ 个分量,其中 $e,\rho$ 都蕴含在状态方程之中:$\rho=\rho(p,T),e=e(p,T)$,所以实际上独立的变量个数是 $5$ 个。 $5$ 个独立的方程和 $5$ 个独立的变量,构成了完备的流体力学方程组。

于是,这些方程可以用于作数值模拟。但是数值计算的计算能力和模拟精度是有限的。抛开数值计算,这些流体力学方程是相当复杂的,以至于即使人们研究相当简单的模型,也几乎无法得到解析解。因此我们会首先研究一些流体力学方程组能很好地解决和描述的问题,例如\textbf{涡旋}、\textbf{伯努利公式}、\textbf{重力波}等。
