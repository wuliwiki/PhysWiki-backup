% 张量积(综述)
% license Pub
% type Wiki

本文根据 CC-BY-SA 协议转载翻译自维基百科\href{https://en.wikipedia.org/wiki/Tensor_product}{相关文章}。

在数学中,两个向量空间\( V \)和\( W \)(在相同的域上)的张量积\( V \otimes W \)是一个向量空间,它与一个双线性映射\(V \times W \to V \otimes W\)相关联,该映射将一对\( (v, w) \),其中\( v \in V\), \( w \in W \),映射到\( V \otimes W \)中的一个元素,表示为\( v \otimes w \)。

形式为\( v \otimes w \)的元素称为\( v \)和\( w \)的张量积。\( V \otimes W \)中的元素称为张量,两个向量的张量积有时被称为**初等张量**或**可分解张量**。初等张量生成了\( V \otimes W \),即 \( V \otimes W \) 中的每个元素都是初等张量的和。如果给定 \( V \)和\( W \) 的基,则\( V \otimes W \)的基是所有\( V \)的基元素与\( W \)的基元素的张量积。

两个向量空间的张量积捕捉了所有双线性映射的性质,具体而言,来自\( V \times W \)到另一个向量空间\( Z \)的双线性映射可以通过线性映射\(V \otimes W \to Z\) 唯一地分解(见下文标题为“普遍性质”的部分),即该双线性映射与从张量积 \( V \otimes W \)到\( Z \) 的唯一线性映射相关联。

张量积在许多应用领域中都有使用,包括物理学和工程学。例如,在广义相对论中,重力场通过度量张量来描述,度量张量是一个张量场,每个空间时间流形的点上都有一个张量,并且每个张量属于该点的余切空间与自身的张量积。
\subsection{定义与构造}
两个向量空间的张量积是一个向量空间,它的定义是直到同构的。定义它有几种等价的方式。大多数方式都是显式地定义一个向量空间,称为张量积,通常,这些定义的等价性证明几乎是直接由所定义的向量空间的基本性质得出的。

张量积也可以通过普遍性质来定义;见下文的“普遍性质”部分。与所有的普遍性质一样,所有满足该性质的对象通过一个唯一的同构进行同构,并且这个同构与普遍性质兼容。当使用这种定义时,其他定义可以看作是满足普遍性质的对象的构造,并且证明存在满足普遍性质的对象,即证明张量积的存在。
\subsubsection{从基底出发}
设\( V \)和\( W \) 是定义在域\( F \)上的两个向量空间,分别具有基底\( B_V \)和\( B_W \)。

向量空间\( V \)和\( W \)的张量积\( V \otimes W \)是一个向量空间,其基底是所有形式为 \( v \otimes w \) 的集合,其中 \( v \in B_V \) 且 \( w \in B_W \)。这个定义可以通过以下方式进行形式化(这个形式化通常在实践中不常用,因为前述的非正式定义通常已经足够):  
\( V \otimes W \) 是从笛卡尔积 \( B_V \times B_W \) 到 \( F \) 的所有函数的集合,这些函数有有限个非零值。通过逐点操作,使得 \( V \otimes W \) 成为一个向量空间。将 \( (v, w) \) 映射为 1,其他 \( B_V \times B_W \) 中的元素映射为 0 的函数表示为 \( v \otimes w \)。

集合  
\[
\{v \otimes w \mid v \in B_V, w \in B_W\}
\]  
然后直接构成了 \( V \otimes W \) 的基底,这个基底被称为基底 \( B_V \) 和 \( B_W \) 的张量积。