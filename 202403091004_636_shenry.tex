% 一元函数的对称与周期性
% keys 对称轴|周期性|中心对称|函数
% license Xiao
% type Tutor

\pentry{函数\nref{nod_functi}}{nod_6e85}

\textbf{轴对称},即当横坐标到对称轴的距离相等时,函数值也相等
\begin{equation}
f\left( {a + x} \right) = f\left( {a - x} \right)~.
\end{equation}
\textbf{中心对称},即当横坐标到对称轴的距离相等时,函数值互为相反数
\begin{equation}
f\left( {a + x} \right) =  - f\left( {a - x} \right)~.
\end{equation}
\textbf{周期性},即横坐标经过一定长度后的函数值相等,
\begin{equation}
f\left( {a + x} \right) = f\left( x \right)~.
\end{equation}
其中,$a$ 分别为对称轴、对称中心和周期 $T$。定义\textbf{最小正周期}为其字面意思——周期函数最小的一个正周期,不是所有的周期函数都有最小正周期,例如常函数 $f(x) = c$ 就不存在最小正周期。$\sin x$ 和 $\cos x$ 的最小正周期都是 $2\pi$。

现在来看同时含有参数 $a$ 和参数 $b$ 的情况
\begin{equation}
\text{轴对称:} f\left( {a + x} \right) = f\left( {b - x} \right)~.
\end{equation}
\begin{equation}
\text{中心对称:} f\left( {a + x} \right) =  - f\left( {b - x} \right)~.
\end{equation}
\begin{equation}
\text{周期性:} f\left( {a + x} \right) = f\left( {b + x} \right)~.
\end{equation}
其对称轴、对称中心、周期分别为:$x = (a + b)/2$, $((a + b)/2, 0)$ 和 $T = \left| {b - a} \right|~.$

若 $f(x)$ 关于直线 $x=a$ 与直线 $x=b$ 对称, 则 $f(x)$ 的一个周期为 $2\left| {b - a} \right|~.$

若 $f(x)$ 关于点 $(a,0)$ 与点 $(b,0)$ 对称,则 $f(x)$ 的一个周期为 $2\left| {b - a} \right|~.$

若 $f(x)$ 关于直线 $x=a$ 与点 $(b,0)$ 对称,则 $f(x)$ 的一个周期为 $4\left| {b - a} \right|.$
