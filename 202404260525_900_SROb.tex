% 斜坐标系表示洛伦兹变换
% keys 斜坐标|洛伦兹变换|拉伸比例|斜率
% license Xiao
% type Tutor

\pentry{洛伦兹变换\nref{nod_SRLrtz},斜坐标系\nref{nod_ObSys}}{nod_7d36}
\subsection{洛伦兹变换的斜坐标表示}

一维空间的洛伦兹变换表述为

\begin{equation}\label{eq_SROb_1}
\leftgroup{
&x' = \frac{x - vt}{\sqrt{1 - v^2/c^2}}\\
&t' = \frac{t - vx/c^2}{\sqrt{1 - v^2/c^2}}
}~.
\end{equation}

如果把 $(x,t)$ 作为直角坐标系,尝试用斜坐标系来表示 $x'$ 和 $t'$,那么我们需要确定四个关键参数:斜坐标轴的斜率 $T_x$ 和 $T_t$,以及其拉伸比例 $r_x$ 和 $r_t$。

\begin{exercise}{}

如果用直角坐标表示铁轨系中的事件坐标,那么这些事件在火车系中的坐标可以一个斜坐标系来表示。将一维空间洛伦兹变换的表达式和斜坐标系的坐标转换进行比较,推导出这个斜坐标系的坐标轴斜率和拉伸比例。

\end{exercise}

答案如图所示。

\begin{figure}[ht]
\centering
\includegraphics[width=5cm]{./figures/ea54150cf54620cb.pdf}
\caption{洛伦兹变换的斜坐标表达,其中 $\tan{\theta}=v/c$,两坐标轴的拉伸比例均为 $\sqrt{(1+v^2/c^2)}/\sqrt{(1-v^2/c^2)}$。} \label{fig_SROb_1}
\end{figure}




