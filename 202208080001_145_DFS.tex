% 深度优先搜索(DFS)
% keys DFS|算法|C++|搜索

深度优先搜索(DFS,Depth First Search)简称深搜或者爆搜,DFS 的搜索顺序是按照深度优先搜索,简单来说就是“一条路走到黑”,搜索是把所有方案都试一遍,再判断是不是一个可行解.搜索与“递归”和“栈”有很大的联系,递归是实现搜索的一种方式,而栈则是计算机实现递归的方式.每个搜索过程都对应着一棵\textbf{递归搜索树},递归搜索树可以让我们更加容易的理解 DFS.
整个搜索过程就是基于该搜索树完成的,为了不重复遍历每个结点,会对每个节点进行标记,也可以对树中不可能是答案的分支进行删除,从而更高效的找到答案,这种方法被称为\textbf{剪枝}.如果搜索树在某个子树中搜索到了叶结点,想继续搜索只能返回上个或多个状态,返回的过程称为\textbf{回溯},回溯要记得\textbf{恢复状态},才能保证接下来的搜索过程可以正常进行.


\subsection{普通搜索}
来看一道\href{https://www.luogu.com.cn/problem/P1706}{具体例题}学习 DFS

题意:输出 $n$ 的全排列

思路:以 $n$ 为 $3$ 举例,枚举每个位置上该填什么数,但是每一位上的数不能和其他位置上的数一样,填满了 $3$ 位就输出,然后回溯继续搜索.

\begin{figure}[ht]
\centering
\includegraphics[width=14.25cm]{./figures/DFS_1.png}
\caption{递归搜索树} \label{DFS_fig1}
\end{figure}

上图则是一棵递归搜索树,就是搜索的过程形象化的显示出来.从第一个数字开始填,在填第二个数字,填过的数字记得要标记一下“使用过了”,不然会导致三个数字有两个数字是重复的.上文提到的回溯,就是填完了三个数字,先输出答案,无法在继续搜索下去就要回溯,回溯记得要恢复状态,不然接下来无法填数.

\textbf{算法思路:}
\begin{enumerate}
\item 用 \verb|path| 数组保存排列.

\item 用 \verb|st| 数组保存每个数的状态,\verb|st[i]| 为 $\mathtt{true}$ 则表示数字 $i$ 被用过,反之没被用过.

\end{enumerate}
\begin{lstlisting}[language=cpp]
int path[10], st[10], n;

void dfs(int u)
{
    if (u == n) // u == n 则说明填满了三个数
    {
        for (int i = 0; i < n; i ++ )
            cout << path[i] << ' ';     // 输出答案
        cout << endl;
        return;     // 返回,进行回溯操作
    }
    
    for (int i = 1; i <= n; i ++ )
        if (!st[i])
        {
            st[i] = true;   // 标记数字 i 被使用过
            path[u] = i;    // 第 u 个位置上的数是 i
            
            dfs(u + 1);     // 搜下一个位置
            
// 如果开始执行这段代码了,说明已经填满了 3 个数,正在进行回溯操作,则需要恢复状态
            st[i] = false;  
// 第 u 个位置变成空了,但这句话其实没必要写,在回溯完毕准备填数的时候则会被覆盖
            path[u] = 0;    
        }
        
    return;     // 回溯
}
\end{lstlisting}

DFS 的思想和代码很容易理解,这里不再赘述,但初学者学不懂 DFS 的原因主要是不理解递归而理解不了 DFS.这里来详细的讲解一下递归的执行过程.

\begin{figure}[ht]
\centering
\includegraphics[width=14.25cm]{./figures/DFS_3.png}
\caption{递归调用的过程} \label{DFS_fig3}
\end{figure}


进入入口则是主函数调用,然后在执行一段代码然后递归调用(对应上面的代码就是 \verb|dfs(u+1)|),如果某一次递归的过程的中触发了判断条件,则返回(回溯),回溯到上次执行我这次的递归的代码的下面再继续执行代码,如图中的第一根红线,对应上面的代码是先输出了 $1 2 3$,然后是第一次触发判断条件,这是 $u$ 为 $3$,就返回到 $u$ 为 $2$ 的这次递归上,然后不执行 \verb|dfs(u+1)|,直接执行下面的代码.然后没代码可以执行了,就只能执行第 $27$ 行的代码继续返回(回溯),然后再继续做接下来的操作.


\subsection{搜索的剪枝}