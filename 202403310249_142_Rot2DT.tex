% 平面旋转变换
% keys 线性代数|平面旋转变换|三角恒等式|坐标系|极坐标系
% license Xiao
% type Tutor

\begin{issues}
\issueDraft
\end{issues}
% Giacomo:重新梳理预备知识
% 这篇文章要作为几何向量的线性变换的例子使用

\pentry{三角恒等式\nref{nod_TriEqv}, 极坐标系\nref{nod_Polar},}{nod_7933}


\begin{figure}[ht]
\centering
\includegraphics[width=5cm]{./figures/8bde4ff50cf3e0e4.pdf}
\caption{把矢量绕原点旋转 $\alpha$ 角} \label{fig_Rot2DT_1}
\end{figure}

已知直角坐标系中一点 $P(x,y)$, $P$ 绕原点逆时针旋转 $\alpha $ 角($\alpha  \in R$) 之后变为 $P'(x',y')$ 则有
\begin{align}\label{eq_Rot2DT_1}
x' &= (\cos \alpha)x + (- \sin \alpha)y ~,\\
\label{eq_Rot2DT_2}
y' &= (\sin \alpha)x + (\cos \alpha)y~.
\end{align}
其逆变换如下,即已知 $P'(x',y')$ 求 $P(x,y)$ 
\begin{align}\label{eq_Rot2DT_3}
x &= ( \cos \alpha  )x' + ( \sin \alpha  )y' ~,\\
\label{eq_Rot2DT_4}
y &= ( - \sin \alpha)x' + ( \cos \alpha )y'~,
\end{align}
这相当于把 $(x', y')$ 顺时针旋转 $\alpha$ 得到 $(x, y)$。

\subsubsection{绕任点旋转}
要绕任意点 $(x_0, y_0)$ 旋转, 只需要先把矢量末端平移 $(-x_0, -y_0)$, 绕原点旋转后再平移 $(x_0, y_0)$ 即可
\begin{equation}\label{eq_Rot2DT_5}
\begin{aligned}
x' &= ( \cos \alpha  )(x-x_0) + ( \sin \alpha  )(y-y_0) + x_0 ~,\\
y' &= ( - \sin \alpha)(x-x_0) + ( \cos \alpha )(y-y_0) + y_0~.
\end{aligned}
\end{equation}

\begin{example}{旋转双曲线}
我们来证明函数 $y' = 1/x'$ 的曲线为双曲线\upref{Hypb3}。 由于双曲线标准方程表示的双曲线是关于 $x$ 轴对称的, 我们需要把 $(x', y')$ 顺时针旋转 $\pi/4$ 得到 $(x, y)$, 即上面的 $\alpha = \pi/4$。 把\autoref{eq_Rot2DT_1} 和\autoref{eq_Rot2DT_2} 代入 $y' = 1/x'$ 得
\begin{equation}
\frac{x^2}{2} - \frac{y^2}{2} = 1~.
\end{equation}
这符合双曲线的标准方程, 所以 $y' = 1/x'$ 是一个双曲线。
\end{example}

\subsection{推导}

平面上一点 $P(x,y)$ 也可以用极坐标 $(r, \theta)$ 表示,一般情况下令极点与原点重合,极径与 $x$ 轴重合,则有
\begin{equation}
x = r\cos \theta~, \qquad y = r\sin \theta ~.
\end{equation}     
把点 $P$ 绕原点逆时针旋转 $\alpha $ 角变为 $P'$, 则 $P'$ 极坐标为 $(r, \theta  + \alpha)$。 根据上式计算为 $P'$ 的直角坐标 $(x', y')$ 并用两角和公式(\autoref{eq_TriEqv_1}~\upref{TriEqv})化简如下
\begin{align}
x' &= r\cos(\theta  + \alpha) = r\cos\theta \cos\alpha  - r\sin\theta \sin\alpha  = x\cos\alpha  - y\sin\alpha ~,\\
y' &= r\sin(\theta  + \alpha) = r\sin\theta \cos\alpha  + r\cos\theta \sin\alpha  = x\sin\alpha  + y\cos\alpha ~.
\end{align} 
这就证明了\autoref{eq_Rot2DT_1} 和\autoref{eq_Rot2DT_2} 两式。

若要证\autoref{eq_Rot2DT_3} 和\autoref{eq_Rot2DT_4} 有两种方法。一是将\autoref{eq_Rot2DT_1} 和\autoref{eq_Rot2DT_2} 式中的 $x, y$ 看成未知数,解二元一次方程组。另一种方法的思路是,既然 $P$ 逆时针旋转 $\alpha $ 角为 $P'$, 那么把 $P'$ 顺时针旋转 $\alpha$ 角可得到 $P$。 而“顺时针旋转 $\alpha$ 角”就是“逆时针旋转 $-\alpha $ 角”。把变换\autoref{eq_Rot2DT_1} 和\autoref{eq_Rot2DT_2} 中的 $\alpha$ 换为 $-\alpha$ 再化简得
 \begin{align}
x &= \cos(-\alpha) x' - \sin(-\alpha) y' = \cos\alpha x' + \sin\alpha y'~,\\
y &= \sin(-\alpha) x' + \cos(-\alpha) y' =  -\sin\alpha x' + \cos\alpha y'~.
\end{align}
证毕。
