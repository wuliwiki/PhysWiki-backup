% 电离辐射
% license CCBYSA3
% type Wiki

(本文根据 CC-BY-SA 协议转载自原搜狗科学百科对英文维基百科的翻译)

\textbf{致电离辐射(电离辐射)}是指能量足够高而能使原子或分子中的电子解离、也就是使他们电离的辐射。电离辐射通常包括高能亚原子粒子和离子、高速运动的原子(通常大于光速的1\%),以及高能电磁波。

$Y$射线、$X$射线,以及紫外线中的高能部分属于电离辐射,紫外线中低能部分以及所有紫外线以下的所有频谱,包括可见光(包括几乎所有类型的激光)、红外线、微波、无线电波则属于非电离辐射。因为不同的分子和原子在具有不同的电离能,紫外线中电离辐射和非电离辐射之间没有明确的边界。一般习惯将边界置于10eV 和33eV之间。

典型的致电离亚原子粒子来自放射性衰变,包括α粒子、β粒子和中子。几乎所有放射性衰变产物都是致电离的,因为放射性衰变的能量通常远远高于电离所需的能量。其他自然产生的亚原子电离粒子有μ子、介子、正电子以及宇宙射线与地球大气层相互作用后产生的产生的次级宇宙射线中的其他粒子。[1][2]宇宙射线是由恒星和某些天体事件产生的,例如超新星爆炸。宇宙射线也可能在地球上产生放射性同位素(例如碳14),其会发生衰变并产生电离辐射。宇宙射线和的衰变放射性的同位素的衰变是地球上自然电离辐射的主要来源,被称为背景辐射。电离辐射也可以通过以下方式人工产生:$X$射线管、粒子加速器以及任何产生放射性同位素人工行为。

电离辐射不能直接被人类感觉到,因此必须使用辐射检测仪器(如盖革计数器)来指示并测量它们。一些高强度的电离辐射可以与物质相互作用发出可见光,如契伦科夫辐射和辐射发光现象。电离辐射可用于各种领域,如医学、核电、研究、制造、建筑和许多其他领域,但如果不采取适当的屏蔽措施,则会对健康造成危害。暴露于电离辐射会对活体组织造成损伤,并可能导致辐射灼伤、细胞损伤、放射病、癌症和死亡。

\subsection{类型}
\begin{figure}[ht]
\centering
\includegraphics[width=6cm]{./figures/42d0cf06dd16f8f8.png}
\caption{阿尔法(α)辐射由快速移动的氦-4(4He)细胞核并被一张纸挡住。Beta(β)由电子组成的辐射被铝板阻挡。伽马(γ)由高能光子组成的辐射在穿透致密材料时最终被吸收。中子(n)辐射由被轻元素阻挡的自由中子组成,如氢,它减缓和/或俘获它们。未示出:银河宇宙射线,由高能带电核组成,例如质子、氦核和被称为 HZE 离子的高电荷核。} \label{fig_DLFS_1}
\end{figure}
\begin{figure}[ht]
\centering
\includegraphics[width=6cm]{./figures/9e60e1ff6d93f21f.png}
\caption{云室是电离辐射可视化的几种方法之一。在粒子物理学的早期,它们主要用于研究,但今天仍然是一种重要的教学工具。} \label{fig_DLFS_2}
\end{figure}
电离辐射根据产生电离效应的粒子或电磁波的性质进行分类。它们有不同的电离机制,可以分为直接电离和间接电离。

\subsubsection{1.1 直接电离辐射}
任何带电的有质量粒子只要具有足够的动能都可以通过库仑力作用直接电离原子,这些粒子包括电子、$\mu$介子、带电的$\pi$介子、质子和失去电子的高能带电原子核。当在相对论性速度下移动时,这些粒子有足够的动能产生电离,但相对论性速度不是必需的。例如,典型的α粒子是致电离的,其速度约5\% 光速,而33 eV(足以产生电离)的电子速度约为1\% 光速。

要最先发现的两种致电离粒子源被赋予了的特殊名称:从原子核中发射的氦核被称为$\alpha$粒子,而通常(但不总是)在相对论性速度发射的电子被称为$\beta$粒子。

天然宇宙射线主要由相对论性质子组成,也包括较重的原子核,如氦离子和其他高能重离子。在大气中,这种粒子通常被空气分子阻止,产生短寿期的带电$\pi$介子,它们很快衰变为$\mu$子,$\mu$子是到达地面(并在一定程度上穿透地面)的一种主要宇宙射线辐射。$\pi$介子也可以在粒子加速器中大量产生。

\textbf{$\alpha$粒子}

α粒子由两个质子和两个中子组成,与氦核相同。α粒子辐射通常在α衰变过程中产生,也可能以其他方式产生。α粒子是以希腊字母表中的第一个字母命名。α粒子的符号是α或α2+。因为它们与氦核相同,它们有时也被写作$He^{2+}$或者$^{4}$
$2^{He^2+}$表示氦离子带有+2电荷(缺少两个电子)。如果离子从环境中获得电子,$\alpha$粒子可以写成正常的(电中性的)氦原子4
2He。

α粒子是一种电离能力极强的粒子辐射。当它们由放射性α衰变产生时,它们的穿透深度很浅,只能穿透几厘米的空气,无法穿透皮肤死层。三元裂变产生的α粒子能量是其三倍,并且在空气中穿透得更远。氦核构成宇宙射线的10-12\%,通常也比核衰变过程产生α粒子的能量高得多,因此当能够穿过人体和密度较高的屏蔽层。然而,这种类型的辐射能够被地球大气层大大减弱,其相当于一个大约10米深水构成的防辐射罩。[3]