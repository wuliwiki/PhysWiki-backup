% 连续映射和同胚
% 同胚|映射|连续映射|同构|连续性|嵌入|进制性

\pentry{映射\upref{map},拓扑空间\upref{Topol},函数的连续性\upref{contin}}

给定两个拓扑空间 $(X, \mathcal{T}_X)$ 和 $(Y, \mathcal{T}_Y)$,定义一个映射 $f:X\rightarrow Y$。在集合论意义上,这个映射 $f$ 可以有任何可能的形式,但是在拓扑学意义上,我们只研究满足特定条件的一类映射,称为连续映射。那么什么是连续映射呢?

\subsection{连续映射}
\begin{definition}{连续映射\footnote{这个定义是类比实函数中逐点连续的 $\epsilon-\delta$ 表达。}}
给定两拓扑空间 $(X, \mathcal{T}_X)$ 和 $(Y, \mathcal{T}_Y)$。映射 $f:X\rightarrow Y$ 在某一点 $x_o\in X$ 上连续,当且仅当对于任意 $Y$ 中的开集 $U\ni f(x_0)$,存在 $X$ 中的开集 $V\ni x_0$,使得 $f(V)\subseteq U$。

如果 $f$ 在 $X$ 中的任意一点上都连续,那么称 $f$ 是一个\textbf{连续映射(continuous mapping)}。
\end{definition}

\begin{exercise}{连续映射的等价定义}\label{exe_Topo1_1}
给定两拓扑空间 $X$,$Y$,那么 $f:X\rightarrow Y$ 是连续映射,当且仅当对于任意的 $U\in\mathcal{T}_Y$,有 $f^{-1}(U)\in\mathcal{T}_X$。即开集的原像还是开集。

证明这一点。
\end{exercise}

\autoref{exe_Topo1_1} 中的等价定义,可以简单记为“开集的原像还是开集”,类似地,容易证明它等价于“闭集的原像还是闭集”。这个定义和微积分中实函数的连续性\autoref{the_contin_1} 本质上是一样的。在微积分和实变函数中,这个等价定义的用途似乎没那么多,但是在点集拓扑中要更为常用。接下来讨论的同胚概念就是一个例子。

\subsection{同胚}

在代数学中我们提到了同构和同态的概念\upref{Group2}。在任何理论框架中,同构的两个对象都被看成是同一个对象,这就要求同构的两个对象在指定理论框架中有完全一样的行为。拓扑学的框架和代数学不一样,因此同构的定义也不一样。由于历史原因,拓扑学意义上的同构,被称为同胚。

\begin{definition}{同胚}
对于拓扑空间 $X$ 和 $Y$,如果存在一个双射 $f:X\rightarrow Y$,使得对于任意的开集 $A\in\mathcal{T}_X$ 和 $B\in\mathcal{T}_Y$,都有 $f(A)\in\mathcal{T}_Y$ 和 $f^{-1}(B)\in\mathcal{T}_X$,那么我们就说 $X$ 和 $Y$ 是\textbf{同胚(homeomorphic)}的;这个双射 $f$ 被称为一个\textbf{同胚映射(homeomorphic mapping)}或者简称为\textbf{同胚(homeomorphism)},记为 $X\approx Y$。
\end{definition}

这个定义其实就是直接要求存在一个双射,使得两个空间的开集一一对应。这个定义很直观,因为拓扑学就是定义开集的理论,开集之间一一对应的话,在拓扑意义下就有完全一样的行为了;但实践中一般不可能一个一个地比较两个空间的开集,看它们是不是都互相配对了,所以我们就引入了以下等价的定义:

\begin{theorem}{同胚的等价定义}\label{the_Topo1_1}
对于拓扑空间 $X$ 和 $Y$,如果存在一个双射 $f:X\rightarrow Y$,使得 $f$ 和 $f^{-1}$ 都是连续映射,那么我们有 $X\cong Y$。
\end{theorem}

证明是很简单的,直接应用\autoref{exe_Topo1_1} 的定义就可以。这个定义也可以简单记为:“同胚映射就是\textbf{双向连续双射}”,即同胚映射要求是个双射,并且从两个方向来看都是连续映射。

下面这个例子就是一个单向连续双射,不成为同胚:

\begin{example}{}

考虑单位圆$S^1$,表示为复平面上的集合$S^1=\{\E^{\I x}\mid x\in[0, 1)\}$。考虑映射$f:[0, 1)\to S^1$,定义为$f(x)=\E^{\I x}$,则这是一个双射,但只有$f:S^1\to[0, 1)$连续,反过来的$f^{-1}:[0, 1)\to S^1$不连续。

\end{example}

接下来是若干同胚的例子和反例。

\begin{example}{}\label{ex_Topo1_1}
考虑一维实数空间 $\mathbb{R}$,令所有开区间的集合为拓扑基,定义其上的拓扑。取一个开区间 $I=(0, 1)$,将 $I$ 视为 $\mathbb{R}$ 的子空间。那么这两个空间是同胚的。同胚映射取 $f:I\rightarrow\mathbb{R}$,其中 $\forall x\in I, f(x)=\tan{(\pi\cdot(x-\frac{1}{2}))}$ 即可。
\end{example}

\begin{example}{}\label{ex_Topo1_2}
类似\autoref{ex_Topo1_1},$\mathbb{R}^2\cong I\times I$。其中 $I\times I$ 是一个正方形区域,去除了边界。
\end{example}

\begin{example}{反例}\label{ex_Topo1_3}
通常的度量空间 $\mathbb{R}^2$ 和一个球表面 $S^2$ 不同胚。在 $\mathbb{R}^2$ 平面上挖去一个点得到的空间也和 $\mathbb{R}^2$ 不同胚。
\end{example}

我们还可能经常接触到和同胚类似的一个概念,称为“嵌入”。把拓扑空间 $X$ 嵌入到空间 $Y$,就是说让 $X$ 和 $Y$ 的一个子空间同构,由于同构的空间都被看成是同一个,这就相当于 $X$ 成为了 $Y$ 的一部分。

\begin{definition}{拓扑嵌入}
设拓扑空间 $X$ 和 $Y$。若存在连续映射 $f:X\rightarrow Y$,使得 $X\cong f(X)$,那么称 $f$ 是一个\textbf{拓扑嵌入(映射)(topological embedding)},称 $X$ 由 $f$\textbf{嵌入(embed)}到 $Y$ 中。在明确讨论范围为点集拓扑时,也可简称\textbf{嵌入(embedding)}。
\end{definition}

考虑到同胚就是“双向连续双射”,如果单连续映射的逆映射还是连续的,那它就是嵌入映射。注意,我们这里讨论的范围是较抽象的“点集拓扑”,在更具体的“微分拓扑”或“微分几何”中,(微分)嵌入的概念要求自然会更严格一些。




\subsection{同胚不变性}

数学中至关重要的一种研究思想,就是关注\textbf{不变性(invariance)}。我们所研究的对象哪些地方是恒定不变的,标志了这个对象的本质。

比如说,一个 $8\times8$ 的国际象棋棋盘,可以用 $1\times2$ 的长方形木条铺满;如果把棋盘的左上角和右下角的格子去掉,还能不能铺满呢?答案是否定的,因为如果你考虑棋盘黑白格子的分布,那么每个木条肯定要覆盖一个黑色格子和一个白色格子;去掉的两个格子肯定是同色的,这样一来两种颜色的格子数量就不一样了,肯定不会被铺满。在这个例子中,被覆盖的黑白格子数量无论如何都会是相同的,利用这个不变性就可以很轻松地解答问题。

在物理学中也常常见到不变性的身影。牛顿力学中有伽利略不变性\upref{GaliTr},在任何惯性系下加速度都是一样的;但是由另一个不变性:光速不变原理,却可以导出不同于牛顿力学的狭义相对论。

在点集拓扑中,一个拓扑空间可以有各种各样的性质,比如单位区间 $I=(0,1)$ 具有长度 $1$,而 $\mathbb{R}$ 具有长度 $\infty$。但我们从\autoref{ex_Topo1_1} 中知道,这两个拓扑空间应该是同胚的,所以“长度”这一属性不是它们的本质属性。为了得到拓扑空间的本质属性,我们可以探索存在哪些\textbf{同胚不变性},即对于任何同胚的空间来说都一样的性质。

最重要的四个同胚不变性,分别是:\textbf{紧致性}\upref{Topo2},\textbf{连通性}\upref{Topo3},\textbf{道路连通性}\upref{Topo4}以及\textbf{分离性}\upref{Topo5}。对每一个性质,小时百科都单独开辟了一个词条,请点击性质名称后面的链接查看。
