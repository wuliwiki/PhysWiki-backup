% 索菲斯·李(综述)
% license CCBYSA3
% type Wiki

本文根据 CC-BY-SA 协议转载翻译自维基百科 \href{https://en.wikipedia.org/wiki/Sophus_Lie}{相关文章}。

\begin{figure}[ht]
\centering
\includegraphics[width=6cm]{./figures/a7728e1c8b169d85.png}
\caption{} \label{fig_SFSL_1}
\end{figure}
马留斯·索福斯·李(Marius Sophus Lie,/liː/,挪威语:[liː];1842年12月17日-1899年2月18日)是一位挪威数学家。他在连续对称性理论方面做出了奠基性的贡献,并将其应用于几何和微分方程的研究中。他还在代数学的发展中做出了重要贡献。
\subsection{生平与职业生涯}
马留斯·索福斯·李于1842年12月17日出生在挪威小镇诺尔德菲尤尔。他是路德教牧师约翰·赫尔曼·李与其妻子所生的六个孩子中最小的一个,母亲出身于特隆赫姆的一个知名家族。[1]

他在挪威东南部的莫斯接受了初等教育,之后在奥斯陆(当时称为克里斯蒂安尼亚)读高中。高中毕业后,他原本希望投身军事事业,但由于视力不佳而被军队拒绝,于是改而进入皇家弗雷德里克大学(即今日奥斯陆大学)就读。

索福斯·李的第一篇数学论文《平面几何中虚数的表示》于1869年由克里斯蒂安尼亚科学院与《克雷勒期刊》共同发表。同年,他获得了一项奖学金前往柏林,自9月起在当地停留至1870年2月。在那里,他结识了费利克斯·克莱因,两人迅速成为挚友。离开柏林后,李前往巴黎,克莱因也于两个月后与他会合。在巴黎,他们结识了卡米耶·若尔当与加斯顿·达布。然而,1870年7月19日,普法战争爆发,克莱因因身为普鲁士人而不得不迅速离开法国。李则前往枫丹白露,却被误认为是德国间谍而遭到逮捕,这在挪威为他带来了一定的名声。在达布的干预下,李被关押一个月后获释。[2]

李在1871年于皇家弗雷德里克大学(今奥斯陆大学)获得博士学位,其博士论文题为《一类几何变换》(挪威语:Over en Classe geometriske Transformationer,英文:On a Class of Geometric Transformations)。[3] 这篇论文后来被达布称为“现代几何中最优美的发现之一”。次年,挪威议会为他特别设立了一个教授职位。同年,李拜访了正在埃尔朗根任教的克莱因,后者当时正在发展著名的“埃尔朗根纲领”。

1872年,李与彼得·路德维希·迈德尔·西洛一起合作,花了八个月时间编辑并出版了挪威同胞尼尔斯·亨里克·阿贝尔的数学著作。

1872年底,索福斯·李向当时年仅18岁的安娜·比奇求婚,并于1874年结婚。这对夫妇育有三个孩子:玛丽(Marie,生于1877年)、达格妮(Dagny,生于1880年)和赫尔曼(Herman,生于1884年)。

从1876年起,他与医生雅各布·沃姆-穆勒以及生物学家乔治·奥西安·萨尔斯共同编辑《数学与自然科学档案》期刊。
