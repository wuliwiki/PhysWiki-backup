% 欧几里得矢量空间的正交化、同构及正交群
% 欧氏矢量空间|正交化|同构|正交矩阵|正交群

\pentry{欧几里得矢量空间\upref{EuVS}}
对欧几里得矢量空间 $V$(\autoref{EuVS_def1}~\upref{EuVS}),其上任一基底都可正交标准化,而空间 $V$ 上的对称双线性型 $(*|*)$ 实际上给出了 $V$ 中的度量性质,即长度(\autoref{EuVS_def2}~\upref{EuVS})和夹角(\autoref{EuVS_def3}~\upref{EuVS}),这意味着就度量性质来说,欧几里得矢量空间 $V$ 和 $\mathbb{R}^n$ 没有差别.由于标准正交基底的重要作用,往往要研究不同标准正交基底之间的转换关系,即两基底对应的过渡矩阵(或转换矩阵),与两标准正交基底对应的过渡矩阵称为正交矩阵,它们将构成一个群,称之正交群.
\subsection{正交化过程}
\begin{theorem}{标准正交化过程}\label{EVOIOG_the1}
设 $\bvec e_1,\cdots,\bvec e_m$ 是 $n$ 维欧几里得矢量空间 $V$ 中的一组 $m$ 个线性无关的矢量.那么,存在一个标准正交矢量组 $\bvec e'_1,\cdots,\bvec e'_m$ ,使得线性包络(或称张成空间\autoref{VecSpn_def1}~\upref{VecSpn})
\begin{equation}
L_i=\langle\bvec e_1,\cdots,\bvec e_i\rangle
\end{equation}
和
\begin{equation}
L'_i=\langle\bvec e'_1,\cdots,\bvec e'_i\rangle
\end{equation}
当 $i=1,\cdots,m$ 时都重合,$m\leq n$.
\end{theorem}
\textbf{证明:}令 $\bvec e'_1=\lambda\bvec e_1,\;\lambda=\norm{\bvec e_1}^{-1}$,则 $L_1=\langle\bvec e_1\rangle=\langle \bvec e'_1\rangle=L'_1$ 显然成立,即 $m=1$ 的情形.

设当 $1\leq k<m$ 时,已构造出所需的矢量组 $\bvec e'_1,\cdots,\bvec e'_k$,使得 $L_1=L'_i,\;i=1,\cdots,k$.我们来找出 $\bvec e'_{k+1}$.

首先,矢量 $\bvec e_{k+1}$ 不包含在 $L_k=L'_k$ 中(否则,$\bvec e_{k+1}$ 可用 $\bvec e_1,\cdots,\bvec e_k$ 线性表示).令
\begin{equation}
\bvec v=\bvec e_{k+1}-\sum_{i=1}^{k}\lambda_i\bvec e'_i
\end{equation}
而 $\lambda_1,\cdots,\lambda_k$ 是任意纯量(显然 $\bvec v\neq\bvec 0$).于是
\begin{equation}
L_{k+1}=\langle\bvec e_1,\cdots,\bvec e_k,\bvec e_{k+1}\rangle=\langle\bvec e_1,\cdots,\bvec e_k,\bvec v\rangle
\end{equation}
如果 $\bvec v\perp L'_k$,那么便找到了 $\bvec e'_{k+1}=\frac{\bvec v}{\norm{\bvec v}}$ .而做到这一点的充要条件是,对任意 $j=1,\cdots,k$,成立
\begin{equation}
\begin{aligned}
&0=\qty(\bvec v|\bvec e'_j)=\qty(\bvec e_{k+1}|\bvec e'_j)-\qty(\sum_{i=1}^k\lambda_i\bvec e'_i|\bvec e'_j)\\
&=\qty(\bvec e_{k+1}|\bvec e'_j)-\sum_{i=1}^k\lambda_i(\bvec e'_i|\bvec e'_j)=\qty(\bvec e_{k+1}|\bvec e'_j)-\lambda_j
\end{aligned}
\end{equation}
故只需取 $\lambda_j=\qty(\bvec e_{k+1}|\bvec e'_j)$ 即可.于是,便得到了标准正交组 $\bvec e'_1,\cdots,\bvec e'_k$,且 $L_{k+1}=L'_{k+1}$.

由数学归纳法,\textbf{定理得证}!

\begin{theorem}{}
设 $L$ 是有限维欧几里得矢量空间 $V$ 的一个子空间,$L^{\perp}$ 是它的正交补,那么
\begin{equation}
V=L\oplus L^{\perp},\quad L^{\perp\perp}=L.
\end{equation}
\end{theorem}
\textbf{证明:}在 $L$ 中取任一标准正交基 $(\bvec e_1,\cdots,\bvec e_m)$,如\autoref{EVOIOG_the1} 中找出 $\bvec v$ 的过程一样,可知对 $\forall \bvec w\in V$,矢量 $\bvec v$:
\begin{equation}
\bvec v=\bvec w-\sum_{i=1}^m(\bvec w|\bvec e_i)\bvec e_i
\end{equation}
正交于 $L$,即 $\bvec w=\bvec u+\bvec v$,其中 $\bvec u=\sum_{i=1}^m(\bvec w|\bvec e_i)\bvec e_i\in L$ ,而 $\bvec v\in L^{\perp}$,亦即 $V=L+L^{\perp}$.

要证 $V=L\oplus L^{\perp}$ ,只需证 $L\cap L^{\perp}=\bvec 0$.设 $\bvec x\in L\cap L^{\perp}$,则 $(\bvec x|L)=0$,又 $\bvec x\in L$,于是 $(\bvec x|\bvec x)=0$,由纯量积的正定性,$\bvec x=\bvec 0$,从而 $V=L\oplus L^{\perp}$ .

任意 $\bvec w\in L^{\perp\perp}$,由 $V=L\oplus L^{\perp}$ ,有 $\bvec w=\bvec u+\bvec v(\bvec u\in L,\bvec v\in L^{\perp})$,于是 $(\bvec w|\bvec v)=\norm{\bvec v}^2=0$,故 $\bvec w=\bvec u\in L$,于是 $L^{\perp\perp}\subset L$ .其次,由 $L^{\perp\perp}=(L^{\perp})^{\perp}$ ,而 $(L|L^{\perp})=0$,从而 $L\subset L^{\perp\perp}$.于是 $L^{\perp\perp}=L$.

\textbf{证毕!}
\subsection{欧几里得矢量空间的同构}
\begin{theorem}{}
任意两个维数相同的欧几里得矢量空间 $V,V'$ 都是同构的.即存在矢量空间的同构映射 $f:V\rightarrow V'$,它还保持纯量乘积,即
\begin{equation}
(\bvec x|\bvec y)=(f(\bvec x)|f(\bvec y))'
\end{equation}
其中,$(*|*)'$ 是 $V'$ 上的纯量乘积.
\end{theorem}
\textbf{证明:} 设 $(\bvec e_1,\cdots,\bvec e_n)$ 和 $(\bvec e'_1,\cdots,\bvec e'_n)$ 分别是 $V$ 和 $V'$ 的一个标准正交基底.则映射:
\begin{equation}
f:\bvec x=\sum_{i=1}^{n}x_i\bvec e_i\mapsto\bvec x'=\sum_{i=1}^{n}x_i\bvec e'_i
\end{equation}
显然是个线性的双射\footnote{若线性的双射 $f$ 将矢量空间 $V$ 映到矢量空间 $W$,则 $f$ 就称为 $V$ 到 $W$ 的同构映射,而称 $V$ 和 $W$ 同构.}.并且由于选取的基底皆为标准正交基,所以纯量乘积 $(\bvec x|\bvec y)$ 和 $(\bvec x'|\bvec y')'$ 都按同一公式进行.

\textbf{证毕!}

若固定矢量 $\bvec v$,则映射
\begin{equation}
\varphi_\bvec{v}=(\bvec v|*):V\rightarrow\mathbb{R}
\end{equation}
是 $V$ 上的线性映射,即 $(\bvec v|*)\in V^*$\footnote{$V^*$即 $V$ 的对偶空间.}.
\begin{theorem}{}
映射 $\varphi:\bvec v\mapsto (\bvec v|*)=\varphi_\bvec v$ 是矢量空间 $V$ 到 $V^*$ 的自然同构\footnote{即该同构不依赖于基的选择}.在此同构下,$V$ 的基底 $\bvec e_1,\cdots,\bvec e_n$ 被映射到 $V^*$ 中与其对偶的基底 $e^1,\cdots,e^n$,且 $\varphi_\bvec{e_i}=e^i$.
\end{theorem}
\textbf{证明:}$\varphi$ 是线性的:
\begin{equation}
\varphi_{(\alpha\bvec u+\beta\bvec v)}=(\alpha\bvec u+\beta\bvec v|*)=\alpha(\bvec u|*)+\beta(\bvec v|*)=\alpha\varphi_\bvec{u}+\beta\varphi_\bvec{v}.
\end{equation}
因为 
\begin{equation}
\bvec v\in \mathrm{Ker}\,\varphi\Rightarrow(\bvec v|\bvec x)=0,\forall\bvec x\in V
\end{equation}
所以 $(\bvec v|\bvec v)=0\Rightarrow\bvec v=\bvec 0$ ,故$\mathrm{Ker}\,\varphi=\bvec 0$ ,于是 $\varphi$ 为单射.

$V^*$ 上任意元素必可由 $V^*$ 的基底线性表示,故若 $V^*$ 的任一基底都有 $V$ 的元与之对应,则 $\varphi$ 为满射.特别地
\begin{equation}
\varphi_{\bvec e_i}=(\bvec e_i|*)=\sum_{j=1}^{n} a_{ij}e^j
\end{equation}
因为  $\bvec e_1,\cdots,\bvec e_n$ 是标准正交基底,所以
\begin{equation}
a_{ij}=\sum_{k=1}^n a_{ik}e^k(\bvec e_j)=(\bvec e_i|\bvec e_j)=\delta_{ij}
\end{equation}
于是有 $\varphi_{\bvec e_i}=e^i$.这就证明了满射性.进而,$\varphi$ 是双射.显然,定义 $\varphi$ 的过程并没有选择特定的基底,即 $\varphi$ 是矢量空间 $V$ 到 $V^*$自然同构.

\textbf{证毕!}

由于同构的双方可认为是同一事物的不同表现形式,这意味着,欧几里得矢量空间中每一矢量 $\bvec v$ 都可看成是一个线性函数 $\bvec v:V\rightarrow\mathbb{R}$.(即把 $\bvec v$ 看成 $V$ 的线性函数时,$\bvec v$ 相当于 $\varphi_{\bvec v}=(\bvec v|*)$).
\subsection{正交群}
设 $(\bvec e_1,\cdots,\bvec e_n)$ 和 $(\bvec e'_1,\cdots,\bvec e'_n)$ 是矢量空间 $V$ 的不同标准正交基底.设 $\bvec e'_i$ 到 $\bvec e_i$ 的转换矩阵为 $A=(a_{ij})$,即
\begin{equation}
\bvec e'_j=\sum_{i=1}a_{ij}\bvec e_i
\end{equation}

