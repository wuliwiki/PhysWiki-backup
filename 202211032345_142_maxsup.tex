% 最值和确界
% keys max|maximal|sup|supremum|min|minimal|inf|infimum

\pentry{序关系\upref{OrdRel}}

\subsection{最值}

\addTODO{插入极大值和最大值的引用}

对于一个全序集来说,极大值(TODO:引用)和最大值(TODO:引用)没有区别,但不是所有的全序集都有最大值,比如自然数集 $\mathbb{N}$,比如 $(0, 1)$ (即 $0, 1$ 之间的全体实数的集合).

如果把 $\mathbb{N}$ 和 $(0, 1)$ 都视作 $\mathbb{R}$ 的一个子集的话,我们会发现对任何实数,我们都能找到一个比它还要大的自然数,即 $\mathbb{R}$ 中不存在比任何自然数都要大的实数;但是 $(0, 1)$ 不同, $1$ 比任何 $(0, 1)$ 中的元素都要大,而且任意比 $1$ 还大的数都比 $(0, 1)$ 要大.我们把类似这些元素称为 $(0, 1)$ 的上界.

\begin{definition}{上界}
在一个全序集 $(A, \leq)$ 的一个子集 $S$ 中,如果$A$ 中的一个元素 $a$ 满足 $\forall s \in S, s \leq a$,那么就称 $a$ 是 $S$ 的一个\textbf{上界(upper bound)},
\end{definition}


直觉上说, 的“最大值”是 $1$,只不过由于 $1 \not\in (0, 1)$,叫它最大值显得并不合适.