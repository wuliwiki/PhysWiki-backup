% http 协议笔记
% license Xiao
% type Note

\begin{issues}
\issueDraft
\end{issues}

\subsubsection{常识}
\begin{itemize}
\item HTTP 是一个在计算机世界里专门在「两点」之间「传输」文字、图片、音频、视频等「超文本」数据 的「约定和规范」。
\item HTTP 的每个 message (不是 package)都是文本的
\end{itemize}

http 包的结构就是 head + body, 其中 head 除了第一行都是 \verb`key : value` 的形式
\begin{lstlisting}[language=none]
<method> <URI> HTTP/<version>
<Header1>: <value1>
<Header2>: <value2>
...
<HeaderN>: <valueN>
(空行)
<body>
\end{lstlisting}

\begin{itemize}
\item URI 就是 URL, 例如 \verb`/` 代表网站根目录
\end{itemize}

例子:
\begin{lstlisting}[language=none]
GET / HTTP/1.1
Host: www.example.com
User-Agent: Mozilla/5.0 Chrome/58.0.3029.110 Safari/537.36
Accept: text/html,application/xhtml+xml,application/xml;
Accept-Language: en-US,en;q=0.5
Accept-Encoding: gzip, deflate
Connection: keep-alive
Upgrade-Insecure-Requests: 1
\end{lstlisting}
这个 request 没有 body, 因为是简单的 GET。

\subsection{方法}
\begin{itemize}
\item \verb|method| 有 \verb|GET, POST, PUT, DELETE, HEAD, OPTIONS, PATCH, CONNECT, TRACE|
\end{itemize}


\subsection{服务器回应}
服务器回应的消息有不同的结构
\begin{lstlisting}[language=none]
HTTP/<version> <Status> <Reason Phrase>
<Header1>: <value1>
<Header2>: <value2>
...
<HeaderN>: <valueN>
(空行)
<body>
\end{lstlisting}

例子:
\begin{lstlisting}[language=none]
HTTP/1.1 200 OK
Date: Fri, 29 Dec 2023 12:00:00 GMT
Server: Apache/2.4.18 (Ubuntu)
Last-Modified: Fri, 29 Dec 2023 10:00:00 GMT
Content-Type: text/html; charset=UTF-8
Content-Length: 1234
Connection: close
Set-Cookie: UserID=JohnDoe; Max-Age=3600; Version=1

<!DOCTYPE html>
<html>
<head>
    <title>Welcome to Example.com!</title>
</head>
<body>
    <h1>Hello, World!</h1>
    <p>Welcome to our website.</p>
</body>
</html>
\end{lstlisting}

\subsection{响应状态码}
\begin{itemize}
\item 1xx 中间状态,还需要后续操作。实际用到的比较少
\item 2xx 成功,已被正确处理, 如 \verb`200 OK`, \verb`204 No Content`(响应中没有 body), \verb`206 Partial Content`(分块下载或断点续传)
\item 3xx 重定向,资源位置改变,需要重新请求: \verb`301 Moved Permanently`, \verb`302 Found`(临时重定向)。 \verb`Location` 字段用于指明要跳转的 URL。 \verb`304 Not Modified` \verb`重定向已存在的缓冲文件,也称缓存重定 向,用于缓存控制`
\item 4xx 请求有误,无法处理。 \verb`400 Bad Request` 笼统错误。 \verb`403 Forbidden` 禁止访问资源。 \verb`404 Not Found` 资源未找到。
\item 5xx 服务器错误。 \verb`500 Internal Server Error` 笼统错误。 \verb`501 Not Implemented` 即将开业,敬请期待。 \verb`502 Bad Gateway` (网关或代理时返回的错误码,服务器自身工作正常,但最终资源的服务器错误)
\item \verb`503 Service Unavailable` 网络服务正忙,请稍 后重试
\end{itemize}

\subsection{常见字段}
\begin{itemize}
\item \verb`Host` 如 \verb`Host: www.A.com`。可以用于区分同一个服务器中的不同网站(只有 IP 的话就区分不了)
\item \verb`Content-Length` 回应的数据长度, 如 \verb`Content-Length: 1000`
\item \verb`Connection` 常用于客户端要求服务器使用 TCP 持久连接。 \verb`Connection: keep-alive`。 HTTP/1.1 版本的默认连接都是持久连接,但为了兼容老版本的 HTTP。这不是标准字段。
\item \verb`Content-Type`,相应的数据是什么格式, 如 \verb`Content-Type: text/html; charset=utf-8`
\item \verb`Accept` 用于请求某种格式的内容, 如 \verb`Accept: */*` 表示都行。
\item \verb`Accept-Encoding` 用于请求某种压缩格式: \verb`Accept-Encoding: gzip, deflate`。
\end{itemize}
