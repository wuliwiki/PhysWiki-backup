% 有界算子
% keys 有界性|算子
% license Usr
% type Tutor

\begin{issues}
\issueDraft
\end{issues}

\cite{Ke1}\pentry{有界集\nref{nod_BudSet},拓扑线性空间中的线性算子\nref{nod_TLinO}}{nod_8722}
拓扑线性空间的集可以定义有界性,相应的拓扑线性空间中的线性算子也可以定义有界性。

\begin{definition}{有界算子}
设 $E,E_1$ 是两个拓扑线性空间, $A:D_A\rightarrow E_1$ 是 $E$ 到 $E_1$ 的\enref{线性算子}{TLinO}。若 $A$ 把 $E$ 的属于 $D_A$ 的每一有界集都映到 $E_1$ 的 有界集,则称 $A$ 是\textbf{有界的}。
\end{definition}

线性算子的有界性和连续性有着密切的联系,这可以由下面的定理看出。

\begin{theorem}{}
1.线性连续算子必有界;

2.若 $A:E\rightarrow E_1$ 是线性有界算子,且 $E$ 满足第一可数性公理,则 $A$ 连续。
\end{theorem}
\textbf{证明:}1. 令 $A:D_A\rightarrow E_1$ 是 $E$ 到 $E_1$ 上的线性连续算子。 我们利用反证法证明:设 $M\subset D_A$ 是有界集,而 $AM\subset E_1$ 无界。那么存在 $E_1$ 中的零邻域 $V$,使得任意 $n>0$,都有 $\abs{\lambda}\geq n,AM\nsubseteq\lambda V$ (\autoref{def_BudSet_1})。即存在 $x\in M$,而 $Ax\nsubseteq \lambda V$。选取收敛于无穷大的正数序列 $\{m_i\}$ ,则存在 $\abs{\lambda_i}\geq m_i,x_i\in M$,使得 
\begin{equation}\label{eq_BoundO_1}
Ax_i\nsubseteq \lambda_iM\quad\Rightarrow\quad \frac{1}{\lambda_i}A x_i\nsubseteq V. ~
\end{equation}
因为 $\{\frac{1}{\lambda_i}\}$ 是收敛于0的正数列,而 $M$ 有界,所以 $\{\frac{1}{\lambda_i}x_i\}$ 收敛于0(\autoref{the_BudSet_1} 第一点)。结合\autoref{eq_BoundO_1} ,就有尽管 $\{\frac{1}{\lambda_i}x_i\}$ 收敛于0,但是 $\{\frac{1}{\lambda_i}A x_i\}$ 不收敛于0,因此 $A$ 不连续。这一矛盾表明 $AM$ 必有界。

2.同样利用反证法,设 $A$ 不连续,则存在 $E_1$ 中的零邻域 $V$ ,使得任意 $E$ 的零邻域 $U$,都有 $x\in U,Ax\notin V$。选取 $E$ 的零邻域系 $\{U_n\}$,其中 $U_{n+1}\subset U_n$(因为第一可数性保证任意零邻域 $U_n$,都存在零邻域 $U_{n+1}$,使得 $U_{n+1}\subset U_n$),于是 $\frac{1}{n^2} U_n$ 是 $E$ 的零邻域(\enref{数乘的连续性}{tvs})。因此存在 $x_n\in \frac{1}{n^2}U_n$,使得 $Ax_n\notin V\Rightarrow A (\frac{1}{n}y_n)\notin n V,y_n\in U_n$。

现在,我们证明 $\{(1/n) y_n\}$ 在 $E$ 中有界:任意零邻域 $U$,存在 $U_i\subset U$( $\{U_n\}$  时确定邻域系)。又因为 $y_n\in U_n$,所以 $y_n\in U_n\subset U_i\subset U,n\geq i$,即 $\{y_n\}\rightarrow 0$。由\autoref{lem_BudSet_1} ,$\{t_ny_n\}\Rightarrow0$。由\autoref{the_BudSet_1} 第2点, $\{(1/n) y_n\}$ 在 $E$ 中有界。

$\{(1/n) y_n\}$ 有界,而 $\{A(1/n y_n)\}$ 在 $E_1$ 中无界(因为它不属于零邻域 $V$),因此 $A$ 无界。此矛盾证明了命题。

 
\textbf{证毕!}


