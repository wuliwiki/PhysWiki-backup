% 南京理工大学 2008 年 研究生入学考试试题 普通物理(B)
% license Usr
% type Note

\textbf{声明}:“该内容来源于网络公开资料,不保证真实性,如有侵权请联系管理员”

\subsection{填空题(每空2分,共30分)}

1. 一质点作直线运动,运动方程为$x=3+2t^2+t^3(t>0)\text{(SI制)}$ ,则该质点在 时,,在$t = 3s$ 时,质点开始作减速直线运动。

2. 力度系数为 $K$ 的轻弹簧,一端固定,另一端连接一小质量为 $m$ 的物体,m 与地面间的滑动摩擦系数为 $\mu_k$。在弹簧为原长时,对静止物体 m 施一沿 X 轴方向的恒力为 $F (F > f, f = \mu_k mg)$ 时,则该弹簧的最大伸长量为 $\underline{\hspace{3cm}}$,该过程恒力 $F$ 作功为 $\underline{\hspace{3cm}}$。

3. 一质量为 $m$,长为 $4r$ 的均匀直尺,一端系于 O 点,另一端连接一质量为$2m$,半径为 $r$ 的匀质小圆盘边缘。该系统可绕通过 O 点垂直于纸面的轴转动,则该系统对 O 轴的转动惯量 $I = \underline{\hspace{3cm}}$,直尺在水平位置静止并开始转动时的角加速度 $\beta = \underline{\hspace{3cm}}$,直尺转到竖直位置时的角速度 $\omega = \underline{\hspace{3cm}}$。

4. 对于刚性双原子分子理想气体,其定容摩尔热容 $C_v =\underline{\hspace{3cm}}$,等温摩尔热容 $C_T = \underline{\hspace{3cm}}$。

5. $t = 27^\circ C$ 下,$1 \\ mol$ 氧气分子的平均动能为 $\underline{\hspace{3cm}}$;$t = 27^\circ C$ 下,$1 \ mol$ 氧气分子的总动能为 $\underline{\hspace{3cm}}$。

6. 一质点作简谐振动的位移时间曲线如图所示,振幅为 $A$,周期为 $T$,则质点振动的初相角是 $\underline{\hspace{3cm}}$,振动方程为 $\underline{\hspace{3cm}}$。