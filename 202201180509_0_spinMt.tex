% 自旋角动量矩阵
% 自旋|角动量|矩阵|算符|轨道角动量|升降算符

\pentry{自旋, 简谐振子升降算符归一化\upref{QSHOnr}}%未完成: 自旋词条

我们已知 $1/2$ 自旋的矩阵可以用 $\hbar/2$ 乘以泡利矩阵得到. 以下我们试图计算任意 $n/2$ 自旋粒子的三个自旋分量算符 $S_x$, $S_y$ 和 $S_z$ 对应的矩阵. 这些矩阵一般使用 $S_z$ 的本征态 $\ket{s, m}$ 作为基底(另外两组基底同理可得). 注意当 $s$ 为整数时, 本文的结论同样适用于轨道角动量(把算符 $S$ 和量子数 $s$ 分别替换为 $L$ 和 $l$ 即可).

基本思路是先求出升降算符 $S_\pm = S_x \pm \I S_y$ 的矩阵. 再把它们分别相加和相减得到 $S_x$ 和 $S_y$ 的矩阵. $S_z$ 算符在其本征态基底下是本征值组成的对角矩阵.

已知归一化系数为
\begin{equation}
S_\pm \ket{s,m} = \hbar \sqrt{s(s + 1) - m(m \pm 1)} \ket{s,m\pm1}
\end{equation}
所以
\begin{equation}
\mel{s,m}{S_\pm}{s,m'} = \delta_{m, m'\pm1} \hbar \sqrt{s(s + 1) - mm'}
\end{equation}
可见 $\mat S_+$ 只有下方的子对角线不为零, $\mat S_-$ 只有上方的子对角线不为零.

最后得三个矩阵为
\begin{equation}
\mel{s,m}{S_x}{s,m'} = \frac12(\delta_{m, m'+1} + \delta_{m, m'-1})\hbar \sqrt{s(s + 1) - mm'} 
\end{equation}
\begin{equation}
\mel{s,m}{S_y}{s,m'} = \frac{1}{2\I}(\delta_{m, m'+1} - \delta_{m, m'-1}) \hbar \sqrt{s(s + 1) - mm'}
\end{equation}
\begin{equation}
\mel{s,m}{S_z}{s,m'} = \delta_{m,m'} m\hbar 
\end{equation}
可以验证当 $s = 1/2, m = \pm1/2$ 时, 我们就得到了三个泡利矩阵.
