% 量子力学的基本原理(量子力学)
% 量子力学|态矢

\pentry{线性代数,微积分,经典力学}
\pentry{量子力学的基本原理(科普)\upref{QM0}}


在介绍量子力学的基本原理之前,我们要对“什么是物理理论”做一个简单分析.这是因为量子力学常因“反直觉”而让初学者迷惘,我希望以下讨论能帮助初学者理清概念,从而自然地接受量子力学的语言.这些讨论归入\autoref{QMPrcp_sub1} ,但读者可以跳过.

建议使用\textbf{Stern-Gerlach实验}\upref{SGExp}词条作为例子来理解本词条列举的概念.

\subsection{从牛顿理论到量子理论}\label{QMPrcp_sub1}

从现代科学哲学的视角看,一个物理理论是一个数学模型,模型中有一些概念有现实对应.这就是说,一个物理理论首先是一个数学理论,而使它区分于数学、成为物理的因素即是“实验”,可以直观理解为“有能在仪器上看到、用感官观测到”的量,通常称之为“可观测量”.

以牛顿力学为例.牛顿力学可以认为是四维空间中的几何学,其中“点的坐标”这一概念就是可观测量,它可以显示为尺子上的数值.更准确地说,考虑到牛顿力学中时间的绝对性,该理论应该是一维空间上处处沾了一片三维空间的“纤维丛”上的几何学,不同的观察者眼中会有不同的三维空间坐标,但是时间坐标不变.

光是几何学,那就不是物理理论了,因此牛顿力学还规定了质点运动的三大定律,相当于限定哪些几何轨迹是“合法”的.这三大定律定义了一个概念,“力”.力本身不是可观测量,但我们可以借助此概念来描述物体运动的规律.试举一例:质量为$m$的物体被劲度系数为$k$、原长为$l$的弹簧拉着,做角速度为$\omega$的匀速圆周运动,则规律预言,弹簧的伸长量是$\frac{l\omega^2m}{k-m\omega^2}$.伸长量是可观测量,所以我们可以做实验,看看测出来的伸长量是否是这个值,以此来判断牛顿理论的准确性.
\addTODO{弹簧的长度是心算的,可能有误.核算后再删除此“未完成”.}

牛顿力学怎么定义质点的状态?时间坐标、空间坐标以及空间坐标对时间的导数等.这种定义方式很直观,但我们要跳出直觉,理解牛顿力学的“数学结构”,才能平滑地过渡到量子理论的数学结构.

量子理论则用了截然不同的数学模型,它可以被认为是希尔伯特空间中的线性代数理论.它讨论系统的量子态,并将量子态定义为一个希尔伯特空间中的矢量,这些矢量可以表示为波函数.和力一样,态矢量、波函数等概念都是不可观测的.和牛顿三定律一样,态矢量也不是任意变化的,约束态矢量变化的就是著名的薛定谔方程.因此,许多课本中会告诉你,薛定谔方程在量子力学中的地位,就和三定律在牛顿力学中的地位一样.

牛顿力学中,质点的坐标是可观测量;量子力学中,\textbf{厄米算符}的\textbf{本征值}\footnote{也称特征值.}是可观测量.这些厄米算符对应的是一次观测,观测所得到的值总是对应算符的某个本征值,并且量子态还有一个神奇规律:被观测后,会变成对应本征值的一个本征态,这个过程被称作\textbf{坍缩}.



\subsection{量子力学的基本假设}


\subsubsection{薛定谔方程}

\begin{definition}{量子态(波函数)}
一个系统所处的\textbf{量子态}被表示为一个复值函数,称为\textbf{态函数(state function)}.

如果该函数的自变量是空间位置,则称此时是在\textbf{位置表象}下进行讨论;自变量是动量,则称\textbf{动量表象}.
\end{definition}

量子态的演化规律满足薛定谔方程,参见\textbf{薛定谔方程(单粒子一维)}\upref{TDSE11}、\textbf{定态薛定谔方程(单粒子一维)}\upref{SchEq}、\textbf{薛定谔方程(单粒子多维)}\upref{QMndim}、\textbf{薛定谔方程 2(单粒子多维)}\upref{TDSE}和\textbf{多体薛定谔方程}\upref{NbdQM}等词条,或在百科中搜索关键词“薛定谔方程”.

\subsubsection{左矢和右矢}

表示量子态的态函数全体构成了一个线性空间,因此量子态也有等价的代数诠释.位置表象和动量表象的态函数的不同,相当于选择了不同的基矢量后矢量坐标不同.

\begin{definition}{量子态(狄拉克符号)}\label{QMPrcp_def4}
一个系统所处的\textbf{量子态}被表示为复\textbf{希尔伯特空间}\upref{Hilber}上的一个矢量,用\textbf{狄拉克符号}\footnote{见\textbf{狄拉克符号}\upref{braket}或\textbf{对偶空间}\autoref{DualSp_sub1}~\upref{DualSp}.}表示.

每个量子态都有\textbf{两个}矢量表示,数学上可以理解为这两个“表示矢量”来自两个同构的复希尔伯特空间,这两个空间之间给定了一个同构$\sigma$,使得如果矢量$\bvec{v}$表示一个量子态,则$\sigma(\bvec{v})$表示同一个量子态.

量子态的两个“表示空间”中的矢量分别被称为\textbf{左矢}和\textbf{右矢},分别用$\bra{*}$和$\ket{*}$表示.$*$号处填表示这个量子态的符号.

\end{definition}

如果一个量子态的右矢表示为$\ket{s}$,那其左矢表示应为$\bra{s}=\sigma(\ket{s})$,这里的$\sigma$就是\autoref{QMPrcp_def4} 中提到的两个希尔伯特空间(左矢空间和右矢空间)之间的同构,从右矢空间到左矢空间.

\begin{definition}{}\label{QMPrcp_def2}

若已知态$\ket{s}$,则对于任意复数$c\in\mathbb{C}$,定义$c\ket{s}$和$\ket{s}$是\textbf{同一个态},$c\ket{s}$也可表示为$\ket{cs}$.态$\ket{s}$的对偶矢量表示为$\bra{s}$.

对于复数$a$,规定$\ket{as}$的对偶是$\bra{a^*s}$.
\end{definition}


\subsubsection{内积}


\begin{definition}{内积}\label{QMPrcp_def1}
一个左矢$\bra{a}$和一个右矢$\ket{b}$可以相乘得到一个复数,记为$\braket{a}{b}$.该运算称为“\textbf{内积(inner product)}”,定义为“满足\textbf{埃尔米特矢量空间(酉空间)}\upref{HVorUV}内积的性质\footnote{注意\textbf{埃尔米特矢量空间(酉空间)}\upref{HVorUV}中描述的是同一个空间中的两个向量相乘,而左矢和右矢是在不同的空间中.}”的运算,即
\begin{equation}
\leftgroup{
    \braket{a}{b}&=\braket{b}{a}^*\\
    \braket{a}{c_1b_1+c_2b_2}&=c_1\braket{a}{b_1}+c_2\braket{a}{b_2}\\
    \braket{a}{a}&\geq 0, \quad\text{等号仅在}\ket{a}=\bvec{0}\text{时成立}
} 
\end{equation}


\end{definition}
有了内积的概念后,我们就有了矢量的模长和矢量间正交的概念.为了方便,可以规定仅使用模长为$1$的矢量来表示量子态.

如果取右矢空间的标准正交基$\{\ket{s_\alpha}\}$,则由已有的同构关系$\sigma$,可以得到左矢空间的标准正交基$\{\sigma(\ket{s_\alpha})\}$.在这两个基下,任意左矢和右矢都可以用坐标来表示.

\begin{definition}{态矢量的坐标表示}\label{QMPrcp_def3}
态的右矢$\ket{s}$的坐标用\textbf{列矩阵}表示,左矢$\bra{s}$用\textbf{行矩阵}表示.
\end{definition}


按照\autoref{QMPrcp_def2} 、\autoref{QMPrcp_def1} 和\autoref{QMPrcp_def3} ,分量离散的态矢量的内积可以用矩阵乘法表示.



如果一个量子态的矢量表示是$\ket{s_i}$,态函数表示是$\psi_i$,且态函数取值连续(即态矢量的分量是连续的),那么内积对应态函数的积分:
\begin{equation}
\braket{s_2}{s_1} = \int_{\text{整个态空间}}\psi_2^*\psi_1\dd \bvec{x}
\end{equation}
这里$x$表示态空间上的坐标.

$\ket{s_i}$可以和$\psi_i$视为等价,有时也可以把这个态表示为$\ket{\psi_i}$.


\subsubsection{算符}

\begin{definition}{算符}
\textbf{算符(operator)}是把一个态变为另一个态的\textbf{映射}.
\end{definition}

注意,这里的“变为”并不是说物理上把态改变了,而是指“映射到了”的意思.

从代数角度来说,量子力学中的算符大都是线性空间上的线性变换,非线性算符包括\textbf{时间反演算符}.对于线性算符,我们关心的是这个变换本身的性质.
\addTODO{有了讨论时间反演算符的词条或相关内容后,在此引用.}

给定右矢空间的基以后,算符对量子态的作用可以表示为方阵乘在量子态的右矢或左矢矩阵上,该方阵称为该算符的\textbf{坐标}.如果用波函数表示一个量子态,那么算符也可能是求导算符或者其与其它算符相结合的形式.由于规定右矢坐标是列矩阵,因此为了与矩阵乘法配合,算符$X$应从\textbf{左边}作用在\textbf{右矢}$\ket{s}$上,得到另一个右矢$X\ket{s}$,而从\textbf{右边}作用在左矢$\bra{s}$上,得到另一个左矢$\bra{s}X$.


\begin{definition}{}

如果对于所涉及的\textbf{任意}态右矢$\ket{s}$都有
\begin{equation}
X\ket{s}=Y\ket{s}
\end{equation}
那么称算符$X$和$Y$\textbf{相等},记为$X=Y$.


如果对于\textbf{任意}态右矢$\ket{s}$都有
\begin{equation}
X\ket{s}
\end{equation}
则称$X$为一个\textbf{零算符},记为$X=0$.

\end{definition}




由于矢量之间可以相加,算符之间也可以相加:
\begin{equation}
(X+Y)\ket{s} = X\ket{x}+Y\ket{s}
\end{equation}
由矢量加法的结合性,容易证明算符加法具有结合性:
\begin{equation}
(X+Y)+Z=X+(Y+Z)
\end{equation}

如果算符$X$是线性的,则
\begin{equation}
X(a\ket{s_a}+b\ket{s_b}) = aX\ket{s_a}+bX\ket{s_b}
\end{equation}

相应地,算符总是从\textbf{右边}作用在\textbf{左矢}上:
\begin{equation}
\bra{s}\cdot X = \bra{s}X
\end{equation}

一般地,$X\ket{s}$和$\bra{s}X$\textbf{并不是}同一个量子态的表示.$X\ket{s}$的对偶表示定义为$\bra{s}X^\dagger$,如此定义的$X^\dagger$称为$X$的\textbf{共轭算符(conjugate operator)},或\textbf{厄米共轭(hermitian conjugate)}.

若$X$在某个基下的坐标是矩阵$\mathcal{M}$,则$X^\dagger$的坐标$\mathcal{M}^\dagger$是$\mathcal{M}$的共轭转置,即元素全部取共轭后进行转置,或者反过来先转置再共轭.

算符的乘法定义为
\begin{equation}
(XY)\ket{s} = X(Y\ket{s})
\end{equation}
可以理解为映射的复合,可以表示为矩阵乘法.算符乘法和映射复合一样,\textbf{通常}不可交换:
\begin{equation}
XY\neq YX
\end{equation}
却可复合:
\begin{equation}
(XY)Z=X(YZ)
\end{equation}

注意
\begin{equation}
(XY)^\dagger = Y^\dagger X^\dagger
\end{equation}

以下几个例子给出的是重要的算符.其中,动量算符和能量算符可由德布罗意关系得出,因为$E=h\nu$和$p=h/\lambda$意味着一个简谐波应写为$\exp{(\frac{\I \bvec{p}}{\hbar}\cdot\bvec{x}-\I Et)}$.角动量算符继承经典力学中的关系:$\bvec{L}=\bvec{r}\times\bvec{p}$.注意$\hbar=h/2\pi$.

\begin{example}{位置表象下的算符}
\textbf{位置表象}下,单粒子态函数的
\begin{itemize}
\item \textbf{位置算符}为$x$(一维情况)或$\bvec{x}$(三维情况);
\item \textbf{动量算符}为$-\I\hbar\frac{\partial}{\partial x}$(一维情况)或$-\I\hbar\nabla$(三维情况);
\item \textbf{能量算符}为$\I \frac{\partial}{\partial t}$;
\item \textbf{角动量算符}为$\bvec{x}\times (-\I\hbar\nabla)$,展开后得
\begin{equation}
\hat{\bvec{L}}=
-\I\hbar
\pmat{
    y\frac{\partial}{\partial_z}-z\frac{\partial}{\partial_y}\\
    z\frac{\partial}{\partial_x}-x\frac{\partial}{\partial_z}\\
    x\frac{\partial}{\partial_y}-y\frac{\partial}{\partial_x}
}
\end{equation}
\end{itemize}
\end{example}


\begin{example}{动量表象下的算符}
\textbf{动量表象}下,单粒子态函数的
\begin{itemize}
\item \textbf{位置算符}为$\I\hbar\frac{\partial}{\partial p}$(一维情况)或$\I\hbar\pmat{\frac{\partial}{\partial p_x}, \frac{\partial}{\partial p_y}, \frac{\partial}{\partial p_z}}\Tr$(三维情况);
\item \textbf{动量算符}为$p$(一维情况)或$\bvec{p}$(三维情况);
\item \textbf{能量算符}为$\I \frac{\partial}{\partial t}$.
\item \textbf{角动量算符}为$\I\hbar\pmat{\frac{\partial}{\partial p_x}, \frac{\partial}{\partial p_y}, \frac{\partial}{\partial p_z}}\Tr\times \bvec{p}$,展开后得
\begin{equation}
\hat{\bvec{L}}=
\I\hbar
\pmat{
    *\\
    *\\
    *
}
\end{equation}
\end{itemize}
\end{example}





\begin{example}{投影算符}

设选定了态空间的一组基矢量$\{\ket{s_\alpha}\}_{\alpha\in \Gamma}$,其中指标集$\Gamma$可以是离散的,也可以是连续的.任意态矢量$\ket{a}$的下标为$\alpha$的坐标分量称之为其在基矢量$\ket{s_\alpha}$方向上的\textbf{投影(projectiong)},容易证明该投影是
\begin{equation}
\ket{s_\alpha}\braket{s_\alpha}{a}
\end{equation}
因此$\ket{s_\alpha}\bra{s_\alpha}$是一个算符,称之为\textbf{投影算符(projection operator)}.

思考:列矩阵左乘行矩阵得到什么样的矩阵?这和$\ket{a}\bra{b}$有什么关系?

\end{example}



\subsubsection{测量}

每一个测量行为被表示为一个算符.测量假设认为,进行测量后,量子态会坍缩成该测量算符的一个\textbf{本征}矢量,称为其\textbf{本征态(eigenstate)},同时返回该本征态的\textbf{本征值}作为测量值.

\begin{example}{}
用\textbf{位置算符}表示的测量,测量值被认为是粒子的位置,测量后粒子坍缩为位置算符的本征态.
\end{example}

如果一个量子态$\ket{a}$不是测量算符$X$的本征态,那么它是本征值不同的本征态的线性组合:
\begin{equation}\label{QMPrcp_eq1}
\ket{a} = \sum_{i=1}^{n} c_i\ket{s_i}
\end{equation}
或
\begin{equation}\label{QMPrcp_eq2}
\ket{a} = \int c(x)\ket{s_x}\dd x
\end{equation}
其中\autoref{QMPrcp_eq1} 是离散情况,\autoref{QMPrcp_eq2} 是连续情况,$\ket{s_i}$和$\ket{s_x}$都表示$X$的本征态,$c_i$和$c(x)$表示各本征态的本征值.

离散情况下,对$\ket{a}$进行$X$测量后,得到测量值$c_i$且$\ket{a}$坍缩为$\ket{s_i}$的概率为$\abs{c_i}^2$

\begin{example}{}
考虑光的偏振实验.

把光的量子态定义为沿着偏振角度的一个矢量\footnote{注意,根据前面的规定,$\pm\ket{s}$表示同一个态,因此无所谓正负方向.},构成一个态空间.取水平方向的偏振,表示为$\ket{h}$,和竖直方向的偏振,表示为$\ket{v}$,则$\{\ket{h}, \ket{v}\}$构成一组基.

易证,与水平方向成$\theta$角的偏振态被表示为$\cos\theta\ket{h}+\sin\theta\ket{v}$.

现在设置一个检偏器,其与水平面的角度为$0$.“让光通过检偏器”就是一个测量行为,如果定义“光成功通过”用测量值$1$代表,“光未通过”用测量值$0$代表,那么这个测量行为的算符在给定的基下表示为
\begin{equation}
\pmat{
    1&0\\0&0
}
\end{equation}
该测量算符的本征值为$1$和$0$,分别对应本征态$\ket{h}$和$\ket{v}$.

再设置一个检偏器,其与水平面的角度为$\pi/2$.则该测量行为的算符表示为
\begin{equation}
\pmat{
    0&0\\0&1
}
\end{equation}
该测量算符的本征值为$0$和$1$,分别对应本征态$\ket{h}$和$\ket{v}$.

现在让一个初始态为$\cos\theta\ket{h}+\sin\theta\ket{v}$的光子依次通过这两个检偏器.光子有$\cos^2\theta$的概率成功通过第一个检偏器,并坍缩为$\ket{h}$态;有$\sin^2\theta$的概率不通过.考虑通过的光子,它有$0$的概率成功通过第二个检偏器.因此,这两个检偏器在理想情况下能完全阻绝光.

现在在两个检偏器中间插入一个与水平面成$\pi/4$角的检偏器,它的本征值为$1$和$0$,分别对应本征态$\frac{\sqrt{2}}{2}\ket{h}+\frac{\sqrt{2}}{2}\ket{v}$和$\frac{\sqrt{2}}{2}\ket{h}-\frac{\sqrt{2}}{2}\ket{v}$,因此这个检偏器的测量算符表示为
\begin{equation}
\pmat{
    \frac{\sqrt{2}}{2}&\frac{\sqrt{2}}{2}\\\frac{\sqrt{2}}{2}&\frac{\sqrt{2}}{2}
}
\end{equation}

让一个初始态为$\cos\theta\ket{h}+\sin\theta\ket{v}$的光子依次通过这三个检偏器.光子有$\cos^2\theta$的概率成功通过第一个检偏器,并坍缩为$\ket{h}$态;由于
\begin{equation}
\ket{h}=\frac{\sqrt{2}}{2}\qty(\frac{\sqrt{2}}{2}\ket{h}+\frac{\sqrt{2}}{2}\ket{v})+\frac{\sqrt{2}}{2}\qty(\frac{\sqrt{2}}{2}\ket{h}-\frac{\sqrt{2}}{2}\ket{v})
\end{equation}
因此光子有$(\sqrt{2}/2)^2=1/2$的概率通过第二个检偏器,并坍缩为$\frac{\sqrt{2}}{2}\ket{h}+\frac{\sqrt{2}}{2}\ket{v}$态;接下来,光子有$(\sqrt{2}/2)^2=1/2$的概率通过第三个检偏器.于是可以计算出光子成功通过三个检偏器的概率为$\cos^2\theta\cdot\frac{1}{2}\cdot\frac{1}{2}=\frac{\cos^2\theta}{4}$.

\end{example}


%算符的特征值必为实数;因此算符必为Hermitian的.
%讨论期待值的计算.

































