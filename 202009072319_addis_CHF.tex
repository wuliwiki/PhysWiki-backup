% 柯西—阿达玛公式
% 幂级数|收敛半径

\pentry{幂级数\upref{anal}}

\subsection{表述与证明}

\textbf{柯西—阿达玛公式(Cauchy-Hadamard formula)}是计算幂级数收敛半径的一般公式.

\begin{theorem}{柯西—阿达玛公式}
设有幂级数
\begin{equation}\label{CHF_eq1}
\sum_{n=0}^\infty c_n(z-a)^n.
\end{equation}
则幂级数的收敛半径$R$由公式
$$
\frac{1}{R}=\limsup_{n\to\infty}|c_n|^{1/n}
$$
计算.
\end{theorem}
证明并不困难. 如果$|z-a|>R$, 那么可取$\delta>0$如此之小, 使得$|z-a|>R(1+\delta)$, 从而有
$$
|c_n||z-a|^n>(|c_n|^{1/n}R(1+\delta))^n.
$$
而按照上极限的定义, 有无穷多个$n$使得
$$
|c_n|^{1/n}>\frac{1}{R(1+\delta)}.
$$
于是\autoref{CHF_eq1} 的一般项不趋于零, 从而这级数不可能收敛. 而如果$|z-a|<R$, 那么可取$\delta>0$如此之小, 使得$|z-a|<R(1+2\delta)^{-1}$; 而按照上极限的定义, 从某个$n$开始总有$|c_n|^{1/n}<(1+\delta)/R$, 因此对于充分大的$n$就有
$$
|c_n||z-a|^n\leq\frac{(1+\delta)^n}{(1+2\delta)^n}.
$$
从而幂级数的一般项由收敛的几何级数控制.

\subsection{应用举例}
对于幂级数
$$
\sum_{n=0}^\infty c_n(z-a)^n,
$$
逐项微分和逐项积分的级数分别是
$$
\sum_{n=1}^\infty nc_n(z-a)^{n-1},\,
\sum_{n=0}^\infty \frac{c_n}{n+1}(z-a)^{n+1}.
$$
按照柯西—阿达玛公式, 这两个幂级数的收敛半径都与原幂级数相同. \textbf{因此逐项微分或逐项积分不改变幂级数的收敛半径.}

对于$\lambda>0$, 按照柯西—阿达玛公式, 幂级数
$$
\sum_{n=0}^\infty\frac{z^n}{n^{n/\lambda}}
$$
的收敛半径是无穷大, 而且实际上存在 $A,B>0$ 使得
$$
\left|\sum_{n=0}^\infty\frac{z^n}{n^{n/\lambda}}\right|
\leq Ae^{B|z|^\lambda}.
$$

以$d_1(n)$表示$n$的非1因子个数. 幂级数
$$
\sum_{n=1}^\infty d_1(n)z^n
$$
的系数涨落很没有规律: 当 $n$ 为素数时$d_1(n)=1$, 而一般的正整数却可能有很多个因子. 但显然$1\leq d_1(n)<n$, 因此
$$
\lim_{n\to\infty}d_1(n)^{1/n}=1,
$$
所以按照柯西—阿达玛公式, 此幂级数的收敛半径是 1.