% 估算理想气体的熵
% license CCBY4
% type Note

% 一篇娱乐文章,可能有点bug...
\pentry{理想气体的熵:纯微观分析\upref{IdeaS}}

\subsection{估算熵}
我们知道Sackur-Tetrode 公式 \upref{IdeaS}:
\begin{equation}
S = k\ln \Omega  = Nk \qty(\ln \frac{V}{N} \frac{(2\pi mkT)^{3/2}}{h^3} + \frac52)~,
\end{equation}

我们借此估算标准状态($p=10^5 \Si{Pa}, T=298\Si{K}$)下$1 \Si{mol}$氢气的熵。

我们知道气体体积\upref{Igas} 
$$V=\frac{NkT}{p}~,$$
氢气分子质量 
$$m = \frac{2}{1000 N_A} = \frac{1}{500 N_A}~,$$
物理学常数\upref{Consts} : 
$$
\begin{aligned}
k &= 1.38 \times 10^{-23}\Si{J/K}\\
N_A &= 6.02 \times 10^23 \Si{mol^{-1}}\\
h &= 6.62\times10^{-34} \Si{J\cdot s}
\end{aligned}
~.$$

先计算对数部分,
$$
\begin{aligned}
 & \ln \frac{V}{N} \frac{(2\pi mkT)^{3/2}}{h^3} \\
 = & \ln \frac{kT}{p h^3} \qty(\frac{2\pi kT}{500 N_A})^{3/2}\\
 = & \frac{5}{2} \ln k + \ln T - \ln p - 3 \ln h + \frac{3}{2} \ln (2 \pi T) - \frac{3}{2} \ln 500 - \frac{3}{2} \ln N_A \\
 \approx & 11.64\\
\end{aligned}~,
$$
因此
$$
\begin{aligned}
S \approx Nk (11.62+5/2) \approx 117.5 \Si{J/K}
\end{aligned}~,
$$
与物理化学中氢的标准生成熵 $S = 130.57 \Si{J/K}$ \footnote{数据来自朱文涛等人的《简明物理化学》附录} 在同一数量级,相对误差在$10\%$左右。

\subsection{估算微观状态数}
我们知道:
\begin{equation}
S = k \ln \Omega~,
\end{equation}
因此,
$$
\Omega = e^{S/k}~ \approx e^{10^{25}} ~.
$$
可见,粗略估计气体的微观状态数为$10^{10^{24}}$,即$1$后面跟着$10^{24}$个零(不是$24$个零!)。

如果把这个数打印在纸条上,假定每一个数字宽 $1 \Si{mm} = 10^{-3} \Si{m}$ ,那么纸条的长度大概是 $10^{21} \Si{m} = 10^6 \Si{ly} = 1000 \Si{kly}$\footnote{$\Si{ly}$是光年,$\Si{kly}$是千光年,$1 \Si{ly} = 9.46\times10^{15} \Si{m}$, $1 \Si{kly}$ = $1000 \Si{ly}$},即 $1000$千光年。相比之下,银河系直径大概为$100 \Si{kly}$左右\footnote{数据来源:NASA}。看起来这个纸条能绕银河系三圈!

