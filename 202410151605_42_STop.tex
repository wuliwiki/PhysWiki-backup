% 共轭空间中的强拓扑
% keys 强拓扑|完备
% license Usr
% type Tutor
\pentry{共轭空间与代数共轭空间\nref{nod_ConSpa},线性算子的范数\nref{nod_ONorm}}{nod_0a2f}

\subsection{赋范空间的强拓扑}

由\autoref{ex_tvs_1} 可知,赋范空间是一个拓扑线性空间,因此其上自然由\enref{线性连续泛函}{LinCon}的定义。而赋范空间上根据
\begin{equation}\label{eq_STop_1}
\norm{f}:=\sup_{x\neq0}\frac{\abs{f(x)}}{\norm{x}}~
\end{equation}
可引入线性连续泛函的范数,证明\autoref{eq_STop_1} 满足\enref{范数}{norm}的定义和\autoref{ex_ONorm_1} 的证明完全一样。因此赋范空间的共轭空间可赋予赋范空间的自然结构。范数可以用来定义度量,度量有一个自然定义开集的方式,即在赋范空间上有一个自然定义的拓扑,在赋范空间的共轭空间中这样定义的拓扑就是强拓扑。
\begin{definition}{强拓扑}
设 $E$ 是赋范空间,则其\enref{共轭空间}{ConSpa} $E^*$ 上由\autoref{eq_STop_1} 定义的范数相应的拓扑称为 $E^*$ 的\textbf{强拓扑}。
\end{definition}

当希望把 $E^*$ 当作赋范空间时,我们将其写作 $(E^*,\norm{*})$。

\begin{theorem}{}
设 $E$ 是赋范空间,则 $(E^*,\norm{*})$ 是\enref{完备}{ComSpa}的。
\end{theorem}




