% 约翰·冯·诺依曼
% license Usr
% type Wiki

(本文根据 CC-BY-SA 协议转载自原搜狗科学百科对英文维基百科的翻译)

约翰·冯·诺伊曼(/vɒn ˈnɔɪmən/;匈牙利语:纽曼·雅诺斯·拉霍斯,发音;1903年12月28日-1957年2月8日)是匈牙利裔美国数学家、物理学家、计算机科学家和博学多才者。冯·诺伊曼被普遍认为是他那个时代最重要的数学家,[1] 并被称为“伟大数学家的最后代表”; 他是一个擅长融合纯科学和应用科学的天才。

他在许多领域做出了重大贡献,包括数学(数学基础、函数分析、遍历理论、表象理论、算子代数、几何、拓扑和数值分析)、物理(量子力学、流体力学和量子统计力学)、经济学(博弈论)、计算(冯·诺依曼架构、线性规划、自我复制机器、随机计算)和统计学。

他是将算子理论应用于量子力学发展函数分析的先驱,也是博弈论和细胞自动机概念、通用构造器和数字计算机发展的关键人物。

他一生发表了150多篇论文:大约60篇是纯数学论文,60篇是应用数学论文,20篇是物理论文,其余的是特殊数学或非数学学科的论文。[2]他的最后一部作品是他住院期间写的一份未完成的手稿,后来以书的形式出版了《计算机和大脑》。

他对自我复制结构的分析先于脱氧核糖核酸(DNA)结构的发现。在他提交给国家科学院(National Academy of Sciences)的一份罗列了他一生研究工作简短的清单表中,他说,“我认为最重要的研究工作是量子力学的研究,它是1926年在哥廷根发展起来的,随后又于1927-1929年在柏林发展。此外,我1930年在柏林大学和1935-1939年在普林斯顿大学期间,研究了各种形式的算子理论;1931-1932年在普林斯顿大学,我研究了遍历定理。”

第二次世界大战期间,冯·诺伊曼与理论物理学家爱德华·泰勒(Edward Taller)、数学家斯塔尼斯瓦夫·乌兰(Stanislaw Ulam)等人一起参与曼哈顿计划,解决热核反应和氢弹中核物理的关键步骤问题。他创建了内爆型核武器中使用的爆炸透镜背后的数学模型,并创造了“千吨”(梯恩梯TNT)一词,作为所产生爆炸力的量度。

战后,他在美国原子能委员会总顾问委员会任职,并为许多组织提供咨询,包括美国空军、陆军弹道研究实验室、武装部队特种武器项目和劳伦斯·利弗莫尔国家实验室。作为一名匈牙利移民,他担心苏联会取得核优势,于是设计并推行了相互保证毁灭的政策,以限制军备竞赛。

\subsection{早期生活和教育}
\subsubsection{1.1 家庭背景}
\begin{figure}[ht]
\centering
\includegraphics[width=6cm]{./figures/0570a56209846ea4.png}
\caption{图为冯·诺伊曼的出生地,布达佩斯巴索利街16号。自1968年以来,它一直是约翰·冯·诺依曼计算机协会的所在地。} \label{fig_von_1}
\end{figure}
冯·诺伊曼原名为诺伊曼·雅诺斯·拉霍斯,他生于一个富裕的、适应新文化的、不循规蹈距的犹太家庭(在匈牙利,姓氏排在第一位)。他的名字在英语中等同于约翰·路易斯。

冯·诺伊曼出生在匈牙利王国布达佩斯,当时匈牙利是奥匈帝国的一部分。[3][4] 他是三个兄弟中的老大;他的两个弟弟妹妹是米哈伊尔(英文:迈克尔·冯·诺依曼;1907-1989)和米克尔斯(尼古拉斯·冯·诺伊曼,1911-2011)。[5]他的父亲诺依曼·米卡(马克斯·冯·诺依曼,1873-1928)是一名银行家,拥有法学博士学位。19世纪80年代末,他从佩奇搬到布达佩斯。[6]米卡的父亲和祖父都出生在匈牙利北部泽姆普伦(Zemplen)县的翁德(Ond)(现在是泽伦茨(Szerencs)镇的一部分)。约翰的母亲是坎恩·玛姬特(英文:玛格丽特·坎恩)[7];她的父母是Jakab Kann和Katalin Meisels。[8]坎恩家族的三代人都住在布达佩斯坎恩-赫勒办公室上方的宽敞公寓里;冯·诺依曼一家住在顶层的一套有18个房间的公寓里。[9]

1913年2月20日,弗朗兹·约瑟夫皇帝将约翰的父亲提升为匈牙利贵族,以表彰他对奥匈帝国的贡献。诺伊曼家族因此获得了“玛吉塔(Margitta)”的世袭称谓,意思是“玛吉塔”(今天罗马尼亚,玛吉塔)。这个家庭与这个城镇没有任何关系;这个称呼是根据玛格丽特选择的,正如他们选择的描绘三个玛格丽特的纹章一样。诺伊曼·诺伊曼·雅诺斯后来改名为诺玛姬塔·诺依曼·雅诺斯(约翰·诺伊曼·德·玛姬塔),后来改为德国约翰·冯·诺伊曼。
\subsubsection{1.2 神童}
冯·诺依曼是个神童。当他6岁的时候,他能在脑子里进行两个8位数的除法[10][11] ,还能用古希腊语交谈。当6岁的冯·诺依曼发现母亲漫无目的地盯着他时,他问她,“你在盘算什么?”[12]

匈牙利的儿童直到10岁才开始正式上学;家庭女教师教导冯·诺依曼、他的兄弟和堂兄弟姐妹。马克斯认为除了匈牙利语之外,语言知识也很重要,所以孩子们接受了英语、法语、德语和意大利语的辅导[13]。到8岁时,冯·诺伊曼对微积分就很熟悉了,但他对历史特别感兴趣。他阅读了威廉·昂肯(Wimhelm Oncken)的46卷本《Allgemeine Geschichte in Einzeldarslellungen》。马克斯购买的一个私人图书馆里有一份副本。公寓里的一个房间被改造成了图书馆和阅览室,书架从天花板延伸到了地板。[14]

冯·诺伊曼于1911年加入路德教( Lutheran Fasori Evangélikus Gimnázium)。尤金·维格纳比冯·诺依曼早一年进入路德教会学校,并很快成为了他的朋友[15]。这是布达佩斯最好的学校之一,也是为精英设计的优秀教育体系的一部分。在匈牙利体制下,孩子们在一所体育馆接受所有教育。匈牙利的学校体系造就了以智力成就著称的一代人,其中包括西奥多·冯·卡尔曼(1881年)、乔治·德·赫维希(1885年)、迈克尔·波兰尼(1891年)、莱昂斯·西拉德(1898年)、丹尼斯·加博尔(1900年)、维格纳(1902年)、爱德华·泰勒(1908年)和保罗·erdős(1913年)。[16]这批人有时被称为“火星人”。[17]

虽然马克斯坚持冯·诺依曼在适合他年龄的年级上学,但他同意聘请私人教师,在他表现出天赋的领域给他高级指导。15岁时,他开始在著名分析师Gábor·szegő.的指导下学习高级微积分[15],他们第一次见面时,Szegő被这个男孩的数学天赋惊呆了,他激动地留下了眼泪。冯·诺依曼对Szegő在微积分中提出的问题的一些即时解决方案被描绘在他父亲的信纸上,现在仍在布达佩斯的冯·诺依曼档案馆展出[15]。到19岁时,冯·诺伊曼已经发表了两篇重要的数学论文,其中第二篇给出了序数的现代定义[18],取代了乔治·康托的定义。在体育馆结束学业后,冯·诺伊曼参加并获得了国家数学奖——厄茨奖。[19]
\begin{figure}[ht]
\centering
\includegraphics[width=10cm]{./figures/532c35616045e04b.png}
\caption{} \label{fig_von_2}
\end{figure}
\subsubsection{1.3 大学学习}
根据他的朋友西奥多·冯·卡门的说法,冯·诺依曼的父亲希望约翰跟随他进入工业界,从而把他的时间投入到比数学更有经济价值的工作中。事实上,他的父亲要求西奥多·冯·卡门说服他的儿子不要把数学作为他的专业。[20]冯·诺依曼和他的父亲决定最好的职业道路是成为一名化学工程师。冯·诺伊曼对此知之甚少,所以他被安排在柏林大学学习两年的非学位化学课程,之后他参加了著名苏黎世联邦理工学院(ETH Zurich)的入学考试[21],并于1923年9月通过了[22]。与此同时,冯·诺伊曼(von Neumann)也进入布达佩斯的帕兹曼·彼得大学,作为数学博士候选人。在论文中,他选择的主题为康托集合论的公理化。[23][24] 他于1926年从苏黎世联邦理工学院毕业,成为一名化学工程师(尽管魏格纳说冯·诺依曼从来就不太喜欢化学这门学科)[25],并在获得化学工程学位的同时通过了数学博士学位的最后的考试,其中魏格纳写道,“显然,博士论文和考试并没有给他带来太大的压力。”[25]然后,他在洛克菲勒基金会的资助下进入了哥廷根大学,在戴维·希尔伯特的指导下学习数学。[26]

\subsection{早期职业和私人生活}
冯·诺伊曼在1927年12月13日完成了他的教授论文,并于1928年作为一名编外讲师在柏林大学开始了他的授课,[27]成为该大学历史上任何学科中最年轻的一位授课教师。[28]到1927年底,冯·诺伊曼已经发表了12篇数学专业学术论文,到1929年底,以每月近一篇专业学术论文的速度发表了32篇论文。[29]他具有很强的记忆和回忆的能力,这使他能够快速记忆电话簿的页面,并背诵其中的姓名、地址和号码。[30]1929年,他短暂地成为汉堡大学的私人教师,那里成为终身教授的前景更好[30],但同年10月,当他被邀请到新泽西州普林斯顿的普林斯顿大学时,一个更好的机会出现了。[31]

1930年的元旦,冯·诺伊曼娶了玛丽埃塔·柯维斯(Marietta Kovesi),她曾在布达佩斯大学学习经济学。[31]冯·诺依曼和玛丽埃塔有一个孩子,女儿玛丽娜,出生于1935年,截至2017年,她是密歇根大学工商管理和公共政策的杰出教授。[32]这对夫妇于1937年离婚。1938年10月,冯·诺伊曼与克拉拉·丹结婚,克拉拉·丹是他在二战爆发前最后一次回布达佩斯时遇到的。[33]

冯·诺依曼在与玛丽埃塔结婚之前,于1930年受洗成为天主教徒。[34]冯·诺依曼的父亲马克斯于1929年去世。马克斯在世时,家里没有一个人皈依基督教,但后来都皈依了。[35]

1933年,当新泽西高等研究院任命赫尔曼·韦勒为终身教授的计划落空时,他被授予终身教授职位。[36]他在那里一直担任数学教授,直到去世,尽管他已经宣布打算辞职,成为加州大学的一名讲座教授。[37]1939年,他的母亲、兄弟和姻亲跟随冯·诺依曼来到美国。[38]冯·诺依曼把他的名字英语化为约翰,保留了冯·诺依曼的德国贵族姓氏。他的兄弟们把他们的名字换成了“诺依曼”和“冯尼曼”。[39]冯·诺伊曼(Von Neumann)于1937年加入美国国籍,并立即试图成为美国陆军军官后备队的中尉。他轻而易举地通过了考试,但最终由于年龄原因被拒绝了。[39]他战前对法国如何对抗德国的分析经常被引用:“哦,法国不重要。”[40]

克拉拉和约翰·冯·诺依曼在当地学术界非常活跃。[41]他在韦斯特科特路26号的白色隔板房子是普林斯顿最大的私人住宅之一。[42]他非常注重自己的衣服,总是穿正式的西装。有一次,他骑着骡子穿过大峡谷时,身上穿了一套三件套的细条纹套装。[43]据说希尔伯特在冯·诺伊曼1926年的博士考试中曾问:“请问,这位候选人的裁缝是谁?”,因为他从未见过如此漂亮的晚礼服。[44]

冯·诺依曼一生热爱古代史,以其惊人的历史知识而闻名。普林斯顿的拜占庭历史教授曾说冯·诺伊曼在拜占庭历史方面比他更专业。[45]

冯·诺伊曼喜欢吃喝;他的妻子克拉拉说,除了卡路里,他什么都能计数。他喜欢意第绪语和“低俗的”幽默(尤其是打油诗)。[46]他不吸烟。[46]在普林斯顿,他因经常在留声机上播放极其响亮的德国进行曲而受到投诉,这些音乐让包括阿尔伯特·爱因斯坦在内的在邻近办公室工作的邻居们都无法专心工作。[47]冯·诺伊曼在嘈杂、混乱的环境中完成了一些最出色的工作,他曾经告诫他的妻子为他准备一间安静的书房,但他从来没有用过它,更喜欢在他们夫妇的客厅里大声放着电视。[48]尽管他是一个出了名的坏司机,但他仍然喜欢开车——经常是一边看书一边开车——这导致了多次被逮捕和多起交通事故。当库斯伯特·赫德雇用他为IBM公司的顾问时,赫德经常悄悄地为他的交通罚单支付罚款。[49]

冯·诺伊曼在美国最亲密的朋友是数学家斯塔尼斯瓦乌拉姆(Stanistlaw Ulam)。乌兰后来的一个朋友吉安-卡尔洛·罗塔写道,“他们会花几个小时没完没了地闲聊和咯咯笑,交换犹太笑话,并时不时地聊到数学话题。”当冯·诺依曼在医院里奄奄一息时,每次乌拉姆来访,他都会带着一套新的笑话来逗他开心。[50]他相信他的许多数学思想都是凭直觉产生的,他经常会带着一个未解决的问题睡觉,一醒来就能知道答案。[48]乌兰指出,冯·诺依曼的思维方式可能不是视觉的,而是听觉的。[51]