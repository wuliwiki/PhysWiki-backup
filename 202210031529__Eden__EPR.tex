% EPR 佯谬与定域隐变量理论
% EPR佯谬|量子力学|物理实在|定域|隐变量
\addTODO{未完成}
\pentry{量子力学的基本原理(量子力学)\upref{QMPrcp},自旋角动量\upref{Spin}}

1935年,Einstein,Podolsky 和 Rosen 发表一篇名为《Can Quantum Mechanical Description of Physical Reality be Considered Complete》文章,以定域实在论为出发点论证量子力学的不完备性,以佯谬的形式对量子力学的哥本哈根诠释提出了批评.在这篇论文中 Einstein 谈到:“我不相信上帝会掷骰子.”(原文是“I can't believe that God plays dice”.) 

EPR 佯谬是基于这样一个思想实验\footnote{后来 Bohm 对这个思想实验进行了改进,使之具有更直观的物理图像,也就是我们现在看到的版本.}.我们有一个粒子源,它能够同时产生两个自旋 $1/2$ 的粒子(为了方便表述,下面假定它为电子).单个电子的自旋方向是不确定的,但是粒子源保证了这两个电子的自旋总是相反的,即双电子体系的总自旋为 $0$.具体而言,双粒子波函数可以写为
\begin{equation}
\ket{\Psi^-}=\frac{1}{\sqrt{2}}(\ket{\uparrow}\ket{\downarrow}-\ket{\downarrow}\ket{\uparrow})
\end{equation}

\begin{figure}[ht]
\centering
\includegraphics[width=14cm]{./figures/EPR_1.png}
\caption{EPR 佯谬思想实验} \label{EPR_fig1}
\end{figure}

其中一个电子我们设为 $A$,它经过左侧的斯特恩–格拉赫\footnote{详见Stern-Gerlach 实验\upref{SGExp}.}实验装置,被测量其自旋沿某一方向上的投影分量.根据量子力学的\textbf{坍缩假设},电子的状态会坍缩到沿这一方向上自旋为 $1/2$ 或 $-1/2$ 的本征态,它们是相应自旋算符的本征态:例如可以是 $S_z$ 或 $S_x$.以 $S_z$ 的两个本征态 $\ket{\uparrow},\ket{\downarrow}$ 为自旋空间的基底,那么自旋算符在数学上可以写为泡利算符乘以 $1/2$.(本文中我们采用自然单位制,即忽略所有的 $c$ 或 $\hbar$.)

由于 $A,B$ 两粒子体系的量子态是自选单态,在对 $A$ 作测量时,根据正统量子力学,$B$ 粒子的状态“似乎也发生了坍缩”.例如当 $A$ 粒子测得的 $S_z=1/2$ 时,$B$ 粒子测得的 $S_z$ 就一定是 $-1/2$,这是两电子的总自旋为 $0$ 的必然要求.那么,似乎有一种鬼魅般的超距作用在起作用.这是否符合我们对因果律的认知呢?是否真的存在某种非定域的超距作用?又似乎如同 EPR 论文中所说的那样存在某个正统量子力学的波函数之外的隐变量能够解释这一切?下面我们将先介绍 EPR 论文所基于的几个假设:\textbf{定域性(Locality)和实在论}.

\subsection{定域实在论假设}
\subsection{定域隐变量原理}
\subsection{Bell 不等式简介}