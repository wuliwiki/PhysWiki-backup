% 二阶常系数齐次微分方程
% keys 微积分|微分方程|常微分方程|二阶常系数齐次微分方程|特征方程

\pentry{常微分方程\upref{ODE}, 指数函数(复数)\upref{CExp}}

\begin{issues}
\issueOther{需要补充例题}
\end{issues}

\footnote{参考 \cite{同济高}}二阶常系数齐次微分方程形式如下
\begin{equation}
ay'' + by' + cy = 0
\end{equation}
注意到指数函数 $y = C \E^{rx}$ 第 $n$ 阶导数为 $r^n \E^{rx}$, 不妨尝试把指数函数代入方程,得
\begin{equation}
(a r^2 + br + c) \E^{rx} = 0
\end{equation}
由于 $\E^{rx} \ne 0$, 必有 $a r^2 + br + c = 0$. 把这个二次函数叫做\textbf{特征方程},解特征方程,就可以得到方程的解.根据根的分布, 有如下四种情况

\begin{enumerate}
\item 有两个不同的实根 $r_1$ 和  $r_2$( $b^2 - 4ac > 0$), 方程的通解为
 \begin{equation}
y = C_1 \E^{r_1 x} + C_2 \E^{r_2 x}
\end{equation}
\item 有一个重根 $r$ ($b^2 - 4ac = 0$), 方程的通解为
\begin{equation}\label{Ode2_eq4}
y = C_1 \E^{rx} + C_2 x \E^{rx}
\end{equation}
\item 有两个纯虚数根 $\pm \I \omega_0$( $b = 0,\,\, b^2 - 4ac < 0$), 方程的通解为
\begin{equation}
y = C_1\cos(\omega_0 x) + C_2 \sin(\omega_0 x)
\end{equation}
或 
\begin{equation}
y = C_1\cos(\omega_0 x + C_2)
\end{equation} 
其中 $\omega_0 = \sqrt{c/a}$. 

\item 有两个复数根 $r \pm \I\omega$ ($b \ne 0,\,\, b^2 - 4ac < 0$), 方程的通解为
\begin{equation}\label{Ode2_eq7}
y = \E^{rx} [ C_1\cos(\omega x) + C_2\sin(\omega x) ]
\end{equation} 
或 
\begin{equation}
y = C_1\E^{rx}\cos(\omega x + C_2)
\end{equation} 
其中
\begin{equation}
r =  - \frac{b}{2a} \qquad \omega  = \frac{1}{2a}\sqrt{4ac - b^2} 
\end{equation}

%若令 ${\omega_0} = \sqrt {\frac{c}{a}} $,  $\gamma  = \frac{b}{2\sqrt {ac} }$,  则 $r =  - {\omega_0}\gamma $,  $\omega  = {\omega_0}\sqrt {1 - {\gamma ^2}} $. 满足 ${r^2} + {\omega ^2} = \omega_0^2$. 
\end{enumerate}

应用常微分方程解的存在唯一性定理, 即皮卡-林德勒夫定理\upref{PiLin}, 我们可以确认: \textbf{方程 (1) 的通解一定是上面四种情况之一.}

\subsection{详细推导}

情况 1 的结论是显然的, 我们先来看情况 3. 根据 $y = C\E^{rx}$ 的假设, 通解应该是
\begin{equation}
y = C_1 \E^{\I\omega x} + C_2 \E^{-\I\omega x}
\end{equation}
如果这里的 $C_1$ 和 $C_2$ 取任意复数, 那么上式就是方程在复数域的通解, 其中包含了实数域的通解. 这个通解还有另一种等效的形式, 令
\begin{equation}
C_1 = \frac{C_3}{2} + \frac{C_4}{2\I} \qquad C_2 = \frac{C_3}{2} - \frac{C_4}{2\I}
\end{equation}
代入上式得
\begin{equation}\ali{
y &= C_3 \frac{\E^{\I\omega x} + \E^{-\I\omega x}}{2} + C_4 \frac{\E^{\I\omega x} - \E^{-\I\omega x}}{2\I}\\
&=C_3 \cos(\omega x) + C_4 \sin(\omega x)
}\end{equation}
注意如果 $C_3, C_4$ 取任意复数, 该式仍然是复数域的通解(因为任何$C_1, C_2$ 都可以找到对应的 $C_3, C_4$), 但只要把 $C_3, C_4$ 限制在实数域中, 该式就是实数域的通解.

情况 4 的结论可以类比情况 3 得出, 最后我们来看情况 2. 我们可以把情况 2 看做情况 4 的一个极限, 即 $\omega \to 0$ 时的情况. 如果\autoref{Ode2_eq7} 中的 $C_1, C_2$ 都是普通常数, 则取该极限时可以得到\autoref{Ode2_eq4} 的第一项 $C_1 \E^{rx}$. 那如何得到第二项呢? 我们不妨令\autoref{Ode2_eq7} 中的 $C_1 = 0$, $C_2 = C_3/\omega$, 再来取极限, 得
\begin{equation}
\lim_{\omega\to 0} C_3 \E^{rx} \frac{\sin(\omega x)}{\omega x} x = C_3 x \E^{rx}
\end{equation}
这里用到了“小角正弦值极限\upref{LimArc}” 中的结论.

