% 约翰·伯努利(综述)
% license CCBYSA3
% type Wiki

本文根据 CC-BY-SA 协议转载翻译自维基百科\href{https://en.wikipedia.org/wiki/Johann_Bernoulli}{相关文章}。

\begin{figure}[ht]
\centering
\includegraphics[width=6cm]{./figures/040c46e497ce024e.png}
\caption{约翰·伯努利(约翰·鲁道夫·胡贝尔绘制的肖像,约1740年)} \label{fig_YHBNL_1}
\end{figure}
\textbf{约翰·伯努利}(Johann Bernoulli,也称为法语中的Jean或英语中的John;1667年8月6日(旧历7月27日)-1748年1月1日)是一位瑞士数学家,是伯努利家族中许多杰出数学家之一。他因在微积分学上的贡献以及在莱昂哈德·欧拉少年时期的教育而闻名。
\subsection{传记}  
\subsubsection{早期生活}  
约翰·伯努利出生于巴塞尔,父亲是药剂师尼古拉斯·伯努利,母亲是玛格丽特·施翁高尔。他在巴塞尔大学开始学习医学。他的父亲希望他学习商业,以便能够接管家族的香料贸易,但约翰·伯努利并不喜欢商业,最终说服父亲允许他改学医学。约翰·伯努利开始在课余时间与他的哥哥雅各布·伯努利一起学习数学。[5] 在巴塞尔大学的学习过程中,伯努利兄弟共同合作,花费大量时间研究新发现的微积分。他们是最早不仅学习和理解微积分,而且将其应用于各种问题的数学家之一。[6] 1690年,[7] 他完成了医学学位论文,[8] 由莱布尼茨审阅,[7] 论文题为《肌肉运动论与发酵与发泡现象的研究》[9]。
Here is the translation of the provided text:

**成年生活**  
从巴塞尔大学毕业后,约翰·伯努利开始教授微分方程。后来,在1694年,他娶了巴塞尔市议员的女儿多萝西亚·法尔克纳,并很快接受了格罗宁根大学数学教授的职位。应岳父的要求,伯努利于1705年开始返回家乡巴塞尔的旅程。就在出发后不久,他得知哥哥因肺结核去世。伯努利原本打算回到巴塞尔大学担任希腊语教授,但他最终接管了哥哥的数学教授职位。作为莱布尼茨微积分学派的学生,伯努利在1713年支持莱布尼茨,在莱布尼茨与牛顿关于谁应为微积分的发现负责的争论中站在了莱布尼茨一方。伯努利通过展示莱布尼茨用其方法解决了牛顿未能解决的某些问题来为莱布尼茨辩护。伯努利还推崇笛卡尔的漩涡理论,而非牛顿的万有引力理论。这最终延迟了牛顿理论在欧洲大陆的接受。[10]