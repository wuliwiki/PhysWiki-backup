% 函数的极限(简明微积分)

\pentry{数列的极限(简明微积分)\upref{Lim0}, 充分必要条件\upref{SufCnd}, 函数}

实函数 $f(x)$ 可以看成是一种 “连续” 的数列, 只不过把元素编号从离散的 $n$ 改为连续的 $x$. 类比数列的极限, 我们也可以定义\textbf{函数趋于正无穷的极限}.

\begin{definition}{函数趋于正无穷的极限}\label{FunLim_def1}
考虑实函数 $f(x)$. 若无论要求 $f(x)$ 和一确定实数 $A$ 的距离$\epsilon$ 有多小(但$\epsilon>0$), 都存在 实数$X$ ,使得所有$x>X$都满足$\abs{f(x)-A}<\epsilon$, 那么我们说 $A$ 是函数 $f(x)$ 在 $x$ 趋于正无穷时的极限, 记为
\begin{equation}
\lim\limits_{x\to +\infty} f(x) = A
\end{equation}
\end{definition}

可以看到该定义和数列极限的定义(\autoref{Lim0_def2}~\upref{Lim0})非常相似, 只是简单做了替换.不过,函数并不是简单地把数列的概念拓展到连续的情况. 数列的编号只能朝着一个方向增大, 但函数的自变量 $x$ 既可以趋近正无穷也可以奔向负无穷, 另外, 由于 $x$ 是连续取值的, 也可以考察自变量 $x$ 不断趋近某一点 $x_0$ 的极限, 即 $x\to x_0$.

\addTODO{画图, 画出函数曲线, 距离要求就是两条直线之间的范围, 等等}

\begin{exercise}{}
请仿照\autoref{FunLim_def1} 给出函数趋于负无穷时极限的定义
\begin{equation}
\lim\limits_{x\to -\infty} f(x) = A
\end{equation}
\end{exercise}

如何描述 “自变量趋于一个给定的实数 $x_0$” 呢? 只需要取自变量 $x$ 使得二者间的距离 $\abs{x-x_0}$ 越来越接近 $0$ 即可.

\begin{definition}{函数在某点的极限}\label{FunLim_def3}
考虑实函数 $f(x)$. 若无论要求 $f(x)$ 和确定实数 $A$ 的距离 $\epsilon>0$ 有多小, 都存在一个自变量的取值半径$\delta>0$,使得只要$\abs{x-x_0} < \delta$,就有$\abs{f(x)-\delta}<\epsilon$,
% 能通过不等式 $\abs{x-x_0} < \delta$ ($\delta$ 是一确定实数)使要求成立, 
那么我们说 $A$ 是函数 $f(x)$ 在 $x$ 趋于 $x_0$ 时的极限, 记为
\begin{equation}
\lim\limits_{x\to x_0}f(x)=A
\end{equation}
\end{definition}

\addTODO{画图, 画出函数曲线, 距离要求就是两条直线之间的范围, 等等}

\begin{example}{}
求一些简单的函数在某个值处的极限时, 通常可以直接代入数值计算, 如
\begin{equation}
\lim_{x\to 1} 2x + 1 = 3 \qquad \lim_{x\to 2}\frac{x + 1}{x + 2} = \frac34
\end{equation}

当无穷大与常数相加时, 可以忽略常数, 如
\begin{equation}
\lim_{x\to +\infty} \frac{x + 1}{2x + 2} = \lim_{x\to +\infty} \frac{x}{2x} = \frac12
\end{equation}
\end{example}

但注意 $x\to x_0$ 的极限并不要求函数 $f(x)$ 在 $x_0$ 这点有定义, 因为定义中只考虑 $x$ 慢慢接近 $x_0$ 的过程, 而不考虑 $x = x_0$ 的情况. 即使我们把这点从函数定义域中挖去, 极限是否存在, 以及极限值是多少都不会被改变. 例如以后在 “小角正弦极限\upref{LimArc}” 中会看到, 虽然 $\sin x/ x$ 在 $x = 0$ 处没有定义, 但其极限却等于 $1$.

我们还可以区分函数在某点的\textbf{左极限(left limit)}和\textbf{右极限(right limit)}. 简而言之就是 $x$ 分别从左边和右边两个方向趋近 $x_0$ 时的极限, 具体定义留做思考. 左右极限记为
\begin{equation}
\lim_{x\to x_0^-} f(x) = A_- \qquad \lim_{x\to x_0^+} f(x) = A_+
\end{equation}

\begin{example}{}
函数
\begin{equation}
\theta(x) = \leftgroup{
0 \qquad (x < 0)\\
1 \qquad (x \ge 0)
}\end{equation}
在 $x_0 = 0$ 处的左右极限分别为 $0$ 和 $1$, 但它在 $x_0 = 0$ 处不存在极限.
\end{example}

从直觉上容易得出一些定理(证明略)

\begin{theorem}{}
函数在某点存在极限的充分必要条件是它左右极限都存在并相等.
\end{theorem}

\begin{theorem}{}
若两个函数分别存在极限 $\lim_{x\to a} f(x)$ 和 $\lim_{x\to a} g(x)$ ($a$ 可取 $\pm \infty$), 那么有
\begin{equation}
\lim_{x\to a} [f(x) \pm g(x)] = \lim_{x\to a}f(x) \pm  \lim_{x\to a} g(x)
\end{equation}
\begin{equation}
\lim_{x\to a} [f(x) g(x)] = \lim_{x\to a}f(x) \lim_{x\to a} g(x)
\end{equation}
\begin{equation}
\lim_{x\to a} [f(x)/g(x)] = \lim_{x\to a}f(x)/\lim_{x\to a} g(x) \qquad (\lim_{x\to a} g(x) \ne 0)
\end{equation}
\end{theorem}
