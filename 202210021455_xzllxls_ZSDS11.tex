% 浙江理工大学 2011 年数据结构
% 2011年浙江理工大学991数据结构考研真题

\subsection{一、单选题}
在每小题的四个备选答案中选出一个正确答案,每小题3分,共45分.

1. 若线性表最常用的操作是存取第$i$个元素及其前趋的值,则采用(  )存储方式节省时间. \\
A.单链表 $\qquad$ B.双链表 $\qquad$ C.单循环链表 $\qquad$ D.顺序表

2. 设输入序列为$1$、$2$、$3$、$4$,则借助栈所得到的输出序例不可能是(  ) \\
A.1、2、3、4 \\
B.4、1、2、3 \\
C.1、3、4、2 \\
D.4、3、2、1

3. 常对数组进行的两种基本操作是(  ). \\
A.建立与删除 \\
B.插入与修改 \\
C.查找与修改 \\
D.查找与插入

4. 数组$Q[n]$用来表示个循环队列,$f$为当前队列头元素的前一位置, $r$为队尾元素的位置,假定队列中元素的个数小于$n$ ,计算队列中元素的公式为(  ) \\
A. r-f \\
B. (n+f-r)\%n \\
c. n+r-f \\
D. (n+r-f)\%n

5.广义表((a, b, C, d))的表尾是(  ) \\
A. a \\
B.( ) \\
C.(a,b,c, d) \\
D.((a,b,c,d))

6.实现任意二叉树的后序遍历的非递归算法而不使用栈结构,最佳方案是二叉树采用(    )存储结构. \\
A.三叉链表 \\
B.广义表存储结构 \\
C.二叉链表 \\
D.顺序表存储结构

7.在线索化二叉树中, $P$所指的结点没有左子树的充要条件是(    ). \\
A. P->left==null \\
B. P->ltag=1 \\
C. P->ltag==1且P->left==null \\
D. 以上都不对

8.稀疏矩阵一般的压缩存储方法有两种,即(    ). \\
A.二维数组和三维数组 \\
B.三元组和散列 \\
C.三元组和十字链表 \\
D.散列和十字链表

9.有$n$个结点的有向图的边数最多有(    ). \\
A. n \\
B. n(n-1) \\
C. n (n-1)/2 \\
D. 2n

10.带权有向图$G$用邻接矩阵$A$存储,则顶点$i$的入度等于$A$中(    ) \\
A.第$i$行非无穷元素之和 \\
B.第$i$列非无穷元素之和 \\
C.第$i$行非零且非无穷元素个数 \\
D.第$i$列非零且非无穷元素个数

11.采用邻接表存储的图的深度优先遍历算法类似于二叉树的_____. \\
A.先序遍历 \\
B.中序遍历 \\
C.后序遍历 \\
D.按层遍历

12.链表适用于()查找. \\
A.顺序 \\
B.二分法 \\
C.顺序,也能二分法 \\
D.随机

