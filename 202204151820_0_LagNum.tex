% 拉格朗日方程的数值解

\begin{issues}
\issueDraft
\end{issues}

\pentry{拉格朗日方程\upref{Lagrng}, 哈密顿正则方程\upref{HamCan}}

若程序中给出拉格朗日量的数值函数 $L(q, \dot q, t)$, 输入和输出均为数值(例如双精度实数), 那么如何数值求解运动方程呢?
\addTODO{写一个 Matlab 程序}

如果可以写出系统哈密顿量的表达式, 那么数值解哈密顿正则方程是首选的做法, 因为它本身已经是. 但对于较复杂的系统, 哈密顿量比拉格朗日量难算得多, 甚至可能没有解析表达式.



相比之下可以发现拉格朗日方程的数值解比哈密顿正则方程要难一些, 且用差分计算微分会带来一定的误差. 相比之下, 哈密顿方程已经是一阶常微分方程, 可以直接进行数值解. 

问题的关键是如何列出一个一阶常微分方程组.
\begin{equation}
\dv{t} \pdv{L}{\dot q_i} = \pdv{L}{q_i}
\quad (i=1,\dots,N)
\end{equation}
根据全微分, 左边有
\begin{equation}
\dv{t} \pdv{L}{\dot q_i} = \sum_j\pdv{q_j}\pdv{L}{\dot q_i}\dot q_j + \sum_j\pdv{\dot q_j}\pdv{L}{\dot q_i}\ddot q_j + \pdv{t}\pdv{L}{\dot q_i}
\end{equation}
代入得
\begin{equation}
\sum_j\pdv{\dot q_j}\pdv{L}{\dot q_i}\ddot q_j = \pdv{L}{q_i} - \sum_j\pdv{q_j}\pdv{L}{\dot q_i}\dot q_j - \pdv{t}\pdv{L}{\dot q_i}
\end{equation}
这样, 二阶导数只存在于左边的 $\ddot q_j$, 其他项都只是 $q,\dot q, t$ 的函数. 该式中的偏微分全部可以通过差分来计算. 数值解线性方程, 就可以得到(令 $v_i = \dot q_i$)
\begin{equation}
\leftgroup{
&\dot v_i = f_i(q, v_i, t)\\
&\dot q_i = v_i
}\qquad (i = 1,\dots,N)
\end{equation}
就得到了 $2N$ 条方程组成的一阶常微分方程组, 变量一共有 $2N$ 个. 可以使用四阶龙格库塔法\upref{OdeRK4} 等方法求解.
