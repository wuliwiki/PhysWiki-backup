% 麦克斯韦方程组(综述)
% license CCBYSA3
% type Wiki

本文根据 CC-BY-SA 协议转载翻译自维基百科\href{https://en.wikipedia.org/wiki/Maxwell\%27s_equations}{相关文章}。

\textbf{麦克斯韦方程组},或称\textbf{麦克斯韦–赫维赛德方程组},是一组耦合偏微分方程,与洛伦兹力定律一起构成了经典电磁学、经典光学、电路和磁路的基础。这些方程为电学、光学和无线电技术(如发电、电动机、无线通信、透镜、雷达等)提供了数学模型。它们描述了电场和磁场如何由电荷、电流及场的变化产生。[注1] 这些方程以物理学家和数学家詹姆斯·克拉克·麦克斯韦的名字命名,他在1861年和1862年首次发表了包含洛伦兹力定律的早期方程形式。麦克斯韦最早使用这些方程提出光是一种电磁现象。方程的现代形式及其最常见的表述归功于奥利弗·赫维赛德。[1]