% 复变函数的导数 柯西—黎曼条件
% 柯西|黎曼|导数

\begin{issues}
\issueDraft
\end{issues}

\pentry{复变函数\upref{Cplx}, 全微分\upref{TDiff}}

类似实函数, 定义复变函数 $w = f(z)$ ($z$ 和函数值都是复数)的导数为
\begin{equation}\label{CauRie_eq4}
f'(z) = \lim_{h\to 0} \frac{f(z + h) - f(z)}{h}
\end{equation}
其中 $h$ 也是一个复数. 极限 $h \to 0$ 在这里是指 $h$ 可以沿复平面上的任意方向趋近于 $0$ 都得到同一个极限值. 对于实变量的函数, 只有正负两个方向, 而复平面上有无穷多个方向.

复平面上, 把 $f(z)$ 看作两个实函数, 令
\begin{equation}
z = x + y\I
\end{equation}
\begin{equation}
f(z) = u(x, y) + \I v(x, y)
\end{equation}
要使\autoref{CauRie_eq4} 的导数存在, 除了要求 $u, v$ 函数在一定区域内处处存在偏导, 还要满足\textbf{柯西—黎曼条件(Cauchy-Riemann condition)}, 即
\begin{equation}\label{CauRie_eq1}
\pdv{u}{x} = \pdv{v}{y} \qquad
\pdv{u}{y} = - \pdv{v}{x}
\end{equation}

\subsubsection{推导}
根据全微分\upref{TDiff}
\begin{equation}\label{CauRie_eq2}
\dd{z} = \dd{u} + \I \dd{v}
\end{equation}
其中
\begin{equation}\label{CauRie_eq3}
\dd{u} = \pdv{u}{x} \dd{x} + \pdv{u}{y} \dd{y} \qquad
\dd{v} = \pdv{v}{x} \dd{x} + \pdv{u}{y} \dd{y}
\end{equation}
如果直接将该式代入 $\dv*{w}{z} = (\dd{u} + \I \dd{v})/(\dd{x} + \I \dd{y})$ 会发现结果和 $\dd{y}/\dd{x}$ 有关, 即与\autoref{CauRie_eq4} 中 $h$ 趋近于零点的方向有关. 所以我们换一种方式. 令
\begin{equation}
c = \dv{z}{w} = a + b\I
\end{equation}
写成微分形式
\begin{equation}
\dd{z} = c\dd{w}
\end{equation}
\begin{equation}
\dd{u} + \I \dd{v} = (a + b\I)(\dd{x} + \I\dd{y}) = (a \dd{x} - b\dd{y}) + \I (b \dd{x} + a\dd{y})
\end{equation}
该式对比\autoref{CauRie_eq2} 和\autoref{CauRie_eq3}, 得
\begin{equation}
a = \pdv{u}{x} = \pdv{v}{y} \qquad
b = -\pdv{u}{y} = \pdv{v}{x}
\end{equation}
这样, 不仅得到了柯西—黎曼条件, 也得到了导数的表达式.

% 未完成, 举例子, 例如 exp, z^2, sin 等都满足柯西—黎曼

\subsection{调和场}
\pentry{调和场\upref{HarmF}}

柯西—黎曼条件的充分必要条件是, $(u, -v)$ 是二维调和场(散度和旋度都为零), 所以存在势函数 $f$, 使
\begin{equation}
u = \pdv{f}{x} \qquad -v = \pdv{f}{y}
\end{equation}

(未完成)
