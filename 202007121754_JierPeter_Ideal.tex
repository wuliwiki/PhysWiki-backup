% 环的理想
\pentry{环\upref{Ring}}

\subsection{概念的来源}

环$R$的理想$I$,是一种“正规子环”,即它是$R$的子环,同时使得商集$R/I$能自然成环,就像正规子群的作用一样.之所以不叫正规子环,是因为理想最初来自代数数论,库默尔(Ernst Eduard Kummer)定义了一个他称为“理想数”的概念,证明了费马大定理在$n<100$时大多数情况成立.后来,戴德金(Julius Wilhelm Richard Dedekind)发现,库默尔定义的理想数,正是能诱导出商环的“正规子环”,于是直接借用了“理想数”的名字,将其命名为“理想”.

给定环$R$和它的子环$I$,那么$I$要满足什么条件才能使得$R/I$成环呢?

显然,$I$关于环的加法,得构成一个正规子群,而这是天然满足的,因为环的加法群是阿贝尔群,而阿贝尔群的一切子群都是正规子群.

$R/I$中的每个元素,被定义为$I$作为子群的陪集.元素$a\in R$所在的陪集就是$a+I$.陪集之间显然可以进行加法运算:
\begin{equation}
(a+I)+(b+I)=a+b+I+I=(a+b)+I
\end{equation}

这满足诱导运算的要求:$a$的陪集加$b$的陪集,等于$a$加$b$的陪集.

为了让$R/I$诱导一个环乘法,我们还需要:$a$的陪集乘$b$的陪集,等于$a$乘$b$的陪集.也就是说,
\begin{equation}

\end{equation}


