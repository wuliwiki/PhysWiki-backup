% 导数的性质与构造(高中)
% keys 导数|性质|构造|恒等
% license Usr
% type Tutor

\begin{issues}
\issueDraft
\end{issues}

\subsection{近似代替}

在导数的\aref{几何含义}{sub_HsDerv_1}中就提到过“以直代曲”。

\begin{equation}
f(x_0+\Delta x)\approx f(x_0)+f'(x_0)\Delta x~.
\end{equation}

\subsection{单调性和极值点}

在介绍\aref{导函数}{sub_HsDerv_2}时,提及区间的中函数的增减与导函数的符号相关。
\begin{theorem}{单调性与导数的关系}
导函数  $f'(x)$  代表了原函数  f(x)  图像在每一点的切线斜率。
\begin{itemize}
\item 在$f'(x)>0$的区间上,原函数的图像单调递增。
\item 在$f'(x)<0$的区间上,原函数的图像单调递减。
\item 当$f'(x) = 0$的区间上,原函数的图像是水平的。
\end{itemize}
\end{theorem}

$f'(x) = 0$的点表示函数的输出值停止增加或减少的点,被称为\textbf{驻点}。在该点是水平的,可能是极值点。


\subsection{高阶导数}

导函数作为原函数,则又可以求得它的导函数,这也被称为高阶导数。

凹凸性

\subsection{常用构造}