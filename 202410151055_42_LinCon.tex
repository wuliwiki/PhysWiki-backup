% 线性连续泛函
% keys 线性|连续|泛函
% license Usr
% type Tutor

\pentry{拓扑向量空间\nref{nod_tvs},泛函与线性泛函\nref{nod_Funal}}{nod_5b23}
拓扑线性(或拓扑向量)空间的拓扑空间性质表明其上映射的连续性具有基本的重要性,而其线性空间性表明线性映射具有基本的重要性。因此,拓扑线性空间上的线性连续映射具有基本的重要性。特别,对拓扑线性空间上的泛函,线性连续泛函具有基本的重要性。
\begin{definition}{连续}
设 $f$ 是拓扑线性空间 $E$ 上的\enref{线性泛函}{Funal}。若对任意 $x_0\in E,0<\epsilon\in\mathbb R$,存在 $x_0$ 的邻域 $U$,使得 $x\in U$ 就有
\begin{equation}
\abs{f(x)-f(x_0)}<\epsilon,~
\end{equation}
则称 $f$ 是\textbf{线性连续泛函}。
\end{definition}

一般定义映射的连续,往往是先定义映射在一点的连续。然而上面的定义是对每一点都连续,而没有事先定义在一点的连续。事实上,在拓扑线性空间中,线性映射在一点连续必定在全空间连续。这由下面的定理指定。
\begin{theorem}{在一点处连续的线性泛函处处连续}
设
\end{theorem}


