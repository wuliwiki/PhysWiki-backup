% 最小二乘法拟合函数(Matlab)
% keys 最小二乘法|数值拟合|极小值
% license Xiao
% type Tutor

\pentry{最小二乘法\upref{LstSqr}, Nelder-Mead 算法\upref{NelMea}}
我们在 “最小二乘法\upref{LstSqr}” 见到的三个例子中, 方差函数都是待定系数的线性组合, 这种情况下我们令偏导为零后得到的是线性方程组, 便于求解 %(见 Matlab 或 Mathematica)链接未完成。 
。 然而当方差不是待定系数的线性组合时, 得到的方程组往往非常复杂, 这时就需要借助数值计算。 相比用数值计算解 $N$ 元的非线性方程组, 更简单的方法是直接用数值方法寻找方差函数的极小值(如Nelder-Mead 算法) 。 实践证明, 大多数情况下极小值点仅有一个, 那就是最小值点。

为了验证结果的正确性, 我们先来用数值方法拟合 $A\cos (x + \varphi_0) + C$, 并与 “最小二乘法\upref{LstSqr}” 中的方法比较结果。

\begin{figure}[ht]
\centering
\includegraphics[width=8cm]{./figures/b15d27bff0f69e5e.pdf}
\caption{运行结果} \label{fig_CurFit_1}
\end{figure}

Matlab 代码如下, 需要 \verb|NelderMead.m| 函数\upref{NelMea}:
\begin{lstlisting}[language=matlab, caption=curveFit\_test.m]
function curveFit_test
close all;
% 随机生成简谐曲线
N = 20;
x0 = linspace(0, 2*pi, N);
y0 = 5*rand * sin(x0 + 2*pi * rand) + 10 * rand;
y0 = y0 + 2*rand(1,20); % 产生随机误差
scatter(x0, y0, '+'); % 画出散点
hold on;

% Nelder-Mead 求方差最小值点
f = @(x) norm( x(1)*sin(x0 + x(2)) + x(3) - y0 )^2;
c = NelderMead(f, [5, 1, 5], 1e-7);
% 画拟合结果
x = linspace(0, 2*pi, 50);
y1 = c(1) * sin(x + c(2)) + c(3);
plot(x, y1);

% 解线性方程组求方差最小值点
c = sinfit(x0, y0);
% 画拟合结果
y2 = c(1)*cos(x) + c(2)*sin(x) + c(3);
plot(x, y2, '.');
end

% 拟合简谐曲线
% y = C(1)*cos(x) + C(2)*sin(x) + C(3)
function C = sinfit(x, y)
N = numel(x);
cosx = cos(x); sinx = sin(x);
sc = sum(sinx.*cosx);
s = sum(sinx); c = sum(cosx);
% 系数矩阵
M = [sum(cosx.^2), sc          , c;
               sc, sum(sinx.^2), s;
                c,            s, N];
b = [sum(y.*cosx); sum(y.*sinx); sum(y)];
C = M\b; % 解线性方程组
end
\end{lstlisting}

运行结果如\autoref{fig_CurFit_1} 所示, 可见两种方法拟合出的曲线一致(红色的曲线和黄色的点线)。 注意第 13 行使用了“Nelder-Mead 算法\upref{NelMea}” 中的函数 \verb|NelderMead| 求函数句柄 \verb|f| 的最小值。

一般参数函数的最小二乘法拟合代码为
\begin{lstlisting}[language=matlab, caption=curveFit.m]
% 最小二乘法拟合一元实函数,(x0,y0) 是数据点
% fun 是函数句柄, 格式 y = fun(x, p), 需要支持矢量 x 输入
% p0 是函数初始参数, double 矢量
% 输出最优的 p 以及(局部)最小方差 s
% 使方差 sum(abs(fun(x0, p) - y0).^2) 最小
% p_err 是 NelderMead 算法中 p 的绝对误差
% 返回方均差 err2

function [p, err2] = curveFit(x0, y0, fun, p0, p_err)
% Nelder-Mead 求方差最小值点
f = @(p) sum((fun(x0, p) - y0).^2);
[p, sum_s2] = NelderMead(f, p0, p_err);
err2 = sum_s2 / numel(x0);
end
\end{lstlisting}

\subsection{例子: 拟合高斯函数}
\begin{figure}[ht]
\centering
\includegraphics[width=10cm]{./figures/837874ae25f75b6b.png}
\caption{高斯函数\upref{GausPD} $A\exp[-(x-x_m)^2/(2\sigma^2)]$ 的拟合} \label{fig_CurFit_2}
\end{figure}

\begin{lstlisting}[language=matlab,ca]
% fit a gaussian distribution
% return mean square error
function [A, xmid, sigma, s2] = fit_gaussian(x0, y0, p_err)
    fun = @(x, p) p(1) * exp(-(x-p(2)).^2 / (2*p(3)^2));
    [A, ind] = max(y0);
    xmid = x0(ind);
    sigma = abs(x0(end)-x0(1));
    p0 = [A, xmid, sigma];
    [p, s2] = curveFit(x0, y0, fun, p0, p_err);
    A = p(1); xmid = p(2); sigma = p(3);
end
\end{lstlisting}

\begin{lstlisting}[language=matlab,caption=fit\_gaussian\_test.m]
% === params ===
N = 10;
A = 1.23; xmid = 0.12; sigma = 1.23;
% ==============

x = linspace(x0(1), x0(end), 500);
x0 = linspace(-6, 6, N);
y0 = A*exp(-(x0-xmid).^2/(2*sigma^2));
y0_with_err = y0 + 0.2*(rand(1, N)-0.5);
[A_fit, xmid_fit, sigma_fit, s2] = fit_gaussian(x0, y0_with_err, 1e-10);

figure;
plot(x, A*exp(-(x-xmid).^2/(2*sigma^2)), 'k');
hold on;
plot(x, A_fit*exp(-(x-xmid_fit).^2/(2*sigma_fit^2)), 'r--');
plot(x0, y0_with_err, 'k.');
grid on;
xlabel x; ylabel y;
title("mean sqr error = " + mean(s2));
legend({'original', 'fitted', 'data'});
\end{lstlisting}
