% 一维有限深方势阱
% 束缚态|能级|薛定谔方程|定态

\begin{issues}
\issueTODO
\end{issues}

% 束缚态的平均动量为零

% 这是区分束缚态(E<0) 和连续态(E>0)的最简单例子了。
% 当我们有势阱时, 都会有这种规律, 例如氢原子。

\pentry{定态薛定谔方程\upref{SchEq}}

\footnote{参考 Wikipedia \href{https://en.wikipedia.org/wiki/Finite_potential_well}{相关页面} 以及\cite{GriffQ}。}本文使用原子单位制\upref{AU}。 一维定态薛定谔方程\autoref{SchEq_eq1}~\upref{SchEq}为
\begin{equation}
-\frac{1}{2m}\dv[2]{x}\psi(x) + V(x) \psi(x) = E \psi(x)
\end{equation}

%\addTODO{图: 势能函数图, 左中右三个部分分别标注 1,2,3 区间}
\begin{figure}[ht]
\centering
\includegraphics[width=7cm]{./figures/FSW_1.pdf}
\caption{有限深方势阱} \label{FSW_fig1}
\end{figure}
令势能函数为
\begin{equation}
V(x) = \begin{cases}
-V_0 \quad &(-L/2 \leqslant x \leqslant L/2)\\
0 \quad &(\text{其他})
\end{cases}
\end{equation}
该势能叫做\textbf{有限深势阱(finite square well)}。

有限深势阱既包含连续的本征态(散射态), 一定包含有限个离散的束缚态(证明见下文)。 有限深势阱是研究一维散射问题的一个简单模型。 在这个问题中, 束缚态的 $E<0$, 散射态的 $E>0$。

\subsection{束缚态}
对于束缚态,$E<0$,而且 $E> V_{min}=-V_0$。这可以用反证法来证明:如果 $E<V_{min}$,那么由定态薛定谔方程 $\frac{\dd {^2}\psi}{\dd x^2}=2m[V(x)-E]\psi$ 就会得出 $\psi$ 和它的二次导数符号相同,这种情况下波函数是不可归一化的。因此我们有 $-V_0<E<0$。

由于 $V(x)$ 是对称的,所以不失一般性,我们假设波函数是奇函数或偶函数来简化问题(\autoref{SchEq_eq3}~\upref{SchEq})。令
\begin{equation}\label{FSW_eq4}
\kappa = \sqrt{-2mE}, \qquad k = \sqrt{2m(E + V_0)}
\end{equation}
第 1,3 区间的通解为 (在这个区间上 $V(x) > E$)
\begin{equation}
\psi(x) = C_1 \E^{\kappa x} + C_2 \E^{-\kappa x}
\end{equation}
为了让无穷远处波函可归一化, 所以
\begin{equation}
\psi_1(x) = A \E^{\kappa x}, \qquad \psi_3(x) = D\E^{-\kappa x}
\end{equation}
第 2 区间的通解为 (在这个区间上 $V(x) < E$)
\begin{equation}
\psi_2(x) = B \cos(k x) + C\sin(k x)
\end{equation}

\subsubsection{奇波函数}
当波函数为奇函数时, 易得 $\psi(0) = 0$ 即 $B = 0$, 且 $A = -D$。 再考虑 $x = L/2$ 处波函数及一阶导数连续有
\begin{equation}\label{FSW_eq1}
\begin{aligned}
&C\sin(k L/2) = D \exp(-\kappa L/2)\\
&k C \cos(k L/2) = -\kappa D \exp(-\kappa L/2)
\end{aligned}
\end{equation}
其中可以把 $E$ 看成未知量, 决定 $\kappa, k$, $C,D$ 也是未知量。 两式相除得
\begin{equation}\label{FSW_eq2}
-\cot(k L/2) = \kappa /k
\end{equation}
% \addTODO{图未完成, \autoref{FSW_eq2} 和\autoref{FSW_eq3} 的图解超越方程,参考 griffiths 的图}
这是一个超越方程, 可能存在解。 解出后再次代入\autoref{FSW_eq1} 可以求得比值 $D/C$。归一化即可确定 $C, D$。现在我们尝试从图像上考察它可能的解。

利用 \autoref{FSW_eq4} 的关系,将 $\kappa$ 用 $k$ 表示,可以得到

\begin{equation}\label{FSW_eq5}
-\cot(k L/2)=\sqrt{2mV_0/k^2-1}
\end{equation}

设 $z=k L/2$,我们可以在同一坐标系中绘制 $-\cot(x)$ 和 $\sqrt{(z_0/z)^2-1}$ 的图像($z_0=L\sqrt{2mV_0}/2$,表征了势阱“大小”),两个图像的每一个交汇点都对应着一个奇函数解。\autoref{FSW_fig2} 显示了 $z_0=7$ 的情况,可以看到两个图像共有 $2$ 个交点。

\begin{figure}[ht]
\centering
\includegraphics[width=12cm]{./figures/FSW_2.png}
\caption{方程 $-\cot z=\sqrt{(z_0/z)^2-1}$ 的解,$z_0=7$ 的情况(奇态)} \label{FSW_fig2}
\end{figure}

对于宽深势阱($z_0=L\sqrt{2mV_0}/2$ 很大),两个图像一定有交。特别是当 $z_0\rightarrow\infty$ 时,交汇点的位置都可以近似为 $n\pi$($n$ 为正整数)。我们将在下面讨论偶波函数的时候具体分析这种情况。对于浅窄势阱,值得注意的是,当 $z_0<\pi/2$ 时无解,这可以从图像中看出。但此时仍有偶函数解,也就意味着至少存在一个束缚态。这将在后面进行讨论。

\subsubsection{偶波函数}
当波函数为奇函数时, 易得 $\psi'(0) = 0$ 即 $C = 0$, 且 $A = D$。 与奇函数的情况同理得
\begin{equation}
\begin{aligned}
&B\cos(k L/2) = D \exp(-\kappa L/2)\\
&k B \sin(k L/2) = \kappa D \exp(-\kappa L/2)
\end{aligned}
\end{equation}
相除得
\begin{equation}\label{FSW_eq3}
\tan(k L/2) = \kappa /k
\end{equation}
利用 \autoref{FSW_eq4} 的关系,将 $\kappa $ 用 $k$ 表示,可以得到

\begin{equation}
\tan(k L/2)=\sqrt{2mV_0/k^2-1}
\end{equation}
设 $z=k L/2$,我们可以在同一坐标系中绘制 $\tan(x)$ 和 $\sqrt{(z_0/z)^2-1}$ 的图像($z_0=L\sqrt{2mV_0}/2$,表征了势阱“大小”),两个图像的每一个交汇点都对应着一个偶函数解。\autoref{FSW_fig3} 显示了 $z_0=7$ 的情况,可以看到两个图像共有 $3$ 个交点。

\begin{figure}[ht]
\centering
\includegraphics[width=12cm]{./figures/FSW_3.png}
\caption{方程 $\tan z=\sqrt{(z_0/z)^2-1}$ 的解,$z_0=7$ 的情况(偶态)} \label{FSW_fig3}
\end{figure}

现在来考虑两种情况:

\textbf{1. 宽深势阱。}即 $z_0=L\sqrt{2mV_0}/2$ 非常大。那么由图像可观察出,交汇点在略小于 $z_n=n\pi/2$($n$ 为奇数)的位置。所以有

\begin{equation}
E_n+V_0\approx \frac{\pi^2}{2mL^2}n^2\quad (n=1,3,5\cdots)
\end{equation}

当 $V_0\rightarrow \infty$ 时,有限深势阱转化为无限深势阱,上式右侧刚好是一维无限深势阱\upref{ISW}的能级(只包括了其中的一半,因为 $n$ 是奇数。另一半来自于奇函数)。当 $V_0$ 有限时,仅有有限多个束缚态。

\textbf{2. 浅窄势阱。}当 $z_0$ 降低时,束缚态越来越少,直到最后当 $z_0<\pi/2$ 时,仅存在一个束缚态。但无论势阱多么“浅小”,总是至少存在一个束缚态,这可以由图像看出。

\subsection{散射态}
% 已画图验证

散射态的计算只需要把 “方势垒\upref{SqrPot}” 中 $E > V_0$ 的情况下取 $V_0 < 0$ 即可, 这里不再赘述。
