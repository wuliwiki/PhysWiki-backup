% 环同态
% 环|同态|环同态|环同态基本定理

\pentry{环的理想和商环\upref{Ideal},}

环同态的概念和群同态是类似的,只不过我们需要兼顾两个运算的性质.


\begin{definition}{环同态}
给定环$R_1$和$R_2$,如果存在映射$f:R_1\rightarrow R_2$,使得对于任意的$r, s\in R_1$都有$f(r)+f(s)=f(r+s)$和$f(rs)=f(r)f(s)$,那么我们称$f$是一个从$R_1$到$R_2$的\textbf{环同态映射(ring homomorphism)},而环$R_1$和$R_2$是\textbf{同态(homomorphic)}的环.
\end{definition}

\begin{definition}{环同构}
如果一个环同态是双射,那么我们称它为一个\textbf{环同构(ring isomorphism)},相应的两个环就是\textbf{同构(isomorphic)}的.
\end{definition}

同构的两个环具有完全相同的运算结构,而同态的两个环的运算结构似而不同,和群的同态、同构类似.

\subsection{环同态基本定理}\pentry{群同态基本定理:\autoref{Group2_exe1}~\upref{Group2}}

\begin{definition}{核}
给定环同态$f:R_1\rightarrow R_2$,记$\opn{ker}f=\{r\in R_1|f(r)=0\}$,称为同态$f$的\textbf{核}.
\end{definition}


和群同态基本定理类似,我们有如下环同态基本定理:

\begin{definition}{环同态基本定理}
给定环同态$f:R_1\rightarrow R_2$,则
\end{definition}





