% 东南大学 2012 年 考研 量子力学
% license Usr
% type Note

\textbf{声明}:“该内容来源于网络公开资料,不保证真实性,如有侵权请联系管理员”

\textbf{1.(15 分)}以下叙述是否正确:(1)电子的自旋态空间是3维的:(2)全同玻色子体系的波函数具有交换反对称性;(3)时间反演对称性导致能量守恒;(4)三维各向同性谐振子的所有能级均是非简并的;(5)处于中心力场中的无自旋单粒子的角动量一定是守恒量。

\textbf{2.(15 分)}质量为 $m$ 的粒子处于以下势阱中:
\[
V(x) = m \omega^2 x^2/2, \quad (x > 0);~
\]
\[
V(x) = \infty, \quad (x < 0)~
\]
试求能量本征值。

\textbf{3.(15 分)}质量为 $m$ 的粒子以能量 $E > 0$ 从左入射,碰到势 $V(x) = \gamma \delta(x) (\gamma > 0)$。

\begin{enumerate}
    \item 试用公式
    \[
    j(x,t) = -(i\hbar/2m)(\psi^* \partial \psi/\partial x - \psi \partial \psi^*/\partial x)~
    \]
    求入射几率流密度 $j_i$,反射几率流密度 $j_r$,透射几率流密度 $j_t$ 的表达式;
    
    \item 试证明波函数 $\psi$ 满足
    \[
    \psi'(0^+) - \psi'(0^-) = (2m\gamma/\hbar^2)\psi(0);~
    \]
    
    \item 求透射系数 $t$。
\end{enumerate}

\textbf{4.(15分)}试利用测不准关系估算:(1)一维谐振子的基态能:(2)氢原子的基态能。


\textbf{5.(15分)}


\textbf{6.(15分)}