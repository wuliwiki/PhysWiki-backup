% 如何自学物理
% license CCBYSA3
% type Tutor

\begin{issues}
\issueDraft
\end{issues}

要谈论这个话题,就不得不提荷兰诺贝尔物理学奖得主 Gerard 't Hooft 的文章 \href{https://webspace.science.uu.nl/~hooft101/theorist.html}{How to become a GOOD Theoretical Physicist} (\href{https://xialab.pku.edu.cn/kytdyw1/zdylm.m.jsp?wbtreeid=1011&tstreeid=11956&_t_uid=2945&language=en&homepageuuid=BF649325C5584FC683CE0B601D21AC65&templateuuid=4CC182410BA14FF8B55ED726FB2087FB&producttype=0&_tmode_=99&tsitesapptype=zdylm}{一个翻译}), 我们不妨就这篇文章的核心内容展开讨论。

\subsection{物理学的主要研究方向}
首先,虽然文章标题说的是 “理论物理学家”, 但大致来说也\textbf{同样适用于物理学其他研究方向}(除了推荐书目中较为高深的内容,基础部分无论对什么方向都是适用的)。 对于公众来说,可能一谈到物理学马上就会想到那些喜闻乐见的理论物理话题例如宇宙、黑洞、弦论等。 这些话题的确非常引人入胜,但远非物理学的全部。 事实上只有一少部分从事物理学研究的人会研究这些领域。

\begin{figure}[ht]
\centering
\includegraphics[width=14.25cm]{./figures/b898880f835f999d.png}
\caption{美国一项调查中物理博士的研究领域占比(参考\href{https://ww2.aip.org/statistics/trends-in-physics-phds}{来源})调查中的,从上到下分别是凝聚态物理、粒子和场、天体物理/宇宙学、原子分子光学、生物物理、核物理、材料/纳米/表面、光学/光子学、计算物理、等离子/聚变物理、应用/工程/能源研究、量子基础/信息理论、复杂系统/统计/非线性/热物理、相对论/引力、软物质/聚合物物理、其他} \label{fig_SdyPhy_1}
\end{figure}

\subsection{关于民间科学家}
\textbf{民间科学家}简称\textbf{民科},原本是指不在相关研究机构学习或工作但却仍自行从事科学研究的人。 但许多情况下,这个词具有贬义,指的是自行发明一些错误理论的科学爱好者。 数学和物理学都是民科的高发地,许多喜欢独立思考的人如果被某个门槛(例如英语、数学、或正规本科教育)卡住而无法学习从事物理学研究的基本方法,则很可能因为对科普内容或生活经验等过度思考而误入歧途,成为所谓的民科。 “思而不学则殆”, “吾尝终日而思矣,不如须臾之所学也”,以及王阳明的 “格竹” 都是古人对过度思考而不学习的警示。 作为诺奖获得者, Hooft 自然也会经常被各种民科用他们 “一无是处” 的理论骚扰,于是在上面的文章之后,他又写了一片讽刺民科的文章:

\href{https://webspace.science.uu.nl/~hooft101/theoristbad.html#:~:text=On\%20your\%20way\%20towards\%20becoming,have\%20your\%20work\%20published\%20anyway.}{How to become a bad theoretical physicist}(\href{https://zhuanlan.zhihu.com/p/38680467}{一个翻译})

完全的 “灌输式教育” 固然不可取,但现代物理学俨然已经是一座摩天大楼了, 只有熟练掌握相当多的基础知识才有可能为其添砖加瓦,所以若为了 “独立思考” 而走向另一个极端,企图去重建这个大厦是更危险的。

另一方面,即使你天赋过人可以从头建立一套更好的物理学,那么为了让现有的物理学家愿意了解你的成果而不把你误会成其他民科,你也要先精通他们已经掌握了了什么样的理论,它还有哪些缺陷,才能论证你的理论是否比现有的更好。 注意这里的了解并不是从文字的角度了解,而是了解物理理论本来的面貌——它们往往使用数学来表达。 例如你可能听说过广义相对论说 “引力来自于时空扭曲”, 但如果你不知道描述时空扭曲的数学描述——黎曼几何的度量张量,这句话就不知所云。

\subsection{英语}
在进一步讲解首先提到的就是英语,可见英语的重要性。 注意 Hooft 博士这篇文章是面向全世界读者的,所以对中国人也同样适用。 英语是事实上的国际学术交流语言。 在物理领域(以及其他大部分自然科学领域),绝大多数论文都是用英语在欧美期刊上发表的,包括国内的研究者。 哪怕是为数较少的国产 SCI 期刊,也多数是英语或双语的。 也就是说,即使是国内的物理研究者之间想要互相了解研究进展,也一般需要读英语论文。

即使抛开学术期刊只谈物理科普和教学方面的网络资源(其他自然科学也基本适用),英语内容无论是数量还是质量都要\textbf{远超中文}。 所以如果你用中文在百度搜索感兴趣的话题但没发现什么有价值的资源,你要意识到这远非互联网上的全部内容。 正如 Hooft 博士所说,你成为理论物理学家所需的一切都可以在互联网上找到, 但关键是你需要分辨哪些是真正对你有用的。 具体到语言,他所说的这些资源几乎都是以英语书写的。

所以无论是学习还是科研,一定程度的英语水平是进入物理领域的门槛, 多早开始学英语(尤其是物理方面的)都不为过。

\begin{figure}[ht]
\centering
\includegraphics[width=8cm]{./figures/636ff07a6af97580.png}
\caption{附:世界网站数量的语言分布,中文仅略高于印度尼西亚(来源:\href{https://w3techs.com/technologies/overview/content_language}{W3Tech})} \label{fig_SdyPhy_2}
\end{figure}

\subsection{数学}
\addTODO{待续}
