% test

\subsection{7-1 图的基本概念}
%\subsection{测试}
%\subsubsection{测试}
\begin{definition}{}
一个图是一个三元组$G=<V(G),E(G),\varphi_G>$,其中$V(G)$是一个非空的节点结合,$E(G)$是边集合,是从边集合$E$到节点无序偶(有序偶)集合上的函数.
如果把图中的边$e_i$看作总是与两个节点关联,那么一个图亦可以简记为$G=<V,E>$,其中$V$是非空结点集,$E$是连接结点的边集.\\
若边$e_i$与结点无序偶$(v_i,v_j)$相关联,则称改边为无向边.\\
若边$e_i$与结点有序偶$<v_i,v_j>$相关联,则称改边为有向边.
\end{definition}
\begin{definition}{}
在图$G=<V,E>$中,与结点$v(v \in V)$关联的边数,称作是该结点度数,记做$deg(v)$.
\end{definition}
\begin{theorem}{}
每个图中,结点度数的总和等于边数的两倍.
$$\sum\limits _ { v \in V } \operatorname { deg } ( v ) = 2 | E |$$
\end{theorem}
\begin{theorem}{}
在任何图中,度数为奇数的结点必是偶数个.
\end{theorem}
\begin{definition}{}
在有向图中,射入一个结点的边数成为该节点的入度,有一个结点射出的边数称为该结点的出度.出度与入度之和为该结点的度数.
\end{definition}
\begin{definition}{}
含有平行边的任何一个图称为多重图.
不含平行边和环的图称为简单图.
\end{definition}

\begin{definition}{}
简单图$G=<V,E>$中,若每一对结点间都有边,则称该图为完全图.
有$n$个结点的完全图称为$K_n$.
\end{definition}
\begin{definition}{}
给定一个图$G,补图记做\overline { G }$.
\end{definition}


\begin{definition}{}
同构$G \simeq G ^ { \prime }$
\end{definition}
\subsection{7-2 路与回路}
\begin{theorem}{}
在具有$n$个结点的图中,如果从结点$v_j$到结点$v_k$存在一条路,则从结点$v_j$到结点$v_k$存在不多于$n-1$条边的路.
\end{theorem}
\begin{definition}{}
设无向图$G=〈V,E)$为连通图,若有点集$V_1 \in V$
,使图$G$删除了$V_1$的所有结点后,所得的子图是不连通图,而删
除了$V_1$的任何真子集后,所得到的子图仍是连通图,则称$V_1$是$G$
的一个点割集.若某一个结点构成一个点割集,则称该结点为割点.

\end{definition}
若G不是完全图,我们定义$k(G)=min \{ |V1|\ | V1$是$G$的点割集$\}$为G的点连通度(或连通度).



连通度$k(G)$是为了产生一个不连通图需要删去的点的最少数目.于是一个不连通图的连通度等于$0$,存在割点的连通图其连通度为$1$.完全图$K_p$中,删去任何$m$个$(m<p-1)$点后仍是连通图,但是删去了$p-1$个点后产生了个平凡图,故定义$k(K_p)=p-1$.
\begin{definition}{}
设无向图$G=〈V,E)$为连通图,若有边集$E_1 \in V$
,使图$G$删除了$V_1$的所有边后,所得的子图是不连通图,而删
除了$E_1$的任何真子集后,所得到的子图仍是连通图,则称$E_1$是$G$
的一个边割集.若某一个边构成一个边割集,则称该结点为割边(或桥).
\end{definition}

\begin{theorem}{}
对于任意一个图$G$有
$$k(G)\leq \lambda(G) \leq \delta(G)$$
\end{theorem}
\begin{theorem}{}
一个连通无向图G中的结点v是割点的充分必要条件是存两个结点u和w,是的u和w的每一条路都经过v.
\end{theorem}

\begin{definition}{}
在有向简单图G中,任何一队结点间,至少有一个结点
\end{definition}

