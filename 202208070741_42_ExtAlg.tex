% 外代数
% keys 外代数|外微分|线性空间|格拉斯曼代数|Grassmann|外积|矢量分析|向量分析|外积|外乘|楔积|楔乘|exterior algebra|exterior derivative|wedge product|反对称

\begin{issues}
\issueMissDepend
\issueOther{预备知识似乎有误}
\end{issues}


\pentry{张量代数\upref{TenAlg},域上的代数\upref{AlgFie}}

\addTODO{与张量代数的关系}
\addTODO{与对称积,张量积的关系}

张量代数\upref{TenAlg}一节最后提到,通过将张量积作为乘法不能使得 $\mathbb T_p^0(V)$ 的对称子空间和斜对称子空间的外直和 
\begin{equation}
\begin{aligned}
&\mathbb T^+(V^*)=\mathbb F\oplus\mathbb T_1^+(V)\oplus\mathbb T_2^+(V)\oplus\cdots\\
&\Lambda(V^*)=\mathbb F\oplus \Lambda^1(V^*)\oplus\Lambda^2(V^*)\oplus\cdots
\end{aligned}
\end{equation}
成为 
\begin{equation}
\mathbb T(V^*)=\mathbb F\oplus\mathbb T_1^0(V)\oplus\mathbb T_2^0(V)\oplus\cdots
\end{equation}
的子代数.这使得可以通过定义其它的乘积,使得这二者各自成为代数.对于斜对称的 $\Lambda(V^*)$,就是本节强调的重点.使 $\Lambda(V^*)$ 成为一个代数的乘法称为\textbf{外积}(\textbf{wedge}),记号为 $\wedge$,它是通过交错化映射 $A$(\autoref{SIofTe_def1}~\upref{SIofTe})定义的,具体便是
\begin{equation}
Q\wedge R=A(Q\otimes R)
\end{equation}

\subsection{外积}
事实上,基于张量的对称化和交错化\upref{SIofTe}引文处提到的理由,讨论任意张量的对称性和斜对称性,与 $(0,p)$型或 $(p,0)$ 型没有任何区别.为了多样性起见,这里考虑 $(0,p)$ ,即取空间($\Lambda^p(V)$ 的定义可见\autoref{SIofTe_def2}~\upref{SIofTe}后的段落.) 
\begin{equation}
\Lambda(V)=\mathbb F\oplus \Lambda^1(V)\oplus\Lambda^2(V)\oplus\cdots
\end{equation}
使用\autoref{SIofTe_def2}~\upref{SIofTe} 的术语,$\Lambda^p(V)$ 的元素称为 $p$ 矢量.

交错化映射 $A$ 将 $\mathbb T_0^p(V)$ 的张量映射到其上的斜对称张量子空间 $\Lambda^p(V)$ 上(\autoref{SIofTe_the1}~\upref{SIofTe}),而 $\Lambda^p(V)$ 都是 $\Lambda(V)$ 的子空间,因此有可能通过 $A$ 定义一种乘法,使得 $\Lambda(V)$ 在该乘法下封闭.注意 $ \Lambda(V)$ 上的任意元素 $Q$ 可记作(\autoref{TenAlg_eq5}~\upref{TenAlg} )
\begin{equation}\label{ExtAlg_eq3}
Q=\sum_{i=0}^\infty Q_i=(Q_0,Q_1,\cdots),\quad Q_i\in\Lambda^i(V)
\end{equation}

此时,为了让张量积不跳出 $\Lambda(V)$,我们试着用 $A$ 作用到 $\Lambda(V)$ 上任意两元素 $Q,R$ 的张量积 $Q\otimes R$ 上:
\begin{equation}
\begin{aligned}
&A(Q\otimes R)=A(\sum_{k=0}^\infty h_k)=\sum_{k=0}^\infty  A(h_k),\\
&h_k=\sum_{i=0}^k Q_i\otimes R_{k-i}\in\mathbb T_0^k(V)
\end{aligned}
\end{equation}
因此 $A(h_k)\in \Lambda^{k}(V)$,即
\begin{equation}
A(Q\otimes R)=\sum_{k=0}^\infty  A(h_k),\quad A(h_k)\in\Lambda^k(V)
\end{equation}
注意\autoref{ExtAlg_eq3} ,就有 $A(Q\otimes R)\in\Lambda(V)$.

于是通过 $A(Q\otimes R)\in\Lambda(V)$ 来定义 $Q$ 和 $R$ 的乘法就得到了 $\Lambda(V)$ 上乘法的封闭性.显然,通过张量积运算的性质,$1\in\mathbb F$ 是该乘法的单位元.
\begin{definition}{外积}
对任意 $q$ 矢量 $Q$ 和 任意 $r$ 矢量 $R$,运算 $\wedge:\Lambda(V)\times\Lambda(V)\rightarrow\Lambda(V)$:
\begin{equation}
Q\wedge R=A(Q\otimes R)
\end{equation}
称为 $\Lambda(V)$ 上的\textbf{外积运算}.
\end{definition}
\begin{theorem}{外积的性质}
外积具有以下的性质:
\begin{enumerate}
\item \begin{equation}\label{ExtAlg_eq1}
\wedge:\Lambda^q(V)\times\Lambda^r(V)\rightarrow\Lambda^{q+r}(V)
\end{equation}
\item \textbf{双线性:}任意 $\alpha,\beta\in\mathbb F$,有
\begin{equation}
\begin{aligned}
Q\wedge(\alpha R+\beta T)&=\alpha(Q\wedge R)+\beta(Q\wedge T)\\
(\alpha Q+\beta S)\wedge R&=\alpha (Q\wedge R)+\beta (S\wedge R)
\end{aligned}
\end{equation}
\item \textbf{结合性:}
\begin{equation}
P\wedge (Q\wedge R)=(P\wedge Q)\wedge R
\end{equation}
\item 任意 $x_{i_1},\cdots,x_{i_p}\in V$,满足
\begin{equation}
x_{i_{\pi 1}}\wedge\cdots \wedge x_{i\pi p}=\epsilon_\pi x_{i_1}\wedge\cdots\wedge x_{i_p}
\end{equation}

\end{enumerate}

\end{theorem}
\textbf{证明:}1.设 $Q$ 为 $q$ 矢量,$R$ 为 $r$ 矢量,那么由\autoref{CofTen_the2}~\upref{CofTen}, $Q\otimes R\in\mathbb T_0^{q+r}$ .由\autoref{SIofTe_the1}~\upref{SIofTe} ,对 $\mathbb T_0^{q+r}(V)$ 上的 $A$,其像 $\Im A=\Lambda ^{q+r}(V)$ .于是\autoref{ExtAlg_eq1} 成立.

2.由张量积和 $A$ 的线性
\begin{equation}
\begin{aligned}
Q\wedge(\alpha R+\beta T)&=A(Q\wedge(\alpha R+\beta T))=A(\alpha Q\otimes R+\beta Q\otimes T)\\
&=\alpha A(Q\otimes R)+\beta A(Q\otimes T)=\alpha Q\wedge R+\beta Q\wedge T
\end{aligned}
\end{equation}
第2式同理.

3.由\autoref{SIofTe_the2}~\upref{SIofTe}
\begin{equation}
\begin{aligned}
(P\wedge Q)\wedge R&=A(A(P\otimes Q)\otimes R)=A((P\otimes Q)\otimes R)\\
&=A(P\otimes (Q\otimes R))=A(P\otimes A(Q\otimes R))\\
&=P\wedge (Q\wedge R)
\end{aligned}
\end{equation}

4.先证 $x\wedge y=-y\wedge x,\;\forall x,y\in V$:
\begin{equation}
x\wedge y=A(x\otimes y)=\frac{1}{2}(x\otimes y-y\otimes x)
\end{equation}
因此
\begin{equation}\label{ExtAlg_eq2}
x\wedge y=-y\wedge x,\quad x\wedge x=0
\end{equation}

对一般情形,可由\autoref{ExtAlg_eq2} 和外积的结合性证得,见\autoref{AntMap_ex1}~\upref{AntMap}.

\textbf{证毕!}

\subsection{外代数}

从上面知道,由 $\wedge$



外代数是一种利用已有线性空间构造“代数”这一对象的通用方法,同时蕴含了对三维矢量分析中代数结构的本质解释.

\subsection{外代数的概念}

给定线性空间 $V$,任取 $x, y\in V$,定义 $x\wedge y\not\in V$ 是一个新的元素,其中符号 $\wedge$ 称作\textbf{外积(exterior product)},有时也叫做\textbf{楔积(wedge product)},前者是因为这个运算得到的是 $V$ 以外的新元素,后者是由于符号长得像个楔子.注意,为了方便,我们没有使用线性代数中常见的粗体正体符号来表示向量.

利用各 $x\wedge y$ 构造新的线性空间:定义 $x\wedge y=-y\wedge x$ 对所有 $x, y\in V$ 成立,这同时意味着 $x\wedge x=0$.定义一个加法 $+$,使得对于 $x_1, x_2, y\in V$,都有 $(x_1+x_2)\wedge y=x_1\wedge y+x_2\wedge y$;再定义数乘为对于任意基本域中的数字 $a$,都有 $a(x\wedge y)=(ax)\wedge y=x\wedge(ay)$.这样,集合 $\{x\wedge y|x, y\in V\}$ 构成一个线性空间,记为 $\bigwedge^2 V$.同时,为了统一考虑,记 $V=\bigwedge^1 V$.

$\bigwedge^1 V$ 和 $\bigwedge^2 V$ 之间也可以进行楔积,并且满足\textbf{结合律}:$x\wedge(y\wedge z)=(x\wedge y)\wedge z$,由此可以拿掉结合括号,定义 $x\wedge y\wedge z=x\wedge(y\wedge z)=(x\wedge y)\wedge z$.集合 $\{x\wedge y\wedge z|x, y, z\in V\}$ 张成的线性空间,记为 $\bigwedge^3 V$.

同理,我们可以构造出任意阶的 $\bigwedge^k V$.要注意的是,如果 $k>\opn{dim} V$,那么 $\bigwedge^k V=\{0\}$.另外,把 $V$ 的基本域 $\mathbb{F}$ 看成一个一维线性空间,记 $\mathbb{F}=\bigwedge^0 V$.

不同线性空间之间可以用直和组合在一起,因此以上这些空间也都可以作直和,得到一个 $\bigwedge V=\bigoplus_k\bigwedge^k V=\mathbb{F}\oplus\bigwedge^1V\oplus\bigwedge^2V\cdots$.这个 $\bigwedge V$,就被称作 $V$ 上的\textbf{外积空间(exterior product space)}或\textbf{楔积空间(wedge product space)}.$\mathbb{F}$ 是 $V$ 的基域,视为 $\mathbb{F}$ 自身上的一维线性空间.

\begin{theorem}{外代数}
任给域 $\mathbb{F}$ 上的线性空间 $V$,则外积 $\wedge$ 是一个 $\bigwedge V$ 上的向量乘法,并且满足结合性,有单位元 $1\in \bigwedge^0 V$,故构成一个 $\mathbb{F}$ 上的代数.称这个代数为 $V$ 上的\textbf{外代数(exterior algebra)}或\textbf{格拉斯曼代数(Grassmann algebra)}.
\end{theorem}

外代数中的元素可以有形象的几何理解.$\bigwedge^1 V$ 中的元素就是 $V$ 中的元素,我们可以想象成箭头.$\bigwedge^2 V$ 中的元素可以看成箭头对,或者是箭头对表示的平行四边形.同样,$\bigwedge^k V$ 中的元素都可以看成是 $k$ 个箭头张成的一个 $k$ 维对象.

外代数有一个重要的性质,我们用\autoref{ExtAlg_exe1} 和\autoref{ExtAlg_exe2} 来阐述 :

\begin{exercise}{}\label{ExtAlg_exe1}
证明:如果 $k>\opn{dim} V$,那么 $\bigwedge^kV=\{0\}$.
\end{exercise}

\begin{exercise}{}\label{ExtAlg_exe2}
证明:对于 $\opn{dim} V=k$,有 $\opn{dim} \bigwedge V=2^k$.思路提示:考虑各 $\opn{dim}\bigwedge^iV$ 的值,再对比 $(1+1)^k$ 的二项式展开.
\end{exercise}




三维欧几里得空间 $\mathbb{R}^3$ 中的叉乘实际上就是外积.这是因为,$\opn{dim}\mathbb{R}^3=\opn{dim}\bigwedge^2\mathbb{R}^3$,这样一来,如果给定 $\mathbb{R}^3$ 的标准正交基 $\{x, y, z\}$,那么我们可以建立同构 $*: \bigwedge^2\mathbb{R}\rightarrow\mathbb{R}^3$,使得 $*(x\wedge y)=z, *(y\wedge z)=x, *(z\wedge x)=y$,这样就可以通过这个同构来把外积变成 $\mathbb{R}^3$ 内部的向量积.这一映射也是叉乘的“右手定则”的来源,我们也完全可以规定 $*(x\wedge y)=-z, *(y\wedge z)=-x, *(z\wedge x)=-y$,这样定义出来的叉乘就是符合左手定则的了.

三维线性空间是唯一可以构造反交换代数的非平凡空间,就是因为只有三维的 $V$ 才满足 $\opn{dim}V=\opn{dim}\bigwedge^2V$,因而可以建立 $\bigwedge^2 V$ 和 $V$ 之间的同构,从而把楔积变成叉积.相应地,比复数更高维的可除代数只有四元数.

外代数是一个“\textbf{分次线性空间(graded vector space)}”,就是说,它作为一个线性空间,每个向量具有一个“次数”,定义如下:每个 $\bigwedge^kV$ 中的向量,其\textbf{次数(grade)}就是 $k$;对于任意向量 $v\in \bigwedge V$,我们总可以把它拆分成各 $\bigwedge^kV$ 中基向量的线性组合,这些基向量中次数最高的就定义为 $v$ 的次数.


\subsection{对偶空间的外代数}

实流形上极为常见的外代数是微分形式生成的外代数,也就是所谓的外微分.由于一个微分 $k$-形式可以看成是将 $k$ 个向量场变成一个光滑函数的映射,也就是在每个切点处都是一个将 $k$ 个切向量变成一个实数的张量,因此要研究微分形式的外代数,首先就要搞清楚对偶空间的外代数.

对偶空间中的元素,都是主空间中的线性函数.$k$ 个线性函数的外积,被定义为“将 $k$ 个主空间向量映射为一个数”的多重线性映射.我们自然想到将 $k$ 个向量的映射联系到对偶空间外代数中的 $k$ 阶元素.

具体来说,假如我们有两个多重线性映射 $f$ 和 $g$,分别是 $n$ 阶和 $m$ 阶的,那么我们定义 $f\wedge g$ 为如下 $n+m$ 阶多重线性映射:
\begin{equation}
\begin{aligned}
&f\wedge g(x_1, x_2, \cdots, x_{n+m})\\=
&\sum\limits_{\sigma\in S_{n+m}}\opn{sgn}(\sigma) f(x_{\sigma(1)}, x_{\sigma(2)}, \cdots, x_{\sigma(n)})\cdot g(x_{\sigma(n+1)}, \cdots, x_{\sigma(n+m)})
\end{aligned}
\end{equation}
其中 $\sigma$ 是一个 $n+m$ 元置换,当 $\sigma$ 为奇变换时 $\opn{sgn}(\sigma)$ 为 $-1$,$\sigma$ 为偶变换时 $\opn{sgn}(\sigma)$ 为 $1$.

最简单的例子,就是两个对偶向量的外积.设 $V^*$ 是线性空间 $V$ 的对偶空间,令 $f, g\in V^*$,那么对于任意 $\bvec{v}, \bvec{u}\in V$,我们有:
\begin{equation}
f\wedge g(\bvec{v}, \bvec{u})=f(\bvec{v})g(\bvec{u})-f(\bvec{u})g(\bvec{v})
\end{equation}
其中一共涉及置换群 $S_2$ 中的两个置换,$(1)$ 和 $(1\phantom{2}2)$,分别是偶置换和奇置换.
