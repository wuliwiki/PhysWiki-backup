% 齐次函数的欧拉定理

首先介绍一下什么是齐次函数.

\begin{definition}{齐次函数}
假设$f: V \to W $是域$ F $内的两个向量空间之间的函数.

我们说$f$是$k$\textbf{次齐次函数},如果对于所有非零的$\alpha \in F$ 和$\mathbf{v} \in V$ ,都有:
\begin{equation}
f(\alpha \mathbf{v}) = \alpha^k f(\mathbf{v}) 
\end{equation}
即是,在欧几里得空间,$f(\alpha \mathbf{v}) = f(k) \ f(\mathbf{v})$ , 其中$f(k)$为指数函数.
\end{definition}

\begin{example}{}
$f(x,y,z)=x^5y^2z^3$是$10$次齐次函数,因为$(\alpha x)^5(\alpha y)^2(\alpha z)^3=\alpha^{10}x^5y^2z^3$.

$f(x,y)=x^5 + 2 x^3 y^2 + 9 x y^4$是$5$次齐次函数.
\end{example}

齐次函数的欧拉定理表述如下:

\begin{theorem}{齐次函数的欧拉定理}

\end{theorem}