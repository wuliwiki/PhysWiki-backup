% 场与粒子的相互作用
% 场论|相互作用|interaction|电磁场

\pentry{从分析力学到场论\upref{CFa1}}

本文的闵可夫斯基空间度规为$\opn{diag}(1, -1, -1, -1)$.

\subsection{自由粒子和自由场}

拉格朗日函数能描述场和粒子的运动规律.如果宇宙中只存在一个场或者粒子,我们就说它是自由的,因为不存在任何其它东西与它相互作用.

自由粒子的拉格朗日函数如何确定?注意到拉格朗日的积分,即作用量,作为运动轨迹的泛函,和坐标的选取无关,因此我们可以猜想用只和轨迹相关的某个泛函作为作用量.最直接的想法是什么呢?轨迹的“长度”:
\begin{equation}
S(\Gamma) = \int_\Gamma \dd s
\end{equation}
这里的$\Gamma$为给定的轨迹



















