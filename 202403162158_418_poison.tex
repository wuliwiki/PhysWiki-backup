% 泊松括号
% keys 哈密顿量|泊松括号|守恒量
% license Xiao
% type Tutor

\begin{issues}
\issueDraft
\end{issues}

\pentry{哈密顿正则方程\nref{nod_HamCan}}{nod_efae}
在理论力学里,泊松括号的引入能更加简洁地表明运动积分需要满足的条件。所谓运动积分,可以简单理解为在某一动力学系统中,不随时间改变的常数。动力学系统总满足二阶微分方程,所以我们总可以找到这样的运动积分。现在设$f(q,p,t)$为粒子关于动量、坐标和时间的函数,且为运动积分,则根据定义我们有:
\begin{equation}
\frac{df}{dt}=\frac{\partial f}{\partial t}+\frac{\partial f}{\partial q}\dot{q}+\frac{\partial f}{\partial p}\dot{ p}~.
\end{equation}
结合正则方程,我们可以把上式改写为
\begin{equation}\label{eq_poison_10}
\frac{df}{dt}=\frac{\partial f}{\partial t}+\pb{H}{f}~.
\end{equation}
引入的泊松记号定义如下:


对于任意两个函数 $u(q, p, t)$ 和 $v(q, p, t)$, 泊松括号的“作用”为
\begin{equation}\label{eq_poison_1}
\pb{u}{v} = \sum_i \pdv{u}{q_i}\pdv{v}{p_i} - \pdv{v}{q_i}\pdv{u}{p_i}~,
\end{equation}
其中$i$为系统的自由度。显然,如果函数$f$不显含时间,且为运动积分(也就是一守恒量),则该函数与系统哈密顿量的泊松括号为0.

\subsection{泊松括号的性质}
根据定义,我们可以证明泊松括号满足如下几条性质。
\begin{enumerate}
\item 反对称性
$\{u,v\}=-\{v,u\}$
\item 双线性

$\pb{au+bv}{\phi}=a\pb{u}{\phi}+b\pb{v}{\phi}$
\item 
$\pb{uv}{f}=\pb{u}{f}v+u\pb{v}{f}$

\item 轮换性,即雅克比恒等式
\begin{equation}
\pb{f}{\pb{g}{h}}+\pb{g}{\pb{h}{f}}+\pb{h}{\pb{f}{g}}=0~.
\end{equation}
\end{enumerate}
前三条性质的证明只需套用定义,读者可自行验证。至于雅克比恒等式的证明,这里提供一条相对简单的思路。注意到,若固定函数$u$,则泊松括号$\pb{u}{v}$可视作对$v$的线性映射,实际上是一线性微分算符。那么进行两次泊松括号的运算,实际上是对$v$求二阶偏导数。雅克比恒等式的三项都套了两层括号,那么展开后的式子是$f,g,h$中任意函数的二阶偏导。通过观察底层括号,你会发现,$f,g,h$都出现了两次。那么如果我们能证明,进行两次泊松运算后,对其中一个函数的二阶偏导数是两两抵消的(只剩一阶偏导数),自然就证明了这条性质。具体的推导参见朗道的力学。

\subsection{泊松定理}
如果在一个系统里,我们能找到两个运动积分$f$和$g$,那么可以证明$\pb{f}{g}$也是一运动积分。证明如下:
\begin{equation}
\frac{d\pb{f}{g}}{dt}=\pb{\frac{df}{dt}}{g}+\pb{f}{\frac{dg}{dt}}=0~.
\end{equation}

\subsection{量子力学中的“泊松”括号}
如果读者已经学习过量子力学,可能会发现对易运算与泊松括号的相似之处。对易子作为量子力学的公设之一,其运算导出了力学量的运动方程,形式上也与本篇式2一致(当然,经典力学与量子力学的相似之处不止如此)。如果我们遵循某种额外施加的条件并视之为原则,那么我们可以找到这两种运算的直接联系。

具体而言,我们假设,量子力学的对易运算其性质与经典力学相同,且算符不对易。那么我们有两种方法计算$\pb{uv}{fg}$
\begin{equation}
\begin{aligned}
\pb{uv}{fg}&=f\pb{u}{g}v+\pb{u}{f}gv+uf\pb{v}{g}+u\pb{v}{f}g\\
&=f\pb{u}{g}v+\pb{u}{f}vg+fu\pb{v}{g}+u\pb{v}{f}g~.
\end{aligned}
\end{equation}
因此,我们有
\begin{equation}
\pb{u}{f}(gv-vg)=(uf-fu)\pb{v}{g}~.
\end{equation}
此式对$u,f$永远成立,则需要$(uf-fu)$与$\pb{u}{f}$线性依赖,所以
\begin{equation}
i\hbar\pb{u}{f}=[u,f]~.
\end{equation}

为什么这里会出现$i\hbar$?我想应该是因为我们关心的,往往是能被实际观测到的物理量,假定这些物理量为厄米算符,那么厄米算符的对易子一般是非厄米的,所以左边的泊松括号需要乘以复数$i$,$\hbar$则为量纲为角动量的实常数.这可以是一个构建经典与量子之间对应的思路,譬如最基本的对易关系为$[p,q]$。

