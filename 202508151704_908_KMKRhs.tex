% 卡迈克尔函数(综述)
% license CCBYSA3
% type Wiki

本文根据 CC-BY-SA 协议转载翻译自维基百科\href{https://en.wikipedia.org/wiki/Carmichael_function}{相关文章}。

在数论(数学的一个分支)中,正整数 $n$ 的卡迈克尔函数$\lambda(n)$ 定义为满足下列条件的最小正整数 $m$:
$$
a^{m} \equiv 1 \pmod{n}~
$$
其中 $a$ 为任意与 $n$ 互素的整数。从代数的角度来看,$\lambda(n)$ 是模 $n$ 的整数乘法群的指数。由于这是一个有限阿贝尔群,必然存在一个元素,其阶等于该指数 $\lambda(n)$。这样的元素被称为模 $n$ 的本原 $\lambda$-根(primitive $\lambda$-root modulo $n$)。
\begin{figure}[ht]
\centering
\includegraphics[width=14.25cm]{./figures/133eee57c8a8a0bb.png}
\caption{卡迈克尔 λ 函数:$1 \le n \le 1000$ 时的 $\lambda(n)$(与欧拉 φ 函数对比)} \label{fig_KMKRhs_1}
\end{figure}
卡迈克尔函数以美国数学家罗伯特·卡迈克尔的名字命名,他于 1910 年首次给出了这一函数的定义\(^\text{[1]}\)。它也被称为卡迈克尔 $\lambda$ 函数、约化欧拉函数以及最小通用指数函数。

模 $n$ 的整数乘法群的阶是 $\varphi(n)$,其中 $\varphi$ 是**欧拉函数**。由于有限群中任一元素的阶都会整除群的阶,所以 $\lambda(n)$ 一定整除 $\varphi(n)$。

下表比较了 $\lambda(n)$(OEIS 数列 A002322)与 $\varphi(n)$ 的前 36 个值(当二者不同的时候,$\varphi(n)$ 用粗体表示;使它们不同的那些 $n$ 的值列在 OEIS 数列 A033949 中)。
\begin{figure}[ht]
\centering
\includegraphics[width=14.25cm]{./figures/309a6eadd8675298.png}
\caption{} \label{fig_KMKRhs_2}
\end{figure}
\subsection{数值例子}
\begin{itemize}
\item $ n = 5$小于且与 5 互素的数集合是 $\{1, 2, 3, 4\}$。因此,欧拉函数的值为 $\varphi(5) = 4$,而卡迈克尔函数 $\lambda(5)$ 的值必须是 4 的约数。约数 1 不满足卡迈克尔函数的定义,因为除了$a \equiv 1 \pmod{5}$之外,$a^1 \not\equiv 1 \pmod{5}$。约数 2 也不行,因为$2^2 \equiv 3^2 \equiv 4 \not\equiv 1 \pmod{5}$。因此,$\lambda(5) = 4$。实际上:$1^4 \equiv 2^4 \equiv 3^4 \equiv 4^4 \equiv 1 \pmod{5}$。其中 2 和 3 是模 5 的本原 $\lambda$-根,同时它们也是模 5 的本原根。
\item $ n = 8$小于且与 8 互素的数集合是 $\{1, 3, 5, 7\}$。因此 $\varphi(8) = 4$,而 $\lambda(8)$ 必须是 4 的约数。事实上,$\lambda(8) = 2$,因为:$
1^2 \equiv 3^2 \equiv 5^2 \equiv 7^2 \equiv 1 \pmod{8}$。模 8 的本原 $\lambda$-根是 3、5 和 7,但模 8 没有本原根。
\end{itemize}
\subsection{λ(n) 的递推公式}
质数幂的卡迈克尔$\lambda$函数可以用欧拉函数表示。任何不是 1 且不是质数幂的整数,都可以唯一地分解为不同质数幂的乘积,此时该数的$\lambda$值等于这些质数幂的 $\lambda$值的最小公倍数。具体来说,$\lambda(n)$ 由以下递推式给出:
$$
\lambda(n) =
\begin{cases}
\varphi(n), & \text{如果 } n \text{ 是 1、2、4,或奇质数幂}, \\[6pt]
\dfrac{1}{2} \varphi(n), & \text{如果 } n = 2^{r},\ r \geq 3, \\[8pt]
\operatorname{lcm}\bigl(\lambda(n_{1}),\lambda(n_{2}),\dots,\lambda(n_{k})\bigr), & \text{如果 } n = n_{1} n_{2} \dots n_{k}, \ \text{且 } n_{1},n_{2},\dots,n_{k} \text{ 是不同质数的幂}.
\end{cases}~
$$
其中,质数幂 $p^r$($p$ 为质数且 $r \geq 1$)的欧拉函数为:
$$
\varphi(p^{r}) = p^{r-1}(p - 1)~
$$
\subsection{卡迈克尔定理}
卡迈克尔证明了两个定理,它们共同确立了这样一个事实:如果将 $\lambda(n)$ 按照上一节的递推公式来定义,那么它就满足引言中所述的性质——即它是满足以下条件的最小正整数 $m$:$a^{m} \equiv 1 \pmod{n}$对于所有与 $n$ 互素的整数 $a$,上述同余式都成立。

\textbf{定理 1} —— 如果 $a$ 与 $n$ 互素,则:$a^{\lambda(n)} \equiv 1 \pmod{n}$。\(^\text{[2]}\) 

这意味着:模 $n$ 的整数乘法群中每个元素的阶都整除 $\lambda(n)$。卡迈克尔将这样一种元素 $a$ 称为模 $n$ 的本原 $\lambda$-根(primitive $\lambda$-root modulo $n$):它的最小正整数指数,使得$a^{\lambda(n)} \equiv 1 \pmod{n}$\(^\text{[3]}\)(不要与模 $n$ 的本原根混淆,卡迈克尔有时会称其为模 $n$ 的本原 $\varphi$-根)。

\textbf{定理 2} —— 对于任意正整数 $n$,都存在一个模 $n$ 的本原 $\lambda$-根。此外,如果 $g$ 是这样一个根,那么所有与 $g$ 的幂同余的本原 $\lambda$-根的个数为:$\varphi\bigl(\lambda(n)\bigr)$。\(^\text{[4]}\) 

如果 $g$ 是定理所保证存在的模 $n$ 的本原 $\lambda$-根之一,那么同余式$g^{m} \equiv 1 \pmod{n}$在正整数范围内没有小于 $\lambda(n)$ 的解。这说明不存在正整数 $m < \lambda(n)$,能使得对于所有与 $n$ 互素的 $a$,都满足$a^{m} \equiv 1 \pmod{n}$。

定理 2 的第二条结论并不意味着模 $n$ 的所有本原 $\lambda$-根都与某一个本原根 $g$ 的幂同余\(^\text{[5]}\)。例如,当 $n = 15$ 时,$\lambda(n) = 4$,而 $\varphi(n) = 8$,$\varphi(\lambda(n)) = 2$。模 15 的本原 $\lambda$-根共有 4 个,分别是 2、7、8 和 13,因为:$1 \equiv 2^{4} \equiv 8^{4} \equiv 7^{4} \equiv 13^{4} \pmod{15}$。其中 2 和 8 互为幂同余,7 和 13 互为幂同余,但 7 和 13 都不与 2 或 8 的幂同余,反之亦然。模 15 的乘法群中的另外四个元素——1、4(满足$4 \equiv 2^{2} \equiv 8^{2} \equiv 7^{2} \equiv 13^{2} \pmod{15}$)、11 和 14——都不是模 15 的本原 $\lambda$-根。

再看一个对比性的例子:当 $n = 9$ 时,$\lambda(n) = \varphi(n) = 6$,且 $\varphi(\lambda(n)) = 2$。模 9 的本原 $\lambda$-根有两个,分别是 2 和 5,并且它们互为对方的五次幂同余。它们同时也是模 9 的本原 $\varphi$-根。
\subsection{卡迈克尔函数的性质}
在本节中,若存在一个整数 $k$,使得
$$
n = k m,~
$$
则称整数 $n$ 可被非零整数 $m$ 整除,记作:
$$
m \mid n~
$$
\subsubsection{λ(n) 最小性的推论}
假设对于所有与 $n$ 互素的整数 $a$,都有$a^{m} \equiv 1 \pmod{n}$,那么 $\lambda(n) \mid m$。

\textsl{证明:}

设$m = k \lambda(n) + r, \quad 0 \le r < \lambda(n)$,则对于所有与 $n$ 互素的整数 $a$,有:
$$
a^{r} = 1^{k} \cdot a^{r} \equiv \left(a^{\lambda(n)}\right)^{k} \cdot a^{r} = a^{k\lambda(n) + r} = a^{m} \equiv 1 \pmod{n}~
$$
由于 $r < \lambda(n)$ 且 $\lambda(n)$ 是使得上述同余式对所有与 $n$ 互素的 $a$ 成立的最小正整数指数,因此必须有 $r = 0$。
\subsubsection{$\lambda(n)$ 整除 $\varphi(n)$}
这可以由初等群论推出,因为任何有限群的指数都必须整除该群的阶。$\lambda(n)$ 是模 $n$ 的整数乘法群的指数,而 $\varphi(n)$ 是该群的阶。

特别地,当该乘法群是循环群时(因为存在本原根),二者相等。这种情况发生在奇质数幂的情形下。

因此,我们可以把卡迈克尔定理看作是对欧拉定理的一个强化。
\subsubsection{可整除性}
$$
a \mid b \ \Rightarrow \ \lambda(a) \mid \lambda(b)~
$$
\textbf{证明:}

由定义可知,对于任意满足$\gcd(k, b) = 1$(因此也有 $\gcd(k, a) = 1$)的整数 $k$,有:$b \mid \bigl(k^{\lambda(b)} - 1\bigr)$,因此也有:$a \mid \bigl(k^{\lambda(b)} - 1\bigr)$。这说明对于所有与 $a$ 互素的 $k$,都有:$k^{\lambda(b)} \equiv 1 \pmod{a}$。根据前面已证明的最小性推论,我们得到:$
\lambda(a) \mid \lambda(b)$。
\subsubsection{复合性质}
对于所有正整数 $a$ 和 $b$,都有:
$$
\lambda\bigl(\mathrm{lcm}(a, b)\bigr) = \mathrm{lcm}\bigl(\lambda(a), \lambda(b)\bigr)~
$$
这是卡迈克尔函数递推公式的直接推论。
\subsubsection{指数循环长度}
如果$r_{\mathrm{max}} = \max_{i} \{ r_{i} \}$是整数 $n$ 的质因数分解$n = p_{1}^{r_{1}} p_{2}^{r_{2}} \cdots p_{k}^{r_{k}}$中最大的指数,那么对于所有整数 $a$(包括与 $n$ 不互素的 $a$)以及所有 $r \ge r_{\mathrm{max}}$,都有:
$$
a^{r} \equiv a^{\lambda(n) + r} \pmod{n}~
$$
特别地,当 $n$ 是平方因子自由数(square-free,$r_{\mathrm{max}} = 1$)时,对于所有 $a$ 都有:
$$
a \equiv a^{\lambda(n) + 1} \pmod{n}~
$$
\subsubsection{平均值}
对于任意 $n \ge 16$\(^\text{[6][7]}\):
$$
\frac{1}{n} \sum_{i \le n} \lambda(i) = \frac{n}{\ln n} e^{B \,(1 + o(1)) \,\frac{\ln \ln n}{\ln \ln \ln n}}~
$$
(下文称为埃尔德什近似,Erdős approximation),其中常数
$$
B := e^{-\gamma} \prod_{p \in \mathbb{P}} \left( 1 - \frac{1}{(p - 1)^{2}(p + 1)} \right) \approx 0.34537~
$$

$\gamma \approx 0.57721$ 为欧拉–马歇罗尼常数。

下表给出了 $\lambda$ 函数前 $2^{26} - 1 = 67\,108\,863$ 个值的概览,包括其精确平均值和埃尔德什近似值。

另外,还给出了一个更易计算的 “对数比对数” 值(LoL 值):
$$
\mathrm{LoL}(n) := \frac{\ln \lambda(n)}{\ln n}~
$$
其中:
\begin{itemize}
\item 
$$
\mathrm{LoL}(n) > \frac{4}{5} \quad \Leftrightarrow \quad \lambda(n) > n^{\frac{4}{5}}~
$$
\end{itemize}
在表格中,第 26 行 
\begin{itemize}
\item \% LoL > $\frac{4}{5}$ 列的 60.49 表示:在 $1 \le n \le 67\,108\,863$ 的整数中,约有 60.49\%(≈ 40,000,000 个)满足 $\lambda(n) > n^{\frac{4}{5}}$
\end{itemize}
这意味着,大多数 $\lambda$ 值在输入 $n$ 的长度 $l := \log_{2} n$ 上是指数级的,具体为:
$$
\left( 2^{\frac{4}{5}} \right)^{l} = 2^{\frac{4l}{5}} = \left( 2^{l} \right)^{\frac{4}{5}} = n^{\frac{4}{5}}~
$$
\begin{figure}[ht]
\centering
\includegraphics[width=14.25cm]{./figures/4ed3ab8e6fc653b9.png}
\caption{} \label{fig_KMKRhs_3}
\end{figure}
\subsubsection{主导区间}
对于任意整数 $N$,以及除了 $o(N)$\(^\text{[8]}\)之外的所有正整数 $n \le N$(即“主导性的大多数”),都有:
$$
\lambda(n) = \frac{n}{(\ln n)^{\ln \ln \ln n + A + o(1)}}~
$$
其中常数\(^\text{[7]}\):
$$
A := -1 + \sum_{p \in \mathbb{P}} \frac{\ln p}{(p - 1)^{2}} \approx 0.2269688~
$$
\subsubsection{下界}
对于任意足够大的数 $N$ 以及任意 $\Delta \ge (\ln \ln N)^{3}$,在所有 $n \le N$ 的正整数中,满足 $\lambda(n) \le n e^{-\Delta}$\(^\text{[9]}\)的个数至多为:
$$
N \exp\left( -0.69 (\Delta \ln \Delta)^{\frac{1}{3}} \right)~
$$
\subsubsection{最小阶}
对于任意正整数序列$n_{1} < n_{2} < n_{3} < \cdots$任意常数 $0 < c < \frac{1}{\ln 2}$,以及任意足够大的 $i$\(^\text{[10][11]}\):
$$
\lambda(n_{i}) > \left( \ln n_{i} \right)^{c \, \ln \ln \ln n_{i}}~
$$
\subsubsection{小值}
对于任意常数 $c$ 和任意足够大的正数 $A$,存在整数 $n > A$ 使得\(^\text{[11]}\):
$$
\lambda(n) < \left( \ln A \right)^{c \, \ln \ln \ln A}~
$$
此外,$n$ 的形式为:
$$
n = \prod_{\substack{q \in \mathbb{P} \\ (q - 1) \mid m}} q~
$$
其中 $m$ 是平方因子自由整数,且 $m < (\ln A)^{c \, \ln \ln \ln A}$\(^\text{[10]}\)。
\subsubsection{函数的值域}
卡迈克尔函数取值的集合,其计数函数为\(^\text{[12]}\):
$$
\frac{x}{(\ln x)^{\eta + o(1)}},~
$$
其中:
$$
\eta = 1 - \frac{1 + \ln \ln 2}{\ln 2} \approx 0.08607~
$$
\subsection{在密码学中的应用}
卡迈克尔函数在密码学中具有重要意义,尤其是在 RSA 加密算法 中的应用。
\subsection{定理 1 的证明}
当 $n = p$ 为质数时,定理 1 等价于费马小定理:
$$
a^{p-1} \equiv 1 \pmod{p} \qquad \text{对于所有与 } p \text{ 互素的 } a~
$$
对于质数幂 $p^{r}$($r > 1$),如果存在整数 $h$ 使得:
$$
a^{p^{\,r-1}(p - 1)} = 1 + h p^{r},~
$$
那么将等式两边同时取 $p$ 次幂,得到:
$$
a^{p^{\,r}(p - 1)} = 1 + h' p^{r+1},~
$$
其中 $h'$ 是某个整数。由归纳法可得:$a^{\varphi(p^{r})} \equiv 1 \pmod{p^{r}}$对所有与 $p$ 互素(因此也与 $p^{r}$ 互素)的 $a$ 都成立。这就证明了定理在 $n = 4$ 或任意奇质数幂的情形下成立。
\subsubsection{对 2 的高次幂情形的结果强化}
对于与(2 的幂)互素的 $a$,可写成$a = 1 + 2h_{2}$其中 $h_{2}$ 为某个整数。于是:
$$
a^{2} = 1 + 4h_{2}(h_{2} + 1) = 1 + 8 \binom{h_{2} + 1}{2} =: 1 + 8h_{3},~
$$
其中 $h_{3}$ 为整数。取 $r = 3$,可写作:
$$
a^{2^{\,r-2}} = 1 + 2^{r} h_{r}~
$$
将等式两边平方得:
$$
a^{2^{\,r-1}} = \left( 1 + 2^{r} h_{r} \right)^{2} = 1 + 2^{r+1} \left( h_{r} + 2^{r-1} h_{r}^{2} \right) =: 1 + 2^{r+1} h_{r+1},~
$$
其中 $h_{r+1}$ 为整数。由归纳法可得:
$$
a^{2^{\,r-2}} = a^{\frac12 \varphi(2^{r})} \equiv 1 \pmod{2^{r}}~
$$
对所有 $r \ge 3$ 且与 $2^{r}$ 互素的 $a$ 都成立\(^\text{[13]}\)。
\subsubsection{含有多个质因子的整数}
根据唯一分解定理,任意 $n > 1$ 都可以唯一地写成:
$$
n = p_{1}^{r_{1}} p_{2}^{r_{2}} \cdots p_{k}^{r_{k}}~
$$
其中 $p_{1} < p_{2} < \dots < p_{k}$ 是质数,$r_{1}, r_{2}, \dots, r_{k}$ 是正整数。质数幂情形的结论表明,对于 $1 \le j \le k$:
$$
a^{\lambda\left(p_{j}^{r_{j}}\right)} \equiv 1 \pmod{p_{j}^{r_{j}}}~
$$
对所有与 $n$ 互素(因此也与 $p_{i}^{r_{i}}$ 互素)的 $a$ 都成立。

由此可知:
$$
a^{\lambda(n)} \equiv 1 \pmod{p_{j}^{r_{j}}}~
$$
对所有与 $n$ 互素的 $a$ 都成立,其中根据递推公式:
$$
\lambda(n) = \mathrm{lcm}\bigl(\lambda(p_{1}^{r_{1}}), \lambda(p_{2}^{r_{2}}), \dots, \lambda(p_{k}^{r_{k}})\bigr)~
$$
由中国剩余定理可知:
$$
a^{\lambda(n)} \equiv 1 \pmod{n}~
$$
对所有与 $n$ 互素的 $a$ 都成立。
\subsection{另见}
\begin{itemize}
\item 卡迈克尔数
\end{itemize}
\subsection{注释}
\begin{enumerate}
\item Carmichael, Robert Daniel(1910),“关于一个新的数论函数的说明”,《美国数学学会会刊》,第 16 卷(第 5 期):232–238,doi:10.1090/S0002-9904-1910-01892-9。
\item Carmichael(1914),第 40 页。
\item Carmichael(1914),第 54 页。
\item Carmichael(1914),第 55 页。
\item Carmichael(1914),第 56 页。
\item Erdős(1991),定理 3。
\item Sándor & Crstici(2004),第 194 页。
\item Erdős(1991),定理 2,3. 正常阶,第 365 页。
\item Friedlander(2001),定理 5。
\item Erdős(1991),定理 1。
\item Sándor & Crstici(2004),第 193 页。
\item Ford, Kevin;Luca, Florian;Pomerance, Carl(2014 年 8 月 27 日),“卡迈克尔 λ 函数的值域”,《代数与数论》,第 8 卷(第 8 期):2009–2026,arXiv:1408.6506,doi:10.2140/ant.2014.8.2009,S2CID 50397623。
\item Carmichael(1914),第 38–39 页。
\end{enumerate}
\subsection{参考文献}
\begin{itemize}
\item Erdős, Paul;Pomerance, Carl;Schmutz, Eric(1991),“卡迈克尔 λ 函数”,《算术学报》,第 58 卷(第 4 期):363–385,doi:10.4064/aa-58-4-363-385,ISSN 0065-1036,MR 1121092,Zbl 0734.11047。
\item Friedlander, John B.;Pomerance, Carl;Shparlinski, Igor E.(2001),“幂生成器的周期与卡迈克尔函数的小值”,《计算数学》,第 70 卷(第 236 期):1591–1605, 1803–1806,doi:10.1090/s0025-5718-00-01282-5,ISSN 0025-5718,MR 1836921,Zbl 1029.11043。
\item Sándor, Jozsef;Crstici, Borislav(2004),《数论手册 II》,多德雷赫特:克鲁威学术出版社(Kluwer Academic),第 32–36 页、第 193–195 页,ISBN 978-1-4020-2546-4,Zbl 1079.11001。
\item Carmichael, Robert D. [1914],《数论理论》,古登堡计划。
\end{itemize}
