% 轻杆模型
% keys 轻杆|伸长|约束|角速度|求导
% license Xiao
% type Tutor

\pentry{速度、加速度\nref{nod_VnA}}{nod_a217}

\textbf{轻杆}在这里指是质量可以忽略不计, 且不能伸缩的刚性杆。 轻杆和质点一样都是一种理想化的模型。 我们先来讨论轻杆的运动学特性, 而把动力学放到后面。

\subsection{不可伸长条件}
\subsubsection{速度约束}
我们来思考 “不可伸长” 这一条件会对杆两端的速度和加速度带来什么约束。 令杆的长度为 $L$, 两端点的位置矢量分别为 $\bvec r_1$ 和 $\bvec r_2$, 则杆的长度平方为 $L^2 = (\bvec r_2 - \bvec r_1)^2$。 另外记端点的速度为 $\bvec v_1, \bvec v_2$。

如果我们将 $\bvec r_1$ 指向 $\bvec r_2$ 的单位矢量记为 $\uvec r$, 那么易得
\begin{equation}\label{eq_rod_7}
\uvec r = \frac{\bvec r_2 - \bvec r_1}{L}~.
\end{equation}
长度平方随时间的变化率为\footnote{为什么我们要用长度的平方而不是直接对 $\abs{\bvec r_2 - \bvec r_1}$ 求导? 我们的确可以这么做并得到同样的结果, 但是使用平方会使计算更方便。}
\begin{equation}\label{eq_rod_2}
\dv{t} L^2 = \dv{t} (\bvec r_2 - \bvec r_1)^2~.
\end{equation}
由矢量求导法则\autoref{eq_DerV_5} 得
\begin{equation}\label{eq_rod_1}
\dv{t} L^2 = 2L \dv{L}{t} = 2 (\bvec r_2 - \bvec r_1) \vdot (\bvec v_2 - \bvec v_1) = 2L (\bvec v_2 - \bvec v_1) \vdot \uvec r~,
\end{equation}
即
\begin{equation}\label{eq_rod_6}
\dv{L}{t} = \bvec v_2 \vdot \uvec r - \bvec v_1 \vdot \uvec r~.
\end{equation}
所以要满足长度恒定不变, 只需要满足(\enref{充分必要条件}{SufCnd})
\begin{equation}\label{eq_rod_3}
\bvec v_2 \vdot \uvec r = \bvec v_1 \vdot \uvec r~.
\end{equation}
该式说明, 两个端点的速度延杆的分量在任意时刻都相等。

\begin{example}{}\label{ex_rod_1}
如\autoref{fig_rod_2}, 杆的两端被固定在两个垂直的轨道上运动, 两端的运动速度分别为 $v_1, v_2$, 已知杆与水平轨道的夹角为 $\theta$, 求 $v_1$ 和 $v_2$ 的关系。
\begin{figure}[ht]
\centering
\includegraphics[width=6cm]{./figures/47dabed1358f64cd.pdf}
\caption{杆的运动} \label{fig_rod_1}
\end{figure}
解: 由\autoref{eq_rod_3}, 两速度在杆方向的分量应该相等, 即
\begin{equation}
v_2 \cos\theta = v_1 \sin\theta~,
\end{equation}
即
\begin{equation}
v_2 = v_1 \tan\theta~.
\end{equation}
\end{example}

\begin{example}{人拉船模型}\label{ex_rod_2}
如\autoref{fig_rod_3}, 人在岸上通过一小滑轮用绳子以速度 $v$ 拉船, 绳子与水平面的夹角为 $\theta$, 求船的速度 $u$。
\begin{figure}[ht]
\centering
\includegraphics[width=8cm]{./figures/f2377ac9dba24516.pdf}
\caption{人拉船模型} \label{fig_rod_3}
\end{figure}

解: 我们可以把从滑轮到船的这段绳子的长度缩短的速度用\autoref{eq_rod_6} 来计算, 这个速度也就是人的速度
\begin{equation}
v = -\dv{L}{t} = u \cos\theta~,
\end{equation}
所以
\begin{equation}
u = \frac{v}{\cos\theta}~.
\end{equation}
\end{example}

\begin{exercise}{}
三个小朋友 $ACB$ 两两相距 $30\Si{m}$, 接下来的任意时刻 $A$ 向着 $B$ 跑, $B$ 向着 $C$ 跑, $C$ 向着 $A$ 跑, 速度都相等。 求他们相遇时, 各跑了多少路程。
\end{exercise}

\subsubsection{角速度}
\pentry{圆周运动的加速度\nref{nod_CMAD}}{nod_2f9c}

首先假设杆长度不变, 若已知两端点的速度, 如何求出角速度呢? 我们可以在以 $\bvec r_1$ 为原点的无转动的参考系中观察,此时轻杆另一端的速度为 $\bvec v_2' = \bvec v_2 - \bvec v_1$ 且与 $\uvec r$ 垂直, 所以角速度大小为
\begin{equation}\label{eq_rod_8}
\omega = \frac{\abs{\bvec v_2 - \bvec v_1}}{L}~.
\end{equation}
角速度矢量(\autoref{sub_CMVD_1})为
\begin{equation}\label{eq_rod_4}
\bvec \omega = \uvec r \cross \frac{\bvec v_2 - \bvec v_1}{L}~.
\end{equation}
注意即使杆长度改变, 该式同样成立。

\subsubsection{加速度约束}
若我们想知道轻杆对其两端加速度的约束, 只需对\autoref{eq_rod_1} 两边再求一次时间导数, 即\autoref{eq_rod_2} 的二阶时间导数。
\begin{equation}
\dv[2]{t} (\bvec r_2 - \bvec r_1)^2 = 2(\bvec v_2 - \bvec v_1)^2 + 2(\bvec r_2 - \bvec r_1) \vdot (\bvec a_2 - \bvec a_1) = 0~,
\end{equation}
用\autoref{eq_rod_8} 把式中的 $(\bvec v_2 - \bvec v_1)^2$ 替换为 $\omega^2 L^2$, 再两边除以 $L$ 得
\begin{equation}\label{eq_rod_5}
\bvec a_1 \vdot \uvec r - \bvec a_2 \vdot \uvec r  = \omega^2 L~.
\end{equation}
$\omega$ 是杆的瞬时角速度的大小。 这就是说, 杆两端的加速度之和(向内为正)等于 $\omega^2L$。

一个简单的特例是若杆的一端固定, 另一端的\enref{向心加速度}{CMAD}就是熟悉的 $\omega^2L$。

另外若杆允许伸缩, 也不难证明
\begin{equation}
\dv[2]{L}{t} = \omega^2 L + \bvec a_2 \vdot \uvec r - \bvec a_1 \vdot \uvec r~.
\end{equation}


\begin{example}{}
在\autoref{ex_rod_1} 的模型中, 若已知端点的速度 $v_1, v_2$ 和 $\theta$, 求两点加速度 $a_1, a_2$ 的关系。
\begin{figure}[ht]
\centering
\includegraphics[width=6.5cm]{./figures/a562eaa8dd11e6fe.pdf}
\caption{杆的运动} \label{fig_rod_2}
\end{figure}

解: 由\autoref{eq_rod_4} 和\autoref{eq_rod_5} 得,
\begin{equation}
\omega = (v_1 \cos\theta + v_2 \sin\theta)/r~,
\end{equation}
\begin{equation}
a_2 \cos\theta - a_1 \sin\theta = \omega^2 r = (v_1 \cos\theta + v_2 \sin\theta)^2/r~.
\end{equation}
\end{example}

\begin{exercise}{}
在\autoref{ex_rod_2} 中, 若拉绳的人不仅有速度 $v$ 还有加速度 $a$, 求滑块的加速度。
\end{exercise}

\subsection{动力学}
\pentry{转动惯量\nref{nod_RigRot}}{nod_dddf}
轻杆没有质量, 也没有转动惯量。 假设我们只能在轻杆的两端对其施加两个力 $\bvec F_1, \bvec F_2$, 这两个力会满足什么条件呢? 首先, 轻杆受到的合力必须为 $\bvec 0$, 否则他就会马上被加速到无限快。 这意味着这两个力大小相等, 方向相反($\bvec F_1 + \bvec F_2 = \bvec 0$)。 其次, 它受到的和力矩必须也为 $\bvec 0$, 否则就会瞬间拥有无限大的角速度。 这意味着两个力必须共线, 即都延杆的方向。 于是我们可以令 $\bvec F_1 = F \uvec r$, $\bvec F_2 = -F \uvec r$。

现在, 无论轻杆如何运动, 这两个力对轻杆做功的功率为
\begin{equation}
P = \bvec F_1 \vdot \bvec v_1 + \bvec F_2 \vdot \bvec v_2 = F (\uvec r \vdot \bvec v_1 - \uvec r \vdot \bvec v_2)~,
\end{equation}
由\autoref{eq_rod_3} 可知功率恒为 0。
