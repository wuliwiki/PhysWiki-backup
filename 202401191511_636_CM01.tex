% 位力定理
% license Xiao
% type Tutor


位力定理是质点组力学在统计上的一个应用。在保守系下,该定理展示了“长时间”后系统的动能平均值及势能平均值的关系。作为牛顿力学的推论,位力定理可用于热力学中玻意耳定律的证明,可用于大尺度星系质量的估算。经典力学和量子力学的关系如此密切,你也很容易猜到,位力定理必然也会“出现”于量子力学中。

(注:本文使用爱因斯坦求和约定,即$x_i y_i=\Sigma x_i y_i$。另,物理量头上一点代表对时间求导)

设n个质点组成一质点系,$G=\bvec{r}_i\cdot \bvec{p}_i$,由链式法则我们有:

\begin{equation}
\frac{\dd G}{\dd t}=\dot{\bvec{r}}_i\cdot \bvec{p}_i+\bvec{r}_i\cdot\dot{\bvec{p}}_i=2T+\bvec{r}_i\cdot \bvec{F}_i~.
\end{equation}
在统计上,某物理量 $F$ 的时间平均值常被定义为 $\overline{F}=\frac{1}{t}\int Fdt $。同理
\begin{equation}
\overline{\frac{\dd G}{\dd t}}=\frac{1}{t}\int \frac{\dd G}{\dd t}\dd t=2\overline{ T}+\overline{{\bvec{r}_i\cdot \bvec{F}_i}}~.
\end{equation}

\begin{theorem}{位力定理}
对于 $\overline{\frac{\dd G}{\dd t}}=0,
\overline{ T}=-\frac{1}{2}\overline{{\bvec{r}_i\cdot \bvec{F}_i}}~.$
\end{theorem}
由于 $\frac{1}{t}\int \frac{\dd G}{\dd t}\dd t=\frac{1}{t}G\big|_{t_0}^{t_1}$,位力定理成立的场景很多,比如束缚态体系,$\lim \limits_{t \to \infty}G\big|_{t_0}^{t_1}\leq \left|{G_{max}-G_{min}}\right|$,即分母为有限值,则无穷远时间后定理所需的条件成立。
\begin{corollary}{势能为齐次线性函数}
容易证明,当系统为保守系,势能为齐次函数,即 $V(ax,ay,az)=a^nV(x,y,z)$ 时,有 $n\overline{V}=2\overline{T}$
\end{corollary}
proof.对于保守系,我们有$r_i\cdot F_i=r_i\cdot \nabla_i V$根据齐次函数的欧拉定理,我们有$r_i\cdot \nabla_i V=n V$。定理得证。
\begin{exercise}{玻意耳定律的证明}
\end{exercise}
提示:利用高斯定理。
\subsubsection{星系质量估算}  
在星系质量的估算中,把星系看成一质点系,简化动能形式即可利用位力定理求得星系质量。
