% 金属材料科普(草稿)

\begin{issues}
\issueDraft
\end{issues}
\footnote{本文内容与图片参考自刘智恩的《材料科学基础》以及Calister的 Material Science and Engineering An Introduction}

\subsection{原子、晶体、晶胞}
如果你的视力\footnote{可见光的波长(约为300-700nm)远大于此,因此凭光学显微镜是不可能看到如此细小的结构的.这也是为什么我们发明了电子显微镜)}好到可以看见纳米级别(大概$10^{-9}m =10^{-6} mm$)的金属结构,那么你会发现金属好像是由大量原子层层叠叠、有序堆积起来的.堆积的具体方式与金属的种类\footnote{有些金属的堆积方式不止一种,与温度、压力等有关,互称为同素异形体}有关.

\begin{figure}[ht]
\centering
\includegraphics[width=5cm]{./figures/MetInt_1.png}
\caption{晶体中原子的排列示意图} \label{MetInt_fig1}
\end{figure}

\begin{definition}{晶体}
原子(或分子)在三维空间按一定规律作周期性排列而形成的固体
\end{definition}

既然晶体中原子的排列是规律重复的,我们自然就能找出其中最小的一个重复单元,以反映这种排列方式的特征.这种最小单元被称为晶胞.
\begin{definition}{晶胞}
能够完全反应晶体几何特征的最小单元
\end{definition}
例如金属铁的晶胞是体心立方(BCC)结构,即正方体中心的一个原子与周围八个原子均相切,有点像这样:
\begin{figure}[ht]
\centering
\includegraphics[width=5cm]{./figures/MetInt_2.png}
\caption{铁的晶胞}} \label{MetInt_fig2}
\end{figure}

\subsection{缺陷}
如果金属中的原子都完全按这种理想的方式整整齐齐地排列,那材料科学也未免太无趣了(材料科学的书至少会薄半本!但同时,我们能用材料科学做的事也会少很多!).实际上,如果你的眼光放得长远一些,不只局限于几个原子的大小,就会发现真实的金属晶体往往存在原子排列偏离理想方式的区域,称为缺陷.

\begin{definition}{缺陷}
实际金属中原子偏离理想排列而出现的不完整区域
\end{definition}

\textbf{金属中缺陷的占比虽然不多,但却对材料的性能起到了决定性影响}.“缺陷”这个词往往让人以为缺陷对于材料性质都是不利的,但事实并非如此,有些缺陷对于材料某一方面的性能可以起到积极作用.夸张点说,半个材料科学与制造技术的研究对象,就是理解并利用缺陷,以制造符合预期的材料.

根据缺陷的空间尺度,缺陷一般被分为点缺陷、线缺陷与面缺陷.

\subsubsection{点缺陷}
点缺陷指的是单独少数原子的错误排列.
\begin{figure}[ht]
\centering
\includegraphics[width=8cm]{./figures/MetInt_3.png}
\caption{空位缺陷与间隙缺陷} \label{MetInt_fig3}
\end{figure}
空位:原子离开了自己的理想位置,形成了一个空位

间隙:原子插入了本不应存在原子的位置,一般是晶体的间隙

有时,一些其他种类的原子也会混入到金属晶体之中
\begin{figure}[ht]
\centering
\includegraphics[width=8cm]{./figures/MetInt_4.png}
\caption{杂原子的置换与间隙} \label{MetInt_fig4}
\end{figure}

(杂原子)置换:其他种类的原子替换了格点上的原子

(杂原子)间隙:其他种类的原子插入了间隙之中

\subsubsection{线缺陷}
线缺陷又称为位错,可细分为刃位错、螺位错、以及二者的组合 混合位错.位错的假设源自于材料性能的理论值与实际值过大的差异.位错相关理论于1930年代被提出,并在1950年代在电子显微镜下实际观察到位错后得到证实.(某种意义上说,位错理论是一个新鲜的概念,考虑到狭义相对论在1905年就被发现了!)

刃位错可以理解为完整晶体中插入了(或失去了)半个额外的原子面,或者说上部分额外多滑移了一个原子间距.
\begin{figure}[ht]
\centering
\includegraphics[width=10cm]{./figures/MetInt_6.png}
\caption{刃位错示意图1} \label{MetInt_fig6}
\end{figure}
\begin{figure}[ht]
\centering
\includegraphics[width=10cm]{./figures/MetInt_5.png}
\caption{刃位错示意图2} \label{MetInt_fig5}
\end{figure}

螺位错也可以理解为上半部分相对于下半部分的额外滑移,但是滑移的方向与刃位错不一致.
\begin{figure}[ht]
\centering
\includegraphics[width=10cm]{./figures/MetInt_7.png}
\caption{螺位错示意图1} \label{MetInt_fig7}
\end{figure}
\begin{figure}[ht]
\centering
\includegraphics[width=10cm]{./figures/MetInt_8.png}
\caption{螺位错示意图2} \label{MetInt_fig8}
\end{figure}

总之,位错使材料的两部分并不完整对齐.更抽象、数学地描述位错时,我们往往使用位错线、滑移面、burgers矢量等,不过这部分枯燥的内容暂时按下不谈.

% \subsubsection{面缺陷}
% 面缺陷包括金属的外表面(就是你能看到的那部分)、晶界、相界等等,后者将在下一节简要讨论.

\subsubsection{缺陷与材料热力学性质}
缺陷对材料的机械力学、热力学、化学乃至电学性能等都有深刻影响,可谓遇事不决缺陷背锅(?).此处,我们先简要介绍一下缺陷对材料热力学性能对影响.

以置换点缺陷为例.当一个额外的大原子置换了原本的小原子,会发生什么呢?假设你在一台电梯中,好巧不巧又挤进来一位善良的胖子,那么你会感觉到一股更大的压力.事实上原子也是如此,大原子挤压了相邻的其余原子,并造成了额外的压力.
\begin{figure}[ht]
\centering
\includegraphics[width=5cm]{./figures/MetInt_9.png}
\caption{置换大原子造成了内压应力} \label{MetInt_fig9}
\end{figure}
可以说,点缺陷在其周围形成了额外的力场,提升了系统总的能量.这个结论可以推广至所有缺陷,即\textbf{缺陷提升了系统的总能量}.

此外,大多数的缺陷(空位点缺陷是一个特例)还\textbf{提高了材料的自由能},这似乎意味着\textsl{缺陷是热力学不稳定的}.现实中,材料经由(传统手段)缓慢冷却,确实能得到较为完美的晶体;然而由于材料中存在大量形成缺陷的机制等,仍无法完全消除缺陷.

\begin{example}{鹤立鸡群的空位点缺陷}
根据热力学定律,一个稳定的系统需要同时满足熵最大与能量最低,即自由能最低.

当每个原子都安安静静乖乖巧巧地呆在自己的位置上时,尽管此时系统能量最低,但此时的系统非常有序,熵不高,自由能不处于最低.因此,总会有一些原子自发离去留下空位,以升高系统的熵并降低自由能.这就是为什么空位点缺陷反而是热力学稳定的.
\end{example}

\begin{example}{冷加工金属不适用于高温环境}
冷加工(这是强化金属的一种方法,即在低温下变形材料,使金属的强度更高,但塑形更差)后,金属内的位错含量大幅升高,金属处于热力学不稳定状态.

这使金属的回复温度(在较高的温度下,金属自发减少缺陷并降低机械强度)更低、速度更快.在高温下,随着金属的回复,机械性能将下降.
\end{example}

\subsection{微结构}
见完了缺陷的小打小闹,是时候继续调高眼界、往大的看了(大致是光学显微镜级别),就会发现...更大的缺陷了.这类在光学显微镜下可见的结构可以被称为微结构.微结构的种类繁多,这里主要举一些典型的例子.

\subsubsection{晶粒与晶界}
或许你还对\autoref{MetInt_fig1} (提示:原子的规则排列)记忆犹新.在整块金属中,原子还是按同一个方向相同地排序吗?答案当然是...否定的.实际中,金属的微观结构可能更像这样:
\begin{figure}[ht]
\centering
\includegraphics[width=5cm]{./figures/MetInt_10.png}
\caption{Atomsk随机生成的晶胞} \label{MetInt_fig10}
\end{figure}
在一定区域内,金属原子的排列位向相同;但在不同区域内,原子的排列位向就不相同了.这样,天然存在一个边界划分这些区域.这些边界被称为晶界,而晶界内所围成的区域称为晶胞.换句话说,金属整体可以看作是由一块块晶粒构成的.
\begin{figure}[ht]
\centering
\includegraphics[width=10cm]{./figures/MetInt_11.png}
\caption{晶界示意图,注意两侧晶粒位向的不同} \label{MetInt_fig11}
\end{figure}

\begin{figure}[ht]
\centering
\includegraphics[width=10cm]{./figures/MetInt_13.png}
\caption{光学显微镜下的铁($4.5\%$ C)} \label{MetInt_fig13}
\end{figure}

\subsubsection{孪晶界}
还有一种特殊类型的晶界.如果晶界两侧的晶体呈对称关系,那么这样的晶界也被称为孪晶界.
\begin{figure}[ht]
\centering
\includegraphics[width=10cm]{./figures/MetInt_12.png}
\caption{孪晶界} \label{MetInt_fig12}
\end{figure}

\subsubsection{相界}

\subsection{宏观材料}
