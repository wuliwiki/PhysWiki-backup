% 图
% keys 图
% license Usr
% type Tutor
\begin{issues}
\issueOther{该领域(图论)的相关概念尚不完善}
\issueOther{缺少参考文献,}
\issueTODO
\end{issues}
\pentry{二元关系\nref{nod_Relat}}{nod_84d8}

直观上,图要描述的是一些带有两端点的边和一些点构成的对象。这样的对象有以下几种情况:

两种可能的边:1.有方向的边,这是指边的端点有起止点之分的情形,此时的图称为有向图;2.无方向的边,这是指边的端点没有起止的差异,无论两个点在边的哪一头都被认为是相同的,这样的图叫无向图。

两种可能的点:1.有边连接的点,即这样的点是某些边的端点;2.无边连接的点,即这样的点不是任何边的端点,这样的点叫做孤立点。

3种可能的图:1.无点无边的图,这称为空图;2.有点无边的图;3.有点有边的图。因为边必然包含点,所以没有无点有边的图。

图论的所有基本概念都是为了建立起以上图像的严格化语言。


\subsection{基础定义}

当建立起某一对象的数学语言时,我们应当首先借助集合论的基本工具。因此,图包含两个集合,它们称为点和边。而边是连接两个端点的边,因此边可以通过指定它的端点确定。也就是说,可以把边认为是点集的二元子集。当边有向时,只需用有序的点对来表示即可。为方便描述起见,我们将用一个记号来表示集合的 $n$ 元子集构成的全体。

\begin{definition}{$n$ 元子集的族}
设 $A$ 是集合,则记 $A$ 的所有 $n$ 元子集的全体为 $[A]^n$。即
\begin{equation}
[A]^n=\{\{x_1,\cdots,x_n\}|x_i\in A,i=1,\cdots n\}.~
\end{equation}
\end{definition}




\begin{definition}{图,顶点,边,阶数}
设 $V$ 是一个集合,$E\subset [A]^2$,则称二元组 $G=(V,E)$ 为\textbf{图}(graph)。$V$ 的元素称为 $G$ 的\textbf{顶点}(vertex),$E$ 的元素称为 $G$ 的\textbf{边}(edge)。$V$ 的基数(元素个数)称为图 $G$ 的\textbf{阶数}(order),记作 $\abs{G}$,而 $E$ 的基数记作 $\norm{G}$。
\end{definition}

当然,顶点也可以称为节点(node)或点(point),边还能叫做线(line),具有顶点集 $V$ 的图也能称为 $V$ 上的图(a graph on $V$),往往图 $G$ 的顶点集记作 $G(V)$,边集记作 $G(E)$,而 $x\in G(V),e\in G(E)$ 直接记作 $x\in G,w\in G$ 等等,这都是习惯的问题,熟悉它们往往是有好处的。

有些地方会约定 $E\cap V=\emptyset$ 来避免记号的混淆。这里不这样做,因为 $E$ 是 $V$ 的二元子集构成的集,其元素是二元子集,不是二元子集的元素本身,因此不会有相交的情况。

\begin{definition}{图的惯用表示}
按照习惯,在绘制图的时候,用\textbf{圆圈}来代表点,而在两点间画一条\textbf{线}来表示连接它们的边。
\end{definition}
如何绘制圆圈和线是不重要的,重要的是正确体现点对之间是否有边。

根据图的阶的分类,又将图分为有限图、无限图和可数图。
\begin{definition}{有限图,无限图,可数图}
设 $G$ 是图,若 $\abs{G}\in\mathbb N$,则称 $G$ 是\textbf{有限的}(fini) 
\end{definition}


%\begin{definition}{图;顶点;边}
%图$G(V,E)$是集合V上的一种二元关系$E$。

%集合$V$的元素称为图的顶点,若两个顶点之间有这种确定的二元关系,则称有一条边连这两个点。

%一个图的顶点的数目称为这个图的阶,记 作$|G|$,图的边的数目称为它的度,记作$||G||$。
%\end{definition}
\begin{definition}{关联;相邻}
\begin{itemize}
\item 若有一条边连一个图的某两个顶点,则称这两个顶点相邻,并称这两个顶点为这条边的端点。
\item 若某一顶点是某一条边的端点,则称这个顶点和这条边关联。
\item 若两条边和同一顶点关联,则称这两条边相邻.
\end{itemize}
\end{definition}
\subsection{2.特殊图元素}
\begin{definition}{特殊点;特殊边}
\begin{itemize}
\item 两个端点是同一个顶点的边称为环。
\item 若某条边的两个端点不是同一个顶点,且只有一条边连这两个顶点,则称这条边为杆。
\item 以某两顶点为端点的边可能不止一条,这时称连这两个顶点的边为重边。
\end{itemize}
\end{definition}
\begin{definition}{特殊图}
\begin{itemize}
\item 只有一个顶点而没有边的图称为平凡图,没有边的图称为孤立图。
\item 既可以有环,也可以有重边的图称为准图.
\\没有环而可能有重边的图称为带重图.
\\没有重边而可能有环的图称为带环图.
\\既没有重边也没有环的图称为简单图,每两个顶点都相邻的简单图称为完全图。n阶完全图记作$K^{n}$
\item 若一个图的阶是有限的,则称这个图为有限图,否则称这个图为无限图。
\item 若一个n阶图的点用 1 , 2 , … , n 来代表,则称它为标定图
\\若在图的每一条边上赋以一个实数或者对于每个节点赋以一个实数,则称它为赋权图。
\end{itemize}
\end{definition}
\begin{theorem}{n阶完全图$K^{n}$的度}
\begin{equation}
||K^{n}||=\frac{n(n-1)}{2}~.
\end{equation}
证明:使用第一数学归纳法:
\\当n=1时,完全图为孤立图,故||K||=0,下设n时成立,考虑n+1的情形:
\\由于$K^{n+1}$可以由$K^{n}$添加1个顶点及n条边得到知:
\\ $||K^{n+1}||=||K^{n}||+n=\frac{n(n-1)}{2}+n=\frac{n(n+1)}{2}$
\\得证.
\end{theorem}
\subsection{3.点的性质}
\begin{definition}{顶点的次(度)}
设点$v \in V$,称图$G$中以顶点$v$作为端点的边数为点$v$的度,记作$d(v)$
\\注:顶点上的环计数时计两次
\end{definition}
\begin{theorem}{简单图中顶点度与边数的关系}
在简单图$G$中,有:
$\sum_{v \in V}{d(v)}=2||G||$\footnote{有时称为握手引理(Handshaking Lemma)}
\\证明:由简单图中的每一条边有且仅有两个端点,且两个相邻的顶点间仅有一条边立得
\end{theorem}
\begin{corollary}{简单图中奇度点个数}
在简单图$G$中,有:$2\mid \sum\limits_{v\in V\atop 2\nmid d(v)}1$
\end{corollary}

\begin{definition}{与度有关的特殊图元素}

\end{definition}
注:以上内容参考了《数学辞海》卷二
