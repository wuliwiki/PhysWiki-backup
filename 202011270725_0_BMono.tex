% 磁单极子
% keys 磁单极子|麦克斯韦方程组|高斯单位

\begin{issues}
\issueDraft
\end{issues}

\pentry{麦克斯韦方程组\upref{MWEq}}

\footnote{参考 Wikipedia \href{https://en.wikipedia.org/wiki/Magnetic_monopole}{相关页面}.}若定义磁荷 $q_m$ 的单位为 $\Si{Am}$(安培·米), $\rho_m$ 为磁荷的体密度, 则
\begin{align}
&\div \bvec E = \frac{\rho}{\epsilon_0}\\
&\curl \bvec E = - \mu_0 \bvec j_m -\pdv{\bvec B}{t}\\
&\div \bvec B = \mu_0 \rho_m \\
&\curl \bvec B = \mu_0 \bvec j + \mu_0\epsilon_0 \pdv{\bvec E}{t}
\end{align}
洛伦兹力
\begin{equation}
\bvec F = q \qty(\bvec E + \frac{\bvec v}{c}\cross \bvec B)
\end{equation}

\subsubsection{高斯单位}
高斯单位制\upref{GaussU}下, 麦克斯韦方程组和洛伦兹力具有更对称的形式
\begin{align}
&\div \bvec E = 4\pi\rho\\
&\curl \bvec E = -\frac{1}{c}\pdv{\bvec B}{t}  - \frac{4\pi}{c}\bvec j_m\\
&\div \bvec B = 4\pi\rho_m \\
&\curl \bvec B = \frac{1}{c}\pdv{\bvec E}{t} + \frac{4\pi}{c} \bvec j
\end{align}
洛伦兹力
\begin{equation}
\bvec F = q \qty(\bvec E + \frac{\bvec v}{c}\cross \bvec B) + q_m \qty(\bvec B - \frac{\bvec v}{c}\cross \bvec E)
\end{equation}
