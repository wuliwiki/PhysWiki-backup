% 夹逼定理
% keys 极限|函数|数列|序列
% license Usr
% type Tutor

\begin{issues}
\issueDraft
\end{issues}
\pentry{极限定义 \nref{nod_Lim},极限运算法则 \nref{nod_LimOp}}{nod_014c}
夹逼定理(Squeeze Theorem,或称夹挤定理)

\subsection{数列极限}

\subsection{函数极限}

若$g(x)\leq f(x)\leq h(x)$,且$\lim _{x\to x_0}h(x)=\lim _{x\to x_0}g(x)=a$,则
\begin{equation}
\lim _{x\to x_0}f(x)=a.~
\end{equation}

\begin{lemma}{不等式一端为0时的极限}
\begin{equation}
0\leq f(x)\leq h(x),\lim _{x\to x_0}h(x)=0\implies\lim _{x\to x_0}f(x)=0.~
\end{equation}

证明:

由函数极限定义,$\forall\varepsilon 0,\exists\delta 0,0<|x-x_0|<\delta\implies|h(x)|<\varepsilon$。
 
对$0<|x-x_0|<\delta$,由$0\leq f(x)\leq h(x)$,$|f(x)|\leq |h(x)|<\varepsilon$,即$\lim _{x\to x_0}f(x)=0$。

\end{lemma}

 由$g(x)\leq f(x)\leq h(x)$,有$0\leq f(x)-g(x)\leq h(x)-g(x)$。
 
 $\lim _{x\to x_0}(h(x)-g(x))\overset{\mathrm{1}}{=}\lim _{x\to x_0}h(x)-\lim _{x\to x_0}g(x)=a-a=0$,由引理可知$\lim _{x\to x_0}(f(x)-g(x))=0$。
 
 $$\begin{align*} \lim _{x\to x_0}f(x) &= \lim _{x\to x_0}[(f(x)-g(x))+g(x)] \\ 
 &\overset{\mathrm{1}}{=} \lim _{x\to x_0}(f(x)-g(x))+\lim _{x\to x_0}g(x) \\ 
 &= 0+a\\ 
 &=a \end{align*}.~$$
 等号1:[[极限#加减]]运算