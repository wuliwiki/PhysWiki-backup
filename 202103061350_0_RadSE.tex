% 球坐标系中的定态薛定谔方程
% keys 球坐标|柱坐标|定态薛定谔方程|拉普拉斯算子|径向动量

\pentry{定态薛定谔方程\upref{SchEq}, 球坐标系中的亥姆霍兹方程\upref{SphHHz}, 球坐标系中的角动量算符\upref{SphAM}}

本文使用原子单位制\upref{AU}. 我们希望在球坐标中求解定态薛定谔方程
\begin{equation}\label{RadSE_eq2}
-\frac{1}{2m}\laplacian \psi(\bvec r) + V(\bvec r)\psi(\bvec r) = E\psi(\bvec r)
\end{equation}
当势能 $V(\bvec r)$ 处处为零时, 这就变成了球坐标系中的亥姆霍兹方程\upref{SphHHz}. 以下考虑一般情况, 但过程也和求解亥姆霍兹方程类似.

使用球坐标的拉普拉斯算子(\autoref{SphNab_eq4}~\upref{SphNab}) 可以将哈密顿算符表示为
\begin{equation}
H = -\frac{1}{2m} \laplacian + V(\bvec r) =  K_r + \frac{L^2}{2mr^2} + V(\bvec r)
\end{equation}
其中径向动量算符和角动量平方算符为(\autoref{SphAM_eq1}~\upref{SphAM})分别为
\begin{equation}
K_r =-\frac{1}{2m} \laplacian_r =  - \frac{1}{2m} \qty(\pdv[2]{r} + \frac2r \pdv{r}) = -\frac{1}{2mr^2} \dv{r} \qty(r^2\pdv{r})
\end{equation}
\begin{equation}\label{RadSE_eq3}
L^2 = -\laplacian_\Omega = -\qty[ \frac{1}{\sin\theta} \pdv{\theta} \qty(\sin \theta \pdv{u}{\theta}) + \frac{1}{\sin^2 \theta} \pdv[2]{u}{\phi}]
\end{equation}
注意角动量算符不含 $r$.

定态薛定谔方程\autoref{RadSE_eq2} 变为
\begin{equation}
\qty(K_r + \frac{L^2}{2mr^2} + V - E)\Psi(\bvec r) = 0
\end{equation}
我们假设势能函数只与粒子到原点的距离有关, 即 $V = V(r)$. 两边乘以 $r^2$ 可以将 $r$ 与角向变量 $\theta, \phi$ (简写为 $\uvec r$)分离, 令 $\Psi = R(r)Y(\uvec r)$.

解得 $Y(\uvec r)$ 为球谐函数 $Y_{l,m}(\uvec r)$ 满足
\begin{equation}
L^2 Y_{l,m}(\uvec r) = l(l+1) Y_{l,m}(\uvec r)
\end{equation}
分离变量后 $R(r)$ 满足的方程一般被称为\textbf{径向薛定谔方程}
\begin{equation}\label{RadSE_eq6}
K_r R_l(r) + \qty[V(r) + \frac{l(l+1)}{2mr^2} ]R_l(r) = ER(r)
\end{equation}
我们可以通过变量替换将其化为更简洁的形式. 实际上是不同 $l$ 的一系列方程.

定义(\textbf{约化径向波函数(scaled radial wave function)}为
\begin{equation}
\psi_l(r) = r R_l(r)
\end{equation}
代入\autoref{RadSE_eq6}, 第一项变为
\begin{equation}
K_r R_l =  - \frac{1}{2m} \qty(\dv[2]{R_l}{r} + \frac2r \dv{R_l}{r}) = -\frac{1}{2mr^2} \dv{r} \qty(r^2\dv{R_l}{r}) =  -\frac{1}{2mr} \dv[2]{\psi_l}{r}
\end{equation}
所以\autoref{RadSE_eq6} 两边乘 $r$ 后化简为
\begin{equation}\label{RadSE_eq1}
-\frac{1}{2m} \dv[2]{\psi_l}{r} + \qty[V(r) + \frac{l(l + 1)}{2mr^2}]\psi_l = E\psi_l
\end{equation}
这是径向薛定谔方程更常见的形式, 方括号中可以看作\textbf{一维等效势能}(类比经典力学的情况\autoref{CenFrc_eq6}~\upref{CenFrc}). 该方程的解取决于 $V(r)$ 的具体形式. 一般来说, 对于每个较小的 $l$, 可能找到有限个束缚态, 我们把它用一个整数 $n$ 编号, 叫做\textbf{主量子数(principal quantum number)}, 而 $l$ 叫做\textbf{角量子数(angular quantum number)}, 束缚态能量 $E_{n,l}$ 由这两个量子数共同决定. 随着 $l$ 不断增加, 方括号中的等效势能越来越浅, 最终有可能导致径向方程不存在束缚态解. 氢原子是一个显著的例外, 由于 $1/r$ 势能的特殊性, 它不但有无穷个束缚态, 而且束缚态能量和 $l$ 无关, 这在一般情况下是不成立的. 显然,对于有限深的势能函数 $V(r)$ 只存在有限个束缚态.

% 未完成, 讨论一下束缚态和连续态
% 令束缚态的解为 $u_{n,l}(r)$, 对应能量(本征值)为 $E_n$

总波函数体积分要求
\begin{equation}
\int \abs{R}^2 \abs{Y}^2 r^2 \dd{\Omega}\dd{r}  = 1
\end{equation}
球谐函数已经满足 $\int \abs{Y}^2 \dd{\Omega} = 1$,  所以, 要求
\begin{equation}
\int \abs{R}^2 r^2 \dd{r}  = 1
\end{equation}
即
\begin{equation}
\int \abs{\psi_l(r)}^2 \dd{r}  = 1
\end{equation}
