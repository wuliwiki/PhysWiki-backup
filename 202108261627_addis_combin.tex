% 组合
% 排列|组合|阶乘

\begin{issues}
\issueDraft
\end{issues}

\pentry{阶乘\upref{factor}}

\footnote{参考 Wikipedia \href{https://en.wikipedia.org/wiki/Combination}{相关页面}.}数学上, \textbf{组合(combination)}用于计算从 $n$ 个不同的物体中选取 $0 \leqslant r \leqslant n$ 个, 有几种方法. 国内高中课本常用 $C_n^r$ 来表示, 但我们使用 $\pmat{n\\r}$.

\begin{equation}
\pmat{n\\r} = \frac{n!}{(n-r)!r!}
\end{equation}
我们可以从排列来推导组合,在组合中认为 ${a,b}$, ${b,a}$ 是等效的, 在排列中认为两者是不等的, 现在我们有 $n$ 个元素,我们要从中选取 $m$ 个元素, 即 $C_n^m$.
我们先根据组合的知识可以知道 $A_n^m$ 表示 $n$ 个元素,从中选取 $m$ 个再对选中的 $m$ 个元素进行全排,我们就可以得到如下公式
\begin{equation}
A_n^m = C_n^m \cdot A_m^m
\end{equation}
对等式进行变换,可得
\begin{equation}\label{combin_eq1}
C_n^m = \frac{A_n^m}{A_m^m}
\end{equation}
对于这个公式,我们可以理解为,从排列中排除组合中认为等效的组合.
我们将\autoref{combin_eq1} 展开,可得
\begin{equation}\label{combin_eq2}
C_n^m = \frac{n(n - 1) \cdots (n - m + 1)}{m(m-1)\cdots 1}
\end{equation}
进一步变换,可得
\begin{equation}
C_n^m = \frac{n!}{m!(n-m)!}
\end{equation}
