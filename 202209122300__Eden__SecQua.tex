% 二次量子化
% 二次量子化|多体系统|玻色统计|费米统计

\pentry{全同粒子\upref{IdPar},全同粒子的统计\upref{IdParS}}

薛定谔方程是关于单粒子的量子力学,而如果我们考虑一个多粒子体系,研究粒子间相互作用对多体系统的影响,则有必要建立一个关于多粒子的量子力学.一个直接的想法是,讲单个时空坐标变量的波函数拓展为 $N$ 个变量的波函数 $\psi(x_1,\cdots,x_N)$,波函数的模方具有概率的意义,因此可以乘上一个系数使它归一化.

然而当我们讨论全同粒子的时候,例如 $N$ 个电子组成的体系,上述波函数的定义并没有体现粒子的全同性,波函数不具有交换对称性或交换反对称性\footnote{玻色统计满足交换对称性 $\psi(x_1,x_2,\cdots)=\psi(x_2,x_1,\cdots)$,而费米统计满足交换反对称性,$\psi(x_1,x_2,\cdots)=-\psi(x_2,x_1,\cdots)$.}.因此有必要将波函数对称化.我们令 $\xi=\pm 1$,$\xi=1$ 对应玻色统计,而 $\xi=-1$ 对应费米统计.那么可以定义新的对称化的波函数为
\begin{equation}
\psi'(x_1,\cdots,x_N)\propto \sum_{P\in S_N} \xi^P \psi(x_{P_1},x_{P_2},\cdots,x_{P_N})
\end{equation}
其中 $\xi^P$ 中的指数 $P$ 看作是排列 $P\in S_N$ 的逆序数,每交换排列 $P$ 的某两个 $P_i,P_j$,逆序数都会改变 $\pm 1$.

$N$ 体系统的波函数所在的 Hilbert 空间记为 $\mathcal{H}_N$.完整的 Hilbert 空间被称为 Fock 空间,记为 $\mathcal{F}$,那么我们可以将 $\mathcal{F}$ 表示为一切 $\mathcal{H}_N$ 的直和,即
\begin{equation}
\mathcal{F}=\mathcal{H}_1 \oplus \mathcal{H}_2 \oplus \cdots
\end{equation}
为了更清楚地了解 $\mathcal{F}$ 的结构,我们需要从几个方面进行研究:首先是研究 $\mathcal{H}_N$ 的一组正交完备基底,有了基底以后我们可以用这些基函数的线性组合来表示任意 $N$ 体系统,经过合适的基底构造还可以将单粒子态与多粒子态联系起来;然后我们需要研究产生算符和湮灭算符,它们将不同粒子数的 Hilbert 空间之间关联了起来.最后,我们需要研究多粒子系统的算符,这里面包括能量算符、动量算符、相互作用势能算符等等,这是二次量子化的一个非常重要的内容.
\subsection{Fock 空间的基底}
最简单的一组基底为坐标表象下的基底,我们将它记为 $\ket{x_1\cdots x_N}$(类似于单粒子 Hilbert 空间的 $\ket{x}$ 基底),经过适当的归一化以后这组基底满足正交完备性.对于多体系统的这组基底,我们约定每交换两个坐标 $x_i,x_j$,态矢 $\ket{x_1\cdots x_N}$ 将改变一个 $\xi$ 的因子,即满足交换对称性
\begin{equation}
\ket{\cdots x_i \cdots x_j \cdots} =
\xi \ket{\cdots x_j \cdots x_i \cdots} 
\end{equation}

我们希望从单粒子波函数出发进行构造,定义这样的多粒子态
\begin{equation}
\ket{\psi_1\cdots\psi_N}=\frac{1}{\sqrt{N!}}\sum_P \xi^P\psi_{P_1}(x_1)\psi_{P_2}(x_2)\psi_{P_3}(x_3)\cdots \psi_{P_N}(x_N)
\end{equation}
定义它的坐标表象为
\begin{equation}
\braket{x_1\cdots x_N}{\psi_1\cdots\psi_N}
\end{equation}
