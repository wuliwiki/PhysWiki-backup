% 张量积(综述)
% license Pub
% type Wiki

本文根据 CC-BY-SA 协议转载翻译自维基百科\href{https://en.wikipedia.org/wiki/Tensor_product}{相关文章}。

在数学中,两个向量空间\( V \)和\( W \)(在相同的域上)的张量积\( V \otimes W \)是一个向量空间,它与一个双线性映射\(V \times W \to V \otimes W\)相关联,该映射将一对\( (v, w) \),其中\( v \in V\), \( w \in W \),映射到\( V \otimes W \)中的一个元素,表示为\( v \otimes w \)。

形式为\( v \otimes w \)的元素称为\( v \)和\( w \)的张量积。\( V \otimes W \)中的元素称为张量,两个向量的张量积有时被称为**初等张量**或**可分解张量**。初等张量生成了\( V \otimes W \),即 \( V \otimes W \) 中的每个元素都是初等张量的和。如果给定 \( V \)和\( W \) 的基,则\( V \otimes W \)的基是所有\( V \)的基元素与\( W \)的基元素的张量积。

两个向量空间的张量积捕捉了所有双线性映射的性质,具体而言,来自\( V \times W \)到另一个向量空间\( Z \)的双线性映射可以通过线性映射\(V \otimes W \to Z\) 唯一地分解(见下文标题为“普遍性质”的部分),即该双线性映射与从张量积 \( V \otimes W \)到\( Z \) 的唯一线性映射相关联。

张量积在许多应用领域中都有使用,包括物理学和工程学。例如,在广义相对论中,重力场通过度量张量来描述,度量张量是一个张量场,每个空间时间流形的点上都有一个张量,并且每个张量属于该点的余切空间与自身的张量积。
\subsection{定义与构造}
两个向量空间的张量积是一个向量空间,它的定义是直到同构的。定义它有几种等价的方式。大多数方式都是显式地定义一个向量空间,称为张量积,通常,这些定义的等价性证明几乎是直接由所定义的向量空间的基本性质得出的。

张量积也可以通过普遍性质来定义;见下文的“普遍性质”部分。与所有的普遍性质一样,所有满足该性质的对象通过一个唯一的同构进行同构,并且这个同构与普遍性质兼容。当使用这种定义时,其他定义可以看作是满足普遍性质的对象的构造,并且证明存在满足普遍性质的对象,即证明张量积的存在。
\subsubsection{从基底出发}
设\( V \)和\( W \) 是定义在域\( F \)上的两个向量空间,分别具有基底\( B_V \)和\( B_W \)。

向量空间\( V \)和\( W \)的张量积\( V \otimes W \)是一个向量空间,其基底是所有形式为\( v \otimes w \)的集合,其中\( v \in B_V \)且 \( w \in B_W \)。这个定义可以通过以下方式进行形式化(这个形式化通常在实践中不常用,因为前述的非正式定义通常已经足够):  
\( V \otimes W \)是从笛卡尔积\( B_V \times B_W \)到 \( F \)的所有函数的集合,这些函数有有限个非零值。通过逐点操作,使得\( V \otimes W \)成为一个向量空间。将\( (v, w) \)映射为 1,其他\( B_V \times B_W \)中的元素映射为 0 的函数表示为\( v \otimes w \)。

集合\(\{v \otimes w \mid v \in B_V, w \in B_W\}\)然后直接构成了\( V \otimes W \)的基底,这个基底被称为基底\( B_V \)和\( B_W \)张量积。

我们可以等价地定义\( V \otimes W \)为\( V \times W \)上的双线性形式的集合,这些双线性形式在\( B_V \times B_W \)的有限个元素上是非零的。为了证明这一点,给定 \( (x, y) \in V \times W \)和一个双线性形式 \( B: V \times W \to F \),我们可以将\( x \)和\( y \)在基底 \( B_V \)和\( B_W \)上分解为:
\[
x = \sum_{v \in B_V} x_v v \quad \text{和} \quad y = \sum_{w \in B_W} y_w w~
\]
其中,只有有限个\( x_v \)和\( y_w \)是非零的,利用\( B \)的双线性特性,我们得到:
\[
B(x, y) = \sum_{v \in B_V} \sum_{w \in B_W} x_v y_w B(v, w)~
\]
因此,我们看到,对于任何\( (x, y) \in V \times W \),\( B \) 的值是唯一且完全由它在\( B_V \times B_W \)上的取值决定的。这使我们能够将之前在\( B_V \times B_W \) 上定义的映射 \( v \otimes w \)扩展为双线性映射\( v \otimes w: V \times W \to F \),通过定义:
\[
(v \otimes w)(x, y) := \sum_{v' \in B_V} \sum_{w' \in B_W} x_{v'} y_{w'} (v \otimes w)(v', w') = x_v y_w~
\]
然后,我们可以将任何双线性形式 \( B \) 表示为(可能是无限的)形式线性组合,基于\( v \otimes w \)映射,如下所示:
\[
B = \sum_{v \in B_V} \sum_{w \in B_W} B(v, w) (v \otimes w)~
\]
使得这些映射类似于向量空间\( \text{Hom}(V, W; F) \)的 Schauder 基底,其中\( \text{Hom}(V, W; F) \)是所有双线性形式的集合,定义在 \( V \times W \) 上。如果要使其成为一个合适的 Hamel 基底,则只需要添加要求 \( B \)在\( B_V \times B_W \)的有限个元素上非零,并考虑这些映射的子空间。

在任何一种构造中,两个向量的张量积是从它们在基底上的分解定义的。更精确地说,取\( x \in V \)和\( y \in W \)的基底分解,如前所述:
\[
\begin{aligned}
x \otimes y &= \left( \sum_{v \in B_V} x_v v \right) \otimes \left( \sum_{w \in B_W} y_w w \right) \\
&= \sum_{v \in B_V} \sum_{w \in B_W} x_v y_w v \otimes w
\end{aligned}~
\]
这个定义显然是从 \( B(v, w) \) 在 \( B(x, y) \) 双线性展开中的系数中推导出来的,使用了基底 \( B_V \) 和 \( B_W \),正如上面所做的那样。然后,可以很容易地验证,通过这个定义,映射\(\otimes : (x, y) \mapsto x \otimes y\)是从\( V \times W \)到\( V \otimes W \)的双线性映射,满足张量积的普遍性质,即任何张量积的构造都满足该性质(见下文)。

如果将其排列成一个矩形阵列,\( x \otimes y \)的坐标向量就是\( x \)和 \( y \)的坐标向量的外积。因此,张量积是外积的一种推广,即它是超越坐标向量的抽象。

这个张量积定义的一个局限性是,如果改变基底,定义的张量积会不同。然而,在一个基底上分解另一个基底的元素定义了两个向量空间的张量积之间的典范同构,这使得它们可以被识别。此外,与以下两种替代定义相反,这个定义不能扩展为环上模的张量积的定义。
\subsubsection{作为商空间}
可以通过以下方式构造一个与基底无关的张量积。

设\( V \)和\( W \)是定义在域\( F \)上的两个向量空间。

首先考虑一个向量空间\( L \),它的基底是笛卡尔积\( V \times W \)。也就是说,\( L \)的基底元素是形如\( (v, w) \) 的对,其中\( v \in V \)和\( w \in W \)。为了得到这样的向量空间,可以将其定义为从\( V \times W \to F \) 的函数的向量空间,这些函数有有限个非零值,并将\( (v, w) \)与在\( (v, w) \)上取值为 1、其他地方取值为 0 的函数等同。

设\( R \)是\( L \)的一个线性子空间,它由张量积必须满足的关系所生成。更精确地说,\( R \)由以下形式的元素所生成:
\[
\begin{aligned}
(v_1 + v_2, w) &- (v_1, w) - (v_2, w), \\
(v, w_1 + w_2) &- (v, w_1) - (v, w_2), \\
(sv, w) &- s(v, w), \\
(v, sw) &- s(v, w),
\end{aligned}~
\]
其中\( v, v_1, v_2 \in V \),\( w, w_1, w_2 \in W \),且\( s \in F \)。

然后,张量积定义为商空间:
\[
V \otimes W = L / R~
\]
在这个商空间中,\( (v, w) \)的像表示为\( v \otimes w \)。

可以直接证明,这种构造的结果满足下文讨论的普遍性质。(一个非常类似的构造也可以用来定义模的张量积。)
\subsubsection{普遍性质}
\begin{figure}[ht]
\centering
\includegraphics[width=6cm]{./figures/61d9a75be843e63e.png}
\caption{张量积的普遍性质:如果 \( h \) 是双线性的,则存在一个唯一的线性映射 \( \tilde{h} \),使得图形是可交换的(即,\( h = \tilde{h} \circ \phi \))。} \label{fig_ZLJ_1}
\end{figure}
在本节中,将描述张量积满足的普遍性质。与所有普遍性质一样,满足该性质的两个对象通过一个唯一的同构相互关联。因此,这是一种(非构造性的)定义两个向量空间张量积的方法。在这种背景下,前面定义的张量积构造可以看作是已定义张量积存在性的证明。

这种方法的一个结果是,张量积的每个性质都可以从普遍性质中推导出来,实际上,人们可以忽略用于证明其存在性的具体方法。

张量积的“普遍性质定义”如下(回想一下,双线性映射是一个在其每个参数上分别线性的函数):

两个向量空间\( V \)和\( W \)的张量积是一个向量空间,记作\( V \otimes W \),以及一个双线性映射\(\otimes : (v, w) \mapsto v \otimes w\)从\( V \times W \) 到 \( V \otimes W \),使得,对于每一个双线性映射\( h: V \times W \to Z \),存在一个唯一的线性映射\( \tilde{h}: V \otimes W \to Z \),使得\(h = \tilde{h} \circ \otimes\)(即,对于每个\( v \in V \)和\( w \in W \),有\( h(v, w) = \tilde{h}(v \otimes w) \))。
\subsubsection{线性不相交}
像上面的普遍性质一样,以下的表述也可以用来判断给定的向量空间和给定的双线性映射是否构成张量积。\(^\text{[1]}\)

\textbf{定理} — 设 \( X \)、\( Y \)和\( Z \)为复向量空间,且\( T: X \times Y \to Z \) 是一个双线性映射。则 \( (Z, T) \)是\( X \)和\( Y \)的张量积,当且仅当\(^\text{[1]}\)\( T \)的像生成整个\( Z \)(即,\( \text{span} \ T(X \times Y) = Z \)),并且\( X \) 和 \( Y \)是 \( T \)-线性不相交的,定义上是指对于所有正整数\( n \)和所有元素\( x_1, \dots, x_n \in X \) 和 \( y_1, \dots, y_n \in Y \),如果\(\sum_{i=1}^{n} T(x_i, y_i) = 0\), 则  
\begin{enumerate}
\item 如果所有 \( x_1, \dots, x_n \) 线性无关,则所有 \( y_i \) 都为 0;  
\item 如果所有 \( y_1, \dots, y_n \) 线性无关,则所有 \( x_i \) 都为 0。
\end{enumerate}
等价地,\( X \) 和 \( Y \)是\( T \)-线性不相交的,当且仅当对于所有在\( X \)中线性无关的序列\( x_1, \dots, x_m \)和在\( Y \)中线性无关的序列\( y_1, \dots, y_n \),向量\(\{ T(x_i, y_j) : 1 \leq i \leq m, 1 \leq j \leq n \}\)
是线性无关的。

例如,立即可以得出,如果\( X = \mathbb{C}^m \)和\( Y = \mathbb{C}^n \),其中 \( m \)和\( n \)是正整数,则可以设置\( Z = \mathbb{C}^{mn} \),并定义双线性映射为:
\[
\begin{aligned}
T: \mathbb{C}^m \times \mathbb{C}^n &\to \mathbb{C}^{mn}\\
(x, y) = \left( (x_1, \ldots, x_m), (y_1, \ldots, y_n) \right) &\mapsto (x_i y_j)_{i=1, \ldots, m, j=1, \ldots, n}
\end{aligned}~
\]
从而形成\( X \)和\( Y \)的张量积。\(^\text{[2]}\)通常,这个映射\( T \)被表示为\( \otimes \),因此有:\(x \otimes y = T(x, y)\)

作为另一个例子,假设\( \mathbb{C}^S \)是定义在集合\( S \)上的所有复值函数的向量空间,集合\( S \) 上的加法和标量乘法按点定义(意味着\( f + g \) 是映射\( s \mapsto f(s) + g(s) \) 和 \( c f \) 是映射\( s \mapsto c f(s) \))。设 \( S \) 和 \( T \) 是任意集合,对于任何\( f \in \mathbb{C}^S \) 和 \( g \in \mathbb{C}^T \),让\( f \otimes g \in \mathbb{C}^{S \times T} \)表示由\( (s, t) \mapsto f(s) g(t) \)定义的函数。如果\( X \subseteq \mathbb{C}^S \)和\( Y \subseteq \mathbb{C}^T \)是向量子空间,则向量子空间 \( Z := \text{span} \{ f \otimes g : f \in X, g \in Y \} \) 与双线性映射一起:
\[
\begin{aligned}
X \times Y &\to Z \\
(f, g) &\mapsto f \otimes g
\end{aligned}~
\]
构成\( X \)和\( Y \)的张量积。\(^\text{[2]}\)
\subsection{性质}
\subsubsection{维度}  
如果\( V \)和\( W \)是有限维向量空间,那么\( V \otimes W \)也是有限维的,且其维度是\( V \)和\( W \)维度的乘积。

这是因为\( V \otimes W \) 的基底是通过取\( V \)的基底元素与\( W \)的基底元素的所有张量积来构成的。
\subsubsection{结合性} 
张量积是结合的,意味着给定三个向量空间\( U \)、\( V \)、\( W \),存在一个典范同构:
\[
(U \otimes V) \otimes W \cong U \otimes (V \otimes W)~
\]
该同构将\( (u \otimes v) \otimes w \) 映射到 \( u \otimes (v \otimes w) \)。

这使得在多个向量空间或向量的张量积中可以省略括号。
\subsubsection{作为向量空间运算的交换性}
两个向量空间\( V \)和\( W \)的张量积是交换的,意味着存在一个典范同构:
\[
V \otimes W \cong W \otimes V~
\]
该同构将\( v \otimes w \)映射到\( w \otimes v \)。

另一方面,即使\( V = W \),向量的张量积也不是交换的;即,一般情况下\( v \otimes w \neq w \otimes v \)。

从\( V \otimes V \)到自身的映射\( x \otimes y \mapsto y \otimes x \)引入了一个线性自同构,称为编织映射(braiding map)。更一般地,照常(见张量代数),设\( V^{\otimes n} \)表示\( V \)的\( n \) 个副本的张量积。对于前\( n \)个正整数的每一个置换\( s \),映射:
\[
x_1 \otimes \cdots \otimes x_n \mapsto x_{s(1)} \otimes \cdots \otimes x_{s(n)}~
\]
引入了一个从\( V^{\otimes n} \)到\( V^{\otimes n} \)的线性自同构,这个映射被称为编织映射。