% 氢原子薛定谔方程数值解

\pentry{薛定谔方程\upref{TDSE}, 原子单位\upref{AU}, 类氢原子的定态波函数\upref{HWF}}

虽然最直观的方法是使用直角坐标, 但计算效率太低. 实际中一般使用球坐标系, 用球谐函数展开波函数. 如果 Hamiltonian 是轴对称的, 那么只需要 $m = 0$ 的球谐函数, 即勒让德多项式.

\begin{equation}
\Psi(\bvec r, t) = \frac{1}{r}\sum_{l',m'} \psi_{l',m'}(r) Y_{l',m'}(\bvec r)
\end{equation}
其中 $\psi_{l,m}(r)$ 是 (scaled)径向波函数.

在薛定谔方程中, 我们往往使用经典的电场, 即令
\begin{equation}
V(\bvec r) = -\frac{Z}{r} - q\bvec E(t) \bvec r
\end{equation}
这叫做长度规范.% 引用未完成
此外我们还可以使用速度规范, 也是等效的.% 引用未完成

\subsection{线性极化场}
若我们取电场极化方向为 $\uvec z$, 则角动量 $L_z$ 是一个守恒量. 假设初始波函数关于 $\uvec z$ 轴对称, 那么在波函数的整个演化过程中, 我们只需要 $m=0$ 的球谐函数展开波函数, 即
\begin{equation}
\Psi(\bvec r, t) = \frac{1}{r}\sum_{l'} \psi_{l'}(r) Y_{l', 0}(\uvec r)
\end{equation}
带入薛定谔方程, 并左乘 $\bra{Y_{l,0}}$ 得
\begin{equation}
\qty[ -\frac{1}{2m} \pdv[2]{r} -\frac{Z}{r} + \frac{l(l+1)}{2mr^2}]\psi_{l,m}(r) + \sum_{l'} E(t)rF(l, l')\psi_{l'}(r) = \I \pdv{\psi_{l,m}}{t}
\end{equation}
