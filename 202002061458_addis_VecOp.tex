% 矢量算符

\pentry{矢量内积\upref{Dot}, 叉乘\upref{Cross}, 偏微分算符\upref{ParOp}}

我们先区分两种函数, 第一种是普通的多元函数, $f(x, y, z)$ 即 $x, y, z$ 是实数, 因变量也是实数(一些情况下也可以是复数). 另一种是所谓的矢量函数, 以下用粗体加以区分, 如 $\bvec f(x, y, z)$, 即因变量是一个 $N$ 维列矢量.

定义三维的\textbf{矢量算符}为 (也叫 nabla 或 del 算符)
\begin{equation}
\Nabla = \uvec x \pdv{x} + \uvec y \pdv{y} + \uvec z \pdv{z}
\end{equation}
它作用在标量函数(因变量是实数或复数) $f(x, y, z)$ 上的结果称为该函数的\textbf{梯度}, 结果是一个矢量函数
\begin{equation}
\Nabla f(x, y, z) = \uvec x \pdv{f}{x} + \uvec y \pdv{f}{y} + \uvec z \pdv{f}{z}
\end{equation}


它与矢量函数(因变量是矢量) $\bvec f(x, y, z)$ 的作用有两种, 第一是 “点乘”, 得到该函数的\textbf{散度}
\begin{equation}
\Nabla \vdot \bvec f(x, y, z) = \pdv{f}{x} + \pdv{f}{y} + \pdv{f}{z}
\end{equation}


(未完成: 例如梯度散度旋度算符, 例如量子力学角动量算符, 要区分对矢量函数作用还是对标量函数作用)
