% 定态薛定谔方程
% 量子力学|波函数|薛定谔方程|哈密顿|束缚态

\begin{issues}
\issueTODO
\issueAbstract
\end{issues}

\pentry{量子力学基本假设\upref{QMPos}, 矢量算符\upref{VecOp}}

\subsection{定态薛定谔方程}
定态薛定谔方程就是哈密顿算符的本征方程, 本征值就是能量 $E$.
\begin{equation}
H \psi(\bvec r) = E \psi(\bvec r)
\end{equation}

单个粒子问题中, 哈密顿算符对应粒子的总能量, 总能量算符可以表示为动能算符和势能算符之和
\begin{equation}
H = T + V
\end{equation}

\subsection{一维定态薛定谔方程}
一维运动的单个质点, 波函数是坐标 $x$ 的函数 $\Psi(x)$
\begin{equation}
T = -\frac{\hbar^2}{2m} \pdv[2]{x} \qquad V = V(x)
\end{equation}
所以定态薛定谔方程为
\begin{equation}\label{SchEq_eq1}
-\frac{\hbar^2}{2m} \pdv[2]{\Psi}{x} + V(x)\Psi = E \Psi
\end{equation}

\subsection{束缚态和散射态}
当 $E > V(\pm \infty)$ 时, \autoref{SchEq_eq1} 必定存在一个 2 维的解空间(二阶常微分方程必有两个线性无关的解).% 链接未完成
当 $E < V(\pm \infty)$

例子: 无限深势阱\upref{ISW}, 有限深势阱\upref{FSW}, 简谐振子(升降算符)\upref{QSHOop}.

\subsection{多维}
二维或三维的情况下, 波函数是位置矢量\upref{Disp}的函数 $\Psi(\bvec r)$
\begin{equation}
T = -\frac{\hbar^2}{2m} \laplacian \qquad V = V(\bvec r)
\end{equation}
定态薛定谔方程为
\begin{equation}
-\frac{\hbar^2}{2m} \laplacian {\Psi} + V(\bvec r)\Psi = E \Psi
\end{equation}
例子: 三维简谐振子(球坐标)\upref{SHOSph}.

\addTODO{为什么束缚态必须是实函数(或者乘以一个相位因子)? 为什么实函数的平均动量必须等于 0(实函数的傅里叶变换对对称函数)? 物理上意味着什么? 如果势能是偶函数,那么波函数为什么一定是基函数或者偶函数?这可以通过哈密顿算符和宇称算符的对易性来解释.}.

\subsection{束缚态}

\subsection{波函数的对称性}

