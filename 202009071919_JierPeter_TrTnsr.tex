% 张量的坐标变换
\begin{issues}
\issueTODO
\end{issues}

\pentry{张量\upref{Tensor}, 过渡矩阵\upref{TransM}}

%高阶张量的坐标变换应该使用什么语言描述?


在\textbf{张量}\upref{Tensor}词条中我们看到,张量表示为矩阵依赖于相关的各线性空间中基的选择.本节将讨论基的变换是如何影响张量的矩阵表示的.

\subsection{一阶张量的坐标变换}
一阶张量是将一个向量映射为一个数,因此只涉及一个线性空间,最为简单.

给定$k$维线性空间$V$,及其上一个张量$f:V\rightarrow\mathbb{R}$.如果$V$的基是$\{\bvec{e}_1, \cdots\bvec{e}_k\}$,那么坐标为$\bvec{c}=(x_1\cdots x_k)\Tr$的向量$\bvec{v}$被映射为:
\begin{equation}
f(\bvec{v})=\sum\limits_{i=1}^k x_if(\bvec{e}_i)
\end{equation}

因此,$f$可以表示为$V$中的一个行向量$\bvec{F}$,坐标为$(f(\bvec{e}_1), \cdots, f(\bvec{e}_k))$.对于$V$中任何向量$\bvec{v}$,都有$f(\bvec{v})=\bvec{F}\bvec{c}$(按矩阵乘法).

若取另一个基$\{\bvec{e}_1', \cdots, \\bvec{e}_k'\}$,其中过渡矩阵为$\bvec{Q}$.如果在新的基下$\bvec{v}$的坐标变为$\bvec{c}'$,那么$\bvec{Q}\bvec{c}'=\bvec{c}$\footnote{见过渡矩阵\upref{TransM}.}.

设在新的基下,$f$表示为行向量$\bvec{F}'$,那么应有$f(\bvec{v})=\bvec{F}\bvec{c}=\bvec{F}'\bvec{c}'$.考虑到$\bvec{Q}\bvec{c}'=\bvec{c}$,我们可知对于任何坐标$\bvec{c}, \bvec{c}'$都有$\bvec{F}\bvec{Q}\bvec{c}'=\bvec{F}'\bvec{c}'$,因此

\begin{equation}
\bvec{F}\bvec{Q}=\bvec{F}'
\end{equation}

这就是一阶张量的坐标变换.
\subsection{二阶张量的坐标变换}

二阶张量涉及两个同构的线性空间$V$,而且是在物理学中最为常见的张量形式,因此我们将详细讨论该情况.在这里,我们把二阶张量$f$理解为从$V_1$和$V_2$到标量域$\mathbb{K}$的一个线性映射,其中$V_1$和$V_2$同构,$\opn{dim}V_1=\opn{dim}V_2=n$.

给$V_1$指定一组基$\{\bvec{a}_1, \bvec{a}_2, \cdots, \bvec{a}_n\}$, 给$V_2$指定一组基$\{\bvec{b}_1, \bvec{b}_2, \cdots, \bvec{b}_n\}$,在这两组基下,张量$f$被表示为一个矩阵$\bvec{F}$,而向量$\bvec{v}_1\in V_1$和$\bvec{v}_2\in V_2$在这两组基下的坐标列矩阵分别为$\bvec{c}(\bvec{v}_1)$和$\bvec{c}(\bvec{v}_2)$.此时,$f(\bvec{v_1},\bvec{v_2})=\bvec{c}(\bvec{v_2})\Tr\bvec{F}\bvec{c}(\bvec{v_1})$.

如果给$V_1$和$V_2$进行基变换,过渡矩阵分别为$\bvec{P}$和$\bvec{Q}$,则在新的基下,两向量的坐标分别为$\bvec{c}'(\bvec{v}_1)=\bvec{P}^{-1}\bvec{c}(\bvec{v}_1)$和$\bvec{c}'(\bvec{v}_2)=\bvec{Q}^{-1}\bvec{c}(\bvec{v}_2)$,那么$f(\bvec{v}_1, \bvec{v}_2)=\bvec{F}\bvec{v}_1\bvec{v}_2$.此时有$f(\bvec{v_1},\bvec{v_2})=\bvec{c}(\bvec{v_2})\Tr\bvec{F}\bvec{c}(\bvec{v_1})=\bvec{c}'(\bvec{v_2})\Tr\bvec{Q}\Tr\bvec{F}\bvec{P}\bvec{c}'(\bvec{v_1})$.因此在新基下,$f$的矩阵为$\bvec{Q}\Tr\bvec{F}\bvec{P}$.

为了方便推广,将矩阵$\bvec{F}$记为$\Bmat {f_{ij}}$,其中$f_{ij}$为$\bvec{F}$第$i$行第$j$列的元素;类似地,将$\bvec{c}(\bvec{v_1})$记为$\Bmat {a_i}$,$\bvec{c}(\bvec{v_2})$记为$\Bmat {b_i}$,那么$f(\bvec{v_1},\bvec{v_2})=\sum\limits_{i, j=1}^{n} b_ia_jf_{ij}$.

将矩阵$\bvec{P}$记为$\Bmat {p_{ij}}$,$\bvec{Q}$记为$\Bmat {q_{ij}}$,则$\bvec{Q}\Tr$被记为$\Bmat{q_{ji}}$.
\subsection{高阶张量的坐标变换}








