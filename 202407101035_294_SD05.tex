% 苏州大学 2005 年硕士考试试题
% keys 苏州大学|考研|物理|2005年
% license Copy
% type Tutor

\begin{enumerate}
\item 质量为 $M$,半径为$r$的均匀圆柱体放在粗糙的水平面上,柱的外围绕有轻绳,绳子绕过一个很轻的滑轮,并悬挂一质量为$m $的物体。设圆柱只滚不滑,并且圆柱体与滑轮间的绳子是水平的,求圆柱体质心加速度$ a_1$、物体的加速度$ a_2$、及绳中张力T。
\begin{figure}[ht]
\centering
\includegraphics[width=6cm]{./figures/821de792fe1174ee.png}
\caption{} \label{fig_SD05_1}
\end{figure}
\item 如图所示,长为$l$的质细杆,一端悬于O点,自由下垂。在O点同时悬一单摆,摆长也是$l$,摆的质量为 $m$,单摆从水平位置由静止开始自由下摆,与自由下垂的细杆作完全弹性碰撞,碰撞后单摆恰好静止。求:\\
(1)细棒的质量 $M;$\\
(2)细棒摆动的最大角度$\theta$。
\item 一弹簧振子作简偕振动,振幅 $A=0.20m$,如果弹簧的劲度系数 $k=2.0N/m$所系物体的质量 $m=0.50kg$,试求:\\
(1)当动能和势能相等时,物体的位移是多少?\\
(2)设$t=0$时,物体在正最大位移处,达到动能和势能相等处所需的时间是多少(在一个周期内)?
\item 已知一沿$x$轴正方向传播的平面余弦波在$t=\frac{1}{3}$秒时的波形如图所示,且周期$T=2S$。求\\
(1)写出O点和P点的振动表达式。\\
(2)写出该波的波动表达式。\\
(3)求P 点离 Os点的距离。
\item 图中球形区域a<r<b,设其体密度为p=A/r,在封闭空腔的中心(1=0)有一个电量为Q的点电荷。证明:当A=Q/2 πa 时,球形区域(a<r<b)中的电场具有恒定值。
\end{enumerate}