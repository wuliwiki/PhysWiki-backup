% 氢原子的精细能级结构

\begin{issues}
\issueDraft
\end{issues}

\footnote{参考 \cite{GriffQ} 相关章节.}讨论氢原子时,我们将哈密顿量取为:\begin{equation}
H=-\frac{h^2}{2m}\laplacian-\frac{e^2}{4\pi\epsilon_0 r}
\end{equation}

但是电子的动能加库仑势能之和并不是完整的内容.我们讨论过了对原子核运动的修正,也就是把$m$替换成约化质量.在我们研究氢原子时,还有个极为重要的现象,那就是由相对论效应修正和自旋-轨道耦合所带来的精细结构.比起数量级为$\alpha^2 mc^2$的波尔能量,精细结构是一个非常微小的扰动,其数量级为$\alpha^4 mc^2$,其中$\alpha$就是精细结构常数.
\begin{equation}
\alpha = \frac{e^2}{4\pi\epsilon_0\hbar c} \approx 0.0072973525693(11) \approx \frac{1}{137.036}
\end{equation}

\subsection{总结}
以下会看到, 哈密顿量的相对论修正为
\begin{equation}
H'_r = -\frac{p^4}{8m^3 c^2}
\end{equation}
引起的能级偏移为
\begin{equation}
\Delta E_r = -\frac{E_n^2}{2mc^2} \qty(\frac{4n}{l+1/2}-3)
\end{equation}
轨道—自旋耦合修正为
\begin{equation}
H'_{SO} = \qty(\frac{e^2}{8\pi\epsilon_0}) \frac{1}{m^2c^2 r^3} \bvec S \vdot \bvec L
\end{equation}
可以证明 $H'_{SO}$ 和总角动量 $J^2$ 和 $J_z$ 对易, $\bvec J = \bvec L + \bvec S$. 好量子数为 $L, S, J, J_z$
\begin{equation}
J^2 = L^2 + S^2 + 2\bvec L \vdot \bvec S
\end{equation}
引起的能级偏移为
\begin{equation}
\Delta E_{SO} = \frac{e^2\hbar^2}{16\pi\epsilon_0 m^2 c^2} \frac{j(j+1) - l(l+1) - 3/4}{l(l+1/2)(l+1)n^3 a^3}
\end{equation}

于是, 精细能级结构的总能量修正为
\begin{equation}
\Delta E_{fs} = \frac{E_n^2}{2mc^2}\qty(3 - \frac{4n}{j + 1/2})
\end{equation}
下面我们来详细讨论.

\subsection{相对论修正}
哈密顿量的首项为动能:
\begin{equation}\label{HfineS_eq1}
T=\frac{1}{2}mv^2=\frac{p^2}{2m}
\end{equation}
动量$\mathbf p$的正则替换(canonical substitution)为$-\I\hbar\Nabla$,由此可得动能算符:
\begin{equation}
T=-\frac{\hbar^2}{2m}\laplacian
\end{equation}
不过,注意到\autoref{HfineS_eq1} 为经典动能的表达式;现在我们考虑相对论表达式:
\begin{equation}
T=\frac{mc^2}{\sqrt{1-(v/c)^2}}-mc^2
\end{equation}
其中的第一项为总的相对论能量,第二项为静能.那么两项的差就是动能.这里我们需要用到相对论的动量代替速度来表示动能$\mathbf T$
\begin{equation}
p=\frac{mv}{\sqrt{1-(v/c)^2}}
\end{equation}
由于
\begin{equation}
p^2c^2+m^2c^4=\frac{m^2v^2c^2+m^2c^4[1-(v/c)^2]}{1-(v/c)^2}=\frac{m^2c^4}{1-(v/c)^2}=(T+mc^2)^2
\end{equation}
因此
\begin{equation}
T=\sqrt{p^2c^2+m^2c^4}-mc^2
\end{equation}
我们将其从$p/mc$级数展开得到近似:
\begin{align}
T &= mc^2\left[\sqrt{1+\left(\frac{p}{mc}\right)^2}\right]\\ 
&=mc^2\left[1+\left(\frac{p}{mc}\right)^2-\frac{1}{8}\left(\frac{p}{mc}\right)^4\cdots -1\right]\\
&=\frac{p^2}{2m}-\frac{p^4}{8m^3c^2}+\cdots
\end{align}
因此我们对哈密顿量的最低阶相对论修正为:
\begin{equation}
H'_r=-\frac{p^4}{8m^3c^2}
\end{equation}
在一阶近似微扰理论中对$E_n$的修正是由$H'$在无微扰态\autoref{TIPT_eq6}~\upref{TIPT}中的期待值:
\begin{equation}\label{HfineS_eq16}
E_r^1=\langle H'_r\rangle=-\frac{1}{8m^3c^2}\langle\psi|p^4\psi\rangle=-\frac{1}{8m^3c^2}\langle p^2\psi|p^2\psi\rangle
\end{equation}
由无微扰态的薛定谔方程得出:
\begin{equation}
T=E-V, \ p^2\psi = 2m(E-V)\psi
\end{equation}
因此我们可以得出:
\begin{equation}
E_r^1=-\frac{1}{2mc^2}(E-V)^2=-\frac{1}{2mc^2}[E^2-2E\langle V\rangle+\langle V^2\rangle]
\end{equation}
以上是一般情况下对$E_n$的修正,接下来我们重点考虑氢原子作为一个运用的实例.根据库伦定律我们可以得到氢原子的势能为
\begin{equation}
V(r)=-\frac{e^2}{4\pi\epsilon_0}\frac{1}{r}
\end{equation}
因此:
\begin{equation}
E_r^1=-\frac{1}{2mc^2}\left[E_n^2+2E_n\frac{e^2}{4\pi\epsilon_0}\left\langle \frac{1}{r}\right\rangle+\left(\frac{e^2}{4\pi\epsilon_0}\right)^2\left\langle \frac{1}{r^2}\right\rangle\right]
\end{equation}
其中$E_n$为无微扰的波尔能级.为了得到最终的结果,我们还需要再无微扰态$\psi_{nlm}$下求得$1/r$和$1/r^2$的期待值,首先$1/r$的平均值的计算比较简单(见\autoref{HfineS_exe1} ):
\begin{equation}
\left\langle\frac{1}{r}\right\rangle = \frac{1}{n^2a}
\end{equation}
其中的$a$为玻尔半径:
\begin{equation}
a=\frac{4\pi\epsilon_0\hbar^2}{me^2}=0.529\times 10^{-10}\rm{m}
\end{equation}
下一个$1/r^2$的期待值(见\autoref{HfineS_exe2} ):
\begin{equation}
\left\langle \frac{1}{r^2}\right\rangle = \frac{1}{(l+1/2)n^3a^2}
\end{equation}
这样我们就得到了:
\begin{equation}
E_r^1=-\frac{1}{2mc^2}\left[E_n^2+2E_n\frac{e^2}{4\pi\epsilon_0}\frac{1}{n^2a}+\left(\frac{e^2}{4\pi\epsilon_0}\right)^2\frac{1}{(l+1/2)n^3a^2}\right]
\end{equation}
进而我们还可以用消去$a$得到:
\begin{equation}\label{HfineS_eq21}
E_r^1=-\frac{(E_n)^2}{2mc^2}\left[\frac{4n}{l+1/2}-3\right]
\end{equation}
其中允许能级$E_n$有著名的玻尔公式:
\begin{equation}
E_{n} =-\left[\frac {m_e}{2\hbar^{2}} \left(\frac {e^ {2}}{4\pi \epsilon_0}\right)^ {2}\right]  \frac {1}{n^ {2}}  =  \frac {E_ {1}}{n^ {2}}, \ \  n=1,2,3, \cdots 
\end{equation}
由此,我们可以看出相对论修正的$E^1_n$小于$E_n$,其比例系数约为$E_n/(mc^2)=2\times 10^{-5}$.
你可能已经注意到,尽管氢原子是高度简并的,但是在这个计算中我们仍然使用了非简并微扰理论 \autoref{HfineS_eq16} ,但是微扰是球对称的,所以它与$L^2$和$L_z$是对易的.并且对于能量$E_n$的$n^2$个态,这些算符的本征态都有不同的本征值.好在波函数$\psi_{nlm}$在该问题上都是好的量子态,也就是说$n,l,m$都是好的量子数,因此我们这里对非简并微扰理论的运用是合规的.

由\autoref{HfineS_eq21} 可见,一些在第$n$能级上的简并度被提高了.由于旋转对称在这种扰动下保持不变,因此$m$中的$(2l+1)$重简并保持不变.另一方面,$l$中“偶然”发生的简并现象消失了,因为它是源自于另一种对于$1/r$势所特有的对称,因此简并度会被任何微扰所打破.
\begin{example}{}
试着用精细结构常数表达出玻尔能量公式:
\begin{align}
E_n&=-\left[\frac {m_e}{2\hbar^{2}} \left(\frac {e^ {2}}{4\pi \epsilon_0}\right)^ {2}\right]  \frac {1}{n^2}\\
&=-\frac{mc^2}{2n^2}\left(\frac {e^{2}}{4\pi \epsilon_0\hbar c}\right)^2=-\frac{mc^2}{2n^2}\alpha^2
\end{align}
\end{example}

如果你能够从第一性原理,不依赖于任何经验性常数$\epsilon_0,e,c,\hbar$计算出精细结构常数,那么你将会是物理学史上的巨佬,诺贝尔奖不在话下.因为精细结构常数是物理学中最为基本的无量纲常数,它将电磁学的电子电荷量,相对论的光速,和怕量子力学的普朗克常数都联系在了一起.
\subsection{自旋-轨道耦合}
电子围绕原子核做轨道运动,相对的以电子作为参考系,质子围绕着电子做轨道运动.在假设电子为静止的坐标系下,这样一个围绕电子做圆周运动的带正电荷的质子就会产生一个磁场$\bvec B$. 因此就会产生一个作用于有自旋的电子的力矩,使得其自旋方向趋同于磁场的方向.那么就有哈密顿量为:
\begin{equation}\label{HfineS_eq26}
H=-\bvec\mu \cdot \bvec B
\end{equation}
因此我们需要找到质子的磁场和电子自旋的磁矩.

首先对于\textbf{质子的磁场}来说,若是我们从电子静止的角度看质子的运动,并将其看作一个连续的圆环线电流,那么根据电动力学中的比奥萨伐尔定律\upref{BioSav} 可得:
\begin{equation}
B=\frac{\mu_0I}{2r}
\end{equation}
其中的等效电流为
\begin{equation}
I=e/T
\end{equation}
其中$e$为质子的质量,$T$为轨道运动的周期.从另一个原子核为静止的角度看电子的运动,其轨道角动量$L$为
\begin{equation}
L=rmv=\frac{2\pi r^2}{T}
\end{equation}
\begin{figure}[ht]
\centering
\includegraphics[width=5cm]{./figures/HfineS_1.pdf}
\caption{氢原子-从电子静止的角度下看质子的轨道运动} \label{HfineS_fig1}
\end{figure}
如\autoref{HfineS_fig1} 所示,根据右手定则$\bvec B$和$\bvec L$是同一个方向的.因此结合前三个式子,并且用
\begin{equation}
c=\frac{1}{\sqrt{\epsilon_0\mu_0}}
\end{equation}
来消去$\mu_0$并用$\epsilon_0$来替代得出:
\begin{equation}\label{HfineS_eq31}
\bvec B = \frac{1}{4\pi\epsilon_0}\frac{e}{mc^2r^3}\bvec L
\end{equation}
接下来我们看到\textbf{电子的磁偶极矩},自旋的带电荷粒子的磁偶极矩是于其自身的自旋角动量相关的,我们将它们的比例系数$\gamma$称之为\textbf{旋磁比(gyromagnetic
ratio)},也就是有:
\begin{equation}
\bvec\mu=\gamma\bvec S
\end{equation}
现在我们将要利用经典电动力学的方法得进行推到.我们首先考虑带电荷为$q$的一个均匀分布在在半径为$r$的圆环上的带电体,绕着$z$轴以周期$T$旋转,类似\autoref{HfineS_fig1} .

我们知道磁偶极矩为:
\begin{equation}
\mu = IA
\end{equation}
这个例子中的电流为:
\begin{equation}
I=\frac{q}{T}
\end{equation}
面积为:
\begin{equation}
A= \pi r^2
\end{equation}
因此
\begin{equation}
\mu=\frac{q\pi r^2}{T}
\end{equation}
假设圆环的质量为$m$,那么由于角动量为:
\begin{equation}
\bvec S = I\bvec\omega
\end{equation}
这个例子中的转动惯量为:
\begin{equation}
I = mr^2
\end{equation}
角速度为:
\begin{equation}
\bvec \omega = \frac{2\pi}{T}
\end{equation}
因此
\begin{equation}
S = \frac{2\pi m r^2}{T}
\end{equation}
显然对于这样的带电圆环体,磁旋比为:
\begin{equation}
\gamma = \frac{\mu}{S} = \frac{q}{2m}
\end{equation}
注意到它和$r$或者$T$是无关的.此外,如果考虑更加复杂的球体,我们也可以将其分割为小的圆环,在将所有圆环累加在一起来得到$\mu,S$的取值.此外,$\bvec \mu$和$\bvec S$的方向当电荷为正时是同向的,当电荷为负时是相反的,因此:
\begin{equation}
\bvec \mu = \frac{q}{2m}\bvec S
\end{equation}
不过,上述推论都是完全建立在经典的理论上的.实际的实验表明,电子的磁矩是经典结论的两倍,也就是:
\begin{equation}
\bvec \mu_e = -\frac{e}{m}\bvec S
\end{equation}
狄拉克的相对论电子理论对于这一现象给出了解释.
%例题未完成 
在之前的例题中,我们就已经认识到直接的将电子看作为一个自旋的球体是危险的,这也就引发了经典推论的错误结论.实际结果于经典结果偏差的被称之为\textbf{$g$因子}:
\begin{equation}\label{HfineS_eq43}
\bvec \mu = g\frac{q}{2m}\bvec S
\end{equation}
在非相对论量子力学理论下考虑自旋-轨道作用时,当然有$g=1$.不过在狄拉克所提出来的相对论量子力学理论中,电子的$g$因此恰好是$2$.在量子电动力学中,因为电子与真空能量的电磁涨落相互作用,朱利安·施温格(Julian Schwinger)对其有微小的修正:
\begin{equation}
g_e = 2+\frac{a}{\pi}+\cdots = 2.002\cdots
\end{equation}
这些对于电子反常磁矩的实验测量和理论推导是近代物理最为辉煌的成就之一.

综上所述结合 \autoref{HfineS_eq26} \autoref{HfineS_eq31} \autoref{HfineS_eq43} 可得:
\begin{equation}
H=\left(\frac{e^2}{4\pi\epsilon_0}\right)\frac{1}{m^2c^2r^3}\bvec S \cdot \bvec L
\end{equation}
不过,这样的计算还是有一个严重的漏洞.于之前我们引入$g$因子类似的.我们所讨论静止的电子系并非一个惯性参考系.因为它在围绕原子核旋转时存在一定的加速度,因此我们还要引入一个适当的动力学修正——\textbf{托马斯(Thomas)旋进}.也就是所我们再引入一个$\frac{1}{2}$的因子就能够解决这一问题,最终得到微扰项为:
\begin{equation}
H'_{so} = \frac{e^2}{8\pi \epsilon_0 m^2 c^2} \frac{\bvec S \vdot \bvec L}{r^3}
\end{equation}

\begin{exercise}{}\label{HfineS_exe1}
利用\textbf{维里定理(virial theorem)}证明计算并得出等式:
\begin{equation}
\left\langle\frac{1}{r}\right\rangle = \frac{1}{n^2a}
\end{equation}
在习题中我们用维里定理论证了其在氢原子运用中可得出以下的结论:
\begin{equation}
\langle V\rangle =2E_n
\end{equation}
接下来我们将氢原子的库伦势能和波尔能量公式带入可得:
\begin{align}
-\frac{e^2}{4\pi\epsilon_0}\left\langle \frac{1}{r}\right\rangle&=-\left[\frac {m_e}{\hbar^{2}} \left(\frac {e^ {2}}{4\pi \epsilon_0}\right)^ {2}\right]  \frac {1}{n^ {2}}\\
\left\langle \frac{1}{r}\right\rangle&=\frac{me^2}{4\pi\epsilon_0\hbar^2n^2}=\frac{1}{n^2a}
\end{align}


\end{exercise}
\begin{exercise}{}\label{HfineS_exe2}
计算并得出等式:
\begin{equation}
\left\langle \frac{1}{r^2}\right\rangle = \frac{1}{(l+1/2)n^3a^2}
\end{equation}
\end{exercise}

