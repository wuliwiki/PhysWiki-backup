% 小时百科 App 设计

\subsection{主页}
上方的搜索栏.

\subsection{百科部分}
\begin{figure}[ht]
\centering
\includegraphics[width=9cm]{./figures/app_1.png}
\caption{百科页面草图, 顶部图标: 退出,信息. 底部图标从左到右: 返回,讨论,收藏,分享,显示, 搜索,目录和收藏夹}\label{app_fig1}
\end{figure}

\subsubsection{按钮说明}
\begin{itemize}
\item 退出:退出百科到 app 首页.
\item 信息:包括作者以及贡献占比(从备份里面通过 git diff 决定). 点击每个作者可进入其主页,选择关注等.
\item 返回:返回上一次链接跳转之前的位置. 链接跳转可能是页面内的跳转也可能是不同页面的.
\item 讨论:类似知乎 app 的评论页面(\autoref{app_fig2}), 可选择按照 “默认” 和 “最新” 排序. 评论顶部设置点赞功能并显示点赞数.
\item 收藏:类似知乎的,可以自己建立不同类别,可以选择公开或者不公开. 若收藏了则图标改变.
\item 分享:转发给微信或 qq 好友, 或者复制链接.
\item 显示:字体大小、 背景色(可以设置护眼的米黄色).
\item 搜索(包括最近浏览): 显示搜索框,可以选择只搜索标题或者搜索全文. 如果不输入内容,下面默认显示最近打开的若干词条,可以拖动调整顺序,每个行右边设置关闭按钮.
\item 目录和收藏夹: 可以查看目录或自己的收藏夹, 打开时保持上次的展开状态和滚动位置.
\end{itemize}

\begin{figure}[ht]
\centering
\includegraphics[width=9cm]{./figures/app_2.png}
\caption{类似知乎 app 的评论页面} \label{app_fig2}
\end{figure}

\subsubsection{操作说明}
\begin{itemize}
\item 向下滚动时,上方标题栏收起, 向上滚动时拉出.
\item 从屏幕左边向右滑动,跳到上一个词条. 从屏幕右边向左滑动,跳到上一个词条.
\item 点击蓝色标题, 可以折叠内容, 直到下一个标题或文末.
\end{itemize}
