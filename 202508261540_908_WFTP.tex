% 微分拓扑(综述)
% license CCBYSA3
% type Wiki

本文根据 CC-BY-SA 协议转载翻译自维基百科\href{https://en.wikipedia.org/wiki/Differential_topology}{相关文章}

在数学中,微分拓扑是研究光滑流形的拓扑性质和光滑性质\(^\text{[a]}\)的领域。从这个意义上说,微分拓扑不同于密切相关的微分几何。微分几何关注的是光滑流形的几何性质,包括大小、距离和刚性形状等概念;而微分拓扑关注的是更为粗略的性质,例如流形中有多少个“洞”、它的同伦型或其微分同胚群的结构。由于这些粗略性质中的许多可以通过代数手段来刻画,微分拓扑与代数拓扑有着紧密的联系\(^\text{[1]}\)。
\begin{figure}[ht]
\centering
\includegraphics[width=6cm]{./figures/6d7860d55520d137.png}
\caption{} \label{fig_WFTP_1}
\end{figure}
微分拓扑领域的核心目标是对所有光滑流形按微分同胚(diffeomorphism)进行分类**。由于维数在微分同胚意义下是光滑流形的不变量,这一分类工作通常通过分别研究各个维度中的(连通)流形来进行:
\begin{itemize}
\item 一维(dimension 1)在一维中,按微分同胚分类后,唯一的光滑流形包括:圆、实数轴,以及带边界的半闭区间 $[0,1)$ 和闭区间 $[0,1]$\(^\text{[2]}\)。
\item 二维(dimension 2)在二维中,每一个闭曲面都可以通过其亏格(genus,洞的数量,或者等价地说其欧拉示性数)以及是否可定向来按微分同胚进行分类。这就是著名的闭曲面分类定理\(^\text{[3][4]}\)[。然而,即便在二维情况下,非紧曲面的分类也变得困难,例如由于存在“Jacob's ladder”这类特殊空间,使问题更加复杂。
\item 三维(dimension 3)在三维中,威廉·瑟斯顿提出的几何化猜想(,由格里高利·佩雷尔曼证明,给出了紧三维流形的部分分类。其中包括著名的庞加莱猜想,它指出任何闭的、单连通的三维流形都与三维球面(3-sphere)同胚(实际上也是微分同胚的)。
\end{itemize}
\begin{figure}[ht]
\centering
\includegraphics[width=6cm]{./figures/fb73b0fbf14774d0.png}
\caption{一个余流形$(W; M, N)$,它推广了微分同胚的概念。} \label{fig_WFTP_2}
\end{figure}
从四维开始,分类问题变得更加复杂,原因有两点[5][6]:

1. 任意有限表示群都可以作为某个四维流形的基本群。由于基本群是微分同胚不变量,这意味着四维流形的分类至少和有限表示群的分类一样困难。而群的分类涉及群的词问题,该问题等价于停机问题,因此无法得到完整的拓扑分类。

2. 在四维及更高维中,存在同胚但非微分同胚的光滑流形。即使是欧几里得空间 $\mathbb{R}^4$,也存在许多异构的 $\mathbb{R}^4$(exotic $\mathbb{R}^4$)结构。这意味着,四维及更高维流形的微分拓扑研究必须使用超出普通连续拓扑范畴的工具。

微分拓扑中的一个核心未解难题是四维光滑庞加莱猜想:是否每个与四维球面同胚的光滑四维流形都与四维球面微分同胚?换句话说,四维球面是否只存在一种光滑结构?在1维、2维和3维中,这一猜想根据前述分类结果是成立的;但在7维中,这一猜想已知是错误的,因为存在米尔诺球这种非标准的光滑结构。
