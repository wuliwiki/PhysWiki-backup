% 南京理工大学 普通物理 B(845)模拟五套卷 第五套
% license Usr
% type Note

\textbf{声明}:“该内容来源于网络公开资料,不保证真实性,如有侵权请联系管理员”

\subsection{一、 填空题 I(24 分,每空 2 分)}
1.一质点沿半径为 $1m$ 的圆周运动,运动方程为$\theta=2+3t^2$ ,式中 以弧度计,$t$以秒计,则 $t=2s$ 时,质点的切向加速度是__________,法向加速度是__________。

2. 一定滑轮质量为$M$,半径为 $R$,对水平轴的转动惯量$J=\frac{1}{2}MR^2$ ,在滑轮的边缘绕一细绳,绳的下端挂一物体,绳的质量可以忽略且不能伸长,滑轮与轴承间无摩擦,物体下落的加速度为 $a$,则绳中的张力 $T=$_________。

3. 如图所示,一静止的均匀细棒,长为 $L$,质量为 $M$,可绕通过棒的端点且垂直于棒长的光滑固定轴 $O$ 在水平面内转动,转动惯量为$\frac{1}{3}ML^2$ ,一质量为 $m$、速率为 $v$ 的子弹在水平面内沿与棒垂直的方向射出并穿出棒的自由端,设穿过棒后 的子弹速率为$\frac{1}{2}v$,则此时棒的角速度应为 ______________ 。
\begin{figure}[ht]
\centering
\includegraphics[width=6cm]{./figures/defd6c917efd6fb6.png}
\caption{} \label{fig_NJUD5_1}
\end{figure}
4. 一弹簧振子沿 $x$ 轴作简谐振动,已知振动物体最大位移为 $x_m=0.4m$,最大恢复
力为 $F_m=0.8N$,最大速度为 $v_m=0.8\pi m/s$,又知 t=0 的初位移为$+0.2m$,且初速度与所选 $x$ 轴方向相反,则振动能量为_____________,此振动的表达式为___________。

5. 若入射波方程为$y_1=A\cos(\omega t+\frac{2\pi x}{\lambda})$ ,在 $x=0$ 处反射,若反射端为固定端,则反射波的方程为 $y_2=$_________(假设振幅不变),合成波方程为__________,波节点的位置为____________。

6. 产生动生电动势的非静电力是__________力,产生感生电动势的非静电力是________力。

7. 以一定量的理想气体作为工作物质,在 $P-T$ 图中经图示的循环过程。图中$a\to b$ 及 $c\to d$ 为两个绝热方程,则循环过程为__________循环,其效率为______
\begin{figure}[ht]
\centering
\includegraphics[width=6cm]{./figures/ec935594987ad66c.png}
\caption{} \label{fig_NJUD5_2}
\end{figure}
\subsection{二、 填空题 II(18 分,每空 2 分)}
1 单色平行光束垂直照射在宽为 $1.0mm$ 的单缝上,在缝后放一焦距为 $20m$ 的会聚透镜,已知位于透镜焦面处的屏幕上的中央明条纹宽度为 $2.5mm$,则入射光波长为_____________。

2.均匀带电的半径为 $R$ 的金属球,带电 $Q$,在距球心为 $a(a<R)$的一点 $P$ 处的电场强度大小 $E=$____________,电势大小 $U=$____________。

3.有一单缝,宽 $a=0.10mm$,在缝后放一焦距为 $50cm$ 的会聚透镜,用平行绿光$(\lambda=546.0nm)$垂直照射单缝,则位于透明焦面处的屏幕上的中央明条纹及第二级明条纹宽_____________。

4. 两根无限长的均匀带电直线相互平行,相距为 $2a$,线电荷密度分别为$+\lambda$ 和$-\lambda$ ,则每单位长度的带电直线所受的作用力为______________。

5.光滑的水平桌面上有一长为 $2L$、质量为 $m$ 的匀质细杆,可绕过其重点且垂直于杆的竖直光滑固定轴 $O$ 自由转动,其转动惯量为$\frac{1}{3}mL^2$ ,起初杆静止,桌面上有两个质量均为 $m$ 的小球,各自在垂直于杆的方向上,正对着杆的一端,以相同速率 $v$ 相向运动,如图所示,当两小球同时与杆的两个端点发生完全非弹性碰撞后,就与杆粘在一起运动,则这一系统碰撞后的转动角速度应为 _____________。
\begin{figure}[ht]
\centering
\includegraphics[width=6cm]{./figures/ad416e811acc8237.png}
\caption{} \label{fig_NJUD5_3}
\end{figure}
6. $S'$ 系相对 $S$ 系以速度 $0.8c$ 沿 $X$ 轴正向运动。两参考系的原点在 $t'=t=0$ 时重合,一事件在 $S'$ 系中发生在 $x'=300 m$. $(y'=z'=0), t'=2\times10^{-7} s$,则该事件在 $S$ 系中发生的空间位置为________ ,时间为 _______。

7. 某一宇宙射线中 $\pi$ 介子的动能 $E_k = 7 m_0 c^2,m_0$ 为 $\pi$ 介子的静止质量,  则实验室中观察到它的寿命是它的固有寿命的 _______ 倍。
\subsection{三、(14 分)}
一链条总长为 l,质量为 m,放在桌面上,并使其下垂,下垂一端的长度为a,设链条与桌面之间的滑动摩擦系数为 ,令链条由静止开始运动,则:\\
(1)链条离开桌面的过程中,摩擦力对链条作了多少功?(2)链条离开桌面时的速率是多少?