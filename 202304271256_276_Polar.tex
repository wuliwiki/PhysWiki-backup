% 极坐标系
% 坐标系|极坐标系|直角坐标系|矢量

\pentry{平面直角坐标系, 位置矢量\upref{Disp}, 四象限 Arctan 函数\upref{Arctan}}
\begin{figure}[ht]
\centering
\includegraphics[width=5.2cm]{./figures/6c9b9449edf1c482.pdf}
\caption{极坐标系和两个单位矢量}\label{fig_Polar_1}
\end{figure}

平面上的极坐标系如\autoref{fig_Polar_1}, 在平面上上取一个点作为原点, 过原点作一条轴称为\textbf{极轴}, 并选定极轴的正方向,规定单位长度。 该平面上某点与原点连成的线段叫做\textbf{极径}, 其长度一般用 $r$ (或 $\rho$ )表示。若 $r$ 为负值, 则表示反方向的长度。 极径与极轴的夹角叫做极角(规定逆时针旋转极角增加,顺时针旋转则减少),用 $\theta $ 表示。 $\theta$ 的值通常表示成弧度, 取值范围一般选 $(-\pi, \pi]$ 或 $[0, 2\pi)$。 于是任何一点都可以用两个有序实数 $(r,\theta)$ 来表示其在该平面上的位置,这就是一个点的\textbf{极坐标}。

为了表示一个坐标对应的单位矢量, 我们一般把坐标变量名记为粗体并在上方加一个标记。 例如直角坐标系中, $\uvec x, \uvec y, \uvec z$ (有时也记为 $\uvec i, \uvec j, \uvec k$ )代表 $x,y,z$ 轴方向的单位矢量。 在极坐标中, 定义 $\uvec r$ 为 $r$ 增加的方向的单位矢量, $\uvec \theta$ 为 $\theta$ 坐标增加方向的单位矢量(即 $\uvec r$ 逆时针旋转 $\pi/2$ 的方向)。 $\uvec r$ 与 $\uvec \theta$ 互相垂直,构成一对\textbf{单位正交基底},平面上的任意矢量都可以正交分解到这两个方向上。 我们通常把 $\uvec r$ 的方向叫做\textbf{径向}(法向),把 $\uvec \theta $ 的方向叫做\textbf{切向}。 要注意极坐标中的两个单位矢量是 $\theta$ 的函数, 对于不同的 $\theta$, 它们的方向也不同。

\begin{exercise}{}
试证明极坐标方程 $r = r_0/\cos(\theta - \theta_0)$ 和 $r = 2R\cos(\theta - \theta_0)$ 分别表示一条直线和一个圆。
\end{exercise}

\subsection{与直角坐标的转换}\label{sub_Polar_1}
要在极坐标系的基础上建立一个直角坐标系, 习惯的做法是取原点相同, 且令 $x$ 轴与极轴重合, $y$ 轴取 $\theta = \pi/2$ 的方向。 这样将 $\bvec r$ 用 $\uvec x$ 和 $\uvec y$ 展开, 就得到
\begin{equation}
\bvec r = r\cos\theta\,\uvec x + r\sin\theta\,\uvec y~,
\end{equation}
即
\begin{equation}\label{eq_Polar_2}
\begin{cases}
x = r\cos\theta\\
y = r\sin\theta
\end{cases}~.
\end{equation}
这样就从极坐标转换成为直角坐标。

要从直角坐标转换为极坐标, 首先由勾股定理有 $r^2 = x^2 + y^2$。 使用反正切函数, 我们可以表示 $x >0$ 或 $\theta\in(-\pi/2,\pi/2)$ 时的 $\theta$, 即 $\theta = \arctan(y/x)$。 为了表示任意情况我们可以使用 $\Arctan$ 函数\autoref{eq_Arctan_1}~\upref{Arctan}。 这样, 从直角坐标转到极坐标的转换就可以表示为
\begin{equation}\label{eq_Polar_4}
\begin{cases}
r = \sqrt{x^2 + y^2}\\
\theta = \Arctan(y, x)~.
\end{cases}
\end{equation}
根据 $\Arctan$ 的定义(\autoref{eq_Arctan_1}~\upref{Arctan}), $\theta$ 的范围是 $(-\pi, \pi]$。
