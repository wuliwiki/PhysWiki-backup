% Mathematica 基础

\begin{issues}
\issueDraft
\end{issues}

\pentry{复数\upref{CplxNo}}

\begin{itemize}
\item 调整字体大小 \verb|Ctrl + 鼠标滚轮|.
\item 快捷键 \verb|Shift + Enter| 执行选中的行.
\item 一行输入多个表达式用分号, 分号也可以用于抑制输出. \verb|a = 3; b = 5; a b|
\item 函数 \verb|N[表达式, 有效位数]| 把表达式的结果变为数值, 四舍五入到指定的有效数字, 第二个参数可省略. 例如 \verb|N[Pi]| 计算圆周率的前 5 位, \verb|N[Pi, 1000]| 计算 1000 位.
\item 也可以用小数点产生数值解, 例如 \verb|Sqrt[2.]|.
\item 用 \verb|(*注释*)| 写注释. 注释不会被执行. 也可以选中后按 \verb|Alt + /|.
\item 空格表示乘号, 相当于 \verb|*|. 数字和字母间相乘可以不需要空格或 \verb|*|.
\item 科学计数法如 \verb|2.3*^70| 表示 $2.3\e{7}$.
\item 快捷键 \verb|Ctrl+L| 可以调用最近的输入或输出.
\item 定义函数 \verb|f[x_, y_] := x^2 + y^2|. 左边的自变量需要用下划线.
\item 中断执行 \verb|Alt + ,| (笔记本界面) 或者 \verb|Ctrl + C| (文本界面).
\end{itemize}

常数都以大写字母开头.
\begin{lstlisting}[language=Mathematica]
Pi (*圆周率*)
E (*自然对数*)
Degree (* 1° 角的弧度*)
I (*虚数单位*)
Infinity (*无穷*)
\end{lstlisting}

\begin{itemize}
\item 输入符号用 \verb|Esc| 键搜索, 例如搜索 \verb|pi| 按回车可以插入 \verb|π|, 相当于 \verb|Pi| 又例如搜索 \verb|inf| 按回车可以插入 \verb|∞|, 相当于 \verb|Infinity|. 如果你知道 TeX, 也可以搜索 tex 命令, 例如 \verb|\varphi| 插入 \verb|φ|.
\item 快捷键 \verb|Ctrl+_| 输入下标. 相当于 \verb|Subscript[x, y]|.
\item 快捷键 \verb|Ctrl+/| 输入分式, 相当于 \verb|x / y|.
\item 快捷键 \verb|@| 输入根式, 相当于 \verb|Sqrt[x]|.
\end{itemize}

\subsection{变量}
\begin{itemize}
\item \verb|?变量| 显示某个变量的定义
\item \verb|Clear[变量名]| 清除某个变量的定义.
\item 数组 \verb|data = {a, 2, c, d}|, \verb|T = {{1, 2}, {3, 4}}|
\item \verb|Tableform[T]| 用于显示矩阵(没有括号).
\item \verb|MatrixForm[T]| 用于显示矩阵(有圆括号).
\end{itemize}

\subsection{算符}
\begin{itemize}
\item \verb|=| 立即赋值(赋值时理科计算右边, 最终结果赋给左边)
\item \verb|:=| 延迟赋值(每次需要左边时都替换为右边)
\item \verb|!=|
\item 连接字符串 \verb|"abc" <> "defg" <> "hij"|.
\item 单变量的函数 \verb|f[...]| 也可以用后缀形式写成 \verb|... // f|.
\item \verb|/.| \verb|ReplaceAll[]| 的简写形式. (什么意思??)
\item \verb|Sin'[x]| 相当于 \verb|D[Sin[x], x]|
\item \verb|x =.| 清除符号 \verb|x| 的定义.
\item \verb|Remove[x]| 完全清除符号.
\end{itemize}

\subsection{常用函数}
\begin{itemize}
\item \verb|Print["a = ", a, " b = ", b]|.
\item 函数参数中, 下标在上标之前给出, 例如勒让德多项式 $P_n^m(x)$ 为 \verb|LegendreP[n, m, x]|.
\item \verb|a + I b| 表示复数.
\item \verb|Re[z], Im[z]| 计算复数的实部和虚部.
\item \verb|Conjugate[z]| 复共轭
\item \verb|Abs[z]| 复数的绝对值
\item \verb|Arg[z]| 复数的幅角
\item \verb|Factor[x^25-1]| 因式分解
\item \verb|Expand[(a+b)^3]| 多项式展开
\item \verb|DSolve[{y''[x] + b y'[x] == 0, y[0] == 0, y'[0] == 1}, y[x], x]| 解常微分方程.
\item \verb|Piecewise[{{x^2, x < 0}, {x, x > 0}}]| 分段函数.
\item \verb|Table[i^2, {i,3}]| 输出 \verb|{1, 4, 9}|.
\item \verb|FullSimplify[表达式]| 简化表达式.
\item \verb|Clear["Global`*"]| 清除当前进程中的所有定义.
\item \verb|Range[5]|, \verb|Range[2, 5]| 输出 \verb|{1,2,3,4,5}|, \verb|{2,3,4,5}|.
\item \verb|Map[f, Range[3]]| 输出 \verb|{f[1], f[2], f[3]}|.
\item \verb|/@| 是 \verb|Map| 的简写形式 \verb|f /@ Range[3]|.
\item 分段函数 \verb|f[x_] := 0 /; x < -1| 以及 \verb|f[x_] := 1 /; x > -1 && x < 1| 等.
\item Listable 的函数可以直接 \verb|Sin[Range[3]]|. \verb|Atributes[Sin]|
\end{itemize}

\subsubsection{微积分}
\begin{itemize}
\item 求和 \verb|Sum[f, {i, imin, imax}]|
\item \verb|Solve[左边==右边, x]| 求解关于 $x$ 的方程.
\item \verb|Limit[f, x->x0]| 极限 $\lim_{x\to x_0} f$.
\item \verb|Minimize[f, x]|, \verb|Maximize[f, x]| 求函数 $f(x)$ 的最小值/最大值.
\item \verb|Series[f, {x, x0, order}]| 把函数展开成泰勒级数.
\item \verb|D[Sin[x]]|, \verb|D[Sin[x+y],x]|, 导数和偏导数.
\item \verb|Integrate[1/(x^3-1),x]| 不定积分, \verb|Integrate[f,{x, xmin, xmax}]| 定积分, \verb|Integrate[f, {x, xmin, xmax},{y, ymin, ymax},...]| 重积分.
\item \verb|Integrate[(f[x])\[Conjugate] f[
   x], {x, -\[Infinity], \[Infinity]}, 
 Assumptions -> {\[Sigma] > 0, Subscript[k, 0] > 0, x0 > 0}]|
\item \verb|FourierSeries[函数, 自变量, 项数]|.
\item \verb|FourierTransform[Exp[-x^2], x, k]| 傅里叶变换.
\item \verb|Plot[y[x], {x, -5, 5}]| 简单的画图.
\item \verb|Plot3D[Sin[y+Sin[3x]], {x,-3,3}, {y,-3,3}]| 简单的画图.
\end{itemize}

\subsubsection{数值运算}
\begin{itemize}
\item \verb|NIntegrate[f, {x, xmin, xmax}]| 数值积分.
\item \verb|NSolve[左边==右边, x]| 求解关于 $x$ 的方程.
\item \verb|NMinimize[f, x]|, \verb|NMaximize[f, x]| 求函数 $f(x)$ 的最小值/最大值.
\end{itemize}
