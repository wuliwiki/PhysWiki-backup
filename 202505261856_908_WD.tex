% 弗朗索瓦·韦达(综述)
% license CCBYSA3
% type Wiki

本文根据 CC-BY-SA 协议转载翻译自维基百科 \href{ https://en.wikipedia.org/wiki/Fran\%C3\%A7ois_Vi\%C3\%A8te}{相关文章}。

\begin{figure}[ht]
\centering
\includegraphics[width=6cm]{./figures/34ef13936a106163.png}
\caption{} \label{fig_WD_1}
\end{figure}
弗朗索瓦·韦达(法语:[fʁɑ̃swa vjɛt];1540年-1603年2月23日),拉丁文名为弗朗西斯库斯·维埃塔(Franciscus Vieta),是一位法国数学家。他在新代数学方面的研究是通向现代代数学的重要一步,原因在于他创新性地使用字母作为方程中的参数。他本职是一名律师,曾担任法国亨利三世和亨利四世的枢密顾问。
\subsection{生平}
\subsubsection{早年生活与教育}
韦达出生于现今法国旺代省的丰特奈勒孔特。他的祖父是来自拉罗谢尔的一位商人,父亲埃蒂安·韦达是丰特奈勒孔特的一名律师,并在勒比索担任公证人。他的母亲是法国天主教同盟势力上升时期,议会首任主席兼法官巴尔纳贝·布里松的姑妈。

韦达曾在方济各会学校就读,并于1558年前往普瓦捷学习法律,次年取得法学学士学位。一年后,他在家乡开始了律师生涯。起初他就被委以重任,办理了一些重要案件,包括替法国国王弗朗索瓦一世的遗孀在普瓦图地区处理租金问题,并代表苏格兰女王玛丽维护其在法国的利益。
\subsubsection{在帕尔特奈任职}
1564年,韦达进入苏比斯夫人安托瓦内特·多贝泰尔的服务。她是帕尔特奈-苏比斯的让五世的妻子,而让五世是胡格诺派(法国新教)主要的军事领袖之一。韦达随他前往里昂,收集有关他在前一年英勇保卫该城抵御萨伏依的雅克(第二代内穆尔公爵)军队的资料。

同年,韦达在现今法国旺代省穆尚普公社的帕尔克-苏比斯担任苏比斯夫人12岁女儿凯瑟琳·德·帕尔特奈的家庭教师。他教授她科学和数学,并为她撰写了多部关于天文学和三角学的论文,其中一些作品至今仍留存。在这些著作中,韦达使用了十进制数(比斯特文发表相关论文早20年),并注意到行星运动的椭圆轨道(这比开普勒早了40年,也早于布鲁诺之死20年)。

帕尔特奈的让五世向法国国王查理九世推荐了韦达。韦达撰写了帕尔特奈家族的谱系,并在让五世于1566年去世后为他写了传记。

1568年,苏比斯夫人安托瓦内特将女儿凯瑟琳嫁给了查尔斯·德·凯勒内克男爵,韦达随她前往拉罗谢尔。在那里,他接触到加尔文主义贵族的最高层,包括科利尼、孔代以及纳瓦拉的女王让娜·达尔布雷及其子纳瓦拉的亨利——即后来的法国国王亨利四世。

1570年,韦达拒绝为苏比斯夫人及其女儿在一桩臭名昭著的诉讼中出庭辩护。在那场官司中,她们指控凯勒内克男爵无法(或不愿)生育继承人。
\subsubsection{初到巴黎}
1571年,韦达在巴黎注册成为律师,并持续探望他的学生凯瑟琳。他经常居住在丰特奈勒孔特,担任一些市政职务。他开始出版《通用观察:论数学正割表的一部专著》,并在夜晚或闲暇时间从事新的数学研究。据他的朋友雅克·德·图记载,他常常连续三天专注于一个问题,手肘支在书桌上,几乎不挪动位置地进食。

1572年,韦达恰好身在巴黎,亲历了“圣巴托洛缪大屠杀”。当晚,凯勒内克男爵在试图营救阿德米拉尔·科利尼未果后被杀。同年,韦达结识了加尔纳什的夫人弗朗索瓦丝·德·罗昂,并成为她的顾问,协助她对抗内穆尔公爵雅克。

1573年,他被任命为雷恩高等法院的顾问。两年后,他说服安托瓦内特·多贝泰尔同意将凯瑟琳·德·帕尔特奈嫁给弗朗索瓦丝的弟弟——雷恩公爵勒内·德·罗昂。

1576年,雷恩公爵亨利·德·罗昂给予韦达特别庇护,并在1580年推荐他担任“申诉事务主管”。1579年,韦达完成了《通用观察》的印刷工作(由梅泰耶出版社出版),该书作为两部三角函数表的附录出版:《数学正割表,或关于三角形》,即《通用观察》标题中提到的“canon”),以及《有理边三角形表》。一年后,他被任命为巴黎高等法院的申诉事务主管,正式为国王服务。同年,他在内穆尔公爵与弗朗索瓦丝·德·罗昂的诉讼中成功为后者辩护,因而引来顽固的天主教同盟的怨恨。
\subsubsection{流亡丰特奈}
1583年至1585年间,在天主教同盟的劝说下,法国国王亨利三世将韦达罢免,因为他被指同情新教事业。在罗昂的推动下,纳瓦拉的亨利(即后来的亨利四世)于1585年3月3日和4月26日分别致信法国国王亨利三世,试图促使韦达重返原职,但未能成功。

随后,韦达与弗朗索瓦·德·罗昂一同退居丰特奈和博瓦尔叙梅尔。他潜心数学研究,度过了四年,并于1591年完成了其名作《新代数学》。
\subsubsection{为两位国王破译密码}
1589年,亨利三世在布卢瓦避难,并命令王室官员于当年4月15日前抵达图尔。韦达是最早返回图尔的人之一。他成功破译了天主教同盟及其他国王敌人的密信。后来,他与古典学者约瑟夫·朱斯特·斯卡利杰发生争论,并在1590年战胜了对方。

亨利三世去世后,韦达成为纳瓦拉的亨利(即法国国王亨利四世)的枢密顾问。\(^\text{[4]: 75–77  }\)国王非常欣赏韦达,钦佩他的数学才华。韦达被任命为图尔高等法院的顾问。1590年,韦达破译了一种由500多个字符组成的西班牙密码系统,这使得法国方面获得的所有西班牙文密电都能被轻易解读。\(^\text{[5]}\)

亨利四世公开了一封由指挥官莫雷奥写给西班牙国王的信件。这封信经韦达解读,揭露了法国天主教同盟首领夏尔·德·梅讷企图篡位,取代亨利四世登上法国王位。信件的公布成为法国宗教战争和解的重要转折点。西班牙国王甚至指控韦达使用了“魔法”来破译密码。

1593年,韦达发表了自己反驳斯卡利杰的论证。从1594年起,他被专门任命负责破译敌方的秘密密码。
\subsubsection{格里高利历改革}
1582年,教皇格里高利十三世发布通谕《Inter gravissimas》,命令天主教国家的国王执行从儒略历向格里高利历的变更。该历法改革基于卡拉布里亚医生阿洛伊修斯·利利乌斯(Aloysius Lilius,又名路易吉·利利奥或路易吉·吉利奥)的计算。在他去世后,其工作由教皇的科学顾问克里斯托弗·克拉维乌斯接手并推进。

1600年,韦达在一系列小册子中指责克拉维乌斯随意引入修正和中间日,并误解了其前任(利利乌斯)在月相周期计算中的原意。韦达提出了一个新的历法时间表,然而在他去世后,克拉维乌斯在其著作《解释》(Explicatio,1603年)中巧妙地予以反驳。\(^\text{[6]}\)

据说韦达是错的。毫无疑问,他自视甚高,正如数学史学家东布尔所说,他相信自己是某种“时间之王”。\(^\text{[7]}\)韦达确实对克拉维乌斯评价不高,正如雅克·德·图所记:

他曾说克拉维乌斯确实非常擅长解释数学原理,也能清晰地理解前人所发明的内容,并撰写了多种论文,将前人著作汇编整理而成,但未加引用来源。因此,他的作品虽然更有条理,但原始著作中的内容却被他重新组织、易于理解,却也掩盖了最初的出处。
\subsubsection{阿德里安·范·罗门的问题}
1596年,斯卡利杰在莱顿大学重新发起对韦达的攻击。次年,韦达作出了最终的回应。同年3月,阿德里安·范·罗门(Adriaan van Roomen)向欧洲顶尖数学家提出一个挑战:求解一个45次多项式方程。法国国王亨利四世从荷兰大使那里听到冷嘲热讽:大使声称法国根本没有数学家,并说之所以没有法国人被邀请解决这个问题,仅仅是因为荷兰数学家范·罗门从未请法国人出手。

韦达出场,看了一眼这个问题,然后在窗前站了几分钟,便解出了该题。这个方程实际上是 $\sin(x)$ 与 $\sin(x/45)$ 之间的关系。他立刻解决了它,并表示他可以(事实上第二天就)将另外22道类似难题一并解出并交给大使。他后来回忆道:“一看便解”。此外,韦达还给范·罗门出了一个新的问题,要求他用欧几里得工具(即直尺和圆规)来解出亚历山大时代数学家佩尔加的阿波罗尼乌斯所提出的一个早已失传的几何问题。范·罗门面对这个问题无力应对,不得不借助一个“巧计”才能解出(见下文详细说明)。
\subsubsection{晚年岁月}
1598年,韦达获准特别休假。然而,亨利四世仍委托他平息一场公证人叛乱——国王命令这些公证人退还所收费用,引发了他们的不满。由于长期操劳,身体欠佳,韦达于1602年12月辞去了王室职务。他获得了2万埃居的酬劳,这笔钱后来在他去世时被发现放在他的床边。

在去世前几周,他还撰写了一篇关于密码学问题的最后论文,该论文使当时所有现行的加密方法瞬间过时。他于1603年2月23日去世,如德·图所记。\(^\text{[8]}\)他留下了两个女儿:珍妮,其母为芭芭拉·科特罗;苏珊娜,其母为朱莉安娜·勒克莱尔。长女珍妮于1628年去世,曾嫁给布列塔尼议会顾问让·加布里奥;苏珊娜于1618年1月在巴黎去世。

韦达的死因不明。他的学生、著作出版人亚历山大·安德森称其为“praeceps et immaturum autoris fatum”(即“作者英年早逝,命途多舛”)。\(^\text{[5][9]}\)
\subsection{著作与思想}
\subsubsection{新代数学}
\textbf{背景}
\begin{figure}[ht]
\centering
\includegraphics[width=6cm]{./figures/ac6194bfc397d35e.png}
\caption{《著作集》,1646年} \label{fig_WD_2}
\end{figure}
16世纪末,数学仍处于希腊几何与阿拉伯方程求解法则的双重影响之下。在韦达所处的时代,代数学因此在两极之间摇摆:一方面是看似由规则列表组成的算术,另一方面是看上去更为严谨的几何学。与此同时,意大利数学家卢卡·帕乔利、斯奇皮奥内·德尔·费罗、尼科洛·丰塔纳·塔塔利亚、杰罗拉莫·卡尔达诺、洛多维科·费拉里,尤其是拉斐尔·邦贝利(1560年)等人,都发展了三次方程的解法,预示着一个新时代的到来。

另一方面,来自德国 Coss 学派的传统,以及威尔士数学家罗伯特·里科德(1550年)和荷兰人西蒙·斯特文(1581年)的努力,引入了早期的代数符号体系,如小数和指数的使用。然而,复数在当时最多只是哲学上的思考工具。几乎一个世纪后,笛卡尔才将它们视为“虚数”来使用。当时仍只考虑正实数解,几何证明仍是常规方法。

当时数学家的任务实际上是双重的:一方面,需要以更几何化的方式构建代数学,为其提供严谨的基础;另一方面,也要使几何更具代数特性,使平面上的解析计算成为可能。韦达与笛卡尔正是完成了这场“双重革命”的关键人物。

\textbf{韦达的符号代数学}

首先,韦达为代数学奠定了与几何学同等坚实的基础。他结束了以操作步骤为主的代数学传统(即“代数与相等法”——al-Jabr 与 al-Muqabala),创立了最早的符号代数学,并宣称凭借这种方法,“没有任何问题不能解决”。\(^\text{[10][11]}\)

在他为《导论》所写的献词中,韦达致凯瑟琳·德·帕尔特奈写道:

“凡是新的事物,在起初通常都显得粗糙而无形,必须由后世加以打磨与完善。看啊,我所呈现的这门艺术虽然是新的,但实际上却是古老的,只不过被野蛮人破坏与玷污得面目全非。因此,为了在其中引入一种全新的形式,我认为有必要重新构思并出版一套新的词汇体系,将所有伪术语清除掉……”\(^\text{[12]}\)

韦达尚未使用后来的“乘号”符号(由威廉·奥特雷德于1631年引入),也未使用等号“=”。这种缺失尤其引人注目,因为罗伯特·里科德早在1557年就采用了现代等号,而威廉姆斯·克西兰德自1575年起已使用平行竖线来表示等值。\(^\text{[5]}\)此外还值得注意的是,1572年拉斐尔·邦贝利曾用类似“u”的符号并在其上方标数字,来表示未知数的幂次。\(^\text{[13]}\)

韦达既缺乏时间,也没有得力的学生来精彩地展现他的方法。他出版作品耗时多年(因为他非常严谨),更重要的是,他做出了一个非常独特的选择:将未知数与参数明确区分,用元音字母表示未知数,用辅音字母表示参数。在这一记号体系上,他或许是借鉴了一些同时代的学者,比如彼得鲁斯·拉穆斯,后者在几何图形中以元音标示点位,仅在用尽后才用辅音字母如 R、S、T 等。\(^\text{[5]}\)这种选择在后来的数学家中并不受欢迎,包括笛卡尔在内的许多人更倾向于用字母表的前几个字母表示参数,后几个字母表示未知数。

韦达在某些方面也仍受制于其时代背景。首先,作为拉穆斯思想的继承者,他并未将线段长度视为“数”。他的书写体系严格遵循同类量的一致性,这使得表达更为繁琐难懂。他未能承认邦贝利提出的复数概念,并且常常需要借助几何构造来反复验证其代数解。尽管他完全明白自己的新代数学已经足以提供解法,这种“让步”仍在一定程度上损害了他的声誉。

尽管如此,韦达依然带来了诸多创新:他提出了二项式公式,后来被帕斯卡和牛顿进一步发展;他还首次建立了多项式的系数与其根的和与积之间的关系,即著名的“韦达公式”。

\textbf{几何代数学}

韦达擅长多种当时最新的数学技巧,致力于通过引入与原始未知量相关的新量来简化方程。他的另一部作品《几何构造的规范审查》具有现代特征,后来被称为“代数几何”——它是一系列使用直尺和圆规构造代数表达式的方法规则汇编。这些著作通俗易懂,因此具有极高的教学价值。然而,他最早提出的“同类性原则”(homogeneity principle)则超前于时代,以至于大多数读者未能真正理解这一思想。古典时代的希腊数学家曾经使用过这一原则,但在后来的数学家中,只有希罗(Hero)、丢番图(Diophantus)等少数人敢于将线段和面积视为纯粹的数值,并将其相加得到新的数值(即它们的和)。\(^\text{[5]}\)

这种关于“加和”的研究,在丢番图的作品中已有体现,或许正是促使韦达提出如下原则的动因:一个方程中的所有量必须是同类的——要么都是线段、要么都是面积、体积,甚至更高维的“超体积”;纯粹的数值间方程被视为不可接受。在韦达之后的几个世纪中,数学界对这一问题的看法经历了多次转变。现代数学家则倾向于将非同类的方程化为齐次(同类)形式,以获得更对称的表达。韦达本人尚未走得如此之远,然而,他间接地启发了这一方向。他还提出了解析求解二次、三次和四次方程的一般方法,这些方法不同于西皮奥内·达尔·费罗和洛多维科·费拉里的解法,而他当时并不知道他们的工作。他还设计了二次和三次方程的数值近似解法,虽然在他之前斐波那契可能已有相关尝试,但那种方法后来完全失传。\(^\text{[5]}\)

最重要的是,韦达是第一位不仅为“未知数”,也为“问题本身”引入符号表示的数学家。\(^\text{[10]}\)正因如此,他的代数学不再只是一套规则陈述,而是一种高效的“计算代数学”——在其中,各种运算直接作用于字母符号,最后只需简单地代入具体数值即可得解。这种方法,正是当代代数方法的核心,成为数学发展的关键一步。\(^\text{[14]}\)借此,韦达不仅为中世纪代数学(从花剌子密到斯特文)画上句号,也正式开启了代数学的现代时代。
\subsubsection{物种逻辑}
由于经济宽裕,韦达开始自费出版其系统的数学理论著作,赠予遍布欧洲各国的朋友与学者。他将这一理论称为“物种逻辑”(species logistic,其中 species 指“符号”),即“符号运算艺术”(1591年)。\(^\text{[15]}\)

他将解决数学问题的过程分为三个阶段:

\begin{itemize}
\item 第一步是将问题用方程的形式加以表达,韦达将此阶段称为“探求阶段”。在这一阶段,他用辅音字母(如 B、D 等)表示已知量,用元音字母(如 A、E 等)表示未知量。
\item 第二步是分析问题,他称之为“结论阶段”。在此阶段,数学家需分析方程并求解,得出问题的性质与特征(即“推论”——porisma),从而可以进入下一阶段。
\item 第三步是“解释性分析”,即回到原始问题,通过几何或数值构造来呈现解答,并以之前的 porisma 为基础加以验证。
\end{itemize}

韦达曾用这种方法处理过多类问题,包括完整解出形式为$X^2 + Xb = c$的二次方程,以及形式为$X^3 + aX = b$的三次方程(韦达将其化简为二次方程处理)。他了解方程的正根(在他那个时代,根通常只指正数根)与未知数各次幂的系数之间的联系(即韦达公式,以及其在二次方程中的应用)。他还发现了一个公式,可由已知角的正弦值推导出倍角的正弦值,并考虑到了正弦函数的周期性。据推测,韦达早在1593年便已知晓此公式。\(^\text{[5]}\)
\subsubsection{韦达公式}
1593年,韦达在几何推理的基础上,结合他精通的三角计算,首次在数学史上发现了一个无穷乘积,并以此给出了圆周率 π 的表达式,这个表达式如今被称为“韦达公式”:[16]
$$
\pi = 2 \times \frac{2}{\sqrt{2}} \times \frac{2}{\sqrt{2 + \sqrt{2}}} \times \frac{2}{\sqrt{2 + \sqrt{2 + \sqrt{2}}}} \times \frac{2}{\sqrt{2 + \sqrt{2 + \sqrt{2 + \sqrt{2}}}}} \times \cdots~
$$
他还使用阿基米德的方法,将一个正六边形反复倍增至边数为 $6 \times 2^{16} = 393,216$ 的多边形,借此计算出了$\pi$的前10位小数。
\subsubsection{阿德里安·范·罗门的挑战与阿波罗尼乌斯问题}
这段著名的争论由塔勒芒·德·雷奥在其作品《趣闻录:十七世纪史料回忆录》第一卷第46则中如此描述:

“在亨利四世在位时期,有一位荷兰人,名叫阿德里安努斯·罗马努斯,是一位学识渊博的数学家,但并不像他自己所认为的那样高明。他出版了一本论文集,向全欧洲的数学家提出了一道问题,但却未邀请任何法国人参与。不久之后,一位国家大使来到枫丹白露觐见国王。国王带他参观各处名胜,并说他的王国在各行各业都有杰出人才。大使答道:‘可是,陛下,据阿德里安努斯·罗马努斯的说法,您这儿却没有数学家,他的目录里完全没有提及法国人。’国王说:‘不,我们有。我有一位极优秀的人。去找韦达先生来。’于是韦达——当时也在枫丹白露——立刻被召来。大使取来阿德里安努斯的书,向韦达展示了他提出的问题。韦达刚到画廊,还未等国王出来,就已经用铅笔写下了两个解法。到当晚,他又将数个解法送交大使。”

1595年,韦达发表了对阿德里安·范·罗门所设问题的回应。他提出回到古老的阿波罗尼乌斯问题,即:构造一条同时与三条给定圆相切的圆。范·罗门提出了一个使用双曲线的解法,但韦达并不赞同这种方法,因为他期望该问题能以欧几里得工具(即直尺和圆规)来解决。

韦达在1600年发表了他自己的解法,收录于其著作《高卢的阿波罗尼乌斯》中。在这篇论文中,韦达利用了“两个圆的相似中心”这一几何概念。\(^\text{[5]}\)据他的朋友德·图记载,阿德里安·范·罗门得知此事后,立刻离开了维尔茨堡大学,骑马前往韦达居住地丰特奈勒孔特。据说他在那里逗留了一个月,向韦达学习新代数学的方法。两人后来成为好友,韦达还支付了范·罗门返回维尔茨堡前在法期间的所有开销。

这一解法在当时几乎立刻引发了全欧洲的关注,也使韦达赢得了历代数学家的赞誉。韦达并未分类讨论各种情况(如圆相切、相离等),但他认识到解的个数取决于三个圆的相对位置,并大致勾勒出了十种可能的情况。笛卡尔在1643年完成了对阿波罗尼乌斯“三圆定理”的推广,结果得到了一个包含87项的二次方程,其中每一项都是六个因子的乘积(这种方法使得实际的几何作图几乎无法实现)。\(^\text{[17]}\)
\subsubsection{宗教与政治信仰}
韦达曾被天主教同盟指控为新教徒,但他其实并非胡格诺派。据多姆布尔所说,他的父亲才是新教徒。\(^\text{[18]}\)韦达本人在宗教问题上持冷淡态度,既未接受帕尔特奈地区的加尔文教信仰,也未采纳其另一位庇护者罗昂家族的宗教立场。他被任命为雷恩高等法院成员这一事实恰恰证明了他的宗教立场并非新教徒。在1574年4月6日于布列塔尼高等法院宣誓就任时,他还曾公开宣读一份天主教信仰声明。\(^\text{[18]}\)

尽管如此,韦达一生都在为新教徒辩护与提供保护,因此也遭到了天主教同盟的愤怒报复。对他而言,国家的稳定显然比国王的宗教信仰更为重要。在当时,这类人被称为“政治派”。

此外,韦达临终时拒绝忏悔。一位朋友不得不劝说他接受天主教圣礼,否则他的女儿将难以找到合适的婚姻对象。至于韦达是否是无神论者,学界至今仍有争议。\(^\text{[18]}\)
\subsubsection{著作出版}
\textbf{按时间顺序列出:}
\begin{itemize}
\item 1564年至1568年间,韦达为其学生凯瑟琳·德·帕尔特奈编写了一些天文学与三角学教材,并撰写了一部未出版的著作:《天体和谐论》。
\item 1579年,他出版了三角函数表《数学正割表,或论三角形》,以及《有理边三角形表》和三角学著作《通用观察:论数学正割表》。这几部作品均由他自费出版,印刷过程极为困难。这些文本包含大量正弦与余弦的公式,特别之处在于使用了十进制数。在当时,它的三角表精度超过了雷根蒙塔努斯(Triangulate Omnimodis, 1533)和瑞提库斯(1543,附于哥白尼的《天体运行论》中)的成果。(另有1589年再版的扫描版本)
\item 1589年,《解读莫雷奥指挥官致西班牙国王的信件》。
\item 1590年,《解读莫雷奥指挥官致其主西班牙国王的信件》,图尔:梅泰耶出版社。
\item 1591年:
\item In artem analyticem isagoge(《分析艺术导论》),又称《新代数学》,图尔:梅泰耶出版社,九开本初版。
\item Zeteticorum libri quinque(《探求五书》),图尔:梅泰耶出版社,二十四开本;该书收录了丢番图提出的问题,并用分析方法加以求解。
\item 1591年至1593年间:《几何构作的规范审查》,图尔:梅泰耶出版社,七开本。
\item 1593年:
\item 《几何补编》,图尔:弗朗西斯出版社,二十一开本。
\item 《弗朗西斯·韦达论数学事务的多种回应,第八卷》,图尔:梅泰耶出版社,四十九开本;主要回应斯卡利杰提出的挑战。
\item 《论数学事务的多种回应,第八卷》,即著名的“第八本回应集”,其中韦达讨论了诸如三等分角问题(他承认该问题可归结为三次方程)、化圆为方、构造正七边形等问题。
\item 1594年:
  《反对新圆周率法的护辩》,巴黎:梅泰耶出版社,四开本,八开页数;再次针对斯卡利杰的批评作出回应。
\item 1595年:
  《针对阿德里安·罗马努斯向全世界数学家提出的构作问题,弗朗西斯·韦达的回应》,巴黎:梅泰耶出版社,四开本,十六开页数;论述关于阿德里安·范·罗门挑战的问题。
\item 1600年:
\item 《关于幂之数值解析解法》,巴黎:勒克莱尔出版社,三十六开本;该作提供了提取根和求解最多六次方程的方法。
\item 《弗朗西斯·韦达:高卢的阿波罗尼乌斯》,巴黎:勒克莱尔出版社,四开本,十三页;韦达在书中自称为“法国的阿波罗尼乌斯”。
\item 1600年至1602年间:
\item 《丰特奈枢密官关于真正格里高利历呈报给教皇克雷孟八世的报告》,巴黎:梅泰耶出版社,四开本,四十页。
\item 《弗朗西斯·韦达对克里斯托弗·克拉维乌斯的抗议》,巴黎:梅泰耶出版社,四开本,八页;即他反驳克拉维乌斯的论文。
\end{itemize}
\subsubsection{身后出版物}
\begin{itemize}
\item 1612年:
\item 《高卢阿波罗尼乌斯补编》,由马林·盖塔尔迪编辑。
\item 《复兴的阿波罗尼乌斯补编,或至今尚未解决之阿波罗尼乌斯问题的分析研究,依据佩尔加的阿波罗尼乌斯的学说,并由来自拉古萨的贵族马林·盖塔尔迪近来创立》,由亚历山大·安德森编辑。
\item 1615年**:
\item 《关于角的分割的解析定理,首次由弗朗西斯·韦达构想,但未附证明,如今终于附上证明》,由亚历山大·安德森编辑。
\item 《为所编辑之〈复兴的阿波罗尼乌斯补编〉中的阿波罗尼乌斯问题探求部分所作之补充说明,对盖塔尔迪在文中顺带提出的意见予以回应》,由亚历山大·安德森编辑。
\item 《弗朗西斯·韦达·丰特奈人之〈关于方程的识别与校正〉两部论述》,由亚历山大·安德森编辑。
\item 1617年:
\item 《对克莱门特·西里亚科新近发布的〈评韦达〉一文的简短驳辨》,由亚历山大·安德森编辑。
\item 1619年:
\item 《十则数学探究·第一集》,由亚历山大·安德森编辑。
\item 1631年:
\item 《分析艺术导论;及其关于符号运算的早期注解,首次出版》巴黎:博德里出版社,十二开本;为《导论》的第二版,收录了韦达身后出版的《关于符号运算的早期注解》。
\end{itemize}
\subsection{评价与影响}
在天主教同盟势力上升期间,韦达的书记官是纳撒尼尔·塔波利,他可能是16世纪英格兰最有趣且最神秘的数学家之一。塔波利回到伦敦后,成为托马斯·哈里奥特的亲密朋友之一。

除凯瑟琳·德·帕尔特奈之外,韦达的其他著名学生还包括:来自奥尔良的法国数学家雅克·阿莱奥姆、拉古萨的马林·盖塔尔迪、让·德·博格朗,以及苏格兰数学家亚历山大·安德森。他们通过出版韦达的著作并延续其方法,进一步阐发了他的理论。在韦达去世后,他的继承人将其手稿交给了彼得·阿莱奥姆。\(^\text{[19]}\)以下是几部最重要的身后出版版本:
\begin{itemize}
\item 1612年:马林·盖塔尔迪出版《高卢阿波罗尼乌斯补编》。
\item 1615年至1619年:亚历山大·安德森出版《对弗朗西斯·韦达的批评》以及《弗朗西斯·韦达·丰特奈人关于方程识别与修正的两部论述》,巴黎:拉克艾出版社,1615年,四开本,135页。不幸的是,安德森的去世中断了后续出版工作。
\item 1630年,出版了法文版的《解析艺术导论,或现代代数学》,由数学家 J.L. 德·沃勒扎尔翻译并加注解,巴黎:雅坤出版社。
\item 同年,出版了法文版并附评论的《弗朗索瓦·韦达〈探求五书〉》,由沃勒扎尔翻译并增补,巴黎:雅坤出版社,219页。
\end{itemize}
同年,安托万·瓦塞(Antoine Vasset,即克洛德·阿尔迪 Claude Hardy 的笔名)出版了 Isagoge,次年博格朗又出版了其拉丁文译本,据说笛卡尔曾阅读过该版本。

1648年,荷兰莱顿大学教授弗朗斯·范·斯科腾整理出版了韦达的数学著作全集,由埃尔泽维尔出版社印刷。他的助手包括雅克·戈里乌斯和梅森。

英国数学家托马斯·哈里奥特和艾萨克·牛顿、荷兰物理学家威利布罗德·斯涅尔、法国数学家皮埃尔·费马和布莱兹·帕斯卡都使用了韦达的符号体系。

大约在1770年,意大利数学家塔尔焦尼·托泽蒂在佛罗伦萨发现了韦达的手稿《天体和谐论》。韦达在其中写道:“Describat Planeta Ellipsim ad motum anomaliae ad Terram”(行星绕地运行的轨道呈椭圆形)。这表明他不仅采纳了哥白尼的体系,而且在开普勒之前就已理解了行星轨道的椭圆形态。\(^\text{[21]}\)

1841年,法国数学家米歇尔·沙尔成为最早重新评估韦达在现代代数学发展中作用的学者之一。

1847年,法国科学院(巴黎)终身秘书弗朗索瓦·阿拉戈在一封信中宣布他打算撰写弗朗索瓦·韦达的传记。

在1880年至1890年间,居住在丰特奈勒孔特的理工学院出身的弗雷德里克·里特成为韦达著作的第一位译者,并与本杰明·菲永一道,成为韦达的首批当代传记作者。
\subsubsection{笛卡尔对韦达的看法}
在韦达去世34年后,哲学家勒内·笛卡尔发表了他的数学方法,并出版了一本几何学著作,这本书改变了代数学的面貌,并在韦达工作的基础上发展了新的理论。他将代数应用于几何,同时摆脱了“同类性”的约束。对于让-巴蒂斯特·肖沃——一位拉弗莱什学院的旧同学——指责他借用了韦达思想,笛卡尔于1639年2月在写给梅森(Mersenne)的信中回应称,他从未阅读过韦达的著作。[22]

笛卡尔接受了韦达对数学的看法,即数学的研究应强调结果的自明性,并将这一理念贯彻到他用几何推理诠释符号代数的方式中。[23] 他还采用了“普遍数学”这一术语,称其为“已有悠久使用传统的术语”,此词源自范·罗门的著作《普遍数学》。[24]

“我不认识这位测量员,他说我们曾在巴黎一起研究过韦达的著作,这让我感到惊讶。因为我在法国时,连那本书的封面是否见过我都记不清了。”

此外,笛卡尔还曾表示,韦达的记号体系令人困惑,并使用了不必要的几何论证。在一些信件中,他表现出对《分析艺术导论》主张的理解;但在另一些信件中,他又公然嘲讽韦达的观点。他的传记作者查尔斯·亚当[25] 指出这种矛盾:

“这些话令人惊讶,因为他(笛卡尔)在前几行才说,他在自己的几何学中只写下他认为‘既不为韦达,也不为任何人所知’的内容。那么他对韦达知道些什么显然是知情的,他之前肯定读过韦达的著作。”

当前的研究尚未明确指出韦达的著作对笛卡尔的直接影响程度。这种影响可能是通过阿德里安·范·罗门或雅克·阿莱奥姆在海牙的工作形成的,也可能是通过让·德·博格朗的著作所间接传递的。[26]

在写给梅森的信中,笛卡尔刻意淡化了他前辈工作的独创性与深度。他曾说:“我从韦达停止的地方开始。”他的观点在17世纪逐渐成为主流,使数学家获得了一种不再依赖“同类性”约束的清晰代数语言。许多当代研究重新评估了这位帕尔特奈数学家的贡献,认为他有双重功绩:既引入了最早的字母运算元素,也初步建立了代数学的公理体系。[27]

虽然韦达并不是第一个用字母表示未知数的人——约尔达努斯·内莫拉里乌斯在更早之前已有类似尝试——但我们仍可以合理地认为,仅仅将韦达的创新归结为这一发现是过于简单的。他应被视为处于16世纪末至17世纪初代数变革交汇点上的关键人物。
\subsection{参见}
\begin{itemize}
\item 韦达公式
\item 迈克尔·斯蒂费尔
\item 拉斐尔·邦贝利
\end{itemize}
\subsection{注释翻译}

\begin{enumerate}
\item 康托尔,1911年,第57页。
\item Bernard R. Goldstein(1998),《开普勒新天文学中有什么新内容?》,收录于 John Earman 和 John D. Norton 编辑的《科学的宇宙:探索论文集》,匹兹堡–康斯坦茨科学哲学与科学史丛书,匹兹堡大学出版社,第3–23页,ISBN 9780822972013。特别参见第21页:“韦达的一份未出版手稿中包含了关于行星模型中椭圆轨道的数学讨论。”
\item 山姆·金瑟,《雅克-奥古斯特·德·图的作品》,谷歌图书。
\item 巴什马科娃与斯米尔诺娃,《代数学的起源与演进》,华盛顿特区:美国数学协会,2000年,第75–77页。
\item 康托尔,1911年,第58页。
\item 克拉维乌斯,《数学全集第五卷》,包含克里斯托弗·克拉维乌斯的《罗马论述》,由 Anton Hierat 和 Johann Volmar 出版,1612年,出版地点:巴黎皇家广场。
\item 迈克尔·奥特与马尔科·潘扎,《数学中的分析与综合》,谷歌图书。
\item 德·图,圣安德鲁斯大学资料馆,2008年7月8日存档版本。
\item 沃尔特·威廉·罗斯·鲍尔,《数学史简述》,谷歌图书。
\item H. J. M. Bos,《重新定义几何的精确性:笛卡尔的转变》,谷歌图书。
\item 雅各布·克莱因:《希腊数学思想与代数学的起源》,谷歌图书。
\item 理查德·W·哈登(1994):《商人肩上的世界:早期现代欧洲的交换与数学自然观》,纽约:纽约州立大学出版社,ISBN 0-585-04483-X。
\item 杰奎琳·安妮·斯泰德尔(2000):《关于代数学的一篇长篇论述:约翰·沃利斯1685年〈代数学论〉》(学位论文),开放大学出版社。
\item 海伦娜·M·皮肖尔(Helena M. Pycior):《符号、不可能的数与几何纠缠:英国代数学……》,谷歌图书。
\item 彼得·墨菲:《证据、证明与事实:资料选编》,谷歌图书。
\item Variorum de rebus Mathèmaticis Responsorum Liber VIII,第30页。
\item 亨克·J. M. 博斯:《笛卡尔、伊丽莎白与阿波罗尼乌斯问题》,收录于《勒内·笛卡尔通信录 1643》,Quæstiones Infinitæ,第202–212页。乌得勒支:Zeno哲学研究所,由特奥·费尔贝克主编,埃里克-扬·博斯与杰罗恩·范德文整理,2003年。
\item 让·东布尔:《弗朗索瓦·韦达与宗教改革》,可在 cc-parthenay.fr 上查阅,[2007年9月11日存档版本](法语)。
\item 雅克-奥古斯特·德·图:见于《世界通史》(L'histoire universelle,法文版)和《Universal History》(英文版),[2008年7月8日存档版本]。
\item 弗朗索瓦·韦达(1983):《分析艺术》,T·理查德·维特默英译,俄亥俄州肯特:肯特州立大学出版社。
\item 关于《天体和谐论》的文章:Adsabs.harvard.edu,《弗朗索瓦·韦达的行星理论(第一部分)》。
\item 笛卡尔致梅森的信(1639年2月20日),PDF文件,来源:Pagesperso-orange.fr(法语)。
\item 马滕·布林克(2018):《数学中的“日常性”:论数学实践在历史书写中的适用性》(The 'Everyday' in Mathematics: On the usability of mathematical practices for doing history,预印本),第10–11页。
\item 保罗·博克斯塔尔(2009):《介于韦达与笛卡尔之间:阿德里安·范·罗门与普遍数学》,载于《精确科学史档案》,第63卷第4期,第433–470页,doi:10.1007/s00407-009-0043-4,JSTOR编号:41134318。
\item Archive.org:查尔斯·亚当,《笛卡尔生平与著作》,巴黎:L. Cerf出版社,1910年,第215页。
\item 佐佐木干:《笛卡尔的数学思想》,第259页。
\item 例如:E. 海尔(2008):《历史中的分析学》,纽约:施普林格出版社,第6页,ISBN 9780387770314。
\end{enumerate}
\subsection{参考书目}
\begin{itemize}
\item 玛丽莲·贝利·奥格尔维、乔伊·多萝西·哈维著:《女性科学家传记词典:L–Z 卷》,谷歌图书,第985页。
\item 伊莎贝拉·G·巴赫马科娃、E.I. 斯拉武京:〈弗朗索瓦·韦达的《三角学的起源》与他在不定分析中的研究〉,载《精确科学史档案》,第16卷第4期,1977年,第289–306页。
\item 伊莎贝拉·格里戈里耶芙娜·巴什马科娃、加丽娜·S·斯米尔诺娃、阿贝·谢尼策:《代数学的起源与演进》,谷歌图书,第75页起。
\item 乔埃尔·比亚尔、鲁什迪·拉希德:《笛卡尔与中世纪》,巴黎:弗兰出版社,1998年,谷歌图书(法语)。
\item 大卫·M·伯顿(1985):《数学史导论》,马萨诸塞州牛顿:阿林与培根出版社。
\item 弗洛伦斯·卡约里(1919):《数学史》,第152页起。
\item 罗纳德·卡林格(主编)(1995):《数学经典选读》,新泽西恩格尔伍德克利夫斯:普伦蒂斯·霍尔出版社。
\item 约翰·德比郡(2006):《未知量:代数学的实与虚的历史》,Scribd.com,[2009年12月21日存档版本]。
\item 霍华德·伊夫斯(1980):《数学的伟大时刻(1650年以前)》,美国数学协会,谷歌图书。
\item J. 格里萨尔(1968):《弗朗索瓦·韦达,十六世纪末的数学家:一项生平与文献研究》,第三阶段博士论文,巴黎高等研究实践学院,科学与技术史研究中心,巴黎。(法语)
\item 加斯顿·戈达尔:《弗朗索瓦·韦达(1540–1603),现代代数学之父》, Père de l’Algèbre Moderne),巴黎第七大学,法国《旺代研究》,ISSN 1257-7979(法语)
\item 理查德·W·哈登:《商人肩上的世界》,谷歌图书。

\end{itemize}