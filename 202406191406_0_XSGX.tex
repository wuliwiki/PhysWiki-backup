% 电磁波吸收(综述)
% license CCBYSA3
% type Wiki

(本文根据 CC-BY-SA 协议转载自原搜狗科学百科对英文维基百科的翻译)

在物理学中,电磁辐射的吸收是指物质(通常是由原子中的电子吸收)吸收光子能量的方式——从而将电磁能量转化为物质的内部能量(例如热能)。[1] 一个显著的效应(衰减)是,当光波在介质中传播时,其强度会逐渐降低。尽管光波的吸收通常不取决于它们的强度(线性吸收),但在某些条件下(光学),介质的透明度会随着波强度的函数而变化,并且发生饱和吸收(或非线性吸收)。

\begin{figure}[ht]
\centering
\includegraphics[width=14cm]{./figures/c99a808abde08f5c.png}
\caption{请电磁辐射吸收概述。该示例讨论了使用可见光作为具体示例的一般原理。 发射多个波长光的白色光源聚焦在样品上(互补色对由黄色虚线表示)。光子在撞击样品时,与存在的分子的能隙匹配的光子(在该实例中为绿光)被吸收,激发分子。其他光子不受影响地传输,如果辐射在可见光区域(400-700 nm),则透射光显示为互补色(此处为红色)。通过记录各种波长的光衰减,可以获得吸收光谱。} \label{fig_XSGX_1}
\end{figure}

\subsection{量化吸收}

许多方法都有可能量化辐射吸收,主要例子如下。

\begin{itemize}
\item 吸收系数以及一些密切相关的衍生量
\item 衰减系数(NB不常用,与“吸收系数”意思相同)
\item 摩尔衰减系数(也被称为“摩尔吸收率”),即吸收系数除以摩尔浓度(另见比尔-兰伯特定律)
\item 质量衰减系数(也称为“质量消光系数”),即吸收系数除以密度
\item 吸收截面和散射截面分别与吸收和衰减系数密切相关
\item 天文学中的“消光”,相当于衰减系数
\item 其他辐射吸收测量,包括穿透深度和趋肤效应、传播常数、衰减常数、相位常数和复波 数、复折射率和消光系数、复介电常数、电阻率和电导率。
\item 相关测量,包括吸光度(也称为“光学密度”)和光学深度(也称为“光学厚度”)
\end{itemize}

所有这些量至少在某种程度上衡量了介质吸收辐射的程度。其中从业者使用的方法因领域和技术而有差异,通常仅仅是根据惯例。

\subsection{测量吸收}

物体的吸收率可以量化有多少入射光被它吸收(而不是被反射或折射)。根据比尔-兰伯特定律,这可能与物体的其他属性有关。

对多种波长吸光度进行精确测量,可以允许通过吸收光谱来识别物质,在吸收光谱中,从一侧照射样品,并测量从样品向各个方向射出的光的强度。吸收的几个例子是紫外-可见光谱、红外光谱和X-射线吸收光谱。

\subsection{应用}


\begin{figure}[ht]
\centering
\includegraphics[width=14cm]{./figures/3e8a258964651b70.png}
\caption{地球大气对不同波长电磁辐射(包括可见光)的透射率(或不透明度)的粗略图} \label{fig_XSGX_2}
\end{figure}

\begin{itemize}
\item 在无线电传播中,它表现为非视距传播。例如,参见卫星链路设计中使用的大气中无线电波衰减的计算。
\item 在气象学和气候学中,全球和局部温度部分取决于大气气体(如温室效应)以及陆地和海洋表面对辐射的吸收(见反照率)。
\item 在医学上,X-射线被不同的组织(特别是骨骼)进行不同程度地吸收,这是X-射线成像的基础。
\item 在化学和材料科学中,不同的材料和分子以不同的频率对不同程度的辐射进行吸收,这可以用来识别材料。
\item 在光学领域,太阳镜、彩色滤光片、染料和其他此类材料是专门针对它们吸收的可见光波段以及吸收比例而设计的。
\item 在生物学中,光合有机物需要适当波长的光在叶绿体的活性区域内被吸收,这样光能就可以转化为糖和其他分子内的化学能。
\item 在物理学中,地球电离层的D区可以显著吸收高频电磁波谱中的无线电信号是众所周知的。
\item 在核物理中,核辐射的吸收可用于液位、密度或厚度的测量。[2]
\end{itemize}

\subsection{参考文献}

\begin{enumerate}
\item West, William. "Absorption of electromagnetic radiation". AccessScience. McGraw-Hill. doi:10.1036/1097-8542.001600. Retrieved 8 April 2013..
\item M. Falahati; et al. (2018). "Design, modelling and construction of a continuous nuclear gauge for measuring the fluid levels". Journal of Instrumentation. 13 (2): P02028. Bibcode:2018JInst..13P2028F. doi:10.1088/1748-0221/13/02/P02028..
\end{enumerate}
