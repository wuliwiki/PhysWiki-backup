% 郑州大学 2010 年 考研 量子力学
% license Usr
% type Note

\textbf{声明}:“该内容来源于网络公开资料,不保证真实性,如有侵权请联系管理员”

\subsection{(30分,每小题10分)}
\subsubsection{(1)}
证明如果体系在 $t=0$ 时刻的波函数 $\Psi(x, 0)$ 是归一化的,那么在以后任意时刻波函数也是归一化的。

证明:
$$\frac{d}{dt} \int_{-\infty}^{\infty} \left| \Psi(x,t) \right|^2 dx = \int_{-\infty}^{\infty} \left( \frac{\partial \Psi^*}{\partial t} \Psi + \Psi^* \frac{\partial \Psi}{\partial t} \right) dx~$$
代入薛定谔方程:
$$i\hbar \frac{\partial \Psi}{\partial t} = \left[-\frac{\hbar^2}{2\mu} \frac{\partial^2}{\partial x^2} + V(x) \right] \Psi~$$
可得:
$$\frac{d}{dt} \int_{-\infty}^{\infty} \left| \Psi(x,t) \right|^2 dx = 0~$$
从而:
$$\int_{-\infty}^{\infty} \left| \Psi(x,t) \right|^2 dx = \text{常数}~$$
若 $\Psi(x, 0)$ 是归一化的,则 $\Psi(x,t)$ 也是归一化的。
\subsubsection{(2)}
证明对易关系:
$$\left[x, f(p_x)\right] = i\hbar \frac{df}{dp_x}~$$
证明:

在动量表象 $x = i\hbar \frac{\partial}{\partial p_x}$,设 $\Phi(p_x)$ 是任意函数,有:
