% 格奥尔格·欧姆(综述)
% license CCBYSA3
% type Wiki

本文根据 CC-BY-SA 协议转载翻译自维基百科\href{https://en.wikipedia.org/wiki/Georg_Ohm}{相关文章}。

\begin{figure}[ht]
\centering
\includegraphics[width=6cm]{./figures/33a59a3a2faf0fbf.png}
\caption{乔治·西蒙·欧姆  1789年3月16日  埃尔朗根,勃兰登堡-拜罗伊特(现为德国)} \label{fig_GOM_1}
\end{figure}

乔治·西蒙·欧姆(Georg Simon Ohm,/oʊm/;德语:[ˈɡeːɔʁk ˈʔoːm];1789年3月16日 – 1854年7月6日)是德国物理学家和数学家。作为一名学校教师,欧姆开始研究由意大利科学家亚历山德罗·伏打发明的新的电化学电池。通过使用他自己制作的设备,欧姆发现导体两端施加的电位差(电压)与产生的电流之间存在直接的正比关系。这个关系被称为欧姆定律,而电阻的国际单位“欧姆”(Ω)也以他的名字命名。
\subsection{传记}
\subsubsection{早年生活}  
乔治·西蒙·欧姆出生于一个新教家庭,地点是埃尔朗根,勃兰登堡-拜罗伊特(当时属于神圣罗马帝国)。他是锁匠约翰·沃尔夫冈·欧姆和埃尔朗根裁缝的女儿玛丽亚·伊丽莎白·贝克的儿子。虽然他的父母并未接受正式教育,但欧姆的父亲是一位受人尊敬的人,他通过自学达到了较高的学识水平,并能够通过自己的教导为儿子们提供优良的教育。[4] 在家族的七个孩子中,只有三人活到了成年:乔治·西蒙、他的弟弟马丁(后来成为一位著名数学家)和他的妹妹伊丽莎白·巴巴拉。他的母亲在他十岁时去世。

从小,乔治和马丁就由父亲亲自教育,父亲将他们培养到较高的数学、物理、化学和哲学水平。乔治·西蒙从十一岁到十五岁就读于埃尔朗根中学,但在学校的科学训练上并未得到很多指导,这与他和马丁在父亲那里接受的启发性教育形成了鲜明对比。这一特点使欧姆家族与伯努利家族相似,正如埃尔朗根大学的教授卡尔·克里斯蒂安·冯·朗斯多夫所指出的。
\subsubsection{大学生涯}
Georg Ohm的父亲担心儿子浪费了教育机会,于是将Ohm送往瑞士。1806年9月,Ohm接受了在Gottstadt bei Nidau的一所学校担任数学教师的职位。

1809年初,Karl Christian von Langsdorf离开了埃尔兰根大学,去海德堡大学任职。Ohm希望能在Langsdorf的指导下重新开始数学学习。然而,Langsdorf建议Ohm独立进行数学研究,并建议他阅读欧拉、拉普拉斯和拉克鲁瓦的著作。Ohm有些不情愿地接受了这个建议,但他于1809年3月离开了Gottstatt修道院的教职,转而成为Neuchâtel的私人家教。在两年时间里,他一边担任家教,一边按照Langsdorf的建议继续私人学习数学。直到1811年4月,他才返回埃尔兰根大学。
\subsubsection{教学生涯} 
欧姆的学术研究为他攻读博士学位做了准备,他于1811年10月25日从埃尔朗根大学获得博士学位。此后,他立即加入该校数学系,担任讲师,但由于前景不佳,他在三个学期后离开了。作为讲师,他的薪水无法维持生计。巴伐利亚政府向他提供了一个在班贝格一所质量较差的学校教授数学和物理的职位,欧姆于1813年1月接受了这个职位。对工作不满的欧姆开始编写一本基础几何教材,试图证明自己的能力。那所学校在1816年2月关闭。随后,巴伐利亚政府将欧姆派往班贝格的一所过于拥挤的学校,协助数学教学。
\begin{figure}[ht]
\centering
\includegraphics[width=6cm]{./figures/dd69cc86728b5f2d.png}
\caption{慕尼黑工业大学(Theresienstrasse 校区)为欧姆所立的纪念碑(由威廉·冯·吕曼设计)} \label{fig_GOM_2}
\end{figure}
在班贝格的工作结束后,欧姆将他完成的手稿寄给了普鲁士国王威廉三世。国王对欧姆的书籍感到满意,并在1817年9月11日为欧姆提供了科隆耶稣会文理中学的职位。这所学校以良好的科学教育而著称,欧姆不仅需要教授数学,还需要教授物理。该校的物理实验室设备齐全,使得欧姆能够开始进行物理实验。作为一名锁匠的儿子,欧姆在机械装置方面有一定的实践经验。

欧姆于1827年出版了《电池电路的数学研究》(*Die galvanische Kette, mathematisch bearbeitet*)。然而,欧姆所在的学院并未欣赏他的研究成果,于是他辞去了职位。随后,他向纽伦堡的工艺学校提出申请,并成功被聘用。欧姆于1833年到达纽伦堡工艺学校,并于1852年成为慕尼黑大学实验物理学教授。

1849年,欧姆出版了《分子物理学》(*Beiträge zur Molecular-Physik*)。在这部作品的序言中,他表示希望能写出第二卷和第三卷,“如果上帝赐予我长寿的话,第四卷也许能完成”。然而,在发现书中所记载的一项原创发现已被一位瑞典科学家预见之后,他决定不再出版,称:“这一事件使我对‘人算不如天算’这一说法有了全新的、深刻的理解。最初推动我进行此项研究的计划已经消失无踪,而一个新的计划,非我所预见,却在不经意间得以完成。”

欧姆于1854年在慕尼黑去世,并葬于旧南方墓地。他的家族信件被编成一本德文书籍,其中显示他曾在信件的签名处写道“Gott befohlen, G S Ohm”,意思是“托付于上帝”。
\subsection{欧姆定律的发现}  
欧姆定律首次出现在他那本著名的书籍《电池电路的数学研究》(*Die galvanische Kette, mathematisch bearbeitet*,1827年)中,在这本书中他给出了完整的电学理论。在这部作品中,他阐述了他的定律,即电动势在电路任何部分两端的作用是电流强度与该部分电阻的乘积。

这本书首先介绍了理解其余部分所需的数学背景。虽然他的工作对电流电学的理论和应用产生了巨大影响,但在当时却遭遇了冷淡的反响。欧姆将他的理论呈现为一种“相邻作用”的理论,这与远程作用的概念相对立。欧姆认为电的传输发生在“相邻粒子”之间,这是他自己使用的术语。这篇论文探讨了这一思想,特别是阐明了欧姆与约瑟夫·傅里叶(Joseph Fourier)和克劳德-路易·纳维尔(Claude-Louis Navier)科学方法之间的区别。

托马斯·阿奇博尔德(Thomas Archibald)对欧姆在推导欧姆定律时所使用的概念框架进行了研究。欧姆的工作标志着电路理论学科的早期起步,尽管直到19世纪末,这一领域才成为重要的学科。
\subsection{欧姆的声学定律}  
欧姆的声学定律,有时称为声学相位定律或简单地称为欧姆定律,指出音乐声音是由一组纯谐波音调构成,并且人耳将其感知为这些音调的集合。这个定律已被证明并不完全正确。
\subsection{研究与出版}  
欧姆的第一篇论文发表于1825年,探讨了随着电线长度增加,电磁力的衰减情况。1826年,他描述了基于傅里叶热传导研究的电路导电性。这篇论文延续了欧姆从实验证据中推导结果的方法,特别是在第二篇中,他提出了一些定律,这些定律在很大程度上解释了其他研究电流电学的学者的结果。最重要的成果是他于1827年在柏林出版的小册子,标题为《电池电路的数学研究》(*Die galvanische Kette mathematisch bearbeitet*)。这项工作的雏形在前两年已在施魏格尔(Schweigger)和波戈多夫(Poggendorff)的期刊中出现,对电流理论和应用的发展产生了重要影响。欧姆的名字已被纳入电学术语中,如欧姆定律(他首次在《电池电路的数学研究》中发表)、电流与电阻器中电压的比例关系,并作为电阻的国际单位欧姆(符号Ω)。

尽管欧姆的工作对理论产生了深远的影响,但最初并未受到热烈欢迎。然而,他的工作最终获得了皇家学会的认可,并于1841年获得了科普利奖章。1842年,他成为皇家学会的外籍会员,1845年成为巴伐利亚科学院与人文学科学院的正式会员。某种程度上,查尔斯·惠特斯通(Charles Wheatstone)引起了人们对欧姆在物理学领域所引入定义的关注。
\subsubsection{作品}  
\begin{itemize}
\item Grundlinien zu einer zweckmäßigen Behandlung der Geometrie als höheren Bildungsmittels an vorbereitenden Lehranstalten [几何学在\item 预备教育机构中的适当处理指南](德文)。Palm und Enke,1817年 — 通过Google图书。  
\item Die galvanische Kette: mathematisch bearbeitet [电池电路:数学研究](德文)。柏林:T.H. Riemann,1827年 — 通过Google图书。  
英文翻译:The Galvanic Circuit Investigated Mathematically,由William Francis翻译。纽约:D. Van Nostrand Company,1891年 — 通过Google图书。
\item Beiträge zur Molecular-Physik. Erster Band. Elemente der analytischen Geometrie im Raume am schiefwinkligen Coordinatensysteme [分子物理学贡献。第一卷。关于斜坐标系的解析几何元素](德文)。纽伦堡:Schrag,1849年 — 通过Internet Archive。  
\item Erklärung aller in einaxigen Krystallplatten zwischen geradlinig polarisirstem Lichte wahrnehmbaren Interferenz-Erscheinungen in mathematischer Form mitgetheilt* [在单轴晶体板中,直线偏振光下可观察到的干涉现象的数学形式解释](德文)。慕尼黑:巴伐利亚科学院,  
\item 第一部分,1852年 — 通过慕尼黑数字化中心  
\item 第二部分,1853年 — 通过Google图书 
\item Grundzüge der Physik als Compendium zu seinen Vorlesungen [物理学基础:讲座概要]。纽伦堡:Schrag,  
\item 第一部分,一般物理学,1854年 — 通过Google图书  
\item 第二部分,特殊物理学,1854年 — 通过Google图书
\end{itemize}
\subsection{另见}  
欧姆(单位)
\subsection{注释} 
欧姆定律,即电流与电势差成正比,最早由亨利·卡文迪许(Henry Cavendish)发现,但卡文迪许在生前没有发表他的电学发现,这些发现直到1879年才为人所知,远在欧姆独立发现并发表这一规律之后。因此,这一定律最终以欧姆的名字命名。
\subsection{参考文献}  
\begin{enumerate}
\item "Ohm". Random House Webster's Unabridged Dictionary.  
\item Dudenredaktion; Kleiner, Stefan; Knöbl, Ralf (2015) [首次出版1962年]. Das Aussprachewörterbuch [发音词典](德文)(第7版)。柏林:Dudenverlag,第398、645页。ISBN 978-3-411-04067-4。  
\item Krech, Eva-Maria; Stock, Eberhard; Hirschfeld, Ursula; Anders, Lutz Christian (2009). Deutsches Aussprachewörterbuch [德语发音词典](德文)。柏林:Walter de Gruyter,第536、788页。ISBN 978-3-11-018202-6。  
\item Keithley, Joseph F. (1999). The Story of Electrical and Magnetic Measurements: From 500 BC to the 1940s. John Wiley & Sons。  
\item 前述某些句子包含来自现已进入公有领域的出版物的文本:Chisholm, Hugh, ed. (1911). "Ohm, Georg Simon". Encyclopædia Britannica. 第20卷(第11版)。剑桥大学出版社,第34页。  
\item Chisholm 1911,第34页。  
\item Kneller, Karl Alois; Kettle, Thomas Michael (1911).Christianity and the Leaders of Modern Science: A Contribution to the History of Culture in the Nineteenth Century. 弗莱堡,第17–18页。  
\item Georg Simon Ohm (2002), *Georg Simon Ohm: Nachgelassene Schriften und Dokumente aus seinem Leben: mit Schriftstücken seiner Vorfahren und Briefen seines Bruders Martin*. Palm und Enke,第216、219页。
\end{enumerate}