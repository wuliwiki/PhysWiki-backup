% 2010 年计算机学科专业基础综合全国联考卷
% keys 考研 计算机 全国联考

\subsection{一、单项选择题}
第1~40 小题,每小题2 分,共80 分.下列每题给出的四个选项中,只有一个选项最符合试题要求.

1. 若元素a、b、c、d、e、f 依次进栈,允许进栈、退栈操作交替进行,但不允许连续三次进行退栈操作,则不.可能得到的出栈序列是______. \\
A. d c e b f a $\quad$ B. c b d a e f $\quad$ C. b c a e f d $\quad$ D. a f e d c b

2. 某队列允许在其两端进行入队操作,但仅允许在一端进行出队操作.若元素a、b、c、d、e 依次入此队列后再进行出队操作,则不.可能得到的出队序列是______ \\
A. b a c d e $\quad$ B. d b a c e $\quad$ C. d b c a e $\quad$ D. e c b a d

3. 下列线索二叉树中(用虚线表示线索),符合后序线索树定义的是______.\\
\begin{figure}[ht]
\centering
\includegraphics[width=14.25cm]{./figures/CSN10_1.png}
\caption{第3题图} \label{CSN10_fig1}
\end{figure}

4. 在右图所示的平衡二叉树中,插入关键字48 后得到一棵新平衡二叉树.在新平衡二叉树中,关键字37 所在结点的左、右子结点中保存的关键字分别是______. \\
\begin{figure}[ht]
\centering
\includegraphics[width=5cm]{./figures/CSN10_2.png}
\caption{第4题图} \label{CSN10_fig2}
\end{figure}
A.13,48 $\quad$ B.24,48 $\quad$ C.24,53 $\quad$ D、24,90

5. 在一棵度为4的树T 中,若有20个度为4 的结点,10个度为3 的结点,1个度为2 的结点,10个度为1的结点,则树T 的叶结点个数是______. \\
A.41 $\quad$ B.82 $\quad$ C.113 $\quad$ D.122

6. 对n(n≥2)个权值均不相同的字符构造成哈夫曼树.下列关于该哈夫曼树的叙述中,\textbf{错误}的是______ \\
A.该树一定是一棵完全二叉树. \\
B.树中一定没有度为1 的结点. \\
C.树中两个权值最小的结点一定是兄弟结点. \\
D.树中任一非叶结点的权值一定不小于下一层任一结点的权值.

7. 若无向图G=(V, E)中含有7 个顶点,要保证图G 在任何情况下都是连通的,则需要的边数最少是_____. \\
A.6 $\quad$ B.15 $\quad$ C.16 $\quad$ D.21

8. 对下图进行拓扑排序,可以得到不同的拓扑序列的个数是_____. \\
\begin{figure}[ht]
\centering
\includegraphics[width=10cm]{./figures/CSN10_3.png}
\caption{第8题图} \label{CSN10_fig3}
\end{figure}
A.4 $\quad$ B.3 $\quad$ C.2 $\quad$ D.1

9. 已知一个长度为16的顺序表L,其元素按关键字有序排列.若采用折半查找法查找一个L中不存在的元素,则关键字的比较次数最多的是_____. \\
A.4 $\quad$ B.5 $\quad$ C.6 $\quad$ D.7

10. 采用递归方式对顺序表进行快速排序.下列关于递归次数的叙述中,正确的是______. \\
A.递归次数与初始数据的排列次序无关. \\
B.每次划分后,先处理较长的分区可以减少递归次数. \\
C.每次划分后,先处理较短的分区可以减少递归次数. \\
D.递归次数与每次划分后得到的分区的处理顺序无关.

11. 对一组数据(2,12,16,88,5,10)进行排序,若前三趟排序结果如下: \\
第一趟排序结果:2,12,16,5,10,88 \\
第二趟排序结果:2,12,5,10,16,88 \\
第三趟排序结果:2,5,10,12,16,88 \\
则采用的排序方法可能是______. \\
A.起泡排序 $\quad$ B.希尔排序 $\quad$ C.归并排序 $\quad$ D.基数排序

12. 下列选项中,能缩短程序执行时间的措施是. \\
Ⅰ. 提高CPU 时钟频率 \\
Ⅱ. 优化数据通路结构 \\
Ⅲ. 对程序进行编译优化 \\
A.仅Ⅰ 和Ⅱ $\quad$ B.仅Ⅰ 和Ⅲ $\quad$ C.仅Ⅱ 和Ⅲ $\quad$ D.Ⅰ 、Ⅱ 和Ⅲ

13. 假定有4个整数用8位补码分别表示$r1=FEH$,$r2=F2H$,$r3=90H$,$r4=F8H$,若将运算结果存放在一个8位
寄存器中,则下列运算中会发生溢出的是. \\
A.r1 x r2 $\quad$ B.r2 x r3 $\quad$ C.r1 x r4 $\quad$ D.r2 x r4

14. 假定变量$i$、$f$和$d$的数据类型分别为int,float 和double(int 用补码表示,float 和double分别用IEEE754单精度和双精度浮点数格式表示),已知$i=785$,$f=1.5678e3$,$d=1.5e100$.若在$32$位机器中执行下列关系表达式,则结果为“真”的是. \\
(I)i == (int)(float)i $\quad$ (II)f == (float)(int)f \\
(III)f == (float)(double)f $\quad$ (IV)(d+f)-d == f \\
A.仅I 和II $\quad$ B.仅I 和III $\quad$ C.仅II 和III $\quad$ D.仅III 和IV

15.假定用若干个2kx4 位的芯片组成一个8kx8位的存储器,则地址0B1FH所在芯片的最小地址是. \\
A.0000H $\quad$ B.0600H $\quad$ C.0700H $\quad$ D.0800H

16. 下列有关RAM和ROM的叙述中,正确的是. \\
I RAM是易失性存储器,ROM是非易失性存储器 \\
II RAM和ROM都采用随机存取方式进行信息访问 \\
III RAM和ROM都可用作Cache \\
IV RAM和ROM都需要进行刷新 \\
A.仅I 和II $\quad$ B.仅II 和III $\quad$ C.仅I,II 和IV $\quad$ D.仅II,III 和IV

18. 下列寄存器中,汇编语言程序员可见的是. \\
A.存储器地址寄存器(MAR) $\quad$ B.程序计数器(PC) \\
C.存储器数据寄存器(MDR) $\quad$ D.指令寄存器(IR)

19. 下列选项中,\textbf{不}会引起指令流水线阻塞的是. \\
A.数据旁路(转发) $\quad$ B.数据相关 \\
C.条件转移 $\quad$ D.资源冲突

20. 下列选项中的英文缩写均为总线标准的是______. \\
A.PCI、CRT、USB、EISA \\
B.ISA、CPI、VESA、EISA \\
C.ISA、SCSI、RAM、MIPS \\
D.ISA、EISA、PCI、PCI-Express

21. 单级中断系统中,中断服务程序内的执行顺序是______. \\
I 保护现场 $\quad$ II 开中断 $\quad$ III 关中断 $\quad$ IV 保存断点 \\
V 中断事件处理 $\quad$ VI 恢复现场 $\quad$ VII 中断返回 \\
A.I->V->VI->II->VII $\quad$ B.III->I->V->VII \\
C.III->IV->V->VI->VII $\quad$ D.IV->I->V->VI->VII

22. 假定一台计算机的显示存储器用DRAM 芯片实现,若要求显示分辨率为1600*1200,颜色深度为24位,
帧频为85Hz,显存总带宽的50\%用来刷新屏幕,则需要的显存总带宽至少约为______. \\
A.245Mbps $\quad$ B.979Mbps $\quad$ C.1958Mbps $\quad$ D.7834Mbps

23. 下列选项中,操作系统提供给应用程序的接口是_____. \\
A.系统调用 $\quad$ B.中断 $\quad$ C.库函数 $\quad$ D.原语

24. 下列选项中,导致创建新进程的操作是______. \\
Ⅰ 用户登录成功 $\quad$ Ⅱ 设备分配 $\quad$ Ⅲ 启动程序执行 \\
A.仅Ⅰ 和Ⅱ $\quad$ B.仅Ⅱ 和Ⅲ $\quad$ C.仅Ⅰ 和Ⅲ $\quad$ D.Ⅰ 、Ⅱ 和Ⅲ

25. 设与某资源关联的信号量初值为3,当前值为1.若M表示该资源的可用个数,N表示等待该资源的进程
数,则M、N 分别是______. \\
A.0、1 $\quad$ B.1、0 $\quad$ C.1、2 $\quad$ D.2、0

26. 下列选项中,降低进程优先级的合理时机是_____. \\
A. 进程的时间片用完 \\
B. 进程刚完成I/O,进入就绪列队 \\
C. 进程长期处于就绪列队中 \\
D. 进程从就绪态转为运行态 \\

