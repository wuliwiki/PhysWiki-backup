% 旋转参考系的机械能守恒

\pentry{科里奥利力\upref{Corio}, 机械能守恒(单个质点)\upref{ECnst}}

在匀速旋转参考系中, 如果加入惯性力的修正, 则牛顿定律同样适用. 那么我们是否同样有某种修正版的单质点机械能守恒呢? 答案是肯定的, 在旋转参考系中, 我们仍然定义每个质点的动能为 $E_k = mv^2/2$ ($v$ 是相对旋转参考系的速度), 假设质点所受的所有非惯性力都是保守力\upref{V}, 对应的势能函数 $V(\bvec r)$. 那么机械能修正后的守恒量为
\begin{equation}
\frac{1}{2}m v^2 + V(\bvec r) - \frac{1}{2}\omega^2 r^2 = \text{const}
\end{equation}
其中 $\omega$ 是旋转参考系相对于惯性系的角速度.

对于质点系, 如果质点两两之间的力也是保守力, 那么

\subsection{推导}
证明的思路很简单, 在 “机械能守恒(单个质点)\upref{ECnst}” 中证明的基础上, 我们只需要额外考虑惯性力的做功即可.

由于科里奥利力始终与每个质点的运动方向垂直, 所以对系统不做功. 而离心力却有可能会做功. 每个质点所受的离心力之和位置有关, 于是我们可以得到一个离心力场
\begin{equation}
F_c(\bvec r) = m\omega^2 \bvec r
\end{equation}
 这是一个中心力场, 所以必定是一个保守场\upref{CenFrc}, 沿径向积分容易得到对应的势能函数为
\begin{equation}
V_c(\bvec r) = -\frac{1}{2}\omega^2 r^2
\end{equation}
在旋转参考系中, 物体受到的非惯性力也都是保守力, 我们也可以写出

当我们把质点的总机械能加上这个等效势能作为修正, 就能得到修正的机械能守恒
