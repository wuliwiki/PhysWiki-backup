% KL散度(相对熵)
% Kullback–Leibler divergence 熵 统计距离


\textbf{$KL$散度}(Kullback–Leibler divergence, 缩写KLD)是一种统计学度量,表示的是一个概率分布相对于另一个概率分布的差异程度,在信息论中又称为\textbf{相对熵}.

设离散概率空间$X$上有两个概率分布$P$和$Q$,那么$P$相对于$Q$的$KL$散度定义如下:
\begin{equation}
D_{KL}(P||Q)=\sum_{x\in X}P(x)ln(\frac{P(x)}{Q(x)})=\sum_{x\in X}P(x)(ln(P(x))-ln(Q(x)))
\end{equation}

对于连续型随机变量,设概率空间$X$上有两个概率分布$P$和$Q$,其概率密度分别为$p$和$q$,那么,$P$相对于$Q$的$KL$散度定义如下:
\begin{equation}
D_{KL}(P||Q)=\int_{-\infty}^{+\infty}p(x)ln(\frac{p(x)}{q(x)})dx
\end{equation}

显然,当$P=Q$时,$D_{KL}=0$.

根据$KL$散度的定义,可以知道此度量是没有对称性的.也就是说,$P$相对于$Q$和$KL$散度一般并不等于$Q$相对于$P$的.由于$KL$散度能够衡量两个概率分布之间的差异,现在广泛用于机器学习中,用于评估生成模型所产生的数据分布与实际数据之间的差异程度.