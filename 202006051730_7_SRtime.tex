% 时间的变换与钟慢效应

\pentry{尺缩效应\upref{SRsmt}}




伽利略变换中时间是绝对而独立的,和空间无关.但我们在事件与尺缩效应\upref{SRsmt}词条中已经知道,引入光速不变原理之后会发现,两个独立事件\footnote{注意,我们已经声明过将同时同地的事件都看作同一个,因此此处的“独立事件”不可能是同时同地的.}是否同时,取决于观察的惯性系是哪个.

\subsection{测量事件发生时间}

在一个一维空间中,发生了一个事件,我们自然能知道这个事件发生的位置.实际上,因为我们要求事件发生的时候都同时发射一道球面光,在空间各个地方布满光感探测器以后,就可以通过探测的结果来反推球面光的形状,进而计算出球心位置.当然,我们没必要真的这么麻烦地计算,因为毕竟我们是在做思想实验,只需要明确“球面波的形状可测量”,进而得出“事件发生的位置是可知的”就行了.

但是事件发生的时间需要一点技巧去求得.很容易想到,利用光速不变原理就可以把对位置的知识转化成对时间的知识:首先让时间和空间的原点处($t=0, x=0$)发生一个事件,不妨称之为\textbf{参考事件}.在某时某地发生了另一个需要我们观测的事件,那么参考事件的球面光和待观测事件的球面光相遇也是一个事件.相遇事件发生的位置到原点的距离,除以光速,就可以得到相遇事件发生的时间;同理也可以得到相遇事件和待观测事件所发生的时间差,进而计算出待观测事件发生的时间.

\begin{figure}[ht]
\centering
\includegraphics[width=14cm]{./figures/SRtime_1.pdf}
\caption{请添加图片描述} \label{SRtime_fig1}
\end{figure}

这个方法在我们推导两个惯性系中时间的变换时非常好用.

\subsection{时间的变换}

沿用铁轨系$K_1$和火车系$K_2$的设定,令$K_2$相对$K_1$以速度$v$运动.我们现在想计算一下,在$K_1$中任意点$(x_1,t_1)$发生的事件,在$K_2$中应该对应哪个坐标$(x_2,t_2)$.
