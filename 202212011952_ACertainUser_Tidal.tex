% 潮汐力

\begin{issues}
\issueDraft
\end{issues}
\begin{figure}[ht]
\centering
\includegraphics[width=8cm]{./figures/Tidal_1.pdf}
\caption{潮汐力分布示意图(地球参考系)} \label{Tidal_fig1}
\end{figure}

我们先以月球对地球的潮汐力为例,并忽略地球自传、公转等因素.

\begin{figure}[ht]
\centering
\includegraphics[width=12cm]{./figures/Tidal_2.pdf}
\caption{月球对地球各点的引力不同} \label{Tidal_fig2}
\end{figure}

众所周知,引力与两点距离有关:
$$\bvec F_\text{引力}= - \frac{GMm}{r^2}\hat R \qquad a_\text{引力}=-\frac{GM}{r^2}\hat R$$

如\autoref{Tidal_fig2}  所示,由于月球到地球各点的距离不同,所以准确的说,月球对地球各点的引力也不同,而潮汐现象正是由于引力的这种细微差异.我们分别分析 地球表面上正对月球$A$点、背对月球的$B$点的引力:
\begin{equation}
\begin{aligned}
F_A &= - \frac{GM_\text{月}\Delta m}{(R-r)^2} \hat R\\
F_B &= -\frac{GM_\text{月} \Delta m}{(R+r)^2}\hat R\\
\end{aligned}
\end{equation}
以及月球对地球整体施加的引力加速度%这里我不大会表述...不是地球的自转
\begin{equation}
a_\text{地球} = -\frac{GM_\text{月}}{R^2}\hat R
\end{equation}
其中$r$是地球半径,$R$是地月距离,$\hat R$是从月球指向地球的单位方向向量,负号表示引力是指向月球的.

\begin{figure}[ht]
\centering
\includegraphics[width=10cm]{./figures/Tidal_3.pdf}
\caption{地球表面上,正对与背对月球的两点都被拉起} \label{Tidal_fig3}
\end{figure}

而从地球这一非惯性参考系上看,各点都还受一个离心加速度,大小等于地球质心的加速度,方向与其相反$$a_\text{离}=\frac{GM_\text{月}}{R^2}\hat R$$.如\autoref{Tidal_fig3} 所示,考虑到这个因素后,有趣的事情发生了:在一个地球人看来,地球表面上正对与背对月球的两点似乎都正被拉起,这就是潮汐力.某种意义上,潮汐力也是由于我们处于地球这一非惯性参考系而带来的错觉.

A点处的潮汐力为
$$
\bvec F_A' =\bvec F_A + f_\text{惯性} =  -\frac{GM_\text{月}\Delta m}{(R-r)^2}\hat R + \frac{GM_\text{月}\Delta m}{R^2} \hat R
\approx -\frac{2rGM_\text{月}\Delta m}{R^3} \hat R
$$
B点处的潮汐力为
$$
-\frac{GM_\text{月}}{(R+r)^2}\hat R + \frac{GM_\text{月}}{R^2} \hat R
\approx \frac{2rGM_\text{月}}{R^3} \hat R
$$

再强调一次,无论是正对月球的一面,还是背对月球的一面,潮汐力的方向都是离开地表的.因此,潮汐力不能简单地理解为“因为地球的一面离月球更近,所以那一侧的海洋被吸起了”.由于潮汐力的这一非同寻常的性质,因此一天(月球绕地球运动一圈)之内,地球出现两次潮汐现象.
\begin{figure}[ht]
\centering
\includegraphics[width=12cm]{./figures/Tidal_4.png}
\caption{一天内,地球出现两次潮汐现象} \label{Tidal_fig4}
\end{figure}

同时,我们发现潮汐力是与距离的三次方成反比的,这意味着感性上说“潮汐力比引力衰减得更快”.
\begin{exercise}{谁的潮汐力大?}
你可能觉得太阳的质量远比月球大,因此太阳对地球产生的潮汐作用更明显.试查找相关数据,并判断哪个产生了更大的潮汐力\footnote{如果你没算错的话,月球产生的潮汐力大概是太阳的二倍多.}.
\end{exercise}




绘制地球表面潮汐力分布的octave/matlab代码:
\begin{lstlisting}[language=matlab]
clc
clear

[x y z] = sphere(20);
xs = -60;
scatter3(xs,0,0);

u = x - xs;
v = y;
w = z;

r = sqrt(u.^2+v.^2+w.^2);
mag = 10^4./(r.^2);

u = -mag.*u./r;
v = -mag.*v./r;
w = -mag.*w./r;
u = u + 10^4/(xs^2); %地球参考系这一非惯性参考系中的离心力

hold on
axis equal
axis off
surf(x,y,z,'FaceColor','none','EdgeColor',[0.8 0.8 0.8]);
axis([-1.2 1.2 -1.2 1.2 -1.2 1.2])
quiver3(x,y,z,u,v,w);
view(-30,30);
hold off

\end{lstlisting}