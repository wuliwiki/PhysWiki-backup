% 经典形核理论

\begin{figure}[ht]
\centering
\includegraphics[width=14cm]{./figures/NCLT_1.png}
\caption{形核时,虽然体积自由能降低,但表面自由能升高}} \label{NCLT_fig1}
\end{figure}

虽然相变是降低自由能的过程;但相变时新产生的相界面又提高了自由能,二者间存在竞争.过冷液体中形成一半径为r的固体晶胚前后,总自由能变化 
\begin{equation}
\Delta G  = \Delta G_V +\Delta G_S
\end{equation}

\begin{itemize}
\item $\Delta G_V = \frac{4}{3}\pi r^3 \Delta G_B$是体积自由能变
\item $\Delta G_B = \Delta H \frac{\Delta T}{T_M}$ 是单位体积自由能变
\item $\Delta G_S = 4\pi r^2 \gamma$是表面自由能变
\end{itemize}

随后,系统将沿自由能减少的方向自发运动.
\begin{figure}[ht]
\centering
\includegraphics[width=5cm]{./figures/NCLT_2.png}
\caption{请添加图片描述} \label{NCLT_fig2}
\end{figure}


令 $\dv{\Delta G}{r} = 0$, 解得$r_k=-\frac{2\gamma}{\Delta G_B}$.此为临界形核半径;只有初始大小r>rk的晶胚能自发长大、形成晶核;更小的晶胚将自发衰亡