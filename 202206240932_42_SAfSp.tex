% 仿射子空间
% 仿射子空间|平面|超平面|方向子空间|平行

\begin{issues}
\issueTODO
\end{issues}

\pentry{仿射空间\upref{AfSp}}
本节将引入仿射子空间的概念,仿射子空间也称为仿射空间中的平面.0维的仿射子空间是个点,1维的是直线,$n-1$ 维的则是超平面($n$ 为仿射空间的维数).本节将证明,仿射空间中的平面本身也是一个仿射空间,且任何的平面包含通过平面上两不同点的直线.此外,平面作为仿射空间,其装备了一个方向子空间(即与其相配备的矢量空间),若两平面的方向子空间相同,就称它们平行,平行的平面必能通过相互平移得到.抛去仿射空间的内容,这些都与我们通常的几何直觉相一致.以上内容都能在本节得到.
\subsection{仿射子空间}
\begin{definition}{}
设 $(\mathbb A,V)$ 是个 $n$ 维的仿射空间,$U$ 是 $V$的 矢量子空间\upref{SubSpc}.在 $\mathbb A$ 中固定一点 $\dot p$,称集合
\begin{equation}\label{SAfSp_eq1}
\Pi=\dot p+U=\{\dot p+u|u\in U\}
\end{equation}
是 $\mathbb A$ 的一个 $m=\dim U$ 维的\textbf{平面}(或\textbf{仿射子空间}).当 $m=0$ 是,$\Pi$ 称为\textbf{点};$m=1$ 称为\textbf{直线};$m=n-1$ 称为\textbf{超平面}.$U$ 称为 $\Pi$ 的\textbf{方向子空间}.显然,由于 $\dot p=\dot p+0\in \Pi$ \footnote{这里0是 $V$ 中的矢量},故也称 $\Pi$ 是经过点 $\dot p$ 的在方向子空间 $U$ 上的仿射子空间.
\end{definition}
\begin{theorem}{}
仿射空间 $(\mathbb A,V)$ 中的平面 $\Pi=\dot p+U$ 本身也是个仿射空间,它与矢量空间 $U$ 相伴.
\end{theorem}
\textbf{证明:}由于 $U\in V$ 且 $U$ 是个矢量子空间,故
\begin{equation}
\dot q+0=\dot q,\quad (\dot q+v)+u=\dot q+(v+u)
\end{equation}
显然对 $\forall \dot q\in\mathbb A,\forall u,v\in U$ 成立.于是便得到了仿射空间定义中的性质1(\autoref{AfSp_def1}~\upref{AfSp}).

其次,由\autoref{SAfSp_eq1} ,对 $\forall \dot q,\dot q'\in\Pi$ ,$\exists u,u'\in U$ ,使得 $\dot q=\dot p+u,\dot q'=\dot p+u'$.则
\begin{equation}
\vec{qq'}=\vec{pq'}-\vec{pq}=u'-u\in U
\end{equation}
存在.并由 $\dot q,\dot q'$ 在 $V$ 中对应矢量的唯一性,$\vec{qq'}\in U$ 必唯一.于是得到仿射空间定义中的性质2.

\textbf{证毕!}

该定理表明,$\dot p$ 完全确定了 $\Pi$ 到 $U$ 的双射(\autoref{AfSp_sub1}~\upref{AfSp} 中性质1).于是 $U=\{\vec{pq}|q\in \Pi\}$ ,即方向子空间 $U$ 完全由平面 $\Pi$ 确定.

对 1 维的直线 $\Pi$,设矢量空间 $V$ 定义在域 $\mathbb F$ 上,则
\begin{equation}
\Pi=\{\dot p+\lambda\vec{pq}|\lambda\in \mathbb F\}
\end{equation}
\begin{theorem}{}
子集 $\Pi \subset\mathbb A$ 是个过点 $\dot p$ 的平面,当且仅当,它能整个包含通过其上两个不同点的直线(域 $\mathbb F$ 的特征(\autoref{field_def2}~\upref{field}) $\mathrm{ch}\,\mathbb F\neq 2$ ).
\end{theorem}
\textbf{证明:}1.
$\Rightarrow$

由 $\Pi$ 是个平面,则 $\Pi=\dot p+U$,$U\in V$ 是个矢量子空间.$\forall\dot q_1,\dot q_2\in \Pi$,过 $\dot q_1,\dot q_2$ 直线上的点必形为
\begin{equation}
\dot q_1+\lambda\vec{q_1q_2}=\dot p+\vec{pq_1}+\lambda\vec{q_1q_2}
\end{equation}
设 $\dot q_1=\dot p+u_1,\dot q_2=\dot p+u_2$ ,则
\begin{equation}
\begin{aligned}
&u_1,u_2\in U,\quad \vec{q_1q_2}=\vec{pq_2}-\vec{pq_1}=u_2-u_1\\
&\Downarrow\\
&\dot q_1+\lambda\vec{q_1q_2}=\dot p+u_1+\lambda(u_2-u_1)\in\dot p+U=\Pi
\end{aligned}
\end{equation}

2.$\Leftarrow$

设 $U=\{\vec{pq}|\dot q\in \Pi\}$,则只需证明 $U$ 是个矢量子空间.据条件,对 $\forall \dot q_1,\dot q_2\in\Pi$
\begin{equation}
\dot q_1+\lambda\vec{q_1q_2}=\dot p+\vec{pq_1}+\lambda(\vec{pq_2}-\vec{pq_1})\in\Pi
\end{equation}
 即
 \begin{equation}
 \vec{pq_1},\vec{pq_2}\in U\Rightarrow \vec{pq_1}+\lambda(\vec{pq_2}-\vec{pq_1})\in U
 \end{equation}
 取 $\dot{q_1}=\dot{p}$ ,并注意到 $\vec{pp}=0$,带入上式即得 
 \begin{equation}\label{SAfSp_eq2}
 \vec{pq_2}\in U\Rightarrow\lambda\vec{pq_2}\in U 
 \end{equation}
取 $\lambda=\frac{1}{2}$,则 
\begin{equation}
\vec{pq_1},\vec{pq_2}\in U\Rightarrow \frac{1}{2}(\vec{pq_2}+\vec{pq_1})\in U\Rightarrow \vec{pq_2}+\vec{pq_1}\in U
\end{equation}
由矢量子空间的定义,于是 $U$ 是 $V$ 的矢量子空间.

\textbf{证毕!}
\begin{corollary}{}
对 $(\mathbb A,V)$ 上的两个平面 $(\Pi',U'),(\Pi'',U'')$,那么 $\Pi=\Pi'\cap\Pi''$ 要么为空集,要么也是一个平面.当 $\Pi$ 为平面时,设 $U$ 是其方向子空间,则 $U=U'\cap U''$.
\end{corollary}
\textbf{证明:}设 $\dot p\in \Pi'\cap\Pi''$ ,那么 $\Pi'=\dot p+U',\Pi''=\dot p+U''$ .此时,对 $\forall\dot q\in\Pi'\cap\Pi''$,就有 \begin{equation}
\dot q=\dot p+u'=\dot p+u'',\quad u'\in U',u''\in U''
\end{equation}
于是, $u'=u''\in U'\cap U''$.
\subsection{平行}