% 包络线
% keys 包络|包络线

曲线族的\textbf{包络}是这样一条曲线,在这曲线上的每一点,都有曲线族中一条曲线与它相切.即是说,在这曲线上的每一点,这曲线与曲线族中通过这点的曲线有公切线.称这条曲线为曲线族的\textbf{包络线}.

若曲线族为 
\begin{equation}\label{Velope_eq1}
F(x,y,C)=0
\end{equation}
其中 $C$ 为任意常数(参数).则它的包络线上的任一点 $M(x,y)$ 必须同时满足方程组
\begin{equation}
\left\{\begin{aligned}
&F(x,y,C)=0\\
&F_C(x,y,C)=0
\end{aligned}\right.
\end{equation}
\subsection{证明}
首先确定曲线族\autoref{Velope_eq1} 的切线斜率,对等式\autoref{Velope_eq1} 求微商,并注意 $y$ 是 $x$ 的函数, $c$ 是任意常数,就得到
\begin{equation}
F_x+F_y\dv{y}{x}=0
\end{equation}
由此
\begin{equation}\label{Velope_eq5}
\dv{y}{x}=-\frac{F_x}{F_y}
\end{equation}

设包络线方程为
\begin{equation}\label{Velope_eq2}
R(x,y)=0
\end{equation}
由于包络线上任一点 $M(x,y)$同时在曲线族中的一曲线上,所以包络线方程\autoref{Velope_eq2} 可写为
\begin{equation}\label{Velope_eq3}
R(x,y)=F(x,y,C)
\end{equation}
由此可确定 $C$ 是 $x$ 与 $y$ 的什么函数.于是我们可以找到形如\autoref{Velope_eq1} 的曲线族的包络线方程,只不过这里 $C$ 不是常数,而是 $x$ 与 $y$ 的未知函数.\autoref{Velope_eq3} 求微分,注意 $c$ 已经不是常数,得到
\begin{equation}\label{Velope_eq4}
F_x\dd x+F_y\dd y+F_C\dd C=0
\end{equation}

由条件,包络线切线的斜率 $\dv{y}{x}$ ,应当与曲线族\autoref{Velope_eq1} 中过这切点的曲线的切线斜率相同,即\autoref{Velope_eq4} 给出的 $\dv{y}{x}$ 应当与\autoref{Velope_eq5} 的表达式相同.比较可知
\begin{equation}
F_C\dd C=0
\end{equation}
$\dd C$ 给出 $C$ 是常数,得到的是曲线族中的曲线,而不是包络,为得到包络
,应有
\begin{equation}
F_C=0
\end{equation}
由这方程确定出 $C$ 是 $(x,y)$