% 随机变量 概率分布函数
% 随机变量|概率|概率分布|随机分布|分布函数

% 未完成, 讲概念, 归一化, 标准差, 方差等.

\pentry{定积分\upref{DefInt}}

生活中有许多现象可以看做是随机的, 例如掷骰子的点数. 事实上骰子作为一个宏观物体, 其运动可以用一个复杂的动力学方程来精确描述. 但经过诸如 “摇匀” 这种混沌过程后, 方程的最终结果对初始条件极为敏感, 使结果难以预测. 这时我们就有充分的理由将该结果看作是随机的, 并用一个变量来表示可能的结果(就像把方程中的未知数用 $x$ 表示). 我们把这样的变量称为\textbf{随机变量}.

随机变量可以是\textbf{离散}的也可以是\textbf{连续}的, 例如掷骰子的点数只能取 1 到 6 的离散值, 而打靶时子弹离靶心的距离就可以用一个连续的随机变量表示. 一些更复杂事件的结果可能需要用到不止一个随机变量来描述, 本文只讨论单个随机变量, 但结论容易拓展到多个变量.

对于一些离散的随机变量, 可能发现每个离散值得到的概率也都是恒定的. 对一个公平的骰子, 所有的点数得到的概率都是 $1/6$; 对一个公平的硬币, 掷到正反两面的概率都是 $1/2$. 如果骰子或硬币是不公平的, 不同结果会对应不同的概率, 但这些概率也是固定的. 对于连续的随机变量, 得到不同值的概率可能也是固定的, 然而这些值有无穷多个, 应该如何描述他们对应的概率呢?

\subsection{连续随机变量的分布函数}
我们可以用\textbf{概率分布函数(probability distribution function, PDF)}来描述一个变量取各个值的概率. 假设一个连续随机变量 $x$ 可以在某个区间内取值, 我们就把该区间分为 $n$ 份, 第 $i$ 个子区间的长度为 $\Delta x_i$ 然后我们做大量的实验(记为 $N$ 次), 把随机变量得到的每个值分类归入这 $n$ 个子区间中, 并把第 $i$ 个区间中值的个数记为 $N_i$. 现在我们可以画出一种表示概率的\textbf{直方图(histogram)}, 令第 $i$ 个区间的长方形高度为 $y_i = N_i/(N \Delta x_i)$, 则每个长方形的面积 $y_i \Delta x_i = N_i/N$ 表示随机变量的值落在第 $i$ 个区间的概率, 注意所有长方形的面积之和为 1.

% 未完成: 此处应有图, 左图是 N 为有限值, 右图是 PDF

现在, 我们令区间数 $n\to \infty$ 且每个区间长度 $\Delta x_i \to 0$, 则离散的 $y_i$ 值就可以表示为函数 $y = f(x)$. 我们可以用定积分来表示 “所有长方形的面积之和为 1” , 即\footnote{注意积分上下限是 $x$ 取值的区间, 以下为了方便表示, 我们取整个实数域, 可以理解为超出区间的部分概率分部函数为 0.}
\begin{equation}\label{RandF_eq1}
\int_{-\infty}^{+\infty} f(x)\dd x = 1
\end{equation}
该式叫做概率分布函数的\textbf{归一化}. 满足归一化意味着, 所有情况发生的概率总和为 1.

若我们要求随机变量落在区间 $[a,b]$ 内的概率, 就求 $[a,b]$ 区间内分布函数下方的面积即可. 更常见地, 我们可以用微分式
\begin{equation}
\dd{P} = f(x) \dd{x}
\end{equation}
表示 $x$ 处长度为 $\dd{x}$ 的区间微元对应的概率 $\dd{P}$. 所以 $f(x)$ 又被称为\textbf{概率密度(probability density)}.

\subsection{平均值}
大学物理中, 随机变量 $x$ 的平均值通常被表示为 $\bar x$ 或者 $\ev{x}$, 我们以后都会使用.

对于离散的情况, 某个量的平均值等于每个可能的值出现的概率乘以该值再求和, 即
\begin{equation}
\ev{x} = \sum_i x_i P_i
\end{equation}

要求某个分布的平均值,我们同样可以将整个区间划分为 $n$ 个子区间, 每个区间的概率近似为 $P_i = f(x_i) \Delta x_i$, 则平均值为
\begin{equation}
\ev{x} \approx \sum_{i=0}^n x_i P_i = \sum_{i=1}^n x_i f(x_i) \Delta x_i
\end{equation}
用定积分\upref{DefInt}的思想, 当子区间无限多且取无限小时, 上式变为
\begin{equation}\label{RandF_eq7}
\ev{x} = \int_{-\infty}^{+\infty} x f(x) \dd{x}
\end{equation}

\subsection{方差}
离散情况下, 若已知平均值 $\ev{x}$, \textbf{方差}(每个数据点离平均值距离的平方的平均值) 可定义为
\begin{equation}
\sigma_x^2 \approx \sum_{i=0}^n (x_i - \bar x)^2 P_i
\end{equation}
与计算平均值的思路类似, 将方差拓展到连续变量的情况得
\begin{equation}\label{RandF_eq6}
\sigma_x^2 = \int_{-\infty}^{+\infty} \qty(x-\bar x)^2 f(x) \dd{x}
\end{equation}

\begin{exercise}{}
某直流电源存在微小误差, 其电压随时间的函数为
\begin{equation}
U(t) = U_0 + \varepsilon \sin(\omega t)
\end{equation}
为衡量误差大小, 请计算电压的方差(用 $\varepsilon$ 表示). 提示: 由于电压变化是周期性的, 可以只在一个周期内积分.
\end{exercise}

\subsection{任意函数的平均}
更一般地, 我们可以对离散的随机变量 $x_i$ 定义任意函数 $g(x)$ 的平均值为
\begin{equation}\label{RandF_eq3}
\ev{g} = \sum_{i=0}^n g(x_i) P_i
\end{equation}
例如在计算平均值和方差时, $g(x)$ 分别取 $x$ 和 $(x - \bar x)^2$.

拓展到连续的随机变量, 有
\begin{equation}\label{RandF_eq2}
\ev{g} = \int_{-\infty}^{+\infty} g(x) f(x) \dd{x}
\end{equation}

\begin{example}{分子的平均动能}
某气体中含有大量分子(阿伏伽德罗常数数量级: $10^{23}$), 若假设某时刻它们的速度大小 $v$ 的分布函数为
\begin{equation}
f(v) = A \sin[2](\frac{\pi v}{v_{m}}) \qquad (v \in [0, v_{m}])
\end{equation}
其中 $A$ 为常数. 请分别计算:
\begin{enumerate}
\item 常数 $A$, 使 $f(v)$ 满足归一化(\autoref{RandF_eq1})
\item 分子速度大小的平均值
\item 分子速度大小方差
\item 分子动能 $E_k = mv^2/2$ 的平均值
\item 分子动能的方差
\end{enumerate}

解:

1. 将 $f(v)$ 代入分布函数的归一化条件\autoref{RandF_eq1} 
\begin{equation}\label{RandF_eq5}
A\int_{0}^{v_{m}} \sin[2](\frac{\pi v}{v_{m}})\dd v= 1
\end{equation}
而由
\begin{equation}
\int \sin^2 x\dd x = \int\frac{(1-\cos 2x)}{2}\dd x = \frac{1}{2}x-\frac{1}{4}\sin 2x+C
\end{equation}
知
\begin{equation}\label{RandF_eq4}
\begin{aligned}
&\int_{0}^{v_{m}} \sin[2](\frac{\pi v}{v_{m}})\dd v=\frac{v_{m}}{\pi}\int_{0}^{v_{m}} \sin[2](\frac{\pi v}{v_{m}})\dd (\frac{\pi v}{v_{m}})\\
&=\frac{v_{m}}{\pi}\qty (\frac{1}{2}\frac{\pi v}{v_{m}}-\frac{1}{4}\sin \frac{2\pi v}{v_{m}})\Bigg|_{0}^{v_{m}}=\frac{v_{m}}{\pi}\qty(\frac{\pi}{2}-0)=\frac{v_{m}}{2}
\end{aligned}
\end{equation}
\autoref{RandF_eq4} 代入\autoref{RandF_eq5} ,得 
\begin{equation}
A=\frac{2}{v_{m}}
\end{equation}

2.由\autoref{RandF_eq2} 
\begin{equation}\label{RandF_eq8}
\begin{aligned}
\ev{v}&=A\int_{0}^{v_{m}} v\sin[2](\frac{\pi v}{v_{m}})\dd v=\frac{A}{2}\int_0^{v_m}v\qty[1-\cos\qty(\frac{2\pi v}{v_{m}})]\dd v\\
&=\frac{A}{2}\qty{\frac{v^2}{2}-\qty[\sin(\frac{2\pi v}{v_{m}})\cdot\frac{v}{\frac{2\pi}{v_m}}+\cos(\frac{2\pi v}{v_{m}})\cdot\frac{1}{\qty(\frac{2\pi}{v_m})^2}]}\Bigg|_{0}^{v_{m}}\\
&=\frac{A}{2}\qty(\frac{v_{m}^2}{2}-\frac{v_{m}^2}{4\pi^2}+\frac{v_{m}^2}{4\pi^2})=\frac{v_{m}}{2}
\end{aligned}
\end{equation}

3.由方差的定义\autoref{RandF_eq6} 和平均值的定义\autoref{RandF_eq7} ,我们可将方差写为
\begin{equation}
\sigma_x^2=\ev{(x-\ev{x})^2}=\ev{x^2-2x\ev{x}+\ev{x}^2}=\ev{x^2}-\ev{x}^2
\end{equation}
那么
\begin{equation}\label{RandF_eq10}
\sigma_v^2=\ev{v^2}-\ev{v}^2
\end{equation}
\begin{equation}\label{RandF_eq9}
\begin{aligned}
\ev{v^2}&=A\int_{0}^{v_{m}} v^2\sin[2](\frac{\pi v}{v_{m}})\dd v=\frac{A}{2}\int_{0}^{v_{m}}v^2\qty[1-\cos\qty(\frac{2\pi v}{v_{m}})]\dd v\\
&=\frac{A}{2}\qty{\frac{v^3}{3}-\qty[\sin(\frac{2\pi v}{v_{m}})\qty[\frac{v^2}{\qty(\frac{2\pi }{v_{m}})^2}-\frac{2}{\qty(\frac{2\pi }{v_{m}})^3}]+\cos(\frac{2\pi v}{v_{m}})\qty[\frac{2v}{\qty(\frac{\pi}{v_{m}})^2}]]}\Bigg|_0^{v_{m}}\\
&=A\qty(\frac{v_{m}^3}{\pi}-\frac{4v_{m}^3}{\pi^3})=\frac{2v_{m}^2}{\pi}-\frac{8v_{m}^2}{\pi^3}
\end{aligned}
\end{equation}
\autoref{RandF_eq8} 和\autoref{RandF_eq9} 代入\autoref{RandF_eq10}, 方差为
\begin{equation}
\sigma_v^2=\frac{2v_{m}^2}{\pi}-\frac{8v_{m}^2}{\pi^3}-\frac{4v_{m}^2}{\pi^2}
\end{equation}

4.
\begin{equation}\label{RandF_eq11}
\ev{E_k}=\ev{\frac{mv^2}{2}}=\frac{m}{2}\ev{v^2}=\frac{mv_{m}^2}{\pi}-\frac{4mv_{m}^2}{\pi^3}
\end{equation}

5.
\begin{equation}\label{RandF_eq12}
\sigma_{E_k}=\ev{E_k^2}-\ev{E_k}^2
\end{equation}
而
\begin{equation}\label{RandF_eq13}
\ev{E_k^2}=\frac{m^2}{4}\ev{v^4}
\end{equation}
\begin{equation}
\begin{aligned}
\ev{v^4}&=A\int_0^{v_{m}}v^4\sin[2](\frac{\pi v}{v_{m}})\dd v \\
&=A\qty{\sin(\frac{\pi v}{v_{m}})\qty[\frac{4v^3}{(\frac{\pi}{v_{m}})^2}-\frac{24v}{(\frac{\pi}{v_{m}})^4}]-\cos(\frac{\pi v}{v_{m}})\qty[\frac{v^4}{\frac{\pi}{v_{m}}}-\frac{12v^2}{(\frac{\pi}{v_{m}})^3}+\frac{24}{(\frac{\pi}{v_{m}})^5}]}\bigg|_0^{v_{m}}\\
&=A\qty(\frac{v_{m}^5}{\pi}-\frac{12v_{m}^5}{\pi^3}+\frac{24v_{m}^5}{\pi^5}+\frac{24v_{m}^5}{\pi^5})\\
&=\frac{2v_{m}^4}{\pi}\qty(1-\frac{12}{\pi^2}+\frac{48}{\pi^4})
\end{aligned}
\end{equation}
联立 \autoref{RandF_eq11}  、\autoref{RandF_eq12} 、 和\autoref{RandF_eq13} 
\begin{equation}
\begin{aligned}
\sigma_{E_k}&=\frac{m^2v_{m}^4}{2\pi}\qty(1-\frac{12}{\pi^2}+\frac{48}{\pi^4})-\qty(\frac{mv_{m}^2}{\pi}-\frac{4mv_{m}^2}{\pi^3})^2\\
&=\frac{m^2v_{m}^4}{2\pi}\qty(1-\frac{2}{\pi}-\frac{12}{\pi^2}+\frac{16}{\pi^3}+\frac{48}{\pi^4}-\frac{32}{\pi^5})
\end{aligned}
\end{equation}
\end{example}
