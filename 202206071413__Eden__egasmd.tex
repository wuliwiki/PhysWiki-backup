% 近自由电子模型
% 自由电子气体|布洛赫波|晶格|能带

\pentry{金属中的自由电子气体\upref{mfcgas},布洛赫理论\upref{Bloch}}

在自由电子模型中,我们将电子视作了满足周期性边界条件的平面波(理想气体单粒子能级密度\upref{IdED1}),也就是将电子气体视作了理想气体,来采用相应的能态密度 $D(\epsilon)$.在这样的近似下,我们得到了重要的结果,例如不同能量电子的分布,电子气体对金属热容的贡献等.但这种模型许多严重的缺陷,例如它不能解释为什么有些固体是金属,而有些固体是绝缘体;它不能解释固体丰富的热学、电学或光学性质.为此,我们需要的是\textbf{近自由电子模型},将固体中周期势场作为微扰项考虑进来,考察周期势场下的电子波函数的行为及能带关系.近自由电子模型几乎给出了关于金属中电子行为所有的定性问题的答案.
\subsection{近自由电子的哈密顿量}
设周期性势场 为 $V(x)$,它包括固体中固定不动的离子实产生的库伦势,以及固体中的巡游电子在 $x$ 处产生的平均势能.由于我们采用平均势作为近自由电子所受到的势能的近似,我们常称这种方法为\textbf{平均场近似}.
那么我们可以将固体中一个电子的哈密顿量可以近似地表达为
\begin{equation}
H=-\frac{\hbar^2}{2m}\nabla^2+V(x)=H_0+V
\end{equation}

\subsection{带隙的来源}
\subsection{微扰论与布洛赫函数}