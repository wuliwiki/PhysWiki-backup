% 逆序数
% 逆序数|排列

\begin{issues}
\issueDraft
\end{issues}

\pentry{排列\upref{permut}}

\subsection{逆序对}
\footnote{参考 Wikipedia \href{https://en.wikipedia.org/wiki/Inversion_(discrete_mathematics)}{相关页面}.}我们把集合 $\qty{1,\dots,N}$ 的第 $n$ 种排列记为 $p_n$, 该排列的元素按照顺序分别记为 $p_{n,1}, \dots, p_{n,N}$. 对于任意 $i < j$, 如果满足 $p_{n,i} > p_{n,j}$ 我们就把 $i, j$ 或者 $p_{n,i}, p_{n,j}$ 称为排列 $p_n$ 的一个\textbf{逆序对}.

\subsection{逆序数}
\textbf{逆序数(Inversion number)}的定义为
\begin{equation}\label{InvNum_eq5}
N_p = \sum_{i=2}^N \text{满足}\, j<i \,\text{且}\, P_n(j) > P_{n,i} \, \text{的}\, j\, \text{的个数} 
\end{equation}

在行列式和 Levi-Civita 符号中, 我们只对逆序数的奇偶性感兴趣.

交换排列中任意两个数, 逆序数奇偶性改变. 
