% 阿尔-花拉子米(综述)
% license CCBYSA3
% type Wiki

本文根据 CC-BY-SA 协议转载翻译自维基百科\href{https://en.wikipedia.org/wiki/Al-Khwarizmi}{相关文章}。
\begin{figure}[ht]
\centering
\includegraphics[width=6cm]{./figures/6cab70e0324244c5.png}
\caption{20世纪木刻版画,描绘阿尔-花拉子米} \label{fig_HLZM_1}
\end{figure}
穆罕默德·伊本·穆萨·阿尔-花拉子米(约780年–约850年),或简称阿尔-花拉子米,是一位波斯学者,致力于数学、天文学和地理学的研究,并在阿拉伯语学术领域产生了深远的影响。约在公元820年,他在巴格达的智慧之家工作,该地是阿拔斯王朝的首都。

他在代数领域的代表性著作《阿尔-贾布尔》(《计算的简明书》)编写于813年至833年之间,提出了线性方程和二次方程的系统解法。他在代数方面的成就之一是通过完成平方法来解二次方程,并为此提供了几何证明。由于阿尔-花拉子米是第一个将代数作为独立学科来研究的人,并且引入了“简化”和“平衡”方法(将减去的项转移到方程的另一边,即在方程两边相同项的消去),因此他被誉为代数的奠基人或创始人。英语中的“algebra”一词就来源于他那本著作的简写标题(阿尔-贾布尔,意为“完成”或“重新联合”)。他的名字还衍生出了英语中的“algorism”和“algorithm”以及西班牙语、意大利语和葡萄牙语中的“algoritmo”;以及西班牙语中的“guarismo”和葡萄牙语中的“algarismo”,均表示“数字”。

12世纪时,阿尔-花拉子米关于印度算术的教科书《印度人数字算法》(Algorithmo de Numero Indorum)的拉丁文翻译,规范了印度数字系统,并将基于十进制的位值计数系统引入西方世界。同样,阿尔-贾布尔由英学者罗伯特·查斯特在1145年翻译成拉丁文,并一直作为欧洲大学的主要数学教材,直到16世纪。

阿尔-花拉子米修订了公元2世纪由罗马学者克劳狄乌斯·托勒密撰写的《地理学》,列出了城市和地方的经纬度。他还制作了一套天文表格,并撰写了关于历法的作品,以及关于天文仪器和日晷的研究。阿尔-花拉子米在三角学方面也作出了重要贡献,编制了准确的正弦和余弦表,并制作了第一个正切表。
\subsection{生活}
\begin{figure}[ht]
\centering
\includegraphics[width=6cm]{./figures/27f5f3869a4239e7.png}
\caption{马德里的大学城(Ciudad Universitaria)中穆罕默德·伊本·穆萨·阿尔·胡瓦里兹米的纪念碑} \label{fig_HLZM_2}
\end{figure}
很少有关于穆罕默德·伊本·穆萨·阿尔·胡瓦里兹米(al-Khwarizmi)生活的确切细节。伊本·纳迪姆(Ibn al-Nadim)将他的出生地定为胡瓦尔兹姆(Khwarazm),他通常被认为来自这个地区。作为波斯人,他的名字意味着“来自胡瓦尔兹姆”,这个地区曾是大伊朗的一部分,现在属于土库曼斯坦和乌兹别克斯坦。

塔巴里(al-Tabari)给出的名字是穆罕默德·伊本·穆萨·阿尔·胡瓦里兹米·阿尔·马久西·阿尔·库特鲁布布利(Muḥammad ibn Musá al-Khwārizmī al-Majūsī al-Qūṭrubbullī)。其中的“阿尔·库特鲁布布利”(al-Qutrubbulli)可能表示他来自巴格达附近的库特鲁布尔(Qutrubbul)。然而,罗什迪·拉谢德(Roshdi Rashed)对此表示否认,认为这只是早期手稿的错误。在另一种看法中,大卫·A·金(David A. King)认为他是来自库特鲁布尔地区,因为他被称为“阿尔·胡瓦里兹米·阿尔·库特鲁布布利”,可能是因为他出生在巴格达附近。

关于他的宗教信仰,托默(Toomer)写道,塔巴里给他的另一个称号“阿尔·马久西”(al-Majūsī)似乎表明他是古老的琐罗亚斯德教信徒。这种信仰在那个时代对于伊朗裔人士来说依然有可能,但胡瓦里兹米的《代数》序言表明他是正统的穆斯林,因此“阿尔·马久西”这一称号可能仅仅意味着他的祖先,甚至可能是他年轻时曾是琐罗亚斯德教徒。

伊本·纳迪姆的《历史大典》(Al-Fihrist)中有简短的胡瓦里兹米传记及其著作清单。胡瓦里兹米的工作主要集中在813至833年之间。穆斯林征服波斯后,巴格达成为了科学研究和贸易的中心。大约820年,他被任命为天文学家,并成为智慧之宫(House of Wisdom)图书馆的馆长。智慧之宫是由阿拔斯哈里发阿尔·马蒙(al-Ma'mūn)创立的。胡瓦里兹米研究了包括翻译希腊文和梵文科学手稿在内的各种科学和数学内容。他还是一位历史学家,曾被塔巴里等人引用。

在阿尔·瓦西克(al-Wathiq)统治时期,他据说参与了两次使节任务,其中之一是前往可萨(Khazars)。道格拉斯·莫顿·邓洛普(Douglas Morton Dunlop)建议,穆罕默德·伊本·穆萨·阿尔·胡瓦里兹米可能与穆罕默德·伊本·穆萨·伊本·沙基尔(Muḥammad ibn Mūsā ibn Shākir),即三兄弟班努·穆萨(Banū Mūsā)中的长子是同一个人。
\subsection{贡献}
\begin{figure}[ht]
\centering
\includegraphics[width=6cm]{./figures/dfca5e36d9defc42.png}
\caption{《阿尔·胡瓦里兹米的代数》中的一页} \label{fig_HLZM_3}
\end{figure}
阿尔·胡瓦里兹米在数学、地理、天文学和制图学方面的贡献为代数和三角学的创新奠定了基础。他系统化的线性和二次方程求解方法促成了代数的诞生,而“代数”这一词汇源自他关于这一主题的著作《Al-Jabr》(《完备与平衡计算书》)。

大约在820年写成的《使用印度数字的计算方法》在中东和欧洲传播了印度-阿拉伯数字系统。当该书在12世纪被翻译成拉丁文,名为《Algoritmi de numero Indorum》(阿尔·胡瓦里兹米关于印度算术的著作)时,“算法”这一术语被引入西方世界。

他的一些工作基于波斯和巴比伦天文学、印度数字和希腊数学。

阿尔·胡瓦里兹米对托勒密关于非洲和中东的资料进行了系统化和修正。另一部重要著作是《Kitab surat al-ard》(《地球的图像》;翻译为《地理学》),其中给出了基于托勒密《地理学》中的坐标值,改进了地中海、亚洲和非洲的值。

他还撰写了有关天文仪器如天体仪和日晷的书籍。他参与了一项测定地球周长的项目,并为阿尔·马蒙哈里发绘制了一张世界地图,监督了70位地理学家的工作。当他的著作通过拉丁文翻译在12世纪传播到欧洲时,对欧洲数学的发展产生了深远的影响。
\subsubsection{代数}
\begin{figure}[ht]
\centering
\includegraphics[width=6cm]{./figures/ff3c66dbb23f817d.png}
\caption{《阿尔·卡瓦里兹米的代数书》原始阿拉伯文印刷手稿} \label{fig_HLZM_6}
\end{figure}
\begin{figure}[ht]
\centering
\includegraphics[width=6cm]{./figures/773a045cdfee199e.png}
\caption{《阿尔·卡瓦里兹米的代数》由弗雷德里克·罗森翻译的英文版中的一页} \label{fig_HLZM_7}
\end{figure}
《代数学》(Al-Jabr,阿拉伯文:الكتاب المختصر في حساب الجبر والمقابلة,al-Kitāb al-mukhtaṣar fī ḥisāb al-jabr wal-muqābala)是一本大约在公元820年左右写成的数学书籍。该书在哈里发阿尔·马蒙的鼓励下编写,作为一本关于计算的流行著作,书中充满了许多示例和应用,涵盖了贸易、测量和法律继承等问题。[49] “代数”一词来源于该书中描述的基本方程操作之一(*al-jabr*,意为“恢复”,指的是在方程两边加上一个数以合并或取消项)。该书由罗伯特·查尔斯特(Robert of Chester)于1145年将其翻译为拉丁文《代数与相互平衡之书》(*Liber algebrae et almucabala*),因此形成了“代数”这一术语。杰拉德·克雷莫纳(Gerard of Cremona)也进行过翻译。一本独特的阿拉伯文手稿保存在牛津大学,并由F. 罗斯恩(F. Rosen)于1831年翻译。一份拉丁文翻译本保存在剑桥大学。[50]

该书提供了关于解决二次方程的详尽说明,并讨论了“还原”和“平衡”的基本方法,指的是将方程的项移到方程的另一边,即在方程两边取消相同的项。[51][52]

阿尔-花拉兹米(Al-Khwārizmī)解线性和二次方程的方法首先将方程化简为六种标准形式之一(其中 \(b\) 和 \(c\) 是正整数):
\begin{itemize}
\item 平方等于根(\(ax^2 = bx\))
\item 平方等于数(\(ax^2 = c\))
\item 根等于数(\(bx = c\))
\item 平方与根等于数(\(ax^2 + bx = c\))
\item 平方与数等于根(\(ax^2 + c = bx\))
\item 根与数等于平方(\(bx + c = ax^2\))
\end{itemize}
通过除去平方项的系数并使用两个操作,\textbf{al-jabr}(阿拉伯语:الجبر,意为“恢复”或“完成”)和 al-muqābala(意为“平衡”)。\textbf{al-jabr}是通过向方程两边加上相同的数值,从而消去负数项、根和平方项的过程。例如,\(x^2 = 40x - 4x^2\) 被化简为 \(5x^2 = 40x\)。al-muqābala 是将同类型的量移到方程同一边的过程。例如,\(x^2 + 14 = x + 5\) 被化简为 \(x^2 + 9 = x\)。

以上讨论使用了现代数学符号来表示书中讨论的问题类型。然而,在阿尔-花拉兹米的时代,大部分符号尚未被发明,因此他必须用普通文本来呈现问题和其解法。例如,对于一个问题,他写道(摘自1831年的翻译):

如果有人说:“你把十分成两部分:将其中一部分乘以自身;它将等于另一部分的八十一倍。” 计算:你说,十减去某物,乘以自身,等于一百加上一个平方减去二十个某物,这等于八十一个某物。从一百和一个平方中分离出二十个某物,并加到八十一上。然后它就会变成一百加上一个平方,等于一百零一个根。将根的二分之一求出,结果是五十又半。将其平方,得到二千五百五十又四分之一。从中减去一百,剩余二千四百五十又四分之一。提取平方根,结果是四十九又半。从根的二分之一,即五十又半中减去它,剩下的是一,这是两个部分之一。

在现代符号中,这个过程,其中 \(x\) 是“东西”(阿拉伯语:شيء,shayʾ)或“根”,可以通过以下步骤表示:
\[
(10 - x)^2 = 81x~
\]
\[
100 + x^2 - 20x = 81x~
\]
\[
x^2 + 100 = 101x~
\]
设方程的根为 \(x = p\) 和 \(x = q\)。那么:\(\frac{p + q}{2} = 50 \frac{1}{2}\)\(pq = 100\)
并且:
\[
\frac{p - q}{2} = \sqrt{\left(\frac{p + q}{2}\right)^2 - pq} = \sqrt{2550 \frac{1}{4} - 100} = 49 \frac{1}{2}~
\]
因此,根是:
\[
x = 50 \frac{1}{2} - 49 \frac{1}{2} = 1~
\]
有几位作者曾以《Kitāb al-jabr wal-muqābala》为题出版过著作,包括阿布·哈尼法·迪纳瓦里(Abū Ḥanīfa Dīnawarī)、阿布·卡米尔(Abū Kāmil)、阿布·穆罕默德·阿德利(Abū Muḥammad al-'Adlī)、阿布·尤素福·米斯西(Abū Yūsuf al-Miṣṣīṣī)、阿卜杜·哈米德·伊本·图尔克('Abd al-Hamīd ibn Turk)、辛德·伊本·阿里(Sind ibn 'Alī)、萨赫尔·伊本·比什尔(Sahl ibn Bišr)和沙拉夫·阿尔·丁·图西(Sharaf al-Dīn al-Ṭūsī)等。

所罗门·甘兹(Solomon Gandz)曾将阿尔-花拉兹米誉为代数学的奠基人:

阿尔-花拉兹米的代数学被认为是科学的基础和基石。从某种意义上讲,阿尔-花拉兹米比狄奥凡图斯更有资格被称为“代数学之父”,因为阿尔-花拉兹米是第一个以基础形式并专门讲授代数学的人,而狄奥凡图斯主要关注的是数论。[53]

维克托·J·凯茨(Victor J. Katz)补充道:

至今仍然存在的第一部真正的代数教材是穆罕默德·伊本·穆萨·阿尔-花拉兹米于公元825年左右在巴格达编写的《代数与平衡》一书。[54]

约翰·J·奥康纳和埃德蒙·F·罗伯逊在《MacTutor数学史档案》中写道:

阿拉伯数学的一个最重要的进步之一是阿尔-花拉兹米的工作,也就是代数学的起源。理解这一新思想的重要性是至关重要的。这是一次革命性的突破,摆脱了希腊数学的概念,后者本质上是几何学。代数学是一种统一的理论,它允许有理数、无理数、几何量等都可以作为“代数对象”来处理。它为数学提供了一条全新的发展路径,概念上远比之前存在的数学更加广阔,并为未来的学科发展提供了一个载体。代数思想的引入的另一个重要方面是,它使得数学能够以一种以前从未发生过的方式应用于自身。[55]

罗什迪·拉谢德和安吉拉·阿姆斯特朗写道:

阿尔-花拉兹米的文本不仅与巴比伦的泥板不同,而且与狄奥凡图斯的《算术》也有所不同。它不再是一个个待解的问题系列,而是一个从原始术语开始的阐述,这些组合必须给出所有可能的方程原型,方程也明确构成了研究的真正对象。另一方面,方程作为一个独立对象的思想从一开始就出现,可以说是以一种通用的方式出现,因为它不仅仅是在解决问题的过程中产生的,而是专门被用来定义一个无限的类问题。[56]

根据瑞士裔美国数学史学家弗洛里安·卡乔里的说法,阿尔-花拉兹米的代数学与印度数学家的工作不同,因为印度人没有像“恢复”和“减少”这样的规则。[57] 关于阿尔-花拉兹米的代数工作与印度数学家布拉马古普塔的工作的不同及其重要性,卡尔·B·博耶写道:

确实,在两个方面,阿尔-花拉兹米的工作比狄奥凡图斯的工作有所退步。首先,它远远比狄奥凡图斯问题中的代数要基础得多;其次,阿尔-花拉兹米的代数是完全修辞性的,没有希腊《算术》或布拉马古普塔工作中的省略符号!甚至数字也写成了文字,而不是符号!阿尔-花拉兹米很可能不知道狄奥凡图斯的工作,但他必须至少熟悉布拉马古普塔的天文和计算部分;然而,阿尔-花拉兹米和其他阿拉伯学者并未使用省略符号或负数。尽管如此,《代数》更接近于今天的初等代数,而不是狄奥凡图斯或布拉马古普塔的作品,因为这本书并不涉及不确定分析中的难题,而是对方程(尤其是二次方程)求解的直接且基础的阐述。阿拉伯人通常喜欢从前提到结论的清晰论证以及系统的组织结构——在这些方面,狄奥凡图斯和印度人都不如阿尔-花拉兹米。[58]
\subsubsection{算术}
\begin{figure}[ht]
\centering
\includegraphics[width=6cm]{./figures/c86ea77c4ab8bde1.png}
\caption{算法学派与算盘学派的争论,描绘于公元1508年的一幅素描中} \label{fig_HLZM_4}
\end{figure}
阿尔-花拉兹米的第二部最具影响力的作品是关于算术的,虽然原始阿拉伯文已失传,但该作品通过拉丁文翻译流传下来。他的著作包括《印度计算书》 (kitāb al-ḥisāb al-hindī) 和可能更加基础的《印度算术的加法与减法书》 (kitāb al-jam' wa'l-tafriq al-ḥisāb al-hindī) [60][61]。这些文本描述了可以在尘板上执行的十进制数字(印度-阿拉伯数字)算法。阿拉伯语中称此为“takht”(拉丁语:tabula),即用薄薄的尘土或沙子覆盖的板子,供进行计算,数字可以用尖笔写在上面,并在必要时轻松擦除或更换。阿尔-花拉兹米的算法被使用了近三百年,直到被阿尔-乌克利迪西的算法所取代,这些算法可以用笔和纸进行计算。[62]

作为12世纪阿拉伯科学通过翻译传入欧洲的浪潮的一部分,这些文本在欧洲引发了革命性的影响。[63] 阿尔-花拉兹米的拉丁化名字“Algorismus”演变成了用于计算的方法名称,并且至今存在于“算法”(algorithm)这一术语中。它逐渐取代了欧洲之前使用的算盘计算方法。[64]
\begin{figure}[ht]
\centering
\includegraphics[width=6cm]{./figures/6cd64d08f25039ba.png}
\caption{拉丁文翻译中的一页,开头为“Dixit algorizmi”} \label{fig_HLZM_5}
\end{figure}
虽然没有哪一部被认为是阿尔-花拉兹米原著的字面翻译,但仍然保留下了四部采用阿尔-花拉兹米方法的拉丁文著作:[60]

\begin{itemize}
\item 《Dixit Algorizmi》(1857年出版,标题为《Algoritmi de Numero Indorum》[65])[66]
\item 《Liber Alchoarismi de Practica Arismetice》
\item 《Liber Ysagogarum Alchorismi》
\item 《Liber Pulveris》
\end{itemize}

《Dixit Algorizmi》("阿尔-花拉兹米如是说")是剑桥大学图书馆一份手稿的开头短语,通常被称为1857年的《Algoritmi de Numero Indorum》。它归功于1126年翻译过天文表的巴斯的阿德拉尔德。这部作品也许最接近阿尔-花拉兹米的原著。[66]

阿尔-花拉兹米关于算术的工作为将基于印度数学中发展出的印度-阿拉伯数字系统的阿拉伯数字引入西方世界做出了贡献。术语“算法”源自于“algorism”,即阿尔-花拉兹米开发的使用印度-阿拉伯数字进行算术运算的技术。“算法”和“algorism”这两个词都源自阿尔-花拉兹米名字的拉丁化形式,分别为“Algoritmi”和“Algorismi”。[67]
\subsubsection{天文学}
\begin{figure}[ht]
\centering
\includegraphics[width=6cm]{./figures/5f588450d50b8326.png}
\caption{《科珀斯克里斯蒂学院手稿283页》,为阿尔·花拉子米的《天文表》(Zīj)的拉丁文翻译。} \label{fig_HLZM_8}
\end{figure}
《阿尔·花拉子米的《天文表》》(Zīj as-Sindhind,阿拉伯文:زيج السند هند,意为“《悉檀达天文表》”)是一部包含约37章的天文和历法计算的著作,内有116个表格,涉及历法、天文和占星数据,还包括正弦值表。这是第一部基于印度天文方法(称为悉檀达)的阿拉伯天文表(Zij)。"Sindhind"一词是梵文"Siddhānta"的变形,Siddhānta通常是指天文教科书。实际上,阿尔·花拉子米表格中的天体运动数据来源于“修正版布拉马悉檀达”(Brahmasphutasiddhanta),即布拉马吉普塔的天文学著作。

这部作品包含了当时已知的太阳、月亮和五大行星的运动表格。它标志着伊斯兰天文学的一个转折点。在此之前,穆斯林天文学家主要采用研究的方法,翻译他人的作品,并学习已发现的知识。

这部作品的原始阿拉伯文版本(约公元820年)已经遗失,但西班牙天文学家马斯拉马·马吉里提(约1000年)所写的版本,经过拉丁文翻译后得以流传,推测由巴斯的阿德拉尔(Adelard of Bath)于1126年1月26日完成翻译。现存的四部拉丁文手稿保存在以下图书馆:沙特尔公共图书馆、马萨林图书馆(巴黎)、西班牙国家图书馆(马德里)和博德利图书馆(牛津)。
\subsubsection{三角学}  
阿尔·花拉子米的《天文表》包含了正弦和余弦的三角函数表。[69] 一部关于球面三角学的相关论文也归于他。[55]

阿尔·花拉子米制作了精确的正弦和余弦表,并且是首个编制正切表的人。[72][73]
\subsubsection{《地理学》}
\begin{figure}[ht]
\centering
\includegraphics[width=14.25cm]{./figures/a2df7dfdfb5508b5.png}
\caption{Gianluca Gorni重建的《阿尔-花拉子密世界地图》中有关印度洋的部分。阿尔-花拉子密使用的大多数地名与托勒密、马特尔卢斯和贝海姆的地名相符。沿海的总体形状在塔普罗班那和卡蒂加拉之间相同。龙尾,即印度洋的东部入口,在托勒密的描述中并不存在,但在阿尔-花拉子密的地图上有很少的细节描绘,尽管在马特尔卢斯的地图和后来的贝海姆版本中则非常清晰和精确。} \label{fig_HLZM_9}
\end{figure}
阿尔·花拉子米的第三部重要著作是《地球图像书》(阿拉伯语:كتاب صورة الأرض,"Book of the Description of the Earth"),也称为他的《地理学》,完成于833年。这是对托勒密二世纪《地理学》的重大修订,内容包括在一个总引言之后列出了2402个城市和其他地理特征的坐标。[75]

《地球图像书》仅存一份副本,现保存在斯特拉斯堡大学图书馆。[76][77] 一份拉丁文翻译保存在西班牙马德里的国家图书馆。[78] 这本书以天气区的纬度和经度顺序开篇,即按纬度的区块进行排列,在每个天气区内则按经度顺序排列。正如保罗·加莱兹(Paul Gallez)所指出的,这种系统可以推导出许多纬度和经度,即使原文档的状况非常糟糕,几乎无法辨认。无论是阿拉伯文原本还是拉丁文翻译,都没有包含世界地图;然而,休伯特·道尼赫特(Hubert Daunicht)通过坐标列表成功重建了缺失的地图。他读取了手稿中沿海点的纬度和经度,或者从无法辨认的上下文中推导出它们。然后,他将这些点转移到绘图纸上,并用直线将它们连接起来,从而获得了接近原始地图的海岸线图。他还对河流和城镇进行了相同的处理。[79]
\begin{figure}[ht]
\centering
\includegraphics[width=6cm]{./figures/9ae26dc993dcb7c3.png}
\caption{一份15世纪的托勒密《地理学》版本,供对比使用。} \label{fig_HLZM_10}
\end{figure}
阿尔·花拉子米修正了托勒密对地中海长度的过高估计[80],从加那利群岛到地中海东岸,托勒密估计为63度经度,而阿尔·花拉子米则几乎准确地估计为接近50度经度。他“将大西洋和印度洋描绘为开放水域,而非像托勒密那样画成封闭的海洋。”[81] 因此,阿尔·花拉子米的本初子午线位于幸运岛附近,约比马里努斯和托勒密使用的本初子午线东偏10°。大多数中世纪的穆斯林地理学家继续使用阿尔·花拉子米的本初子午线。[80]
\subsubsection{《犹太历法》}  
阿尔·花拉兹米还写了几部其他作品,其中包括一部关于希伯来历法的论著,题为《Risāla fi istikhrāj ta'rīkh al-yahūd》(阿拉伯语:رسالة في إستخراج تأريخ اليهود,“提取犹太纪元”)。该书描述了梅托尼周期,即19年的插入周期;确定犹太历法新年第一天(提什雷月的第一天)所在星期几的规则;计算犹太纪年(Anno Mundi)与塞琉古纪年之间的间隔;并给出使用希伯来历法确定太阳和月亮的平均经度的规则。类似的内容也出现在阿尔·比鲁尼和迈蒙尼德的著作中。[37]