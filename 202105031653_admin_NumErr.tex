% 数值计算的误差

\subsection{误差}
误差(Errors)的基本概念在高中物理里面应该有所涉及,这里就不仔细展开了. 对于科学计算而言,我们所关注的主要是\textbf{相对误差(relative error)}和\textbf{绝对误差(absolute error)}.

如果我们把一个数据的实际值记做 $x$ , 把它的近似值记为 $\hat x$. 在这个近似的过程,误差产生的原因主要有:
\begin{itemize}
\item 测量中的误差
\item 运算和计算机表达中的不精确所引入的机器误差或者舍入误差(round-off error)
\item 数值方法和离散化带来的误差(discretization error)
\end{itemize}
尤其是后面两种,是科学计算中要面临的两个主要的误差来源,下面我们会对它们进行逐一分析. 因此,在评估和使用数值方法时,我们需要系统的衡量这两项误差,并且要\textbf{掌握它们的来源},对它们的\textbf{大小有足够的控制}.

\subsection{范数}
在了解误差以前,我们先来回顾一下线性代数中的一个常规概念,范数.

科学计算中经常会涉及到向量(vector),矩阵(matrix)和张量(tensor). 计算与它们有关的误差,需要使用更为一般化的“绝对值”函数,也就是范数 $\norm{\cdot}$.

下面是几个常见的范数,其中向量由小写字母表示,矩阵由大写字母表示.
\begin{itemize}
\item 1-范数:  $\|v\|_1=(|v_1|+|v_2|+\cdots+|v_N|)$
\item 2-范数:  $\|v\|_2=(v_1^2+v_2^2+\cdots+v_N^2)^{\frac{1}{2}}$
\item p-范数:  $\|v\|_p=(v_1^p+v_2^p+\cdots+v_N^p)^{\frac{1}{p}},\quad p&gt;0$
\item $\infty$  -范数,  $\|v\|_{\infty}=\max_{1\le i \le N}|v_i|$
\item 矩阵的p-范数:  $\|A\|p=\max_{\|u\|_p=1}\|Au\|_p$
\item Frobenius-范数:  $\|A\|_F=(\sum_{i=1}^m\sum_{j=1}^n|a_{ij}|^2)^{\frac{1}{2}}$
\end{itemize}

这些范数可以用 Matlab 的 \verb|norm()| 函数或者用 Python 中的 \href{https%3A//docs.scipy.org/doc/numpy/reference/generated/numpy.linalg.norm.html}{numpy.linalg.norm} 函数进行计算. 那么绝对误差的用范数表达式为  $\epsilon=\|x-\hat{x}\|$  ,相对误差为  $\tau=\frac{\|x-\hat{x}\|}{\|x\|}$  .

例子1:  \verb|absrelerror(corr, approx)| 函数使用默认的2-范数,计算 \verb|corr| 和 \verb|approx| 之间的绝对和相对误差.




舍入误差(Round-off error)

舍入误差,有时也称为机器精度(machine epsilon),简单理解就是<b>由于用计算机有限的内存来表达实数轴上面无限多的数,而引入的近似误差.

换一个角度来看,这个误差应该不超过计算机中可以表达的相邻两个实数之间的间隔. 通过一些巧妙的设计,例如

这个间隔的绝对值可以随计算机所要表达的值的大小变化,并始终保持它的相对值在一个较小的范围. 关于浮点数标准,我们会在后面的一章 “<i>计算机算术”</i>里面具体解释和分析.

例子2:机器精度

使用之前定义的 \verb|absrelerror()| 函数,在 Python3 环境下,运行下面的程序.</p><div class="highlight"><pre><code class="language-python"><span class="n">a</span> <span class="o">=</span> <span class="mi">4</span><span class="o">/</span><span class="mi">3</span>
<span class="n">b</span> <span class="o">=</span> <span class="n">a</span><span class="o">-</span><span class="mi">1</span>
<span class="n">c</span> <span class="o">=</span> <span class="n">b</span> <span class="o">+</span> <span class="n">b</span> <span class="o">+</span><span class="n">b</span>
<span class="k">print</span><span class="p">(</span><span class="n">f</span><span class="s1">&#39;c = {c}&#39;</span><span class="p">)</span>
<span class="n">absrelerror</span><span class="p">(</span><span class="n">c</span><span class="p">,</span> <span class="mi">1</span><span class="p">)</span>| </pre></div><p>输出:</p><div class="highlight"><pre><code class="language-pycon">c = 0.9999999999999998
*----------------------------------------------------------*
This program illustrates the absolute and relative error.
*----------------------------------------------------------*
Absolute error: 2.220446049250313e-16
Relative error: 2.2204460492503136e-16| </pre></div><h3>例子3: 机器精度2</h3><p>运行下面的四行代码,观察输出的结果是否存在误差.</p><div class="highlight"><pre><code class="language-python"><span class="k">print</span><span class="p">(</span><span class="mf">1e-20</span><span class="o">+</span><span class="mf">1e-34</span><span class="p">)</span>
<span class="k">print</span><span class="p">(</span><span class="mf">1e-20</span><span class="o">+</span><span class="mf">1e-35</span><span class="p">)</span>
<span class="k">print</span><span class="p">(</span><span class="mf">1e-20</span><span class="o">+</span><span class="mf">1e-36</span><span class="p">)</span>
<span class="k">print</span><span class="p">(</span><span class="mf">1e-20</span><span class="o">+</span><span class="mf">1e-37</span><span class="p">)</span>| </pre></div><p>输出:</p><div class="highlight"><pre><code class="language-pycon">1.0000000000000099e-20
1.000000000000001e-20
1.0000000000000001e-20
1e-20| </pre></div><p>由上面两个例子可以观察到,机器精度约为  $10^{-16}$  . 值得注意的是,在例子2中,如果我们直接令 \verb|b=1/3| ,再求和,则不会得到任何误差. 事实上,这个舍入误差出现在 \verb|b=a-1| 时,并且延续到了后面的求和计算. 而对于例子3,求和时当两个数的相对大小差距超过  $10^{-16}$  (或  $10^{16}$  )时,较小的数的贡献会消失.这种现象在有些地方也被称作cancellation error. 事实上,它只是舍入误差的一种表现. 至于具体的原理,也会在后面计算机算术这一章和浮点数标准中详细介绍.</p><h3>例子4: 矩阵运算中的误差</h3><p>下面定义一个测试函数 \verb|testErrA(n)| ,随机生成  $n\times n$  矩阵  $A$  ,然后计算  $A^{-1}A$  与单位矩阵  $I$  之间的误差.

理论上,这两个值应该完全相等,但是由于存在舍入误差, $A^{-1}A$ 并不完全等于单位矩阵.</p><div class="highlight"><pre><code class="language-python"><span class="c1"># Generate a random nxn matrix and compute A^{-1}*A which should be I analytically</span>
<span class="k">def</span> <span class="nf">testErrA</span><span class="p">(</span><span class="n">n</span> <span class="o">=</span> <span class="mi">10</span><span class="p">):</span>
    <span class="n">A</span> <span class="o">=</span> <span class="n">np</span><span class="o">.</span><span class="n">random</span><span class="o">.</span><span class="n">rand</span><span class="p">(</span><span class="n">n</span><span class="p">,</span><span class="n">n</span><span class="p">)</span>
    <span class="n">Icomp</span> <span class="o">=</span> <span class="n">np</span><span class="o">.</span><span class="n">matmul</span><span class="p">(</span><span class="n">np</span><span class="o">.</span><span class="n">linalg</span><span class="o">.</span><span class="n">inv</span><span class="p">(</span><span class="n">A</span><span class="p">),</span> <span class="n">A</span><span class="p">)</span>
    <span class="n">Iexact</span> <span class="o">=</span> <span class="n">np</span><span class="o">.</span><span class="n">eye</span><span class="p">(</span><span class="n">n</span><span class="p">)</span>
    <span class="n">absrelerror</span><span class="p">(</span><span class="n">Iexact</span><span class="p">,</span> <span class="n">Icomp</span><span class="p">)</span>| </pre></div><p>调用函数可得到类似如下的输出:</p><div class="highlight"><pre><code class="language-pycon">testErrA()
*----------------------------------------------------------*
This program illustrates the absolute and relative error.
*----------------------------------------------------------*
Absolute error: 6.626588434229317e-15
Relative error: 2.095511256869353e-15| </pre></div><p>由于矩阵  $A$  的随机性,这里的输出并不会完全一致,但相对误差始终保持在 $10^{-14}$  至  $10^{-16}$  这个范围. 我们也可以尝试不同大小的矩阵,误差也会随着矩阵尺寸变大而增大.

<b>注1</b>:对于矩阵求逆运算的机器精度估算,涉及到矩阵的条件数(condition number),这个会在后面求解线性方程时具体分析.

<b>注2</b>:矩阵求逆的运算复杂度约为  $\mathcal{O}(n^2)$  或以上,因此继续增大  $n$  有可能会让程序的运行时间大大增加.</p><h2>离散化误差(Discretization error)</h2><p>顾名思义,是由数值方法的离散化所引入的误差.通常情况下,离散化误差的大小与离散化尺寸直接相关. 以前向差分(forward difference)为例,函数  $f(x)$  的一阶导数可以近似为:

 $ f&#39;(x)\approx \frac{f(x+h)-f(x)}{h},\\$  

其中,  $h$  为网格尺寸或步长. 通过泰勒展开(Taylor expansion)可知,前向差分的离散化误差为  $\mathcal{O}(h)$  ,即每当  $h$  缩小到它的一半,则误差也相应的缩小一半.

类似的,对于中央差分(central difference)和五点差分(five-points difference)

 $f&#39;(x)\approx \frac{f(x+h)-f(x-h)}{2h},\\$  

 $f&#39;(x)\approx \frac{-f(x+2h)+8f(x+h)-8f(x-h)+f(x-2h)}{12h}.\\ $ 

它们的离散化误差分别为 $\mathcal{O}(h^2)$ 和  $\mathcal{O}(h^4)$  .对应的,当  $h$  缩小一半,误差分别变为原来的1/4和1/16.</p><h3>例子5: 前向差分的离散化误差</h3><p>取  $f(x)=e^x$  ,可知  $f&#39;(x)=e^x$  .我们用前向差分来计算  $f(x)$  的一阶导数,把它们的值和真实值对比,并分别画出  $h=0.2, 0.1$  和  $0.05$  的结果.</p><div class="highlight"><pre><code class="language-python"><span class="kn">import</span> <span class="nn">matplotlib.pyplot</span> <span class="kn">as</span> <span class="nn">plt</span>
<span class="kn">import</span> <span class="nn">numpy</span> <span class="kn">as</span> <span class="nn">np</span>

<span class="k">def</span> <span class="nf">ForwardDiff</span><span class="p">(</span><span class="n">fx</span><span class="p">,</span> <span class="n">x</span><span class="p">,</span> <span class="n">h</span><span class="o">=</span><span class="mf">0.001</span><span class="p">):</span>
    <span class="c1"># ForwardDiff(@fx, x, h);</span>
    <span class="k">return</span> <span class="p">(</span><span class="n">fx</span><span class="p">(</span><span class="n">x</span><span class="o">+</span><span class="n">h</span><span class="p">)</span> <span class="o">-</span> <span class="n">fx</span><span class="p">(</span><span class="n">x</span><span class="p">))</span><span class="o">/</span><span class="n">h</span>

<span class="k">def</span> <span class="nf">testDiscretization</span><span class="p">(</span><span class="n">h</span><span class="o">=</span><span class="mf">0.1</span><span class="p">,</span> <span class="n">l</span><span class="o">=</span><span class="mi">0</span><span class="p">,</span> <span class="n">u</span><span class="o">=</span><span class="mi">2</span><span class="p">):</span>
    <span class="c1"># compute the numerical derivatives</span>
    <span class="n">xh</span> <span class="o">=</span> <span class="n">np</span><span class="o">.</span><span class="n">linspace</span><span class="p">(</span><span class="n">l</span><span class="p">,</span> <span class="n">u</span><span class="p">,</span> <span class="nb">int</span><span class="p">(</span><span class="nb">abs</span><span class="p">(</span><span class="n">u</span><span class="o">-</span><span class="n">l</span><span class="p">)</span><span class="o">/</span><span class="n">h</span><span class="p">))</span>
    <span class="n">fprimF</span> <span class="o">=</span> <span class="n">ForwardDiff</span><span class="p">(</span><span class="n">np</span><span class="o">.</span><span class="n">exp</span><span class="p">,</span> <span class="n">xh</span><span class="p">,</span> <span class="n">h</span><span class="p">)</span>
    <span class="k">return</span> <span class="n">xh</span><span class="p">,</span> <span class="n">fprimF</span>

<span class="c1"># The exact solution</span>
<span class="n">Nx</span> <span class="o">=</span> <span class="mi">400</span>
<span class="n">l</span><span class="p">,</span> <span class="n">u</span> <span class="o">=</span> <span class="mi">0</span><span class="p">,</span> <span class="mi">2</span>
<span class="n">x</span> <span class="o">=</span> <span class="n">np</span><span class="o">.</span><span class="n">linspace</span><span class="p">(</span><span class="n">l</span><span class="p">,</span> <span class="n">u</span><span class="p">,</span> <span class="n">Nx</span><span class="p">)</span>
<span class="n">f_exa</span> <span class="o">=</span> <span class="n">np</span><span class="o">.</span><span class="n">exp</span><span class="p">(</span><span class="n">x</span><span class="p">)</span>

<span class="c1"># Plot</span>
<span class="n">fig</span><span class="p">,</span> <span class="n">axs</span> <span class="o">=</span> <span class="n">plt</span><span class="o">.</span><span class="n">subplots</span><span class="p">(</span><span class="mi">3</span><span class="p">,</span> <span class="mi">1</span><span class="p">,</span> <span class="n">figsize</span><span class="o">=</span><span class="p">(</span><span class="mi">6</span><span class="p">,</span><span class="mi">6</span><span class="p">))</span>
<span class="n">fig</span><span class="o">.</span><span class="n">tight_layout</span><span class="p">(</span><span class="n">pad</span><span class="o">=</span><span class="mi">0</span><span class="p">,</span> <span class="n">w_pad</span><span class="o">=</span><span class="mi">0</span><span class="p">,</span> <span class="n">h_pad</span><span class="o">=</span><span class="mi">2</span><span class="p">)</span>

<span class="k">for</span> <span class="n">i</span><span class="p">,</span> <span class="n">ax</span> <span class="ow">in</span> <span class="nb">zip</span><span class="p">(</span><span class="nb">range</span><span class="p">(</span><span class="mi">4</span><span class="p">),</span> <span class="n">axs</span><span class="p">):</span>
    <span class="n">ax</span><span class="o">.</span><span class="n">plot</span><span class="p">(</span><span class="n">x</span><span class="p">,</span> <span class="n">f_exa</span><span class="p">,</span> <span class="n">color</span><span class="o">=</span><span class="s1">&#39;blue&#39;</span><span class="p">)</span>
    <span class="n">xh</span><span class="p">,</span> <span class="n">fprimF</span> <span class="o">=</span> <span class="n">testDiscretization</span><span class="p">(</span><span class="n">h</span><span class="o">=</span><span class="mf">0.2</span><span class="o">*</span><span class="mf">0.5</span><span class="o">**</span><span class="n">i</span><span class="p">)</span>
    <span class="n">ax</span><span class="o">.</span><span class="n">plot</span><span class="p">(</span><span class="n">xh</span><span class="p">,</span> <span class="n">fprimF</span><span class="p">,</span> <span class="s1">&#39;ro&#39;</span><span class="p">,</span> <span class="n">clip_on</span><span class="o">=</span><span class="bp">False</span><span class="p">)</span>
    <span class="n">ax</span><span class="o">.</span><span class="n">set_xlim</span><span class="p">([</span><span class="mi">0</span><span class="p">,</span> <span class="mi">2</span><span class="p">])</span>
    <span class="n">ax</span><span class="o">.</span><span class="n">set_ylim</span><span class="p">([</span><span class="mi">1</span><span class="p">,</span><span class="nb">max</span><span class="p">(</span><span class="n">fprimF</span><span class="p">)])</span>
    <span class="n">ax</span><span class="o">.</span><span class="n">set_xlabel</span><span class="p">(</span><span class="sa">r</span><span class="s1">&#39;$x$&#39;</span><span class="p">)</span>
    <span class="n">ax</span><span class="o">.</span><span class="n">set_ylabel</span><span class="p">(</span><span class="s1">&#39;Derivatives&#39;</span><span class="p">)</span>
    <span class="n">ax</span><span class="o">.</span><span class="n">legend</span><span class="p">([</span><span class="s1">&#39;Exact Derivatives&#39;</span><span class="p">,</span><span class="s1">&#39;Calculated Derivatives&#39;</span><span class="p">])</span>| </pre></div><figure data-size="normal"><noscript> $h$  的减小,逐渐靠近真实值.</p><h2>小结:舍入误差 vs. 离散化误差</h2><p>我们通过下面这个例子来分析和对比这两个误差.</p><h3>例子6: 对比两个误差</h3><p>继续例子4中的函数  $f(x)=e^x$  和它的导数  $f&#39;(x)=e^x$  .取  $x=1$  ,分别用前面提到的三种有限微分方法求出数值导数,并与真实值比较计算出相对误差.

这里我们来观察,当  $h$  取不同值( $10^{-1}$  至  $10^{-15}$ )的时候,相对误差的变化情况.</p><div class="highlight"><pre><code class="language-python"><span class="kn">import</span> <span class="nn">matplotlib.pyplot</span> <span class="kn">as</span> <span class="nn">plt</span>
<span class="kn">import</span> <span class="nn">numpy</span> <span class="kn">as</span> <span class="nn">np</span>

<span class="k">def</span> <span class="nf">ForwardDiff</span><span class="p">(</span><span class="n">fx</span><span class="p">,</span> <span class="n">x</span><span class="p">,</span> <span class="n">h</span><span class="o">=</span><span class="mf">0.001</span><span class="p">):</span>
    <span class="c1"># Forward difference</span>
    <span class="k">return</span> <span class="p">(</span><span class="n">fx</span><span class="p">(</span><span class="n">x</span><span class="o">+</span><span class="n">h</span><span class="p">)</span> <span class="o">-</span> <span class="n">fx</span><span class="p">(</span><span class="n">x</span><span class="p">))</span><span class="o">/</span><span class="n">h</span>

<span class="k">def</span> <span class="nf">CentralDiff</span><span class="p">(</span><span class="n">fx</span><span class="p">,</span> <span class="n">x</span><span class="p">,</span> <span class="n">h</span><span class="o">=</span><span class="mf">0.001</span><span class="p">):</span>
    <span class="c1"># Central difference</span>
    <span class="k">return</span> <span class="p">(</span><span class="n">fx</span><span class="p">(</span><span class="n">x</span><span class="o">+</span><span class="n">h</span><span class="p">)</span> <span class="o">-</span> <span class="n">fx</span><span class="p">(</span><span class="n">x</span><span class="o">-</span><span class="n">h</span><span class="p">))</span><span class="o">/</span><span class="n">h</span><span class="o">*</span><span class="mf">0.5</span>

<span class="k">def</span> <span class="nf">FivePointsDiff</span><span class="p">(</span><span class="n">fx</span><span class="p">,</span> <span class="n">x</span><span class="p">,</span> <span class="n">h</span><span class="o">=</span><span class="mf">0.001</span><span class="p">):</span>
    <span class="c1"># Five points difference </span>
    <span class="k">return</span> <span class="p">(</span><span class="o">-</span><span class="n">fx</span><span class="p">(</span><span class="n">x</span><span class="o">+</span><span class="mi">2</span><span class="o">*</span><span class="n">h</span><span class="p">)</span> <span class="o">+</span> <span class="mi">8</span><span class="o">*</span><span class="n">fx</span><span class="p">(</span><span class="n">x</span><span class="o">+</span><span class="n">h</span><span class="p">)</span> <span class="o">-</span> <span class="mi">8</span><span class="o">*</span><span class="n">fx</span><span class="p">(</span><span class="n">x</span><span class="o">-</span><span class="n">h</span><span class="p">)</span> <span class="o">+</span> <span class="n">fx</span><span class="p">(</span><span class="n">x</span><span class="o">-</span><span class="mi">2</span><span class="o">*</span><span class="n">h</span><span class="p">))</span> <span class="o">/</span> <span class="p">(</span><span class="mf">12.0</span><span class="o">*</span><span class="n">h</span><span class="p">)</span>

<span class="c1"># choose h from 0.1 to 10^-t, t&gt;=2</span>
<span class="n">t</span> <span class="o">=</span> <span class="mi">15</span>
<span class="n">hx</span> <span class="o">=</span> <span class="mi">10</span><span class="o">**</span><span class="n">np</span><span class="o">.</span><span class="n">linspace</span><span class="p">(</span><span class="o">-</span><span class="mi">1</span><span class="p">,</span><span class="o">-</span><span class="n">t</span><span class="p">,</span> <span class="mi">30</span><span class="p">)</span>

<span class="c1"># The exact derivative at x=1</span>
<span class="n">x0</span> <span class="o">=</span> <span class="mi">1</span>
<span class="n">fprimExact</span> <span class="o">=</span> <span class="n">np</span><span class="o">.</span><span class="n">exp</span><span class="p">(</span><span class="mi">1</span><span class="p">)</span>

<span class="c1"># Numerical derivative using the three methods</span>
<span class="n">fprimF</span> <span class="o">=</span> <span class="n">ForwardDiff</span><span class="p">(</span><span class="n">np</span><span class="o">.</span><span class="n">exp</span><span class="p">,</span> <span class="n">x0</span><span class="p">,</span> <span class="n">hx</span><span class="p">)</span>
<span class="n">fprimC</span> <span class="o">=</span> <span class="n">CentralDiff</span><span class="p">(</span><span class="n">np</span><span class="o">.</span><span class="n">exp</span><span class="p">,</span> <span class="n">x0</span><span class="p">,</span> <span class="n">hx</span><span class="p">)</span>
<span class="n">fprim5</span> <span class="o">=</span> <span class="n">FivePointsDiff</span><span class="p">(</span><span class="n">np</span><span class="o">.</span><span class="n">exp</span><span class="p">,</span> <span class="n">x0</span><span class="p">,</span> <span class="n">hx</span><span class="p">)</span>

<span class="c1"># Relative error</span>
<span class="n">felF</span> <span class="o">=</span> <span class="nb">abs</span><span class="p">(</span><span class="n">fprimExact</span> <span class="o">-</span> <span class="n">fprimF</span><span class="p">)</span><span class="o">/</span><span class="nb">abs</span><span class="p">(</span><span class="n">fprimExact</span><span class="p">)</span>
<span class="n">felC</span> <span class="o">=</span> <span class="nb">abs</span><span class="p">(</span><span class="n">fprimExact</span> <span class="o">-</span> <span class="n">fprimC</span><span class="p">)</span><span class="o">/</span><span class="nb">abs</span><span class="p">(</span><span class="n">fprimExact</span><span class="p">)</span>
<span class="n">fel5</span> <span class="o">=</span> <span class="nb">abs</span><span class="p">(</span><span class="n">fprimExact</span> <span class="o">-</span> <span class="n">fprim5</span><span class="p">)</span><span class="o">/</span><span class="nb">abs</span><span class="p">(</span><span class="n">fprimExact</span><span class="p">)</span>

<span class="c1"># Plot</span>
<span class="n">fig</span><span class="p">,</span> <span class="n">ax</span> <span class="o">=</span> <span class="n">plt</span><span class="o">.</span><span class="n">subplots</span><span class="p">(</span><span class="mi">1</span><span class="p">)</span>
<span class="n">ax</span><span class="o">.</span><span class="n">loglog</span><span class="p">(</span><span class="n">hx</span><span class="p">,</span> <span class="n">felF</span><span class="p">)</span>
<span class="n">ax</span><span class="o">.</span><span class="n">loglog</span><span class="p">(</span><span class="n">hx</span><span class="p">,</span> <span class="n">felC</span><span class="p">)</span>
<span class="n">ax</span><span class="o">.</span><span class="n">loglog</span><span class="p">(</span><span class="n">hx</span><span class="p">,</span> <span class="n">fel5</span><span class="p">)</span>
<span class="n">ax</span><span class="o">.</span><span class="n">autoscale</span><span class="p">(</span><span class="n">enable</span><span class="o">=</span><span class="bp">True</span><span class="p">,</span> <span class="n">axis</span><span class="o">=</span><span class="s1">&#39;x&#39;</span><span class="p">,</span> <span class="n">tight</span><span class="o">=</span><span class="bp">True</span><span class="p">)</span>
<span class="n">ax</span><span class="o">.</span><span class="n">set_xlabel</span><span class="p">(</span><span class="sa">r</span><span class="s1">&#39;Step length $h$&#39;</span><span class="p">)</span>
<span class="n">ax</span><span class="o">.</span><span class="n">set_ylabel</span><span class="p">(</span><span class="s1">&#39;Relative error&#39;</span><span class="p">)</span>
<span class="n">ax</span><span class="o">.</span><span class="n">legend</span><span class="p">([</span><span class="s1">&#39;Forward difference&#39;</span><span class="p">,</span><span class="s1">&#39;Central difference&#39;</span><span class="p">,</span> <span class="s1">&#39;Five points difference&#39;</span><span class="p">])</span>
<span class="o">&lt;</span><span class="n">matplotlib</span><span class="o">.</span><span class="n">legend</span><span class="o">.</span><span class="n">Legend</span> <span class="n">at</span> <span class="mh">0x12066aad0</span><span class="o">&gt;</span>| </pre></div><figure data-size="normal"><noscript> $h$  的不同,呈现出两种不同的特性.  </li><li>当  $h$  较大时(右半侧),误差曲线相对规则,是由离散化误差主导的. </li><li>当  $h$  较小时(左半侧),误差曲线波动较大,是由舍入误差主导.</li></ul><p><b>注</b>:舍入误差主导且随着  $h$  减小而增大的主要原因是: </p><ul><li>当  $h$  较小时,有限差分法的分子上相近的数相减会造成类似例子2和3中的舍入误差.  </li><li>由于这个例子中的  $f(x)$  在  $e$  附近,因此这个误差应为  $10^{-16}$  左右.它除以较小的  $h$  时,就会被相应的放大,当  $h$  越小,这个舍入误差越大.</li></ul><p>因此,<b>当我们使用上述方法时,需要注意,尽可能取</b>  $h$  <b>在误差曲线的右半侧,这样我们对于误差才有完全的控制.</b>

同时,我们也观察到,对于越是高阶的差分(如五点差分),它的离散化误差随  $h$  的下降速率越大,但也越早到达舍入误差的区域. 因此,当我们遇到类似问题上,应<b>选择合适阶数的有限差分方法,并根据它的特性选择适合的</b>  $h$  <b>值.并不一定是越高阶的方法越好,</b>  $h$  <b>越小越好.</b></p><p class="ztext-empty-paragraph"><br/>

本文中所用函数的完整代码见github: </p><a target="_blank" href="https://link.zhihu.com/?target=https%3A//github.com/enigne/ScientificComputingBridging/blob/master/Lab/L2/measureErrors.py" data-draft-node="block" data-draft-type="link-card" class="LinkCard old LinkCard--noImage"><span class="LinkCard-content"><span class="LinkCard-text"><span class="LinkCard-title" data-text="true">measureErrors.py</span><span class="LinkCard-meta"><span style="display:inline-flex;align-items:center">​<svg class="Zi Zi--InsertLink" fill="currentColor" viewBox="0 0 24 24" width="17" height="17"><path d="M13.414 4.222a4.5 4.5 0 1 1 6.364 6.364l-3.005 3.005a.5.5 0 0 1-.707 0l-.707-.707a.5.5 0 0 1 0-.707l3.005-3.005a2.5 2.5 0 1 0-3.536-3.536l-3.005 3.005a.5.5 0 0 1-.707 0l-.707-.707a.5.5 0 0 1 0-.707l3.005-3.005zm-6.187 6.187a.5.5 0 0 1 .638-.058l.07.058.706.707a.5.5 0 0 1 .058.638l-.058.07-3.005 3.004a2.5 2.5 0 0 0 3.405 3.658l.13-.122 3.006-3.005a.5.5 0 0 1 .638-.058l.069.058.707.707a.5.5 0 0 1 .058.638l-.058.069-3.005 3.005a4.5 4.5 0 0 1-6.524-6.196l.16-.168 3.005-3.005zm8.132-3.182a.25.25 0 0 1 .353 0l1.061 1.06a.25.25 0 0 1 0 .354l-8.132 8.132a.25.25 0 0 1-.353 0l-1.061-1.06a.25.25 0 0 1 0-.354l8.132-8.132z"></path></svg></span>github.com</span></span><span class="LinkCard-imageCell"><div class="LinkCard-image LinkCard-image--default"><svg class="Zi Zi--Browser" fill="currentColor" viewBox="0 0 24 24" width="32" height="32"><path d="M11.991 3C7.023 3 3 7.032 3 12s4.023 9 8.991 9C16.968 21 21 16.968 21 12s-4.032-9-9.009-9zm6.237 5.4h-2.655a14.084 14.084 0 0 0-1.242-3.204A7.227 7.227 0 0 1 18.228 8.4zM12 4.836A12.678 12.678 0 0 1 13.719 8.4h-3.438A12.678 12.678 0 0 1 12 4.836zM5.034 13.8A7.418 7.418 0 0 1 4.8 12c0-.621.09-1.224.234-1.8h3.042A14.864 14.864 0 0 0 7.95 12c0 .612.054 1.206.126 1.8H5.034zm.738 1.8h2.655a14.084 14.084 0 0 0 1.242 3.204A7.188 7.188 0 0 1 5.772 15.6zm2.655-7.2H5.772a7.188 7.188 0 0 1 3.897-3.204c-.54.999-.954 2.079-1.242 3.204zM12 19.164a12.678 12.678 0 0 1-1.719-3.564h3.438A12.678 12.678 0 0 1 12 19.164zm2.106-5.364H9.894A13.242 13.242 0 0 1 9.75 12c0-.612.063-1.215.144-1.8h4.212c.081.585.144 1.188.144 1.8 0 .612-.063 1.206-.144 1.8zm.225 5.004c.54-.999.954-2.079 1.242-3.204h2.655a7.227 7.227 0 0 1-3.897 3.204zm1.593-5.004c.072-.594.126-1.188.126-1.8 0-.612-.054-1.206-.126-1.8h3.042c.144.576.234 1.179.234 1.8s-.09 1.224-.234 1.8h-3.042z"></path></svg></div></span></span></a><p></p></div></div><div class="ContentItem-time">发布于 2020-03-28</div><div class="Reward"><div><div class="Reward-tagline">真诚赞赏,手留余香</div><button class="Reward-rewardBtn">赞赏</button></div><div class="Reward-countZero">还没有人赞赏,快来当第一个赞赏的人吧!</div></div><div class="Post-topicsAndReviewer"><div class="TopicList Post-Topics"><div class="Tag Topic"><span class="Tag-content"><a class="TopicLink" href="//www.zhihu.com/topic/19608617" target="_blank"><div class="Popover"><div id="null-toggle" aria-haspopup="true" aria-expanded="false" aria-owns="null-content">科学计算</div></div></a></span></div><div class="Tag Topic"><span class="Tag-content"><a class="TopicLink" href="//www.zhihu.com/topic/19793027" target="_blank"><div class="Popover"><div id="null-toggle" aria-haspopup="true" aria-expanded="false" aria-owns="null-content">误差</div></div></a></span></div><div class="Tag Topic"><span class="Tag-content"><a class="TopicLink" href="//www.zhihu.com/topic/19662420" target="_blank"><div class="Popover"><div id="null-toggle" aria-haspopup="true" aria-expanded="false" aria-owns="null-content">计算数学</div></div></a></span></div></div></div><div><div class="Sticky RichContent-actions is-bottom"><div class="ContentItem-actions"><span><button aria-label="赞同 7 " type="button" class="Button VoteButton VoteButton--up"><span style="display:inline-flex;align-items:center">​<svg class="Zi Zi--TriangleUp VoteButton-TriangleUp" fill="currentColor" viewBox="0 0 24 24" width="10" height="10"><path d="M2 18.242c0-.326.088-.532.237-.896l7.98-13.203C10.572 3.57 11.086 3 12 3c.915 0 1.429.571 1.784 1.143l7.98 13.203c.15.364.236.57.236.896 0 1.386-.875 1.9-1.955 1.9H3.955c-1.08 0-1.955-.517-1.955-1.9z" fill-rule="evenodd"></path></svg></span>赞同 7</button><button aria-label="反对" type="button" class="Button VoteButton VoteButton--down"><span style="display:inline-flex;align-items:center">​<svg class="Zi Zi--TriangleDown" fill="currentColor" viewBox="0 0 24 24" width="10" height="10"><path d="M20.044 3H3.956C2.876 3 2 3.517 2 4.9c0 .326.087.533.236.896L10.216 19c.355.571.87 1.143 1.784 1.143s1.429-.572 1.784-1.143l7.98-13.204c.149-.363.236-.57.236-.896 0-1.386-.876-1.9-1.956-1.9z" fill-rule="evenodd"></path></svg></span></button></span><style data-emotion-css="qbubgm">.css-qbubgm{margin-left:0;}</style><div class="css-qbubgm"><button type="button" class="Button BottomActions-CommentBtn Button--plain Button--withIcon Button--withLabel"><span style="display:inline-flex;align-items:center">​<svg class="Zi Zi--Comment Button-zi" fill="currentColor" viewBox="0 0 24 24" width="1.2em" height="1.2em"><path d="M10.241 19.313a.97.97 0 0 0-.77.2 7.908 7.908 0 0 1-3.772 1.482.409.409 0 0 1-.38-.637 5.825 5.825 0 0 0 1.11-2.237.605.605 0 0 0-.227-.59A7.935 7.935 0 0 1 3 11.25C3 6.7 7.03 3 12 3s9 3.7 9 8.25-4.373 9.108-10.759 8.063z" fill-rule="evenodd"></path></svg></span>1 条评论</button></div><div class="Popover ShareMenu"><div class="ShareMenu-toggler" id="null-toggle" aria-haspopup="true" aria-expanded="false" aria-owns="null-content"><button type="button" class="Button Button--plain Button--withIcon Button--withLabel"><span style="display:inline-flex;align-items:center">​<svg class="Zi Zi--Share Button-zi" fill="currentColor" viewBox="0 0 24 24" width="1.2em" height="1.2em"><path d="M2.931 7.89c-1.067.24-1.275 1.669-.318 2.207l5.277 2.908 8.168-4.776c.25-.127.477.198.273.39L9.05 14.66l.927 5.953c.18 1.084 1.593 1.376 2.182.456l9.644-15.242c.584-.892-.212-2.029-1.234-1.796L2.93 7.89z" fill-rule="evenodd"></path></svg></span>分享</button></div></div><button type="button" class="Button ContentItem-action Button--plain Button--withIcon Button--withLabel"><span style="display:inline-flex;align-items:center">​<svg class="Zi Zi--Heart Button-zi" fill="currentColor" viewBox="0 0 24 24" width="1.2em" height="1.2em"><path d="M2 8.437C2 5.505 4.294 3.094 7.207 3 9.243 3 11.092 4.19 12 6c.823-1.758 2.649-3 4.651-3C19.545 3 22 5.507 22 8.432 22 16.24 13.842 21 12 21 10.158 21 2 16.24 2 8.437z" fill-rule="evenodd"></path></svg></span>喜欢</button><button type="button" class="Button ContentItem-action Button--plain Button--withIcon Button--withLabel"><span style="display:inline-flex;align-items:center">​<svg class="Zi Zi--Star Button-zi" fill="currentColor" viewBox="0 0 24 24" width="1.2em" height="1.2em"><path d="M5.515 19.64l.918-5.355-3.89-3.792c-.926-.902-.639-1.784.64-1.97L8.56 7.74l2.404-4.871c.572-1.16 1.5-1.16 2.072 0L15.44 7.74l5.377.782c1.28.186 1.566 1.068.64 1.97l-3.89 3.793.918 5.354c.219 1.274-.532 1.82-1.676 1.218L12 18.33l-4.808 2.528c-1.145.602-1.896.056-1.677-1.218z" fill-rule="evenodd"></path></svg></span>收藏</button><button type="button" class="Button ContentItem-action Button--plain Button--withIcon Button--withLabel"><span style="display:inline-flex;align-items:center">​<svg class="Zi Zi--Deliver Button-zi" fill="currentColor" viewBox="0 0 24 24" width="1.2em" height="1.2em"><path d="M5.171 4H18.83a1.5 1.5 0 0 1 1.455 1.136l2.597 10.386a4 4 0 0 1 .119.97V19s0 2-2.002 2H3c-2 0-2-2-2-2v-2.508a4 4 0 0 1 .12-.97L3.715 5.136A1.5 1.5 0 0 1 5.171 4zm1.074 2a1 1 0 0 0-.97.761l-2.123 8.62a.5.5 0 0 0 .486.619h4.717a1 1 0 0 1 .892.548C9.906 17.85 10.824 18.5 12 18.5c1.176 0 2.094-.65 2.753-1.952a1 1 0 0 1 .892-.548h4.717a.5.5 0 0 0 .486-.62l-2.122-8.619A1 1 0 0 0 17.755 6H6.245zM8 9c0-.552.453-1 .997-1h6.006c.55 0 .997.444.997 1 0 .552-.453 1-.997 1H8.997A.996.996 0 0 1 8 9zm-1.5 4c0-.552.445-1 .996-1h9.008c.55 0 .996.444.996 1 0 .552-.445 1-.996 1H7.496a.995.995 0 0 1-.996-1z"></path></svg></span>申请转载</button><div class="Post-ActionMenuButton"><div class="Popover"><div id="null-toggle" aria-haspopup="true" aria-expanded="false" aria-owns="null-content"><button type="button" class="Button Button--plain Button--withIcon Button--iconOnly"><span style="display:inline-flex;align-items:center">​<svg class="Zi Zi--Dots Button-zi" fill="currentColor" viewBox="0 0 24 24" width="1.2em" height="1.2em"><path d="M5 14a2 2 0 1 1 0-4 2 2 0 0 1 0 4zm7 0a2 2 0 1 1 0-4 2 2 0 0 1 0 4zm7 0a2 2 0 1 1 0-4 2 2 0 0 1 0 4z" fill-rule="evenodd"></path></svg></span></button></div></div></div></div></div></div></article><div class="Post-Sub Post-NormalSub"><div class="PostIndex-Contributions"><h3 class="BlockTitle">文章被以下专栏收录</h3><ul><div class="ContentItem Column-ColumnItem"><div class="ContentItem-main"><div class="ContentItem-image"><a class="ColumnLink" href="//www.zhihu.com/column/c_1226443594048942080"><div class="Popover"><div id="null-toggle" aria-haspopup="true" aria-expanded="false" aria-owns="null-content"> $科学计算"/></div></div></a></div><div class="ContentItem-head"><h2 class="ContentItem-title"><a class="ColumnLink ColumnItem-Title" href="//www.zhihu.com/column/c_1226443594048942080"><div class="Popover"><div id="null-toggle" aria-haspopup="true" aria-expanded="false" aria-owns="null-content">科学计算</div></div></a></h2><div class="ContentItem-meta">系统介绍科学计算相关知识</div></div></div></div></ul></div></div></div></main></div></div><script id="js-clientConfig" type="text/json">{"host":"zhihu.com","protocol":"https:","wwwHost":"www.zhihu.com","videoHost":"video.zhihu.com","fetchRoot":{"www":"https:\u002F\u002Fwww.zhihu.com","api":"https:\u002F\u002Fapi.zhihu.com","lens":"https:\u002F\u002Flens.zhihu.com","zhuanlan":"https:\u002F\u002Fzhuanlan.zhihu.com","walletpay":"https:\u002F\u002Fwalletpay.zhihu.com","captcha":"https:\u002F\u002Fcaptcha.zhihu.com"}}</script><script id="js-initialData" type="text/json">{"initialState":{"common":{"ask":{}},"loading":{"global":{"count":0},"local":{"env\u002FgetIpinfo\u002F":false,"article\u002Fget\u002F":false,"brand\u002FgetUrl\u002F":false}},"club":{"tags":{},"admins":{"data":[]},"members":{"data":[]},"explore":{"candidateSyncClubs":{}},"profile":{},"checkin":{},"comments":{"paging":{},"loading":{},"meta":{},"ids":{}},"postList":{"paging":{},"loading":{},"ids":{}},"recommend":{"data":[]},"silences":{"data":[]},"application":{"profile":null}},"entities":{"users":{"c0d9b4bba7b55a4c382c09e4ae43102a":{"uid":40977397972992,"userType":"people","id":"c0d9b4bba7b55a4c382c09e4ae43102a"},"c-g-2-11":{"isFollowed":true,"avatarUrlTemplate":"https:\u002F\u002Fpic3.zhimg.com\u002Fv2-9fd1680b3e9ea906ec038a9448a46d78.jpg?source=172ae18b","uid":"35018097295360","userType":"people","isFollowing":true,"urlToken":"c-g-2-11","id":"77fbc1230a7a1438551696a01b2f0d4b","description":"数学爱好者","name":"Gong Cheng","isAdvertiser":false,"headline":"计算科学博士","gender":1,"url":"\u002Fpeople\u002F77fbc1230a7a1438551696a01b2f0d4b","avatarUrl":"https:\u002F\u002Fpic1.zhimg.com\u002Fv2-9fd1680b3e9ea906ec038a9448a46d78_l.jpg?source=172ae18b","isOrg":false,"type":"people","levelInfo":{"exp":15799,"level":4,"nicknameColor":{"color":"","nightModeColor":""},"levelIcon":"https:\u002F\u002Fpic4.zhimg.com\u002Fv2-8f86c92ac45b818694ebe5d277f11389_l.png","iconInfo":{"url":"https:\u002F\u002Fpic4.zhimg.com\u002Fv2-8f86c92ac45b818694ebe5d277f11389_l.png","nightModeUrl":"https:\u002F\u002Fpic4.zhimg.com\u002Fv2-378e97600d3f445d1896897f7277d13f_l.png","width":93,"height":51}},"badge":[],"badgeV2":{"title":"","mergedBadges":[],"detailBadges":[],"icon":"","nightIcon":""},"exposedMedal":{"medalId":"972477022068568064","medalName":"备受瞩目","avatarUrl":"https:\u002F\u002Fpic4.zhimg.com\u002Fv2-56635eff4121dfc06cf271f46fd700d4_r.png?source=172ae18b","miniAvatarUrl":"https:\u002F\u002Fpic1.zhimg.com\u002Fv2-d1adfc6e446f0712d968ba64cf567370_l.png?source=172ae18b","description":"被 100 个人关注"}}},"questions":{},"answers":{},"articles":{"118757498":{"trackUrl":["https:\u002F\u002Fsugar.zhihu.com\u002Fplutus_adreaper\u002Fcontent_monitor_log?si=__SESSIONID__&ti=__ATOKEN__&at=view&pf=__OS__&ed=BiBUKF0xBSkqGGFTAWl7B1j1E7egfwtRZw==&idfa=__IDFA__&imei=__IMEI__&androidid=__ANDROIDID__&oaid=__OAID__&ci=__CREATIVEID__&zid=__ZONEID__"],"id":118757498,"title":"科学计算基础(1)——误差","type":"article","articleType":"normal","excerptTitle":"","url":"https:\u002F\u002Fzhuanlan.zhihu.com\u002Fp\u002F118757498","imageUrl":"https:\u002F\u002Fpic1.zhimg.com\u002Fv2-8fb1f6237306f2141adb0f9aa9f14ccd_720w.jpg?source=172ae18b","titleImage":"https:\u002F\u002Fpic4.zhimg.com\u002Fv2-8fb1f6237306f2141adb0f9aa9f14ccd_720w.jpg?source=172ae18b","excerpt":"\u003Cimg src=\"https:\u002F\u002Fpic1.zhimg.com\u002Fv2-2561d87c8c771311d55ece66b2beb444_200x112.png\" data-caption=\"\" data-size=\"normal\" data-rawwidth=\"447\" data-rawheight=\"459\" data-watermark=\"watermark\" data-original-src=\"v2-2561d87c8c771311d55ece66b2beb444\" data-watermark-src=\"v2-cc0367b1c9d22e30f4218ec1b969b15a\" data-private-watermark-src=\"\" class=\"origin_image inline-img zh-lightbox-thumb\" data-original=\"https:\u002F\u002Fpic1.zhimg.com\u002Fv2-2561d87c8c771311d55ece66b2beb444_r.png\"\u002F\u003E误差(Errors)误差的基本概念在高中物理里面应该有所涉及,这里就不仔细展开了. 对于科学计算而言,我们所关注的主要是相对误差(relative error)和绝对误差(absolute error). 如果我们把一个数据的实际值记做 x , 把它的近似值极为 \\hat{x} . 在这个近…","created":1585333041,"updated":1585333041,"author":{"isFollowed":true,"avatarUrlTemplate":"https:\u002F\u002Fpic3.zhimg.com\u002Fv2-9fd1680b3e9ea906ec038a9448a46d78.jpg?source=172ae18b","uid":"35018097295360","userType":"people","isFollowing":true,"urlToken":"c-g-2-11","id":"77fbc1230a7a1438551696a01b2f0d4b","description":"数学爱好者","name":"Gong Cheng","isAdvertiser":false,"headline":"计算科学博士","gender":1,"url":"\u002Fpeople\u002F77fbc1230a7a1438551696a01b2f0d4b","avatarUrl":"https:\u002F\u002Fpic1.zhimg.com\u002Fv2-9fd1680b3e9ea906ec038a9448a46d78_l.jpg?source=172ae18b","isOrg":false,"type":"people","levelInfo":{"exp":15799,"level":4,"nicknameColor":{"color":"","nightModeColor":""},"levelIcon":"https:\u002F\u002Fpic4.zhimg.com\u002Fv2-8f86c92ac45b818694ebe5d277f11389_l.png","iconInfo":{"url":"https:\u002F\u002Fpic4.zhimg.com\u002Fv2-8f86c92ac45b818694ebe5d277f11389_l.png","nightModeUrl":"https:\u002F\u002Fpic4.zhimg.com\u002Fv2-378e97600d3f445d1896897f7277d13f_l.png","width":93,"height":51}},"badge":[],"badgeV2":{"title":"","mergedBadges":[],"detailBadges":[],"icon":"","nightIcon":""},"exposedMedal":{"medalId":"972477022068568064","medalName":"备受瞩目","avatarUrl":"https:\u002F\u002Fpic4.zhimg.com\u002Fv2-56635eff4121dfc06cf271f46fd700d4_r.png?source=172ae18b","miniAvatarUrl":"https:\u002F\u002Fpic1.zhimg.com\u002Fv2-d1adfc6e446f0712d968ba64cf567370_l.png?source=172ae18b","description":"被 100 个人关注"}},"commentPermission":"all","copyrightPermission":"need_review","state":"published","imageWidth":650,"imageHeight":435,"content":"\u003Ch2\u003E误差(Errors)\u003C\u002Fh2\u003E\u003Cp\u003E误差的基本概念在高中物理里面应该有所涉及,这里就不仔细展开了. 对于科学计算而言,我们所关注的主要是相对误差(relative error)和绝对误差(absolute error).\u003C\u002Fp\u003E\u003Cp\u003E如果我们把一个数据的实际值记做 \u003Cimg src=\"https:\u002F\u002Fwww.zhihu.com\u002Fequation?tex=x\" alt=\"x\" eeimg=\"1\"\u002F\u003E , 把它的近似值极为 \u003Cimg src=\"https:\u002F\u002Fwww.zhihu.com\u002Fequation?tex=%5Chat%7Bx%7D\" alt=\"\\hat{x}\" eeimg=\"1\"\u002F\u003E . 在这个近似的过程,误差产生的原因主要有:\u003C\u002Fp\u003E\u003Cul\u003E\u003Cli\u003E测量中的误差; \u003C\u002Fli\u003E\u003Cli\u003E运算和计算机表达中的不精确所引入的机器误差或者舍入误差(round-off error); \u003C\u002Fli\u003E\u003Cli\u003E数值方法和离散化带来的误差(discretization error).\u003C\u002Fli\u003E\u003C\u002Ful\u003E\u003Cp\u003E尤其是后面两种,是科学计算中要面临的两个主要的误差来源,下面我们会对它们进行逐一分析. 因此,在评估和使用数值方法时,我们需要系统的衡量这两项误差,并且要\u003Cb\u003E掌握它们的来源\u003C\u002Fb\u003E,对它们的\u003Cb\u003E大小有足够的控制\u003C\u002Fb\u003E.\u003C\u002Fp\u003E\u003Ch2\u003E范数(Norm)\u003C\u002Fh2\u003E\u003Cp\u003E在了解误差以前,我们先来回顾一下线性代数中的一个常规概念,\u003Cb\u003E范数\u003C\u002Fb\u003E.\u003C\u002Fp\u003E\u003Cp\u003E科学计算中经常会涉及到向量(vector),矩阵(matrix)和张量(tensor). 计算与它们有关的误差,需要使用更为一般化的“绝对值”函数,也就是\u003Cb\u003E范数\u003C\u002Fb\u003E \u003Cimg src=\"https:\u002F\u002Fwww.zhihu.com\u002Fequation?tex=+%5C%7C%5Ccdot%5C%7C\" alt=\" \\|\\cdot\\|\" eeimg=\"1\"\u002F\u003E .\u003C\u002Fp\u003E\u003Cp\u003E下面是几个常见的范数,其中向量由小写字母表示,矩阵由大写字母表示.\u003C\u002Fp\u003E\u003Cul\u003E\u003Cli\u003E1-范数: \u003Cimg src=\"https:\u002F\u002Fwww.zhihu.com\u002Fequation?tex=%5C%7Cv%5C%7C_1%3D%28%7Cv_1%7C%2B%7Cv_2%7C%2B%5Ccdots%2B%7Cv_N%7C%29\" alt=\"\\|v\\|_1=(|v_1|+|v_2|+\\cdots+|v_N|)\" eeimg=\"1\"\u002F\u003E \u003C\u002Fli\u003E\u003Cli\u003E2-范数: \u003Cimg src=\"https:\u002F\u002Fwww.zhihu.com\u002Fequation?tex=%5C%7Cv%5C%7C_2%3D%28v_1%5E2%2Bv_2%5E2%2B%5Ccdots%2Bv_N%5E2%29%5E%7B%5Cfrac%7B1%7D%7B2%7D%7D\" alt=\"\\|v\\|_2=(v_1^2+v_2^2+\\cdots+v_N^2)^{\\frac{1}{2}}\" eeimg=\"1\"\u002F\u003E \u003C\u002Fli\u003E\u003Cli\u003Ep-范数: \u003Cimg src=\"https:\u002F\u002Fwww.zhihu.com\u002Fequation?tex=%5C%7Cv%5C%7C_p%3D%28v_1%5Ep%2Bv_2%5Ep%2B%5Ccdots%2Bv_N%5Ep%29%5E%7B%5Cfrac%7B1%7D%7Bp%7D%7D%2C%5Cquad+p%3E0\" alt=\"\\|v\\|_p=(v_1^p+v_2^p+\\cdots+v_N^p)^{\\frac{1}{p}},\\quad p&gt;0\" eeimg=\"1\"\u002F\u003E \u003C\u002Fli\u003E\u003Cli\u003E\u003Cimg src=\"https:\u002F\u002Fwww.zhihu.com\u002Fequation?tex=%5Cinfty\" alt=\"\\infty\" eeimg=\"1\"\u002F\u003E -范数, \u003Cimg src=\"https:\u002F\u002Fwww.zhihu.com\u002Fequation?tex=%5C%7Cv%5C%7C_%7B%5Cinfty%7D%3D%5Cmax_%7B1%5Cle+i+%5Cle+N%7D%7Cv_i%7C\" alt=\"\\|v\\|_{\\infty}=\\max_{1\\le i \\le N}|v_i|\" eeimg=\"1\"\u002F\u003E \u003C\u002Fli\u003E\u003Cli\u003E矩阵的p-范数: \u003Cimg src=\"https:\u002F\u002Fwww.zhihu.com\u002Fequation?tex=%5C%7CA%5C%7Cp%3D%5Cmax_%7B%5C%7Cu%5C%7C_p%3D1%7D%5C%7CAu%5C%7C_p\" alt=\"\\|A\\|p=\\max_{\\|u\\|_p=1}\\|Au\\|_p\" eeimg=\"1\"\u002F\u003E \u003C\u002Fli\u003E\u003Cli\u003EFrobenius-范数: \u003Cimg src=\"https:\u002F\u002Fwww.zhihu.com\u002Fequation?tex=%5C%7CA%5C%7C_F%3D%28%5Csum_%7Bi%3D1%7D%5Em%5Csum_%7Bj%3D1%7D%5En%7Ca_%7Bij%7D%7C%5E2%29%5E%7B%5Cfrac%7B1%7D%7B2%7D%7D\" alt=\"\\|A\\|_F=(\\sum_{i=1}^m\\sum_{j=1}^n|a_{ij}|^2)^{\\frac{1}{2}}\" eeimg=\"1\"\u002F\u003E \u003C\u002Fli\u003E\u003C\u002Ful\u003E\u003Cp\u003E这些范数可以用Matlab的\u003Ccode\u003Enorm()\u003C\u002Fcode\u003E函数或者用Python中的\u003Ccode\u003E\u003Ca href=\"https:\u002F\u002Flink.zhihu.com\u002F?target=https%3A\u002F\u002Fdocs.scipy.org\u002Fdoc\u002Fnumpy\u002Freference\u002Fgenerated\u002Fnumpy.linalg.norm.html\" class=\" wrap external\" target=\"_blank\" rel=\"nofollow noreferrer\"\u003Enumpy.linalg.norm()\u003C\u002Fa\u003E\u003C\u002Fcode\u003E函数进行计算.那么绝对误差的用范数表达式为 \u003Cimg src=\"https:\u002F\u002Fwww.zhihu.com\u002Fequation?tex=%5Cepsilon%3D%5C%7Cx-%5Chat%7Bx%7D%5C%7C\" alt=\"\\epsilon=\\|x-\\hat{x}\\|\" eeimg=\"1\"\u002F\u003E ,相对误差为 \u003Cimg src=\"https:\u002F\u002Fwww.zhihu.com\u002Fequation?tex=%5Ctau%3D%5Cfrac%7B%5C%7Cx-%5Chat%7Bx%7D%5C%7C%7D%7B%5C%7Cx%5C%7C%7D\" alt=\"\\tau=\\frac{\\|x-\\hat{x}\\|}{\\|x\\|}\" eeimg=\"1\"\u002F\u003E .\u003C\u002Fp\u003E\u003Ch3\u003E例子1: \u003Ccode\u003Eabsrelerror(corr, approx)\u003C\u002Fcode\u003E函数使用默认的2-范数,计算\u003Ccode\u003Ecorr\u003C\u002Fcode\u003E和\u003Ccode\u003Eapprox\u003C\u002Fcode\u003E之间的绝对和相对误差.\u003C\u002Fh3\u003E\u003Cdiv class=\"highlight\"\u003E\u003Cpre\u003E\u003Ccode class=\"language-python\"\u003E\u003Cspan class=\"kn\"\u003Eimport\u003C\u002Fspan\u003E \u003Cspan class=\"nn\"\u003Enumpy\u003C\u002Fspan\u003E \u003Cspan class=\"kn\"\u003Eas\u003C\u002Fspan\u003E \u003Cspan class=\"nn\"\u003Enp\u003C\u002Fspan\u003E\n\n\u003Cspan class=\"k\"\u003Edef\u003C\u002Fspan\u003E \u003Cspan class=\"nf\"\u003Eabsrelerror\u003C\u002Fspan\u003E\u003Cspan class=\"p\"\u003E(\u003C\u002Fspan\u003E\u003Cspan class=\"n\"\u003Ecorr\u003C\u002Fspan\u003E\u003Cspan class=\"o\"\u003E=\u003C\u002Fspan\u003E\u003Cspan class=\"bp\"\u003ENone\u003C\u002Fspan\u003E\u003Cspan class=\"p\"\u003E,\u003C\u002Fspan\u003E \u003Cspan class=\"n\"\u003Eapprox\u003C\u002Fspan\u003E\u003Cspan class=\"o\"\u003E=\u003C\u002Fspan\u003E\u003Cspan class=\"bp\"\u003ENone\u003C\u002Fspan\u003E\u003Cspan class=\"p\"\u003E):\u003C\u002Fspan\u003E\n    \u003Cspan class=\"s2\"\u003E&#34;&#34;&#34; \n\u003C\u002Fspan\u003E\u003Cspan class=\"s2\"\u003E    Illustrates the relative and absolute error.\n\u003C\u002Fspan\u003E\u003Cspan class=\"s2\"\u003E    The program calculates the absolute and relative error. \n\u003C\u002Fspan\u003E\u003Cspan class=\"s2\"\u003E    For vectors and matrices, numpy.linalg.norm() is used.\n\u003C\u002Fspan\u003E\u003Cspan class=\"s2\"\u003E\n\u003C\u002Fspan\u003E\u003Cspan class=\"s2\"\u003E    Parameters\n\u003C\u002Fspan\u003E\u003Cspan class=\"s2\"\u003E    ----------\n\u003C\u002Fspan\u003E\u003Cspan class=\"s2\"\u003E    corr : float, list, numpy.ndarray, optional\n\u003C\u002Fspan\u003E\u003Cspan class=\"s2\"\u003E        The exact value(s)\n\u003C\u002Fspan\u003E\u003Cspan class=\"s2\"\u003E    approx : float, list, numpy.ndarray, optional\n\u003C\u002Fspan\u003E\u003Cspan class=\"s2\"\u003E        The approximated value(s)\n\u003C\u002Fspan\u003E\u003Cspan class=\"s2\"\u003E\n\u003C\u002Fspan\u003E\u003Cspan class=\"s2\"\u003E    Returns\n\u003C\u002Fspan\u003E\u003Cspan class=\"s2\"\u003E    -------\n\u003C\u002Fspan\u003E\u003Cspan class=\"s2\"\u003E    None\n\u003C\u002Fspan\u003E\u003Cspan class=\"s2\"\u003E    &#34;&#34;&#34;\u003C\u002Fspan\u003E\n\n    \u003Cspan class=\"k\"\u003Eprint\u003C\u002Fspan\u003E\u003Cspan class=\"p\"\u003E(\u003C\u002Fspan\u003E\u003Cspan class=\"s1\"\u003E&#39;*----------------------------------------------------------*&#39;\u003C\u002Fspan\u003E\u003Cspan class=\"p\"\u003E)\u003C\u002Fspan\u003E\n    \u003Cspan class=\"k\"\u003Eprint\u003C\u002Fspan\u003E\u003Cspan class=\"p\"\u003E(\u003C\u002Fspan\u003E\u003Cspan class=\"s1\"\u003E&#39;This program illustrates the absolute and relative error.&#39;\u003C\u002Fspan\u003E\u003Cspan class=\"p\"\u003E)\u003C\u002Fspan\u003E\n    \u003Cspan class=\"k\"\u003Eprint\u003C\u002Fspan\u003E\u003Cspan class=\"p\"\u003E(\u003C\u002Fspan\u003E\u003Cspan class=\"s1\"\u003E&#39;*----------------------------------------------------------*&#39;\u003C\u002Fspan\u003E\u003Cspan class=\"p\"\u003E)\u003C\u002Fspan\u003E\n\n    \u003Cspan class=\"c1\"\u003E# Check if the values are given, if not ask to input\u003C\u002Fspan\u003E\n    \u003Cspan class=\"k\"\u003Eif\u003C\u002Fspan\u003E \u003Cspan class=\"n\"\u003Ecorr\u003C\u002Fspan\u003E \u003Cspan class=\"ow\"\u003Eis\u003C\u002Fspan\u003E \u003Cspan class=\"bp\"\u003ENone\u003C\u002Fspan\u003E\u003Cspan class=\"p\"\u003E:\u003C\u002Fspan\u003E\n        \u003Cspan class=\"n\"\u003Ecorr\u003C\u002Fspan\u003E \u003Cspan class=\"o\"\u003E=\u003C\u002Fspan\u003E \u003Cspan class=\"nb\"\u003Efloat\u003C\u002Fspan\u003E\u003Cspan class=\"p\"\u003E(\u003C\u002Fspan\u003E\u003Cspan class=\"nb\"\u003Einput\u003C\u002Fspan\u003E\u003Cspan class=\"p\"\u003E(\u003C\u002Fspan\u003E\u003Cspan class=\"s1\"\u003E&#39;Give the correct, exact number: &#39;\u003C\u002Fspan\u003E\u003Cspan class=\"p\"\u003E))\u003C\u002Fspan\u003E\n    \u003Cspan class=\"k\"\u003Eif\u003C\u002Fspan\u003E \u003Cspan class=\"n\"\u003Eapprox\u003C\u002Fspan\u003E \u003Cspan class=\"ow\"\u003Eis\u003C\u002Fspan\u003E \u003Cspan class=\"bp\"\u003ENone\u003C\u002Fspan\u003E\u003Cspan class=\"p\"\u003E:\u003C\u002Fspan\u003E\n        \u003Cspan class=\"n\"\u003Eapprox\u003C\u002Fspan\u003E \u003Cspan class=\"o\"\u003E=\u003C\u002Fspan\u003E \u003Cspan class=\"nb\"\u003Efloat\u003C\u002Fspan\u003E\u003Cspan class=\"p\"\u003E(\u003C\u002Fspan\u003E\u003Cspan class=\"nb\"\u003Einput\u003C\u002Fspan\u003E\u003Cspan class=\"p\"\u003E(\u003C\u002Fspan\u003E\u003Cspan class=\"s1\"\u003E&#39;Give the approximated, calculated number: &#39;\u003C\u002Fspan\u003E\u003Cspan class=\"p\"\u003E))\u003C\u002Fspan\u003E\n\n    \u003Cspan class=\"c1\"\u003E# be default 2-norm\u002FFrobenius-norm is used\u003C\u002Fspan\u003E\n    \u003Cspan class=\"n\"\u003Eabserror\u003C\u002Fspan\u003E \u003Cspan class=\"o\"\u003E=\u003C\u002Fspan\u003E \u003Cspan class=\"n\"\u003Enp\u003C\u002Fspan\u003E\u003Cspan class=\"o\"\u003E.\u003C\u002Fspan\u003E\u003Cspan class=\"n\"\u003Elinalg\u003C\u002Fspan\u003E\u003Cspan class=\"o\"\u003E.\u003C\u002Fspan\u003E\u003Cspan class=\"n\"\u003Enorm\u003C\u002Fspan\u003E\u003Cspan class=\"p\"\u003E(\u003C\u002Fspan\u003E\u003Cspan class=\"n\"\u003Ecorr\u003C\u002Fspan\u003E \u003Cspan class=\"o\"\u003E-\u003C\u002Fspan\u003E \u003Cspan class=\"n\"\u003Eapprox\u003C\u002Fspan\u003E\u003Cspan class=\"p\"\u003E)\u003C\u002Fspan\u003E\n    \u003Cspan class=\"n\"\u003Erelerror\u003C\u002Fspan\u003E \u003Cspan class=\"o\"\u003E=\u003C\u002Fspan\u003E \u003Cspan class=\"n\"\u003Eabserror\u003C\u002Fspan\u003E\u003Cspan class=\"o\"\u003E\u002F\u003C\u002Fspan\u003E\u003Cspan class=\"n\"\u003Enp\u003C\u002Fspan\u003E\u003Cspan class=\"o\"\u003E.\u003C\u002Fspan\u003E\u003Cspan class=\"n\"\u003Elinalg\u003C\u002Fspan\u003E\u003Cspan class=\"o\"\u003E.\u003C\u002Fspan\u003E\u003Cspan class=\"n\"\u003Enorm\u003C\u002Fspan\u003E\u003Cspan class=\"p\"\u003E(\u003C\u002Fspan\u003E\u003Cspan class=\"n\"\u003Ecorr\u003C\u002Fspan\u003E\u003Cspan class=\"p\"\u003E)\u003C\u002Fspan\u003E\n\n    \u003Cspan class=\"c1\"\u003E# Output\u003C\u002Fspan\u003E\n    \u003Cspan class=\"k\"\u003Eprint\u003C\u002Fspan\u003E\u003Cspan class=\"p\"\u003E(\u003C\u002Fspan\u003E\u003Cspan class=\"n\"\u003Ef\u003C\u002Fspan\u003E\u003Cspan class=\"s1\"\u003E&#39;Absolute error: {abserror}&#39;\u003C\u002Fspan\u003E\u003Cspan class=\"p\"\u003E)\u003C\u002Fspan\u003E\n    \u003Cspan class=\"k\"\u003Eprint\u003C\u002Fspan\u003E\u003Cspan class=\"p\"\u003E(\u003C\u002Fspan\u003E\u003Cspan class=\"n\"\u003Ef\u003C\u002Fspan\u003E\u003Cspan class=\"s1\"\u003E&#39;Relative error: {relerror}&#39;\u003C\u002Fspan\u003E\u003Cspan class=\"p\"\u003E)\u003C\u002Fspan\u003E\u003C\u002Fcode\u003E\u003C\u002Fpre\u003E\u003C\u002Fdiv\u003E\u003Ch2\u003E舍入误差(Round-off error)\u003C\u002Fh2\u003E\u003Cp\u003E舍入误差,有时也称为机器精度(machine epsilon),简单理解就是\u003Cb\u003E由于用计算机有限的内存来表达实数轴上面无限多的数,而引入的近似误差.\u003C\u002Fb\u003E\u003C\u002Fp\u003E\u003Cp\u003E换一个角度来看,这个误差应该不超过计算机中可以表达的相邻两个实数之间的间隔. 通过一些巧妙的设计,例如 \u003C\u002Fp\u003E\u003Ca href=\"https:\u002F\u002Flink.zhihu.com\u002F?target=https%3A\u002F\u002Fieeexplore.ieee.org\u002Fdocument\u002F8766229\" data-draft-node=\"block\" data-draft-type=\"link-card\" class=\" wrap external\" target=\"_blank\" rel=\"nofollow noreferrer\"\u003E754-2019 - IEEE Standard for Floating-Point Arithmetic - IEEE Standard\u003C\u002Fa\u003E\u003Cp\u003E这个间隔的绝对值可以随计算机所要表达的值的大小变化,并始终保持它的相对值在一个较小的范围. 关于浮点数标准,我们会在后面的一章“\u003Ci\u003E计算机算术”\u003C\u002Fi\u003E里面具体解释和分析.\u003C\u002Fp\u003E\u003Ch3\u003E例子2:机器精度\u003C\u002Fh3\u003E\u003Cp\u003E使用之前定义的\u003Ccode\u003Eabsrelerror()\u003C\u002Fcode\u003E函数,在Python3环境下,运行下面的程序.\u003C\u002Fp\u003E\u003Cdiv class=\"highlight\"\u003E\u003Cpre\u003E\u003Ccode class=\"language-python\"\u003E\u003Cspan class=\"n\"\u003Ea\u003C\u002Fspan\u003E \u003Cspan class=\"o\"\u003E=\u003C\u002Fspan\u003E \u003Cspan class=\"mi\"\u003E4\u003C\u002Fspan\u003E\u003Cspan class=\"o\"\u003E\u002F\u003C\u002Fspan\u003E\u003Cspan class=\"mi\"\u003E3\u003C\u002Fspan\u003E\n\u003Cspan class=\"n\"\u003Eb\u003C\u002Fspan\u003E \u003Cspan class=\"o\"\u003E=\u003C\u002Fspan\u003E \u003Cspan class=\"n\"\u003Ea\u003C\u002Fspan\u003E\u003Cspan class=\"o\"\u003E-\u003C\u002Fspan\u003E\u003Cspan class=\"mi\"\u003E1\u003C\u002Fspan\u003E\n\u003Cspan class=\"n\"\u003Ec\u003C\u002Fspan\u003E \u003Cspan class=\"o\"\u003E=\u003C\u002Fspan\u003E \u003Cspan class=\"n\"\u003Eb\u003C\u002Fspan\u003E \u003Cspan class=\"o\"\u003E+\u003C\u002Fspan\u003E \u003Cspan class=\"n\"\u003Eb\u003C\u002Fspan\u003E \u003Cspan class=\"o\"\u003E+\u003C\u002Fspan\u003E\u003Cspan class=\"n\"\u003Eb\u003C\u002Fspan\u003E\n\u003Cspan class=\"k\"\u003Eprint\u003C\u002Fspan\u003E\u003Cspan class=\"p\"\u003E(\u003C\u002Fspan\u003E\u003Cspan class=\"n\"\u003Ef\u003C\u002Fspan\u003E\u003Cspan class=\"s1\"\u003E&#39;c = {c}&#39;\u003C\u002Fspan\u003E\u003Cspan class=\"p\"\u003E)\u003C\u002Fspan\u003E\n\u003Cspan class=\"n\"\u003Eabsrelerror\u003C\u002Fspan\u003E\u003Cspan class=\"p\"\u003E(\u003C\u002Fspan\u003E\u003Cspan class=\"n\"\u003Ec\u003C\u002Fspan\u003E\u003Cspan class=\"p\"\u003E,\u003C\u002Fspan\u003E \u003Cspan class=\"mi\"\u003E1\u003C\u002Fspan\u003E\u003Cspan class=\"p\"\u003E)\u003C\u002Fspan\u003E\u003C\u002Fcode\u003E\u003C\u002Fpre\u003E\u003C\u002Fdiv\u003E\u003Cp\u003E输出:\u003C\u002Fp\u003E\u003Cdiv class=\"highlight\"\u003E\u003Cpre\u003E\u003Ccode class=\"language-pycon\"\u003Ec = 0.9999999999999998\n*----------------------------------------------------------*\nThis program illustrates the absolute and relative error.\n*----------------------------------------------------------*\nAbsolute error: 2.220446049250313e-16\nRelative error: 2.2204460492503136e-16\u003C\u002Fcode\u003E\u003C\u002Fpre\u003E\u003C\u002Fdiv\u003E\u003Ch3\u003E例子3: 机器精度2\u003C\u002Fh3\u003E\u003Cp\u003E运行下面的四行代码,观察输出的结果是否存在误差.\u003C\u002Fp\u003E\u003Cdiv class=\"highlight\"\u003E\u003Cpre\u003E\u003Ccode class=\"language-python\"\u003E\u003Cspan class=\"k\"\u003Eprint\u003C\u002Fspan\u003E\u003Cspan class=\"p\"\u003E(\u003C\u002Fspan\u003E\u003Cspan class=\"mf\"\u003E1e-20\u003C\u002Fspan\u003E\u003Cspan class=\"o\"\u003E+\u003C\u002Fspan\u003E\u003Cspan class=\"mf\"\u003E1e-34\u003C\u002Fspan\u003E\u003Cspan class=\"p\"\u003E)\u003C\u002Fspan\u003E\n\u003Cspan class=\"k\"\u003Eprint\u003C\u002Fspan\u003E\u003Cspan class=\"p\"\u003E(\u003C\u002Fspan\u003E\u003Cspan class=\"mf\"\u003E1e-20\u003C\u002Fspan\u003E\u003Cspan class=\"o\"\u003E+\u003C\u002Fspan\u003E\u003Cspan class=\"mf\"\u003E1e-35\u003C\u002Fspan\u003E\u003Cspan class=\"p\"\u003E)\u003C\u002Fspan\u003E\n\u003Cspan class=\"k\"\u003Eprint\u003C\u002Fspan\u003E\u003Cspan class=\"p\"\u003E(\u003C\u002Fspan\u003E\u003Cspan class=\"mf\"\u003E1e-20\u003C\u002Fspan\u003E\u003Cspan class=\"o\"\u003E+\u003C\u002Fspan\u003E\u003Cspan class=\"mf\"\u003E1e-36\u003C\u002Fspan\u003E\u003Cspan class=\"p\"\u003E)\u003C\u002Fspan\u003E\n\u003Cspan class=\"k\"\u003Eprint\u003C\u002Fspan\u003E\u003Cspan class=\"p\"\u003E(\u003C\u002Fspan\u003E\u003Cspan class=\"mf\"\u003E1e-20\u003C\u002Fspan\u003E\u003Cspan class=\"o\"\u003E+\u003C\u002Fspan\u003E\u003Cspan class=\"mf\"\u003E1e-37\u003C\u002Fspan\u003E\u003Cspan class=\"p\"\u003E)\u003C\u002Fspan\u003E\u003C\u002Fcode\u003E\u003C\u002Fpre\u003E\u003C\u002Fdiv\u003E\u003Cp\u003E输出:\u003C\u002Fp\u003E\u003Cdiv class=\"highlight\"\u003E\u003Cpre\u003E\u003Ccode class=\"language-pycon\"\u003E1.0000000000000099e-20\n1.000000000000001e-20\n1.0000000000000001e-20\n1e-20\u003C\u002Fcode\u003E\u003C\u002Fpre\u003E\u003C\u002Fdiv\u003E\u003Cp\u003E由上面两个例子可以观察到,机器精度约为 \u003Cimg src=\"https:\u002F\u002Fwww.zhihu.com\u002Fequation?tex=10%5E%7B-16%7D\" alt=\"10^{-16}\" eeimg=\"1\"\u002F\u003E . 值得注意的是,在例子2中,如果我们直接令\u003Ccode\u003Eb=1\u002F3\u003C\u002Fcode\u003E,再求和,则不会得到任何误差. 事实上,这个舍入误差出现在\u003Ccode\u003Eb=a-1\u003C\u002Fcode\u003E时,并且延续到了后面的求和计算. 而对于例子3,求和时当两个数的相对大小差距超过 \u003Cimg src=\"https:\u002F\u002Fwww.zhihu.com\u002Fequation?tex=10%5E%7B-16%7D\" alt=\"10^{-16}\" eeimg=\"1\"\u002F\u003E (或 \u003Cimg src=\"https:\u002F\u002Fwww.zhihu.com\u002Fequation?tex=10%5E%7B16%7D\" alt=\"10^{16}\" eeimg=\"1\"\u002F\u003E )时,较小的数的贡献会消失.这种现象在有些地方也被称作cancellation error. 事实上,它只是舍入误差的一种表现. 至于具体的原理,也会在后面计算机算术这一章和浮点数标准中详细介绍.\u003C\u002Fp\u003E\u003Ch3\u003E例子4: 矩阵运算中的误差\u003C\u002Fh3\u003E\u003Cp\u003E下面定义一个测试函数\u003Ccode\u003EtestErrA(n)\u003C\u002Fcode\u003E,随机生成 \u003Cimg src=\"https:\u002F\u002Fwww.zhihu.com\u002Fequation?tex=n%5Ctimes+n\" alt=\"n\\times n\" eeimg=\"1\"\u002F\u003E 矩阵 \u003Cimg src=\"https:\u002F\u002Fwww.zhihu.com\u002Fequation?tex=A\" alt=\"A\" eeimg=\"1\"\u002F\u003E ,然后计算 \u003Cimg src=\"https:\u002F\u002Fwww.zhihu.com\u002Fequation?tex=A%5E%7B-1%7DA\" alt=\"A^{-1}A\" eeimg=\"1\"\u002F\u003E 与单位矩阵 \u003Cimg src=\"https:\u002F\u002Fwww.zhihu.com\u002Fequation?tex=I\" alt=\"I\" eeimg=\"1\"\u002F\u003E 之间的误差.\u003C\u002Fp\u003E\u003Cp\u003E理论上,这两个值应该完全相等,但是由于存在舍入误差,\u003Cimg src=\"https:\u002F\u002Fwww.zhihu.com\u002Fequation?tex=A%5E%7B-1%7DA\" alt=\"A^{-1}A\" eeimg=\"1\"\u002F\u003E并不完全等于单位矩阵.\u003C\u002Fp\u003E\u003Cdiv class=\"highlight\"\u003E\u003Cpre\u003E\u003Ccode class=\"language-python\"\u003E\u003Cspan class=\"c1\"\u003E# Generate a random nxn matrix and compute A^{-1}*A which should be I analytically\u003C\u002Fspan\u003E\n\u003Cspan class=\"k\"\u003Edef\u003C\u002Fspan\u003E \u003Cspan class=\"nf\"\u003EtestErrA\u003C\u002Fspan\u003E\u003Cspan class=\"p\"\u003E(\u003C\u002Fspan\u003E\u003Cspan class=\"n\"\u003En\u003C\u002Fspan\u003E \u003Cspan class=\"o\"\u003E=\u003C\u002Fspan\u003E \u003Cspan class=\"mi\"\u003E10\u003C\u002Fspan\u003E\u003Cspan class=\"p\"\u003E):\u003C\u002Fspan\u003E\n    \u003Cspan class=\"n\"\u003EA\u003C\u002Fspan\u003E \u003Cspan class=\"o\"\u003E=\u003C\u002Fspan\u003E \u003Cspan class=\"n\"\u003Enp\u003C\u002Fspan\u003E\u003Cspan class=\"o\"\u003E.\u003C\u002Fspan\u003E\u003Cspan class=\"n\"\u003Erandom\u003C\u002Fspan\u003E\u003Cspan class=\"o\"\u003E.\u003C\u002Fspan\u003E\u003Cspan class=\"n\"\u003Erand\u003C\u002Fspan\u003E\u003Cspan class=\"p\"\u003E(\u003C\u002Fspan\u003E\u003Cspan class=\"n\"\u003En\u003C\u002Fspan\u003E\u003Cspan class=\"p\"\u003E,\u003C\u002Fspan\u003E\u003Cspan class=\"n\"\u003En\u003C\u002Fspan\u003E\u003Cspan class=\"p\"\u003E)\u003C\u002Fspan\u003E\n    \u003Cspan class=\"n\"\u003EIcomp\u003C\u002Fspan\u003E \u003Cspan class=\"o\"\u003E=\u003C\u002Fspan\u003E \u003Cspan class=\"n\"\u003Enp\u003C\u002Fspan\u003E\u003Cspan class=\"o\"\u003E.\u003C\u002Fspan\u003E\u003Cspan class=\"n\"\u003Ematmul\u003C\u002Fspan\u003E\u003Cspan class=\"p\"\u003E(\u003C\u002Fspan\u003E\u003Cspan class=\"n\"\u003Enp\u003C\u002Fspan\u003E\u003Cspan class=\"o\"\u003E.\u003C\u002Fspan\u003E\u003Cspan class=\"n\"\u003Elinalg\u003C\u002Fspan\u003E\u003Cspan class=\"o\"\u003E.\u003C\u002Fspan\u003E\u003Cspan class=\"n\"\u003Einv\u003C\u002Fspan\u003E\u003Cspan class=\"p\"\u003E(\u003C\u002Fspan\u003E\u003Cspan class=\"n\"\u003EA\u003C\u002Fspan\u003E\u003Cspan class=\"p\"\u003E),\u003C\u002Fspan\u003E \u003Cspan class=\"n\"\u003EA\u003C\u002Fspan\u003E\u003Cspan class=\"p\"\u003E)\u003C\u002Fspan\u003E\n    \u003Cspan class=\"n\"\u003EIexact\u003C\u002Fspan\u003E \u003Cspan class=\"o\"\u003E=\u003C\u002Fspan\u003E \u003Cspan class=\"n\"\u003Enp\u003C\u002Fspan\u003E\u003Cspan class=\"o\"\u003E.\u003C\u002Fspan\u003E\u003Cspan class=\"n\"\u003Eeye\u003C\u002Fspan\u003E\u003Cspan class=\"p\"\u003E(\u003C\u002Fspan\u003E\u003Cspan class=\"n\"\u003En\u003C\u002Fspan\u003E\u003Cspan class=\"p\"\u003E)\u003C\u002Fspan\u003E\n    \u003Cspan class=\"n\"\u003Eabsrelerror\u003C\u002Fspan\u003E\u003Cspan class=\"p\"\u003E(\u003C\u002Fspan\u003E\u003Cspan class=\"n\"\u003EIexact\u003C\u002Fspan\u003E\u003Cspan class=\"p\"\u003E,\u003C\u002Fspan\u003E \u003Cspan class=\"n\"\u003EIcomp\u003C\u002Fspan\u003E\u003Cspan class=\"p\"\u003E)\u003C\u002Fspan\u003E\u003C\u002Fcode\u003E\u003C\u002Fpre\u003E\u003C\u002Fdiv\u003E\u003Cp\u003E调用函数可得到类似如下的输出:\u003C\u002Fp\u003E\u003Cdiv class=\"highlight\"\u003E\u003Cpre\u003E\u003Ccode class=\"language-pycon\"\u003EtestErrA()\n*----------------------------------------------------------*\nThis program illustrates the absolute and relative error.\n*----------------------------------------------------------*\nAbsolute error: 6.626588434229317e-15\nRelative error: 2.095511256869353e-15\u003C\u002Fcode\u003E\u003C\u002Fpre\u003E\u003C\u002Fdiv\u003E\u003Cp\u003E由于矩阵 \u003Cimg src=\"https:\u002F\u002Fwww.zhihu.com\u002Fequation?tex=A\" alt=\"A\" eeimg=\"1\"\u002F\u003E 的随机性,这里的输出并不会完全一致,但相对误差始终保持在\u003Cimg src=\"https:\u002F\u002Fwww.zhihu.com\u002Fequation?tex=10%5E%7B-14%7D\" alt=\"10^{-14}\" eeimg=\"1\"\u002F\u003E 至 \u003Cimg src=\"https:\u002F\u002Fwww.zhihu.com\u002Fequation?tex=10%5E%7B-16%7D\" alt=\"10^{-16}\" eeimg=\"1\"\u002F\u003E 这个范围. 我们也可以尝试不同大小的矩阵,误差也会随着矩阵尺寸变大而增大.\u003C\u002Fp\u003E\u003Cp\u003E\u003Cb\u003E注1\u003C\u002Fb\u003E:对于矩阵求逆运算的机器精度估算,涉及到矩阵的条件数(condition number),这个会在后面求解线性方程时具体分析.\u003C\u002Fp\u003E\u003Cp\u003E\u003Cb\u003E注2\u003C\u002Fb\u003E:矩阵求逆的运算复杂度约为 \u003Cimg src=\"https:\u002F\u002Fwww.zhihu.com\u002Fequation?tex=%5Cmathcal%7BO%7D%28n%5E2%29\" alt=\"\\mathcal{O}(n^2)\" eeimg=\"1\"\u002F\u003E 或以上,因此继续增大 \u003Cimg src=\"https:\u002F\u002Fwww.zhihu.com\u002Fequation?tex=n\" alt=\"n\" eeimg=\"1\"\u002F\u003E 有可能会让程序的运行时间大大增加.\u003C\u002Fp\u003E\u003Ch2\u003E离散化误差(Discretization error)\u003C\u002Fh2\u003E\u003Cp\u003E顾名思义,是由数值方法的离散化所引入的误差.通常情况下,离散化误差的大小与离散化尺寸直接相关. 以前向差分(forward difference)为例,函数 \u003Cimg src=\"https:\u002F\u002Fwww.zhihu.com\u002Fequation?tex=f%28x%29\" alt=\"f(x)\" eeimg=\"1\"\u002F\u003E 的一阶导数可以近似为:\u003C\u002Fp\u003E\u003Cp\u003E\u003Cimg src=\"https:\u002F\u002Fwww.zhihu.com\u002Fequation?tex=+f%27%28x%29%5Capprox+%5Cfrac%7Bf%28x%2Bh%29-f%28x%29%7D%7Bh%7D%EF%BC%8C%5C%5C\" alt=\" f&#39;(x)\\approx \\frac{f(x+h)-f(x)}{h},\\\\\" eeimg=\"1\"\u002F\u003E \u003C\u002Fp\u003E\u003Cp\u003E其中, \u003Cimg src=\"https:\u002F\u002Fwww.zhihu.com\u002Fequation?tex=h\" alt=\"h\" eeimg=\"1\"\u002F\u003E 为网格尺寸或步长. 通过泰勒展开(Taylor expansion)可知,前向差分的离散化误差为 \u003Cimg src=\"https:\u002F\u002Fwww.zhihu.com\u002Fequation?tex=%5Cmathcal%7BO%7D%28h%29\" alt=\"\\mathcal{O}(h)\" eeimg=\"1\"\u002F\u003E ,即每当 \u003Cimg src=\"https:\u002F\u002Fwww.zhihu.com\u002Fequation?tex=h\" alt=\"h\" eeimg=\"1\"\u002F\u003E 缩小到它的一半,则误差也相应的缩小一半.\u003C\u002Fp\u003E\u003Cp\u003E类似的,对于中央差分(central difference)和五点差分(five-points difference)\u003C\u002Fp\u003E\u003Cp\u003E\u003Cimg src=\"https:\u002F\u002Fwww.zhihu.com\u002Fequation?tex=f%27%28x%29%5Capprox+%5Cfrac%7Bf%28x%2Bh%29-f%28x-h%29%7D%7B2h%7D%EF%BC%8C%5C%5C\" alt=\"f&#39;(x)\\approx \\frac{f(x+h)-f(x-h)}{2h},\\\\\" eeimg=\"1\"\u002F\u003E \u003C\u002Fp\u003E\u003Cp\u003E\u003Cimg src=\"https:\u002F\u002Fwww.zhihu.com\u002Fequation?tex=f%27%28x%29%5Capprox+%5Cfrac%7B-f%28x%2B2h%29%2B8f%28x%2Bh%29-8f%28x-h%29%2Bf%28x-2h%29%7D%7B12h%7D.%5C%5C+\" alt=\"f&#39;(x)\\approx \\frac{-f(x+2h)+8f(x+h)-8f(x-h)+f(x-2h)}{12h}.\\\\ \" eeimg=\"1\"\u002F\u003E\u003C\u002Fp\u003E\u003Cp\u003E它们的离散化误差分别为\u003Cimg src=\"https:\u002F\u002Fwww.zhihu.com\u002Fequation?tex=%5Cmathcal%7BO%7D%28h%5E2%29\" alt=\"\\mathcal{O}(h^2)\" eeimg=\"1\"\u002F\u003E和 \u003Cimg src=\"https:\u002F\u002Fwww.zhihu.com\u002Fequation?tex=%5Cmathcal%7BO%7D%28h%5E4%29\" alt=\"\\mathcal{O}(h^4)\" eeimg=\"1\"\u002F\u003E .对应的,当 \u003Cimg src=\"https:\u002F\u002Fwww.zhihu.com\u002Fequation?tex=h\" alt=\"h\" eeimg=\"1\"\u002F\u003E 缩小一半,误差分别变为原来的1\u002F4和1\u002F16.\u003C\u002Fp\u003E\u003Ch3\u003E例子5: 前向差分的离散化误差\u003C\u002Fh3\u003E\u003Cp\u003E取 \u003Cimg src=\"https:\u002F\u002Fwww.zhihu.com\u002Fequation?tex=f%28x%29%3De%5Ex\" alt=\"f(x)=e^x\" eeimg=\"1\"\u002F\u003E ,可知 \u003Cimg src=\"https:\u002F\u002Fwww.zhihu.com\u002Fequation?tex=f%27%28x%29%3De%5Ex\" alt=\"f&#39;(x)=e^x\" eeimg=\"1\"\u002F\u003E .我们用前向差分来计算 \u003Cimg src=\"https:\u002F\u002Fwww.zhihu.com\u002Fequation?tex=f%28x%29\" alt=\"f(x)\" eeimg=\"1\"\u002F\u003E 的一阶导数,把它们的值和真实值对比,并分别画出 \u003Cimg src=\"https:\u002F\u002Fwww.zhihu.com\u002Fequation?tex=h%3D0.2%2C+0.1\" alt=\"h=0.2, 0.1\" eeimg=\"1\"\u002F\u003E 和 \u003Cimg src=\"https:\u002F\u002Fwww.zhihu.com\u002Fequation?tex=0.05\" alt=\"0.05\" eeimg=\"1\"\u002F\u003E 的结果.\u003C\u002Fp\u003E\u003Cdiv class=\"highlight\"\u003E\u003Cpre\u003E\u003Ccode class=\"language-python\"\u003E\u003Cspan class=\"kn\"\u003Eimport\u003C\u002Fspan\u003E \u003Cspan class=\"nn\"\u003Ematplotlib.pyplot\u003C\u002Fspan\u003E \u003Cspan class=\"kn\"\u003Eas\u003C\u002Fspan\u003E \u003Cspan class=\"nn\"\u003Eplt\u003C\u002Fspan\u003E\n\u003Cspan class=\"kn\"\u003Eimport\u003C\u002Fspan\u003E \u003Cspan class=\"nn\"\u003Enumpy\u003C\u002Fspan\u003E \u003Cspan class=\"kn\"\u003Eas\u003C\u002Fspan\u003E \u003Cspan class=\"nn\"\u003Enp\u003C\u002Fspan\u003E\n\n\u003Cspan class=\"k\"\u003Edef\u003C\u002Fspan\u003E \u003Cspan class=\"nf\"\u003EForwardDiff\u003C\u002Fspan\u003E\u003Cspan class=\"p\"\u003E(\u003C\u002Fspan\u003E\u003Cspan class=\"n\"\u003Efx\u003C\u002Fspan\u003E\u003Cspan class=\"p\"\u003E,\u003C\u002Fspan\u003E \u003Cspan class=\"n\"\u003Ex\u003C\u002Fspan\u003E\u003Cspan class=\"p\"\u003E,\u003C\u002Fspan\u003E \u003Cspan class=\"n\"\u003Eh\u003C\u002Fspan\u003E\u003Cspan class=\"o\"\u003E=\u003C\u002Fspan\u003E\u003Cspan class=\"mf\"\u003E0.001\u003C\u002Fspan\u003E\u003Cspan class=\"p\"\u003E):\u003C\u002Fspan\u003E\n    \u003Cspan class=\"c1\"\u003E# ForwardDiff(@fx, x, h);\u003C\u002Fspan\u003E\n    \u003Cspan class=\"k\"\u003Ereturn\u003C\u002Fspan\u003E \u003Cspan class=\"p\"\u003E(\u003C\u002Fspan\u003E\u003Cspan class=\"n\"\u003Efx\u003C\u002Fspan\u003E\u003Cspan class=\"p\"\u003E(\u003C\u002Fspan\u003E\u003Cspan class=\"n\"\u003Ex\u003C\u002Fspan\u003E\u003Cspan class=\"o\"\u003E+\u003C\u002Fspan\u003E\u003Cspan class=\"n\"\u003Eh\u003C\u002Fspan\u003E\u003Cspan class=\"p\"\u003E)\u003C\u002Fspan\u003E \u003Cspan class=\"o\"\u003E-\u003C\u002Fspan\u003E \u003Cspan class=\"n\"\u003Efx\u003C\u002Fspan\u003E\u003Cspan class=\"p\"\u003E(\u003C\u002Fspan\u003E\u003Cspan class=\"n\"\u003Ex\u003C\u002Fspan\u003E\u003Cspan class=\"p\"\u003E))\u003C\u002Fspan\u003E\u003Cspan class=\"o\"\u003E\u002F\u003C\u002Fspan\u003E\u003Cspan class=\"n\"\u003Eh\u003C\u002Fspan\u003E\n\n\u003Cspan class=\"k\"\u003Edef\u003C\u002Fspan\u003E \u003Cspan class=\"nf\"\u003EtestDiscretization\u003C\u002Fspan\u003E\u003Cspan class=\"p\"\u003E(\u003C\u002Fspan\u003E\u003Cspan class=\"n\"\u003Eh\u003C\u002Fspan\u003E\u003Cspan class=\"o\"\u003E=\u003C\u002Fspan\u003E\u003Cspan class=\"mf\"\u003E0.1\u003C\u002Fspan\u003E\u003Cspan class=\"p\"\u003E,\u003C\u002Fspan\u003E \u003Cspan class=\"n\"\u003El\u003C\u002Fspan\u003E\u003Cspan class=\"o\"\u003E=\u003C\u002Fspan\u003E\u003Cspan class=\"mi\"\u003E0\u003C\u002Fspan\u003E\u003Cspan class=\"p\"\u003E,\u003C\u002Fspan\u003E \u003Cspan class=\"n\"\u003Eu\u003C\u002Fspan\u003E\u003Cspan class=\"o\"\u003E=\u003C\u002Fspan\u003E\u003Cspan class=\"mi\"\u003E2\u003C\u002Fspan\u003E\u003Cspan class=\"p\"\u003E):\u003C\u002Fspan\u003E\n    \u003Cspan class=\"c1\"\u003E# compute the numerical derivatives\u003C\u002Fspan\u003E\n    \u003Cspan class=\"n\"\u003Exh\u003C\u002Fspan\u003E \u003Cspan class=\"o\"\u003E=\u003C\u002Fspan\u003E \u003Cspan class=\"n\"\u003Enp\u003C\u002Fspan\u003E\u003Cspan class=\"o\"\u003E.\u003C\u002Fspan\u003E\u003Cspan class=\"n\"\u003Elinspace\u003C\u002Fspan\u003E\u003Cspan class=\"p\"\u003E(\u003C\u002Fspan\u003E\u003Cspan class=\"n\"\u003El\u003C\u002Fspan\u003E\u003Cspan class=\"p\"\u003E,\u003C\u002Fspan\u003E \u003Cspan class=\"n\"\u003Eu\u003C\u002Fspan\u003E\u003Cspan class=\"p\"\u003E,\u003C\u002Fspan\u003E \u003Cspan class=\"nb\"\u003Eint\u003C\u002Fspan\u003E\u003Cspan class=\"p\"\u003E(\u003C\u002Fspan\u003E\u003Cspan class=\"nb\"\u003Eabs\u003C\u002Fspan\u003E\u003Cspan class=\"p\"\u003E(\u003C\u002Fspan\u003E\u003Cspan class=\"n\"\u003Eu\u003C\u002Fspan\u003E\u003Cspan class=\"o\"\u003E-\u003C\u002Fspan\u003E\u003Cspan class=\"n\"\u003El\u003C\u002Fspan\u003E\u003Cspan class=\"p\"\u003E)\u003C\u002Fspan\u003E\u003Cspan class=\"o\"\u003E\u002F\u003C\u002Fspan\u003E\u003Cspan class=\"n\"\u003Eh\u003C\u002Fspan\u003E\u003Cspan class=\"p\"\u003E))\u003C\u002Fspan\u003E\n    \u003Cspan class=\"n\"\u003EfprimF\u003C\u002Fspan\u003E \u003Cspan class=\"o\"\u003E=\u003C\u002Fspan\u003E \u003Cspan class=\"n\"\u003EForwardDiff\u003C\u002Fspan\u003E\u003Cspan class=\"p\"\u003E(\u003C\u002Fspan\u003E\u003Cspan class=\"n\"\u003Enp\u003C\u002Fspan\u003E\u003Cspan class=\"o\"\u003E.\u003C\u002Fspan\u003E\u003Cspan class=\"n\"\u003Eexp\u003C\u002Fspan\u003E\u003Cspan class=\"p\"\u003E,\u003C\u002Fspan\u003E \u003Cspan class=\"n\"\u003Exh\u003C\u002Fspan\u003E\u003Cspan class=\"p\"\u003E,\u003C\u002Fspan\u003E \u003Cspan class=\"n\"\u003Eh\u003C\u002Fspan\u003E\u003Cspan class=\"p\"\u003E)\u003C\u002Fspan\u003E\n    \u003Cspan class=\"k\"\u003Ereturn\u003C\u002Fspan\u003E \u003Cspan class=\"n\"\u003Exh\u003C\u002Fspan\u003E\u003Cspan class=\"p\"\u003E,\u003C\u002Fspan\u003E \u003Cspan class=\"n\"\u003EfprimF\u003C\u002Fspan\u003E\n\n\u003Cspan class=\"c1\"\u003E# The exact solution\u003C\u002Fspan\u003E\n\u003Cspan class=\"n\"\u003ENx\u003C\u002Fspan\u003E \u003Cspan class=\"o\"\u003E=\u003C\u002Fspan\u003E \u003Cspan class=\"mi\"\u003E400\u003C\u002Fspan\u003E\n\u003Cspan class=\"n\"\u003El\u003C\u002Fspan\u003E\u003Cspan class=\"p\"\u003E,\u003C\u002Fspan\u003E \u003Cspan class=\"n\"\u003Eu\u003C\u002Fspan\u003E \u003Cspan class=\"o\"\u003E=\u003C\u002Fspan\u003E \u003Cspan class=\"mi\"\u003E0\u003C\u002Fspan\u003E\u003Cspan class=\"p\"\u003E,\u003C\u002Fspan\u003E \u003Cspan class=\"mi\"\u003E2\u003C\u002Fspan\u003E\n\u003Cspan class=\"n\"\u003Ex\u003C\u002Fspan\u003E \u003Cspan class=\"o\"\u003E=\u003C\u002Fspan\u003E \u003Cspan class=\"n\"\u003Enp\u003C\u002Fspan\u003E\u003Cspan class=\"o\"\u003E.\u003C\u002Fspan\u003E\u003Cspan class=\"n\"\u003Elinspace\u003C\u002Fspan\u003E\u003Cspan class=\"p\"\u003E(\u003C\u002Fspan\u003E\u003Cspan class=\"n\"\u003El\u003C\u002Fspan\u003E\u003Cspan class=\"p\"\u003E,\u003C\u002Fspan\u003E \u003Cspan class=\"n\"\u003Eu\u003C\u002Fspan\u003E\u003Cspan class=\"p\"\u003E,\u003C\u002Fspan\u003E \u003Cspan class=\"n\"\u003ENx\u003C\u002Fspan\u003E\u003Cspan class=\"p\"\u003E)\u003C\u002Fspan\u003E\n\u003Cspan class=\"n\"\u003Ef_exa\u003C\u002Fspan\u003E \u003Cspan class=\"o\"\u003E=\u003C\u002Fspan\u003E \u003Cspan class=\"n\"\u003Enp\u003C\u002Fspan\u003E\u003Cspan class=\"o\"\u003E.\u003C\u002Fspan\u003E\u003Cspan class=\"n\"\u003Eexp\u003C\u002Fspan\u003E\u003Cspan class=\"p\"\u003E(\u003C\u002Fspan\u003E\u003Cspan class=\"n\"\u003Ex\u003C\u002Fspan\u003E\u003Cspan class=\"p\"\u003E)\u003C\u002Fspan\u003E\n\n\u003Cspan class=\"c1\"\u003E# Plot\u003C\u002Fspan\u003E\n\u003Cspan class=\"n\"\u003Efig\u003C\u002Fspan\u003E\u003Cspan class=\"p\"\u003E,\u003C\u002Fspan\u003E \u003Cspan class=\"n\"\u003Eaxs\u003C\u002Fspan\u003E \u003Cspan class=\"o\"\u003E=\u003C\u002Fspan\u003E \u003Cspan class=\"n\"\u003Eplt\u003C\u002Fspan\u003E\u003Cspan class=\"o\"\u003E.\u003C\u002Fspan\u003E\u003Cspan class=\"n\"\u003Esubplots\u003C\u002Fspan\u003E\u003Cspan class=\"p\"\u003E(\u003C\u002Fspan\u003E\u003Cspan class=\"mi\"\u003E3\u003C\u002Fspan\u003E\u003Cspan class=\"p\"\u003E,\u003C\u002Fspan\u003E \u003Cspan class=\"mi\"\u003E1\u003C\u002Fspan\u003E\u003Cspan class=\"p\"\u003E,\u003C\u002Fspan\u003E \u003Cspan class=\"n\"\u003Efigsize\u003C\u002Fspan\u003E\u003Cspan class=\"o\"\u003E=\u003C\u002Fspan\u003E\u003Cspan class=\"p\"\u003E(\u003C\u002Fspan\u003E\u003Cspan class=\"mi\"\u003E6\u003C\u002Fspan\u003E\u003Cspan class=\"p\"\u003E,\u003C\u002Fspan\u003E\u003Cspan class=\"mi\"\u003E6\u003C\u002Fspan\u003E\u003Cspan class=\"p\"\u003E))\u003C\u002Fspan\u003E\n\u003Cspan class=\"n\"\u003Efig\u003C\u002Fspan\u003E\u003Cspan class=\"o\"\u003E.\u003C\u002Fspan\u003E\u003Cspan class=\"n\"\u003Etight_layout\u003C\u002Fspan\u003E\u003Cspan class=\"p\"\u003E(\u003C\u002Fspan\u003E\u003Cspan class=\"n\"\u003Epad\u003C\u002Fspan\u003E\u003Cspan class=\"o\"\u003E=\u003C\u002Fspan\u003E\u003Cspan class=\"mi\"\u003E0\u003C\u002Fspan\u003E\u003Cspan class=\"p\"\u003E,\u003C\u002Fspan\u003E \u003Cspan class=\"n\"\u003Ew_pad\u003C\u002Fspan\u003E\u003Cspan class=\"o\"\u003E=\u003C\u002Fspan\u003E\u003Cspan class=\"mi\"\u003E0\u003C\u002Fspan\u003E\u003Cspan class=\"p\"\u003E,\u003C\u002Fspan\u003E \u003Cspan class=\"n\"\u003Eh_pad\u003C\u002Fspan\u003E\u003Cspan class=\"o\"\u003E=\u003C\u002Fspan\u003E\u003Cspan class=\"mi\"\u003E2\u003C\u002Fspan\u003E\u003Cspan class=\"p\"\u003E)\u003C\u002Fspan\u003E\n\n\u003Cspan class=\"k\"\u003Efor\u003C\u002Fspan\u003E \u003Cspan class=\"n\"\u003Ei\u003C\u002Fspan\u003E\u003Cspan class=\"p\"\u003E,\u003C\u002Fspan\u003E \u003Cspan class=\"n\"\u003Eax\u003C\u002Fspan\u003E \u003Cspan class=\"ow\"\u003Ein\u003C\u002Fspan\u003E \u003Cspan class=\"nb\"\u003Ezip\u003C\u002Fspan\u003E\u003Cspan class=\"p\"\u003E(\u003C\u002Fspan\u003E\u003Cspan class=\"nb\"\u003Erange\u003C\u002Fspan\u003E\u003Cspan class=\"p\"\u003E(\u003C\u002Fspan\u003E\u003Cspan class=\"mi\"\u003E4\u003C\u002Fspan\u003E\u003Cspan class=\"p\"\u003E),\u003C\u002Fspan\u003E \u003Cspan class=\"n\"\u003Eaxs\u003C\u002Fspan\u003E\u003Cspan class=\"p\"\u003E):\u003C\u002Fspan\u003E\n    \u003Cspan class=\"n\"\u003Eax\u003C\u002Fspan\u003E\u003Cspan class=\"o\"\u003E.\u003C\u002Fspan\u003E\u003Cspan class=\"n\"\u003Eplot\u003C\u002Fspan\u003E\u003Cspan class=\"p\"\u003E(\u003C\u002Fspan\u003E\u003Cspan class=\"n\"\u003Ex\u003C\u002Fspan\u003E\u003Cspan class=\"p\"\u003E,\u003C\u002Fspan\u003E \u003Cspan class=\"n\"\u003Ef_exa\u003C\u002Fspan\u003E\u003Cspan class=\"p\"\u003E,\u003C\u002Fspan\u003E \u003Cspan class=\"n\"\u003Ecolor\u003C\u002Fspan\u003E\u003Cspan class=\"o\"\u003E=\u003C\u002Fspan\u003E\u003Cspan class=\"s1\"\u003E&#39;blue&#39;\u003C\u002Fspan\u003E\u003Cspan class=\"p\"\u003E)\u003C\u002Fspan\u003E\n    \u003Cspan class=\"n\"\u003Exh\u003C\u002Fspan\u003E\u003Cspan class=\"p\"\u003E,\u003C\u002Fspan\u003E \u003Cspan class=\"n\"\u003EfprimF\u003C\u002Fspan\u003E \u003Cspan class=\"o\"\u003E=\u003C\u002Fspan\u003E \u003Cspan class=\"n\"\u003EtestDiscretization\u003C\u002Fspan\u003E\u003Cspan class=\"p\"\u003E(\u003C\u002Fspan\u003E\u003Cspan class=\"n\"\u003Eh\u003C\u002Fspan\u003E\u003Cspan class=\"o\"\u003E=\u003C\u002Fspan\u003E\u003Cspan class=\"mf\"\u003E0.2\u003C\u002Fspan\u003E\u003Cspan class=\"o\"\u003E*\u003C\u002Fspan\u003E\u003Cspan class=\"mf\"\u003E0.5\u003C\u002Fspan\u003E\u003Cspan class=\"o\"\u003E**\u003C\u002Fspan\u003E\u003Cspan class=\"n\"\u003Ei\u003C\u002Fspan\u003E\u003Cspan class=\"p\"\u003E)\u003C\u002Fspan\u003E\n    \u003Cspan class=\"n\"\u003Eax\u003C\u002Fspan\u003E\u003Cspan class=\"o\"\u003E.\u003C\u002Fspan\u003E\u003Cspan class=\"n\"\u003Eplot\u003C\u002Fspan\u003E\u003Cspan class=\"p\"\u003E(\u003C\u002Fspan\u003E\u003Cspan class=\"n\"\u003Exh\u003C\u002Fspan\u003E\u003Cspan class=\"p\"\u003E,\u003C\u002Fspan\u003E \u003Cspan class=\"n\"\u003EfprimF\u003C\u002Fspan\u003E\u003Cspan class=\"p\"\u003E,\u003C\u002Fspan\u003E \u003Cspan class=\"s1\"\u003E&#39;ro&#39;\u003C\u002Fspan\u003E\u003Cspan class=\"p\"\u003E,\u003C\u002Fspan\u003E \u003Cspan class=\"n\"\u003Eclip_on\u003C\u002Fspan\u003E\u003Cspan class=\"o\"\u003E=\u003C\u002Fspan\u003E\u003Cspan class=\"bp\"\u003EFalse\u003C\u002Fspan\u003E\u003Cspan class=\"p\"\u003E)\u003C\u002Fspan\u003E\n    \u003Cspan class=\"n\"\u003Eax\u003C\u002Fspan\u003E\u003Cspan class=\"o\"\u003E.\u003C\u002Fspan\u003E\u003Cspan class=\"n\"\u003Eset_xlim\u003C\u002Fspan\u003E\u003Cspan class=\"p\"\u003E([\u003C\u002Fspan\u003E\u003Cspan class=\"mi\"\u003E0\u003C\u002Fspan\u003E\u003Cspan class=\"p\"\u003E,\u003C\u002Fspan\u003E \u003Cspan class=\"mi\"\u003E2\u003C\u002Fspan\u003E\u003Cspan class=\"p\"\u003E])\u003C\u002Fspan\u003E\n    \u003Cspan class=\"n\"\u003Eax\u003C\u002Fspan\u003E\u003Cspan class=\"o\"\u003E.\u003C\u002Fspan\u003E\u003Cspan class=\"n\"\u003Eset_ylim\u003C\u002Fspan\u003E\u003Cspan class=\"p\"\u003E([\u003C\u002Fspan\u003E\u003Cspan class=\"mi\"\u003E1\u003C\u002Fspan\u003E\u003Cspan class=\"p\"\u003E,\u003C\u002Fspan\u003E\u003Cspan class=\"nb\"\u003Emax\u003C\u002Fspan\u003E\u003Cspan class=\"p\"\u003E(\u003C\u002Fspan\u003E\u003Cspan class=\"n\"\u003EfprimF\u003C\u002Fspan\u003E\u003Cspan class=\"p\"\u003E)])\u003C\u002Fspan\u003E\n    \u003Cspan class=\"n\"\u003Eax\u003C\u002Fspan\u003E\u003Cspan class=\"o\"\u003E.\u003C\u002Fspan\u003E\u003Cspan class=\"n\"\u003Eset_xlabel\u003C\u002Fspan\u003E\u003Cspan class=\"p\"\u003E(\u003C\u002Fspan\u003E\u003Cspan class=\"sa\"\u003Er\u003C\u002Fspan\u003E\u003Cspan class=\"s1\"\u003E&#39;$x$&#39;\u003C\u002Fspan\u003E\u003Cspan class=\"p\"\u003E)\u003C\u002Fspan\u003E\n    \u003Cspan class=\"n\"\u003Eax\u003C\u002Fspan\u003E\u003Cspan class=\"o\"\u003E.\u003C\u002Fspan\u003E\u003Cspan class=\"n\"\u003Eset_ylabel\u003C\u002Fspan\u003E\u003Cspan class=\"p\"\u003E(\u003C\u002Fspan\u003E\u003Cspan class=\"s1\"\u003E&#39;Derivatives&#39;\u003C\u002Fspan\u003E\u003Cspan class=\"p\"\u003E)\u003C\u002Fspan\u003E\n    \u003Cspan class=\"n\"\u003Eax\u003C\u002Fspan\u003E\u003Cspan class=\"o\"\u003E.\u003C\u002Fspan\u003E\u003Cspan class=\"n\"\u003Elegend\u003C\u002Fspan\u003E\u003Cspan class=\"p\"\u003E([\u003C\u002Fspan\u003E\u003Cspan class=\"s1\"\u003E&#39;Exact Derivatives&#39;\u003C\u002Fspan\u003E\u003Cspan class=\"p\"\u003E,\u003C\u002Fspan\u003E\u003Cspan class=\"s1\"\u003E&#39;Calculated Derivatives&#39;\u003C\u002Fspan\u003E\u003Cspan class=\"p\"\u003E])\u003C\u002Fspan\u003E\u003C\u002Fcode\u003E\u003C\u002Fpre\u003E\u003C\u002Fdiv\u003E\u003Cfigure data-size=\"normal\"\u003E\u003Cnoscript\u003E\u003Cimg src=\"https:\u002F\u002Fpic3.zhimg.com\u002Fv2-cc0367b1c9d22e30f4218ec1b969b15a_b.jpg\" data-caption=\"\" data-size=\"normal\" data-rawwidth=\"447\" data-rawheight=\"459\" class=\"origin_image zh-lightbox-thumb\" width=\"447\" data-original=\"https:\u002F\u002Fpic3.zhimg.com\u002Fv2-cc0367b1c9d22e30f4218ec1b969b15a_r.jpg\"\u002F\u003E\u003C\u002Fnoscript\u003E\u003Cimg src=\"data:image\u002Fsvg+xml;utf8,&lt;svg xmlns=&#39;http:\u002F\u002Fwww.w3.org\u002F2000\u002Fsvg&#39; width=&#39;447&#39; height=&#39;459&#39;&gt;&lt;\u002Fsvg&gt;\" data-caption=\"\" data-size=\"normal\" data-rawwidth=\"447\" data-rawheight=\"459\" class=\"origin_image zh-lightbox-thumb lazy\" width=\"447\" data-original=\"https:\u002F\u002Fpic3.zhimg.com\u002Fv2-cc0367b1c9d22e30f4218ec1b969b15a_r.jpg\" data-actualsrc=\"https:\u002F\u002Fpic3.zhimg.com\u002Fv2-cc0367b1c9d22e30f4218ec1b969b15a_b.jpg\"\u002F\u003E\u003C\u002Ffigure\u003E\u003Cp\u003E可以观察到我们用前向差分得到的导数随着 \u003Cimg src=\"https:\u002F\u002Fwww.zhihu.com\u002Fequation?tex=h\" alt=\"h\" eeimg=\"1\"\u002F\u003E 的减小,逐渐靠近真实值.\u003C\u002Fp\u003E\u003Ch2\u003E小结:舍入误差 vs. 离散化误差\u003C\u002Fh2\u003E\u003Cp\u003E我们通过下面这个例子来分析和对比这两个误差.\u003C\u002Fp\u003E\u003Ch3\u003E例子6: 对比两个误差\u003C\u002Fh3\u003E\u003Cp\u003E继续例子4中的函数 \u003Cimg src=\"https:\u002F\u002Fwww.zhihu.com\u002Fequation?tex=f%28x%29%3De%5Ex\" alt=\"f(x)=e^x\" eeimg=\"1\"\u002F\u003E 和它的导数 \u003Cimg src=\"https:\u002F\u002Fwww.zhihu.com\u002Fequation?tex=f%27%28x%29%3De%5Ex\" alt=\"f&#39;(x)=e^x\" eeimg=\"1\"\u002F\u003E .取 \u003Cimg src=\"https:\u002F\u002Fwww.zhihu.com\u002Fequation?tex=x%3D1\" alt=\"x=1\" eeimg=\"1\"\u002F\u003E ,分别用前面提到的三种有限微分方法求出数值导数,并与真实值比较计算出相对误差.\u003C\u002Fp\u003E\u003Cp\u003E这里我们来观察,当 \u003Cimg src=\"https:\u002F\u002Fwww.zhihu.com\u002Fequation?tex=h\" alt=\"h\" eeimg=\"1\"\u002F\u003E 取不同值(\u003Cimg src=\"https:\u002F\u002Fwww.zhihu.com\u002Fequation?tex=10%5E%7B-1%7D\" alt=\"10^{-1}\" eeimg=\"1\"\u002F\u003E 至 \u003Cimg src=\"https:\u002F\u002Fwww.zhihu.com\u002Fequation?tex=10%5E%7B-15%7D\" alt=\"10^{-15}\" eeimg=\"1\"\u002F\u003E)的时候,相对误差的变化情况.\u003C\u002Fp\u003E\u003Cdiv class=\"highlight\"\u003E\u003Cpre\u003E\u003Ccode class=\"language-python\"\u003E\u003Cspan class=\"kn\"\u003Eimport\u003C\u002Fspan\u003E \u003Cspan class=\"nn\"\u003Ematplotlib.pyplot\u003C\u002Fspan\u003E \u003Cspan class=\"kn\"\u003Eas\u003C\u002Fspan\u003E \u003Cspan class=\"nn\"\u003Eplt\u003C\u002Fspan\u003E\n\u003Cspan class=\"kn\"\u003Eimport\u003C\u002Fspan\u003E \u003Cspan class=\"nn\"\u003Enumpy\u003C\u002Fspan\u003E \u003Cspan class=\"kn\"\u003Eas\u003C\u002Fspan\u003E \u003Cspan class=\"nn\"\u003Enp\u003C\u002Fspan\u003E\n\n\u003Cspan class=\"k\"\u003Edef\u003C\u002Fspan\u003E \u003Cspan class=\"nf\"\u003EForwardDiff\u003C\u002Fspan\u003E\u003Cspan class=\"p\"\u003E(\u003C\u002Fspan\u003E\u003Cspan class=\"n\"\u003Efx\u003C\u002Fspan\u003E\u003Cspan class=\"p\"\u003E,\u003C\u002Fspan\u003E \u003Cspan class=\"n\"\u003Ex\u003C\u002Fspan\u003E\u003Cspan class=\"p\"\u003E,\u003C\u002Fspan\u003E \u003Cspan class=\"n\"\u003Eh\u003C\u002Fspan\u003E\u003Cspan class=\"o\"\u003E=\u003C\u002Fspan\u003E\u003Cspan class=\"mf\"\u003E0.001\u003C\u002Fspan\u003E\u003Cspan class=\"p\"\u003E):\u003C\u002Fspan\u003E\n    \u003Cspan class=\"c1\"\u003E# Forward difference\u003C\u002Fspan\u003E\n    \u003Cspan class=\"k\"\u003Ereturn\u003C\u002Fspan\u003E \u003Cspan class=\"p\"\u003E(\u003C\u002Fspan\u003E\u003Cspan class=\"n\"\u003Efx\u003C\u002Fspan\u003E\u003Cspan class=\"p\"\u003E(\u003C\u002Fspan\u003E\u003Cspan class=\"n\"\u003Ex\u003C\u002Fspan\u003E\u003Cspan class=\"o\"\u003E+\u003C\u002Fspan\u003E\u003Cspan class=\"n\"\u003Eh\u003C\u002Fspan\u003E\u003Cspan class=\"p\"\u003E)\u003C\u002Fspan\u003E \u003Cspan class=\"o\"\u003E-\u003C\u002Fspan\u003E \u003Cspan class=\"n\"\u003Efx\u003C\u002Fspan\u003E\u003Cspan class=\"p\"\u003E(\u003C\u002Fspan\u003E\u003Cspan class=\"n\"\u003Ex\u003C\u002Fspan\u003E\u003Cspan class=\"p\"\u003E))\u003C\u002Fspan\u003E\u003Cspan class=\"o\"\u003E\u002F\u003C\u002Fspan\u003E\u003Cspan class=\"n\"\u003Eh\u003C\u002Fspan\u003E\n\n\u003Cspan class=\"k\"\u003Edef\u003C\u002Fspan\u003E \u003Cspan class=\"nf\"\u003ECentralDiff\u003C\u002Fspan\u003E\u003Cspan class=\"p\"\u003E(\u003C\u002Fspan\u003E\u003Cspan class=\"n\"\u003Efx\u003C\u002Fspan\u003E\u003Cspan class=\"p\"\u003E,\u003C\u002Fspan\u003E \u003Cspan class=\"n\"\u003Ex\u003C\u002Fspan\u003E\u003Cspan class=\"p\"\u003E,\u003C\u002Fspan\u003E \u003Cspan class=\"n\"\u003Eh\u003C\u002Fspan\u003E\u003Cspan class=\"o\"\u003E=\u003C\u002Fspan\u003E\u003Cspan class=\"mf\"\u003E0.001\u003C\u002Fspan\u003E\u003Cspan class=\"p\"\u003E):\u003C\u002Fspan\u003E\n    \u003Cspan class=\"c1\"\u003E# Central difference\u003C\u002Fspan\u003E\n    \u003Cspan class=\"k\"\u003Ereturn\u003C\u002Fspan\u003E \u003Cspan class=\"p\"\u003E(\u003C\u002Fspan\u003E\u003Cspan class=\"n\"\u003Efx\u003C\u002Fspan\u003E\u003Cspan class=\"p\"\u003E(\u003C\u002Fspan\u003E\u003Cspan class=\"n\"\u003Ex\u003C\u002Fspan\u003E\u003Cspan class=\"o\"\u003E+\u003C\u002Fspan\u003E\u003Cspan class=\"n\"\u003Eh\u003C\u002Fspan\u003E\u003Cspan class=\"p\"\u003E)\u003C\u002Fspan\u003E \u003Cspan class=\"o\"\u003E-\u003C\u002Fspan\u003E \u003Cspan class=\"n\"\u003Efx\u003C\u002Fspan\u003E\u003Cspan class=\"p\"\u003E(\u003C\u002Fspan\u003E\u003Cspan class=\"n\"\u003Ex\u003C\u002Fspan\u003E\u003Cspan class=\"o\"\u003E-\u003C\u002Fspan\u003E\u003Cspan class=\"n\"\u003Eh\u003C\u002Fspan\u003E\u003Cspan class=\"p\"\u003E))\u003C\u002Fspan\u003E\u003Cspan class=\"o\"\u003E\u002F\u003C\u002Fspan\u003E\u003Cspan class=\"n\"\u003Eh\u003C\u002Fspan\u003E\u003Cspan class=\"o\"\u003E*\u003C\u002Fspan\u003E\u003Cspan class=\"mf\"\u003E0.5\u003C\u002Fspan\u003E\n\n\u003Cspan class=\"k\"\u003Edef\u003C\u002Fspan\u003E \u003Cspan class=\"nf\"\u003EFivePointsDiff\u003C\u002Fspan\u003E\u003Cspan class=\"p\"\u003E(\u003C\u002Fspan\u003E\u003Cspan class=\"n\"\u003Efx\u003C\u002Fspan\u003E\u003Cspan class=\"p\"\u003E,\u003C\u002Fspan\u003E \u003Cspan class=\"n\"\u003Ex\u003C\u002Fspan\u003E\u003Cspan class=\"p\"\u003E,\u003C\u002Fspan\u003E \u003Cspan class=\"n\"\u003Eh\u003C\u002Fspan\u003E\u003Cspan class=\"o\"\u003E=\u003C\u002Fspan\u003E\u003Cspan class=\"mf\"\u003E0.001\u003C\u002Fspan\u003E\u003Cspan class=\"p\"\u003E):\u003C\u002Fspan\u003E\n    \u003Cspan class=\"c1\"\u003E# Five points difference \u003C\u002Fspan\u003E\n    \u003Cspan class=\"k\"\u003Ereturn\u003C\u002Fspan\u003E \u003Cspan class=\"p\"\u003E(\u003C\u002Fspan\u003E\u003Cspan class=\"o\"\u003E-\u003C\u002Fspan\u003E\u003Cspan class=\"n\"\u003Efx\u003C\u002Fspan\u003E\u003Cspan class=\"p\"\u003E(\u003C\u002Fspan\u003E\u003Cspan class=\"n\"\u003Ex\u003C\u002Fspan\u003E\u003Cspan class=\"o\"\u003E+\u003C\u002Fspan\u003E\u003Cspan class=\"mi\"\u003E2\u003C\u002Fspan\u003E\u003Cspan class=\"o\"\u003E*\u003C\u002Fspan\u003E\u003Cspan class=\"n\"\u003Eh\u003C\u002Fspan\u003E\u003Cspan class=\"p\"\u003E)\u003C\u002Fspan\u003E \u003Cspan class=\"o\"\u003E+\u003C\u002Fspan\u003E \u003Cspan class=\"mi\"\u003E8\u003C\u002Fspan\u003E\u003Cspan class=\"o\"\u003E*\u003C\u002Fspan\u003E\u003Cspan class=\"n\"\u003Efx\u003C\u002Fspan\u003E\u003Cspan class=\"p\"\u003E(\u003C\u002Fspan\u003E\u003Cspan class=\"n\"\u003Ex\u003C\u002Fspan\u003E\u003Cspan class=\"o\"\u003E+\u003C\u002Fspan\u003E\u003Cspan class=\"n\"\u003Eh\u003C\u002Fspan\u003E\u003Cspan class=\"p\"\u003E)\u003C\u002Fspan\u003E \u003Cspan class=\"o\"\u003E-\u003C\u002Fspan\u003E \u003Cspan class=\"mi\"\u003E8\u003C\u002Fspan\u003E\u003Cspan class=\"o\"\u003E*\u003C\u002Fspan\u003E\u003Cspan class=\"n\"\u003Efx\u003C\u002Fspan\u003E\u003Cspan class=\"p\"\u003E(\u003C\u002Fspan\u003E\u003Cspan class=\"n\"\u003Ex\u003C\u002Fspan\u003E\u003Cspan class=\"o\"\u003E-\u003C\u002Fspan\u003E\u003Cspan class=\"n\"\u003Eh\u003C\u002Fspan\u003E\u003Cspan class=\"p\"\u003E)\u003C\u002Fspan\u003E \u003Cspan class=\"o\"\u003E+\u003C\u002Fspan\u003E \u003Cspan class=\"n\"\u003Efx\u003C\u002Fspan\u003E\u003Cspan class=\"p\"\u003E(\u003C\u002Fspan\u003E\u003Cspan class=\"n\"\u003Ex\u003C\u002Fspan\u003E\u003Cspan class=\"o\"\u003E-\u003C\u002Fspan\u003E\u003Cspan class=\"mi\"\u003E2\u003C\u002Fspan\u003E\u003Cspan class=\"o\"\u003E*\u003C\u002Fspan\u003E\u003Cspan class=\"n\"\u003Eh\u003C\u002Fspan\u003E\u003Cspan class=\"p\"\u003E))\u003C\u002Fspan\u003E \u003Cspan class=\"o\"\u003E\u002F\u003C\u002Fspan\u003E \u003Cspan class=\"p\"\u003E(\u003C\u002Fspan\u003E\u003Cspan class=\"mf\"\u003E12.0\u003C\u002Fspan\u003E\u003Cspan class=\"o\"\u003E*\u003C\u002Fspan\u003E\u003Cspan class=\"n\"\u003Eh\u003C\u002Fspan\u003E\u003Cspan class=\"p\"\u003E)\u003C\u002Fspan\u003E\n\n\u003Cspan class=\"c1\"\u003E# choose h from 0.1 to 10^-t, t&gt;=2\u003C\u002Fspan\u003E\n\u003Cspan class=\"n\"\u003Et\u003C\u002Fspan\u003E \u003Cspan class=\"o\"\u003E=\u003C\u002Fspan\u003E \u003Cspan class=\"mi\"\u003E15\u003C\u002Fspan\u003E\n\u003Cspan class=\"n\"\u003Ehx\u003C\u002Fspan\u003E \u003Cspan class=\"o\"\u003E=\u003C\u002Fspan\u003E \u003Cspan class=\"mi\"\u003E10\u003C\u002Fspan\u003E\u003Cspan class=\"o\"\u003E**\u003C\u002Fspan\u003E\u003Cspan class=\"n\"\u003Enp\u003C\u002Fspan\u003E\u003Cspan class=\"o\"\u003E.\u003C\u002Fspan\u003E\u003Cspan class=\"n\"\u003Elinspace\u003C\u002Fspan\u003E\u003Cspan class=\"p\"\u003E(\u003C\u002Fspan\u003E\u003Cspan class=\"o\"\u003E-\u003C\u002Fspan\u003E\u003Cspan class=\"mi\"\u003E1\u003C\u002Fspan\u003E\u003Cspan class=\"p\"\u003E,\u003C\u002Fspan\u003E\u003Cspan class=\"o\"\u003E-\u003C\u002Fspan\u003E\u003Cspan class=\"n\"\u003Et\u003C\u002Fspan\u003E\u003Cspan class=\"p\"\u003E,\u003C\u002Fspan\u003E \u003Cspan class=\"mi\"\u003E30\u003C\u002Fspan\u003E\u003Cspan class=\"p\"\u003E)\u003C\u002Fspan\u003E\n\n\u003Cspan class=\"c1\"\u003E# The exact derivative at x=1\u003C\u002Fspan\u003E\n\u003Cspan class=\"n\"\u003Ex0\u003C\u002Fspan\u003E \u003Cspan class=\"o\"\u003E=\u003C\u002Fspan\u003E \u003Cspan class=\"mi\"\u003E1\u003C\u002Fspan\u003E\n\u003Cspan class=\"n\"\u003EfprimExact\u003C\u002Fspan\u003E \u003Cspan class=\"o\"\u003E=\u003C\u002Fspan\u003E \u003Cspan class=\"n\"\u003Enp\u003C\u002Fspan\u003E\u003Cspan class=\"o\"\u003E.\u003C\u002Fspan\u003E\u003Cspan class=\"n\"\u003Eexp\u003C\u002Fspan\u003E\u003Cspan class=\"p\"\u003E(\u003C\u002Fspan\u003E\u003Cspan class=\"mi\"\u003E1\u003C\u002Fspan\u003E\u003Cspan class=\"p\"\u003E)\u003C\u002Fspan\u003E\n\n\u003Cspan class=\"c1\"\u003E# Numerical derivative using the three methods\u003C\u002Fspan\u003E\n\u003Cspan class=\"n\"\u003EfprimF\u003C\u002Fspan\u003E \u003Cspan class=\"o\"\u003E=\u003C\u002Fspan\u003E \u003Cspan class=\"n\"\u003EForwardDiff\u003C\u002Fspan\u003E\u003Cspan class=\"p\"\u003E(\u003C\u002Fspan\u003E\u003Cspan class=\"n\"\u003Enp\u003C\u002Fspan\u003E\u003Cspan class=\"o\"\u003E.\u003C\u002Fspan\u003E\u003Cspan class=\"n\"\u003Eexp\u003C\u002Fspan\u003E\u003Cspan class=\"p\"\u003E,\u003C\u002Fspan\u003E \u003Cspan class=\"n\"\u003Ex0\u003C\u002Fspan\u003E\u003Cspan class=\"p\"\u003E,\u003C\u002Fspan\u003E \u003Cspan class=\"n\"\u003Ehx\u003C\u002Fspan\u003E\u003Cspan class=\"p\"\u003E)\u003C\u002Fspan\u003E\n\u003Cspan class=\"n\"\u003EfprimC\u003C\u002Fspan\u003E \u003Cspan class=\"o\"\u003E=\u003C\u002Fspan\u003E \u003Cspan class=\"n\"\u003ECentralDiff\u003C\u002Fspan\u003E\u003Cspan class=\"p\"\u003E(\u003C\u002Fspan\u003E\u003Cspan class=\"n\"\u003Enp\u003C\u002Fspan\u003E\u003Cspan class=\"o\"\u003E.\u003C\u002Fspan\u003E\u003Cspan class=\"n\"\u003Eexp\u003C\u002Fspan\u003E\u003Cspan class=\"p\"\u003E,\u003C\u002Fspan\u003E \u003Cspan class=\"n\"\u003Ex0\u003C\u002Fspan\u003E\u003Cspan class=\"p\"\u003E,\u003C\u002Fspan\u003E \u003Cspan class=\"n\"\u003Ehx\u003C\u002Fspan\u003E\u003Cspan class=\"p\"\u003E)\u003C\u002Fspan\u003E\n\u003Cspan class=\"n\"\u003Efprim5\u003C\u002Fspan\u003E \u003Cspan class=\"o\"\u003E=\u003C\u002Fspan\u003E \u003Cspan class=\"n\"\u003EFivePointsDiff\u003C\u002Fspan\u003E\u003Cspan class=\"p\"\u003E(\u003C\u002Fspan\u003E\u003Cspan class=\"n\"\u003Enp\u003C\u002Fspan\u003E\u003Cspan class=\"o\"\u003E.\u003C\u002Fspan\u003E\u003Cspan class=\"n\"\u003Eexp\u003C\u002Fspan\u003E\u003Cspan class=\"p\"\u003E,\u003C\u002Fspan\u003E \u003Cspan class=\"n\"\u003Ex0\u003C\u002Fspan\u003E\u003Cspan class=\"p\"\u003E,\u003C\u002Fspan\u003E \u003Cspan class=\"n\"\u003Ehx\u003C\u002Fspan\u003E\u003Cspan class=\"p\"\u003E)\u003C\u002Fspan\u003E\n\n\u003Cspan class=\"c1\"\u003E# Relative error\u003C\u002Fspan\u003E\n\u003Cspan class=\"n\"\u003EfelF\u003C\u002Fspan\u003E \u003Cspan class=\"o\"\u003E=\u003C\u002Fspan\u003E \u003Cspan class=\"nb\"\u003Eabs\u003C\u002Fspan\u003E\u003Cspan class=\"p\"\u003E(\u003C\u002Fspan\u003E\u003Cspan class=\"n\"\u003EfprimExact\u003C\u002Fspan\u003E \u003Cspan class=\"o\"\u003E-\u003C\u002Fspan\u003E \u003Cspan class=\"n\"\u003EfprimF\u003C\u002Fspan\u003E\u003Cspan class=\"p\"\u003E)\u003C\u002Fspan\u003E\u003Cspan class=\"o\"\u003E\u002F\u003C\u002Fspan\u003E\u003Cspan class=\"nb\"\u003Eabs\u003C\u002Fspan\u003E\u003Cspan class=\"p\"\u003E(\u003C\u002Fspan\u003E\u003Cspan class=\"n\"\u003EfprimExact\u003C\u002Fspan\u003E\u003Cspan class=\"p\"\u003E)\u003C\u002Fspan\u003E\n\u003Cspan class=\"n\"\u003EfelC\u003C\u002Fspan\u003E \u003Cspan class=\"o\"\u003E=\u003C\u002Fspan\u003E \u003Cspan class=\"nb\"\u003Eabs\u003C\u002Fspan\u003E\u003Cspan class=\"p\"\u003E(\u003C\u002Fspan\u003E\u003Cspan class=\"n\"\u003EfprimExact\u003C\u002Fspan\u003E \u003Cspan class=\"o\"\u003E-\u003C\u002Fspan\u003E \u003Cspan class=\"n\"\u003EfprimC\u003C\u002Fspan\u003E\u003Cspan class=\"p\"\u003E)\u003C\u002Fspan\u003E\u003Cspan class=\"o\"\u003E\u002F\u003C\u002Fspan\u003E\u003Cspan class=\"nb\"\u003Eabs\u003C\u002Fspan\u003E\u003Cspan class=\"p\"\u003E(\u003C\u002Fspan\u003E\u003Cspan class=\"n\"\u003EfprimExact\u003C\u002Fspan\u003E\u003Cspan class=\"p\"\u003E)\u003C\u002Fspan\u003E\n\u003Cspan class=\"n\"\u003Efel5\u003C\u002Fspan\u003E \u003Cspan class=\"o\"\u003E=\u003C\u002Fspan\u003E \u003Cspan class=\"nb\"\u003Eabs\u003C\u002Fspan\u003E\u003Cspan class=\"p\"\u003E(\u003C\u002Fspan\u003E\u003Cspan class=\"n\"\u003EfprimExact\u003C\u002Fspan\u003E \u003Cspan class=\"o\"\u003E-\u003C\u002Fspan\u003E \u003Cspan class=\"n\"\u003Efprim5\u003C\u002Fspan\u003E\u003Cspan class=\"p\"\u003E)\u003C\u002Fspan\u003E\u003Cspan class=\"o\"\u003E\u002F\u003C\u002Fspan\u003E\u003Cspan class=\"nb\"\u003Eabs\u003C\u002Fspan\u003E\u003Cspan class=\"p\"\u003E(\u003C\u002Fspan\u003E\u003Cspan class=\"n\"\u003EfprimExact\u003C\u002Fspan\u003E\u003Cspan class=\"p\"\u003E)\u003C\u002Fspan\u003E\n\n\u003Cspan class=\"c1\"\u003E# Plot\u003C\u002Fspan\u003E\n\u003Cspan class=\"n\"\u003Efig\u003C\u002Fspan\u003E\u003Cspan class=\"p\"\u003E,\u003C\u002Fspan\u003E \u003Cspan class=\"n\"\u003Eax\u003C\u002Fspan\u003E \u003Cspan class=\"o\"\u003E=\u003C\u002Fspan\u003E \u003Cspan class=\"n\"\u003Eplt\u003C\u002Fspan\u003E\u003Cspan class=\"o\"\u003E.\u003C\u002Fspan\u003E\u003Cspan class=\"n\"\u003Esubplots\u003C\u002Fspan\u003E\u003Cspan class=\"p\"\u003E(\u003C\u002Fspan\u003E\u003Cspan class=\"mi\"\u003E1\u003C\u002Fspan\u003E\u003Cspan class=\"p\"\u003E)\u003C\u002Fspan\u003E\n\u003Cspan class=\"n\"\u003Eax\u003C\u002Fspan\u003E\u003Cspan class=\"o\"\u003E.\u003C\u002Fspan\u003E\u003Cspan class=\"n\"\u003Eloglog\u003C\u002Fspan\u003E\u003Cspan class=\"p\"\u003E(\u003C\u002Fspan\u003E\u003Cspan class=\"n\"\u003Ehx\u003C\u002Fspan\u003E\u003Cspan class=\"p\"\u003E,\u003C\u002Fspan\u003E \u003Cspan class=\"n\"\u003EfelF\u003C\u002Fspan\u003E\u003Cspan class=\"p\"\u003E)\u003C\u002Fspan\u003E\n\u003Cspan class=\"n\"\u003Eax\u003C\u002Fspan\u003E\u003Cspan class=\"o\"\u003E.\u003C\u002Fspan\u003E\u003Cspan class=\"n\"\u003Eloglog\u003C\u002Fspan\u003E\u003Cspan class=\"p\"\u003E(\u003C\u002Fspan\u003E\u003Cspan class=\"n\"\u003Ehx\u003C\u002Fspan\u003E\u003Cspan class=\"p\"\u003E,\u003C\u002Fspan\u003E \u003Cspan class=\"n\"\u003EfelC\u003C\u002Fspan\u003E\u003Cspan class=\"p\"\u003E)\u003C\u002Fspan\u003E\n\u003Cspan class=\"n\"\u003Eax\u003C\u002Fspan\u003E\u003Cspan class=\"o\"\u003E.\u003C\u002Fspan\u003E\u003Cspan class=\"n\"\u003Eloglog\u003C\u002Fspan\u003E\u003Cspan class=\"p\"\u003E(\u003C\u002Fspan\u003E\u003Cspan class=\"n\"\u003Ehx\u003C\u002Fspan\u003E\u003Cspan class=\"p\"\u003E,\u003C\u002Fspan\u003E \u003Cspan class=\"n\"\u003Efel5\u003C\u002Fspan\u003E\u003Cspan class=\"p\"\u003E)\u003C\u002Fspan\u003E\n\u003Cspan class=\"n\"\u003Eax\u003C\u002Fspan\u003E\u003Cspan class=\"o\"\u003E.\u003C\u002Fspan\u003E\u003Cspan class=\"n\"\u003Eautoscale\u003C\u002Fspan\u003E\u003Cspan class=\"p\"\u003E(\u003C\u002Fspan\u003E\u003Cspan class=\"n\"\u003Eenable\u003C\u002Fspan\u003E\u003Cspan class=\"o\"\u003E=\u003C\u002Fspan\u003E\u003Cspan class=\"bp\"\u003ETrue\u003C\u002Fspan\u003E\u003Cspan class=\"p\"\u003E,\u003C\u002Fspan\u003E \u003Cspan class=\"n\"\u003Eaxis\u003C\u002Fspan\u003E\u003Cspan class=\"o\"\u003E=\u003C\u002Fspan\u003E\u003Cspan class=\"s1\"\u003E&#39;x&#39;\u003C\u002Fspan\u003E\u003Cspan class=\"p\"\u003E,\u003C\u002Fspan\u003E \u003Cspan class=\"n\"\u003Etight\u003C\u002Fspan\u003E\u003Cspan class=\"o\"\u003E=\u003C\u002Fspan\u003E\u003Cspan class=\"bp\"\u003ETrue\u003C\u002Fspan\u003E\u003Cspan class=\"p\"\u003E)\u003C\u002Fspan\u003E\n\u003Cspan class=\"n\"\u003Eax\u003C\u002Fspan\u003E\u003Cspan class=\"o\"\u003E.\u003C\u002Fspan\u003E\u003Cspan class=\"n\"\u003Eset_xlabel\u003C\u002Fspan\u003E\u003Cspan class=\"p\"\u003E(\u003C\u002Fspan\u003E\u003Cspan class=\"sa\"\u003Er\u003C\u002Fspan\u003E\u003Cspan class=\"s1\"\u003E&#39;Step length $h$&#39;\u003C\u002Fspan\u003E\u003Cspan class=\"p\"\u003E)\u003C\u002Fspan\u003E\n\u003Cspan class=\"n\"\u003Eax\u003C\u002Fspan\u003E\u003Cspan class=\"o\"\u003E.\u003C\u002Fspan\u003E\u003Cspan class=\"n\"\u003Eset_ylabel\u003C\u002Fspan\u003E\u003Cspan class=\"p\"\u003E(\u003C\u002Fspan\u003E\u003Cspan class=\"s1\"\u003E&#39;Relative error&#39;\u003C\u002Fspan\u003E\u003Cspan class=\"p\"\u003E)\u003C\u002Fspan\u003E\n\u003Cspan class=\"n\"\u003Eax\u003C\u002Fspan\u003E\u003Cspan class=\"o\"\u003E.\u003C\u002Fspan\u003E\u003Cspan class=\"n\"\u003Elegend\u003C\u002Fspan\u003E\u003Cspan class=\"p\"\u003E([\u003C\u002Fspan\u003E\u003Cspan class=\"s1\"\u003E&#39;Forward difference&#39;\u003C\u002Fspan\u003E\u003Cspan class=\"p\"\u003E,\u003C\u002Fspan\u003E\u003Cspan class=\"s1\"\u003E&#39;Central difference&#39;\u003C\u002Fspan\u003E\u003Cspan class=\"p\"\u003E,\u003C\u002Fspan\u003E \u003Cspan class=\"s1\"\u003E&#39;Five points difference&#39;\u003C\u002Fspan\u003E\u003Cspan class=\"p\"\u003E])\u003C\u002Fspan\u003E\n\u003Cspan class=\"o\"\u003E&lt;\u003C\u002Fspan\u003E\u003Cspan class=\"n\"\u003Ematplotlib\u003C\u002Fspan\u003E\u003Cspan class=\"o\"\u003E.\u003C\u002Fspan\u003E\u003Cspan class=\"n\"\u003Elegend\u003C\u002Fspan\u003E\u003Cspan class=\"o\"\u003E.\u003C\u002Fspan\u003E\u003Cspan class=\"n\"\u003ELegend\u003C\u002Fspan\u003E \u003Cspan class=\"n\"\u003Eat\u003C\u002Fspan\u003E \u003Cspan class=\"mh\"\u003E0x12066aad0\u003C\u002Fspan\u003E\u003Cspan class=\"o\"\u003E&gt;\u003C\u002Fspan\u003E\u003C\u002Fcode\u003E\u003C\u002Fpre\u003E\u003C\u002Fdiv\u003E\u003Cfigure data-size=\"normal\"\u003E\u003Cnoscript\u003E\u003Cimg src=\"https:\u002F\u002Fpic1.zhimg.com\u002Fv2-f01d1fa174061e3261b5460ec545ee60_b.jpg\" data-caption=\"\" data-size=\"normal\" data-rawwidth=\"411\" data-rawheight=\"270\" class=\"content_image\" width=\"411\"\u002F\u003E\u003C\u002Fnoscript\u003E\u003Cimg src=\"data:image\u002Fsvg+xml;utf8,&lt;svg xmlns=&#39;http:\u002F\u002Fwww.w3.org\u002F2000\u002Fsvg&#39; width=&#39;411&#39; height=&#39;270&#39;&gt;&lt;\u002Fsvg&gt;\" data-caption=\"\" data-size=\"normal\" data-rawwidth=\"411\" data-rawheight=\"270\" class=\"content_image lazy\" width=\"411\" data-actualsrc=\"https:\u002F\u002Fpic1.zhimg.com\u002Fv2-f01d1fa174061e3261b5460ec545ee60_b.jpg\"\u002F\u003E\u003C\u002Ffigure\u003E\u003Cp\u003E这是一个非常有趣的结果,我们发现\u003C\u002Fp\u003E\u003Cul\u003E\u003Cli\u003E这三种方法的相对误差,随着 \u003Cimg src=\"https:\u002F\u002Fwww.zhihu.com\u002Fequation?tex=h\" alt=\"h\" eeimg=\"1\"\u002F\u003E 的不同,呈现出两种不同的特性.  \u003C\u002Fli\u003E\u003Cli\u003E当 \u003Cimg src=\"https:\u002F\u002Fwww.zhihu.com\u002Fequation?tex=h\" alt=\"h\" eeimg=\"1\"\u002F\u003E 较大时(右半侧),误差曲线相对规则,是由离散化误差主导的. \u003C\u002Fli\u003E\u003Cli\u003E当 \u003Cimg src=\"https:\u002F\u002Fwww.zhihu.com\u002Fequation?tex=h\" alt=\"h\" eeimg=\"1\"\u002F\u003E 较小时(左半侧),误差曲线波动较大,是由舍入误差主导.\u003C\u002Fli\u003E\u003C\u002Ful\u003E\u003Cp\u003E\u003Cb\u003E注\u003C\u002Fb\u003E:舍入误差主导且随着 \u003Cimg src=\"https:\u002F\u002Fwww.zhihu.com\u002Fequation?tex=h\" alt=\"h\" eeimg=\"1\"\u002F\u003E 减小而增大的主要原因是: \u003C\u002Fp\u003E\u003Cul\u003E\u003Cli\u003E当 \u003Cimg src=\"https:\u002F\u002Fwww.zhihu.com\u002Fequation?tex=h\" alt=\"h\" eeimg=\"1\"\u002F\u003E 较小时,有限差分法的分子上相近的数相减会造成类似例子2和3中的舍入误差.  \u003C\u002Fli\u003E\u003Cli\u003E由于这个例子中的 \u003Cimg src=\"https:\u002F\u002Fwww.zhihu.com\u002Fequation?tex=f%28x%29\" alt=\"f(x)\" eeimg=\"1\"\u002F\u003E 在 \u003Cimg src=\"https:\u002F\u002Fwww.zhihu.com\u002Fequation?tex=e\" alt=\"e\" eeimg=\"1\"\u002F\u003E 附近,因此这个误差应为 \u003Cimg src=\"https:\u002F\u002Fwww.zhihu.com\u002Fequation?tex=10%5E%7B-16%7D\" alt=\"10^{-16}\" eeimg=\"1\"\u002F\u003E 左右.它除以较小的 \u003Cimg src=\"https:\u002F\u002Fwww.zhihu.com\u002Fequation?tex=h\" alt=\"h\" eeimg=\"1\"\u002F\u003E 时,就会被相应的放大,当 \u003Cimg src=\"https:\u002F\u002Fwww.zhihu.com\u002Fequation?tex=h\" alt=\"h\" eeimg=\"1\"\u002F\u003E 越小,这个舍入误差越大.\u003C\u002Fli\u003E\u003C\u002Ful\u003E\u003Cp\u003E因此,\u003Cb\u003E当我们使用上述方法时,需要注意,尽可能取\u003C\u002Fb\u003E \u003Cimg src=\"https:\u002F\u002Fwww.zhihu.com\u002Fequation?tex=h\" alt=\"h\" eeimg=\"1\"\u002F\u003E \u003Cb\u003E在误差曲线的右半侧,这样我们对于误差才有完全的控制.\u003C\u002Fb\u003E\u003C\u002Fp\u003E\u003Cp\u003E同时,我们也观察到,对于越是高阶的差分(如五点差分),它的离散化误差随 \u003Cimg src=\"https:\u002F\u002Fwww.zhihu.com\u002Fequation?tex=h\" alt=\"h\" eeimg=\"1\"\u002F\u003E 的下降速率越大,但也越早到达舍入误差的区域. 因此,当我们遇到类似问题上,应\u003Cb\u003E选择合适阶数的有限差分方法,并根据它的特性选择适合的\u003C\u002Fb\u003E \u003Cimg src=\"https:\u002F\u002Fwww.zhihu.com\u002Fequation?tex=h\" alt=\"h\" eeimg=\"1\"\u002F\u003E \u003Cb\u003E值.并不一定是越高阶的方法越好,\u003C\u002Fb\u003E \u003Cimg src=\"https:\u002F\u002Fwww.zhihu.com\u002Fequation?tex=h\" alt=\"h\" eeimg=\"1\"\u002F\u003E \u003Cb\u003E越小越好.\u003C\u002Fb\u003E\u003C\u002Fp\u003E\u003Cp class=\"ztext-empty-paragraph\"\u003E\u003Cbr\u002F\u003E\u003C\u002Fp\u003E\u003Cp\u003E本文中所用函数的完整代码见github: \u003C\u002Fp\u003E\u003Ca href=\"https:\u002F\u002Flink.zhihu.com\u002F?target=https%3A\u002F\u002Fgithub.com\u002Fenigne\u002FScientificComputingBridging\u002Fblob\u002Fmaster\u002FLab\u002FL2\u002FmeasureErrors.py\" data-draft-node=\"block\" data-draft-type=\"link-card\" class=\" wrap external\" target=\"_blank\" rel=\"nofollow noreferrer\"\u003EmeasureErrors.py\u003C\u002Fa\u003E\u003Cp\u003E\u003C\u002Fp\u003E","adminClosedComment":false,"topics":[{"url":"https:\u002F\u002Fwww.zhihu.com\u002Fapi\u002Fv4\u002Ftopics\u002F19608617","type":"topic","id":"19608617","name":"科学计算"},{"url":"https:\u002F\u002Fwww.zhihu.com\u002Fapi\u002Fv4\u002Ftopics\u002F19793027","type":"topic","id":"19793027","name":"误差"},{"url":"https:\u002F\u002Fwww.zhihu.com\u002Fapi\u002Fv4\u002Ftopics\u002F19662420","type":"topic","id":"19662420","name":"计算数学"}],"voteupCount":7,"voting":0,"heavyUpStatus":"allow_heavy_up","column":{"description":"最近因为疫情赋闲在家,准备在这个专栏整理过去几年教过的几门科学计算硕士课程,主要是把自己的讲义和课件翻译成中文.目前的坑包括:科学计算基础,有限元分析,非线性最优化.后面有可能的话还会加一些其他的小内容.","canManage":false,"intro":"系统介绍科学计算相关知识","isFollowing":false,"urlToken":"c_1226443594048942080","id":"c_1226443594048942080","articlesCount":6,"acceptSubmission":true,"title":"科学计算","url":"https:\u002F\u002Fzhuanlan.zhihu.com\u002Fc_1226443594048942080","commentPermission":"all","created":1585186939,"updated":1591293389,"imageUrl":"https:\u002F\u002Fpic1.zhimg.com\u002Fv2-73da67fd77a1b1a04136a65904c4d575_720w.jpg?source=172ae18b","author":{"isFollowed":false,"avatarUrlTemplate":"https:\u002F\u002Fpic1.zhimg.com\u002Fv2-9fd1680b3e9ea906ec038a9448a46d78.jpg?source=172ae18b","uid":"35018097295360","userType":"people","isFollowing":false,"urlToken":"c-g-2-11","id":"77fbc1230a7a1438551696a01b2f0d4b","description":"数学爱好者","name":"Gong Cheng","isAdvertiser":false,"headline":"计算科学博士","gender":1,"url":"\u002Fpeople\u002F77fbc1230a7a1438551696a01b2f0d4b","avatarUrl":"https:\u002F\u002Fpic2.zhimg.com\u002Fv2-9fd1680b3e9ea906ec038a9448a46d78_l.jpg?source=172ae18b","isOrg":false,"type":"people"},"followers":2,"type":"column"},"commentCount":1,"contributions":[{"id":23453398,"state":"accepted","type":"first_publish","column":{"description":"最近因为疫情赋闲在家,准备在这个专栏整理过去几年教过的几门科学计算硕士课程,主要是把自己的讲义和课件翻译成中文.目前的坑包括:科学计算基础,有限元分析,非线性最优化.后面有可能的话还会加一些其他的小内容.","canManage":false,"intro":"系统介绍科学计算相关知识","isFollowing":false,"urlToken":"c_1226443594048942080","id":"c_1226443594048942080","articlesCount":6,"acceptSubmission":true,"title":"科学计算","url":"https:\u002F\u002Fzhuanlan.zhihu.com\u002Fc_1226443594048942080","commentPermission":"all","created":1585186939,"updated":1591293389,"imageUrl":"https:\u002F\u002Fpic1.zhimg.com\u002Fv2-73da67fd77a1b1a04136a65904c4d575_720w.jpg?source=172ae18b","author":{"isFollowed":false,"avatarUrlTemplate":"https:\u002F\u002Fpic1.zhimg.com\u002Fv2-9fd1680b3e9ea906ec038a9448a46d78.jpg?source=172ae18b","uid":"35018097295360","userType":"people","isFollowing":false,"urlToken":"c-g-2-11","id":"77fbc1230a7a1438551696a01b2f0d4b","description":"数学爱好者","name":"Gong Cheng","isAdvertiser":false,"headline":"计算科学博士","gender":1,"url":"\u002Fpeople\u002F77fbc1230a7a1438551696a01b2f0d4b","avatarUrl":"https:\u002F\u002Fpic2.zhimg.com\u002Fv2-9fd1680b3e9ea906ec038a9448a46d78_l.jpg?source=172ae18b","isOrg":false,"type":"people"},"followers":2,"type":"column"}}],"isTitleImageFullScreen":false,"upvotedFollowees":[],"commercialInfo":{"isCommercial":false,"plugin":{}},"suggestEdit":{"status":false,"reason":"","tip":"","url":"","title":""},"reason":"","annotationAction":[],"canTip":true,"tipjarorsCount":0,"isLabeled":false,"hasPublishingDraft":false,"isFavorited":false,"favlistsCount":23,"isNormal":true,"status":0,"shareText":"科学计算基础(1)——误差 - 来自知乎专栏「科学计算」,作者: Gong Cheng https:\u002F\u002Fzhuanlan.zhihu.com\u002Fp\u002F118757498 (想看更多?下载 @知乎 App:http:\u002F\u002Fweibo.com\u002Fp\u002F100404711598 )","canComment":{"status":true,"reason":""},"mcnFpShow":-1,"isVisible":true,"isLiked":false,"likedCount":2,"visibleOnlyToAuthor":false,"hasColumn":true,"republishers":[],"isNewLinkCard":false,"emojiReaction":{"cryFaceCount":0,"cryFaceHasSet":false,"hugCount":0,"hugHasSet":false,"likeCount":2,"likeHasSet":false,"onlookerCount":0,"onlookerHasSet":false},"abParam":{"hbAnsCard":"0"}}},"columns":{"c_1226443594048942080":{"description":"最近因为疫情赋闲在家,准备在这个专栏整理过去几年教过的几门科学计算硕士课程,主要是把自己的讲义和课件翻译成中文.目前的坑包括:科学计算基础,有限元分析,非线性最优化.后面有可能的话还会加一些其他的小内容.","canManage":false,"intro":"系统介绍科学计算相关知识","isFollowing":false,"urlToken":"c_1226443594048942080","id":"c_1226443594048942080","articlesCount":6,"acceptSubmission":true,"title":"科学计算","url":"https:\u002F\u002Fzhuanlan.zhihu.com\u002Fc_1226443594048942080","commentPermission":"all","created":1585186939,"updated":1591293389,"imageUrl":"https:\u002F\u002Fpic1.zhimg.com\u002Fv2-73da67fd77a1b1a04136a65904c4d575_720w.jpg?source=172ae18b","author":{"isFollowed":false,"avatarUrlTemplate":"https:\u002F\u002Fpic1.zhimg.com\u002Fv2-9fd1680b3e9ea906ec038a9448a46d78.jpg?source=172ae18b","uid":"35018097295360","userType":"people","isFollowing":false,"urlToken":"c-g-2-11","id":"77fbc1230a7a1438551696a01b2f0d4b","description":"数学爱好者","name":"Gong Cheng","isAdvertiser":false,"headline":"计算科学博士","gender":1,"url":"\u002Fpeople\u002F77fbc1230a7a1438551696a01b2f0d4b","avatarUrl":"https:\u002F\u002Fpic2.zhimg.com\u002Fv2-9fd1680b3e9ea906ec038a9448a46d78_l.jpg?source=172ae18b","isOrg":false,"type":"people"},"followers":2,"type":"column"}},"topics":{},"roundtables":{},"favlists":{},"comments":{},"notifications":{},"ebooks":{},"activities":{},"feeds":{},"pins":{},"promotions":{},"drafts":{},"chats":{},"posts":{},"clubs":{},"clubTags":{},"zvideos":{},"zvideoContributions":{}},"currentUser":"c0d9b4bba7b55a4c382c09e4ae43102a","account":{"lockLevel":{},"unlockTicketStatus":false,"unlockTicket":null,"challenge":[],"errorStatus":false,"message":"","isFetching":false,"accountInfo":{},"urlToken":{"loading":false}},"settings":{"socialBind":null,"inboxMsg":null,"notification":{},"email":{},"privacyFlag":null,"blockedUsers":{"isFetching":false,"paging":{"pageNo":1,"pageSize":6},"data":[]},"blockedFollowees":{"isFetching":false,"paging":{"pageNo":1,"pageSize":6},"data":[]},"ignoredTopics":{"isFetching":false,"paging":{"pageNo":1,"pageSize":6},"data":[]},"restrictedTopics":null,"laboratory":{}},"notification":{},"people":{"profileStatus":{},"activitiesByUser":{},"answersByUser":{},"answersSortByVotesByUser":{},"answersIncludedByUser":{},"votedAnswersByUser":{},"thankedAnswersByUser":{},"voteAnswersByUser":{},"thankAnswersByUser":{},"topicAnswersByUser":{},"zvideosByUser":{},"articlesByUser":{},"articlesSortByVotesByUser":{},"articlesIncludedByUser":{},"pinsByUser":{},"questionsByUser":{},"commercialQuestionsByUser":{},"favlistsByUser":{},"followingByUser":{},"followersByUser":{},"mutualsByUser":{},"followingColumnsByUser":{},"followingQuestionsByUser":{},"followingFavlistsByUser":{},"followingTopicsByUser":{},"publicationsByUser":{},"columnsByUser":{},"allFavlistsByUser":{},"brands":null,"creationsByUser":{},"creationsSortByVotesByUser":{},"creationsFeed":{},"infinity":{},"batchUsers":{}},"env":{"ab":{"config":{"experiments":[{"expId":"launch-li_video-3","expPrefix":"li_video","isDynamicallyUpdated":true,"isRuntime":false,"includeTriggerInfo":false},{"expId":"launch-qa_art2qa_new-3","expPrefix":"qa_art2qa_new","isDynamicallyUpdated":true,"isRuntime":false,"includeTriggerInfo":false},{"expId":"launch-qa_cl_guest-2","expPrefix":"qa_cl_guest","isDynamicallyUpdated":true,"isRuntime":false,"includeTriggerInfo":false},{"expId":"launch-qa_column_invite-2","expPrefix":"qa_column_invite","isDynamicallyUpdated":true,"isRuntime":false,"includeTriggerInfo":false},{"expId":"launch-qa_repost-2","expPrefix":"qa_repost","isDynamicallyUpdated":true,"isRuntime":false,"includeTriggerInfo":false},{"expId":"launch-tp_zrec-8","expPrefix":"tp_zrec","isDynamicallyUpdated":true,"isRuntime":false,"includeTriggerInfo":false},{"expId":"launch-us_adjust_new-11","expPrefix":"us_adjust_new","isDynamicallyUpdated":true,"isRuntime":false,"includeTriggerInfo":false},{"expId":"launch-us_noti_count-8","expPrefix":"us_noti_count","isDynamicallyUpdated":true,"isRuntime":false,"includeTriggerInfo":false},{"expId":"launch-vd_andplay_d-2","expPrefix":"vd_andplay_d","isDynamicallyUpdated":true,"isRuntime":false,"includeTriggerInfo":false},{"expId":"launch-vd_hookupplay_an-2","expPrefix":"vd_hookupplay_an","isDynamicallyUpdated":true,"isRuntime":false,"includeTriggerInfo":false},{"expId":"launch-vd_ios_play-10","expPrefix":"vd_ios_play","isDynamicallyUpdated":true,"isRuntime":false,"includeTriggerInfo":false},{"expId":"launch-vd_timeguide-2","expPrefix":"vd_timeguide","isDynamicallyUpdated":true,"isRuntime":false,"includeTriggerInfo":false},{"expId":"launch-vd_video_replay-3","expPrefix":"vd_video_replay","isDynamicallyUpdated":true,"isRuntime":false,"includeTriggerInfo":false},{"expId":"launch-vd_zvideo_link-10","expPrefix":"vd_zvideo_link","isDynamicallyUpdated":true,"isRuntime":false,"includeTriggerInfo":false},{"expId":"launch-zanswer-2","expPrefix":"zanswer","isDynamicallyUpdated":false,"isRuntime":false,"includeTriggerInfo":false},{"expId":"launch-recnew_2th-3","expPrefix":"recnew_2th","isDynamicallyUpdated":false,"isRuntime":false,"includeTriggerInfo":false},{"expId":"se_infinity6-2","expPrefix":"se_infinity6","isDynamicallyUpdated":true,"isRuntime":false,"includeTriggerInfo":false},{"expId":"se_meta_ss-4","expPrefix":"se_meta_ss","isDynamicallyUpdated":true,"isRuntime":false,"includeTriggerInfo":false},{"expId":"qa_cdzixun-1","expPrefix":"qa_cdzixun","isDynamicallyUpdated":true,"isRuntime":false,"includeTriggerInfo":false},{"expId":"qa_np-1","expPrefix":"qa_np","isDynamicallyUpdated":true,"isRuntime":false,"includeTriggerInfo":false},{"expId":"qa_fo_recom_h-3","expPrefix":"qa_fo_recom_h","isDynamicallyUpdated":true,"isRuntime":false,"includeTriggerInfo":false},{"expId":"qa_recmessa-3","expPrefix":"qa_recmessa","isDynamicallyUpdated":true,"isRuntime":false,"includeTriggerInfo":false},{"expId":"vd_video_tab-9","expPrefix":"vd_video_tab","isDynamicallyUpdated":true,"isRuntime":false,"includeTriggerInfo":false},{"expId":"li_tp_paidanswer-10","expPrefix":"li_tp_paidanswer","isDynamicallyUpdated":true,"isRuntime":false,"includeTriggerInfo":false},{"expId":"tp_content-4","expPrefix":"tp_content","isDynamicallyUpdated":true,"isRuntime":false,"includeTriggerInfo":false},{"expId":"gw_qrlogin-1","expPrefix":"gw_qrlogin","isDynamicallyUpdated":true,"isRuntime":false,"includeTriggerInfo":false},{"expId":"xuanpinceshi-5_v3","expPrefix":"xuanpinceshi","isDynamicallyUpdated":false,"isRuntime":false,"includeTriggerInfo":false},{"expId":"office_xp-2_v2","expPrefix":"office_xp","isDynamicallyUpdated":false,"isRuntime":false,"includeTriggerInfo":false},{"expId":"xuanpinab-2_v4","expPrefix":"xuanpinab","isDynamicallyUpdated":false,"isRuntime":false,"includeTriggerInfo":false},{"expId":"pc_editorplugin-1_v5","expPrefix":"pc_editorplugin","isDynamicallyUpdated":false,"isRuntime":false,"includeTriggerInfo":false},{"expId":"meta_ebook-2_v2","expPrefix":"meta_ebook","isDynamicallyUpdated":false,"isRuntime":false,"includeTriggerInfo":false},{"expId":"webpImg-1_v3","expPrefix":"webpImg","isDynamicallyUpdated":false,"isRuntime":false,"includeTriggerInfo":false},{"expId":"vdox_cmt-2_v2","expPrefix":"vdox_cmt","isDynamicallyUpdated":false,"isRuntime":false,"includeTriggerInfo":false},{"expId":"ffzix_xgdzwz-1_v1","expPrefix":"ffzix_xgdzwz","isDynamicallyUpdated":false,"isRuntime":false,"includeTriggerInfo":false},{"expId":"hwtj_bottom_btn-3_v2","expPrefix":"hwtj_bottom_btn","isDynamicallyUpdated":false,"isRuntime":false,"includeTriggerInfo":false},{"expId":"se_cvr_boost-5_v1","expPrefix":"se_cvr_boost","isDynamicallyUpdated":false,"isRuntime":false,"includeTriggerInfo":false},{"expId":"rec_new2th-2_v4","expPrefix":"rec_new2th","isDynamicallyUpdated":false,"isRuntime":false,"includeTriggerInfo":false},{"expId":"se_zp_five-2_v1","expPrefix":"se_zp_five","isDynamicallyUpdated":false,"isRuntime":false,"includeTriggerInfo":false},{"expId":"hw_aa_30-1_v2","expPrefix":"hw_aa_30","isDynamicallyUpdated":false,"isRuntime":false,"includeTriggerInfo":false},{"expId":"hw_aa_50-2_v1","expPrefix":"hw_aa_50","isDynamicallyUpdated":false,"isRuntime":false,"includeTriggerInfo":false},{"expId":"se_filter-4_v7","expPrefix":"se_filter","isDynamicallyUpdated":false,"isRuntime":false,"includeTriggerInfo":false},{"expId":"se_sug_exp-1_v4","expPrefix":"se_sug_exp","isDynamicallyUpdated":false,"isRuntime":false,"includeTriggerInfo":false},{"expId":"ge_v_rank_v3-1_v1","expPrefix":"ge_v_rank_v3","isDynamicallyUpdated":false,"isRuntime":false,"includeTriggerInfo":false},{"expId":"phrase_recall-4_v5","expPrefix":"phrase_recall","isDynamicallyUpdated":false,"isRuntime":false,"includeTriggerInfo":false},{"expId":"test_3-2_v1","expPrefix":"test_3","isDynamicallyUpdated":false,"isRuntime":false,"includeTriggerInfo":false},{"expId":"book_kpwz-3_v1","expPrefix":"book_kpwz","isDynamicallyUpdated":false,"isRuntime":false,"includeTriggerInfo":false},{"expId":"se_mixer_roll-1_v2","expPrefix":"se_mixer_roll","isDynamicallyUpdated":false,"isRuntime":false,"includeTriggerInfo":false},{"expId":"se_cp_answer-1_v1","expPrefix":"se_cp_answer","isDynamicallyUpdated":false,"isRuntime":false,"includeTriggerInfo":false},{"expId":"se_cp_post3-2_v1","expPrefix":"se_cp_post3","isDynamicallyUpdated":false,"isRuntime":false,"includeTriggerInfo":false},{"expId":"se_resnet-3_v2","expPrefix":"se_resnet","isDynamicallyUpdated":false,"isRuntime":false,"includeTriggerInfo":false}],"params":[{"id":"se_cp_question","type":"Int","value":"0","chainId":"_gene_","layerId":"se_cp_question","key":704},{"id":"ge_dipin_pre","type":"String","value":"0","chainId":"_gene_","layerId":"gese_layer_1000","key":3124},{"id":"se_dssm","type":"Int","value":"0","chainId":"_gene_","layerId":"se_dssm","key":392},{"id":"comment","type":"Int","value":"0","chainId":"_gene_","layerId":"comment","key":644},{"id":"se_cp_answer","type":"Int","value":"0","chainId":"_gene_","layerId":"se_cp_answer","key":705},{"id":"se_warmup","type":"Int","value":"0","chainId":"_gene_","layerId":"zVTj","key":729},{"id":"web_heifetz_grow_ad","type":"String","value":"1","layerId":"webgw_layer_3"},{"id":"ge_newyanzhi","type":"String","value":"0","chainId":"_gene_","layerId":"geus_layer_1019","key":2788},{"id":"ge_newcard","type":"String","value":"3","chainId":"_gene_","layerId":"geus_layer_839","key":2997},{"id":"office_xp","type":"Int","value":"6666","layerId":"office_xp"},{"id":"rec_new2th","type":"Int","value":"1","chainId":"_gene_","layerId":"Hump","key":320},{"id":"tp_topic_style","type":"String","value":"0","chainId":"_all_","layerId":"tptp_layer_4"},{"id":"tp_zrec","type":"String","value":"1","chainId":"_all_","layerId":"tptp_layer_619"},{"id":"gue_v_serial","type":"String","value":"1","layerId":"guevd_layer_695"},{"id":"club_fn","type":"Int","value":"1","layerId":"club_fn"},{"id":"test_3","type":"Int","value":"1","chainId":"_gene_","layerId":"dABe","key":637},{"id":"pf_adjust","type":"String","value":"1","chainId":"_all_","layerId":"pfus_layer_9"},{"id":"hw_aa_30","type":"Int","value":"0","chainId":"_gene_","layerId":"hw_aa_30","key":361},{"id":"tp_contents","type":"String","value":"1","chainId":"_all_","layerId":"tptp_layer_627"},{"id":"gue_vid_tab","type":"String","value":"7","layerId":"guevd_layer_900"},{"id":"gue_visit_n_artcard","type":"String","value":"1","layerId":"gueqa_layer_579"},{"id":"gue_art_ui","type":"String","value":"2","layerId":"gueqa_layer_647"},{"id":"ge_usercard1","type":"String","value":"0","chainId":"_gene_","layerId":"gese_layer_742","key":2740},{"id":"pfd_newbie2","type":"Int","value":"0","chainId":"_gene_","layerId":"pfd_newbie2","key":71},{"id":"ge_newbie3","type":"Int","value":"0","chainId":"_gene_","layerId":"ge_newbie3","key":180},{"id":"se_zp_five","type":"Int","value":"0","chainId":"_gene_","layerId":"se_zp_five","key":344},{"id":"gue_q_share","type":"String","value":"0","layerId":"gueqa_layer_647"},{"id":"click_new_tb","type":"Int","value":"0","chainId":"_gene_","layerId":"1vYQ","key":422},{"id":"phrase_recall","type":"Int","value":"1","chainId":"_gene_","layerId":"phrase_recall","key":603},{"id":"se_mixer_km","type":"Int","value":"0","chainId":"_gene_","layerId":"se_mixer_km","key":649},{"id":"ge_hard_s_ma","type":"String","value":"0","chainId":"_gene_","layerId":"geli_layer_856","key":3031},{"id":"zr_slotpaidexp","type":"String","value":"1","chainId":"_all_","layerId":"zrrec_layer_5"},{"id":"ts_refresh","type":"Int","value":"1","layerId":"EBn6"},{"id":"correct_cnn","type":"Int","value":"0","chainId":"_gene_","layerId":"correct_cnn","key":397},{"id":"web_collection_guest","type":"String","value":"1","layerId":"webqa_layer_4"},{"id":"gue_video_replay","type":"String","value":"2","layerId":"guevd_layer_3"},{"id":"se_cp_post3","type":"Int","value":"0","chainId":"_gene_","layerId":"se_cp_post3","key":706},{"id":"web_column_auto_invite","type":"String","value":"1","layerId":"webqa_layer_1"},{"id":"gue_zvideo_link","type":"String","value":"1","layerId":"guevd_layer_2"},{"id":"gue_bulletmb","type":"String","value":"0","layerId":"guevd_layer_812"},{"id":"recnew_2th","type":"Int","value":"21","chainId":"_gene_","layerId":"l-recnew_2th","key":67},{"id":"ad_com_zhi","type":"Int","value":"0","chainId":"_gene_","layerId":"ad_com_zhi","key":571},{"id":"web_answer_list_ad","type":"String","value":"1","layerId":"webqa_layer_4"},{"id":"ge_entity","type":"String","value":"0","chainId":"_gene_","layerId":"gese_layer_946","key":3036},{"id":"ge_v_rank_v3","type":"Int","value":"0","chainId":"_gene_","layerId":"FPcB","key":581},{"id":"se_prank_decay","type":"Int","value":"0","chainId":"_gene_","layerId":"se_prank_decay","key":703},{"id":"gue_recmess","type":"String","value":"1","layerId":"gueqa_layer_795"},{"id":"zr_expslotpaid","type":"String","value":"1","chainId":"_all_","layerId":"zrrec_layer_11"},{"id":"xuanpinab","type":"Int","value":"0","layerId":"xuanpinab"},{"id":"gue_fo_recom","type":"String","value":"1","layerId":"gueqa_layer_780"},{"id":"ge_relation2","type":"String","value":"1","chainId":"_gene_","layerId":"gese_layer_815","key":2796},{"id":"gue_profile_video","type":"String","value":"1","layerId":"guevd_layer_5"},{"id":"gue_iosplay","type":"String","value":"1","layerId":"guevd_layer_810"},{"id":"haojiakp","type":"Int","value":"0","layerId":"haojiakp"},{"id":"se_sug_exp","type":"Int","value":"0","chainId":"_gene_","layerId":"9ZDG","key":522},{"id":"li_edu_page","type":"String","value":"old","chainId":"_all_","layerId":"lili_layer_580"},{"id":"pf_noti_entry_num","type":"String","value":"2","chainId":"_all_","layerId":"pfus_layer_718"},{"id":"webpImg","type":"Int","value":"1","layerId":"JnVt"},{"id":"li_vip_verti_search","type":"String","value":"0","chainId":"_all_","layerId":"lili_layer_2"},{"id":"gue_card_test","type":"String","value":"1","layerId":"gueqa_layer_2"},{"id":"meta_ebook","type":"Int","value":"1","layerId":"meta_ebook"},{"id":"zanswer","type":"Int","value":"1","layerId":"l-zanswer"},{"id":"ge_upload","type":"String","value":"0","chainId":"_gene_","layerId":"geus_layer_839","key":2892},{"id":"se_4a","type":"Int","value":"0","chainId":"_gene_","layerId":"rtiq","key":335},{"id":"web_answerlist_ad","type":"String","value":"0","layerId":"webqa_layer_1"},{"id":"se_personal_ab","type":"Int","value":"0","chainId":"_gene_","layerId":"9Zt1","key":554},{"id":"se_v082","type":"Int","value":"0","chainId":"_gene_","layerId":"se_v082","key":725},{"id":"web_sem_ab","type":"String","value":"1","layerId":"webgw_layer_3"},{"id":"km_detail_knn","type":"Int","value":"0","layerId":"TY59"},{"id":"pfd_newbie","type":"Int","value":"0","chainId":"_gene_","layerId":"pfd_newbie","key":63},{"id":"an_video_tab","type":"Int","value":"0","chainId":"_gene_","layerId":"an_video_tab","key":727},{"id":"gue_messrec","type":"String","value":"0","layerId":"gueqa_layer_769"},{"id":"ge_search_ui","type":"String","value":"1","chainId":"_gene_","layerId":"gese_layer_838","key":2898},{"id":"se_ffzx_jushen1","type":"String","value":"0","chainId":"_all_","layerId":"sese_layer_4"},{"id":"ts_refresh2","type":"Int","value":"1","layerId":"heWh"},{"id":"se_zhiplus_cpc","type":"Int","value":"0","chainId":"_gene_","layerId":"se_zhiplus_cpc","key":716},{"id":"ge_prf_rec","type":"String","value":"0","chainId":"_gene_","layerId":"getop_layer_991","key":3040},{"id":"test_rec","type":"Int","value":"10","layerId":"test_rec"},{"id":"se_v083","type":"Int","value":"0","chainId":"_gene_","layerId":"se_v083","key":734},{"id":"qap_question_visitor","type":"String","value":" 0","chainId":"_all_","layerId":"qapqa_layer_2"},{"id":"ge_meta_ss","type":"String","value":"1","chainId":"_gene_","layerId":"gese_layer_834","key":3079},{"id":"se_dwd_all","type":"Int","value":"0","chainId":"_gene_","layerId":"se_dwd_all","key":621},{"id":"se_lr","type":"Int","value":"0","chainId":"_gene_","layerId":"se_lr","key":698},{"id":"ts_cardtitle","type":"Int","value":"0","layerId":"ZED7"},{"id":"Full_ans_fav","type":"Int","value":"0","chainId":"_gene_","layerId":"Pnd6","key":372},{"id":"se_timebox","type":"Int","value":"0","chainId":"_gene_","layerId":"se_timebox","key":689},{"id":"gue_cdzixun","type":"String","value":"0","layerId":"gueqa_layer_3"},{"id":"ge_sug_rep","type":"String","value":"1","chainId":"_gene_","layerId":"gese_layer_1034","key":3158},{"id":"ge_item","type":"String","value":"0","chainId":"_gene_","layerId":"gese_layer_945","key":2971},{"id":"Test_Punk","type":"Int","value":"0","layerId":"Test_Punk"},{"id":"vdox_cmt","type":"Int","value":"1","layerId":"vdox_cmt"},{"id":"se_fix_ebook","type":"Int","value":"0","chainId":"_gene_","layerId":"se_fix_ebook","key":103},{"id":"iosserial","type":"Int","value":"0","chainId":"_gene_","layerId":"iosserial","key":714},{"id":"gue_repost","type":"String","value":"1","layerId":"gueqa_layer_671"},{"id":"top_test_4_liguangyi","type":"String","value":"1","chainId":"_all_","layerId":"iosus_layer_1"},{"id":"xuanpinceshi","type":"Int","value":"3399","layerId":"xuanpinceshi"},{"id":"pc_editorplugin","type":"Int","value":"0","layerId":"pc_editorplugin"},{"id":"web_login","type":"String","value":"0","layerId":"webgw_layer_759"},{"id":"ge_video","type":"String","value":"1","chainId":"_gene_","layerId":"geli_layer_856","key":2831},{"id":"web_audit_01","type":"String","value":"case1","layerId":"webre_layer_1"},{"id":"edu_cd1","type":"Int","value":"0","layerId":"l18Y"},{"id":"andserial","type":"Int","value":"0","chainId":"_gene_","layerId":"andserial","key":711},{"id":"li_sp_mqbk","type":"String","value":"0","chainId":"_all_","layerId":"lili_layer_748"},{"id":"ge_emoji","type":"String","value":"0","chainId":"_gene_","layerId":"getp_layer_827","key":3209},{"id":"hw_aa_50","type":"Int","value":"1","chainId":"_gene_","layerId":"hw_aa_50","key":362},{"id":"book_kpwz","type":"Int","value":"2","chainId":"_gene_","layerId":"Zyox","key":671},{"id":"se_filter","type":"Int","value":"2","chainId":"_gene_","layerId":"2NQZ","key":363},{"id":"gue_bullet_guide","type":"String","value":"发个弹幕聊聊…","layerId":"guevd_layer_0"},{"id":"gue_bullet_second","type":"String","value":"1","layerId":"guevd_layer_1"},{"id":"gue_playh_an","type":"String","value":"1","layerId":"guevd_layer_622"},{"id":"use_bff_profit","type":"Int","value":"0","layerId":"use_bff_profit"},{"id":"ffzix_xgdzwz","type":"Int","value":"0","layerId":"ffzix_xgdzwz"},{"id":"gue_video_guide","type":"String","value":"1","layerId":"guevd_layer_625"},{"id":"se_mixer_roll","type":"Int","value":"0","chainId":"_gene_","layerId":"se_mixer_roll","key":699},{"id":"ge_yuzhi_v1","type":"String","value":"1","chainId":"_gene_","layerId":"gese_layer_1029","key":3127},{"id":"web_scl_rec","type":"String","value":"0","layerId":"webgw_layer_759"},{"id":"gue_goods_card","type":"String","value":"0","layerId":"gueqa_layer_1"},{"id":"ge_guess","type":"String","value":"0","chainId":"_gene_","layerId":"gese_layer_938","key":2912},{"id":"gue_andplayd","type":"String","value":"1","layerId":"guevd_layer_686"},{"id":"gue_sharp","type":"String","value":"1","layerId":"guevd_layer_686"},{"id":"hwtj_bottom_btn","type":"Int","value":"3","layerId":"jrhg"},{"id":"rec_3th","type":"Int","value":"11","chainId":"_gene_","layerId":"rec_3th","key":697},{"id":"qap_question_author","type":"String","value":"0","chainId":"_all_","layerId":"qapqa_layer_2"},{"id":"ge_sxzx","type":"String","value":"0","chainId":"_gene_","layerId":"gere_layer_990","key":3060},{"id":"se_cvr_boost","type":"Int","value":"2","chainId":"_gene_","layerId":"se_cvr_boost","key":183},{"id":"se_resnet","type":"Int","value":"1","chainId":"_gene_","layerId":"t8JE","key":730},{"id":"li_paid_answer_exp","type":"String","value":"0","chainId":"_all_","layerId":"lili_layer_3"},{"id":"tp_dingyue_video","type":"String","value":"0","chainId":"_all_","layerId":"tptp_layer_4"},{"id":"io_video_tab","type":"Int","value":"0","chainId":"_gene_","layerId":"io_video_tab","key":728},{"id":"ge_rec_2th","type":"String","value":"11","chainId":"_gene_","layerId":"geli_layer_965","key":3023},{"id":"ge_infinity6","type":"String","value":"1","chainId":"_gene_","layerId":"gese_layer_815","key":2817},{"id":"se_images","type":"Int","value":"0","chainId":"_gene_","layerId":"se_images","key":652},{"id":"li_panswer_topic","type":"String","value":"1","chainId":"_all_","layerId":"lili_layer_602"},{"id":"show_ad","type":"Int","value":"0","chainId":"_gene_","layerId":"show_ad","key":27},{"id":"gue_art2qa","type":"String","value":"0","layerId":"gueqa_layer_579"}],"chains":[{"chainId":"_all_"}],"encodedParams":"CoIBwAI0DIgBhALBAtkC5Aq1C0ABfQJpAbQKRwC0AFgBpgFbAokC1wuNAcICQwA7AtwLRQK\u002FAuwKCgJMC08BKgLVAj8A1wJSC8wC4AveAgcMbQK6AnQBsQJWDJsLZwDKAg8LxwKJDGoBnwJrAbsCNwxgC7kC9Au3ANoC2ALPCwELjAIbABJBAAAAAAAAAAMBAQAAAAAAAAEAAAAAFQAAAAABAAAAAAAAAAEAAAABAAAAAAEAAAABAAABAgIAAQALAAIBAAsBAAA="},"triggers":{}},"userAgent":{"Edge":false,"IE":false,"Wechat":false,"Weibo":false,"QQ":false,"MQQBrowser":false,"Qzone":false,"Mobile":false,"Android":false,"iOS":false,"isAppleDevice":false,"Zhihu":false,"ZhihuHybrid":false,"isBot":false,"Tablet":false,"UC":false,"Sogou":false,"Qihoo":false,"Baidu":false,"BaiduApp":false,"Safari":false,"GoogleBot":false,"AndroidDaily":false,"iOSDaily":false,"WxMiniProgram":false,"BaiduMiniProgram":false,"QQMiniProgram":false,"JDMiniProgram":false,"isWebView":false,"isMiniProgram":false,"origin":"Mozilla\u002F5.0 (Windows NT 10.0; Win64; x64) AppleWebKit\u002F537.36 (KHTML, like Gecko) Chrome\u002F90.0.4430.93 Safari\u002F537.36"},"appViewConfig":{},"ctx":{"path":"\u002Fp\u002F118757498","query":{},"href":"http:\u002F\u002Fzhuanlan.zhihu.com\u002Fp\u002F118757498","host":"zhuanlan.zhihu.com"},"trafficSource":"production","edition":{"beijing":false,"baidu":false,"sogou":false,"baiduBeijing":false,"sogouBeijing":false,"sogouInput":false,"baiduSearch":false,"googleSearch":false,"miniProgram":false,"xiaomi":false},"theme":"light","enableShortcut":true,"referer":"","xUDId":"ALAfwfGY3BKPTvQQXG3Q779TY1l53800P-Q=","mode":"ssr","conf":{},"xTrafficFreeOrigin":"","ipInfo":{"cityName":"","countryName":"美国","regionName":"美国","countryCode":"US"},"logged":true,"vars":{"passThroughHeaders":{}}},"me":{"columnContributions":[]},"label":{"recognizerLists":{}},"ecommerce":{},"comments":{"pagination":{},"collapsed":{},"reverse":{},"reviewing":{},"conversation":{},"parent":{}},"commentsV2":{"stickers":[],"commentWithPicPermission":{},"notificationsComments":{},"pagination":{},"collapsed":{},"reverse":{},"reviewing":{},"conversation":{},"conversationMore":{},"parent":{}},"pushNotifications":{"default":{"isFetching":false,"isDrained":false,"ids":[]},"follow":{"isFetching":false,"isDrained":false,"ids":[]},"vote_thank":{"isFetching":false,"isDrained":false,"ids":[]},"currentTab":"default","notificationsCount":{"default":0,"follow":0,"vote_thank":0}},"messages":{"data":{},"currentTab":"common","messageCount":0},"register":{"registerValidateSucceeded":null,"registerValidateErrors":{},"registerConfirmError":null,"sendDigitsError":null,"registerConfirmSucceeded":null},"login":{"loginUnregisteredError":false,"loginBindWechatError":false,"loginConfirmError":null,"sendDigitsError":null,"needSMSIdentify":false,"validateDigitsError":false,"loginConfirmSucceeded":null,"qrcodeLoginToken":"","qrcodeLoginScanStatus":0,"qrcodeLoginError":null,"qrcodeLoginReturnNewToken":false},"switches":{},"captcha":{"captchaNeeded":false,"captchaValidated":false,"captchaBase64String":null,"captchaValidationMessage":null,"loginCaptchaExpires":false},"sms":{"supportedCountries":[]},"chat":{"chats":{},"inbox":{"recents":{"isFetching":false,"isDrained":false,"isPrevDrained":false,"result":[],"next":null,"key":null},"strangers":{"isFetching":false,"isDrained":false,"isPrevDrained":false,"result":[],"next":null,"key":null},"friends":{"isFetching":false,"isDrained":false,"isPrevDrained":false,"result":[],"next":null,"key":null},"search":{"isFetching":false,"isDrained":false,"isPrevDrained":false,"result":[],"next":null,"key":null},"config":{"newCount":0,"strangerMessageSwitch":false,"strangerMessageUnread":false,"friendCount":0}},"global":{"isChatMqttExisted":false}},"emoticons":{"emoticonGroupList":[],"emoticonGroupDetail":{}},"creator":{"currentCreatorUrlToken":null,"homeData":{"recommendQuestions":[]},"tools":{"question":{"invitationCount":{"questionFolloweeCount":0,"questionTotalCount":0},"goodatTopics":[]},"customPromotion":{"itemLists":{}},"recommend":{"recommendTimes":{}}},"explore":{"academy":{"tabs":[],"article":{}}},"rights":[],"rightsStatus":{},"levelUpperLimit":10,"account":{"growthLevel":{}},"mcn":{},"applyStatus":{},"videoSupport":{},"mcnManage":{},"tasks":{},"recentlyCreated":[],"analysis":{"all":{},"answer":{},"zvideo":{},"article":{},"pin":{},"singleContent":{}},"announcement":{},"school":{"tabs":[],"contents":[],"banner":null,"entities":{}}},"answers":{"voters":{},"copyrightApplicants":{},"favlists":{},"newAnswer":{},"concernedUpvoters":{},"simpleConcernedUpvoters":{},"paidContent":{},"settings":{}},"recommendation":{"homeRecommendations":[]},"shareTexts":{},"articles":{"voters":{},"concernedUpvoters":{}},"previewPost":{},"favlists":{"relations":{}},"columns":{"voters":{}},"reward":{"answer":{},"article":{},"question":{}},"video":{"data":{},"shareVideoDetail":{},"last":{}},"topstory":{"recommend":{"isFetching":false,"isDrained":false,"afterId":0,"items":[],"next":null},"follow":{"isFetching":false,"isDrained":false,"afterId":0,"items":[],"next":null},"room":{"meta":{},"isFetching":false,"afterId":0,"items":[],"next":null},"followWonderful":{"isFetching":false,"isDrained":false,"afterId":0,"items":[],"next":null},"sidebar":null,"announcement":{},"hotListCategories":[],"hotList":[],"guestFeeds":{"isFetching":false,"isDrained":false,"afterId":0,"items":[],"next":null},"followExtra":{"isNewUser":null,"isFetched":false,"followCount":0,"followers":[]},"hotDaily":{"data":[],"paging":{}},"hotHighlight":{"isFetching":false,"isDrained":false,"data":[],"paging":{}},"banner":{}},"readStatus":{},"column":{},"requestColumn":{"categories":[],"error":null},"articleContribution":{"contributeRequests":[],"deleteContributeIdList":[],"handledContributeIdList":[],"recommendedColumns":[],"pinnedColumns":[],"sentContributeRequestsIdList":[null,null,null,null,null,null,null,null,null,null,null,null,null,null,null,null,null,null,"c_1226443594048942080"]},"columnContribution":{"contributeRequests":[],"autoInviteEnabled":false,"recommendedContributors":[],"contributionInvitation":null},"draftHistory":{"history":{},"drafts":{}},"upload":{},"articleDraft":{"titleImage":"","titleImageSize":{},"isTitleImageFullScreen":false,"canTitleImageFullScreen":false,"title":"","titleImageUploading":false,"error":"","content":"","draftLoading":false,"updating":false,"globalLoading":false,"pendingVideo":{"resource":null,"error":null},"deleteFail":{"fail":false},"recommendTopics":[],"selectedColumn":0,"articleDisclaimers":[]},"articleDrafts":{"isDrained":false,"isLoading":false,"items":[]},"columnAutocomplete":{"users":[],"friends":[]},"columnCollection":{},"userProfit":{"permission":{"permissionStatus":{"zhiZixuan":0,"recommend":-1,"task":0,"plugin":0,"infinity":0},"visible":false}},"mcn":{"bindInfo":{},"memberCategoryList":[],"producerList":[],"categoryList":[],"lists":{},"banners":{},"protocolStatus":{"isAgreedNew":true,"isAgreedOld":true},"probationCountdownDays":0},"zvideos":{"campaigns":{},"tagoreCategory":[],"recommendations":{},"insertable":{},"recruit":{"form":{"platform":"","nickname":"","followerCount":"","domain":"","contact":""},"submited":false,"ranking":[]},"club":{},"qyActivityData":{},"batchVideos":{},"contribution":{"selectedContribution":null,"configs":{},"contributionLists":{},"recommendQuestions":{"isLoading":true,"paging":{"isEnd":false,"isStart":true,"totals":0},"data":[]},"questionSearchResults":{"isLoading":true,"paging":{"isEnd":false,"isStart":true,"totals":0},"data":[]}}},"republish":{}},"fetchHost":"www.zhihu.com","subAppName":"column"}</script><script crossorigin="" src="https://static.zhihu.com/heifetz/vendor.bfefaa5cfee5584ea98c.js"></script><script crossorigin="" src="https://static.zhihu.com/heifetz/column.app.5ea762694f40b6b8e704.js"></script></body><script src="https://hm.baidu.com/hm.js?98beee57fd2ef70ccdd5ca52b9740c49" async=""></script></html>
