% 从分析力学到场论
% keys 经典场|狭义相对论|作用量|拉格朗日函数

%在相对论中创建目录:经典场论

\addTODO{预备知识待确定.}

\pentry{流形上的张量场\upref{TenMan}}

所谓场论,就是研究场的运动,或者说变化的理论.

牛顿力学认为物质是由粒子构成的,而粒子是一种无大小的数学对象,并且被赋予了一个描述其特征的标量,称为“质量”.牛顿动力学所研究的,就是“粒子”这种对象运动的规律.最古典的处理方式,是用\textbf{牛顿三定律}来描述这种规律,相当于描述每时每刻粒子的运动状态和其改变;拉格朗日力学则提出了另一种研究范式,即从整体着手,研究哪些粒子运动轨迹是允许的,此时描述规律的方式变成了\textbf{最小作用量原理}.角度虽有不同,但两种处理方式都是描述“粒子”的运动.

现在,我们要讨论,是否可以用场论来处理牛顿力学.

考虑一个粒子在三维空间中的运动.粒子的位置由一个三维向量函数$\bvec{r}(t)$






















