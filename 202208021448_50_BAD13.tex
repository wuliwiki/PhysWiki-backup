% 北京航空航天大学2013年数据结构与C语言程序设计
% 北京航空航天大学 2013 数据结构 C语言程序设计 计算机 考研


考生注意:所有答题务必书写在考场提供的答题纸上,写在本试题单上的答题一律无效(本题单不参与阅卷).

\subsection{一、单项选择题}
(本题20分,每小题各2分)

1. 对于K度为n的线性表,建立其对应的单链表的时问复杂度为____. \\
A. $O(1)$ $\quad$ B $O(log_2n)$ $\quad$ C. $O(n)$ $\quad$ D. $O(n^2)$

2. 一般情况下.在一个双向链表中插一个新的链结点,____. \\
A.需要器改4个指针域内的指针 $\quad$ B.需要修改3个指针域内的指针 \\
C.需要修改2个指针域内的指针 $\quad$ D.只需要修改1个指针域内的指针

3. 假设用单个字母表示中缀表达式中的一个运算数(或称运算对象).井利用堆栈产生中缀表达式对应的后辍表达式.对于中辍表达式$A+B*(C/D-E)$,当从左至右扫描到运算数$E$时,堆栈中的运算符依次是____.(注:不包含表达式的分界符) \\
A. $+*/-$  $\quad$  B. $+*(/-$ $\quad$  C. $+*-$ $\quad$ D. $+*(-$

4. 若某二叉排序树的前序遍历序列为50,20,40,30,80,60,70.则后序遍历序列____. \\
A.30.40 20.50.70.60.80 $\quad$ B.30.40.20.W.60.80.50. \\
C.70.60.80.50.30.40.20 $\quad$ D.70.60.90.30.40.20.50.

5. 分别以6、3、8、12、5、7对应叶结点的权值构造的哈夫曼(Huffman)树的深度____. \\
A.6  $\quad$ B.5 $\quad$ C.4 $\quad$ D.3

6. 下列关于图的叙述中,\textbf{错误}的是 \\
A. 根据图的定义,图中至少有一个顶点 \\
B. 根据图的定义,图中至少有一个个顶点和一条边(弧) \\
c. 具有$n$个顶点的无向图最多有$n\times(n-1)/2$条边 \\
D. 具有$n$个顶点的有向图最多有$n\times(n-1)$条边(弧) \\

7. 若在有向图$O$的拓扑序列中,顶点$v_i$,在顶点$v_j$之前,则下列$4$种情形中不可出现的是____. \\
A. $G$中有弧$<v_i,v_j>$ \\
B. $G$中没有弧$<v_i,v_j>$ \\
C. $G$中有一条从顶点$v_i$到顶点$v_j$的路径 \\
D. $G$中有一条从顶点$v_j$到顶点$v_i$的路径.

8. 下列关于查找操作的叙述中,\textbf{错误}的是____ \\
A. 在顺序表中查找元素可以采用顺序查找法,也可以采用折半查找法 \\
B. 在链表中查找结点只能采用顺序查找法,不能采用折半查找法 \\
C. 一般情况下,顺序查找法不如折半查找法的时间效率高 \\
D. 折半查找的过程可以用一棵称之为“判定树”的二叉树来描述

9. 在一棵$m$阶$B$-树中.除根结点之外的任何分支结点包含关键字的个数至少是____. \\
A. $m/2-1$ $\quad$ B. $m/2$ $\quad$ C. $\lceil m/2-1 \rceil $ $\quad$ D. $\lceil m/2 \rceil $

9. 在一棵$m$阶$B$-树中,除根结点之外的任何分支结点包含关键字的个数至少是____. \\
A. $m/2-1$ $\quad$ B. $m/2$ $\quad$ C. $\lceil m/2 \rceil-1$ $\quad$ D. $\lceil m/2 \rceil$.

10. 若对序列(49. 38,65,97,76,13. 27,49')进行快速捧序,则第一趟排序结束(即确定了第1个分界元素的最终位置)时,序列的状态是___. \\
A. (13. 27, 49', 38, 49, 76, 97, 65) $\quad$ B.(13. 38. 27. 49'. 49. 76. 97. 65). \\
C. (13. 38. 49'. 27. 49. 97. 76. 65) $\quad$ D.(13. 38. 49'. 27. 49. 76. 97. 65)

\subsection{二、填空题}
本题共20分,每小题各2分

1.非空线性表在采用____存储结构的情况下,删除表的一个数据元素平均需要移动表中近一半元素的位置.

2.将个长度为$n$的单链表链接到一个长度为$m$的单链表后面,该算法的时间复杂度用大$O$符号表示为____.

3.若完全二叉树的叶结点的数目为$k$,且最下面一层的结点数大干$1$,则该完全二叉树的深度为____.

4.若潍度为8的完全=叉树的第7层有10十叶结点,则&:卫树的结点总数*____.

5在具有n十顶点的有自凹十,每十顶点的度最大可“达到____.
  6若对有向固进行拓扑排序,则能够得到拓扑序列的条件是____.
  7已知长度为10的顺序表中数据元素接值从小到大排列.若在该表十进行折半查找,
则平均查找长度(ASL)是____.
  g若在棵m阶B-树的某十结点中插^十新的关键宇值而H起结点产生分裂,
则诖结点十原有的关键字值的数目是____.
  9有种#序方法可能台出现这种情况:最后一a捧序开始之村,序列中所有的元
素部币在其最终应该在的位置L.这种排序方诘是____.
  10若按照咆排序法的思想将序列(2,12,J6,5 IO仲元襄拄值从小到大进行排序,
整十排序过程十所进行的元素之问的盹较次数为____.
