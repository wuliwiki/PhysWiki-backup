% 编译器/解释器简介

\begin{issues}
\issueDraft
\end{issues}

\begin{itemize}
\item 参考\href{https://www.toptal.com/scala/writing-an-interpreter}{这篇文章},还有\href{https://www.youtube.com/watch?v=Xu4RtLlm42I}{这个}

\item lexer
\item parser
\item interpreter
\item abstract syntax tree (AST)
\item domain-specific languages (DSLs)
\item 一个纯手写 c 编译器\href{https://norasandler.com/2017/11/29/Write-a-Compiler.html}{教程}.
\item baltam 用的是 \href{https://www.gnu.org/software/bison/}{bison} 和 \href{https://www.genivia.com/doc/reflex/html/}{RE/flex}
\item 用 bison 实现一个简单科学计算器(包括变量赋值和基本函数) 的例子 \href{http://web.mit.edu/gnu/doc/html/bison_5.html}{Multi-Function Calculator: mfcalc}.
\item \href{https://en.wikipedia.org/wiki/LR_parser}{LR parser}
\item 然而 TeX 的编译器要复杂得多(如果要实现全部功能)见\href{https://groups.google.com/g/comp.text.tex/c/E1736iEOxNI}{这个讨论}.
\end{itemize}
