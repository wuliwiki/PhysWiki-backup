% 函数的凹凸性(极简微积分)
% license Xiao
% type Tutor

\pentry{导数(极简微积分)\nref{nod_Der},高阶导数(极简微积分)\nref{nod_HigDer}}{nod_49a3}

\begin{figure}[ht]
\centering
\includegraphics[width=6 cm]{./figures/1d7dc03e8b0fe495.pdf}
\caption{$y=x^2 (x>0)$与$y=\sqrt{x}$的函数图像(比例经过调整)} \label{fig_CONCV_1}
\end{figure}

让我们先观察$y=x^2 (x>0)$与$y=\sqrt{x}$的函数图像\autoref{fig_CONCV_1}。尽管他们都是单调的增函数,但他们的增长方式却似乎不太一样。

\begin{figure}[ht]
\centering
\includegraphics[width=6 cm]{./figures/1dab6895d358d75a.pdf}
\caption{$y=x^2 (x>0)$的切线越来越陡,而$y=\sqrt{x}$ 的则愈发平缓(比例经过调整)} \label{fig_CONCV_2}
\end{figure}

如果你和我想的一样,他们的不同之处在于,$y=x^2 (x>0)$增长得越来越快,而$y=\sqrt{x}$增长得越来越慢\footnote{尽管$y=\sqrt{x}$增长得越来越慢,但仍有 $\lim_{x\to+\infty} \sqrt{x} = +\infty$。关乎“无穷”的问题总是有点反直觉。}。
\autoref{fig_CONCV_2} 用切线直观反映了这个问题:随着$x$增加,$y=x^2 (x>0)$的切线越来越陡,而$y=\sqrt{x}$ 的则愈发平缓。

在数学上,我们把这种“增长得越来越快”的函数称为凹(“下凸”)函数,而“增长得越来越慢”的称为凸(“上凸”)函数。

\addTODO{同济高数和全世界反过来的,我的建议是放弃凹凸,只叫上下凸}

\addTODO{应当给出凸函数的(连续)定义,再去结合二阶导数}

那么我们怎么数学地形容“函数增长得越来越快”?我们知道,函数的变化速率可以用相应函数的\enref{导数}{Der} $f'(x)$表示;而“增长得越来越快”意味着$f'(x)$越来越大;而$f'(x)$越来越大意味着$f'(x)$的导数$f''(x)>0$,即函数的\enref{二阶导数}{HigDer}大于零。按照物理的话术说,“速度越来越快”意味着“加速度大于零”\footnote{防杠声明:假定$v,a$同向}。

由此,我们联系了函数的凹凸性与函数的二阶导数,总结如下(假定$f(x)$单调增长,即$f'(x)>0$):

$f(x)\text{是凹函数} \Rightarrow f(x)\text{增长得越来越快}\Rightarrow f'(x)\text{在增长} \Rightarrow f''(x)>0~,$

$f(x)\text{是凸函数} \Rightarrow f(x)\text{增长得越来越慢}\Rightarrow f'(x)\text{在减小} \Rightarrow f''(x)<0~.$

\begin{exercise}{}
将凹凸性的概念推广至$f'(x)<0$的情况。
\end{exercise}
