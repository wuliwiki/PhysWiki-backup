% 经典力学笔记(科普)

\begin{issues}
\issueDraft
\end{issues}

\begin{itemize}
\item \textbf{经典力学}一般是指\textbf{牛顿力学}, 但在一些较现代的文献中也包含相对论。
\item 经典力学中物体受力后的运动规律一般由牛顿三定律描述。
\item 牛顿定律不讨论物体受到的力是如何产生的。
\item 物理中, \textbf{粒子}也叫\textbf{质点}, 是指在当前问题的一般尺度下大小可以忽略不记的物体。 我们假设它们不存在转动, 只存在平移。
\item 物理中, 在研究物体的运动规律时往往先研究粒子的运动规律, 再把所有物体都看作由许多粒子组成的系统。
\item 牛顿三定律本身描述的对象只是质点。 经过一些推论才可以把它拓展到一般的物体。
\end{itemize}
