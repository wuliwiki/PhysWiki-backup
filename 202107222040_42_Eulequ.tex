% 欧拉方程
% 欧拉方程

\begin{issues}
\issueDraft
\issueTODO
\end{issues}

具有以下形式的方程称为\textbf{欧拉方程}.
\begin{equation}\label{Eulequ_eq1}
\sum_{i=0}^{n}a_ix^{n-i}y^{(n-i)}=0
\end{equation}
其中 $a_0=1,a_1,\cdots,a_n$为常数.

欧拉方程可化为具有如下形式的常系数线性方程
\begin{equation}
\varphi(D_t)y=0
\end{equation}
其中
\begin{equation}
\begin{aligned}
&x=e^t\\
&D_t=\dv{}{t}\\
&\varphi(D_t)=\sum_{i=0}^{n-1}a_iD_t(D_t-1)\cdots(D_t-n+i+1)+a_n=0
\end{aligned}
\end{equation}

欧拉方程\autoref{Eulequ_eq1} 具有如下形式的解
\begin{equation}
y=\sum_{i=1}^mx^{r_i}P_{k_i-1}(\ln x)
\end{equation}
其中,$r_s$是下面方程的根
\begin{equation}\label{Eulequ_eq2}
\sum_{i=0}^{n-1}a_ir(r-1)\cdots(r-n+i+1)+a_n=0
\end{equation}
$k_s$ 是根 $r_s$ 的重数,$m$是不同根的个数, $P_{k_s-1}(\ln x)$ 是具有任意系数的 $k_s-1$ 次多项式. 

特别的,当\autoref{Eulequ_eq2} 的所有根均为单根时,欧拉方程\autoref{Eulequ_eq1} 的解为
\begin{equation}
y=\sum_{i=1}^{n}C_ix^{r_i}
\end{equation}

\subsection{证明}
令
\begin{equation}
x=e
\end{equation}

\addTODO{证明结论}