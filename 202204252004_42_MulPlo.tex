% 多元函数的极值
% 极值|充要条件|二次型


\pentry{二元函数的极值(简明微积分)\upref{F2Exm}}
\subsection{极值}
设函数 $u=f(x_1,\cdots,x_n)$ 定义于区域 $\mathcal{D}$ 中,且 $(x_1^0,\cdots,x_n^0)$ 是这区域的内点.

\begin{definition}{极值}\label{MulPlo_def1}
若点 $(x_1^0,\cdots,x_n^0)$ 有这样一个领域
\[(x_1^0-\delta,x_1^0+\delta;\cdots;x_n^0-\delta,x_n^0+\delta)\]
使对于其中一切点都能成立不等式
\begin{equation}
\begin{aligned}
f(x_1,\cdots,x_n)&\leq f(x_1^0,\cdots,x_n^0)\\
&(\geq)
\end{aligned}
\end{equation}
就说,函数 $f(x_1,\cdots,x_n)$ 在点 $(x_1^0,\cdots,x_n^0)$ 处有\textbf{极大值}(\textbf{极小值}).

若在除去点 $(x_1^0,\cdots,x_n^0)$ 本身以外区域中的每一点都能成立严格不等式
\begin{equation}
\begin{aligned}
f(x_1,\cdots,x_n)&< f(x_1^0,\cdots,x_n^0)\\
&(>)
\end{aligned}
\end{equation}
就说,函数 $f(x_1,\cdots,x_n)$ 在点 $(x_1^0,\cdots,x_n^0)$ 处有\textbf{真正的}极大值(极小值);否则,极大值(极小值)就称为\textbf{广义的}.

极大值和极小值总称为\textbf{极值}.
\end{definition}
\subsection{极值的必要条件}
\begin{theorem}{}
若函数 $f$ 在某一点 $(x_1^0,\cdots,x_n^0)$ 处有极值,且在这一点处存在着(有限)偏导数
\[f'_{x_1}(x_1^0,\cdots,x_n^0),\cdots,f'_{x_n}(x_1^0,\cdots,x_n^0)\]
则这些偏导数都为0.
\end{theorem}
\textbf{证明:}令 $x_2=x_2^0,\cdots,x_n=x_n^0$ ,而 $x_1$ 仍保持为变量;那么,就得到 $x_1$ 的一元函数:
\begin{equation}
u=f(x_1,x_2^0,\cdots,x_n^0)
\end{equation}
因为函数在点 $(x_1^0,\cdots,x_n^0)$ 有极值(为明确,设为极大值)存在,由极值\autoref{MulPlo_def1} ,在点 $x_1=x_1^0$ 的某一领域 $(x_1^0-\delta,x_1^0+\delta)$ 内,必成立不等式
\begin{equation}
f(x_1,x_2^0\cdots,x_n^0)\leq f(x_1^0,\cdots,x_n^0)
\end{equation}
于是上述一元函数在点 $x_1=x_1^0$ 将有极大值,由费马定理\autoref{MeanTh_the1}~\upref{MeanTh},就得
\begin{equation}
f'_{x_1}(x_1^0,\cdots,x_n^0)=0
\end{equation}
同样的方法可证明在点 $(x_1^0,\cdots,x_n^0)$ 处其它偏导数也都为0.

\textbf{证毕!}

于是,一阶偏导数等于0是极值存在的必要条件.

因此,对极值的“怀疑”就是那些一阶偏导数全为0的点,它们的坐标可由解方程组
\begin{equation}
\begin{aligned}
f'_{x_1}(x_1,\cdots,x_n)&=0,\\
&\vdots\\
f'_{x_n}(x_1,\cdots,x_n)&=0
\end{aligned}
\end{equation}
求出.这种点称为\textbf{静止点}.
\subsection{极值的充分条件}
设函数 $f(x_1,\cdots,x_n)$ 是在某一静止点 $(x_1^0,\cdots,x_n^0)$ 的领域内定义着的连续并有一阶及二阶连续导数.
\begin{issues}
\issueOther{需链接多元函数的泰勒公式}
\end{issues}
按照多元函数的泰勒公式展开下式到二阶项(由于是在静止点,一阶项为0)
\begin{equation}
\Delta=f(x_1,\cdots,x_n)-f(x_1^0,\cdots,x_n^0)
\end{equation}
得
\begin{equation}
\begin{aligned}
\Delta=&\frac{1}{2}\mathrm{d}^2 f(x_0+\theta\Delta x_0,\cdots,x_n+\theta\Delta x_n)\\
=&\frac{1}{2}[f''_{x_1^2}\Delta x_1^2+f''_{x_2^2}\Delta x_2^2+\cdots+f''_{x_n^2}\Delta x_n^2+2f''_{x_1x_2}\Delta x_1\Delta x_2\\
&+2f''_{x_1x_3}\Delta x_1\Delta x_3+\cdots+2f''_{x_{n-1}x_n}\Delta x_{n-1}\Delta x_n\large]
 \\
 =&\frac{1}{2}\sum_{i,k=1}^nf''_{x_ix_k}\Delta x_i\Delta x_k \quad (0<\theta<1)
\end{aligned}
\end{equation}
