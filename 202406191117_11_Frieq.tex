% Friedmann 方程
% license Usr
% type Tutor
\subsection{弗里德曼方程与加速度方程}

\textbf{弗里德曼方程(Friedmann Equation)}是爱因斯坦方程的时间部分,即:
\begin{equation}G_{00}+\Lambda g_{00}=8\pi GT_{00}\quad\Rightarrow\quad\left(\frac{\dot{a}}{a}\right)^2=H^2=\frac{8\pi G}{3}\rho+\frac{\Lambda}{3}-\frac{k}{a^2}~,\end{equation}
可见,如果宇宙常数项足够大,$\dot a$将永不为0,宇宙将一直膨胀。而如果宇宙常数项很小,或者为负值,宇宙则有可能在将来经历停止膨胀,然后收缩的命运(比之如今的宇宙热历史理论,该方程是时间反演不变的。)\footnote{有一个理论是宇宙常数并非“常数”,}

对上式进行求导,便得到\textbf{加速度方程(acceleration equation)}:

\begin{equation}
\Rightarrow\quad\frac{\ddot{a}}{a}-\frac{\Lambda}{3}=-\frac{4\pi G}{3}(\rho+3P)~,
\end{equation}
又称作\textbf{雷乔杜里方程(Raychaudhuri equation)}。丛该方程上看,$\rho,P$的作用是使膨胀减速,可以理解为引力作用;宇宙常数项则能促进宇宙膨胀。

