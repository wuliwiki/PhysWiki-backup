% 电荷在磁场中的圆周运动(高中物理)
% license Pub
% type Note

电荷在磁场中的圆周运动是高中物理的经典问题。根据磁场力始终垂直于速度方向的特性 ($\bvec F = q\bvec v \times \bvec B$),我们可以做出如下推论:
\begin{itemize}
\item 运动圆心一定在垂直入射速度的直线上
\item 出射速度也垂直于与圆心的连线
\item 电荷速率从不改变
\end{itemize}

此外,我们还有几个重要公式:
\begin{itemize}
\item $R=\frac{mv}{qB}$
\item $T=\frac{2\pi m}{qB}$
\item $t = \frac{\theta}{2\pi} T$
\end{itemize}
其中,$R$是圆周运动半径,$m$是电荷质量,$v$是电荷入射速率,$q$是电荷带电量,$B$是磁场强度,$T$是周期(电荷在磁场中完整运动一圈所需的时间),$t$是运动时长,$\theta$是圆心角。

推导也很简单:
$$
\left \{
\begin{aligned}
F&=m\frac{v^2}{R}\\
F&=qvB\\
\end{aligned}
\Rightarrow
R=\frac{mv}{qB}
\right.
~.
$$

$$
\left \{
\begin{aligned}
v=R\cdot\omega\\
\omega = \frac{2\pi}{T}\\
R=\frac{mv}{qB}
\end{aligned}
\Rightarrow
R=\frac{mv}{qB}
\right.
~.
$$