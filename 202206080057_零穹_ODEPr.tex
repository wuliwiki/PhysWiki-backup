% 基本知识(常微分方程)
% 常微分方程|小结

\pentry{常微分方程简介\upref{ODEint}}
\subsection{常微分方程}
在常微分方程简介\upref{ODEint}一节中,可以知道,如果一个方程未知的是函数,且方程含有未知函数的导数,则称这样的方程为\textbf{微分方程}.若微分方程中的未知函数是一元函数(即未知函数仅有一个自变量),则称着方程为\textbf{常微分方程};相反的,若未知函数是多元函数,则称\textbf{偏微分方程}.所以,在自变量为 $x$,且仅有一个未知函数 $y$ 时,一般的常微分方程形式为
\begin{equation}
F\qty(x,y,y',\cdots,y^{(n)})=0
\end{equation}

其中, $y^{(n)}=\dv[n]{y}{x},\;y'=y^{(1)}$.

常微分方程理论通常也研究更一般的方程组.通常,常微分方程组中方程的个数与其中出现的未知函数个数相同,而所有未知函数都是同一自变量的函数.所以,一般的常微分方程组形式为
\begin{equation}\label{ODEPr_eq1}
\leftgroup{
&F_1\qty(x,y_1,y_1',\cdots,y_1^{(n_1)},\cdots,y_n,y_n',\cdots,y_n^{(n_m)})=0\\
\cdots\\
&F_n\qty(x,y_1,y_1',\cdots,y_1^{(n_1)},\cdots,y_n,y_n',\cdots,y_n^{(n_m)})=0
}
\end{equation}
这里,$y_i,(i=1,\cdots,n)$ 都是 $x$ 的未知函数,函数 $F_i$ 是 $\qty(\sum\limits_{j=1}^{m}n_j+n+1)$ 个变量的函数.函数 $F_i$ 可能并不是对变量的所有值都有定义,所以要讨论 $F_i$ 的定义区域 $B$ (假定每个 $F_i$ 的定义区域都是 $B$).这里的区域是  $\qty(\sum\limits_{j=1}^{m}n_j+n+1)$ 个变量 
\begin{equation}
x,y_1,y_1',\cdots,y_1^{(n_1)},\cdots,y_n,y_n',\cdots,y_n^{(n_m)}
\end{equation}
 的$\qty(\sum\limits_{j=1}^{m}n_j+n+1)$ 维坐标空间中的区域\footnote{区域是指这样的集合,其中每一点都有一个邻域属于该集合}.其中,函数 $y_i$ 的最大阶数 $n_i$ 称维方程组\autoref{ODEPr_eq1} \textbf{关于 $y_i$ 的阶},而称数 $\sum\limits_{i=1}^m n_m$ 为\textbf{方程组\autoref{ODEPr_eq1} 的阶}.如果自变量 $x$ 的函数 $y_i=\varphi_i(x)$ 在区间
 \begin{equation}
 r_1<x<r_2
 \end{equation}
上有定义,并把 $\varphi_i(x)$ 代入\autoref{ODEPr_eq1} 时,得到在区间 $r_1<x<r_2$ 上关于 $x$ 的恒等式,则称 $y_i=\varphi_i(x)$ 为方程组\autoref{ODEPr_eq1} 的\textbf{解};而称区间 $r_1<x<r_2$ 为解 $\varphi_i(x)$ 的\textbf{定义区间}.

明显的,仅当函数 $\varphi_i(x)$ 在整个区间 $r_1<x<r_2$ 上有直到 $n_i$ 阶导数时才能在方程组中作代换 $y_i=\varphi_i(x)$,而为了作代换,还须对区间 $r_1<x<r_2$ 中的变量 $x$ 的任一值,以 $\qty(x,y_1,y_1',\cdots,y_1^{(n_1)},\cdots,y_n,y_n',\cdots,y_n^{(n_m)})$ 为坐标的点属于函数 $F$ 的定义区域 $B$.

如果方程组\autoref{ODEPr_eq1} 关于变量 $y_i^{(n_i)}$ 是可解的,那么方程组\autoref{ODEPr_eq1} 可写为等价的形式: 
\begin{equation}\label{ODEPr_eq2}
\begin{aligned}
&y_i^{(n_i)}=f_i\qty(x,y_1,y_1',\cdots,y_1^{(n_1-1)},\cdots,y_n,y_n',\cdots,y_n^{(n_m-1)}),\\
& i=1,\cdots n
\end{aligned}
\end{equation}
\autoref{ODEPr_eq2} 称为是\textbf{已解出最高阶导数的}.
\subsection{微分方程中关于解的重心}
处理微分方程时,所面临的主要问题是求出它的解.正如在代数学中一样,关于所谓“寻找方程的解”的问题可用不同方式来理解.在代数学中,首先企图运用“开根求解任意次方程”来找出解的一般公式,后来证明了,运用开根求解四次以上的方程的一般公式是不存在的.然而,近似求解具有数值系数的方程以及研究方程的根对系数的依赖关系还是可能的.在微分方程理论中,关于解的概念的演变大致也是这样的,开始时总是力图以“求积方式积分出微分方程”.以后,当弄清只是对少数类型的方程才存在这种意义的解时,理论的重心就转移到\textbf{研究解的性态(稳定性,有界性,渐近性等)的一般规律}.
\subsection{标准常微分方程}
在代数学中,解决各种代数方程组解的个数问题的定理起了很大的作用,例如,断定 $n$ 次多项式恰有 $n$ 个根(计入重数)的代数学基本定理.同样,在微分方程理论中,也要问微分方程解的个数问题.可以证明,\textbf{每一微分方程组的解的集合有连续统的势}\footnote{即解的个数与区间 $[0,1]$ 中点的个数一样多.},所以并不关心解的个数问题,而提如何描述\textbf{给定微分方程解的集合}的问题.这一问题由微分方程的存在及唯一性定理(链接)回答.

存在及唯一性定理都是对外表上有某种特殊性的方程组叙述和描述的,而比较一般的方程可以化到这种方程组.为方便叙述起见,称这种特殊的方程组是\textbf{标准的}.

\begin{definition}{标准常微分方程}
形如
\begin{equation}
y_i'=f_i(x,y_1,\cdots,y_n),\quad i=1,\cdots,n
\end{equation}
的常微分方程称为\textbf{标准的}.

其中, $x$ 是自变量, $y_i,\; i=1,\cdots, n$ 是变量 $x$ 的未知函数,而函数 $f_i$ 是定义在 $n+1$ 维空间的某一区域 $\Gamma$ 上的 $n+1$ 个变量的函数.
\end{definition}

今后总假设\footnote{可通过隐函数存在定理\upref{impli}理解.},函数 
\begin{equation}
f_i(t,y_1,\cdots,y_n),\quad i=1,\cdots,n
\end{equation}
在区域 $\Gamma$ 中是\textbf{连续}的,同时它的偏导数
\begin{equation}
\pdv{f_i}{y_i},\quad i,j=1,\cdots,n
\end{equation}
在区域 $\Gamma$ 上也是\textbf{连续}的.
\begin{example}{一阶常微分方程}
在\autoref{ODEPr_eq1} 中,若 $n=n_1=1$,则方程组变为
\begin{equation}\label{ODEPr_eq3}
F_1(x,y_1,y_1')=0
\end{equation}
因为只有一方程,可将标号 “1” 省略而写为
\begin{equation}\label{ODEPr_eq4}
F(x,y,y')=0
\end{equation}

此时,微分方程\autoref{ODEPr_eq4} 便是\textbf{一阶常微分方程}.其对应的标准常微分方程为
\begin{equation}\label{ODEPr_eq5}
y'=f(x,y)
\end{equation}
也称\autoref{ODEPr_eq5} 是已解出导数的.
\end{example}
\subsection{一阶常微分方程的几何解释}
引进变量 $x,y$ 的坐标平面 $P$,取 $x$ 为横坐标,$y$ 为纵坐标.则按假设,\autoref{ODEPr_eq5} 中的函数 $f$ 的定义域是平面 $P$ 上的一区域 $\Gamma$,且 $f$ 及其偏导数 $\pdv{f}{y}$ 在 $\Gamma$ 中是连续的.

方程
\begin{equation}
y'=f(x,y)
\end{equation}
的解 $y=\varphi(x)$ 在平面 $P$ 上的几何表示是