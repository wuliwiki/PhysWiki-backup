% 数学归纳法(高中)
% keys 数学归纳法|递推|归纳法
% license Usr
% type Tutor

\begin{issues}
\issueDraft
\end{issues}

\pentry{数列\nref{nod_HsSeFu}}{nod_aa25}

数列可以被视为一个从自然数集合到实数集合的映射,其中每个自然数 $n$ 唯一对应一个实数 $a_n$。数列的通项公式不仅提供了这种对应关系的计算规则,还可以理解为一种筛选机制。筛选的过程就是通过不断验证命题“自然数 $n$ 与其对应的 $a_n$满足通项公式 $a_n = f(n)$ ”是否为真来找到数列的每个值。例如,若通项公式定义为 $a_n = 2n + 1$,选择 $n = 3$ 。设对应的 $a_3 = 6$,代入公式则得 $2 \cdot 3 + 1 = 7 \neq 6$,命题为假,说明 $a_3 = 6$ 不满足公式;设取对应的 $a_3 = 7$,代入验证有 $2 \cdot 3 + 1 = 7$,命题为真,表明 $a_3 = 7$ 满足公式。因此$a_3=7$。

由此可见,通项公式就像一系列逻辑命题,通过逐一验证自然数的取值,从庞大实数集合中筛选出符合定义的值,从而构成数列。这种“逐一验证”的思想可以进一步推广到更广泛的命题形式,包括等式、不等式,以及没有明确代数表达的陈述。只要命题与自然数相关,就可以类比数列的逐一验证过程,设计一个系统化的证明方法。这种证明方法在数学中被称为数学归纳法。数学归纳法不仅在数列的研究中具有重要作用,还广泛应用于多项式、几何等领域的证明。作为高中数学的核心工具之一,数学归纳法能够证明几乎所有与自然数相关的命题,展现出数学的强大力量和广泛适用性。

针对高中考试而言,数学归纳法通过将复杂的证明过程转化为对最终结果的猜想,提供了一种高效且直观的方法。其特点是,学生可以绕过繁杂的推导,直接利用其他方法得到结果,并通过数学归纳法验证结果的正确性。这一过程的核心在于猜想的提出,而在高中阶段,通常可以通过观察题目提供的模式、直接计算几项具体值来总结出可能的规律或者直接通过一些超纲的方法得到原本需要复杂推理的得到的结论。这样即便绕过了复杂的推导过程,学生仍然可以使用数学归纳法对结果的正确性进行证明。这一特点使得数学归纳法在高中阶段不仅成为理解和解决问题的工具,也成为考试中处理复杂证明问题的高效方法。

针对高中考试而言,数学归纳法通过将复杂的证明过程转化为对最终结果的猜想,提供了一种高效且直观的方法。其特点在于,学生可以绕过繁琐的推导,通过观察题目提供的模式、计算几项具体值总结规律,或者直接借助一些超纲的方法得出结论。即使跳过了完整的推理过程,学生依然可以使用数学归纳法验证结果的正确性。这一特性使得数学归纳法在高中阶段不仅成为理解和解决问题的工具,也成为应对复杂证明问题的高效手段。同时,也将猜想的提出提高到了这一过程的核心,如何猜本身也是数学neng。


\subsection{从多米诺骨牌开始}

在介绍数学归纳法之前,先来看看一种常见的常见连锁效应——多米诺骨牌。(具体描述和形容一下多米诺骨牌搭建和推倒的过程,以防有人不知道。)。设想一下,该如何用最简单的方法来描述这个过程?

一个直接的思路是,搭建的人员在搭建时,要保证每一张骨牌倒下时,都会带动下一个骨牌倒下。然后,在推倒时,保证自己能推倒第一块。这样,第一块倒下就会带动第二块,第二块倒下就会带动第三块,一直这样倒下去,那么所有的骨牌都会倒下。

这种“传递性”的现象,就是数学归纳法背后的原理。

\begin{definition}{数学归纳法}
对某个与自然数相关的命题$P(n)$,利用\textbf{数学归纳法}证明$P(n)$成立的过程分为三步:
\begin{enumerate}
\item 验证:检查命题对某个特定起始点(通常是最小值)是否成立,即命题$P(1)$或$P(0)$成立成立。
\item 假设:假设命题在某个自然数$k$成立,即假设 $n = k$ 时命题$P(k)$成立。
\item 推导:证明命题在归纳假设的前提下,可以推导出$n = k+1$ 时命题$P(k+1)$也成立。
\end{enumerate}
如果以上三步都完成,就可以得出结论:该命题对所有自然数$n$ 都成立。
\end{definition}

可以看到,验证对应的就是第一块能推倒,假设就是指某张骨牌倒下,而推导得出的结论就是下一张骨牌也会倒下。

注意事项与常见误区
	1.	遗漏基础验证
如果未验证初始点 P(1),推导将失去起点支撑。
	2.	归纳推导不严谨
推导步骤必须从归纳假设出发,完整证明 P(k+1)。
	3.	适用范围不明确
在使用归纳法前,需明确命题的适用范围。

尽管通常称为数学归纳法,但本身只是叫做归纳法,“数学”是与其他领域分隔开。虽然数学归纳法名字中有“归纳”,但是数学归纳法并非逻辑上不严谨的\aref{归纳推理法}{sub_HsLogi_1},它属于完全严谨的演绎推理法。数学归纳法是一种从特殊到一般的递归式证明,与反证法等方法相比,更适用于递推关系明显的问题。
数学归纳法是一种公理模式存在,如果满足则判断正确。也就是说它本身是不可以证明的。其实它是定义自然数的公理之一,也就是说,只要有自然数存在的场合,就天然存在数学归纳法。

效果:
\begin{itemize}
\item 保证自然数的完备性:确保性质对所有自然数都成立,不必逐一检查每一个自然数。
\item 建立递归定义的基础:提供严格的证明这些递归定义正确性的方法。
\end{itemize}
\subsection{实际应用}


所有与自然数有关的命题都可以通过数学归纳法来证明。读者可以自行尝试一下利用数学归纳法证明“\enref{恒等式与恒成立不等式}{HsIden}”中给出的表达式。

数学归纳法的实际应用
前面介绍的数学归纳法比较抽象,尽管过程已经很清晰了,但由于之前从未接触过,具体使用却仍让人感到陌生。下面以几个实例来感受一下数学归纳法的强大。
数学归纳法在以下方面有广泛应用:

\subsubsection{证明恒等式}
下面的例题,曾经应用在\aref{等比数列}{sub_HsGmPg_1}中给出证明。通常,类似的恒等式都可以通过数学归纳法证明。
\begin{example}{证明:对于任意正整数$n$都有$a^n-b^n=\left(a-b\right)\left(a^{n-1}+a^{n-2}b+\cdots+b^{n-1}\right)$。}
证明:

当 $n = 1$ 时:
\begin{equation}
a^1 - b^1 = (a - b)~.
\end{equation}
因此,命题在 $n = 1$ 时成立。

假设命题对 $n = k$ 成立,即:
\begin{equation}
a^k - b^k = (a - b)\left(a^{k-1} + a^{k-2}b + \cdots + b^{k-1}\right)~.
\end{equation}
需证明命题对 $n = k+1$ 成立,代入假设有:
\begin{equation}
\begin{aligned}
a^{k+1} - b^{k+1} &= a^{k+1}-ab^k+ab^k - b^{k+1}\\
&= a \cdot (a^k - b^k) + b^k(a - b)\\
&=a \cdot (a - b)\left(a^{k-1} + a^{k-2}b + \cdots + b^{k-1}\right) + b^k(a - b)\\
&=(a - b)\left[a \cdot \left(a^{k-1} + a^{k-2}b + \cdots + b^{k-1}\right) + b^k\right]\\
&=(a - b)\left(a^{k} + a^{k-1}b + \cdots + ab^{k-1} + b^k\right)~.
\end{aligned}
\end{equation}
与题设形式一致。

综上由数学归纳法可知:
\begin{equation}
a^n - b^n = (a - b)\left(a^{n-1} + a^{n-2}b + \cdots + b^{n-1}\right)~.
\end{equation}
对所有正整数 $n \geq 1$ 成立。
\end{example}

\subsubsection{不等式恒成立}

有一些问题,结论并不那么显然,在代入处理时,需要利用其他已知条件来辅助放缩。

\begin{example}{证明 $\E^n \geq n+1$ 对 $n \geq 0$ 成立。}
证明:

当 $n = 0$ 时,$\E^0 = 1\geq 0+1$,成立。

假设 $\E^k \geq k+1$ 成立。需要证明 $n = k+1$ 时命题也成立。
由于$\E>2$且$k\geq0$,有:
\begin{equation}
(k + 1) \cdot \E > (k + 1) \cdot 2 = 2k + 2 \geq k + 2~.
\end{equation}
根据假设,可得:
\begin{equation}
\begin{aligned}
\E^{k+1} &= \E \cdot \E^k\\
&\geq \E(k+1)\\
&\geq k+2~.
\end{aligned}
\end{equation}
与命题的形式一致。综上,利用数学归纳法,可以证明$\E^n \geq n+1$对所有整数$n \geq 0$ 都成立。
\end{example}

\subsubsection{整除问题}

由于整除是一个与自然数关系很大的命题,因此很多与整除有关的命题也可以利用数学归纳法来证明。它也成为数论领域的重要工具。

\begin{example}{证明:对任意正整数$n$,$9^{n} - 1$ 能被 $8$ 整除。}
证明:

当 $n = 1$ 时:$9^1 - 1 = 9 - 1 = 8$。显然$8\text{÷}8=1$,命题在 $n = 1$ 时成立。

假设当 $n = k$ 时,命题成立,这意味着存在整数 $m$,使得:
\begin{equation}
9^k - 1 = 8m~.
\end{equation}
当 $n = k+1$ 时,代入假设,有:
\begin{equation}
\begin{aligned}
9^{k+1} - 1 &= 9\cdot9^{k} - 1\\
&= 9\cdot(8m+1)- 1\\
&= 8\cdot9m+9- 1\\
&= 8\cdot(9m+1)~.
\end{aligned}
\end{equation}
显然,$9^{k+1} - 1$ 是 $8$ 的倍数,所以能被 $8$ 整除。

由数学归纳法可知,对所有正整数$n$,$9^{n} - 1$ 能被 $8$ 整除。
\end{example}
除却上面的例子,如从数列的递推公式得到通项公式、多边形内角和公式以及排列组合或其他算法的递归性质上,都可以窥见数学归纳法的身影。可以这么说,数学归纳法使得很多过去觉得显然的命题,有了逻辑依赖。数学归纳法并不只在做题中有用,通常想要证明一个命题对实数都成立,可以先利用数学归纳法证明它对自然数成立,然后再扩展至有理数,最后利用极限再扩展至实数。因此,数学归纳法是很多命题的源头。



\subsection{*数学归纳法的变式}

下面介绍的方法都是可以从数学归纳法推知的其他类型的数学归纳法。因此,可以看作这些归纳法都是在数学归纳法的前提下给出了一些其他的条件来辅助。由于使用了一些条件,使得证明过程更简单。这些内容在高中阶段完全不涉及,仅作开阔视野。

\subsubsection{从$N$开始的数学归纳法}
比如从 n = 2 或 n = 0 起步。

\subsubsection{强归纳法}

:假设命题对多个前面情况成立,再推导出后续。

\subsubsection{逆归纳法}
