% 中国科技大学 2013 年考研普通物理
% keys 中科大|考研|普通物理|普物


\subsection{ 简答题(每题10分,共30分)}
\begin{enumerate}
\item 将一颗人造卫星发射到半径为 $R$ 的圆形轨道上,将第二颗卫星发射到半径为1.01 $R$ 的轨道上.问第二个卫星的周期比第一个卫星的周期是长些还是短些,或者相同?周期如果不同,那么相差百分之几?

\item 一强度为 $p$ 的电偶极子,在均匀电场 $E$ 中受力多大?

\item 两个点电荷的距离为 $d$ ,分别带电量 $Q$ 和 $-Q$ ,把它们分开到无穷远处,需要多大的能量.
\end{enumerate}
\subsection{计算题(每题20分,共120分)}
\begin{enumerate}
\item 一质点以初速率 $v_0$ 做直线运动,所受阻力与其速率成正比.试求当质点速率减为 $\frac{v_0}{n}(n>1)$ 时,质点经过的距离.

\item 如图1,细杆绕端点 $O$ 在平面内匀速旋转,角速度为 $\omega$ ,杆上一小环(可看作质点)相对杆做匀速运动,相对速度为 $v$ .设 $t=0$ 时刻小环位于杆的端点 $O$ .试求小环在任意时刻的速度和加速度.\\
\begin{figure}[ht]
\centering
\includegraphics[width=7cm]{./figures/USTC13_1.png}
\caption{计算题2图示} \label{USTC13_fig1}
\end{figure}

\item 如图2,在相距为 $l$ 的两平行弹性墙壁之间,有质量为 $m$ 的弹性小球以垂直于墙壁的初速度 $v_0$ 往返弹跳.设墙壁的质量远大于 $m$ ,碰撞时完全弹性的重力和空气阻力可以忽略.试求:(1)每墙壁所受的平均作用力;(2)若用外力使左墙壁以速度 $V(V<<v_0)$ 缓缓右移,证明外力作功等于小球动能的增加.\\
\begin{figure}[ht]
\centering
\includegraphics[width=5cm]{./figures/USTC13_2.png}
\caption{计算题3图示} \label{USTC13_fig2}
\end{figure}

\item 将两个电容器 $(C1,C2)$ 分别充电达相同的电势差 $(V)$ ,随后将一个电容器的正极与另一电容器的负极相联,然后将其他两极短路.(1)计算每个电容器的最终电荷.(2)计算电场能的损失.

\item 求产生球对称电势 $V(r)=e^{-\lambda r}/r$ 的电荷分布.

\item 理想气体经历 $V=\frac{1}{K} \ln \frac{p_{0}}{p}$ 的热力学过程,其中 $p_0$ 和 $K$ 是常数.试问:(1)当系统按此体积扩大一倍时,系统对外做了多少功?(2)在这一过程中的热容是多少?(理想气体的物态方程为: $pV=\mathrm v RT$ )
\end{enumerate}
