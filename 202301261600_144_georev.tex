% 地球的公转

\begin{lstlisting}[language=matlab]
clc
clear

function c=CIRCLE(x,y,z,r)
  t = linspace(0,2*pi,40);
  cx = r*cos(t)+x;
  cy = r*sin(t)+y;
  cz = zeros(size(t))+z;
  c=plot3(cx,cy,cz);
end

function s = SHPERE(x,y,z,r)
[X,Y,Z] = sphere(20);
X = r*X+x;
Y = r*Y+y;
Z = r*Z+z;
s = surf(X,Y,Z);
set(s,'EdgeColor','none')
end

R=5;

for t=linspace(0,365,365*70) 
%t unit in day, 0-1 is one day, 0-365 is one year.
  clf;

  hold on
  axis equal
  axis([-7 7 -7 7 -2 2])
  xlabel('X')
  ylabel('Y')
  zlabel('Z')
  view(0,30)

  sun = SPHERE(0,0,0,0.5);  %绘制太阳
  set(sun,'facecolor','r')
  CIRCLE(0,0,0,5); %地球公转轨道
  [mx my] = meshgrid(-R-2:R+2);
  mz = zeros(size(mx));
  surf(mx,my,mz,'edgecolor','none',
  'facecolor','y','faceAlpha',0.3) %黄道面

  ex = R*cos(2*pi/365*t);
  ey = R*sin(2*pi/365*t);
  earth = SPHERE(ex,ey,0,1); %绘制地球公转
  set(earth,'EdgeColor',[0.9 0.9 0.9],'FaceColor',[0.8 0.8 1]);
  rotate(earth,[0 0 1],mod(t,1)*360,[ex ey 0]); %自转
  rotate(earth,[0 1 0],23.5,[ex ey 0]); %自转倾角
  l = line([ex ex],[ey ey],[-5 5]);  %绘制自转轴
  rotate(l,[0 1 0],23.5,[ex ey 0]);

  [mx my] = meshgrid(ex-2:ex+2, ey-2:ey+2);
  mz = zeros(size(mx));
  p = surf(mx,my,mz,'edgecolor','none','facecolor','r','faceAlpha',0.3);
    %绘制赤道面
  rotate(p,[0 1 0],23.5,[ex ey 0]);

  rc = 1.01;
  cx = (R-rc)*ex/R;
  cy = (R-rc)*ey/R;
  c = CIRCLE(ex,ey,0,rc); %绘制直射纬度
  rotate(c,[0 1 0],23.5,[cx cy 0]);

  c = CIRCLE(ex,ey,0,1.01); %绘制晨昏线
  rotate(c,[0 1 0],90,[ex ey 0]);
  rotate(c,[0 0 1],2/365*t*180,[ex ey 0]);

  hold off

  pause(0.05);
end

\end{lstlisting}