% 克里斯蒂安·哥德巴赫(综述)
% license CCBYSA3
% type Wiki

本文根据 CC-BY-SA 协议转载翻译自维基百科\href{https://en.wikipedia.org/wiki/Christian_Goldbach}{相关文章}。

克里斯蒂安·哥德巴赫(/ˈɡoʊldbɑːk/ GOHLD-bahk,德语:[ˈkʁɪsti̯a(ː)n ˈɡɔltbax];1690年3月18日 – 1764年11月20日)是一位普鲁士数学家,参与了若干重要的研究,主要集中在数论领域;他还学习过法律,并对俄罗斯宫廷产生了兴趣并参与其中。[1][2] 在他早期的欧洲旅行之后,他于1725年以圣彼得堡科学院的教授身份来到俄罗斯。[3] 哥德巴赫于1737年共同领导了科学院。[4] 然而,他在1742年放弃了科学院的职务,并在俄罗斯外交部工作,直至1764年去世。[4] 他今天因哥德巴赫猜想和哥德巴赫–欧拉定理而被铭记。[1] 他与著名数学家莱昂哈德·欧拉有着深厚的友谊,并成为欧拉数学追求的灵感来源。[2]
\subsection{传记}  
\subsubsection{早年生活}  
哥德巴赫出生于普鲁士公国的首都哥尼斯堡(现在的加尔东),当时是勃兰登堡-普鲁士的一部分,父亲是位牧师。[2] 他在皇家阿尔伯图斯大学学习。[2][5] 完成学业后,他于1710年到1724年进行了长时间的欧洲学术旅行,访问了其他德国诸国、英格兰、荷兰、意大利和法国,期间与许多著名数学家见面,如戈特弗里德·莱布尼茨、莱昂哈德·欧拉和尼古拉斯·I·伯努利。这些交往激发了哥德巴赫对数学的兴趣。[6] 他于1713年短暂地就读于牛津大学,在那里,哥德巴赫与约翰·沃利斯和艾萨克·牛顿一起学习数学。[3] 此外,哥德巴赫的旅行还促进了他对语言学、考古学、形而上学、弹道学和医学的兴趣。[6] 在1717到1724年之间,哥德巴赫发表了他的几篇论文,虽然这些论文较为简略,但也证明了他的数学才能。回到哥尼斯堡后,他与乔治·伯恩哈德·比尔芬格和雅各布·赫尔曼相识。[2]
\subsubsection{圣彼得堡科学院}
\begin{figure}[ht]
\centering
\includegraphics[width=6cm]{./figures/20d59c10a9b68323.png}
\caption{圣彼得堡科学院大楼,名为‘艺术珍品馆’,建于1728年。} \label{fig_Goldba_1}
\end{figure}
哥德巴赫于1725年跟随比尔芬格和赫尔曼前往新开设的圣彼得堡科学院。[4] 克里斯蒂安·沃尔夫邀请了所有前往圣彼得堡参加学院的德国人,并为他们写了推荐信,除了哥德巴赫。[3] 哥德巴赫写信给学院的指定院长,请求在学院中获得一个职位,利用他过去的出版物以及在医学和法律方面的知识作为资格。[3][4] 随后,哥德巴赫被聘为数学教授和学院历史学家,签订了为期五年的合同。[3][4] 作为学院历史学家,他从1725年学院开校直到1728年1月,记录了每次学院会议的内容。[4] 哥德巴赫与著名数学家如莱昂哈德·欧拉、丹尼尔·伯努利、约翰·伯努利和让·勒朗·达朗贝尔合作。[5] 哥德巴赫还在欧拉决定将学术追求转向数学而非医学方面发挥了作用,巩固了数学在1730年代成为学院首要研究领域的地位。[3]
\subsubsection{俄罗斯政府工作} 
1728年,当彼得二世成为俄罗斯沙皇时,哥德巴赫成为彼得二世及其表妹安娜的导师。[4] 1729年,彼得二世将俄罗斯宫廷从圣彼得堡迁至莫斯科,于是哥德巴赫也随他去了莫斯科。[2][4] 哥德巴赫于1729年开始与欧拉通信,在信中可以找到一些哥德巴赫最重要的数学贡献。[2][5] 彼得二世于1730年去世后,哥德巴赫停止了教学,但继续协助安娜女皇。[4] 1732年,哥德巴赫回到了圣彼得堡科学院,并在安娜女皇将宫廷迁回圣彼得堡后继续在俄罗斯政府工作。[2][4] 返回科学院后,哥德巴赫被任命为通讯秘书。[3] 随着哥德巴赫的回归,他的朋友欧拉也继续在学院进行教学和研究。[3] 1737年,哥德巴赫和J.D.舒马赫共同接管了科学院的管理工作。[4] 此外,哥德巴赫还在安娜女皇的统治下担任了俄罗斯宫廷的职务。[2][4] 在安娜去世和伊丽莎白女皇执政后,他成功地保持了自己在宫廷中的影响力。[2] 1742年,他进入了俄罗斯外交部,再次离开了科学院。[4] 由于他在俄罗斯政府中的优秀表现和地位的提升,哥德巴赫获得了土地和更高的薪水。[2] 1760年,哥德巴赫为皇室子女的教育制定了新的指南,这些指南将持续了100年。[2][4] 他于1764年11月20日在莫斯科去世,享年74岁。

克里斯蒂安·哥德巴赫精通多种语言——他用德语和拉丁语写日记,信件则用德语、拉丁语、法语和意大利语书写,官方文件则使用俄语、德语和拉丁语。[7]
\subsection{贡献}
\begin{figure}[ht]
\centering
\includegraphics[width=6cm]{./figures/0b683c0c5cb7d58d.png}
\caption{哥德巴赫致欧拉的信,1742年} \label{fig_Goldba_2}
\end{figure}
哥德巴赫最著名的是与莱布尼茨、欧拉和伯努利的通信,特别是在1742年他写给欧拉的信中提出了哥德巴赫猜想。他还研究并证明了一些关于完美幂的定理,如哥德巴赫–欧拉定理,并对分析学做出了若干重要贡献。[1] 他还证明了一个关于费马数的结果,被称为哥德巴赫定理。
\subsubsection{对欧拉的影响} 
哥德巴赫与欧拉的通信包含了哥德巴赫对数学,特别是数论领域的一些最重要贡献。[2] 哥德巴赫与欧拉的友谊延续了哥德巴赫于1728年移居莫斯科之后,二人继续保持通信。[3] 他们的通信跨越了35年,共196封信,信件用拉丁语、德语和法语书写。[6] 这些信件涵盖了广泛的主题,包括各种数学话题。[2] 哥德巴赫是欧拉在数论方面兴趣和工作的主要影响者。[3] 大多数信件讨论了欧拉在数论和微分学方面的研究。[3] 直到1750年代末,欧拉关于数论研究的通信几乎完全是与哥德巴赫的通信。[3]
\begin{figure}[ht]
\centering
\includegraphics[width=6cm]{./figures/d9c4ca841b970fed.png}
\caption{莱昂哈德·欧拉的肖像,他是历史上最杰出的数学家之一} \label{fig_Goldba_3}
\end{figure}
哥德巴赫早期的数学工作和他在写给欧拉的信中的想法直接影响了欧拉的一些工作。1729年,欧拉解决了两个关于数列的问题,这些问题曾令哥德巴赫困惑。[3] 随后,欧拉向哥德巴赫概述了这些问题的解法。[3] 此外,在1729年,哥德巴赫对巴塞尔问题进行了精确的近似,这激发了欧拉的兴趣,并促成了欧拉的突破性解决方案。[3] 通过他的信件,哥德巴赫在1730年代通过与欧拉讨论费马猜想,帮助欧拉保持对数论的关注。[3] 随后,欧拉提供了对该猜想的证明,并将哥德巴赫视为将他引入这一子领域的人。[3] 欧拉随后写下了560篇著作,这些著作在他去世后以《欧拉全集》四卷本出版,其中一些作品受到了哥德巴赫的影响。[3] 哥德巴赫的著名猜想以及他与欧拉的通信证明了他是少数几位能在费马关于数论的革命性思想的背景下理解复杂数论的数学家之一。[8]
\subsection{著作}  
\begin{itemize}
\item (1729) 《级数变换论》  
\item (1732) 《级数的一般项论》
\end{itemize}
\subsubsection{另见}  
\begin{itemize}
\item 哥德巴赫彗星
\end{itemize}
\subsection{参考文献} 
\begin{enumerate}
\item Rosen, Kenneth H. (2004). 《初等数论》(第五版)。Addison-Wesley. ISBN 0-321-23707-2.  
\item "Christian Goldbach - 传记". 数学历史. 检索于2022-10-11.  
\item Calinger, Ronald C (2015). 《莱昂哈德·欧拉:启蒙时代的数学天才》。普林斯顿大学出版社. 第50–51, 66–80, 328–345页. ISBN 978-0691119274.  
\item "Christian Goldbach". 《大英百科全书》. 大英百科全书公司. 2013年8月16日. 检索于2014年10月26日.  
\item Aliprandini, Michael (2017-01-08). "Christian Goldbach". 《伟大科技出版》:1–2 – 通过EBSCO.  
\item Haas, Robert (2014-02-01). "哥德巴赫、赫尔维茨与质数的无穷性:跨世纪的证明编织*". 《数学智者》. 36 (1): 54–60. doi:10.1007/s00283-013-9402-8. ISSN 1866-7414. S2CID 253817631.  
\item Adolf Juskevic, Judith Kopelevic: 《克里斯蒂安·哥德巴赫1690-1764》(数学生命),Birkhäuser出版社,1994年,ISBN 3764326786,第XII页.  
\item "Goldbach, Christian". 《科学传记全词典》。查尔斯·斯克里布内公司,2008年,第448–451页. 检索于2022-10-20 – 通过Gale In Context: 美国历史.
\end{enumerate}
\subsection{外部链接} 
\begin{itemize}
\item 与克里斯蒂安·哥德巴赫相关的媒体,见于维基共享资源  
\item O'Connor, John J.; Robertson, Edmund F., "克里斯蒂安·哥德巴赫",圣安德鲁斯大学《麦克图尔数学历史档案》  
\item 欧拉与哥德巴赫的电子通信副本  
\item 《最新天文图集》,1799年 - 完整数字化副本,琳达·霍尔图书馆
\end{itemize}