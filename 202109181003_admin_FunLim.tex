% 函数的极限(简明微积分)

\pentry{数列的极限(简明微积分)\upref{Lim0}}

实函数 $f(x)$ 可以看成是一种 “连续” 的数列, 只不过把元素编号从离散的 $n$ 改为连续的 $x$. 类比数列的极限, 我们也可以定义\textbf{函数在正无穷的极限}.

\begin{definition}{函数趋于正无穷时的极限}\label{FunLim_def1}
考虑实函数 $f(x)$. 若存在实数 $A$, 使得对于\textbf{任意} $\epsilon>0$, 总存在\textbf{正实数} $X_\epsilon$, 使得对于所有 $x>X_\epsilon$, 都有 $\abs{f(x)-A}<\epsilon$, 那么我们说 $A$ 是函数 $f(x)$ 在 $x$ 趋于正无穷时的极限, 记为
\begin{equation}
\lim\limits_{x\to +\infty} f(x) = A
\end{equation}
\end{definition}

可以看到该定义和数列极限的定义(\autoref{Lim0_def2}~\upref{Lim0})非常相似, 只是简单做了替换.不过,函数并不是简单地把数列的概念拓展到连续的情况.数列的编号只能朝着一个方向增大,但实函数的自变量就自由得多,它可以奔向负无穷,也可以集中到一点 $x_0$.

同理, 我们也可以定义 $\lim\limits_{x\to -\infty} f(x) = A$(留做习题).

如何描述 “自变量趋于一个给定的实数 $x_0$” 呢? 只需要取自变量 $x$ 使得 $\abs{x-x_0}$ 越来越接近 $0$ 即可.

% 现在问题来了,什么叫“越来越”呢?在讨论数列极限的时候,我们没有在意这个细节,因为我们也只能考虑数列编号增大的情况,而这里的“越来越”也自然表示“随着数列编号的增大”了.但是讨论函数自变量趋于给定实数$x_0$的时候,就有些麻烦了,我们没有一个衡量“时间流逝”的自然标准了.要解决这个问题,最好还是再次把数列给请出来.

% 下面,我直接给出函数极限的定义,请仔细咀嚼,看看数列是怎么用来准确描述函数极限的.

% \begin{definition}{函数的极限}\label{Lim0_def3}
% 考虑实函数$f(x)$,并给定一个实数$x_0$.

% 取数列$\{x_n\}$,使得$\lim\limits_{n\to\infty}=x_0$.这样,$\{f(x_n)\}$也构成一个数列.

% 如果对于\textbf{任意}的满足上述要求的数列$\{x_n\}$,都有实数$A$使得$\lim\limits_{n\to\infty}f(x_n)=A$,那么我们说$f(x)$在$x$趋近于$x_0$时的极限为$A$.

% 将这个极限表述为$\lim\limits_{x\to x_0}f(x)=A$.
% \end{definition}

% 注意看\autoref{Lim0_def3} 中是如何从$\lim\limits_{n\to\infty}=x_0$和$\lim\limits_{n\to\infty}f(x_n)=A$过渡到$\lim\limits_{x\to x_0}f(x)=A$的.这里,从离散到连续的桥梁,正是“任意”二字.也就是说,“连续”就是“任意的离散”,比如“连续地接近时满足的条件”就是“任意一种离散地接近时都满足的条件”.

% \autoref{Lim0_def3} 同时也是最为完整的函数极限定义,只需要把$x_0$替换为$\pm\infty$即可囊括无穷的情况.

% % 与数列不同的是, 对于函数我们还可以定义\textbf{函数在负无穷的极限} $\lim\limits_{x\to -\infty} f(x)$(把以上定义的 $>$ 号改成 $<$ 号即可).

% % 另外可以定义 \textbf{$f(x)$ 在 $x_0$ 处的极限} $A$, 即“ 对 $\forall \varepsilon > 0$, $\exists \delta > 0$, 当 $\abs{x - x_0} < \delta$ 时,有 $\abs{f(x) - A} < \varepsilon$”. 注意 $f(x)$ 不需要在 $x_0$ 处有定义.

% \begin{example}{}
% 求函数在某个值处的极限时, 通常可以直接代入数值计算, 如
% \begin{equation}
% \lim_{x\to 1} 2x + 1 = 3 \qquad \lim_{x\to 2}\frac{x + 1}{x + 2} = \frac34
% \end{equation}

% 当无穷大与常数相加时, 可以忽略常数, 如
% \begin{equation}
% \lim_{x\to +\infty} \frac{x + 1}{2x + 2} = \lim_{x\to +\infty} \frac{x}{2x} = \frac12
% \end{equation}
% \end{example}

% \begin{example}{}\label{Lim0_ex2}
% 考虑符号函数$\opn{sgn}(x)$,其定义为:$x>0$时,$\opn{sgn}(x)=1$,$\opn{sgn}(-x)=-1$,且有$\opn{sgn}(0)=0$.也就是说,正数的函数值为$1$,负数的为$2$,$0$的就是$0$.

% 现在考虑$\opn{sgn}(x)$在$0$处的极限值.我们首先要研究那些趋近于$0$的数列.

% 如果我们取数列$g_n=\frac{1}{2^n}$,那么$\opn{sgn}(g_n)$中各项都恒为$1$,因此$\lim\limits_{n\to\infty}\opn{sgn}(g_n)=1$.

% 但是如果取数列$h_n=-\frac{1}{2^n}$,那么$\lim\limits_{n\to\infty}\opn{sgn}(h_n)=-1$.

% 如果再取数列$j_n=0$,那么$\lim\limits_{n\to\infty}\opn{sgn}(j_n)=0$.

% 以上三个数列都趋于$0$,但由它们构造的$\opn{sgn}(g_n)$、$\opn{sgn}(h_n)$和$\opn{sgn}(j_n)$的极限却各不相同.这就意味着$\opn{sgn}(x)$在$0$处\textbf{并没有极限值}.
% \end{example}

% \begin{definition}{函数的敛散性}
% 拓展数列的敛散性的\autoref{Lim0_def4} .若函数在一点处有极限值,则称之为\textbf{收敛}的;否则,称之为\textbf{发散}的.
% \end{definition}

% \subsection{左极限和右极限}

% 自变量趋于无穷的过程,只有一个方向,要么是不停增大(正无穷),要么是不停减小(负无穷).但如果自变量是趋近一个实数$x_0$,那么至少就有两个方向,从大于$x_0$的点开始减小(正向接近),和从小于$x_0$的点开始增大(负向)接近.

% 由于极限的定义是“怎么接近都可以”,因此若$f(x)$在$x=x_0$处存在极限,无论怎么取接近$x_0$的数列,正向接近也好反向接近也罢,哪怕是一会儿正一会儿负地反复横跳,只要接近,这些数列的极限值都是一样的.

% 但是有些函数则不然.考虑这个函数:$f(x)$,其中$x<0$时$f(x)=0$,$x\geq 0$时$f(x)=1$.在$x=0$处,如果\textbf{只考虑}正向接近的数列$\{x_n\}$,那么计算出来的$\{f(x_n)\}$的极限就是$1$;但如果\textbf{只考虑}负向接近的数列,那么计算出来的极限是$0$.按照定义,这意味着$f(x)$在$x=0$处没极限.

% 但是这种情况,我们说它是有左极限和右极限的.

% \begin{definition}{左极限和右极限}
% 对于函数$f(x)$,给定实数$x_0$.

% 如果取任意\textbf{正}向接近的数列$\{x_n\}$,所得到的数列$\{f(x_n)\}$的极限都是$A$,那么称$A$是$f(x)$在$x=x_0$处的\textbf{右极限(right limit)}.

% 如果取任意\textbf{负}向接近的数列$\{x_n\}$,所得到的数列$\{f(x_n)\}$的极限都是$B$,那么称$B$是$f(x)$在$x=x_0$处的\textbf{左极限(left limit)}.
% \end{definition}

% 一个很容易想到的定理是,如果函数在某点的左右极限都存在且相等,那么函数的极限存在且等于左右极限.证明留作思考题,要注意的是,左右极限相等并没有直接说明“左右横跳”式的数列,其极限也等于左右极限.





% \subsection{无穷小的阶}
% 如果令 $x\to 0$, 我们就说 $x$ 是\textbf{无穷小}. 但一些无穷小会更快地趋近于 $0$, 若 $x$ 的某个函数 $\alpha(x)$ 满足
% \begin{equation}
% \lim_{x\to 0} \frac{\alpha(x)}{x} = 0
% \end{equation}
% 那 $\alpha(x)$ 就是 $x$ 的\textbf{高阶无穷小}. 若
% \begin{equation}
% \lim_{x\to 0} \frac{\alpha(x)}{x^n} \ne 0
% \end{equation}
% 则称 $\alpha(x)$ 为 $x$ 的 $n$ 阶无穷小. 例如, $c x^n$ ($c$ 为常数)就是 $x$ 的 \textbf{$n$ 阶无穷小}, 记为 $\order{x^n}$.

% 在求极限时, 若高阶无穷小与低阶无穷小相加, 通常可以忽略高阶无穷小. 另外由定义不难推出
% \begin{equation}
% \order{x^n} x^m = \order{x^{n + m}} \qquad (m > -n)
% \end{equation}

% 在物理中, 当我们用一个函数 $g(x)$ 来近似另一个函数 $f(x)$ 并记为 $f(x) = g(x) + \order{h^n}$ 时(这里 $x$ 是函数的自变量, $h$ 是函数表达式中一个较小的常数), 就说 $g(x)$ 的误差为 $\order{h^n}$.
