% 一维 delta 势能晶格
% 能带|导体|绝缘体|delta 函数|循环边界条件

\begin{issues}
\issueDraft
\end{issues}

\pentry{狄拉克 delta 函数\upref{Delta}, 布洛赫理论\upref{Bloch}}

\footnote{参考\cite{GriffQ}}本文使用原子单位制\upref{AU}. 势能为
\begin{equation}
V(x) = \alpha \sum_{j\in \mathbb Z} \delta(x - ja)
\end{equation}
\begin{equation}
k = \sqrt{2mE}
\end{equation}
在 $x = ja$ 处匹配边界条件, 得
\begin{equation}
\cos(K a) = \cos(ka) + \frac{m\alpha}{k}\sin(ka)
\end{equation}
由此可以对不同的 $K$ 解出不同的 $k$ 或能量 $E$, 每个解对应一个 “束缚态”.

解出来以后会发现得到了许多\textbf{能带(band)}, 每个能带中有间隔密集的能级, 能带之间存在\textbf{能隙或带隙(band gap)}.

如果电子把能带刚好填满, 那么就是\textbf{绝缘体}; 如果没有满就是\textbf{导体}. 如果势能有参杂, 导致出现了空穴以及多余的电子跑到上一个能带, 那就是\textbf{半导体}.
