% QED的重整化理论(单圈修正)
% keys 重整化|单圈修正|量子电动力学

\pentry{QED的费曼规则\upref{qedfey}}

裸的拉氏量为
\begin{equation}
\begin{aligned}
\mathcal{L}=
\bar{\psi}_0 (i\not\partial -m_0)\psi_0 -\frac{1}{4}(F_0^{\mu\nu})^2 - e_0\bar\psi_0 \gamma_\mu\psi_0 A_0^\mu
\end{aligned}
\end{equation}
为了消除圈图的紫外发散,采取一定的正规化方案(例如我们最常使用的维数正规化方案),然后再调整裸参数使得可观测量的计算结果与实验相符,即满足一定的重整化条件。或者我们采取 OS 方案或 $\overline{MS}$ 方案,不同的重整化方案实际上是对场量进行了平移和缩放。设从裸的拉氏量出发,得到的电子和光子的两点编时格林函数为
\begin{equation}
\begin{aligned}
&G^{(2)}(p^2)=\int\dd[4]{x} e^{ipx} \bra{\Omega}\psi_0(x)\psi_0(0)\ket{\Omega}=\frac{iZ_2}{\not p-m+i\epsilon} + \cdots\\
&D_{\mu\nu}^{(2)}(p^2)=\int\dd[4]{x} e^{ipx} \bra{\Omega}A_{0\mu}(x)A_{0\nu}(0)\ket{\Omega}=\frac{iZ_3}{\not p-m+i\epsilon}+\cdots
\end{aligned}
\end{equation}
因此我们可以定义新的场量 $\psi_0=\sqrt{Z_2} \psi,A_{0\mu} = \sqrt{Z_3} A_\mu$,得到新的拉氏量
\begin{equation}
\mathcal{L} = Z_2\bar\psi (i\not\partial - m_0)\psi - \frac{1}{4}Z_3 (F^{\mu\nu})^2-e_0Z_2Z_3^{1/2} \bar\psi \gamma^\mu\psi A_\mu
\end{equation}
再将 $m_0,e_0$ 吸收进新的参数中,定义:
\begin{equation}
Z_m = Z_2 m_0 / m,\quad Z_1 = e_0Z_2Z_3^{1/2} / e
\end{equation}
其中 $e,m$ 为物理电荷量(在大的空间尺度上测得的量)和物理质量。利用这些新的重整化参数 $Z_1,Z_2,Z_3,Z_m$,我们写出重整化的拉氏量。
\begin{equation}
\begin{aligned}
\mathcal{L} = \bar\psi (iZ_2\not\partial - Z_m m)\psi -\frac{1}{4}Z_3F^{\mu\nu}F_{\mu\nu}-Z_1 e\bar\psi \gamma^\mu \psi A_\mu
\end{aligned}
\end{equation}
提取出其中自由场的部分后,微扰的拉氏量就是
\begin{equation}
\mathcal{L}_1 = - Z_1 e \bar\psi \gamma^\mu\psi A_\mu + \mathcal{L}_\text{CT},\quad \mathcal{L}_\text{CT} = \bar\psi (i\delta_2 \not\partial - \delta_m)\psi - \frac{1}{4}\delta_3 F^{\mu\nu}F_{\mu\nu}
\end{equation}
由于重整化不改变理论的对称性,我们要求重整化的拉氏量仍然满足 $U(1)$ 规范对称性,那么重整化参数要满足 $Z_1=Z_2$。这里的证明实际上是不那么严格的,之后会利用 Ward 等式给出一个严格的证明。
\subsection{OS 重整化方案}
\subsubsection{重整化条件}
在 OS 重整化方案下,我们要求电子的正规传播子(两点编时格林函数的傅里叶变换)在物理质量处的留数为 $1$,光子的正规传播子在 $p^2=0$ 处留数为 $1$。除此以外,我们还要求电子光子顶点 $-ie\Gamma^\mu(p'-p)$ 在 $q=p'-p$ 趋于 $0$ 时的极限为 $-ie\gamma^\mu$。这四个重整化条件可以用于确定四个重整化参数。
\begin{equation}\label{qedlop_eq1}
\begin{aligned}
\Sigma(\not p=m) &= 0;\\
\left.\frac{\dd }{\dd p}\Sigma(\not p)\right|_{\not p=m}&=0;\\
\Pi(q^2=0)&=0;\\
-ie\Gamma^\mu(p'-p=0)&=-ie\gamma^\mu
\end{aligned}
\end{equation}
其中 $i\Pi^{\mu\nu}$ 为光子的单粒子不可约图的贡献,$-i\Sigma(\not p)$ 为电子的单粒子不可约图的贡献。
其中自能函数 $\Sigma(\not p)$ 是将 $\not p$ 看作这个矩阵代数中的一个元素,由于自能函数的洛伦兹结构,$\Sigma(\not p)$ 一定可以表示成关于 $\not p, p^2$,而 $\not p$ 同这个代数中的其他任何元素都可交换,因此\autoref{qedlop_eq1} 的第一第二行可以被良好地定义。
\subsubsection{电子自能单圈修正}
计算 $-i\Sigma(\not p)$ 的单圈修正:
\begin{equation}
\begin{aligned}
-i\Sigma(\not p)
&=(-ie)^2 \int\frac{\dd[4]{k}}{(2\pi)^4} \gamma^\mu\frac{i}{\not k-m+i\epsilon} \gamma^\nu  \frac{-ig_{\mu\nu}}{(p-k)^2+i\epsilon}+i(\delta_2\not p -\delta_m)\\
&=-e^2\tilde{\mu}^{4-d} \int\frac{\dd[d]{k}}{(2\pi)^d} \frac{\gamma^\mu \not k \gamma_\mu + m\gamma^\mu\gamma_\mu}{(k^2-m^2+i\epsilon)((p-k)^2+i\epsilon)}+i(\delta_2\not p -\delta_m)
\\
&=-e^2\tilde{\mu}^{4-d} \int\frac{\dd[d]{k}}{(2\pi)^d} \frac{(2-d)\not k + dm}{(k^2-m^2+i\epsilon)((p-k)^2+i\epsilon)}+i(\delta_2\not p -\delta_m)
\end{aligned}
\end{equation}
出于以后的考虑,我们引入一个光子质量项 $m_\gamma^2$,并期望在最后的计算中取 $m_\gamma\rightarrow 0$ 的极限。利用\autoref{qedlop_eq2} 和\autoref{qedlop_eq3} 的计算结果,我们有
\begin{equation}
\begin{aligned}
-i\Sigma(\not p)
&=-e^2\tilde{\mu}^{4-d} \int\frac{\dd[d]{k}}{(2\pi)^d} \frac{(2-d)\not k + dm}{(k^2-m^2+i\epsilon)((p-k)^2-m_\gamma^2+i\epsilon)}+i(\delta_2\not p -\delta_m)
\\
&= -i\frac{e^2}{(4\pi)^{d/2}}\tilde{\mu}^{4-d}
\int_0^1 \dd x \frac{\Gamma(2-d/2)}{((1-x)m^2+x m_\gamma^2-x(1-x)p^2)^{2-d/2}}((2-d)x\not p+dm)
+i(\delta_2\not p -\delta_m)
\\
(d\rightarrow 4)&= -i\frac{e^2}{(4\pi)^2}  \int_0^1 \dd x \left[\frac{2}{\epsilon}+\log\left(\frac{\mu^2}{D}\right)\right]\cdot \left[(4-\epsilon)m -(2-\epsilon)x\not p\right]
+i(\delta_2\not p -\delta_m),\\
&=-i\frac{e^2}{8\pi^2}\left[(4m-\not p)\frac{1}{\epsilon}-\left(m-\frac{\not p}{2}\right)+ \int_0^1\dd x (2m-x\not p)\log\left(\frac{\mu^2}{D}\right) \right]
+i(\delta_2\not p -\delta_m)\\
D&=(1-x)m^2+xm_\gamma^2-x(1-x)p^2
\end{aligned}
\end{equation}
因此在 OS 方案下,根据自能函数的重整化条件\autoref{qedlop_eq1} ,我们可以得到
\begin{equation}
\begin{aligned}
m\delta_2 - \delta_m &= \frac{e^2}{8\pi^2}m\left[3\frac{1}{\epsilon}-\frac{1}{2}+ \int_0^1\dd{x} (2-x)\log\left(\frac{\mu^2}{D_0}\right) \right],\\
\delta_2 &= \frac{e^2}{8\pi^2}\left[-\frac{1}{\epsilon}+\frac{1}{2}-\int_0^1 \dd{x}\ x\log\left(\frac{\mu^2}{D_0} \right) 
+\int_0^1 \dd{x} (xm-2m)\frac{-2x(1-x)m}{D_0}
\right]
\end{aligned}
\end{equation}
这里 $D_0 = D|_{p^2=m^2}=(1-x)^2 m^2+x m_\gamma^2 $。继续对 $\delta_2$ 化简:
\begin{equation}
\begin{aligned}
\delta_2 &= \frac{e^2}{8\pi^2}\left[-\frac{1}{\epsilon}+\frac{1}{2}-\int_0^1 x\log\left(\frac{\mu^2}{D_0} \right) 
+\int_0^1\dd{x} \frac{2x(2-x)(1-x)m^2}{(1-x)^2m^2+xm_\gamma^2}
\right]
\end{aligned}
\end{equation}
那么自能函数就是
\begin{equation}
\begin{aligned}
-i\Sigma(\not p)=i\frac{e^2}{8\pi^2}\left[\int_0^1\dd x(2m-x\not p)\log\left(\frac{D}{D_0}\right)+(\not p-m)\int_0^1\dd{x}\frac{2x(2-x)(1-x)m^2}{(1-x)^2m^2+xm_\gamma^2} \right]
\end{aligned}
\end{equation}
可以看到自能函数中的 $1/\epsilon$ 发散被抵消项消去。但是这里潜藏着一个问题,当我们将光子质量项 $m_\gamma^2$ 取为 $0$,积分将是发散的。这个问题也被称为\textbf{红外发散},我们将在红外发散的词条中具体讨论这个问题。
\subsubsection{光子自能单圈修正}


\subsection{维数正规化下 Feynman 积分计算}
\begin{exercise}{}
计算以下 Feynman 积分
\begin{equation}
I^\mu = \tilde{\mu}^{4-d}\int\frac{\dd[d]{k}}{(2\pi)^d} \frac{k^\mu}{(k^2-m^2+i\epsilon)((p-k)^2+m_\gamma^2+i\epsilon)}
\end{equation}
\end{exercise}
先对该 Feynman 积分进行一些分析。由于根据洛伦兹结构,上述积分是一个洛伦兹矢量,且是关于 $p^\mu$ 的函数,因此
\begin{equation}
I^\mu = A(p^2) p^\mu
\end{equation}
我们再来分析其紫外发散。根据 power counting,可以看到它最多包括线性发散和对数发散,而在紫外区域被积函数是关于 $k^\mu$ 的奇函数,所以线性发散实际上为 $0$。下面我们来作具体的计算。

利用 Feynman 参数化
\begin{equation}\label{qedlop_eq2}
\begin{aligned}
I^\mu 
&= \tilde{\mu}^{4-d} \int\frac{\dd[d]{k}}{(2\pi)^d}\int_0^1\dd x \frac{k^\mu}{(k^2+ xp^2-x m_\gamma^2-2xpk-(1-x)m^2 +i\epsilon)^2}\\
&=\tilde{\mu}^{4-d}\int_0^1\dd x\int\frac{\dd[d]{k}}{(2\pi)^d}  \frac{k^\mu}{((k-xp)^2-\Delta+i\epsilon)^2},\quad \Delta = (1-x)m^2+x m_\gamma^2-x(1-x)p^2\\
&=\tilde{\mu}^{4-d} \int_0^1 \dd x \int\frac{\dd[d]{l}}{(2\pi)^d} \frac{l^\mu + x p^\mu}{(l^2-\Delta+i\epsilon)^2}\\
&=\tilde{\mu}^{4-d} \int_0^1 \dd x(x p^\mu)\cdot\int \frac{\dd[d]{l}}{(2\pi)^d} \frac{1}{(l^2-\Delta+i\epsilon)^2}\\
&=p^\mu \tilde{\mu}^{4-d} \int_0^1 \dd x (x)\cdot \frac{i}{(4\pi)^{d/2}}\Gamma(2-d/2)\frac{1}{\Delta^{2-d/2}}\\
&=p^\mu \frac{i}{(4\pi)^{d/2}} \tilde{\mu}^{4-d} \int_0^1 \dd x \frac{x\cdot \Gamma(2-d/2)}{((1-x)m^2+x m_\gamma^2-x(1-x)p^2)^{2-d/2}}
\end{aligned}
\end{equation}
$d=4$ 为该积分的极点,因此有对数发散。令 $\epsilon = 4-d$,那么
\begin{equation}\label{qedlop_eq3}
\begin{aligned}
I^\mu = p^\mu \frac{i}{(4\pi)^2} \int_0^1 \dd x \left[\frac{2}{\epsilon}+\log\left(\frac{4\pi \tilde{\mu}^2 e^{-\gamma}}{(1-x)m^2+x m_\gamma^2 - x(1-x)p^2}\right)\right]\cdot x\\
= p^\mu \frac{i}{(4\pi)^2} \int_0^1 \dd x \left[\frac{2}{\epsilon}+\log\left(\frac{\mu^2}{(1-x)m^2+x m_\gamma^2 - x(1-x)p^2}\right)\right]\cdot x
\end{aligned}
\end{equation}
