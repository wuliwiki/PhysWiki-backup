% 深度学习 CNN 入门
% keys 深度学习
% license Usr
% type Tutor
现在人工智能很火,其中卷积神经网络在计算机视觉等各个领域都有所运用,此文主在解释其背后原理。
而CNN的运用将在下一章进行详细介绍(主要是对计算机视觉领域)。

在了解卷积神经网络前,先得知道神经网络是一种受到生物神经系统启发的人工智能模型,它重现了大脑中神经元之间相互连接的方式。神经网络在诸多领域中取得了显著成就,如图像识别、自然语言处理和语音识别等。这篇博客将为您解释神经网络的构造,让您能够理解这个令人着迷的领域的基本工作原理。
\subsection{神经元 }
首先需要了解神经元,这是神经网络的基本构建块。

1.神经元的结构:每个神经元都由细胞体、树突和轴突组成。细胞体包含核心部分,树突接收来自其他神经元的信号,而轴突将信号传递给其他神经元。
2.信号传递:神经元之间的通信是通过电化学信号完成的。当信号通过树突传递到细胞体时,如果达到一定阈值,神经元就会触发并将信号传递给下一个神经元。
\begin{figure}[ht]
\centering
\includegraphics[width=12cm]{./figures/c714d3a47bfcc267.png}
\caption{神经元} \label{fig_CNN1_1}
\end{figure}

\subsection{神经元数学模型 }
我们将生物神经元的概念转化为数学模型。人工神经元是神经网络的基本构建块,负责对输入进行处理和传递信号。输入可以类比为神经元的树突,而输出可以类比为神经元的轴突,计算则可以类比为细胞核。

输入和权重:人工神经元接收多个输入,每个输入都有一个相关联的权重,这相当于人工神经网络的记忆。这些权重决定了每个输入对神经元的影响程度。\begin{figure}[ht]
\centering
\includegraphics[width=13cm]{./figures/98ed639285060e32.png}
\caption{神经元数学模型} \label{fig_CNN1_2}
\end{figure}
激活函数:在人工神经元中,激活函数决定了神经元是否激活(发送信号)。常见的激活函数包括Sigmoid、ReLU和Tanh。
\begin{figure}[ht]
\centering
\includegraphics[width=13cm]{./figures/33e99444a6098720.png}
\caption{激活函数} \label{fig_CNN1_3}
\end{figure}

\textbf{神经网络}:是由大量的节点(或称“神经元”)和之间相互的联接构成。而由两层神经元组成的神经网络称之为--“感知器”(Perceptron),感知器只能线性划分数据。在输入和权值的线性加权和叠加了一个函数g(激活函数),加权计算公式为:
\begin{equation}
g(W * x) = z~.
\end{equation}
\begin{figure}[ht]
\centering
\includegraphics[width=13cm]{./figures/06c91b454d4f9f50.png}
\caption{神经网络} \label{fig_CNN1_4}
\end{figure}
现在我们可以将多个人工神经元组合在一起,形成神经网络。神经网络由多个层组成,包括输入层、隐藏层和输出层,也称为多层感知器。
在神经网络中需要默认增加偏置神经元(节点),这些节点是默认存在的。它本质上是一个只含有存储功能,且存储值永远为1的单元。在神经网络的每个层次中,除了输出层以外,都会含有这样一个偏置单元。(如下图)\begin{figure}[ht]
\centering
\includegraphics[width=13cm]{./figures/76533dbf87705924.png}
\caption{多层神经网络} \label{fig_CNN1_5}
\end{figure}
输入层:接收原始数据的输入,例如图像像素或文本单词。

隐藏层:这是神经网络的核心部分,包含多个层次的神经元。隐藏层负责从输入中学习特征并生成有用的表示。

输出层:根据学到的特征生成最终的输出,可以是分类标签、数值或其他任务相关的结果。

需要注意,输入层的节点数:与特征的维度匹配。输出层的节点数:与目标的维度匹配。

中间层的节点数:目前业界没有完善的理论来指导这个决策。一般是根据经验来设置。较好的方法就是预先设定几个可选值,通过切换这几个值来看整个模型的预测效果,选择效果最好的值作为最终选择。

结构图里的关键不是圆圈(代表“神经元”),而是连接线(代表“神经元”之间的连接)。每个连接线对应一个不同的权重(其值称为权值),这是需要训练得到的。
