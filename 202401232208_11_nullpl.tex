% 零化多项式
% license Usr
% type Tutor


\begin{issues}
\issueDraft 
\end{issues}

\begin{definition}{}
设$f(x)$是以域$\mathbb F$中元素为系数的一元多项式,$A$是线性空间$V$上的线性变换。若$f(A)=0$(即零变换),则称$f$是$A$的一个\textbf{零化多项式}(null polynomial)。
\end{definition}
可以验证,若$f$是线性变换$A$的零化多项式,那么也是其任意基下矩阵所对应的零化多项式,即无论选取什么基底,代入多项式的最终结果为零矩阵。这是因为若$A=Q^{-1}BQ$,我们有$f(A)=Q^{-1}f(B)Q$。


下面一条定理及其推论表明了寻找零化多项式的重要意义。
\begin{theorem}{}
给定域$\mathbb F$上的线性空间$V$,域上的多项式$h$可以分解为互素多项式:$h=fg$。对于定义在该线性空间的线性变换$A$,我们有:
\begin{equation}
\opn{ker}h(A)=\opn{ker}f(A)\oplus\opn{ker}g(A)~.
\end{equation}
\end{theorem}
Proof.

首先证明$\opn{ker}f\cap \opn{ker}g=0$。由互素得,存在多项式$u,v$使得$uf+vg=I$。设$\bvec x\in \opn{ker}f\cap \opn{ker}g$,则$(uf+vg)\bvec x=0=\bvec x$。矛盾,因而$f,g$无交集,和为直和。

下证$\opn{ker}h=\opn{ker}f\oplus\opn{ker}g$。
第一步,先证$\opn{ker}h\subset\opn{ker}f\oplus\opn{ker}g$。设$\{\bvec x_i\}$和$\{\bvec y_i\}$分别是$\opn{ker}f$和$\opn{ker}g$的基,则$fg(a^i\bvec x_i+b^i \bvec y_i)=a^igf(\bvec x_i)+b^ifg(\bvec y_i)=0$,第一步得证。

第二步证明$\opn{ker}h\supset\opn{ker}f\oplus\opn{ker}g=\opn{ker}f+\opn{ker}g$。设$\bvec x\in \opn{ker}h$,由之前的证明过程知:$\bvec x=(uf+vg)\bvec x$,只要把这两项分配给$f,g$的核即可。显然,$g(uf\bvec x)=f(vg\bvec x) =0$,得证。

\begin{corollary}{}
$\mathbb F,V,A,h$同上设。$h$可以分解为\textbf{两两互素}的多项式乘积:$h=h_1h_2...h_n$,则:
\begin{equation}
\opn{ker}h(A)=\opn{ker}h_1(A)\oplus\opn{ker}h_2(A)\oplus...\oplus\opn{ker}h_n(A)~.
\end{equation}
\end{corollary}

若$h(A)=0$,则$\opn{ker} h=V$,该条定理意味着若我们找到任意线性变换$A$的零化多项式,则可以把线性空间分解为互素多项式的核。若$f$是$A$的零化多项式,且$\bvec x\in \opn{ker}f(A)$,则$f(A)A x=0$,即任意多项式的核都是对应线性变换的不变子空间。综合前文,这意味着我们可以把$A$分解为\textbf{块对角矩阵}。
在达到这个目的之前,我们还需要两个准备工作:找到零化多项式以及分解互素多项式的简便方法。

Cayley-Hamilton定理告诉我们,线性变换的特征多项式就是一个零化多项式。
\begin{theorem}{Cayley-Hamilton定理}
给定复线性空间$V$上的线性变换$A$,若$f(\lambda)=\opn{det}(A-\lambda I)$为其特征多项式,则$f(A)=0$
\end{theorem}
证明思路是利用复线性空间的任意矩阵都可相似于上三角矩阵,上三角矩阵的零化多项式即特征多项式,以及零化多项式不随基的改变而改变($f(A)=Q^{-1}f(B)Q=0$)。现在只证明第二点,其余读者可自证。

\begin{theorem}{线性空间第一分解定理}

\end{theorem}