% 含时微扰理论

\pentry{1}

一般波函数表示为束缚态的求和以及连续态的积分
\begin{equation}
\ket{\Psi(t)} = \sum_n c_n(t)\ket{n}\E^{-\I\omega_n} + \int c_{\bvec k}(t)\ket{\bvec k}\E^{-\I k^2 t/2} \dd[3]{k}
\end{equation}
其中 $\ket{n}$, $\ket{\bvec k}$ 一起构成一组正交归一基. 为了书写方便我们把上式记为
\begin{equation}
\ket{\Psi(t)} = \sumint_\alpha c_\alpha(t) \ket{\alpha(t)} = \sumint_\alpha c_\alpha(t) \ket{\alpha}\E^{-\I\omega_\alpha t} 
\end{equation}
$\alpha$ 既包括离散的部分又包括连续的部分, $\sumint$ 对离散的项求和, 对连续的项积分.

无误差的薛定谔方程(\autoref{TDPT_eq11}~\upref{TDPT}) 变为
\begin{equation}
\sumint_\alpha c_\alpha(t) H'(t) \ket{\alpha(t)} = \I \hbar \sumint_\alpha \dot c_\alpha(t) \ket{\alpha(t)}
\end{equation}
投影后(\autoref{TDPT_eq2}~\upref{TDPT}) 变为
\begin{equation}
\sumint_\alpha c_\alpha(t) \mel{\alpha'}{H'(t)}{\alpha} \E^{\I \omega_{\alpha',\alpha} t} = \I \hbar \dot c_{\alpha'} (t)
\end{equation}
各阶分离后(\autoref{TDPT_eq6}~\upref{TDPT})变为
\begin{equation}
\dot c_{\alpha'}^{(n)} = \frac{1}{\I\hbar} \sumint_\alpha \mel{\alpha'}{H'(t)}{\alpha} \E^{\I \omega_{\alpha',\alpha} t} c_{\alpha}^{(n-1)} 
\end{equation}
令 $H'(t) = W f(t)$, $\mathcal F[f(t)] = \tilde f(\omega)$, 简单的一阶微扰为
\begin{equation}
\dot c_{\alpha'}^{(1)} = \frac{1}{\I\hbar} \mel{\alpha'}{H'(t)}{\alpha} 
\end{equation}
