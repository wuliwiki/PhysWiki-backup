% 东南大学 2015 年 考研 量子力学
% license Usr
% type Note

\textbf{声明}:“该内容来源于网络公开资料,不保证真实性,如有侵权请联系管理员”

\textbf{1.}以下对称性是否导致一个守恒量,如果是,请指出相应的守恒量(1)空间反演对称性:(2)空间平移对称性;(3)空间转动对称性;(4)时间反演对称性;(5)时间平移对称性.

\textbf{2.判断题}
\begin{enumerate}
\item 一维线性谐振子的是子态空间是无穷维的
\item 全同 bose子系统的波函数具有交换反对称性
\item Hermie 算符$\hat{A}$与$\hat{B}$不对易, 则$\hat{A}$与$\hat{B}$一定无共同本征态
\item 三维各向同性谐振子的所有能级均是简并的.
\item 中心力场无自旋的角动量一定是守恒量
\end{enumerate}

\textbf{3.}证明 Bloch 函数
$$\psi_k(r) = \exp(ik \cdot r)\phi_k(r), \quad \phi_k(r) = \phi_k(r + a),~$$
是平移算符 $\hat{D}(a) = \exp\left(-ia \cdot \hat{p}/\hbar\right)$ 的本征态。相应的本征值为 $\exp(-ik \cdot a)$。

\textbf{4.}质量为 $m$ 带电量为 $q$ 的粒子在均匀电场 $\mathbf{E} = (0, \epsilon, 0)$ 和均匀磁场 $\mathbf{B} = (0, 0, B)$ 中运动,磁场的矢量势选为 $\mathbf{A} = (-By, 0, 0)$。

\begin{enumerate}
    \item 写出粒子的哈密顿算符 $\hat{H}$,并证明动量 $\hat{p}_x$ 和 $\hat{p}_z$ 为守恒量。
    \item 求守恒量完全集 $\{\hat{H}, \hat{p}_x, \hat{p}_z\}$ 的共同本征函数及相应的本征值。
\end{enumerate}

\textbf{5.}设 $\hat{\sigma}$ 为泡利算符,$\hat{\sigma}_z$ 的归一化本征态为 $\lvert + \rangle, \lvert - \rangle$。

\begin{enumerate}
    \item  利用 $\hat{\sigma}_z \hat{\sigma}_x = -\hat{\sigma}_x \hat{\sigma}_z = i\hat{\sigma}_y$ 和 $\hat{\sigma}_x^2 = 1$ 证明:$\hat{\sigma}_x \lvert + \rangle = e^{i\alpha}\lvert - \rangle$,($\alpha^* = \alpha$);
    \item  取 $\alpha = 0$,求 $\hat{\sigma}_x \lvert - \rangle$, $\hat{\sigma}_y \lvert + \rangle$。
\end{enumerate}

\textbf{6.}一维线性谐振子振动频率为 $\omega$,初态 $$\psi(x, 0) = \sum_{n=0}^{\infty} a_n \varphi_n(x)~$$,其中 $\varphi_n(x)$ 为能量本征函数。已知 $$\sum_{n=0}^{\infty} |a_n|^2 = 1~$$。
\begin{enumerate}
    \item 求 $\psi(x, t)$。
    \item 证明 $\psi(x, t) = \psi(x, 0) $。
\end{enumerate}

\textbf{7.}质量为 $m$ 的粒子受到势 $V(r)$ 的作用,其中 $V(r) = V(|r|)$ 已知 $\psi(r)$ 是本征值 $E$ 所对应的一个能量本征函数。证明:
\begin{enumerate}
    \item  $\psi(r)$ 也是 $E$ 对应的本征函数,能保证 $E$ 是简并态吗?
    \item  叙述一个能使 $E$ 对应的能量本征态非简并的条件。
\end{enumerate}

\textbf{8.}