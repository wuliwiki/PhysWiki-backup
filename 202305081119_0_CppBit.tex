% C 和 C++ 的比特运算笔记

\begin{issues}
\issueDraft
\end{issues}

\begin{itemize}
\item \verb|bits & mask| 可以把 mask 为 \verb|0| 的地方清零, 其余不变。
\item \verb`bits | mask` 可以把 mask 为 \verb|1| 的地方开启, 其余不变。
\item \verb|bits ^ mask| 可以把 mask 为 \verb|1| 的地方翻转, 其余不变(异或算符 XOR, 相反时为 1, 否则为零)。
\item \verb|~(bits1 ^ bits2)| 可以实现 NXOR 同或(没有专门的同或算符)。
\item 要把 \verb|bits| 的一些地方设为指定 pattern, 就先用 \verb|&| 把该部分清零再用 \verb`|` 设置 pattern。
\item \verb|<<| 和 \verb|>>| 的右边不可以是负数, 结果 undefined。
\item 对负数使用 \verb|>>| 是 implementation defined 的(大部分会在左边加 1), 所以要用 \verb|unsigned| 类型。
\item 如果 \verb|n << m| 或者 \verb|n >> m| 中如果 \verb|m >= sizeof(n)|, 结果是未定义的, 有一些编译器会得到 \verb|n << (m % sizeof(n))|。 如果直接用超出范围的常数, 编译器可能会警告。
\item 无论使用什么 endian, 无符号的 \verb|n << 1| 等效于乘 2, \verb|n >> 1| 等效于除 2。
\item 要得到右边 \verb|n| 个 \verb|1| 的 mask, 就用 \verb|(1 << n) - 1|。 注意没有必要把他转换为比 \verb|int| 更小的类型, 因为在参与运算时仍然会根据 integral promotion 转成 \verb|int| 或者 \verb|unsigned int|(如果 \verb|int| 不能表示整个范围)。
\item 要得到左边 \verb|n| 个 \verb|1| 的 “单字节” mask, 就用 \verb|255 << (8-n)|
\item 要得到左边第 \verb|m| bit 开始的 \verb|n| 个 \verb|1| 的 “单字节” mask, 就用 \verb|((1 << n) - 1) << (8-n-m)|
\item 操作多字节整数和浮点数的时候千万要注意如果系统是 little endian。 此时 \verb|<<| 要先把 bytes 翻过来, shift 完以后再次翻回去。
\item 一位 16 进制是 4 bit, 两位是一个字节。 \verb|0001, 0011, 0111, 1111| 分别是 \verb|0x1, 0x3, 0x7, 0xF|。 不熟练的话也可以用 \verb|0b101010|, 但看起来比较业余。
\item \verb|char| 可能是 \verb|signed| 也可能不是。 \verb|int16_t|, \verb|uint16_t| 保证有两个字节(\verb|short| 不保证), \verb|int32_t|, \verb|uint32_t| 保证有四个字节(\verb|int| 不保证)。
\end{itemize}
