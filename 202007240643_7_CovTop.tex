% 覆叠空间

\pentry{连续映射和同胚\upref{Topo1}}

覆叠空间是一种常见的简化空间描述的方式,可以更轻松地描述许多复杂空间的性质.覆叠的思想在微分几何中极为重要,而在理论物理中也偶尔会使用该方法.

\begin{definition}{覆叠映射和覆叠空间}
设有拓扑空间之间的\textbf{满}连续映射$p:C\rightarrow X$.如果对于任意的$x\in X$,存在开集$U_x\in\mathcal{T}_X$且$x\in U_x$,使得$p^{-1}(U_x)$是$C$中若干不相交开子集的并,并且每个这样的开子集都通过$p$和$U_x$同胚,那么称$C$是$X$的\textbf{覆叠空间(covering space)},$p$是其\textbf{覆叠映射(covering map)}或\textbf{覆叠投影(covering projection)}.
\end{definition}

