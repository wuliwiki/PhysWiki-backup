% ZFS (Zettabyte File System)笔记

\begin{issues}
\issueDraft
\end{issues}

\begin{itemize}
\item Ubuntu/Debian 安装 zfs: \verb|sudo apt install zfsutils-linux|
\item 创建一个 500MB 文件: \verb|dd if=/dev/zero of=虚拟硬盘文件 bs=1M count=500|
\item 制作 loopback 设备 \verb|sudo losetup -fP --show 虚拟硬盘文件|。会返回设备路径如 \verb|/dev/loop0|
\item \verb|sudo zpool create pool名 /dev/loop0|
\item 检查状态: \verb|sudo zpool status|; \verb|sudo zfs list|
\item 创建文件系统: \verb|sudo zfs create pool名/fs名|
\item 创建测试文件: \verb|echo "This is a test file." > /pool名/fs名/文件|
\item 创建文件系统镜像: \verb|sudo zfs snapshot pool名/fs名@镜像1|
\item 改变测试文件: \verb|echo "This is some additional data." >> /pool名/fs名/文件|
\item 对比当前文件系统和上一次镜像 \verb|sudo zfs diff pool名/fs名@镜像1 pool名/fs名|。 输出如 \verb|M	/pool名/fs名/文件|。 其中 \verb|M| 和 git 中一样表示 modified。
\item 恢复镜像: \verb|sudo zfs rollback pool名/fs名@镜像1|。现在可以检查 \verb|文件| 恢复到了之前的版本。
\item 添加另一个 loop 设备: \verb|dd if=/dev/zero of=虚拟硬盘文件2 bs=1M count=500|
\item \verb|sudo losetup -fP --show 虚拟硬盘文件2|, 输出例如 \verb|loop1|
\item 添加设备到已有的 pool 中: \verb|sudo zpool add pool名 /dev/loop1|
\item 所以 zfs 不需要再给硬盘分区,而是把硬盘加入 pool, 再在 pool 中创建文件系统。 每个文件系统有自己的属性,有自己的镜像,共享整个 pool 的容量。 当然,作为高级功能, zfs 也允许不把整块硬盘加入 pool 而是把一个分区加入 pool。
\item 添加设备后, pool 中的所有文件系统会自动按需使用新增到 pool 中的设备,无需手动 resize。
\item 用完以后,删除 pool: \verb|sudo zpool destroy pool名|, \verb|sudo losetup -d /dev/loop0|
\end{itemize}

\subsection{常识}
\begin{itemize}
\item \verb|zfs| 的程序是非常底层的,系统甚至不知道这些硬盘已经被挂载了。 \verb|disk| 会显示未挂载。 System Monitor 也不会显示磁盘的读写速度。
\item \verb|vdev| 是 virtual device, 一个 pool 中的数据在所有 vdev 上面 strip。
\item 每个 \verb|vdev| 可以有不同的 RAID (\verb|z1| 代表一个冗余相当于普通的 RAID5, \verb|z2| 代表两个,等),也可以是单个盘。 但是一个 pool 中只要有一个 vdev 数据丢失,那么就会导致数据彻底无法恢复。
\item \verb|vdev| 可以有不同的大小,可以不断往 pool 中加入。 zfs 会根据每个 vdev 剩余的容量按比例 strip。
\item \verb|vdev| 一旦加入 pool 就不能删除。
\item 只要把一个 pool 的所有盘用任何方法让 linux 识别到, 那么用 \verb|zpool import [pool名]| 就会识别出这个 pool。
\item zfs 自带 checksum, 也就是说它在读取数据的时候(或者让它检查完整性的时候)会自动检测 checksum 是否一致, 若不一致, 若有数据冗余, 则会自动修复数据。
\item \verb|zpool scrub pool名| 可以马上检查整个 pool 的 checksum 是否一致。
\item \verb|zpool list -v| 查看 pool, fs 和各个硬盘的占用空间状态和其他详细数据。
\end{itemize}

\subsection{RAID}
\begin{itemize}
\item 设置 RAID-Z1: \verb|sudo zpool create pool名 raidz1 /dev/sdb /dev/sdc ...|。 这会自动挂载到 \verb|/pool名| 目录。
\item 创建文件系统: \verb|sudo zfs create pool名/fs名|。
\item \verb|sudo zpool status yue| 可以查看各种状态以及 RAID 设置, 以及读写错误。
\item 4 块 8TB 硬盘用 \verb|raidz1| 组合后 \verb|zpool list -v| 显示 \verb|29.1TB| 容量。
\item 改变挂载目录: \verb|sudo zfs set mountpoint=/挂载目录 pool名/fs名|。
\item \verb|sudo zfs unmount pool名| 弹出 pool, 用 \verb|zfs mount| 重新挂载。
\item \verb|sudo zfs set compression=lz4 pool名/fs名| 开启 lz4 压缩(cpu 会间断性占用 70\% 左右, 8 核 16 线程)。 可以随时打开或者关闭,只对新数据生效。 \verb|lz4| 是 zfs 最流行的压缩算法。 压缩会占用 cpu 但反而有可能会让读写变快! 因为真正需要读写的数据少了。 实测在我的 8cpu 电脑上开启最高压缩率算法 \verb|gzip-9| 并不会让写入变慢。 其实如果大文件主要是视频啥的,压缩率其实非常小。
\item 【严重不推荐】\verb|sudo zfs set dedup=on pool名/fs名| 开启查重(重复文件可以节约空间,但会使写入速度严重变慢——实测慢一半)。 可以随时打开或者关闭,只对新数据生效。
\item 把 \verb|set| 改成 \verb|inherit| 就相当于关闭这些选项。
\item \verb|sudo zfs get all pool名/fs名| 可以列出所有当前选项。 把 \verb|all| 换成 \verb|选项名| 可以列出制定选项。
\item \verb|compressratio| 属性可以显示当前的平均压缩率, 这只用于查看不能手动设置。
\item 的确,以上设置(无论是否开启 \verb|dedup|)都需要 \verb|1GB/1TB| 的内存。
\item 亲测 4 盘 8TB 希捷 exos, RAID-z1,lz4 压缩, 关闭 dedup, 大 mp4 文件写入平均 50-70MB/s,比较稳定。 一开始复制文件可以达到 100MB/s 以上,但这应该主要是在写入缓存。
\item GPT 说 RAID-z1 的写入速度其实比使用单个硬盘更慢。 但是读取速度会更快。
\item 由于磁盘碎片, 数据占总容量 80\% 的时候就会导致性能下降。
\item \verb|zpool replace pool名 /dev/sda /dev/sdb| 可以把 pool 中的某块磁盘(可以离线)替换成一块新的。 新磁盘将没有任何碎片。 这种方法理论上可以用来去碎片化(但有可能花很长时间)。
\end{itemize}

\subsection{镜像和克隆}
\begin{itemize}
\item 由于 zfs 使用 copy-on-write, 制作镜像非常轻量。
\item 可以把镜像挂在到一个指定的文件夹中(只读)。
\item \verb|zfs list -t snapshot| 查看所有镜像。
\item 删除镜像会自动释放不需要的空间(哪怕是删除中间的镜像)。
\item 可以直接把镜像挂载到某个目录: \verb|zfs mount -o ro -t snapshot pool名/fs名@镜像名 /镜像挂载目录|
\item 挂载后的镜像是只读的, 如果要写入, 可以先对镜像做一个克隆: \verb|zfs clone pool名/fs名@镜像名 pool名/fs克隆名|。 挂载克隆也是一样的语法: \verb|zfs mount pool名/fs克隆名 /镜像克隆挂载目录|
\end{itemize}

\subsection{测试 dedup}
\begin{itemize}
\item 2TB 的备份文件拷进去, 开启 \verb|dedup|, 然后再把其中的 1.5TB 拷进另一个文件夹。 开始时的 \verb|zfs list -v|(省略了一些空白列):
\begin{lstlisting}[language=none]
(now is 9:55)
NAME        SIZE  ALLOC   FREE  FRAG    CAP  DEDUP
yue        29.1T  4.81T  24.3T  0%    16%  1.00x
  raidz1   29.1T  4.81T  24.3T  0%  16.5%      -
\end{lstlisting}
\item 32GB 内存爆满, cpu 16 线程占用大部分
\item 四小时测试
\end{itemize}
