% Noether 定理
% keys 对称性|守恒量
% license Usr
% type Tutor

\pentry{最小作用量、哈密顿原理\nref{nod_HamPrn}}{nod_a5cc}
\cite{AZee}Nother定理是关于对称性和守恒量的,它于1915年由Noether证明,其表明每一个使拉氏量不变的变换都对应一个守恒量。正如理论物理中最深刻的定理一样,Noether定理的证明极其的简单。

\subsection{Noether定理}
\begin{theorem}{Noether定理}
设 $L(q(t),\dot q(t))$ ($\dot{}:=\dv{}{t}$) 是拉氏量,其中 $q=(q_1,\ldots,q_n),\dot q=(\dot q_1,\ldots,\dot q_n)$。 则对每一个使得拉氏量 $L(q(t),\dot q(t))$ 不变的变换,都有一个守恒量存在。即若无穷小变换
\begin{equation}
q(t)\rightarrow q(t)+\delta q(t),~
\end{equation}
使得 $L$ 不变,这是指 $\delta L=0$,那么存在守恒量 $Q:=\frac{\delta L}{\delta\dot q}\delta q$,即 $\dot Q=0$。
\end{theorem}
\textbf{证明:}
在无穷小变换 $q\rightarrow q+\delta q$ 下,有
\begin{equation}
\begin{aligned}
\dv{q}{t}\rightarrow\dv{}{t}(q+\delta q)&=\dv{q}{t}+\dv{\delta q}{t}\\
&\Downarrow\\
\delta\qty(\dv{q}{t})&=\dv{\delta q}{t}
\end{aligned}.~
\end{equation}
即 $\delta \dot q=\dot{\delta q}$
因此,
\begin{equation}
\begin{aligned}
0=&\delta L=\frac{\delta L}{\delta \dot q}\dot{\delta q}+\frac{\delta L}{\delta q}\delta q\\
&\overset{\text{Euler-Lagrange方程}}{=}\frac{\delta L}{\delta \dot q}\dot{\delta q}+\dv{}{t}\qty(\frac{\delta L}{\delta\dot q})\delta q\\
=&\dv{}{t}\qty(\frac{\delta L}{\delta\dot q}\delta q)\\
:=&\dot Q
\end{aligned}.~
\end{equation}
证毕!

\textbf{注意:}在无穷小变换下拉氏量不变被理解成一阶 $\delta L=0$ 即可,是因为一阶变分 $\delta L$,例如 $\delta L^2$ 在无穷小时的和即变成积分,刚好就是和的极限值,此时积分值是精确的。这相当于高阶无穷小不起作用。





