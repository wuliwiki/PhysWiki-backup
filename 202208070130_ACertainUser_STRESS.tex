% 应力

对于一块材料,我们很容易用截面法\upref{INTFRC}分析出材料某一截面上的受力情况;但是,截面法只能告诉我们整个截面上的“总受力”,却不能告诉我们截面上“某一点”处的受力.

\subsection{微元体}
为了更好的处理材料内一点处的受力,类似于积分划分小块体积的思想,我们假定材料是由无数小块组成的,每一小块被称作“微元体 Element”.

\begin{figure}[ht]
\centering
\includegraphics[width=10cm]{./figures/STRESS_1.png}
\caption{微元体}} \label{STRESS_fig1}
\end{figure}

微元体的每一个面上受3个力(的分量),包括一个正应力$\sigma$与两个切应力$\tau$.看起来,一个微元体上共有$3\times6=18$个力的分量.但对每一个小块运用刚体的静力平衡\upref{RBSt},可以证明事实上\textbf{一个微元体上只有6个相互独立的力},包括三个正应力与三个切应力.

因此,一个微元体的受力情况可以记为一个矩阵:
\begin{equation}
\mat \sigma=
\begin{bmatrix}
\sigma_{xx} & \sigma_{xy} & \sigma_{xz} \\
\sigma_{yx} & \sigma_{yy} & \sigma_{yz} \\
\sigma_{zx} & \sigma_{zy} & \sigma_{zz} \\
\end{bmatrix}
\end{equation}
