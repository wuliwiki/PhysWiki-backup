% 2012 年计算机学科专业基础综合全国联考卷
% keys 计算机 考研 全国卷

\subsection{一、单项选择题} \\
第1~40小题,每小题2分,共80分.下列每题给出的四个选项中,只有一个选项最符合试题要求.

1.求整数n(n≥0)阶乘的算法如下,其时间复杂度是 . \\
\begin{lstlisting}[language=cpp]
int fact(int n){
  if (n<=1) return 1;
  return n*fact(n-1);
}
\end{lstlisting}
A. $O(log2n)$ $\quad$ B. $O(n)$ $\quad$ C. $O(nlog2n)$ $\quad$ D. $O(n2)$

2.已知操作符包括„+‟、„-‟、„*‟、„/‟、„(‟和„)‟.将中缀表达式a+b-a*((c+d)/e-f)+g转换为等价的后缀表达式ab+acd+e/f-*-g+时,用栈来存放暂时还不能确定运算次序的操作符,若栈初始时为空,则转换过程中同时保存在栈中的操作符的最大个数是 . \\
A.5 $\quad$ B.7 $\quad$ C.8 $\quad$ D.11



