% 电流产生磁场
% keys 电流|磁场|线圈|电磁学
% license Xiao
% type Tutor

\addTODO{这属于科普部分,主要定性说明, 电流周围会产生闭合磁场, 随距离减弱, 线圈内部会有近似直线的磁场, 另外提一下两个线圈的那个装置叫做什么?画两个图}

在很长的一段时间内,科学家们认为电学和磁学是两个不同的分支,电学是研究正电荷负电荷的学科,导体内定向移动的电荷导致电流,库仑发现了电荷之间的相互作用的定律——库仑定律\upref{ClbFrc},法拉第提出电场概念表示电荷激发的一种力场。而磁学是研究磁体之间的相互作用,磁铁有正负两极,产生磁场,磁场会对位于其中的铁磁性物质或者磁铁有一定相互作用。那时的人们还无法想象,电和磁能在一个统一的理论框架下共同描述。

1820 年 4 月的一天,丹麦科学家奥斯特发现通电导线附近,指南针的指向会受到影响。经过反复的实验奥斯特最终确定了电流周围会产生磁场,这对那时的学界带来了巨大的震撼,电学和磁学从此成为了两个联系紧密的学科,人们开始通过更精细的定性定量实验研究它们之间的联系。
\subsection{奥斯特实验与电流的磁效应}
奥斯特将一个指针放在通电直导线的附近,发现指针的 $S$-$N$ 会受到力的作用,最终朝向与电流方向垂直的切向方向,而当导线不通电的时候,指针放在导线附近不会受到力的作用。因此奥斯特推断,通电直导线附近会激发一个环绕导线的磁场。如果将磁场线画出来,磁场线会形成一个环绕导线的一个闭合回路,离电流越远的地方磁场越弱,离电流越近的地方磁场越强。

根据麦克斯韦方程\upref{MWEq} $\curl \bvec B=\mu_0 \bvec J$ 以及 stokes 公式,可以推断出通电直导线附近的磁场大小与距离的关系式是:
\begin{equation}
2\pi R B = \mu_0 I,\Rightarrow B=\frac{\mu_0 I }{2\pi R}~.
\end{equation}
其中 $R$ 为
\subsection{通电螺线管}
