% 牛顿第二定律的矢量形式

我们高中最熟悉的是一维直线运动的牛顿第二定律, 也就是常见的标量形式 $F = ma$. 高中物理已经明确过力和加速度都是矢量, 本书中矢量用黑体和正体表示, 即 $\bvec F$ 和 $\bvec a$. 我们知道, 矢量既有长度也有方向, 但是在一维情况下, 矢量只有两个方向, 我们把其中一个定义为正方向, 那么另一个就是反方向. 

物理中有一个简单的约定, 就是一维的矢量可以用标量(也就是一个实数)表示, 对于正方向的矢量, 就使用正数, 反方向的矢量就使用负数. 所以在一维情况下, 矢量于标量可以一一对应.

在更高维空间中, 例如二维平面和三维空间中, 牛顿第二定律适用于任何曲线运动, 质点速度的方向和大小都可以随时间任意变化. 它的矢量形式是 $\bvec F = m\bvec a$, 比标量形式所包含的意义要丰富得多.

一个典型的例子就是在匀速圆周运动中, 质点存在向心加速度, 然而和圆心之间的距离却没有改变, 但却存在加速度, 而根据牛顿第二定律, 做该运动的质点必须要受一个向心力才能保持匀速圆周运动.

那么问题来了, 为什么圆周运动会有向心加速度呢? 这就取决于矢量加速度 $\bvec a$ 的定义了. 相信在高中一些课堂中老师会用所谓的 “微元法” 推导一次, 但大部分学生过一段时间以后仅仅只是记得向心加速度的公式 $a = v^2/r = \omega r^2$, 却已经忘记这个加速度从何而来.

圆周运动的向心加速度关键就在于速度和加速度都必须看成矢量. 我们先回顾直线运动中 “标量加速度” 的定义为
\begin{equation}
a
\end{equation}


% \begin{equation}
% \bvec p(t) = m\bvec v(t) = m\int_{t=t_0, \bvec v(t_0) = 0}^t \bvec a(t') \dd{t'} = \int_{t=t_0, \bvec v(t_0) = 0}^t \bvec F(t') \dd{t'}
% \end{equation}
