% 乔治·布尔(综述)
% license CCBYSA3
% type Wiki

本文根据 CC-BY-SA 协议转载翻译自维基百科\href{https://en.wikipedia.org/wiki/George_Boole#}{相关文章}。

\begin{figure}[ht]
\centering
\includegraphics[width=6cm]{./figures/228c41af0dd53eca.png}
\caption{布尔的肖像,摘自《插图伦敦新闻》,1865年1月21日} \label{fig_George_1}
\end{figure}
乔治·布尔(George Boole,发音:/buːl/ 布尔,1815年11月2日-1864年12月8日)是一位主要自学成才的英国数学家、哲学家和逻辑学家,他的大部分短暂职业生涯都在爱尔兰科克的女王学院担任数学教授。他在微分方程和代数逻辑领域有所研究,最著名的作品是《思维的法则》(The Laws of Thought,1854),该书中包含了布尔代数。布尔逻辑对计算机编程至关重要,并被认为为信息时代奠定了基础。

布尔是一个鞋匠的儿子。他接受了初等教育,并通过各种方式学习了拉丁语和现代语言。16岁时,他开始教授工作以养家糊口。19岁时,他创办了自己的学校,后来在林肯经营了一所寄宿学校。布尔积极参与当地社团活动,并与其他数学家合作。1849年,他被任命为爱尔兰科克女王学院(现为科克大学)首任数学教授,在那里他遇见了未来的妻子玛丽·埃弗雷斯特。他继续参与社会事业,并保持与林肯的联系。1864年,布尔因患肺炎引发的胸膜积液而去世。

布尔一生发表了约50篇文章和几本单独的著作。他的一些关键作品包括关于早期不变性理论的论文和《逻辑的数学分析》(The Mathematical Analysis of Logic),该书引入了符号逻辑。布尔还写了两部系统性的专著:《微分方程论》和《有限差分法则论》。他对线性微分方程理论和有理函数的留数和研究作出了贡献。1847年,布尔发展了布尔代数,这一二进制逻辑的基本概念为逻辑代数传统奠定了基础,并构成了数字电路设计和现代计算机科学的基石。布尔还试图在概率论中发现一种通用方法,重点研究如何通过逻辑连接给定概率的事件来确定其后果概率。

布尔的工作得到了许多学者的拓展,如查尔斯·桑德斯·皮尔士和威廉·斯坦利·杰文斯等。布尔的思想后来得到了实际应用,当克劳德·香农和维克托·谢斯塔科夫利用布尔代数优化机电继电器系统的设计时,推动了现代电子数字计算机的发展。他对数学的贡献为他赢得了各种荣誉,包括皇家学会的首个数学金奖、基思奖章以及都柏林大学和牛津大学的名誉学位。科克大学在2015年庆祝布尔诞辰200周年,强调了他对数字时代的重大影响。
\subsection{早年生活}
\begin{figure}[ht]
\centering
\includegraphics[width=6cm]{./figures/ae46c5bae0377e72.png}
\caption{3号波特门街的房子和学校} \label{fig_George_2}
\end{figure}
\begin{figure}[ht]
\centering
\includegraphics[width=6cm]{./figures/f2dd0a96ad09ef13.png}
\caption{来自房子的纪念 plaque(牌匾)} \label{fig_George_3}
\end{figure}
布尔于1815年出生在英格兰林肯市,父亲是鞋匠约翰·布尔(John Boole Snr,1779-1848),[7]母亲是玛丽·安·乔伊斯(Mary Ann Joyce)。[8]他接受了初等教育,并从父亲那里获得了教学,但由于家庭生意的严重衰退,他没有接受更多的正式和学术教育。[9]林肯的书商威廉·布鲁克(William Brooke)可能帮助他学习拉丁语,他也可能在托马斯·贝恩布里奇(Thomas Bainbridge)学校学习过拉丁语。[2]他在现代语言方面是自学成才的。[10]事实上,当当地报纸刊登了他翻译的拉丁诗歌时,一位学者指控他抄袭,声称他不可能取得这样的成就。16岁时,布尔成为家庭的经济支柱,负责养活父母和三个年幼的弟妹,并在唐卡斯特的海厄姆学校担任初级教师。[11]他还曾在利物浦短暂任教。[1]
\begin{figure}[ht]
\centering
\includegraphics[width=6cm]{./figures/c17cd6d82ded5147.png}
\caption{林肯的灰修道院,曾是机械学会的所在地} \label{fig_George_4}
\end{figure}
布尔参与了位于林肯灰修道院的林肯机械学会,该学会成立于1833年。[2][12]通过该机构,爱德华·布罗姆赫德(Edward Bromhead)认识了约翰·布尔,并帮助乔治·布尔获得了数学书籍。[13]此外,林肯圣斯威辛教堂的乔治·史蒂文斯·迪克森(Rev. George Stevens Dickson)向他赠送了西尔维斯特·弗朗索瓦·拉克鲁瓦(Sylvestre François Lacroix)的微积分教材。[14]由于没有教师指导,他花了很多年才掌握微积分。[1]

19岁时,布尔在林肯成功地创办了自己的学校:自由学校巷(Free School Lane)。[15]四年后,在罗伯特·霍尔去世后,他接管了位于林肯外的瓦丁顿的霍尔学院(Hall's Academy)。[1]1840年,他搬回林肯并经营了一所寄宿学校。布尔立即参与了林肯地理学会,担任委员会成员,并提交了一篇题为《论多神教的起源、发展和趋势,尤其是在古埃及、波斯和现代印度的表现》的论文。[16]

布尔成为了当地的知名人物,并且是约翰·凯(John Kaye)主教的崇拜者。[17]他参与了当地的早期关门运动。[2]1847年,他与埃德蒙·拉肯(Edmund Larken)等人共同成立了一个建筑协会。[18]他还与查尔斯主义者托马斯·库珀(Thomas Cooper)交往,库珀的妻子与他有亲戚关系。[19]

从1838年起,布尔开始与英国学术数学家建立联系,并广泛阅读。他研究了代数,尤其是符号方法的形式,并开始发表研究论文。[1]
\subsection{教授职务与在科克的生活}
\begin{figure}[ht]
\centering
\includegraphics[width=6cm]{./figures/566ee9f7903c9c2e.png}
\caption{} \label{fig_George_5}
\end{figure}
布尔作为数学家的地位得到了认可,1849年,他被任命为爱尔兰科克女王学院(现为科克大学UCC)的首任数学教授。他于1850年在科克遇见了未来的妻子玛丽·埃弗雷斯特,当时她正在探望她的叔叔约翰·里尔(John Ryall),他是希腊语教授。两人于1855年结婚。[20][21]他始终保持与林肯的联系,并与埃德蒙·R·拉肯一起参与了一项旨在减少卖淫的运动。[22]

1861年,布尔参与了爱尔兰女王法庭的一项判决案件,案件的被告是斯莱戈克雷格豪斯的约翰·休伊特·惠特利(John Hewitt Wheatley),他欠款400英镑。根据判决,惠特利的财产和在科克郡马亨(Maghan/Mahon)土地上的权益转交给了布尔。[23]

1863年3月,布尔租下了科克的利奇菲尔德小屋(Litchfield Cottage),他将在那里与妻子玛丽一起生活,直到第二年12月去世。根据契约,该房产被描述为“所有该住所,名为利奇菲尔德小屋的住宅及其附属建筑,后院的花园和围墙田地”。布尔的遗嘱将他在利奇菲尔德小屋租赁中的所有“财产、权益和租期”留给了他的妻子。1865年8月,布尔去世约8个月后,玛丽(此时住在伦敦哈雷街68号)将该房子转让给了科克巴林坦普尔的弗朗西斯·赫德(Francis Heard)先生,他是她陛下第87南科克团的上尉。
\subsection{荣誉与奖项}
\begin{figure}[ht]
\centering
\includegraphics[width=6cm]{./figures/b664232823a3010d.png}
\caption{细节描绘了他最喜欢的圣经章节(内容由他的遗孀建议),即上帝呼召先知撒母耳(撒母耳记上 3:1–10),一个由父母献给上帝的孩子。} \label{fig_George_6}
\end{figure}
\begin{figure}[ht]
\centering
\includegraphics[width=6cm]{./figures/ffca4d379b4e18c2.png}
\caption{} \label{fig_George_7}
\end{figure}
1844年,布尔的论文《分析中的一般方法》获得了皇家学会颁发的首个数学金奖。1855年,他获得了爱丁堡皇家学会的基思奖章,并于1857年当选为皇家学会院士(FRS)。他还获得了都柏林大学和牛津大学的法学荣誉学位(LL.D.)。