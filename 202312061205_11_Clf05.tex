% 单向量的运算
% license Xiao
% type Tutor

在上一节,我们介绍了单向量的基本运算。本节着重讨论外积、对偶以及左内积运算在单向量集合上的性质,即这些运算在单向量集合上是否封闭,以及运算结果与子空间的关系。需要注意的是,左内积以及对偶运算往往需要我们取一组正交基,因为用集合语言讨论非常方便,所以对于退化二次型,我们需要仔细检验某些结论是否成立。
\subsubsection{互反基}
由于几何代数定义在线性空间上,所以我们也可以定义“对偶基”的概念。稍后我们可以发现,对单向量取对偶后其子空间和“对偶基”的关系。

\begin{definition}{}
给定\textbf{非退化}的几何代数$\mathcal G(V,q)$,$\{e_1,e_2...e_k\}$是$V$上的一组基,则可以定义该基的\textbf{互反基(reciprocal basis)}$\{e^1,e^2...e^k\}$,使得
\begin{equation}
e^i*e_j=\delta^i_j~.
\end{equation}
\end{definition}
reciprocal也可翻译为“互逆”、“对偶”和“倒易”等。

由于几何代数的系数取自域,所以非退化的几何代数里总存在标准正交基。在取了标准正交基后,其互反基实际上就是这组基本身:$e^i=e_i$。
duiy