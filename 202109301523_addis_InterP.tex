% 多项式插值

\pentry{范德蒙矩阵\ 范德蒙行列式\upref{VandDe}}

若有 $n$ 个数据点 $x_1, \dots, x_n$, $y_1, \dots, y_n$($x_i$ 互不相等), 使用多项式
\begin{equation}
p_{m-1}(x) = c_0 + c_1 x + \dots, c_{m-1}x^{m-1}
\end{equation}
插值. 需要用 $m-1$ 阶多项式插值, 可以列出线性方程组
\begin{equation}\label{InterP_eq1}
\pmat{1 & x_1 & x_1^2 & \dots & x_1^{m-1}\\
1 & x_2 & x_2^2 & \dots & x_2^{m-1}\\
1 & x_3 & x_3^2 & \dots & x_3^{m-1}\\
\vdots & \vdots & \vdots & \ddots & \vdots\\
1 & x_n & x_n^2 & \dots & x_n^{m-1}}
\pmat{c_0\\ c_1\\ c_2\\ \vdots \\ c_{m-1}}
=
\pmat{y_0\\ y_1\\ y_2\\ \vdots \\ y_{m-1}}
\end{equation}
现在来讨论什么时候该方程有唯一解. 事实上系数矩阵就是范德蒙矩阵\upref{VandDe}, 当 $x_i$ 互不相等时, 它的秩等于 $\min\qty{m, n}$. 所以当 $m > n$ 时, 存在无穷个解; 当 $n > m$ 时一般无解(见超定方程组\upref{OvrDet}); 当 $n = m$ 时能确保有且仅有唯一解, 因为此时系数矩阵是满秩矩阵.

\subsection{龙格现象}
若点的个数较多, 做多项式插值时会出现\textbf{龙格现象(Runge's phenomenon)}, 即多项式在两个端点处剧烈振动. 一种解决方法是使用多个低阶多项式插值(未完成: splint interp). 另一种方法是不用插值而改为拟合, 当 $m-1 < 2\sqrt{n}$ 时
