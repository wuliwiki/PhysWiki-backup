% 力场 势能
% 势能|力场|矢量场|做功|引力|重力

\pentry{矢量场\upref{Vfield}, 功\upref{Fwork}, 牛顿—莱布尼兹公式\upref{NLeib}}

\subsection{力场}
高中物理中我们已经学过一些场的概念,即质点受场的力取决于质点在场中的位置. 例如地球表面局部的引力场可以近似看做一个恒力场( 称为为\textbf{重力场}), 即在一定区域内, 质点总受向下的,大小恒为 $mg$ 的重力(矢量式 $\bvec F = m\bvec g$). 又例如水平面上一根原长忽略不计的弹簧,一端固定在原点,另一端连接质点,那么质点受力总指向原点,大小等于劲度系数和位矢模长的之积 $kr$. 用矢量的方法表示,就是 $\bvec F = -k\bvec r$.

总结到一般情况,\textbf{力场} 可以用场对质点施加的力(矢量)关于质点位置(即位矢\upref{Disp})的矢量函数表示, 所以力场是一种矢量场.


\begin{example}{引力场}\label{V_ex1}
球坐标原点处质量为 $M$ 的质点在周围造成的引力场为
\begin{equation}
\bvec F(\bvec r) =  - G\frac{M}{r^2}\,\uvec r
\end{equation} 
若位矢用 $\bvec r$ 来表示( $\bvec r = r\,\uvec r$), 则
\begin{equation}
\bvec F(\bvec r) =  - G\frac{M}{r^3}\,\bvec r
\end{equation}
现在变换到直角坐标系中,有
\begin{equation}
\leftgroup{
\bvec r = x\,\uvec x + y\,\uvec y + z\,\uvec z\\
r = \sqrt{x^2 + y^2 + z^2} 
}\end{equation} 
代入上式,展开得
\begin{equation}
\bvec F(\bvec r) =  - \frac{GMx}{(x^2+y^2+z^2)^{3/2}}\uvec x - \frac{GMy}{(x^2 +y^2+z^2)^{3/2}}\uvec y - \frac{GMz}{(x^2+y^2+z^2)^{3/2}}\uvec z 
\end{equation} 

显然球坐标系中的引力场表达式比直角坐标系中的要简洁得多.由此可见,对不同的矢量场选择适当的坐标系往往可以简化问题.
\end{example}

若质点从场的一点移动到另一点的过程中, 力场对质点做的功\upref{Fwork} 只与初末位置有关,而与质点移动的路径无关, 那么这个力场就是一个\textbf{保守场}. 这时我们可以给该质点定义一个\textbf{势能函数}, 势能函数是一个关于位矢的标量函数,一般记为 $V(\bvec r)$, 具有能量量纲. 当质点从一点以任意路径移动到另一点时,场对质点做的功等于质点初位置的势能减末位置的势能, 即
\begin{equation}\label{V_eq1}
\int_{\bvec r_1}^{\bvec r_2} \bvec F(\bvec r) \vdot \dd{\bvec r} = V(\bvec r_1)-V(\bvec r_2)
\end{equation}

\subsection{一维势能函数}

现在先假设质点只能沿一条直线运动, 且力也始终与直线平行. 显然质点从一点到另一点的路径只可能有一条, 所以任何一维力场都是保守场. 若给直线定义一个正方向, 单位矢量为 $\uvec x$, 任何一维力场可以记为
\begin{equation}\label{V_eq2}
\bvec F(x) = F(x) \uvec x
\end{equation}
质点的位置矢量可记为 $\bvec r = x \uvec x$. 由于 $\uvec x \vdot \uvec x = 1$, 质点从 $x = a$ 移动到 $x=b$ 过程中场做的功为
\begin{equation}
\int_{\bvec r_1}^{\bvec r_2} \bvec F(\bvec r) \vdot \dd{\bvec r}
= \int_a^b F(x) \dd{x}
\end{equation}
根据势能的定义, 对任意的 $a$ 和 $b$, 上式应该等于 $V(a) - V(b)$. 根据牛顿—莱布尼兹公式\upref{NLeib}, 势能函数恰好就是 $F(x)$ 的负原函数, 所以 $F(x)$ 是 $V(x)$ 负导函数.
\begin{equation}\label{V_eq8}
V(x) = -\int F(x) \dd{x}
\qquad
F(x) = -\dv{V(x)}{x}
\end{equation}

需要注意的是, 由于原函数有无穷多个(由不定积分中任意常数的取值决定), 所以势能函数也存在无穷多个, 且都相差一个常数. 为了确定势能函数, 我们需要指定场中某一点的势能值, 如果令某点势能为零, 那么这点就叫做\textbf{零势点}.

\begin{example}{弹簧的势能}
一个原长可忽略的轻弹簧劲度系数为 $k$, 一端固定在原点, 另一端连接质点. 质点只能沿 $\uvec x$ 方向运动, 规定质点在原点时势能为 $0$, 求弹簧的势能关于质点位置坐标 $x$ 的函数. 

由题意,\autoref{V_eq2} 中 $F(x) = -kx$, 不定积分并取负值得到含有待定常数的势能函数
\begin{equation}
V(x) = -\int (-kx) \dd{x} = \frac12 k x^2 - C
\end{equation}
为了确定待定常数, 代入 $V(0) = 0$, 解得 $C = 0$. 所以所求势能为
\begin{equation}
V(x) =  \frac12 k x^2
\end{equation}
\end{example}


\subsection{多维势能函数}
\pentry{梯度定理\upref{Grad}}

假设力场 $\bvec F(\bvec r)$ 是平面或三维空间中的保守场, 对应势能为 $V(\bvec r)$, 初始点为 $\bvec r_i$, 终点为 $\bvec r_f$. 对 $-V(\bvec r)$ 使用梯度定理\upref{Grad} 得
\begin{equation}
\int_{\bvec r_i}^{\bvec r_f} \grad [-V(\bvec r)] \vdot \dd{\bvec l} = V(\bvec r_i) - V(\bvec r_f)
\end{equation}
我们把该式与\autoref{V_eq1} 比较, 不难发现力场是势能函数的负梯度
\begin{equation}
\bvec F(\bvec r) = - \grad V(\bvec r)
\end{equation}
即在保守场的某点中, 力的方向是势能下降最快的方向, 大小是该方向的负方向导数. 由梯度的定义, 力场的各个分量分别为对应方向的负偏导数
\begin{equation}\label{V_eq9}
F_x(\bvec r) = -\pdv{V(\bvec r)}{x} \quad F_y(\bvec r) = -\pdv{V(\bvec r)}{y} \quad \ldots
\end{equation}

\begin{example}{二维简谐振子}
若已知二维的势能函数为 $V(x,y) = k_1 (x+y)^2/2 + k_2 (x-y)^2/2$, 求力场. 若已知场函数求势能函数, 又该如何求?

把势能函数代入\autoref{V_eq9} 中, 求偏导, 得场为
\begin{equation}\ali{
\bvec F(\bvec r) &= -\pdv{V}{x} \uvec x - \pdv{V}{y} \uvec y\\
&= -[(k_1 + k_2) x  + (k_1-k_2) y] \uvec x - [(k_1-k_2)x + (k_1+k_2)y] \uvec y
}\end{equation}

现在我们根据“梯度定理\upref{Grad}”中的\autoref{Grad_eq23} 从场逆推势能. 首先对力场的 $x$ 分量和 $y$ 分量分别关于 $x$ 和 $y$ 做不定积分得到任意两个原函数并记为 $G_x$ 和 $G_y$ 得
\begin{equation}
G_x(x,y) = - \frac12 (k_1 + k_2) x^2 -  (k_1-k_2) xy
\end{equation}
\begin{equation}
G_y(x,y) =  - (k_1 - k_2) xy - \frac12 (k_1+k_2) y^2
\end{equation}
代入得(注意这里的场是势能函数的的负梯度而不是梯度,另外注意下式中的常数项都并入 $C$ 中)
\begin{equation}\ali{
V(x,y) &= -G_y(x,y) + G_y(x,y_0) - G_x(x,y_0) + C\\
&= \frac12 (k_1 + k_2) x^2 + (k_1-k_2)xy + \frac12 (k_1+k_2) y^2 + C\\
&= \frac12 k_1 (x+y)^2 +\frac12 k_2 (x-y)^2 + C
}\end{equation}
若规定零势点 $V(0,0) = 0$, 代入上式得 $C=0$.
\end{example}

\subsection{两质点间的势能}

如果两质点 $A$ 和 $B$ 的位矢分别为 $\bvec r_A$ 和 $\bvec r_B$, 相对位移为 $\bvec R = \bvec r_B - \bvec r_A$, 两质点距离为 $R = \abs{\bvec R}$. 且 $A$ 对 $B$ 的作用力为 $\bvec F = F(R) \uvec R$, $B$ 对 $A$ 的反作用力为 $-\bvec F$.现在考虑一个过程中力对两质点做的总功.

在一段微小时间 $\dd{t}$ 内, 两质点分别移动了 $\dd{\bvec r_A}$, 和 $\dd{\bvec r_B}$, 则相互作用力对二者做功为
\begin{equation}
\dd{W} = \bvec F \vdot \dd{\bvec r_B} + (- \bvec F) \vdot \dd{\bvec r_A} = \bvec F \vdot \dd{\bvec R}
= F(R)\uvec R\vdot \dd{\bvec R} = F(R) \dd{R}
\end{equation}
(最后一步的证明见“位置矢量\upref{Disp}”中的\autoref{Disp_ex1})定积分得
\begin{equation}
W = \int_{R_1}^{R_2}  F(R) \dd{R}
\end{equation}
现在我们借用一维势能的定义\autoref{V_eq8} 来定义势能函数为 $F(R)$ 的负原函数, 则力在一段时间内对两质点做的总功就等于末势能减初势能
\begin{equation}\label{V_eq20}
W = V(R_2) - V(R_1)
\end{equation}


\subsection{含时势能}
以上的讨论中, 我们默认力场的分布不随时间变化, 所得势能显然也不随时间变化. 但在一些情况下, 我们也可以定义随时间变化的势能.

%未完成

%例如\autoref{V_ex1} 中如果中心天体随时间变化, 那么力场

%把这种假想的位移叫做虚位移

%这个偏导可以理解为




