% 导数的性质与构造(高中)
% keys 导数|性质|构造|恒等
% license Usr
% type Tutor

\begin{issues}
\issueDraft
\end{issues}

\subsection{近似代替}

在导数的\aref{几何含义}{sub_HsDerv_1}中就提到过“以直代曲”。

\begin{equation}
f(x_0+\Delta x)\approx f(x_0)+f'(x_0)\Delta x~.
\end{equation}

\subsection{单调性}

\subsection{高阶导数}

导函数作为原函数,则又可以求得它的导函数,这也被称为高阶导数。

凹凸性

\subsection{常用构造}