% 前缀和
% keys 前缀和|算法|C++

前缀和算法分为一维和二维,一维前缀和可以很快速的求序列中某一段的和.而二维前缀和可以快速求一个矩阵中某个子矩阵的和(以下讲的是以 $x_1, y_1$ 为左上角,$x_2, y_2$ 为右下角的子矩阵 ).

\subsection{一维前缀和}
一维前缀和的朴素做法是一个 \verb|for| 循环求和,时间复杂度为:$\mathcal{O}(n \times m)$.$m$ 为询问次数.

而前缀和的时间复杂度为:预处理 $\mathcal{O}(n)$,查询 $\mathcal{O}(m)$.

一维前缀和:
\begin{equation}
\sum^{r}_{i=l}a_i
\end{equation}

预处理:
\begin{equation}
S_i=\sum^{i}_{k=1}a_k
\end{equation}

如何预处理?

\begin{equation}
\because 
\sum^{i}_{k=1}a_k-\sum^{i-1}_{k=1}a_{k}=a_k \\
 \therefore \sum^{i}_{k=1}a_k=\sum^{i-1}_{k=1}a_{k}+a_k \\
\end{equation}

即:

\begin{equation}
\\S_k-S_{k-1}=a_k\\
 \\S_k=S_{k-1}+a_k\\
\end{equation}

如何求答案?

\begin{equation}
\sum^{r}_{i=l}a_i=\sum^{r}_{i=1}a_i-\sum^{l-1}_{i=1}a_i=S_r-S_{l-1}
\end{equation}

还有要注意的是 $S_0 = 0$,因为如果想求 $1 \sim n$ 一段数的和可以直接输出 $S_{10} - S_0 = S_{10} $.因此可以从下标为 $1$ 开始存数,这样就不需要做复杂的处理.

\textbf{朴素代码}
\begin{lstlisting}[language=cpp]
const int N = 100010;
int a[N];

int main()
{
    int n, m;
    cin >> n >> m;
    for (int i = 1; i <= n; i ++ ) cin >> a[i];
    
    while (m -- )
    {
        int l, r;
        cin >> l >> r;
        
        int res = 0;
        for (int i = l; i <= r; i ++ ) res += a[i];
        cout << res << endl;
    }
    
    return 0;
}
\end{lstlisting}

\textbf{前缀和}
\begin{lstlisting}[language=cpp]
const int N = 100010;
int n, m, a[N], s[N];

int main() 
{
    cin >> n >> m;
    for (int i = 1; i <= n; i ++ ) cin >> a[i];
    for (int i = 1; i <= n; i ++ ) s[i] = s[i - 1] + a[i];
    
    while (m -- )
    {
        int l, r;
        cin >> l >> r;
        cout << s[r] - s[l - 1] << endl;    // 前缀和公式,O(1)
    }
    
    return 0;
}
\end{lstlisting}

\subsection{二维前缀和}

二维前缀和的朴素做法是两重循环暴力求和.时间复杂度为:$\mathcal{O}(q \times nm)$.$q$ 为询问次数,$n$、$m$ 为矩阵大小.

而二维前缀和的时间复杂度为 $\mathcal{O}(n \times m)$

二维前缀和:

\begin{equation}
\sum^{x2}_{i = x1}\sum^{y2}_{j = y1} a_{ij}
\end{equation}

预处理:

\begin{equation}
S_{ij} = \sum^{i}_{k_1 = 1} \sum^{j}_{k_2 = 1} a_{k_1 k_2}
\end{equation}

如何预处理?

在预处理 $S_{ij}$ 时,要注意 $S_{i - 1, m}$ 和 $S_{i, j - 1}$ 都已经被算过了,因为预处理是逐行逐列来计算的(重复地方不重复计算).
\begin{equation}
S_{ij} = \sum^{i}_{k_1 = 1} \sum^{j}_{k_2 = 1} a_{k_1 k_2} = \sum^{i - 1}_{k_1 = 1} \sum^{j}_{k_2 = 1} + \sum^{i}_{k_1 = 1} \sum^{j - 1}_{k_2 = 1} - \sum^{i- 1}_{k_1 = 1} \sum^{j - 1}_{k_2 = 1} + a_{i, j}
\end{equation}

即:

\begin{equation}
S_{i, j} = S_{i - 1, j} + S_{i, j - 1} - S_{i - 1, j - 1} + a_{i, j}
\end{equation}

如何求答案?

\begin{equation}
\sum^{x_2}_{i = x_1}\sum^{y_2}_{j = y_1} a_{ij} = \sum^{x_2}_{k_1 = 1} \sum^{y_2}_{k_2 = 1} a_{k_1 k_2} - \sum^{x_1 - 1}_{k_1 = 1} \sum^{y_2}_{k_2 = 1} a_{k_1 k_2} - \sum^{x_2}_{k_1 = 1} \sum^{y_1 - 1}_{k_2 = 1} a_{k_1 k_2} + \sum^{x_1 - 1}_{k_1 = 1} \sum^{y_1 - 1}_{k_2 = 1} a_{k_1 k_2}
\end{equation}

即:

\begin{equation}
S_{x_2, y_2} - S_{x_1 - 1, y_2} - S_{x_2, y_1 - 1} + S_{x_1 - 1, y_1 - 1}
\end{equation}

二维前缀和可能有点抽象,可以根据几张图片来对二维前缀和有个更好的理解.

\begin{figure}[ht]
\centering
\includegraphics[width=14.25cm]{./figures/PS_2.png}
\caption{预处理 $1$} \label{PS_fig2}
\end{figure}


\begin{figure}[ht]
\centering
\includegraphics[width=14.25cm]{./figures/PS_1.png}
\caption{预处理 $2$ } \label{PS_fig1}
\end{figure}

