% 2010 年计算机学科专业基础综合全国联考卷
% 考研 计算机 全国联考

\subsection{一、单项选择题}
第1~40 小题,每小题2 分,共80 分.下列每题给出的四个选项中,只有一个选项最符合试题要求.

1. 若元素a、b、c、d、e、f 依次进栈,允许进栈、退栈操作交替进行,但不允许连续三次进行退栈操作,则不.可能得到的出栈序列是______. \\
A. d c e b f a $\quad$ B. c b d a e f $\quad$ C. b c a e f d $\quad$ D. a f e d c b

2. 某队列允许在其两端进行入队操作,但仅允许在一端进行出队操作.若元素a、b、c、d、e 依次入此队列后再进行出队操作,则不.可能得到的出队序列是______ \\
A. b a c d e $\quad$ B. d b a c e $\quad$ C. d b c a e $\quad$ D. e c b a d

3. 下列线索二叉树中(用虚线表示线索),符合后序线索树定义的是______.\\
\begin{figure}[ht]
\centering
\includegraphics[width=14.25cm]{./figures/CSN10_1.png}
\caption{第3题图} \label{CSN10_fig1}
\end{figure}

4. 在右图所示的平衡二叉树中,插入关键字48 后得到一棵新平衡二叉树.在新平衡二叉树中,关键字37 所在结点的左、右子结点中保存的关键字分别是______. \\
\begin{figure}[ht]
\centering
\includegraphics[width=5cm]{./figures/CSN10_2.png}
\caption{第4题图} \label{CSN10_fig2}
\end{figure}
A.13,48 $\quad$ B.24,48 $\quad$ C.24,53 $\quad$ D、24,90

