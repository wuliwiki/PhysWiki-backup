% 上海海事大学 2011 年数据结构
% keys 上海海事大学 2011 年数据结构

\subsection{一.判断题(本题20分,每小题2分)}
1.为了很方便地插入和删除数据,可以使用双向链表存放数据.

2.两个栈共享一片连续内存空间时,为了提高内存利用率,减少溢出机会,应把两个栈的栈底分别设在这片内存空间的两端.

3.数组是同类型值的集合.

4.在查找树(二叉排序树)中插入一个新结点,总是插入到叶子结点的下面.

5.用邻接矩阵存储一个图时,在不考虑压缩存储的情况下,所占用的存储空间大小与图中顶点的个数有关,而与图的边数无关.

6.顺序存储方式只能用于存储线性结构,不能用于存储二叉树.

7.在执行某个排序算法过程中,出现了排序码朝着最终排序序列位置相反方向移动,则该算法是不稳定的.

8.数据的逻辑结构被分为集合结构、线性结构、树型结构、图结构四种,

9.将一棵树转换成二叉树后,根结点没有左子树.

10.哈夫曼树是带权路径长度最短的树,路径上权值较大的结点离根较近.


\subsection{二.填空题(本题30分,每空2分)}
1.分析下列程序段,其时间复杂度分别为:( (1) ),( (2) ).
\begin{lstlisting}[language=cpp]
i=1;
while(i<=n)
    i=i*3;

void testint m) {
    int i=0, s=0;
    while (s<n) {
        i++;
        s=s+i;
    }
}
\end{lstlisting}
2.堆栈的插入和删除操作都是在栈顶位置进行,而队列的__ (3)_ _操作在队尾进
行,_ (4) 操作在队头进行.
3.对具有n个结点的二叉树采用二叉链表存储结构,则该链表中有__ (5)_ 个指针
域,其中有__ (6)_个指针域用于链接孩子结点,_ (7) 个 指针域空闲存放着NULL.
4.对线性表采用折半查找方法,该线性表必须采用__ (8) 存储结构, 并且数据元
素按值(9)
5.除了顺序存储结构与链式存储结构之外,数据的存储结构通常还有__ (0) 结构
和
(1)_ 结构.
6.已知具有4行6列的矩阵A采用行序为主序方式存储,每个元素占用4个存储
单元,并且a[3][4]的存储地址为1234,元素a[1][1]的存储地址是___ (02)
7.对于长度为n的线性表,采用顺序存储结构存储,插入或删除一个元素的时间复
杂度为_ (13) _;

1.分析下列程序段,其时间复杂度分别为:_ (1)__ ._ (2)
i= 1;
void test(int m) {
while(i<=n)
int i=0, s=0;
i=i* 3;
while (s<n) {
i++;
s=s+i;
2.堆栈的插入和删除操作都是在栈顶位置进行,而队列的__ (3)_ _操作在队尾进
行,_ (4) 操作在队头进行.
3.对具有n个结点的二叉树采用二叉链表存储结构,则该链表中有__ (5)_ 个指针
域,其中有__ (6)_个指针域用于链接孩子结点,_ (7) 个 指针域空闲存放着NULL.
4.对线性表采用折半查找方法,该线性表必须采用__ (8) 存储结构, 并且数据元
素按值(9)
5.除了顺序存储结构与链式存储结构之外,数据的存储结构通常还有__ (0) 结构
和
(1)_ 结构.
6.已知具有4行6列的矩阵A采用行序为主序方式存储,每个元素占用4个存储
单元,并且a[3][4]的存储地址为1234,元素a[1][1]的存储地址是___ (02)
7.对于长度为n的线性表,采用顺序存储结构存储,插入或删除一个元素的时间复
杂度为_ (13) _;
