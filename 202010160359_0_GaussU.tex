% 高斯单位制
% keys 厘米|克|秒|国际单位|CGS

\begin{issues}
\issueDraft
\end{issues}

\pentry{厘米—克—秒单位制\upref{CGS}}

\subsubsection{电荷}
CGS 单位满足库伦定律
\begin{equation}
F = \frac{q_1 q_2}{r^2}
\end{equation}
注意与国际单位相比, 电荷的量纲发生变化, 单位为 $\Si{\sqrt{cm^3 g}/s}$, 为了方便我们不妨记为\footnote{这是笔者发明的记号} $C_g$. 电荷的转换常数为\footnote{如果我们认为 $\Si{C}$ 和 $\Si{C_g}$ 是相同的量纲, 则 $\beta_q = 1$.}
\begin{equation}
\beta_q = \sqrt{4\pi\epsilon_0\beta_F} \beta_x = 3.3356409510736\times 10^{-10} \Si{C/C_g}
\end{equation}

\subsubsection{电场}
电场需要满足
\begin{equation}
\bvec F = \bvec E q
\end{equation}
电场单位为 $\Si{\sqrt{g/cm}/s}$, 量纲是 SI 的磁场量纲乘以速度.
\begin{equation}
\beta_{\mathcal E} = \beta_F/\beta_q = 2.997924580815998\times 10^4 \Si{\frac{C_g\cdot m\cdot kg}{C\cdot cm\cdot g}}
\end{equation}

\subsubsection{磁场}
高斯单位制中, 洛伦兹力为
\begin{equation}
\bvec F = \frac{q}{c} \bvec v \cross \bvec B
\end{equation}
可得电磁场具有相同的单位. 转换常数为(量纲为速度倒数而不是 1)\footnote{2020 新国际单位标准以前, 这个数值精确等于 $1/100$, 新标准需要乘以 $\sqrt{\mu_0/(4\times 10^{-7}\pi)}$ 的数值.}
\begin{equation}
\beta_B = \frac{\beta_m}{c\beta_q} = \frac{\beta_{\mathcal E}}{c\beta_x} = 1.000000000272\times 10^{-2} \Si{\frac{C_g\cdot kg \cdot s}{C \cdot g \cdot m}}
\end{equation}
