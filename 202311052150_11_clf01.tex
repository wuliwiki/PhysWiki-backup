% Clifford 代数
% keys 环|二次型
% license Xiao
% type Tutor

如果要构建有限维结合代数,我们需要对basis的结合作出约束。例如,Grassmann代数要求反对称性:$\mathrm {e_1e_2}=-\mathrm {e_2e_1}$。实际上,Grassmann代数是几何代数的一个平凡特例。本节先给出线性空间上的几何代数定义,再用集合语言将其拓展为Clifford代数。
\subsection{几何代数}
\begin{theorem}{线性空间的理想}
给定域$\mathbb F$上的线性空间$V$,其上有一二次型$B_q(B_q(v,w)=\frac{1}{2}(q(v+w)-q(v)-q(w)))$.令$\mathcal T(V)$为$V$上的张量代数,那么如下定义的$\mathcal {I}_q(V)$是它的理想:
\begin{equation}
\mathcal{I}_{q}(V)=\left\{\sum A_{k} \otimes(v_k \otimes v_k-q(v_k)) \otimes B_{k} \mid v \in V, A_{k}, B_{k} \in \mathcal{T}(V)\right\}~,
\end{equation}
\end{theorem}
proof.
环理想首先是加法子群,其次对乘法有“吸收律”。该定理可以简化成一个更简单的形式。即对于环$R$上的一个非空子集$S$,我们可以证明该子集生成的理想为
\begin{equation}
\mathcal {I}_S=\left\{\sum _k a_k s_kb_k|k\in \mathbb N ,a_k,b_k\in R,s_k\in S\right\}~,
\end{equation}
检查理想的定义,该集合确实构成加法子群。其次,无论是左乘还是右乘环元素,都能表示为该形式,因而是理想。对于张量代数,乘法为张量积。

\begin{definition}{几何代数}
给定域$\mathbb F$上的线性空间$V$,其上有一二次型$B_q$.理想同上定义,商代数则为几何代数(\textbf{geometric algebra}),即
\begin{equation}
\mathcal{G}(V, q) \stackrel{\text { def }}{=} \mathcal{T}(V) / \mathcal{I}_{q}(V)~,
\end{equation}
另外,称$V$为$\mathcal{G}(V, q)$的底空间(\textbf{base space})。
\end{definition}
划分等价类后,把$\mathcal{G}(V, q)$上的向量积称为\textbf{几何积},或者\textbf{Clifford积},符号可以用$\cdot$或者不写。

观察等价类,我们会发现一项特殊的等价关系。即
\begin{equation}
v\cdot v=q(v)~,
\end{equation}

理想首先是正规子群。回想对正规子群求商集时,若$a,b$等价,即属于同一左陪集,那么$a^{-1}b$也属于该正规子群。因而,上式的等价关系实际上指的是
$v \otimes v-q(v)$在理想里,显然这是成立的。该等价关系得以让我们把重复的$\mathrm {e_i}$约掉。如果$\{\mathrm{e_i}\}$还是线性空间中的正交基,我们有
\begin{equation}
\begin{aligned}
q(e_i+e_j,e_i+e_j)&=q(e_i)+q(e_j)+2 B_q(e_i,e_j)\\
&=e_ie_i+e_je_j+e——i~,
\end{aligned}
\end{equation}

因而$e_ie_j=-e_je_i$



\subsection{Clifford代数的形式化定义}
几何代数的基域为线性空间,而\textbf{Clifford代数}的基域是交换幺环。
\subsubsection{分次结构}

