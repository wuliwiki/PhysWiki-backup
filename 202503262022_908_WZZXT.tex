% 微正则系统(综述)
% license CCBYSA3
% type Wiki

本文根据 CC-BY-SA 协议转载翻译自维基百科\href{https://en.wikipedia.org/wiki/Microcanonical_ensemble}{相关文章}。

在统计力学中,微正则系综是一个统计系综,表示总能量被精确指定的机械系统的可能状态。\(^\text{[1]}\)假设系统是孤立的,即它不能与环境交换能量或粒子,因此(根据能量守恒)系统的能量随着时间不发生变化。

微正则系综的主要宏观变量是系统中的总粒子数(符号:\( N \))、系统的体积(符号:\( V \))以及系统中的总能量(符号:\( E \))。这些变量在系综中被假设为常数。因此,微正则系综有时被称为\(NVE\)系综。

简单来说,微正则系综通过为每个能量落在以\( E \)为中心的范围内的微观状态分配相等的概率来定义。所有其他微观状态的概率为零。由于概率总和必须为 1,因此概率 \( P \)是能量范围内微观状态数\( W \)的倒数,
\[
P = 1/W,~
\]
然后能量范围的宽度被逐渐缩小,直到它变得无限窄,仍然以\( E \)为中心。在这个过程中,当宽度趋于零时,得到微正则系综。\(^\text{[1]}\)
\subsection{适用性}  
由于与平衡统计力学的基本假设(特别是先验相等概率的假设)相关,微正则系综是该理论中的一个重要概念性基石。\(^\text{[2]}\)它有时被认为是平衡统计力学的基本分布。它在一些数值应用中也很有用,例如分子动力学。\(^\text{[3][4]}\)另一方面,大多数非平凡系统在微正则系综中的数学描述繁琐,而且在熵和温度的定义上也存在一些模糊性。因此,在理论计算中,通常更倾向于使用其他系综。\(^\text{[2][5][6]}\)

微正则系综对现实世界系统的适用性取决于能量波动的重要性,这些波动可能源于系统与环境之间的相互作用以及在准备系统时的不可控因素。通常,如果系统在宏观上很大,或者系统的能量已被精确地确定并且之后几乎与环境隔离,那么波动可以忽略不计。\(^\text{[7]}\)在这种情况下,微正则系综是适用的。否则,其他系综更为合适——例如正则系综(能量波动)或巨正则系综(能量和粒子数波动)。
\subsection{性质}  
\subsubsection{热力学量}  
微正则系综的基本热力学势是熵。熵有至少三种可能的定义,每种定义都通过相空间体积函数 \( v(E) \) 表示。在经典力学中,\( v(E) \) 是相空间中能量小于 \( E \) 的区域的体积。在量子力学中,\( v(E) \) 大致是能量小于 \( E \) 的能量本征态的数量;然而,这必须进行平滑处理,以便我们能够求其导数(有关如何处理的详细信息,请参见“精确表达式”部分)。微正则熵的定义如下:
\begin{itemize}
\item 玻尔兹曼熵 \( S_B \):
\[
S_B = k \log W = k \log \left( \omega \frac{dv}{dE} \right)~
\]
玻尔兹曼熵依赖于所谓的“能量宽度” \( \omega \),这是一个具有能量单位的任意量,通常取为小值,引入它是为了使我们对一个无量纲量取对数,因为 \( \frac{dv}{dE} \) 的单位是 1/能量。
\item ‘体积熵’:
\[
S_v = k \log v~
\]
\item ‘表面熵’:
\[
S_s = k \log \frac{dv}{dE} = S_B - k \log \omega~
\]
\end{itemize}
在表面熵中,我们对具有逆能量单位的量取对数,因此改变能量单位会通过加性常数改变该量。玻尔兹曼熵可以看作是表面熵的一种变体,避免了这个问题。

在微正则系综中,温度是一个派生量,而不是外部控制参数。它被定义为所选熵关于能量的导数。[8] 例如,可以定义“温度” \( T_v \)和\( T_s \)如下:
\[
1/T_v= dS_v/dE,~
\]
\[
1/T_s= dS_s/dE=dS_B/dE.~
\]
像熵一样,在微正则系综中有多种方式理解温度。更一般地说,这些基于系综的定义与它们的热力学对应物之间的关系并不完美,尤其对于有限系统。

微正则系综中的压强和化学势由以下公式给出:\(^\text{[9]}\)
\[
\frac{p}{T} = \frac{\partial S}{\partial V}; \qquad \frac{\mu}{T} = -\frac{\partial S}{\partial N}.~
\]
\subsubsection{相变}  
根据严格的定义,相变对应于热力学势或其导数中的非解析行为。\(^\text{[10]}\) 使用这个定义,微正则系综中的相变可以发生在任何大小的系统中。这与正则系综和巨正则系综形成对比,后者的相变只能在热力学极限中发生——即,在具有无限多自由度的系统中。\(^\text{[10][11]}\)粗略来说,定义正则系综或巨正则系综的热库引入了波动,这些波动“平滑”了有限系统中自由能的任何非解析行为。对于宏观系统来说,这种平滑效应通常可以忽略不计,因为这些系统足够大,能够非常好地近似自由能的非解析行为。然而,在小系统的理论分析中,系综的技术性差异可能是重要的。\(^\text{[11]}\)
\subsubsection{信息熵}  
对于给定的机械系统(固定\( N \)、\( V \))和给定的能量范围,微观状态上概率 \( P \)的均匀分布(如在微正则系综中)最大化了系综平均值\( -\langle \log P \rangle \)。\(^\text{[1]}\)
\subsection{热力学类比}  
路德维希·玻尔兹曼在统计力学方面的早期工作导致了他为具有给定总能量的系统提出的同名熵方程\( S = k \log W \),其中 \( W \)是系统在该能量下可达的不同状态的数量。玻尔兹曼并未深入阐述到底什么构成了一个系统的不同状态集合,除了理想气体的特殊情况。这个话题由乔赛亚·威拉德·吉布斯进行了彻底的研究,他为任意机械系统发展了广义统计力学,并定义了本文描述的微正则系综。\(^\text{[1]}\)吉布斯仔细研究了微正则系综与热力学之间的类比,特别是在自由度较少的系统中它们是如何失效的。他引入了两个不依赖于\( \omega \)的微正则熵的进一步定义——上述的体积熵和表面熵。(请注意,表面熵与玻尔兹曼熵的不同之处仅在于\( \omega \)-依赖的偏移量。)

体积熵\( S_v \)和相关温度\( T_v \)与热力学熵和温度密切类比。可以准确地证明
\[
dE = T_v dS_v - \langle P \rangle dV,~
\]
(\( \langle P \rangle \)是系综平均压强),这与热力学第一定律一致。对于表面熵\( S_s \)(或玻尔兹曼熵\( S_B \))及其相关温度\( T_s \),也可以得到类似的方程,然而,这个方程中的“压强”是一个复杂的量,与平均压强无关。\(^\text{[1]}\)