% Cauchy-Schwarz 不等式
% keys 不等式
% license Usr
% type Tutor
Cauchy-Schwarz不等式是一个在众多背景下都有应用的不等式,\ 例如线性代数,\ 数学分析,\ 概率论,\ 向量代数以及其他许多领域,\ 这就意味着它有很多等价形式或者推广形式.\ 它被认为是数学中最重要的不等式之一.\ 此不等式最初于1821年被Cauchy提出,\ 其积分形式在1859被Buniakowsky提出,\ 而积分形式的现代证明则由Schwarz于1888年给出.\ 

在给出Cauchy-Schwarz不等式之前,\ 先给出几个常用不等式作为引理并且给出相关证明.\ 
\begin{lemma}{三角不等式}
$\forall \boldsymbol{x},\boldsymbol{y}\in\mathbb{R}^n$,满足
\begin{equation}
\lvert\lvert\boldsymbol{x}\rvert-\lvert\boldsymbol{y}\rvert\rvert\leqslant\lvert\boldsymbol{x}\pm\boldsymbol{y}\rvert\leqslant\lvert\boldsymbol{x}\rvert+\lvert\boldsymbol{y}\rvert~
\end{equation}
\end{lemma}
\begin{lemma}{均值不等式}
$\forall \boldsymbol{x}\in\mathbb{R}^n$,定义以下和式
\begin{align}
H_n&=\frac{1}{\frac{1}{n}\sum_{i=1}^n\frac{1}{x_i}}=\frac{n}{}\\
G_n&=\sqrt[n]{\prod_{i}^n x_i}\\
A_n&=\frac{1}{n}\sum_{i=1}^n x_i\\
Q_n&=\sqrt{\frac{1}{n}\sum_{i=1}^n x_i^2}~
\end{align}
\end{lemma}
