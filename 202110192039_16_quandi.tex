% 量子化狄拉克场
% 量子化|狄拉克场|费米子

我们来总结一下狄拉克场的量子化的方法
\begin{equation}
\psi(x) = \int \frac{d^3 p}{(2\pi)^3} \frac{1}{\sqrt{2 E_{\mathbf p}}} \sum_s (a_{\mathbf p}^s u^s (p) e^{-ip\cdot x} + b_{\mathbf p}^{s\dagger} v^s(p) e^{ip\cdot x})~,
\end{equation}
\begin{equation}
\bar \psi (x) = \int \frac{d^3 p}{(2\pi)^3} \frac{1}{\sqrt{2 E_{\mathbf p}}} \sum_s (b_{\mathbf p}\bar v ^s(p) e^{- i p\cdot x}+ a_{\mathbf p}^{s\dagger} \bar u^s(p)e^{i p \cdot x})~.
\end{equation}
产生湮灭算符满足下面的对易关系
\begin{equation}
\{a_{\mathbf p}^r,a_{\mathbf q}^{s\dagger}\} = \{b_{\mathbf p}^r,b_{\mathbf q}^{s\dagger}\} = (2\pi)^3\delta^{(3)}(\mathbf p - \mathbf q)\delta^{rs}~.
\end{equation}
$\psi$和$\psi^\dagger$的等时对易关系为
\begin{equation}
\{\psi_a(\mathbf x),\psi_b^\dagger(\mathbf y)\} = \delta^{(3)}(\mathbf p - \mathbf q)\delta^{rs}~.
\end{equation}
\begin{equation}
\{\psi_a(\mathbf x),\psi_b(\mathbf y)\} = \{\psi_a^\dagger(\mathbf x),\psi_b^\dagger(\mathbf y)\} = 0 ~.
\end{equation}
真空$|0\rangle$定义为
\begin{equation}
a_{\mathbf p }^s|0\rangle = b_{\mathbf p}^s|0\rangle = 0
\end{equation}
哈密顿量定义为
\begin{equation}
H = \int \frac{d^3p}{(2\pi)^3}\sum_s E_{\mathbf p} (a^{s\dagger}_{\mathbf p}a^{s}_{\mathbf p}+b^{s\dagger}_{\mathbf p}b_{\mathbf p}^s)~.
\end{equation}
注意这里,我们已经把无穷大的常数扔掉了.动量算符定义为
\begin{equation}
\mathbf P = \int d^3 x \psi^\dagger (-i\nabla)\psi = \int \frac{d^3 p}{(2\pi)^3} \sum_s \mathbf p (a_{\mathbf p}^{s\dagger} a_{\mathbf p}^s+b_{\mathbf p}^{s\dagger}b_{\mathbf p}^s)~.
\end{equation}
其中$b_{\mathbf p}^{s\dagger}$和$a_{\mathbf p}^{s\dagger}$都是产生能量是$E_{\mathbf p}$, 动量是$\mathbf p$的粒子的算符.我们就把$a_{\mathbf p}^{s\dagger}$产生的粒子称为费米子.$b_{\mathbf p}^{s\dagger}$产生的粒子称为反费米子.

现在我们来定义单粒子态
\begin{equation}
|\mathbf p,s\rangle \equiv \sqrt{2 E_{\mathbf p}} a_{\mathbf p}^{s\dagger} | 0 \rangle 
\end{equation}
现在我们来定义归一化条件
\begin{equation}
\langle \mathbf p,r|\mathbf q,s\rangle = 2 E_{\mathbf p}(2\pi)^3\delta^{(3)}(\mathbf p - \mathbf q) \delta^{rs}
\end{equation}
是洛仑兹不变的.















