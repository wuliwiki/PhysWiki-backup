% 斜对称双线性型的规范型
% 斜对称|双线性型|辛平面|规范型

\pentry{二次型的规范型\upref{GuaOQu}}
在二次型\upref{QuaFor}一节中,已经知道每一对称双线性型 $f$ 对应一个二次型 $q$ ,并且每一对称双线性型 $f$ 都有一规范基底,在此基底下,二次型 $q$ 呈现一种简单的形式
\begin{equation}
q(\bvec x)=\sum_{i}f_{ii}x_i^2
\end{equation}
现在我们转向斜对称的双线性型\autoref{MulMap_def1}~\upref{MulMap},也就是满足 
\begin{equation}
f(\bvec x,\bvec y)=-f(\bvec y,\bvec x)\quad \forall \bvec x,\bvec y\in V
\end{equation}
的2-线性函数.

若 $V_0$ 是斜对称双线性型 $f$ 的\textbf{核},也就是子空间\upref{SubSpc}
\begin{equation}
V_0=\mathrm{Ker} f=\{\bvec v\in V|f(\bvec v,\bvec x)=0,\forall\bvec x\in V\}
\end{equation}
那么 $f$ 在 $V_0$ 的补空间(\autoref{DirSum_def1}~\upref{DirSum}) $V_1$ 上的限制 $f|_{V_1}$ 必为非退化的斜对称型.这是因为,如果 $\bvec a\neq0\in V_1$ 且 $f(\bvec a,\bvec x_1)=0$ 对所有的 $\bvec x_1\in V$ 都成立,那么任意向量 $\bvec x=\bvec x_0+\bvec x_1\in V \;(\bvec x_0\in V_0)$ ,有
\begin{equation}
f(\bvec a,\bvec x)=f(\bvec a,\bvec x_0+\bvec x_1)=f(\bvec a,\bvec x_0)+f(\bvec a,\bvec x_1)=-f(\bvec x_0,\bvec a)=0
\end{equation}
 这就是说 $\bvec a\in V_0$,这是不可能的.这说明对任意 $\bvec a\neq0\in V_1$,总有 $\bvec x_1\in V_1$ 使得 $f(\bvec a,\bvec x_1)\neq0$ .