% 引力波
% license CCBYSA3
% type Wiki

(本文根据 CC-BY-SA 协议转载自原搜狗科学百科对英文维基百科的翻译)


引力波是时空曲率中的扰动,由加速的质量所产生,并以光速从源头向外传播。它们是由Henri Poincaré在1905年提出的[1],后来在1916年被爱因斯坦根据他的广义相对论所预测。[2][3][4] 引力波以引力辐射的形式传输能量,这是一种类似于电磁辐射的辐射能。[5] 作为经典力学的一部分,牛顿的万有引力定律并没有规定它们(引力波)的存在,因为该定律是基于物理相互作用瞬时(以无限大速度)传播的假设,这展示了经典物理学方法无法解释相对论现象的一个例子。

引力波天文学是观测天文学的一个分支,它使用引力波来收集关于可探测引力波来源的观测数据,例如由白矮星、中子星和黑洞组成的双星系统,和超新星等事件,以及大爆炸后不久的早期宇宙的形成。

1993年,Russell A. Hulse和Joseph H. Taylor, Jr.因为发现和观测赫尔斯-泰勒双星而获得诺贝尔物理学奖,这是引力波存在的第一个间接证据。[6]

2016年2月11日,激光干涉引力波天文台和处女座干涉仪科学合作组织宣布他们首次直接观测到引力波。这一观察发生在五个月前,即2015年9月14日,在这次观测中使用了增进LIGO探测器。这次事件的引力波起源于一对合并的黑洞。 在第一次探测到引力波的消息宣布之后,激光干涉引力波天文台的仪器又探测到了两次确认的引力波事件以及一次潜在的引力波事件。[7][8] 2017年8月,两个激光干涉引力波天文台的仪器和处女座干涉仪观测到第四次来自合并黑洞的引力波,[9] 和来自双星合并的第五次引力波。 另外几个引力波探测器正在计划或建造中。[10]

2017年,Rainer Weiss、Kip Thorne和Barry Barish因他们在引力波直接探测中所作的贡献而获得了诺贝尔物理学奖。

\begin{figure}[ht]
\centering
\includegraphics[width=6cm]{./figures/b96d7130530cad7b.png}
\caption{对两个黑洞的碰撞的模拟。除了形成深重力阱并合并成一个更大的黑洞外,当黑洞相互旋转时引力波将向外传播。} \label{fig_YLB_1}
\end{figure}

\subsection{介绍}

在爱因斯坦的广义相对论中,引力被视为时空曲率导致的一种现象。这种曲率是由质量的存在所引起的。一般来说,在一个给定体积的空间中包含的质量越大,其边界处的时空曲率就越大。[11] 当有质量的物体在时空中移动时,曲率会发生改变,反映出这些物体位置的变化。在某些情况下,加速的物体会引发曲率的变化,并以波的形式向外以光速传播。这些传播现象被称为引力波。

\begin{figure}[ht]
\centering
\includegraphics[width=6cm]{./figures/7857d71a3c37254e.png}
\caption{线性极化引力波} \label{fig_YLB_2}
\end{figure}

当引力波经过观察者时,观察者会发现时空被应变的影响扭曲了。物体之间的距离随着波的传播有节奏地增加和减少,频率等于波的频率。尽管这些自由物体从未受到不平衡的力的作用,这种情况还是会发生。这种效应的大小与其跟引力波源的距离成反比。[12] 由于它们的质量在它们彼此靠近的轨道上有非常大的加速度,互相螺旋靠近合并的双中子星被预测为引力波的强大来源。然而,由于我们与这些辐射源之间巨大的天文距离,在地球上测量到的影响预计非常小,应变不到$1/10^{20}$。科学家已经用越来越灵敏的探测器证明了这些波的存在。最灵敏的探测器完成了由LIGO和VIRGO天文台提供的$1/{5 \times 10^{22}}$截至2012年)的灵敏度测量任务。[13] 欧洲空间局目前正在开发一个名为激光干涉空间天线的空基天文台。


\subsection{历史}

\subsection{引力波经过时的影响}

\subsection{来源}

\subsubsection{4.1 双星系统}

\textbf{致密双星系统}

\subsubsection{4.2 黑洞双星}

\subsubsection{4.3 超新星}

\subsubsection{4.4 自转的中子星}

\subsubsection{4.5 暴胀}

\subsection{性质和行为}

\subsubsection{5.1 能量、动量和角动量}

\subsubsection{5.2 红移}

\subsubsection{5.3 量子引力、波粒方面和引力子}

\subsubsection{5.4 对研究早期宇宙的意义}

\subsubsection{5.5 确定运动方向}

\subsection{引力波天文学}

\subsection{探测}

\subsubsection{7.1 间接探测}

\subsubsection{7.2 困难}

\subsubsection{7.3 地面探测器}

\textbf{共振天线}

\textbf{干涉仪}

\textbf{Einstein@Home}

\subsubsection{7.4 天基干涉仪}

\subsubsection{7.5 使用脉冲星计时阵列}

\subsubsection{7.6 原初引力波}

\subsubsection{7.7 激光干涉引力波天文台和处女座干涉仪的观测}

\subsection{8 在小说中}

\subsection{参考文献}