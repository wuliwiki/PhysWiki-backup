% 丹尼尔·伯努利(综述)
% license CCBYSA3
% type Wiki

本文根据 CC-BY-SA 协议转载翻译自维基百科\href{https://en.wikipedia.org/wiki/Michael_Faraday}{相关文章}。

\begin{figure}[ht]
\centering
\includegraphics[width=6cm]{./figures/12ffafd992474bb6.png}
\caption{丹尼尔·伯努利的肖像,约1720-1725年} \label{fig_BNL_1}
\end{figure}
丹尼尔·伯努利 FRS(/bɜːrˈnuːli/ 伯-努-利;瑞士标准德语:[ˈdaːni̯eːl bɛrˈnʊli];1700年2月8日[公历1月29日] – 1782年3月27日)是一位瑞士数学家和物理学家,并且是来自巴塞尔的伯努利家族中的众多杰出数学家之一。他特别以其将数学应用于力学,尤其是流体力学,以及在概率和统计学领域的开创性工作而闻名。[3] 他的名字在伯努利原理中得以纪念,这一原理是能量守恒的一个具体例子,描述了支撑20世纪两项重要技术——化油器和飞机机翼——运作原理的数学机制。[4][5]