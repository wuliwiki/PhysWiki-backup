% 电容的串联和并联

\pentry{电容\upref{Cpctor}}

\begin{figure}[ht]
\centering
\includegraphics[width=9cm]{./figures/Ccomb_1.pdf}
\caption{电容并联(左)和串联(右)} \label{Ccomb_fig1}
\end{figure}

\subsubsection{电容的并联}
如\autoref{Ccomb_fig1} 左边为两电容器的并联,由于并联电压相等
\begin{equation}
U_1=U_2=U
\end{equation}

由电容的定义\autoref{Cpctor_eq2}~\upref{Cpctor}知道
\begin{equation}
\begin{aligned}
Q_1=C_1U\\
Q_2=C_2U
\end{aligned}
\end{equation}
所以,两电容器并联后总电容为
\begin{equation}
C=\frac{Q_1+Q_2}{U}=\frac{Q_1}{U}+\frac{Q_2}{U}=C_1+C_2
\end{equation}
\begin{equation}
C = C_1 + C_2
\end{equation}

对于$n$个电容器的并联,可采取同样的证明方法,结果有
\begin{equation}
C=\sum_{i=1}^{n}C_i
\end{equation}
或者用数学归纳法证明,我们将此作为习题.


\subsubsection{电容的串联}
如\autoref{Ccomb_fig1} 右边为两电容器的串联.对于串联,两电容器电荷量$Q$相等
\begin{equation}
Q_1=Q_2=Q
\end{equation}

由电容的定义\autoref{Cpctor_eq2}~\upref{Cpctor}知道
\begin{equation}
\begin{aligned}
U_1=\frac{Q}{C_1}\\
U_2=\frac{Q}{C_2}
\end{aligned}
\end{equation}
所以,两电容器串联后总电容为
\begin{equation}
{C} = \frac{Q}{U_1+U_2} = \frac{Q}{\frac{Q}{C_1}+\frac{Q}{C_2}} = \frac{C_1C_2}{C_1 + C_2}
\end{equation}
或
\begin{equation}
\frac{1}{C}=\frac{1}{C_1}+\frac{1}{C_2}
\end{equation}

与电容的并联类似,可证明对于$n$个电容器的串联,有
\begin{equation}
\frac{1}{C}=\sum_{i=1}^{n}\frac{1}{C_i}
\end{equation}
\begin{exercise}{}
用数学归纳法分别证明$n$个电容器的串联和并联公式
\end{exercise}
%推导类比弹簧的串联和并联\upref{Spring} 以及电阻的串联和并联\upref{Rcomb}.
