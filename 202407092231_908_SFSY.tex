% 双缝实验
% license CCBYSA3
% type Wiki

(本文根据 CC-BY-SA 协议转载自原搜狗科学百科对英文维基百科的翻译)

双缝实验(double-slit experiment,或称双狭缝实验)是一种演示光子或电子等等微观物体的波动性与粒子性的实验。

在现代物理学中,双缝实验证明光和物质可以显示经典的波和粒子的特性;此外,它显示了量子力学现象的基本概率性质。这个实验是托马斯·杨在1801年首次用光进行的。1927年,戴维孙和格默证明了电子也有相同的性质,这种性质后来扩展到原子和分子。早在量子力学和波粒二象性概念出现之前,托马斯·杨的光实验就是经典物理学的一部分。他认为这证明了光波理论是正确的,他的实验有时被称为杨氏实验[1] 或杨氏狭缝。

该实验属于一类普通的“双路径”实验,其中一个波被分成两个独立的波,然后合并成一个波。两种波的路径长度的变化会导致相移,从而产生干涉图样。另一个版本是马赫-曾德尔干涉仪,它用镜子分割光束。

在这个实验的基本版本中,相干光源,例如激光束,照亮有两个平行狭缝的平板,穿过狭缝的光在平板后面的屏幕上被观察到。[2] 光的波动性质导致穿过两个狭缝的光波发生干涉,在屏幕上产生亮带和暗带——如果光是由经典粒子组成的话,这种结果是不可能出现的。[2][3] 然而,人们总是发现光在屏幕上的离散点被吸收,作为单个粒子(而不是波)打在屏幕上,通过粒子密度的变化会显示出干涉图案。[4] 此外,包括狭缝处探测器的实验版本发现,每个探测到的光子穿过一个狭缝(就像经典粒子一样),而不是穿过两个狭缝(就像波一样)。[5][6][7][8][9] 然而,这些实验证明,如果检测到粒子穿过哪个狭缝,它们就不会形成干涉图样。这些结果证明了波粒二象性原理。[10][11]

当向双缝发射时,发现其他原子级实体,如电子,表现出相同的行为。 此外,个别离散撞击的探测被观察到具有内禀的概率性,这用经典力学是无法解释的。

这个实验可以用比电子和光子大得多的实体来完成,尽管随着尺寸的增加会变得更加困难。进行双缝实验的最大实体是每个包含810个原子的分子(其总质量超过10,000个原子质量单位)。

双缝实验(及其变体)已成为经典的思想实验,因为它清晰地表达了量子力学的核心难题。因为它证明了观察者预测实验结果能力的根本局限性,理查德·费曼称之为“一种无法用任何经典方式解释的现象,这种现象蕴含着量子力学的核心。事实上,它包含了量子力学中唯一的神秘。

\subsection{概观}
\begin{figure}[ht]
\centering
\includegraphics[width=8cm]{./figures/423775cb916e2722.png}
\caption{相同的双缝配置(缝间0.7 mm);在上图中,一条狭缝是闭合的。在单缝图像中,由于狭缝的非零宽度,形成了衍射图样(主带两侧的微弱点)。在双缝图像中也可以看到衍射图样,但其强度是单缝图像的两倍,并增加了许多较小的干涉条纹。} \label{fig_SFSY_1}
\end{figure}
如果光严格地由普通粒子或经典粒子组成,这些粒子通过狭缝以直线发射,并被允许照射到另一侧的屏幕上,我们将会看到与狭缝的大小和形状相对应的图案。然而,当实际进行这个“单缝实验”时,屏幕上的图案是光扩散的衍射图案。狭缝越小,扩散角度越大。图像的顶部显示了当红色激光照射狭缝时形成的图案的中心部分,如果仔细观察,还有两个微弱的边带。使用更精细的仪器可以看到更多的波段。这可以用衍射来解释,即图案是光波从狭缝干涉的结果。
\begin{figure}[ht]
\centering
\includegraphics[width=10cm]{./figures/0726aeb0a385322b.png}
\caption{请添加图片标题} \label{fig_SFSY_2}
\end{figure}
