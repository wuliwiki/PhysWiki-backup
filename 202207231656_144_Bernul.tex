% 伯努利方程
% 流体|密度|压强|功率|横截面

% \begin{issues}
% \issueDraft
% \end{issues}

不可压缩流体的方程.\footnote{参考 Wikipedia \href{https://en.wikipedia.org/wiki/Bernoulli-principle}{相关页面}.}

\begin{theorem}{伯努利方程}
假设液体不可被压缩、没有粘滞性、与管壁也没有摩擦阻力,那么处处有
\begin{equation}
\frac{v^2}{2} + gz + \frac{p}{\rho} = \text{常数}
\end{equation}
其中 $g$ 是重力加速度, $z$ 是高度, $p$ 是液体的压强, $\rho$ 是液体的密度
\end{theorem}
可以根据伯努利原理设计液体测速计等设备.

\subsubsection{推导}
\footnote{推导过程还参考了安宇教授等的《大学物理》课程}
\begin{figure}[ht]
\centering
\includegraphics[width=12cm]{./figures/Bernul_1.png}
\caption{那根管子} \label{Bernul_fig1}
\end{figure}
如图, 一根管子的粗细不同两部分的横截面面积分别为 $A_1, A_2$, 压强分别为 $p_1, p_2$, 高度分别为$h_1, h_2$;其中流过的液体密度为$\rho$,在两处的速度分别为 $v_1, v_2$.

假设Δt时间内,1处有质量为m1液体进入水管,其流速为v1;同时2处有m2液体流出水管,其流速为v2.

根据不可压缩的假设,流入水管的水量=流出水管的水量,$A_1v_1\Delta t=A_2v_2\Delta t=V$,即$m1=m2$

两部分液体的机械能:
\begin{equation}
E_1=\frac{1}{2}mv_1^2+mgh_1
\end{equation}
\begin{equation}
E_2=\frac{1}{2}mv_2^2+mgh_2
\end{equation}

那么机械能差值:
\begin{equation}
\Delta E = E_2 - E_1 = \frac{1}{2}m(v_2^2-v_1^2)+mg(h_2-h_1)
\end{equation}

再考虑液体压力的做功.1处液体流入管道时,液体压力对液体做正功;而2处液体流出管道时,液体压力对液体做负功.
\begin{equation}
W_1=p_1v_1A_1\Delta t=p_1V 
\end{equation}
\begin{equation}
W_2=-p_2v_2A_2 \Delta t=-p_2V 
\end{equation}
总功
\begin{equation}
W=(p_1-p_2)V
\end{equation}
根据机械能定理 $W=\Delta E$,
\begin{equation}
(p_1-p_2)V= \frac{1}{2}m(v_2^2-v_1^2)+mg(h_2-h_1)
\end{equation}
得
\begin{equation}
p_1V+\frac{1}{2}mv_1^2+mgh_1=p_2V+\frac{1}{2}mv_2^2+mgh_2
\end{equation}
事实上1,2的位置可以任意选取,因此有
\begin{equation}
pV+\frac{1}{2}mv^2+mgh=\text{常数}
\end{equation}
两边同除以V,
\begin{equation}
p+\frac{1}{2}\rho v^2+\rho gh=\text{常数}
\end{equation}
还可以再同除以$\rho$,
\begin{equation}
\frac{p}{\rho} + \frac{v^2}{2} + gh = \text{常数}
\end{equation}

% 穿过横截面 $i = 1, 2$ 的功率分别为
% \begin{equation}
% P_i = p_i A_i v_i + \frac{1}{2} (\rho v_i A_i) v_i^2 + \rho v_i A_i g z_i
% \end{equation}
% 而且我们要求
% \begin{equation}\label{Bernul_eq1}
% P_1 = P_2
% \end{equation}

% 另外由于液体不可压缩, 有
% \begin{equation}
% A_1v_1 = A_2 v_2
% \end{equation}
% 所以\autoref{Bernul_eq1} 两边同时除以 $\rho v_i A_i$ 得
% \begin{equation}
% \frac{v_1^2}{2} + gz_1 + \frac{p_1}{\rho} = \frac{v_2^2}{2} + gz_2 + \frac{p_2}{\rho}
% \end{equation}

\addTODO{这样的推导如何拓展到开放空间的情况呢? 举例: 水龙头下的乒乓球, 香蕉球, 机翼, 两张纸中间吹气}
