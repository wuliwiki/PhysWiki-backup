% 牛顿第二定律的矢量形式

\pentry{高中牛顿第二定律}

我们高中最熟悉的是一维直线运动的牛顿第二定律, 也就是常见的标量形式 $F = ma$. 高中物理已经明确过力和加速度都是\textbf{矢量}, 本书中矢量用黑体和正体表示, 即 $\bvec F$ 和 $\bvec a$. 我们知道, 矢量既有长度也有方向, 但是在一维情况下, 矢量只有两个方向, 我们把其中一个定义为正方向, 那么另一个就是反方向. 

物理中有一个简单的约定就是, 一维的矢量可以用\textbf{标量}也就是一个实数表示, 对于正方向的矢量, 就使用正数表示, 反方向的矢量就使用负数. 所以\textbf{在一维情况下, 矢量于标量可以一一对应}.

在更高维空间的运动中, 也就是二维平面和三维空间中, 牛顿第二定律适用于任何曲线运动, 质点\textbf{速度矢量}的方向和大小都可以随时间任意变化. 牛顿第二定律的矢量形式是
\begin{equation}
\bvec F = m\bvec a
\end{equation}
它比标量形式所包含的意义要丰富得多. 由该式可知 $\bvec F$ 和 $\bvec a$ 的方向时刻相同, 长度相差一个比例系数 $m$.

一个典型的曲线运动例子就是在匀速圆周运动中, 质点和圆心之间的距离却没有改变, 但却存在向心加速度. 根据牛顿第二定律, 做该运动的质点必须要受一个向心力才能保持匀速圆周运动. 相信在高中一些课堂中老师会用所谓的 “微元法” 推导一次向心加速度, 但大部分学生过一段时间以后仅仅只是记得向心加速度的公式 $a = v^2/r = \omega r^2$, 却已经忘记这个加速度从何而来, 为什么会出现.

乍看之下, 圆周运动半径不变, 连向心速度都恒为零, 何来加速度? 问题的关键就在于如何定义一般的矢量加速度 $\bvec a$. 我们先来看\textbf{直线运动}中 “标量版加速度” 的定义
\begin{equation}\label{New2_eq1}
a(t) = \lim_{\Delta t\to 0}\frac{v(t+\Delta t) - v(t)}{\Delta t}
\end{equation}
这里的 $\lim_{\Delta t\to 0}$ 表示 $\Delta t$ 需趋近于 0 使后面的表达式取得极限. 该式的意义是在一段微小时间内, 速度随时间变化的快慢. 注意该式仅限于直线运动, 所以一旦速度方向发生改变, 那么这个定义将立即不适用. 千万不要误以为在曲线运动时, 该式中 $a$ 代表 $\bvec a$ 的大小, $v$ 代表 $\bvec v$ 的大小——这是非常严重的错误. 例如在匀速圆周运动中, 若这么认为, 那么由于速度大小不变, 将得到加速度为零的结论. \textbf{矢量加速度不是速度大小的变化率, 而是速度矢量的变化率}. 后者的定义是:
\begin{equation}\label{New2_eq2}
\bvec a(t) = \lim_{\Delta t\to 0}\frac{\bvec v(t+\Delta t) - \bvec v(t)}{\Delta t}
\end{equation}
该式和\autoref{New2_eq1} 中最关键的不同就是, 当两个速度相减时, 使用的是\textbf{矢量相减}而不是模长相减. 矢量的减法需要用三角形法则, 这在高中物理中经常出现, 简单来说就是先把两矢量起点放在一起, 再从第二个矢量的尖端指向第一个矢量的尖端, 详见 “几何矢量的运算\upref{GVecOp}”.
\addTODO{把三角形的图搬运过来}
只有在直线运动的特殊情况下, 矢量相减才简化为标量相减, 才能使用\autoref{New2_eq1} 的定义.

为什么\autoref{New2_eq2} 定义的加速度才能满足牛顿第二定律? 因为牛顿第二定律就是这么规定的, 牛顿第二定律属于牛顿力学框架的基本假设, 它的成立不需要原因.

明确了矢量加速度的定义后, 我们重新来看圆周运动的向心加速度

\addTODO{推导, 直接搬运 “圆周运动的加速度\upref{CMAD}”: 在一个时刻, 速度矢量是这样, 过了 $\Delta t$ 以后, 是这样. 注意数学上几何矢量的两个要素只有长度和方向而没有起点, 所以把它们任意平移都还是同一矢量. 把速度的起点画在质点上只是为了形象而不是为了强调起点不同. 在相减时, 按照三角形法则我们必须要先把它们移动到同一起点再连接两个终点, 而不是在移动前直接连接.}


% \begin{equation}
% \bvec p(t) = m\bvec v(t) = m\int_{t=t_0, \bvec v(t_0) = 0}^t \bvec a(t') \dd{t'} = \int_{t=t_0, \bvec v(t_0) = 0}^t \bvec F(t') \dd{t'}
% \end{equation}
