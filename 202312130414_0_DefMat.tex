% 正定矩阵(复数)
% license Xiao
% type Tutor

% 如果做出任何修改,需要考虑对 DefMaR 也做同样修改

\pentry{厄米矩阵的本征问题\upref{HerEig},正定矩阵(实数)\upref{DefMaR}}

\footnote{参考 Wikipedia \href{https://en.wikipedia.org/wiki/Definite_matrix}{相关页面}。}
本文把 “正定矩阵(实数)\upref{DefMaR}” 拓展到复数矩阵。 在看复数的正定矩阵的定义前,我们先看另一个定理以帮助理解

\begin{theorem}{}
对任意厄米矩阵\upref{HerMat} $\mat A$ 以及任意复数列向量 $\bvec v$, $\bvec v\Her \mat A \bvec v$ 必为实数。
\end{theorem}
\textbf{证明:} 把 $\bvec v\Her \mat A \bvec v$ 看作 $1\times 1$ 的矩阵,需要证明 $(\bvec v\Her \mat A \bvec v)\Her = \bvec v\Her \mat A \bvec v$ 根据\autoref{eq_HerMat_2}~\upref{HerMat} 有 $(\bvec v\Her \mat A \bvec v)\Her = \bvec v\Her \mat A\Her \bvec v$,而厄米矩阵满足 $\mat A\Her=\mat A$, 得证。

\textbf{正定矩阵(positive definite matrix)}定义如下。
\begin{definition}{}
若一个厄米矩阵 $\mat A$, 对任意非零复数列向量 $\bvec v$ 都满足
\begin{equation}\label{eq_DefMat_1}
\bvec v\Her \mat A\bvec v > 0~,
\end{equation}
那么它就是\textbf{正定矩阵}。

类似地, 也可以定义\textbf{半正定矩阵}(把\autoref{eq_DefMat_1} 中 $>$ 替换为 $\geqslant$), \textbf{负定矩阵}(把 $>$ 替换为 $<$), \textbf{半负定矩阵}。
\end{definition}
其中 $\bvec v\Her$ 表示 $\bvec v$ 的厄米共轭\upref{HerMat}(即先转置\upref{Mat} 再把每个矩阵元取复共轭)。

当 $\mat A$ 是对称矩阵时, 它对应一个对称 2-线性函数, $q(v) = \bvec v\Tr \mat A \bvec v$ 是对应的二次型\upref{QuaFor}。 当 $\mat A$ 为厄米矩阵时, 则对应一个对称的半双线性形式\upref{sequil}。

\begin{theorem}{}
一个矩阵 $\mat A$ 是正定矩阵当且仅当其本征值都大于零。 半正定矩阵和(半)负定矩阵的定义也类似。
\end{theorem}

证明: 令 $\mat A$ 的本征矢为 $\{\uvec u_i\}$ (一组正交归一基底), 对应本征值为 $\lambda_i$(实数), 令非零矢量为 $\bvec v = \sum_i c_i \uvec u_i$ ($c_i$ 不全为零)。 那么
\begin{equation}
\bvec v\Her \mat A \bvec v = \sum_i \lambda_i \abs{c_i}^2~,
\end{equation}
可见若所有 $\lambda_i > 0$, 结果必然是正的。 若要求对任意不全为零的 $c_1,c_2,\dots$ 等式都大于零, 那么也能反推出所有 $\lambda_i > 0$。

\begin{theorem}{}
正定(负定)矩阵都是满秩\upref{MatRnk}的。
\end{theorem}
证明: 对于非零 $\bvec x$, $\bvec x\Her \mat A \bvec x \ne \mat 0$, 说明 $\mat A\bvec x \ne \bvec 0$, 所以齐次方程组 $\mat A\bvec x = \bvec 0$ 唯一的解就是 $\bvec x = \bvec 0$, 所以\upref{LinEq}矩阵是满秩的, 证毕。

\begin{example}{}
求二维厄米矩阵
\begin{equation}
H = \pmat{a & b\\ b^* & d}~
\end{equation}
正定的充分必要条件。

用特征多项式直接求本征值
\begin{equation}
(\lambda - a)(\lambda - d) - \abs{b}^2 = 0~.
\end{equation}
$\lambda$ 必定有解, 利用求根公式, 两个解大于零的充要条件是
\begin{equation}
ad > \abs{b}^2, \qquad
a > 0~.
\end{equation}
\end{example}
