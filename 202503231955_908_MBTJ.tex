% 麦克斯韦-玻尔兹曼统计(综述)
% license CCBYNCSA3
% type Wiki

本文根据 CC-BY-SA 协议转载翻译自维基百科\href{https://en.wikipedia.org/wiki/Maxwell\%E2\%80\%93Boltzmann_statistics}{相关文章}。

在统计力学中,麦克斯韦–玻尔兹曼统计描述了经典物质粒子在热平衡中不同能级上的分布。当温度足够高或粒子密度足够低,以至于量子效应可以忽略不计时,这一统计方法适用。

对于麦克斯韦–玻尔兹曼统计,能量为\(\varepsilon_i\)的粒子的期望数为:
\[
\langle N_i \rangle = \frac{g_i}{e^{(\varepsilon_i - \mu) / kT}} = \frac{N}{Z} g_i e^{-\varepsilon_i / kT}~
\]
其中:
\begin{itemize}
\item \(\varepsilon_i\)是第\( i \)个能级的能量,
\item \( \langle N_i \rangle \)是能量为\( \varepsilon_i \)的状态集合中粒子的平均数,
\item \( g_i \)是第\( i \)个能级的简并度,即具有能量 \( \varepsilon_i \)的状态的数量,这些状态可以通过其他方式加以区分[注1],
\item \( \mu \)是化学势,
\item \( k \)是玻尔兹曼常数,
\item \( T \)是绝对温度,
\item \( N \)是总粒子数:
  \[
  N = \sum_i N_i~
  \]
v\( Z \) 是配分函数:
  \[
  Z = \sum_i g_i e^{-\varepsilon_i / kT}~
  \]
\item \( e \) 是欧拉数。
\end{itemize}