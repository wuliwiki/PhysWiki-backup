% 电磁场张量
% keys 场张量|相对论|电磁场|张量变换|麦克斯韦张量|法拉第张量|麦克斯韦双矢量


\pentry{张量的分类\upref{CatTns}, 电动力学, 闵可夫斯基空间\upref{MinSpa}}
我们继续使用自然单位制,令 $\mu_0=\epsilon_0=c=1$ 来简化表达.依照习惯,上下标使用希腊字母如 $\mu, \nu$ 时,取值范围为 $\{0, 1, 2, 3\}$;使用拉丁字母如 $i, j$ 时,取值范围为 $\{1, 2, 3\}$.

一个参考系中的电磁场需要用六个实函数数来刻画,三个用来刻画电场,三个用来刻画磁场.六个实数太过复杂,我们希望寻求一种简单的方式来简化表达.把六个实数合成一个对象的方法,最直接的当然是使用一个六维向量——不过这样并不能带来实质上的简化.实践中我们使用的其实是一个反对称张量场,用它来表示电磁场.

在狭义相对论里,时空是一个线性空间,事件的时空坐标随着基的不同而不同,而不同的基就代表不同的观察者,事件的坐标分量就是观察者的测量值.

和向量一样,任何张量只有给定了空间的基,才有“坐标分量”的概念.换句话说,只有有了观察者,才有观察者的测量值.张量本身不随基的选择而改变,改变的只是坐标分量.对于电磁场张量来说,其坐标分量,或称观察者的测量值,就是电场强度和磁场强度的空间分量,一共六个实函数.

%电磁场张量的推导暂缺.建议用常规推导,同时考虑编写外导数推导的相关词条后引用.
\subsection{电磁场张量的定义}
\begin{definition}{电磁场张量}
$F^{\mu\nu}=\partial^{\mu}A^{\nu}-\partial^{\nu}A^{\mu}$,其中 $\partial^{\mu}=(-\frac{\partial}{\partial t},\frac{\partial}{\partial x},\frac{\partial}{\partial y},\frac{\partial}{\partial z})$,$A^{\mu}=(\phi,A_x,A_y,A_z)$,其中 $\phi,\bvec A$ 满足洛伦兹规范\upref{LoGaug} $\frac{\partial \phi}{\partial t}+\nabla \cdot \bvec A=0$\footnote{在 SI 单位制下,洛伦兹规范应当写为 $\frac{1}{c^2}\frac{\partial \phi}{\partial t}+\nabla \cdot \bvec A=0$.而这里用的是自然单位制,将 $1/c^2$ 略去.},洛伦兹规范在这里可以简写为 $\partial_\mu A^{\mu}=0$.
\end{definition}
从上面的定义式可以看出,电磁场张量 $F^{\mu\nu}$ 是个反对称张量\footnote{“反对称”意味着 $F^{\mu\nu}=-F^{\nu\mu}$.},且具有洛伦兹协变性,即满足
\begin{equation}\label{EMFT_eq4}
F'_{\mu\nu}=L_\mu^\alpha L_\nu^\beta F_{\alpha\beta}
\end{equation}

电磁场张量是一个二阶张量,为了方便可以写成矩阵形式(假定 $F^{\mu\nu}$ 的第一个指标为行数,第二个指标为列数).矩阵的元素是关于 $\phi$ 和 $\bvec A$ 的表达式,所以它反映了电磁场的性质,进一步地\autoref{EMFT_eq4} 反映了电磁场的变化规律.

根据 $\phi,\bvec A$ 的定义,我们有
\begin{equation}
\begin{aligned}
&\bvec E=-\nabla \phi+\frac{ \partial \bvec A}{\partial t}
\\
&\bvec B=\nabla\times \bvec A
\end{aligned}
\end{equation}
所以对任意的 $i=0,1,2$
\begin{equation}
F_{0,i}=
\end{equation}

下面的定义描述了电磁场张量的坐标分量与坐标系对应的电磁场分量的联系.

\begin{definition}{电磁场张量}
一个伪黎曼流形上的电磁场是一个二阶\textbf{反对称}张量 $F^{\mu\nu}$.若在某基下其分量为 $F^{01}=E_x, F^{02}=E_y, F^{03}=E_z, F^{23}=B_x, F^{31}=B_y, F^{12}=B_z$,那么在这个基对应的观察者所观察到的电场就是 $\pmat{E_x, E_y, E_z}\Tr$,磁场就是 $\pmat{B_x, B_y, B_z}\Tr$.
电磁场张量可以写成以下矩阵形式:
\begin{equation}\label{EMFT_eq1}
F^{\mu\nu}=\pmat{
&0, &E_x, &E_y, &E_z\\
&-E_x, &0, &B_z, &-B_y\\
&-E_y, &-B_z, &0, &B_x\\
&-E_z, &B_y, &-B_x, &0
}_{\mu\nu}
\end{equation}
\end{definition}
即假定 $F^{\mu\nu}$ 的第一个指标为行数,第二个指标为列数\footnote{为了方便,我们常借用二阶矩阵来表示电磁张量,但是要记住,这是一个不规范的表达.}.

用反对称张量来表示电磁场的方式不止一种,还可以使用以下定义的对偶张量.

\begin{definition}{电磁张量的对偶张量}
电磁场张量 $F^{\mu\nu}$ 的\textbf{对偶张量},记为 $G^{\mu\nu}$.如果 $F^{\mu\nu}$ 在某基下的分量为 $F^{01}=E_x, F^{02}=E_y, F^{03}=E_z, F^{23}=B_x, F^{31}=B_y, F^{12}=B_z$,那么有 $G^{01}=-B_x, G^{02}=-B_y, G^{03}=-B_z, G^{23}=E_x, G^{31}=E_y, G^{12}=E_z$.
对偶张量有一个简单的记法:
\begin{equation}
G_{\mu\nu}=\epsilon_{\mu\nu\rho\sigma}F^{\rho\sigma}
\end{equation}
其中 $\epsilon_{\mu\nu\rho\sigma}$ 是一个四维反对称张量,$(\mu\nu\rho\sigma)$ 为偶排列的时候 $\epsilon$ 取 $1$,如果是奇排列则 $\epsilon$ 取 $-1$,其他情况(有重复指标的时候) $\epsilon$ 取 $0$.
\end{definition}

对偶张量,用类似\autoref{EMFT_eq1} 的方式写出来,就是

\begin{equation}
\begin{aligned}
G^{\mu\nu}=\eta^{\mu\mu'}\eta^{\nu\nu'}G_{\mu'\nu'}=
\pmat{
&0, &-B_x, &-B_y, &-B_z\\ 
&B_x, &0, &E_z, &-E_y\\ 
&B_y, &-E_z, &0, &E_x\\
&B_z, &E_y, &-E_x, &0
}_{\mu\nu}
\end{aligned}
\end{equation}

\subsection{电磁场张量的参考系变换}


\subsection{闵可夫斯基时空中的电动力学}

\subsubsection{场源的描述}

\begin{definition}{电流密度4-矢量}
在闵可夫斯基空间中,电流密度被表示为一个4-矢量 $J^\mu$.如果在某观察者看来,电荷密度的分布是 $\rho$,而电流密度的分布是 $\pmat{J_x, J_y, J_z}\Tr$,那么在这个观察者的坐标系下,电流密度4-矢量的坐标就是 $\pmat{\rho, J_x, J_y, J_z}\Tr$.
\end{definition}

由电流密度的定义,我们知道
\begin{equation}
\nabla\cdot\bvec{J}=-\frac{\partial\rho}{\partial t}
\end{equation}
即
\begin{equation}
\frac{\partial}{\partial x^i}J^i+\frac{\partial}{\partial t}\rho=0
\end{equation}
故得
\begin{equation}
\frac{\partial}{\partial x^\mu}J^\mu=0
\end{equation}

这就是电流密度的定义对电流密度4-矢量的限制条件.


\subsubsection{四维形式的麦克斯韦方程组}

利用四维形式的电磁场张量和电流密度4-矢量,我们可以将麦克斯韦方程组表示为以下两个方程
\begin{equation}\label{EMFT_eq2}
\leftgroup{
    &\frac{\partial}{\partial x^\nu}F^{\mu\nu}=J^\mu\\
    &\frac{\partial}{\partial x^\nu}G^{\mu\nu}=0
}
\end{equation}

注意\autoref{EMFT_eq2} 中的真指标是 $\mu$,赝指标是 $\nu$.

\autoref{EMFT_eq2} 的第一个方程是一个四维矢量方程,可以拆成一个标量方程和一个三维矢量方程,其中标量方程就是 $\nabla\cdot\bvec{E}=\rho$,而矢量方程就是 $\nabla\times\bvec{B}=\bvec{J}+\partial\bvec{E}/\partial t$. 

类似地,\autoref{EMFT_eq2} 的第二个方程分别代表了 $\nabla\cdot\bvec{B}=0$ 和 $\nabla\times\bvec{E}=-\partial\bvec{B}/\partial t$.

\subsubsection{电荷受力}

对于一个带电荷 $q$ 的粒子,如果它在某参考系下的4-速度为 $U^\mu$,那么在电磁场 $F^{\mu\nu}$ 中,该电荷受到的闵可夫斯基力为
\begin{equation}\label{EMFT_eq3}
K^{\mu}=qU_\mu F^{\mu\nu}
\end{equation}
注意,其中 $U_\mu=U^ag_{a\mu}$,而 $g_{ij}$ 是闵可夫斯基度规.换句话说,如果粒子在该参考系下的3-速度为 $\pmat{v_x, v_y, v_z}\Tr$,那么有 $U_\mu=\gamma\pmat{1, -v_x, -v_y, -v_z}$.

同样,我们把\autoref{EMFT_eq3} 分为一个标量方程和一个三维向量方程.

标量方程展开来就是
\begin{equation}
\begin{aligned}
&\frac{\dd}{\dd\tau}(\gamma m)=\gamma \bvec{v}\cdot q\bvec{E}
\end{aligned}
\end{equation}
它对应经典物理中的\textbf{功能原理}:$\dd E/\dd t=\bvec{v}\cdot\bvec{F}$.

向量方程展开则得到的是
\begin{equation}
\begin{aligned}
&\bvec{K}=q\gamma(\bvec{E}+\bvec{v}\times\bvec{B})\\
\Rightarrow &\bvec{F}=q(\bvec{e}+\bvec{v}\times\bvec{B})
\end{aligned}
\end{equation}
它对应经典物理中的\textbf{动量原理},其中力是洛伦兹力.






