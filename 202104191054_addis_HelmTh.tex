% 亥姆霍兹分解
% 散度|旋度|调和场|亥姆霍兹

\begin{issues}
\issueTODO
\end{issues}

\pentry{散度的逆运算\upref{DivInv}, 旋度的逆运算\upref{HlmPr2}, 调和场\upref{HarmF}}

\footnote{参考 \cite{GriffE} 相关章节.}任何处处可导的矢量场都可以分解为一个无旋场 $\bvec F_{d}(\bvec r)$, 一个无散场 $\bvec F_{c}(\bvec r)$, 和一个调和场\upref{HarmF} $\bvec H(\bvec r)$.
\begin{equation}\label{HelmTh_eq3}
\bvec F(\bvec r) = \bvec F_{d}(\bvec r) + \bvec F_{c}(\bvec r) + \bvec H(\bvec r)
\end{equation}
一种分解的计算方法是, 令
\begin{equation}
\rho(\bvec r) \equiv \div \bvec F(\bvec r) \qquad
\bvec j(\bvec r) \equiv \curl \bvec F(\bvec r)
\end{equation}
\begin{equation}\label{HelmTh_eq1}
\bvec F_{d}(\bvec r) \equiv \frac{1}{4\pi}\int \frac{\rho(\bvec r') \bvec R}{R^3} \dd{V'}
\end{equation}
\begin{equation}\label{HelmTh_eq2}
\bvec F_{c}(\bvec r) \equiv \frac{1}{4\pi}\int \frac{\bvec j(\bvec r')\cross \bvec R}{R^3} \dd{V'}
\end{equation}
\begin{equation}\label{HelmTh_eq4}
\bvec H(\bvec r) \equiv \bvec F(\bvec r) - \bvec F_{d}(\bvec r) - \bvec F_{c}(\bvec r)
\end{equation}
则\autoref{HelmTh_eq3} 成立. 其中 $\bvec r, \bvec r'$ 分别是坐标原点指向三维直角坐标 $(x, y, z)$ 和 $(x', y', z')$ 的位置矢量, $\bvec R = \bvec r' - \bvec r$, $R = \abs{\bvec R}$, 体积分 $\int\dd{V'} = \int\dd{x'}\dd{y'}\dd{z'}$ 的区域是整个三维空间或者其中 $\bvec F$ 不为零的区域, $\cross$ 表示矢量叉乘\upref{Cross}.

注意\autoref{HelmTh_eq3} 的分解不是唯一的, 我们也可以给\autoref{HelmTh_eq1} 和\autoref{HelmTh_eq2} 右边分别加上一个任意调和场, 分解依然成立. 可以证明如此定义的 $\bvec F_{d}$, $\bvec F_{c}$ 和 $\bvec H$ 分别是无旋场,无散场和调和场.

特殊地(例如在电磁学中), 如果要求 $\bvec F, \bvec F_d, \bvec F_c$ 在无穷远处都趋于零, 这要求调和场 $\bvec H(\bvec r)$ 恒为零(\autoref{HarmF_the1}~\upref{HarmF}), 且\autoref{HelmTh_eq1} 和\autoref{HelmTh_eq2} 右边不能加上非零调和场, 所以该条件下亥姆霍兹分解是唯一的.

\autoref{HelmTh_eq1} 和\autoref{HelmTh_eq2} 可以分别类比静电学中的库伦定律(\autoref{Efield_eq9}~\upref{Efield})和静磁学中的比奥萨法尔定律(\autoref{BioSav_eq3}~\upref{BioSav}): 把 $\bvec F_c$ 看作静电场, $\rho(\bvec r)$ 看作电荷密度, 满足高斯定律(\autoref{EGauss_eq1}~\upref{EGauss}); 把 $\bvec F_c$ 看作静磁场, $\bvec j(\bvec r)$ 看作电流密度, 满足安培环路定理(\autoref{AmpLaw_eq2}~\upref{AmpLaw}).

\subsubsection{证明}
根据\autoref{DivInv_the1}~\upref{DivInv}以及\autoref{HlmPr2_the2}~\upref{HlmPr2}, \autoref{HelmTh_eq1} 和\autoref{HelmTh_eq2} 中定义的 $\bvec F_{d}(\bvec r)$ 和 $\bvec F_{c}(\bvec r)$ 分别是无旋场和无散场, 且满足
\begin{equation}
\div \bvec F_d(\bvec r) = \rho(\bvec r) = \div \bvec F(\bvec r)
\end{equation}
\begin{equation}
\curl \bvec F_c(\bvec r) = \bvec j(\bvec r) = \curl \bvec F(\bvec r)
\end{equation}
容易证明 $\bvec H(\bvec r)$ 必定是调和场(无散且无旋):
\begin{equation}
\div \bvec H = \div (\bvec F - \bvec F_d) = 0
\end{equation}
\begin{equation}
\curl \bvec H = \curl (\bvec F - \bvec F_d) = \bvec 0
\end{equation}
证毕.

\subsection{另一种表示}
上述分解采用的是积分的方法, 我们也可以用微分来描述. 无旋场总能表示为某个标量函数 $V(\bvec r)$ 的梯度(证明见势能\upref{V}), 而无散场总能表示为另一个矢量场 $\bvec A$ 的旋度\upref{HlmPr2}, 所以
\begin{equation}
\bvec F_{d} = \grad V\qquad \bvec F_{c} = \curl \bvec A
\end{equation}
所以亥姆霍兹分解也可以记为
\begin{equation}
\bvec F = \grad V + \curl \bvec A + \bvec H
\end{equation}
但是求解 $V(\bvec r)$ 和 $\bvec A$ 并不那么直接, 一般还是需要先用上一节中的办法进行分解才能求得.
