% 字符串的分片与索引
% 字符串|分片|索引

\pentry{Python 基本变量类型\upref{PyType}\upref{Sample}}

\textbf{字符串}可以通过string[x] 的方式进行索引、分片,也就是加一个[](像不像一把刀).字符串的分片(slice)实际上可以看作是从字符串这个大面包\autoref{Strsi_fig1} 中找出来你要吃的那片美味(复制出来一小段你要的长度),放在你的嘴里(储存在另一个地方),而不会对字符串这个源文件改动.分片获得的每个字符串可以看作是原字符串的一个副本(面包片).

\begin{figure}[ht]
\centering
\includegraphics[width=10cm]{./figures/Strsi_1.png}
\caption{切片} \label{Strsi_fig1}
\end{figure}

我们来看一段程序:
\begin{lstlisting}[language=python]
name = 'My name is Mike'
print(name[0])
'M'
print(name[-4])
'M'
print(name[11:14]) # 从第一个到第十四个,第十四个不包括在内
'Mik'
print(name[11:15]) # 从第一个到第十五个,第十五个不包括在内
'Mike'
print(name[5:])  #代表着从编号为5的字符到结束的字符串分片.
'me is Mike'
print(name[:5]) #从编号为0的字符开始到编号为5但不包含第5个字符
'My na'
\end{lstlisting}
\textbf{\textsl{:}}两边分别代表着字符串的分割从哪里开始,并到哪里结束.我们不妨列个表格\autoref{Strsi_tab1} 来说明字符的对应关系
\begin{table}[ht]
\centering
\caption{对应关系}\label{Strsi_tab1}
\begin{tabular}{|c|c|c|c|c|c|c|c|c|c|c|c|c|c|c|c|}
\hline
 字符  & M & y &   & n & a & m & e &   & i & s &   & M & i & k &e\\
\hline
 序号  & 0 & 1 & 2 & 3 & 4 & 5 & 6 & 7 & 8 & 9 & 10 & 11 & 12 & 13 & 14\\
\hline
 反序  & -15 & -14 & -13 & -12 & -11 & -10 & -9 & -8 & -7 & -6 & -5 & -4 & -3 & -2 & -1\\
\hline
\end{tabular}
\end{table}
\begin{example}{”朋友中的魔鬼“}
这个文字小游戏代码如下:
\begin{lstlisting}[language=python]
word = 'friends'
find_the_evil_in_your_friends = word[0]+word[2:4]+word[-3:-1] 
print(find_the_evil_in_your_friends)
\end{lstlisting}
没有问题的话,你会得到结果:
\begin{lstlisting}[language=python]
fiend
\end{lstlisting}
也就发现了朋友中的魔鬼(get到了吗?)
\end{example}