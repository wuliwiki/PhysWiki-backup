% 格林函数与静电边值问题
% 格林函数|静电学|边值问题|泊松方程

\pentry{静电势的泊松方程\upref{EPoiEQ}}

静电学问题中有一类非常常见的边值问题,例如已知接地导体的空腔中有电荷分布,求空腔内的电势;求带电导体在空间中产生的电场等.使用格林定理可以帮助我们高效率地计算边值问题.

设真空中的电荷分布为 $\rho(\bvec r)$,则空间中的静电势满足泊松方程
\begin{equation}
\nabla^2 \phi = -\frac{\rho}{\epsilon_0}
\end{equation}

定义函数 $\psi(r,r')$:
\begin{equation}
\psi(r,r')=\frac{1}{|r-r'|}
\end{equation}
它则满足以下泊松方程
\begin{equation}
\nabla'^2 \psi(r,r')=-4\pi\delta(r-r')
\end{equation}

\begin{theorem}{格林定理}
设 $\phi,\psi$ 为区域 $V$ 上的标量函数,则通常有
\begin{equation}
\int_V (\phi \nabla^2 \psi -\psi\nabla^2\phi)\dd V=\int_{\partial V}(\phi \nabla \psi-\psi\nabla\phi)\dd {\bvec S} 
\end{equation}
\end{theorem}
\textbf{证明:} 由高斯定理,$\int_V \nabla\cdot \bvec F\dd V=\int_{\partial V}\bvec F \dd {\bvec S}$,
\begin{equation}
\begin{aligned}
\int_{\partial V}&(\phi\nabla\psi-\psi\nabla\phi)\dd{\bvec S}=\int_V \nabla \cdot (\phi\nabla\psi-\psi\nabla\phi)\dd V\\
&=\int_V((\nabla\phi)\cdot(\nabla\psi)+\phi\nabla^2\psi-(\nabla\psi)\cdot(\nabla\phi)-\psi\nabla^2\phi)\dd V\\
&=\int_V(\phi\nabla^2\psi-\psi\nabla^2\phi)\dd V
\end{aligned}
\end{equation}

下面我们用格林定理来解决静电边值问题.