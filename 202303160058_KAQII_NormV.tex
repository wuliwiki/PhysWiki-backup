% 范数、赋范空间
% 线性代数|赋范空间|线性空间|三角不等式

\begin{issues}
\issueOther{应当移动到线性代数,至少把巴拿赫空间以外的部分移走}
\end{issues}


\pentry{矢量空间\upref{LSpace}, 度量空间\upref{Metric}}

\textbf{范数(norm)}可以看作几何矢量\upref{GVec}的模长在一般矢量空间上的拓展。
\begin{definition}{}\label{NormV_def1}
设 $X$ 是实数或复数域上的矢量空间。 $X$ 上的范数是满足如下条件的非负函数 $\|\cdot\|$:
\begin{enumerate}
\item $\norm{x} \geqslant 0$ (正定)
\item $\norm{x} = 0$ 当且仅当 $x = 0$
\item $\|\lambda x\| = |\lambda|\|x\|$
\item $\|x_1+x_2\| \leqslant \|x_1\|+ \|x_2\|$ (三角不等式)
\end{enumerate}
如果一个矢量空间中定义了范数, 我们就把它称为\textbf{赋范空间(normed space)}。
\end{definition}

赋范线性空间都具有度量
\begin{equation}
d(x,y) := \norm{x-y}
\end{equation}
所以赋范空间都是度量空间\upref{Metric}。 作为度量空间时完备的赋范空间称为\textbf{巴拿赫空间(Banach space)}\upref{banach}。

一个线性空间上可能可以定义许多个范数。 线性空间 $X$ 上的两个范数 $\|\cdot\|_1$ 和 $\|\cdot\|_2$ 称为\textbf{等价 (equivalent)} 的, 如果有正实数 $C>1$ 使得如下不等式对于任何 $x\in X$ 都成立:
$$
C^{-1}\|x\|_{1}\leq\|x\|_2\leq C\|x\|_1.
$$

\textbf{内积空间(inner product space)}\upref{InerPd} 是非常重要的特殊的赋范空间。如果 $H$ 是内积空间,$\langle\cdot,\cdot\rangle\to\mathbb C$ 是其上的内积, 则若命 $\|x\|=\sqrt{\langle x,x\rangle }$, $H$ 便成为一个赋范空间。

\subsection{有限维空间上的范数}
设 $p\geq1$。 定义 $\mathbb R^N$ 或 $\mathbb C^N$ 空间(即 $N$ 维实数或复数列矢量空间) 的 \textbf{$p$-范数}为
\begin{equation}
\norm{ x}_p = \qty(\sum_{i=1}^N \abs{x_i}^p)^{1/p}
\end{equation}
物理中常见的是 \textbf{2-范数}, 也叫\textbf{欧几里得范数(Euclidean norm)} 即
\begin{equation}
\norm{ x}_2 = \sqrt{\abs{x_1}^2 + \abs{x_2}^2 + \dots+|x_N|^2}
\end{equation}
它是由内积
$$
\langle x,y\rangle=\sum_{i=1}^Nx_i\bar y_i
$$
诱导的。

在极限 $p \to \infty$ 之下, 绝对值最大的 $x_i$ 对求和的贡献将远大于其他分量, 所以可定义\textbf{无穷范数(infinity norm)}为
\begin{equation}
\norm{\bvec x}_\infty = \max \qty{\abs{x_i}}.
\end{equation}

除此之外, 有限维实或复线性空间上还可以定义许多种不同的范数。 不过, 有限维实或复线性空间上的任意两个范数必然彼此等价。 它们都给出空间上唯一的一个自然拓扑(即使得所有线性泛函均连续的拓扑)。 在任何范数之下, 有限维实或复线性空间都是巴拿赫空间\upref{banach}。

\subsection{赋范空间中的极限}
既然赋范空间属于度量空间, 那么我们可以延用度量空间中序列极限的定义。 只需要令距离函数 $d(x, y) = \norm{x - y}$ 即可。

在赋范空间中, 极限
\begin{equation}
\lim_{n\to\infty} x_n = x
\end{equation}
的定义是
\begin{equation}
\lim_{n\to\infty} \|{x_n - x}\|= 0
\end{equation}

\subsection{函数的范数}
欧氏空间上的函数可以定义多种范数。 对于函数 $f$, 最常见的范数有极大范数
$$
\|f\|_{L^\infty}:=\sup_{x\in\mathbb{R}^N}|f(x)|;
$$
还有 $L^p$ 范数(这里 $p\geq1$):
$$
\|f\|_{L^p}:=\left(\int_{\mathbb{R}^N}|f(x)|^pdx\right)^{1/p}.
$$
特别地, 量子力学中经常要考虑的范数是 $L^2$ 范数, 它是由内积
$$
\langle f,g\rangle=\int_{\mathbb{R}^N}f(x)\bar g(x)dx
$$
诱导的。 这些定义都可以推广到一般的拓扑空间或测度空间上。

\subsection{赋范空间上的线性算子}
两个赋范线性空间 $(X,\|\cdot\|_X),(Y,\|\cdot\|_Y)$ 之间的线性算子 $T$ 若满足
$$
\|Tx\|_Y\leq C\|x\|_X
$$
则称为是\textbf{有界的(bounded)}; 使得上面不等式成立的 $C$ 的下确界称作算子 $T$ 的\textbf{算子范数 (operator norm)}, 常记为 $\|T\|_{X\to Y}$, 不至于混淆时也可简略记为 $\|T\|$。 赋范线性空间之间的有界线性算子与赋范线性空间之间的连续线性算子是同一个数学对象。 由线性性质,
$$
\|T\|_{X\to Y}=\sup_{\|x\|_X\leq 1}\|Tx\|_Y.
$$
从 $X$ 到 $Y$ 的有界线性算子的集合常记为 $\mathfrak{B}(X,Y)$。 若赋予算子范数, 则它也是一个赋范线性空间。

如果存在一有界线性算子 $T:X\to Y$, 使得 $T$ 是双射, 且其逆映射也是有界的, 则称赋范线性空间 $X,Y$ 是\textbf{同构的(isomorphic)}, 算子 $T$ 称为两者之间的同构映射(isomorphism)。 此时 $T$ 既是线性空间范畴下的可逆算子, 也是两个度量空间之间的同胚。 因此, 任何拓扑性质, 例如完备性, 可分性等等, 在同构映射下都不变。 

如果有 $\|Tx\|_Y=\|x\|_X$, 则称 $T$ 是赋范线性空间之间的\textbf{等距(isometry)}。 如果既是同构映射也是等距映射, 则称为等距同构(isometric isomorphism), 此时两赋范线性空间则称为等距同构的。 注意, 对于无穷维线性空间来说, 一个空间到自己的单线性映射不一定是满的, 这与有限维线性空间截然不同。 例如在平方可和序列空间 $l^2$ 中, 平移映射
$$
(x_1,x_2,x_3,...)\to(0,x_1,x_2,x_3,...)
$$
是单的, 而且甚至是等距映射, 但它不是满的。

\subsection{赋范空间上的结构}
给定两个赋范线性空间 $(X,\|\cdot\|_X),(Y,\|\cdot\|_Y)$ 之后, 其直积 $X\times Y$ 上可定义一些相互等价的范数而使之成为赋范线性空间, 例如
$$
\|(x,y)\|_{1}=\|x\|_X+\|y\|_Y,\qquad \|(x,y)\|_{\infty }=\max(\|x\|_X,\|y\|_Y).
$$
更一般地, 可定义
$$
\|(x,y)\|_{p}=\left(\|x\|_X^p+\|y\|_Y^p\right)^{1/p},\,1\leq p\leq\infty
$$
它们上述两个范数都等价。

如果 $X$ 是赋范线性空间, 则其子空间也是赋范线性空间。 如果 $M\subset X$ 是的闭子空间, 则商空间 $X/M$ 上可定义范数为
$$
\|x+M\|:=\inf \limits _{m\in M}\|x+m\|.
$$
此时 $X/M$ 也成为赋范线性空间。 赋范线性空间之间的有界线性算子 $T:X\to Y$ 可以被分解为
$$
T=T_{1}\circ \pi ,\ \ \ T:X\ {\overset {\pi }{\longrightarrow }}\ X/\operatorname {Ker} (T)\ {\overset {T_{1}}{\longrightarrow }}\ Y,
$$
其中 $\pi:x\to x+\text{Ker}(T)$ 是商投影, $T_1:x+\text{Ker}(T)\to T(x)$ 是商映射, 它是到像空间 $\text{Ran}(T)$ 的一一映射, 但逆映射却不一定有界。 关于有界算子的像空间的性质, 详见巴拿赫定理\upref{BanThm}。 %链接未完成

对于赋范线性空间 $X$ 的真闭子空间 $M$, 成立\textbf{里斯引理(Riesz's lemma)}: 任给 $\alpha\in(0,1)$, 都存在 $x$ 使得 $\|x\|=1$, 且 $\text{dist}(x,M)\geq\alpha$。 详见\upref{RiLem}。
