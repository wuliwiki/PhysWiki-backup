% 天津大学 2015 年考研量子力学答案
% 考研|天津大学|量子力学|2015|答案

\begin{issues}
\issueDraft
\issueTODO
\end{issues}

\subsection{ }
由归一化条件$\displaystyle (\sqrt{\frac{1}{3}}A)^{2} + (\sqrt{\frac{2}{3}}A)^{2} = 1 $可得到$A=1$.故归一化函数为:\\

$\displaystyle \psi(x) = \sqrt{\frac{1}{3}}\phi_{210}(x)+\sqrt{\frac{2}{3}} \phi_{310}(x)$ \\

\subsection{ }
\begin{enumerate}
\item 由题可得:
\begin{equation}
\begin{aligned}
\left[L^{2},L_{x}\right] =& [ L^{2}_{x} + L^{2}_{y} + L^{2}_{z} , L_{x}] \\
=& 0 + [L^{2}_{y},L_{x}] + [L^{2}_{z},L_{x}] \\
=& 0 - L_y[L_y,L_x]+[L_y,L_x]L_y \\
=& 0 - i\hbar L_y L_z - i\hbar L_z L_y + i\hbar L_z L_y + i\hbar L_y L_z \\
=& 0
\end{aligned}
\end{equation}
同理可得:
\begin{equation}
\begin{aligned}
\left[\hat{L}_+,\hat{L}_z\right]Y_{lm}(\theta ,\phi) =& \left[\hat{L}_{x}+i\hat{L}_{y} ,\hat{L}_z \right]Y_{lm}(\theta ,\phi) \\
=& \left[\hat{L}_x ,\hat{L}_z \right]Y_{lm}(\theta ,\phi) + i\left[\hat{L}_x ,\hat{L}_z \right]Y_{lm}(\theta ,\phi) \\
=& -\hbar \hat{L}_{+}Y_{lm}(\theta ,\phi)
\end{aligned}
\end{equation}

\begin{equation}
\begin{aligned}
\left[\hat{L}_{-},\hat{L}_{z}\right]Y_{lm}(\theta ,\phi) =& \left[\hat{L}_{x}-i\hat{L}_{y} ,\hat{L}_z \right]Y_{lm}(\theta ,\phi) \\
=& \left[\hat{L}_x ,\hat{L}_z \right]Y_{lm}(\theta ,\phi) - i\left[\hat{L}_x ,\hat{L}_z \right]Y_{lm}(\theta ,\phi) \\
=& \hbar \hat{L}_{-}Y_{lm}(\theta ,\phi)
\end{aligned}
\end{equation}

\begin{equation}
\begin{aligned}
\left[\hat{L}_{+},\hat{L}_{-}\right]Y_{lm}(\theta ,\phi) =& \left[\hat{L}_{x} + i\hat{L}_{y} ,\hat{L}_{x}-i\hat{L}_{y} \right]Y_{lm}(\theta ,\phi) \\
=& 2\hbar \hat{L}_{z}Y_{lm}(\theta ,\phi)
\end{aligned}
\end{equation}

\item 答:%经典物理不能解释黑体辐射,光电效应等效应,普朗克提出能量量子化后,解决了经典物理不能解释的黑体辐射;在普朗克的启发下,爱因斯坦引入了光量子的概念,解决了经典物理不能解释的光电效应,并由康普顿效应证实光具有粒子性.
\item 由题可得:
\begin{equation}
\begin{aligned}
e^{i \rho_{y} \partial} =& \sum_{n=0}^{\infty} \frac{(i \rho_y \partial)^{n}}{n!} \\
=& \sum (i\rho_y \partial - \frac{i\rho_y \partial^3}{3!} + \frac{i\rho_y \partial^5}{5!} + \dots) + \sum (1 - \frac{\partial^2}{2!} + \frac{\partial^4}{4!} + \dots) \\
=& i\rho_y \sin{\theta} + \cos{\theta} 
\end{aligned}
\end{equation}
故
\end{enumerate}