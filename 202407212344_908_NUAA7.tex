% 南京航空航天大学 2015 量子真题
% license Usr
% type Note

\textbf{声明}:“该内容来源于网络公开资料,不保证真实性,如有侵权请联系管理员”

\subsection{简答题(20 分,每题 10 分)}
①证明:薛定谔方程中如果 $V(x)$是偶函数,即 $V(-x)=V(x)$,那么波函数 $\psi(x)$总
可以取作偶函数或奇函数。

②经典物理中一个矢量 $\vec r$ 与自身的矢量积(叉乘)恒为零$\vec r \times \vec r\equiv
0$   ,对于量子力学
中矢量算符这一结论仍然普遍成立吗?试举例说明。

\subsection{二}
如果算符 $\hat{P}$ 满足等幂性,即 $\hat{P}^2 = \hat{P}$,那么我们称为 $\hat{P}$ 投影算符,试证明两
投影算符 $\hat{P}_1$ 、 $\hat{P}_1$ 之和 21 ˆˆ PP + 也为为投影算符的重要条件为这两个投影算符对易
[ ] 0ˆ,ˆ 21 =PP 。(20 分)