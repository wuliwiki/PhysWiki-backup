% 有界变差
% keys 实分析|数学分析|黎曼-斯蒂尔杰斯积分
% license Usr
% type Wiki

有界变差(Bounded Variation)函数的总变差是有限的。这一概念是后面学习黎曼-斯蒂尔杰斯积分的关键基础。有界变差是描述实数轴上函数变化幅度的一种度量方式。

考虑一个实值函数$f(x)$,在闭区间$a\leq x\leq b$上定义并有限。我们将这个区间分成一些小区间,形成一个分割
\begin{align} 
\Gamma=\{x_{0},\,x_{1},\,\ldots,\,x_{m}\} ~.
\end{align},也就是$\Gamma$是点$x_{i}$的集合,满足$x_{0}=a$,$x_{m}=b$,且$x_{i-1}<x_{i}$。

对于每个分割$\Gamma$,我们计算一种和$S_{\Gamma}$,表示相邻点$f(x_{i})$和$f(x_{i-1})$的绝对差的总和。也就是,
\begin{align} 
S_{\Gamma}=S_{\Gamma}[f;a,b]=\sum_{i=1}^{m}|f(x_{i})-f(x_{i-1})|~.
\end{align}
函数$f$在$[a,b]$上的变差定义为
\begin{align} 
V=V[f;a,b]=\sup_{\Gamma}S_{\Gamma}~, 
\end{align}
其中$\sup$取遍$[a,b]$的所有分割$\Gamma$。由于$0\leq S_{\Gamma}<+\infty$,我们有$0\leq V\leq+\infty$。如果$V$有限,那么$f$在$[a,b]$上的变差有界;如果$V$是无穷大,那么$f$在$[a,b]$上的变差无界。

下面列举几个简单有界变差函数的例子:

\textbf{例子 1}:假设$f$在$[a,b]$上单调。那么,显然,每个$S_{\tau}$都等于$|f(b)-f(a)|$,因此$V=|f(b)-f(a)|$。

\textbf{例子 2}:假设$f$的图形可以分为有限数量的单调弧段;即假设$[a,b]=\bigcup_{i=1}^{k}[a_{i}a_{i+1}]$,并且$f$在每个$[a_{i}a_{i+1}]$上是单调的。那么$V=\sum_{i=1}^{k}|f(a_{i+1})-f(a_{i})|$。

\textbf{例子 3}:设$f$是狄利克雷函数,定义为$f(x)=1$对于有理数$x$,$f(x)=0$对于无理数$x$。那么,显然,对于任何区间$[a,b]$,$V[a,b]=+\infty$。

\textbf{例子 6}:定义在$[a,b]$上的函数$f$被称为在$[a,b]$上满足Lipschitz条件,如果存在常数$C$使得
\begin{align} 
|f(x)-f(y)|\leq C|x-y|, \
\forall x,y\in[a,b]~.
\end{align}
这样的函数显然是有界变差的,$V[f;a,b]\leq C(b-a)$。例如,如果$f$在$[a,b]$上有连续的导数,那么(根据中值定理)$f$在$[a,b]$上满足Lipschitz条件。

\begin{theorem}{}
\begin{enumerate}
\item 如果$f$在$[a,b]$上的变差有界,那么$f$在$[a,b]$上有界。
\item 设$f$和$g$在$[a,b]$上的变差有界。那么对于任意实常数$c$,$cf$,$f+g$和$fg$在$[a,b]$上的变差也有界。此外,如果存在$\varepsilon>0$使得对于$[a,b]$中的$x$,$|g(x)|\geq\varepsilon$,那么$f/g$在$[a,b]$上的变差也有界。
\end{enumerate}
\end{theorem}

\begin{example}{证明:}

\end{example}
\begin{theorem}{}
\begin{enumerate}
\item 如果$[a^{\prime},b^{\prime}]$是$[a,b]$的子区间,则$V[a^{\prime},b^{\prime}]\leq V[a,b]$;即,随着区间的增加,变差也增加。
\item 
\end{enumerate}
\end{theorem}
\begin{example}{设 $f(x)=x\sin\left(1/x\right)$ 为 $0<x\leq1$ 且 $f(0)=0$。 证明 $f$ 在 $[0,1]$ 上有界且连续,但不是有界变差$V[f;0,1]=+\infty$。}
\end{example}

