% 凸函数(补)
% keys 凸集合;凸函数
% license Usr
% type Tutor

\begin{definition}{凸集}
对于一个集合$C\subseteq\mathbb{R}^n,\forall\boldsymbol{x}_1,\boldsymbol{x}_2,\theta\in[0,1],$满足
$$\theta\boldsymbol{x}_1+(1-\theta)\boldsymbol{x}_2\in C~$$
称$C$是凸集.
\end{definition}
\begin{definition}{凸集}
对于一个集合$C\subseteq\mathbb{R}^n,\forall\boldsymbol{x}_1,\cdot,\boldsymbol{x}_n,\theta_k\in[0,1],k=1,2,\cdots,n$满足
$$\theta_1\boldsymbol{x}_1+\cdots+\theta_n\boldsymbol{x}_n\in C~$$
称$C$是凸集.
\end{definition}
\begin{definition}{凸集}
对于一个集合$C\subseteq\mathbb{R}^n,$存在映射$p:\mathbb{R}^n\to\mathbb{R},p(\boldsymbol{x}),\int_Cp(\boldsymbol{x})\mathrm{d}\boldsymbol{x}=1,\forall\boldsymbol{x}\in C,$满足
$$\int_C\boldsymbol{x}p(\boldsymbol{x})\mathrm{d}\boldsymbol{x}\in C~$$
称$C$是凸集.
\end{definition}
\begin{definition}{凸函数}
存在映射$f:\mathbb{R}^n\to\mathbb{R},\mathrm{dom} f$是凸集合,如果$\forall\boldsymbol{x_1},\boldsymbol{x_2}\in\mathrm{dom} f, \theta\in[0,1],$满足
$$
f(\theta\boldsymbol{x}_1+(1-\theta)\boldsymbol{x}_2)\leqslant\theta f(\boldsymbol{x}_1) + (1-\theta)f(\boldsymbol{x}_2)~
$$
\end{definition}
称映射$f$是定义在$\mathrm{dom} f$上的凸函数.如果满足
$$
f(\theta\boldsymbol{x}_1+(1-\theta)\boldsymbol{x}_2)\leq\theta f(\boldsymbol{x}_1) + (1-\theta)f(\boldsymbol{x}_2)~
$$
称映射$f$是定义在$\mathrm{dom} f$上的强凸函数.