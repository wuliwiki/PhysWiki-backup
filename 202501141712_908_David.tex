% 大卫·希尔伯特(综述)
% license CCBYSA3
% type Wiki

本文根据 CC-BY-SA 协议转载翻译自维基百科\href{https://en.wikipedia.org/wiki/David_Hilbert}{相关文章}。

\begin{figure}[ht]
\centering
\includegraphics[width=6cm]{./figures/9019106ae7482c98.png}
\caption{1912年的希尔伯特} \label{fig_David_1}
\end{figure}
大卫·希尔伯特(David Hilbert,发音:/ˈhɪlbərt/;德语:[ˈdaːvɪt ˈhɪlbɐt];1862年1月23日 – 1943年2月14日)是德国数学家和数学哲学家,是他那个时代最具影响力的数学家之一。

希尔伯特发现并发展了广泛的基础性思想,包括不变理论、变分法、交换代数、代数数论、几何学基础、算子谱理论及其在积分方程中的应用、数学物理学,以及数学基础(特别是证明理论)。他采纳并捍卫了乔治·康托尔的集合论和超限数理论。1900年,他提出了一系列问题,为20世纪的数学研究指明了方向。

希尔伯特及其学生为建立严格的数学理论做出了贡献,并发展了现代数学物理中重要的工具。他是证明理论和数学逻辑的共同创始人。
\subsection{生活}  
\subsubsection{早期生活与教育}  
希尔伯特是奥托(Otto),一位县法官,和玛丽亚·特蕾莎·希尔伯特(Maria Therese Hilbert,原姓Erdtmann,一位商人的女儿)的长子和唯一的儿子。他出生在普鲁士省(当时属于普鲁士王国),具体地点是哥尼斯堡(根据希尔伯特本人所说)或哥尼斯堡附近的维劳(自1946年起称为兹南门斯克),当时他的父亲在该地工作。希尔伯特的祖父是大卫·希尔伯特,一名法官和秘密顾问(Geheimrat)。母亲玛丽亚对哲学、天文学和质数有兴趣,而父亲奥托则教他普鲁士的美德。父亲成为市法官后,家庭迁至哥尼斯堡。大卫的妹妹伊丽丝(Elise)在他六岁时出生。他在八岁时开始上学,比通常的入学年龄晚了两年。

1872年底,希尔伯特进入了弗里德里希中学(Friedrichskolleg Gymnasium,亦称哥尼斯堡皇家学院,是伊曼努尔·康德140年前曾就读的学校);然而,在一段不愉快的时期后,他于1879年底转学并于1880年初从更注重科学的威廉中学(Wilhelm Gymnasium)毕业。毕业后,希尔伯特于1880年秋季入读哥尼斯堡大学(“阿尔贝尔蒂纳”大学)。1882年初,赫尔曼·闵可夫斯基(Hermann Minkowski,希尔伯特比他年长两岁,同为哥尼斯堡人,但曾到柏林学习了三个学期)回到哥尼斯堡并进入了这所大学。希尔伯特与这位害羞但才华横溢的闵可夫斯基建立了终生的友谊。
\subsubsection{职业生涯}
\begin{figure}[ht]
\centering
\includegraphics[width=6cm]{./figures/2e091f1db63eaea3.png}
\caption{1886年的希尔伯特} \label{fig_David_2}
\end{figure}
\begin{figure}[ht]
\centering
\includegraphics[width=6cm]{./figures/f6ef6825f2a61abb.png}
\caption{1907年的希尔伯特} \label{fig_David_3}
\end{figure}
1884年,阿道夫·赫尔维茨(Adolf Hurwitz)从哥廷根大学来到哥尼斯堡大学,担任外籍教授(即副教授)。三人之间开始了密切且富有成果的学术交流,尤其是闵可夫斯基和希尔伯特,他们在各自的科学事业中多次互相影响。希尔伯特于1885年获得博士学位,博士论文题为《Über invariante Eigenschaften spezieller binärer Formen, insbesondere der Kugelfunktionen》(《关于特殊二元形式的不变性质,特别是球面谐波函数》),该论文是在费尔迪南·冯·林德曼(Ferdinand von Lindemann)的指导下写的。

希尔伯特于1886年到1895年期间,担任哥尼斯堡大学的私人讲师(Privatdozent)。1895年,在费利克斯·克莱因(Felix Klein)的帮助下,他获得了哥廷根大学数学教授的职位。在克莱因和希尔伯特的领导下,哥廷根大学成为了数学界的顶尖学府。他在那里度过了余生。
\subsubsection{哥廷根学派}
\begin{figure}[ht]
\centering
\includegraphics[width=8cm]{./figures/e40cafaed8f9a186.png}
\caption{哥廷根数学研究所。其新建筑由洛克菲勒基金会资助,希尔伯特和库朗于1930年共同揭幕。} \label{fig_David_4}
\end{figure}
希尔伯特的学生包括赫尔曼·外尔(Hermann Weyl)、国际象棋冠军埃马努埃尔·拉斯克(Emanuel Lasker)、恩斯特·策梅洛(Ernst Zermelo)和卡尔·古斯塔夫·亨佩尔(Carl Gustav Hempel)。约翰·冯·诺伊曼(John von Neumann)曾是他的助手。在哥廷根大学,希尔伯特与20世纪一些最重要的数学家共同工作,他的社交圈中包括了艾米·诺瑟(Emmy Noether)和阿隆佐·丘奇(Alonzo Church)等人。

希尔伯特在哥廷根的69名博士生中,有许多人后来成为了著名的数学家,包括(及其论文答辩年份):奥托·布卢门塔尔(Otto Blumenthal,1898年)、费利克斯·伯恩斯坦(Felix Bernstein,1901年)、赫尔曼·外尔(Hermann Weyl,1908年)、理查德·库朗(Richard Courant,1910年)、埃里希·黑克(Erich Hecke,1910年)、雨果·施泰因豪斯(Hugo Steinhaus,1911年)和威廉·阿克曼(Wilhelm Ackermann,1925年)。  

1902年至1939年间,希尔伯特担任《数学年刊》(Mathematische Annalen)的编辑,这是当时最重要的数学期刊之一。1907年,他被选为美国国家科学院的国际会员。
\subsubsection{个人生活}
\begin{figure}[ht]
\centering
\includegraphics[width=6cm]{./figures/734202e44a335bb0.png}
\caption{希尔伯特与他的妻子凯瑟·耶罗施(1892年)} \label{fig_David_5}
\end{figure}
\begin{figure}[ht]
\centering
\includegraphics[width=6cm]{./figures/70a233823a56b9b1.png}
\caption{弗朗茨·希尔伯特} \label{fig_David_6}
\end{figure}
1892年,希尔伯特与凯瑟·耶罗施(Käthe Jerosch,1864–1945)结婚,她是哥尼斯堡一位商人的女儿,“是一位直言不讳、思想独立的年轻女士,与希尔伯特的独立思想不谋而合。”在哥尼斯堡期间,他们有了唯一的孩子,弗朗茨·希尔伯特(Franz Hilbert,1893–1969)。弗朗茨一生饱受精神疾病困扰,在他被送入精神病诊所后,希尔伯特曾说:“从今以后,我必须认为自己没有儿子。”他对弗朗茨的态度给凯瑟带来了相当大的痛苦。

希尔伯特认为数学家赫尔曼·闵可夫斯基是他“最好的和最忠实的朋友”。

希尔伯特在普鲁士福音教会接受洗礼并成长为一名加尔文主义者。[a] 后来他离开了教会,成为了一名不可知论者。[b] 他还认为,数学真理独立于上帝的存在或其他先验假设。[c][d] 当伽利略·伽利莱因未能坚持他的日心说理论时,希尔伯特对此提出异议:“但[伽利略]并不是傻瓜。只有傻瓜才会认为科学真理需要殉道;这在宗教中或许是必要的,但科学结果终会自己证明。”[e]
\begin{figure}[ht]
\centering
\includegraphics[width=6cm]{./figures/b030fb58ef9d3b4d.png}
\caption{凯瑟·希尔伯特与康斯坦丁·卡拉塞奥多里(1932年之前)} \label{fig_David_7}
\end{figure}
\subsubsection{晚年} 
像阿尔伯特·爱因斯坦一样,希尔伯特与柏林学派保持着密切联系,该学派的主要创始人曾在哥廷根大学向希尔伯特学习(包括库尔特·格雷林(Kurt Grelling)、汉斯·赖兴巴赫(Hans Reichenbach)和瓦尔特·杜比斯拉夫(Walter Dubislav))。[18]

大约在1925年,希尔伯特患上了恶性贫血,这是一种当时无法治疗的维生素缺乏症,其主要症状是极度疲劳;他的助手尤金·维格纳(Eugene Wigner)描述他经历了“极度的疲劳”,并表示他“看起来非常衰老”。即使在最终被诊断并接受治疗后,他“几乎不再是一个科学家”,而且“毫无疑问,1925年以后他不再是希尔伯特”。[19]

希尔伯特于1932年当选为美国哲学学会会员。[20]

希尔伯特活到了纳粹在1933年清洗哥廷根大学许多著名教员的时期。[21] 被迫离开的包括赫尔曼·外尔(Hermann Weyl)(他在1930年希尔伯特退休后接替了希尔伯特的职位)、艾米·诺瑟(Emmy Noether)和埃德蒙·兰道(Edmund Landau)。其中一位不得不离开德国的保罗·伯奈耶(Paul Bernays),他曾与希尔伯特合作过数学逻辑,并共同撰写了重要著作《数学基础》(Grundlagen der Mathematik)[22](该书最终分两卷出版,分别是1934年和1939年)。这本书是希尔伯特与阿克曼(Ackermann)1928年出版的《数学逻辑原理》一书的续集。赫尔曼·外尔的继任者是赫尔穆特·哈塞(Helmut Hasse)。

大约一年后,希尔伯特参加了一场宴会,并与新任教育部长伯恩哈德·鲁斯特(Bernhard Rust)同座。鲁斯特问道:“数学研究所真的因为犹太人的离去而遭受如此重大的打击吗?”希尔伯特回答道:“打击?它不再存在了,不是吗?”[23][24]
\subsubsection{死亡}
\begin{figure}[ht]
\centering
\includegraphics[width=6cm]{./figures/46768d3dffe299bf.png}
\caption{希尔伯特的墓碑: 我们必须知道  我们将知道} \label{fig_David_8}
\end{figure}
希尔伯特于1943年去世时,纳粹几乎完全更换了哥廷根大学的教职员工,因为许多前任教员要么是犹太人,要么是犹太人的配偶。希尔伯特的葬礼只有不到十人参加,其中只有两位是同行学者,其中包括阿诺德·索末菲尔德(Arnold Sommerfeld),一位理论物理学家,也是哥尼斯堡人。[25] 他的死讯直到他去世几个月后才为外界所知。[26]

他在哥廷根墓碑上的碑文摘自他在1930年9月8日向德国科学家与医生协会发表退休演讲时所说的著名话语。这些话是对拉丁格言“Ignoramus et ignorabimus”或“我们不知道,也不会知道”的回应:[27]

Wir müssen wissen.  
Wir werden wissen.

我们必须知道。  
我们将知道。

在希尔伯特在1930年德国科学家与医生协会年会上发表这些话的前一天,库尔特·哥德尔(Kurt Gödel)——在与该协会会议共同举行的认识论会议的圆桌讨论中——初步宣布了他不完备定理的第一个表述。[f] 哥德尔的不完备定理表明,即使是像佩亚诺算术这样简单的公理系统,也要么是自相矛盾的,要么包含无法在该系统内证明或反驳的逻辑命题。
\subsection{对数学和物理的贡献}
\subsubsection{解决戈尔丹问题}  
希尔伯特的第一次关于不变函数的研究导致他在1888年证明了著名的有限性定理。二十年前,保罗·戈尔丹(Paul Gordan)曾使用复杂的计算方法证明了二次型生成元的有限性定理。试图将他的计算方法推广到具有多于两个变量的函数时,由于涉及的计算难度极大,这些尝试都失败了。为了应对这一被一些学者称为“戈尔丹问题”的难题,希尔伯特意识到必须采取完全不同的方法。因此,他证明了希尔伯特基定理,展示了任意多个变量的量子不变量的有限生成元的存在,但这种证明是抽象的。也就是说,虽然证明了这种集合的存在,但它并不是一个构造性证明——它没有展示“一个具体的对象”——而是一种存在性证明[28],并依赖于在无限扩展中使用排中律。

希尔伯特将他的结果提交给了《数学年刊》(Mathematische Annalen)。戈尔丹,这本期刊上负责不变量理论的专家,未能理解希尔伯特定理的革命性,并拒绝了这篇文章,批评其内容因为不够全面。戈尔丹的评论是:

Das ist nicht Mathematik. Das ist Theologie.  
这不是数学。这是神学。[29]

另一方面,费利克斯·克莱因(Felix Klein)认识到这项工作的重大意义,并保证这篇文章会未经修改地发表。在克莱因的鼓励下,希尔伯特在第二篇文章中扩展了他的方法,提供了最小生成元集的最大度数估计,并再次将其提交给《数学年刊》。在阅读了这篇手稿后,克莱因写信给他,表示:

Without doubt this is the most important work on general algebra that the Annalen has ever published.  
毫无疑问,这是《数学年刊》上发布的最重要的代数工作。[30]

后来,在希尔伯特方法的实用性被普遍认可之后,戈尔丹自己也表示:

I have convinced myself that even theology has its merits. 
我已经说服自己,甚至神学也有它的优点。[31]

尽管希尔伯特取得了如此多的成功,但他证明的性质却带来了比他想象的更多麻烦。虽然克罗内克(Kronecker)已经承认了这一点,希尔伯特后来回应类似的批评时曾说:“许多不同的构造都可以归结为一个基本的思想”——换句话说(引用Reid的话):“通过存在性证明,希尔伯特能够获得一个构造”;“这个证明”(即页面上的符号)就是“对象”[31]。并非所有人都信服。虽然克罗内克很快去世,但他的构造主义哲学仍然由年轻的布劳威尔(Brouwer)和他正在发展的直觉主义“学派”继承下去,这给希尔伯特的晚年带来了极大的困扰[32]。事实上,希尔伯特最终会失去他的“天才学生”外尔(Weyl),他转向了直觉主义——“希尔伯特对他曾经学生对布劳威尔思想的迷恋感到不安,这唤起了他对克罗内克的记忆”[33]。直觉主义者布劳威尔尤其反对对无限集使用排中律(正如希尔伯特所做的那样)。希尔伯特回应道:

Taking the Principle of the Excluded Middle from the mathematician ... is the same as ... prohibiting the boxer the use of his fists.
从数学家那里剥夺排中律的原则……就像是……禁止拳击手使用拳头。[34]
\subsubsection{零点定理(Nullstellensatz)}
在代数学科中,一个域被称为\textbf{代数闭合的},当且仅当它上面的每个多项式都有一个根在这个域中。在这个条件下,希尔伯特给出了一个标准,用于判断一组多项式 \((p_{\lambda})_{\lambda \in \Lambda}\) (有 \(n\) 个变量)是否有公共根:当且仅当不存在多项式 \(q_1, \dots, q_k\) 和指数 \(\lambda_1, \dots, \lambda_k\),使得:

\[
1 = \sum_{j=1}^{k} p_{\lambda_j}(\vec{x}) q_j(\vec{x})~
\]

这个结果被称为\textbf{希尔伯特根定理}(Hilbert's Root Theorem),或在德语中称为\textbf{希尔伯特零点定理}(Hilberts Nullstellensatz)。他还证明了,消失理想与其消失集之间的对应关系在仿射簇与 \( \mathbb{C}[x_1, \dots, x_n] \) 中的极小理想之间是一一对应的。
\subsubsection{曲线}
\begin{figure}[ht]
\centering
\includegraphics[width=6cm]{./figures/928fa7ce33d45ec2.png}
\caption{替代规则} \label{fig_David_9}
\end{figure}
1890年,朱塞佩·皮亚诺(Giuseppe Peano)在《数学年鉴》(Mathematische Annalen)上发表了一篇文章,描述了历史上第一个填充空间的曲线。作为回应,希尔伯特设计了他自己的这种曲线构造,现称为希尔伯特曲线。该曲线的逼近是通过迭代构造的,依据本节第一张图片中的替代规则进行。曲线本身则是逐点极限。
\begin{figure}[ht]
\centering
\includegraphics[width=6cm]{./figures/91bbeda6a9cddcff.png}
\caption{希尔伯特曲线的前六个逼近} \label{fig_David_10}
\end{figure}
\subsubsection{几何的公理化}
希尔伯特于1899年出版的《几何基础》(Grundlagen der Geometrie,英文译名《Foundations of Geometry》)提出了一组正式的公理集,称为**希尔伯特公理**,用来替代传统的欧几里得公理。这些公理避免了欧几里得公理中被发现的弱点,欧几里得的作品在当时仍作为教科书使用。要具体说明希尔伯特使用的公理,必须参考《几何基础》的出版历史,因为希尔伯特曾多次修改和调整这些公理。原版专著很快就有了法文翻译,其中希尔伯特增加了V.2,即完备性公理(Completeness Axiom)。由E.J. Townsend翻译的英文版在1902年获得了希尔伯特的授权,并注册版权。[35][36] 该翻译版本包含了法文版的修改,因此被认为是第二版的翻译。希尔伯特继续修改文本,并且该书在德文版中出现了多个版本,第七版是希尔伯特去世时的最后一版。第七版之后又有新版本出版,但主要文本基本没有修改。[g]

希尔伯特的这一方法标志着现代公理化方法的转变。在这方面,希尔伯特的工作受到了莫里茨·帕施(Moritz Pasch)1882年工作的预示。公理不被视为自明的真理。几何学可以处理一些我们直观上非常理解的事物,但不需要为这些未定义的概念指定任何明确的意义。元素,如点、线、平面等,可以被替换成,如希尔伯特对舍恩弗里斯和科特尔所说的,桌子、椅子、啤酒杯以及其他类似的物体。[37] 讨论的重点是它们之间的定义关系。

希尔伯特首先列举了未定义的概念:点、线、平面、在(点与线、点与平面、线与平面之间的关系)、相对位置、点对(线段)的全等性以及角度的全等性。这些公理将欧几里得的平面几何和立体几何统一为一个系统。
\subsubsection{23个问题}
希尔伯特在1900年巴黎国际数学家大会上提出了一份高度影响力的清单,包含23个未解问题。这通常被认为是由单个数学家提出的最成功、最深刻的未解问题集合。[来源需要]  

在重新研究经典几何的基础之后,希尔伯特有可能将其推演到整个数学领域。他的方法与后来的“基础主义者”拉塞尔–怀特黑德(Russell-Whitehead)或“百科全书派”布尔巴基(Nicolas Bourbaki)以及同时代的朱塞佩·皮亚诺(Giuseppe Peano)有所不同。整体数学界可以参与他认为是数学重要领域的关键问题的解决。

这些问题集合作为一次演讲“数学问题”在巴黎举行的第二届国际数学家大会上首次提出。希尔伯特在演讲的引言中说:

“我们之中有谁不愿意揭开隐藏未来的面纱,目睹我们科学的未来发展,探索其在未来几个世纪的发展奥秘?未来一代数学家的精神将会朝着什么目标前进?新世纪将揭示数学思想这一广阔而丰富领域中的哪些方法和新事实?”[38]

在大会上,希尔伯特提出了不到一半的问题,这些问题被收录在大会记录中。随后,他扩展了这一视野,并提出了现在被视为经典的23个希尔伯特问题。另见希尔伯特的第二十四个问题。完整的文本非常重要,因为对这些问题的解释仍然是一个不可避免的辩论话题,每当讨论有多少个问题已被解决时,都会涉及这些问题。

其中一些问题在短时间内得以解决。另一些问题则在整个20世纪被讨论过,少数问题至今仍被认为过于开放,难以得出结论。一些问题至今仍然是挑战。

以下是希尔伯特在《美国数学会公报》1902年翻译中所列的23个问题的标题:

\begin{enumerate}
\item 康托尔的连续体基数问题。
\item 算术公理的一致性。
\item 两个底面和高度相等的四面体体积相等问题。
\item 直线是两点之间最短距离的问题。
\item 不假设定义群的函数可微的李群的连续群概念。
\item 物理学公理的数学处理。
\item 某些数的无理性与超越性。
\item 素数问题(“黎曼假设”)。
\item 任何数域中最一般的互反律的证明。
\item 迪奥方程解的可解性判定。
\item 具有任意代数数值系数的二次型。
\item 克罗内克定理关于阿贝尔域的扩展到任何代数有理性领域。
\item 一般七次方程的解不能仅通过两个变量的函数求解。
\item 某些完备函数系统的有限性证明。
\item 施伯特的枚举微积分的严格基础。
\item 代数曲线和曲面拓扑问题。
\item 通过平方表示定向形式。
\item 通过全等多面体构建空间。
\item 变分法中的常规问题的解是否总是解析的?
\item 边界值问题的一般问题(偏微分方程的边界值问题)。
\item 存在具有规定单群的线性微分方程的证明。
\item 通过自同构函数对解析关系进行统一化。
\item 变分法方法的进一步发展。
\end{enumerate}
\subsubsection{形式主义}
在20世纪中叶成为标准的论述中,希尔伯特的23个问题集合被视为一种宣言,为形式主义学派的发展铺平了道路,形式主义是20世纪数学的三大主要学派之一。形式主义者认为,数学是根据约定的形式规则对符号的操作。因此,数学是一种独立的思维活动。
\subsubsection{计划}
1920年,希尔伯特提出了一个元数学研究项目,后来被称为希尔伯特计划。他希望数学能够在一个坚实且完整的逻辑基础上进行表述。他认为,从原则上讲,这可以通过以下两点来实现:
\begin{enumerate}
\item 数学的所有内容都可以从一个正确选择的有限公理系统中推导出来;并且
\item 通过某种方法(如ε演算),证明某种公理系统的一致性。
\end{enumerate}
他似乎有技术和哲学两方面的原因来提出这个建议。这也表明了他对当时德国思想界“ignorabimus”(我们不知道且永远不会知道)观念的反感,这一观点可以追溯到埃米尔·杜·布瓦-雷蒙(Emil du Bois-Reymond)提出的表述。

这个计划在数学哲学的主流观点中依然可以识别出来,通常被称为形式主义。例如,布尔巴基小组采纳了它的一个简化和选择性版本,认为这是满足其双重项目(a)编写百科全书式的基础工作,以及(b)支持公理化方法作为研究工具的要求。这个方法在与希尔伯特的代数和泛函分析工作相关的领域取得了成功并产生了影响,但在与他对物理学和逻辑的兴趣方面,未能同样取得相应的成果。

希尔伯特在1919年写道:

“我们在这里所说的并不是任何意义上的任意性。数学不像一场游戏,其任务是由任意规定的规则决定的。相反,它是一个具有内在必然性的概念系统,只能是这样,而绝不是其他方式。”[40]

希尔伯特在其两卷本著作《数学基础》一书中发表了他对数学基础的看法。
\subsubsection{哥德尔的工作}
希尔伯特和与他共同参与这一事业的数学家们全力投入了这个项目。他试图通过明确的原则来支持公理化数学,以消除理论上的不确定性,但最终以失败告终。

哥德尔证明了,任何非矛盾的形式系统(只要足够全面,至少能包含算术)都无法通过自身的公理证明其完备性。1931年,哥德尔的不完备定理表明,希尔伯特的宏大计划在所述的形式下是不可能实现的。第二点与第一点在合理的方式下无法兼容,前提是公理系统确实是有限的。

尽管如此,随后的证明理论成就至少澄清了与数学家关心的理论相关的一致性问题。希尔伯特的工作开启了逻辑澄清的道路;而对哥德尔工作的理解需求,随后促使了递归理论的发展,并最终在1930年代催生了数学逻辑作为一门独立学科。后来的理论计算机科学的基础,尤其是阿隆佐·丘奇和艾伦·图灵的工作,也直接源自这一“辩论”[41]。
\subsubsection{泛函分析}
大约在1909年,希尔伯特将精力投入到微分方程和积分方程的研究;他的工作对现代泛函分析的许多重要部分产生了直接影响。为了进行这些研究,希尔伯特引入了无限维欧几里得空间的概念,后来被称为希尔伯特空间。他在这部分分析中的工作为接下来二十年间物理学数学的重大贡献奠定了基础,尽管它的方向出乎意料。后来,斯特凡·巴拿赫扩展了这一概念,定义了巴拿赫空间。希尔伯特空间是泛函分析领域中重要的对象类,特别是在自伴线性算子的谱理论方面,在20世纪的发展中起到了关键作用。
\subsubsection{物理学}
直到1912年,希尔伯特几乎完全是一位纯数学家。当计划从波恩来访时,他的数学家朋友兼同事赫尔曼·闵可夫斯基开玩笑说,他必须先在隔离区待上10天才能拜访希尔伯特。事实上,闵可夫斯基似乎在1912年之前负责了希尔伯特大部分的物理学研究,包括他们在1905年联合举办的关于物理的研讨会。

1912年,在朋友去世三年后,希尔伯特几乎将注意力完全集中在物理学上。他安排为自己找了一位“物理导师”[42]。他开始研究动力学气体理论,随后转向初步的辐射理论和物质的分子理论。即使在1914年战争爆发后,他仍继续举办研讨会和课程,紧跟阿尔伯特·爱因斯坦等人的研究成果。

到1907年,爱因斯坦已经提出了引力理论的基本框架,但随后他花费了近8年的时间,才将该理论完成定型。[43] 到1915年初,希尔伯特的物理兴趣集中在广义相对论上,并邀请爱因斯坦到哥廷根就该主题进行为期一周的讲座。[44] 爱因斯坦在哥廷根受到了热烈的欢迎。[45] 在整个夏季,爱因斯坦了解到希尔伯特也在研究场方程,并加大了自己的努力。1915年11月,爱因斯坦发表了几篇论文,最终提出了《引力场方程》(见爱因斯坦场方程)。几乎同时,希尔伯特发表了《物理学的基础》,这是对场方程的公理化推导(见爱因斯坦–希尔伯特作用量)。希尔伯特完全承认爱因斯坦是该理论的创始人,在他们一生中,关于场方程的公众优先权争议从未发生。[i] 更多内容见优先权问题。

此外,希尔伯特的工作预示并促进了量子力学数学表述的若干进展。他的研究是赫尔曼·魏尔和约翰·冯·诺依曼关于维尔纳·海森堡的矩阵力学和厄尔温·薛定谔的波动方程数学等价性的研究的关键组成部分,他以自己名字命名的希尔伯特空间在量子理论中起着重要作用。1926年,冯·诺依曼证明,如果量子态被理解为希尔伯特空间中的向量,它们将与薛定谔的波函数理论和海森堡的矩阵相对应。[j]

在全身心投入物理学研究的过程中,希尔伯特致力于为物理学的数学提供严格性。尽管物理学高度依赖高等数学,物理学家们往往在使用数学时显得“马虎”。对像希尔伯特这样的纯数学家来说,这既丑陋又难以理解。当他开始理解物理学以及物理学家如何使用数学时,他为他所发现的内容发展出了一个连贯的数学理论——最重要的是在积分方程领域。当他的同事理查德·库朗编写了现在经典的《数学物理方法》(Methoden der mathematischen Physik)一书,并包含了希尔伯特的一些思想时,尽管希尔伯特并没有直接参与写作,他仍然在书中添加了希尔伯特的名字作为作者。希尔伯特曾说:“物理学对物理学家来说太难了”,暗示所需的数学通常超出了他们的能力;库朗–希尔伯特的书使得物理学家们能够更容易理解这一点。
\subsubsection{数论}
希尔伯特通过他在1897年发表的论文《Zahlbericht》(字面意思为“数字报告”)统一了代数数论领域。他还解决了由沃林(Waring)在1770年提出的一个重要数论问题。与有限性定理类似,他使用了一种存在性证明,表明这个问题必定有解,而不是提供一种产生答案的机制。[46] 随后,他在这个主题上几乎没有更多的出版物;但在他的一位学生的论文中出现了希尔伯特模形式,因此他的名字进一步与这一重要领域相关联。

他还对类域理论提出了一系列猜想。这些概念具有深远的影响,他的贡献仍体现在希尔伯特类域和局部类域理论中的希尔伯特符号这一名称上。通过高桥贞次(Teiji Takagi)的工作,这些结果大多数在1930年之前得到了证明。[k]

希尔伯特并未直接从事解析数论的核心领域,但他的名字因希尔伯特–波利亚猜想而为人所知,原因有些轶事性质。希尔伯特的学生恩斯特·海林格曾告诉安德烈·韦伊,希尔伯特在20世纪初期的一个研讨会上宣布,他预计黎曼猜想的证明将是弗雷德霍尔姆(Fredholm)关于带对称核的积分方程的工作的一个推论。[47]
\subsection{作品}
希尔伯特的文集(Gesammelte Abhandlungen)已经多次出版。原始版本的论文包含了“许多不同程度的技术性错误”[48];当文集首次出版时,这些错误被修正,并且发现修正工作可以在不对定理陈述做重大更改的情况下完成,唯一的例外是——一项被声称的连续统假设的证明。[49][50] 然而,这些错误如此之多且如此重要,以至于奥尔加·陶斯基-托德(Olga Taussky-Todd)花了三年时间才完成修正工作。[50]
\subsection{另见}
\subsubsection{概念}
以大卫·希尔伯特命名的事物列表:
\begin{itemize}
\item 几何基础
\item 希尔伯特 C* 模块
\item 希尔伯特立方体
\item 希尔伯特曲线
\item 希尔伯特矩阵
\item 希尔伯特度量
\item 希尔伯特-穆姆福德准则
\item 希尔伯特数
\item 希尔伯特环
\item 希尔伯特-庞加莱级数
\item 希尔伯特级数和希尔伯特多项式
\item 希尔伯特空间
\item 希尔伯特谱
\item 希尔伯特系统
\item 希尔伯特变换
\item 希尔伯特的端算术
\item 希尔伯特的“大酒店悖论”
\item 希尔伯特-施密特算子
\item 希尔伯特-史密斯猜想
\end{itemize}
\subsubsection{定理}
\begin{itemize}
\item 希尔伯特-伯奇定理
\item 希尔伯特不可约性定理
\item 希尔伯特的零点定理 (Hilbert's Nullstellensatz)
\item 希尔伯特定理(微分几何)
\item 希尔伯特定理90
\item 希尔伯特的Syzygy定理
\item 希尔伯特-斯佩塞尔定理
\end{itemize}
\subsubsection{其他}
\begin{itemize}
\item 布劳威尔-希尔伯特争议
\item 变分法中的直接法
\item 可判定性问题(Entscheidungsproblem)
\item 《几何与想象》
\item 广义相对论优先权争议
\end{itemize}
\subsection{脚注}

a.希尔伯特一家此时已经离开了他们曾受洗和结婚的加尔文派新教教会。— Reid 1996,第91页\\
b.大卫·希尔伯特似乎是不可知论者,并且与神学或宗教并无关联。康斯坦斯·里德讲述了一个关于此的故事:

   希尔伯特一家在大约1902年时,已经离开了他们曾受洗和结婚的改革宗新教教会。在哥廷根流传着这样一个故事:当大卫·希尔伯特的儿子弗朗茨开始上学时,他无法回答“你是什么宗教信仰?”这个问题。(1970年,第91页)\\
在1927年的汉堡演讲中,希尔伯特宣称:“数学是无前提的科学(die Mathematik ist eine voraussetzungslose Wissenschaft)”以及“为了建立数学,我不需要一个善良的上帝([z]u ihrer Begründung brauche ich weder den lieben Gott)”(1928年,第85页;van Heijenoort,1967年,第479页)。然而,从《数学问题》(1900年)到《自然认识与逻辑》(1930年),他把自己对人类精神和纯粹思想的几乎宗教般的信仰寄托于数学的力量上。他深信每一个数学问题都可以通过纯粹的理性解决:无论是数学还是任何自然科学的部分(通过数学),都不存在“我们无法知道的事”(Hilbert,1900年,第262页;1930年,第963页;Ewald,1996年,第1102、1165页)。因此,寻找数学的内在绝对基础成为了希尔伯特一生的事业。他从未放弃这个立场,具有象征意义的是,他在1930年哥尼斯堡演讲中说的“我们必须知道,我们将知道”这句话,被刻在了他的墓碑上。在这里,我们遇到了已逝神学的幽灵(用乔治·贝克莱的话来说),因为将人类认知绝对化意味着默默地将其与神圣的认知等同起来。—Shaposhnikov, Vladislav (2016). "Theological Underpinnings of the Modern Philosophy of Mathematics. Part II: The Quest for Autonomous Foundations". *Studies in Logic, Grammar and Rhetoric*, 44 (1): 147–168. doi:10.1515/slgr-2016-0009\\
c.“数学是一门无前提的科学。为了奠定它的基础,我不需要像克罗内克尔那样依赖上帝,也不需要像庞加莱那样假设我们理解的特殊能力与数学归纳法原理相适应,也不需要布劳威尔的原始直觉,最后,也不需要像拉塞尔和怀特黑德那样依赖无穷、可约性或完备性的公理,这些公理实际上是实际的、有内容的假设,无法通过一致性证明来弥补。”——大卫·希尔伯特,《数学基础》,希尔伯特计划,22C:096,爱荷华大学\\  
d.迈克尔·R·马修斯(2009),《科学、世界观与教育》,斯普林格,129页,ISBN 978-90-481-2779-5。众所周知,希尔伯特拒绝了利奥波德·克罗内克尔在数学基础问题上的“上帝”假设。\\  
e.康斯坦斯·里德;赫尔曼·外尔(1970),《希尔伯特》,斯普林格,92页,ISBN 978-0-387-04999-1。或许来宾们会讨论伽利略的审判,有人会指责伽利略未能坚持自己的信念。“但他不是傻子,”希尔伯特会反驳,“只有傻子才会认为科学真理需要殉道;这在宗教中可能是必要的,但科学成果会随着时间的推移自我证明。”\\
f.“《精确科学的认识论会议》持续了三天,从9月5日到7日”(道森 1997:68)。“它...与第九十一届德国科学家与医生年会和第六届德国物理学家与数学家大会同时举行,并且紧接在这些大会之前...戈德尔的发言是在1930年9月6日(星期六)下午3点到3点20分进行的,会议在第二天(星期日)结束时进行了一场关于第一天演讲的圆桌讨论。在这场讨论中,戈德尔毫无预警地、几乎随口地宣布:“甚至可以举出一些命题的例子(实际上是像哥德巴赫猜想或费马大定理这样的命题),它们内容上是正确的,但在经典数学的形式体系中是无法证明的”(道森:69)。“事实上,希尔伯特本人也出席了在哥尼斯堡举行的会议,尽管显然他并未参加认识论会议。圆桌讨论的第二天,他在德国科学家与医生协会的年会上发表了开幕演讲——他的著名演讲《自然认识与逻辑》(Naturerkennen und Logik),在演讲的最后,他宣称:“对于数学家来说,没有‘无知’(Ignorabimus),我认为自然科学也同样没有…我认为没人能够找到一个无法解决的问题,真正的原因是,没有无法解决的问题。与愚蠢的‘无知’相对,我们的信条是:我们必须知道,我们将知道”(道森:71)。戈德尔的论文于1930年11月17日收到(参见里德,第197页,范·海耶诺特 1976:592),并于1931年3月25日出版(道森 1997:74)。但戈德尔在此之前已经做过一次讲座...“1930年10月,汉斯·哈恩向维也纳科学院提交了摘要”(范·海耶诺特:592);这篇摘要和完整论文都出现在范·海耶诺特:583页及以后。\\
g.独立且同时,19岁的美国学生罗伯特·李·穆尔(Robert Lee Moore)发布了一组等效的公理。部分公理相同,而穆尔体系中的某些公理在希尔伯特的体系中是定理,反之亦然。[需要引用]\\
h.随着时间的推移,将引力场方程与希尔伯特的名字联系起来的情况变得越来越少。一个显著的例外是P.约旦(《引力与宇宙》,布伦瑞克,维维出版社,1952),他称真空中的引力方程为“爱因斯坦–希尔伯特方程”。(莱奥·科里,《大卫·希尔伯特与物理学公理化》,第437页)”\\
i.自1971年以来,关于哪位学者最早提出现今公认的场方程形式,展开了激烈且学术性的讨论。“希尔伯特曾坦率承认,并在讲座中经常表示,这个伟大的想法是爱因斯坦的:‘哥尼斯堡街上的每个小男孩比爱因斯坦更懂四维几何,’他曾如此说。‘然而,尽管如此,做这项工作的还是爱因斯坦,而不是数学家。’”(里德 1996,第141–142页,也参见艾萨克森 2007:222,引用自索恩,第119页)。\\
j.1926年,即马克斯·玻恩和维尔纳·海森堡提出量子力学矩阵力学形式的次年,数学家约翰·冯·诺依曼成为希尔伯特在哥尼斯堡的助手。当冯·诺依曼于1932年离开时,他基于希尔伯特数学的量子力学数学基础的书籍《量子力学的数学基础》出版了。参见:诺曼·麦克雷(1999)《约翰·冯·诺依曼:开创现代计算机、博弈论、核威慑及更多领域的科学天才》(美国数学学会再版)和里德(1996)。\\
k.这项工作确立了高田为日本第一位具有国际声誉的数学家。\\
\subsection{引用文献}
\begin{enumerate}
\item Weyl, H. (1944). "David Hilbert. 1862–1943". 《皇家学会会士讣告》。4 (13): 547–553. doi:10.1098/rsbm.1944.0006. S2CID 161435959.
\item David Hilbert at the Mathematics Genealogy Project
\item "Hilbert". 《随机出版社韦伯斯特未删节词典》.
\item Joyce, David. "David Hilbert的数学问题". Clark University. 检索于2021年1月15日.
\item Hilbert, David. "数学问题". 检索于2021年1月15日.
\item Zach, Richard (2003年7月31日). "Hilbert的程序". 《斯坦福哲学百科全书》。检索于2009年3月23日.
\item Reid 1996,第1–3页;同样在第8页,Reid提到Hilbert的出生地有一些歧义。Hilbert本人曾表示他出生在哥尼斯堡。
\item Reid 1996,第4–7页。
\item Reid 1996,第11页。
\item Reid 1996,第12页。
\item Weyl, Hermann (2012), "David Hilbert及其数学工作", 见Peter Pesic(编),《无限的层次/数学与哲学的精选著作》,Dover出版社,第94页,ISBN 978-0-486-48903-2
\item Suzuki, Jeff (2009), 《历史背景中的数学》,美国数学协会,第342页,ISBN 978-0-88385-570-6
\item "数学家族树项目 – David Hilbert". 检索于2007年7月7日。
\item "David Hilbert". www.nasonline.org. 检索于2023年6月30日。
\item Reid 1996,第36页。
\item Reid 1996,第139页。
\item Reid 1996,第121页。
\item Milkov, Nikolay; Peckhaus, Volker (2013年1月1日). "柏林学派与维也纳圈子:亲缘与分歧". 《柏林学派与逻辑实证主义哲学》 (PDF). 《波士顿科学哲学与历史系列》。第273卷,第20页。doi:10.1007/978-94-007-5485-0_1. ISBN 978-94-007-5485-0. OCLC 7325392474. 归档 (PDF) 自原始文献 2014年8月20日。检索于2021年5月19日。
\item 1992年(由Andrew Szanton叙述)。《尤金·P·维格纳的回忆录》。Plenum出版社,ISBN 0-306-44326-0
\item "APS会员历史". search.amphilsoc.org. 检索于2023年6月30日。
\item "Göttingen的'耻辱'". 从原文档案存档于2013年11月5日。检索于2013年6月5日。(Hilbert的同事被流放)
\item Milne-Thomson, L (1935). "《数学基础》摘要". 《自然》. 136 (3430): 126–127. doi:10.1038/136126a0. S2CID 4122792. 检索于2023年12月15日。这可能是自Whitehead和Russell的《数学原理》以来最重要的数学基础书籍。
\item Eckart Menzler-Trott: 《Gentzens问题。纳粹德国的数学逻辑》, Birkhäuser, 2001, ISBN 3-764-36574-9, Birkhäuser出版社;2001年版,第142页。
\item Hajo G. Meyer: 《悲剧命运。德国犹太人和历史力量的影响:应用历史哲学的练习》,Frank & Timme,2008年,ISBN 3-865-96174-6,第202页。
\item Reid 1996,第213页。
\item Reid 1996,第214页。
\item Reid 1996,第192页。
\item Reid 1996,第36–37页。
\item Reid 1996,第34页。
\item Reid 1996,第195页。
\item Reid 1996,第37页。
\item 参见 Reid 1996,第148–149页。
\item Reid 1996,第148页。
\item Reid 1996,第150页。
\item Hilbert 1950
\item G. B. Mathews(1909) 《从自然出发的几何基础》Nature 80:394,5 (#2066)
\item Otto Blumenthal (1935). David Hilbert (编辑). 《生活史》。《全集》。第3卷。Julius Springer出版社,第388–429页。已从2016年3月4日的原档案存档。检索于2018年9月6日。此处:第402-403页。
\item "存档副本"(PDF)。已从原文档案存档于2009年5月30日。检索于2012年9月11日。存档来自 [www.seas.harvard.edu/courses/cs121/handouts/Hilbert.pdf]
\item Finkelstein, Gabriel (2013). 《Emil du Bois-Reymond: 神经科学、自我与19世纪德国社会》。剑桥;伦敦:MIT出版社,第265–289页,ISBN 978-0262019507。
\item Hilbert, D. (1919–20), 《自然与数学认识:1919-1920年在哥尼斯堡的讲座》。由Paul Bernays编写(由David E. Rowe编辑并附英文介绍),巴塞尔,Birkhäuser出版社(1992年)。
\item Reichenberger, Andrea (2019年1月31日). "从可解性到形式可判定性:重访Hilbert的'非Ignorabimus'",《人文数学期刊》。9 (1): 49–80. doi:10.5642/jhummath.201901.05. ISSN 2159-8118. S2CID 127398451.
\item Reid 1996,第129页。
\item Isaacson 2007:218
\item Sauer 1999; Fölsing 1998【需要页码】; Isaacson 2007:212
\item Isaacson 2007:213
\item Reid 1996,第114页。
\item Endres, S.; Steiner, F. (2009), "Berry–Keating算子在 \( L^{2}({\mathbb{R}}_{>},{\rm d}x) \) 和一般自伴实现的紧量子图上的应用", 《物理学A:数学与理论》,43 (9): 37, arXiv:0912.3183v5, doi:10.1088/1751-8113/43/9/095204, S2CID 115162684
\item Reid 1996,第13章。
\item Sieg 2013,第284-285页。
\item Rota G.-C. (1997), "我希望曾被教过的十课", 《美国数学会通知》,44: 22–25。
\end{enumerate}
\subsection{来源}
\subsubsection{英文翻译版的主要文献}
\begin{itemize}
\item Ewald, William B., 编. (1996). 《从康德到希尔伯特:数学基础的源头书》。牛津,英国:牛津大学出版社。
  \item 1918年:“公理化思维”,1114–1115页。
  \item 1922年:“数学的新基础:第一次报告”,1115–1133页。
  \item 1923年:“数学的逻辑基础”,1134–1147页。
  \item 1930年:“逻辑与自然知识”,1157–1165页。
  \item 1931年:“初等数论的基础”,1148–1156页。
  \item 1904年:“关于逻辑和算术的基础”,129–138页。
  \item 1925年:“关于无限”,367–392页。
  \item 1927年:“数学基础”,附有Weyl的评论和Bernays的附录,464–489页。
\item van Heijenoort, Jean (1967). 《从弗雷格到哥德尔:数学逻辑的源头书,1879–1931》。哈佛大学出版社。
\item Hilbert, David (1950) [1902]. 《几何的基础》[Grundlagen der Geometrie](PDF)。由E.J. Townsend翻译(第2版)。La Salle, IL:Open Court出版社。存档(PDF)自2005年12月28日的原文。
\item Hilbert, David (1990) [1971]. 《几何的基础》[Grundlagen der Geometrie]。由Leo Unger翻译(第二版英文版)。La Salle, IL:Open Court出版社。ISBN 978-0-87548-164-7。从第10版德文版翻译。
\item Hilbert, David; Cohn-Vossen, Stephan (1999). 《几何与想象力》。美国数学学会。ISBN 978-0-8218-1998-2。最初为哥尼斯堡市民讲授的一系列通俗讲座。
\item Hilbert, David (2004). Hallett, Michael; Majer, Ulrich (编.). 《David Hilbert的数学与物理基础讲座,1891–1933》。柏林与海德堡:Springer-Verlag出版社。ISBN 978-3-540-64373-9。
\end{itemize}