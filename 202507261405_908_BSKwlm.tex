% 博苏克-乌拉姆定理(综述)
% license CCBYSA3
% type Wiki

本文根据 CC-BY-SA 协议转载翻译自维基百科\href{https://en.wikipedia.org/wiki/Borsuk\%E2\%80\%93Ulam_theorem}{相关文章}。

在数学中,博苏克–乌拉姆定理指出:

每一个从 $n$ 维球面 $S^n$ 到 $n$ 维欧几里得空间 $\mathbb{R}^n$ 的连续函数,必定存在一对对踵点被映射到同一个点。这里,对踵点指的是位于球面上、从球心看方向完全相反的两点。

形式化表述:若$f: S^n \to \mathbb{R}^n$是一个连续函数,则存在$x \in S^n$使得$f(-x) = f(x)$。

特例说明:当 $ n = 1$:这意味着地球赤道上总存在一对正对着的点,它们的**温度相同**。这个结论也适用于任何圆周。注意这依赖于温度在空间中**连续变化**这一假设,而这在现实中不一定成立【1】。

当 $ n = 2 $:可以解释为地球表面在任意时刻总存在一对对踵点,它们的**温度和气压完全相同**(假设这两个物理量都在空间中连续变化)。

与奇函数等价的其他表述:记$ S^n $为 n 维球面,\( B^n \) 为 n 维单位球体:

- 若  
  \[
  g: S^n \to \mathbb{R}^n
$$

是一个**连续奇函数**(即 $g(-x) = -g(x)$),则存在

$$
x \in S^n
$$

使得

$$
g(x) = 0
\]。

- 若  
\[
g: B^n \to \mathbb{R}^n
$$

是一个连续函数,且在边界 $S^{n-1}$ 上为奇函数,那么必存在

$$
x \in B^n
$$

使得

$$
g(x) = 0
\]。
$$
