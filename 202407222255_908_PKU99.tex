% 北京大学 1999 年 考研 量子力学
% license Usr
% type Note

\textbf{声明}:“该内容来源于网络公开资料,不保证真实性,如有侵权请联系管理员”



1. (25分) 简要回答以下问题\\
(a) 简述“不确定原理”(测不准关系),说明其意义。\\
(b) 试述“态的叠加原理”,说明其意义。\\
(c) 全同粒子有什么特性?对波函数有什么要求?举例说明之。\\

2. (10分) 已知在 $\hat{J}$ 和 $\hat{L}_z$ 的共同表象中, $\hat{L}_x = \frac{\hbar}{\sqrt{2}} \begin{pmatrix} 
0 & 1 & 0 \\
1 & 0 & 1 \\
0 & 1 & 0 
\end{pmatrix}$, 试求其本征值和本征函数,并写出在自身表象中的矩阵表示。

3. (15分) 一个原子在$z$向磁场$B$中,除了能级的塞曼分裂外,还受到 $\Delta \hat{H}_d = \frac{\mu_B^2}{2c^2a_0} B^2 n^2 \sin^2 \theta$ (c.g.s) 的微扰,\\
(a)] 已知H原子基态, $\Psi (1s) = \frac{1}{\sqrt{\pi a_0^3}} e^{-r/a_0}$, 求一级微扰能 $ \Delta E_d$。\\
(b)] 估计这项修正的量级(设$B=10^4$高斯),与塞曼分裂( $\mu_B B$ 量级)比较。\\
(c)] 分析这个修正的物理意义。\\

4. (15分) 氢原子基态 $I^2S_{1/2}$,氢原子核的自旋 $I=1/2$,核自旋与电子相互作用使能级产生超精细分裂。已知超精细相互作用哈密顿量是 $\Delta\hat H = A \vec{I} \cdot \vec{J}$,式中 $\vec{I}, \vec{J}$ 分别是核自旋角动量和电子总角动量,A是常数\\
(a) 用 $I J F$ 和表示总角动量 ( $\vec{F} = \vec{I} + \vec{J}$)\\
(b) 目前 $I = 1/2, J = 1/2$,求基态 $I^2S_{1/2}$ 能级的超精细分裂,并作图表示之。\\

5. (15分) 在一维无限深势阱 $V(x) = 
\begin{cases} 
0 & 0 \leq x \leq a \\\\
\infty & x < 0, 0 > a 
\end{cases}$
中

(a) 求一个粒子在此势阱中的能量本征值及相应的本征函数。

(b) 一个粒子开始时处于基态,如突然使势阱宽度扩展为 $2a$,问该粒子在扩展后仍处于基态的几率是多少?