% 圆周运动:为什么卫星不会掉下来(科普)
% license Usr
% type Tutor


\begin{issues}
\issueDraft
\end{issues}

\subsection{牛顿大炮:用抛物线近似圆周运动}
牛顿有一个很形象的思想实验: 若不考虑大气阻力,大炮在水平方向发射的炮弹开始时近似做抛物线,由于地球是圆的,但当炮弹的初速度足够大时,它将飞出地球。 那么在落向地面和飞出地球之间就很可能存在一个临界速度,可以使得炮弹绕地球做圆周运动(不考虑大气阻力)。

\addTODO{配图}

那么使用所谓的微元法\footnote{高中经常把微积分中不断细分的思想称为 “微元法”,但要严格地证明这样的方法是对的,应该学习微积分。},我们是否可以认为当炮弹做圆周运动时,它在每一小段可以近似看作是在做平抛运动(水平发射的抛物线)呢?答案是可以的,且这些小段分割得越短,这个近似就越准确。


\begin{figure}[ht]
\centering
\includegraphics[width=8cm]{./figures/e7cd9a57108af76d.pdf}
\caption{假设炮弹做平抛运动,当飞出一段距离后,稍微调整重力的方向,并以此时速度为初速度再次做平抛运动。} \label{fig_CMintr_1}
\end{figure}

\begin{figure}[ht]
\centering
\includegraphics[width=6cm]{./figures/5ff4dea9b3eb717b.pdf}
\caption{当平抛运动重复多次,我们就得到了近似的圆周运动。 每次抛物运动的时间越短结果就越精确。} \label{fig_CMintr_2}
\end{figure}




这就是为什么月亮和人造卫星不会掉下来。 从某种意义上他们每时每刻都在往下掉!但这要相对于直线运动而言——如果没有地球的引力,他们将做直线运动飞出地球。

通过这种方式,我们甚至可以推导出圆周运动的向心加速度,过程见最后。


\subsection{割圆术: 用多边形运动推导圆周运动}
另一种用微元法解释圆周运动的方式是使用线段代替圆。 这也是古代人在计算圆周率时使用的 “割圆术”。

想象看似做圆周运动的质点实际上实在沿着一个正 $N$ 边形做匀速运动。显然,在每个线段中匀速运动时,根据牛顿第一定律,质点不需要受任何外力。 但在拐点处,我们可以想象它和一面墙发生了完全弹性碰撞(也就是没有能量损失的碰撞,类似镜面反射,碰撞后速度不变)。 每次这样的碰撞中,墙壁都正面向圆心的方向。

当多边形的边数变得足够多时,多边形逐渐逼近圆形,且不连续的碰撞对球产生的脉冲力也可以近似看成是连续的向心力了。 如果把这些力平均一下,同样可以推导出向心加速度的公式。

\addTODO{图未完成}

\subsection{由平抛运动推导向心加速度}
通过这种方式,我们甚至可以推导出圆周运动的向心加速度。且这个加速度就是平抛运动中物体向下的加速度。

每个抛物运动中,若时间为 $\Delta t$,则转过的角度为
\begin{equation}
\Delta\theta = \tan^{-1}\frac{g\Delta t}{v} \approx \frac{g\Delta t}{v}~.
\end{equation}
当 $\Delta\theta$ 变小时, 式中的近似会越来越精确。

每段抛物线的长度近似为
\begin{equation}
\Delta l \approx v \Delta t~.
\end{equation}
同样,当 $\Delta\theta$ 变小时, 近似会越来越精确。

注意 $\Delta\theta$ 不但是炮弹转过的角度,也是炮弹绕圆心旋转的角度。

另外考虑到圆的弧长公式 $\Delta l = R\Delta\theta$,那么
\begin{equation}
R = \frac{\Delta l}{\Delta\theta} \approx \frac{v^2}{g}~.
\end{equation}
也就是说
\begin{equation}
g \approx \frac{v^2}{R}~.
\end{equation}
而 $g$ 就是重力加速度,也就是圆周运动的向心加速度。这就是我们高中熟悉的圆周运动向心加速度的公式
\begin{equation}
a = \frac{v^2}{R}~.
\end{equation}

\subsection{由多边形推导向心加速度}
为了更好地理解为何向心力会产生圆周运动, 我们可以假设一个小球在光滑水平面运动的过程中被反弹若干次, 使得它的轨道称为闭合的正 $N$ 边形, 每个顶点距离中心为 $R$。 每次撞击后, 速度不变, 方向改变 $2\theta  = 2\pi/N$。
\addTODO{图}
所以在每次碰撞时, 我们可以把速度垂直分解为径向(指向圆心)和角向速度。 碰撞后角向速度不变, 径向速度增量为 $2v\sin\theta$, 即动量改变为
\begin{equation}
\Delta p = 2mv\sin\theta~.
\end{equation}
正 $N$ 边形每条边的长度为 $L = 2R\sin(\theta)$, 所以相邻两次碰撞的时间为
\begin{equation}
\Delta t = \frac{2R\sin\theta}{v}~.
\end{equation}
我们可以令 $N \to \infty$, 这样, 多边形轨迹就趋近于圆周运动, 每次撞击时, 冲量趋近于 0, 时间间隔也趋近于 0, 我们就可以把这些离散的撞击看成时一个连续的力。 平均来说, 力的大小等于动量改变除以时间
\begin{equation}
F_c = \frac{\Delta p}{\Delta t} = \frac{mv^2}{R} = m\omega^2 R~,
\end{equation}
注意该式甚至不需要取极限 $\Delta t \to 0$ 就可以成立。
