% 氢原子隧道电离

\footnote{参考 Wikipedia \href{https://en.wikipedia.org/wiki/Tunnel_ionization}{相关页面}.}本文使用原子单位制\upref{AU}.

氢原子的电离率(单位时间的概率)在 $E \ll E_a$ 时为
\begin{equation}
\omega = \frac{4}{\abs{E}} \exp[-\frac{2}{3\abs{E}}]
\end{equation}


注意若给氢原子的哈密顿算符添加恒定电场项 $\bvec E \vdot \bvec r$, 那么该系统将不存在严格的束缚态. 因为当 $\bvec r$ 和 $\bvec E$ 夹角大于 $90^\circ$ 且当 $r\to\infty$ 时 $\bvec E \vdot \bvec r \to -\infty$ (\autoref{Htunnl_fig1} ). 所以隧道电离不存在严格的\textbf{阈值(threshold)},理论上任何强度的恒定电场都会产生隧道电离.

\begin{figure}[ht]
\centering
\includegraphics[width=12cm]{./figures/Htunnl_1.png}
\caption{隧道电离示意图} \label{Htunnl_fig1}
\end{figure}

假设氢原子开始时处于某束缚态且无外电场, 能量为 $E_n$, 经过一段时间后外电场出现(图中向左), 这时在原子核的右边就可能会出现一个势垒, 局部势能大于 $E_n$, 但随着 $r$ 增大势能最终小于 $E_n$. 这时根据隧道效应,波函数会以一定的速率穿过该势垒, 单位时间穿过势垒的波函数的概率就叫做\textbf{电离率(ionization rate)}. 电场越弱, 势垒越高越宽, 电离率就越小. 相反, 若电场太强, 使得右边的势能突起比 $E_n$ 要小, 那么同样不存在隧道效应, 而是直接电离.
