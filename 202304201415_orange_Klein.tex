% 克莱因-戈登传播子
% 克莱因|戈登|传播子
\pentry{标量场的量子化\upref{quanti}}

相对论的因果性要求:对于任意两个类空间隔的时空点 $x,y$,在两处进行的任何物理测量都是没有相互影响的。对于 Klein-Gordon 场,我们要求在类空间隔的两个点上 $\phi(x)$ 和 $\phi(y)$ 对易,即原则上是可以同时测量的。
\begin{equation}
[\phi(x),\phi(y)]=0,\forall x,y:(x-y)^2<0~.
\end{equation}
下面我们通过计算验证这一结果。利用 \autoref{eq_quanti_3}~\upref{quanti} 及产生湮灭算符的对易关系进行计算,可以得到
\begin{equation}
[\phi(x),\phi(y)]=\int \frac{\dd{}^3 \bvec p}{(2\pi)^3} \frac{1}{2\omega_{\bvec p}}
\left( e^{-ip(x-y)}-e^{ip(x-y)}\right)
\end{equation}
注意上式中的两项各自都是洛伦兹不变的,并且各自都不等于 $0$。由于 $\dd{}^3 \bvec k/((2\pi)^3 2\omega_{\bvec k})$ 是洛伦兹不变的积分测度,所以对易子 $[\phi(x),\phi(y)]$ 也是洛伦兹不变的,这也可以通过 \autoref{eq_quanti_4}~\upref{quanti} 验证。对于类空间隔的两个点 $x,y$,可以通过洛伦兹变换使 $x,y$ 互换,从而使上式的第二项变成与第一项完全相同,因此两项的差恒等于 $0$,即对易子等于 $0$。对于类时间隔的两个点 $x,y$,不能通过洛伦兹变换做到这一点,因此一般来说对易子不等于 $0$。



我们现在来看 $[\phi(x),\phi(y)]$ 这个量。因为这个量是一个c数,所以我们有
\begin{equation}
[\phi(x),\phi(y)] = \langle 0 | [\phi(x),\phi(y)] | 0 \rangle
\end{equation}
我们可以用四维形式来表示它。我们首先假设 $x^0>y^0$.
\begin{equation}\label{eq_Klein_1}
\begin{aligned}
& \langle 0|[\phi(x), \phi(y)]| 0\rangle\\
=&\int \frac{d^{3} p}{(2 \pi)^{3}} \frac{1}{2 E_{\mathbf{p}}}\left(e^{-i p \cdot(x-y)}-e^{i p \cdot(x-y)}\right) \\ 
=& \int \frac{d^{3} p}{(2 \pi)^{3}}\left\{\left.\frac{1}{2 E_{\mathbf{p}}} e^{-i p \cdot(x-y)}\right|_{p^{0}=E_{\mathbf{p}}}+\left.\frac{1}{-2 E_{\mathbf{p}}} e^{-i p \cdot(x-y)}\right|_{p^{0}=-E_{\mathbf{p}}}\right\} \\
=& \int \frac{d^{3} p}{(2 \pi)^{3}} \int \frac{d p^{0}}{2 \pi i} \frac{-1}{p^{2}-m^{2}} e^{-i p \cdot(x-y)} 
\end{aligned}
\end{equation}
\begin{figure}[ht]
\centering
\includegraphics[width=14cm]{./figures/66d7e95ac59a97f1.png}
\caption{在最后一步中,$p^0$ 的积分围道如图所示。} \label{fig_Klein_1}
\end{figure}
对于 $x^0>y^0$ 我们从下面闭合围道,包围两个极点。对于 $x^0<y^0$ 我们从上面闭合围道,结果是0。因此\autoref{eq_Klein_1} 的的最后一行以及闭合围道的办法可以用下式表示
\begin{equation}
D_R(x-y)\equiv \theta(x^0-y^0)\langle 0 | [\phi(x),\phi(y)]|0 \rangle
\end{equation}
可以证明,$D_R(x-y)$ 满足
\begin{equation}
(\partial^2+m^2)D_R(x-y) = -i \delta^{(4)}(x-y)
\end{equation}
$D_R(x-y)$ 在 $x^0<y^0$ 时为零,所以被称为\textbf{延迟格林函数}。对延迟格林函数进行傅里叶变换可得。
\begin{equation}
D_R(x-y) = \int \frac{d^4p}{(2\pi)^4} e^{-ip(x-y)} \tilde D_R (p)
\end{equation}
其中动量空间的格林函数 $\tilde D_R(p)$ 满足下式
\begin{equation}\label{eq_Klein_2}
(-p^2+m^2) \tilde D_R(p) = -i
\end{equation}
由\autoref{eq_Klein_2} 可推出
\begin{equation}
D_R (x-y) = \int \frac{d^4p}{(2\pi)^4} \frac{i}{p^2-m^2} e^{-ip\cdot(x-y)}
\end{equation}
积分围道如图所示。这种积分围道被称为\textbf{Feynman prescription}。一种方便的记法是写为
\begin{equation}
D_F(x-y) \equiv \int \frac{d^4p}{(2\pi)^4} \frac{i}{p^2-m^2+i\epsilon} e^{-ip\cdot(x-y)}
\end{equation}
极点分别位于 $p^0 = \pm(E_{\mathbf p}-i\epsilon)$。当 $x^0>y^0$ 的时候,我们可以从下方闭合 $p^0$ 的积分。当 $x^0<y^0$ 的时候,我们可以从上面闭合积分。
\begin{figure}[ht]
\centering
\includegraphics[width=14cm]{./figures/b7fb430aadb27d55.png}
\caption{Feynmann prescription的积分围道。} \label{fig_Klein_2}
\end{figure}
费曼传播子的定义是
\begin{equation}\label{eq_Klein_3}
\begin{aligned}
D_{F}(x-y) &= \begin{cases}D(x-y) & \text { for } x^{0}>y^{0} \\
D(y-x) & \text { for } x^{0}<y^{0}\end{cases} \\
&=\theta\left(x^{0}-y^{0}\right)\langle 0|\phi(x) \phi(y)| 0\rangle+\theta\left(y^{0}-x^{0}\right)\langle 0|\phi(y) \phi(x)| 0\rangle \\
& \equiv\langle 0|T \phi(x) \phi(y)| 0\rangle
\end{aligned}
\end{equation}
最后一行定义了时间顺序算符 $T$。我们可以在 $D_F(x-y)$ 左边加上 $(\partial^2+m^2)$,你可以验证 $D_F$ 是克莱因戈登算符的格林函数。$D_F(x-y)$ 被称为克莱因戈登粒子的费曼传播子。

