% 指数函数(高中)
% keys 指数|指数函数|自然常数
% license Usr
% type Tutor

\begin{issues}
\issueDraft
\end{issues}

\pentry{函数\nref{nod_functi},函数的性质\nref{nod_HsFunC},幂运算与幂函数\nref{nod_power}}{nod_d767}

在幂运算的基础上。

\subsection{自然常数$e$}

这里要先介绍一个特殊的常数$e \approx 2.71828$。它和早已在小学时就接触过的$\pi$有许多相似的性像。

他们都是无理数,这意味着它们不能表示为两个整数的比值。它们的小数部分是无限且不循环的,也就是说,在任何整数进制中它们都永远不会终止或重复。

他们也都是超越数,意思是它们不能作为任何有理系数多项式方程的解。换句话说,它们不能通过根式表示。这比无理数的要求更加严格。$e$由查尔斯·埃尔米特(Charles Hermite)在1873年证明,$\pi$由费迪南德·冯·林德曼(Ferdinand von Lindemann)在1882年证明。

二者都可以用无穷展开的方式来表示,下面给出两个常见的展开方式\footnote{关于求和符号可以参考\enref{求和符号(高中)}{SumSym},关于阶乘可以参考\enref{阶乘(高中)。}{factor}}:
\begin{equation}
\pi=4\sum_{n=0}^\infty\frac{(-1)^i}{2i+i}~.
\end{equation}
\begin{equation}
e=\sum_{n=0}^\infty\frac{1}{i!}~.
\end{equation}

$e$的定义有很多种方式,极限定义是被广为了解的,但它有些抽象。本文将另给出一个定义,这个定义比较简单,但在了解微分运算之后,才能体会这个定义的简洁。
假设某个初始值进行增长,随着增长次数变得无限多且增长的频率越来越频繁,最终增长的极限值是 e。

\begin{equation}
e = \lim_{n \to \infty} \left( 1 + \frac{1}{n} \right)^n~.
\end{equation}


微积分中的自然对数函数:$e$ 是使得
\begin{equation}
\frac{d}{dx} f(x) = f(x)~.
\end{equation}
成立的指数函数的底数,意味着以 $e$ 为底的指数函数是唯一的保持自身斜率不变的增长函数。

\subsection{指数函数}

回看幂运算的\aref{定义}{def_power_1},如果将底数作为参数,指数作为自变量的函数就称为指数函数,指数函数的名称指的就是自变量的在指数位置上,注意不要与幂函数相混淆。

\begin{definition}{指数函数}
形如
\begin{equation}
f(x) = a^x~.
\end{equation}
的函数称作\textbf{指数函数(exponential function)},其中 $a\in\mathbb R^+$。
\end{definition}

$f(x)>0$恒成立。

\subsection{指数函数的性质}

同样,根据a的性质讨论指数函数的性质。

恒过定点$(0,1)$。

指数函数在定义域上是单调的,
增减则根据a的性质有区分
单调性、无界性和始终为正值

\subsection{指数爆炸}

指数爆炸意味着一个函数值随自变量呈指数级别的快速增长,它的显著特征是初期增速缓慢,但随后会急剧加速。指数函数的增长速度非常快,对于初等函数而言,当参数 x 足够大时,指数函数的增长速度是最快的。具体来说,若参数 $a > 1$,在第一象限内($x > 0$)的典型函数增长速度从慢到快通常满足以下顺序:

\begin{equation}
 a < \log_a{x} <x^a < a^x~.
\end{equation}

式子中,常数 $a$ 是一个固定值,不随 $x$ 改变,或者说不增加。对数函数 $\log_a{x}$ 的值在 $x$越大时,仍在增加,但增速会越来越慢,仅略大于不增。幂函数 $x^a$ 和指数函数 $a^x$的增长速度都会随着 $x$ 增加,$x^a$ 增速逐渐加快,但$x^a$比指数函数 $a^x$ 慢。指数增长会呈现“爆炸式”的加速,远超其他初等函数。

\subsection{柯西函数方程}

事实上,指数函数就是满足柯西函数方程$f(x+y)=f(x)f(y)$的一个解。