% 太阳能电池
% license CCBYSA3
% type Wiki

(本文根据 CC-BY-SA 协议转载自原搜狗科学百科对英文维基百科的翻译)

\begin{figure}[ht]
\centering
\includegraphics[width=6cm]{./figures/af8830ce3bab8b15.png}
\caption{传统的晶体硅太阳能电池。由汇流条(较大的银色带)和手指(较小的带)制成的电触点印刷在硅片上。} \label{fig_TYNDC_1}
\end{figure}

太阳能电池,或称光伏电池,是一种通过光伏效应将光能直接转化为电能的电力设备,光伏效应是一种物理和化学现象。 它是光电池的一种形式,被定义为当暴露在光线下时,其电特性如电流、电压或电阻会发生变化的装置。单个太阳能电池装置可以组合成模块,也称为太阳能电池板。 基本上,单结硅太阳能电池可以产生大约0.5至0.6伏的最大开路电压。[1]

太阳能电池被描述为光伏电池,无论其来源是阳光还是人造光。它们被用作光电探测器(例如红外探测器),探测可见光范围内的光或其他电磁辐射,或测量光强。

光伏电池的运行需要三个基本属性:
\begin{itemize}
\item 光的吸收,产生电子空穴对或激子。
\item 相反类型电荷载流子的分离。
\item 将这些载流子单独提取到外部电路中。
\end{itemize}

相比之下,太阳能集热器是通过吸收阳光来提供热量,用于直接加热或间接发电。另一方面,“光电分解电池”(光电化学电池)是指一种光伏电池(像亚历山大·爱德蒙·贝克勒尔和现代染料敏化太阳能电池开发的),或者是指一种仅利用太阳能光照将水直接分解成氢和氧的装置。

\subsection{应用}



\subsubsection{1.1 电池、模块、面板和系统}



\subsection{历史}



\subsubsection{2.1 空间应用}



\subsubsection{2.2 降价}



\subsubsection{2.3 研究和工业生产}



\subsection{成本下降和指数增长}



\subsubsection{3.1 补贴和电网平价}



\subsection{理论}



\subsection{效率}



\subsection{材料}



\subsubsection{6.1 结晶硅}



\textbf{单晶硅}



\textbf{外延硅发展}



\textbf{多晶硅}



\textbf{带状硅}



\textbf{类单晶硅多硅(MLM)}



\subsubsection{6.2 薄膜}



\textbf{碲化镉}



\textbf{铜铟硒化镓薄膜电池}



\textbf{硅薄膜}



\textbf{砷化镓薄膜}



\subsubsection{6.3 多结电池}



\textbf{GaInP/Si双结太阳能电池}



\subsection{太阳能电池研究}

\subsubsection{7.1 钙钛矿太阳能电池}



\subsubsection{7.2 双面太阳能电池}



\subsubsection{7.3 中间带}



\subsubsection{7.4 液体墨水}



\subsubsection{7.5 上转化和下转化}



\subsubsection{7.6 吸光染料}



\subsubsection{7.7 量子点}



\subsubsection{7.8 有机/聚合物太阳能电池}



\subsubsection{7.9 自适应电池}



\subsubsection{7.10 表面纹理}



\subsubsection{7.11 包装}



\su