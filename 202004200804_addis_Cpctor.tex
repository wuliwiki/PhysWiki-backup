% 电容
% 电容量|导体|电势差|电压

\pentry{电势 电势能\upref{QEng}}

\subsubsection{单个导体的电容量}

一个导体的电容等于电荷量除以电势


\begin{example}{导体球的电容}
若规定无穷远处为零势点, 由高斯定理, 半径为 $R$, 带电为 $Q$ 的导体球的电势为 $V = Q/(4\pi\epsilon_0 R)$, 所以其电容为
\begin{equation}\label{Cpctor_eq1}
C = \frac{Q}{V} = 4\pi\epsilon_0 R
\end{equation}
\end{example}


\subsubsection{两导体之间的电容量}
若两导体带等量异种电荷\footnote{这里的 $Q$ 不一定要求大于零, $\pm$ 号只是一个用于区分两个导体的记号而已, 不代表电荷的正负.} $\pm Q$, 电势差为 $V$, 则两导体间的电容量同样被定义为
\begin{equation}
C = \frac{Q}{V}
\end{equation}
现在来证明两导体间的电容只与他们的形状, 相对位置以及空间中的电介质分布有关, 而与电荷量无关. 这就要求证明电势差始终与电荷 $Q$ 成正比, 如果我们假设量导体表面的电荷面密度 $\sigma$ 始终与 $Q$ 成正比(证明见%未完成, 参考 Griffiths
, 那么由库仑定律% 引用公式未完成,引用对 \rho 三重积分的那条
可知空间中任意一点的场强也与 $Q$ 成正比. 而电势差
\begin{equation}
V = \int_+^- \bvec E(r) \cdot \dd{\bvec l}
\end{equation}
(式中的 “$\pm$” 分别代表带电荷量为 $\pm Q$ 的导体表面上的一点, 积分路径任意选取)和场强成正比也就是和 $Q$ 成正比. 证毕.

\begin{example}{平行板电容器}\label{Cpctor_exe2}
% 图未完成
电容器中最常见也是最基本的模型就是平行板电容器, 我们假设空间中存在均匀的电介质, 介电常数为 $\epsilon$ 空间中两块面积为 $S$ 的方形导体板相距为 $d$ 平行放置, 带电量分别是 $\pm Q$, 如果忽略边缘效应(即假设只有两板之间的长方体空间中存在匀强电场), 由高斯定理,%引用例题未完成
两板之间的电场为 $E = {\sigma}/{\epsilon} = Q/(S\epsilon)$, 电势差为 $V = Ed = Qd/(S\epsilon)$, 所以电容等于
\begin{equation}\label{Cpctor_eq4}
C = \frac{Q}{V} = \epsilon \frac sd
\end{equation}
可见在相同介质中, 平行板电容器的电容量与板的面积成正比, 与距离成反比, 比例系数为 $\epsilon$. 这就是为什么 $\epsilon$ 也被称为电容率. 当不存在电介质时, $\epsilon = \epsilon_0$, 所以 $\epsilon_0$ 被称为真空中的电容率.
\end{example}

\begin{example}{同心球壳电容器}
若有两个半径分别为 $r$ 和 $R$ 的同心球壳导体 ($r < R$), 电荷量分别为 $+Q$ 和 $-Q$, 求两球壳间的电容量.

由高斯定理%未完成:最好给出例题
两球壳间的电场只与小球壳的电荷量有关, 所以两球壳的电势差为
\begin{equation}
V = \frac{Q}{4\pi\epsilon_0} \qty(\frac 1r - \frac 1R)
\end{equation}
所以电容量为
\begin{equation}\label{Cpctor_eq6}
C = \frac QV = \frac{4\pi\epsilon_0 rR}{R - r}
\end{equation}
当 $r\to R$ 时, 球壳的面积为 $S = 4\pi R^2$, 两板间距为 $d = R - r$, 则上式的电容趋于平行板电容器的电容.
\end{example}

对比\autoref{Cpctor_eq6} 和\autoref{Cpctor_eq1} 可以发现两个足够近的导体获得的电容量要比单个导体大得多, 这也是为什么电路中的电容都有两个极.

\subsection{电压与电流的关系}
若电容器两端的电压随时间变化为 $V(t)$, 那么如何计算流过电容器的电流\footnote{我们在讨论电路的时候习惯性地说电流“流过电容器”, 而事实上并没有电荷在电容器内部从一极移动到另一极.}呢? 如果我们定义从 $+Q$ 极流入从 $-Q$ 极流出的电流方向为正方向, 那么 $I = \dv*{Q}{t}$, 对 $Q = CV$ 两边求时间导数得
\begin{equation}
I = C\dv{V}{t}
\end{equation}
\begin{exercise}{}
试证明在交流电路中, 流经电容的电流的相位比电压的相位提前 $\pi/2$.
\end{exercise}

\subsection{电容器的能量}
为了改变电容器的电荷量 $Q$ , 我们需要将电荷从一个导体移动到另一个导体, 为了不影响电荷及电场的分布, 从 $Q = 0$ 开始, 我们每次只从 $-Q$ 移动极少量的电荷 $\Delta Q$ 到 $+Q$,移动的过程中外力需要克服电场力做功 $V\Delta Q$, 而 $V$ 又是 $Q$ 的函数, 第 $i$ 次移动前, 电荷量为 $\pm Q_i$, 由定积分\upref{DefInt} 的思想
\begin{equation}
W = \int V(Q) \dd{Q} = \int \frac{Q}{C} \dd{Q} = \frac12 \frac{Q^2}{C} = \frac12 CV^2
\end{equation}
可见电容器的能量与电荷量或电势的平方成正比. 注意电容器的能量本质上是导体表面上连续电荷分布的电势能, 所以另一种计算上式的方法就是直接使用\autoref{QEng_eq8}\upref{QEng} 得
\begin{equation}
W = \frac12 [QV_+ + (-Q)V_-] = \frac12 QV = \frac12 \frac{Q^2}{C} = \frac12 CV^2 
\end{equation}
