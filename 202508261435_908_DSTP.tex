% 代数拓扑(综述)
% license CCBYSA3
% type Wiki

本文根据 CC-BY-SA 协议转载翻译自维基百科\href{https://en.wikipedia.org/wiki/Algebraic_topology}{相关文章}

\begin{figure}[ht]
\centering
\includegraphics[width=6cm]{./figures/1e3777d61cc0bb5a.png}
\caption{环面,代数拓扑中最常被研究的对象之一。} \label{fig_DSTP_1}
\end{figure}
代数拓扑是数学的一个分支,它利用抽象代数的工具研究拓扑空间。其基本目标是寻找能够刻画拓扑空间的代数不变量,从而将这些空间按同胚分类,尽管在实际研究中,大多数情况下是按同伦等价进行分类的。

虽然代数拓扑主要是用代数方法研究拓扑问题,但有时也可以利用拓扑方法来解决代数问题。例如,代数拓扑就能给出一个简便的证明:自由群的任意子群仍然是自由群。
\subsection{主要分支}
以下是代数拓扑中研究的一些主要方向:
\subsubsection{同伦群}
在数学中,同伦群用于代数拓扑中对拓扑空间进行分类。第一个也是最简单的同伦群是基本群,它记录了一个空间中环路的信息。直观地说,同伦群记录了拓扑空间的基本形状或“洞”的信息。
\subsubsection{同调}
在代数拓扑和抽象代数中,同调(部分源自希腊语“ὁμός homos”,意为“相同”)是一种通用方法,用于将一个给定的数学对象(例如拓扑空间或群)关联到一个阿贝尔群或模的序列中。
\subsubsection{上同调}
在同调理论和代数拓扑中,上同调是一个总称,指的是由上链复形定义的一系列阿贝尔群。换句话说,上同调是一种对上链、上循环和上边界的抽象研究。上同调可以看作是一种赋予拓扑空间代数不变量的方法,这些不变量具有比同调更精细的代数结构。上同调起源于同调构造的代数对偶化过程。用不那么抽象的语言来说,上链从根本意义上讲,应该是为同调理论中的链赋予某种“量”的方法。
\subsubsection{流形}
流形是一种拓扑空间,它在每个点附近都与欧几里得空间相似。常见的例子有平面、球面和环面,这些都可以在三维空间中实现;还有克莱因瓶和实射影平面,它们无法嵌入三维空间,但可以嵌入四维空间。通常,代数拓扑中的研究结果聚焦于流形的整体性、不可微分的性质;例如庞加莱对偶性。
\subsubsection{结理论}
结理论研究的是数学中的“结”。虽然它的灵感来源于日常生活中鞋带或绳索打结的现象,但数学上的“结”有所不同,其两端是连在一起的,因此无法简单地解开。用精确的数学语言表述,一个结是将一个圆嵌入三维欧几里得空间$\mathbb{R}^3$中的映射。如果两个数学结可以通过对 $\mathbb{R}^3$ 本身进行连续变形(称为“环境同胚”或 ambient isotopy)而互相转化,则它们被视为等价的。这些变形对应于我们操作一根绳子时,绳子没有被剪断或穿过自身的过程。
\subsubsection{复形}