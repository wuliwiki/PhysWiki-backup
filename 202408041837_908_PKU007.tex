% 北京大学 2007 年 考研 普通物理
% license Usr
% type Note

\textbf{声明}:“该内容来源于网络公开资料,不保证真实性,如有侵权请联系管理员”

\subsection{(16分)}
假设质点沿平面极坐标系中的圆周曲线 $r = 2R \cos \theta$ 运动(其中 $R$ 为大于零的常数,$\left| \theta \right| < \pi/2$),而且二倍面积速度 $r^2 \dot{\theta} = h > 0$ 为常量。试求(用常量 $R, h$ 和 $r$ 表示):
\begin{enumerate}
\item 质点所受的合外力;
\item 质点的切向加速度和法向加速度,
\end{enumerate}
\subsection{(16分)}
质量为$M$、倾斜角为$\theta$的大木块静止于光滑水平面上。$t=0$时把质量为$m$的小木块置于大木块的光滑斜面的顶端,然后令其从静止开始沿斜面下滑,试求:
\begin{enumerate}
\item 大木块的加速度
\item 小木块相对于大木块的加速度
\item 大小木块之间的相互作用力
\item 大木块对地面的压力
\end{enumerate}
\subsection{(18分)}
半径为$r$、质量为m的匀质小球体在半径为$R(>r)$的半球内表面从静止开始作无滑滚动,初始时小球球心与大球球心的连线与竖直线的夹角为$\theta$=60°。
\begin{enumerate}
\item 求小球滚到最低点时小球的转动角速度和其质心的速度:
\item 求$\theta$所满足的二阶微分方程,并求小球在最低点附近做小振动的周期。已知小球体对于过其球心的轴的转动惯量为$I=2mr^2/5$.
\end{enumerate}
\subsection{15分}
三个点电荷的带电体系,电荷带电量分别为$q_1,q_2,q_3,$电荷所在处的电势分别为么$u_1,u_2,u_3$。证明该带电体系的静电能为$(q_1u_1,q_2,u,+qu;)/2。