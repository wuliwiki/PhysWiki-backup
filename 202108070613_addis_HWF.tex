% 类氢原子的束缚态
% 氢原子|波函数|束缚态|薛定谔方程

\pentry{球坐标系中的定态薛定谔方程\upref{RadSE}, 原子单位制\upref{AU}} % 预备知识未完成

本文使用原子单位制\upref{AU}.\textbf{类氢原子(hydrogen-like atom)}被定义为原子核有 $Z$ 个质子(核电荷为 $+Ze$) 有一个核外电子的原子/离子, 例如氢原子和失去一个电子的氦原子 $\mathrm{He}^+$, 失去两个电子的锂离子 $\mathrm{Li}^{++}$. % 这个定义应该放在玻尔模型里面

类氢原子的定态薛定谔方程为
\begin{equation}\label{HWF_eq1}
-\frac{1}{2m} \laplacian \psi(\bvec r) - \frac{Z}{r} \psi(\bvec r) = E \psi(\bvec r)
\end{equation}
类氢原子是唯一存在解析解的原子(离子).

我们这里只讨论束缚态, 即 $E < 0$ 的解.  从数学上, $E$ 取小于零的任意值时我们都能找到解, 但只有当 $E$ 取特定离散值的时候这些波函数才能归一化(否则没有物理意义). 由于类氢原子具有球对称性, 球坐标下的波函数具有最简单的形式. 波函数的表达式为
\begin{equation}\label{HWF_eq3}
\psi_{nlm} (r,\theta ,\phi) = R_{nl}(r) Y_{l,m}(\theta, \phi)
\end{equation}
其中 $n$ 是\textbf{主量子数}($n = 1, 2, \dots$), $l$ 是\textbf{角量子数}($l = 0, 1, \dots, n - 1$), $m$ 是\textbf{磁量子数}($m = -l, -l+1, \dots, l$). $R_{nl}(r)$ 是归一化的\textbf{径向波函数}, $Y_{l,m}(\theta, \phi)$ 是归一化的\textbf{球谐函数}(见“球谐函数列表\upref{YlmTab}”).

\begin{figure}[ht]
\centering
\includegraphics[width=14.25cm]{./figures/HWF_2.png}
\caption{氢原子波函数的概率密度 $\abs{\psi_{nlm}(\bvec r)}^2$ 的 $x$-$z$ 截面, 大小成比例. 每个图中的三个数字分别是量子数 $n, l, m$, 电子出现在白色圆圈内部的概率为 0.95. 色条对应的数值是线性的, 每个子图中的色条取值范围不相同. 源码见 “氢原子波函数画图程序(Matlab)\upref{Hplot}”.} \label{HWF_fig2}
\end{figure}

\subsubsection{径向波函数}
\begin{equation}\label{HWF_eq2}
R_{nl}(r) = \sqrt{\qty(\frac{2 Z}{n})^3 \frac{(n - l - 1)!}{2n (n + l)!}} \qty(\frac{2Zr}{n})^l  L_{n-l-1}^{2l+1}\qty(\frac{2Zr}{n}) \E^{-Zr/n}
\end{equation}
其中 $L_n^l(x)$ 是\textbf{连带拉盖尔多项式(associated Laguerre polynomial)}. 注意 $Z$ 的作用是把径向波函数关于原点收缩 $Z$ 倍, 并保持波函数归一化.

$Z = 1$ 时 $r R_{n,l}(r)$ 的函数图\footnote{我们以后会看到 $r R_{l,m}(r)$ 比 $R_{l,m}(r)$ 更常用}如\autoref{HWF_fig1}
\begin{figure}[ht]
\centering
\includegraphics[width=13cm]{./figures/HWF_1.pdf}
\caption{径向波函数函数图(使用原子单位)} \label{HWF_fig1}
\end{figure}
以下给出前几个径向波函数, 注意所有径向波函数的值都是实数.
\begin{equation} % 已数值验证归一化和正交性
n = 1 \qquad
R_{10}(r) = 2 Z^{3/2}\E^{-Zr}
\end{equation}
\begin{equation}
n = 2 \qquad
\leftgroup{
R_{20}(r) &= \frac{1}{\sqrt 2} Z^{3/2} \qty(1 - \frac12 Zr) \E^{-Zr/2}\\
R_{21}(r) &= \frac{1}{2\sqrt{6}} Z^{3/2} Zr \E^{-Zr/2}
}\end{equation}
\begin{equation}
n = 3 \qquad
\leftgroup{
R_{30}(r) &= \frac{2}{3\sqrt {3}} Z^{3/2} \qty(1 - \frac23 Zr + \frac{2}{27} Z^2r^2) \E^{-Zr/3}\\
R_{31}(r) &= \frac{8}{27\sqrt 6} Z^{3/2} \qty(1 - \frac16 Zr) Zr \E^{-Zr/3}\\
R_{32}(r) &= \frac{4}{81\sqrt {30}} Z^{3/2} Z^2r^2 \E^{-Zr/3}}
\end{equation}
更多 $R_{n,l}$ 表达式可以用 Mathematica\upref{Mma} 或者 \href{https://www.wolframalpha.com/}{Wolfram Alpha} 生成, 函数定义如
\begin{lstlisting}[language=Mathematica]
HydrogenR[Z_, n_, l_, r_] := Sqrt[(2*Z/n)^3*
   Factorial[n-l-1]/(2*n*Factorial[n+l])] (2*Z*r/n)^l*
   LaguerreL[n-l-1, 2l+1, 2Z*r/n]*Exp[-Z*r/n]
Hwf[Z_, n_, l_, m_, r_, \[Theta]_, \[Phi]_] := 
   HydrogenR[Z, n, l, r] SphericalHarmonicY[l, m, \[Theta], \[Phi]]
\end{lstlisting}
或者用 Matlab 进行数值计算
\begin{lstlisting}[language=matlab, caption=hydrogen\_Rnl.m]
function ret = hydrogen_Rnl(Z,n,l,r)
ret = sqrt((2*Z/n)^3*factorial(n-l-1)/(2*n*factorial(n+l))) *...
        (2*Z*r/n).^l .* laguerreL(n-l-1, 2*l+1, 2*Z*r/n) .* exp(-Z*r/n);
end
\end{lstlisting}

\subsection{性质}
我们要求氢原子每个束缚态满足归一化条件
\begin{equation}
\int \abs{\Psi_{n,l,m}(\bvec r)} \dd[3]{r} = \int \int_0^\infty \abs{R_{n,l}(r) Y_{l,m}(\uvec r)}^2 r^2 \dd{r}\dd{\Omega} = 1
\end{equation}
根据\autoref{RadSE_eq5}~\upref{RadSE} 和\autoref{RadSE_eq7}~\upref{RadSE} 有
\begin{equation}
\int_0^\infty R_{n', l}\Cj(r) R_{n,l}(r) r^2 \dd{r} = \delta_{n,n'}
\end{equation}
注意对相同的 $n$ 和不同的 $l$, 径向波函数并不正交.

\subsection{径向概率分布}
我们来求径向概率分布 $P(r)$. $P(r)$ 的定义为: 发现粒子在 $r \in [a, b]$ (厚球壳)内的概率为 $\int_a^b P(r) \dd{r}$. 由于波函数的模长平方就是三维的概率密度, 有
\begin{equation}
\int_a^b P(r) \dd{r} = \int_a^b \int \abs{\frac1r \psi_{l,m}(r) Y_{l,m}(\uvec r)}^2 \dd{\Omega} r^2 \dd r
= \int_a^b \abs{\psi_{l,m}(r)}^2 \dd{r}
\end{equation}
对任意 $a, b > 0$ 都成立, 所以有
\begin{equation}
P(r) = \abs{\psi_{l,m}(r)}^2
\end{equation}

任意波函数可以表示为所有本征波函数的叠加
\begin{equation}
\Psi(\bvec r) = \frac1r \sum_{l,m} \psi_{l,m}(r) Y_{l,m}(\uvec r)
\end{equation}
其径向概率分布为
\begin{equation}\label{HWF_eq4}
\int_a^b P(r) \dd{r} = \int_a^b \int \abs{\frac1r \sum_{l,m}\psi_{l,m}(r) Y_{l,m}(\uvec r)}^2 \dd{\Omega} r^2 \dd r
= \sum_{l,m} \int_a^b \abs{\psi_{l,m}(r)}^2 \dd{r}
\end{equation}
对任意 $a, b > 0$ 都成立, 所以有
\begin{equation}
P(r) = \sum_{l,m} \abs{\psi_{l,m}(r)}^2
\end{equation}

\subsection{动量表象\ 动量分布}
要求动量表象下的波函数, 我们需要将位置表象的波函数投影到归一化的动量的本征矢上, 即三维傅里叶变换
\begin{equation}
\psi_{nlm}(\bvec p) = \braket{\bvec p}{\psi} = \frac{1}{\sqrt{2\pi}} \int \exp(-\I \bvec p \vdot \bvec r) \psi(\bvec r) \dd[3]{r}
\end{equation}
这个积分在球坐标中完成才是最方便的, 具体方法我们将举例子说明(见\autoref{Pl2Ylm_ex1}~\upref{Pl2Ylm}).

正如位置表象下位置的分布函数是 $\abs{\psi(\bvec r)}^2$, 动量表象下动量的分布函数是 $\abs{\psi(\bvec p)}^2$ (也符合测量理论).
