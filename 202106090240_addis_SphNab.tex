% 球坐标系中的矢量算符
% 球坐标系|梯度|散度|旋度|拉普拉斯算符

\pentry{正交曲线坐标系中的矢量算符\upref{CVecOp}}

球坐标系中标量函数 $u(r, \theta, \phi)$ 和矢量函数 $\bvec v(r, \theta, \phi)$ 的梯度, 散度, 旋度和拉普拉斯算符的公式如下. 其中 $r$ 是极径,$\theta $ 是极角,$\phi $ 是方位角. 推导见下文.

\begin{itemize}
\item \textbf{梯度算符}
\begin{equation}\label{SphNab_eq1}
\grad u = \pdv{u}{r}\uvec r + \frac{1}{r} \pdv{u}{\theta}\uvec \theta  + \frac{1}{r\sin \theta }\pdv{u}{\phi}\uvec \phi
\end{equation}
\item \textbf{散度算符}
\begin{equation}\label{SphNab_eq2}
\div \bvec v = \frac{1}{r^2} \pdv{r} (r^2 v_r) + \frac{1}{r\sin \theta} \pdv{\theta} (\sin\theta v_\theta) + \frac{1}{r\sin \theta}\pdv{v_\phi}{\phi}
\end{equation}
\item \textbf{旋度算符}
\begin{equation}\ali{
\curl \bvec v = & \frac{1}{r\sin \theta} \qty[\pdv{\theta} (\sin \theta v_\phi) - \pdv{v_\theta}{\phi}]\uvec r  + \frac1r \qty[\frac{1}{\sin \theta}\pdv{v_r}{\phi} - \pdv{r} (r v_\phi)]\uvec \theta\\
&+ \frac1r \qty[\pdv{r} (r v_\theta) - \pdv{v_r}{\theta}]\uvec \phi
}\end{equation}
\item \textbf{拉普拉斯算符}
\begin{equation}\label{SphNab_eq4}
\laplacian u = \div (\grad u) = \frac{1}{r^2} \pdv{r} \qty(r^2 \pdv{u}{r}) + \frac{1}{r^2 \sin\theta}\pdv{\theta} \qty(\sin \theta \pdv{u}{\theta}) + \frac{1}{r^2 \sin^2 \theta} \pdv[2]{u}{\phi}
\end{equation}
\end{itemize}

\subsubsection{推导}
位置矢量的微分可以表示为
\begin{equation}\label{SphNab_eq12}
\dd{\bvec r} = \uvec r \dd{r} + r\uvec \theta\dd{\theta} + r\sin\theta\uvec \phi\dd{\phi}
\end{equation}
\addTODO{该式的推导应该在正交曲线坐标系\upref{CurCor} 中完成}.
代入\autoref{CVecOp_eq4}~\upref{CVecOp}到\autoref{CVecOp_eq6}~\upref{CVecOp}即可完成推导.


\subsection{拉普拉斯算符的径向和角向分解}
为了书写方便本书中定义两个算符 $\laplacian_r$ 和 $\laplacian_\Omega$ 满足
\begin{equation}\label{SphNab_eq3}
\laplacian = \laplacian_r + \frac{\laplacian_\Omega}{r^2}
\end{equation}
其中 $\laplacian_r$ 只对 $r$ 求偏导(\autoref{SphNab_eq4} 右边第一项), $\laplacian_\Omega$ 只对 $\theta,\phi$ 求偏导(\autoref{SphNab_eq4} 右边后两项). 该分解有助于使用分离变量法求解球坐标系中的拉普拉斯方程\upref{SphLap}.

$\laplacian_\Omega$ 还可以进一步分解为两个矢量算符的点乘
\begin{equation}
\laplacian_\Omega = \grad_\Omega\vdot \grad_\Omega
\end{equation}
其中
\begin{equation}
\begin{aligned}
\grad_\Omega &= \bvec r \cross \grad\\
&= \uvec \phi \pdv{\theta} - \uvec \theta\frac{1}{\sin\theta} \pdv{\phi}\\
&= -\qty(\sin\phi\pdv{\theta} + \cot\theta\cos\phi\pdv{\phi})\uvec x
+ \qty(\cos\phi\pdv{\theta} - \cot\theta \sin\phi \pdv{\phi}) \uvec y
+ \pdv{\phi} \uvec z
\end{aligned}
\end{equation}
该分解在量子力学中有应用, 见 “球坐标系中的角动量算符\upref{SphAM}”.

推导: 只需要把 $\bvec r = r \uvec r$ 和\autoref{SphNab_eq1} 代入 $\bvec r \cross \grad$, 再把结果换到直角坐标系中即可.
