% 逆序数
% keys 逆序数|排列

\begin{issues}
\issueDraft
\end{issues}

\pentry{排列\upref{permut}}

\footnote{参考 Wikipedia \href{https://en.wikipedia.org/wiki/Inversion_(discrete_mathematics)}{相关页面}.}我们把集合 $\qty{1,\dots,N}$ 的第 $n$ 种排列记为 $p_n$, 该排列的元素按照顺序分别记为 $p_{n,1}, \dots, p_{n,N}$. 对于任意 $i < j$, 如果满足 $p_{n,i} > p_{n,j}$ 我们就把 $i, j$ 或者 $p_{n,i}, p_{n,j}$ 称为排列 $p_n$ 的一个\textbf{逆序对(inversion)}. 一个排列中所有逆序对的个数就叫\textbf{逆序数(inversion number)}.

在行列式和 Levi-Civita 符号中, 我们只对逆序数的奇偶性感兴趣. 

\begin{theorem}{}
交换排列中任意两个数, 逆序数奇偶性改变. 
\end{theorem}

\subsubsection{证明}
在排列 $p_n$ 中, 考虑第 $i, j$ ($i < j$)两个元素的置换, 不妨把他们的值分别记为 $m, n$. 不妨假设 $m < n$ ($n < m$ 的讨论同理). 为了方便讨论, 我们把任意两个不同元素都用弧线连起来, 弧线的条数就是逆序数. 我们可以把这些连线分为 4 类来讨论:
\begin{enumerate}
\item 对于不连接到 $m$ 或 $n$ 的线, 该置换不对他们产生影响, 不改变逆序数.
\item 从 $m$ 或 $n$ 连接到 $m$ 左边或 $n$ 右边的那些线, 置换同样不会他们的数量产生影响, 不改变逆序数.
\item 从 $m$ 或 $n$ 连接到 $m, n$ 之间某个元素 $q$ 的线, 必须满足 $m > q > n$, 有两条. 置换后, 两条线都会消失.
\end{enumerate}

我们先看. 再来看, 而这并不会改变逆序数的奇偶性. 最后还有一根特殊的线, 即置换前 $m$ 和 $n$ 之间没有, 但置换后却产生的那条, 这使得逆序数必定增加或减少奇数个, 所以任何置换必定会改变逆序数的奇偶性. 证毕.
