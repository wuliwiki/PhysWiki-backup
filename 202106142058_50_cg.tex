% 计算机图形学
% 计算机科学 计算机图形学

\subsection{基本概念}
计算机图形学(Computer Graphics, CG)是研究计算机在在硬件和软件的帮助下创建计算机图形的科学学科,是计算机科学的一个重要分支领域.
计算机图形学主要研究如何用计算方法来操作视觉和几何信息.它主要聚焦于图像生成和处理的数学和计算基础,而不只是纯艺术方面.

\subsection{分支学科}
几何:研究表面的表示和处理方法

动画:研究运动的表示和处理方法

渲染(绘制):研究模拟光线传递的再现算法

成像:图像获取或图像编辑


\subsection{渲染}
渲染是用计算机程序的方法从三维模型生成二维图像的过程.场景文件包含用图形学语言或数据结构严格定义的对象;主要包含虚拟场景的几何、视角、纹理、光照和明暗信息.渲染程序通常内置于计算机图形学软件中,也有的是做成插件或者独立的程序.



\subsection{计算机动画}
计算机动画(Computer animation)通常指场景中任何随着时间推移而发生的视觉变化.除了对象的平移、旋转之外,计算机动画还可以随着时间的进展而改变对象大小、颜色、透明度和表面纹理等.很多计算机动画还需要有真实感的显示.在科学和工程研究中,研究人员常使用随时间而变化的伪彩色或抽象形体来表示物理量,从而有助于理解物理学过程的本质.在影视和娱乐广告中,为了产生逼真的视觉效果,也会用计算机来生成场景的精确表示.


\subsection{应用}
计算机图形学广泛应用于多种领域,包括科学、艺术、工程、医药、影视和娱乐等诸多方面.


\subsection{参考文献}
\begin{enumerate}
\item Donald Hearn, Pauline Baker, Carithers著, 蔡士杰, 杨若瑜译. 计算机图形学[M]. 电子工业出版社. 2014
\item https://en.wikipedia.org/wiki/Computer_graphics
\end{enumerate}