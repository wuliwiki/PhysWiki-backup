% Julia 常用功能

\begin{issues}
\issueDraft
\end{issues}

\begin{lstlisting}[language=julia]
println("hello world")
\end{lstlisting}

\begin{lstlisting}[language=julia]
function sphere_vol(r)
    return 4/3*pi*r^3
end
\end{lstlisting}

随机数
\begin{lstlisting}[language=julia]
rand(ComplexF64, Nr1, Nr2, Npw)
\end{lstlisting}

hash
\begin{lstlisting}[language=julia]
hash(矩阵)
\end{lstlisting}

当前时间 \verb|time()|, 零向量 \verb|zeros(整数)|

矩阵切割 \verb|Psi[:, j, :]|

\verb|size(Psi, 维度)|



\subsubsection{命令行}
\begin{itemize}
\item 为了区分, 系统的控制行叫做 terminal, 而 julia 的控制行叫做 **REPL** (read-eval-print-loop)
\item `Ctrl + D` 退出或者 `exit()` 退出
\item `ans` 和 Matlab 一样
\end{itemize}

\subsubsection{}
