% 北京大学 2004 年 考研 普通物理
% license Usr
% type Note

\textbf{声明}:“该内容来源于网络公开资料,不保证真实性,如有侵权请联系管理员”

力学
\subsection{(16 分)}
杆 $AB$ 斜靠在高为$h$的桌子的棱角上,$A$ 端在地面沿 $cx$ 轴作直线运动,杆身保持与桌子接触并在竖直面内运动。已知杆的转动角速度$\dot{\theta}=\omega_{0}$为常量。
\begin{enumerate}
\item 求$A$端的速度和加速度:
\item 求杆与桌子的交点$C$(杆上点)的速度和加速度。
\end{enumerate}
(结果用任意时刻的 $\theta$ 及 $\omega$。和 $h$ 表示)
\subsection{(18 分)}
质量为$m$、半径为$r$的匀质小圆球自静止开始从半径为$R$、固定于地面的粗糙大球面顶端向下作无滑动滚动,两球连心线与竖直线所成的角度记为$\omega$。
\begin{enumerate}
\item 求小圆球的转动角速度和$\theta$的关系:
\item 求小圆球下滚到任意$\theta$角时,小圆球的转动角速度和质心的速度:
\item $\theta$为多大时小圆球离丌大球面?已知小球对其对称轴的转动惯量为$\frac{2}{4}mr^2$。
\end{enumerate}
\begin{figure}[ht]
\centering
\includegraphics[width=8cm]{./figures/4e36648d880cc755.png}
\caption{} \label{fig_PKU004_1}
\end{figure}
\subsection{(16 分) }
\textbf{人造卫星在进入预定的圆轨道(轨道半径为$R$)之前,通常需要一个椭圆轨道过渡,椭圆轨道半长轴为$R$半短轴为$r$,}如图。
\begin{enumerate}
\item 求人造卫星在椭圆轨道近地点和远地点的速率;
\item 若要在远地点 $A$ 从椭圆轨道变为圆轨道问需要增加多少能量。已知地球和人造卫星的质量分别为$M$ 和 $m$,万有引力常量为G。
\end{enumerate}
\begin{figure}[ht]
\centering
\includegraphics[width=14.25cm]{./figures/f341f1ad7ef39576.png}
\caption{} \label{fig_PKU004_2}
\end{figure}

电磁学

\subsection{(15 分)}
无穷大导体平面上方$h$处,有一个带电量$q$的点电荷(如图),求导体平面附近的电场分布。(导体平面附近是指导体外,无限接近导体平面的区域)
\begin{figure}[ht]
\centering
\includegraphics[width=8cm]{./figures/a5002716a1ac7d89.png}
\caption{} \label{fig_PKU004_3}
\end{figure}