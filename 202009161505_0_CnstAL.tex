% 匀加速直线运动

\begin{issues}
\issueTODO
\end{issues}

\pentry{速度 加速度(一维)\upref{VnA1}}

质点做匀速直线运动时, 我们延运动的直线建立坐标轴 $x$, 则最一般的运动方程
\begin{equation}\label{CnstAL_eq1}
x(t) = x_0 + v_0 (t - t_0) +  \frac12 a_0 (t - t_0)^2
\end{equation}
其中 $x_0, v_0, a_0$ 分别是 $t_0$ 时刻的位置, 速度和加速度. 注意沿 $x$ 轴正方向的速度和加速度取正号, 沿反方向取负号.

速度变化为
\begin{equation}\label{CnstAL_eq2}
v(t) = v_0 + a_0 (t - t_0)
\end{equation}
另外有一条不含时间的公式
\begin{equation}\label{CnstAL_eq3}
v_2^2 - v_1^2 = 2a_0 (x_2 - x_1)
\end{equation}
其中 $v_1, v_2$ 分别是质点经过点 $x_1$ 和 $x_2$ 时的速度.

\addTODO{插入一些 EP1 例题}

\subsection{推导}
做匀速直线运动的质点在 $t_0$ 时的位置为 $x_0$, 速度为 $v_0$, 且加速度始终等于常数 $a_0$, 求任意时刻的速度和加速度 $x(t)$.

我们首先把 $a(t) = a_0$ 代入\autoref{VnA1_eq8}~\upref{VnA1}
\begin{equation}
v(t) = v_0 + \int_{t_0}^t a_0 \dd{t'}
\end{equation}
马上得到\autoref{CnstAL_eq2} . 再次积分(\autoref{VnA1_eq5}~\upref{VnA1})
\begin{equation}
x(t) = x_0 + \int_{t_0}^t [v_0 + a_0 (t' - t_0)] \dd{t'}
\end{equation}
得\autoref{CnstAL_eq1} . 这个过程相当于直接使用\autoref{VnA1_eq2}~\upref{VnA1}.

要得到\autoref{CnstAL_eq3}, 由\autoref{CnstAL_eq1} 和\autoref{CnstAL_eq2} 分别得
\begin{equation}
x_2 - x_1 = v_1 (t_2 - t_1) +  \frac12 a_0 (t_2 - t_1)^2
\end{equation}
\begin{equation}
t_2 - t_1 = \frac{v_2 - v_1}{a_0}
\end{equation}
代入消去 $t$ 得\autoref{CnstAL_eq3} . 直观来看, 如果画速度—时间图, $x_2 - x_1$ 可以表示成梯形的面积, 利用梯形公式可以得到.