% 自旋 1/2 粒子的非相对论波函数
% keys 自旋|薛定谔方程|泡利方程

\pentry{薛定谔方程(单粒子多维)\upref{QMndim},自旋角动量\upref{Spin},自旋角动量矩阵\upref{spinMt}}

在量子力学发展的早期,薛定谔首先提出了 Klein Gordon 方程,企图描绘遵从相对论变换的电子波动方程,但却遭遇失败.之后薛定谔退而求其次,转而求它的非相对论近似下的方程,得到了著名的薛定谔方程:
\begin{equation}
i\hbar \frac{\partial }{\partial t}\psi = \hat H\psi=\qty[\frac{ 1}{2m}\qty(-\frac{i}{\hbar}\nabla)^2+V(x)]\psi
\end{equation}
换言之,粒子波动方程的能量由 $\hat H$
原因是电子是一个自旋为 1/2 的粒子,