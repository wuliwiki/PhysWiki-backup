% 分裂域
% keys splitting field|正规扩张|regular extension

域的扩张\upref{FldExp}

\pentry{域的扩张\upref{FldExp}}

本节我们要介绍一个在代数中非常基础且重要的概念:分裂域.简单来说,分裂域就是在一个域中添加某个多项式的全体根所得到的扩域.

\begin{definition}{分裂域}
给定域$\mathbb{F}$及其上一个多项式$f(x)$.若存在扩域$\mathbb{K}/\mathbb{F}$,使得$f(x)$在$\mathbb{K}$上可以分解为$f(x)=\prod_{i=1}^n (x-a_i)$,且$\mathbb{K}=\mathbb{F}(a_1, a_2, \cdots, a_n)$,则称$\mathbb{K}$是$f(x)\in \mathbb{F}[x]$上的\textbf{分裂域(splitting field)}.
\end{definition}

定义看起来有些绕口,先说$f$在$\mathbb{K}$中可以分解,也就是说每一个根都存在,再说$\mathbb{K}$可以看成用这些根对$\mathbb{F}$进行扩域的结果.这么定义是因为我们要先确定元素$a_i$都存在,而为此就需要先确定$\mathbb{K}$存在.但是定义中只说了“若$\mathbb{K}$”存在,这个假设到底成立与否呢?答案是肯定的.

\begin{theorem}{分裂域的存在性}
给定域$\mathbb{F}$及其上一个多项式$f(x)$,则$f(x)\in \mathbb{F}[x]$上的分裂域存在.
\end{theorem}

\textbf{证明}:

首先在环$\mathbb{F}[x]$上对元素$f(x)$进行因式分解\footnote{也就是画出它的一棵\textbf{真因子树}\upref{FctTre}.},得到其不可约

\textbf{证毕}.




























