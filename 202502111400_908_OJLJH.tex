% 欧几里得几何(综述)
% license CCBYSA3
% type Wiki

本文根据 CC-BY-SA 协议转载翻译自维基百科\href{https://en.wikipedia.org/wiki/Euclidean_geometry#}{相关文章}。

\begin{figure}[ht]
\centering
\includegraphics[width=6cm]{./figures/29202c1c81a65fe8.png}
\caption{拉斐尔的《雅典学派》中的细节,展示了一位希腊数学家——可能代表欧几里得或阿基米德——正在使用圆规绘制几何构图。} \label{fig_OJLJH_1}
\end{figure}
欧几里得几何是归功于古希腊数学家欧几里得的数学体系,他在其几何学教材《几何原本》中对其进行了描述。欧几里得的方法是假设一小组直观上令人信服的公设(公理),并从这些公理中推导出许多其他命题(定理)。尽管欧几里得的许多结果早已被提出,[1] 但他是第一个将这些命题组织成一个逻辑系统的人,其中每个结果都是从公理和先前证明的定理推导出来的。[2]

《几何原本》以平面几何开始,至今仍在中学(高中)教授,作为第一个公理化系统和数学证明的初步例子。它接着讲解了三维的立体几何。《几何原本》中的许多内容阐述了现在被称为代数和数论的结果,用几何语言来表达。[1]

在超过两千年的时间里,“欧几里得”这个形容词是不必要的,因为欧几里得的公理似乎是如此直观明显(平行公设可能是唯一例外),以至于从这些公理中推导出的定理被认为是绝对正确的,因此没有其他类型的几何被认为是可能的。然而,今天许多自洽的非欧几里得几何已被发现,最早的几何形式是在19世纪初发现的。爱因斯坦的广义相对论理论的一个含义是,物理空间本身并非欧几里得空间,欧几里得空间仅在短距离内(相对于引力场的强度)对其进行良好近似。[3]

超过两千年来,“欧几里得”这个形容词并不必要,因为欧几里得的公理看起来非常直观明显(平行公设可能是个例外),从这些公理推导出来的定理被认为是绝对正确的,因此没有其他形式的几何被认为是可能的。然而,今天我们知道许多其他自洽的非欧几里得几何,最早的发现是在19世纪初。阿尔伯特·爱因斯坦的广义相对论理论的一个含义是,物理空间本身并非欧几里得的,欧几里得空间只有在短距离(相对于引力场的强度)内才是一个很好的近似。

欧几里得几何是合成几何的一个例子,因为它从描述几何对象(如点和线)的基本属性的公理出发,逻辑地推导出关于这些对象的命题。这与近两千年后由勒内·笛卡尔引入的解析几何形成对比,后者通过坐标使用代数公式来表达几何属性。
\subsection{《几何原本》}
《几何原本》主要是对早期几何知识的系统化。它在比早期的几何处理方法中取得了显著的进步,这一点很快被认可,因此几乎没有人再对保留早期的几何作品感兴趣,而这些作品如今几乎都已遗失。

《几何原本》共包含13卷:

第I至IV卷和第VI卷讨论平面几何。许多关于平面图形的定理被证明,例如“任意三角形中,任意两个角之和小于两个直角。”(第I卷命题17)以及“在直角三角形中,斜边的平方等于两直角边的平方和。”(第I卷命题47)

第V卷和第VII至X卷涉及数论,数被几何地处理,作为线段的长度或平面区域的面积。介绍了素数、有理数和无理数等概念。并且证明了素数的个数是无限的。

第XI至XIII卷涉及立体几何。一个典型的定理是圆锥体和底面和高度相同的圆柱体的体积比为1:3。并且构造了柏拉图立体。

\begin{figure}[ht]
\centering
\includegraphics[width=6cm]{./figures/8f816c81e5e3a080.png}
\caption{平行公设(公设 5):如果两条直线与第三条直线相交,并且在一侧的内角之和小于两个直角,那么这两条直线如果延伸足够远,必定会在该侧相交。} \label{fig_OJLJH_2}
\end{figure}
欧几里得几何是一种公理化系统,其中所有定理(“真命题”)都从少数简单的公理推导出来。在非欧几里得几何出现之前,这些公理被认为在物理世界中显而易见,因此所有的定理也都被认为是同样真实的。然而,欧几里得从假设到结论的推理依然独立于物理现实有效。[4]

在《几何原本》第一卷的开头,欧几里得给出了五个平面几何公设(公理),用构造的方式来表述(根据托马斯·希斯的翻译):[5]

假设如下:
\begin{enumerate}
\item 从任意一点到任意一点画一条直线。
\item 将一条有限的直线继续延伸成一条直线。
\item 以任意中心和任意距离(半径)画一个圆。
\item 所有的直角都相等。
\item [平行公设]:如果一条直线与两条直线相交,并且在同一侧形成的内角小于两个直角,那么这两条直线如果无限延伸,必定会在角小于两个直角的那一侧相交。
\end{enumerate}

尽管欧几里得明确地只主张构造物体的存在,但在他的推理中,他也隐含地假设这些物体是唯一的。