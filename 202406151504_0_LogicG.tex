% 逻辑门、布尔运算
% license Usr
% type Tutor

\begin{issues}
\issueDraft
\end{issues}

\textbf{德摩根定理(De Morgan's laws)}又称\textbf{对偶律}

\begin{itemize}
\item \textbf{与、或、非(AND, OR, NOT)}是三种最常见的逻辑门
\item CPU 里面所有功能都是逻辑门构成的。 例如做整数加法的电路。
\item 一个\textbf{布尔变量}的值只能是 0 或 1。
\item OR, AND, NOT 运算可以表示为 $A + B$, $A\cdot B$, $\overline A$
\item 另一种常用的符号是 $A\lor B$(或), $A\land B$(与), $\neg A$(非)。 这类似于并集、交集和补集。
\item \textbf{结合律(associative laws)} $A\cdot(B\cdot C) = (A\cdot B)\cdot C$, $A+(B+C)=(A+B)+C$
\item \textbf{交换律(commutative laws)} $A+B=B+A$,$A\cdot B=B\cdot A$
\item \textbf{有界律(identity laws)} $A+0=A$,$A\cdot 1=A$,$A\cdot 0 = 0$,$A+1=1$(这就是为什么要用加号和乘号,很多性质都很相似)
\item \textbf{幂等律(idempotent laws)} $A+A=A$,$A\cdot A=A$
\item \textbf{互补律(complement laws)} $A+\overline A = 1$, $A\cdot \overline A = 0$
\item \textbf{分配律(distributive laws)} $A\cdot(B+C)=A\cdot B+A\cdot C$, $A+(B\cdot C) = (A+B)\cdot(A+C)$
\item \textbf{吸收律(absorption laws)} $A\cdot(A+B) = A$, $A+A\cdot B = A$ (分配律的特殊情况)
\item \textbf{德摩根律(De Morgan's laws)} $\overline{A+B} = \overline A \cdot \overline B$; $\overline{A \cdot B} = \overline A + \overline B$
\item 德摩根律可拓展到多个变量如 $\overline{A+B+C} = \overline A \cdot \overline B \cdot \overline C$, $\overline{A \cdot B\cdot C} = \overline A + \overline B + \overline C$
\item \textbf{两边取 NOT} 等式依然成立。
\item 德摩根定律两边取 NOT,就有 $A+B = \overline{\overline A \cdot \overline B}$; $A \cdot B = \overline{\overline A + \overline B}$。 这说明\textbf{与或非中只有两个是必须的}。
\item \textbf{共识定理(consensus theorem)} $A + \overline A \cdot B = A + B$
\item \textbf{与或非中非门是必须的}(无法用 AND 和 OR 构成)。
\item \textbf{与非门 (NAND,NOT AND)} $\overline{A\cdot B}$ 可以构成任意门(两个输入相连就是 NOT 门)
\item \textbf{或非门(NOR,NOT OR)} $\overline{A+B}=\overline A\cdot\overline B$ 检查是否输入都为零,也可以构成任意门(两个输入相连就是 NOT 门)。
\item \textbf{异或门(XOR)} 当两个输入不同时输出 1,$A\cdot\overline B+\overline A\cdot B$。
\item \textbf{同或门(XNOR)} 当两个输入相同时输出 1,即异或取非。
\item NAND 闪存的名字就是 NAND 门命名的。
\item 任意个输入变量,经过逻辑运算得到一个输出变量。 如果已知每种输入值对应的输出,那么可以用 NOT 把每种输出为 1 的情况都做 OR 即可。但复杂情况下初始表达式可能较为复杂,化简也麻烦,所以设计电路还是需要一些技巧。
\item 例如要 $ABC$ 中是否只有一个 1,就写出 $A\cdot\overline B\cdot\overline C + \overline A\cdot B\cdot\overline C + \overline A\cdot\overline B\cdot C$
\item NOR,XOR,AND 可以分别检测两个输入之和是否为 0,1,2,可以用于加法电路或计数电路。
\item 如何创建 2 或 3 个变量的计数电路(也就是加法电路)?(结果分别是 4 或 5 bit)
\item 如何创建两个 $N$ 位二进制数的加法电路?(每个数的每 bit 是一个变量)
\item 如何创建 $2^N$ 个变量的计数电路?(结果表示为二进制数,每个 bit 一个变量)
\end{itemize}
