% 离散傅里叶变换
% keys DFT|FFT|快速傅里叶变换|离散傅里叶变换
% license Xiao
% type Tutor

\textbf{离散傅里叶变换(Discrete Fourier Transform)(DFT)}是一个复数域的线性变换\upref{LTrans}。 对两组有序数列 $f_0, f_1, \dots, f_{N-1}$ 和 $g_0,g_2,\dots, g_{N-1}$,正变换和逆变换分别为\footnote{工程上的定义常常是正变换没有 $1/\sqrt{N}$ 因子,逆变换的 $1/\sqrt{N}$ 因子变为 $1/N$。 这样的好处是节省运算量。 我们使用的定义好处是变换为幺正变换,有保持归一化的特点。}
\begin{equation}\label{eq_DFT_1}
g_p = \frac{1}{\sqrt{N}}\sum_{q=0}^{N-1} \exp(-\frac{2\pi\I}{N} p q ) f_q~,
\end{equation}
\begin{equation}\label{eq_DFT_2}
f_q = \frac{1}{\sqrt{N}}\sum_{p=0}^{N-1} \exp(\frac{2\pi\I}{N} p q) g_p~.
\end{equation}
一个更常见的名词是\textbf{快速傅里叶变换(Fast Fourier Transform)(FFT)}, 其数学定义与离散傅里叶变换一样,只是优化了算法使程序运行更快\footnote{参考 Numerical Recipes 3ed}。

\subsection{与傅里叶变换的关系}
\pentry{傅里叶变换(指数函数)\upref{FTExp}}
在详细分析 DFT 的性质之前, 我们先看看它与解析的傅里叶变换如何对应。 函数 $f(x)$ 和 $g(k)$ 间的傅里叶变换\upref{FTExp}为
\begin{equation}\label{eq_DFT_3}
g(k) = \frac{1}{\sqrt{2\pi}} \int_{-\infty}^{+\infty} f(x)\E^{-\I kx} \dd{x}~,
\end{equation}
\begin{equation}\label{eq_DFT_4}
f(x) = \frac{1}{\sqrt{2\pi}} \int_{-\infty}^{+\infty} g(k)\E^{\I kx} \dd{k}~.
\end{equation}
如果 $f(x)$ 和 $g(k)$ 分别只在区间 $[-L_x/2, L_x/2]$ 和 $[-L_k/2, L_k/2]$ 内不为零, 积分就可以只在这两个区间内进行。 如果函数不为零的区间 $[a,b]$ 并不关于原点对称, 那么使用平移公式即可(\autoref{sub_DFT_1})。

\subsubsection{格点}
\begin{figure}[ht]
\centering
\includegraphics[width=8cm]{./figures/cad82c3334b8d489.pdf}
\caption{$N = 4$ 和 $N=5$ 时的 $x$ 格点, $\Delta x = L_x/N$。 $k$ 轴格点同理。} \label{fig_DFT_1}
\end{figure}
为了做离散傅里叶变换, 我们给两个区间划出 $N$ 个等间距的格点(\autoref{fig_DFT_1}) $\dots, x_{-1}, x_0, x_1,\dots$ 和 $\dots, k_{-1}, k_0, k_1,\dots$ 令 $x_0 = k_0 = 0$, 并规定相邻格点的间距为
\begin{equation}\label{eq_DFT_5}
\Delta x = L_x/N ~,\qquad \Delta k = L_k/N~.
\end{equation}
注意第一个和最后一个格点未必在区间的两端, 首尾格点分别相距 $L_x(N-1)/N$ 和 $L_k(N-1)/N$。 若 $N$ 是奇数, 我们令中间的格点为 $x_0$ 和 $k_0$, 如果 $N$ 是偶数, 我们令中间偏右的格点为 $x_0$ 和 $k_0$, 即
\begin{equation}
\leftgroup{
&x_n = n \Delta x\\
&k_n = n \Delta k~,
}\qquad n = \leftgroup{
&-N/2 \dots N/2 - 1 &\quad &(\text{偶数 } N)\\
&-(N-1)/2 \dots (N-1)/2 &\quad &(\text{奇数 } N)~.
}\end{equation}

\subsubsection{求和代替积分}
现在我们用求和近似\autoref{eq_DFT_3} \autoref{eq_DFT_4} 的积分得
\begin{equation}\label{eq_DFT_6}
g(k_p) \approx \frac{1}{\sqrt{2\pi}} \sum_q f(x_q) \E^{-\I k_p x_q} \Delta x~,
\end{equation}
\begin{equation}\label{eq_DFT_7}
f(x_q) \approx \frac{1}{\sqrt{2\pi}} \sum_p g(k_p) \E^{\I k_p x_q} \Delta k~.
\end{equation}
式中 $k_p x_q = (\Delta x \Delta k)pq$, 对比 DFT (\autoref{eq_DFT_1} \autoref{eq_DFT_2})中的指数项得
\begin{equation}\label{eq_DFT_8}
\Delta x\Delta k = \frac{2\pi}{N}~.
\end{equation}
但我们会发现这里的 $p, q$ 可以是负整数, 而 DFT 中的 $p, q$ 都是非负整数。 但稍加计算就会发现当 $p$ 或 $q$ 是负值时, 把它们加上 $N$, 指数项并不改变。 例如 $k_{p+N} x_q = k_p x_q + 2\pi$, 并不影响指数项。 所以, 我们只要将所有小于零的格点编号加上 $N$ 并重新排列即可。 例如 $N = 4$ 时, $x$ 格点为 $x_{-2}, x_{-1}, x_0, x_1$, 令 $x_2 = x_{-2}, x_3 = x_{-1}$, 这四个格点的名字就变为 $x_0, x_1, x_2, x_3$, 数值不变。 现在再来对比 DFT 和\autoref{eq_DFT_6} \autoref{eq_DFT_7}, 就只是相差两个常数了:
\begin{equation}
g(k_i) = \sqrt{\frac{N}{2\pi}} \Delta x \cdot g_i, \qquad
f(x_i) = \sqrt{\frac{N}{2\pi}} \Delta k \cdot f_i~.
\end{equation}

\autoref{eq_DFT_8} 是 DFT 一个重要的性质。 结合\autoref{eq_DFT_5} 得
\begin{equation}\label{eq_DFT_9}
N\Delta x \Delta k = L_x \Delta k = L_k \Delta x = \frac{L_x L_k}{N} = 2\pi~.
\end{equation}
离散傅里叶变换只有两个自由度, 只要决定 $N, L_x, \Delta x, L_k, \Delta k$ 中的任意两个, 就可以完全决定变换公式。 注意 $L_x$ 与 $\Delta k$ 成反比, $L_k$ 与 $\Delta x$ 成反比, $L_xL_k$ 与 $N$ 成正比。

总结起来, 要用 DFT 数值近似函数 $f(x)$ 的傅里叶变换, 就先用上述方法生成的格点将该函数等间距采样, 然后左半和右半调换(负脚标加上 $N$)得到 $f_i$, 代入 DFT 公式(在程序中使用 FFT)得到 $g_i$ 再左半和右半调换得到 $g(k)$ 的离散点。 反傅里叶变换同理。

下文如无特殊说明, 默认 DFT 前后各将数列的左半右半交换一次。

\subsection{与傅里叶级数的关系}
要用 DFT 计算傅里叶级数\upref{FSExp} 的系数(\autoref{eq_FSExp_2}~\upref{FSExp}), 只需要把\autoref{eq_DFT_3} 乘以 $\sqrt{2\pi}/L_x$ 即可, 所以
\begin{equation}\label{eq_DFT_10}
c_n = \frac{\sqrt{2\pi}}{L_x} g(k_i) = \frac{1}{\sqrt{N}} g_i~,
\end{equation}
另外可以发现 DFT 中 $k_n = 2\pi n/L_x$ 的离散值恰好就是傅里叶级数级数中需要的。

\subsection{DFT 变换矩阵}
\pentry{正交矩阵、酉矩阵\upref{UniMat}}
DFT 是一个线性变换, 我们来看变换矩阵的性质。 把变换和逆变换的系数矩阵用
 $\mat F$ 和 $\mat F^{-1}$ 来表示, 令列矢量 $\bvec f = (f_0\ f_1\dots f_{N-1})\Tr$, $\bvec g = (g_0\ g_1\dots g_{N-1})\Tr$, 变换和逆变换分别记为
\begin{equation}
\bvec g = \mat F \bvec f ~,\qquad
\bvec f = \mat F^{-1} \bvec g~.
\end{equation}
其中
\begin{equation}
F_{pq} = \frac{1}{\sqrt{N}} \exp(-\frac{2\pi\I}{N} p q)~.
\end{equation}
下面证明, $\mat F$ 是一个酉矩阵\upref{UniMat}, 所以逆变换矩阵就是 $\mat F$ 的厄米共轭。
\begin{equation}
\mat F^{-1} = \mat F\Her~,
\end{equation}
不难验证\autoref{eq_DFT_2} 的变换矩阵的确等于 $\mat F\Her$。

根据酉矩阵的定义,我们需要证明
\begin{equation}
\sum_{p=0}^{N-1} F^*_{pq_1} F_{pq_2} = 0 \quad (q_1 \ne q_2)~,
\end{equation}
而
\begin{equation}\label{eq_DFT_14}
\sum_{p=0}^{N-1} F^*_{pq_1}F_{pq_2}
= \frac1N \sum_{p=0}^{N-1} \exp\qty[\frac{2\pi\I}{N} (q_2-q_1) p]~.
\end{equation}
注意到求和的每一项在复平面上都对应模长为 $1/N$, 幅角为 $(q_2-q_1)p/N$ 个圆周的矢量,
而 $N$ 条矢量恰好向不同方向均匀分布,所以相加为 $0$。证毕。

\subsection{采样定理}\label{sub_DFT_2}
从以上的分析中, DFT 似乎只是一种近似, 且有种种限制, 例如我们只能取关于原点对称的区间, 又例如变换只有两个自由度。 我们不禁想定义更广义的离散傅里叶变换, 使\autoref{eq_DFT_6} 和\autoref{eq_DFT_7} 中 $x_i$ 和 $k_j$ 可以取任意区间的任意多个等间距格点。 但事实上, 这样的定义并不比 DFT 更精确, 而且不能用 FFT 算法提高运算速度。

这里给出\textbf{奈奎斯特—香农采样定理(Nyquist–Shannon sampling theorem)}: 如果 $g(k)$ 不超出 $[-L_k/2, L_k/2]$(即 $f(x)$ 是\textbf{有限带宽的}), 那么我们只需要用步长 $\Delta x \leq 2\pi/L_k$ (DFT 恰好取等号)来对  $f(x)$  采样就可以用以下插值公式精确还原出 $f(x)$:\footnote{其实我并不确定是否需要 $x_0 = 0$ 或者 $x_i$ 的位置有什么其他要求, 现在姑且认为 $x_0 = 0$ 好了。}
\begin{equation}\label{eq_DFT_15}
f(x) = \sum_{n = -\infty}^{\infty} f(x_n)\sinc[\pi(x - x_n)/\Delta x]~,
\end{equation}
其中 $\sinc x = \sin x/x$ (且定义 $\sinc 0 = 1$)。

现在我们将插值公式代入傅里叶变换得
\begin{equation}
g(k) =  \frac{1}{\sqrt{2\pi}} \sum_{n = -\infty}^{\infty} f(x_n) \int_{-\infty}^{+\infty}\sinc[\pi(x - x_n)/\Delta x] \E^{-\I k x} \dd{x}~,
\end{equation}
结果是
\begin{equation}\label{eq_DFT_17}
g(k) = \leftgroup{
&\frac{1}{\sqrt{2\pi}} \sum_{n = -\infty}^{\infty} f(x_n) \E^{-\I kx_n} \Delta x \quad &(\abs{k} \leqslant L_k/2)\\
&0 & (\abs{k} > L_k/2)
}~.
\end{equation}
当 $k = k_p$ 时, 这正是\autoref{eq_DFT_6} 右边(注意求和从 $N$ 项拓展为无穷项)。 不仅如此, 该式也可以作为 $g(k)$ 的插值公式。

有时候我们并不知道 $f(x)$ 的带宽, 如何决定采样点的最大步长 $\Delta x$ 呢? 一个简单的方法是先选一个较小的 $\Delta x$, 使 $f(x_i)$ 散点看起来较为连续。%这里应该用 exp(-x^2)*exp(10i*x) 画图说明。
这时解析傅里叶变换就可以用 DFT 近似。 得到 $g(k_j)$ 散点后, 再判断合理的 $L_k$ 并计算相应的 $\Delta x$ 即可。

由于傅里叶变化和反变换是对称的, 所以以上的分析中将 $f(x)$ 和 $g(k)$ 调换同样成立。 一个有趣的问题是, 是否存在一些情况使 $f(x)$ 和 $g(x)$ 都是有限带宽的? 严格来说这种情况是不存在的\footnote{我们可以用反证法做一个不太严谨的证明: 如果这种情况存在, 那么 $f(x)$ 就可以用有限个 $\sinc(x)$ 函数的线性组合来表示, 然而从定义上来看 $\sinc(x)$ 函数在 $x$ 的区间外不可能恒为零。}, 但许多时候我们可以近似认为函数在区间外等于零, 其中典型的例子就是高斯分布的傅里叶变换(\autoref{ex_FTExp_1}~\upref{FTExp})。

\subsection{归一化}
我们知道傅里叶变换可以保持函数的模长不变, 或者说保持函数的归一化。%链接未完成

假设采样定理的条件成立, 我们来看如果通过离散点计算函数的模长。 以 $f(x)$ 为例, 用插值公式来计算模长的平方, 得
\begin{equation}\ali{
\int_{-\infty}^{\infty} &f^*(x) f(x) \dd{x} = \\
&\sum_{m = -\infty}^{\infty}\sum_{n = -\infty}^{\infty} f(x_m)^*f(x_n) \int_{-\infty}^{\infty} \sinc\qty[\frac{\pi(x - x_m)}{\Delta x}]\sinc\qty[\frac{\pi(x - x_n)}{\Delta x}]\dd{x}~.
}\end{equation}
果然, 这个积分的结果是
\begin{equation}
\sum_{i = -\infty}^{\infty} \abs{f_i}^2\Delta x~.
\end{equation}
即函数的模方等于列矢量的模方乘以步长, 这对 $g(k)$ 也同样成立。 既然 DFT 是精确的, 我们预期变换矩阵 $\mat F$ 能保证变换前后列矢量的模长相等, 而这恰好是酉矩阵的性质。%链接未完成

\subsection{插值与添零}
% 未完成: 把 “DFT1.m” 的内容作为一个例子

上面我们看到, 如果 $f(x)$ 是有限带宽的, 就可以使用\autoref{eq_DFT_15} 进行插值, 称为\textbf{sinc 插值}\footnote{注意实现的时候只能对有限项求和。}。 然而在实践中, 由于直接用 sinc 插值计算量太大, 我们往往用下面介绍的\textbf{添零法} 代替。

假设我们要对 $N$ 个散点 $f(x_i)$ 插值, 可以先做 DFT 得到 $N$ 个 $g(k_j)$, 由\autoref{eq_DFT_9} 可知, 要缩小 $\Delta x$ 而保持 $L_x$ 不变, 就要保持 $\Delta k$ 不变且增加 $L_k$ 和 $N$。 所以我们可以在 $g(k_j)$ 数列的两边添加相同数量的 0, 然后再做反 DFT 变换即可\footnote{不难证明如果新的长度是 $N$ 的整数倍, 反变换后在老格点处仍能得到同样的 $f(x_i)$。}。

然而这么做与 sinc 插值得到的结果并不完全相同。 sinc 插值得到的 $f(x)$ 往往会超出原来的区间(因为 sinc 函数只以 $\Delta x/x$ 的速度衰减)。 所以我们首先需要在误差能接受的范围内取一个更大 $L_x$(即更小的 $\Delta k$)。 由于 sinc 插值函数的带宽严格小于 $L_k = 2\pi/\Delta x$, 所以 $L_k$ 不需要改变。 现在我们可以认为两个函数都是有限带宽的了, 要得到“包含所有信息” 的 DFT, 就要先保持 $\Delta x$ 不变, 对 $f(x_i)$ 在新区间补零, 做 DFT 得到 $g(k_j)$。 现在两个函数的地位完全平等, 都可以用\autoref{eq_DFT_15} 和\autoref{eq_DFT_17} 中的任意一个来精确插值。 而注意补零插值法就相当于\autoref{eq_DFT_17}。 所以这时再对 $g(k_j)$ 补任意多的零, 就可以得到 $f(x)$ 任意密度的插值。 在误差范围内, 现在得到的 $f(x)$ 就与 sinc 插值得到的相等了。

如果直接按照 DFT 公式计算添零法, 程序的速度未必比 sinc 插值要快, 但如果用 FFT, 那添零法将会更快\footnote{如果 FFT 中只有部分格点不为零, 那么 FFT 理论上可以变得更快, 称为 pruned FFT, 然而 FFTW 的相关页面中介绍, 只要非零格点的个数大于 1\%, pruned FFT 都不会有显著的性能提升。}。

\subsection{任意区间的 DFT}\label{sub_DFT_1}
上面提到 DFT 或 FFT 的另一个限制就是只能在以零点为中心的两个区间之间变换 $f(x)$ 和 $g(k)$。 但事实上我们只需要把数据稍作处理就可以在两个任意的区间 $[x_a, x_b]$ 和 $[k_a, k_b]$ 进行变换。

我们先定义两个新函数\footnote{\autoref{eq_DFT_20} 也可以改为 $f_1(x) = f(x + x_0)\E^{-\I k_0 (x+x_0)}$ 和 $g_1(k) = g(k + k_0)\E^{\I k x_0}$。}
\begin{equation}\label{eq_DFT_20}
f_1(x) = f(x + x_0)\E^{-\I k_0 x}, \qquad
g_1(k) = g(k + k_0)\E^{\I (k+k_0) x_0}~.
\end{equation}
其中 $x_0 = (x_a + x_b)/2$,  $k_0 = (k_a + k_b)/2$。 另外定义 $L'_x = x_b - x_a$, $L'_k = k_b-k_a$, $\Delta x' = 2\pi/L'_k$, $\Delta k' = 2\pi/L'_x$。 根据傅里叶变换的平移性质(\autoref{eq_FTExp_4}~\upref{FTExp})\footnote{容易证明, DFT 也精确满足这个性质。}, $f_1(x)$ 的傅里叶变换就是 $g_1(k)$。 注意 $f_1(x)$ 和 $g_1(k)$ 的 DFT 区间 $[-L_x/2, L_x/2]$ 和 $[-L_k/2, L_k/2]$ 分别关于原点对称。 现在我们只需要做 $f_1(x)$ 和 $g_1(k)$ 间的  DFT, 就能得到 $f(x)$ 和 $g(k)$ 间的离散变换, 不妨称为\textbf{广义 DFT}。

经过简单的推导可以发现广义 DFT 仍然可以用\autoref{eq_DFT_6} 和\autoref{eq_DFT_7} 表示, 只是 $x$ 和 $k$ 的 $N$ 个格点分别落在 $[x_a, x_b]$ 和 $[k_a, k_b]$ 内\footnote{同样, 如果 $N$ 是奇数, $x_0, k_0$ 是中间格点, 否则就是中间靠右格点。}。 但在程序中, 最高效的做法还是先将 $\bvec f$ 先变为 $\bvec f_1$, 做 FFT 得到 $\bvec g_1$ 再变为 $\bvec g$。

由采样定理易证, 如果 $g(k)$ 只在 $[k_a, k_b]$ 不为零, 那么只需用 $\Delta x' = 2\pi/L'_k$ 对 $f(x)$ 采样就可以用以下插值公式精确还原出\footnote{对离散的 $f_1(x)$ 使用之前的插值公式, 再由 $f_1(x)$ 求 $f(x)$ 就可以推导出\autoref{eq_DFT_21}。} $f(x)$。
\begin{equation}\label{eq_DFT_21}
f(x) = \sum_{n = -\infty}^{\infty} f(x_n)\sinc\qty[\frac{\pi(x - x_n)}{\Delta x}] \E^{\I k_0 (x-x_n)}~,
\end{equation}

实际应用中同样也可以用补零法来插值。

广义 DFT 的好处是什么呢? 例如有时候非零区间 $[x_a, x_b]$ 或(和) $[k_a, k_b]$ 很窄但却远离原点, 则 $L_k$ 或 $L_x$ 仍需取较大的值, 这时 DFT 就没有发挥最大效率, 因为大部分格点的函数值都是 0。 但如果用广义 DFT, 就可以大大减少格点数。

\begin{example}{高斯波包}
我们现在来用 FFT 数值计算以下高斯波包的傅里叶变换。
\begin{equation}
f(x) = \E^{-x^2}\E^{100 \I x}~.
\end{equation}
为了便于验证数值结果, 解析的傅里叶变换结果是% 未完成:给出“高斯积分”中的相关链接
\begin{equation}
g(k) = \frac{1}{\sqrt2}\E^{-(k-100)^2/4}~,
\end{equation}
我们分别取 $x, k$ 的区间为 $[-5, 5]$ 和 $[90, 110]$ (区间外的函数值不超过 $1.4\e{-11}$)。 所以我们采用\autoref{eq_DFT_21} 的傅里叶变换, 令 $x_0 = 0$, $k_0 = 100$, 则 $L_x \geqslant 10$, $L_k \geqslant 20$。 为了满足 $L_x L_k = 2\pi N$ (见\autoref{eq_DFT_9}), $N$ 取最小值 32。 所以不妨令 $L_x = 10, L_k = 2\pi N/L_x = 20.106...$。
所以 $\Delta x = L_x/N = 0.313...$, $\Delta k = L_k/N = 2\pi/L_x = 0.628...$。 现在我们就可以生成 $x, k$ 的格点并且做 FFT 了。
% 未完成: 接下来说明如何使用 matlab 的 fft(), ifft() 以及 fftshift, ifftshift. 给出代码。
\end{example}
