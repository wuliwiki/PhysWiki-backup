% 量子信道
% keys 量子信道|量子操作

\pentry{量子系综\upref{qEns},量子测量\upref{qmeas},量子力学基本原理\upref{QMPrcp}}

在量子系综和量子测量的章节,我们反复强调了推广标准量子力学中纯态和投影测量的必要性。这种必要性可以粗略分为三个方面:
\begin{enumerate}
\item 当系统和更大的环境发生了相互作用并且耦合在了一起,这时即使总系统的态是纯态,上面的测量是投影测量,对应到子系统上并非一定如此,往往会变成密度矩阵和POVM测量。
\item 当我们并未掌握一个物理系统的全部信息的话,就必须使用系综的语言来描述这个系统。如果这个系统是量子系统的话,那么对应的合适的描述就应当是密度矩阵和POVM测量。
\item 我们可以定义一些概率性的态制备和测量协议,在这些情况之下,由于经典概率的参与,我们需要把纯态和投影测量推广到密度矩阵和POVM测量。
\end{enumerate}

这里我们仍然继续这样的思路,不过这次我们想要推广的是量子力学中的演化。我们将会看到,在量子信息的语言下,演化可以被推广成为量子信道,它描述了我们能够对一个量子态进行的所有“合法”的变换,包括演化,擦除信息,甚至之前提到的POVM测量。因为这样的特性,量子信道成为了量子信息科学中的标准工具和语言。

\subsection{量子态的合法变换}

首先不妨回顾一下标准量子力学中是如何描述量子态演化的。假设在Hilbert空间$\mathcal{H}$上有一个初态$\ket{\psi}$,现在通过某种演化算符$O:\mathcal{H}\to\mathcal{H}$将其变换到末态$\ket{\phi}$。由于归一化条件的限制:
\begin{equation}
\braket{\psi}{\psi}=\braket{\phi}{\phi}=1=\bra{\psi}O^\dagger O\ket{\psi}~.
\end{equation}
我们便知道了此时的演化算符$O$会是一个$\mathcal{H}$上的幺正算符。

我们可以对混态问同样的问题:假设在Hilbert空间$\mathcal{H}$上有一个初态$\rho$,现在通过某种操作