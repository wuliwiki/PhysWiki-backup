% 达朗贝尔原理(综述)
% license CCBYSA3
% type Wiki

本文根据 CC-BY-SA 协议转载翻译自维基百科\href{https://en.wikipedia.org/wiki/D\%27Alembert\%27s_principle}{相关文章}。

\begin{figure}[ht]
\centering
\includegraphics[width=8cm]{./figures/f3acd791f8ea3d45.png}
\caption{让·勒朗·达朗贝尔的《动力学论》,1743年。在这本书中,这位法国学者阐述了动量原理,也被称为“达朗贝尔原理”。} \label{fig_DAP_1}
\end{figure}
达朗贝尔原理,也称为拉格朗日–达朗贝尔原理,是经典运动定律的基本表述。它以其发现者法国物理学家和数学家让·勒朗·达朗贝尔以及意大利-法国数学家约瑟夫·路易·拉格朗日命名。达朗贝尔原理通过引入惯性力,将虚功原理从静力系统推广到动力系统,当这些惯性力与系统中的外力相加时,系统达到动态平衡状态。[1][2]

达朗贝尔原理可以应用于依赖于速度的运动约束情形。[1]: 92  该原理不适用于不可逆位移(如滑动摩擦),并且需要更一般的不可逆性说明。[3][4]
\subsection{原理陈述}
该原理表述为:作用在一个由大量粒子组成的系统上的力与系统本身的动量的时间导数之间的差的和,在投影到与系统约束一致的任何虚位移上时,其总和为零。[需要澄清] 因此,用数学符号表示,达朗贝尔原理写作:
\[
\sum_{i} \left( \mathbf{F}_i - m_i \dot{\mathbf{v}}_i - \dot{m}_i \mathbf{v}_i \right) \cdot \delta \mathbf{r}_i = 0,~
\]
其中:
\begin{itemize}
\item \( i \) 是一个整数,用于表示(通过下标)系统中对应于特定粒子的变量,
\item \( \mathbf{F}_i \) 是作用在第 \( i \) 个粒子上的总外力(不包括约束力),
\item \( m_i \) 是第 \( i \) 个粒子的质量,
\item \( \mathbf{v}_i \) 是第 \( i \) 个粒子的速度,
\item \( \delta \mathbf{r}_i \) 是与约束一致的第 \( i \) 个粒子的虚位移。
\end{itemize}
牛顿的点符号用于表示相对于时间的导数。上述方程通常被称为达朗贝尔原理,但这种变分形式首次是由约瑟夫·路易·拉格朗日书写的。[5] 达朗贝尔的贡献在于证明在一个动态系统的整体中,约束力消失。也就是说,广义力 \( \mathbf{Q}_j \) 不需要包含约束力。该原理等价于更为繁琐的高斯最小约束原理。
\subsection{推导}
\subsubsection{质量变化的一般情况}
\begin{figure}[ht]
\centering
\includegraphics[width=6cm]{./figures/b44472e436612a72.png}
\caption{让·达朗贝尔 (1717–1783)} \label{fig_DAP_2}
\end{figure}
达朗贝尔原理的一般表述中提到了“系统动量的时间导数”。根据牛顿第二定律,动量的第一时间导数即为力。第 \( i \) 个质量的动量是其质量与速度的乘积:
\[
\mathbf{p}_i = m_i \mathbf{v}_i~
\]
其时间导数为:
\[
\dot{\mathbf{p}}_i = \dot{m}_i \mathbf{v}_i + m_i \dot{\mathbf{v}}_i.~
\]
在许多应用中,质量是恒定的,此时该方程简化为:
\[
\dot{\mathbf{p}}_i = m_i \dot{\mathbf{v}}_i = m_i \mathbf{a}_i.~
\]
然而,一些应用涉及质量变化(例如,链条卷起或展开),在这些情况下,\(\dot{m}_i \mathbf{v}_i\) 和 \(m_i \dot{\mathbf{v}}_i\) 两项都必须保留,因此得到:
\[
\sum_{i} \left( \mathbf{F}_i - m_i \mathbf{a}_i - \dot{m}_i \mathbf{v}_i \right) \cdot \delta \mathbf{r}_i = 0.~
\]
\subsubsection{恒定质量的特殊情况}

考虑对于恒定质量粒子系统的牛顿定律,每个粒子 \( i \) 的总力为[6]
\[
\mathbf{F}_i^{(T)} = m_i \mathbf{a}_i,~
\]
其中:
\begin{itemize}
\item \( \mathbf{F}_i^{(T)} \) 是作用在系统中各粒子上的总力,
\item \( m_i \mathbf{a}_i \) 是由总力产生的惯性力。
\end{itemize}
将惯性力移到左侧,可以得到一个可以视为准静态平衡的表达式,但实际上这只是牛顿定律的小代数操作:[6]
\[
\mathbf{F}_i^{(T)} - m_i \mathbf{a}_i = 0.~
\]
考虑总力和惯性力通过系统的任意虚位移 \( \delta \mathbf{r}_i \) 所做的虚功 \( \delta W \),由于每个粒子上的总力和惯性力相互抵消,得到一个恒等式零:[6]
\[
\delta W = \sum_i \mathbf{F}_i^{(T)} \cdot \delta \mathbf{r}_i - \sum_i m_i \mathbf{a}_i \cdot \delta \mathbf{r}_i = 0.~
\]
通过认识到虚功表达式必须适用于任意位移,可以恢复原始向量方程。将总力分为外力 \( \mathbf{F}_i \) 和约束力 \( \mathbf{C}_i \),得到:[6]
\[
\delta W = \sum_i \mathbf{F}_i \cdot \delta \mathbf{r}_i + \sum_i \mathbf{C}_i \cdot \delta \mathbf{r}_i - \sum_i m_i \mathbf{a}_i \cdot \delta \mathbf{r}_i = 0.~
\]
如果假设任意虚位移的方向与约束力正交(这通常不是一般情况,因此该推导仅适用于特殊情况),则约束力不做功,\(\sum_i \mathbf{C}_i \cdot \delta \mathbf{r}_i = 0.\)这样的位移称为与约束一致的位移。[7] 这引出了达朗贝尔原理的表述,即对于动态系统,外力和惯性力之差不做虚功:[6]
\[
\delta W = \sum_i (\mathbf{F}_i - m_i \mathbf{a}_i) \cdot \delta \mathbf{r}_i = 0.~
\]
对于静态系统,还有一个对应的原理,称为外力的虚功原理。
\subsection{达朗贝尔的惯性力原理}
达朗贝尔指出,可以通过添加所谓的“惯性力”和“惯性力矩”或“惯性转矩”,将一个加速的刚体转换为一个等效的静力系统。惯性力必须作用在质心上,而惯性力矩可以作用在任何位置。这样,该系统可以像一个受惯性力矩和外力作用的静态系统那样被分析。其优势在于,在等效静力系统中,可以围绕任意一点取力矩(不仅限于质心)。这通常会简化计算,因为可以通过选择适当的力矩方程应用点(力矩和为零),依次消去力矩方程中的任一力。在机械学动力学和运动学基础课程中,这一原理有助于分析运动过程中作用在机构连杆上的力。在工程动力学教材中,这一原理有时被称为达朗贝尔原理。

一些教育者警告说,使用达朗贝尔的惯性力学的尝试可能导致学生频繁出现符号错误。[8] 这些错误的潜在原因之一是惯性力的符号。在相对于惯性参考系加速的非惯性参考系中,惯性力可以用来描述一种表观力。在这样的非惯性参考系中,一个在惯性参考系中没有受到力作用并且处于静止状态的质量,会表现出一个加速度 \(-\mathbf{a}\) 以及一个表观的惯性力,或称伪力或虚假力 \(-m\mathbf{a}\) 似乎作用在它上面:在这种情况下,惯性力带有负号。[8]
\subsection{动态平衡}
达朗贝尔形式的虚功原理表明,当刚体系统在任意虚位移下,所受外力和惯性力的总虚功为零时,该系统处于动态平衡。因此,对于具有 \( m \) 个广义坐标的 \( n \) 刚体系统,其动态平衡条件为:
\[
\delta W = \left(Q_1 + Q_1^{*}\right) \delta q_1 + \dots + \left(Q_m + Q_m^{*}\right) \delta q_m = 0,~
\]
其中,\( \delta q_j \) 是任意一组虚位移,\( Q_j \) 是广义外力,\( Q_j^{*} \) 是广义惯性力。该条件产生 \( m \) 个方程:
\[
Q_j + Q_j^{*} = 0, \quad j = 1, \dots, m,~
\]
也可以写成:
\[
\frac{d}{dt} \frac{\partial T}{\partial \dot{q}_j} - \frac{\partial T}{\partial q_j} = Q_j, \quad j = 1, \dots, m.~
\]
结果是一组 \( m \) 个运动方程,定义了刚体系统的动力学。
\subsection{使用拉格朗日函数的表述}
达朗贝尔原理可以用系统的拉格朗日函数 \( L = T - V \) 重写,作为哈密顿原理的广义版本,如下所示:
\[
\delta \int_{t_1}^{t_2} L(\mathbf{r}, \dot{\mathbf{r}}, t) \, dt + \sum_i \int_{t_1}^{t_2} \mathbf{F}_i \cdot \delta \mathbf{r}_i \, dt = 0,~
\]
其中:
\begin{itemize}
\item \( \mathbf{r} = (\mathbf{r}_1, \dots, \mathbf{r}_N) \) 表示系统中各粒子的位置,
\item \( \mathbf{F}_i \) 是作用在第 \( i \) 个粒子上的外力,
\item \( \delta \mathbf{r}_i \) 是与约束一致的第 \( i \) 个粒子的虚位移,\( \sum_i \mathbf{C}_i \cdot \delta \mathbf{r}_i = 0 \),表示约束条件。
\item 满足约束的关键曲线也满足以下约束条件:\(\sum_i \mathbf{C}_i \cdot \dot{\mathbf{r}}_i = 0.\)
\end{itemize}
使用拉格朗日函数
\[
L(\mathbf{r}, \dot{\mathbf{r}}, t) = \sum_i \frac{1}{2} m_i \dot{\mathbf{r}}_i^2,~
\]
可以恢复达朗贝尔原理的前述表述。