% 最小生成树
% Pirm|最小生成树|算法|C++

\begin{issues}
\issueMissDepend
\end{issues}

生成树的定义:是指在一个带权的无向联通图中选择 $n$ 个点和 $n - 1$ 条边构成的无向联通子图.

最小生成树的定义即为边权最小的生成树.

求最小生成树最常用的两种算法为:Prim 和 Kruskal.Prim 常用于稠密图,Kruskal 则相反.

\subsection{Prim 算法}

Prim 算法的思路与 Dijkstra 算法非常相似.
定义 $S$ 为当前已经确定了属于最小生成树的结点,$T$ 为集合外的结点.使用 \verb|dist| 数组存储每个结点到 $S$ 集合的距离,距离定义为如果有多个结点指向 $S$ 集合,则距离最短的边为这个结点到 $S$ 集合的距离.最开始初始化所有结点到 $S$ 集合的距离为 $+\infty$,$1$ 号点到 $S$ 集合的距离为 $0$.一共进行