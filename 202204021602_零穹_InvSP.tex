% 不变子空间
% 不变子空间

\pentry{线性算子代数\upref{LiOper}}
\begin{definition}{不变子空间}
子空间 $U\in V$ 相对于线性算子\upref{LiOper} $\mathcal{A}:V\rightarrow V$ 是\textbf{不变的},如果 $\mathcal{A}U\subset U$.
\end{definition}
\begin{example}{}
算子 $\mathcal{A}$ 的\textbf{核} $\mathrm{Ker}\;\mathcal A$ 和\textbf{像} $\mathrm{Im}\;\mathcal{A}$
\begin{equation}
\begin{aligned}
\mathrm{Ker}\;\mathcal{A}&=\{\bvec v\in V|\mathcal{A}\bvec v=\bvec 0\}\\
\mathrm{Im}\;\mathcal{A}&=\{\bvec w\in V|\bvec w=\mathcal{A}\bvec v,\forall\bvec v\in V\}
\end{aligned}
\end{equation}
都是 $\mathcal A$ 的不变子空间.
\end{example}
\begin{theorem}{}
$n$ 维空间 $V$ 是算子 $\mathcal{A}$ 的不变子空间 $U$( $m$ 维) 和 $W$ ($(n-m)$ 维)的直和,即 $V=U\oplus W$,当且仅当算子 $\mathcal{A}$ 的矩阵 $A$ 在某基底下具有分块对角形式
\begin{equation}
A=\begin{pmatrix}
A_U&0\\
0&A_W
\end{pmatrix}
\end{equation}
\end{theorem}
\textbf{证明:}1.\begin{equation}
V=U\oplus W
\end{equation}
