% 复合函数的偏导 链式法则
% keys 多元微积分|导数|偏导数|全微分|复合函数求导|链式法则

\pentry{全微分\upref{TDiff}, 复合函数求导\upref{ChainR}}


先来看一个二元函数的例子: 若已知二元函数 $z = f(u,v)$,$z$ 是 $u, v$ 的函数,但若 $u$ 和 $v$ 都又是 $x$ 和 $y$ 的函数,则 $z$ 最终是 $x$ 和 $y$ 的函数,即
\begin{equation}\label{PChain_eq1}
z(x,y) = f[u(x,y),v(x,y)]
\end{equation}
那如何求 $z(x,y)$ 的偏导数呢?我们先来看全微分关系\upref{TDiff}
\begin{equation}
\dd{z} = \pdv{f}{u} \dd{u} + \pdv{f}{v} \dd{v}
\end{equation}
而 $u$ 和 $v$ 作为 $x, y$ 的函数, 它们的微小变化又都是由 $x$ 和 $y$ 的微小变化引起的, 所以有全微分
\begin{equation}
\dd{u} = \pdv{u}{x} \dd{x} + \pdv{u}{y} \dd{y}
\quad
\dd{v} = \pdv{v}{x} \dd{x} + \pdv{v}{y} \dd{y}
\end{equation}
代入上式得
\begin{equation}\ali{
\dd{z} &= \pdv{f}{u} \qty( \pdv{u}{x} \dd{x} + \pdv{u}{y} \dd{y} ) + \pdv{f}{v} \qty( \pdv{v}{x}\dd{x} + \pdv{v}{y} \dd{y} ) \\
   &= \qty( \pdv{f}{u}\pdv{u}{x} + \pdv{f}{v}\pdv{v}{x} )\dd{x} + \qty( \pdv{f}{u}\pdv{u}{y} + \pdv{f}{v}\pdv{v}{y} ) \dd{y}
}\end{equation}
这就是 $z$ 关于 $x$ 和 $y$ 的全微分关系.根据偏导数的定义
\begin{equation}
\pdv{z}{x} = \pdv{f}{u}\pdv{u}{x} + \pdv{f}{v}\pdv{v}{x}
\end{equation}
\begin{equation}
\pdv{z}{y} = \pdv{f}{u}\pdv{u}{y} + \pdv{f}{v}\pdv{v}{y}
\end{equation}
这也叫偏导的\textbf{链式法则}. 为了方便我们也会把 $\pdv*{z}{x}$ 和 $\pdv*{z}{y}$ 分别记为 $\pdv*{f}{x}$ 和 $\pdv*{f}{y}$.

我们可以把以上的例子拓展到任意多个变量的情况, 即令
\begin{equation}\label{PChain_eq2}
z(x_1, \dots, x_N) = f[u_1(x_1, \dots, x_N), \dots, u_M(x_1, \dots, x_N)]
\end{equation}
这时链式法则可以记为
\begin{equation}
\pdv{f}{x_i} = \sum_j \pdv{f}{u_j}\pdv{u_j}{x_i}
\end{equation}

\begin{example}{}
令 $u(x,y) = x + y$, $v(x,y) = x - y$, $f(u, v) = u^2 + v^2$, 求 $\pdv*{f}{x}$ 和 $\pdv*{f}{y}$.

解: 直接套用\autoref{PChain_eq1} 得
\begin{equation}
\pdv{f}{x} = 2u \cdot 1 + 2v \cdot 1 = 4x
\end{equation}
\begin{equation}
\pdv{f}{y} = 2u \cdot 1 + 2v \cdot (-1) = 4y
\end{equation}

为了验证,我们也可以直接写出
\begin{equation}
f(x, y) = (x+y)^2 + (x-y)^2 = 2x^2 + 2y^2
\end{equation}
再直接求偏导得 $\pdv*{f}{x} = 4x$, $\pdv*{f}{y} = 4y$, 可见结果也是相同的.
\end{example}

\subsection{通用函数名}
物理中常常会出现一种容易混淆的情况,就是当一个因变量可以有几套自变量(例如上面的 $z(u,v)$ 和 $z(x,y)$)时,通常直接用因变量($z$)作为函数名而另外不定义函数名($f$).然而 $z(u,v)$ 与 $z(x,y)$ 中的 $z$ 并不是同一个函数.以下举例说明

\begin{example}{}\label{PChain_ex1}
在二维直角坐标系中,定义势能函数为
\begin{equation}\label{PChain_eq7}
V=f(x,y)=x^2+y^2+2x
\end{equation}
而若用极坐标\upref{Polar}描述该势能, 则函数变为
\begin{equation}\label{PChain_eq8}
V = g(r,\theta) = f(r\cos \theta , r\sin \theta ) = r^2 + 2r\cos \theta
\end{equation}
但许多物理书为了表述方便并不用 $f$ 和 $g$ 区分两个不同的函数, 而是使用 $V(x,y)$ 表示\autoref{PChain_eq7} 和 $V(r,\theta)$ 表示\autoref{PChain_eq8}.这样后者就有可能被误解为
\begin{equation}
V(r,\theta) = r^2+\theta^2+2r \quad \text{(错)}
\end{equation}
这就需要从语境中判断是否使用了\textbf{通用函数名}\footnote{“通用函数名”是笔者起的名字, 不清楚是否有其他叫法.}.

使用通用函数名时,要注意判断偏导数使用的是哪一套变量,例如 $\pdv*{V}{x}$ 默认使用 $V(x,y)$ 求偏导, 即把 $y$ 看成常数; $\pdv*{V}{r}$ 默认使用 $V(r,\theta)$ 求偏导, 即把 $\theta$ 看成常数.一种更复杂的情况如 $(\pdv*{V}{x})_\theta$. 按照定义\footnote{见偏导数\upref{ParDer}中的\autoref{ParDer_eq1}.},应该是仅用 $x$ 和 $\theta$ 表示 $V$,然后求偏导.考虑极坐标的定义,$\theta$ 不变意味着 $y$ 与 $x$ 成正比即 $y=x\tan\theta$,代入\autoref{PChain_eq7} 得
\begin{equation}
V(x,\theta)=x^2(1+\tan^2 \theta) + 2x
\end{equation}
现在再对 $x$ 求偏导即可(略).
\end{example}

\subsection{显含}
在物理中, 尤其在分析力学中, 我们通常会遇见\textbf{显含(explicitly depends on)}的概念. 当
\begin{equation}
f(u_1, \dots, u_M)
\end{equation}
中的某个 $u_i$ 满足 $\pdv*{f}{u_i}$ 不恒为零时, 我们说 $f$ \textbf{显含} $u_i$. 特殊地, 当某个 $u_i(x_1, \dots, x_N) = x_j$ 时, 我们就说函数 $f$ \textbf{显含} $x_j$, 否则就说 $f$ \textbf{不显含} 或者 \textbf{隐含(implicitly depends on)} $x_j$.

\begin{example}{}
一个质点延着一条静止轨道 $y = 0$ 运动, 它的动能\upref{KELaw1}为 $E(v_x) = m v_x^2$. 虽然速度 $v_x$ 是时间 $t$ 的函数, 但我们说 $E(v_x)$ 不显含时间.

若轨道在 $y$ 方向具有随时间变化的速度 $v_y(t) = a t^2$, 如果把动能记为 $E(v_x, v_y) = m(v_x^2 + v_y^2)$, 那么该函数仍然不显含 $t$.

仍然令 $v_y(t) = a t^2$, 但把动能记为 $E(v_x, t) = m v_x^2 + m (a t^2)^2$, 那么该函数就显含 $t$.

虽然在后两种情况中物理情景都是一样的, 但 $E(v_x, v_y)$ 和 $E(v_x, t)$ 在数学上却是两个不同的函数, 使用了通用函数名 $E$. 这两个函数分别是 $f(x, y) = m(x^2 + y^2)$ 和 $g(x, y) = mx^2 + ma^2 y^4$, 显然不是同一个函数. 它们都显含 $x, y$, 不显含其他任何变量.
\end{example}
