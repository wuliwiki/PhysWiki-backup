% 最大期望算法
% license CCBYSA3
% type Wiki

(本文根据 CC-BY-SA 协议转载自原搜狗科学百科对英文维基百科的翻译)

在统计学中,期望最大化算法(EM)是一种迭代方法,用于寻找统计模型中参数的最大似然或最大后验估计,其中模型依赖于未观察到的潜变量。EM迭代在执行求期望(E)步骤和最大化(M)步骤之间交替进行,求期望(E)步骤创建了一个函数,是用参数当前估计值所估计得到的对数似然的期望的函数,最大化(M)步骤计算参数,使其能够最大化在E步骤中计算得到的对数似然的期望。然后这些参数估计值被用于确定下一个E步骤中的潜变量分布。

\begin{figure}[ht]
\centering
\includegraphics[width=8cm]{./figures/66edbd8e9bf77549.png}
\caption{Old Faithful eruption数据集(R语言中典型的数据集)上的EM聚类过程。随机初始化的模型(由于轴之间的规模不同,显示的是两个非常扁平又宽的球体)用于拟合观测数据。在第一次迭代中,模型的变化非常可观,但之后就在两步之间间歇收敛。由ELKI可视化。} \label{fig_ZDQW_1}
\end{figure}

\subsection{历史}



\subsection{介绍}



\subsection{描述}



\subsection{性能}



\subsection{正确性的证明}



\subsection{作为最大化-最大化过程}



\subsection{应用}



\subsection{滤波和平滑电磁算法}



\subsection{变体}



\subsubsection{9.1 α-EM算法}



\subsection{与变分贝叶斯方法的关系}



