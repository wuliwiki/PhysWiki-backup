% 线性方程组的解
% keys 线性方程组|矢量空间|秩|矩阵
% license Xiao
% type Tutor

\begin{issues}
\issueDraft
\end{issues}
% Giacomo:应该拆散进不同文字中

\pentry{线性方程组\nref{nod_LinEqu},矩阵与线性变换\nref{nod_MatLS}, 高斯消元法\nref{nod_GAUSS}, 矩阵的秩\nref{nod_MatRnk}}{nod_1edb}

线性方程组可以记为
\begin{equation}\label{eq_LinEq_1}
\mat A \bvec x = \bvec b~,
\end{equation}
其中 $\mat A$ 是 $M \times N$ 的矩阵, $\bvec x$ 是 $N$ 维列矢量, $\bvec b$ 是 $M$ 维列矢量, $\mat A \bvec x$ 表示矩阵与列矢量相乘(\autoref{eq_Mat_4}~\upref{Mat})。 $\mat A$ 和 $\bvec b$ 是已知的, $\bvec x$ 是未知的, 被称为方程组的\textbf{解(solution)}。

线性方程组$\mat A \bvec x=\bvec b$可以有零个解(无解)、一个解(唯一解)或无数解;但不可能只有 $2,3,4,\dots$ 个解。

% 求解$\mat A x = b$即相当于求解一组线性方程:
% \begin{equation}
% \begin{cases}
% A_{11}x_1+A_{12}x_2+A_{13}x_3+...&=b_1\\
% A_{21}x_1+A_{22}x_2+A_{23}x_3+...&=b_2\\
% A_{31}x_1+A_{32}x_2+A_{33}x_3+...&=b_3\\
% ...
% \end{cases}
% \end{equation}

\subsection{线性方程组的几何含义}\label{sub_LinEq_2}
可以分别从行与列的角度来理解线性方程组的几何含义。\footnote{本文参考了Gilbert Strang的《线性代数》课程,J. Leon 的Linear Algebra with Applications,以及李永乐等的线代考研课程}更深刻、数学的表示可以见线性方程组的仿射解释\upref{AS2LF}。

\subsubsection{行视角}
$$
\ali{\mat A \bvec x=\bvec b &\Longleftrightarrow 
\begin{pmatrix}
A_{11}&A_{12}&A_{13}&...\\
A_{11}&A_{12}&A_{13}&...\\
A_{11}&A_{12}&A_{13}&...\\
...
\end{pmatrix}
\begin{pmatrix}
x_{1}\\
x_{2}\\
x_{3}\\
...\\
\end{pmatrix}
=
\begin{pmatrix}
b_{1}\\
b_{2}\\
b_{3}\\
...\\
\end{pmatrix}\\
&\Longleftrightarrow
\begin{cases}
b_1&=A_{11}x_1+A_{12}x_2+A_{13}x_3+...\\
b_2&=A_{21}x_1+A_{22}x_2+A_{23}x_3+...\\
b_3&=A_{31}x_1+A_{32}x_2+A_{33}x_3+...\\
...&\\
\end{cases}}
~
$$

这相当于求解一组线性方程。其中每一条方程可以看作 $N$ 维欧几里得空间中的 $N-1$ 维超平面(一维超平面就是直线,二维超平面是通常的平面等),方程组的解就是这些超平面的交点的集合。

\begin{example}{}
例如,求解
$$
\begin{pmatrix}
2&-1\\
-1&2\\
\end{pmatrix}
\begin{pmatrix}
x_1\\
x_2\\
\end{pmatrix}
=
\begin{pmatrix}
0\\
3\\
\end{pmatrix}~.
$$
即
$$
\begin{cases}
2x_1-x_2=0\\
-x_1+2_2=3~.\\
\end{cases}
$$
\begin{figure}[ht]
\centering
\includegraphics[width=8cm]{./figures/ed701f2472333a9c.pdf}
\caption{行视角下的线性方程组。解可以理解为直线的交点。仿自Strang的《线性代数》} \label{fig_LinEq_2}
\end{figure}
\end{example}
从这个角度可以很直观的理解“无解”、“无数解”。很显然,一组直线不能仅有两个交点,所以线性方程组也不可能只有两个解。
\begin{figure}[ht]
\centering
\includegraphics[width=12cm]{./figures/d89a22b5cc50341b.pdf}
\caption{“无解(没有交点)”、“唯一解(有唯一交点)”、“无数解(有无数交点)”} \label{fig_LinEq_3}
\end{figure}

或者用分块矩阵\upref{BlkMat}的视角,将$\mat A$划分为若干行向量$\bvec \alpha_1, \bvec \alpha_2,\bvec \alpha_3...$,让表述变得更加简洁。
$$
\mat A \bvec x=\bvec b \Leftrightarrow 
\begin{pmatrix}
\bvec \alpha_1\\
\bvec \alpha_2\\
\bvec \alpha_3\\
\vdots
\end{pmatrix}
\bvec x
=
\begin{pmatrix}
b_{1}\\
b_{2}\\
b_{3}\\
\vdots
\end{pmatrix}
\Leftrightarrow 
\begin{cases}
b_1&=\bvec \alpha_1 \bvec x\\
b_2&=\bvec \alpha_2 \bvec x\\
b_3&=\bvec \alpha_3 \bvec x\\
&\vdots
\end{cases}~
$$

\subsubsection{列视角}
运用分块矩阵\upref{BlkMat}的视角,将$\mat A$划分为若干列向量$\bvec \alpha_1, \bvec \alpha_2,\bvec \alpha_3...$
$$
\mat A \bvec x=\bvec b \Leftrightarrow 
\begin{pmatrix}
\bvec \alpha_1& \bvec \alpha_2& \bvec \alpha_3&...
\end{pmatrix}
\begin{pmatrix}
x_{1}\\
x_{2}\\
x_{3}\\
...\\
\end{pmatrix}
=
\bvec b
\Leftrightarrow 
\bvec b = \bvec \alpha_1 x_1+\bvec \alpha_2 x_2+\bvec \alpha_3 x_3+...~
$$
此时,$\bvec b$可以看作是一系列$\bvec \alpha_i$的线性组合,而解$\bvec x$是各个列向量的“系数”。方程组无解的含义即为“$\bvec b$不能由$\bvec \alpha_i$线性组合得到”,无数解的含义即为“有无数种方法线性组合$\bvec \alpha_i$以得到$\bvec b$”,这暗示了这一系列$\bvec \alpha_i$中存在线性相关的项。

\begin{example}{}
还是例如求解
$$
\begin{pmatrix}
2&-1\\
-1&2\\
\end{pmatrix}
\begin{pmatrix}
x_1\\
x_2\\
\end{pmatrix}
=
\begin{pmatrix}
0\\
3\\
\end{pmatrix}~,
$$
即
$$
\begin{pmatrix}
2\\
-1\\
\end{pmatrix}
x_1
+
\begin{pmatrix}
-1\\
2\\
\end{pmatrix}
x_2
=
\begin{pmatrix}
0\\
3\\
\end{pmatrix}~.
$$
\begin{figure}[ht]
\centering
\includegraphics[width=8cm]{./figures/a340417f57880d30.pdf}
\caption{列视角下的线性方程组。解可理解为各列向量的“系数”。仿自Strang的《线性代数》} \label{fig_LinEq_4}
\end{figure}
\end{example}
看起来,在线性方程组中,矩阵$\mat A$的行向量与列向量存在一种微妙的关联。

\subsection{判断线性方程组解的个数}
\begin{figure}[ht]
\centering
\includegraphics[width=14.25cm]{./figures/0cb8501eb2baf4ae.pdf}
\caption{$Ax=b$ 的解} \label{fig_LinEq_1}
\end{figure}

\addTODO{需要补充证明}%我不会证明QAQ

关于线性方程组的解,我们有如下定理。记m为$\mat A$的行数,n为$\mat A$的列数,r为$\mat A$的矩阵的秩\upref{MatRnk}。可参考\autoref{fig_LinEq_1} 的分类。

\begin{theorem}{$\mat A \bvec x=\bvec b$解的唯一性}
若 $n-r=0$,若解存在,则解唯一。

若 $n-r>0$,若解存在,则解不唯一。
\end{theorem}
n-r事实上是 $\mat A$ 的零空间的基个数 $\opn{dim}(\opn{Nul}(\mat A))=n-r$。

\begin{theorem}{$\mat A \bvec x=\bvec b$解的存在性}
若 $m-r=0$,则解一定存在。

若 $m-r>0$,则解可能不存在。
\end{theorem}
m-r事实上是$\mat A$的左零空间的基个数$dim(Nul(\mat A^T))=m-r$。

\begin{theorem}{$\mat A \bvec x=\bvec b$ 解的存在性 2}
若$\bvec b$是$\mat A$的列向量的线性组合($\bvec b \in Col(\mat A)$),则$\mat A \bvec x=\bvec b$一定存在解;否则无解。
\end{theorem}
由列视角(\autoref{sub_LinEq_2} )看,这是显然的。

\begin{corollary}{}
设 $\overline{\mat A} = [\mat A | \bvec b] $ 为A的增广矩阵。

$\mat A \bvec x=\bvec b$无解:$\opn{rank}(\overline{\mat A})=\opn{rank}(\mat A)+1$

唯一解:$\opn{rank}(\overline{\mat A})=\opn{rank}(\mat A)=n$

无数解:$\opn{rank}(\overline{\mat A})=\opn{rank}(\mat A)<n$
\end{corollary}
考虑到秩的含义,结合上述定理,也容易理解该推论。

\begin{theorem}{$\mat A \bvec x=\bvec 0$解的唯一性}
若 $n-r=0$,则只有一个解 $\bvec x = \bvec 0$.

若 $n-r>0$,则解不唯一,且存在(n-r)个线性无关的解。
\end{theorem}
这(n-r)个解即为零空间的(n-r)个基。

\begin{theorem}{$\mat A \bvec x=0$解的存在性}
一定存在平凡解 $\bvec x=0$
\end{theorem}

\begin{theorem}{解的结构}
$\mat A \bvec x=\bvec b$的通解等于$\mat A \bvec x=\bvec b$的一个特解加上$\mat A \bvec x=0$的各个线性无关的解的线性组合,即$\bvec x = \bvec x_p +k_1\bvec x_{n1}+k_2\bvec x_{n2}+...$

因此,通常的解题套路是先求解$\mat A \bvec x=\bvec b$的一个特解,再求解$\mat A \bvec x=0$的各个线性无关的解。

(若n-r=0,则$\mat A \bvec x=0$仅有解$\bvec x_n=0$,$\mat A \bvec x=\bvec b$自然只有一个解$\bvec x = \bvec x_p$)
\end{theorem}

更深入的探讨详见下文:

从矢量空间\upref{LSpace}的角度来看, $\bvec x$ 是一个 $N$ 维矢量空间(以下称为 $X$ 空间)中一个矢量关于某组基底的坐标, $\bvec b$ 是一个 $M$ 维矢量空间(以下称为 $Y$ 空间)中一个矢量关于某组基底的坐标。 矩阵 $\mat A$ 可以将 $X$ 空间中的任意矢量映射到 $Y$ 后的坐标。

我们知道 $\mat A$ 的第 $i$ 列代表的矢量就是 $X$ 空间中的第 $i$ 个基底映射到 $Y$ 空间的对应矢量。 我们把 $A$ 的 $N$ 列对应的 $N$ 个矢量记为 $\{\bvec \alpha_i\}$。 先来看一个定理

\subsection{满秩方阵}
我们知道矩阵的秩 $R$ 等于线性无关的行数或列数\upref{MatRnk}, 下面来根据秩来分类讨论方程组的解空间结构。 最简单的情况是 $\mat A$ 为满秩, 即 $R = M = N$。 这时由于 $\{\bvec \alpha_i\}$ 两两线性无关, 它们可以作为 $Y$ 空间的一组基底, 与 $X$ 空间的基底一一对应。 那么这个映射既是单射又是满射。%未完成:引用
对于 $Y$ 空间的任意矢量 $\bvec b$, $X$ 空间都存在唯一的解 $\bvec x$。 特殊地,当 $\bvec b = \bvec 0$ 时(即方程是\textbf{齐次}的),唯一解就是 $\bvec x = \bvec 0$。

\subsection{$R = M < N$}\label{sub_LinEq_1}

当 $\mat A$ 的秩等于 $M$ 且小于 $N$ 时, 映射变为从 $N$ 维空间到更小的 $M$ 维空间。 即非单射: 虽然任意的 $\bvec x$ 仍然映射到唯一的 $\bvec b$, 但任意的 $\bvec b$ 却对应无穷多个 $\bvec x$。 

% 举例未完成:三维矢量投影到二维矢量

\addTODO{引用矩阵版本的零空间}
当方程是齐次的时候, 零空间$X_0$ 是 $N- M$ 维的(为什么?)。 这种情况下,我们希望能解出零空间的 $N - M$ 个基底,使得这组基底的任意线性组合都是齐次方程的解。

对于非齐次方程, 我们可以先求对应的齐次方程组的零空间的一组基底,再求出非齐次方程的任意一个解(\textbf{特解}), 那么非齐次方程组的\textbf{解集}(所有解的集合)就等于零空间中的所有矢量与特解相加。 注意非齐次方程的解集并不构成一个矢量空间, 因为它不包含零矢量($\bvec x = \bvec 0$ 总是对应 $\bvec b = \bvec 0$, 所以不可能是非齐次方程组的解),解集中若干矢量的线性组合也不一定仍然属于解集。
\addTODO{证明}

\subsection{$R < M$}
当 $R < M$ 时, $\{\bvec \alpha_i\}$ 中只有 $R$ 个线性无关, 它们在 $Y$ 空间中张成\upref{VecSpn}一个 $R$ 维子空间 $Y_0$。 如果 $\bvec b$ 在 $Y_0$ 中(可以通过 $\bvec b$ 是否与 $\{\bvec \alpha_i\}$ 线性无关来判断), 方程组就存在解, 如果落在子空间外, 方程组就无解。

\footnote{另见 “线性变换与矩阵的代数关系\upref{linmat}” 的 \autoref{the_linmat_2}。}
