% 刘维尔定理

多维空间中的流体, 跟随一点时, 周围密度不随时间变化.

$(q, p)$ 空间称为相空间. $t$ 等于零时在相空间中取一块小区域, 具有边界 $\mathcal B$. 可以证明随着时间变化, 虽然边界开始变形, 但边界两边的点不会跨越边界. 也可以证明, 这个区域的体积始终保持不变.

刘维尔定理的一个最直接推论是, 如果开始时相空间中这种流体的密度处处相同, 那么接下来在任意时刻 $t$, 流体密度仍然处处相同. 而这种状态通常就是热力学系统的平衡态.

\subsection{难点}
对于足够复杂的系统, 即使开始时流体密度并不均匀, 在经过足够长的时间以后也会在相空间中变得均匀. 这看似违背了刘维尔定理, 但从严格的数学角度来说是没有的, 只是原来的体积被非常均匀地 “揉” 到了整个相空间中. 想象我们在一块白色的橡皮泥中混入一小块黑色橡皮泥, 假设两种橡皮泥之间始终保持明确的边界, 当我们不停地随机揉动橡皮泥, 经过很长一段时间之后, 虽然边界依然明确, 但是从更宏观的角度来看两种橡皮泥已经混合均匀了.
