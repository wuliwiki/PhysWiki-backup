% 反射变换(高等代数)
% license Usr
% type Tutor

\begin{issues}
\issueMissDepend
\end{issues}

反射变换的概念来源于平面上的轴反射。根据初中几何学里就已接触的轴反射定义,我们可以用矩阵表示关于$y$轴的线性变换。
\begin{equation}
\mat M=\pmat{0&1\\-1&0}~.
\end{equation}
可以验证,$\mat M\bvec e_x=-\bvec e_x,\mat M \bvec e_y=\bvec e_y$。因此这确实是一个保持向量在平行轴方向的分量不变,垂直轴的方向反向的轴反射。

任意图像在轴反射后形状不变,也就是说,这是一个保距变换。同理,我们也可以对任意向量作关于平面的反射,或者更延伸一些,在$\mathbb R^n$中作关于超平面$\mathbb R^{n-1}$的反射。(下文将超平面简称为平面)
\begin{definition}{}
设$S$是$n$维向量空间$\mathbb R^n$中的平面,$\bvec n$是其\textbf{单位法向量},对于任意$\bvec x\in \mathbb R^n$,定义其\textbf{反射(reflection)}$R:\mathbb R^n\rightarrow \mathbb R^n$为
\begin{equation}
R(\bvec x)=\bvec x-2(\bvec x,\bvec n)\bvec n~.
\end{equation}
\end{definition}
可以验证,$R$是保距变换。


\begin{theorem}{}
保距变换一定是反射变换的复合。
\end{theorem}
\textbf{证明:\footnote{参考来源:\href{https://kconrad.math.uconn.edu/blurbs/grouptheory/isometryRn.pdf}{Notes of KEITH CONRAD}。}}

在$n=1$的时候,保距变换要么是恒等变换,要么是反射变换,因为只有这两种变换不改变基向量的范数。因此,定理自然成立。

假设定理对$1<n<k$成立,设$\mathbb R^{k-1}$上的保距变换可以写为至多$m$个反射变换的复合。 

假设$f$是$\mathbb R^k$中的保距变换,且不是恒等变换。设$f(\bvec x)=\bvec y,\bvec z=\bvec y-\bvec x$,$S_z$为以$\bvec z$为法向量的超平面,并用$R_z$表示关于该超平面的反射变换。不失一般性,设$\bvec z$为单位法向量,可以验证:$R_z\circ f(\bvec x)=\bvec x$且$R_z\circ f(\bvec y)=x$\footnote{也就是$R_z(\bvec x)=\bvec y$,可直接验证得到。},则$R_z\circ f(\bvec z)=\bvec z$。即$\bvec z$是$R_z\circ f$的不变子空间。在欧几里得空间中,保距变换即为正交变换,且其复合依然保距,根据\autoref{lem_ortho_2},$S_z$也为$R_z\circ f$的不变子空间。

根据假设,$R_z\circ f|_{S_z}$可写为最多$m$个反射变换的复合。设为
\begin{equation}
R_z\circ f|_{S_z}=R_1R_2...R_{m}~,
\end{equation}
设$\bvec e_i\in S_z$,是对应超平面的单位法向量。

拓展$R_i$为对全空间向量作用的$R'_i$,各单位法向量对应的超平面从$n-2$维变为$n-1$维,且$R'_i(\bvec z)=\bvec z$,容易验证该定义是自洽合理的\footnote{即保持在超平面上的限制形式依旧。}。

现在我们有$R_z \circ f=R'_1R'_2...R'_{m}$,则$f=R_zR'_1R'_2...R'_{m}$,所以不仅定理成立,而且\textbf{$n$维欧几里得空间中保持原点不变的保距变换是最多$n$个反射变换的复合\footnote{从$n=1$开始推导,可知非平凡的保距变换必是反射变换,且次数小于或等于空间维数。}}。
\addTODO{还需要增加一些实例}