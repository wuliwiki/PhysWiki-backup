% 傅里叶变换(三角)
% 微积分|傅里叶级数|傅里叶变换|三角傅里叶变换|极限

\begin{issues}
\issueDraft
\end{issues}

\pentry{傅里叶级数(三角)\upref{FSTri}}
平方可积的函数都可以用正弦和余弦函数来展开
\begin{equation}
f(x) = \frac{1}{\sqrt{\pi}}\int_0^{+\infty} A(k) \cos(kx) + B(k)\sin(kx) \dd{k}
\end{equation}
其中
\begin{equation}
\begin{aligned}
&A(k) = \frac{1}{\sqrt{\pi}}\int_{-\infty}^{+\infty} \cos(kx) f(x) \dd{x}\\
&B(k) = \frac{1}{\sqrt{\pi}}\int_{-\infty}^{+\infty} \sin(kx) f(x) \dd{x}
\end{aligned}
\end{equation}

为了把整个实数范围的非周期函数用三角函数展开,在三角傅里叶级数中我们取极限 $l\to\infty$ 这时角频率的间隔趋于 0.
\begin{equation}
\Delta k = k_{n+1} - k_n = \frac{(n+1)\pi}{l} - \frac{n\pi}{l} = \frac{\pi}{l} \to 0
\end{equation}
