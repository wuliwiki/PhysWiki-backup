% 变分
% 变分|微分

\pentry{可取曲线(变分学)\upref{DesCur},绝对极值与相对极值(变分学)\upref{AbPol}}
泛函是函数概念中自变数用函数替换的推广,而变分则是函数微分在泛函里的推广.简单来说,正如微分为函数改变量的主要线性部分,变分是泛函改变量的主要线性部分.另一方面,若将函数和泛函用参数的形式表示,则函数$\phi(t)=f(x_1+th_1,\cdots,x_n+th_n) $ 的微分是 $\phi(t)$ 对于 $t$ 在 $t=0$ 上的微商,而泛函 $\varphi(t)=J(y+t\eta)$ 的变分是 $\varphi(t)$ 对于参变数 $t$ 在 $t=0$ 时的微商.下面来具体讨论!
\subsection{微分}
先简略复习一下多元函数微分的定义是有必要的.

设已给函数 $f(x_1,x_2,\cdots,x_n)$ 具有一阶连续偏微商,则有
\begin{equation}
f(x_1+h_1,x_2+h_2,\cdots,x_n+h_n)-f(x_1,x_2,\cdots,x_n)=\sum_{i=1}^{n} \pdv{f}{x_i}h_i+\epsilon
\end{equation}
其中 $\epsilon$ 是相对于 $\abs{h_i}$ 中最大者(或 $\sqrt{h_1^2+h_2^2+\cdots+h_n^2}$ )的 高阶无穷小.而 $\sum\limits_{i=1}^{n} \pdv{f}{x_i}h_i$ 是改变量 $h_1,h_2,\cdots,h_n$ 的线性函数,称为函数 $f(x_1,x_2,\cdots,x_n)$ 的\textbf{微分}.