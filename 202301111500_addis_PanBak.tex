% 用网盘增量备份文件
% 备份|网盘|增量备份|极速秒传

对任何人来说,文件备份有多重要这里就不多说了,有能力的同学完全可以自己搭建一个云备份系统,但本文介绍一种成本极低且安全高效的备份方案。

首先要介绍一些网盘的 “极速秒传” 功能: 当你要上传一个文件, 如果你或者其他任何网盘用户已经传过一个内容完全相同的文件(文件名不同没关系),那么客户端就会帮你秒传。 也就是说客户端并不需要真的把这个文件重新上传一次, 而是直接在你网盘中生成服务器上某个已有文件的 “快捷方式”。 这是一个十分强大的功能,但也有弊端: 如果网盘审核出某个文件违反了 “相关规定”, 那么网盘就会将其替换成一个违规通知, 那么这个文件的所有“快捷方式”也都变成了这个通知。 另一方面,如果你的文件没有加密,你的文件可能有泄露的风险(例如账号被盗)。

这里推荐的方法是, 把要备份的文件夹(假设名为 \verb|backup|)中的每个文件都分别加密压缩, 然后上传到网盘, 将其重命名为例如 \verb|backup-2020-01-01|。 注意是每个文件压缩而不是整个文件夹压缩成一个大压缩包(方法见下文)。 如果文件夹较大,第一次上传无疑需要较长时间(因为网盘不可能已经有你加密过的文件, 无法秒传)。 但从第二次开始我们就可以 “增量备份” 了, 例如过了几天, \verb|backup| 文件夹中增添/删除/重命名了一少部分文件, 我们再次将这个文件夹的每个文件分别用\textbf{相同的密码}加密压缩再上传到网盘,重命名为 \verb|backup-2020-01-07|。 这时你会发现没有改变过的文件(包括重命名的)都会被秒传, 只有内容改变了的或者新增的文件才需要真正上传, 所以这个过程所用的时间将比第一次大大缩短!这就是为什么我们要将每个文件分别压缩而不是一起压缩。

注意无论是否秒传,网盘都会按照文件的实际大小计算你所用的空间(这也是为什么网盘能以较低的会员费提供如此巨大的空间)。 如果经过多次备份空间不够用了,那么可以在客户端上找到 “清理重复文件” 的功能, 清理时可以选择保留最新版本的文件。 清理完成后, 所有内容完全一样的文件都只会再最新版本的备份文件夹中保留, 使占用空间大大降低。 但这样的弊端是, 旧文件夹中的重复文件被删除。

\begin{lstlisting}[language=bash]
# 文件加密
openssl aes-256-cbc -nosalt -in 文件名 -out 加密文件名 -pass pass:密码
# 文件解密
openssl aes-256-cbc -nosalt -d -in 加密文件名 -out 文件名 -pass pass:密码
# 字符串加密
echo '需要加密的字符串' | openssl enc -base64 -e -aes-256-cbc -nosalt -pass pass:密码
# 字符串解密
echo "需要解密的字符串" | openssl enc -base64 -d -aes-256-cbc -nosalt -pass pass:密码
\end{lstlisting}

\subsection{批量加密压缩文件夹中的文件}

(未完成:这里发现一个问题,似乎即使用同一个命令压缩出来的文件 hash 也是不同的, 所以并不能触发秒传, 我自己写过一个程序可以保证结果一致, 但有没有 linux 的自带程序可以办到?)

关于这个操作,相信对 Bash 命令行\upref{Linux}有一定了解的同学来说不是难事。

在 \verb|backup| 文件夹中打开 bash 命令行(Win10 系统可以使用 WSL, Mac 系统直接用 Terminal)中安装 \verb|7zip| 压缩程序:
\begin{lstlisting}[language=bash]
sudo apt install p7zip-full
\end{lstlisting}
将每个文件(包括子文件夹中的)分别用 \verb|密码| 加密压缩, 保存到 \verb|backup| 所在文件夹的 \verb|输出| 文件夹()。
\begin{lstlisting}[language=bash]
find . -type f -exec 7z a ../输出/{}.7z -p密码 {} \;
\end{lstlisting}
要解压, 在 \verb|输出| 中打开命令行, 以下命令将每个子文件夹中的压缩包分别解压并删除压缩包
\begin{lstlisting}[language=bash]
find . -type f -name "*.7z" -execdir 7z x {} -p密码 \; -exec rm {} \;
\end{lstlisting}
