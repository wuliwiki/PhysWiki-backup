% 陕西师范大学 2005 年 考研 量子力学
% license Usr
% type Note

\textbf{声明}:“该内容来源于网络公开资料,不保证真实性,如有侵权请联系管理员”

\subsection{(15 分)}
粒子在一维势场$U(x)$中运动,证明属于同一能级的两个東缚定态波函数$\psi_1$与$\psi_2$,只相
差一常数,即$\psi_1=C\psi_2$,$C$为一常数。
\subsection{(20 分)}
已知一维谱振子的哈帟顿算符为$$\hat{H} = -\frac{\hbar^2}{2\mu} \frac{d^2}{dx^2} + \frac{1}{2}\mu \omega^2 x^2~$$处于$\psi(x) = C(2\alpha^2x^2 - 1)e^{-\frac{1}{2}\alpha^2x^2}$状态中,式中$\alpha = \left( \frac{\mu \omega}{\hbar} \right)^{\frac{1}{2}}$,$C$为一常数。
\begin{enumerate}
\item 证明它处于能量的本征态,本征值是多少?
\item 它的动是是否具有确定值,为什么?
\end{enumerate}
\subsection{(15 分)}
什么是厄密算符?以下算符是否为厄密算符,要说明详细理由?
\begin{enumerate}
\item $\frac{d}{dx}$
\end{enumerate}
