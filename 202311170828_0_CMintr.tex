% 圆周运动(科普)
% license Usr
% type Tutor

牛顿有一个很形象的思想实验: 若不考虑大气阻力,大炮在水平方向发射的炮弹开始时近似做抛物线,但当炮弹的初速度足够大时,它将绕地球做圆周运动永远不落到地面。

那么使用 “微元法” 的思想,我们是否可以认为当炮弹做圆周运动时,它在每一小段可以近似看作是在做平抛运动(水平发射的抛物线)呢?答案是可以的,且这些小段分割得越短,这个近似就越准确。

\begin{figure}[ht]
\centering
\includegraphics[width=6cm]{./figures/f5e66bd1fac10d50.png}
\caption{假设炮弹做平抛运动,当飞出一段距离后,稍微调整重力的方向,并以此时速度为初速度再次做平抛运动。} \label{fig_CMintr_1}
\end{figure}

\begin{figure}[ht]
\centering
\includegraphics[width=6cm]{./figures/3646d689f4b24526.png}
\caption{当平抛运动重复多次,我们就得到了近似的圆周运动。 每次抛物运动的时间越短结果就越精确。} \label{fig_CMintr_2}
\end{figure}

通过这种方式,我们甚至可以推导出圆周运动的向心加速度。且这个加速度就是平抛运动中物体向下的加速度。
