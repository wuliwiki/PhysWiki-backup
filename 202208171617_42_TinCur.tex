% 曲线坐标系下的张量坐标变换(仿射空间)
% 坐标变换|张量|曲线坐标

\pentry{仿射空间中的曲线坐标系\upref{CFinAf},张量的坐标变换\upref{TrTnsr}}
在仿射空间中的曲线坐标系\upref{CFinAf}一节中,我们知道,在区域 $\Omega$ 中给定一个曲线坐标,就相当于在 $\Omega$ 中的每一点 $M$ 上给出了一个局部标架 $\{M;\partial_1 x,\cdots,\partial_n x\}$,其中 $x$ 是点 $M$ 的向径.在相当多的情形下,都是认为仿射空间取任意的曲线坐标 $x^i$ ,因而在每一点 $M$ 产生一个局部标架,于是点 $M$ 处的张量 $T(M)$ 的坐标,都是在这一标架下取的.这些坐标简单的叫作张量 $T(M)$ 在已知曲线坐标系 $x^i$ 中的坐标. 在 $\Omega$ 上每一点 $M$ 处给定一个张量 $T(M)$ 就叫作在 $\omega$ 上给定了一个张量场.

