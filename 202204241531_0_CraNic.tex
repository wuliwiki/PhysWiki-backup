% Crank-Nicolson 算法解一维含时薛定谔方程
% 算法|薛定谔方程|数值解|偏微分方程

\begin{issues}
\issueDraft
\end{issues}

\pentry{薛定谔方程\upref{TDSE}, 矩阵\upref{Mat}}

\footnote{参考 \cite{NR3}.}薛定谔方程为
\begin{equation}
-\frac12 \pdv[2]{\psi}{x} + V\psi = \I \pdv{\psi}{t}
\end{equation}
用 Crank-Nicolson 或 Caley scheme\footnote{二者是一回事, 见 \cite{NR3} 19.2 节.} 得到的结果是
\begin{equation}\label{CraNic_eq2}
\qty(1+\frac{\I}{2}\mat H_{n+1}\Delta t)\bvec\psi_{n+1} = \qty(1-\frac{\I}{2}\mat H_n\Delta t)\bvec\psi_n
\end{equation}
其中 $\bvec\psi_n$ 是时刻 $t_n$ 的波函数列矢量(已知), $\bvec\psi_{n+1}$ 为时刻 $t_{n+1}$ 的波函数列矢量(未知), $\mat H_n$ 是 $t_n$ 时刻的哈密顿矩阵.

但事实上, 还可以继续减少计算量. 若近似认为\footnote{或者更公平地, 把\autoref{CraNic_eq5} 中 $\mat H_n$ 改为 $\mat H_{n+1/2}$, 即 $(t_n+t_{n+1})/2$ 时的哈密顿矩阵.} $\mat H_{n+1}\approx \mat H_{n}$, 将\autoref{CraNic_eq2} 整理后得
\begin{equation}\label{CraNic_eq5}
\qty(\frac12 + \frac{\I}{4}\mat H_n\Delta t)\qty(\bvec\psi_{n+1}+\bvec\psi_n) = \bvec\psi_n
\end{equation}
解这个方程, 再减去 $\bvec \psi_n$ 即可.

\subsection{等间距网格}
对于等间距网格, 可以用差分法(\autoref{DerDif_eq5}~\upref{DerDif})计算二阶导数, 表示为矩阵有
\begin{equation}
\mat D_2 = \frac{1}{\Delta x^2}\pmat{-2 & 1 & 0 & 0 & \dots\\
1 & -2 & 1 & 0 & \dots\\
0 & 1 & -2 & 1 & \dots\\
0 & 0 & 1 & -2 & \dots\\
\vdots & \vdots & \vdots & \vdots & \ddots}
\end{equation}
那么 $\mat H_n = -\mat D_2/(2m) + \mat V_n$, 其中 $\mat V_n$ 是对角矩阵, 第 $i$ 个对角元为 $V(x_i, t_n)$. 现在, 就可以代入\autoref{CraNic_eq2} 或\autoref{CraNic_eq5} 求解了.

\subsection{虚时间}
使用虚时间后, \autoref{CraNic_eq2} 和\autoref{CraNic_eq5} 分别变为
\begin{equation}
\qty(1+\frac12\mat H\Delta t)\bvec\psi_{n+1} = \qty(1-\frac12\mat H\Delta t)\bvec\psi_n
\end{equation}
\begin{equation}
\qty(\frac12 + \frac14\mat H\Delta t)\qty(\bvec\psi_{n+1}+\bvec\psi_n) = \bvec\psi_n
\end{equation}

Matlab 代码如下

\begin{lstlisting}[language=matlab]
% Crank-Nicolson 法解一维薛定谔方程
% 等间距网格,稀疏矩阵
function TDSE_1d_failed
% ==== 参数设置 ======
m = 1; % 质量,角频率
xmin = -10; xmax = 30; Nx = 300; % x 网格
tmin = 0; tmax = 10; Nt = 1000; % 时间网格
Nplot = 10; % 画图步数
t0 = 0; % 高斯波包的初始时间
p0 = 5; % 初始动量
x0 = 0; sig_x = 2; % 初始位置, 波包宽度
V = @(x,t) zeros(size(x));
% ===================
close all;
psi_gs = @(x) 1/(2*pi*sig_x^2)^0.25/sqrt(1 + 1i*t0/(2*m*sig_x^2))...
      *exp(-(x-x0-p0*t0/m).^2/(2*sig_x)^2/(1 + 1i*t0/(2*m*sig_x^2)))...
      .*exp(1i*p0*(x-p0*t0/(2*m)));
x = linspace(xmin, xmax, Nx); dx = (xmax-xmin)/(Nx-1);
t = linspace(tmin, tmax, Nt); dt = (tmax-tmin)/(Nt-1);
psi = psi_gs(x).';
% 准备稀疏矩阵
ind1 = [1:Nx, 1:Nx-1, 2:Nx]; % 行标
ind2 = [1:Nx, 2:Nx, 1:Nx-1]; % 列标
% 动能矩阵非零元
T = -1/(2*m)*[-2*ones(1,Nx), ones(1,2*Nx-2)]/dx^2;
A = (1i*dt/4)*T; % 对角元稍后更新
figure; plot(x, real(psi)); hold on;
plot(x, imag(psi));
for it = 1:Nt
    % 更新对角元
    A(1:Nx) = 0.5 + (1i*dt/4)*(T(1:Nx) + V(x, t(it)+dt/2));
    Asp = sparse(ind1, ind2, A, Nx, Nx);
    % figure; spy(Asp);
    tmp = Asp \ psi;
    % err = norm(Asp*tmp - psi);
    psi = tmp - psi;
    if mod(it, Nplot) == 0
        clf;
        plot(x, real(psi)); hold on;
        plot(x, imag(psi));
        axis([xmin,xmax,-0.5,0.5]);
        pause(0.2);
    end
end
end
\end{lstlisting}
