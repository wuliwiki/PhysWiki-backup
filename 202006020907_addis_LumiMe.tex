% 普通光源的发光机理

一般普通光源(指非激光光源)发光的机理是处于激发态的原子(或分子)的自发辐射,即光源中的原子(或分子)吸收了外界能量而处于激发态,这些激发态是极不稳定的,它会自发地回到低能量的激发态或基态,在这过程中,原子向外发射电磁波(光波).每个原子的发光是间歇的. 一个原子经一次发光后,只有在重新获得足够能量后才会再次发光每次发光的持续时间极短,约为$10^{-8}\mathrm s$.可见原子发射的光波是一段频率一定、振动方向一定、有限长的光波,通常称为光波列\autoref{LumiMe_fig1}.
\begin{figure}[ht]
\centering
\includegraphics[width=5.5cm]{./figures/LumiMe_1.png}
\caption{光波波列} \label{LumiMe_fig1}
\end{figure}

同一原子在不同时刻所发出的波列之间振动方向和相位也各不相同.在普通光源中,大量原子在发光,各个原子的激发和辐射参差不齐,而且彼此之间没有联系,是一种随机过程,因而不同原子在同一时刻所发出的波列在频率、振动方向和相位上各自独立,可见,普通光源中原子发光,可谓此起彼伏、瞬息万变.