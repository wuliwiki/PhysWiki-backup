% 皮埃尔-西蒙·拉普拉斯(综述)
% license CCBYSA3
% type Wiki

本文根据 CC-BY-SA 协议转载翻译自维基百科\href{https://en.wikipedia.org/wiki/Pierre-Simon_Laplace}{相关文章}。

\begin{figure}[ht]
\centering
\includegraphics[width=6cm]{./figures/12e7e5c72d2ec562.png}
\caption{} \label{fig_LPLS_1}
\end{figure}
皮埃尔-西蒙·拉普拉斯(Pierre-Simon, Marquis de Laplace,1749年3月23日-1827年3月5日)是法国学者,他的工作对工程学、数学、统计学、物理学、天文学和哲学的发展具有重要意义。他在五卷本《天体力学》(Mécanique céleste)(1799–1825)中总结并扩展了前人的工作。这部著作将经典力学的几何学研究转化为基于微积分的研究,从而开辟了更广泛的研究领域。[2]拉普拉斯还推广并进一步确认了艾萨克·牛顿爵士的工作。在统计学中,贝叶斯概率解释主要是由拉普拉斯发展而来的。[3]

拉普拉斯提出了拉普拉斯方程,并开创了拉普拉斯变换,这在数学物理的许多分支中都有应用,而数学物理领域正是他主导发展的一个重要领域。广泛应用于数学中的拉普拉斯算子也以他的名字命名。他重新阐述并发展了太阳系起源的星云假说,是最早提出类似黑洞概念的科学家之一,[4] 斯蒂芬·霍金曾表示:“拉普拉斯基本上预言了黑洞的存在”。[1]他提出了拉普拉斯恶魔,这是一个假设的全能预测智慧。他还改进了牛顿关于声速的计算,得出了更精确的测量结果。[5]

拉普拉斯被认为是历史上最伟大的科学家之一。有时被称为“法国的牛顿”或“法国的牛顿”,他被描述为拥有卓越的数学天赋,超越了几乎所有同时代的人。[6] 1785年,拉普拉斯曾担任拿破仑从巴黎军事学院毕业时的考官。[7] 拉普拉斯于1806年成为帝国伯爵,并在1817年波旁王朝复辟后被封为侯爵。
\subsection{早年}
\begin{figure}[ht]
\centering
\includegraphics[width=6cm]{./figures/1abf9bfb2f2cc079.png}
\caption{} \label{fig_LPLS_2}
\end{figure}
皮埃尔-西蒙·拉普拉斯的肖像,作者:约翰·恩斯特·海因修斯(1775年)

拉普拉斯生活中的一些细节不为人知,因为1925年与他的曾曾孙科尔贝尔-拉普拉斯伯爵的家族 château(位于利厄市附近的圣朱利安·德·梅约)一起被烧毁。还有一些记录早在1871年,当他的位于巴黎附近阿尔居伊的住所被掠夺时就已被销毁。[8]