% Visual C++ 的简单画图库 MatPlot
% keys VC++|MatPlot|Matlab|画图

\begin{figure}[ht]
\centering
\includegraphics[width=3cm]{./figures/MtPlot_1.png}
\caption{MatPlot 图标} \label{MtPlot_fig1}
\end{figure}

MatPlot 是笔者写的一个简单易用的 c++ 科学绘图函数库,可以在 windows 系统使用.其函数与 Matlab 中的画图函数相似,目前主要提供的函数有例如 \verb|figure()|,\verb|close()|,\verb|plot()|,\verb|scatter()|,\verb|clf()|,\verb|axis()| 等,暂时不支持 3D 画图. GitHub 页面点\href{https://github.com/MacroUniverse/MatPlot}{这里}.


\subsection{例 1:最简单的例子}

\begin{figure}[ht]
\centering
\includegraphics[width=8cm]{./figures/MtPlot_2.png}
\caption{最简单的例子(控制台程序)} \label{MtPlot_fig2}
\end{figure}

\begin{figure}[ht]
\centering
\includegraphics[width=12cm]{./figures/MtPlot_3.png}
\caption{结果 1} \label{MtPlot_fig3}
\end{figure}

注意 MatPlot 的所有函数都具有 \verb|namespace MatPlot|.\verb|MatPlotInit()| 和 \verb|MatPlotClose()| 分别是初始化和退出 MatPlot 的命令,在程序中分别只能使用一次.所有画图函数必须在调用 \verb|MatPlotInit()| 之后调用 \verb|MatPlotClose()| 之前使用.为了让程序不立即终止,可以用 \verb|Sleep()| 函数暂停程序,输入的数字是暂停的毫秒数.

\verb|plot()| 函数的完整格式为
\verb|plot(double *px, double *py, int N, char color = 'a');|
其中 \verb|px| 和 \verb|py| 既可以是数组也可以是指针,\verb|N| 是 \verb|px| 和 \verb|py| 的长度,\verb|color| 是颜色,默认值 \verb|'a'| 代表自动颜色,这种模式下每使用一次 \verb|plot()| (或 \verb|scatter|),颜色就会改变一次. \verb|color| 可选 \verb|'r'|(red), \verb|'g'|(green), \verb|'b'|(blue), \verb|'y'|(yellow), \verb|'m'|(magenta), \verb|'c'|(cyan), \verb|'k'|(black), 'w'(white). plot() 默认在“当前画图框”(见下文)上绘制,如果没有画图框,则自动生成画图框 figure1.

\verb|scatter()| 函数用于画散点图,使用格式与 \verb|plot()| 完全相同.如果在例 1 中的 \verb|plot()| 下面加上代码
\verb|scatter(x, y, N);|
运行结果如下

\begin{figure}[ht]
\centering
\includegraphics[width=10cm]{./figures/MtPlot_4.png}
\caption{结果 2} \label{MtPlot_fig4}
\end{figure}

注意散点的颜色变成了橙色.

\verb|figure()| 函数用于生成新的画图框,其的格式为
\verb|figure(int FigNo = -1);|
默认值 \verb|-1| 的情况下 \verb|figure()| 按照顺序给画图框编号.如果输入 FigNo,则生成编号为 FigNo 的画图框.figure() 函数的另一个功能是指定当前画图框(Current Figure)为 FigNo.plot(),scatter() 以及其他许多函数都是对当前画图框进行操作.

close() 函数用于关闭画图框,其格式为
close(int FigNo = -1);
默认值 -1 的情况下,close() 关闭当前画图框,指定 FigNo 时关闭指定的画图框.如果 close() 关闭的是当前画图框(无论指定 FigNo 与否),则当前画图框自动变为下一个编号更大的画图框,如果不存在,则自动变为下一个编号更小的画图框.另外,单击画图框右上角的 x 按钮与使用 close() 效果相同.

close_all() 函数用于关闭所有画图框.

clf() 函数(clean figure)用于清空当前画图框,并把坐标范围设为 0 到 1.

axis() 函数用于调整坐标范围,格式为
axis(double xmin, double xmax, double ymin, double ymax);
其中 xmin,xmax 分别为 x 轴的最小和最大值,ymin 和 ymax 分别为 y 轴的最小和最大值.

axis_auto() 用于设置坐标轴为自动调节(新生成的画图框默认为自动调节),如果 plot() 或者 scatter() 函数的数据点超出当前的坐标范围,坐标轴会自动调整以包含所有数据点.

axis_manual() 用于设置坐标范围固定不变.

xticks() 和 yticks() 函数用于设置 x 轴和 y 轴的坐标点,格式都为
xticks(double tickmin, double tickmax, int Nticks = 26, int txtstart = 0, int txtspace = 5, int Ndigits = 2);
其中 tickmin 是和 tickmax 分别是最小和最大坐标点的坐标,Nticks 是坐标点的个数,txtspace 是每几个坐标点下方显示坐标,txtstart 是从第几个坐标点开始显示坐标,Ndigits 是显示的小数位数(目前还不支持科学计数法).

\subsection{例 2: 设置坐标}

目前坐标点和坐标显示的规划功能并不像 Matlab 中的那么完善.先来看一段画正弦函数的代码

\begin{figure}[ht]
\centering
\includegraphics[width=9cm]{./figures/MtPlot_5.png}
\caption{例 2} \label{MtPlot_fig5}
\end{figure}

\begin{figure}[ht]
\centering
\includegraphics[width=10cm]{./figures/MtPlot_6.png}
\caption{运行结果为} \label{MtPlot_fig6}
\end{figure}

可以发现 x 轴的坐标点安排不合理.我们可以用 xticks() 函数重新设置.在 plot() 函数下方插入
xticks(0, 6.2, 32, 0, 5, 1);
即设置最小的坐标点为 0,最大的为 6.2,共有 32 个坐标点,从第一个开始,每隔五个打印一个坐标,并保留 1 位小数.运行结果变为

\begin{figure}[ht]
\centering
\includegraphics[width=10cm]{./figures/MtPlot_7.png}
\caption{运行结果 2} \label{MtPlot_fig7}
\end{figure}

为提供更友好的用户界面,MatPlot 在初始化的时候生成并隐藏了控制面板,可以用于临时调用 MatPlot 的常用函数.用 show_control() 函数可以显示控制面板.用 hide_control() 可以将其再次隐藏. 为了使用控制面板,可以在程序中用 Sleep() 函数.

\begin{figure}[ht]
\centering
\includegraphics[width=5cm]{./figures/MtPlot_8.png}
\caption{控制面板} \label{MtPlot_fig8}
\end{figure}

控制面板中标有函数名的按钮用于调用对应的函数,按钮旁的文本框用于输入对应函数的参数.其中 plot(...) 和 scatter(...) 按钮分别绘制 DataNo*sin(x) 的函数图和散点图.选择或取消 autoaxis(...) 选项相当于调用 axis_auto() 和 axis_manual 函数.CurrentFig 显示当前画图框的编号(注意 0 代表 figure1,1 代表 figure2...)按 OK 或 Cancel 退出 MatPlot.
