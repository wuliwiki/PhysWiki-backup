% 李导数
% license Usr
% type Tutor
\pentry{流\nref{nod_flow}}{nod_fbf5}\pentry{前推\nref{nod_pfw}}{nod_a77f}
\subsection{对向量场作用}
区别于仿射联络,\textbf{李导数(Lie derivative)}$\mathcal{L}:\mathfrak{X}(M)\times\mathfrak{X}(M)\to\mathfrak{X}(M)$是对向量场微分的另一种思路。

\begin{definition}{}
设$V$为任意光滑向量场,令$\theta_t:M\rightarrow M$为该向量场的流,定义对$W$关于$V$的李导数为
\begin{equation}(\mathcal{L}_VW)_p=\left.\frac{d}{dt}\right|_{t=0}(\theta_{-t})_*W_{\theta_t(p)}=\lim_{t\to0}\frac{(\theta_{-t})_*W_{\theta_t(p)}-W_p}{t}~.\end{equation}
\end{definition}
由前推定义可知,$\theta_{t*}W_q\in T_{\theta_t(q)}M$,因而$(\theta_{-t})_*W_{\theta_t(p)}\in T_p M$,所以该定义是合理的,我们利用前推得到了切空间的两个向量并进行微分。

\subsection{对张量场作用}
\subsection{对微分形式作用}
\subsection{李导数的基本性质}