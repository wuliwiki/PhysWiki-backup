% OpenACC 笔记
% license Usr
% type Note

\pentry{C++ 基础\nref{nod_Cpp0}}{nod_ad0d}

\begin{itemize}
\item \href{https://www.openacc.org/}{OpenACC} 可以像 OpenMP 那样,通过添加少量的 \verb`#pragma` (预编译指令)给已有的程序进行 GPU 加速支持在 Linux 上运行的 C++ 和 Fortran 程序。本文以 C++ 为例。 它由于易用性,它常用于高性能科学计算中。
\item 和 OpenMP 一样, OpenACC 需要编译器支持。一般使用英伟达显卡专用的 \href{https://developer.nvidia.com/hpc-sdk}{NVIDIA HPC SDK}(包含 CUDA Toolkit\upref{CUDA})(202301:暂不支持 Windows) 或者 \href{https://www.openacc.org/tools/gcc-for-openacc}{GCC}。
\item 【ubuntu22.04 原装 g++ 测试失败】运行中显示 \verb`Not executing on GPU.`,应该需要安装 OpenACC 版本的 GCC,或者直接用 NVIDIA SDK。
\item 先确保 NVIDIA GPU 驱动安好, \verb`nvidia-smi` 能用, 然后安装好 NVIDIA SDK 以后, 要在 \verb`.bashrc` 最后添加以下几行设置环境(见\href{https://docs.nvidia.com/hpc-sdk//hpc-sdk-install-guide/index.html}{这个文档})
\begin{lstlisting}[language=none]
NVARCH=`uname -s`_`uname -m`; export NVARCH
NVCOMPILERS=/opt/nvidia/hpc_sdk; export NVCOMPILERS
MANPATH=$MANPATH:$NVCOMPILERS/$NVARCH/23.11/compilers/man; export MANPATH
PATH=$NVCOMPILERS/$NVARCH/23.11/compilers/bin:$PATH; export PATH
\end{lstlisting}
\item \verb`source ~/.bashrc` 以后, 就可以试试 \verb`nvc --version`, \verb`nvcc --version` 了。
\item \verb`nvc/nvc++` 编译器支持 OpenACC 和 OpenMP, 而 \verb`nvcc` 则用于 CUDA\upref{CUDA} 编程。
\end{itemize}

\subsection{入门例程}
\begin{itemize}
\item 【装了 GPU 驱动仍然编译失败】在 GCC 中,直接加上 \verb`-fopenacc` 选项即可! 例如 \verb`g++ -fopenacc test1.cpp -o test1.x`。 在编译时可以加上 \verb`-fopt-info` 选项输出更多信息。
\item 【测试成功】用 Nvidia 编译器如 \verb`nvc -acc -gpu=managed -Minfo=accel test1.cpp -o test1.x`
\end{itemize}

\begin{lstlisting}[language=none,caption=test1.cpp]
#include <iostream>
#include <vector>
#include <openacc.h>

// Function to initialize the vectors with values
void initialize(std::vector<double>& a, std::vector<double>& b, int n) {
	for(int i = 0; i < n; ++i) {
		a[i] = static_cast<double>(i);
		b[i] = static_cast<double>(2 * i);
	}
}

// detect if GPU is actually running
void detect_gpu()
{
	double a[100], b[100];
	#pragma acc parallel loop
	for (int i = 0; i < 100; ++i) {
		if (i == 10) {
			if (acc_on_device(acc_device_not_host))
				printf("Executing on GPU.\n");
			else
				printf("Not executing on GPU.\n");
		}
		a[i] += b[i];
	}
}

int main() {
	const int n = 1000000; // Size of the vectors
	std::vector<double> a(n), b(n), c(n);
	double *pa = a.data(), *pb = b.data(), *pc = c.data();

	// Initialize vectors a and b
	initialize(a, b, n);

	detect_gpu();

	// Using OpenACC to offload the following computation to an accelerator
	// and explicitly handle data movement
#pragma acc data copyin(pa[0:n], pb[0:n]) copyout(pc[0:n])
	{
#pragma acc parallel loop
		for(int i = 0; i < n; ++i)
			pc[i] = pa[i] + pb[i];
	}

	// Display the first 10 results
	for(int i = 0; i < 10; ++i) {
		std::cout << "c[" << i << "] = " << c[i] << std::endl;
	}
}
\end{lstlisting}

\subsection{编译器选项}
\begin{itemize}
\item \verb`-ta=tesla`: Compiler option to target NVIDIA Tesla GPUs.
\item \verb`-Minfo=accel`: Provides feedback about the code generated by the compiler.
\end{itemize}

\subsection{常用命令}
\subsubsection{循环}
\begin{itemize}
\item \verb`#pragma acc parallel`: GPU 并行运算
\item \verb`#pragma acc kernels`: Identifies a code block for parallelization, allowing the compiler to automatically manage parallelism.
\item \verb`#pragma acc loop`: Used within parallel or kernels regions to indicate loops that should be parallelized.
\end{itemize}

\subsubsection{函数和变量}
\begin{itemize}
\item \verb`#pragma acc routine`: 让一个函数可以在 GPU 代码中被调用(也可以在 CPU 代码调用)。
\item \verb`#pragma acc declare`: Used for declaring variables or creating a data region.
\end{itemize}

\subsubsection{数据传输}
\begin{itemize}
\item \verb`#pragma acc data`: Manages data movement to and from the GPU.
\item \verb`#pragma acc enter data`: Specifies data that should be moved to the GPU.
\item \verb`#pragma acc exit data`: Specifies data to be moved back from the GPU.
\item \verb`#pragma acc update`: Synchronizes data between the host and the GPU.
\item \verb`copy, copyin, copyout`, create, present: Clauses for data construct to define how data is handled (e.g., whether it's copied to/from the GPU or just created there).
\end{itemize}

\subsubsection{线程精细控制}
\begin{itemize}
\item \verb`gang, worker, vector`: Used with loop directive to control how loop iterations are distributed over parallel execution units.
\item \verb`collapse(n)`: Collapses nested loops to enhance parallelism.
\item \verb`reduction(operator:list)`: Performs a reduction operation (like sum, max) across parallel elements.
\end{itemize}

