% 选择公理(综述)
% license CCBYSA3
% type Wiki

本文根据 CC-BY-SA 协议转载翻译自维基百科\href{https://en.wikipedia.org/wiki/Axiom_of_choice}{相关文章}。

\begin{figure}[ht]
\centering
\includegraphics[width=8cm]{./figures/ce05c80ac3fab240.png}
\caption{} \label{fig_AC_1}
\end{figure}
在数学中,选择公理(简称AC或AoC)是集合论的一个公理,它等价于“非空集合的笛卡尔积是非空的”这一命题。非正式地说,选择公理表明,给定任何一个集合的集合,每个集合至少包含一个元素,便可以通过从每个集合中选择一个元素来构造一个新集合,即使这个集合是无限的。形式上,它声明,对于每一个索引族\( (S_i)_{i \in I} \)的非空集合,存在一个索引集合\( (x_i)_{i \in I} \),使得对于每个\( i \in I \),都有\( x_i \in S_i\)。\(^\text{[1]}\)选择公理是由恩斯特·泽梅洛在1904年提出的,旨在形式化他的良序定理证明。选择公理等价于每个划分都有一个横切集的命题。\(^\text{[2]}\)

在许多情况下,通过选择元素构造的集合可以在不使用选择公理的情况下完成,特别是当从中选择元素的集合个数是有限的,或者如果有一个标准规则来选择元素——即对于每个集合,存在某种区分性特性,恰好有一个元素满足该特性。一个典型的例子是从自然数中选取的集合。从这样的集合中,总是可以选择最小的数字,例如,给定集合 {{4, 5, 6}, {10, 12}, {1, 400, 617, 8000}},包含每个集合最小元素的集合是 {4, 10, 1}。在这种情况下,“选择最小的数字”就是一个选择函数。即使从自然数中收集了无限多个集合,也总是可以从每个集合中选择最小的元素来构造一个集合。也就是说,选择函数提供了所选元素的集合。但对于所有实数的非空子集的集合,尚未发现一个确定的选择函数。在这种情况下,必须调用选择公理。\(^\text{[3]}\)
\begin{figure}[ht]
\centering
\includegraphics[width=6cm]{./figures/482b3dc7355b02c6.png}
\caption{\((S_i)\) 是一个以实数 \(\mathbb{R}\) 为索引的无限索引集合族;即,对于每个实数 \(i\),都有一个集合 \(S_i\),上面展示的是一个小样本。每个集合包含至少一个元素,且可能包含无限多个元素。选择公理允许我们从每个集合中选择一个元素,形成一个相应的元素族 \((x_i)\),该族也以实数为索引,且每个 \(x_i\) 来源于 \(S_i\)。通常来说,这些集合可以通过任何集合 \(I\) 进行索引(称为索引集,其元素作为集合中元素的索引),不仅仅是 \(\mathbf{R}\)。} \label{fig_AC_2}
\end{figure}
伯特兰·罗素提出了一个类比:对于任何(即使是无限的)鞋子对的集合,可以从每对鞋子中挑选出左鞋,得到一个合适的鞋子集合(即集合);这使得可以直接定义一个选择函数。而对于一对一对的袜子的无限集合(假设它们没有区分特征,如左袜子和右袜子),没有显而易见的方式来构造一个函数,从每对袜子中选择一只袜子来形成一个集合,而不使用选择公理。

虽然最初具有争议,选择公理现在已被大多数数学家毫无保留地使用,\(^\text{[4]}\)并且被包括在标准的公理化集合论中,即泽梅洛-弗兰克尔集合论与选择公理(ZFC)。这样做的动机之一是,许多广泛接受的数学结果,如提霍诺夫定理,要求选择公理来证明。现代集合论家还研究与选择公理不兼容的公理,如确定性公理。选择公理在某些构造性数学的变体中被避免,尽管在某些构造性数学的变体中,选择公理是被采纳的。
\subsection{陈述}  
选择函数(也称为选择器或选择器函数)是一个定义在非空集合族\(X\)上的函数\(f\),满足对于\(X\)中的每个集合\(A\),\(f(A)\)是\(A\)中的一个元素。基于这一概念,选择公理可以表述为:

\textbf{公理}——对于任何非空集合族\(X\),存在一个选择函数\(f\),该函数定义在\(X\)上,并将\(X\)中的每个集合映射到该集合的一个元素。

形式上,这可以表示为:
\[
\forall X\left[\varnothing \notin X \implies \exists f\colon X \rightarrow \bigcup_{A \in X} A \quad \forall A \in X\,(f(A) \in A)\right].~
\]
因此,选择公理的否定可以表示为存在一个非空集合族,它没有选择函数。形式上,这可以通过利用以下逻辑等价关系来推导:
\[
\neg \forall X\left[P(X) \to Q(X)\right] \quad \iff \quad \exists X\left[P(X) \land \neg Q(X)\right].~
\]
每个非空集合族\(X\)上的选择函数是\(X\)中集合的笛卡尔积的一个元素。这并不是笛卡尔积中最一般的情况,其中一个集合可以作为因子多次出现;然而,我们可以关注那些每次当某个集合作为因子出现时,选择相同元素的笛卡尔积中的元素,这些元素对应于该家族中所有不同集合的笛卡尔积的一个元素。选择公理断言存在这样的元素;因此,它等价于:

给定任何非空集合族,它们的笛卡尔积是一个非空集合。
\subsubsection{术语}  
在本文及其他关于选择公理的讨论中,以下缩写是常见的:
\begin{itemize}
\item AC – 选择公理。较少使用 AoC。\(^\text{[5]}\)  
\item ZF – 泽梅洛–弗兰克尔集合论,省略选择公理。  
\item ZFC – 泽梅洛–弗兰克尔集合论,扩展以包括选择公理。
\end{itemize}