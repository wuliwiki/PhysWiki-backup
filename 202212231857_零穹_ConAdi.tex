% 浸渐不变量
% 浸渐不变量|绝热不变量

当系统由一些缓变参数 $\lambda_i$ 确定时,系统在运动过程中保持不变的量称为系统的\textbf{浸渐不变量}。这样的系统可以理解为处于一个外场当中,而参数 $\lambda_i$ 描述了系统所处外场的性质。例如处于三维静电场中的二维平面上的电荷系统,系统受到的场的作用与平面所处的位置有关,那么系统所处电场的性质可以用平面所处的 $z$ 坐标来描述(以平面作为 $xOy$ 平面)。为简单起见,我们假设只有一个参数 $\lambda$。

所谓的“缓变”,是指在一个运动周期 $T$ 内 $\lambda$ 的变化很小,即 
\begin{equation}
\frac{\lambda(t+T)-\lambda(t)}{\lambda(t)}\ll 1
\end{equation}
由于 $\lambda(t+T)-\lambda(t)\approx T\dv{\lambda}{t}$,上式可写为
\begin{equation}
T\dv{\lambda}{t}\ll\lambda
\end{equation}
