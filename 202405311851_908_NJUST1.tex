% 南京理工大学 2011 量子真题
% license Usr
% type Note

\textbf{声明}:“该内容来源于网络公开资料,不保证真实性,如有侵权请联系管理员”

\subsection{简答题(请考生在下列12题中选作10题,每题6分,共60分):}

1. 一粒子的波函数为 $\psi (\mathbf{r}) = \psi (x, y, z)$,写出粒子位于 $x \sim x + d x$ 间的几率;用球坐标表示,粒子波函数表示为 $\psi (r, \theta, \varphi)$,写出粒子在球壳 $(r, r + \mathrm{d}r)$ 中被测到的几率,及在 $(\theta, \varphi)$ 方向的立体角 $\mathrm{d}\Omega$ 内找到粒子的几率;

2. 2.何谓正常塞曼效应?何谓反常塞曼效应?何谓斯塔克效应?

3. 何谓量子纠缠?何谓量子隐形传态?何谓测量塌缩?

4. 写出狄拉克 $\delta$ 函数的傅里叶变换及拉普拉斯变换的表示式。

5. 一体积分面两个空间的电偶极子 $ \mathbf{d}_1$,粒子 $a$ 可用如下电量矩阵描述,给定两个可能的电量,分别为 $\mathbf{d}_1 = (q_1, \mathbf{r}_1)$ 和 $\mathbf{d}_2 = (q_2, \mathbf{r}_2)$,向你写出可能的表示为几人?

6. $\mathcal{H}$ 力 $V$ 能的力矩如何单独相干叠加?

7. 对一个粒子保持近于某一力学空间的轨道,测量结果与求和方案单位有什么关系?两个力学空间时间具有确定值的条件是什么?

8. 量子系统的一些基本特征是什么?有啥区别?

9. 下列波函数所描写的状态是否为定态?并说明其理由。

\begin{enumerate}
  \item $\varphi_1 (x, t) = \varphi (x) e^{-i \omega t} + \varphi (x) e^{i \omega t}$
  \item $\psi_2 (x, t) = u (x) e^{-i \omega t} + v (x) e^{i \omega t}$
\end{enumerate}

10. 已知 $L \cdot x = i \hbar \frac{\partial}{\partial t}$,量子 $L$ 符合条件是什么?请根据原因简要说明。

11. 与自由粒子相关联的变换是什么?写出表示有无偏移?

12. 谱学序列在实验量子物理的关系写为 [\textbf{A} \textit{B} \textit{C}],写出形式定义和数学表示。

13. 写出迪拉克基态自旋的动力学关系,目标函数与自旋周期相等。
