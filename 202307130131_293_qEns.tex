% 量子系综
% keys 量子系综|混态

\pentry{量子比特\upref{Qubit}}


在传统的量子力学学习中,我们已经对密度矩阵\upref{denMat}和量子系综有了一定的了解。现在我们将会从信息学的角度来重新理解这一概念,并且讨论经典与量子概率这个相当底层的话题。于此同时,我们还会讨论量子系综是如何在描述量子子系统的过程中自然出现的。

\subsection{统计与系综}

我们不妨从Stern Gerlach实验中来引入量子系综的概念。

\pentry{Stern Gerlach实验\upref{SGExp}, 系综理论}
在对单比特纯态的讨论中,我们发现了一个重要的结论:对于任意的单比特纯态,总是存在着一个方向$\vec{n}\cdot\vec{\sigma}$,使得这个纯态是它的本征值为1的本征态。这意味着这样一个重要的结论:只要银原子的自旋态是纯态,那么总是存在着一个方向,只要将非均匀磁场调整到这个方向,那么这些银原子就只会在屏幕上产生一个斑点。

但是这与实验结论是不相符的。从炉子中产生的银原子满足这样的性质:不管将磁场调整到什么方向,屏幕上都会产生两个斑点。这一结论只能说明,这些银原子处于一种不能用纯态描述的状态。

什么状态能够描述这些银原子呢?经典统计物理能够给我们提供一些灵感。回想统计物理中,我们定义了系综这个概念,它指的是大量的,拥有相同宏观性质的系统(但微观状态不一定相同)的集合,从这样的大量系统的样本中进行取样,就可以确定系统代表点在相空间中的概率分布。在明确了这一点后,我们就可以把逻辑反过来:在给定了一个概率分布$\{p_i\}$和系统的一系列可能拥有(微观)状态$\{\text{态}_i\}$后,我们就可以定义一个系综为
\begin{equation}
\mathcal{E}:=\{p_i, \text{态}_i\}_{i\in\mathcal{I}}~.
\end{equation}
其中$\mathcal{I}$是指标集。在经典物理中,系统的状态就是相空间中的一个几何点,因此系综往往不用(1)中的记号来描述,而直接将其与相空间上的一个概率密度分布$\rho$建立等同。

在量子力学中,我们仍然希望将这一套方法仍然成立。幸运的是,只需要将(1)中的态从“相空间中的几何点”换成“希尔伯特空间中的射线”即可。也就是说,一个量子系综应当被定义为
\begin{equation}
\mathcal{E}:=\{p_i,\ket{\psi_i}\}_{i\in\mathcal{I}}~.
\end{equation}
给定一些量子纯态$\psi_i$和一个概率分布,我们即可定义一个量子系综。

回到Stern-Gerlach实验中。由于自旋的方向对银原子在炉子里的能量没有影响,那么统计力学的经验会告诉我们,如果用系综来对炉子中的银原子进行描述的话,那么它关于自旋态是等概率分布的,也就是说$p_i$是一个均匀分布,而$\ket{\psi_i}$取遍整个Bloch球面。

\subsection{密度矩阵}

现在我们要考虑在一个量子系综上测量一个可观测量,并且获得它的平均值。容易得到对$\mathcal{E}$来说,这个平均值是\begin{equation}
\langle O\rangle_\mathcal{E}=\sum_i p_i\bra{\psi_i}O\ket{\psi_i}~,
\end{equation}
即先在每一个纯态$\ket{\psi_i}$上取期望,然后按照$p_i$进行混合。注意到利用trace的性质,(3)可以进一步写为
\begin{equation}
\langle O\rangle_\mathcal{E}=\sum_i p_i\operatorname{Tr}\left[\ket{\psi_i}\bra{\psi_i}O\right]=\operatorname{Tr}\left[\left(\sum_i p_i\ket{\psi_i}\bra{\psi_i}\right)O\right]~.
\end{equation}
这暗示着我们可以把小括号中的表达式$\sum_i p_i\ket{\psi_i}\bra{\psi_i}$看作是$\mathcal{E}$的一种等价表示。我们把形如这样的表达式叫做密度矩阵(density matrix)。容易得到密度矩阵既可以描述一个量子纯态(只需要仅让一个$p_i=1$即可),也可以描述量子系综。因此在量子信息科学中,我们更习惯使用密度矩阵来表示量子系统的状态。
\begin{exercise}{}
密度矩阵的等价定义是希尔伯特空间上厄米,trace为1的半正定矩阵。请证明任何一个满足这种性质的矩阵都可以描述一个量子系综。
\end{exercise}

如(1)中表示的系综概念是可以进一步推广的,$\text{态}_i$可以进一步地用$\rho_i$来表示,也就是说我们可以定义一个“系综的系综”
$$\mathcal{E}=\{p_i,\rho_i\}_{i\in\mathcal{I}}=\sum_i p_i\rho_i~.$$

在引入了密度矩阵的定义后,我们拥有了对量子系统的更加完整的描述:每一个量子系统都对应一个密度矩阵,而且在系综意义上的混合就是密度矩阵的相加。

\subsection{经典和量子概率}

在得到了量子系综和其密度矩阵描述之后,我们可以对量子力学中的概率获得更加深入的理解了。考虑一个由纯态构成的系综$\mathcal{E}=\{p_i,\ket{\psi_i}\}$,我们发现这里出现了两种概率分布:首先$p_i$本身是一个概率分布,其次如果我们得到了系综中的一个状态$\ket{\psi_i}$,对它的测量也会产生一个概率分布。如何理解这两种概率呢?

为了回答这个问题,我们需要回到一开始对量子纯态的理解。当最初定义量子系统的状态由希尔伯特空间中的射线表示的时候,我们其实隐含了一个假设,那就是我们已经获得了这个系统的全部信息。而这一点并不一定总能够得到保证。

不妨考虑一个简单的态制备和测量协议:Alice首先抛出一枚均匀的硬币,根据硬币何面朝上来制备不同的态。如果正面朝上就制备一个$\ket{+}$态,如果反面朝上就制备一个$\ket{-}$态。然后她再在$\{\ket{0},\ket{1}\}$这个基底下对她制备的态进行测量。

对Alice来说,这个态肯定是处于纯态的,因为她知道抛硬币的结果,或者说知道这个系统的全部信息。那么对于不知道抛硬币结果的Eve来说呢?一些系统的信息对她来说是未知的,她只知道Alice以50\%的概率制备了$\ket{+}$,以50\%的概率制备了$\ket{-}$。这时候在她看来系统就处于一个混态
\begin{equation}
\rho=\frac{1}{2}\ket{+}\bra{+}+\frac{1}{2}\ket{-}\bra{-}=\frac{1}{2}\begin{pmatrix}
1 & 0\\
0 & 1
\end{pmatrix}~.
\end{equation}

如果Eve拥有着极强的计算能力,并且获得了Alice抛硬币时的力度,角度,环境的风速,湿度等一切信息,原则上说,她可以计算出硬币的结果。那这时她也相当于获得了系统的全部信息,在她看来这个量子态也变成了纯态。从这个意义上来说,扔硬币所得到的概率就相当于是一种“经典概率”,它也是一种“外在”的概率,因为它可以被认为仅仅是我们没有获得系统的全部信息导致的\footnote{这也和概率论中的贝叶斯学派观点一致。}。

接下来Alice对她手里的量子比特进行测量,这个拥有着极强计算能力的Eve能否预测出测量结果呢?很遗憾,(如果量子力学是正确的话\footnote{这也是当年爱因斯坦所质疑的,但是被贝尔不等式的一系列实验所证实。})并没有任何办法。不管是Eve还是Alice本人,都只知道测量会有50\%的概率得到0,有50\%的概率得到1。那么这个概率就是一种“量子概率”,因为这是量子理论中所独有的,由于对叠加态测量所导致的概率。它也是一种“内禀”的概率,因为即使知道了系统的全部信息,这个概率也仍然存在。

对经典概率与量子概率的区分和讨论是量子信息的基本问题之一。本节的基本思想在后面的章节中也会反复涉及。

\subsection{作为子系统的量子系综}

\pentry{约化密度矩阵\upref{partra}}

我们现在从量子系统的子系统角度来研究量子系综。考虑$AB$系统中的量子态$\ket{\psi}_{AB}$,在操作的意义上,原则上我们总是可以只考虑$A$系统或者$B$系统的演化、测量等操作。比如说两个系统可以保持类空间隔,以至于因果律保证着一个系统的操作不会对另一个系统产生任何影响。在这种情况下,我们就可以定义对系统$A$进行的局部操作$O_A$,这是一个定义在$\mathcal{H}_A$上的算符。为了实现它只需要在$AB$系统上实现操作$O=O_A\otimes\mathbb{I}_B$。

我们已经知道了此时$A$系统的状态由约化密度矩阵来描述,即
\begin{equation}
\rho_A=\operatorname{Tr}_B[\ket{\psi}\bra{\psi}_{AB}]
\end{equation}
对于局部操作$O_A$来说,可以认为是$O$在$AB$系统上的操作,也可以看作是



