% 树与图的深度优先搜索
% DFS|算法|树与图的深度优先搜索|C++

\begin{issues}
\issueTODO
\end{issues}

本文将介绍树与图的深度优先搜索.

\textbf{树与图的存储:}
存储可以使用邻接表,邻接表的实现可以使用前面学的单链表\upref{List},邻接表就是 $n$ 个单链表,邻接表的所使用的数组需要多开一个 \verb|head| 数组,表示每个单链表的表头.邻接表的插入一般都是\textbf{头插法},即从单链表的表头插入新结点.

\begin{figure}[ht]
\centering
\includegraphics[width=14.25cm]{./figures/DFS2_1.png}
\caption{邻接表插入一个数}} \label{DFS2_fig1}
\end{figure}

可见,对于一张 $n$ 个点 $m$ 条边的图,可以用 $n$ 个单链表构成,$\forall x\in \text{Graph}$ 要想找到 $x$ 的所有出边,可以依据 $x$ 的表头依次访问.

\begin{lstlisting}[language=cpp]
void add(int a, int b)
{
    e[idx] = b, ne[idx] = h[a], h[a] = idx ++ ;
}
\end{lstlisting}