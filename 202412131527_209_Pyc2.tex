% Python 第一步
% license Usr
% type Tutor

\subsection{创建一个Python项目}

打开 Pycharm , 新建项目

\begin{figure}[ht]
\centering
\includegraphics[width=14.25cm]{./figures/2ada35917a742b32.png}
\caption{选项选取} \label{fig_Pyc2_2}
\end{figure}

右键项目名——新建——Python 文件

\begin{figure}[ht]
\centering
\includegraphics[width=14.25cm]{./figures/44ea62f4da2d69c8.png}
\caption{新建Python文件} \label{fig_Pyc2_1}
\end{figure}

给文件命名

\begin{figure}[ht]
\centering
\includegraphics[width=14.25cm]{./figures/cd91ca45cc505f0c.png}
\caption{命名} \label{fig_Pyc2_3}
\end{figure}

恭喜你,建立了你的第一个Python文件!

\subsection{输入与输出}

\begin{figure}[ht]
\centering
\includegraphics[width=14.25cm]{./figures/8bb25afdd70aa412.png}
\caption{编写代码位置} \label{fig_Pyc2_4}
\end{figure}

打出 \footnote{单词 print 的中文意思是打印,所以括号里面的内容是打印出来的?}print() ,其中括号里面的内容就是输出内容,例如下图

\begin{lstlisting}[language=python]
print(1234)
\end{lstlisting}

然后右键文件名,选择运行。(操作如下图)

\begin{figure}[ht]
\centering
\includegraphics[width=14.25cm]{./figures/b5c6db6c52a0d894.png}
\caption{运行} \label{fig_Pyc2_5}
\end{figure}

接着会弹出运行框。框内就是结果

\begin{figure}[ht]
\centering
\includegraphics[width=14.25cm]{./figures/9c001a49cf8c70ed.png}
\caption{输出结果} \label{fig_Pyc2_6}
\end{figure}

现在你已经会如何运行编写好的程序了,那我们继续讨论 Print。

现在打出

\begin{lstlisting}[language=python]
print(1+2)
\end{lstlisting}

运行,结果如下

\begin{lstlisting}[language=bash]
3
\end{lstlisting}

可以发现,得到的结果是 3,而不是1+2

这就是Print的另一个功能,可以计算括号内的结果并输出。

继续打出

\begin{lstlisting}[language=python]
print(你好)
\end{lstlisting}

运行后,结果如下

\begin{figure}[ht]
\centering
\includegraphics[width=14.25cm]{./figures/c18bf6eea4e75d37.png}
\caption{运行结果} \label{fig_Pyc2_7}
\end{figure}

并没有输出 你好 而是报错了(出现了红字),这是为什么呢?

这是因为

