% 角动量
% 角动量

\begin{issues}
本词条有误,需大改
\end{issues}
\subsection{物体最一般的运动}
处于一个空间当中的物体,若不考虑物体发生形变(不发生形变指物体中任意两点的位置关系不发生任何改变,这样的物体称\textbf{刚体}),物体最一般的运动是怎样的呢?物体的运动可用物体在空间中的位置随时间的变化来描述,而在观察物体的运动时,我们总是在某一时刻观察物体的位置,紧接着在下一时刻继续观察物体的位置,在观察时间段里,当知道了每一个观察时刻物体对应的位置,我们就认为我们知道了物体的运动,为了更精细的得到物体的运动状况,我们只是要求两相邻观察时刻的时间间隔能充分的小.然而无论如何,我们只是在将一段时间用很多时刻点来进行划分,并观察每一时刻物体的位置.这是因为在一段时间内所能做到的对物体的观察的次数只能是有限的,而任意两时刻间的时刻是无限的.幸运的是,微积分告诉我们,只需要相邻观察时刻充分的小,所获得的物体的运动就越接近于物体的真实运动,而物体的真实运动就是当相邻观察时刻的时间间隔趋于0的极限运动.所以为描述物体最一般的运动,只需要研究两个时刻物体的最一般的位置变化即可.一个叫 \textbf{初位置},一个叫\textbf{末位置}.
\addTODO{角动量性质、转动惯量、添加例子}