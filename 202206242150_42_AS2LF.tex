% 线性方程组的仿射解释
% 仿射线性函数|线性方程组|几何解释

\pentry{仿射空间\upref{AfSp},线性方程组\upref{LinEqu}}
在解释线性方程组的仿射几何意义前,先总结一下矩阵和矢量观点下的意义.

一般的线性方程组具有如下形式:
\begin{equation}\label{AS2LF_eq1}
\leftgroup{
&a_{11}x_1 + a_{12}x_2 + \dots + a_{1n}x_n\;\;\;=y_1\\
&a_{21}x_1 + a_{22}x_2 + \dots + a_{2n}x_n\;\;\;=y_2\\
&\qquad \qquad \dots  \qquad \dots \qquad  \dots\\
&a_{m1}x_1 + a_{m2}x_2 + \dots + a_{mn}x_n=y_m}\\
\end{equation}
写成\textbf{矩阵形式}则为
\begin{equation}\label{AS2LF_eq2}
\mat A \bvec x=\bvec y
\end{equation}
其中
\begin{equation}
\mat A=\begin{pmatrix}
a_{11}&a_{12}&\cdots&a_{1n}\\
a_{21}&a_{22}&\cdots&a_{2n}\\
\vdots&\vdots&\vdots&\vdots\\
a_{m1}&a_{m2}&\cdots&a_{mn}
\end{pmatrix}
,\quad \bvec x=\begin{pmatrix}
x_1\\x_2\\\vdots\\x_n
\end{pmatrix}
,\quad \bvec y=\begin{pmatrix}
y_1\\y_2\\\vdots\\y_n
\end{pmatrix}
\end{equation}
在\textbf{矢量观点}下,\autoref{AS2LF_eq1} 又可写为
\begin{equation}
\sum_{i=1}^n x_ia_i=y
\end{equation}
其中
\begin{equation}
a_i=(a_{1i},a_{2i},\cdots,a_{mi})^T,\quad y=(y_1,y_2,\cdots,y_m)^T
\end{equation}
在矩阵观点下,\autoref{AS2LF_eq2} 有解,当且仅当 $\rank$

