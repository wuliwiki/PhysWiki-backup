% 一阶常系数线性微分方程组(常微分方程)
% 常微分方程|ODE|ordinary differential equation|方程组|线性变换|矩阵|矩阵指数

\pentry{矩阵指数\upref{MatExp},常系数线性微分方程\upref{ODEb2}}

\subsection{齐次方程}

一阶常系数\textbf{齐次}线性微分方程组形如
\begin{equation}\label{eq_ODEb3_1}
\leftgroup{
    \frac{\dd}{\dd t}x_1&=a_{11}x_1+a_{12}x_2+\cdots+a_{1n}x_n\\
    \frac{\dd}{\dd t}x_2&=a_{21}x_1+a_{22}x_2+\cdots+a_{2n}x_n\\
    &\vdots\\
    \frac{\dd}{\dd t}x_n&=a_{n1}x_1+a_{n2}x_2+\cdots+a_{nn}x_n~,
}
\end{equation}
其中各 $x_i$ 是关于 $t$ 的未知函数,各 $a_{ij}$ 是已知常数。我们要研究的是如何解出这个方程组中的各未知函数。

别看这个方程有那么多变量 $x_i(t)$,实际上我们可以把它们放到一起,构成一个 $n$ 维向量 $\bvec{x}(t)=(x_1(t), x_2(t), \cdots, x_n(t))$,这样就可以理解为还是只有一个自变量,只不过自变量从标量变成向量了。

以上述向量理解的方式来看,\autoref{eq_ODEb3_1} 右边部分就是一个线性变换 $\mat{M}\bvec{x}(t)$,其中
\begin{equation}\label{eq_ODEb3_4}
\mat{M}=\pmat{
    &a_{11} &a_{12} &\cdots &a_{1n}\\
    &a_{21} &a_{22} &\cdots &a_{2n}\\
    &\vdots &\vdots &\ddots &\vdots\\
    &a_{n1} &a_{n2} &\cdots &a_{nn}
    }~
\end{equation}
是已知常数矩阵。

这样,我们就还可以把\autoref{eq_ODEb3_1} 写成
\begin{equation}\label{eq_ODEb3_3}
\frac{\dd}{\dd t}\bvec{x}=\mat{M}\bvec{x}~
\end{equation}
的形式,看起来和一元方程
\begin{equation}\label{eq_ODEb3_2}
\frac{\dd}{\dd t}x=ax~
\end{equation}
非常像。

\subsubsection{矩阵指数解法}


\autoref{eq_ODEb3_2} 的通解是 $x=C\E^{at}$,其中 $C$ 为常数。事实上,\autoref{eq_ODEb3_3} 的通解也可以类似地用\textbf{矩阵指数}\upref{MatExp}来表示:
\begin{equation}\label{eq_ODEb3_5}
\bvec{X}=\E^{\mat Mt}~.
\end{equation}

\begin{definition}{基解矩阵}
如果 $n\times n$ 矩阵 $\mat X$ 的每一列作为一个向量,都是\autoref{eq_ODEb3_1} 或者说\autoref{eq_ODEb3_3} 的线性无关解\footnote{于是有了 $n$ 个线性无关解,根据方程组的存在与唯一性定理,任何一个解都可以用基解矩阵里的列向量线性组合得出。},那么称 $\mat X$ 是 \autoref{eq_ODEb3_1} 或者说\autoref{eq_ODEb3_4} 的\textbf{基解矩阵}。
\end{definition}

\begin{theorem}{}
\autoref{eq_ODEb3_5} 是\autoref{eq_ODEb3_1} 或者说\autoref{eq_ODEb3_4} 的基解矩阵。
\end{theorem}

证明思路很简单,根据\autoref{the_MatExp_1}~\upref{MatExp},直接得到 $\frac{\dd}{\dd t}\mat{X}=\mat M\mat X$,又由于求导是对每个矩阵元独立进行的,因此把 $\mat X$ 看成是列向量排列而成的行矩阵,则每个列向量都是一个解。另外,由于 $\mat X|_{t=0}=\mat E$,其中 $\mat E$ 为单位矩阵,故 $\opn{tr}\mat X=n\not=0$,故根据\autoref{the_MatExp_2}~\upref{MatExp},$\abs{\mat{X}}$ 的行列式不为零,故各解向量线性无关。这样,线性无关的一组解就构成了基解矩阵。


基解矩阵的地位和基本解组一样,都能用来线性组合出所有的解。我们看一个实例:

\begin{example}{}
考虑方程组
\begin{equation}\label{eq_ODEb3_6}
\leftgroup{
    \frac{\dd}{\dd t}x_1&=x_1+x_2\\
    \frac{\dd}{\dd t}x_2&=x_2~,
}
\end{equation}
且已知初值
\begin{equation}\label{eq_ODEb3_8}
\leftgroup{
    x_1(0)&=3\\
    x_1(0)&=2~.
}
\end{equation}
如何求解此方程呢?

首先我们要求出其基解矩阵:
\begin{equation}
\begin{aligned}
\mat X=\E^{\pmat{1&1\\0&1}t}&=\sum_{n=0}^\infty\frac{1}{n!}\pmat{1&n\\0&1}t^n\\
&=\sum_{n=0}^\infty\frac{1}{n!}\pmat{t^n&nt^n\\0&t^n}\\
&=\pmat{\E^t&t\E^t\\0&\E^t}~.
\end{aligned}
\end{equation}

容易验证,
\begin{equation}
x_1=\E^t~, x_2=0~
\end{equation}
和
\begin{equation}
x_1=t\E^t~, x_2=\E^t~
\end{equation}
都是\autoref{eq_ODEb3_6} 的解,且它们彼此在任何区间上都线性无关,因此 $\mat X$ 确实是\autoref{eq_ODEb3_6} 的基解矩阵。

现在设
\begin{equation}\label{eq_ODEb3_7}
\leftgroup{
x_1&=A\E^t+Bt\E^t\\x_2&=B\E^t~,
}
\end{equation}
其中 $A, B$ 是待定常数。

将\autoref{eq_ODEb3_7} 代入初值条件\autoref{eq_ODEb3_8} ,解出 $A, B$,最终得到满足初值条件的特解
\begin{equation}
\leftgroup{
    x_1&=3\E^t+2t\E^t\\
    x_2&=2\E^t~.
}
\end{equation}


\end{example}

\subsubsection{特征向量解法}

如果 $n\times n$ 矩阵 $\mat M$ 有 $n$ 个线性无关的特征向量 $\bvec{v}_i$,其特征值分别为 $\lambda_i$,那么\autoref{eq_ODEb3_5} 的另一种基解矩阵就可以写为
\begin{equation}\label{eq_ODEb3_9}
\mat X=\pmat{\E^{\lambda_{1}t}\bvec{v}_1&\E^{\lambda_{2}t}\bvec{v}_2&\cdots&\E^{\lambda_{n}t}\bvec{v}_n}~.
\end{equation}

验证是很容易的:$\frac{\dd}{\dd t}\E^{\lambda_{i}t}\bvec{v}_i=\lambda_i\E^{\lambda_{i}t}\bvec{v}_i=\mat M\E^{\lambda_{i}t}\bvec{v}_i$。

如果 $\mat M$ 没那么多线性无关的特征向量(换句话来说,就是无法对角化),那么\autoref{eq_ODEb3_9} 虽然仍旧是解,却不够全面,从而不是基解矩阵。我们接下来讨论这种情况下的基解矩阵求法。

我们的目标是脱离无穷级数来计算矩阵指数 $\E^{\mat Mt}$,这是因为无穷级数常常难以计算。虽然接下来使用的方法也常有很大计算量,但这也是没办法的事,$\mat M$ 无法对角化导致过程不会像\autoref{eq_ODEb3_9} 那么简单了。

首先注意到,如果设列向量 $\uvec{e}_i$ 为“第 $i$ 行为 $1$,其余为 $0$”,那么我们应该有
\begin{equation}\label{eq_ODEb3_12}
\E^{\mat Mt}=\pmat{\E^{\mat Mt}\uvec{e}_1&\E^{\mat Mt}\uvec{e}_2&\cdots&\E^{\mat Mt}\uvec{e}_n}~,
\end{equation}

问题转化为计算 $\E^{\mat Mt}\uvec{e}_i$。

设 $\mat E$ 是\textbf{单位矩阵},那么我们有
\begin{equation}
\E^{\lambda_it}\E^{-\lambda_i\mat Et}=\mat E~.
\end{equation}

因此,
\begin{equation}\label{eq_ODEb3_10}
\begin{aligned}
\E^{\mat Mt}\uvec{e}_j&=\E^{\mat Mt}\mat E\uvec{e}_j\\
&=\E^{\mat Mt}\E^{\lambda_it}\E^{-\lambda_i\mat Et}\uvec{e}_j\\
&=\E^{\lambda_it}\E^{(\mat M-\lambda_i\mat E)t}\uvec{e}_j\\
&=\E^{\lambda_it}\qty(\sum_{k=0}^\infty \frac{t^k}{k!}(\mat M-\lambda_i\mat E)^k)\uvec{e}_j~,
\end{aligned}
\end{equation}

\autoref{eq_ODEb3_10} 里仍有无穷级数。

将 $\uvec{e}_j$ 进行\textbf{特征分解},则对于其每一个特征分量 $\bvec{v}$,设其\textbf{重数}为 $m$,都有
\addTODO{引用特征向量与特征分解相关词条。}
\begin{equation}\label{eq_ODEb3_11}
\qty(\sum_{k=0}^\infty \frac{t^k}{k!}(\mat M-\lambda_i\mat E)^k)\bvec{v}=\qty(\sum_{k=0}^n \frac{t^k}{k!}(\mat M-\lambda_i\mat E)^k)\bvec{v}~,
\end{equation}

这样就把无穷级数计算化为多个有限和计算了。

这种计算方式规避了直接计算无穷级数,但复杂点在于要先关于 $\mat M$ 对各 $\uvec{e}_j$ 进行特征分解,再分别计算用特征分解计算\autoref{eq_ODEb3_11} ,最后再利用\autoref{eq_ODEb3_10} 、\autoref{eq_ODEb3_12} 组合成整个 $\E^{\mat Mt}$。









