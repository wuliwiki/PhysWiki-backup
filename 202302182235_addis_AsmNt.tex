% 汇编语言笔记(GAS, x86-64)

\begin{issues}
\issueDraft
\end{issues}

一个例子:
\begin{lstlisting}[language=none]
.section .data
    msg: .string "Hello,\nWor\nabcdefg"

.section .text
    .globl main

main:
    movl $4, %eax # The number 4 is the system call for write
    movl $1, %ebx # The number 1 is the file descriptor for stdout
    movl $msg, %ecx # The address of the message to be written
    movl $18, %edx # The length of the message
    int $0x80 # Trigger the system call

    movl $1, %eax # The number 1 is the system call for exit
    xorl %ebx, %ebx # The exit status code (0 in this case)
    int $0x80 # Trigger the system call
\end{lstlisting}

编译: \verb|gcc -no-pie -o hello.x hello.s|

\begin{itemize}
\item assembler 是汇编语言的编译器
\item 汇编语言的语法依赖于 CPU 架构和具体的 assembler。
\item \verb|gcc| 使用的 assembler 是 \textbf{GNU Assembler (GAS)}
\item 注释用 \verb|#|
\item \verb|movl $4, %eax| 把 register \verb|eax| 的值设为 4。 \verb|%| 表示 register, 
\item \verb|变量名 db '字符串', 0| 定义一个字符串。 其中 \verb|db| 是 \textbf{define byte}, \verb|0| 是字符串结尾的 Null。
\item 也可以用 \verb|变量名: .asciz "字符串\n"|
\item \verb|int 0x80| 开始 software interrupt, 进行 system call。
\item \verb|mov eax, 1| 是 system call \verb|exit()|
\item \verb|xor ebx, ebx| 可以把 register \verb|ebx| 变为 0。
\end{itemize}
