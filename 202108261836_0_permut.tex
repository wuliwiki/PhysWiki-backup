% 排列
% 排列|集合|序列|阶乘

% 未完成: 可以补充更多内容

\pentry{映射\upref{map}, 阶乘\upref{factor}}

\footnote{参考 Wikipedia \href{https://en.wikipedia.org/wiki/Permutation}{相关页面}.}我们讨论含有 $N$ 个元素的任意集合, 由于集合中元素的名称不重要, 我们以下将它记为 $\qty{1,2,\dots, N}$. 注意集合的是没有顺序的, 例如 $\qty{1,2,3}$ 和 $\qty{1,3,2}$ 是同一个集合. 当我们把集合 $S$ 中的的元素按照某种顺序排列成一个序列时, 就称为它是集合 $S$ 的一种\textbf{排列(permutation)}. 每个排列可以看作一个映射 $f:\qty{1,\dots,N}\to\qty{1,\dots,N}$.

那么 $N$ 个元素的集合一共有几种不同的排列呢? 第 1 个位置有 $N$ 种不同的可能, 确定之后第 2 个位置有 $N-1$ 种不同的可能, 第 3 个位置有 $N-2$ 种…… 最后一个位置只有 1 种. 所以可能性的种数可以用阶乘\upref{factor} 表示, 记为 $A_N$
\begin{equation}
A_N = N! = N(N-1)(N-2)\dots 1
\end{equation}

我们可以把第 $i$ 种排列记为 $p_i$, 该排列的元素按照顺序分别记为 $p_{i,1}, p_{i,2}, \dots, p_{i,N}$.

一般地,从 $n$ 个不同元素中任取 $m$ ($m \leq n$)个元素, 按照一定的顺序排成一列,称为从 $n$ 个不同元素中取出 $m$ 个元素的一个排列.

根据一个排列的定义,两个排列相同的含义为:组成排列的元素相同,并且元素的排列顺序也相同.

从 $n$ 个不同元素中取出 $m$ ($m \leq n$)个元素的所有排列的个数,称为从n个不同元素中取出m个元素的排列数,用符号 $A_n^m$ 表示.

根据分步乘法计数原理可得排列数公式\begin{equation}
A_n^m = n (n - 1)(n - 2) \cdots (n - m + 1)
\end{equation}

我们对排列数公式进行变形,\begin{equation}\label{permut_eq1}
\begin{aligned}
A_n^m &= \frac{n(n - 1)(n - 2)\cdots	(n - m + 1)(n - m)\cdots 3\cdot 2 \cdot 1}{(n - m)(n - m -1)\cdots 3\cdot 2 \cdot 1}\\
&=\frac{n!}{(n - m)!}
\end{aligned}
\end{equation}
\autoref{permut_eq1} 为排列数公式的另一种表达形式.
