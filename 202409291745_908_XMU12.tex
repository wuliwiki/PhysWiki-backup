% 厦门大学 2012 年 考研 量子力学
% license Usr
% type Note

\textbf{声明}:“该内容来源于网络公开资料,不保证真实性,如有侵权请联系管理员”

\subsection{一 、}

(1)波函数归一化条件的物理意义是什么?物理上对波函数有哪些要求?

(2)什么是幺正算符?若$ A,B,C$ 为幺正算符则它们的积 $ABC$ 是不是幺正算符?为什么?

(3)光的辐射分成几种过程?若粒子由能级 $E2$ 跃迁到能级 $E1$,写出辐射光子的频率。

(4)对处于某量子态的电子,如沿 $Z$ 轴方向测量其自旋,总是得到$+\frac{h}{2}$
 的结果,那么沿 $X$ 轴方向测量其自旋会得到什么结果?

(5)若算符 $A$ 不显含时间且与体系的哈密顿量 $H$ 对易,即$[A,H]=0$,那么$A$是体系的守恒量吗?说出你的判断理论。
\subsection{二、}
设粒子处在下列一维无限深势阱中,
\[V(x) = \begin{cases} 0, & 0 \leq x \leq a \\\\\infty, & \text{其余区域}\end{cases}~\]

(1) 写出粒子波函数所满足的边界条件;

(2) 写出粒子的能级 $E_n$ 以及相应的波函数 $\Psi_n(x)$;

(3) 若粒子的初始波函数为 $\Psi(x,0) = A\sin^3 \left(\frac{\pi x}{a}\right)$,$0 \leq x \leq a$,式中 $A$ 为归一化常数。

[提示:$\sin 3\theta = 3 \sin \theta - 4 \sin^3 \theta$,不知道这个式子对不对]

①求出 $A$ 和 $t$ 时刻的波函数$\Psi(x,t)$;

②计算粒子的坐标平均值$x$.
\subsection{三、}
设算符 $A$ 和 $B$ 不对易,$[A,B]=C$,但 $A$ 和 $B$ 都与 $C$ 对易,即
$[A,C]=0,[B,C]=0$ 试证明:

(1)$[A, B^m] =n CB^{m-1} ,m$ 为整数;

(2)$[A, e^B] =Ce^B$转子绕一固定点转动;

(3)$e^{A+B} =e^A+e^B+e^{-\frac{c}{2}}$ 
\subsection{四、}

采用自然单位制,类氢离子中电子处于能量本征态
$$\Psi(r,\theta)=\frac{1}{81}\sqrt{\frac{2}{\pi}}Z^{3/2}(6-Zr)e^{Zr/3}\cos\theta~$$
其中 $r$ 是以玻尔半径 $a$ 为单位表示的。

(1)求主量子数 $n$,轨道量子数 $l$ 和磁量子数 $m$ 的值;

(2)当一个电子处于$\Psi(r,\theta)$态且 $Z=1$ 时,计算 $r$ 的最可几半径;

(3)由$\Psi(r,\theta)$态出发构造另一个具有相同 $n,l$ 值,但是磁量子
数为 $m+1$ 的能量本征态.

[提示:在球坐标下$Y_{10}=\sqrt{\frac{3}{4 \pi}}\cos\theta,L_+=e^{i\varphi}\frac{\alpha}{\alpha\theta}+i\cot\theta e^{i\varphi}\frac{\alpha}{\alpha\varphi}$]
\subsection{五、}
一个由三个自旋 1/2 的离子构成的系统,三个粒子的自旋算
符分别为$\vec{S_1},\vec{S_2},\vec{S_3}$

(1)先考虑其中两个粒子的自旋耦合$\vec{S_{12}}=\vec{S_1}+\vec{S_2}$写出$\vec{S_{12}}$⃗的自旋量子数$S_{12}$的可能取值;

(2)再考虑三个粒子的自旋耦合$\vec{S}=\vec{S_12}+\vec{S_3}$写出其自旋量子数$s$的可能取值;

(3)假设系统的哈密顿量为$H=\frac{A}{\hbar^2}\vec{S_1}\cdot\vec{S_2}+\frac{B}{\hbar^2}(\vec{S_1}+\vec{S_2})\cdot\vec{S_3}$式中 $A,B$ 为常数,求系统的能级以及能级简并度。

[提示:选取系统的力学量完全集为($\vec{S_{12}} , \vec{S^2}  , \vec{S_Z}$ )]
\subsection{六、}
考虑一个受扰的处于无限深势阱中的粒子,其哈密顿量$H=\frac{p^2}{2m}+V(x)+H^\prime$式中为 $m$ 的粒子在二维无限深势阱中运动
