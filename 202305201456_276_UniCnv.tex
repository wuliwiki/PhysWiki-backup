% 一致收敛
% 数学分析|一致收敛|极限



\addTODO{需添加预备知识,收敛或逐点收敛,以及一致连续。}

\subsection{一致收敛的概念}

函数列 $f_n$ 逐点收敛到 $f$,只要求对于任意固定的 $x_0$,数列 $\{f_n(x_0)\}$ 收敛到 $f(x_0)$ 即可。接下来介绍的\textbf{一致收敛}则是一个更强的要求。

阅读过程中要牢记,谈一致收敛的时候一定是限定了定义域的。同一个函数可能在一个区间里一致收敛,但是拓展一下定义域就不一致收敛了。\autoref{ex_UniCnv_1} 和\autoref{ex_UniCnv_2} 中的反例,如果把定义域缩小为 $[1/3, 1/2]$,那么它们其实也是一致收敛的。

一个函数列 $f_n(x)$ 一致收敛到 $f(x)$ 的定义是: 当对于任意 $\epsilon > 0$, 存在 $N$, 当 $n \geqslant N$ 时对任意 $x$ 都有
\begin{equation}\label{eq_UniCnv_1}~,
\abs{f_n(x) - f(x)} < \epsilon
\end{equation}
或者记为
\begin{equation}
\lim_{n\to\infty} \qty(\max \abs{f_n(x) - f(x)}) = 0
\end{equation}

一致收敛是比(逐点)收敛更强的条件。

\begin{example}{逐点收敛但不一致收敛的例子}\label{ex_UniCnv_1}
在区间 $(0, 1)$ 上考虑函数列 $f_n(x)=x^n$ 和函数 $f(x)=0$,则容易验证 $f_n$ 逐点收敛到 $f$。

但是,对于任意 $\epsilon>0$ 和任意 $f_n$,总可以找到 $x\in(\epsilon^{1/n}, 1)$,使得 $f_n(x)>\epsilon$。因此按照定义,$f_n$\textbf{不}一致收敛到 $f$。

\end{example}

\begin{example}{逐点收敛但不一致收敛的例子}\label{ex_UniCnv_2}
在区间 $[0, 1]$ 上考虑函数列 $f_n(x)=\sin(\pi\cdot x^n)$ 和函数 $f(x)=0$,则容易验证 $f_n$ 逐点收敛到 $f$。

但是,对于任意 $\epsilon>0$ 和任意 $f_n$,总可以找到 $x=\frac{1}{2^n}$,使得 $f_n(x)=1>\epsilon$。因此按照定义,$f_n$\textbf{不}一致收敛到 $f$。

\end{example}

逐点收敛只考虑,是不是每个点都收敛。而一致收敛的威力在于,有一个统一的进度,每个点收敛的进度不得比这更慢。\autoref{ex_UniCnv_1} 和\autoref{ex_UniCnv_2} 里举出的反例就是如此,不管你怎么取 $\epsilon$ 作为限定,总存在跟不上进度的点,导致这种限定不是\textbf{一致}的。

如果 $f_n$ 一致收敛到 $f$,那么我们会看到,随着 $n$ 增大,$f_n-f$ 的上下界越来越小,像是把函数列挤压到 $x$ 轴的过程。但是不一致收敛的函数,由于总存在不听话的、跟不上进度的点,就没法把函数压平。

一致收敛还可以从“函数之间的距离”角度来考虑。

\begin{definition}{函数的距离}\label{def_UniCnv_1}

在同一定义域上给定两个函数 $f$ 和 $g$,定义它们之间的距离为 $\opn{d}(f, g)=\opn{sup}\abs{f-g}$,即距离为函数 $\abs{f-g}$ 的\textbf{上确界}\upref{SupInf}。

有时候也记 $\opn{d}(f, g)=\abs{\abs{f, g}}$。

\end{definition}

把每个函数看成“函数的集合”里的一个点,\autoref{eq_UniCnv_1} 给出了衡量各点之间距离的方式。一致收敛的函数列,就是这个集合里一致收敛的点列。由此可见,一致收敛是一种更注重\textbf{函数整体}的性质。


\subsection{一致收敛的性质}

在研究和一致收敛相关的问题时,我们可以只考虑函数列收敛到 $f(x)=0$ 的情况。这是因为 $\{f_n(x)\}$(一致)收敛到 $f(x)$,等价于说 $\{f_n(x)-f(x)\}$ 一致收敛到 $0$。

\begin{theorem}{线性性}
设给定定义域上,函数列 $\{f_n\}$ 和 $\{g_n\}$ 分别一致收敛到 $f$ 和 $g$,且 $a, b$ 是任意常数,那么函数列 $\{af_n+bg_n\}$ 一致收敛到 $af+bg$ 上。
\end{theorem}

\begin{theorem}{有界乘积收敛}\label{the_UniCnv_1}
设给定定义域上,函数列 $\{f_n\}$ 一致收敛到 $f$ 上,且 $g$ 是一个有界函数,那么 $\{gf_n\}$ 一致收敛到 $gf$ 上。
\end{theorem}

\autoref{the_UniCnv_1} 的一个证明思路提示:由于 $g$ 有界(设上下界绝对值中较大的为 $G$),因此 $gf_n$ 和 $gf$ 的偏差也是“一致”的,即不会超过 $Gf$。由此,对于给定的 $n$,如果 $\abs{f_n(x)-f(x)}<\epsilon$ 恒成立,那么 $\abs{gf_n(x)-gf(x)}<G\epsilon$ 恒成立,再由 $\epsilon$ 的任意性即可得证。

如果取 $g$ 是无界函数,那么 $g$ 的趋于无穷的点就会让 $\{gf_n\}$ 出现一个跟不上进度的点,导致 $\{gf_n\}$ 不是一致收敛的,如\autoref{ex_UniCnv_3} 所述。

\begin{example}{}\label{ex_UniCnv_3}
在 $(0, 1)$ 上考虑函数列 $\{f_n(x)=1/n\}$,显然它一致收敛到 $f(x)=0$。

取 $g(x)=1/x$,那么 $g$ 没有上下界。对于任意正整数 $n$ 和 $\epsilon>0$,总有 $x\in(0, \frac{1}{n\epsilon})$,使得 $g(x)f_n(x)>\epsilon$,从而 $\{gf_n\}$ 不一致收敛。
\end{example}

\begin{theorem}{双有界乘积收敛}\label{the_UniCnv_2}
设给定定义域上,函数列 $\{f_n\}$ 和 $\{g_n\}$ 分别一致收敛到 $f$ 和 $g$,且 $f, g$ 都是有界函数,那么 $\{f_ng_n\}$ 一致收敛到 $fg$ 上。
\end{theorem}

\autoref{the_UniCnv_2} 的一个证明思路提示:设 $g$ 上下界中绝对值较大的为 $G$。$g_n$ 一致收敛,由一致收敛的定义,任取 $\epsilon>0$,存在正整数 $N$,使得编号 $n>N$ 时,所有 $g_n$ 都共享上下界 $G+\epsilon$ 和 $-G-\epsilon$。这样一来,问题就可以归结为\autoref{the_UniCnv_1} 的情况,只不过这里固定的有界函数可以取常数函数 $G$。

\begin{theorem}{}\label{the_UniCnv_3}
设给定定义域上,函数列 $\{f_n\}$ 和 $\{g_n\}$ 分别一致收敛到 $f$ 和 $g$,且 $f$ 是有界函数,$\abs{g}$ 有一个下界 $\delta>0$,那么 $\{\frac{f_n}{g_n}\}$ 一致收敛到 $\frac{f}{g}$。
\end{theorem}

\autoref{the_UniCnv_3} 的一个证明思路提示:利用 $\abs{g}$ 有一个非零下界,证明 $1/g_n$ 是一个有界函数,且 $1/g_n$ 一致收敛到 $1/g$。然后问题就可以归结为\autoref{the_UniCnv_2} 的情况。

\begin{theorem}{链式收敛}\label{the_UniCnv_4}
设给定定义域 $X$ 上,函数列 $\{f_n\}$ 一致收敛到 $f$ 上,且在区间 $I$ 上,$h(x)$ 是一个\textbf{一致连续}的函数,另外对于任意 $x\in X$,都有 $f_n(x)\in I, f(x)\in I$。

那么 $h(f_n(x))$ 一致收敛到 $h(f(x))$ 上。
\end{theorem}

\autoref{the_UniCnv_4} 的要点是,$h$ 必须是一致收敛的,不可以把条件减弱成逐点收敛。

\begin{theorem}{柯西收敛原理}\label{the_UniCnv_6}
函数列 $\{f_n\}$ 在定义域上一致收敛的充要条件是,对于任意 $\epsilon>0$,存在 $N$,使得对于任意正整数 $m, n>N$,都有 $\abs{f_n(x)-f_m(x)}>\epsilon$ 恒成立,或者用\autoref{def_UniCnv_1} 的话来说,$\abs{\abs{f_n-f_m}}>\epsilon$。
\end{theorem}

这里的柯西收敛原理,其实就是直接引用了度量空间里的柯西收敛原理。

\begin{theorem}{Dini定理}\label{the_UniCnv_5}
如果\textbf{连续函数}$\{f_n\}$ 在\textbf{闭区间}$[a, b]$ 上\textbf{逐点收敛}到 $f$,且对于任意固定的 $x_0\in [a, b]$,数列 $\{f_n(x_0)\}$ 都是单调的,那么 $\{f_n\}$ 在 $[a, b]$ 上一致收敛到 $f$。
\end{theorem}

\autoref{the_UniCnv_5} 的一个证明思路提示:闭区间上的连续函数必然能取到最大和最小值,因此必然有界。又由于 $\{f_n(x_0)\}$ 都是单调的,进而 $\{f_n(x_0)-f(x_0)\}$ 都是单调的,因此 $f_n-f$ 的上下界也是单调函数。由于 $f_n-f$ 收敛到 $0$,因此 $f_n-f$ 的上下界都要收敛到 $0$,即 $f_n$ 和 $f$ 的距离要收敛到 $0$,从而得证。









