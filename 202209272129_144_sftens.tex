% 表面张力
% 表面张力|液体|表面能

\pentry{亥姆霍兹自由能\upref{HelmF}}

\subsection{表面张力}

液体与自己的蒸汽或另一种介质接触的交界面为液体的表面.实验表明,对液体表面的每一处,假想画一条线,则该线两侧的页面相互存在拉力的作用,该拉力垂直于此线.我们称这个\textbf{位于液面内处处与此线垂直的拉力}为\textbf{表面张力}.\textbf{表面张力系数} 的意思是:单位长度的直线两侧液面的相互拉力.表面张力系数用字母 $\sigma$ 表示.

例如:一个宽 $L$ 的导轨上,用可移动的铁丝拉出一块长方形肥皂泡,铁丝受肥皂泡液面的拉力为 $\sigma L$;由于肥皂泡有两个表面,所以铁丝受到肥皂泡的总拉力为 $2\sigma L$.如果将铁丝移动 $\Delta x$ 的距离使得肥皂泡的面积扩大 $L\Delta x$,就要对铁丝做功 $2\sigma L\Delta x$.注意肥皂泡的\textbf{表面总面积}为 $\Delta S=2 L \Delta x$,所以我们得到做功的表达式:
\begin{equation}
\Delta W=\sigma \Delta S
\end{equation}
该式不仅对上述例子成立,对任意形状的液面都成立.例如我们将肥皂泡吹大,受表面张力的影响,肥皂泡内部的气压大于外部的气压,在吹的过程中肥皂泡被做的功为 $\sigma\cdot 4\pi R^2$.

表面张力系数随温度的升高而降低.一般在分析表面张力的时候,我们通常假定温度不变.

一个关于表面张力微观来源的简要解释可以参考\autoref{MetInt_sub1}~\upref{MetInt}.还可以从能量的角度初步理解:内部的原子与四周的原子都成键;而表面的原子只有下半部分成键,上半部分并未成键;我们知道“成键放能”,因此表面的原子由于未完全成键而能量更高.

\subsection{拉普拉斯公式}
如何计算肥皂泡内外的压强差呢?我们转换成研究更一般的问题:对于介质 $A$ 和介质 $B$(它们可以是液体或气体),假如它们的分界面是一个曲面,并且处于平衡态,那么两个介质在分界面附近存在一个压强差:$\Delta p=p_B-p_A$.为什么会存在这样的压强差呢?原因是两个介质的性质不同,导致其分界面处存在表面张力;而曲面上沿各个方向的表面张力会导致曲面一侧向另一侧的合力,这与压强差相平衡.

拉普拉斯公式给出了对于弯曲分界面两侧的压强差的定量化的计算公式.假设我们要研究弯曲分界面上一点处的张力大小(这一点处沿各个方向的张力是相同的),那么它与该点处曲面的曲率有关.设该点处两个正交的曲率半径分别为 $r_1$ 和 $r_2$,那么分界面两侧的压强差为(假设曲率中心在介质 $B$ 侧时,曲率半径 $r$ 为正,在 $A$ 侧时 $r$ 为负)
\begin{equation}
\Delta p=p_B-p_A=\sigma(\frac{1}{r_1}+\frac{1}{r_2})
\end{equation}
曲率半径越小,压强差越大,这是合乎我们的期待的:在某一点处曲率半径较小,该曲面微元朝各个方向的张力的合力就更大,因此需要更大的压强差去与它平衡.

对于肥皂泡来说,$r_1=r_2$,因此 $\Delta p=2\sigma/r$;但事情并没有结束,肥皂泡的气液分界面一共有两个,所以在两个分界面处都存在压强差.因此肥皂泡内外压强差为 $2\cdot 2\sigma/r=4\sigma/r$.

如何证明拉普拉斯公式呢?首先,压强差只与曲面某一点附近的曲面微元的形状有关,而不与远处的曲面有关,这是显然的.因此我们需要提炼出曲面微元的一些数学量,例如曲面方程的二阶偏导、曲率半径等等.为了能够求出压强差,我们能够利用的信息是,压强差与作用在微元上的张力合力相平衡的,但这很难计算.另一个想法是从能量的角度考虑,利用\textbf{虚功原理}\footnote{可以参考词条虚位移、虚功、虚功原理\upref{VirWrk}}.这意味着,如果我讲曲面朝一侧收缩(或膨胀)一个无穷小距离 $\delta x$,那么这一无穷小过程中表面张力做的功等于压强差做的功,或者准确地来说,这两个功之间的差异是 $O(\delta x^2)$ 级别的,是高阶无穷小量;用热力学的语言来说,总\textbf{自由能}关于 $x$ 的一阶偏导为 $0$,这正保证了两介质的分界面曲面在当前位置是处于受力平衡状态.

现在尝试作具体的计算.曲面微元的两个正交曲率半径分别为 $r_1,r_2$,我们取这样一个曲面微元,其面积为 $s=r_1 \dd \theta_1 \cdot r_2 \dd \theta_2$,$\dd\theta_1,\dd\theta_2$ 是两个曲率半径方向上曲面微元的角度变化量(可以看作是圆心角).如果曲面膨胀一个无穷小距离 $\delta x$,且曲面微元形状不变,$\dd \theta_1,\dd \theta_2$ 不变.那么其面积变化为
\begin{equation}
\delta s=(r_1+\delta x)\dd \theta_1\cdot 
(r_2+\delta x)\dd \theta_2 - r_1 \dd \theta_1 \cdot r_2 \dd \theta_2=(r_1+r_2)\delta x \dd \theta_1\dd \theta_2
\end{equation}
那么张力所做的虚功为 $\sigma \delta s$,压强差所作的功为 $s\Delta p \cdot \delta x$,两者应当相等:
\begin{equation}
\sigma\delta s=s\Delta p \delta x\Rightarrow \sigma(r_1+r_2)=r_1r_2\Delta p
\end{equation}
最终我们就得到了拉普拉斯公式:
\begin{equation}
\Delta p=p_B-p_A=\sigma(\frac{1}{r_1}+\frac{1}{r_2})
\end{equation}


\subsection{表面能}
以上我们利用的虚功原理,实际上在热力学中对应于特定的\textbf{平衡态判据}\footnote{具体参考热动平衡判据\upref{equcri}.}.在分析表面张力时,我们通常作近似处理,假定温度不变,那么表面张力系数也为恒定的常数.对于一个恒温的热力学系统,运用虚功原理就得到了自由能判据\upref{HelmF},压强所作虚功与表面张力所作虚功相抵消,实际上对应于系统总自由能的一阶变分为 $0$.对于有多介质分界面且需考虑表面张力的热力学系统,其自由能需要包含 $\sigma S$ 项.此时系统的自由能也被叫做\textbf{表面能}.

考虑恒温系统,再将整个大系统看成一个恒定体积的封闭容器.根据自由能判据\upref{HelmF},在稳定平衡状态下
\begin{equation}
\delta F=0,\delta^2 F>0
\end{equation}

通过这个判据我们可以推出许多结论.例如容器中充满液体 $A$,液体 $A$ 中悬浮着一滴液体 $B$,其形状近似为半径为 $r$ 的球体.那么总自由能的变分为
\begin{align}
\delta(F_A+F_B)&=-S_A\delta T-p_A\delta V_A-S_B\delta T-p_B\delta V_B+\sigma\delta S\\
&=(p_A-p_B)4\pi r^2\dd r + (\sigma_A+\sigma_B)8\pi r\dd r=0
\end{align}

所以我们发现,$p_A\neq p_B$,在这个热力学稳定平衡的系统中,由于表面张力的存在,有压强差.计算得
\begin{equation}
p_B-p_A=\frac{2\sigma}{r}
\end{equation}

更一般地,有拉普拉斯公式:
\begin{equation}\label{sftens_eq1}
\Delta p = \sigma \left(\frac{1}{r_1}+\frac{1}{r_2}\right)
\end{equation}

其中 $r_1,r_2$ 代表两个正交的曲率半径.

待补充:
弯曲液面附加压强(对饱和蒸气压的影响)
对液滴的分析(临界半径),过饱和蒸汽,沸腾,过热液体