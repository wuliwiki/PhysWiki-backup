% 古斯塔夫·基尔霍夫(综述)
% license CCBYSA3
% type Wiki

本文根据 CC-BY-SA 协议转载翻译自维基百科\href{https://en.wikipedia.org/wiki/Gustav_Kirchhoff}{相关文章}。

\begin{figure}[ht]
\centering
\includegraphics[width=6cm]{./figures/ba04c3fbf9d952c3.png}
\caption{} \label{fig_JRHF_1}
\end{figure}
古斯塔夫·罗伯特·基尔霍夫(德语:[ˈgʊs.taf ˈkɪʁçhɔf];1824年3月12日–1887年10月17日)是德国的物理学家、数学家和化学家,他在电路学、光谱学以及加热物体的黑体辐射发射等基本理解方面做出了重要贡献。[1][2] 他还在1860年提出了“黑体”这一术语。[3]

有几个不同的概念集被称为“基尔霍夫定律”,包括基尔霍夫电路定律、基尔霍夫热辐射定律和基尔霍夫热化学定律。

“本生–基尔霍夫光谱学奖”是以基尔霍夫和他的同事罗伯特·本生的名字命名的。