% 沸腾
% keys 沸腾|暴沸|水

\textbf{沸腾}是在液体表面及液体内部同时发生的剧烈的汽化现象.

我们用水壶烧水时将看到几个不同的阶段.\textbf{烧到一定程度},可以在水壶底看到一些小气泡积聚在水壶底部.一些小气泡可能会脱离底部上升,但在上升过程中会越来越小直至消失.\textbf{再过一段时间},一些气泡能够到达液面变成很小的空气气泡而破裂,此时能听到“吱吱”的声音.\textbf{再后来},气泡在上升的过程中不断增大而冒出液面,整个液体呈现上下翻滚的剧烈汽化状态,这就是\textbf{沸腾现象}.
\begin{figure}[ht]
\centering
\includegraphics[width=6cm]{./figures/ebull_1.png}
\caption{附在容器底部的气泡}} \label{ebull_fig1}
\end{figure}
要解释沸腾现象,我们需要借助一定热学知识.\textbf{小气泡的产生原因}是:空气在水中的溶解度随水温升高而降低,温度较高的下层水的部分空气分子首先脱溶,所以一般气泡会先积聚在底部.现在设液体内部一个半径为 $r$ 的气泡距表面距离为 $h$,附在容器底部.设大气压强为 $p_0$,液体的密度为 $\rho$,则在这个深度上液体压强为
\begin{equation}
p_0+\rho gh 
\end{equation}

再由液体表面张力的\textbf{拉普拉斯公式}(\upref{sftens}\autoref{sftens_eq1}),气泡内气压为
\begin{equation}\label{ebull_eq1}
p=p_0+\frac{2\sigma}{r}+\rho gh
\end{equation}

\textbf{气泡内气体可分为两部分}:从液体中脱溶的空气、液体蒸汽.设空气分子摩尔数为 $\nu$,根据理想气体状态方程(\upref{PVnRT}\autoref{PVnRT_eq1}),空气的分压\upref{PartiP}为 $\nu RT/V$.根据饱和蒸气压方程(\upref{Clapey}\autoref{Clapey_eq2}),气泡内液体蒸汽的分压为 $p_r=A-B/T$(其中 $A,B$ 是依赖于系统的常数,该方程为近似公式).我们可以将前面的\autoref{ebull_eq1} 改写为(通常 $\rho gh\ll p_0$,所以略去该项)
\begin{equation}
\frac{\nu RT}{V}+p_r=p_0+\frac{2\sigma}{r}
\end{equation}

随着温度的增大,饱和蒸气压 $p_r$ 也增大,上式左侧(泡内)压强将大于右侧压强,所以气泡会胀大.但是在胀大的同时,左侧的 $\nu RT/V$ 又会减小,达到一个平衡.