% 理想气体的状态密度(相空间)
% keys 理想气体|状态密度|球体体积|相空间

\begin{issues}
\issueDraft
\end{issues}

\pentry{$N$ 维球体的体积\upref{NSphV},相空间\upref{PhSpace}}

统计力学中一个非常重要的基本假设是\textbf{等概率原理},它讲的是:\textbf{处于平衡状态的孤立系统,系统各个可能的微观状态出现的概率是相等的}.也许你第一眼看到这个原理会百思不得其解,因为它一定程度上违背我们的直觉:这个“可能的微观状态”的“可能”是指什么?系统中每个粒子的速度可以任意大吗?我们平常遇到的这些孤立系统的平衡态,它们都有稳定不变的宏观性质(体积,压强,温度等),任何微观状态的概率怎么会“平权”呢?…… 这一系列问题将在我们的计算和思考过程中得到解决,而我们将会惊叹于等概率原理的绝妙之处.

让我们先来解决几个数学问题,为以后的道路作铺垫:
\subsection{能量和动量的状态密度}
对于一个 $6N$ 维的状态空间($N$ 个粒子,这里只考虑它们的 $x,y,z,p_x,p_y,p_z$ $6$ 个自由度),单个粒子质量为 $m$,系统的总能量为:
\begin{equation}
E=\sum_{i} \frac{p_{i,x}^2+p_{i,y}^2+p_{i,z}^2}{2m}
\end{equation}

能量 $E$ 只和动量 $p_x,p_y,p_z$ 有关\footnote{统计力学主要研究的气体系统中,粒子的相互作用势可忽略不计,所以在计算能量时可以不考虑势能 $V$},所以相空间中能量 $E$ 对应着动量空间的一个球面,这个动量空间的维数是 $3N$.

我们想要知道 $E$~$E+\Delta E$ 中有多少个微观状态(包括了多少个相空间中的状态点).根据统计力学中的量子力学假设,每个状态点占据相空间中 $h^{3N}$ 的体积(称它为相格\upref{PhSpace}),那么状态点的个数就是相空间中 $E$~$E+\Delta E$ 所包括的体积除以 $N!h^{3N}$(注意我们有全同粒子假设,交换两个粒子所对应的微观状态是同一个,所以还要除以 $N!$).这个状态点的个数除以 $\Delta E$,就是所谓的\textbf{状态密度}——单位能量所包括的状态点的个数.

先考虑能量 $<E$ 的状态点的个数.对于位置坐标没有限制,位置空间的体积为 $V^N$.再计算动量空间中半径为 $\sqrt{2Em}$ 的球体的体积\footnote{可以根据体积公式\autoref{NSphV_eq8}~\upref{NSphV}}.相空间中能量小于 $E$ 的状态数为
\begin{equation}\label{IdSDp_eq2}
\Omega_0 = \frac{V^N}{N! h^{3N}} \frac{(2\pi mE)^{3N/2}}{\Gamma(3N/2+1)}
\end{equation}
对 $E$ 求导就能得到能量的状态密度函数:
\begin{equation}\label{IdSDp_eq3}
g(E) = \frac{V^N}{N! h^{3N}} \frac{(2\pi m)^{3N/2}}{\Gamma(3N/2)} E^{3N/2 - 1}
\end{equation}
动量 $p=\sqrt{2Em}$,$g(E)\dd E=g(p)\dd p=g(p)\sqrt{m/(2E)}\dd E$,于是可以解得关于动量绝对值的状态密度函数为
\begin{equation}\label{IdSDp_eq4}
g(p) = \frac{V^N}{N! h^{3N}} \frac{2\pi^{3N/2}}{\Gamma(3N/2)} p^{3N - 1}
\end{equation}


\subsection{推导}
在 $N$ 个独立的,可区分粒子的相空间中, 能量小于 $E$ 的状态数(体积除以 $h^{3N}$, 无量纲)为
\begin{equation}\label{IdSDp_eq1}
\Omega_0 = \frac{1}{N! h^{3N}} \int\limits_{\sum p^2 \leqslant 2mE} \dd[3N]{q} \dd[3N]{p} = \frac{V^N}{N! h^{3N}} \int\limits_{\sum p^2 \leqslant 2mE} \dd[3N]{p}
\end{equation}
其中积分 $\int_{\dots} \dd[3N]{p} $ 可以看做 $n=3N$ 维球体的体积, 半径为 $R = \sqrt{2mE}$. 

$n$ 维空间中球体的体积\upref{NSphV}为
\begin{equation}
V_n = \frac{\pi^{n/2}}{\Gamma(n/2+1)}R^n
\end{equation}
代入 $n=3N$ 和 $R = \sqrt{2mE} $, 得
\begin{equation}
\int\limits_{\sum p^2 \leqslant 2mE} \dd[3N]{p} = \frac{(2\pi mE)^{3N/2}}{\Gamma(3N/2+1)}
\end{equation}
代入\autoref{IdSDp_eq1} 得\autoref{IdSDp_eq2}. 再对 $E$ 求导得\autoref{IdSDp_eq3}.
