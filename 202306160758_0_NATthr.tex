% FRP 内网穿透笔记

\begin{issues}
\issueDraft
\end{issues}

英文叫做 NAT Penetration, 好像也叫 ingress as a service

\subsection{自己搭建}
FRP 是最著名的穿透软件, 可以自己在服务器上面安装。 \href{https://github.com/fatedier/frp}{GitHub 主页}。
frp is a fast \textbf{reverse proxy} to help you expose a local server behind a NAT or firewall to the Internet.

README.md 中 ssh tunnel 的例子:
\begin{itemize}
\item A 是有公共 ip 的服务器(reverse proxy), B 是防火墙后面的服务器。
\item 下载 release 压缩包, \verb|fprs| 和 \verb|fprs.ini| (\verb|s| 表示 server)上传到服务器端, \verb|fprc| 和 \verb|fprc.ini| (\verb|s| 表示 client)
\item 映射 ssh 的例子: 服务器端设置文件:
\begin{lstlisting}[language=none,caption=fprs.ini]
[common]
bind_port = fpr使用的端口
\end{lstlisting}
\item A 上启动服务端 \verb|./frps -c ./frps.ini|
\item B 上设置文件:
\begin{lstlisting}[language=none,caption=fprc.ini]
[common]
server_addr = 公网ip或者域名
server_port = fpr使用的端口

[映射名]
type = tcp或udp
local_ip = 127.0.0.1
local_port = B的端口
remote_port = A的端口
\end{lstlisting}
\item B 上启动 \verb|./frpc -c ./frpc.ini|
\item 要开机启动, 可以设置 systemd\upref{systmd}
\item 现在第三个机器上, 用 \verb|ssh 用户名@公网ip或者域名 -p A的端口| 就可以了连接 ssh 了。
\item 貌似 \verb|[映射名]| 必须在所有的 client 中唯一
\item 如果用 frp 代理了客户端的某个端口, 那么客户端所在的局域网也仍然可以链接该端口。
\end{itemize}

\subsection{一些商业服务}
比自己搭建服务器贵得多
\begin{itemize}
\item Ngrok
\item Natapp
\item 小米球
\item Sunny-Ngrok
\item echosite
\item Ssh、autossh
\item Lanproxy
\item Spike
\item fcn
\item 花生壳
\end{itemize}

其他方案(包括 VPN, 端口映射)参考\href{https://johackim.com/how-to-expose-local-server-behind-firewall}{这里}。

另外参考一个\href{https://www.digitalocean.com/community/tutorials/how-to-forward-ports-through-a-linux-gateway-with-iptables}{具体教程}, 讲解如何设置网关(文章中称为防火墙)的两个网卡间的 port forwarding 可以让外网访问内网的 nginx 服务器。 
