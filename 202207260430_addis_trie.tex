% trie 树(字典树)
% trie树|字典树|数据结构|C++

$\mathtt{Trie}$ 树(字典树)高效的存储和查找字符串集合的数据结构.

假设要存储这些字符串:
\verb|cat|、\verb|her|、\verb|him|、\verb|no|、\verb|nova|.

对应在图上就长这样:\begin{figure}[ht]
\centering
\includegraphics[width=14.25cm]{./figures/trie_1.png}
\caption{$\mathtt{trie}$ 树} \label{trie_fig1}
\end{figure}

插入字符串:从根节点开始看,看有没有这个字母,有的话就走,没有的话就创建一个新的字母结点.存储每个单词一般都会在这个单词的结尾字母打一个标记(如图中的蓝色结点),意思是以这个字母为结尾的路径是有一个单词的.\href{https://pic2.zhimg.com/v2-cb9b476f3856b7ae68a00af6911c07a9_b.gif}{插入一个字符串动图}

查找字符串:比如要查找 \verb|him| 这个单词,我们就从根节点开始走,依次走每个字母,如果找到了要查找的字符串的结尾字母并且这个结点上有标记,就表示找到了这个字符串.\href{https://pic2.zhimg.com/v2-98c24afcfc74582fdc54c7381d29d639_b.gif}{查找字符串动图}

来看一个具体题目:维护一个字符串集合,支持两种操作:

\begin{enumerate}
\item \verb|I x| 向集合中插入一个字符串 x;
\item \verb|Q x| 询问一个字符串在集合中出现了多少次.
\end{enumerate}
共有 $N$ 个操作,输入的字符串总长度不超过 $10^5$,字符串仅包含小写英文字母.

\begin{lstlisting}[language=cpp]
样例:
输入:      输出:
11
I cat       
Q cat       1
Q ca        0
I her
Q her       1
I him
Q him       1
I no 
Q no        1
I nova
Q nova      1
\end{lstlisting}

开一个数组 \verb|sno[N][26]| 来表示每个点的所有儿子,第一维存储结点大小,第二维存储每个字母,因为字符串仅包含小写英文字母,所以第二维只开 $26$ 的大小就可以了.\verb|cnt[N]| 表示以当前字母为结点的字母有多少个(标记)\verb|idx| 的含义和链表里的 \verb|idx| 含义一样.

\begin{lstlisting}[language=cpp]

// 插入(存储)字符串
void insert(string str)
{
    int p = 0;  // 类似指针,从根结点开始
    for (int i = 0; str[i]; i ++ )
    {
        int u = str[i] - 'a';  // 'a' ~ 'z' 映射成 0 ~ 25
        if (!son[p][u]) son[p][u] = ++ idx;  // 该节点不存在,创建节点
        p = son[p][u];     // 使 p 指向(走到)子结点
    }
    
    cnt[p] ++ ;  // 以 p 结点为结尾的单词数量 ++ 
}
\end{lstlisting}

\begin{lstlisting}[language=cpp]

// 查询字符串出现的次数
int find(string str)
{
    int p = 0;
    for (int i = 0; str[i]; i ++ )
    {
        int u = str[i] - 'a';
        if (!son[p][u]) return 0; // 没有的话就返回 0,表示这个字符串在集合中没有出现过
        p = son[p][u];            // 有的话走向子结点
    }

    return cnt[p];   // 返回以 p 结点为结尾的单词数量
}
\end{lstlisting}