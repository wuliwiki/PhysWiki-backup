% 洛伦兹变换
% keys 光速不变|洛伦兹变换|狭义相对论|四维时空

% 未完成: 应该把 “洛伦兹变换” 中的相关内容放到 “狭义相对论” 中

\pentry{光速不变原理\upref{SpeRel}, 伽利略变换\upref{GaliTr}} % 未完成

\subsection{狭义相对论的基本假设}
\begin{enumerate}
\item 相对性原理——任何惯性系中, 物理定律及相同实验的结果都相同
\item 光速不变原理——任何惯性系中, 真空中的光速都不改变
\end{enumerate}

\subsection{狭义相对论的时空}

我们把在一个惯性系中某位置的坐标 $(x, y, z)$ 与某个时间 $t$ 组成一个\textbf{四维矢量}(或坐标) $(x, y, z, t)$ 叫做\textbf{事件}, “事件” 和 “质点”, “点电荷” 一样, 都属于一种模型. 我们忽略事件的空间和时间长度, 假设它精确地在某时刻某坐标点发生. 注意一个事件的四维坐标取决于观察者的参考系. 在伽利略变换% 链接未完成
中, 事件的时间与参考系无关, 而所有时间的位置只需要经过同一个平移就可以变换成另一个参考系中的坐标.

在狭义相对论中, 我们只讨论惯性系, 另外不讨论引力或只讨论引力较弱的情况. 我们仍然假设每个惯性系中的空间都是欧几里得三维空间, 即可以用三维直角坐标系以及高中所学的欧几里得立体几何描述. 但不同的是, 同一个事件在不同惯性系中可能具有不同的时间, 事件在不同惯性系中的位置也不能用简单的平移变换来对应, 而是和时间一起进行更复杂的线性变换\upref{LTrans}, 即下面要介绍的洛伦兹变换.

\subsection{洛伦兹变换}
为方便讨论, 设 $t = 0$ 时两参考系 $S$ 与 $S'$ 的原点重合, $x, y, z$ 轴方向相同. $S'$ 系相对 $S$ 系沿 $x$ 轴方向以速度 $v$ 匀速运动. $S$ 系中的坐标 $(x, y, z, t)$ 一一对应到 $S'$ 系中的坐标 $(x', y', z', t')$.

洛伦兹变换的另一个假设是, 事件的四个坐标在不同惯性系中的变换一定是线性变换, 因为时间和空间是均匀的, 这也叫物理定律的平移对称及时间对称.

可以证明, 满足上述要求的线性变换为
\begin{equation}\label{LornzT_eq1}
\leftgroup{
&x' = \frac{x - vt}{\sqrt{1 - v^2/c^2}}\\
&y'= y\\
&z' = z\\
&t' = \frac{t - vx/c^2}{\sqrt{1 - v^2/c^2}}
}
\qquad
\leftgroup{
&x = \frac{x' + vt'}{\sqrt{1 - v^2/c^2}}\\
&y = y'\\
&z = z'\\
&t = \frac{t' + vx'/c^2}{\sqrt{1 - v^2/c^2}}
}
\end{equation}
该变换被称为\textbf{洛伦兹变换}.

\subsection{推导}
由于两个参考系是完全对称的, 根据相对性原理, 它们之间的坐标变换应该具有完全相同的形式. 但\autoref{LornzT_eq1} 中的变换和逆变换为何有个别符号上的差异呢? 因为习惯上的坐标系的定义使得 $S'$ 相对 $S$ 的速度是正的, 而 $S$ 相对于 $S'$ 速度却是负的. 为了保持绝对的镜像对称我们只需把 $S'$ 的 $x$ 轴取反方向.

先考虑 $(x, t)$ 两个坐标, 令线性变换为
\begin{equation}
\leftgroup{
x' &= ax + bt\\
t' &= mx + nt
}
\end{equation}
解得逆变换为
\begin{equation}
\leftgroup{
x &= \frac{nx' - bt'}{an - bm}\\
t &= \frac{mx' - at'}{bm - an}
}
\end{equation}
由相对性原理, 正变换和逆变换的系数必须完全相同, 对比系数并化简得
\begin{equation}\label{LornzT_eq2}
\leftgroup{
&a = -n\\
&bm - an = 1
}
\end{equation}

为了书写方便, 我们使用无量纲的物理公式\upref{NoUnit}. 令长度单位为真空中的光在单位时间走过的路程, 则速度单位为光速.

由光速不变原理, 当某点延 $x$ 轴以光速运动, 即 $\dv*{x}{t} = 1$ 时, 它在另一个参考系中的速度必须也是光速, 即 $\dv*{x'}{t'} = -1$ (注意两个参考系中 $x$ 轴方向相反). 所以
\begin{equation}\label{LornzT_eq3}
\dv{x'}{t'} = \frac{a\dd{x} + b\dd{t}}{m\dd{x} + n\dd{t}} = \frac{a + b}{m + n} = -1
\end{equation}

根据定义, $S$ 系中的任意不动点($\dv*{x}{t} = 0$)在 $S'$ 系中速度为 $v$, 所以
\begin{equation}\label{LornzT_eq4}
\dv{x'}{t'} = \frac{a\dd{x} + b\dd{t}}{m\dd{x} + n\dd{t}} = \frac{b}{n} = v
\end{equation}

根据两坐标系定义, 我们要求 $t$ 不变时, $x$ 增加 $x'$ 减少, 所以
\begin{equation}\label{LornzT_eq5}
a < 0
\end{equation}

联立\autoref{LornzT_eq2} 到\autoref{LornzT_eq5} 可解得所有系数
\begin{equation}
a = - \frac{1}{\sqrt{1-v^2}} \qquad
b = \frac{v}{\sqrt{1-v^2}} \qquad
m = - \frac{v}{\sqrt{1-v^2}} \qquad
n = \frac{1}{\sqrt{1-v^2}}
\end{equation}

所以变换和逆变换为
\begin{equation}
\leftgroup{
&x' = - \frac{x + vt}{\sqrt{1 - v^2}}\\
&t' = \frac{t + vx}{\sqrt{1 - v^2}}
}
\qquad
\leftgroup{
&x = -\frac{x' + vt'}{\sqrt{1 - v^2}}\\
&t = \frac{t' + vx'}{\sqrt{1 - v^2}}
}
\end{equation}
若按照一般的坐标系习惯, 把 $x'$ 改成 $-x'$, 就与\autoref{LornzT_eq1} 一致了\footnote{要从该式得到国际单位的公式, 只需把 $x, x', v, v'$ 都替换为国际单位的物理量除以转换常数 $c$ 即可}.

下面再来考虑其他维度, 设
\begin{equation}
\leftgroup{
&x' = \frac{x - vt}{\sqrt{1 - v^2}} + ay + bz\\
&t' = \frac{t - vx}{\sqrt{1 - v^2}} + cy + ez
}
\end{equation}
求逆变换,对比系数,得出 $a, b, c, e$ 为零. 同理可得
\begin{equation}
\leftgroup{
y' &= y\\
z' &= z
}
\end{equation}
至此我们就完全推导出了洛伦兹变换.
