% 爱森斯坦判别式
% keys 多项式|Eisenstein|不可约多项式
\pentry{素理想与极大理想\upref{Ideals}}

爱森斯坦判别式可以用来判断多项式是否可约.满足爱森斯坦判别式条件的多项式必然是不可约的,即无法表示为两个多项式的乘积.

最简单的多项式是整系数多项式,也被称为\textbf{整数环上的}多项式,因此我们先介绍此类多项式上的爱森斯坦判别式.这只是更一般的爱森斯坦判别式的特例.

\begin{theorem}{整数环上的爱森斯坦判别式}
设有整系数多项式$f(x)=f_0+f_1x+f_2x^2+\cdots+f_nx^n$,若存在素数$p$使得:
\begin{itemize}
\item $p\not|f_n$;
\item 对于$i\in\{0, 1, 2, \cdots, n-1\}$,都有$p|f_i$;
\item $p^2|f_0$.
\end{itemize}
那么$f(x)$就是不可约多项式.
\end{theorem}

\textbf{证明}:

反设$f(x)$可约,则可以写成$f(x)=(h_0+h_1x+h_2x^2+\cdots+h_rx^r)(g_0+g_1x+g_2x^2+\cdots+g_sx^s)$.

由于$f(x)=f_0+f_1x+f_2x^2+\cdots+f_nx^n$,故$f_i=\sum\limits_{j+k=i}h_jg_k$

\textbf{证毕}.







