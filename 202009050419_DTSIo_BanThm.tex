% 巴拿赫定理
\pentry{范数\upref{NormV}巴拿赫空间\upref{banach}}

在巴拿赫空间的理论中, 巴拿赫定理 (Banach theorems) 是一组关于巴拿赫空间上的有界线性算子的定理. 它们都依赖于巴拿赫空间的完备性, 并且在分析数学中有广泛的应用. 可参考 Kôsaku Yosida: Functional analysis. Grundlehren der mathematischen Wissenschaften 123, Springer-Verlag, 1980 (6th ed.).

\subsection{一致有界原理 (共鸣定理)}
\begin{theorem}{一致有界原理 (共鸣定理)}
设$X$是巴拿赫空间, $Y$是赋范线性空间, $\mathfrak{F}\subset\mathfrak{B}(X,Y)$是有界线性算子的族. 如果对于任何$x\in X$皆有$\sup_{T\in\mathfrak{F}}\|Tx\|_Y<\infty$, 那么实际上必有
$$
\sup_{T\in\mathfrak{F}}\|T\|_{\mathfrak{B}(X,Y)}
=\sup_{T\in\mathfrak{F}}\left(\sup_{\|x\|_X\leq 1}\|Tx\|_Y\right)<\infty.
$$
也就是说, 从巴拿赫空间出发的线性算子的族, 如果是逐点有界的, 则一定是一致有界的. 
\end{theorem}
这个定理的证明用到了贝尔纲定理 (Baire category theorem), 而这依赖于$X$的完备性. 如果$X$不完备, 则定理不成立. 一个反例如下: 取$X$为仅有有限项非零的序列组成的空间, 并赋以范数$\|x\|=\max_{k}|x_k|$. 这个空间不是完备的. 定义$T_n:X\to X$为
$$
T_nx=(x_1,2x_2,...,nx_n,0,...),
$$
则$\|T_n\|=n$, 因此族$\{T_n\}$不是一致有界的, 但对于任何固定的$x\in X$, 当$n$充分大时, $T_nx$将恒等于常值, 因此对于任何固定的$x\in X$, 都有$\sup_n\|T_nx\|<\infty$.

\subsection{开映像原理}

\subsection{闭图像定理}