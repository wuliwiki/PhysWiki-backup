% 范数

\pentry{矢量空间\upref{LSpace}}

一些矢量空间中, 我们可以给每个矢量都定义一个\textbf{范数}, 例如常见的 $N$ 维 “几何矢量\upref{GVec}” 空间的模长就是该空间的一种范数. 同一个矢量空间可以存在多种不同的范数. 如果一个矢量空间中定义了范数, 我们就把它称为\textbf{赋范空间}. 范数必须满足以下条件

\begin{enumerate}
\item 
\end{enumerate}

\subsection{列矢量的范数}
定义矢量的 $N$ 范数为
\begin{equation}
\norm{\bvec x}_n = \qty(\sum_i \abs{x_i}^n)^{1/n}
\end{equation}
最常见的是几何矢量的坐标的 2-范数.


可以证明极限情况 $n \to \infty$ 时, 绝对值最大的 $x_i$ 对求和的贡献将远大于其他分量, 所以定义\textbf{无穷范数}为
\begin{equation}
\norm{\bvec x}_\infty = \max \qty{\abs{x_i}}
\end{equation}

\subsection{函数的范数}
多元函数 $f(x_1, \dots x_N)$ 的范数在物理中常定义为
\begin{equation}
\norm{f} = \int \abs{f(x_1, \dots, x_N)}^2 \dd{x_1}\dots \dd{x_N}
\end{equation}
另一种简单定义是使用函数的最大值
\begin{equation}
\norm{f} = \max\qty{\abs{f(x_1, \dots, x_N)}}
\end{equation}
