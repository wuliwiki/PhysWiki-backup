% 威廉·爱德华·韦伯(综述)
% license CCBYSA3
% type Wiki

本文根据 CC-BY-SA 协议转载翻译自维基百科\href{https://en.wikipedia.org/wiki/Wilhelm_Eduard_Weber}{相关文章}。

\begin{figure}[ht]
\centering
\includegraphics[width=6cm]{./figures/89cd851af1488861.png}
\caption{戈特利布·比尔曼(Gottlieb Biermann)所绘的韦伯(Weber)肖像。} \label{fig_Eduard_1}
\end{figure}
威廉·爱德华·韦伯(Wilhelm Eduard Weber,/ˈveɪbər/[1],德语发音:[ˈvɪlhɛlm ˈeːdu̯aʁt ˈveːbɐ];1804年10月24日—1891年6月23日),德国物理学家,与卡尔·弗里德里希·高斯共同发明了世界上第一台电磁电报。
\subsection{传记}  
\subsubsection{早年经历 } 
韦伯出生在维滕贝格的 Schlossstrasse(城堡街),他的父亲 迈克尔·韦伯(Michael Weber)是该地的神学教授。这栋建筑曾是 亚伯拉罕·瓦特(Abraham Vater)曾经的住所。[2]  

威廉是家中三兄弟中的老二,三兄弟都以出色的科学天赋闻名。1815年,随着维滕贝格大学的解散,他的父亲被调往哈雷。威廉最初由父亲教授基础课程,随后进入哈雷的孤儿院与文法学校(Orphan Asylum and Grammar School)学习。之后,他进入大学,专攻自然哲学。他在课堂上表现突出,并通过原创研究崭露头角。取得博士学位并成为 私人讲师(Privatdozent)后,他被任命为哈雷大学的 特别自然哲学教授(Professor Extraordinary of Natural Philosophy)。
