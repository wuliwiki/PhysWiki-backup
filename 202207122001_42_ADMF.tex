% ADM形式
% ADM形式|时空3+1分解

\begin{issues}
\issueTODO
\end{issues}

将4维时空分解成1维的时间和3维的空间称之为时空的\textbf{3+1分解},用数学的语言来描述就是:若 $W$ 是4维时空(可看成4维矢量空间\upref{LSpace}),$T$ 是1维的时间(1维矢量空间),$V$ 是3维空间 (3维矢量空间),那么$V=T\oplus V$ (\autoref{DirSum_def1}~\upref{DirSum}) 就是时空 $W$ 的 $3+1$ 分解.在一般的框架下对时空进行 $3+1$ 的分解称之为\textbf{ADM 形式}(\textbf{ADM formalism}).ADM 形式来源于Arnowitt, Deser and Misner 1962年的工作,并以三人的首字母命名.
\subsection{时空3+1分解}
考虑4维时空中的空间超曲面 $\bvec X$( $n$ 维空间的超曲面指其 $n-1$ 维的子空间,所以空间超曲面就是我们所处的3维空间,用黑体字母表示暗示着它的每一元素都可用一3维的空间矢量表示),其由3个坐标 $x^i,i=1,2,3$ 所定义,即 
\begin{equation}
\bvec X=\bvec X(x^i)
\end{equation}
在超曲面中的任一点处,都有一基底 $\{\bvec e_i\}$ 与之对应.其中
\begin{equation}
\bvec e_i=\partial_i \bvec X=\pdv{\bvec X}{x^i}
\end{equation}
$\bvec e_i$ 方向的坐标轴当然只有对应的坐标 $x^i$ 变化,由偏导数的定义\upref{ParDer},上式的几何图像很直白.

而垂直于超曲面的单位法矢量 $n$ 满足
\begin{equation}\label{ADMF_eq1}
g_{\mu\nu} e_i^\mu n^\nu=0,\quad g_{\mu\nu}n^\mu n^\nu=-1
\end{equation}
其中 $e_i^\mu,n^\nu$ 分别是四矢量 $\bvec e_i,n$ 对应 $\mu,\nu$ 坐标轴的坐标分量,而 $g_{\mu\nu}$ 为时空的度规(亦即度量,时空的度量显然是闵可夫斯基度量(\autoref{EFSp_def1}~\upref{EFSp})).\autoref{ADMF_eq1} 中,第一式表明 $n$ 垂直于超曲面,第二式表明 $n$ 是单位矢.

现在,以一连续的方式定义超曲面,从而获得一族超曲面 
\begin{equation}
\{\bvec{X}=\bvec X(x^i,t)|t\in\mathbb R\}
\end{equation}

