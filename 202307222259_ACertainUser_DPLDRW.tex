% 多普勒与音爆(Matlab 绘图)

我们知道多普勒效应\upref{Dopler}:当声源在运动时,观察者会发现声源的频率改变了,并且频率的改变与观察者的位置、声源的速度等有关。

以下的octave/matlab程序简要地展示了多普勒效应:每一圈代表一个波峰,两圈之间的间隙相当于半波长。可见,在波源前方,波长被压缩、频率升高;而在后方波长延长、频率降低。

同时,如果波源的速度超过波速(例如超声速,至少目前我们似乎还不能超光速),那么波甚至追不上波源,并非在波源前方产生一个锥形区域。锥形的角度与波源速度有关。

%其实我不确定这个程序有没有各种意义上的bug...