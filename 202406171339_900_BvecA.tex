% 磁矢势
% keys 旋度|磁场|矢势
% license Xiao
% type Tutor

\begin{issues}
\issueDraft
\issueOther{已知电流如何求磁矢势? 见毕奥—萨伐尔定律的旋度形式}
\end{issues}

\pentry{亥姆霍兹分解\nref{nod_HelmTh}}{nod_b477}

由于磁场 $\bvec B(\bvec r)$ 任何情况都是一个\enref{无源场}{MagGau},所以根据 “旋度的逆运算\upref{HlmPr2}” 的\autoref{the_HlmPr2_1}, 必定存在一个矢量场 $\bvec A(\bvec r)$ 使得
\begin{equation}\label{eq_BvecA_2}
\curl \bvec A = \bvec B~.
\end{equation}
且 $\bvec A$ 可以通过下式计算
\begin{equation}\label{eq_BvecA_1}
\bvec A(\bvec r) = \frac{1}{4\pi} \int \bvec B(\bvec r') \cross \frac{\bvec R}{R^3} \dd{V'} + \bvec H(\bvec r)~.
\end{equation}
其中 $\bvec r, \bvec r'$ 分别是坐标原点指向三维直角坐标 $(x, y, z)$ 和 $(x', y', z')$ 的位置矢量, $\bvec R = \bvec r' - \bvec r$, $R = \abs{\bvec R}$, 体积分 $\int\dd{V'} = \int\dd{x'}\dd{y'}\dd{z'}$ 的区域是空间中 $\bvec B$ 不为零的区域, $\cross$ 表示矢量\enref{叉乘}{Cross}, $\bvec H(\bvec r)$ 是一个任意无旋场。

若已知恒定电流分布如何求空间某点的磁矢势呢? 当然我们可以先用\enref{毕奥—萨伐尔定律}{BioSav}求出磁场分布再用\autoref{eq_BvecA_1} 求出磁矢势, 但也而已直接求出, 使用毕奥—萨伐尔定律的旋度形式(\autoref{eq_BioSav_5})
\begin{equation}
\bvec B(\bvec r) = \frac{\mu_0}{4\pi} \curl \int \frac{\bvec j(\bvec r')}{\abs{\bvec r - \bvec r'}}\dd{V'}~,
\end{equation}
对比\autoref{eq_BvecA_2} 可得
\begin{equation}\label{eq_BvecA_3}
\bvec A(\bvec r) = \frac{\mu_0}{4\pi} \int \frac{\bvec j(\bvec r')}{\abs{\bvec r - \bvec r'}}\dd{V'}~.
\end{equation}
注意 $\bvec A$ 不是唯一的, 给它加上任意无旋场同样使\autoref{eq_BvecA_2} 成立。 两个旋度相同的场只可能相差一个无旋场。 无旋场也可以记为任意函数的梯度 $\grad \varphi$。

可以证明静电学条件下\autoref{eq_BvecA_3} 右边第一项是一个无散场, 对第一项的积分求关于 $\bvec r$ 的梯度得
\begin{equation}
\begin{aligned}
&\div \int \frac{\bvec j(\bvec r')}{\abs{\bvec r - \bvec r'}}\dd{V'} = 
\int \qty(\grad\frac{1}{\abs{\bvec r - \bvec r'}}) \bvec j(\bvec r') \dd{V'}\\
&= -\int \qty(\grad' \frac{1}{\abs{\bvec r - \bvec r'}}) \bvec j(\bvec r') \dd{V'}~,
\end{aligned}
\end{equation}
其中 $\grad'$ 代表求关于 $\bvec r'$ 的梯度。 接下来使用多维分部积分\autoref{eq_IntBP2_1}, 令 $f(\bvec r') = 1/\abs{\bvec r - \bvec r'}$, $\bvec A(\bvec r') = \bvec j(\bvec r')$。 面积分取无穷大的球面, 积分为零; 最后一项中由于 $\grad' \vdot \bvec j(\bvec r') = 0$, 积分同样为零。 证毕。

\subsection{规范}
(详见 “\enref{规范变换}{Gauge}”)由于 $\bvec A(\bvec r)$ 不止一种, 我们有时候需要某种\textbf{规范(gauge)}来将其唯一确定下来。 例如在\textbf{库仑规范(Coulomb Gauge)}中, 我们要求
\begin{equation}
\div \bvec A = \bvec 0~.
\end{equation}
根据\autoref{eq_HlmPr2_4}, 我们只需要令 $\bvec H(\bvec r)$ 是一个调和场即可, 事实上库仑规范直接规定 $\bvec H(\bvec r) = 0$。
