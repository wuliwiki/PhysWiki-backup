% 光与物质粒子的统一(相对论点粒子的作用量)
% keys 相对论|非相对论|光|作用量
% license Usr
% type Tutor

本节将从作用量的视角介绍牛顿力学到相对论的自然过渡,最终给出适用于物质粒子和光的作用量。阅读本节需带着“民主”的思想:时间和空间应当被平等对待。
\subsection{Newton力学到狭义相对论}
首先看Newton力学中物质粒子的Euler-Lagrange作用量:
\begin{equation}
S=\int\dd t\qty[\frac{1}{2}m\qty(\dv{\vec x}{t})^2-V(x)]~.
\end{equation}
这一作用量是相当“笨重”的。在这里,让我们只考虑自由粒子,把势 $V(x)$ 给丢掉。那么上式可写为
\begin{equation}
S=\int\dd t\frac{1}{2}m\qty(\dv{\vec x}{t})^2=\int\dd t\frac{1}{2}m\frac{({\dd }\vec x)^2}{\dd t}~.
\end{equation}
