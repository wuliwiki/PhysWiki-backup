% 速度的坐标系变换(无转动)
% keys 速度|坐标系|参考系|速度叠加|相对速度

\pentry{速度\upref{VnA}}

这种情况就是高中物理中所谓的 “绝对速度 = 相对速度 + 牵连速度”: 若某时刻两个坐标系 $S$ 和 $S'$ 之间无相对转动, 那么某时刻两坐标系之间的相对速度是唯一确定的, 即 $S'$ 系中任意一个固定的点相对于 $S$ 系的速度矢量都是相同的。 注意我们不要求 $S$ 系和 $S'$ 系的单位矢量 $\uvec x, \uvec y, \uvec z$ 和 $\uvec x', \uvec y', \uvec z'$ 同方向, 只要求它们的相对方向不随时间变化。

把 $S'$ 系相对于 $S$ 系的速度记为 $\bvec v_r$ 根据高中所学的速度叠加原理, 若某点 $P$ 相对于 $S$ 系的速度瞬时为 $\bvec v$, 相对于 $S'$ 的瞬时速度为 $\bvec v'$ 我们有
\begin{equation}\label{Vtrans_eq1}
\bvec v = \bvec v' + \bvec v_r~.
\end{equation}
其中三个矢量都可以是时间的函数。 注意该式与点 $P$ 的位置无关只和速度有关。

\begin{figure}[ht]
\centering
\includegraphics[width=8cm]{./figures/Vtrans_1.pdf}
\caption{一个简单的例子:在相对于地面运动的火车上看,飞机的速度似乎更慢一些。\textsl{假设火车和飞机的速度远低于光速!}} \label{Vtrans_fig1}
\end{figure}

\begin{figure}[ht]
\centering
\includegraphics[width=8cm]{./figures/Vtrans_2.pdf}
\caption{另一个简单的例子:在相对于地面下坡运动的火车上看,飞机似乎在抬升。} \label{Vtrans_fig2}
\end{figure}

注意\autoref{Vtrans_eq1} 中的矢量都是\textbf{几何矢量}\upref{GVec}, 不能将 $\bvec v_S$ 和 $\bvec v_S'$ 等同于点 $P$ 在 $S$ 系和 $S'$ 系中的三个直角坐标的求导。 如果要将\autoref{Vtrans_eq1} 写成坐标矢量的形式, 三个矢量必须使用同一坐标系(见\autoref{Vtrans_ex2})。

本书中, 粗体正体的字母既可以用于表示几何矢量本身, 又可以表示其坐标\footnote{小时百科符号与规范\upref{Conven}}。 对于后者, 我们可以声明坐标所使用的坐标系
\begin{equation}\label{Vtrans_eq5}
(\bvec v)_{S} = (\bvec v')_{S} + (\bvec v_r)_{S}~,
\end{equation}
\begin{equation}\label{Vtrans_eq6}
(\bvec v)_{S'} = (\bvec v')_{S'} + (\bvec v_r)_{S'}~.
\end{equation}

\begin{example}{}\label{Vtrans_ex2}
令 $S$ 系和 $S'$ 系中的三个单位矢量分别为 $\uvec x, \uvec y, \uvec z$ 和 $\uvec x', \uvec y', \uvec z'$。 他们的关系以及相对速度为
\begin{equation}
\uvec y' = \uvec x~, \quad
\uvec z' = \uvec y~, \quad
\uvec x' = \uvec z~, \quad
\bvec v_r = 2\uvec x = 2\uvec y'~.
\end{equation}
点 $P$ 在 $S$ 系中坐标关于时间的导数为 $(\bvec v)_S = (1, 2, 3)$。 请将\autoref{Vtrans_eq1} 表示为三个分量的形式。

\textbf{解:} 容易得出, 点 $P$ 在 $S'$ 系中的坐标关于时间的导数为 $(\bvec v')_{S'} = (3, -1, 2)$。

我们先来看错误的理解: 将\autoref{Vtrans_eq1} 中 $\bvec v$ 等同于 $(1, 2, 3)$, $\bvec v'$ 等同于 $(3, -1, 2)$,  这时会发现, 无论 $\bvec v_r$ 取 $(\bvec v_r)_S = (2, 0, 0)$ 还是 $(\bvec v_r)_{S'} = (0, 2, 0)$ 都不能让\autoref{Vtrans_eq1} 成立。

正确的做法是将三个矢量都放到同一坐标系中。 可以验证\autoref{Vtrans_eq5} 和\autoref{Vtrans_eq6} 都成立。

对于更一般的情况, 两参考系中矢量的坐标变换需要使用三维旋转矩阵\upref{Rot3D} 。
\end{example}
