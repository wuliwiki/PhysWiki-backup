% 导数(数学分析)
% derivative|微分|differential|斜率|切线|导函数|函数|function|微积分|calculus|数学分析

\pentry{极限\upref{Lim}}



切线的几何描述并不严谨和完整,初学者可能会有很多疑问,比如割线的两个点靠近切点的时候,如果其中一个点到切点的距离始终是另一个点到切点距离的两倍,那它们还算同时达到切点吗?

实际上,在\textbf{极限}\upref{Lim}中我们强调过,极限的概念并不是简单地等同,而是趋近的性质.用割线去接近切线也是一种极限,不存在“达到切点”的情况,只能是越来越接近切点.

和其它类型的极限一样,割线趋近的过程强调“任意性”.为了方便描述,我们需要把一根切线表示为数字.最直接的办法就是用切线的斜率来表示.

\begin{definition}{斜率}
对于用$f(x)=ax+b$描述的直线,定义其斜率为$a$.
\end{definition}

由于我们给定了切点,因此割线接近切线的过程中,肯定也会接近切点,相当于确定了割线的极限是要经过切点的,于是我们只需要考察割线的斜率是不是收敛就行.由于斜率是一个数字,这就使得我们又获得了一个数列,而我们对于数列的极限是很熟悉的了.

\begin{definition}{导数}\label{Der2_def1}
考虑实函数$f(x)$,给定实数$x_0$.对于\textbf{一对}数列$\{a_n\}$、$\{b_n\}$,令$\lim\limits_{n\to\infty}a_n=x_0$、$\lim\limits_{n\to\infty}b_n=x_0$,且各$a_n\not=b_n$\footnote{这一条是保证总能画出割线,因为两点确定一条直线嘛,$a_n=b_n$的话这条割线就画不出来了.}.这样,每一对点$(a_n, f(a_n))$、$ (b_n, f(b_n))$都能唯一确定一条割线,也就确定了割线的斜率$d_n$.

对于\textbf{任意}的上述数列对$\{a_n\}$、$\{b_n\}$,如果其生成的$d_n$都收敛到同一个实数$A$上,那么该实数$A$就是$f(x)$在$x_0$处的\textbf{切线斜率},也称\textbf{导数(derivative)}.
\end{definition}

我们来看几个例子,加深理解.

\begin{example}{}
考虑函数$f(x)=2x+1$.容易验证,无论怎么选一对数列,它们生成的割线斜率都是$2$.因此由定义,$f(x)$在任何一个点处的导数值都是$2$.
\end{example}

\begin{example}{}\label{Der2_ex1}
考虑函数$f(x)=x^2$,尝试计算其在$x=2$处的导数值\footnote{$x=2$处即点$(2, 4)$.也可以说$y=4$且$x>0$处,但肯定是$x=2$处的说法更方便.}.

取数列$a_n=2+g_n$和$b_n=2+h_n$,其中$g_n$和$h_n$都趋近于$0$,且各$g_n\not=h_n$.那么$a_n$和$b_n$所确定的割线就通过点$(2+g_n, 4+4g_n+g_n^2)$和点$(2+h_n, 4+4h_n+h_n^2)$,因此割线斜率是
\begin{equation}
\begin{aligned}
d_n&=\frac{4+4g_n+g_n^2-4-4h_n-h_n^2}{g_n-h_n}\\
&=\frac{4(g_n-h_n)+(g_n+h_n)(g_n-h_n)}{g_n-h_n}\\
&=4+g_n+h_n
\end{aligned}
\end{equation}
由于随着$n$的增大,$g_n$和$h_n$都趋于零,因此$d_n$趋于$4$.

又由于我们没有具体约束$g_n$和$h_n$,使得上述讨论适用于一切可用于构造趋近于给定点割线的数列$a_n$和$b_n$,因此满足\textbf{任意}性,因此$f(x)=x^2$在$x=2$处的导数值就是$4$.
\end{example}

\begin{exercise}{}
考虑函数$f(x)=\abs{x}$,画出这个函数的图像,并证明其在$x=0$处没有导数.
\end{exercise}

\begin{exercise}{}\label{Der2_exe1}
考虑函数$f(x)=[x]$,定义为$[x]$是小于等于$x$的、最接近$x$的整数.比如,$[\pi]=3$,$[e]=2$,$[4.99]=4$, $[-4.99]=-5$.

画出这个函数的图像,并证明其在横坐标为整数的点处没有导数.
\end{exercise}








\subsection{导函数及其计算}

我们上面讨论的是对于函数$f(x)$,在给定点求其导数.但是如果我们把所有点的导数都求出来(“不存在导数”也是一种结果),那么我们就得到了一个新的函数,就叫做$f(x)$的\textbf{导函数},记为$f'(x)$.

如果我们能把导函数的表达式计算出来,那么就可以直接代值去计算各处的导数,没必要挨个像\autoref{Der2_ex1} 那样进行一番冗长的运算了.

% 为了实现这个目的,我们要先研究一番几个基本的函数.

% \begin{example}{多项式函数的导数}
% 多项式是形如$f(x)=a_0+a_1x+a_2x^2+\cdots+a_nx^n$的函数.为了计算其导函数,我们首先要考虑最简单的多项式,$x^n$.

% 对于任意实数$x_0$,取数列$a_n=x_0+g_n$和$b_n=x_0+h_n$,其中$g_n$和$h_n$都趋近于$0$,且各$g_n\not=h_n$.那么$a_n$和$b_n$分别确定点$(x_0+g_n, (x_0+g_n)^n)$和$(x_0+h_n, ((x_0+h_n)^n)$,割线斜率也就是$d_n=\frac{(x_0+g_n)^n-(x_0+h_n)^n}{g_n-h_n}$.

% 应用二项式定理$(a+b)^n=a^n+b^n+$
% \end{example}

为了方便计算,我们要先把导数的定义简化成如下形式:

\begin{definition}{导数}\label{Der2_def2}
考虑实函数$f(x)$,给定实数$x_0$.如果对于任意趋近于$x_0$的数列$x_n$,都有数列$\frac{f(x_n)-f(x_0)}{x_n-x_0}$收敛,且所有这样的数列都收敛于实数$A$.那么$A$就是$f(x)$在$x_0$处的\textbf{导数}.

换句话说,$f(x)$在$x_0$处的导数是$\lim\limits_{h\to 0}\frac{f(x_0+h)-f(x_0)}{h}$.
\end{definition}

\begin{exercise}{两个定义的等价性}
证明\autoref{Der2_def1} 和\autoref{Der2_def2} 是等价的.

提示:\autoref{Der2_def1} 显然是蕴含了\autoref{Der2_def2} 的,因为前者描述的是“所有数列对”的情况,后者则是限定了其中一个数列恒为$x_0$的特殊情况.所以你只需要证明\autoref{Der2_def2} 蕴含了\autoref{Der2_def1} .提示:单独用$a_n$和$b_n$套进\autoref{Der2_def2} 中$x_n$的位置能得到两条割线,而$a_n$和$b_n$在\autoref{Der2_def1} 中又构成第三条割线.这三条割线是可以构成一个三角形的.根据\autoref{Der2_def2} ,前两条割线的斜率都趋于$A$,那么作为三角形第三边的第三条割线,其斜率也不得不跟着趋于$A$,这样就从\autoref{Der2_def2} 推出\autoref{Der2_def1} 了.
\end{exercise}


\autoref{Der2_def2} 的表述可以理解为,我们在$(x_0, f(x_0))$处出发,改变自变量的取值(正向负向的改变均可),算出由此带来的函数值改变,再计算二者的商.这个改变越接近$0$,计算出来的商就越接近导数值$f'(x_0)$.

\begin{definition}{导函数}
给定实函数$f(x)$.如果另一个函数$f'(x)$的取值范围是$f(x)$的全体可求导数的点,且在这些点上$f'$的值就是$f$的导数值,那么我们称$f'(x)$是$f(x)$的导函数.
\end{definition}

\subsubsection{Leibniz表示法}


我们还可以用Leibniz的符号,将$f(x)$的导函数$\lim\limits_{x\to x_0}\frac{f(x)-f(x_0)}{x-x_0}$表示成$\frac{\dd f(x)}{\dd x}$.这里,$\mathrm{d}$不是一个量,而是一个符号,表示“取变化量,然后让这个变化量趋于零但不等于零”.$\dd x$的含义就是$x-x_0$,然后取一个极限过程$\lim\limits_{x-x_0\to 0}$\footnote{取极限过程就是指,取一系列这样的$x$构成一个数列$\{x-x_0\}$,且这个数列趋于$0$.$\lim\limits_{x-x_0\to 0}$等价于$\lim\limits_{x\to x_0}$.}.

要注意的是,$\dd f(x_0)$是取决于$\dd x$的,即$\dd f(x_0)=f(x_0+\dd x)-f(x_0)$.也就是说,只有$x$是自由变化的,而$f(x)$变化的量是被动的.所以$\dd f(x)$可能不会跟着$\dd x$一起趋于零.一个例子就是\autoref{Der2_exe1} 中的符号函数,在$x=0$的地方,$\dd f(x)$的取值只有$\pm 1$两种(别忘了$\dd x$不等于零),导致$\dd x$趋于零的过程中,$\frac{\dd f(x)}{\dd x}$越来越趋近于无穷.


由于$\dd x$表示的是\textbf{任意}接近$0$的数列,因此,如果两个自变量彼此独立、不相互影响,比如$x$和$y$,那么$\dd x$和$\dd y$的关系是不确定的,$\frac{\dd y}{\dd x}$也是无法定义的值.但是如果两个变量之间有关系,比如$y=f(x)$,那么$\frac{\dd y}{\dd x}$就是有意义的了.

举个例子.单给出无关的$x$和$y$,那么可以让$\dd x$对应$x_n=1/2^n$,$\dd y$对应$y_n=1/3^n$,这样$\frac{\dd y}{\dd x}=\lim\limits_{n\to\infty}(\frac{2}{3})^n=0$.但是我们也可以让$\dd x$对应$x_n=1/2^n$,$\dd y$对应$y_n=1/2^n$,这样算出来的$\frac{\dd y}{\dd x}$就是$1$.到底哪个才是真正的$\frac{\dd y}{\dd x}$呢?这没法讨论了.但是呢,如果限定了二者之间的关系,比如$y=x^2$,那么$\dd y$对应的$y_n$就不能随意取了,必须用$\dd x$对应的$x_n$结合$y=x^2$计算出来,而且无论怎么取$x_n$,$\frac{\dd y}{\dd x}$在$x=x_0$处的结果都会是$2x_0$.

为了方便,我们也可以把$\frac{\dd f(x)}{\dd x}$表示为$\frac{\dd}{\dd x}f(x)$.这里,我们将$\frac{\dd}{\dd x}$视为一个操作,它把$f(x)$变成其导函数$f'(x)$.更进一步,我们还可以写出$f'=\frac{\dd}{\dd x}f$.


\subsubsection{两个重要的性质}

\begin{theorem}{导数的线性性}
假设$f(x)$和$g(x)$是两个\textbf{处处可导}\footnote{即每一个点上都有导数.}的函数,那么$\frac{\dd}{\dd x}(f(x)+g(x))=\frac{\dd}{\dd x}f(x)+\frac{\dd}{\dd x}g(x)$.
\end{theorem}

证明很简单,留作习题.

\begin{theorem}{Leibniz律}\label{Der2_the1}
假设$f(x)$和$g(x)$是两个\textbf{处处可导}的函数,那么$\frac{\dd}{\dd x}(f(x)g(x))=f(x)\frac{\dd}{\dd x}g(x)+g(x)\frac{\dd}{\dd x}f(x)$
\end{theorem}

\textbf{证明}:

我们从\autoref{Der2_def2} 出发.

\begin{equation}
\begin{aligned}
\frac{\dd}{\dd x}(f(x)g(x))=&\lim\limits_{h\to 0}\frac{f(x+h)g(x+h)-f(x)g(x)}{h}\\
=&\lim\limits_{h\to 0}\frac{f(x+h)g(x+h)-f(x+h)g(x)+f(x+h)g(x)-f(x)g(x)}{h}\\
=&\lim\limits_{h\to 0}\frac{f(x+h)g(x+h)-f(x+h)g(x)}{h}+\\&\lim\limits_{h\to 0}\frac{f(x+h)g(x)-f(x)g(x)}{h}\\
=&\lim\limits_{h\to 0}f(x+h)\frac{\dd}{\dd x}g(x)+g(x)\frac{\dd}{\dd x}f(x)\\
=&f(x)\frac{\dd}{\dd x}g(x)+g(x)\frac{\dd}{\dd x}f(x)
\end{aligned}
\end{equation}

\textbf{证毕}.

有了这两条性质,我们足以计算很多常见函数的导函数了.

\begin{example}{多项式函数}
多项式是形如$f(x)=a_0+a_1x+a_2x^2+\cdots+a_nx^n$的函数.为了计算其导函数,我们首先要考虑最简单的多项式,$x^n$.

$f(x)=x$的导函数很容易计算,就是处处等于1.现在,把$x^2$表示为$f(x)f(x)$,那么根据\autoref{Der2_the1} ,其导函数就是$1\cdot f(x)+1\cdot f(x)=2f(x)=2x$.

一般地,
\end{example}
















