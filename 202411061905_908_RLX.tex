% 热力学(综述)
% license CCBYSA3
% type Wiki

本文根据 CC-BY-SA 协议转载翻译自维基百科\href{https://en.wikipedia.org/wiki/Thermodynamics}{相关文章}。

\subsection{热力学}
热力学研究热量、功和温度,以及它们与能量、熵以及物质和辐射的物理性质之间的关系。这些量的行为由热力学四大定律所支配,这些定律通过可测量的宏观物理量进行定量描述,但可以通过统计力学从微观成分的角度加以解释。热力学在科学和工程的广泛领域中起着重要作用。

从历史上看,热力学起源于提高早期蒸汽机效率的需求,尤其是在法国物理学家萨迪·卡诺(1824年)的工作下,他认为发动机的效率是帮助法国赢得拿破仑战争的关键。[1] 苏格兰-爱尔兰物理学家开尔文勋爵是第一个在1854年给出简明定义的人,[2] 他说:“热力学是研究热量与作用于物体接触部分之间的力以及热量与电作用之间关系的学科。” 德国物理学家和数学家鲁道夫·克劳修斯重新阐述了卡诺循环,并为热理论提供了更真实、更可靠的基础。他于1850年发表的论文《论热的动因》[3] 首次提出了热力学第二定律。1865年,他引入了熵的概念。1870年,他引入了适用于热的维里定理。[4]

热力学的最初应用于机械热机,随后迅速扩展到化合物和化学反应的研究。化学热力学研究熵在化学反应过程中的作用本质,为该领域的扩展和知识提供了大部分内容。其他形式的热力学也随之出现。统计热力学或统计力学关注从微观行为出发对粒子集体运动的统计预测。1909年,康斯坦丁·卡拉西奥多里提出了一种纯数学方法的公理化表述,这种描述通常被称为几何热力学。
\subsection{引言}
任何热力学系统的描述都运用热力学四大定律,这些定律构成了公理基础。第一定律规定能量可以通过热、功以及物质转移在物理系统之间传递。[5] 第二定律定义了熵这一量的存在,它描述了系统在热力学上可以演变的方向,量化了系统的有序状态,并可用于量化系统中可提取的有用功。[6]

在热力学中,研究并分类大量物体之间的相互作用。热力学系统及其环境的概念在这里是核心。一个系统由粒子组成,这些粒子的平均运动定义了系统的性质,而这些性质则通过状态方程相互关联。属性可以组合以表达内能和热力学势,这对于确定平衡条件和自发过程非常有用。

通过这些工具,热力学可以用来描述系统如何响应环境的变化。这可以应用于科学和工程的许多领域,如发动机、相变、化学反应、传输现象,甚至黑洞。热力学的结果对于物理学的其他领域以及化学、化学工程、腐蚀工程、航空航天工程、机械工程、细胞生物学、生物医学工程、材料科学和经济学等都至关重要。[7][8]

本文主要集中在经典热力学,经典热力学主要研究热力学平衡态的系统。非平衡热力学通常被视为经典处理的扩展,但统计力学为这一领域带来了许多进展。
\subsection{历史}
\begin{figure}[ht]
\centering
\includegraphics[width=8cm]{./figures/3cd5bdcf9569fb74.png}
\caption{热力学的最初八大学派的热力学家。对现代热力学版本影响最深远的学派是柏林学派,特别是鲁道夫·克劳修斯(Rudolf Clausius)1865年出版的《热的机械理论》一书,维也纳学派,卢德维希·玻尔兹曼(Ludwig Boltzmann)的统计力学,以及耶鲁大学的吉布斯学派,威拉德·吉布斯(Willard Gibbs)1876年的著作《异质物质的平衡》及其引发的化学热力学研究。[9]} \label{fig_RLX_1}
\end{figure}
热力学作为一门科学学科的历史通常始于奥托·冯·盖里克(Otto von Guericke),他在1650年制造并设计了世界上第一台真空泵,并通过他的马格德堡半球演示了真空。盖里克之所以制作真空,是为了反驳亚里士多德长期以来认为“自然厌恶真空”的假设。紧接着,英爱物理学家和化学家罗伯特·波义耳(Robert Boyle)得知了盖里克的设计,并在1656年与英国科学家罗伯特·胡克(Robert Hooke)合作,制造了一台空气泵。[10] 使用这台泵,波义耳和胡克注意到压力、温度和体积之间的关联。随着时间的推移,波义耳定律得以提出,表明压力和体积成反比。然后,在1679年,基于这些概念,波义耳的同事丹尼斯·帕平(Denis Papin)制造了一台蒸汽消化器,这是一种密封容器,带有紧密的盖子,能够将蒸汽困在其中,直到产生高压。

后来,设计中加入了一个蒸汽释放阀,以防止机器爆炸。通过观察阀门有节奏地上下运动,帕平构思出了活塞和气缸发动机的想法。然而,他并未继续进行这一设计。尽管如此,在1697年,基于帕平的设计,工程师托马斯·萨弗里(Thomas Savery)制造了第一台发动机,随后托马斯·纽科门(Thomas Newcomen)于1712年进行了改进。尽管这些早期的发动机粗糙且低效,但它们引起了当时领先科学家的关注。

热容和潜热这两个热力学发展的基础概念,是由格拉斯哥大学的约瑟夫·布莱克(Joseph Black)教授提出的,詹姆斯·瓦特(James Watt)曾在该校担任仪器制造师。布莱克和瓦特一起进行实验,但正是瓦特构思了外部冷凝器的想法,这一设计大大提高了蒸汽机的效率。[11] 基于前人的工作,萨迪·卡诺(Sadi Carnot),被誉为“热力学之父”,于1824年出版了《火的动力反思》(Reflections on the Motive Power of Fire),这本书探讨了热、动力、能量和发动机效率。书中概述了卡诺发动机、卡诺循环与动力之间的基本能量关系,标志着热力学作为一门现代科学的开端。[12]

第一本热力学教科书由威廉·兰金(William Rankine)于1859年编写,他最初受过物理学训练,并曾在格拉斯哥大学担任土木与机械工程教授。[13] 热力学的第一定律和第二定律在1850年代同时提出,主要来自威廉·兰金、鲁道夫·克劳修斯和威廉·汤姆森(凯尔文勋爵)的研究工作。统计热力学的基础是由詹姆斯·克拉克·麦克斯韦、卢德维希·玻尔兹曼、马克斯·普朗克、鲁道夫·克劳修斯和J·威拉德·吉布斯等物理学家奠定的。

克劳修斯在1850年发表的论文《论热的动力》中首次阐述了第二定律的基本思想,[3] 他被称为“热力学的奠基人之一”,[14] 并于1865年提出了熵的概念。

在1873至1876年间,美国数学物理学家乔赛亚·威拉德·吉布斯(Josiah Willard Gibbs)发表了三篇重要论文,其中最著名的是《异质物质的平衡》,[15] 在这篇论文中,他展示了如何通过研究热力学系统的能量、熵、体积、温度和压力来图解热力学过程,包括化学反应,从而判断某一过程是否会自发发生。[16] 此外,皮埃尔·杜厄(Pierre Duhem)在19世纪也撰写了关于化学热力学的著作。[17] 在20世纪初,化学家如吉尔伯特·N·刘易斯(Gilbert N. Lewis)、梅尔·兰德尔(Merle Randall)[18] 和E·A·古根海姆(E. A. Guggenheim)[19][20] 将吉布斯的数学方法应用于化学过程的分析。
\subsection{词源}  
热力学有着复杂的词源学背景。[21]

从表面分析来看,这个词由两个部分组成,可以追溯到古希腊语。首先,"thermo-"("热的";如在词汇“温度计”中使用)源自希腊词根 θέρμη (therme),意为“热”。其次,"dynamics"("力学"或"动力学")源自希腊词根 δύναμις (dynamis),意为“力量”或“动力”。[23]

在1849年,威廉·汤姆森(William Thomson)使用了形容词“thermo-dynamic”。[24][25]

在1854年,汤姆森和威廉·兰金(William Rankine)使用名词“thermo-dynamics”来代表广义热机的科学。[25][21]

皮埃尔·佩罗(Pierre Perrot)声称,热力学一词是由詹姆斯·焦耳(James Joule)在1858年创造的,用来指代热与动力之间关系的科学,[12] 然而,焦耳从未使用过该术语,而是使用了“完美的热力学引擎”这一表达,指代汤姆森在1849年使用的术语。[21]
\subsection{热力学分支}  
热力学系统的研究发展出了几个相关的分支,每个分支都使用不同的基本模型作为理论或实验基础,或将热力学原理应用于不同类型的系统。
\subsubsection{经典热力学}  
经典热力学是对接近平衡态的热力学系统状态的描述,它使用宏观的、可测量的属性。经典热力学用于基于热力学定律来建模能量、功和热的交换。形容词“经典”反映了这一学科在19世纪发展过程中所代表的第一层次的理解,描述了系统的变化,主要以宏观的经验(大尺度的、可测量的)参数来表达。后来,统计力学的发展为这些概念提供了微观的解释。
\subsubsection{统计力学}  
统计力学,也称为统计热力学,随着19世纪末和20世纪初原子与分子理论的发展而出现,并通过对单个粒子或量子力学状态之间微观相互作用的解释,补充了经典热力学。该领域将单个原子和分子的微观属性与材料的宏观、可观测的体积属性联系起来,从而解释了经典热力学作为统计学、经典力学和量子理论在微观层面的自然结果。
\subsubsection{化学热力学}  
化学热力学是研究能量与化学反应或物质相态变化之间关系的学科,在热力学定律的框架内进行。化学热力学的主要目标是确定某一转化过程的自发性。[26]
\subsubsection{平衡热力学} 
平衡热力学是研究物质和能量在系统或物体中的传递的学科,这些系统或物体可以通过外界因素的作用,从一个热力学平衡状态转变到另一个平衡状态。术语“热力学平衡”表示一种平衡状态,其中所有宏观流动均为零;对于最简单的系统或物体,其强度属性是均匀的,且其压力垂直于边界。在平衡状态下,系统的宏观不同部分之间没有未平衡的势能或驱动力。平衡热力学的一个核心目标是:在给定系统处于明确初始平衡状态的情况下,结合其周围环境和构成边界,计算在特定热力学操作改变其边界或周围环境后,系统将达到的最终平衡状态。
\subsubsection{非平衡热力学}  
非平衡热力学是热力学的一个分支,研究的是不处于热力学平衡状态的系统。自然界中大多数系统都不处于热力学平衡状态,因为它们不是静止状态,且不断地、间歇性地受到物质和能量的流动,向其他系统输入和输出。对非平衡系统的热力学研究需要比平衡热力学所涉及的更为一般的概念。[27] 许多自然系统至今仍超出了目前已知的宏观热力学方法的范围。