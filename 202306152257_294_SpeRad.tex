% 谱半径
\pentry{有界算子的预解式\upref{BddRsv}}

\begin{definition}{谱半径}
设 $X$ 是复巴拿赫空间, $T:X\to X$ 是有界线性算子。 数
$$
r(T):=\sup_{\lambda\in\sigma(T)}|\lambda|~
$$
称为算子 $T$ 的\textbf{谱半径 (spectral radius)}.
\end{definition}

对预解式 $(z-T)^{-1}$ 进行冯诺依曼级数展开\upref{BddRsv}, 可见谱集包含于圆 $|z|\leq\|T\|$ 内。 所以有 $r(T)\leq\|T\|$. 更精确的定理如下:
\begin{theorem}{}
极限 $\lim_{n\to\infty}\|T^n\|^{1/n}$ 存在, 并且等于 $T$ 的谱半径 $r(T)$. 
\end{theorem}
\textbf{证明。} 

谱半径的重要意义由下述定理刻画:
\begin{theorem}{}
当 $|z|>r(T)$ 时, 冯诺依曼级数
$$
\frac{1}{z}\sum_{n=0}^\infty\frac{T^n}{z^n}~,
$$ 
按照算子范数收敛到 $(z-T)^{-1}$. 当 $|z|<r(T)$ 时, 此冯诺依曼级数按照算子范数是发散的。
\end{theorem}

可见, 谱半径的意义正如同数值幂级数的收敛半径, 而它的计算方法正如同柯西-阿达玛公式\upref{CHF}所揭示的那样。

不论在 $X$ 上取什么样的等价范数, 都不影响上面算子的可逆性和连续性, 自然也不影响它的谱性质。 由于谱半径的定义只用到了算子 $T$ 的谱的性质, 所以谱半径是与 $X$ 上的范数选取无关的。 