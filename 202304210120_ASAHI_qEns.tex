% 量子系综
% 量子系综|混态

\begin{issues}
\issueDraft
\issueTODO
\end{issues}

\pentry{量子比特\upref{Qubit}}


在传统的量子力学学习中,我们已经对密度矩阵\upref{denMat}和量子系综有了一定的了解。现在我们将会从信息的角度来重新理解这一概念。

\subsection{统计与系综}

我们不妨从Stern Gerlach实验中来引入量子系综的概念。

\pentry{Stern Gerlach实验\upref{SGExp}}

在对单比特纯态的讨论中,我们发现了一个重要的结论:对于任意的单比特纯态,总是存在着一个方向$\vec{n}\cdot\vec{\sigma}$,使得这个纯态是它的本征值为1的本征态。这意味着这样一个重要的结论:只要银原子的自旋态是纯态,那么总是存在着一个方向,只要将非均匀磁场调整到这个方向,那么这些银原子就只会在屏幕上产生一个斑点。

但是这与实验结论是不相符的。从炉子中产生的银原子满足这样的性质:不管将磁场调整到什么方向,屏幕上都会产生两个斑点。这一结论只能说明,这些银原子处于一种不能用纯态描述的状态。

什么状态能够描述这些银原子呢?经典统计物理能够给我们提供一些灵感。

\subsection{密度矩阵}

