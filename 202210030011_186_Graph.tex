% 图
% 图
\begin{aligned}
图是图论的研究对象,一个图是一个集合上的一种二元关系.
这个集合的元素称为图的点,若两个点之间有这种确定的二元关系,则称有一条边连这两个点.
在文献中,总是将一般的图记为 G = ( V , E ) ,其中, V = V ( G )和 E = E ( G )分别表示 G 的点集和边集.
一个图的点的数目称为这个图的阶,图的边的数目称为它的度.
\end{aligned}
\begin{aligned}
若有一条边连一个图的某两个节点,则称这两个节点相邻,并称这两个节点为这条边的端点.若某一节点是某一条边的端点,则称这个节点和这条边关联.若两条边和同一节点关联,则称这两条边相邻,两个端点是同一个节点的边称为环.若某条边的两个端点不是同一个节点,且只有一条边连这两个节点,则称这条边为杆.只有一个竹点而没有边的图称为平凡图;没有边的图称为孤立图.以某两节点为端点的边可能不止一条,这时称连这两个节点的边为重边.既可以有
\end{aligned}