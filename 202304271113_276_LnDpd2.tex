% 线性相关与线性组合


\begin{issues}
\issueOther{应该是重复内容, 有待合并}
\end{issues}

\pentry{分块矩阵\upref{BlkMat}, 线性方程组解的结构\upref{LinEq}}

\begin{figure}[ht]
\centering
\includegraphics[width=6cm]{./figures/f1d2d5f8b1462cc0.pdf}
\caption{v1,v2,v3线性相关,而v1,v2,v4线性无关} \label{fig_LnDpd2_1}
\end{figure}

\begin{definition}{线性相关}
对于一组向量$\alpha_1, \alpha_2,\alpha_3,...$,若\textbf{存在不全为0的}$k_1,k_2,...$,使$k_1 \bvec \alpha_1+k_2 \bvec \alpha_2+k_3 \bvec \alpha_3+...=0$,即称$\alpha_1, \alpha_2,\alpha_3,...$线性相关\upref{linDpe}。

反之,若\textbf{只当}$k_1=k_2=...=0$时,$k_1 \bvec \alpha_1+k_2 \bvec \alpha_2+k_3 \bvec \alpha_3+...=0$,则$\alpha_1, \alpha_2,\alpha_3,...$线性无关。
\end{definition}

\begin{definition}{线性组合、线性表出}
对于一组向量$\alpha_1, \alpha_2,\alpha_3,...$与一个向量$\beta$,若存在$k_1,k_2,...$,使$k_1 \bvec \alpha_1+k_2 \bvec \alpha_2+k_3 \bvec \alpha_3+...=\beta$,则称$\beta$是$\alpha_1, \alpha_2,\alpha_3,...$的线性组合。
\end{definition}

\begin{theorem}{}
若$\beta$是$\alpha_1, \alpha_2,\alpha_3,...$的线性组合,则$\beta, \alpha_1, \alpha_2,\alpha_3,...$必线性相关。
\end{theorem}

\begin{theorem}{}
若$\alpha_1, \alpha_2,\alpha_3,...,\alpha_n$线性相关,则$\alpha_1, \alpha_2,\alpha_3,...,\alpha_n,\alpha_{n+1},\alpha_{n+2},...$必线性相关。
\end{theorem}

\begin{theorem}{}
m个n阶向量(m>n)必线性相关。
\end{theorem}

\subsection{线性相关、线性表出与线性方程组}

\subsubsection{线性相关}
根据分块矩阵,$k_1 \bvec \alpha_1+k_2 \bvec \alpha_2+k_3 \bvec \alpha_3+...=0$可记为
$$
\begin{pmatrix}
\bvec \alpha_1& \bvec \alpha_2& \bvec \alpha_3&...
\end{pmatrix}
\begin{pmatrix}
k_{1}\\
k_{2}\\
k_{3}\\
...\\
\end{pmatrix}
=
0
\Leftrightarrow 
\mat A \bvec k = 0~.
$$
这将线性相关问题化为线性方程组问题:若$\mat A \bvec k = 0$有非零解,则线性相关,否则线性无关。

\subsubsection{线性表出}
同理,
$$
\begin{pmatrix}
\bvec \alpha_1& \bvec \alpha_2& \bvec \alpha_3&...
\end{pmatrix}
\begin{pmatrix}
k_{1}\\
k_{2}\\
k_{3}\\
...\\
\end{pmatrix}
=
\bvec b
\Leftrightarrow 
\mat A \bvec k = \bvec b
$$
这将线性组合问题化为线性方程组问题:若$\mat A \bvec k = \bvec b$有解,则$\beta$是$\alpha_1, \alpha_2,\alpha_3,...$的线性组合。
