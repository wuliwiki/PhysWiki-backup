% 导数(高中)
% keys 导数|高中|变化|求导
% license Xiao
% type Tutor

\begin{issues}
\issueDraft
\end{issues}

在学习函数的性质时,曾经提到过函数的\aref{变化率}{sub_HsFunC_4}。其中,\textbf{平均变化率(average rate of change)}的概念类似于计算某段时间内的平均速度,反映了函数在一个区间内的整体变化趋势。当平均变化率所涉及的两个点逐渐靠近,直至几乎重合时,这一变化率便转化为描述局部变化的工具。此时,连线逐渐成为该点处的切线,而平均变化率也演变为\textbf{瞬时变化率(instantaneous rate of change)},即函数在该点的变化速率。瞬时变化率的另一个名称是\textbf{导数(derivative)}。

平均变化率提供了宏观的变化趋势,而导数则通过精确的数学方法刻画了局部的瞬时变化。导数的应用范围极其广泛,几乎所有涉及变化的领域都能发现它的踪迹。例如,在经济学中,导数用于分析股票价格的涨跌;在气象学中,它可以测量温度的变化速度;在生物学中,它帮助研究细胞分裂的速率。作为研究变化的强大工具,导数为人们提供了一种新的思维方式,帮助深入理解和处理动态问题。

导数的理论基础依赖于\textbf{极限(limit)},但由于高中阶段未涉及极限的具体内容,因此高中的导数的学习主要聚焦于以下三个方面:
\begin{enumerate}
\item 理解背景和实际意义:掌握导数的几何意义和实际应用场景,理解它与函数其他性质的关系;
\item 熟练计算:学习导数的常见计算规则,能够快速、准确地对复杂函数求导;
\item 解决实际问题:利用导数分析函数性质,在面对恒成立、不等关系等问题时,知道如何构造辅助函数来使用导数解决问题。
\end{enumerate}

导数不仅是理解函数变化的核心工具,也是未来微积分学习的重要基础。高中阶段对导数的熟练掌握,将为进一步研究函数极限与积分打下坚实的基础。

\subsection{定义}

在谈论导数的定义之前,先从一个生活中的常见情景入手:假设某人开车从家到商场,整个行程花了30分钟,行驶了15公里。通过小学就学过的计算,可以得出这段旅程的\textbf{平均速度}为15公里 ÷ 0.5小时 = 30公里/小时。这种计算方式,是对整个行程的总体描述,表示汽车“平均”每小时行驶的距离。

然而,在实际驾驶过程中,汽车的速度并不是恒定的。坐车时可以看到仪表盘上的速度表,它显示的单位同样是“公里/小时”,但随着行驶情况变化,汽车可能需要减速、加速,甚至短暂停车,它显示的数字也会有时是10公里/小时,有时是80公里/小时,甚至可能短暂为0。这些读数反映的是汽车在某一时刻的速度,也被称为\textbf{瞬时速度}。

瞬时速度与平均速度的差别在于,前者描述的是某一具体时刻的车速,而后者是对整个行程的概括。这两种速度各有用处:平均速度简单直观,但忽略了旅途中车速的变化;瞬时速度则能精确反映某一时刻的状态,但若没有速度表,就没有办法知道瞬时速度。

如果把汽车的运动位置当成一个函数的话,瞬时速度就是这个函数的导数。速度表\footnote{下面探讨的都是电子式的速度表,机械式的原理略有差异。}的显示原理就是在不断地进行导数运算。那么速度表是怎么实现的呢?下面大概介绍一下原理,速度表自己带有一个脉冲信号,他会从0开始不断计数,而且计数间隔的时间非常短。另外,车轮每转一圈它就会记录并重新开始计数。对于一台确定的车,每两个脉冲间隔的时间$t$,车轮的直径为$d$,都是确定的,如果车轮或轴旋转一圈时会触发的脉冲数为$N$,那么速度表的值$v$就是:

\begin{equation}
v={\pi d\over tN}=\frac{\Delta y}{\Delta x}~.
\end{equation}

这里,$\Delta y$表示车轮转一周的路程,$\Delta x$表示对应的时间间隔。这个计算方式本质上仍然是之前熟悉的平均速度计算公式。但如之前所说,人们通常将速度表显示的值视为瞬时速度。那么,为什么这里的平均速度可以作为瞬时速度呢?

这和导数的定义有很大的关系。对于函数的\aref{平均变化率}{def_HsFunC_3}$\displaystyle\frac{\Delta y}{\Delta x}$,如果区间$[a, b]$的两个端点距离非常近,近到几乎无法区分两个端点了,这时一般$\Delta x,\Delta y$都会非常小,这个比值如果稳定存在的话,那么就称这个比值为导数。所以,在实际使用时,如果能够控制这两个值,使得它们达成都比较小的特定条件,就可以近似地认为比值是导数。而这也是几乎所有数字领域实现导数计算的原理。

在速度表的例子中,车轮周长通常在2米左右(即0.002千米),对于“公里/小时”这种单位的速度,速度表需要感知的最低车速约为1公里/小时。也就是说,对应的最小时间间隔约为0.002小时(即7.2秒)。因此,车速较低时,速度表的数值可能会有一定延迟,因为需要完成计数后才能更新,不过这时对于驾驶者影响也不大了。当然前面这是题外话,对于日常使用来说,0.002这个数字已经足够小了,更不用说一般汽车行驶的速度都是几十公里每小时,时间间隔就更小了。因此,才可以基本认为速度表输出的就是瞬时速度。

回过头来看,从刚才的描述中,已经从平均变化率与导数的区别得到了导数定义的核心部分,下面以定义形式给出:

\begin{definition}{导数}
若函数$f(x)$在某点$x_0$处的极限
\begin{equation}
\lim_{\Delta x\to 0}{f(x_0+\Delta x)-f(x_0)\over \Delta x}~.
\end{equation}
存在,则称其为$f(x)$在$x_0$处的\textbf{导数(derivative)},记作$f'(x_0)$。
\end{definition}

在导数的定义中,表达式$f(x_0+\Delta x)-f(x_0)$代表函数值的增量,也就是平均变化率中的$\Delta y$的具体形式,严格对应于自变量的增量$\Delta x$。这一形式表达了函数在某一段增量上的整体变化。

导数定义的关键在于让$\Delta x$趋于0,这一点与速度表中的近似情况有着本质的不同。在速度表的例子中,虽然时间间隔$t$可以很短,在实际应用时已经足够使用精确,但它从未真正达到0,因此计算的速度始终是一种物理近似。数学领域追求的是严谨与普遍性。在速度表的例子中,如果$\Delta x$始终保持有限值,那么导出的速度是有限精度的。但在数学中,讨论的不是“非常小”,而是“无限接近于0”。这是通过极限的概念实现的:当$\Delta x$趋于0时,$\frac{\Delta y}{\Delta x}$的值是否稳定地接近某个固定值?通过让$\Delta x$趋于0,可以从根本上消除误差,获得一个“极限精确”的值。极限的引入不仅确保了定义的严谨性,也拓宽了导数的适用范围,使其能够用于分析任何精确到无穷小尺度的变化。

这里经常会有人产生困惑——“这里的$\Delta x$是不是0?”。这个问题哪怕是在牛顿和莱布尼茨提出微积分时,也仅仅是使用了这个概念,而不是清晰地描述它。这里的$\Delta x$称作\textbf{无穷小}\footnote{注意无穷小并不是负无穷大。},是指在绝对值上可以任意接近 0 的动态量,但它本身并不是 0。无穷小的存在,让数学得以研究“接近”的行为,而不局限于“已经到达”的状态。当$\Delta x$趋于0时,分母的变化量会变得无限小。此时,分子$f(x_0+\Delta x)-f(x_0)$的变化同样变小,但它们之间的比值却可能趋向某个确定值,这一现象是极限的核心所在。


导数也是一个对应关系,即每个自变量都对应一个导数,因此他也是一个函数,这个函数称为\textbf{导函数}(不引起歧义时,简称为导数)。导函数和原本的函数是一一对应的,因此可以根据定义或求导方法,来求一个函数的导函数,这个过程就是\textbf{求导}。

\subsection{记号}
由于历史发展和人们长久以来的使用习惯,导函数逐渐衍生出了许多不同的记法。这些记法不仅仅是使用者的偏好选择,还与特定领域的需求和表达习惯密切相关。既为方便计算和推导,也为强调不同的数学概念。了解这些符号的使用,有助于理解求导这个运算,另外在未来见到时,也不至陌生,不要求完全掌握,看个眼熟就好。下面的符号针对函数$y=f(x)$:
\begin{itemize}
\item 拉格朗日记法——$y'$或$f'(x)$,好处是记法比较简洁,便于书写,缺点是难以表达较为复杂的关系。高中数学主要采用这种表示法。某点$x_0$处的导数记作$f'(x_0)$。
\item 莱布尼茨记法——$\displaystyle\frac{\dd y}{\dd x}$  或  $\displaystyle\frac{\dd}{\dd x}f(x)$,好处是在进行某些复杂运算时,分子与分母可以直接按照乘除法的规则来进行运算,降低推导的复杂度。另外,也在形式上代表着变化率。在大学阶段的数学领域主要采用这种表示方法。某点$x_0$处的导数记作$\displaystyle\eval{\dfrac{\dd y}{\dd x}}_{x=x_0}$ 或$\displaystyle\frac{\dd}{\dd x}f(x_0)$。
\item 牛顿记法——$\dot{y}$。由于在物理学中,时间是一个较为特别的变量,一般用这种方法来表示某个变量相对于时间的导数。基本只在物理学领域使用。
\item 重导数记法——$f_x$,这种记法简洁紧凑,又能在复杂关系出现时,避免拉格朗日记法的问题。在偏微分方程和张量分析中常用,尤其是对多重导数关系。
\item 欧拉记法——$Df(x)$,采用$D$算子。主要在大学阶段的微分方程中使用,好处是$D^n f(x)$可以直接修改$n$来表示进行几次求导运算。另外,将求导(微分)视为一种算子,便于与其他运算符进行组合,适合处理复杂的微分运算。
\item 差分导数——$\Delta f(x)$,对于定义域是离散的函数(一般是$\mathbb{Z}$),通常会用这样的符号来表示它的导数,称为\textbf{差分}。
\end{itemize}

在高中阶段,一般要求只使用拉格朗日记法,即$y'$或$f'(x)$,并且不允许使用其他记法。其余的记法可以这样理解:
\begin{itemize}
\item 用$D$运算符代替$\displaystyle\frac{\dd}{\dd x}f(x)$中的$\displaystyle\frac{\dd}{\dd x}$就成了$Df(x)$;
\item 把$\displaystyle\frac{\dd}{\dd x}f(x)$中最重要的两部分$f,x$拿出来显示就成了$f_x$;
\item 在明确自变量的情况下,只强调$\displaystyle\frac{\dd y}{\dd x}$中的$y$就成了$y'$;
\item $\dot{y}$与$y'$异曲同工,只是更着眼于“时间”;
\item $\Delta f(x)$与$Df(x)$只是因为定义域不同,处理方法不同;
\end{itemize}

\subsection{几何含义}


\subsection{求导法则}

为记录方便,下面记$u=f(x),v=g(x),u'=f'(x),v'=g'(x)$。

\begin{itemize}
\item 加减法:$(u\pm v)'=u'\pm v'$
\item 乘法:$(uv)'=u'v+uv'$
\item 除法:$\displaystyle\left(\frac{u}{v}\right)'=\frac{u'v-uv'}{v^2}$
\item 复合函数:$(f(v))'=f'(v)v'$
\end{itemize}

\subsection{基本初等函数的导数推导}

\subsection{对照表}

这里将常见的函数与导数对照表列出如下,方便查询。具体介绍需查看每个函数自己的页面。

\begin{table}[ht]
\centering
\caption{高中常见函数及其导数}\label{tab_HsDerv1}
\begin{tabular}{|c|c|c|}
\hline
\textbf{函数名称}     & \textbf{函数 $f(x)$}     & \textbf{导函数 $f'(x)$}     \\ \hline
幂函数&$x^n$                    & $n x^{n-1}$                \\ \hline
反比例函数&$\displaystyle\frac{1}{x}$             & $\displaystyle-\frac{1}{x^2}$           \\ \hline
指数函数(e为底)&$e^x$                     & $e^x$                      \\ \hline
对数函数(e为底)&$\ln(x)$                  & $\displaystyle\frac{1}{x}$              \\ \hline
指数函数&$a^x$                     & $a^x\ln a $                      \\ \hline
对数函数&$\log_a(x)$                  & $\displaystyle \frac{1}{x\ln a}$              \\ \hline
正弦函数&$\sin(x)$                 & $\cos(x)$                  \\ \hline
余弦函数&$\cos(x)$                 & $-\sin(x)$                 \\ \hline
正切函数&$\tan(x)$                 & $\displaystyle \frac{1}{\cos^2(x)}$                \\ \hline
\end{tabular}
\end{table}

