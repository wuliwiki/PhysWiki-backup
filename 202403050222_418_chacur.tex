% 偏微分方程的分类与特征线
% keys 特征线|PDE|偏微分方程
% license Usr
% type Tutor

\begin{issues}
\issueMissDepend
\issueTODO
\issueDraft
\end{issues}

特征线法又被称为达朗贝尔法和行波法,一般适用于解初值问题的一阶 PDE,对二阶 PDE 的分类也颇有帮助。本文默认 PDE 的范围是 $-\infty < x, y, \dots < +\infty, t>0$,用 $u'_t$ 表示 $\pdv{u}{t}$,用 $u'_x$ 表示 $\pdv{u}{x}$。。

特征线有如下性质,
\begin{enumerate}
\item 对于 $n$ 维(这里包含时间维)的 PDE,其特征线总是 $n-1$ 维的。
\item 解的“间断”性质通过、且仅能通过特征线传播。
\item 特征线上 PDE 的解是一样的,特征线的斜率是解“传播”的速度。
\end{enumerate}
其中性质 $2$ 对应着如果不存在特征线(特征线不是实的,类似一元二次方程没有实数根称为根不存在),那么 PDE 的解连续(这就是 elliptic PDE)。







\subsection{一维一阶 PDE}
\begin{theorem}{}
对于关于 $u(x, t)$ 的 PDE:
\begin{equation}
\pdv{u}{t} + A(x, t) \pdv{u}{x} + B(x, t) u = f(x, t) ~,
\end{equation}
初值条件 $u(x, 0) = \phi(x), -\infty < x < +\infty$。

设特征线族为 $x = x(t, \tau)$ 是下面 ODE 的解,
\begin{equation}
\left\{
\begin{aligned}
\dv{x}{t} &= A(x, t), \\
x(0) &= \tau ~.
\end{aligned}
\right .
\end{equation}
考虑 $v(t) = u(x(t), t)$,由全微分与偏导的关系,
$$\dv{v}{t} = \pdv{u}{x} \dv{x}{t} + \pdv{u}{t} \dv{t}{t} = \pdv{u}{x} \cdot A(x, t) + \pdv{u}{t} ~,$$
这便是特征线的解。若再要求解原方程组,又有:
\begin{equation}
\left\{
\begin{aligned}
\dv{v}{t} + B(x, t) v &= f(x(t), t) ~, \\
v(0) = u(x(0), 0) &= u(\tau, 0) = \phi(x) ~.
\end{aligned}
\right.
\end{equation}
\end{theorem}








下面利用两个一维一阶的 PDE 举例来说明如何求解 PDE 的特征线。
\begin{example}{}
求解这 PDE 的特征曲线与其本身:
\begin{equation}
\left \{ 
\begin{aligned}
u'_t + (x + t) u'_x + u &= x ~, \\
\eval{u}_{t=0} &= x ~.
\end{aligned}
\right .
\end{equation}

\textbf{解}:特征曲线 $x=x(t)$ 对应下面 ODE,
\begin{equation}
\left \{ 
\begin{aligned}
\dv{x}{t} &= x+t ~, \\
x(0) &= \tau ~.
\end{aligned}
\right .
\end{equation}
可以解得 $x(t) = e^t - t - 1 + \tau \cdot e^t$。


\end{example}
% todo















很多人都对偏微分方程为什么用圆锥曲线来分类有疑惑,一个偏微分方程为什么会跟平面上曲线的分类有关呢?对 PDE(偏微分方程)进行分类又对解其有何帮助?下面来讨论这些问题。

\subsection{二阶线性 PDE 的分类}
在一般的数理方程或介绍 PDE 的书籍中,会把二阶线性 PDE 作为着重点来讲并一般将之分为三类,分别是
\begin{enumerate}
\item 椭圆类(elliptic PDE),例如泊松方程 $\laplacian u = f(x,y,z)$;
\item 抛物线类(parabolic PDE),例如热传导方程 $k \laplacian T = \frac{\partial T}{\partial t}$;
\item 双曲类(hyperbolic PDE),例如波动方程 $\laplacian u = \frac{1}{c^2} \frac{\partial^2 u}{\partial t^2}$。
\end{enumerate}
