% 速度规范
% 长度规范|速度规范|波函数|规范变换|薛定谔方程|麦克斯韦方程组

\pentry{电磁场中的单粒子薛定谔方程\upref{QMEM}, 偶极子近似}

当空间中存在静止的电荷分布时, 我们可以把标量势能分为 $V + \varphi$ 两部分. 前者由静止电荷根据库伦定律计算, 不参与规范变换, 在这里我们甚至可以不把它看成电磁力而只是某种一般的势能. 后者可以随时间变化, 但库仑规范下 $\varphi = 0$. 定义不含时哈密顿算符为
\begin{equation}
H_0 = \frac{\bvec p^2}{2m} + qV
\end{equation}
则库伦规范下, 电磁场中带电粒子的哈密顿量为(\autoref{QMEM_eq4}~\upref{QMEM})
\begin{equation}\label{LVgaug_eq2}
H = H_0 - \frac{q}{m} \bvec A \vdot \bvec p + \frac{q^2}{2m} \bvec A^2
\end{equation}
当我们用偶极子近似时, $\bvec A(t)$ 与位置无关而只是时间的函数. 我们可以利用这个性质方便地得到另外另种规范. 注意他们只有在偶极子近似时才成立.

规范变换为\autoref{QMEM_eq5}~\upref{QMEM}
\begin{equation}\label{LVgaug_eq1}
\bvec A = \bvec A' + \grad \chi
\qquad
\varphi = \varphi' - \pdv{\chi}{t}
\end{equation}
对库仑规范使用规范变换
\begin{equation}\label{LVgaug_eq3}
\Psi(\bvec r, t) = \exp(\I q\chi)\Psi^V(\bvec r, t)
\end{equation}
\begin{equation}\label{LVgaug_eq4}
\chi(t) = \frac{q}{2m} \int^t \bvec A^2(t') \dd{t'}
\end{equation}
注意\autoref{LVgaug_eq1} 中 $\grad \chi = \bvec 0$ 所以 $\bvec A' = \bvec A$. 把\autoref{LVgaug_eq1} 带入\autoref{LVgaug_eq2} 可以消除 $\bvec A^2$ 项得\footnote{该式中省略了 $\bvec A'$ 的瞥}
\begin{equation}
H^V = H_0 - \frac{q}{m} \bvec A \vdot \bvec p
\end{equation}
薛定谔方程为
\begin{equation}
H^V \Psi^V = \I \pdv{t} \Psi^V
\end{equation}
这种规范叫做\textbf{速度规范(velocity gauge)}.
