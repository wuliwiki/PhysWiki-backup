% 克拉伯龙方程
% 相变|相平衡|饱和蒸气压方程

\begin{issues}
\issueDraft
\end{issues}

\pentry{热动平衡判据\upref{equcri}}

\subsection{克拉伯龙方程}
单元系两相平衡共存时,必须满足\upref{equcri} 中所讲的 $T^\alpha=T^\beta=T,P^\alpha=P^\beta=P,\mu^\alpha(T,P)=\mu^\beta(T,P)$.其中相平衡条件中摩尔化学势 $\mu$ 可以看作是温度和压强的函数.此时,相变温度 $T$ 对应着压强 $P$(在压强 $P$ 的环境下,相变温度为 $T$).

现在考虑系统的温度和压强有改变量 $\dd T,\dd P$,要使得两相仍处于平衡条件,就有

\begin{align}
&\mu^\alpha(T,P)=\mu^\beta(T,P)\\
&\mu^\alpha(T+\dd T,P+\dd P)=\mu^\beta(T+\dd T,P+\dd P)
\end{align}

由于化学势就是摩尔吉布斯函数,所以 $\dd\mu=-S_m\dd T+V_m\dd P$($S_m$ 为摩尔熵,$V_m$ 是摩尔体积).由 $\dd \mu^\alpha=\dd \mu^\beta$ 可以推出
\begin{equation}
\frac{\dd P}{\dd T}=\frac{S^\beta_m-S^\alpha_m}{V^\beta_m-V^\alpha_m}
\end{equation}

在实验上熵是不能直接测量的,但我们知道在可逆过程中 $\Delta Q=T\Delta S$.考虑 $1\rm mol$ 物质从 $\alpha$ 相转变到 $\beta$ 相所吸收的\textbf{相变潜热},由于相变时物质温度不变,有 $L=T(S_m^\beta-S_m^\alpha)$.可以得到\textbf{克拉伯龙方程}:

\begin{equation}\label{Clapey_eq1}
\frac{\dd P}{\dd T}=\frac{L}{T(V^\beta_m-V^\alpha_m)}
\end{equation}
\begin{figure}[ht]
\centering
\includegraphics[width=10cm]{./figures/Clapey_1.png}
\caption{水的三相图} \label{Clapey_fig1}
\end{figure}

相平衡曲线的斜率通常是正的,但也存在例外:从水的三相图中看到,水的熔化线斜率 $\dd p/\dd T<0$.而水的摩尔体积比冰的摩尔体积小,密度比冰大,代入克拉伯龙方程确实能得到熔化曲线斜率小于 $0$ 的结果.

\subsection{饱和蒸气压方程}

由克拉伯龙方程可以得出在气相 $\beta$ 与凝聚相(液相或固相)$\alpha$ 之间的相变方程,可以得到饱和蒸气压与温度的关系,也就是\textbf{饱和蒸气压方程}.现在做粗略的近似,如果将气相看作理想气体,那么由 \autoref{Clapey_eq1} 可得
\begin{equation}
\frac{1}{P}\frac{\dd P}{\dd T}=\frac{L}{RT^2}
\end{equation}

再做更粗糙的近似,将相变潜热 $L$ 认为是与温度无关.那么可以积分得:
\begin{equation}
\ln P=-\frac{L}{RT}+A
\end{equation}