% 薛定谔绘景和海森堡绘景



\pentry{时间演化算符(量子力学)\upref{TOprt},转移矩阵\upref{TransM}}

% \footnote{参考 Wikipedia \href{https://en.wikipedia.org/wiki/Heisenberg_picture}{相关页面}.}
薛定谔方程\upref{TDSE}通常使用的是动量表象\upref{moTDSE}和\textbf{薛定谔绘景}, 在海森堡绘景中, 波函数(态矢)不随时间改变, 而测量量的算符随时间改变. 海森堡绘景相当于在薛定谔绘景的基础上做了一个基底变换, 类似于位置和动量表象\upref{moTDSE}的关系.



作为一个物理理论,量子力学关心的是可观测量,包括本征值、概率、期望等,因此我们关心的不是量子态本身如何,而是量子态在可观测量的本征矢下展开的系数.随着时间流逝,这些系数会变化,而薛定谔绘景和海森堡绘景就是两种不同的解释系数变化的方法.

薛定谔绘景认为,可观测量不变,但是态矢量会随着时间变化,导致展开系数变化;海森堡绘景则认为,态矢量不变,但可观测量会变化,造成其本征矢量变化,从而导致态矢量的展开系数变化\footnote{严格来说,薛定谔绘景中也有可观测量会变化的情况,比如磁场在变化,那么哈密顿量$\bvec{S}\cdot \bvec{B}$就可能变化.所以这里说的是薛定谔绘景中不随时间变化的算符,到了海森堡绘景就会随时间变化.如果一个算符在薛定谔绘景中随时间变化,那么到了海森堡绘景只需要再叠加一个基底变换即可,正如本文开头所说,细节则请参见下面的小节.}.

这和线性代数里情况一模一样.一个矩阵可以解释为线性变换,它把一个向量变为另一个向量,导致这个向量的坐标变化了;也可以解释为\textbf{转移矩阵}\upref{TransM},向量本身没变,但是基变了,同样导致坐标变化.

你可以用这样一副图像来抽象地理解两个绘景的关系:薛定谔绘景中,坐标系不变,但向量在顺时针旋转;海森堡绘景中,向量不变,但坐标系在以相同的角速度逆时针旋转.这副图像已经能说明为什么海森堡绘景可以看作是基向量反向演化.



本文中,角标 $H$ 代表海森堡绘景, 角标 $S$ 代表薛定谔绘景. 例如波函数分别记为 $\psi_H(\bvec r, t)$ 和 $\psi_S(\bvec r)$, 后者不是时间的函数, 它的定义是
\begin{equation}
\psi_S(\bvec r) = \psi_H(\bvec r, 0)
\end{equation}

同时,本文使用$\hbar=1$的单位制,时间演化算符按定义为$\mathcal{U}(t)=\E^{-\I H t}$.

% \addTODO{演化子是什么?链接} %Jier: 就是时间演化算符\upref{TOprt},该词条我已经更新了

% 使用演化子(propagator) $U(t)$, 波函数之间的关系为
% \begin{equation}
% \psi_H(\bvec r, t) = U(t) \psi_H(\bvec r, 0) = U(t) \psi_S(\bvec r)
% \end{equation}
% 薛定谔方程在海森堡绘景中可以记为
% \begin{equation}
% H(U\psi_H) = \I\hbar \pdv{t} (U\psi_H)
% \end{equation}
% 由于 $\psi_H$ 不含时, 两边抵消, 得演化子满足方程
% \begin{equation}
% H U(t) = \I\hbar \pdv{t} U(t)
% \end{equation}

% 定义海森堡绘景中的算符为
% \begin{equation}
% Q_H(t) = U\Her(t) Q_S(t) U(t)
% \end{equation}
% 对时间求导得
% \begin{equation}
% \dv{t}Q_H = \frac{\I}{\hbar} [H_H, Q_H(t)] + \qty(\pdv{Q_S}{t})_H
% \end{equation}
% 平均值公式仍然和薛定谔绘景相同
% \begin{equation}
% \mel{\psi_H}{Q_H}{\psi_H} = \mel{\psi_S}{Q_S}{\psi_S}
% \end{equation}
% 证明:
% \begin{equation}
% \mel{\psi_S}{Q_S}{\psi_S} = \mel{\psi_H}{U\Her(t)Q_SU(t)}{\psi_H} = \mel{\psi_H}{Q_H}{\psi_H}
% \end{equation}
% 证毕.

\subsection{薛定谔绘景}

薛定谔绘景,简而言之,是固定算符不变,研究态矢量的演化.这也是我们在\textbf{量子力学的基本原理(量子力学)}\upref{QMPrcp}中使用的描述.

在薛定谔绘景中,算符是恒定的,从而其对应的本征矢量也是恒定的.实际的量子态$\ket{s}$则随着时间$t$演化为$\mathcal{U}(t)\ket{s}$.这个过程可以理解为,映射$\mathcal{U}(t)$作用在矢量$\ket{s}$上,导致$\ket{s}$变化,于是其关于各可观测量的本征态的基底展开系数变化——这些系数的模方就是测量后得到对应本征态的概率(概率密度),因此我们观测到的概率就会变化.


薛定谔绘景下,初态为$\ket{s}$的量子态,其可观测量$X$的期望值随时间演化:
\begin{equation}
\langle X \rangle(t) = \bra{s}\mathcal{U}^\dagger(t)X\mathcal{U}(t)\ket{s}
\end{equation}

\subsubsection{薛定谔方程}

薛定谔绘景下,态右矢对时间的导数为
\begin{equation}\label{HsbPic_eq4}
\ali{
    \frac{\dd}{\dd t}\mathcal{U}(t)\ket{s} &= \frac{\dd}{\dd t}\E^{-\I H t}\ket{s}\\
    &= -\I H \mathcal{U}(t)\ket{s}
}
\end{equation}
如果记$\ket{s, t}=\mathcal{U}(t)\ket{s}$,那\autoref{HsbPic_eq4} 也就是
\begin{equation}\label{HsbPic_eq5}
\I \frac{\dd}{\dd t}\ket{s, t} = H\ket{s, t}
\end{equation}
这就是\textbf{薛定谔方程}.








\subsection{海森堡绘景}

海森堡绘景是另一种描述量子力学的框架,量子态本身不变,但可观测量的算符以及对应的本征态则随时间变化,由此造成量子态的基底展开系数变化.海森堡绘景下计算得到的可观测量的演化规律和薛定谔绘景相同.

注意,时间演化算符$\mathcal{U}(t)=\exp(-\I Ht)$是一个\textbf{幺正算符},即$\mathcal{U}^\dagger(t)=\mathcal{U}^{-1}(t)$.

\subsubsection{离散情况}



设$X^{(S)}$是薛定谔绘景下\textbf{不随时间改变}的可观测量,$X^{(H)}(t)$是在海森堡绘景下的同一个观测量,为了方便,特别地令$X^{(H)}(0)=X^{(S)}$.

% 设能量算子$H^{(S)}$的本征值$E_a$的本征矢为$\ket{a}$,其中$a$是正整数;$H^{(H)}(t)$的对应本征值$E_a$的本征矢则记为$\ket{a; t}$.规定在$t=0$时,$\ket{a; 0}$.

固定$\ket{s}$不变,而让可观测量算符$X^{(H)}(0)$变为$X^{(H)}(t)$.算符改变了,则其本征矢$\ket{a_i; t}$也会改变.

我们断言:$\ket{a_i; t}=\mathcal{U}^{-1}(t)\ket{a_i; 0}$,即海森堡绘景下基的演化,与薛定谔绘景下态矢量的演化是反向的.这直观上很好理解和记忆:考虑一个二维实线性空间,如果说薛定谔绘景下态矢量逆时针旋转、坐标系不变,那么海森堡绘景下就应该是态矢量不变、坐标系顺时针旋转.详细证明如下.

由于可观测量的本征矢能构成一组基,因此任意量子态$\ket{s}$可以用这组基展开\footnote{$c^i$中的$i$是上标,功能类似下标,不是次方.这么写是为了配合\textbf{爱因斯坦求和约定}\upref{EinSum},但该预备知识并非必要.}:
\begin{equation}\label{HsbPic_eq12}
\ket{s} = \sum_i c^i(t)\ket{a_i; t} = \sum_i c^i(0)\ket{a_i; 0}
\end{equation}


基的变换由一个\textbf{新的时间演化算符}$\Omega(t)$表征:
\begin{equation}\label{HsbPic_eq13}
\ket{a_i; t} = \Omega(t)\ket{a_i; 0}
\end{equation}
于是$c^i(t)=\braket{s}{a_i; t}=\bra{s}\Omega(t)\ket{a_i; 0}$.

为方便之后的讨论,我们用\textbf{转移矩阵}\upref{TransM}$\omega^i_j(t)$来描述算符$\Omega(t)$对基的变化:
\begin{equation}
\Omega(t)\ket{a_i; t} = \sum_k\omega^k_i(t)\ket{a_k, t}
\end{equation}


回到薛定谔绘景,此时保持基向量$\ket{a_i; 0}=\ket{a_i}$不变,而态矢量$\ket{s}$按时间演化算符改变.两个绘景下,态矢量的展开系数相同,于是有:
\begin{equation}\label{HsbPic_eq14}
\mathcal{U(t)}\ket{s} = \sum_i c^i(t) \ket{a_i}
\end{equation}






% Jier:注释原因:讲得太云里雾里了,设一堆乱七八糟的,改成上面那样了,逻辑为“测量算符变-本征基变-系数变-与薛定谔绘景的系数变化比较-结论”.

% 我们通过转移矩阵$\Omega(t)=\omega^i_j(t)$实现这个变化:$X^{(H)}(t)=X^{(H)}(0)$,这也可以理解为通过改变基来改变测量算符.如果所用的基是测量算符的本征矢构成的,那改变后的基就是海森堡绘景中改变后的本征矢构成的.

% 设有一初始量子态$\ket{s}=\sum_a c^a\ket{a}=\sum_a c^a\ket{a; 0}$\footnote{这里的$a$是上标,功能类似下标,不是次方.这么写是为了配合\textbf{爱因斯坦求和约定}\upref{EinSum},但该预备知识并非必要.}.在薛定谔绘景中,一段时间$t$后,这个态变为$\sum_a c^a\mathcal{U}(t)\ket{a}=\sum_ac^a\E^{-\I at}\ket{a}$,即系数的变化是$c^a\to c^a\E^{-\I at}$.

% 根据\textbf{转移矩阵}\upref{TransM}的\autoref{TransM_sub1}~\upref{TransM},如果固定$\ket{s}$不变,而基根据转移矩阵$\Omega(t)$改变,那么$\ket{s}$的系数变化是$c^a \to \sum_{j}\omega^a_j(t)c^j$.

% 因此,薛定谔绘景和海森堡绘景的统一性要求,在能量本征态为基时有
% \begin{equation}\label{HsbPic_eq1}
% \E^{-\I at}c^a = \sum_{j}\omega^a_j(t)c^j
% \end{equation}
% 由于能量本征态随时间的演化依旧是能量本征态,因此$\Omega(t)$在能量本征态下的矩阵是对角矩阵,于是\autoref{HsbPic_eq1} 简化为
% \begin{equation}
% \E^{-\I at} = \omega^a_a(t)
% \end{equation}
% 这意味着$\Omega(t)$就是$\E^{-\I H t}=\mathcal{U}(t)$.

于是,海森堡绘景下,测量算符随时间的演化为
\begin{equation}\label{HsbPic_eq2}
X\to \mathcal{U}(t)^\dagger X \mathcal{U}(t)
\end{equation}
相当于测量算符的本征矢量$\ket{a}$随时间的演化为
\begin{equation}\label{HsbPic_eq3}
\ket{a; 0}\to \ket{a; t}=\mathcal{U}(t)^\dagger \ket{a} = \E^{\I Ht}\ket{a}
\end{equation}


\subsubsection{连续情况}

连续情况的讨论和离散情况完全相同,只是要求指标$a$的取值范围为指定范围的实数,并将$\sum_a$都替换为$\int \dd a$.\autoref{HsbPic_eq2} 和\autoref{HsbPic_eq3} 依然成立.


\addTODO{时间相关算符的海森堡绘景?}



\subsubsection{海森堡方程}

同\autoref{HsbPic_eq5} 的导出思路相同,我们可以推导海森堡绘景下算符\footnote{这里的算符$X$在薛定谔绘景下不随时间变化,且注意$X$不一定和$H$对易.}的运动方程:
\begin{equation}\label{HsbPic_eq6}
\ali{
    \frac{\dd}{\dd t} \mathcal{U}(t)^\dagger X \mathcal{U}(t) &= \mathcal{U}'(t)^\dagger X \mathcal{U}(t)+\mathcal{U}(t)^\dagger X \mathcal{U}'(x)\\
    &= \I H \E^{\I H t}X\E^{-\I H t}-\E^{\I H t}X\I H\E^{-\I H t}\\
    &= \I [H, \E^{\I H t}X\E^{-\I H t}]
}
\end{equation}
其中$[*, *]$是李括号.

记$\mathcal{U}(t)^\dagger X \mathcal{U}(t)=X(t)$,整理一下\autoref{HsbPic_eq6} ,就得到\textbf{海森堡方程}:
\begin{equation}\label{HsbPic_eq9}
\I\frac{\dd}{\dd t}X(t) = [X(t), H]
\end{equation}


利用\textbf{算符对易性(量子力学)}\upref{ComOpQ},可算出
\begin{equation}\label{HsbPic_eq8}
    [\bvec{x}, \bvec{p}^2] = 2\I \bvec{p}
\end{equation}
和
\begin{equation}\label{HsbPic_eq7}
[\bvec{p}, V(\bvec{x})] = -\I\nabla V(\bvec{x})
\end{equation}
注意\autoref{HsbPic_eq7} 右边的$\nabla$已经作用在$V$上了,整体是一个函数而非算子.

将\autoref{HsbPic_eq8} 和\autoref{HsbPic_eq7} 代入\autoref{HsbPic_eq9} ,令$H=\frac{\bvec{p}^2}{2m}+V(x)$,则有
\begin{equation}\label{HsbPic_eq10}
\leftgroup{
    \frac{\dd}{\dd t} \bvec{x} &= \frac{\bvec{p}}{m}\\
    \frac{\dd}{\dd t} \bvec{p} &= -\nabla V(x)
}
\end{equation}


\autoref{HsbPic_eq10} 就是常用的海森堡运动方程,在狄拉克的\textsl{The Principles of Quantum Mechanics}中写为

\begin{equation}\label{HsbPic_eq11}
\leftgroup{
    \frac{\dd}{\dd t} q_r &= \frac{\partial H}{\partial p_r}\\
    \frac{\dd}{\dd t} p_r &= -\frac{\partial H}{\partial q_r} 
}
\end{equation}














