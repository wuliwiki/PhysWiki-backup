% Linux 创建网络文件夹(sshfs 和 NFS)

这里介绍两种不同的方法。

\subsection{使用 SSHFS}
\begin{itemize}
\item \href{https://www.digitalocean.com/community/tutorials/how-to-use-sshfs-to-mount-remote-file-systems-over-ssh}{参考这里}。
\item 首先确保 \verb`ssh` 和 \verb|sftp| 可以连接。 注意如果使用 \verb|sudo sshfs| 就要确保 \verb|sudo ssh| 和 \verb|sudo sftp| 可以链接, 后者使用的配置文件在 \verb|/root/.ssh/| 而不是 \verb|~/.ssh/|
\item 安装 \verb`sshfs`
\item 注意 \verb`sshfs` 在 WSL 下无效, 好失望… 希望 wsl 2 (据说要 2020 年)可以
\item 创建需要网盘目录, 一般在 \verb`/mnt/` 目录下, 例如 \verb`/mnt/101`
\item \verb`sudo sshfs -o allow_other -o follow_symlinks addis@10.0.2.101:/home/addis /mnt/101` 注意 \verb`addis@` 一定要有, 否则无法进入网盘。 另外一定要用 ip 不能用 ssh 设置的别名
\item 其中 \verb|-o follow_symlinks| 选项如果没有的话, 使用绝对路径的 \verb|symlink| 就会被误以为是本地计算机上的路径。 但如果使用该选项, 远程的 symlink 在本地就会显示为真正的文件夹, 不能只删掉 link。
\item 要弹出网盘, 用 \verb`umount [dir]`, 然后可以手动删掉路径
\item 理论上 \verb|sudo sftp| (注意有没有 \verb|sudo| 区别很大) 可以连接的话 \verb|sudo sshfs| 就是可以的。 如果不 mount 在 \verb|mnt| 目录, 不用 sudo 应该也是可以的。
\item 如果出现问题, 用 \verb|sudo sshfs -o debug ...| 来 debug
\end{itemize}

\subsubsection{开机自动 mount}
\begin{itemize}
\item \href{https://www.linode.com/docs/networking/ssh/using-sshfs-on-linux/}{参考这里}
\item 要重启以后自动 mount 网盘(permanent mount, 两边重启都不怕), \textbf{一定}要先用上面 \verb`sshfs` 命令先临时 mount 一次 (否则会出错!), 成功以后用 \verb`sudo vim /etc/fstab` 在配置文件中最后一行加入 \verb`addis@192.168.137.1:/ /mnt/Miranda fuse.sshfs x-systemd.automount,_netdev,follow_symlinks,identityfile=/home/addis/.ssh/id_rsa,allow_other,default_permissions,reconnect 0 0`, \verb`/home/addis/.ssh/id_rsa`, 其他的什么意思不清楚也不用管. Use \verb`sudo mount -a` to take effect.
\item 现在可以试一试重启, 如果启动失败, 会进入一个安全命令行, 将 \verb`/etc/fstab` 加入的一行删掉即可。 如果成功的话, 重启后进入 \verb`/mnt/folder` 就会自动 mount
\end{itemize}

\subsection{使用 NFS}
\subsubsection{服务器端}
\begin{itemize}
\item Network File System
\item 安装 \verb|sudo apt-get install nfs-kernel-server|
\item \verb|mkdir 文件夹|
\item \verb|sudo vim /etc/exports|, 添加一行 \verb|/home/用户名/文件夹 *(rw,sync,no_root_squash,no_subtree_check)|
\item \verb|sudo service nfs-kernel-server restart|
\item \verb|nfsstat –s| 可以检查 nfs 的信息, 包括 nfs 版本(v1 到 v4) 等
\end{itemize}

\subsubsection{客户端}
\begin{itemize}
\item 在其他 client 机器上, 创建相同的 \verb|文件夹| 文件夹, 然后 \verb|sudo apt-get install nfs-common|
\item 在 client 机器上, \verb|sudo mount -t nfs [host]:/home/用户名/文件夹 ~/文件夹|。 其中 \verb|[host]| 是 ip 地址或者域名。
\item 用 \verb|df -h| 查看是否成功。
\item 要每次开机都自动 mount, 可以 \verb|sudo vim /etc/fstab|, 添加一行 \verb|[host]:/home/用户名/文件夹 /home/用户名/文件夹 nfs|
\item 使用以上方法, symlink 可以正常使用(server 和 client 都使用 ext4 文件系统)。
\item NFS 的默认端口是 2049
\item 要在 Windows 上面挂在 NFS 网盘似乎设置起来比较复杂。 可以用第三方客户端, 例如 \href{https://sourceforge.net/projects/nfsclient/}{NFSClient}。 但似乎不支持挂载, 传文件就把右边拖动到左边即可。
\begin{figure}[ht]
\centering
\includegraphics[width=10cm]{./figures/NFS_1.png}
\caption{NFSClient} \label{NFS_fig1}
\end{figure}
\end{itemize}
