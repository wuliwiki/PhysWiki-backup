% 系统的动量定理、角动量定理、动能定理及其守恒律
\pentry{合力、力矩和、功之和\upref{sysvr0}, 系统的动量、角动量、动能\upref{sysvr1}}
\footnote{本文受安宇等人的《大学物理》启发,部分例子引用自此。}
类似于单个质点的动量\upref{PLaw1}、角动量\upref{AMLaw1}、动能\upref{KELaw1}定理,系统也存在相关的定理,并具有相似的形式,

\subsection{系统的动量定理}
\begin{equation}
\bvec F = \bvec F_{ext} = \dv{\bvec P}{t}
\end{equation}
系统所受合力等于系统动量的变化率。由于合力等于合外力,因此也可以表述为“系统所受合外力等于系统动量的变化率”

\subsection{系统的角动量定理}
\begin{equation}
\bvec \tau = \bvec \tau_{ext} = \dv{\bvec L}{t}
\end{equation}
系统所受力矩和等于系统角动量的变化率。由于力矩和等于外力矩和,因此也可以表述为“系统所受外力矩和等于系统角动量的变化率”

\subsection{系统的动能定理}
\begin{equation}
\delta w =\delta w_{in} + \delta w_{ext} = \dd E_k
\end{equation}
系统中功之和等于系统动能的增量。

\subsection{守恒律}
守恒律可以理解为相关定理的“推论”。以动量为例,当系统不受外力(或合外力为零)时,根据动量定理,$\dv{\bvec P}{t} = \bvec F_{ext}=0$,即$\bvec P = \bvec P_0$,系统总动量不发生改变。此时我们称“系统动量守恒”。

\begin{equation}
\bvec F_{ext} = 0 \Rightarrow \bvec P = \bvec P_0
\end{equation}

动量守恒有一个简单但巧妙的推论,即“内力不改变系统的总动量”。在电影作品中,我们常看到太空中的主角为了前进而往相反方向抛掷物品。虽然抛掷后,主角与被抛掷物品的动量都发生了改变,但这二者的变化等大反向,主角与物品组成的系统的总动量保持不变。

然而日常中,因为摩擦力、重力等外力的广泛存在,这一定理体现得并不十分明显。或许桌球的撞击是一个很好的例子。

同样地,我们有角动量守恒
\begin{equation}
\bvec \tau_{ext} = 0 \Rightarrow \bvec L = \bvec L_0
\end{equation}
与动能守恒
\begin{equation}
\bvec \delta w = 0 \Rightarrow \bvec E_k = \bvec E_{k,0}
\end{equation}

注意,与动量、角动量不同,内力做功可以改变系统的总动能。一个经典的例子是,烟花爆炸时,碎片的额外动能便来自于内力做功(化学能转换为机械动能)。
