% 剩余类环
% keys 同余式|剩余类环|模n同余
% license Usr
% type Tutor

\pentry{环\upref{Ring}}

定义1 (模 $m$ 同余) : 设 $n, n^{\prime}$ 是两自然数,若用正整数 $m$ 去除它们时余数Q相同,则称 $n, n^{\prime}$ 模 $m$ 同余,记作 $n \equiv n^{\prime}(\bmod m)$ 或 $n \equiv n^{\prime}(m)$ 。
由于除数为 $m$ 时余数为 $\{0,1, \cdots, m-1\}$ 中的一个,我们把在除数 $m$ 时余数为 $r$ 的所有自然 数构成的集合记作 $\{r\}_m$ ,称为模 $m$ 的同余类 $\mathrm{Q}$ (或剩余类),于是
\begin{equation}
\{r\}_m=r+m \mathbb{N}=\{r+m k \mid k \in \mathbb{N}\}.
\end{equation}
这就相当于给自然数集分了类:
\begin{equation}
\mathbb{N}=\{0\}_m \cup\{1\}_m \cdots \cup\{m-1\}_m.
\end{equation}

为了方便叙述,定义
定义2(整除的记号):若正整数 $m$ 整除自然数 ${ }^Q n$ ,即用 $m$ 除 $n$ 余数为 0 ,则记 $m \mid n$.
同余性质1: 若 $n \equiv n^{\prime}(m)$ 且 $n \geq n^{\prime}$ ,则 $m \mid\left(n-n^{\prime}\right)$.
因为模 $m$ 同余的两个自然数,它们余数相同,所以它们之差把余数给消掉了,剩下两个 $m$ 的倍 数之差。
同余性质2: 若 $k \equiv k^{\prime}(m)$ 且 $l \equiv l^{\prime}(m)$ ,则 $k+l=k^{\prime}+l^{\prime}(m), k l \equiv k^{\prime} l^{\prime}(m)$.
因为 $k, k^{\prime}$ 余数相同, $l, l^{\prime}$ 余数也相同,那么 $k+l$ 和 $k^{\prime}+l^{\prime}$ 余数当然相同了。清晰点把:
\begin{equation}
\begin{aligned}
& k \equiv k^{\prime}(m) \Rightarrow k=a_1 m+b, k^{\prime}=a_2 m+b, \\
& l \equiv l^{\prime}(m) \Rightarrow l=n_1 m+r, l^{\prime}=n_2 m+r,
\end{aligned}
\end{equation}
所以 $k+l=a_1 m+n_1 m+(b+r), \quad k^{\prime}+l^{\prime}=a_2 m+n_2 m+(b+r)$
所以 $k+l, k+l^{\prime}$ 被 $m m$ 除的余数和 $b+r$ 被 $m$ 除的余数相同,所以 $k+l=k^{\prime}+l^{\prime}(m)$
,同样的 $k l \equiv k^{\prime} l^{\prime}(m)$ 我就不给你证了,习题自用。