% 中国科学技术大学 2016 年考研普通物理 A 考试试题
% keys 中国科学技术大学|考研|物理|2016年
% license Copy
% type Tutor


\textbf{声明}:“该内容来源于网络公开资料,不保证真实性,如有侵权请联系管理员”

\begin{enumerate}
\item 长为$L$、质量为$m$的均质细棒绕太阳作半径为$R$(细棒中心到太阳的距离)的圆周运动。太阳质量为$M$,万有引力常数为$G$。假设细棒总是沿着径向,试计算细棒中心处的张力。(太阳视为质点,此外不要做任何近似)。
\item 两个质量均为$m$的小球置于光滑的水平桌面上,其中心用原长为$d$、倔强系数为$k$的轻弹簧连接。初始时体系静止,然后在$t=0$时刻给其中一个小球突然施加一个冲击力,使其获得垂直于弹簧的速度$v_0$。\\
(1)导出t>0时弹簧长度$l(t)$所满足的微分方程;
(2)在运动过程中弹簧的最短长度等于多少?
(3)导出弹簧的最大长度所满足的方程,并在$v_0$很小的情形下求其近似解。
\item 六根相同的均质细棒通过光滑的铰链连接成正六边形,并置于光滑的水平桌面上。现在,在细棒1的中点$P$处施加一垂直于该棒的冲击力,在极短时间的作用后细棒1获得速度$u$。试求此时与其相对的细棒4的速度$v$。
\item 两个同心薄导体球壳,半径分别为$a$和 $2a$,带电量分别为$Q$和$2Q$,两球壳之间是介电常数为$2\varepsilon_0$的电介质,求该体系的:\\
(1)宏观静电能;\\
(2)总静电能。
\item 两块长方形导体薄板平行放置,并通以等大反向电流$I$,导体板长 $a$,宽$b$,两板间距$d$,$d$远小于$a$和$b$,两板之间充有绝对磁导率分别为$\mu$和/的两种磁介质,体积各占一半,介质-介质界面垂直于导体板。设导体板的绝对磁导率为/o,求
(1)两磁介质中的磁场强度;
(2)两磁介质表面的磁化面电流密度
20分)有两个半径分别为a和b的同轴圆环,分开距离x,分别载流ia和 i。若 a>>b,求
(1)两圆环互感;
(2)两圆环的相互作用力。
指出下列原子辐射跃迁中哪些是电偶极允许跃迁,哪些是禁戒跃迁,并说明后者所违背的选择定则:(1)氦1s2p'P→1s'S。;(2)氦1s2p’P→1s''S,;(3)碳3p3s’P→2p°P;(4)碳2p3s’P→2p°P;(5)钠 2p'4d’Ds → 2p°3p’P。
已知某原子的一个能级为三重结构,且随能量的增加,两个能级间隔之比为3:5,试由朗德间隔定则确定这些能级的S、L和J值,并写出状态符号。
(15分)处于弱磁场中的某碱金属原子,其多重态能级会发生分裂。请问:(1)'P,和'D,态的g因子各为多少?(2)'P和'D,,态各自的磁分裂能级中,哪个态的相邻磁能级间隔大?(3)’P,和'D,态相距最远的磁能级间隔分别为多少?(用4B来表示)

\end{enumerate}