% 集合论(综述)
% license CCBYSA3
% type Wiki

本文根据 CC-BY-SA 协议转载翻译自维基百科\href{https://en.wikipedia.org/wiki/Set_theory}{相关文章}。

\begin{figure}[ht]
\centering
\includegraphics[width=6cm]{./figures/590e7a061f3044c8.png}
\caption{} \label{fig_Set1_1}
\end{figure}
\textbf{集合论}是数学逻辑的一个分支,研究集合,集合可以非正式地描述为对象的集合。尽管任何类型的对象都可以组成一个集合,但集合论——作为数学的一个分支——主要关注那些与整个数学相关的集合。

现代集合论的研究始于19世纪70年代,由德国数学家理查德·德德金德和乔治·康托尔发起。特别是,乔治·康托尔通常被认为是集合论的创始人。在这个早期阶段研究的非形式化系统被称为朴素集合论。在朴素集合论中发现悖论(如罗素悖论、康托尔悖论和布拉利-福尔蒂悖论)之后,20世纪初提出了各种公理化系统,其中泽梅洛–弗兰克尔集合论(无论是否包含选择公理)仍然是最著名和最研究的。

集合论通常被用作整个数学的基础系统,特别是以泽梅洛–弗兰克尔集合论与选择公理的形式。除了其基础性作用外,集合论还提供了一个框架,用于发展数学中的无穷大理论,并在计算机科学(如关系代数理论)、哲学、形式语义学和进化动力学等领域有着广泛的应用。它的基础性吸引力、与悖论的关系、以及对无穷大的概念及其多重应用的影响,使得集合论成为逻辑学家和数学哲学家关注的主要领域之一。当代集合论的研究涵盖了广泛的主题,从实数线的结构到大基数的一致性研究。