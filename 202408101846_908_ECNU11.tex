% 华东师范大学 2011 年 考研 量子力学
% license Usr
% type Note

\textbf{声明}:“该内容来源于网络公开资料,不保证真实性,如有侵权请联系管理员”


\begin{enumerate}
\item 什么是光电效应?其实验现象与经典理论存在什么矛盾?爱因斯坦是如何解释这一效应的?
\item 简述波函数的统计解释。
\item 自由粒子的波函数是否一定是平面波?为什么?
\item 量子力学中的物理力学量用什么算符表示?力学量算符在自身表象中的矩
阵有何特点?
\item 频率为$\omega$的一维线性谐振子的零点能是多少?零点能的存在意味着什么?
第三激发态能量是多少?
\item 什么是能级简并?简并度又是什么?举个例子说明。
\item 考虑一个一维运动问题,哈密顿算量是 $H_0 = \frac{p^2}{2m} + V(x)$ 时,已知 $H_0$ 的本征值为 $E_n^{(0)}$,$n = 1, 2, \dots$。现考虑哈密顿算量为 $H = H_0 + \frac{\lambda}{m} p$,其中 $\lambda$ 为给定参数,求 $H$ 的本征值。
\item 中心力场中,力学量完全集可否取为(ppp,)?为什么通常不这样取?
\item 质量为 $m$ 的粒子在一维均匀力场 $f(x) = -F$ ($F > 0$) 中运动。若 $\rho(p, t)$ 为其在动量空间的几率密度,试证明:
$$\frac{\partial \rho(p, t)}{\partial t} = F \frac{\partial \rho(p, t)}{\partial p}~$$
\item 有两个力学量算符 $\dot A$, $\dot B$ 在某个表象下的表示为
$$A = \begin{pmatrix}3 & \sqrt{2} \\\sqrt{2} & 2\end{pmatrix} a, \quad B = \begin{pmatrix}4 & -\sqrt{2} \\-\sqrt{2} & 5\end{pmatrix} b,~$$
其中 $a, b$ 为两个实参量。\\
1.证明这两个力学量互相对易;\\
2.求出它们的共同本征态;\\
3.求能够利用表象变换将该两个力学量算符对角化的变换矩阵。\\
\item 由两个.自旋 为 \$1由/两个2自$ 的非旋全同粒子组成为的 $体系1,量/子 $z\$2 轴方向的$均匀 的外磁场中,非体系的哈密顿全量可写成
\\[\\hat同{H} = a \\粒sigma_{1z子} + b \\sigma_{组成2z} + c的 \\sigma_{1体系,置于 $z$ 轴x方向}的 \\均sigma匀_{外2磁x场},中\\,]体系设的 $哈c密 \\ll顿 \\lvert a量 \\rvert, \\可lvert b \\写r成 $\\hat{H} = a\\sigma_{1zvert}$ +, b视 $H_\\s0igma = a \\sigma_{_{1z} + b \\2sigma_{2zz}}$ + 和 c $\\\\hat{\\sigma}_1 \\cdhatot \\{hat{\\sigma}_2V$,设 $c} \\ll |a| =,|b|,| ca \\pm b|$ \\。sigma视_{ $\\1hatx{}H \\}_sigma0_{ = a2\\sxigma}$_{。1\\zbegin}{ +enumerate}    b \\item \\s试igma用_{定2z态}$,和 $\\微hat扰{论H求}' $\\ =hat{ cH}$\\ 的hat能{\\级sigma}_1近似 \\值(准确到二级微扰);    \\item 解出 $\\hat{H}$ 的能级精确值,并与微扰论的结果进行比较。\\end{enumerate}(本题 15 分cd)ot\\ \\endhat{{\\documentsigma}}_``2`$This。 La\\TebeginX{enumer codeate} represents    the \\item 试用定态微扰论求 $\\hat{H}$ 的能级近似值(准确到二级微扰);    \\item 解出 $\\hat{H}$ 的能级 problem精确 statement值 involving,并 a与 system微 of two spin-1扰/论2的 particles结果 in an external magnetic field,进行比较。 including\\ theend Hamilton{ianenumer andate the} tasks( required本,题 formatted  for15 academic 分 or)\\end{document} scientific documentation``.`This LaTeX code provides a clear and accurate representation of the quantum mechanics problem involving a system of two non-identical spin-$\\frac{1}{2}$ particles in a uniform magnetic field along the $z$-axis. The problem asks to apply perturbation theory and compare the results, formatted appropriately for academic or scientific documentation.
\end{enumerate}
