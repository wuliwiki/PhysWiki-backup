% 函数的性质与变换(高中)
% keys 函数|变换|平移|旋转|伸缩|单调|对称|奇偶性|初等函数|周期
% license Xiao
% type Tutor

\begin{issues}
\issueDraft
\end{issues}

\pentry{函数(高中)\nref{nod_functi}}{nod_6f70}

在前面的学习中,我们已经接触了函数的概念,以及如何通过复合运算来将多个函数结合起来。现在,我们将深入探讨一个函数的各种性质,比如它的单调性、奇偶性和周期性。这些性质是理解函数行为和性质的关键,它们能够帮助我们更好地分析函数在不同情况下的表现。

然后,我们还会研究函数的变换,比如平移、缩放和对称等操作。当我们改变函数的形式,比如平移、拉伸或对称,它们的图像和性质也会跟着变化,就像镜子里的影像随手势而动。函数的变换不仅会改变函数的图像,还会影响它的性质和表达方式,这让我们可以用更灵活的方式去理解和处理函数。

最后,我们会大概介绍一下高中阶段需要掌握的几种主要函数类型。这些函数在数学和实际应用中扮演着重要角色,它们构成了函数世界的一大部分。这里会把它们作为两部分来整体介绍,后面的学习中会具体介绍每一个函数的细节。

\subsection{函数的变换}

在实际生活中,变换的概念无处不在。比如,调整照片的大小、改变音乐的速度、甚至是地图应用中缩放和旋转视图,这些操作都与函数变换的原理息息相关。

函数的变换本质上是空间的变换,也就是说函数本身的性质没有变,改变的是函数所在的坐标系。但就像驶远的汽车,从你的视角是汽车开远了,而从车的视角,假设车没动的话就是你在远离。坐标系的改变后,如果认为坐标系不变,那么就相当于是函数在进行变换了。尽管这与我们现实生活中的经验不太一样,毕竟我们想要移动一个画框在墙上的位置一般不会把整面墙移动走,但采用“变换的本质是变换坐标系”这个视角去看待函数的变换问题,不仅比较容易理解,而且在处理时也不用去记一些口诀,也更不容易出错。

\subsubsection{平移}

函数图像的\textbf{平移(Translation)}变换是我们经常用到的一种操作。简单来说,就是将一个函数的图像按照某个方向进行移动,但不改变它的形状。尽管初中时就已经接触到了,或者说平时在生活中我们经常移动某个物品。为了更细致地理解这个过程,下面的内容可以拿两张纸自己比划一下试试,这个例子尽管简单,但对后面的其他变换的理解有好处。

\begin{example}{观察如何“向右平移函数”}
\begin{enumerate}
\item 你有两张纸——上面一张纸代表函数的图像,而下面一张纸代表坐标系。
\item 把上面的纸放在下面的纸上,这时候两张纸是对齐的,图像在坐标系的“原位置”。严格来讲,可以认为两张纸的左下角是原点。
\item 现在,如果想把函数图象向右移动,就需要把上面那张代表图像的纸向右移动(记得不要旋转或改变纸的形状),你会发现图像虽然位置变了,但它本身没有改变——形状还是一样的,只是相对于坐标系的位置发生了移动。
\item 此时,我们可以认为上面的那张纸代表的是原本的坐标系,而下面那张纸代表的是移动后的坐标系。于是,移动后的那张纸的原点出现在了在原本坐标系那张纸的原点的左侧,也就是说,图象向右移动,相当于坐标系向左移动。
\end{enumerate}
\end{example}

同理,如果想要向左、上、下平移函数图像,就需要把坐标系向右、下、上移动。如果要斜着移动,可以把移动的过程拆解成先水平方向移动,再垂直方向移动,也就是移动两次。

理解了“图像与坐标系相对移动”的关系后,接下来看看如何在函数表达式上反映出这种平移变换。

\begin{example}{用表达式描述“函数向右平移$c$个单位”}

假设$(X_0,Y_0)$和$(X_1,Y_1)$分别代表原坐标系和变换后坐标系上的某一点。根据上面的观察,移动后的原点在移动前的原点左侧$c(c>0)$个单位。也就是说,旧的坐标原点的坐标,从$(0,0)$变成了$(c,0)$。而由于整张纸上的所有点的移动与原点一致,因此他们都符合这个关系,即$(X_0,Y_0)$变成了$(X_0+c,Y_0)$,而$(X_1,Y_1)$代表的就是移动后的点,于是:
\begin{equation}
\begin{cases}
X_1=X_0+c\\
Y_1=Y_0
\end{cases}\iff
\begin{cases}
X_0=X_1-c\\
Y_0=Y_1
\end{cases}~.
\end{equation}

现在假设原坐标系上的点  $(X_0, Y_0)$  在函数图像上,也就是说  $Y_0 = f(X_0)$ 。将  $X_0$  和  $Y_0$  的表达式带入平移后的关系式中,可以得到  $Y_1 = f(X_1 - c)$ 。这正好就是移动后的函数图像上对应的点。因此,我们得知,函数向右平移$c$个单位的表达式描述为:
\begin{equation}
y=f(x)\Rightarrow y=f(x-c)~.
\end{equation}
\end{example}

推荐你自己试一试,根据相同的方法可以得到其他情况下函数平移的表达式描述:
\begin{itemize}
\item 函数向左$c(c>0)$个单位为:$y=f(x)\Rightarrow y=f(x+c)$
\item 函数向上$b(b>0)$个单位为:$y=f(x)\Rightarrow y-b=f(x)\Rightarrow y=f(x)+b$
\item 函数向下$b(b>0)$个单位为:$y=f(x)\Rightarrow y+b=f(x)\Rightarrow y=f(x)-b$
\end{itemize}

很多人会称上面的变换规律为“左加右减,上加下减”,这句话是在函数视角下的描述。一般情况下看似没有问题,但涉及到与伸缩及旋转等复合时,便可能引起错误。在坐标系视角下,应为“左加右减,下加上减”,水平方向上,对$x$进行操作,而垂直方向上,则对$y$进行操作。当然,如果掌握原理,随时可以列出变化前后的关系,然后再代入到表达式中推导。

同时,可以观察到,如果认为$c$或$b$可以取负值,那么取负值时则意味着相反方向平移绝对值单位长度,这也与学习负数时的概念一致。

\subsubsection{伸缩}

在了解了平移变换之后,\textbf{伸缩(Scaling)}变换的推导就信手拈来了。

以“把函数的图像在$y$方向拉伸$A(A>1)$倍”为例,这意味着每个点的$y$值都要扩大$A$倍。如果平移对应的是原点的变化,这里则意味着原来$1$单位的长度,变化为$A$个单位长度。

也就是这时有关系:
\begin{equation}
\begin{cases}
X_1=X_0\\
Y_1=AY_0
\end{cases}\iff
\begin{cases}
X_0=X_1\\
\displaystyle Y_0={Y_1\over A}
\end{cases}~.
\end{equation}

将  $X_0$  和  $Y_0$  的表达式带入平移后的关系式中,可以得到  ${Y_1\over A} = f(X_1 )$ ,即:

\begin{equation}
y_1=Af(x_1)~.
\end{equation}

根据相同的方法可以得到其他情况下函数伸缩的表达式描述\footnote{这里请忽略“压缩$A$倍”之类的语病。}:

\begin{itemize}
\item 函数在$y$方向压缩$A(A>1)$倍为:$\displaystyle y=f(x)\Rightarrow y={1\over A}f(x)$
\item 函数在$x$方向拉伸$B(B>1)$倍为:$y=f(x)\Rightarrow y=f(Bx)$
\item 函数在$x$方向压缩$B(B>1)$倍为:$\displaystyle y=f(x)\Rightarrow y=f({x\over B})$
\end{itemize}

与平移类似,可以观察到,如果认为$A$或$B$为任意正数,那么取倒数时则意味着伸缩关系的互换,即拉伸$A$倍意味着压缩$\displaystyle 1\over A$倍。这也与学习倒数时的概念一致。

需要注意,与平移不同的是,伸缩变换前后,原本在$y$方向上距离为$d$的两个点,现在会变为$Ad$。也就是说,伸缩前后两个点的距离会变化。平移和伸缩,就像初中证明时全等与相似的关系。

\subsubsection{*旋转}

这一部分内容需要使用三角函数的知识,如果不太熟悉可以去\addTODO{添加初中部分}回顾。

\textbf{旋转(Rotation)}


就像复合函数可以多个函数一层层复合一样,这些变换也可以依次加在同一个函数上,最终会得到一个新的表达式。在处理时,如果不太熟悉可以把每一步独立考虑,最后再整理就可以了。下面试一试。
\begin{exercise}{把函数$y=f(x)$先在$x$方向上向右平移$a$个单位,再在$y$方向上向下平$b$个单位,先在$x$方向上压缩为原来的$1\over5$,求得到的新函数的表达式。}

\end{exercise}

\addTODO{抛物线的例子: 事实上抛物线只有一个形状,无论如何水平拉伸,竖直拉伸, 都等效于等比例缩放。这也是为什么抛物线的离心率只有一个值(离心率决定圆锥曲线的形状)。}
\subsubsection{复合变换}

注意顺序

总结一下,高中涉及到的函数变换主要针对的是函数图象而言的。旋转和平移统称为\textbf{欧式变换(Euclidean Transformation,或刚体变换,Rigid Transformation)},它是一种保持几何对象的形状和大小的变换,也就是点之间的距离和角度都保持不变。在欧式变换的基础上再加上伸缩,称为\textbf{相似变换(Similarity Transformation)}。相似变换只保持角度保持不变,而距离可以成比例变化。也就是保留了图形比例关系,因此也认为它保留了物体的形状。在代数的视角,相似变换包含的这三类统称为\textbf{线性变换(Liner Transformation)}。

另外,高中不太会接触到的\textbf{仿射变换(Affine Transformation)}保留了直线的性质和平行的关系,但角度和长度不再保持不变;\textbf{摄影变换(Projective Transformation)}则只保留了直线的性质,甚至允许平行线在变换后不再平行。每一类变换都是对前一类的扩展,加入了更多的几何操作自由度。摄影变换是最广泛的,允许最自由的变换;欧式变换则是最保守的,严格保持几何形状和大小。

其实函数的变换包含的范畴非常广泛:除了上面提到的用于分析和操作几何图形及其在空间中的关系的几何变换;频域分析方面,傅里叶变换、拉普拉斯变换、Z变换等变换可以将问题从时域转换到频域,从而揭示出信号和系统中隐藏的特征。当然现在主要先研究高中涉及的部分;学有余力的话,也可以了解几何变换的其他部分,这对理解解析几何的内容会很有帮助;而频域分析的部分在大学阶段才可能会接触到。

\subsection{函数的性质}\label{sub_HsFunC_1}

函数具有一些性质,有一些在高中会接触到,有一些不会接触到。
以后我们会看到一些用\enref{极限}{Lim}和\enref{导数}{Der}描述的性质。 例如 % \addTODO{链接}
, 可导。
\subsubsection{零点}

函数的零点是指使函数值为零的自变量 $x$,从几何角度来看,函数 $f(x)$ 的零点就是其图像与 $x$ 轴的交点。一般情况下\footnote{确实有一些函数的零点是连续的,例如 $f(x) = 0$ 以及由此复合得到的函数。},零点是孤立的,而零点之外的点通常连续构成一个区间。在这些区间中,函数往往表现出某种特性(例如取正值或负值),通常,函数在零点的两侧会发生符号变化\footnote{从正值变为负值,或从负值变为正值}。而零点则作为这些区间的端点,成为函数符号变化的分界线。因此,判断某个点是否属于某个区间时,通常需要将其与零点进行比较。事实上,函数的零点之所以具有特殊意义,正是因为它常常对应某种边界条件,标志着状态的转变或某种变化的界限,比如物体的平衡点以及工程问题中的最优设计或操作条件等。

\begin{definition}{零点}
对于函数 $f(x)$,使得 $f(x) = 0$ 成立的$x$的值,称为 $f(x)$ 的\textbf{零点(zero point)}。
\begin{equation}
x_0 \in \{ x \mid f(x) = 0 \}~.
\end{equation}
\end{definition}

函数的零点与方程有密切联系,零点沟通了函数与方程。任何方程总可以通过移项转换成形如 $f(x) = 0$ 的形式\footnote{高中阶段只涉及一元函数,因此此处指的是一元方程},而这个表达式所描述的,就是函数 $f(x)$ 的零点。这一点在初中学习时,相信你就已经感受过了。

两个函数的交点也可以用零点的形式来表示。假如要求解两个函数 $f(x)$ 和 $g(x)$ 的交点,其实就是寻找满足 $f(x) = g(x)$ 的点。通过设 $F(x) = f(x) - g(x)$,可以将问题转化为求 $F(x)$ 的零点,这样 $F(x)$ 的零点对应的就是 $f(x)$ 和 $g(x)$ 的交点。特别地,当函数 $f(x)$ 取某个固定值 $a$ 时,这个问题可以看作是 $g(x) = a$ 的特殊情况。此时我们设 $F(x) = f(x) - a$。根据函数的平移性质,这相当于将函数 $f(x)$ 向下平移 $a$ 个单位(如果 $a$ 是负值,就向上平移 $|a|$ 个单位)。因此,函数取某个值的点,其实就是平移后的零点。自然,$f(x)$ 的零点就是 $g(x) = 0$ 的特殊情况了。

如果一个零点在方程中出现多次,称为重根。例如,对于 $f(x) = (x - 1)^2$,$x = 1$ 是零点,但它是一个重数为 2 的零点。重数的概念在多项式函数的分析中很重要,零点的重数还与函数在该点的图像行为有关,比如多项式曲线在某些重的重根处“接触”而不是“穿过” $x$ 轴。这是很关键的一点,在不等式部分会着重介绍。

尽管,在高中阶段涉及到的零点通常可以通过代入某些特殊值来求解,但有些时候,并不是需要求解某个具体的零点值,这时只要能够证明在某个区间上存在零点就可以了。于是需要使用介值定理(也称为零点存在定理)。

\begin{definition}{介值定理}
如果连续函数 $f(x)$ 在区间 $[a, b]$ 上满足 $f(a)$ 和 $f(b)$ 符号相反,那么在 $[a, b]$ 上至少存在一个零点。
\end{definition}

在大学阶段,还会学习与介值定理和零点相关的一组定理,称为“中值定理”,他们构成了非常强大的研究工具。

一种求解零点的方法称为“二分法”

在区间 $[a, b]$ 上,如果 $f(a)$ 和 $f(b)$ 符号相反,那么通过不断将区间二分,逐步逼近零点。这种方法虽然简单,但很有效。

当然,除此之外还有“牛顿法”等一系列方法,它们是使用计算机求解方程的重要工具,但在高中并不涉及。


一阶导数的零点:如果 $f'(x) = 0$,那么 $x$ 可能是函数的极值点。通过结合导数符号的变化,可以进一步判断该点是极大值、极小值还是平稳点。二阶导数的零点:二阶导数的零点往往对应于函数的拐点,也就是曲率改变的点。

\subsubsection{单调性}



\begin{definition}{单调性}
设$f(x)$是定义在$D$上的函数,若在$I\subseteq D$上,对$\forall x_1,x_2\in I,x_1< x_2$,函数均满足:
\begin{equation}
f(x_1)<f(x_2)~.
\end{equation}
则称函数$f(x)$在区间$I$上\textbf{单调递增(monotonically increasing})\footnote{此处采取高中教材上的定义。其实,满足这个条件时称作\textbf{严格单调递增 (Strictly increasing)},而单调递增则是指函数满足$f(x_1)\leq f(x_2)$时。递减也相同。这个概念会在大学阶段区分,目前给出作为提醒。},或函数$f(x)$在区间$I$上是\textbf{增函数};若满足
\begin{equation}
f(x_1)>f(x_2)~.
\end{equation}
则称函数$f(x)$在区间$I$上\textbf{单调递减(monotonically decreasing}),或函数$f(x)$在区间$I$上是\textbf{减函数}。

若函数在定义域上单调,则称为\textbf{单调函数 (monotonic functions)}。
\end{definition}



\subsubsection{对称性}

对称性分为两种,一种是轴对称性,一种是中心对称性,这两个性质在初中就有接触过。

\begin{definition}{中心对称}
中心对称
\end{definition}

\begin{definition}{轴对称}
轴对称
\end{definition}

注意这里要区分轴对称性和反函数的区别。

有两个比较特殊的对称性称为奇偶性。

\begin{definition}{奇偶性}
设函数$f(x)$定义在$D$上,且$D$是关于$0$对称的。若对任意的$x\in D$,有:
\begin{itemize}
\item $f(x)=f(-x)$,则称$f(x)$是偶函数。
\item $f(x)=-f(-x)$或$-f(x)=f(-x)$,则称$f(x)$是奇函数。
\end{itemize}
\end{definition}

\subsubsection{周期性}

在生活中,很多事情都是有规律的,比如每天日出日落、四季轮回。数学中用周期性来描述这种“规律”。周期性的函数图象好比一首不断循环的旋律,它遵循着固定的步调,过一段时间就会“回到原点”,再继续以同样的方式变化。

\begin{definition}{周期}
设函数$f(x)$定义在$D$上,且满足
\begin{equation}
\forall x\in D,f(x+T)=f(x)~.
\end{equation}
则称,$T$是函数的一个周期。
\end{definition}


周期函数具有一些特性:
\begin{itemize}
\item 周期函数的图像会在每个周期$T$内重复。无论在$x$轴上平移多少个周期,函数的形状和取值都会保持不变。
\item 两个周期分别为$T_1,T_2$的周期函数,只有满足它们的周期之比$k={T_1\over T_2}$为有理数,即$k={p\over q},p,q\in\mathbb{N}^*$时,他们的和才是周期函数,周期为$T=pT_2=qT_1$。周期相同可以认为是二者周期之比为$1$的特殊情况,此时和仍然是周期函数且周期为原周期。
\item 周期函数在定义域内一定不是单调的,因此周期函数也没有反函数。
\end{itemize}

高中阶段涉及的周期函数主要是两类:一类是抽象函数,也就是不给出表达式,然后利用周期性的特性来等量替换;另一类是三角函数,这将在\enref{三角函数}{HsTrFu}的部分详细讲解。



还有一些性质是高中不会涉及到的,此处给出:
\begin{itemize}
\item \enref{连续性}{contin}, 一致连续
\end{itemize}



\subsection{特殊的函数}

在高中阶段会涉及到的两种特殊的函数包括初等函数和分段函数。

\subsubsection{初等函数}

高中研究的函数都是初等函数。初等函数指的是由基本初等函数经过基本运算(加减乘除)以及复合形成的函数。

初等函数之所以被称为初等函数就是因为它的性质很好,

基本初等函数:
\begin{itemize}
\item 常值函数
\item 幂函数
\item 指数函数
\item 对数函数
\item 三角函数
\end{itemize}

\subsubsection{分段函数}

绝对值函数

取整函数

狄利克雷函数