% 北京大学 1999 年 考研 量子力学
% license Usr
% type Note

\textbf{声明}:“该内容来源于网络公开资料,不保证真实性,如有侵权请联系管理员”



1. (25分) 简要回答以下问题\\
(a) 简述“不确定原理”(测不准关系),说明其意义。\\
(b) 试述“态的叠加原理”,说明其意义。\\
(c) 全同粒子有什么特性?对波函数有什么要求?举例说明之。\\

2. (10分) 已知在 $\hat{J}$ 和 $\hat{L}_z$ 的共同表象中, $\hat{L}_x = \frac{\hbar}{\sqrt{2}} \begin{pmatrix} 
0 & 1 & 0 \\
1 & 0 & 1 \\
0 & 1 & 0 
\end{pmatrix}$, 试求其本征值和本征函数,并写出在自身表象中的矩阵表示。

3. (15分) 一个原子在$z$向磁场$B$中,除了能级的塞曼分裂外,还受到 $\Delta \hat{H}_d = \frac{\mu_B^2}{2c^2a_0} B^2 n^2 \sin^2 \theta$ (c.g.s) 的微扰,\\
(a)] 已知H原子基态, $\Psi (1s) = \frac{1}{\sqrt{\pi a_0^3}} e^{-r/a_0}$, 求一级微扰能 $ \Delta E_d$。\\
(b)] 估计这项修正的量级(设B=10T高斯),与塞曼分裂( $\mu_B B$ 量级)比较。\\
(c)] 分析这个修正的物理意义。\\

