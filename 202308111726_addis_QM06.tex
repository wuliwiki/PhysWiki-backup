% 相互作用绘景
% keys 动力学|微扰近似
% license Xiao
% type Tutor
%\pentry{量子力学的基本原理\upref{QMPrcp}, 薛定谔绘景和海森堡绘景\upref{HsbPic},时间演化算符\upref{TOprt}}
 \pentry{薛定谔绘景和海森堡绘景\upref{HsbPic}}


在本章节中,使用上角标为$(s)$代表薛定谔绘景,上角标为$(I)$代表相互作用绘景,例如$H^{(s)}$,$H^{(I)}$。

\begin{definition}{}

不同于海森堡绘景和薛定谔绘景,在相互作用绘景里,态矢和算符都随时间而改变。薛定谔绘景中的哈密顿量可以分割为:
\begin{equation}
H^{(S)}=H^{(S)}_0+V^{(S)}~.
\end{equation}
分别定义$\mathcal U^{(S)}(t)$为$H^{(S)}$对应的时间演化算符,$\mathcal U^{(S)}_0(t)$为$H^{(S)}_0$对应的时间演化算符,具体定义方法参考\href{https://wuli.wiki/online/TOprt.html}{时间演化算符}。

定义:
\begin{equation}
\begin{aligned}
|\psi^{(I)}(t)\rangle&=\mathcal U^{(S)}_0(t)^\dagger|\psi^{(S)}(t)\rangle=\mathcal U^{(S)}_0(t)^\dagger\mathcal U^{(S)}(t)|\psi^{(S)}(0)\rangle~, \\
A^{(I)}(t)&=\mathcal U^{(S)}_0(t)^\dagger A^{(S)}(t)\mathcal U^{(S)}_0 (t)~.
\end{aligned}
\end{equation}
上式中$A$为任意算符。容易验证,在如此定义的相互作用绘景中算符平均值与在薛定谔绘景中相同。

\end{definition}

考虑时间演化算符满足:
\begin{equation}
\begin{aligned}
&\I\hbar\frac{\partial}{\partial t}\mathcal{U}^{(S)}(t) = H^{(S)}(t)\mathcal{U}^{(S)}(t) \\
&\I\hbar\frac{\partial}{\partial t}\mathcal{U}_0^{(S)}(t) = H^{(S)}_0(t)\mathcal{U}_0^{(S)}(t)~.
\end{aligned}
\end{equation}

计算相互作用绘景中算符与态矢的演化方法,(为了简洁,在推导算符随时间的变化时省略(t),不应该忘记的是,以下涉及的所有算符都是含时的。):
\begin{equation}
\begin{aligned}
\frac{\partial}{\partial t}A^{(I)}&=\frac{\partial}{\partial t}\left(\mathcal U^{(S)\dagger}_0 A^{(S)}\mathcal U^{(S)}_0 \right)\\
&=\left(\frac{\partial}{\partial t}\mathcal U^{(S)\dagger}_0\right) A^{(S)}\mathcal U^{(S)}_0 +\mathcal U^{(S)\dagger}_0 \left(\frac{\partial}{\partial t}A^{(S)}\right)\mathcal U^{(S)}_0+\mathcal U^{(S)\dagger}_0 A^{(S)}\left(\frac{\partial}{\partial t}\mathcal U^{(S)}_0\right)\\
&=-\frac{1}{\I\hbar}\mathcal U^{(S)\dagger}_0 H^{(S)}_0A^{(S)}\mathcal U^{(S)}_0 +\frac{1}{\I\hbar}\mathcal U^{(S)\dagger}_0 A^{(S)}H^{(S)}_0\mathcal U^{(S)}_0+\mathcal U^{(S)\dagger}_0 \frac{\partial}{\partial t}\left(A^{(S)}\right)\mathcal U^{(S)}_0 \\
&=-\frac{1}{\I\hbar}\left(\mathcal U^{(S)\dagger}_0 H^{(S)}_0 \mathcal U^{(S)}_0\right)\left(\mathcal U^{(S)\dagger}_0 A^{(S)}\mathcal U^{(S)}_0\right) +\frac{1}{\I\hbar}\left(\mathcal U^{(S)\dagger}_0 A^{(S)}\mathcal U^{(S)}_0\right)\left(\mathcal U^{(S)\dagger}_0 H^{(S)}_0\mathcal U^{(S)}_0\right)+\left(\frac{\partial}{\partial t}A\right)^{(I)} \\
&=-\frac{1}{\I\hbar} H^{(I)}_0  A^{(I)}+\frac{1}{\I\hbar} A^{(I)} H^{(I)}_0+\left(\frac{\partial}{\partial t}A\right)^{(I)} \\
&=\frac{1}{\I\hbar}\left[A^{(I)},H_0^{(I)}\right]+\left(\frac{\partial}{\partial t}A\right)^{(I)}~.
\end{aligned}
\end{equation}

\begin{equation}
\begin{aligned}
\frac{\partial}{\partial t}|\psi^{(I)}(t)\rangle&=
\frac{\partial}{\partial t}\left(\mathcal U^{(S)}_0(t)^\dagger\mathcal U^{(S)}(t)\right)|\psi^{(S)}(0)\rangle \\
&=\left(\frac{\partial}{\partial t}\mathcal U^{(S)}_0(t)^\dagger\right)\mathcal U^{(S)}(t)|\psi^{(S)}(0)\rangle+\mathcal U^{(S)}_0(t)^\dagger\left(\frac{\partial}{\partial t}\mathcal U^{(S)}(t)\right)|\psi^{(S)}(0)\rangle \\
&=-\frac{1}{\I\hbar}\mathcal U^{(S)}_0(t)^\dagger H^{(S)}_0(t)\mathcal U^{(S)}(t)|\psi^{(S)}(0)\rangle+\frac{1}{\I\hbar}\mathcal U^{(S)}_0(t)^\dagger H^{(S)}(t)\mathcal U^{(S)}(t) |\psi^{(S)}(0)\rangle \\
&=\frac{1}{\I\hbar}\mathcal U^{(S)}_0(t)^\dagger V^{(S)}(t)\mathcal U^{(S)}(t) |\psi^{(S)}(0)\rangle \\
&=\frac{1}{\I\hbar}\left(\mathcal U^{(S)}_0(t)^\dagger V^{(S)}(t)\mathcal U^{(S)}_0(t)\right)\left(\mathcal U^{(S)}_0(t)^\dagger\mathcal U^{(S)}(t) |\psi^{(S)}(0)\rangle\right) \\
&=\frac{1}{\I\hbar}V^{(I)}(t)|\psi^{(I)}(t)\rangle~.
\end{aligned}
\end{equation}

由此可见,在相互作用绘景下算符与态矢都随时间演化,且算符和态矢分别依据哈密顿量的两部分($H_0^{(I)}$和$V^{(I)}$)演化。

同样也可以计算相互作用绘景下的时间演化算符$\mathcal U^{(I)}(t)=\mathcal U^{(S)}_0(t)^\dagger\mathcal U^{(S)}(t)$随时间的演化。:

\begin{equation}
\begin{aligned}
\frac{\partial}{\partial t}\mathcal U^{(I)}(t)&=
\frac{\partial}{\partial t}\left(\mathcal U^{(S)}_0(t)^\dagger\mathcal U^{(S)}(t)\right)\\
&=\left(\frac{\partial}{\partial t}\mathcal U^{(S)}_0(t)^\dagger\right)\mathcal U^{(S)}(t)+\mathcal U^{(S)}_0(t)^\dagger\left(\frac{\partial}{\partial t}\mathcal U^{(S)}(t)\right) \\
&=-\frac{1}{\I\hbar}\mathcal U^{(S)}_0(t)^\dagger H^{(S)}_0(t)\mathcal U^{(S)}(t)+\frac{1}{\I\hbar}\mathcal U^{(S)}_0(t)^\dagger H^{(S)}(t)\mathcal U^{(S)}(t)  \\
&=\frac{1}{\I\hbar}\mathcal U^{(S)}_0(t)^\dagger V^{(S)}(t)\mathcal U^{(S)}(t)  \\
&=\frac{1}{\I\hbar}\left(\mathcal U^{(S)}_0(t)^\dagger V^{(S)}(t)\mathcal U^{(S)}_0(t)\right)\left(\mathcal U^{(S)}_0(t)^\dagger\mathcal U^{(S)}(t) \right) \\
&=\frac{1}{\I\hbar}V^{(I)}(t)U^{(I)}(t)~.
\end{aligned}
\end{equation}

