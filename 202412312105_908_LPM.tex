% 加布里埃尔·李普曼(综述)
% license CCBYSA3
% type Wiki

本文根据 CC-BY-SA 协议转载翻译自维基百科\href{https://en.wikipedia.org/wiki/Gabriel_Lippmann}{相关文章}。

乔纳斯·费迪南德·加布里埃尔·李普曼(1845年8月16日-1921年7月12日)是一位出生于卢森堡的法国物理学家和发明家,因其“基于干涉现象的色彩摄影再现方法”而于1908年获得诺贝尔物理学奖。[2]
\subsection{早年生活和教育}
\begin{figure}[ht]
\centering
\includegraphics[width=6cm]{./figures/2135552cea5927ab.png}
\caption{1908年的李普曼} \label{fig_LPM_1}
\end{figure}
加布里埃尔·李普曼于1845年8月16日出生在卢森堡的博讷沃伊(卢森堡语:Bouneweg)。当时,博讷沃伊是霍勒里奇(卢森堡语:Hollerech)市镇的一部分,这个地方常被认为是他的出生地。(如今,博讷沃伊和霍勒里奇都是卢森堡市的区。)他的父亲伊萨耶(Isaïe)是一位出生在梅斯附近恩内里的法国犹太人,经营着位于博讷沃伊的旧修道院内的家族手套制造生意。1848年,李普曼一家搬到了巴黎,在那里李普曼最初由母亲米里亚姆·罗斯(Lévy)辅导,之后进入拿破仑中学(现为亨利四世中学)。据说他是一个注意力不集中但富有思考的学生,对数学特别感兴趣。1868年,他进入巴黎高等师范学院学习,但未通过能够使他进入教师职业的聚集考试,而是选择专攻物理学。1872年,法国政府派他前往海德堡大学,在古斯塔夫·基尔霍夫的鼓励下,他专攻电学,并于1874年获得“优等”博士学位。李普曼于1875年回到巴黎,继续学习,直到1878年成为索邦大学的物理学教授。在索邦大学,他教授声学和光学课程。
\subsection{职业生涯}  
\begin{figure}[ht]
\centering
\includegraphics[width=6cm]{./figures/aac5aa28858d1eac.png}
\caption{李普曼教授在索邦大学物理研究实验室(索邦大学图书馆,NuBIS)} \label{fig_LPM_2}
\end{figure}
李普曼在多年的职业生涯中对多个物理学领域做出了几项重要贡献。
\subsubsection{毛细电计}
\begin{figure}[ht]
\centering
\includegraphics[width=6cm]{./figures/1ca13b737ad1efd4.png}
\caption{李普曼电计(1872年)} \label{fig_LPM_3}
\end{figure}
李普曼的早期发现之一是电现象与毛细现象之间的关系,这使他能够发明一种敏感的毛细电计,后来被称为李普曼电计,并被用于第一台心电图(ECG)机。在1883年1月17日,约翰·G·麦肯德里克在向格拉斯哥哲学学会报告时描述了该设备,内容如下:

李普曼的电计由一根1米长、直径7毫米的普通玻璃管组成,管两端开放,并通过坚固的支架保持竖直位置。下端被拉成一个毛细点,直到毛细管的直径为0.005毫米。管内充满了水银,毛细点浸入稀硫酸中(体积比为1:6的水),并且在含酸的容器底部有少量水银。每根管内的水银都与一根铂金线连接,最后还设置了使毛细点能通过显微镜(放大250倍)观察到的装置。这样的一种仪器非常敏感;李普曼指出,它能够测量出极小的电势差,甚至可以达到1/10,080 Daniell电池的电势差。因此,它是一种非常精密的观察和(通过补偿法可进行标定)测量微小电动势的工具。[10][11]

李普曼的博士论文,提交给索邦大学于1875年7月24日,研究的是电毛细现象。[12]
\subsubsection{压电效应}
1881年,李普曼预言了反向压电效应。[13]
\subsubsection{彩色摄影}
\begin{figure}[ht]
\centering
\includegraphics[width=6cm]{./figures/7830ee920efd30fa.png}
\caption{李普曼在1890年代拍摄的一张彩色照片。它不含任何颜料或染料。} \label{fig_LPM_4}
\end{figure}
最重要的是,李普曼因发明了一种通过摄影重现颜色的方法,该方法基于干涉现象,这使他获得了1908年的诺贝尔物理学奖。

1886年,李普曼开始关注如何将太阳光谱的颜色固定在摄影板上。1891年2月2日,他向科学院宣布:“我成功地将光谱的图像及其颜色固定在摄影板上,且图像保持不变,可以在日光下保持而不受损坏。” 到1892年4月,他能够报告称,他已成功制作出彩色图像,展示了彩色玻璃窗、一组旗帜、一碗橙子上面放着一朵红色罂粟花以及一只五彩斑斓的鹦鹉。他在1894年和1906年分别向科学院提交了两篇论文,介绍了他使用干涉方法进行彩色摄影的理论。