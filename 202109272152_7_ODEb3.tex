% 一阶常系数线性微分方程组
% 常微分方程|ODE|ordinary differential equation|方程组|线性变换|矩阵|矩阵指数

\pentry{矩阵指数\upref{MatExp},常系数线性微分方程\upref{ODEb2}}

一阶常系数线性微分方程组形如
\begin{equation}\label{ODEb3_eq1}
\leftgroup{
    \frac{\dd}{\dd t}x_1&=a_{11}x_1+a_{12}x_2+\cdots+a_{1n}x_n\\
    \frac{\dd}{\dd t}x_2&=a_{21}x_1+a_{22}x_2+\cdots+a_{2n}x_n\\
    &\vdots\\
    \frac{\dd}{\dd t}x_n&=a_{n1}x_1+a_{n2}x_2+\cdots+a_{nn}x_n
}
\end{equation}
其中各$x_i$是关于$t$的未知函数,各$a_{ij}$是已知常数.我们要研究的是如何解出这个方程组中的各未知函数.

别看这个方程有那么多变量$x_i(t)$,实际上我们可以把它们放到一起,构成一个$n$维向量$\bvec{x}(t)=(x_1(t), x_2(t), \cdots, x_n(t))$,这样就可以理解为还是只有一个自变量,只不过自变量从标量变成向量了.

以上述向量理解的方式来看,\autoref{ODEb3_eq1} 右边部分就是一个线性变换$\mat{M}\bvec{x}(t)$,其中
\begin{equation}
\mat{M}=\pmat{
    &a_{11} &a_{12} &\cdots &a_{1n}\\
    &a_{21} &a_{22} &\cdots &a_{2n}\\
    &\vdots &\vdots &\ddots &\vdots\\
    &a_{n1} &a_{n2} &\cdots &a_{nn}
    }
\end{equation}
是已知常数矩阵.

这样,我们就还可以把\autoref{ODEb3_eq1} 写成
\begin{equation}\label{ODEb3_eq3}
\frac{\dd}{\dd t}\bvec{x}=\mat{M}\bvec{x}
\end{equation}
的形式,看起来和一元方程
\begin{equation}\label{ODEb3_eq2}
\frac{\dd}{\dd t}x=ax
\end{equation}
非常像.

\autoref{ODEb3_eq2} 的通解是$x=C\E^{at}$,其中$C$为常数.事实上,\autoref{ODEb3_eq3} 的通解也可以类似地用矩阵指数来表示:
\begin{equation}
\bvec{x}=C\E^{\mat Mt}
\end{equation}

























