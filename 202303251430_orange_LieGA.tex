% 李群的李代数
% 李群|李代数|Lie group|Lie algebra|切空间|tangent space|向量场|vector field|李括号|Lie bracket

\pentry{李群\upref{LieGrp},李代数\upref{LieAlg}}

\subsection{光滑向量场}

\addTODO{这个话题应该移动到更相关的位置}

我们先回顾光滑向量场的两条性质。

\begin{lemma}{提升映射与向量场的交换性}\label{LieGA_lem1}
设 $F:M\to N$ 是一个\textbf{微分同胚},那么对于任意光滑函数 $f\in C^{\infty}(M)$ 和光滑向量场 $X\in\mathfrak{X}(M)$,都有:
\begin{equation}\label{LieGA_eq1}
(Xf)\circ F^{-1}=F_*(X)(f\circ F^{-1})~,
\end{equation}
其中 $F_*:TM\to TN$ 是 $F$ 的微分。
\end{lemma}

\autoref{LieGA_lem1} 根据“向量场对光滑函数作用”的定义就可以证出。

\addTODO{引用流形上的李括号的定义}

\begin{lemma}{微分和李括号的交换性}\label{LieGA_lem2}
设 $F:M\to N$ 是一个\textbf{微分同胚},那么对于任意光滑向量场 $X, Y\in\mathfrak{X}(M)$,有 $F_*([X, Y])=[F_*(X), F_*(Y)]$。
\end{lemma}

\textbf{证明}:

我们只需要证明 $F_*(XY)(f\circ F^{-1})=F_*(X)F_*(Y)(f\circ F^{-1})$ 对于任意 $f\in C^{\infty}(M)$ 成立即可。

由\autoref{LieGA_lem1} ,
\begin{equation}
\begin{aligned}
F_*(XY)(f\circ F^{-1})&=(XYf)\circ F^{-1}\\
&=F_*(X)(Yf\circ F^{-1})\\
&=F_*(X)(F_*(Y)(f\circ F^{-1}))\\
&=F_*(X)F_*(Y)(f\circ F^{-1})~.
\end{aligned}
\end{equation}

\textbf{证毕}。

\subsection{李群上的左不变向量场}

\begin{definition}{左不变向量场}\label{LieGA_def1}
给定李群 $G$,对于任意 $g\in G$,定义映射 $l_g:G\to G$ 为\textbf{左平移映射},即对于任意 $p\in G$,都有 $l_g(p)=gp$。

根据李群的定义,$l_g$ 是一个光滑映射,因此可以求其微分 $l_{g}^*:TG\to TG$,该微分把一个 $p\in G$ 处的切向量映射为 $gp$ 处的切向量。

如果存在 $G$ 上的切向量场 $X$\footnote{注意,只要求是切向量场,没有要求连续性,更没有要求光滑性。但是稍后我们会看到,所谓的左不变向量场必然是光滑的。},使得对于\textbf{任意}$g\in G$,都有 $l_g(X)=X$;换句话说,就是 $l_{g}^*(X_p)=X_g$,那么称 $X$ 为 $G$ 上的一个\textbf{左不变切向量场(left-invariant tangent vector field)}。
\end{definition}

\autoref{LieGA_def1} 已经直白地说明了,一个左不变向量场 $X$ 唯一地由其在单位元处的取值 $X_e$ 决定的,即 $X_p=l_{p}^*(X_e)$。同时,任意给定 $X_e$,都能由此生成唯一的左不变向量场 $X$。这样,左不变向量场的性质被其在单位元处的取值完全决定,单位元就好像存储了全息信息一样,局部就可以描述整体。

\begin{theorem}{}
左不变切向量场必为光滑向量场。
\end{theorem}

\textbf{证明}:




思路是证明 $X$ 在任意图中都是光滑的。考虑左不变场的性质,只需要证明在某一个图中的情况即可,其它图中可以进行类比\footnote{这是因为如果 $pq=s$,那么 $l_{s}=l_pl_q$。这样一来,左不变场的性质可以由任何一点 $p$ 处的值完全确定,从而任何一个图中的情况都可以类比到其它图中。}。证明的核心是李群中群运算导出光滑映射。

考虑李群 $G$ 上的左不变切向量场 $X$。由\autoref{LieGA_eq1} ,$X$ 完全由 $X_p$ 决定。设在某图 $\phi: G\to\mathbb{R}^n$ 中,$\phi^*(X)_{\phi(e)}=\frac{\dd }{\dd t}\phi(c(t))$,其中 $c(t)$ 是 $I\to G$ 的光滑映射,且 $c(0)=e$,则
\begin{equation}\label{LieGA_eq2}
\phi^*(X)_{\phi(p)}=\frac{\dd}{\dd t}\phi(p\cdot c(t))~.
\end{equation}

由李群的定义,$p\cdot c(t)$ 是 $G\times I\to G$ 的光滑映射\footnote{$c$ 是 $I\to G$ 的光滑映射,因此可以构造 $G\times I\to G\times G$ 的光滑映射 $f(x, t)=(x, c(t))$。由李群的定义,$g(x, y)=x\cdot y$ 是 $G\times G\to G$ 的光滑映射。这样一来,$p\cdot c(t)=g(f(p, t))=g\circ f$ 就是光滑映射的复合,因而也光滑。}。\textbf{不过这一条不重要}。

\textbf{这个才是重要的}:把自变量改为 $\phi(p)$ 和 $t$,令 $g(x, y)=x\cdot y$ 为 $G\times G\to G$ 的光滑映射,则\footnote{映射的乘积定义参见\autoref{map_def2}~\upref{map}}
\begin{equation}
\phi^{-1}(\phi(p))\cdot \phi^{-1}(\phi(c(t)))=g\circ(\phi^{-1}\times c)
\end{equation}
为 $\mathbb{R}^n\times I\to G$ 的光滑映射。

因此,$\phi(p\cdot c(t))=\phi[\phi^{-1}(\phi(p))\cdot \phi^{-1}(\phi(c(t)))]$ 是 $\mathbb{R}^n\times I\to \mathbb{R}^n$ 的光滑映射。

因此\autoref{LieGA_eq2} 是光滑映射的导函数,故是 $\mathbb{R}^n\times I\to \mathbb{R}^n$ 的光滑映射。即,$\phi^*(X)$ 是 $\mathbb{R}^n$ 上的光滑向量场。

故 $X$ 是 $\phi^{-1}(\mathbb{R}^n)$ 上的光滑向量场,进而 $X$ 是 $G$ 上的光滑向量场。



% 只需要证明对于任意 $f\in C^{\infty}(G)$,$Xf$ 都是光滑函数即可。

% $Xf$ 在 $p\in G$ 处的取值可以如下计算,其中取\textbf{光滑}道路 $c:I\to G$ 使得 $\frac{\dd}{\dd t}c(t)|_{t=0}=X_e$,$c(0)=e$,而 $e$ 是 $G$ 的单位元:
% \begin{equation}
% Xf|_p=X_pf=l_{p}^*(X_e)f=\frac{\dd}{\dd t}f(pc(t))|_{t=0}
% \end{equation}

% 这里 $p, c(t)$ 都是群元素,$pc(t)$ 是它们的群乘法。

% 我们只需要证明 $\frac{\dd}{\dd t}f(pc(t))|_{t=0}=\frac{\dd}{\dd t}f\circ l_{p}^*\circ c|_{t=0}$ 作为一个 $G\to \mathbb{R}$ 的函数是光滑的即可。注意,这个函数的自变量是 $p$。

% 为了证明上面这段话,我们又只需要证明 $f\circ l_{p}^*\circ c$ 是一个 $G\times I\to \mathbb{R}$ 的光滑函数即可。

% 由于 $f, l_{p}^*, c$ 都是光滑函数,其组合自然也是光滑函数,由此得证。

\textbf{证毕}。

接下来这条性质是引入李代数的关键。

\begin{exercise}{}
如果 $X, Y$ 是左不变切向量场,那么 $[X, Y]=XY-YX$ 也是。

利用\autoref{LieGA_lem2} ,取左平移映射为所用的微分同胚,证明这一点。
\end{exercise}


\subsection{李群的李代数}

前两节的结论让我们知道了如下事实:李群 $G$ 上全体左不变切向量场的集合,$L(G)$,构成了光滑向量场 $\mathfrak{X}(G)$ 的一个子线性空间。同时,由于左不变切向量场可以被局部描述,我们可以通过单位元上的切空间 $T_eG$ 来描述 $L(G)$;换句话说,$T_eG$ 和 $L(G)$ 是同构的。

\begin{definition}{李群的李代数}
令$X, Y\in L(G)$ 是李群 $G$ 上的两个左不变向量场,任取$p\in G$。定义 $X_p, Y_p\in T_pG$ 的李括号为
\begin{equation}
[X_p, Y_p]=[X, Y]_p~.
\end{equation}
因此 $T_p G$ 是一个李代数,即\textbf{李群的李代数}。
\end{definition}

%是否需要讨论矩阵李群上的李代数呢?恰好就是矩阵乘法的李代数。

\begin{theorem}{复李群的复李代数}
复李群\autoref{LieGrp_def6}~\upref{LieGrp}的李代数是复李代数。
\end{theorem}



\begin{example}{矩阵李群的李代数}

设$M$是一个$\mathbb{C}$上的$n$阶矩阵李群,那么其李代数作为集合是
\begin{equation}
S = \{ \mathcal{X}\in \opn{GL}(n, \mathbb{C}) \mid \E^{t\mathcal{X}}\in M, \forall t\in\mathbb{R} \}~.
\end{equation}
这个李代数可以直接理解为单位元$e$处的切空间,其李括号由矩阵乘法导出。

事实上,$\E^{t\mathcal{X}}$是由切向量$\mathcal{X}\in\opn{T}_e M$生成的左不变切向量场的\textbf{积分曲线}。

\end{example}























