% Egoroff定理
% 叶戈洛夫定理|实变函数|一致收敛

在讨论函数极限的相关问题时,我们常需要函数列具有\textbf{一致收敛}的性质.反过来,观察一个不一致收敛的函数列,比如$\{f_n(x)=x^n\}$在区间$[0, 1]$上就不一致收敛,我们会发现如果把区间挖掉长度$\epsilon$\textbf{任意}小的一部分,那么$\{f_n\}$在$[0, 1-\epsilon]$上总是一致收敛的.这提示我们研究,任意收敛函数列是否可以去掉一个小部分以后是一致收敛的?

答案是肯定的,这就是本节要讨论的Egoroff定理.

\subsection{Egoroff定理}

为方便讨论,我们需要先证明一个引理,证明思路依赖于对集合极限的讨论.

\begin{lemma}{}
任取可测集$E$上的一列实函数$\{f_n\}$,再任取一个$E$上的实函数$f$.

则$\lim\limits_{n\to\infty}f_n(x_0)\neq f(x_0)$,当且仅当对于\textbf{任意}一列\textbf{单调递减}到$0$的\textbf{正}数列$\{\epsilon_k\}$,必有:

\begin{equation}\label{EgrfTh_eq1}
x_0\in\bigcup^\infty_{k=1}\bigcap^\infty_{N=1}\bigcup^\infty_{n=N} \{x\in E|\abs{f_n(x)-f(x)}>\epsilon_k\}
\end{equation}

\end{lemma}

\textbf{证明}:

该引理简单来说,就是“\autoref{EgrfTh_eq1} 为全体使$f_n$不收敛于$f$的点构成的集合”.为了证明引理,我们首先要翻译一下什么叫“$f_n(x)$不收敛于$f(x)$”.

“$f_n(x_0)$\textbf{收敛}于$f(x_0)$”等价于“对于任意$\epsilon>0$,存在$N_\epsilon\in\mathbb{Z}^+$,使得只要$n>N_\epsilon$,就有$\abs{f_n(x_0)-f(x_0)}<\epsilon$”.将该命题进行否定\footnote{回忆否定的过程,即把命题描述中的全称量词和存在量词互换,然后把最后描述的条件取反.},即得到:

“$f_n(x_0)$\textbf{不收敛}于$f(x_0)$”等价于“存在$\epsilon>0$,使得对于任意$N\in\mathbb{Z}^+$,都存在$n>N$,使得$\abs{f_n(x_0)-f(x_0)}\geq\epsilon$”\footnote{到这里,有的读者可能已经看出来怎么证明了:“存在$\epsilon>0$”对应的就是\autoref{EgrfTh_eq1} 中的$\bigcup^\infty_{k=1}$.}.

接下来证明\autoref{EgrfTh_eq1} 和“存在$\epsilon>0$,使得对于任意$N\in\mathbb{Z}^+$,都存在$n>N$,使得$\abs{f_n(x)-f(x)}\geq\epsilon$”是等价命题.

为了方便,记$E_{n, k}=\{x\in E|\abs{f_n(x)-f(x)}>\epsilon_k\}$.那么$\bigcap^\infty_{N=1}\bigcup^\infty_{n=N} E_{n, k}$就是集合的\textbf{上极限}(\autoref{SetLim_def1}~\upref{SetLim},原因见\textbf{集合的极限}\upref{SetLim}词条内容).根据上极限的意义可知,$\bigcap^\infty_{N=1}\bigcup^\infty_{n=N} E_{n, k}$是全体“对于$\epsilon_k$和任意的$N\in\mathbb{Z}^+$,总存在$n>N$使得$\abs{f_n(x)-f(x)}\geq\epsilon$”的$x$构成的集合.

再套上一个$\bigcup^\infty_{k=1}$来取遍所有$\epsilon_k$,得到的\autoref{EgrfTh_eq1} 就等价于“$f_n(x_0)$\textbf{不收敛}于$f(x_0)$”.

\textbf{证毕}.








\begin{theorem}{Egoroff定理}

设$E$是$\mathbb{R}^n$上测度有限的可测集合.如果$\{f_n\}$是$E$上的一列函数,都几乎处处取有限函数值,而存在$E$上的函数$f(x)$使得\footnote{即$f_n$几乎处处收敛于$f$,或者说不收敛点构成一个零测集.}\begin{equation}
\lim\limits_{n\to\infty}f_n(x)=f(x)a. e. 
\end{equation}

那么对于任意正数$\epsilon$,必存在一个可测集$E_\epsilon$,使得$\opn{m}E_\epsilon<\epsilon$,且$f_n$在$E-E_\epsilon$上一致收敛于$f$.

\end{theorem}

\textbf{证明}:

考虑$\{f_n\}$在$E$上\textbf{处处}收敛于$0$的情况.这个情况最简单,并且证明了这个情况也就相当于证明了$\{f_n\}$\textbf{几乎处处}收敛于任意$f$的情况.

\textbf{证毕}.










