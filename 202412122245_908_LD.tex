% 零点能量(综述)
% license CCBYSA3
% type Wiki

本文根据 CC-BY-SA 协议转载翻译自维基百科\href{https://en.wikipedia.org/wiki/Zero-point_energy}{相关文章}。

\begin{figure}[ht]
\centering
\includegraphics[width=8cm]{./figures/9b23f9fc5e8f2f7f.png}
\caption{液态氦由于零点能的存在,在标准大气压下无论温度如何都保持动能,不会冻结。当其温度降到 Lambda 点以下时,它表现出超流体性特性。} \label{fig_LD_1}
\end{figure}
\textbf{零点能(ZPE)}是量子力学系统可能具有的最低能量。与经典力学不同,量子系统即使在最低能量状态下也会不断波动,这可以通过海森堡不确定性原理来描述[1]。因此,即使在绝对零度下,原子和分子也会保持某些振动运动。除了原子和分子外,真空的空旷空间也具有这些性质。根据量子场论,宇宙可以被看作不仅是孤立的粒子,而是连续波动的场:物质场,其量子是费米子(即轻子和夸克),以及力场,其量子是玻色子(例如光子和胶子)。所有这些场都有零点能[2]。这些波动的零点场导致了一种在物理学中重新引入以太的现象[1][3],因为某些系统可以探测到这种能量的存在[需要引用]。然而,如果这个以太要保持洛伦兹不变性,以保证与爱因斯坦的相对论没有矛盾,那么它就不能被视为一种物理介质[1]。

零点能的概念对宇宙学也非常重要,然而,物理学目前缺乏一个完整的理论模型来理解宇宙学中的零点能;特别是理论上与观测到的宇宙真空能量之间的差异,成为了一个重大争议问题[4]。然而,根据爱因斯坦的广义相对论,任何这种能量都会引起引力,而来自宇宙膨胀、暗能量和Casimir效应的实验证据表明,任何这种能量都极其微弱。一个试图解决这一问题的提案是认为费米子场具有负的零点能,而玻色子场具有正的零点能,因此这些能量会以某种方式相互抵消[5][6]。如果超对称是自然界的精确对称性,这个想法是成立的;然而,欧洲核子研究中心的大型强子对撞机至今未找到支持这一理论的证据。此外,已知如果超对称是有效的,它最多也只是一个破缺的对称性,仅在极高的能量下才成立,目前没有人能够展示一个低能宇宙中发生零点能抵消的理论[6]。这一差异被称为宇宙学常数问题,是物理学中最大的未解之谜之一。许多物理学家认为,“真空是理解自然的关键”[7]。
\subsection{词源和术语}  
零点能(ZPE)一词是从德语“Nullpunktsenergie”翻译过来的。[8] 有时,它与零点辐射和基态能量互换使用。零点场(ZPF)一词可以用来指代特定的真空场,例如量子电动力学(QED)真空,它专门处理量子电动力学(如光子、电子和真空之间的电磁相互作用),或者量子色动力学(QCD)真空,它涉及量子色动力学(如夸克、胶子和真空之间的色荷相互作用)。真空可以被视为不是空的空间,而是所有零点场的组合。在量子场论中,这种场的组合被称为真空态,与之相关的零点能量被称为真空能量,平均能量值称为真空期望值(VEV),也称为其凝聚态。
\subsection{概述}
\begin{figure}[ht]
\centering
\includegraphics[width=8cm]{./figures/429e3fe66ba0fbb1.png}
\caption{动能与温度} \label{fig_LD_2}
\end{figure}
在经典力学中,所有粒子都可以被认为具有某种能量,这种能量由它们的势能和动能组成。例如,温度来自于由动能引起的随机粒子运动的强度(称为布朗运动)。当温度降低到绝对零度时,可以认为所有运动都停止,粒子完全静止。然而,实际上,即使在最低的温度下,粒子仍然保持动能。与这种零点能量对应的随机运动永远不会消失;它是量子力学不确定性原理的结果。
\begin{figure}[ht]
\centering
\includegraphics[width=8cm]{./figures/a6ab0822f805ed5d.png}
\caption{零点辐射不断地对电子施加随机冲动,使得电子永远无法完全停止。零点辐射赋予振荡器的平均能量等于振荡频率乘以普朗克常数的一半。} \label{fig_LD_3}
\end{figure}
不确定性原理表明,任何物体无法同时拥有精确的位置和速度值。量子力学物体的总能量(包括势能和动能)由其哈密顿量描述,哈密顿量也描述了该系统作为一个简谐振子或波函数,在不同的能量状态之间波动(参见波粒二象性)。所有量子力学系统即使在其基态下也会经历波动,这是它们波动性本质的结果。不确定性原理要求每个量子力学系统必须具有大于经典势阱最小值的波动零点能量。这导致即使在绝对零度下也会有运动。例如,液氦在大气压力下无论温度如何都不会冻结,这正是由于其零点能量。

根据阿尔伯特·爱因斯坦的质量与能量等价关系 \( E = mc^2 \),任何包含能量的空间点都可以被看作具有质量,从而产生粒子。现代物理学已经发展出了量子场论(QFT),用以理解物质与力之间的基本相互作用;它将空间中的每个点视为一个量子简谐振子。根据量子场论,宇宙由物质场组成,物质场的量子是费米子(如轻子和夸克),以及力场,力场的量子是玻色子(如光子和胶子)。所有这些场都具有零点能量。最近的实验支持这样一个观点:粒子本身可以看作是基础量子真空的激发态,物质的所有属性只是由零点场相互作用引起的真空波动。

‘空’空间可以具有内在能量,并且没有‘真正的真空’这一概念,这一观点似乎是反直觉的。通常认为,整个宇宙完全浸泡在零点辐射中,因此它只能在计算中增加一个常数值。因此,物理测量将仅揭示该值的偏差。[10] 对于许多实际计算,零点能量通常被在数学模型中以强制性的方式忽略,作为一个没有物理效应的项。然而,这种处理方式会引发问题,因为在爱因斯坦的广义相对论中,空间的绝对能量值不是一个任意常数,而是产生了宇宙学常数。几十年来,大多数物理学家认为,存在某种尚未发现的基本原理,能够消除无限的零点能量并使其完全消失。如果真空没有内在的、绝对的能量值,它就不会发生引力效应。人们曾认为,随着宇宙从大爆炸的余波中膨胀,任何单位的空旷空间中所包含的能量将减少,因为总能量扩展以填满宇宙的体积;宇宙中的星系和所有物质应该开始减速。然而,1998年通过发现宇宙的膨胀并没有减缓,而是加速了,这一可能性被排除。这意味着空空间确实具有某种内在的能量。暗能量的发现最好通过零点能量来解释,尽管目前仍然是一个谜,为什么其值与通过理论得到的巨大值相比如此之小——这就是宇宙学常数问题。[5]

许多归因于零点能量的物理效应已经通过实验验证,如自发辐射、卡西米尔力、兰姆位移、电子的磁矩和德尔布吕克散射。[11][12] 这些效应通常被称为‘辐射修正’。[13] 在更复杂的非线性理论中(例如量子色动力学,QCD),零点能量可以引发各种复杂的现象,如多稳定态、对称性破缺、混沌和涌现。当前的研究领域包括虚拟粒子的效应,[14] 量子纠缠,[15] 惯性质量和引力质量之间的差异(如果有的话),[16] 光速变化,[17] 观察到的宇宙学常数的原因,[18] 以及暗能量的性质。[19][20]
\subsection{历史 } 
\subsubsection{早期的以太理论}
\begin{figure}[ht]
\centering
\includegraphics[width=6cm]{./figures/36a1f632beb96010.png}
\caption{詹姆斯·克拉克·麦克斯韦} \label{fig_LD_4}
\end{figure}
零点能量源自关于真空的历史思想。对于亚里士多德来说,真空是 τὸ κενόν,‘空的’;即与物体无关的空间。他认为这个概念违反了基本的物理原则,并主张火、空气、地球和水的元素并非由原子构成,而是连续的。对于原子论者来说,‘空’的概念具有绝对性质:它是存在与不存在的区别。关于真空特性的辩论大多局限于哲学领域,直到文艺复兴时期才开始有所突破,当时奥托·冯·格里克发明了第一个真空泵,科学上可验证的理论才开始出现。人们认为,简单地去除所有气体,就能创造出完全空的空间,这是第一个被普遍接受的真空概念。

然而,到了19世纪末,显然即使在抽空的区域中,仍然存在热辐射。以太存在作为真实空无的替代品是当时最流行的理论。根据基于麦克斯韦电动力学的成功电磁以太理论,这种无所不包的以太被赋予了能量,因此与虚无是非常不同的。电磁现象和引力现象能够在空空间中传播,被认为是它们相关的以太是空间本身的一部分。然而,麦克斯韦指出,这些以太在大多数情况下是临时设定的:

“对于那些坚持认为以太作为哲学原则存在的人来说,自然对真空的厌恶是想象一种包围一切的以太的充分理由……以太被发明出来,让行星在其中游动,构成电气大气层和磁气流,传递从身体一部分到另一部分的感觉等等,直到空间被以太填充了三四次。”

此外,1887年的迈克尔逊–莫雷实验结果是第一次强烈证据表明当时流行的以太理论存在严重缺陷,并启动了最终导致狭义相对论的研究路线,后者完全排除了静止以太的概念。对于当时的科学家来说,似乎可以通过冷却并消除所有辐射或能量来在空间中创造一个真正的真空。由此演变出了第二个实现真正真空的概念:将一个区域的空间冷却至绝对零度温度,然后进行抽空。绝对零度在19世纪技术上是无法实现的,因此这一辩论没有得到解决。
\subsubsection{第二量子理论}
\begin{figure}[ht]
\centering
\includegraphics[width=6cm]{./figures/1755ae4a531d63e4.png}
\caption{普朗克在1918年,因其在量子理论方面的工作获得诺贝尔物理学奖。} \label{fig_LD_6}
\end{figure}
在1900年,马克斯·普朗克推导了单个能量辐射体(例如振动的原子单位)平均能量 \(\varepsilon\) 与绝对温度的关系公式:
\[
\varepsilon = \frac{h\nu}{e^{h\nu / (kT)} - 1}~
\]
其中,\(h\) 是普朗克常数,\(\nu\) 是频率,\(k\) 是玻尔兹曼常数,\(T\) 是绝对温度。零点能量并未对普朗克的原始定律做出贡献,因为在1900年时,普朗克尚不知晓零点能量的存在。

零点能量的概念是由马克斯·普朗克在1911年在德国提出的,作为对他在1900年提出的原始量子理论中零基态公式的修正项。

在1912年,普朗克发表了第一篇关于辐射不连续发射的期刊文章,基于能量的离散量子。他的“第二量子理论”中,共振器连续吸收能量,但仅当它们达到相空间的有限单元边界时,才会以离散的能量量子发射能量,其中它们的能量成为 \(h\nu\) 的整数倍。这一理论促使普朗克得出了新的辐射定律,但在这个版本中,能量共振器具有零点能量,即共振器能够取的最小平均能量。普朗克的辐射方程包含一个残余能量因子 \(\frac{h\nu}{2}\) 作为额外项,这个项依赖于频率 \(\nu\),且大于零(其中 \(h\) 是普朗克常数)。因此,普遍认为“普朗克方程标志着零点能量概念的诞生”。在1911到1913年间的一系列论文中,[29]普朗克发现了振荡器的平均能量为:
\[
\varepsilon = \frac{h\nu}{2} + \frac{h\nu}{e^{h\nu / (kT)} - 1}~
\]
\begin{figure}[ht]
\centering
\includegraphics[width=6cm]{./figures/4b6500c57682c25f.png}
\caption{爱因斯坦在获得1921年诺贝尔物理学奖后的官方肖像。} \label{fig_LD_5}
\end{figure}
不久,零点能量的概念引起了阿尔伯特·爱因斯坦及其助手奥托·斯特恩的关注。[31] 他们在1913年发表了一篇论文,试图通过计算氢气的比热并与实验数据进行比较来证明零点能量的存在。然而,在他们认为自己成功之后,他们很快撤回了对这一理论的支持,因为他们发现普朗克的第二理论可能并不适用于他们的例子。在同年的一封信中,爱因斯坦向保罗·艾伦费斯特声明,零点能量“死得像钉子一样”。[32] 零点能量也被彼得·德拜提及,[33] 他指出,即使温度接近绝对零度,晶格中原子的零点能量也会导致X射线衍射辐射强度的降低。1916年,瓦尔特·能斯特提出,空旷的空间充满了零点电磁辐射。[34] 随着广义相对论的发展,爱因斯坦发现真空的能量密度有助于产生一个宇宙常数,以便得到他场方程的静态解;空旷空间或真空可以具有某种内在能量的观点重新出现了。爱因斯坦在1920年表示:

有一个强有力的论点支持以太假说。否认以太最终意味着假设空旷空间没有任何物理性质。力学的基本事实与这种观点并不和谐……根据广义相对论,空间是赋予物理性质的;因此,在这个意义上,存在一个以太。根据广义相对论,没有以太的空间是不可思议的;因为在这样的空间中,不仅没有光的传播,而且没有时间和空间的标准存在(即测量杆和时钟),因此也就没有物理意义上的时空间隔。但是,这个以太不能被看作是具有可追溯时间的物质特性,也不能应用于可追踪的部分。运动的概念不能应用于它。[35][36]
\begin{figure}[ht]
\centering
\includegraphics[width=6cm]{./figures/d83b5e7eb6dd8e91.png}
\caption{海森堡,1924年} \label{fig_LD_7}
\end{figure}
Kurt Bennewitz 和 Francis Simon(1923年),他们在沃尔特·能斯特(Walther Nernst)位于柏林的实验室工作,研究了低温下化学物质的熔化过程。他们计算了氢气、氩气和水银的熔点,并得出结论,认为这些结果为零点能提供了证据。此外,他们还正确地提出(后来由西蒙(1934年)验证)这一量是导致氦气即使在绝对零度下也难以固化的原因。1924年,罗伯特·穆立肯(Robert Mulliken)通过比较10BO和11BO的带谱,提供了分子振动的零点能的直接证据:如果没有零点能,不同电子能级基态的振动频率的同位素差异应该会消失,这与观察到的谱线相矛盾。仅仅一年后的1925年,随着在维尔纳·海森堡的文章《量子理论对运动学和力学关系的重新解释》中矩阵力学的发展,零点能从量子力学中得到了推导。

1913年,尼尔斯·玻尔(Niels Bohr)提出了现在被称为玻尔模型的原子模型,但尽管如此,为什么电子不会坠入原子核依然是一个谜。根据经典理论,考虑到加速电荷通过辐射损失能量,意味着电子应该会螺旋式地坠入原子核,原子也不应该是稳定的。这个经典力学的问题在1915年被詹姆斯·霍普伍德·吉恩斯(James Hopwood Jeans)总结得很好:“假设力学定律 \(\frac{1}{r^2}\) 在r趋近于零时仍然成立,将会是一个非常真实的困难。因为两电荷在零距离时的相互作用力将是无限大的;我们应该看到异号电荷不断相互吸引,且一旦相遇,没有力会使它们进一步收缩或无限减小。” 这一难题的解决出现在1926年,当时厄尔温·薛定谔(Erwin Schrödinger)提出了薛定谔方程。该方程解释了这样一个新的非经典事实:当电子被限制靠近原子核时,它必然会拥有很大的动能,因此最小的总能量(动能加势能)实际上会出现在某个正的距离上,而不是零距离;换句话说,零点能对于原子稳定性至关重要。
\subsubsection{量子场论及其发展} 
1926年,帕斯夸尔·乔丹(Pascual Jordan)首次尝试对电磁场进行量子化。在与马克斯·玻恩(Max Born)和维尔纳·海森堡(Werner Heisenberg)合作的论文中,他将腔体内的场视为量子谐振子的叠加。在计算中,他发现,除了振荡器的“热能”之外,还必须存在一个无限的零点能项。他得出了与爱因斯坦在1909年得到的相同的波动公式。然而,乔丹并不认为他的无限零点能项是“真实的”,他写信给爱因斯坦说:“这只是计算中的一个量,没有直接的物理意义”。乔丹找到了一种方法来去除这个无限项,并于1928年与泡利(Pauli)共同发表了一篇论文,进行了一次被称为“量子场论中的第一次无限减法或重正化”的操作。
\begin{figure}[ht]
\centering
\includegraphics[width=6cm]{./figures/887a811793832fb1.png}
\caption{保罗·狄拉克,1933年} \label{fig_LD_8}
\end{figure}
基于海森堡等人的工作,保罗·狄拉克在1927年提出的辐射的发射与吸收理论[54] 是量子辐射理论的首次应用。狄拉克的工作被认为对新兴的量子力学领域至关重要;它直接处理了“粒子”实际是如何被创造出来的过程:自发辐射。[55] 狄拉克将电磁场的量子化描述为一个由谐振子组成的集合,并引入了粒子的产生和湮灭算符的概念。该理论表明,自发辐射依赖于电磁场的零点能波动才能开始。[56][57] 在一个光子被湮灭(吸收)的过程中,光子可以被看作是进入了真空态。类似地,当光子被创造(发射)时,偶尔可以认为光子是从真空态过渡到一个实际存在的状态。用狄拉克的话来说:[54]

光量子有一个特性,即它似乎在其某些驻定状态下不存在,即零状态,其中它的动量,因此也它的能量,都是零。当一个光量子被吸收时,可以认为它跳入了这个零状态,而当一个光量子被发射时,可以认为它是从零状态跳到一个它在物理上显现出来的状态,从而看起来像是被创造出来了。由于没有限制可以这样创造的光量子的数量,我们必须假设零状态中存在无限多的光量子...

当代物理学家在被问及自发辐射的物理解释时,通常会引用电磁场的零点能。这一观点由维克多·韦斯科普夫(Victor Weisskopf)在1935年提出并推广:[58]

从量子理论中可以推导出所谓零点振荡的存在;例如,每个振荡器在其最低状态下并不完全静止,而是始终围绕其平衡位置运动。因此电磁振荡也不可能完全停止。因此,电磁场的量子特性导致了场强在最低能量状态下的零点振荡,在这种状态下,空间中没有光量子... 零点振荡作用在电子上与普通的电振荡相同。它们可以改变电子的本征状态,但仅限于过渡到最低能量的状态,因为空旷的空间只能带走能量,而不能给予能量。通过这种方式,自发辐射作为这些与零点振荡相对应的独特场强存在的结果而产生。因此,自发辐射是由空旷空间的零点振荡引发的光量子辐射。

这一观点后来得到了西奥多·韦尔顿(Theodore Welton,1948年)的支持,[59] 他认为自发辐射“可以被视为在波动场的作用下发生的强迫辐射”。这一新理论,狄拉克称之为量子电动力学(QED),预言了即使在没有源的情况下,也存在波动的零点场或“真空场”。

在1940年代,微波技术的改进使得能够更精确地测量氢原子能级的跃迁,现在称为兰姆位移(Lamb shift),[60] 并测量了电子的磁矩。[61] 这些实验结果与狄拉克理论之间的差异促使了将重正化引入QED以处理零点能的无限大的问题。重正化最初由汉斯·克拉梅尔(Hans Kramers)[62] 和维克多·韦斯科普夫(1936年)[63] 提出,并在1947年由汉斯·贝特(Hans Bethe)首次成功地应用于计算兰姆位移的有限值。[64] 关于自发辐射,这些效应可以部分通过与零点场的相互作用来理解。[65][11] 但由于重正化能够从计算中去除一些零点能的无限大,并不是所有的物理学家都愿意给零点能赋予任何物理意义,而是将其视为一个数学工件,认为它可能在某一天被消除。在沃尔夫冈·泡利1945年的诺贝尔讲座[66]中,他明确表示反对零点能的观点,称“显然,这个零点能没有物理现实”。
\begin{figure}[ht]
\centering
\includegraphics[width=6cm]{./figures/26371c1d7af1c598.png}
\caption{亨德里克·卡西米尔(1958年)} \label{fig_LD_9}
\end{figure}
1948年,亨德里克·卡西米尔(Hendrik Casimir)证明了零点场的一个后果是,在两个无电荷、完全导电的平行板之间会产生吸引力,这就是所谓的卡西米尔效应。当时,卡西米尔正在研究胶体溶液的性质。胶体溶液是含有微米级颗粒的液体基质,如油漆和蛋黄酱。这些溶液的性质由范德华力决定——范德华力是一种存在于中性原子和分子之间的短程吸引力。卡西米尔的一位同事,西奥·欧尔贝克(Theo Overbeek)意识到,当时用于解释范德华力的理论,由弗里茨·伦敦(Fritz London)在1930年发展出来的,并未能正确解释胶体实验数据。因此,欧尔贝克要求卡西米尔研究这个问题。在与迪尔克·波尔德(Dirk Polder)合作时,卡西米尔发现,只有考虑光传播有限速度的事实,才能正确描述两个中性分子之间的相互作用。不久之后,在与玻尔(Bohr)讨论零点能的过程中,卡西米尔注意到,这一结果可以用真空波动来解释。他开始思考,如果有两面镜子——而不是两分子——在真空中面对面,会发生什么。正是这项工作导致了他对反射板之间吸引力的预测。卡西米尔和波尔德的工作为范德华力和卡西米尔力的统一理论铺平了道路,并在两种现象之间建立了平滑的过渡。利夫希茨(Lifshitz)在1956年(1956年)做出了相关工作,针对平行介电板的情况完成了这一理论的统一。范德华力和卡西米尔力的统称为色散力,因为它们都是由偶极矩算符的色散引起的。相对论性力的作用在约百纳米的尺度上变得主导。

1951年,赫伯特·卡伦(Herbert Callen)和西奥多·韦尔顿(Theodore Welton)证明了量子涨落-耗散定理(FDT),该定理最初由奈奎斯特(Nyquist)在1928年以经典形式提出,用于解释电路中的约翰逊噪声。涨落-耗散定理表明,当某物耗散能量时,以一种有效的不可逆方式,连接的热浴也必须发生涨落。涨落和耗散是密不可分的,二者不可分离。FDT的意义在于,真空可以被视为与耗散力耦合的热浴,因此能量可以部分地从真空中提取出来,用于潜在的有用工作。FDT已在某些量子非经典条件下被实验验证为真实。

1963年,杰恩斯-卡明斯模型(Jaynes-Cummings model)被提出,描述了二能级原子与量子场模式(即真空)在光学腔中的相互作用。该模型给出了直觉上不容易理解的预测,例如原子的自发辐射可能被有效频率(拉比频率)驱动。在1970年代,实验开始测试量子光学的各个方面,结果表明,原子的自发辐射速率可以通过反射表面进行调控。最初,这些结果在一些领域中被怀疑:人们认为自发辐射速率无法修改,因为,究竟原子如何通过发射光子“看见”它的环境,而一开始就需要发射光子?这些实验催生了腔量子电动力学(CQED),即研究镜子和腔体对辐射修正的影响。自发辐射可以被抑制(或“抑制”)或放大。放大的现象最早由帕塞尔(Purcell)在1946年预测(帕塞尔效应),并已被实验验证。部分而言,这种现象可以通过真空场对原子的作用来理解。
\subsection{不确定性原理}
零点能与海森堡不确定性原理密切相关。粗略地说,不确定性原理表明,互补变量(例如粒子的位置和动量,或场的值和在空间中的导数)不能通过任何给定的量子态同时精确指定。特别地,不可能存在一个系统处于其势阱底部静止不动的状态,因为在这种状态下,位置和动量将会被完全确定,并且可以达到任意高的精度。因此,系统的最低能量状态(基态)必须具有一个位置和动量的分布,这个分布满足不确定性原理,这意味着其能量必须大于势阱的最低点。

在势阱底部附近,通用系统的哈密顿量(给出其能量的量子力学算符)可以近似为量子谐振子:
\[
{\hat {H}} = V_0 + \frac{1}{2}k\left({\hat {x}} - x_0\right)^2 + \frac{1}{2m}{\hat {p}}^2~
\]
其中,\(V_0\) 是经典势阱的最小值。

不确定性原理告诉我们:
\[
\sqrt{\langle \left({\hat {x}} - x_0\right)^2 \rangle} \sqrt{\langle {\hat {p}}^2 \rangle} \geq \frac{\hbar}{2}~
\]
这使得上面动能和势能项的期望值满足:
\[
\langle \frac{1}{2}k\left({\hat {x}} - x_0\right)^2 \rangle \langle \frac{1}{2m}{\hat {p}}^2 \rangle \geq \left( \frac{\hbar}{4} \right)^2 \frac{k}{m}~
\]
因此,能量的期望值至少为:
\[
\langle {\hat {H}} \rangle \geq V_0 + \frac{\hbar}{2} \sqrt{\frac{k}{m}} = V_0 + \frac{\hbar \omega}{2}~
\]
其中,\(\omega = \sqrt{k/m}\) 是系统振荡的角频率。

更深入的处理表明,基态的能量实际上会饱和这个界限,并且是精确的:\(E_0 = V_0 + \frac{\hbar \omega}{2}\)这需要解系统的基态。
\subsection{原子物理学}
\begin{figure}[ht]
\centering
\includegraphics[width=8cm]{./figures/7a47c75222b88cc6.png}
\caption{零点能量 \( E = \frac{\hbar \omega}{2} \) 使得谐振子的基态相位(颜色)发生变化。当多个本征态叠加时,这会产生可测量的效应。} \label{fig_LD_10}
\end{figure}
量子谐振子的概念及其相关的能量可以应用于原子或亚原子粒子。在普通的原子物理学中,零点能是与系统基态相关的能量。专业物理文献通常使用角频率(ω)来测量频率,如上文所示的ν,且定义为ω = 2πν。这导致了一种约定,使用带有横线的普朗克常数(ħ)来表示数量 \( \frac{h}{2\pi} \)。用这种方式,零点能的一个例子是上面的 \( E = \frac{\hbar \omega}{2} \),它与量子谐振子的基态相关。在量子力学中,零点能是系统在基态下哈密顿量的期望值。

如果存在多个基态,则称这些基态是简并的。许多系统具有简并的基态。简并性发生在存在一个酉算符,该算符在基态上非平凡地作用并与系统的哈密顿量对易时。

根据热力学第三定律,处于绝对零度的系统存在于其基态;因此,它的熵由基态的简并度决定。许多系统,如完美的晶格,具有唯一的基态,因此在绝对零度下熵为零。对于表现出负温度的系统,也有可能最高激发态在绝对零度下存在。

一维势阱中粒子的基态波函数是一个半周期的正弦波,在势阱的两个边缘处为零。粒子的能量由以下公式给出:
\[
E = \frac{h^2 n^2}{8mL^2}~
\]
其中,\( h \) 是普朗克常数,\( m \) 是粒子的质量,\( n \) 是能级(\( n = 1 \) 对应基态能量),\( L \) 是势阱的宽度。
\subsection{量子场论}
在量子场论(QFT)中,“空”空间的结构被想象成由场组成,空间和时间中的每一点都可以视为一个量子谐振子,相邻的振荡子相互作用。根据QFT,宇宙由物质场组成,其量子是费米子(如电子和夸克);力场,其量子是玻色子(即光子和胶子);以及一个希格斯场,其量子是希格斯玻色子。物质场和力场都具有零点能量。一个相关的术语是零点场(ZPF),它是特定场的最低能量态。真空可以被视为不是空的空间,而是所有零点场的组合。

在QFT中,真空态的零点能量被称为真空能量,而哈密顿量的平均期望值被称为真空期望值(也叫凝聚态或简写为VEV)。量子电动力学(QED)真空是处理量子电动力学(例如光子、电子与真空之间的电磁相互作用)的真空态的一部分,而量子色动力学(QCD)真空则处理量子色动力学(例如夸克、胶子与真空之间的色荷相互作用)。最近的实验支持这样一个观点:粒子本身可以被看作是潜在量子真空的激发态,并且物质的所有属性仅仅是由于与零点场相互作用而产生的真空波动。

空间中的每一点都贡献 \( E = \frac{\hbar \omega}{2} \),这导致在任何有限体积内计算出无限的零点能量;这也是需要重整化才能使量子场论具有实际意义的一个原因。在宇宙学中,真空能量是解释宇宙学常数和暗能量来源的一个可能解释。

科学家们对于真空中包含多少能量并没有达成一致。量子力学要求能量大,如保罗·狄拉克所宣称,它像一个能量海洋。其他专攻广义相对论的科学家要求能量足够小,以便空间的曲率与观测到的天文学现象相符。海森堡不确定性原理允许能量足够大,以促使量子行为在短暂时间内发生,即使平均能量足够小,满足相对论和平坦空间的要求。为了应对这种分歧,真空能量被描述为一种具有正负能量的虚拟能量势。

在量子扰动理论中,有时会说,一环和多环费曼图对基本粒子传播子的贡献就是真空波动或零点能量对粒子质量的贡献。
\subsubsection{量子电动力学真空 } 
最古老且最著名的量子化力场是电磁场。麦克斯韦方程已被量子电动力学(QED)所取代。通过考虑源自量子电动力学的零点能,可以获得对零点能的特征性理解,这种零点能不仅仅通过电磁相互作用产生,而且在所有量子场论中都有出现。

\textbf{重新定义能量的零点}

在电磁场的量子理论中,经典的波振幅α和α*被运算符a和a†取代,满足以下关系:
\[
[a, a^{\dagger}] = 1~
\]
经典表达式中出现的量|α|²,在量子理论中被光子数算符a†a所替代。事实上:
\[
[a, a^{\dagger}a] \neq 1~
\]
这意味着量子理论不允许辐射场的状态,其中光子数和场振幅可以精确地定义,也就是说,我们无法为a†a和a同时找到特征态。场的波动性和粒子属性的调和通过将概率振幅与经典模式相联系来完成。场模式的计算是完全经典的问题,而场的量子特性则由与这些经典模式相关联的模式“振幅”a†和a承载。

场的零点能形式上来源于a和a†的不对易性。这对任何谐振子都成立:零点能\(\frac{\hbar \omega}{2}\)出现在我们写出哈密顿量时:
\[
H_{cl} = \frac{p^2}{2m} + \frac{1}{2}m\omega^2 q^2 = \frac{1}{2}\hbar \omega (a a^{\dagger} + a^{\dagger} a) = \hbar \omega \left(a^{\dagger} a + \frac{1}{2}\right)~
\]
通常有一种观点认为整个宇宙完全浸泡在零点电磁场中,因此它只能对期望值添加一个常数量。因此,物理测量将只揭示偏离真空状态的变化。因此,通过重新定义能量的零点,或者认为它是常数且因此对海森堡运动方程没有影响,可以将零点能从哈密顿量中去掉。因此,我们可以通过法定声明,假设基态能量为零,举例如下,场哈密顿量可以被替换为:
\[
H_F - \langle 0 | H_F | 0 \rangle = \frac{1}{2} \hbar \omega (a a^{\dagger} + a^{\dagger} a) - \frac{1}{2} \hbar \omega = \hbar \omega (a^{\dagger} a + \frac{1}{2}) - \frac{1}{2} \hbar \omega = \hbar \omega a^{\dagger} a~
\]
而不会影响理论的任何物理预测。新的哈密顿量被称为正规排列(或维克排列),用双点符号表示。正规排列的哈密顿量表示为:\( :H_F: \),即:
\[
H_F: \equiv \hbar \omega (a a^{\dagger} + a^{\dagger} a) \equiv \hbar \omega a^{\dagger} a~
\]
换句话说,在正规排列符号内,我们可以交换a和a†。由于零点能与a和a†的不对易性密切相关,正规排列过程消除了零点场的任何贡献。这在场哈密顿量的情况下尤其合理,因为零点项仅仅添加了一个常数能量,可以通过简单的零点能量重新定义来去除。此外,这个常数能量显然与a和a†对易,因此不会对由海森堡方程描述的量子动力学产生任何影响。

然而,事情并没有那么简单。零点能不能仅通过从哈密顿量中去掉其能量来消除:当我们这么做并求解场算符的海森堡方程时,我们必须包括真空场,它是场算符解的均匀部分。事实上,我们可以证明,真空场对于保持对易子和QED的形式一致性至关重要。当我们计算场能量时,除了可能存在的粒子和力的贡献外,还会得到来自真空场本身,即零点场能量的贡献。换句话说,即使我们可能已将零点能从哈密顿量中删除,它依然会重新出现。

\textbf{自由空间中的电磁场}

根据麦克斯韦方程,"自由"场(即没有源的场)的电磁能量可以通过以下公式描述:
\[
H_F = \frac{1}{8\pi} \int d^3r \left( \mathbf{E}^2 + \mathbf{B}^2 \right) = \frac{k^2}{2\pi} |\alpha(t)|^2~
\]
我们引入满足赫尔姆霍兹方程的“模式函数” \( A_0(r) \):
\[
(\nabla^2 + k^2) \mathbf{A}_0(\mathbf{r}) = 0~
\]
其中 \( k = \frac{\omega}{c} \),并假设它已标准化,使得:
\[
\int d^3r \left| \mathbf{A}_0(\mathbf{r}) \right|^2 = 1~
\]
我们希望对自由空间的电磁能量进行“量子化”,以适应多模式场。自由空间的场强度应与位置无关,因此对于每个场模式,\(|A_0(r)|^2\) 应与 \(r\) 无关。满足这些条件的模式函数为:
\[
\mathbf{A}_0(\mathbf{r}) = e_{\mathbf{k}} e^{i \mathbf{k} \cdot \mathbf{r}}~
\]
其中 \( \mathbf{k} \cdot e_{\mathbf{k}} = 0 \),以便在我们所使用的库仑规中满足横向条件 \( \nabla \cdot A(r,t) = 0 \)。

为了实现所需的标准化,我们假设空间被分割为体积为 \( V = L^3 \) 的立方体,并对场施加周期性边界条件:
\[
\mathbf{A}(x+L, y+L, z+L, t) = \mathbf{A}(x, y, z, t)~
\]
或者等效地:
\[
(k_x, k_y, k_z) = \frac{2\pi}{L}(n_x, n_y, n_z)~
\]
其中 \( n \) 可以取任何整数值。这样,我们可以在任意一个虚拟立方体中考虑场,并定义模式函数:
\[
\mathbf{A}_{\mathbf{k}}(\mathbf{r}) = \frac{1}{\sqrt{V}} e_{\mathbf{k}} e^{i \mathbf{k} \cdot \mathbf{r}}~
\]
该函数满足赫尔姆霍兹方程、横向条件以及“盒子标准化”:
\[
\int_V d^3r \left|\mathbf{A}_{\mathbf{k}}(\mathbf{r})\right|^2 = 1~
\]