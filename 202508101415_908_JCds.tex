% 交错代数(综述)
% license CCBYSA3
% type Wiki

本文根据 CC-BY-SA 协议转载翻译自维基百科\href{https://en.wikipedia.org/wiki/Alternative_algebra}{相关文章}。

在抽象代数中,交替代数是一类代数,其乘法运算不必满足结合律,只需满足交替性。也就是说,对于代数中的任意元素 $x$ 和 $y$,必须有:
\begin{itemize}
\item $x(xy) = (xx)y$
\item $(yx)x = y(xx)$
\end{itemize}
显然,所有结合代数都是交替代数,但也存在一些严格非结合的代数(如八元数)同样满足交替性。
\subsection{结合子}
交替代数之所以得名,是因为它们的结合子是交替的。结合子是一个三线性映射,定义为:
$$
[x, y, z] = (xy)z - x(yz)~
$$
按定义,如果一个多线性映射在其任意两个自变量相等时取零,则称其为交替的。对于一个代数,左交替恒等式与右交替恒等式等价于\(^\text{[1]}\):
$$
[x, x, y] = 0~
$$
$$
[y, x, x] = 0~
$$
将这两个恒等式结合起来,可以推出:
$$
[x, y, x] = [x, x, x] + [x, y, x] - [x, x+y, x+y]~
$$
$$
= [x, x+y, -y]~
$$
$$
= [x, x, -y] - [x, y, y] = 0~
$$
对于所有的 $x$ 和 $y$ 都成立。这等价于柔性恒等式\(^\text{[2]}\):
$$
(xy)x = x(yx)~
$$
因此,交替代数的结合子是交替的。反过来,任何结合子是交替的代数显然也是交替代数。由于对称性,任何满足以下三个恒等式中任意两个的代数:
\begin{itemize}
\item 左交替恒等式:$x(xy) = (xx)y$
\item 右交替恒等式:$(yx)x = y(xx)$
\item 柔性恒等式:$(xy)x = x(yx)$
\end{itemize}
都是交替代数,并且因此满足这三个恒等式全部成立。

一个交替的结合子总是完全反对称的,即:
$$
[x_{\sigma(1)}, x_{\sigma(2)}, x_{\sigma(3)}] = \operatorname{sgn}(\sigma) [x_1, x_2, x_3]~
$$
对于任意排列 $\sigma$ 都成立。只要基域的特征不为 2,反之亦然。
\subsection{例子}
\begin{itemize}
\item 每一个结合代数都是交替代数。
\item 八元数构成一个非结合的交替代数,它是实数域上的 8 维赋范除代数\(^\text{[3]}\)。
\item 更一般地,任意八元数代数都是交替代数。
\end{itemize}
\subsubsection{非例子}
\begin{itemize}
\item 十六元数、三十二元数以及所有更高阶的凯莱–迪克森代数都失去了交替性。
\end{itemize}
\subsection{性质}
“阿廷定理”在此处的含义与原始元阿廷定理不同,后者请参见“原始元定理”条目。

阿廷定理指出,在一个交替代数中,由任意两个元素生成的子代数是结合的\(^\text{[4]}\)。反之,任何满足此性质的代数显然是交替代数。由此可知,在交替代数中,只涉及两个变量的表达式可以不加括号而无歧义地书写。

阿廷定理的一个推广形式表明:在交替代数中,如果三个元素 $x, y, z$ 是结合的(即$
[x, y, z] = 0$成立),那么由它们生成的子代数也是结合的。

阿廷定理的一个推论是:交替代数都是幂结合的,也就是说,由单个元素生成的子代数是结合的\(^\text{[5]}\)。反之则未必成立:例如,十六元数是幂结合的,但不是交替代数。

穆方恒等式

在任意交替代数中,都成立以下恒等式\(^\text{[2]}\):
\begin{itemize}
\item $a(x(ay)) = (axa)y$
\item $((xa)y)a = x(aya)$
\item $(ax)(ya) = a(xy)a$
\end{itemize}
在含幺交替代数中,若乘法逆元存在,则它是唯一的。并且,对于任意可逆元 $x$ 以及任意 $y$,都有:
$$
y = x^{-1}(xy)~
$$
这等价于说,对于所有这样的 $x$ 和 $y$,结合子$[x^{-1}, x, y]$恒为零。

如果 $x$ 和 $y$ 都是可逆的,那么 $xy$ 也是可逆的,并且其逆元为:$(xy)^{-1} = y^{-1}x^{-1}$因此,所有可逆元在乘法下封闭,并构成一个穆方环。在交替环或交替代数中,这个单位元环路与结合环或结合代数中的单位元群是类似的。

克莱因费尔德定理指出:任何简单的非结合交替环都是其中心上的广义八元数代数\(^\text{[6]}\)。交替环的结构理论可参见 Zhevlakov、Slin'ko、Shestakov 和 Shirshov 合著的《Rings That Are Nearly Associative》一书\(^\text{[7]}\)。
**出现**

任意交替除环上的射影平面都是一个**穆方平面**(Moufang plane)。

每一个**构成代数**(composition algebra)都是交替代数,这一点由 Guy Roos 在 2008 年证明\[8]:设 $A$ 是定义在域 $K$ 上的构成代数,它具有一个范数 $n$,该范数是一个乘法同态:

$$
n(a \times b) = n(a) \times n(b)
$$

将 $(A, \times)$ 与 $(K, \times)$ 联系起来。

定义形式

$$
(a : b) = n(a+b) - n(a) - n(b)
$$

其中 $(a : b) \colon A \times A \to K$。那么,元素 $a$ 的**迹**由 $(a : 1)$ 给出,而它的**共轭**定义为

$$
a^* = (a : 1)e - a
$$

其中 $e$ 是 1 的基元。通过一系列习题可以证明,构成代数总是交替代数\[9]。

**另见**

* 域上的代数(Algebra over a field)
* 马尔采夫代数(Maltsev algebra)
* 佐恩环(Zorn ring)
