% 二元关系
% keys 二元关系|等价关系|等价类

\pentry{集合\upref{Set}}

\subsection{关系}

在 “集合\upref{Set}” 中我们只关心了集合的基数,即集合中元素的数目.在这种语境下,任何两个元素数量相同的集合都可以看作是同一个集合.但是仅仅讨论集合的基数未免太过单调,缺少了很多有意思的理论,于是我们希望在集合的元素之间建立一些结构,来进行更细致的划分和研究.

\textbf{关系(relation)}是集合上最基础的一种结构.给定一个关系,我们就可以讨论一些元素之间\textbf{是否满足}这个关系.比如说,如果取一家三口构成一个集合,$\sim$ 代表的是“年龄大于”,那么我们可以说“爸爸对于孩子具有这个关系”,但是反过来“孩子对于爸爸不具有这个关系”.从这个例子可以看出,关系的表达方式很灵活,而且可以是有方向性的.讨论关系时,我们唯一关心的是给定元素之间是否具有这样的关系.

关系可以用在两个元素之间,也可以用在三个元素之间,甚至可以用在不特定的元素之间.

\subsection{二元关系}

在绝大多数数学和物理领域,我们只关心集合上的\textbf{二元关系(binary relation)}.如果 $\sim$ 是在集合 $A$ 上定义的一个二元关系,那么任意给定两个元素,我们都可以讨论它们之间是否具有这种关系,但如果给定三个元素,讨论就没有意义了.比如,如果 $\sim$ 的定义是“年龄大于”,那么把三个人的年龄都拿过来比较就没有意义;不过,如果 $\sim$ 的定义是“比后面两个人的年龄都大”,那么 $\sim$ 就可以用在三个人身上.

对于集合 $A$ 上的二元关系 $\sim$,如果 $x, y\in A$ 满足这个关系,我们可以把这句话表述为 $x\sim y$.如果不满足,则可以表述为 $x\not\sim y$.用这样简洁的表示方法,我们可以把以上“年龄大于”的关系表述如下:
\begin{equation}
\text{爸爸}\sim\text{孩子} \qquad
\text{妈妈}\sim\text{孩子} \qquad
\text{孩子}\not\sim\text{爸爸} \qquad
\text{孩子}\not\sim\text{妈妈} \qquad
\end{equation}
爸爸和妈妈之间,在没有进一步信息的情况下,无法判断是否满足这个关系,但这个关系是存在的.也就是说,元素间的关系一共有两种状态:满足和不满足.

在数学上,二元关系被定义为笛卡尔积$A \times A$的子集,如果集合$A$中的两个元素$a,b$满足关系$R$.则$(a,b) \in R$,否则$(a,b) \notin R$.

\subsection{等价关系}
最基础的一类关系,是\textbf{等价关系(equivalence)}. 在集合 $A$ 上定义的关系 $\sim$ 是一个等价关系,要求满足条件:
\begin{enumerate}
\item \textbf{自反性}:$\forall x\in A, x\sim x$;
\item \textbf{对称性}:$\forall x, y\in A, x\sim y \Rightarrow y\sim x$;
\item \textbf{传递性}:$\forall x, y, z\in A, x\sim y, y\sim z\Rightarrow x\sim z$.
\end{enumerate}
如果把“具有关系 $\sim$”看成是两个元素间相互连接(没有方向性,因为对称性),传递性保证了当多个元素相连时,这些元素也两两互联;自反性甚至保证了每个元素必然和自身相连.由此一来,我们可以根据等价关系把集合 $A$ 划分成\textbf{等价类(equivalence class)},每个等价类都是 $A$ 的一个子集,所包含的元素彼此相连.

等价类的划分是数学中非常重要的思维方法,它可以将每一个等价类中的元素都看成是无差别的,大大简化一个集合的复杂程度.有了等价关系后,我们可以用相应的等价类来组成一个新的集合,这样的集合被称作\textbf{商集(quotient set)}.和原来集合中的元素不一样,商集中的元素是原集合中的子集.

作为例子,我们取全体非负整数的集合 $\mathbb{Z}$,并在上面定义关系 $\sim$ 为“两数的差是3的倍数”,那么容易验证,$1\sim4, 7\sim304, 0\not\sim 77, 77\not\sim 1$. 利用这个等价关系,我们可以把非负整数集合划分成三个等价类,分别是 $\{0, 3, 6, 9, 12\cdots \}$,$\{1, 4, 7, 10, 13\cdots\}$ 和 $\{2, 5, 8, 11, 14\cdots\}$. 这样,我们可以得到一个含有三个元素的商集,用数学专业术语来说,叫做集合 $A$ 模去关系 $\sim$ 所得的商集.这里的\textbf{模(mod)}的意思来自于“除法”,在群论\upref{Group1}中会看到为什么用除法来命名商集.

\subsection{序关系}
另一类比较基础的关系是\textbf{序关系(ordering relation)}.在集合$A$上定义的关系$\prec$(一般也可以写成$\leq$或$\leqslant$)是一个序关系,要求满足条件:
\begin{enumerate}
\item \textbf{自反性}:$\forall x\in A,x\prec x$;
\item \textbf{反对称性}:$\forall x,y\in A,\ x\prec y,\ y\prec x \Rightarrow x = y $;
\item \textbf{传递性}:$\forall x,y,z\in A,\ x\prec y,\ y\prec z \Rightarrow x\prec z $.
\end{enumerate}

\begin{definition}{偏序集、半序集}
如果一个集合$A$上定义了一个序关系$\prec$,那么称这个集合是带有偏序$\prec$的\textbf{偏序集(partially ordered set)}或\textbf{半序集}.
\end{definition}

带有偏序$\prec$的偏序集$A$通常记作$(A,\prec)$.在不致混淆的情况下,可以简称为偏序集$A$.偏序集上的序关系称为\textbf{偏序关系(partially ordered relation)}.

序关系实际上是给集合中的元素排序,因此$a \prec b$又被称为\textbf{$a$在$b$前},反之$b \prec a$称为\textbf{$a$在$b$后}.这种排序方式并不唯一,而且任意两个元素间不一定可以

\begin{example}{}
各种数集(包括$\mathbb{N},\mathbb{Z},\mathbb{Q},\mathbb{R},\mathbb{C}$)按通常的序关系构成偏序集.

其中复数集$\mathbb{C}$上的序关系比较特殊,复数$a_1+b_1\I \leq a_2+b_2\I$当且仅当$a_1\leq a_2$且$b_1 \leq b_2$.复数的序关系通常不使用.
\end{example}

这个例子说明序关系实际上是数与数大小关系的抽象.

\begin{example}{字典序}\label{Relat_ex1}
已知$(A,\prec),(B,\leq)$是两个偏序集,那么笛卡尔积$A\times B$按某个序关系$\leqslant$构成偏序集.这个序关系$\leqslant$满足:
\begin{enumerate}
\item $\forall(a_1,b_1),(a_2,b_2) \in A\times B, a_1\prec a_2 \Rightarrow (a_1,b_1)\leqslant(a_2,b_2)$;
\item $\forall(a_1,b_1),(a_2,b_2) \in A\times B, a_1=a_2, b_1\leq b_2 \Rightarrow (a_1,b_1) \leqslant (a_2, b_2)$.
\end{enumerate}

由于字典通常按照这样的顺序编排\footnote{比如,字典中单词in在a后,在it前.},因此这种序关系称为字典序.
\end{example}
\begin{example}{幂集}
一个集合$A$的幂集\footnote{即集合$A$所有子集的集合.}$2^X$与子集关系一起构成偏序集.
\end{example}

\begin{definition}{全序集}
偏序集$(A,\prec)$称为\textbf{全序集(totally ordered set)}或\textbf{有序集(ordered set)},当且仅当对任意$a,b \in A$,$a \prec b$或$b \prec a$.
\end{definition}
全序集上的序关系称为\textbf{全序关系(totally ordered relation)}.
\begin{example}{}
数集$\mathbb{N},\mathbb{Z},\mathbb{Q},\mathbb{R}$按通常的序关系构成全序集.
\end{example}
\begin{example}{}
将\autoref{Relat_ex1} 中的偏序集$A,B$换成全序集,则按同样的序关系,$A\times B$也构成全序集.
\end{example}

\begin{definition}{极小元、极大元}
已知偏序集$(A,\prec)$:
\begin{enumerate}
\item 如果存在某个$a \in A$,使得对任意$x \in A, x\prec a$都有$x = a$,那么称$a$为偏序集$A$的\textbf{极小元(minimal element)}.
\item 如果存在某个$b \in A$,使得对任意$x \in A, b\prec x$都有$x = b$,那么称$b$为偏序集$A$的\textbf{极大元(maximal element)}.
\end{enumerate}
\end{definition}

\begin{definition}{最小元、最大元}
已知偏序集$(A,\prec)$的某个非空子集$B$:
\begin{enumerate}
\item 如果存在某个$a \in B$,使得$\forall x \in B, a \prec x$.那么$a$被称为$B$的\textbf{最小元(least element)};
\item 如果存在某个$b \in B$,使得$\forall x \in B, x \prec b$.那么$b$被称为$B$的\textbf{最大元(greatest element)}.
\end{enumerate}
\end{definition}

极大元、极小元与最大元、最小元极易混淆.偏序集的极大元、极小元不一定唯一,但最大元、最小元只要存在必然唯一.在全序集中两者统一.

\begin{example}{}
集合$A=\{0,1,2\}$,在上面定义偏序关系$\prec$为$0\prec 0$, $1\prec 1$, $1\prec 2$, $2\prec 2$.那么子集$\{0,1\}$的极大元为0和1,不存在最大元.
\end{example}

\begin{exercise}{}
证明:如果偏序集的某个子集存在最大元,那么它的极大元必然存在且唯一,并且两者相等.
\end{exercise}

\begin{exercise}{}
证明:对于全序集的任意非空子集,如果存在极大元,则最大元必然存在,且两者相等.
\end{exercise}

\begin{definition}{良序集}
全序集$(A,\prec)$称为\textbf{良序集(well-ordered set)},当且仅当它的任意非空子集都有极小元.
\end{definition}

良序集上的序关系称为\textbf{良序关系(well-ordered relation)}.