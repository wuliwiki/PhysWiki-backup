% 本原元素
% keys primitive element|可分扩张|域|多项式|本原元定理|单扩张|代数扩张|域扩张

\pentry{可分扩张\upref{SprbEx}}

%primitive element

有限扩张都是代数扩张,其中最简单的是单代数扩张,因为只需要添加一个元素就可以生成.而我们加下来讨论的“本原元定理”,就准确地描述了哪些有限扩张是单扩张.

\begin{definition}{本原元}

设$\mathbb{K}/\mathbb{F}$是有限次的域扩张,如果存在$a\in\mathbb{K}$使得$\mathbb{K}=\mathbb{F}(a)$,则称$a$是域扩张$\mathbb{K}/\mathbb{F}$的\textbf{本原元素(primitive element)},或者说是$\mathbb{K}$对$\mathbb{F}$的本原元素,简称本原元.

\end{definition}

下面给出一个引理,有助于理解和证明本原元定理.事实上,这个引理正是本原元定理的其中一个方向的核心思路.


\begin{lemma}{}
设$\mathbb{K}=\mathbb{F}(a, b)$,且$\mathbb{K}/\mathbb{F}$只有有限多个中间域,那么必存在$x\in\mathbb{K}$使得$\mathbb{K}=\mathbb{F}(x)$.
\end{lemma}

\textbf{证明}:

任取$c\in\mathbb{F}$,构造单扩域$\mathbb{K}_c=\mathbb{F}(a+cb)$.

\textbf{证毕}.



\begin{theorem}{本原元定理}

设$\mathbb{K}/\mathbb{F}$是有限次的域扩张,则

$\mathbb{K}/\mathbb{F}$是单扩张 $\iff$ $\mathbb{K}/\mathbb{F}$的中间域\footnote{见\autoref{FldExp_the3}~\upref{FldExp}证明后的一句讨论.}只有有限多个.

\end{theorem}

\textbf{证明}:

$\mathbb{F}$是有限域时,由\autoref{FntFld_cor1}~\upref{FntFld},两个条件恒成立.因此下面只考虑$\mathbb{F}$是无限域的情况.

$\Leftarrow$:



\textbf{证毕}.
































