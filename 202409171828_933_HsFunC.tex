% 函数的变换与性质
% keys 函数|变换|平移|旋转|伸缩|单调|对称|奇偶性|初等函数|周期
% license Xiao
% type Tutor

\begin{issues}
\issueDraft
\end{issues}

\pentry{函数\nref{nod_functi}}{nod_6f70}

\subsection{函数的变换}

在实际生活中,变换的概念无处不在。比如,调整照片的大小、改变音乐的速度、甚至是地图应用中缩放和旋转视图,这些操作都与函数变换的原理息息相关。

函数的变换包含的范畴非常广泛:高中涉及到的函数变换主要针对的是函数图象而言的,包括平移、旋转、伸缩,它们统称线性变换;几何方面,相似变换、射影变换、仿射变换等变换用于分析和操作几何图形及其在空间中的关系;频域分析方面,傅里叶变换、拉普拉斯变换、Z变换等变换帮助我们将问题从时域转换到频域,从而揭示出信号和系统中隐藏的特征。当然现在主要先研究高中涉及的部分,几何部分或许也可以了解一番,而频域分析的部分在大学阶段才可能会接触到。

函数的变换本质上是空间的变换,也就是说函数本身的性质没有变,改变的是函数所在的坐标系。但就像驶远的汽车,从你的视角是汽车开远了,而从车的视角,假设车没动的话就是你在远离。坐标系的改变后,如果认为坐标系不变,那么就相当于是函数在进行变换了。尽管这与我们现实生活中的经验不太一样,毕竟我们想要移动一个画框在墙上的位置一般不会把整面墙移动走,但采用“变换的本质是变换坐标系”这个视角去看待函数的变换问题,会比较容易理解。下面的内容会

下面的内容均研究函数$y=f(x)$,假设点$(x,y)$代表着函数上的一个点,函数$y_1=f(x_1)$和点$(x_1,y_1)$分别代表着变换后的函数和变换后的点。介绍也会在“坐标系在变换”的这个视角上进行。如果觉得比较困难,可以先只了解平移部分,待习惯后再了解后面的两个部分。

\subsubsection{平移}
函数图象向右移动
这里可以拿两张纸自己比划一下,假设上面的指代表函数的图象,而下面的纸代表坐标系。

\subsubsection{伸缩}

有了是上面的经验伸缩就容易得多了,如果我们希望把函数的图象在$y$方向上拉伸为$A$倍,那么就要把坐标系在$y$方向上压缩为$\displaystyle 1\over A$,即取$\displaystyle y_1={y\over A}$即可,代入整理后可以得到:
\begin{equation}
y_1=Af(x_1)~.
\end{equation}



\subsubsection{*旋转}

这一部分内容需要使用三角函数的知识,如果不太熟悉可以去\addTODO{添加初中部分}回顾。

就像复合函数可以多个函数一层层复合一样,这些变换也可以依次加在同一个函数上,最终会得到一个新的表达式。在处理时,如果不太熟悉可以把每一步独立考虑,最后再整理就可以了。下面试一试。
\begin{exercise}{把函数$y=f(x)$先在$x$方向上向右平移$a$个单位,再在$y$方向上向下平$b$个单位,先在$x$方向上压缩为原来的$1\over5$,求得到的新函数的表达式。}

\end{exercise}

\subsection{函数的性质}

函数具有一些性质,有一些在高中会接触到,有一些不会接触到。
以后我们会看到一些用\enref{极限}{Lim}和\enref{导数}{Der}描述的性质。 例如 % \addTODO{链接}
, 可导。
\subsubsection{零点}
\begin{definition}{零点}

\end{definition}
\subsubsection{单调性}

\begin{definition}{单调性}

\end{definition}



\subsubsection{对称性}

对称性分为两种,一种是轴对称性,一种是中心对称性,这两个性质在初中就有接触过。

\begin{definition}{中心对称}
中心对称
\end{definition}

\begin{definition}{轴对称}
轴对称
\end{definition}

有两个比较特殊的对称性称为奇偶性。

\begin{definition}{奇偶性}
如果函数$f(x)$的定义域为$D$,且$D$是关于$0$对称的。若对任意的$x\in D$,有:
\begin{itemize}
\item $f(x)=f(-x)$,则称$f(x)$是偶函数。
\item $f(x)=-f(-x)$或$-f(x)=f(-x)$,则称$f(x)$是奇函数。
\end{itemize}
\end{definition}

\subsubsection{周期性}

\begin{definition}{周期}

\end{definition}

还有一些性质是高中不会涉及到的,此处给出:
\begin{itemize}
\item \enref{连续性}{contin}, 一致连续
\end{itemize}


\subsection{特殊的函数}

在高中阶段会涉及到的两种特殊的函数包括初等函数和分段函数。

\subsubsection{初等函数}

高中研究的函数都是初等函数。初等函数指的是由基本初等函数经过基本运算(加减乘除)以及复合形成的函数。

初等函数之所以被称为初等函数就是因为它的性质很好,

基本初等函数:
\begin{itemize}
\item 常值函数
\item 幂函数
\item 指数函数
\item 对数函数
\item 三角函数
\end{itemize}

\subsubsection{分段函数}

绝对值函数

取整函数

狄利克雷函数