% 序列的极限

\pentry{数列的极限(简明微积分)\upref{Lim0}, 极限\upref{Lim}, 实数集的拓扑\upref{ReTop}}

\subsection{基本定义与性质}

序列的极限是分析数学中最基本的定义. 词条 数列的极限(简明微积分)\upref{Lim0} 和 极限\upref{Lim} 已经给出了一些序列极限的例子, 它的形式定义以及背后的直观解释. 为完整起见, 这里再重复一次序列极限的定义:

\begin{definition}{数列的极限}
考虑数列$\{a_n\}$.若存在一个实数$A$,使得对于\textbf{任意}给定的\textbf{正实数} $\varepsilon > 0$(无论它有多么小),总存在正整数 $N_\epsilon$, 使得对于所有编号 $n>N_\epsilon$ ,都有 $\abs{a_n - A} < \varepsilon$ ($A$ 为常数) 成立,那么数列 $a_n$ 的极限就是 $A$.

将“数列$\{a_n\}$的极限是$A$”表示为$\lim\limits_{n\to\infty}a_n=A$.
\end{definition}

正如之前两个词条所解释的, 等式$\lim\limits_{n\to\infty}a_n=A$所表达的含义是"序列$a_n$随着$n$的增大将可以任意地接近$A$"". 或者说, 对于序列$\{a_n\}$进行极限运算, 就是要找到"序列$a_n$越来越接近的那个数". 这种运算显然跟实数的四则运算不一样.

序列的极限运算有如下基本性质:

\begin{theorem}{极限的基本性质}
\begin{itemize}
\item 序列的极限若存在, 则必定是唯一的.
\item 极限运算保持序关系: 如果$\lim\limits_{n\to\infty}a_n=A$, $\lim\limits_{n\to\infty}b_n=B$, 而且从某个$n$开始有$a_n\geq b_n$, 那么必然有$A\geq B$.
\item 设$\lim\limits_{n\to\infty}a_n=A$, $\lim\limits_{n\to\infty}b_n=B$, 则$\lim\limits_{n\to\infty}a_n\pm b_n=A\pm B$, $\lim\limits_{n\to\infty}a_nb_n=AB$.
\item 设$\lim\limits_{n\to\infty}a_n=A$, $\lim\limits_{n\to\infty}b_n=B\neq0$, 则
$$
\lim\limits_{n\to\infty}\frac{a_n}{b_n}=\frac{A}{B}.
$$
\end{itemize}
\end{theorem}

如下基本定理是很重要的:

\begin{theorem}{}
单调有界的实数序列必然有极限.
\end{theorem}
\textbf{证明.} 设$\{a_n\}$是单调递增的序列, 有上界$M$. 按照确界原理\upref{SupInf}, 数集$\{a_n:n\in\mathbb{N}\}$有唯一的上确界$A$, 也就是说成立如下两件事: 第一, $a_n\leq A$对于任何$n$都成立; 第二, 任给$\varepsilin>0$, 数$A-\varepsilon$都不是数集$\{a_n:n\in\mathbb{N}\}$的shang

\textbf{证毕.}

\subsection{序列的上极限与下极限}

\subsection{柯西序列; 柯西收敛准则}