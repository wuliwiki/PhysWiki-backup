% 笛卡尔积
% keys 直积|有序数对|笛卡尔积
% license Usr
% type Tutor

\begin{issues}
\issueDraft
\end{issues}

在刚刚接触到乘法计算时,作为一个运算结果,“积”是作为与“乘法”相关的概念被引入的。后来,随着对向量的学习的深入,内积和外积逐渐也成为了熟悉的概念,二者分别与点乘($\cdot$)和叉乘($\times$)相对应。或许,“卷积”和“张量积”等概念也偶尔会出现在你的视野中。他们往往是与一个逐渐抽象的“乘法”相对应,说他逐渐抽象,是因为他与我们熟知的数的乘法的样子和计算方法相去甚远。而还称呼它是乘法,是因为某种程度上,它保留了乘法的一些特性。

在物理上,常常会通过“乘法”来定义一个新的物理量,比如:功是力与位移的乘积,力矩是力与力臂的乘积,电路中功率是电流与电压的乘积(先忽略这个乘积具体的形式)等。这个新的物理量与原有的两个物理量之间都存在关系。而“加法”往往是在一个概念内部量的多少的计算,基本是不涉及其他概念的。

数学上使用\textbf{直积}(direct product)来组合两个同类的已知对象,从而定义新对象,例如集合、群、模、拓扑空间等都可以进行直积运算。而作用在两个集合上的直积便称为\textbf{笛卡尔积}(Cartesian product)。下面会先直接给出笛卡尔积的定义,然后就定义中的"有序对"进行讲解。

\begin{definition}{笛卡尔积}\label{def_CartPr_1}
从集合$X,Y$中各取出一个元素$x,y$形成的有序对的集合,称作\textbf{笛卡尔积}。
\begin{equation}
X\times Y:=\{(x,y)|x\in X,y\in Y\}~.
\end{equation}

\end{definition}

\subsection{有序对}

现实生活中,“序”这个概念是必不可少的,集合却偏偏有这样一个让人困惑的性质——“集合内的元素的具有无序性”。那“序”这个概念该如何表达呢?有序对(ordered pair)的出现就是为了能够有概念来表示顺序。因此,定义时就希望这个概念能够满足下面两个性质:

\begin{enumerate}
\item 唯一性:每一个有序对是唯一定义的,即$(a, b) = (c, d)\implies (a = c) \land (b = d)$。
\item 顺序性:有序对中的元素顺序是固定的,即$a\neq b\iff(a, b)\neq (b, a)$。
\end{enumerate}

这样,“序”的概念就能在表达序的想法(顺序性)的同时,保持稳定(唯一性)。

\begin{definition}{有序对}
\textbf{有序对}有以下几种常见的定义方法\footnote{参考\href{https://en.wikipedia.org/wiki/Ordered_pair}{Wikipedia}}:
\begin{itemize}
\item 维纳对(Wiener pair,1914年):$(a, b):= \{\{\emptyset,\{ a\}\}, \{\{b\}\}\} $
\item 豪斯多夫对(Hausdorff Pair,1914年):$ (a, b):= \{\{a, 1\}, \{b, 2\}\} $
\item 库拉托夫斯基对(Kuratoswki pair,1921年)$(a, b) := \{\{a\}, \{a, b\}\}$
\end{itemize}
\end{definition}
其中:
\begin{itemize}
\item 维纳对通常配合类型论使用。
\item 豪斯多夫对由于使用了数字作为序的描述,如果要考虑尚未定义数的场景,或需要研究数的序时,可能会造成循环论证。
\item 库拉托夫斯基对定义较为简洁,目前使用也最为广泛。\textbf{下面的描述中,会采取此定义}。
\end{itemize}

需要注意:
\begin{itemize}
\item 每个定义在某些具体领域(如研究集合的理论)使用时,都存在各自的局限性。
\item 定义不仅限于上面三种,由于与内容关系不大,其余的没有在此给出。
\item 有序对的概念针对的是集合的元素,并不局限于数集,通常听到的“有序数组”的概念是“有序对”概念的子概念。
\end{itemize}

定义能够比较好地描述之前提到的特性,下面以唯一性举例。

\begin{example}{唯一性证明}
根据定义有$(a, b) = \{\{a\}, \{a, b\}\} , (c, d) = \{\{c\}, \{c, d\}\} $,若$(a, b)=(c,d)$,即$\{\{a\}, \{a, b\}\}=\{\{c\}, \{c, d\}\}$,则根据集合相等的定义,且$card(\{a\})=1$,必有$\{a\}=\{c\},\{a, b\}=\{c, d\}$。继续根据集合相等,有$a=c$,进而有$b=d$。

证毕。
\end{example}

这样,通过在无序的集合中,构造具有不同特点的子集,实现了有序的概念。这种方法在涉及到集合的构造(如自然数的构建等)时会经常使用。有序对的概念在关系、函数、拓扑等领域都有应用。

\subsection{笛卡尔积}

现在,根据\autoref{def_CartPr_1} 就可以使用笛卡尔积得到新集合了。比如:
\begin{itemize}
\item 若记掷一次骰子的结果为$A=\{1, 2, 3, 4, 5, 6\}$,则掷两次骰子的结果集就可以表示为$A\times A$。
\item 若记一条直线为${\mathbb R}$,则由两条直线张成的平面就是${\mathbb R}\times{\mathbb R}$,也写作${\mathbb R}^2$。
\end{itemize}
可以看出,当前笛卡尔积的结果是对应两个集合的。就像推广乘法到连乘一样,如果想要求解3个或者到n个集合的笛卡尔积,为了能够让他符合直观猜想的样子$(a_1, a_2, \ldots, a_n)$,可以通过递归的方法来将有序对的概念扩展到更多的元,例如:三元组可以定义为有序对的有序对$(a, b, c) := ((a, b), c)$。因此,将有序对的概念推广到任意有序$n$元组,记作:
\begin{equation}
(a_1, a_2, \ldots, a_n) := ((a_1, a_2, \ldots, a_{n-1}), a_n)~.
\end{equation}
进而,笛卡尔积也可以推广到$n$元:
\begin{definition}{n元笛卡尔积}
对于 $n$ 个集合 $A_1, A_2, \ldots, A_n$,其笛卡尔积定义为:
\begin{equation}
A_1 \times A_2 \times \cdots \times A_n = \{(a_1, a_2, \ldots, a_n) \mid \forall i,a_i \in A_i \} ~.
\end{equation}
\end{definition}