% 马克斯·玻恩(综述)
% license CCBYSA3
% type Wiki

本文根据 CC-BY-SA 协议转载翻译自维基百科\href{https://en.wikipedia.org/wiki/Max_Born}{相关文章}。
\begin{figure}[ht]
\centering
\includegraphics[width=6cm]{./figures/4294383d11e9fe1f.png}
\caption{肖像,20世纪30年代} \label{fig_BE_1}
\end{figure}
\textbf{马克斯·玻恩}(Max Born,FRS FRSE)(德语发音:[ˈmaks ˈbɔʁn];1882年12月11日-1970年1月5日)是一位德裔英国物理学家和数学家,对量子力学的发展起到了关键作用。他还在固态物理学和光学领域做出了重要贡献,并在20世纪20年代和30年代指导了许多著名物理学家的研究工作。玻恩因其“对量子力学的基础研究,特别是对波函数统计解释的贡献”于1954年获得了诺贝尔物理学奖。[1]

玻恩于1904年进入哥廷根大学,在那里他结识了三位著名的数学家:费利克斯·克莱因(Felix Klein)、大卫·希尔伯特(David Hilbert)和赫尔曼·闵可夫斯基(Hermann Minkowski)。他以弹性丝和带的稳定性为主题撰写了博士论文,并因此获得了该校哲学系奖项。1905年,他开始与闵可夫斯基研究狭义相对论,随后完成了以汤姆森原子模型为主题的资格论文(habilitationsschrift)。1918年,他在柏林偶然遇到弗里茨·哈伯(Fritz Haber),讨论了一种金属与卤素反应生成离子化合物的过程,这一过程今天被称为\textbf{玻恩–哈伯循环}(Born–Haber cycle)。

在第一次世界大战期间,玻恩最初被安置为无线电操作员,但由于他的专业知识,他被调往从事声测位研究工作。1921年,玻恩返回哥廷根大学,为他长期的朋友和同事詹姆斯·弗兰克(James Franck)安排了另一张教授职位。在玻恩的领导下,哥廷根大学成为当时世界上物理学的主要中心之一。

1925年,玻恩与维尔纳·海森堡(Werner Heisenberg)共同提出了量子力学的矩阵力学表述。次年,他提出了薛定谔方程中 \textbf{ψ\ψ} 概率密度函数的标准解释,这一贡献为他赢得了1954年的诺贝尔物理学奖。

玻恩的影响远远超出了他的个人研究领域。包括马克斯·德尔布吕克(Max Delbrück)、齐格弗里德·弗吕格(Siegfried Flügge)、弗里德里希·洪德(Friedrich Hund)、帕斯夸尔·约尔丹(Pascual Jordan)、玛丽亚·哥佩特-梅耶(Maria Goeppert-Mayer)、洛塔尔·沃尔夫冈·诺德海姆(Lothar Wolfgang Nordheim)、罗伯特·奥本海默(Robert Oppenheimer)和维克多·魏斯科普夫(Victor Weisskopf)等人都在哥廷根大学师从玻恩取得了博士学位。

此外,玻恩的助手中也包括许多著名的物理学家,例如恩里科·费米(Enrico Fermi)、维尔纳·海森堡(Werner Heisenberg)、格哈德·赫兹伯格(Gerhard Herzberg)、弗里德里希·洪德(Friedrich Hund)、沃尔夫冈·泡利(Wolfgang Pauli)、莱昂·罗森菲尔德(Léon Rosenfeld)、爱德华·泰勒(Edward Teller)以及尤金·维格纳(Eugene Wigner)。

1933年1月,纳粹党在德国上台,作为犹太人的玻恩被暂停了他在哥廷根大学的教授职位。他移居到英国,并在剑桥大学圣约翰学院担任职务,同时撰写了一本科普书《不安定的宇宙》(*The Restless Universe*)以及《原子物理学》(*Atomic Physics*),后者很快成为一本标准教材。

1936年10月,他成为爱丁堡大学自然哲学的泰特教授。在那里,他与两位德国出生的助手E.沃尔特·凯勒曼(E. Walter Kellermann)和克劳斯·富克斯(Klaus Fuchs)合作,继续他的物理学研究。

1939年8月31日,也就是第二次世界大战在欧洲爆发的前一天,玻恩成为了英国公民。他一直在爱丁堡工作到1952年,随后退休回到西德的巴特皮尔蒙特(Bad Pyrmont)。1970年1月5日,玻恩在哥廷根的一家医院去世。[2]
\subsection{早年生活}  
马克斯·玻恩于1882年12月11日出生在布雷斯劳(现波兰弗罗茨瓦夫)。在玻恩出生时,布雷斯劳属于德意志帝国普鲁士的西里西亚省,他的家族具有犹太血统。[3] 玻恩是解剖学家和胚胎学家古斯塔夫·玻恩(Gustav Born)与其妻子玛格丽特(Margarethe,昵称格雷琴)·考夫曼(Kauffmann)所生的两个孩子之一。古斯塔夫是布雷斯劳大学的胚胎学教授,[4] 玛格丽特则出身于一个西里西亚工业家家族。玛格丽特于1886年8月29日去世,当时马克斯只有四岁。[5]  

马克斯有一个妹妹,名叫凯特(Käthe),她于1884年出生。马克斯还有一个同父异母的弟弟沃尔夫冈(Wolfgang),是父亲与第二任妻子贝莎·利普斯坦(Bertha Lipstein)所生。沃尔夫冈后来成为纽约城市学院的艺术史教授。[6]
 
玻恩最初在布雷斯劳的国王威廉文理中学(König-Wilhelm-Gymnasium)接受教育,并于1901年进入布雷斯劳大学学习。德国的大学体系允许学生自由地在各大学之间转学,因此他分别于1902年夏季学期在海德堡大学、1903年夏季学期在苏黎世大学学习。在布雷斯劳的同学奥托·托普利茨(Otto Toeplitz)和恩斯特·赫林格(Ernst Hellinger)告诉玻恩有关哥廷根大学的信息,[7] 玻恩于1904年4月进入哥廷根大学。在那里,他遇到了三位著名的数学家:费利克斯·克莱因(Felix Klein)、大卫·希尔伯特(David Hilbert)和赫尔曼·闵可夫斯基(Hermann Minkowski)。  

玻恩到达哥廷根后不久,就与后两位数学家建立了密切关系。从他上希尔伯特的第一堂课起,希尔伯特就发现玻恩具有非凡的能力,并挑选他担任课程记录员,负责为哥廷根大学数学阅览室整理课程笔记。这一职位使玻恩与希尔伯特保持了定期且极具价值的联系。希尔伯特后来选择玻恩担任首位无薪的半官方助理,使他成为玻恩的导师。  

玻恩与闵可夫斯基的初次接触来自他的继母贝尔塔(Bertha),她曾在柯尼斯堡与闵可夫斯基一起参加舞蹈课。通过这层关系,玻恩获得了每周日受邀到闵可夫斯基家中共进晚餐的机会。此外,在担任课程记录员和助理期间,玻恩经常在希尔伯特的家中见到闵可夫斯基。[8][9]

玻恩与克莱因的关系较为复杂。他参加了一场由克莱因以及应用数学教授卡尔·龙格(Carl Runge)和路德维希·普朗特(Ludwig Prandtl)共同主持的关于弹性理论的研讨会。虽然玻恩对这个主题并不特别感兴趣,但他不得不在会上提交一篇报告。他选择了一个简单的案例:研究两端固定的弯曲线条,并利用希尔伯特的变分法计算出最小化势能、从而达到最稳定状态的配置。克莱因对此印象深刻,邀请玻恩以“平面与空间中的弹性稳定性”为题提交论文。这个题目正是克莱因非常感兴趣的研究领域,而且克莱因已经将其设定为哥廷根大学年度哲学系大奖(Philosophy Faculty Prize)的主题。该奖项不仅享有盛誉,而且提交的作品还可以作为博士论文。

然而,玻恩拒绝了这个邀请,因为他并不喜欢应用数学领域的研究。克莱因对此极为愤怒。[10][11]

克莱因在学术界拥有很大的影响力,可以决定学术职业的成败,因此玻恩感到必须弥补过失,最终同意为该奖项提交作品。但由于克莱因拒绝为其指导,玻恩转而请卡尔·龙格担任其导师。此外,沃尔德玛·福伊特(Woldemar Voigt)和卡尔·史瓦西(Karl Schwarzschild)成为了他的其他考官。玻恩从自己的报告入手,进一步发展了稳定性条件的方程。随着研究的深入,他对这一主题越来越感兴趣,并设计了一个实验装置以验证自己的理论预测。

1906年6月13日,校长宣布玻恩赢得了哲学系大奖。一个月后,他通过了口试,以优异成绩(magna cum laude)获得数学博士学位。[12]

毕业后,玻恩不得不履行学生时期延期的兵役义务。他被征召入德国军队,分配到驻扎在柏林的“俄国皇后亚历山德拉第二卫队龙骑兵团”。他的服役时间很短,因为1907年1月的一次哮喘发作使他提前退役。随后,他前往英国,被剑桥大学冈维尔与凯斯学院录取,在卡文迪许实验室师从J. J. 汤姆森、乔治·塞尔(George Searle)和约瑟夫·拉默(Joseph Larmor)学习物理六个月。

返回德国后,玻恩再次被军队征召,并服役于精英部队“第一(西里西亚)生命胸甲骑兵团‘大选帝侯’”。然而,他仅服役六周后再次因健康原因退役。退役后,他回到布雷斯劳,在奥托·卢默(Otto Lummer)和恩斯特·普林斯海姆(Ernst Pringsheim)的指导下工作,试图完成物理学的授课资格考试。然而,玻恩的黑体实验发生了一次小事故——冷却水管破裂导致实验室被水淹,卢默因此告诫他永远不可能成为一名物理学家。[13]

1905年,阿尔伯特·爱因斯坦发表了关于狭义相对论的论文《论动体的电动力学》。玻恩对此非常感兴趣,开始研究这一课题。然而,当他得知赫尔曼·闵可夫斯基(Hermann Minkowski)也在进行类似研究时,他感到非常沮丧。玻恩将自己的研究结果写信告诉闵可夫斯基后,闵可夫斯基邀请他返回哥廷根大学并在那里完成他的授课资格考试。玻恩接受了邀请。

在哥廷根,托普利茨(Toeplitz)帮助玻恩复习矩阵代数,以便他能够使用闵可夫斯基的四维时空矩阵完成将相对论与电动力学相结合的项目。玻恩和闵可夫斯基相处融洽,他们的工作进展顺利。然而,1909年1月12日,闵可夫斯基因阑尾炎突然去世。数学系的学生推举玻恩在闵可夫斯基的葬礼上代表他们发言。[14]

几周后,玻恩试图在哥廷根数学协会的一次会议上展示他们的研究成果。然而,他刚开始介绍,就被反对相对论的克莱因(Felix Klein)和马克斯·亚伯拉罕(Max Abraham)公开质疑,被迫中断了演讲。然而,希尔伯特(David Hilbert)和鲁恩格(Carl Runge)对玻恩的工作产生了兴趣,并在与玻恩讨论后确信他的研究结果的正确性,最终说服他再次发表演讲。这一次,玻恩的演讲没有受到打断,弗伊特(Woldemar Voigt)还主动提出支持玻恩的授课资格论文。[15] 

随后,玻恩将他的演讲内容发表为一篇题为《相对性原理运动学中的刚性电子理论》(\textbf{德文:Die Theorie des starren Elektrons in der Kinematik des Relativitätsprinzips})的文章,[16] 其中首次引入了\textbf{玻恩刚性}的概念。10月23日,玻恩就汤姆森原子模型发表了授课资格讲座。[15]
\subsection{职业生涯}  
\subsubsection{柏林与法兰克福}  
玻恩作为一名青年学者在哥廷根定居,担任私人讲师(Privatdozent)。在哥廷根,玻恩住在一间位于达尔曼街17号(Dahlmannstraße 17)的小型寄宿公寓,由安妮修女(Sister Annie)经营,这间公寓被称为\textbf{El BoKaReBo}。名字来源于住户姓氏的首字母:“El”代表医学专业学生埃拉·菲利普森(Ella Philipson),“Bo”代表玻恩(Born)和物理学学生汉斯·博尔扎(Hans Bolza),“Ka”代表年轻讲师西奥多·冯·卡门(Theodore von Kármán),“Re”代表另一名医学学生阿尔布雷希特·雷纳(Albrecht Renner)。经常光顾这间公寓的还有阿诺·索末菲(Arnold Sommerfeld)的博士生保罗·彼得·埃瓦尔德(Paul Peter Ewald),他是索末菲外借给希尔伯特(Hilbert)的特别助理。数学家兼私人讲师理查德·柯朗(Richard Courant)称这些人是“圈内人”。[17]  

1912年,玻恩结识了赫德维希(赫迪)·埃伦伯格(Hedwig “Hedi” Ehrenberg)。她是莱比锡大学法学教授的女儿,同时也是卡尔·鲁恩格(Carl Runge)女儿伊丽丝(Iris)的朋友。赫迪的父亲是犹太背景,但他在结婚时成为了信奉路德教的基督徒,玻恩的妹妹凯特(Käthe)也是如此。尽管玻恩从未实践过宗教,但他拒绝皈依其他信仰。他们的婚礼于1913年8月2日在花园中举行。不过,玻恩在1914年3月接受了路德教的洗礼,洗礼仪式由主持他们婚礼的同一牧师完成。玻恩认为“宗教宣言和教会无关紧要”。[18] 他决定接受洗礼,部分是为了尊重妻子,部分是为了融入德国社会。[18]  

这段婚姻育有三个孩子:两个女儿,分别是1914年出生的艾琳(Irene)和1915年出生的玛格丽特(昵称格丽特莉,Margarethe/Gritli),以及1921年出生的儿子古斯塔夫(Gustav)。[19] 通过婚姻,玻恩与法学家维克托·埃伦伯格(Victor Ehrenberg,岳父)和鲁道夫·冯·耶林(Rudolf von Jhering,妻子外祖父)有亲戚关系,还与哲学家和神学家汉斯·埃伦伯格(Hans Ehrenberg)有关。此外,玻恩还是英国喜剧演员本·埃尔顿(Ben Elton)的叔祖。[20]

到1913年底,玻恩已发表了27篇论文,包括关于相对论和晶格动力学的重要研究(其中3篇与西奥多·冯·卡门合作),[21] 后者最终成为一本书。[22] 1914年,他收到马克斯·普朗克的来信,信中提到柏林大学新设立了一席理论物理学副教授职位。这个职位原本邀请马克斯·冯·劳厄担任,但他拒绝了,玻恩接受了邀请。[23] 此时第一次世界大战正如火如荼。1915年,他抵达柏林后不久便加入了陆军通信部队。同年10月,他加入了由鲁道夫·拉登堡领导的\textbf{火炮试验委员会}(Artillerie Prüfungskommission),这是一支位于柏林的陆军火炮研究与开发组织。拉登堡成立了一个专门研究新技术声测量的部门。在柏林期间,玻恩与爱因斯坦建立了终生的友谊,爱因斯坦成为玻恩家的常客。[24] 1918年11月停战协定签署后仅几天,普朗克便设法让陆军释放了玻恩。同月,他偶然与弗里茨·哈伯相遇,讨论了一种金属与卤素反应生成离子化合物的机制,这一研究后来被称为\textbf{玻恩-哈伯循环}。[25]  

甚至在玻恩正式接任柏林的教席前,冯·劳厄改变了主意,决定自己想要这一职位。[23] 他与玻恩及相关院系商定,两人互换职位。1919年4月,玻恩成为法兰克福大学科学学院理论物理研究所的正式教授(professor ordinarius)及所长。[22] 在此期间,玻恩受到哥廷根大学的邀请,替代彼得·德拜担任物理研究所所长。[26] 爱因斯坦建议他说:“理论物理学将在你所在的地方繁荣;当今的德国,再也找不到第二个玻恩了。”[27] 在与教育部谈判时,玻恩为其长期朋友兼同事詹姆斯·弗兰克争取到了哥廷根的另一席实验物理学教授职位。[26]  

1919年,伊丽莎白·博尔曼加入理论物理研究所,担任玻恩的助理。[28] 她开发了第一批原子束。与玻恩合作时,博尔曼首次测量了气体中原子的自由路径和分子的大小。[29][30]
\subsubsection{哥廷根}
在马克斯·玻恩和詹姆斯·弗兰克在哥廷根大学任职的12年间(1921年至1933年),玻恩拥有了一位在基础科学概念上持相似观点的合作伙伴——这一点对于教学和研究都十分有益。玻恩与实验物理学家的合作方式类似于阿诺德·索末菲在慕尼黑大学的做法。索末菲是理论物理学的正教授,也是理论物理研究所的主任,他同样是量子理论发展的重要推动者。玻恩和索末菲都与实验物理学家密切合作,以验证和推进他们的理论。

1922年,索末菲在美国威斯康星大学麦迪逊分校讲学期间,将自己的学生维尔纳·海森堡派往玻恩处担任助手。海森堡于1923年返回哥廷根,在1924年完成了在玻恩指导下的资格论文,并成为哥廷根的一名私人讲师(Privatdozent)。[31][32]

在1919年和1920年,马克斯·玻恩因针对爱因斯坦相对论的大量反对意见感到不满,并于1919年冬季发表演讲支持爱因斯坦。玻恩通过这些关于相对论的演讲获得报酬,这在当时快速通货膨胀的年份里帮助他应对经济开支。这些德语演讲后来于1920年出版成书,出版前爱因斯坦还看过书稿。第三版于1922年出版,英语翻译版本则于1924年出版。玻恩在书中将光速描述为曲率的函数,[35] 并指出“在某些光线方向上,光速远远大于其通常值 c,其他物体也能达到更高的速度。”[36]

1925年,玻恩和海森堡提出了量子力学的矩阵力学表示法。7月9日,海森堡将一篇题为《关于动力学和力学关系的量子理论重新解释》(**Über quantentheoretische Umdeutung kinematischer und mechanischer Beziehungen**)的论文交给玻恩审阅并提交发表。在论文中,海森堡通过避免使用具体但不可观察的电子轨道表示,用量子跃迁的过渡概率等参数重新构建量子理论,这需要使用对应初态和终态的双索引。[37][38] 当玻恩阅读论文时,他认识到可以将这种表述转化为矩阵的系统语言并加以扩展。[39] 这与他在布雷斯劳大学师从雅各布·罗萨内斯时所学的矩阵理论知识相关。[40]

在此之前,矩阵很少被物理学家使用,它们被认为属于纯数学的领域。古斯塔夫·米耶(Gustav Mie)在1912年曾在电动力学论文中使用过矩阵,而玻恩在1921年的晶体晶格理论研究中也使用过矩阵。然而,在这些案例中,矩阵的代数性质,尤其是矩阵乘法,并未被纳入研究框架中,就像它们在量子力学的矩阵表述中那样被充分利用。[41] 

在助教兼前学生帕斯库尔·约当(Pascual Jordan)的帮助下,玻恩立即着手对海森堡的研究进行转录和扩展。他们的研究成果迅速完成并提交发表,其论文在海森堡的论文发表后仅60天内就被接受出版。[42] 随后,由三位作者共同完成的后续论文在年末前提交。[43] 

最终的结果是一个令人惊讶的表述:
\[
pq - qp = \frac{h}{2\pi i}I~
\]
其中,\(p\) 和 \(q\) 分别表示位置和动量的矩阵,\(I\) 是单位矩阵。方程左侧不为零,是因为矩阵乘法并不满足交换律。[40] 这一公式完全归功于玻恩,他还确立了矩阵中对角线以外的元素均为零的事实。玻恩认为,他与约当共同发表的论文包含了量子力学的“最重要的基本原理”,甚至扩展到了电动力学领域。[40] 这篇论文为海森堡的方法奠定了坚实的数学基础。[44]

玻恩惊讶地发现,保罗·狄拉克(Paul Dirac)也沿着与海森堡类似的思路进行思考。不久之后,沃尔夫冈·泡利(Wolfgang Pauli)利用矩阵方法计算了氢原子的能级,并发现这些结果与玻尔模型一致。与此同时,另一个重要的贡献来自埃尔温·薛定谔(Erwin Schrödinger),他用波动力学的方法研究了这个问题。由于波动力学似乎可以回归确定性的经典物理学,这在当时吸引了许多人的兴趣。然而,玻恩对此并不认同,因为这一观点与实验所确定的事实相矛盾。[40] 

玻恩于1926年7月发表了关于薛定谔方程中概率密度函数 \( \psi^*\psi \) 的标准解释,这一理论至今广为接受。[45][44]

1926年12月4日,爱因斯坦在写给玻恩的信中提出了他对量子力学的著名评论:

> “量子力学无疑令人印象深刻,但我的内心声音告诉我,这还不是终极真理。这个理论讲述了很多东西,但并没有真正让我们更接近‘老天爷’的秘密。无论如何,我坚信‘祂’不会玩骰子。”[46]

这一引述常被简化为“上帝不掷骰子”。[47]

1928年,爱因斯坦提名海森堡、玻恩和约旦角逐诺贝尔物理学奖。[48][49] 最终,海森堡在1932年独自获得该奖,以表彰他“创立量子力学并应用于发现氢的同素异形体”。[50] 随后,薛定谔和狄拉克于1933年共同获得诺贝尔奖,以表彰他们“在原子理论中发现新的有效形式”。[50]

1933年11月25日,玻恩收到海森堡的一封信。信中海森堡提到,他因“良心不安”而延迟了写信,因为他独自获得了该奖,而这是“在哥廷根与您、约旦和我共同完成的工作”。[51] 海森堡还强调,玻恩和约旦对量子力学的贡献不会因“外界的错误决定”而被抹杀。[51] 1954年,海森堡在一篇纪念普朗克1900年洞见的文章中,明确肯定了玻恩和约旦在矩阵力学数学形式化中的重要作用,并强调他们对量子力学的巨大贡献未能在公众视野中得到充分承认。[52]

在玻恩领导下,许多著名学者在哥廷根完成了博士学位,包括马克斯·德尔布吕克、齐格弗里德·弗吕格、弗里德里希·洪德、帕斯夸尔·约旦、玛丽亚·格佩特-梅耶、洛塔·沃尔夫冈·诺德海姆、罗伯特·奥本海默和维克托·魏斯科普夫。[53][54] 在哥廷根大学理论物理研究所,玻恩的助手包括恩里科·费米、维尔纳·海森堡、格哈德·赫茨贝格、弗里德里希·洪德、帕斯夸尔·约旦、沃尔夫冈·泡利、莱昂·罗森费尔德、爱德华·泰勒和尤金·维格纳。[55] 瓦尔特·海特勒于1928年成为玻恩的助手,并于1929年在他指导下完成了资格论文。

玻恩不仅善于发掘与他合作的天才,而且“允许那些超越他的人自由发展;对于天赋稍逊者,他耐心地布置适合其能力的课题”。[56] 德尔布吕克和格佩特-梅耶后来都获得了诺贝尔奖。[57][58]
\subsection{晚年生活}
1933年1月,纳粹党在德国上台。同年5月,玻恩成为哥廷根大学六位被暂停职务的犹太教授之一,暂停期间仍然领取薪水;而弗朗克此前已经辞职。在过去的十二年里,他们将哥廷根打造成了世界上最重要的物理学中心之一。[59] 玻恩开始寻找新工作,并写信给约翰·霍普金斯大学的玛丽亚·格佩特-梅耶和普林斯顿大学的鲁迪·拉登堡。他接受了剑桥大学圣约翰学院的邀请。[60] 

在剑桥,他撰写了一本面向大众的科普书《不安的宇宙》(*The Restless Universe*),以及一本很快成为标准教材的教科书《原子物理学》(*Atomic Physics*),后者共有七个版本。他的家人很快适应了英国的生活,他的两个女儿伊琳和格莉特莉分别与威尔士人布林利(布林)·牛顿-约翰和英国人莫里斯·普赖斯订婚。玻恩的外孙女奥利维亚·牛顿-约翰是伊琳的女儿。[61][62][63]

玻恩在剑桥的职位只是临时的,他在哥廷根的任期于1935年5月终止。因此,他接受了C.V.拉曼的邀请,于1935年前往班加罗尔。[64] 玻恩曾考虑在那里担任永久职位,但印度科学研究所未能为他设立一个额外的教授职位。[65] 1935年11月,玻恩一家被取消了德国公民身份,成为无国籍人士。几周后,哥廷根大学撤销了玻恩的博士学位。[66] 玻恩考虑了彼得·卡皮查在莫斯科的邀请,并开始从鲁道夫·佩尔斯的俄裔妻子根妮娅那里学习俄语。然而,查尔斯·高尔顿·达尔文询问玻恩是否愿意接替他,担任爱丁堡大学的泰特自然哲学教授。玻恩迅速接受了这一邀请,[67] 并于1936年10月就任。[62]

在爱丁堡,玻恩推动了数学物理的教学。他有两位德国助手:E. Walter Kellermann 和 Klaus Fuchs,以及一位苏格兰助手 Robert Schlapp,[68] 他们共同继续研究电子的神秘行为。[69] 玻恩于1937年成为爱丁堡皇家学会院士,并于1939年3月成为伦敦皇家学会院士。在1939年,玻恩尽力将留在德国的朋友和亲属带出该国,包括他的妹妹凯瑟、姻亲库尔特和玛加,以及朋友海因里希·劳施·冯·特劳本贝格的女儿们。赫迪经营一个家政事务所,为年轻的犹太女性安排工作。1939年8月31日,在欧洲第二次世界大战爆发的前一天,玻恩获得了英国公民的归化证书。[70]

玻恩一直在爱丁堡任职,直到1952年70岁退休。他于1954年搬到西德的巴德皮尔蒙特安享晚年。[71] 同年10月,他接到通知,获颁诺贝尔奖。他的物理学同行从未停止提名他。弗兰克和费米曾在1947年和1948年提名他,表彰其在晶体格方面的研究,多年来他也因在固体物理、量子力学和其他领域的工作而获得提名。[72] 1954年,他因“在量子力学中的基础研究,尤其是对波函数的统计解释”而获奖,[1] 这项工作完全是他独自完成的。[72] 在他的诺贝尔讲座中,他反思了其研究的哲学意义:  

我相信,诸如绝对确定性、绝对精确性、最终真理等观念只是想象的产物,不应在科学的任何领域被接受。另一方面,从理论角度看,任何关于概率的断言都是对或错。这种思想上的松弛(\textbf{Lockerung des Denkens})在我看来是现代科学给予我们的最大祝福。因为对单一真理的信仰以及自认为掌握了真理,是世界上一切罪恶的根源。[73]

退休后,他继续从事科学工作,并为其著作推出新版。1955年,他成为《罗素-爱因斯坦宣言》的签署者之一。他于1970年1月5日在哥廷根的医院去世,享年87岁,[2] 安葬于哥廷根的城市公墓,与瓦尔特·能斯特、威廉·韦伯、马克斯·冯·劳厄、奥托·哈恩、马克斯·普朗克和大卫·希尔伯特同眠。[74]

\subsection{}
\subsection{全球政策}  
他是签署召开制定世界宪法大会协议的签署人之一。[75][76] 这一举措促成了人类历史上首次召开的**世界宪法议会大会**,负责起草并通过《地球联邦宪法》。[77]