% 向量空间的对称/反对称幂
% keys 对称幂|反对称幂
% license Xiao
% type Wiki

% \begin{issues}
% \end{issues}

\pentry{向量空间的张量积\upref{vecTsr},对称/反对称多线性映射\upref{SASmap}}

\subsection{作为子空间的对称/反对称幂}

\subsubsection{对称幂}

我们可以在 $V^{\times n}$ 上定义自然置换作用\autoref{ex_Group3_6}~\upref{Group3},类似的 $n$ 阶张量幂空间 $V^{\otimes n}$ 上也可以定义一个 $S_n$ \upref{Perm}的群作用:
\begin{equation}
\begin{aligned}
\rho(\sigma): V^{\otimes n} &\to V^{\otimes n}~, \\
v_1 \otimes \cdots \otimes v_n &\mapsto v_{\sigma(1)} \otimes \cdots \otimes v_{\sigma(n)}~.
\end{aligned}
\end{equation}


注意:我们把$\rho(\sigma)$ 被定义为线性的,因此只需要考虑 $V^{\otimes n}$ 的一组基的映射就可以了,完整的写法为 $\sum a_v v_1 \otimes \cdots \otimes v_n \mapsto \sum a_v v_{\sigma(1)} \otimes \cdots \otimes v_{\sigma(n)}$,下同。

\begin{example}{}
考虑 $n = 2$,$S_2 = \mathbb{Z}/2\mathbb{Z} = \{e, (1 2)\}$,$\rho(e)$是恒等映射,而
\begin{equation}
\begin{aligned}
\rho((1 2)): V \otimes V &\to V \otimes V~, \\
v_1 \otimes v_2 &\mapsto v_2 \otimes v_1~.
\end{aligned}
\end{equation}
\end{example}

特别的,我们把 $n$ 阶张量幂空间 $V^{\otimes n}$ 的不动点集 $(V^{\otimes n})^{S_n}$ (\autoref{def_Group3_2}~\upref{Group3})称为 $V$ 的 $n$ 阶\textbf{对称幂空间},记做 $\opn{Sym}^n V$ 或者 $S^n(V)$。

\begin{exercise}{}
$v_1 \otimes \cdots \otimes v_n \in \opn{Sym}^n V$ 当且仅当对任意的的 $i, j$,
\begin{equation}
\cdots \otimes v_i \otimes \cdots \otimes v_j \otimes \cdots = \cdots \otimes v_j \otimes \cdots \otimes v_i \otimes \cdots~.
\end{equation}

\end{exercise}

\begin{example}{}\label{ex_vecSAS_1}
考虑 $V = \mathbb{R}^2$;$\opn{Sym}^2 V = \langle e_1 \otimes e_1, e_2 \otimes e_2, e_1 \otimes e_2 + e_2 \otimes e_1 \rangle$ 是一个三维向量空间。
\end{example}

我们定义(向量的)\textbf{对称积}\footnote{如果域的特征 $\opn{char} \mathbb{F}$ 大于零(比如 $\mathbb{F}_p$),$\frac{1}{2}$的存在性依赖于 $\opn{char} \mathbb{F}$ 是否大于$2$,更一般的,$\frac{1}{n!}$ 存在要求$\opn{char} \mathbb{F} > n$;当我们只考虑 $\mathbb{R}, \mathbb{C}$的时候,不需要考虑这些问题。}
\begin{equation}
\begin{aligned}
\cdot: V \times V &\to V^{\otimes 2}~, \\
v \cdot w &:= \frac12 (v \otimes w + w \otimes v)~,
\end{aligned}
\end{equation}
因此\autoref{ex_vecSAS_1} 中 $\opn{Sym}^2 V$ 的基向量可以写成 $e_1 \cdot e_1, e_2 \cdot e_2$ 和 $e_1 \cdot e_2$ (在基向量的意义下,系数不重要)。

注意:我们会说一个向量\textbf{空间} $V$ 的(反)对称\textbf{幂空间},以及两个\textbf{向量}的对称\textbf{积},有时我们会把二阶对称幂空间 $\text{Sym}^2 V$ 称为向量空间 $V$ 的对称积空间,但是这并不严谨。

更一般的,我们可以定义多项对称积
\begin{equation}
\begin{aligned}
\cdot: V^{\times n} &\to V^{\otimes n}~, \\
v_1 \cdots v_n &:= \frac{1}{n!} \sum_{\sigma \in S_n} v_{\sigma(1)} \otimes \cdots \otimes v_{\sigma(n)}~,
\end{aligned}
\end{equation}
比如当 $n = 3$ 时,
\begin{equation}
\begin{aligned}
v_1 \cdot v_2 \cdot v_3 = &\frac16 (v_1 \otimes v_2 \otimes v_3 \\
&+ v_1 \otimes v_3 \otimes v_2 \\
&+ v_2 \otimes v_1 \otimes v_3 \\
&+ v_2 \otimes v_3 \otimes v_1 \\
&+ v_3 \otimes v_1 \otimes v_2 \\
&+ v_3 \otimes v_2 \otimes v_1)~,
\end{aligned}
\end{equation}

可以证明,$v_1 \cdot v_2 \cdots = v_1 \cdot v_3 \cdot v_2 \cdots = \cdots$ 在 $S_n$ 的置换作用下固定,这意味着$v \cdots w \in \text{Sym}^n V$;

取 $V$ 的一组基 $\{e_1, \dots, e_k\}$,可以找到 $\opn{Sym}^n V$ 的一组基
\begin{equation}
\left\{ e_{i_1} \cdot \cdots \cdot e_{i_n} \mid 1 \leq i_1 \leq \dots \leq i_n \leq k \right\}~,
\end{equation}
特别的,$\dim(\opn{Sym}^n V) = \pmat{k + n - 1 \\ n}$(隔板法\upref{BarCom})。

\subsubsection{反对称幂(又称交错幂、外幂)}

张量幂空间 $V^{\otimes n}$ 上存在另一个 $S_n$ 的群作用:
\begin{equation}
\begin{aligned}
\rho(\sigma): V^{\otimes n} &\to V^{\otimes n}~, \\
v_1 \otimes \cdots \otimes v_n &\mapsto \opn{sgn}(\sigma) v_{\sigma(1)} \otimes \cdots \otimes v_{\sigma(n)}~.
\end{aligned}
\end{equation}
其中对于偶置换$\opn{sgn}{\sigma} = 1$、奇置换$\opn{sgn}{\sigma} = -1$;我们把它的不动点集称为 $V$ 的 $n$ 阶\textbf{反对称幂空间}或称\textbf{交错幂}、\textbf{外幂}),记做 ${\large \wedge}^n V$。

同样由于对称群 $S_n$ 由对换生成,我们只需要考虑“交换两项变号”即可。
\begin{theorem}{}
$v_1 \otimes \cdots \otimes v_n \in {\large \wedge}^n V$ 当且仅当对任意的的 $i, j$,
\begin{equation}
\cdots \otimes v_i \otimes \cdots \otimes v_j \otimes \cdots = - \cdots \otimes v_j \otimes \cdots \otimes v_i \otimes \cdots~.
\end{equation}
\end{theorem}

% Giacomo:是否有必要写?
% TODO:交错性($v_i = v_j \implies v_1 \otimes \cdots \otimes v_n = 0$)与反对称性等价

\begin{example}{}\label{ex_vecSAS_2}
考虑 $V = \mathbb{R}^2$;反对称幂子空间 ${\large \wedge}^2 V = \langle e_1 \otimes e_2 - e_2 \otimes e_1 \rangle$,因为
\begin{equation}
\begin{aligned}
&\phantom{=} \rho((1 2))(e_1 \otimes e_2 - e_2 \otimes e_1) \\
&= \rho((1 2))(e_1 \otimes e_2) - \rho((1 2))(e_2 \otimes e_1) \\
&= - e_2 \otimes e_1 + e_1 \otimes e_2
\end{aligned}~
\end{equation}
\end{example}

我们定义(向量的)\textbf{反对称积}(或称\textbf{交错积}、\textbf{外积})
\begin{equation}
\begin{aligned}
\wedge: V \times V &\to V^{\otimes 2}~, \\
v \wedge w &:= \frac12 (v \otimes w - w \otimes v)~,
\end{aligned}
\end{equation}
因此\autoref{ex_vecSAS_2} 中 ${\large \wedge}^2 V$ 的基向量可以写成 $e_1 \wedge e_2$。

反对称积满足,
\begin{itemize}
\item $v \wedge w = - w \wedge v$,特别的,
\item $v \wedge v = 0$;
\end{itemize}
这意味着$v \wedge w \in {\large \wedge}^2 V$。

更一般的,我们可以定义多项反对称积
\begin{equation}
\begin{aligned}
\wedge: V^{\times n} &\to V^{\otimes n}~, \\
v_1 \wedge \cdots \wedge v_n &:= \frac{1}{n!} \sum_{\sigma \in S_n} \text{sgn}(\sigma) v_{\sigma(1)} \otimes \cdots \otimes v_{\sigma(n)}~,
\end{aligned}
\end{equation}

对任意的 $\mu \in S_n$,
\begin{equation}
\begin{aligned}
\rho(\mu)(v_1 \wedge \cdots \wedge v_n) &= \frac{1}{n!} \sum_{\sigma \in S_n} \text{sgn}(\sigma) \rho(\mu)(v_{\sigma(1)} \otimes \cdots \otimes v_{\sigma(n)}) \\
&= \frac{1}{n!} \sum_{\sigma \in S_n} \text{sgn}(\sigma) \text{sgn}(\mu) (v_{\mu(\sigma(1))} \otimes \cdots \otimes v_{\mu(\sigma(n))}) \\
&= \frac{1}{n!} \sum_{\sigma \in S_n} \text{sgn}(\mu \sigma) (v_{\mu \sigma(1)} \otimes \cdots \otimes v_{\mu \sigma(n)}) \\
&= v_1 \wedge \cdots \wedge v_n~,
\end{aligned}
\end{equation}
因此 $v_1 \wedge \cdots \wedge v_n \in {\large \wedge}^n V$。
% Giacomo:
% 如果读者反应不能理解,可以添加更详细的证明

我们还可以定义两个反对称积(的结果)之间的的反对称积:
\begin{equation}
\begin{aligned}
\wedge: {\large \wedge}^{n_1} V \times {\large \wedge}^{n_2} V &\to {\large \wedge}^{n_1 + n_2} V~, \\
\nu \wedge \omega &:= v_1 \wedge \cdots \wedge v_{n_1} \wedge w_1 \wedge \cdots \wedge w_{n_2}~,
\end{aligned}
\end{equation}
此时的反对称积“并不满足反对称律”,取而代之的是
\begin{equation}
\nu \wedge \omega = (-1)^{n_1 n_2} \omega \wedge \nu~.
\end{equation}

最后,类似对称积的情况,取 $V$ 的一组基 $\{e_1, \dots, e_k\}$,可以找到 ${\large \wedge}^n V$ 的一组基
\begin{equation}
\left\{ e_{i_1} \wedge \cdots \wedge e_{i_n} \mid 1 \leq i_1 < \dots < i_n \leq k \right\}~,
\end{equation}
特别的,$\dim({\large \wedge}^n V) = \pmat{k \\ n}$,因此如果 $n > k$,${\large \wedge}^n V$ 就是零空间了。

\subsubsection{二阶张量幂空间的对称反对称分解}

由于 $k$ 维向量空间 $V$ 的二阶张量幂空间可以被分解成对称幂空间和反对称幂空间的直和
\begin{equation}
\begin{aligned}
V^{\otimes 2} &= \opn{Sym}^2 V \oplus {\large \wedge}^2 V \\
v_1 \otimes v_2 &= v_1 \cdot v_2 + v_1 \wedge v_2~.
\end{aligned}
\end{equation}

另一方面, $k$ 维向量空间 $V$ 的二阶张量幂空间 $V^{\otimes 2}$ 的维度为 $k^2$,我们有
\begin{equation}
\begin{aligned}
\pmat{k + 1 \\ 2} + \pmat{k \\ 2} &= \frac{(k + 1)!}{(k - 1)! 2} + \frac{k!}{(k - 2)! 2} \\
&= \frac{(k + 1) k}{2} + \frac{k (k - 1)}{2} \\
&= k^2~.
\end{aligned}
\end{equation}

不过,这个性质对于更高阶的情况不再成立,比如当 $n = 3$ 时
\begin{equation}
\begin{aligned}
\pmat{k + 2 \\ 3} + \pmat{k \\ 3} &= \frac{(k + 2)!}{(k - 1)! 6} + \frac{k!}{(k - 3)! 6} \\
&= \frac{k}{3} ((k + 2)(k + 1) + (k - 1) (k - 2)) \\
&= \frac{k}{3} (k^2 + 4) \\
&< k^3~.
\end{aligned}
\end{equation}


\subsection{作为商空间的对称/反对称幂}

我们可以换一个视角来看待对称/反对称幂。考虑对称积映射 $\cdot: V^{\times n} \to V^{\otimes n}$,由于
\begin{itemize}
\item 它的值域为 $\opn{Sym}^n V$,而且
\item 它是一个多线性映射,
\end{itemize}
我们可以把它改写为一个线性满同态 $\opn{Sym}: V^{\otimes n} \twoheadrightarrow \opn{Sym}^n V$,我们有
\begin{equation}
\ker(\opn{Sym}) = \{v - \rho_\text{perm}(g) v \mid v \in V^{\otimes n}, g \in S_n\}~,
\end{equation}
换言之,我们可以把 $\opn{Sym}^n V$理解成 $V^{\otimes n}$ 的商空间;

类似的,反对称积映射 $\wedge: V^{\times n} \to V^{\otimes n}$ 可以改写成线性满同态 ${\large \wedge}: V^{\otimes n} \twoheadrightarrow {\large \wedge}^n V$,同样有
\begin{equation}
\ker({\large \wedge}) = \{v - \rho_\text{sign}(g) v \mid v \in V^{\otimes n}, g \in S_n\}~.
\end{equation}

\subsubsection{对称/反对称幂的万有性质}

\addTODO{万有性质的词条,包含初/终性质}
对称幂映射 $\opn{Sym}: V^{\times n} \to \opn{Sym}^n V$ 是一个对称映射,而且 $\opn{Sym}^n V$ 是所有到达域中“最小的”的(初对象,相关词条有待补充):任意对称映射 $f: V^{\times n} \to W$,我们都有唯一确定的线性映射 $\bar{f}: \opn{Sym}^n V \to W$,满足 $f = \bar{f} \circ \opn{Sym}$,我们有交换图
\begin{equation}
\begin{CD}
V^{\times n} @>{\opn{Sym}}>> \opn{Sym}^n V \\
@V{f}VV @V{\bar{f}}VV \\
W @= W
\end{CD}~.
\end{equation}
这个性质被被称为对称映射的万有性质,可以对照张量积的万有性质\autoref{def_vecTsr_1}~\upref{vecTsr}进行理解。

类似的,反对称幂映射 ${\large \wedge}: V^{\times n} \to {\large \wedge}^n V$ 是一个反对称映射,任意反对称映射 $f: V^{\times n} \to W$,我们都有唯一确定的线性映射 $\bar{f}: {\large \wedge}^n V \to W$,满足 $f = \bar{f} \circ {\large \wedge}$,我们有交换图
\begin{equation}
\begin{CD}
V^{\times n} @>{{\large \wedge}}>> {\large \wedge}^n V \\
@V{f}VV @V{\bar{f}}VV \\
W @= W
\end{CD}~.
\end{equation}
这个性质被可以被称为反对称映射的万有性质。

万有性质保证了对称/反对称幂空间在同构意义下的唯一性,进而证明了子空间构造和商空间构造是同构的。