% 几何光学介绍
% license Usr
% type Tutor

\begin{issues}
\issueTODO 波粒二象性之前的光学发展
\end{issues}

20世纪初,由于量子论的提出与发展,人们意识到,光的干涉、衍射与偏振现象所证实的光的波动性,以及由黑体辐射、光电效应与康普顿效应所证实的光的量子性——粒子性,都客观反映着光的本质。光实际上具有波粒二象性——波动性和粒子性这两种看上去完全不相同的属性的统一——这个观点使人们对光的本质有了更加深刻的认识。

光的波粒二象性即对应着光学的两个重要组成部分——波动光学和几何光学。波动光学以光的电磁场理论为基础,根据麦克斯韦方程,研究光在各种介质中的传播规律。而几何光学实际上忽略了光的波动效应,即认为光的波长趋近于零,引入光线的概念,研究光的常规宏观尺度下,在各种常规光学元件中的传播规律。
