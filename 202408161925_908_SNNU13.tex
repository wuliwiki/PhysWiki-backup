% 陕西师范大学 2013 年 考研 量子力学
% license Usr
% type Note

\textbf{声明}:“该内容来源于网络公开资料,不保证真实性,如有侵权请联系管理员”

\subsection{填空(每小题3分,共30分)}
\begin{enumerate}
    \item 根据德布罗意假设,对于一定能量 $E$ 和一定动量 $p$ 的粒子,与它相联系的是频率为 $\nu$,波长为 $\lambda$ 的平面波,它们之间的关系为 $E = \underline{\hspace{2cm}}$,$p = \underline{\hspace{2cm}}$。
    
    \item 泡利算符的对易关系是 $\hat \sigma_x \hat \sigma_y - \hat \sigma_y \hat \sigma_x = \underline{\hspace{2cm}}$,泡利算符的反对易关系是 $\hat \sigma_x \hat \sigma_y + \hat \sigma_y \hat \sigma_x = \underline{\hspace{2cm}}$。
    
    \item 氢原子态函数波函数 $\psi_{311} = R_{31}(r) Y_{11}(\theta, \varphi)$ 所描述的态,其轨道角动量的长度为 $\underline{\hspace{2cm}}$,轨道角动量在 z 轴上的投影为 $\underline{\hspace{2cm}}$。
    
    \item 力学量 $F$ 的本征值方程 $F\psi = F\psi$ 中,如果对 $F$ 的一个本征值,有 $n$ 个相互独立(线性无关)的本征函数数,我们就说该本征值 $F$ 是 $\underline{\hspace{2cm}}$ 的,简并度为 $\underline{\hspace{2cm}}$。
    
    \item 在动力学表达式中,动量算符的本征方程表示为 $\underline{\hspace{4cm}}$,坐标算符的本征方程表示为 $\underline{\hspace{4cm}}$。
    
    \item 由 $n$ 个全同粒子组成的体系中,不能有两个或两个以上的粒子处于 $\underline{\hspace{2cm}}$,这就是泡利不相容原理。

\end{enumerate}