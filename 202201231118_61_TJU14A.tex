% 天津大学 2014 年考研量子力学答案
% keys 考研|天津大学|量子力学|2014|答案

\subsection{ }
\begin{enumerate}
\item 对于$\psi(r,\theta,\varphi) = \frac{1}{\sqrt{5}}\psi_{310} + \frac{2}{\sqrt{5}}\psi_{211} $,主量子数$n$可能等于$3,2$.\\
\begin{table}[ht]
\centering
\caption{$\hat{L}_{z},\hat{L}^{2}$的可能值与几率}\label{TJU14A_tab1}
\begin{tabular}{|c|c|c|}
\hline
$\hat{L}_z$ 的可能值 & 0 & $\hbar$  \\
\hline
$\hat{L}^2$ 的可能值 & $2\hbar^{2}$ & $2\hbar^{2}$  \\
\hline
相应几率 & $\frac{1}{5}$ & $\frac{4}{5}$  \\
\hline
\end{tabular}
\end{table}
$\hat{L}_{z}$和$\hat{L}^{2}$的平均值为:\\
\begin{align}
& \overline{\hat{L}_{z}} = 0 \times \frac{1}{5} + \hbar \times \frac{4}{5} = \frac{4\hbar}{5} \\
& \overline{\hat{L}^{2}} = 2\hbar^{2} \times \frac{1}{5} + 2\hbar^{2} \times \frac{4}{5} = 2\hbar^{2}
\end{align}
\item 
\end{enumerate}