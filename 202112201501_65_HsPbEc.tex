% 概率习题(高中)
% 高中|概率|习题

\begin{issues}
\issueDraft
\end{issues}
\subsection{例1}
(全国乙卷数学理6)将5名北京奥运会志愿者分配到花样滑冰、短道速滑、冰球和冰壶4个项目进行培训,每名志愿者只分到1个项目,每个项目至少分配1名志愿者,则不同的分配方案共有()\\
A.60种\\
B.120种\\
C.240种\\
D.480种

\subsubsection{解答:}\\
由题,5名运动员要分成4组,我们需要挑出两个人分为一组,其余一人一组
\begin{equation}
C_5^2 = 10 
\end{equation}
也就是说有十种分组方式.\\
再对四组进行全排
\begin{equation}
A_4^4 = 24
\end{equation}
\begin{equation}
C_5^2A_4^4 =240 
\end{equation}
故答案选A.

\subsection{例2}
(2020年高考一卷19)甲、乙、丙三位同学进行羽毛球比赛,约定赛制如下:

累计负两场者被淘汰;比赛前抽签决定首先比赛的两人,另一人轮空;每场比赛的胜者与轮空者进行下一场比赛,负者下一轮轮空,直至有一人被淘汰;当一人被淘汰后,剩余的两人继续比赛,直至其中一人被淘汰,另一人最终获胜,比赛结束.

经抽签,甲、乙首先比赛,丙轮空. 设每场比赛双方获胜的概率都为 $\frac{1}{2}$.

(1)  求甲连胜四场的概率;

(2)  求需要进行第五场比赛的概率;

(3)  求丙最终获胜的概率.
\subsubsection{解答:}\\
(1)(第一问没有什么难度,注意答题格式规范)\\
设甲连胜四场为事件 $A$
\begin{equation}
P(A) = (\frac{1}{2})^4 = \frac{1}{16}
\end{equation}
(2)
\begin{figure}[ht]
\centering
\includegraphics[width=14.25cm]{./figures/HsPbEc_1.png}
\caption{图解2-2} \label{HsPbEc_fig1}
\end{figure}
如图所示,进行5场比赛有两种情况,负者1胜一定进行5场,负者1负,有 $\frac{1}{2}$ 的概率进行五场\\
设进行5场比赛为事件 $B$
\begin{equation}
P(B) = \frac{1}{2} + \frac{1}{2} \cdot \frac{1}{2} = \frac{3}{4}
\end{equation}
(3)
\begin{figure}[ht]
\centering
\includegraphics[width=14.25cm]{./figures/HsPbEc_2.png}
\caption{图解2-3} \label{HsPbEc_fig2}
\end{figure}\\
设丙获胜为事件 $C$\\
由图可得
\begin{equation}
P(C) = (\frac{1}{2})^4 + (\frac{1}{2})^3 + (\frac{1}{2})^3 + (\frac{1}{2})^3 = \frac{7}{16}
\end{equation}

以上是编者的解法,网络上可以找到其他解法,这里插入一种作为参考
\begin{figure}[ht]
\centering
\includegraphics[width=14.25cm]{./figures/HsPbEc_3.png}
\caption{来自百度文库} \label{HsPbEc_fig3}
\end{figure}