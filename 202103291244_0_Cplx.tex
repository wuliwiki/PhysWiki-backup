% 复变函数
% keys 复数|复变函数|导数

\pentry{复数\upref{CplxNo}, 函数\upref{functi}}
\textbf{复变函数(complex function)}一般是指自变量和函数值都在复数域内取值的函数($f:\mathbb C \to \mathbb C$),通常表示为
\begin{equation}
w = f(z)
\end{equation}
若把自变量 $z$ 拆分成实部 $x$ 和虚部 $y$, 即 $z = x + \I y$, 且函数 $f$ 也拆分成实部函数 $u(x,y)$ 和虚部函数 $v(x,y)$,则复变函数可记为
\begin{equation}\label{Cplx_eq1}
w = u(x,y) + \I v(x,y)
\end{equation}
例如,复指数函数\upref{CExp} 被定义为
\begin{equation}\label{Cplx_eq3}
w = \E^z = \E^x \cos y + \I \E^x\sin y
\end{equation}
即 $u(x, y) = \E^x \cos y$, $v(x, y) = \E^x \sin y$.

更一般地, 函数值为复数的多元函数都可以叫做复变函数. 例如 $f: \mathbb R \to \mathbb C$ 或者 $f:\mathbb R^N \to \mathbb C$. 例如简谐振动可以记为\upref{VbExp}
\begin{equation}
f(t) = \E^{-\I \omega t} = \cos(\omega t) + \I \sin(\omega t) \qquad (t \in \mathbb R)
\end{equation}

\subsection{复变函数与矢量场}
当我们把复变函数记为\autoref{Cplx_eq1} 的形式后, 可以把它看作一个平面矢量场矢量场\upref{Vfield}, 即复平面上的每一点对应一个几何矢量. 令实轴和虚轴分别为 $x$ 轴和 $y$ 轴, 其单位矢量分别为 $\uvec x$ 和 $\uvec y$, 令位置矢量 $\bvec r = x \uvec x + y \uvec y$, 那么这个矢量场就可以表示为
\begin{equation}\label{Cplx_eq4}
\bvec w(\bvec r) = u(\bvec r) \uvec x + v(\bvec r) \uvec y
\end{equation}
这相当于把映射 $f:\mathbb C \to \mathbb C$ 看成 $f: \mathbb R^2 \to \mathbb R^2$.

正因如此, 研究复变函数时可以借助许多矢量分析的工具, 如散度\upref{Divgnc}和旋度\upref{Curl}等. 所以以后讲解时我们也会使用相应的矢量分析内容作为预备知识.

\subsection{复变函数的可视化}
矢量场的概念可以用于把复变函数可视化
\addTODO{把 $\exp(z)$ 分别画成带箭头的矢量场, 以及两个分开的标量场}
\begin{figure}[ht]
\centering
\includegraphics[width=7.5cm]{./figures/Cplx_2.pdf}
\caption{$\E^z$ 对应的二维矢量场(\autoref{Cplx_eq4} ), 图中的箭头与 $\bvec w(\bvec r)$ 成正比.} \label{Cplx_fig2}
\end{figure}
也可以把实部 $u(x,y)$ 虚部 $v(x, y)$ 分别画成标量场. \autoref{Cplx_eq3}
\begin{figure}[ht]
\centering
\includegraphics[width=14.25cm]{./figures/Cplx_3.png}
\caption{$\exp(z)$ 的实部和虚部函数 $u, v$} \label{Cplx_fig3}
\end{figure}


另一种方式是使用变形网格的方式表示平面到平面的映射
\begin{figure}[ht]
\centering
\includegraphics[width=8cm]{./figures/Cplx_1.png}
\caption{复映射} \label{Cplx_fig1}
\end{figure}

\subsection{作为实函数的拓展}
复变函数中很多函数与我们原来我们学过的函数同名,只是定义域和值域从实数拓展到了复数. 例如三角函数,对数函数,指数函数等.拓展后, 复变函数的定义必须要与原来的实函数保持 “兼容”,即当自变量取实数时, 函数值也要与原来相同.

例如,当复指数函数\autoref{Cplx_eq3} 的自变量只在实轴上取值(即 $y = 0$) 时,该函数变为我们原来所熟悉的实函数 $\E^x$.  又如,复正弦函数\upref{CTrig} 可记为
\begin{equation}\label{Cplx_eq2}
\sin z = \sin(x + \I y) = \sin x\cosh y + \I\cos x\sinh y
\end{equation}
其中 $\sinh $ 和 $\cosh $ 是双曲正弦和双曲余弦函数双曲函数\upref{TrigH}.当 $y = 0$ 时,又重新得到实函数 $\sin x$.

为什么一定要定义成\autoref{Cplx_eq2} 呢? 如果只需满足 “兼容” 的条件, 那么拓展的方式可以有无限多种, 但一般我们还需要添加另一个条件, 就是拓展后的函数需要是\textbf{解析}的, 这叫做\textbf{解析拓延}. % 链接未完成
解析拓延只有一种, 例如实函数 $\sin x$ 进行解析拓延后, 只有\autoref{Cplx_eq2} 一种可能的定义.

我们研究的复变函时几乎都是在讨论解析函数, 解析函数要求它可以在复平面的某个区域内可以展开为(复)无穷级数(类比实函数的泰勒级数\upref{Taylor})
\begin{equation}
f(z) = \sum_{n=0}^\infty c_n z^n
\end{equation}

\subsection{复变函数的导数}
由于复变函数相当于两个实数自变量的实值函数,一般情况下求导变得比较复杂, 它不但取决于自变量, 还取决于求导的方向, 这有点类似于方向导数\upref{DerDir}.但如果复变函数在某个区域上\textbf{解析},那么在该区域上导数与方向无关.对于复数域初等基本函数,求导的结果也和实数域的求导一样. 详见 “复变函数的导数 柯西—黎曼条件\upref{CauRie}”.

\subsection{复积分}
复变函数的积分需要在复平面上选取一个路径. 从矢量场的角度考虑, 这可以类比矢量场的线积分\upref{IntL}.
