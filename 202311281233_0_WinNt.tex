% Windows 笔记
% license Xiao
% type Note

\subsection{常用}
\begin{itemize}
\item 把任何程序,脚本,或者他们的快捷方式放\textbf{启动文件夹},可以在用户登录后执行。 路径为: \verb|C:\ProgramData\Microsoft\Windows\Start Menu\Programs\StartUp|(也可以通过 \verb|Win+R| 中输入 \verb|shell:startup| 打开)。
\end{itemize}

\subsubsection{快捷键}
\begin{itemize}
\item 切换屏幕:\verb|Win+Ctrl + 方向键| (或者在触控板设置用四指滑动)
\item 增添屏幕:\verb|Win+Ctrl + D|
\item 半屏/四分之一屏显示窗口:\verb|Win + 方向键|(可以使用两个如左上)
\item \verb|Ctrl+双击文件夹|:在新窗口中打开文件夹
\item \verb|Ctrl+W|:关闭当前窗口或浏览器单个页面。
\item \verb|Win+L|:锁屏
\item \verb|Win+E|:打开文件浏览器
\item \verb|Alt+上|:上级目录
\item 文件浏览器和 Chrome 中用 \verb|Ctrl+L| 选中当前路径
\item \verb|shift+Del|
\end{itemize}

\subsubsection{快捷路径}
在文件浏览器和 cmd 中均适用
\begin{itemize}
\item 在浏览器的 options -> view 中可以设置显示完整路径。
\item \verb|Desktop| 相当于 \verb|C:\Users\用户名\Desktop|。 \verb|Documents| 同理。 浏览器的路径也不会显示完整路径(设置了也没用)。
\item \verb|%USERPROFILE%| 相当于 \verb|C:\Users\用户名|
\item \verb|%APPDATA%| 相当于 \verb|C:\Users\用户名\AppData\Roaming|
\end{itemize}


\subsection{清理电脑}
\begin{itemize}
\item 用 \href{https://windirstat.net/}{WinDirStat} 查看硬盘被谁占用。
\item 要清除休眠文件, cmd 用管理员运行 \verb|powercfg -h off|, 然后删除 \verb|C:/hiberfil.sys| 即可(Win 10/11 亲测)。
\item 要清除 Page file, 在开始菜单搜索 \verb|Adjust the Appearance and Performance of Windows|, 然后在 Advance 面板里面调整 Virtual Memory(内存够大可以直接关掉)。
\item 在开始菜单搜索 \verb|Disk Cleanup|, 清除里面的选项, 然后再 \verb|Cleanup System Files|。
\end{itemize}

\subsection{网络}
\begin{itemize}
\item cmd 中的 \verb|ipconfig| 命令可以查看 ip 地址等网络信息
\item 重新启动后,在登录界面是可以自动连接 wifi 的,如果不行,就在该界面手动连接,并选择可用时自动连接再重启试试。
\item 若插着网线,windows 不会自动连接 wifi (包括登录界面和桌面)。 控制面板的 Network adapter 属性里面的优先级设置无效。
\item 如果有多块网卡都能联网, 把一块中属性中的 IPv4 里面的网关删掉就可以不联网。
\item 查看当前 wifi 密码(若已连接): 在控制面板的网络设置立面,找到当前的 wifi 网卡,右键属性,安全,管理员可以显示密码。
\end{itemize}

\subsection{Windows Sandbox}
\begin{itemize}
\item 最大的缺点是不能保存状态, 每次都是新的, 网上可能有教程可以保存
\item 传文件直接复制粘贴就可以, 网上也有 map 硬盘的的教程
\item 轻量级, 占硬盘很小,很快
\item 还是不能代替 Virtual Box
\end{itemize}
