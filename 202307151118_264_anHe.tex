% 内禀反常霍尔效应
% 晶体|电子|霍尔效应
\pentry{电子运动的准经典模型\upref{cryele}}
\begin{issues}
\issueDraft
\end{issues}
一般霍尔效应的产生需要磁场,并且满带不出现霍尔效应。但是反常霍尔效应不需要这些条件。反常霍尔效应的产生有几种理论解释,我们现在来介绍最基础的内禀反常霍尔效应。
\subsection{介绍}
我们知道一个布洛赫态可以写成:
\begin{equation}
\psi_{n,\bvec{k}}=\E^{\I\bvec{k}\cdot\bvec{r}}u_{n,\bvec{k}}~,
\end{equation}
其哈密顿量为:$\hat H_0=-\frac{\hbar^2}{2m}\nabla^2+V(\bvec{r})$,对应的能量是$E_{n,\bvec{k}}$。其中$V(\bvec{r})$是一个周期函数,有$V(\bvec{r}+\bvec{R})=V(\bvec{r})$,$\bvec{R}$是任意一个格矢。

外力作用下哈密顿量变成 $\hat H =\hat H_0 -\bvec{F}\cdot\bvec{r}$,则 $\dd{t}$ 时间后,布洛赫态变成:
\begin{equation}
\begin{aligned}
\psi(\bvec{r},\dd{t})&=\E^{-\frac{\I\,\hat H\,\dd{t}}{\hbar}}\psi_{n,\bvec{k}} \approx \psi_{n,\bvec{k}}-\frac{\I\dd{t}}{\hbar}(\hat H_0 -\bvec{F}\cdot\bvec{r})\psi_{n,\bvec{k}}\\
&=(1-\I \frac{E_{n,\bvec{k}}\dd{t}}{\hbar}+\I\frac{\bvec{F}\cdot\bvec{r}\dd{t}}{\hbar})\E^{\I\bvec{k}\cdot\bvec{r}}u_{n,\bvec{k}}
\approx \E^{-\frac{\I}{\hbar}E_{n,\bvec{k}}\dd{t}}\E^{\I(\bvec{k}+\frac{\bvec{F}\dd{t}}{\hbar})\cdot \bvec{r}}u_{n,\bvec{k}}~.
\end{aligned}
\end{equation}
在推导电子运动的准经典模型时,我们忽略了$u_{n,\bvec{k}}$到$u_{n,\bvec{k}+\frac{\bvec{F}\dd{t}}{\hbar}}$之间的变化(包络近似),从而得出了$d\bvec{k}=\frac{\bvec{F}\dd{t}}{\hbar}$的结论。现在我们不忽略它的变化,从而推导出反常霍尔效应来。

有:
\begin{equation}\label{eq_anHe_3}
\begin{aligned}
\psi(\bvec{r},\dd{t})&\approx \E^{-\frac{\I}{\hbar}E_{n,\bvec{k}}\dd{t}}\E^{\I(\bvec{k}+\frac{\bvec{F}\dd{t}}{\hbar})\cdot \bvec{r}}u_{n,\bvec{k}}\\
&=\E^{-\frac{\I}{\hbar}E_{n,\bvec{k}}\dd{t}}\E^{\I(\bvec{k}+\frac{\bvec{F}\dd{t}}{\hbar})\cdot \bvec{r}}(u_{n,\bvec{k}}-u_{n,\bvec{k}+\frac{\bvec{F}}{\hbar}\dd{t}}+u_{n,\bvec{k}+\frac{\bvec{F}}{\hbar}\dd{t}})\\
&=\E^{-\frac{\I}{\hbar}E_{n,\bvec{k}}\dd{t}}\E^{\I(\bvec{k}+\frac{\bvec{F}\dd{t}}{\hbar})\cdot \bvec{r}}u_{n,\bvec{k}+\frac{\bvec{F}}{\hbar}\dd{t}}-\E^{-\frac{\I}{\hbar}E_{n,\bvec{k}}\dd{t}}\E^{\I(\bvec{k}+\frac{\bvec{F}\dd{t}}{\hbar})\cdot \bvec{r}}(u_{n,\bvec{k}+\frac{\bvec{F}}{\hbar}\dd{t}}-u_{n,\bvec{k}})\\
&=\E^{-\frac{\I}{\hbar}E_{n,\bvec{k}}\dd{t}}\psi_{n,\bvec{k}+\frac{\bvec{F}\dd{t}}{\hbar}}-\E^{-\frac{\I}{\hbar}E_{n,\bvec{k}}\dd{t}}\E^{\I(\bvec{k}+\frac{\bvec{F}\dd{t}}{\hbar})\cdot \bvec{r}}(\nabla_{\bvec{k}} u\cdot \frac{\bvec{F}}{\hbar}\dd{t})\\
&\approx \E^{-\frac{\I}{\hbar}E_{n,\bvec{k}}\dd{t}}\psi_{n,\bvec{k}+\frac{\bvec{F}\dd{t}}{\hbar}}-\E^{\I\bvec{k}\cdot\bvec{r}}\nabla_{\bvec{k}} u\cdot \frac{\bvec{F}}{\hbar}\dd{t}~,
\end{aligned}
\end{equation}
其中$\nabla_{\bvec{k}} u$表示函数$u$在$\bvec{k}$空间的梯度。最后一条等式的前一项即为经典理论的结果,后一项的导出只保留了 $\dd{t}$ 的一阶小量。
\subsection{深入}
现在我们来研究一下$\nabla_{\bvec{k}} u$。由上可知,有:
\begin{equation}
\hat H_0 \E^{\I\bvec{k}\cdot\bvec{r}}u_{n,\bvec{k}}(\bvec{r})
=E_{n,\bvec{k}}\E^{\I\bvec{k}\cdot\bvec{r}}u_{n,\bvec{k}}(\bvec{r})~,
\end{equation}
所以有:
\begin{equation}
\E^{-i\bvec{k}\cdot\bvec{r}} \hat H_0 \E^{\I\bvec{k}\cdot\bvec{r}}u_{n,\bvec{k}}(\bvec{r})
=E_{n,\bvec{k}}u_{n,\bvec{k}}(\bvec{r})~,
\end{equation}
即 $\hat H_{\bvec k} =\E^{-i\bvec{k}\cdot\bvec{r}} \hat H_0 \E^{\I\bvec{k}\cdot\bvec{r}}$的本征函数是$u_{n,\bvec{k}}(\bvec{r})$。代入$\hat H_0 =-\frac{\hbar^2}{2m}\nabla^2+V(\bvec{r})$,对于任意一个函数$f(\bvec{r})$,即有:
\begin{equation}
\begin{aligned}
\hat H_{\bvec k} f(\bvec{r})&=\E^{-i\bvec{k}\cdot\bvec{r}}(-\frac{h^2}{2m}\nabla^2+V(\bvec{r}))\E^{\I\bvec{k}\cdot\bvec{r}}f(\bvec{r})\\
&=V(\bvec{r})f(\bvec{r})+\E^{-i\bvec{k}\cdot\bvec{r}}(-\frac{\hbar^2}{2m})\nabla(i\bvec{k}\E^{\I\bvec{k}\cdot\bvec{r}}f(\bvec{r})+\E^{\I\bvec{k}\cdot\bvec{r}}\nabla f(\bvec{r}))\\
&=V(\bvec{r})f(\bvec{r})+\E^{-i\bvec{k}\cdot\bvec{r}}(-\frac{\hbar^2}{2m})(-k^2\E^{\I\bvec{k}\cdot\bvec{r}}+2i\bvec{k}\E^{\I\bvec{k}\cdot\bvec{r}}\nabla f(\bvec{r})+\E^{\I\bvec{k}\cdot\bvec{r}}\nabla^2 f(\bvec{r}))\\
&=V(\bvec{r})f(\bvec{r})-\frac{\hbar^2}{2m}(-k^2+2i\bvec{k}\nabla+\nabla^2)f(\bvec{r})~,
\end{aligned}
\end{equation}
即 $\hat H_{\bvec k}$ 的具体形式为
\begin{equation}
\hat H_{\bvec k} =\frac{\hbar^2}{2m}(-i\nabla+\bvec{k})^2+V(\bvec{r})~.
\end{equation}
当$\bvec{k}$变化$\delta\bvec{k}$时,即有:
\begin{equation}
\hat H_{\bvec{k}+\delta\bvec{k}} =\frac{\hbar^2}{2m}(-i\nabla+\bvec{k})^2+V(\bvec{r})+\frac{\hbar^2}{2m}((\bvec{k}+\delta{\bvec{k}})^2-k^2)+\frac{\hbar}{m}\delta\bvec{k}\cdot(-i\hbar \nabla)~.
\end{equation}
等式右边第1、2项即为原先的 $\hat H_{\bvec k}$, 第3、4项可以视为微扰。其中第3项是能量微扰项,不作用在波函数上,第四项为$\bvec{k}\cdot\bvec{p}$微扰项,与动量相关(此处还说明了晶格动量$\hbar\bvec{k}$与电子动量$\bvec{p}$不是同一个东西)。

$\hat H_{\bvec k}$的本征态是不同$n$的$u_{n,\bvec k}$,不同的$n$之间的能级差较大,所以可以用非简并微扰来求解$\hat H_{\bvec{k}+\delta\bvec{k}}$的本征态,即:
\begin{equation}
u_{n,\bvec{k}+\delta\bvec{k}}(\bvec{r})=u_{n,\bvec{k}}(\bvec r)+\frac{\hbar}{m}\delta\bvec{k}\cdot\sum_{l\neq n}\frac{\bra{u_{l,\bvec{k}}}\hat {\bvec p}\ket{u_{n,\bvec{k}}}}{E_{n,\bvec{k}}-E_{l,\bvec{k}}}u_{l,\bvec{k}}(\bvec{r})~.
\end{equation}
因为$u_{n,\bvec{k}+\delta\bvec{k}}=u_{n,\bvec{k}}+\delta \bvec{k}\cdot \nabla_{\bvec{k}} u$,所以我们就终于得出了$\nabla_{\bvec{k}} u$的具体形式,即:
\begin{equation}\label{eq_anHe_10}
\nabla_{\bvec{k}} u = \frac{\hbar}{m}\sum_{l\neq n}\frac{\bra{u_{l,\bvec{k}}}\hat {\bvec p}\ket{u_{n,\bvec{k}}}}{E_{n,\bvec{k}}-E_{l,\bvec{k}}}u_{l,\bvec{k}}(\bvec{r})~.
\end{equation}
\subsection{推导}
将\autoref{eq_anHe_10}  代回到最开始的\autoref{eq_anHe_3}   ,即有:
\begin{equation}\label{eq_anHe_11}
\psi(\bvec{r},\dd{t})\approx \E^{-\frac{\I}{\hbar}E_{n,\bvec{k}}\dd{t}}\psi_{n,\bvec{k}+\frac{\bvec{F}\dd{t}}{\hbar}}(\bvec{r})-\frac{\bvec{F}}{m}\dd{t}\cdot\sum_{l\neq n}\frac{\bra{u_{l,\bvec{k}}}\hat {\bvec p}\ket{u_{n,\bvec{k}}}}{E_{n,\bvec{k}}-E_{l,\bvec{k}}}\psi_{l,\bvec{k}}(\bvec{r})~.
\end{equation}
因为$\frac{\bvec{F}}{m}\dd{t}$是小量,我们再做一个类似于\autoref{eq_anHe_3} 最后化简时的把戏,将\autoref{eq_anHe_11} 改写成:
\begin{equation}
\psi(\bvec{r},\dd{t})\approx \E^{-\frac{\I}{\hbar}E_{n,\bvec{k}}\dd{t}}\psi_{n,\bvec{k}+\frac{\bvec{F}\dd{t}}{\hbar}}(\bvec{r})-\frac{\bvec{F}}{m}\dd{t}\cdot\sum_{l\neq n}\frac{\bra{u_{l,\bvec{k}}}\hat {\bvec p}\ket{u_{n,\bvec{k}}}}{E_{n,\bvec{k}}-E_{l,\bvec{k}}}\psi_{l,\bvec{k}+\frac{\bvec{F}\dd{t}}{\hbar}}(\bvec{r})~.
\end{equation}
由此可见,不使用包络近似下,外力作用会将不同的能带耦合起来,这将使布洛赫电子的运动有别于经典的运动。
当外力$\bvec F$很小时,$\psi$具有稳态解:
\begin{equation}
\psi(\bvec r,t=\infty)\approx\psi_{n,\bvec{k}}-\I\frac{\hbar}{m}\bvec{F}\cdot\sum_{l\neq n}\frac{\bra{u_{l,\bvec{k}}}\hat {\bvec p}\ket{u_{n,\bvec{k}}}}{(E_{n,\bvec{k}}-E_{l,\bvec{k}})^2}\psi_{l,\bvec{k}}~.
\end{equation}

现在我们可以来算速度了,根据隔壁量子力学的基础知识,有:
\begin{equation}
\bvec{v}=\frac{\dd{\ev{\bvec{r}}}}{\dd{t}}=\frac{\I}{\hbar}\ev{[\hat{H},\hat{\bvec{r}}]}=\frac{\I}{\hbar}\ev{[\frac{\hat{\bvec{p}}^2}{2m}+V(\bvec{r})-\bvec{F}\cdot\bvec{r},\hat{\bvec{r}}]}=\ev{\hat{\bvec{P}}}/m~.
\end{equation}
继续忽略外力$\bvec{F}$的高阶项,则:
\begin{equation}\label{eq_anHe_15}
\bvec{v}=\ev{\hat{\bvec{P}}}/m=\frac{1}{m}\bra{\psi_{n,\bvec{k}}}\hat{\bvec{p}}\ket{\psi_{n,\bvec{k}}}+\frac{\I\hbar}{m}\sum_{l\neq n}\frac{1}{(E_{l,\bvec{k}}-E_{l,\bvec{k}})^2}[(\bvec{F}\cdot\bra{u_{n,\bvec{k}}}\hat{\bvec{p}}\ket{u_{l,\bvec{k}}})\bra{\psi_{l,\bvec{k}}}\hat{\bvec{p}}\ket{\psi_{n,\bvec{k}}}-(\bvec{F}\cdot\bra{u_{l,\bvec{k}}}\hat{\bvec{p}}\ket{u_{n,\bvec{k}}})\bra{\psi_{n,\bvec{k}}}\hat{\bvec{p}}\ket{\psi_{l,\bvec{k}}}]~.
\end{equation}
对于$\bra{\psi_{n,\bvec{k}}}\hat{\bvec{p}}\ket{\psi_{l,\bvec{k}}}(l\neq n)$,则将其展开,由于$u_{n,\bvec{k}}$的正交性,有:
\begin{equation}\label{eq_anHe_16}
\bra{\psi_{n,\bvec{k}}}\hat{\bvec{p}}\ket{\psi_{l,\bvec{k}}}=\int\dd{V}\  u_{n,\bvec{k}}\E^{-\I\bvec{k}\cdot\bvec{r}}(-\I\hbar\nabla)(\E^{\I\bvec{k}\cdot\bvec{r}}u_{l,\bvec{k}})=\bra{u_{n,\bvec{k}}}\hat{\bvec{p}}\ket{u_{l,\bvec{k}}}~.
\end{equation}
不妨记$\bra{u_{n,\bvec{k}}}\hat{\bvec{p}}\ket{u_{l,\bvec{k}}}$为$\bvec{p}_{nl,\bvec{k}}$.\\
而对于$\bra{\psi_{n,\bvec{k}}}\hat{\bvec{p}}\ket{\psi_{n,\bvec{k}}}$,注意到:
\begin{equation}
\bra{\psi_{n,\bvec{k}}}\hat{\bvec{p}}\ket{\psi_{n,\bvec{k}}}=\bra{u_{n,\bvec{k}}}\hbar\bvec{k}+\hat{\bvec{p}}\ket{u_{n,\bvec{k}}}=\frac{m}{\hbar}\bra{u_{n,\bvec{k}}}(\nabla_{\bvec{k}}\hat{H_{\bvec{k}}})\ket{u_{n,\bvec{k}}}~.
\end{equation}
并且:
\begin{equation}\label{eq_anHe_18}
\begin{aligned}
\bra{u_{n,\bvec{k}}}\nabla_{\bvec{k}}\hat{H_\bvec{k}}\ket{u_{n,\bvec{k}}}&=\braket{u_{n,\bvec{k}}}{\nabla_{\bvec{k}}(\hat{H_\bvec{k}}u_{n,\bvec{k}})}-\bra{u_{n,\bvec{k}}}\hat{H_\bvec{k}}\ket{\nabla_{\bvec{k}}u_{n,\bvec{k}}}\\
&=\braket{u_{n,\bvec{k}}}{\nabla_{\bvec{k}}(E_{n,\bvec{k}}u_{n,\bvec{k}})}-E_{n,\bvec{k}}\braket{u_{n,\bvec{k}}}{\nabla_{\bvec{k}}u_{n,\bvec{k}}}\\
&=E_{n,\bvec{k}}\braket{u_{n,\bvec{k}}}{\nabla_{\bvec{k}}u_{n,\bvec{k}}}+\bra{u_{n,\bvec{k}}}{\nabla_{\bvec{k}}E_{n,\bvec{k}}\ket{u_{n,\bvec{k}}}}-E_{n,\bvec{k}}\braket{u_{n,\bvec{k}}}{\nabla_{\bvec{k}}u_{n,\bvec{k}}}\\
&=\bra{u_{n,\bvec{k}}}{\nabla_{\bvec{k}}E_{n,\bvec{k}}\ket{u_{n,\bvec{k}}}}=\nabla_{\bvec{k}}E_{n,\bvec{k}}~.
\end{aligned}
\end{equation}
将\autoref{eq_anHe_18} 和\autoref{eq_anHe_16} 代回到\autoref{eq_anHe_15} 里,就有了一个简单的结果:
\begin{equation}
\bvec{v}=\frac{1}{\hbar}\nabla_{\bvec{k}}E_{n,\bvec{k}}+\frac{\I\hbar}{m}\sum_{l\neq n}\frac{1}{(E_{l,\bvec{k}}-E_{l,\bvec{k}})^2}(\bvec{F}\cdot)~.
\end{equation}







