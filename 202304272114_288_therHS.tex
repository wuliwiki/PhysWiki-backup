% 热力学初步(高中)

\begin{issues}
\issueTODO
\end{issues}
% 分子动理论|气体等x定律|固体液体|热力学定律
% 缩减一部分,把第二章的前两小节合并,第二章整体作为一个新的小节
% 或者直接拆分成分子动力学和热力学初步算了,麻烦
% 第二章和第三章作为热力学初步内容

%\pentry{相互作用\upref{HSPM02}}% 分子动力学

\subsection{温度和温标}
\subsubsection{状态参量与平衡态}
以研究容器中气体的热学性质为例,我们的研究对象是一个由大量分子所组成的系统,称之为热力学系统,简称\textbf{系统}。在我们的经验常识中,一小罐热的气体会在室温下逐渐变凉,最终和室温温度保持一致,这个过程中,这一小罐热的气体即是系统,而系统之外与之产生相互作用的其他物体统称为\textbf{外界}。外界影响系统,导致系统的某些物理量发生变化。在这个例子中,热空气温度逐渐下降,最终和外界空气保持一致,热空气(系统)的状态也随之改变。在热血中,为了确定系统的状态,需要用一些物理量来进行描述,这些物理量叫做系统的状态参量,比如体积$V$是描述系统空间范围的几何参量,压强$p$是描述系统之间或内部力的作用的力学参量,温度$T$是确定系统冷热程度的热学参量……

在上述的例子中,可以发现温度逐渐下降,说明系统的热学性质发生了改变。但是,往往在任意时刻确定系统的状态是困难的,只有在系统处于\textbf{平衡态},也即系统经过了足够长的时间的烟花,内部的各个部分的状态参量达到稳定状态不再改变时,我们才可以比较准确的描述系统的状态。

\subsubsection{}
\subsection{气体的等温变化}
\subsection{气体的等压变化}
\subsection{气体的等容变化}
\subsection{固体和液体简介}
\subsection{功、热、内能}
\subsection{热力学第一定律}
\subsection{能量守恒定律}
\subsection{热力学第二定律}



%% 画图时间
