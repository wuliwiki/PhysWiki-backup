% 二次量子化
% keys 二次量子化|多体系统|玻色统计|费米统计

\pentry{全同粒子\upref{IdPar},全同粒子的统计\upref{IdParS}}

薛定谔方程是关于单粒子的量子力学,而如果我们考虑一个多粒子体系,研究粒子间相互作用对多体系统的影响,则有必要建立一个关于多粒子的量子力学.一个直接的想法是,讲单个时空坐标变量的波函数拓展为 $N$ 个变量的波函数 $\psi(x_1,\cdots,x_N)$,波函数的模方具有概率的意义,因此可以乘上一个系数使它归一化.

然而当我们讨论全同粒子的时候,例如 $N$ 个电子组成的体系,上述波函数的定义并没有体现粒子的全同性,波函数不具有交换对称性或反对称性.因此

\begin{equation}
\ket{\psi_1\cdots\psi_N}=\psi_1(x_1)\psi_2(x_2)\psi_3(x_3)\cdots \psi_n(x_N)
\end{equation}
