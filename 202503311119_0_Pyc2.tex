% Python 第一步
% license Xiao
% type Tutor


\begin{issues}
\issueTODO
\end{issues}

\subsection{创建一个Python项目}

打开 Pycharm , 新建项目

\begin{figure}[ht]
\centering
\includegraphics[width=14.25cm]{./figures/2ada35917a742b32.png}
\caption{选项选取} \label{fig_Pyc2_2}
\end{figure}

右键项目名——新建——Python 文件

\begin{figure}[ht]
\centering
\includegraphics[width=14.25cm]{./figures/44ea62f4da2d69c8.png}
\caption{新建Python文件} \label{fig_Pyc2_1}
\end{figure}

给文件命名

\begin{figure}[ht]
\centering
\includegraphics[width=14.25cm]{./figures/cd91ca45cc505f0c.png}
\caption{命名} \label{fig_Pyc2_3}
\end{figure}

恭喜你,建立了你的第一个Python文件!

\subsection{输入}

\begin{figure}[ht]
\centering
\includegraphics[width=14.25cm]{./figures/8bb25afdd70aa412.png}
\caption{编写代码位置} \label{fig_Pyc2_4}
\end{figure}

打出 \footnote{单词 print 的中文意思是打印,所以括号里面的内容是打印出来的?}print() ,其中括号里面的内容就是输出内容,例如下图

\begin{lstlisting}[language=python]
print(1234)
\end{lstlisting}

然后右键文件名,选择运行。(操作如下图)

\begin{figure}[ht]
\centering
\includegraphics[width=14.25cm]{./figures/b5c6db6c52a0d894.png}
\caption{运行} \label{fig_Pyc2_5}
\end{figure}

接着会弹出运行框。框内就是结果

\begin{figure}[ht]
\centering
\includegraphics[width=14.25cm]{./figures/9c001a49cf8c70ed.png}
\caption{输出结果} \label{fig_Pyc2_6}
\end{figure}

现在你已经会如何运行编写好的程序了,那我们继续讨论 print。

现在打出

\begin{lstlisting}[language=python]
print(1+2)
\end{lstlisting}

运行,结果如下

\begin{lstlisting}[language=bash]
3
\end{lstlisting}

可以发现,得到的结果是 3,而不是1+2

这就是print的另一个功能,可以计算括号内的结果并输出。

\subsection{变量}

你可以把变量想象成一个盒子,你可以在这个盒子里放任何东西,比如数字、文字、和更多更复杂的东西。

举个例子,现在有一个数字你想储存在一个变量里

\begin{lstlisting}[language=python]
number = 10
\end{lstlisting}

注:Python中 “=” 指赋值(把等号右侧的赋予左侧),而 “==” 才是数学中的等号,再后文会详细解释。

然后,输出变量

\begin{lstlisting}[language=python]
print(number)
\end{lstlisting}

可以得到

\begin{lstlisting}[language=bash]
10
\end{lstlisting}

注:变量名只能包含字母、数字和下划线,且不能以数字开头。

变量不仅可以储存,还可以参与计算参与计算。如:

\begin{lstlisting}[language=python]
number_1 = 50
number_2 = 20
print(number_1 + number_2)
\end{lstlisting}

运行后

\begin{lstlisting}[language=bash]
70
\end{lstlisting}

Python允许你同时为多个变量赋值。如:

\begin{lstlisting}[language=python]
a = b = c = 1
print(a)
print(b)
print(c)
\end{lstlisting}

可得

\begin{lstlisting}[language=bash]
1
1
1
\end{lstlisting}

您也可以为多个对象指定多个变量。如:

\begin{lstlisting}[language=python]
a, b, c = 1, 2, 8
print(a)
print(b)
print(c)
\end{lstlisting}

可得

\begin{lstlisting}[language=bash]
1
2
8
\end{lstlisting}

\subsection{基本数据类型}

如果我们打出

\begin{lstlisting}[language=python]
print(你好)
\end{lstlisting}

运行后,结果如下

\begin{figure}[ht]
\centering
\includegraphics[width=14.25cm]{./figures/c18bf6eea4e75d37.png}
\caption{运行结果} \label{fig_Pyc2_7}
\end{figure}

并没有输出 你好 而是报错了(出现了红字),这是为什么呢?

如果我们打出

\begin{lstlisting}[language=python]
number = 10
print(number)
\end{lstlisting}

结果为

\begin{lstlisting}[language=bash]
10
\end{lstlisting}

为什么不是 number 呢?

这就要讲到Python 中的基本数据类型了。

常见的数据类型有:

\begin{enumerate}
\item Number(数字)
\item String(字符串)
\item bool(布尔类型)
\item List(列表)
\item Tuple(元组)
\item Set(集合)
\item Dictionary(字典)
\end{enumerate}

注:此外还有一些高级的数据类型,如: 字节数组类型(bytes)。

\subsubsection{Number(数字)}

Python3 支持 int(整数)、float(浮点数——小数)、bool(后文详解)、complex(复数)。

注:type() 可以用来查询变量所指的对象类型。

所以,下图number_1就是整数,number_2就是浮点数

\begin{lstlisting}[language=python]
number_1 = 10 
number_2 = 5.6
print(type(number_1))
print(type(number_2))
\end{lstlisting}

结果如下

\begin{lstlisting}[language=bash]
<class 'int'>
<class 'float'>
\end{lstlisting}

注:1.0也是浮点数。

如果我们想要取一个数字的整数部分,Python支持如下指令:

\begin{lstlisting}[language=python]
number_1 = 3.1415926
print(int(number_1))
\end{lstlisting}

你将得到

\begin{lstlisting}[language=bash]
3
\end{lstlisting}

\subsubsection{String(字符串)}

这里就是解决前面问题的关键

print()

Python 中单引号 ' 和双引号 " 使用完全相同。

使用三引号(''' 或 """)可以指定一个多行字符串。

转义符 \。

反斜杠可以用来转义,使用 r 可以让反斜杠不发生转义。 如 r"this is a line with \n" 则 \n 会显示,并不是换行。

按字面意义级联字符串,如 "this " "is " "string" 会被自动转换为 this is string。

字符串可以用 + 运算符连接在一起,用 * 运算符重复。

Python 中的字符串有两种索引方式,从左往右以 0 开始,从右往左以 -1 开始。

Python 中的字符串不能改变。

Python 没有单独的字符类型,一个字符就是长度为 1 的字符串。

字符串切片 str[start:end],其中 start(包含)是切片开始的索引,end(不包含)是切片结束的索引。

字符串的切片可以加上步长参数 step,语法格式如下:str[start:end:step]

word = '字符串'

sentence = "这是一个句子。"

paragraph = """这是一个段落,

可以由多行组成"""

实例(Python 3.0+)

#!/usr/bin/python3
 
str='123456789'
 
print(str)                 # 输出字符串

print(str[0:-1])           # 输出第一个到倒数第二个的所有字符

print(str[0])              # 输出字符串第一个字符

print(str[2:5])            # 输出从第三个开始到第六个的字符(不包含)

print(str[2:])             # 输出从第三个开始后的所有字符

print(str[1:5:2])          # 输出从第二个开始到第五个且每隔一个的字符(步长为2)

print(str * 2)             # 输出字符串两次

print(str + '你好')         # 连接字符串
 
print('------------------------------')
 
print('hello\nrunoob')      # 使用反斜杠(\)+n转义特殊字符
print(r'hello\nrunoob')     # 在字符串前面添加一个 r,表示原始字符串,不会发生转义
这里的 r 指 raw,即 raw string,会自动将反斜杠转义,例如:

>>> print('\n')       # 输出空行

>>> print(r'\n')      # 输出 \n
\n
>>>
以上实例输出结果:

123456789
12345678
1
345
3456789
24
123456789123456789
123456789你好
------------------------------
hello
runoob
hello\nrunoob

\begin{aligned}
大纲:
变量
print(你哈)——数据类型【数字|字符串】
输入
运算符
换行
注释
\end{aligned}




% print 默认输出是换行的,如果要实现不换行需要在变量末尾加上 end="":

% 实例(Python 3.0+)
% #!/usr/bin/python3
 
% x="a"
% y="b"
% # 换行输出
% print( x )
% print( y )
 
% print('---------')
% # 不换行输出
% print( x, end=" " )
% print( y, end=" " )
% print()

% 继续打出




字符串(String)
Python 中单引号 ' 和双引号 " 使用完全相同。
使用三引号(''' 或 """)可以指定一个多行字符串。
转义符 \。
反斜杠可以用来转义,使用 r 可以让反斜杠不发生转义。 如 r"this is a line with \n" 则 \n 会显示,并不是换行。
按字面意义级联字符串,如 "this " "is " "string" 会被自动转换为 this is string。
字符串可以用 + 运算符连接在一起,用 * 运算符重复。
Python 中的字符串有两种索引方式,从左往右以 0 开始,从右往左以 -1 开始。
Python 中的字符串不能改变。
Python 没有单独的字符类型,一个字符就是长度为 1 的字符串。
字符串切片 str[start:end],其中 start(包含)是切片开始的索引,end(不包含)是切片结束的索引。
字符串的切片可以加上步长参数 step,语法格式如下:str[start:end:step]

