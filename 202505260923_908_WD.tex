% 弗朗索瓦·韦达(综述)
% license CCBYSA3
% type Wiki

本文根据 CC-BY-SA 协议转载翻译自维基百科 \href{ https://en.wikipedia.org/wiki/Fran\%C3\%A7ois_Vi\%C3\%A8te}{相关文章}。

弗朗索瓦·韦达(法语:[fʁɑ̃swa vjɛt];1540年-1603年2月23日),拉丁文名为弗朗西斯库斯·维埃塔(Franciscus Vieta),是一位法国数学家。他在新代数学方面的研究是通向现代代数学的重要一步,原因在于他创新性地使用字母作为方程中的参数。他本职是一名律师,曾担任法国亨利三世和亨利四世的枢密顾问。
\subsection{生平}
\subsubsection{早年生活与教育}
韦达出生于现今法国旺代省的丰特奈勒孔特。他的祖父是来自拉罗谢尔的一位商人,父亲埃蒂安·韦达是丰特奈勒孔特的一名律师,并在勒比索担任公证人。他的母亲是法国天主教同盟势力上升时期,议会首任主席兼法官巴尔纳贝·布里松的姑妈。

韦达曾在方济各会学校就读,并于1558年前往普瓦捷学习法律,次年取得法学学士学位。一年后,他在家乡开始了律师生涯。起初他就被委以重任,办理了一些重要案件,包括替法国国王弗朗索瓦一世的遗孀在普瓦图地区处理租金问题,并代表苏格兰女王玛丽维护其在法国的利益。
\subsubsection{在帕尔特奈任职}
1564年,韦达进入苏比斯夫人安托瓦内特·多贝泰尔的服务。她是帕尔特奈-苏比斯的让五世的妻子,而让五世是胡格诺派(法国新教)主要的军事领袖之一。韦达随他前往里昂,收集有关他在前一年英勇保卫该城抵御萨伏依的雅克(第二代内穆尔公爵)军队的资料。

同年,韦达在现今法国旺代省穆尚普公社的帕尔克-苏比斯担任苏比斯夫人12岁女儿凯瑟琳·德·帕尔特奈的家庭教师。他教授她科学和数学,并为她撰写了多部关于天文学和三角学的论文,其中一些作品至今仍留存。在这些著作中,韦达使用了十进制数(比斯特文发表相关论文早20年),并注意到行星运动的椭圆轨道(这比开普勒早了40年,也早于布鲁诺之死20年)。

帕尔特奈的让五世向法国国王查理九世推荐了韦达。韦达撰写了帕尔特奈家族的谱系,并在让五世于1566年去世后为他写了传记。

1568年,苏比斯夫人安托瓦内特将女儿凯瑟琳嫁给了查尔斯·德·凯勒内克男爵,韦达随她前往拉罗谢尔。在那里,他接触到加尔文主义贵族的最高层,包括科利尼、孔代以及纳瓦拉的女王让娜·达尔布雷及其子纳瓦拉的亨利——即后来的法国国王亨利四世。

1570年,韦达拒绝为苏比斯夫人及其女儿在一桩臭名昭著的诉讼中出庭辩护。在那场官司中,她们指控凯勒内克男爵无法(或不愿)生育继承人。
\subsubsection{初到巴黎}
1571年,韦达在巴黎注册成为律师,并持续探望他的学生凯瑟琳。他经常居住在丰特奈勒孔特,担任一些市政职务。他开始出版《通用观察:论数学正割表的一部专著》,并在夜晚或闲暇时间从事新的数学研究。据他的朋友雅克·德·图(Jacques de Thou)记载,他常常连续三天专注于一个问题,手肘支在书桌上,几乎不挪动位置地进食。

1572年,韦达恰好身在巴黎,亲历了“圣巴托洛缪大屠杀”。当晚,凯勒内克男爵在试图营救阿德米拉尔·科利尼未果后被杀。同年,韦达结识了加尔纳什的夫人弗朗索瓦丝·德·罗昂(Françoise de Rohan),并成为她的顾问,协助她对抗内穆尔公爵雅克。

1573年,他被任命为雷恩高等法院(Parlement of Rennes)的顾问。两年后,他说服安托瓦内特·多贝泰尔同意将凯瑟琳·德·帕尔特奈嫁给弗朗索瓦丝的弟弟——雷恩公爵勒内·德·罗昂(René de Rohan)。

1576年,雷恩公爵亨利·德·罗昂(Henri, duc de Rohan)给予韦达特别庇护,并在1580年推荐他担任“申诉事务主管”(*maître des requêtes*)。1579年,韦达完成了《通用观察》的印刷工作(由梅泰耶出版社出版),该书作为两部三角函数表的附录出版:《数学正割表,或关于三角形》(*Canon mathematicus, seu ad triangula*,即《通用观察》标题中提到的“canon”),以及《有理边三角形表》(*Canonion triangulorum laterum rationalium*)。

一年后,他被任命为巴黎高等法院的申诉事务主管,正式为国王服务。同年,他在内穆尔公爵与弗朗索瓦丝·德·罗昂的诉讼中成功为后者辩护,因而引来顽固的天主教同盟的怨恨。
