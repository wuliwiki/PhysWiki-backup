% 艾萨克·牛顿(综述)
% license CCBYSA3
% type Wiki

本文根据 CC-BY-SA 协议转载翻译自维基百科\href{https://en.wikipedia.org/wiki/Isaac_Newton}{相关文章})

\begin{figure}[ht]
\centering
\includegraphics[width=6cm]{./figures/0c83e6f3dfbe0a8c.png}
\caption{《46岁的牛顿肖像,1689年》} \label{fig_Newton_1}
\end{figure}
艾萨克·牛顿爵士,皇家学会会员(1642年12月25日-1726/27年3月20日[a]),是一位英国博学家,活跃于数学、物理学、天文学、炼金术、神学和写作领域,在他所在的时代被称为自然哲学家。他是科学革命及其后的启蒙运动中的关键人物。他的开创性著作《自然哲学的数学原理》首次出版于1687年,汇集了许多前人的研究成果,奠定了经典力学的基础。牛顿还在光学方面做出了开创性的贡献,并与德国数学家戈特弗里德·威廉·莱布尼茨共同被认为是微积分的创立者,尽管他在莱布尼茨之前几年就已发展了微积分。[10][11]

在《自然哲学的数学原理》中,牛顿制定了运动定律和万有引力定律,这些理论成为数个世纪以来主导性的科学观点,直到相对论的出现。他利用对重力的数学描述推导了开普勒的行星运动定律,解释了潮汐、彗星轨迹、岁差等现象,消除了关于太阳系日心说的疑虑。他展示了地球上的物体和天体的运动可以由相同的原理解释。牛顿推测地球为扁球体,这一推测后来由莫佩尔蒂、拉康达米娜等人的测地测量所证实,使得大多数欧洲科学家信服于牛顿力学的优越性。

他制造了第一个实用的反射望远镜,并基于棱镜将白光分解为可见光谱的颜色的观察,发展出一套精细的颜色理论。他关于光的研究汇集于其极具影响力的著作《光学》中,1704年出版。他提出了一个经验性的冷却定律,这是第一个热传导的表述,首次对声速进行了理论计算,并引入了牛顿流体的概念。此外,他还对电进行了早期研究,他在《光学》一书中的一个设想可以说是电场理论的开端。作为数学家,除了微积分的研究外,他还对幂级数进行了研究,将二项式定理推广至非整数指数,发展出求解函数根的方法,并分类了大部分的三次平面曲线。

牛顿是剑桥大学三一学院的成员,也是剑桥大学的第二任卢卡斯数学教授。他是一位虔诚但非正统的基督徒,私下拒绝三位一体的教义。他拒绝加入英国国教的圣职,这在当时的剑桥大学教员中是少见的。除了数学科学方面的工作之外,牛顿还将大量时间投入到炼金术和《圣经》年代学的研究中,但他在这些领域的大部分作品直到去世后很久才发表。在政治上,他与辉格党有密切联系,并曾在1689-1690年和1701-1702年两次短暂担任剑桥大学的国会议员。1705年,他被安妮女王封为爵士,并在伦敦度过了生命的最后三十年,担任皇家造币厂的监理(1696–1699)和厂长(1699–1727),以及皇家学会会长(1703–1727)。
\subsection{早期生活}  
主要条目:艾萨克·牛顿的早期生活  
艾萨克·牛顿于1642年12月25日(根据当时在英格兰使用的儒略历,公历为1643年1月4日)出生于林肯郡的伍尔斯索普庄园。[17]牛顿的父亲也是名叫艾萨克·牛顿,在他出生前三个月去世。牛顿出生时早产,身体较小;他的母亲汉娜·艾斯考说,他可以放进一个夸脱的杯子里。[18]当牛顿三岁时,母亲再婚,和她的新丈夫巴纳巴斯·史密斯牧师一起生活,留下牛顿由他的外祖母玛格丽·艾斯考(原姓布莱思)照顾。牛顿不喜欢他的继父,并对母亲再婚心存怨恨,这在他19岁之前的一份罪行清单中有所体现:“威胁我的父亲和母亲史密斯要烧掉他们和他们的房子。”[19] 牛顿的母亲在第二次婚姻中生了三个孩子(玛丽、本杰明和汉娜)。[20]
\subsubsection{国王学校}  
牛顿大约在十二岁到十七岁期间,就读于格兰瑟姆的国王学校,该校教授拉丁语和古希腊语,并可能为他打下了坚实的数学基础。1659年10月,他被母亲从学校撤回,回到了伍尔斯索普-拜-科尔斯特沃斯。母亲在第二次丧夫后,试图让他成为一名农民,但他对此职业十分厌恶。国王学校的校长亨利·斯托克斯说服他的母亲让他重返学校。部分出于对校园欺凌者的报复心态,他成为了年级第一的学生,主要通过制作日晷和风车模型而脱颖而出。[24]
\subsubsection{剑桥大学}  
1661年6月,牛顿被剑桥大学的三一学院录取。他的叔叔威廉·艾斯考牧师曾在剑桥学习,推荐他进入该校。在剑桥,牛顿起初以“补助生”的身份入学,通过做杂役来支付学费,直到1664年获得奖学金,覆盖他四年的大学费用,直至获得硕士学位。[25]当时,剑桥的教学以亚里士多德的理论为基础,牛顿与当时的现代哲学家如笛卡尔以及天文学家伽利略·伽利莱和托马斯·斯特里特一起阅读这些作品。他在笔记本中记录了一系列关于机械哲学的“问题”。1665年,他发现了广义二项式定理,并开始发展一个后来成为微积分的数学理论。1665年8月,牛顿在剑桥获得学士学位后,因应对大瘟疫,大学暂时关闭。[26]

尽管他在剑桥大学的学生时代并不显著[27] ,但在接下来的两年里,[28]牛顿在伍尔斯索普的家中进行的私下研究促成了他在微积分、光学和引力定律方面的理论发展。[29][30]

1667年4月,牛顿返回剑桥大学,并在10月被选为三一学院的研究员。[31][32]研究员需要接受圣职并被按立为英国国教牧师,尽管在复辟时期这一要求并未严格执行,符合英格兰教会的声明便足够。他承诺道:“我要么将神学作为我研究的对象,并在这些章程规定的时间(7年)到来时接受圣职,要么就辞去学院职务。”[33]在此之前,他对宗教并没有过多思考,曾两次签署同意《三十九条》,即英格兰教会教义的基础。到1675年,这个问题无法避免,而此时他的非常规观点成为了障碍。[34]

他的学术工作给卢卡斯教授艾萨克·巴罗留下了深刻印象,巴罗渴望发展自己的宗教和行政潜力(他在两年后成为三一学院的院长);在1669年,牛顿接替了他的职位,这距离他获得硕士学位仅一年。卢卡斯教授的任职条款要求持有者不得活跃于教会——可能是为了留出更多时间用于科学研究。牛顿认为这应该使他免于按立的要求,查理二世国王接受了这一论点,因此牛顿的宗教观点与英国国教的正统观念之间的冲突得以避免。[35]

剑桥大学卢卡斯数学教授的职位还包括教授地理的责任。[36][37]在1672年和1681年,牛顿出版了《一般地理》的修订、校正和增补版,这本地理教科书最初由已故的伯纳德·瓦伦纽斯于1650年出版。[38]在《一般地理》中,瓦伦纽斯试图建立一个理论基础,将科学原则与古典地理概念联系起来,并认为地理是科学与应用于量化地球特征的纯数学的结合。[36][39] 虽然不清楚牛顿是否曾讲授地理,但1733年杜格代尔和肖的英文翻译版本中提到,牛顿出版此书是为了让学生在他讲授这一主题时阅读。[36] 《一般地理》被一些人视为地理历史中古代与现代传统的分界线,而牛顿参与后续版本的编辑被认为是这一持久遗产的重要原因之一。[40]

牛顿于1672年被选为皇家学会会员(FRS)。[1]
\subsection{中年}  
\subsubsection{微积分}  
牛顿的工作被认为“显著推动了当时所有研究的数学分支”。[41] 他在这一主题上的研究,通常称为流量(fluxions)或微积分,见于1666年10月的一份手稿,现已被收录在牛顿的数学论文中。[42]他的作品《通过无限项数的方程进行分析》(De analysi per aequationes numero terminorum infinitas)于1669年6月由艾萨克·巴罗发送给约翰·柯林斯,巴罗在同年8月给柯林斯的信中指出,这是“一个非凡天才和在这些领域中精通的作品”。[43]牛顿后来与莱布尼茨发生了关于微积分发展优先权的争论。大多数现代历史学家认为,牛顿和莱布尼茨是独立发展微积分的,尽管他们使用的数学符号大相径庭。然而,已确定牛顿在莱布尼茨之前很早就开始发展微积分。[44][11][45] 莱布尼茨的符号和“微分法”,如今被认为是更加便利的符号,后来被欧洲大陆的数学家采用,并在1820年左右也被英国数学家接受。

他的工作广泛使用基于趋近于零的小量比率的极限值的几何形式微积分:在《自然哲学的数学原理》中,牛顿以“第一和最后比率的方法”这一名称进行了演示,[46] 并解释了为何以这种形式进行阐述,[47]同时也指出“通过这种方式,完成的与无体积法所完成的是相同的。”[48]因此,现代人称《原理》为“一本充满无穷微积分理论和应用的书”,而在牛顿的时代,[49] 几乎所有内容都与这种微积分有关。[50] 他在1684年的《旋转物体运动论》中使用了涉及“一或多个无穷小量阶”的方法,并且在他关于运动的论文中也体现了这一点,[51]“这些论文是在1684年前的二十年内写成的”。[52]

牛顿曾对发表他的微积分持犹豫态度,因为他担心会引发争议和批评。[53]牛顿与瑞士数学家尼古拉·法蒂奥·德·杜伊耶关系密切。1691年,杜伊耶开始撰写牛顿《原理》的新版本,并与莱布尼茨通信。[54]1693年,杜伊耶与牛顿的关系恶化,这本书也未能完成。[55]从1699年开始,皇家学会的其他成员指控莱布尼茨抄袭。[56]1711年,争论进一步升级,皇家学会在一项研究中宣称真正的发现者是牛顿,并将莱布尼茨标记为骗子;后来发现牛顿实际上写下了该研究关于莱布尼茨的结论性评论。由此开始了这场痛苦的争议,影响了牛顿和莱布尼茨两人的生活,直至莱布尼茨于1716年去世。[57]
\begin{figure}[ht]
\centering
\includegraphics[width=6cm]{./figures/d24e0f92cc76e386.png}
\caption{1702年,牛顿的肖像由戈弗雷·奈勒(Godfrey Kneller)绘制。} \label{fig_Newton_2}
\end{figure}
牛顿通常被认为是广义二项式定理的创立者,该定理适用于任何指数。他发现了牛顿恒等式和牛顿法,分类了三次平面曲线(两个变量的三次多项式),对有限差分理论做出了重要贡献,是第一个使用分数指数的人,并采用坐标几何推导丢番图方程的解。他通过对数近似调和级数的部分和(这是欧拉求和公式的前身),并首次自信地使用幂级数及其反演。牛顿对无穷级数的研究受到西蒙·斯特芬的十进制数的启发。[58]
\subsubsection{光学}
\begin{figure}[ht]
\centering
\includegraphics[width=6cm]{./figures/fb2394ae8235a2b3.png}
\caption{牛顿于1672年向皇家学会展示的反射望远镜的复制品(他在1668年制造的第一个望远镜被借给了一位仪器制造商,但之后没有关于其去向的进一步记录)。[59]} \label{fig_Newton_3}
\end{figure}
1666年,牛顿观察到在最小偏转位置,光线通过棱镜后形成的颜色光谱呈椭圆形,即使进入棱镜的光线是圆形的,这意味着棱镜以不同的角度折射不同的颜色。[60][61]这使他得出结论,颜色是光的内在属性——这一点在此之前一直是争论的焦点。

从1670年到1672年,牛顿讲授光学。[62] 在此期间,他研究了光的折射,证明了由棱镜产生的多彩图像(他称之为光谱)可以通过透镜和第二个棱镜重新组合成白光。[63] 现代学术研究表明,牛顿对白光的分析和再合成与粒子炼金术有一定关系。[64] 
\begin{figure}[ht]
\centering
\includegraphics[width=6cm]{./figures/d1e379c1b13d7d69.png}
\caption{牛顿发现的分散棱镜将白光分解成光谱颜色的示意图。} \label{fig_Newton_4}
\end{figure}
通过这项工作,他得出结论,任何折射望远镜的透镜都会受到光的色散影响(色差)。为了证明这一概念,他构建了一台使用反射镜而非透镜作为物镜的望远镜,以绕过这个问题。[66][67]构建这一设计,即今天所称的牛顿望远镜——第一台已知的功能性反射望远镜——涉及解决合适的镜面材料和成型技术的问题。[67]牛顿自己研磨镜面,使用一种高度反射的镜面金属定制合成,并利用牛顿环来判断望远镜光学的质量。到1668年底,[68]他成功制造了这台第一台反射望远镜,长度约为八英寸,能够提供更清晰和更大的图像。1671年,皇家学会要求他演示他的反射望远镜。[69]他们的兴趣鼓励牛顿发表了他的笔记《论颜色》,[70]他后来将其扩展为著作《光学》。当罗伯特·胡克批评牛顿的一些观点时,牛顿感到非常冒犯,以至于他退出了公共辩论。牛顿和胡克在1679年至1680年间有过简短的交流,那时胡克被任命为皇家学会的通信管理者,开始了旨在促使牛顿为皇家学会事务做出贡献的通信,[71]
\begin{figure}[ht]
\centering
\includegraphics[width=6cm]{./figures/8e4d4c58235cd719.png}
\caption{1682年牛顿致威廉·布里格斯的信件的复制品,评论了布里格斯的《新视觉理论》。} \label{fig_Newton_5}
\end{figure}
牛顿认为光是由粒子或微粒组成的,这些粒子在加速进入更密集的介质时会发生折射。他倾向于用声波来解释薄膜的反射和透射的重复模式(《光学》第二卷,第12条),但仍然保留了他的“适合”理论,认为微粒在反射或透射时会受到影响(第13条)。然而,后来的物理学家更倾向于用纯波动的解释来说明光的干涉图样和衍射现象。今天的量子力学、光子和波粒二象性的概念与牛顿对光的理解仅有微小的相似之处。

在1675年的《光的假说》中,牛顿假设以太的存在,以便在粒子之间传递力。[73]与剑桥的柏拉图主义哲学家亨利·摩尔的接触重新激发了他对炼金术的兴趣。他用基于赫尔墨斯吸引与排斥观念的神秘力量取代了以太。约翰·梅纳德·凯恩斯,曾获得牛顿许多关于炼金术的著作,指出“牛顿不是理性时代的第一人:他是最后一位魔法师。”[74]牛顿对科学的贡献无法与他对炼金术的兴趣割裂开来。[73]当时,炼金术和科学之间并没有明确的区分。

在1704年,牛顿出版了《光学》,在书中阐述了他的粒子理论。他认为光由极细微的微粒组成,而普通物质则由更粗大的微粒构成,并推测通过某种炼金术的转化,“粗大物体和光是否可以互相转化……物体是否可以从进入其组成的光粒子中获得大量的活性?”[75]牛顿还构造了一种原始形式的摩擦静电发生器,使用了一个玻璃球。[76]

在他的著作《光学》中,牛顿首次展示了使用棱镜作为光束扩展器的图示,并介绍了多棱镜阵列的应用。[77]大约278年后,多棱镜光束扩展器成为窄线宽可调激光器发展的核心。此外,这些棱镜光束扩展器的使用导致了多棱镜色散理论的产生。[77]

在牛顿之后,许多理论被修正。扬和菲涅尔放弃了牛顿的粒子理论,转而支持惠更斯的波动理论,以展示颜色是光波长的可见表现。科学也逐渐意识到颜色感知与可数学化光学之间的区别。德国诗人兼科学家歌德无法摆脱牛顿的基础理论,但“歌德确实在牛顿的盔甲中找到了一个漏洞……牛顿坚持认为没有颜色的折射是不可能的,因此他认为望远镜的物镜将永远不完美,消色差和折射是不相容的。这一推论后来被多隆证明是错误的。[78]
\subsubsection{重力}
\begin{figure}[ht]
\centering
\includegraphics[width=6cm]{./figures/51b511f20a306b0c.png}
\caption{约翰·范德班克所作的牛顿肖像雕刻。} \label{fig_Newton_6}
\end{figure}
牛顿早在1665年就开始发展他的引力理论。[29][30]1679年,牛顿通过考虑引力及其对行星轨道的影响,重新回到他的天体力学研究,参考了开普勒的行星运动定律。这是受到1679年至1680年与被任命为皇家学会秘书的胡克的一次简短书信交流的刺激,[79]胡克开启了一项旨在促使牛顿为皇家学会事务做出贡献的通信。[71]牛顿对天文学的重新兴趣在1680至1681年的冬季受到了一颗彗星出现的进一步刺激,他与约翰·弗拉姆斯蒂德进行了通信。[80]在与胡克的交流之后,牛顿推导出一个证明,表明行星轨道的椭圆形状源于与半径向量的平方成反比的向心力。牛顿将他的结果以《物体运动论》为题,传达给了埃德蒙·哈雷和皇家学会,这是一篇约九张纸的论文,于1684年12月被抄录入皇家学会的登记册中。[81]这篇论文包含了牛顿发展并扩展形成《原理》的核心内容。
\begin{figure}[ht]
\centering
\includegraphics[width=6cm]{./figures/5dbd8e950865e95f.png}
\caption{牛顿的《原理》自藏本,包含牛顿手写的第二版修正,现在存放于剑桥大学三一学院的温伦图书馆。} \label{fig_Newton_7}
\end{figure}
《原理》于1687年7月5日出版,得到了哈雷的鼓励和财政支持。在这部著作中,牛顿阐述了三条普遍运动定律。这些定律描述了任何物体、作用于其上的力以及由此产生的运动之间的关系,为经典力学奠定了基础。它们对随后不久的工业革命带来了许多进展,并且在200多年内没有得到改进。这些进展中的许多仍然是现代世界中非相对论技术的基础。他使用拉丁词汇“gravitas”(重量)来描述后来被称为重力的效应,并定义了万有引力定律。[82]

在同一部著作中,牛顿提出了一种类似于微积分的几何分析方法,使用“首尾比率”,并首次通过波义耳定律对空气中的声音速度进行了分析性测定。他推断出地球的椭球体形状,并将春分点的进动归因于月球对地球椭圆形的引力作用。他开始研究月球运动中的不规则性,并提供了彗星轨道确定的理论,还有许多其他贡献。[82]牛顿的传记作者大卫·布鲁斯特报道说,将他的引力理论应用于月球运动的复杂性之大,影响了牛顿的健康:“在他处理这一问题的1692至1693年期间,他失去了食欲和睡眠”,并告诉天文学家约翰·马钦,“他只在研究这一主题时才头痛”。据布鲁斯特所述,埃德蒙·哈雷还告诉约翰·康杜伊特,当牛顿被催促完成他的分析时,他“总是回答这让他头痛,常常让他失眠,以至于他再也不想考虑这个问题”。【原文强调】[83]

牛顿明确了他对太阳系的日心观点——这一观点以一种较为现代的方式发展,因为早在1680年代中期,他就意识到太阳相对于太阳系重心的“偏差”。[84]对于牛顿来说,不能准确地认为太阳或任何其他天体是静止的,而应该认为“地球、太阳和所有行星的共同重心被视为世界的中心”,这个重心“要么静止,要么沿直线匀速前进”。(考虑到普遍共识,即无论中心位于何处,都是静止的,牛顿选择了“静止”这一选项。)[85]

牛顿因在科学中引入“神秘力量”而受到批评,因为他假设了一种能够在广阔距离上作用的看不见的力量。[86]在《原理》第二版(1713年)中,牛顿在结尾的《一般学论》中坚决拒绝了这样的批评,写道,现象足以暗示引力的存在,但并未表明其原因,因此构建不由现象所暗示的假设既不必要也不恰当。(在这里,牛顿使用了他著名的表达“Hypotheses non fingo”。)[87]

凭借《原理》,牛顿获得了国际认可。[88]他获得了一批追随者,其中包括瑞士出生的数学家尼古拉·法蒂奥·德·杜伊耶。[89]

在1710年,牛顿找到了78种立方曲线中的72种,并将它们分类为四种类型。[90]1717年,詹姆斯·斯特林在牛顿的帮助下证明了每个立方曲线都是这四种类型之一。牛顿还声称,这四种类型可以通过平面投影从其中一种获得,这一点在1731年得到了证明,距离牛顿去世四年。[91]
\subsection{晚年}   
\subsubsection{皇家铸币厂}
\begin{figure}[ht]
\centering
\includegraphics[width=6cm]{./figures/285008173cc69f79.png}
\caption{1712年牛顿老年的肖像,画家:詹姆斯·索恩希尔爵士。} \label{fig_Newton_8}
\end{figure}
在1690年代,牛顿撰写了一些宗教著作,探讨了《圣经》的字面和象征性解读。他寄给约翰·洛克的一份手稿中,质疑《约翰一书》5章7节——即“约翰的插入句”——与新约原始手稿的一致性,这份手稿直到1785年才得以出版。[92]

牛顿还曾于1689年和1701年担任剑桥大学的英格兰议会议员,但根据一些说法,他的唯一评论是抱怨议会里的冷风,并要求关闭窗户。[93]然而,剑桥的日记作者亚伯拉罕·德·拉·普赖姆注意到,牛顿曾训斥那些通过声称一所房子闹鬼而吓坏当地人的学生。[94]

牛顿于1696年迁往伦敦,担任皇家铸币厂的监护人,这一职位是在威廉三世国王统治期间获得的,得到的是哈利法克斯伯爵查尔斯·蒙塔古的支持,他当时是财政大臣。牛顿负责英国的大规模重铸,曾触怒塔楼的总督卢卡斯勋爵,并为埃德蒙·哈雷争取到了临时切斯特分支的副审计员职位。1699年托马斯·尼尔去世后,牛顿成为最著名的铸币厂厂长,这一职务他一直担任到生命的最后30年。这些任命本是荣誉职务,但牛顿对此十分认真。[95][96]他于1701年辞去了剑桥的职务,利用自己的权力改革货币,惩罚剪币和伪造者。

作为皇家铸币厂的监护人和厂长,牛顿估计在1696年的大重铸中,约20\%的流通硬币是伪造的。伪造货币被视为叛国罪,罪犯将面临绞刑、拖尸和肢解的惩罚。尽管如此,起诉即使是最明显的罪犯也可能极为困难,但牛顿证明了自己能胜任这一任务。[97]