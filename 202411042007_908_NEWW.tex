% 牛顿万有引力定律(综述)
% license CCBYSA3
% type Wiki


本文根据 CC-BY-SA 协议转载翻译自维基百科\href{https://en.wikipedia.org/wiki/Newton\%27s_law_of_universal_gravitation}{相关文章}。

牛顿的万有引力定律指出,宇宙中的每一个粒子都以一种力吸引着其他粒子,这种力与它们质量的乘积成正比,与它们中心之间距离的平方成反比。相隔的物体互相吸引,就好像它们的所有质量都集中在它们的中心一样。该定律的发表被称为“第一次伟大统一”,因为它标志着地球上重力现象与已知的天文行为的统一。[1][2][3]

这是一个从经验观察中通过艾萨克·牛顿所称的归纳推理得出的普遍物理定律。[4] 它是经典力学的一部分,最早由牛顿在其著作《自然哲学的数学原理》(简称《原理》)中提出,该书首次出版于1687年7月5日。

因此,万有引力的方程形式为:
\[
F = G \frac{{m_1 m_2}}{{r^2}}~
\]
其中,\( F \) 是作用在两个物体之间的引力,\( m_1 \) 和 \( m_2 \) 是两个物体的质量,\( r \) 是它们质心之间的距离,\( G \) 是引力常数。