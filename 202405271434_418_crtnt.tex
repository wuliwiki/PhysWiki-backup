% 中国剩余定理
% keys 中国剩余定理
% license Usr
% type Tutor

\pentry{线性同余\nref{nod_linmod}}{nod_3099}
\begin{theorem}{中国剩余定理}
假设一组整数 $m_1, m_2, \dots, m_k$ 两两互素,则对于任意的一组整数 $a_1, a_2, \dots, a_k$,关于 $x$ 的同余方程组
\begin{equation}\left\{
\begin{aligned}
x &\equiv a_1 \pmod{m_1} \\
x &\equiv a_2 \pmod{m_2} \\
\dots   \\
x &\equiv a_k \pmod{m_k}
\end{aligned}\right. ~~
\end{equation}
在模 $m_1m_2\dots m_k$ 意义下有解且仅有一组同余解.
\end{theorem}
\textbf{证明}:设 $M = \prod_i m_i$,而令 $M_a = M / m_a$。由于 $m_i$ 间两两互素,故 $M_a$ 与 $m_a$ 互素,也就是 $(M_a, m_a) = 1$,利用\autoref{the_linmod_4}~\upref{linmod},关于 $N_a$ 的同余方程
\begin{equation}
N_a M_a \equiv 1 \pmod{m_a} ~~
\end{equation}
将仅有一组同余意义下的解。利用各 $N_a$ 与 $M_a$ 可以构造出模 $M$ 意义下的解,具体的,
\begin{equation}
x
\end{equation}
