% 波包的延迟(量子力学)
% keys 相移|wigner|EWS|傅里叶变换|群速度
% license Xiao
% type Tutor

\begin{issues}
\issueDraft
\end{issues}

\pentry{一维散射(量子)\nref{nod_Sca1D}, 群速度\nref{nod_GroupV}}{nod_2115}
广义情况下,一阶微扰得到的波包是(可能做任何运动,在任何形状的势能中), 令 $\ket{k}$ 为精确散射态(无论几维波函数,无论出于什么样的势能函数中)
\begin{equation}
\ket{\psi(t)} = \int C(E)\ket{E}\E^{-\I Et}\dd{k}~.
\end{equation}
其中 $E = k^2/2$。 那么该波包 $\Delta t$ 时间之前的样子就精确是
\begin{equation}
\ket{\psi(t-\Delta t)} = \int C'(E)\ket{E}\E^{-\I Et}\dd{k}~.
\end{equation}
其中 $C'(E) = C(E)\E^{\I E \Delta t}$。 可见,给系数添加一个关于能量的相位 $E \Delta t$ 就会使波包精确落后 $\Delta t$,也添加相位后的波函数要再演化 $\Delta t$ 才能达到添加相位以前的状态。以上的一切都是精确的。

\subsubsection{一维自由粒子}
\begin{equation}
\psi(x,t) = \int A(k)\exp\I\qty[k(x-k_0t - kt/2 + k_0t)] \dd{k}~.
\end{equation}

首先,对于 $V(x)=0$ 的自由粒子,若
\begin{equation}
\psi(x,t) = \int A(k)\E^{\I (kx-Et)} \dd{k}~.
\end{equation}
如果 $A(k)$ 是以 $k_0$ 为中心的\textbf{偶函数且实函数},那么根据傅里叶变换的\autoref{the_FTExp_1}~\upref{FTExp} 和平移性质(\autoref{eq_FTExp_7}~\upref{FTExp}), $\psi(x,0)$ 就是一个\textbf{偶函数且实函数}乘以 $\E^{\I k_0 x}$。 根据 Ehrenfest 定理的\autoref{eq_Ehrenf_2}~\upref{Ehrenf},无论初始形状的具体形状如何 $\ev{x} = k_0 t$ 都是精确成立的。但对任意单调的 $V(x)$, $\ev{x}$ 的轨迹并不精确等于经典粒子,因为 $\mel{\psi}{V(x)}{\psi} \approx V(\mel{\psi}{x}{\psi})$ 通常只是近似成立(势能曲线改变越缓慢,越精确)。

\addTODO{当 $t>0$ 时是否波包的形状也像高斯波包那样关于 $x_c$ 对称的呢(虽然肯定有 chirp)?}

根据以上结论,在此基础上若令 $C(k) = A(k)\E^{\I E\Delta t}$,那么此直线运动的轨迹就是 $x_c = k_0(t-\Delta t)$。而
\begin{equation}
\Delta t = \pdv{E} \arg C(k)~.
\end{equation}

如果基底不是平面波而是一个类似库仑平面波(相位甚至随位置变化,可能导致频率随位置变化),那么只需要把基底中的额外相位都并入到 $C(k)$ 中一起做能量偏导即可。
\addTODO{但时间延迟是相对于哪条直线呢?尤其是在并不太远的距离处。}

\subsubsection{一维散射}
一个一维自由波包用\enref{傅里叶变换}{FTTri}表示为
\begin{equation}\label{eq_tDelay_3}
\psi(x, t) = \int_{-\infty}^{+\infty} A(k) \exp \I[kx - \omega t + \varphi(k)] \dd{k}~,
\end{equation}
其中 $A(k)$ 是一个\textbf{实值函数}。 自由粒子满足 $\omega = k^2/(2m)$。 如果经过一个局部的势阱或势垒, 不同平面波透射后发生相移 $\phi(k)$, 经过后, 波包为
\begin{equation}\label{eq_tDelay_1}
\psi'(x, t) = \int_{-\infty}^{+\infty} A'(k) \exp \I[kx - \omega t + \varphi(k) + \phi(k)] \dd{k}~.
\end{equation}
想象一个特殊情况: 经过势阱后 $A'(k) = a A(k)$ 只相差一个常数, 且相移和 $\omega$ 成正比: $\phi(\omega) = \Delta t \omega$, 那么波函数变为
\begin{equation}
\psi'(x, t) = \int_{-\infty}^{+\infty} A(k) \exp \I[kx - \omega (t - \Delta t) + \varphi(k)] \dd{k}
= \psi(x, t - \Delta t)~.
\end{equation}
这样波包在时间轴上向右平移了 $\Delta t$, 即延迟了 $\Delta t$。 近似来说, 如果波包频率带宽较窄, 中心为 $\omega_0$, 那么在带宽以内可以把 $\phi$ 近似看成是 $\omega$ 的线性函数, 那么延迟近似为
\begin{equation}\label{eq_tDelay_2}
\Delta t(\omega_0) = \eval{\dv{\phi}{\omega}}_{\omega = \omega_0}~.
\end{equation}
如果要取一个与波包形状无关的延迟的定义, 那么这个定义是最佳选择。 注意这样定义的延迟与频率有关。 这个延迟被称为 \textbf{Wigner 延迟}或者 \textbf{EWS (Eisenbud-Wigner-Smith) 延迟} 。

\subsection{驻相法}
使用\textbf{驻相法(stationary phase method)}可以分析出波包\textbf{峰值}的近似位置、速度以及时间延迟。 驻相法比上面的近似要略微严谨一些, 获得的信息也多一些。 同样\textbf{假设波包带宽较窄或者 $A(k)$ 的相位随 $\omega$ 线性变化}。 那么对给定的 $x, t$, 当且仅当\autoref{eq_tDelay_3} 中的积分在被积函数的总相位不随 $k$ 变化时取得模长最大值。 被积函数的总相位等于 $kx - \omega t + \varphi$, 令其对 $k$ 求导为零, 有
\begin{equation}\label{eq_tDelay_5}
x = v_g \qty(t - \dv{\varphi}{\omega})~.
\end{equation}
其中 $v_g = \dv*{\omega}{k} = k$ 是波包的速度, 即\enref{群速度}{GroupV}。 确切来说, 是\textbf{波峰}移动的速度。 该式反应了波峰随时间的变化轨迹。 同样, 如果给 $\varphi$ 加上一个相移 $\phi$, 可得\autoref{eq_tDelay_2}。 可见, 驻相法不仅可以得到相对延迟, 还可以得到\autoref{eq_tDelay_3} 的波峰的绝对延迟
\begin{equation}\label{eq_tDelay_4}
\Delta t(\omega) = \dv{\varphi}{\omega}~.
\end{equation}
若波包是对称的, 那么波峰的位置等于波包的平均位置, 若不对称, 则二者不相等, 绝对延迟是指前者。

\subsection{短程中心势能散射的延迟}

原理和上文也几乎一样。 无论有限远处的势能是什么样, 只要无限远的地方有极限相移, 那么就用该相移来计算即可。

\subsection{库仑势能散射的延迟}
库仑势能的库仑函数在无穷远处不存在相移的极限。 所以这个延迟会是无穷大。
\begin{equation}
\delta_l(E) =  - \frac{\pi l}{2} - \eta\ln(2kr) + \sigma_l(E)~,
\end{equation}
而且这个相移和 $l$ 有关。
