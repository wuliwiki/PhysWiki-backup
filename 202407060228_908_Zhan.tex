% 栈(综述)
% license CCBYSA3
% type Wiki

(本文根据 CC-BY-SA 协议转载自原搜狗科学百科对英文维基百科的翻译)

在计算机科学中,堆栈作为元素的集合是一种抽象数据类型,有两个主要操作:
\begin{itemize}
\item 推送,它将元素添加到集合中。
\item pop,移除最近添加的尚未移除的元素。
\end{itemize}
元素从堆栈中取出的顺序产生了它的另一个名字,叫做后进先出。此外,查看操作可以在不修改堆栈的情况下访问顶部。[1] 这种结构的名称“堆栈”来源于一组堆叠在一起的物理项目的类比,这使得从堆栈顶部取出一个项目变得容易,而到达堆栈中更深处的项目可能需要先取出多个上层的其他项目。[2]

推送和弹出操作被视为线性数据结构,或者更抽象地说是顺序集合,只发生在结构的一端,即堆栈的顶部。这使得将堆栈实现为单个链表和指向顶部元素的指针成为可能。堆栈可以被实现为具有有限容量的空间。如果堆栈已满,并且没有足够的空间接受要推送的实体,则堆栈被认为处于溢出状态。弹出操作从堆栈顶部移除一个项目。

\subsection{历史}
斯塔克斯(Stacks)于1946年进入计算机科学文献,当时艾伦·图灵(Alan M. Turing)使用“埋葬”和“不埋葬”("bury" and "unbury")这两个术语作为调用子程序和从子程序返回的手段。[3] 子程序已经于1945年在康拉德·楚泽的Z4( Konrad Zuse's Z4 )中实现。

慕尼黑理工大学(Technical University Munich)的克劳斯·萨姆森(Klaus Samelson)和弗里德里希·鲍尔(Friedrich L. Bauer)于1955年提出了这个想法,并于1957年申请了专利,[4] 鲍尔(Bauer)于1988年3月因堆栈原理的发明获得了计算机先锋奖。[5] 澳大利亚人查尔斯·伦纳德·汉布林(Charles Leonard Hamblin)在1954年上半年独立发展了同样的概念。[6]

堆栈通常被比喻成自助餐厅中弹簧加载的一堆盘子。[7][2][8] 干净的盘子放在堆叠的顶部,将任何已经在那里的盘子向下推。当一个盘子从堆叠中取出时,它下面的那个就会弹出来成为新的顶部。

\subsection{ 非必要操作}
在许多实现中,堆栈比“推送”和“弹出”有更多的操作。一个例子是“栈顶”,即“读取数据”,它观察最顶层的元素而不将其从栈中移除。[9] 由于这可以通过使用相同数据的“弹出”和“推送”来完成,所以这不是必需的。如果堆栈为空,“堆栈顶部”操作可能会出现下溢情况,与“弹出”操作相同。此外,实现通常有一个函数用于返回堆栈是否为空。

\subsection{软件栈}
\subsubsection{3.1 实现}
堆栈可以通过数组或链表轻松实现。在这两种情况下,将数据结构标识为堆栈的不是实现,而是接口:只允许用户将项目弹出或推送到数组或链表上,几乎没有其他辅助操作。下面将使用伪代码演示这两种实现。

\textbf{数组}
数组可用于实现(有界)堆栈,如下所示。第一个元素(通常在零点偏移处)是底部,导致数组[0]成为被推到堆栈上的第一个元素,最后一个元素弹出。程序必须跟踪堆栈的大小(长度),使用一个可变的顶部记录到目前为止推送的项目数,从而指向数组中下一个元素要插入的位置(假设从零开始的索引约定)。因此,堆栈本身可以有效地实现为三元素结构:
\begin{lstlisting}[language=cpp]
structure stack:
    maxsize : integer
    top : integer
    items : array of item
\end{lstlisting}

\begin{lstlisting}[language=cpp]
procedure initialize(stk : stack, size : integer):
    stk.items ← new array of size items, initially empty
    stk.maxsize ← size
    stk.top ← 0
\end{lstlisting}
在检查溢出后,推送(push)操作添加一个元素并递增顶部索引:
\begin{lstlisting}[language=cpp]
procedure push(stk : stack, x : item):
    if stk.top = stk.maxsize:
        report overflow error
    else:
        stk.items[stk.top] ← x
        stk.top ← stk.top + 1
\end{lstlisting}
类似地,弹出(pop)在检查下溢后递减顶部索引,并返回先前的顶部索引:
\begin{lstlisting}[language=cpp]
procedure pop(stk : stack):
    if stk.top = 0:
        report underflow error
    else:
        stk.top ← stk.top − 1
        r ← stk.items[stk.top]
        return r
\end{lstlisting}
使用动态数组,有可能实现一个可以根据需要增长或收缩的堆栈。堆栈的大小就是动态数组的大小,这是一个非常有效的堆栈实现,因为向动态数组的末尾添加项目或从末尾移除项目需要摊销的O(1)时间。

\textbf{链表}

实现堆栈的另一个选项是使用单链表。堆栈是指向列表“头”的指针,也许还有一个计数器来追踪列表的大小:
\begin{lstlisting}[language=cpp]
structure frame:
    data : item
    next : frame or nil
\end{lstlisting}
\begin{lstlisting}[language=cpp]
structure stack:
    head : frame or nil
    size : integer
\end{lstlisting}
\begin{lstlisting}[language=cpp]
procedure initialize(stk : stack):
    stk.head ← nil
    stk.size ← 0
\end{lstlisting}
推送和弹出项目出现在列表的最前面;在此实现中,溢出是不可能的(除非内存耗尽):
\begin{lstlisting}[language=cpp]
procedure push(stk : stack, x : item):
    newhead ← new frame
    newhead.data ← x
    newhead.next ← stk.head
    stk.head ← newhead
    stk.size ← stk.size + 1
\end{lstlisting}
\begin{lstlisting}[language=cpp]
procedure pop(stk : stack):
    if stk.head = nil:
        report underflow error
    r ← stk.head.data
    stk.head ← stk.head.next
    stk.size ← stk.size - 1
    return r
\end{lstlisting}

\subsubsection{3.2 堆栈和编程语言}