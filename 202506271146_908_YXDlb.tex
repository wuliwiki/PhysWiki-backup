% 伊西多·拉比(综述)
% license CCBYSA3
% type Wiki

本文根据 CC-BY-SA 协议转载翻译自维基百科 \href{https://en.wikipedia.org/wiki/Isidor_Isaac_Rabi}{相关文章}。

\begin{figure}[ht]
\centering
\includegraphics[width=6cm]{./figures/6a4d8eb51e8bbdb5.png}
\caption{1944年的拉比} \label{fig_YXDlb_1}
\end{figure}
以色列·“伊西多”·艾萨克·拉比(Israel "Isidor" Isaac Rabi,/ˈrɑːbi/;意第绪语:איזידאָר יצחק ראַבי,转写:Izidor Yitzkhok Rabi\(^\text{[2]}\);1898年7月29日-1988年1月11日)是一位美国核物理学家,因“发明用于记录原子核磁性特性的共振方法”而获得1944年诺贝尔物理学奖。他也是美国最早研究腔体磁控管的科学家之一,该装置广泛用于微波雷达和微波炉中。

拉比出生于加利西亚赖马努夫的一个传统波兰犹太家庭,婴儿时期随家人移民至美国,在纽约下东区长大。1916年,他以电气工程专业身份进入康奈尔大学学习,但不久后转向化学,后来又对物理学产生了兴趣。他在哥伦比亚大学继续深造,并因研究某些晶体的磁化率而获得博士学位。1927年,他前往欧洲,与当时许多顶尖物理学家会面并共事。

1929年,拉比返回美国,哥伦比亚大学为他提供了教职。与格雷戈里·布赖特合作时,他发展出了布赖特–拉比方程,并预测斯特恩–盖拉赫实验可以通过改进来验证原子核的某些特性。他利用核磁共振技术测定原子的磁矩和核自旋的研究,使他获得了1944年诺贝尔物理学奖。核磁共振技术随后成为核物理和化学中的重要工具,并进一步发展出磁共振成像(MRI)技术,使其在医学领域也具有重要意义。二战期间,拉比在麻省理工学院的辐射实验室从事雷达研究,同时也参与了曼哈顿计划。战后,他担任美国原子能委员会下属的一般顾问委员会(GAC)成员,并于1952年至1956年间出任委员会主席。他还参与了国防动员办公室和陆军弹道研究实验室的科学顾问委员会,并曾担任总统德怀特·艾森豪威尔的科学顾问。

拉比参与创建了布鲁克海文国家实验室,并于1946年推动其成立。作为美国教科文组织代表,他还参与了1952年欧洲核子研究中心(CERN)的创建。1964年哥伦比亚大学设立“大学教授”职位时,拉比是首位获得此殊荣的人。1985年,该校还以他的名字命名了一个特别讲席。他于1967年退休,退出教学工作,但仍积极参与系内事务,并一直保留“荣誉大学教授”与“特别讲席教授”头衔直至去世。
\subsection{早年经历}
以色列·艾萨克·拉比于1898年7月29日出生在奥匈帝国时期的加利西亚地区赖马努夫,该地现属波兰。他出生于一个波兰犹太正统家庭。不久之后,他的父亲戴维·拉比移民到了美国。几个月后,年幼的拉比和母亲谢因德尔也前往美国与父亲团聚,全家搬进了曼哈顿下东区一套两居室的公寓。在家中,他们讲意第绪语。拉比上学时,母亲告诉校方他的名字是“Izzy”,校方人员误以为是“Isidor”的昵称,于是将“Isidor”登记为他的正式名字。从此,这个名字便成为他的官方姓名。后来,为了应对反犹主义,他开始将自己的名字写作“Isidor Isaac Rabi”,并在职业场合以“I.I. Rabi”为人所知。对他的大多数亲友来说,包括1903年出生的妹妹格特鲁德,他通常被简称为“Rabi”。1907年,全家搬到布鲁克林的布朗斯维尔,并在那里经营一家杂货店。\(^\text{[3]}\)

童年时期,拉比对科学产生了浓厚兴趣。他常常从公共图书馆借阅科学书籍阅读,还自己动手制作收音机。他的第一篇科学论文——关于电容器设计的文章——在他还在小学时便发表在《现代电气》杂志上。\(^\text{[4][5]}\)在读到哥白尼的日心说后,他成为了一名无神论者。他对父母说:“这一切都很简单,谁还需要上帝?”\(^\text{[6]}\)作为对父母的妥协,他的成年礼在家中举行,他用意第绪语发表了一篇关于电灯如何工作的演讲。

他就读于布鲁克林的实用技术高中,于1916年毕业。\(^\text{[7]}\)同年晚些时候,他进入康奈尔大学,最初主修电气工程,但很快转向化学专业。1917年美国加入第一次世界大战后,他在康奈尔参加了学生陆军训练团。他的本科毕业论文研究了锰的氧化态。1919年6月,他获得了理学士学位。但当时犹太人在化学工业和学术界的就业机会非常有限,他没有收到任何工作录用通知。他曾短暂在莱德利实验室工作,之后做了一段时间的簿记员。\(^\text{[8]}\)
\subsection{教育经历}
1922年,拉比重返康奈尔大学攻读化学研究生学位,并开始学习物理。1923年,他遇见了亨特学院暑期课程的学生海伦·纽马克,并开始追求她。为了能在她回家时留在她身边,他转学至哥伦比亚大学继续深造,导师是阿尔伯特·威尔斯。1924年6月,拉比在纽约市立学院找到了一份兼职导师的工作。威尔斯专攻磁学,他建议拉比将博士论文题目定为钠蒸气的磁化率。这个题目并未引起拉比的兴趣,但在听完威廉·劳伦斯·布拉格在哥伦比亚关于某些晶体(被称为图顿盐,Tutton's salts)电化率的讲座后,拉比决定研究这些晶体的磁化率,威尔斯也同意指导他的研究。\(^\text{[9]}\)

测量晶体的磁共振首先要培育晶体,这是一个简单的过程,常由小学生完成。但随后需将晶体切割成各向异性的薄片,并精确测量其在磁场中的响应。晶体生长期间,拉比阅读了詹姆斯·克拉克·麦克斯韦1873年的著作《电与磁论》,从中获得了灵感,想出了一个更简单的方法:他将晶体系在玻璃纤维上,再接入一个扭力天平系统,把晶体浸入一个磁化率可调的溶液中并放置在两个磁极之间。当溶液的磁化率与晶体一致时,磁铁的开关不会影响晶体的位置。这个新方法不仅更为省力,还能获得更精确的结果。1926年7月16日,拉比将题为《晶体的主磁化率》的论文投稿至《物理评论》杂志。第二天,他与海伦结婚。这篇论文在学术界没有引起太大关注,尽管卡里亚玛尼卡姆·斯里尼瓦萨·克里希南读过并在研究中加以应用。拉比由此认识到,除了发表研究成果,还需要积极宣传自己的工作。\(^\text{[10][11]}\)

像许多其他年轻物理学家一样,拉比密切关注着欧洲正在发生的重大事件。他对斯特恩–盖拉赫实验感到震惊,这项实验使他确信量子力学的正确性。他与拉尔夫·克罗尼希、弗朗西斯·比特、马克·泽曼斯基等人一起,着手将薛定谔方程推广到对称刚体分子体系,并试图找出该类力学系统的能级状态。问题在于,他们都无法解出所得到的那个二阶偏微分方程。拉比在路德维希·施莱辛格(的《微分方程理论导论》中找到了答案。书中描述了一种由卡尔·古斯塔夫·雅可比最早发展出来的方法。该方程具有超几何方程的形式,而雅可比曾找到过其解。克罗尼希与拉比将他们的研究结果整理成文,并投稿至《物理评论》杂志,该论文于1927年发表。\(^\text{[12][13]}\)
\subsection{欧洲}
1927年5月,拉比获得了巴纳德奖学金的任命。这份奖学金为期从1927年9月至1928年6月,资助金额为1500美元(相当于2024年的约27000美元\(^\text{[14]}\))。他立即向纽约城市学院申请为期一年的休假前往欧洲深造。但因申请被拒,他选择辞职。拉比抵达苏黎世时,本希望能为埃尔温·薛定谔工作,却遇到了两位美国同乡——朱利叶斯·亚当斯·斯特拉顿和莱纳斯·鲍林。他们得知薛定谔即将离开,因他已被任命为柏林弗里德里希·威廉大学理论研究所的所长。于是拉比转而前往慕尼黑大学,寻求阿诺德·索末菲尔德的接纳。到达慕尼黑后,他遇到了另外两位美国人:霍华德·珀西·罗伯逊和爱德华·康登。索末菲尔德接纳拉比为博士后研究员。当时德国物理学家鲁道夫·佩耶尔斯和汉斯·贝特也在索末菲尔德门下,但三位美国人之间关系尤为亲密。\(^\text{[15]}\)

根据导师威尔斯的建议,拉比前往利兹参加了第97届英国科学促进会年会,在会上听取了维尔纳·海森堡关于量子力学的报告。会后,拉比转往哥本哈根,自愿为尼尔斯·玻尔工作。玻尔当时正在度假,拉比便着手研究分子氢的磁化率计算。玻尔于10月归来后,安排拉比与西田吉男一道,继续在汉堡大学与沃尔夫冈·泡利合作进行研究。\(^\text{[16]}\)

尽管拉比是为了与泡利合作而来到汉堡的,但他发现奥托·斯特恩也在此工作,并且身边还有两位讲英语的博士后研究员:罗纳德·弗雷泽和约翰·布拉德肖·泰勒。拉比很快与他们成为朋友,并对他们进行的分子束实验产生了兴趣,\(^\text{[17]}\)正是这项研究使斯特恩在1943年获得了诺贝尔物理学奖。\(^\text{[18]}\)他们的研究涉及非均匀磁场,这类磁场难以操控,且难以准确测量。拉比设想了一种替代方法:使用均匀磁场,并将分子束以掠射角入射,使原子像光线经过棱镜一样发生偏折。这种方法不仅更易操作,而且能得到更精确的结果。在斯特恩的鼓励下,并在泰勒的大力协助下,拉比成功实现了这一设想。根据斯特恩的建议,拉比将自己的研究结果写信投稿给《自然》杂志,\(^\text{[17]}\)该期刊于1929年2月刊登了这项成果,\(^\text{[19]}\)随后他又将题为《Zur Methode der Ablenkung von Molekularstrahlen》(《关于分子束偏转方法》)的论文投至《物理学杂志》,并于同年4月发表。\(^\text{[20]}\)

此时,巴纳德奖学金已到期,拉比与妻子海伦靠着来自洛克菲勒基金会的每月182美元(相当于2024年的约3300美元\(^\text{[14]}\)资助维持生活。他们离开汉堡前往莱比锡,拉比希望能在那里与海森堡合作。在莱比锡,他遇见了来自纽约的罗伯特·奥本海默,这也开启了两人长久的友谊。1929年3月,海森堡前往美国巡回讲学,于是拉比与奥本海默决定前往苏黎世联邦理工学院,当时泡利正任该校物理学教授。拉比在那里结识了许多该领域的领军人物,这极大丰富了他的物理学教育经历,其中包括保罗·狄拉克、瓦尔特·海特勒、弗里茨·伦敦、弗朗西斯·惠勒·卢米斯、约翰·冯·诺依曼、约翰·斯莱特、利奥·西拉德和尤金·维格纳。\(^\text{[21]}\)
\subsection{分子束实验室}
\begin{figure}[ht]
\centering
\includegraphics[width=6cm]{./figures/41d7c72dc08e530f.png}
\caption{拉比(右)与同为诺贝尔奖得主的欧内斯特·O·劳伦斯(左)和恩里科·费米(中)合影} \label{fig_YXDlb_2}
\end{figure}
1929年3月26日,拉比收到了哥伦比亚大学的讲师聘书,年薪为3,000美元。哥伦比亚大学物理系主任乔治·B·皮格拉姆正在寻找一位理论物理学家来教授统计力学和一门关于新兴学科量子力学的高级课程,海森堡推荐了拉比。此时海伦已怀孕,拉比需要一份稳定的工作,而这份工作又在纽约,于是他接受了邀请,并于8月乘坐“罗斯福总统号”轮船返回美国。\(^\text{[22]}\)拉比也因此成为当时哥伦比亚大学唯一一位犹太裔教职员工。\(^\text{[23]}\)

拉比是一位糟糕的授课教师。莱昂·莱德曼回忆道,每次听完拉比的课后,学生们都会跑去图书馆试图搞清楚他到底在讲什么。欧文·卡普兰评价拉比和哈罗德·尤里是“我上过课的最差老师”。\(^\text{[24]}\)诺曼·拉姆齐认为拉比的讲座“相当糟糕”,\(^\text{[24]}\)而威廉·尼伦伯格则觉得他“简直是个可怕的讲师”。\(^\text{[25]}\)尽管在授课方面有所欠缺,他的影响力却非常深远。他激励了许多学生投身物理学,其中一些人后来声名显赫。\(^\text{[26]}\)

拉比的第一个女儿海伦·伊丽莎白于1929年9月出生。\(^\text{[27]}\)第二个女儿玛格丽特·乔艾拉则出生于1934年。\(^\text{[28]}\)在教学工作与家庭生活之间,他几乎没有时间进行科研,在哥伦比亚大学任教的第一年里没有发表任何论文,但即便如此,年终时他仍被提升为助理教授。\(^\text{[27]}\) 1937年,他晋升为正教授。\(^\text{[29]}\)

1931年,拉比重新开始粒子束实验的研究。他与格雷戈里·布赖特合作,提出了布赖特–拉比方程,并预测可以通过改进斯特恩–盖拉赫实验来验证原子核的性质。\(^\text{[30]}\)下一步便是将这一设想付诸实验。在维克托·W·科恩的协助下,\(^\text{[31]}\)拉比在哥伦比亚大学搭建了一套分子束实验装置。他们的思路是采用弱磁场而非强磁场,希望借此探测钠的核自旋。实验进行时,他们观测到四条细束,从而推断出钠的核自旋为 $\frac{3}{2}$。\(^\text{[32]}\)

拉比的分子束实验室逐渐吸引了其他研究者的加入,其中包括研究锂元素以获得博士学位的研究生西德尼·米尔曼。[33][34] 另一位是杰罗尔德·扎卡里亚斯,他认为钠核太难研究,于是提出研究最简单的元素——氢。氢的同位素氘仅在1931年由尤里在哥伦比亚大学发现,他也因此获得1934年诺贝尔化学奖。尤里为他们的实验提供了重水和氘气。尽管氢元素非常简单,汉堡的施特恩研究小组却观察到氢的行为并不符合理论预期。\(^\text{[35]}\)尤里还以另一种方式提供了帮助:他将自己的诺贝尔奖金的一半赠予拉比,用于资助分子束实验室的建设。\(^\text{[36]}\)在分子束实验室开启科研生涯的其他科学家还包括诺曼·拉姆齐、朱利安·施温格、杰罗姆·凯洛格和波利卡普·库施。\(^\text{[37]}\)他们全都是男性,因为拉比不相信女性能成为物理学家。他从未招收过女性博士或博士后学生,并且通常反对女性担任教职候选人。\(^\text{[38]}\)

在 C. J. 戈特尔的建议下,拉比的团队尝试使用交变磁场。\(^\text{[39]}\)这成为了核磁共振方法的基础。1937 年,拉比、库施、米尔曼和扎卡里亚斯使用分子束实验,测量了几种锂化合物的磁矩,包括氯化锂、氟化锂和双原子锂。\(^\text{[40]}\)在将该方法应用于氢时,他们发现质子的磁矩为 2.785±0.02 个核磁子,\(^\text{[41]}\)而当时理论预测应为 1,\(^\text{[42][43]}\)而氘核的磁矩则为 0.855±0.006 个核磁子。\(^\text{[41]}\)这为施特恩团队在 1934 年发现、拉比团队确认的结果提供了更为精确的测量数据。\(^\text{[44][45]}\)由于氘核由一个质子和一个自旋方向一致的中子构成,因此中子的磁矩可以通过用氘核磁矩减去质子磁矩的方式推算得出。结果表明中子的磁矩不为零,并且符号与质子相反。基于这些更为精确的测量中出现的一些奇特特征,拉比提出氘核具有电四极矩。\(^\text{[46][47][48]}\)这一发现意味着氘核的物理形状并非对称,从而为理解束缚核子之间的核力性质提供了重要线索。凭借创建分子束磁共振探测方法,拉比于 1944 年获得了诺贝尔物理学奖。\(^\text{[49]}\)
\subsection{第二次世界大战}
\begin{figure}[ht]
\centering
\includegraphics[width=6cm]{./figures/b978444bc8af013b.png}
\caption{原始空腔磁控管的阳极块,由伯明翰大学的约翰·兰德尔和哈里·布特研发,可见其谐振腔结构。} \label{fig_YXDlb_3}
\end{figure}
1940年9月,拉比成为美国陆军弹道研究实验室科学顾问委员会的成员。\(^\text{[50]}\)同月,英国的蒂泽德代表团将多项新技术带到了美国,其中包括一个腔体磁控管,这是一种高功率装置,通过电子流与磁场的相互作用产生微波。这项装置有望彻底改变雷达技术,因此,美国国家防御研究委员会的阿尔弗雷德·李·卢米斯决定在麻省理工学院(MIT)设立一个新的实验室来开发这项雷达技术。这个实验室被命名为“辐射实验室”,既是为了显得低调,又是向伯克利辐射实验室致敬。卢米斯招募了李·杜布里奇来负责该实验室的运行。\(^\text{[51]}\)

卢米斯和杜布里奇于1940年10月在麻省理工学院举办的一次应用核物理会议上招募新实验室的物理学家。志愿加入者中就包括拉比。他的任务是研究磁控管,这是一项高度机密的技术,必须保存在保险箱中。\(^\text{[52]}\)辐射实验室的科学家们将目标定为在1941年1月6日前制造出一台微波雷达装置,并在3月前将其原型安装到道格拉斯 A-20“浩劫”轰炸机上。这个目标最终得以实现;技术障碍被逐步克服,一套可运行的美制微波雷达设备被制成。磁控管在大西洋两岸得到了进一步改进,使波长从150厘米缩短到10厘米,再缩短到3厘米。实验室随后还开发了用于探测潜艇的空对地雷达、用于火控的 SCR-584 雷达,以及一种远程无线电导航系统——LORAN。\(^\text{[53]}\) 在拉比的推动下,辐射实验室在哥伦比亚大学设立了一个分支机构,由他负责。\(^\text{[54]}\)

1942年,罗伯特·奥本海默试图招募拉比和罗伯特·巴切尔前往洛斯阿拉莫斯实验室,参与一个新的绝密项目。他们说服奥本海默,指出其建立军事实验室的计划不可行,因为科学研究工作需要由平民主导。于是计划得以修改,新的实验室将由加州大学以战争部委托合同的方式作为一个平民机构来运行。最终,拉比仍未前往西部,但他同意担任曼哈顿计划的顾问。\(^\text{[55]}\)1945年7月,拉比出席了“三位一体”核试验。参与 Trinity 项目的科学家组织了一个对爆炸当量的竞猜活动,预测从完全哑弹到45千吨TNT当量不等。拉比到得较晚,只剩下一个18千吨的选项,他便买下了它。\(^\text{[56]}\)他戴上焊工护目镜,与诺曼·拉姆齐和恩里科·费米一同等待结果。\(^\text{[57]}\)爆炸最终被测定为18.6千吨,拉比赢得了竞猜。\(^\text{[56]}\)
\subsection{晚年生活}
1945年,拉比在美国物理教师协会为纪念弗洛伊德·K·里希特迈尔举办的里希特迈尔纪念讲座上提出,原子的磁共振可以作为钟表的基础。威廉·L·劳伦斯为《纽约时报》撰写了相关报道,标题为“计划中的‘宇宙摆钟’”\(^\text{[58][59][60]}\)。不久之后,扎卡里亚斯和拉姆齐就制造出了这样的原子钟。\(^\text{[61]}\)拉比积极从事磁共振的研究直到大约1960年,但他直到去世前仍会在各类会议和研讨会上亮相。\(^\text{[62][63]}\)
\begin{figure}[ht]
\centering
\includegraphics[width=8cm]{./figures/7ff2546cb5de3f40.png}
\caption{1962年,拉比与诺贝尔奖得主约翰·巴丁(左)和维尔纳·海森堡(右)合影。} \label{fig_YXDlb_5}
\end{figure}
1945年至1949年间,拉比担任哥伦比亚大学物理系主任。在此期间,该系聚集了两位诺贝尔奖得主(拉比和恩里科·费米)以及十一位未来的获奖者,其中包括七位教职员工(波利卡普·库施、威利斯·兰姆、玛丽亚·格佩特-梅耶、詹姆斯·雷恩沃特、诺曼·拉姆齐、查尔斯·汤斯和汤川秀树)、一位研究科学家(奥格·玻尔)、一位访问教授(汉斯·贝特)、一位博士生(利昂·莱德曼)和一位本科生(利昂·库珀)。\(^\text{[64]}\)拉比的博士生马丁·L·珀尔于1995年获得诺贝尔奖。\(^\text{[65]}\)拉比曾任哥伦比亚大学尤金·希金斯物理学教授,1964年哥伦比亚大学设立“大学教授”这一职位时,拉比成为首位获此殊荣者。这意味着他可以自由选择研究或教授的方向。\(^\text{[66]}\)他于1967年从教学岗位退休,但仍活跃于物理系,并保留“荣休大学教授”头衔直至去世。\(^\text{[67]}\)1985年,哥伦比亚大学为他设立了以其名字命名的特别教席。\(^\text{[68]}\)

曼哈顿计划的一项遗产是国家实验室体系的建立,但这些实验室最初都不设在美国东海岸。拉比与拉姆齐召集了纽约地区几所大学,组成联盟,游说设立一个属于东部的国家实验室。此事被时任麻省理工学院教授的扎卡里亚斯得知后,他也在麻省理工和哈佛组织起一个竞争团队。拉比随后与曼哈顿计划的总负责人莱斯利·R·格罗夫斯少将进行了会谈。格罗夫斯表示愿意支持新建国家实验室,但前提是只能建一个。虽然曼哈顿计划当时仍有资金,但这套战时体制预期将由新的机构接管并逐步解散。在拉比等人的一番游说与协商之后,这两个竞争团队于1946年1月合并。最终,包括哥伦比亚大学、康奈尔大学、哈佛大学、约翰斯·霍普金斯大学、麻省理工学院、普林斯顿大学、宾夕法尼亚大学、罗切斯特大学和耶鲁大学在内的九所高校组成联盟,并于1947年1月31日与新成立的原子能委员会(AEC,接替了曼哈顿计划)签署合同,正式建立布鲁克海文国家实验室。\(^\text{[69]}\)
\begin{figure}[ht]
\centering
\includegraphics[width=6cm]{./figures/7a033ad993b11ba2.png}
\caption{拉比(右下坐者)与其他诺贝尔奖得主合影(从左至右站立者:瓦尔·菲奇、詹姆斯·克罗宁、丁肇中;左下坐者:杨振宁)。} \label{fig_YXDlb_4}
\end{figure}
拉比曾向埃多阿多·阿马尔迪建议,布鲁克海文国家实验室的模式或许可以为欧洲所借鉴。在他看来,科学是激励并团结当时仍在战后恢复中的欧洲的一种方式。1950年,拉比被任命为美国驻联合国教科文组织代表,这为他提供了一个契机。在1950年6月于佛罗伦萨老宫举行的教科文组织会议上,拉比呼吁建立地区性实验室。这一倡议最终结出了硕果:1952年,来自11个国家的代表联合创立了“欧洲核子研究委员会”(Conseil Européen pour la Recherche Nucléaire,简称 CERN,即欧洲核子研究组织)。拉比随后收到了一封由玻尔、海森堡、阿马尔迪等人联名致信,祝贺他推动这一成就取得成功。他将这封信装裱起来,挂在了家中书房的墙上。\(^\text{[70]}\)
\subsubsection{军事事务}
1946年,《原子能法》设立了美国原子能委员会,并规定成立一个由九人组成的“总咨询委员会”(GAC),为委员会在科学与技术事务上提供建议。拉比于1946年12月被任命为该委员会成员之一。\(^\text{[71]}\)在20世纪40年代后期,GAC极具影响力。但在1950年,该委员会一致反对研发氢弹。拉比比其他多数成员立场更为激进,与费米一起不仅从技术角度,也从道德角度反对氢弹的开发。\(^\text{[72]}\)然而,美国总统哈里·S·杜鲁门无视GAC的建议,下令推进研发工作。\(^\text{[73]}\)拉比后来表示:

“我从未原谅杜鲁门屈服于压力。他根本就不理解这是怎么回事。事实上,在他卸任总统后,仍不相信苏联在1949年就有了原子弹。他就是这么说的。所以他在我们甚至还不知道怎么造氢弹的时候,就向全世界宣布我们要造氢弹,这是他能做的最糟糕的事之一。这也说明了这类事情有多么危险。”\(^\text{[74]}\)

1952年奥本海默任期届满未获连任,拉比接替其出任GAC主席,并任职至1956年。\(^\text{[75]}\)在1954年原子能委员会具有争议的安全听证会上,奥本海默被撤销安全许可,拉比为其出庭作证。虽然许多证人支持奥本海默,但拉比的言辞最为有力:

“所以在我看来,这根本不值得用这种方式来对待……一个像奥本海默博士这样有着巨大成就的人。他有一份实实在在的、积极的履历……我们有了原子弹,而且有了一整系列的原子弹;我们有了超级炸弹,也有了一整系列的超级炸弹,你们还想要什么?美人鱼吗?”\(^\text{[76][77]}\)

1952年,拉比被任命为国防动员办公室科学咨询委员会成员,并于1956年至1957年担任该委员会主席。\(^\text{[78]}\)这一时期正值“斯普特尼克危机”。1957年10月15日,美国总统德怀特·艾森豪威尔与SAC会面,寻求应对苏联人造卫星成功的对策。拉比是最先发言的人之一,他与艾森豪威尔在哥伦比亚大学时便相识。他提出了一系列建议,其中一项是加强该委员会的职能,使其能及时向总统提供建议。这一建议得以实施,数周后SAC升级为“总统科学咨询委员会”。拉比也随之成为艾森豪威尔的科学顾问。\(^\text{[79]}\)

1956年,拉比出席了“诺布斯卡计划”反潜战会议,会议内容涵盖从海洋学到核武器等多个领域。\(^\text{[80]}\) 此时,他还担任美国驻北约科学委员会代表,并正是在那段时间,“软件工程”这一术语首次被提出。在该职务任内,拉比对许多大型软件项目屡屡延期表示遗憾。这促成了一系列讨论,最终促成了第一个软件工程会议的召开。\(^\text{[81]}\)
\subsubsection{荣誉}
在他的一生中,拉比除了获得诺贝尔奖外,还荣获了许多其他奖项。其中包括1942年由富兰克林研究所颁发的埃利奥特·克雷森奖章,\(^\text{[82]}\)1948年获得的美国功绩勋章和英国颁发的“为自由事业服务国王奖章”,\(^\text{[29]}\)1956年被授予法国荣誉军团勋章军官勋位,\(^\text{[83]}\)1960年获得哥伦比亚大学颁发的巴纳德科学杰出贡献奖章,\(^\text{[84]}\)1967年荣获尼尔斯·玻尔国际金质奖章和“和平原子奖”,1982年获得美国物理教师协会颁发的厄斯特德奖章,1985年获得富兰克林与埃莉诺·罗斯福研究所颁发的“四大自由奖”以及美国国家科学院颁发的“公共福利奖章”,1986年荣获美国成就学院的金盘奖\(^\text{[85]}\)和美国国家科学基金会颁发的范尼瓦尔·布什奖。\(^\text{[83][86]}\)他是美国物理学会于1931年选出的会士,\(^\text{[87]}\)并在1950年担任该会会长。他同时也是美国国家科学院、美国哲学会以及美国艺术与科学院的成员,并被国际认可为日本学士院与巴西科学院的成员。1959年,他被任命为以色列魏茨曼科学研究院董事会成员。\(^\text{[29]}\)哥伦比亚大学设立了一项以拉比命名的本科学院最具价值的科研奖学金,用于激励和支持有潜力的年轻科学家。\(^\text{[88]}\)在欧洲核子研究中心法国普雷韦桑园区还有一条以他命名的街道“拉比路”。

哥伦比亚大学的“I. I. Rabi 学者项目”专门支持“一些被认为在科学方面最有前途的哥伦比亚学院新生”。\(^\text{[89]}\)
\subsubsection{逝世}
拉比于1988年1月11日因癌症在曼哈顿里弗赛德大道的家中去世。\(^\text{[68][62]}\)他的妻子海伦在他之后仍然在世,最终于2005年6月18日去世,享年102岁。\(^\text{[90]}\)在生命的最后时光,当医生用磁共振成像(MRI)为他进行检查时,他被提醒起自己最伟大的成就——这种技术正是建立在他开创性的磁共振研究基础之上的。那台机器的内部恰好是反光的,他感慨地说道:“我在那台机器里看到了自己……我从没想过我的研究最终会变成这样。”\(^\text{[91]}\)
\subsubsection{在流行文化中的出现}
在1980年播出的电视迷你剧《奥本海默》中,拉比由巴里·丹宁饰演;在2023年电影《奥本海默》中,则由大卫·克鲁默尔茨饰演。\(^\text{[92][93][94]}\)
\subsection{著作}
\begin{itemize}
\item 拉比,伊西多·艾萨克(1960年)。《我作为物理学家的生涯》。加利福尼亚州克莱蒙特:克莱蒙特学院。OCLC 1071412。
\item 拉比,伊西多·艾萨克;瑟伯,罗伯特;魏斯科普夫,维克多·F;派斯,亚伯拉罕;西博格,格伦·T(1969年)。《奥本海默:20世纪最非凡人物之一的故事》。斯克里布纳出版社。OCLC 223176672。
\item 拉比,伊西多·艾萨克(1970年)。《科学:文化的核心》。纽约:世界出版公司。OCLC 74630。
\end{itemize}
\subsection{注释}
\begin{enumerate}
\item “伊西多·拉比”。数学家族谱项目。
\item “伊西多·艾萨克·拉比的意第绪语拼写”。itranslate.com。
\item Rigden 1987,第17–21页。
\item Rigden 1987,第27页。
\item Ramsey 1993,第312页。
\item Rigden 1987,第23页。
\item Rigden 1987,第27–28页。
\item Rigden 1987,第33–34页。
\item Rigden 1987,第35–40页。
\item Rigden 1987,第41–45页。
\item Rabi 1927,第174–185页。
\item Rigden 1987,第50–53页。
\item Kronig 与 Rabi 1928,第262–269页。
\item 1634–1699:McCusker, J. J. (1997).《这在现实货币中值多少钱?:作为美国经济货币价值折算器的历史价格指数:补遗与更正》(PDF),美国古文物学会。
1700–1799:McCusker, J. J. (1992).《这在现实货币中值多少钱?:作为美国经济货币价值折算器的历史价格指数》(PDF),美国古文物学会。
1800年至今:明尼阿波利斯联邦储备银行。“消费者价格指数(估算)1800–”。检索日期:2024年2月29日。
\item Rigden 1987,第55–57页。
\item Rigden 1987,第57–59页。
\item Rigden 1987,第60–62页。
\item Toennies 等 2011,第1066页。
\item Rabi 1929,第163–164页。
\item Rabi 1929b,第190–197页。
\item Rigden 1987,第65–67页。
\item Rigden 1987,第66–69页。
\item Rigden 1987,第104页。
\item Rigden 1987,第71页。
\item Rigden 1987,第72页。
\item Rigden 1987,第71–72页。
\item Rigden 1987,第70页。
\item Rigden 1987,第83页。
\item “伊西多·艾萨克·拉比 – 生平简介”。诺贝尔传媒。检索时间:2012年8月17日。
\item Rigden 1987,第80页。
\item “讣告:维克托·威廉·科恩(Victor William Cohen)”。《今日物理》(Physics Today),第28卷第1期,第111–112页,1975年1月。Bibcode:1975PhT....28a.111.,doi:10.1063/1.3068792,ISSN 0031-9228。原始内容存档于2013年9月27日。
\item Rigden 1987,第84–88页。
\item Millman 1977,第87页。
\item Rigden 1987,第88–89页。
\item Goldstein 1992,第21–22页。
\item Rigden 1987,第90页。
\item Goldstein 1992,第23页。
\item Rigden 1987,第116页。
\item Goldstein 1992,第33–34页。
\item Rabi 等人 1939,第526–535页。
\item Kellogg 等人 1939,第728页。
\item Rigden 1987,第115页。
\item Breit 与 Rabi 1934,第230–231页。
\item Rabi、Kellogg 与 Zacharias 1934a,第157–163页。
\item Rabi、Kellogg 与 Zacharias 1934b,第163–165页。
\item Rigden 1987,第112–113页。
\item Rabi 等人 1938,第318页。
\item Rabi 等人 1992,第131–133页。
\item Goldstein 1992,第36页。
\item “BRL的科学顾问委员会,1940年。” 美国陆军研究实验室。原始内容存档于2016年12月1日。检索于2016年7月26日。
\item 康纳特,2002年,第209–213页。
\item 里格登,1987年,第131–134页。
\item 里格登,1987年,第135页。
\item 里格登,1987年,第143页。
\item 休利特与安德森,1962年,第230–232页。
\item 罗德斯,1986年,第656页。
\item 里格登,1987年,第155–156页。
\item 伊西多·I·拉比,《射频光谱学》(1945年1月20日在纽约哥伦比亚大学发表的里希特迈耶纪念讲座)。
\item “1945年1月19日至20日纽约会议”,《物理评论》,第67卷,第199–204页(1945年)。
\item 劳伦斯,威廉(1945年1月21日)。“计划中的‘宇宙钟摆’时钟” (PDF),《纽约时报》第34页。检索日期:2012年6月15日。
\item 里格登,1987年,第170–171页。
\item 拉姆齐,1993年,第319页。
\item 里格登,1987年,第15页。
\item “哥伦比亚诺贝尔奖得主”,哥伦比亚大学。原始内容存档于2012年10月29日。检索日期:2012年6月16日。
\item “马丁·L·珀尔——生平简介”,诺贝尔奖,诺贝尔媒体。检索日期:2016年3月19日。
\item 里格登,1987年,第68页。
\item “伊西多·艾萨克‘I. I.’拉比”,《当代美国物理学家档案》。原始内容存档于2012年10月17日。检索日期:2012年6月16日。
\item 伯杰,玛丽莲(1988年1月12日)。“伊西多·艾萨克·拉比,原子物理学的先驱,逝世,享年89岁”。《纽约时报》,第A1、A24页。
\item Rigden 1987,第182–185页。
\item Rigden 1987,第235–237页。
\item Hewlett & Anderson 1962,第648页。
\item Hewlett & Duncan 1969,第380–385页。
\item Hewlett & Duncan 1969,第403–408页。
\item Rigden 1987,第246页。
\item Hewlett & Duncan 1969,第665页。
\item Rigden 1987,第227页。
\item ellerstein, Alex(2015年1月16日),“奥本海默,未删节:第二部分”。《受限数据:核保密博客》。检索日期:2015年2月1日。
\item “白宫科学顾问”,《当代美国物理学家档案》。原始页面存档于2013年7月22日。检索日期:2012W年6月16日。
\item Rigden 1987,第248–251页。
\item Friedman 1994,第109–114页。
\item MacKenzie 2001,第34页。
\item “伊西多·艾萨克·拉比”,富兰克林研究所。2014年1月15日。检索日期:2016年3月19日。
\item Ramsey 1993,第320页。
\item “拉比荣获巴纳德奖章”,《今日物理》,第13卷第8期,第52页,1960年8月。Bibcode:1960PhT....13h..52..,doi:10.1063/1.3057088。ISSN 0031-9228。
\item “美国成就学院金盘奖获得者”,[www.achievement.org,美国成就学院。](http://www.achievement.org,美国成就学院。)
\item “公共福利奖”。美国国家科学院。检索日期:2011年2月18日。
\item “APS 院士档案”。美国物理学会。(搜索年份为1931年,机构为哥伦比亚大学)
\item “I.I. 拉比学者项目 | 本科研究与奖学金”。urf.columbia.edu。检索日期:2022年9月21日。
\item “I.I. 拉比学者项目 | 本科研究与奖学金”。urf.columbia.edu。检索日期:2023年7月24日。
\item “拉比,海伦·纽马克”。《纽约时报》。2005年6月21日。ISSN 0362-4331。检索日期:2016年1月23日。
\item Rigden 1987,第xxi–xxii页。
\item “巴里·丹嫩”。英国电影协会(BFI)。原始页面存档于2017年10月1日。检索日期:2023年8月31日。
\item 克拉克·科利斯(2023年7月21日)。“《奥本海默》演员阵容:克里斯托弗·诺兰真实剧情片中的人物是谁扮演的”。《娱乐周刊》。检索日期:2023年7月24日。
\item 莫莉·莫斯,刘易斯·奈特(2023年7月22日)。“《奥本海默》演员阵容:克里斯托弗·诺兰电影中的完整演员名单”。《广播时报》。检索日期:2023年7月24日。
\end{enumerate}
\subsection{参考文献}
\begin{itemize}
\item Breit, G.; Rabi, I.I.(1934年)。“关于当前核磁矩数值的解释”。《物理评论》,46(3): 230–231。Bibcode: 1934PhRv...46..230B。doi:10.1103/PhysRev.46.230。
\item 康南特,珍妮特(2002年)。《燕尾服公园:一位华尔街大亨与改变二战进程的科学宫殿的秘密》。纽约:西蒙与舒斯特出版社。ISBN 0-684-87287-0。OCLC 48966735。
\item 弗里德曼,诺曼(1994年)。《1945年以来的美国潜艇:设计史图解》。马里兰州安纳波利斯:美国海军研究所。ISBN 1-55750-260-9。OCLC 29477981。
\item 戈德斯坦,杰克·S.(1992年)。《另一种时间:杰罗尔德·R·扎卡赖亚斯的一生》。马萨诸塞州剑桥:麻省理工学院出版社。ISBN 0-262-07138-X。OCLC 24628294。
\item 休利特,理查德·G.;安德森,奥斯卡·E.(1962年)。《新世界,1939–1946》。大学公园:宾夕法尼亚州立大学出版社。ISBN 0-520-07186-7。OCLC 637004643。
\item 休利特,理查德·G.;邓肯,弗朗西斯(1969年)。《原子盾,1947–1952:美国原子能委员会历史》。大学公园:宾夕法尼亚州立大学出版社。ISBN 0-520-07187-5。OCLC 3717478。
翻译如下:
\item Kellogg, J.M.B.;Rabi, I.I.;Ramsey, N.F. Jr.;Zacharias, J.R.(1939年10月)。《质子与氘核的磁矩:在不同磁场中氘的射频谱》。《物理评论》,56(8):728–743。Bibcode: 1939PhRv...56..728K。doi:10.1103/PhysRev.56.728。
\item Kronig, R. de L.;Rabi, I.I.(1928年2月)。《波动力学中对称刚体的研究》。《物理评论》,29(2):262–269。Bibcode: 1927PhRv...29..262K。doi:10.1103/PhysRev.29.262。S2CID 4000903。
\item 麦肯齐,唐纳德(2001年)。《形式化证明:计算、风险与信任》。马萨诸塞州剑桥:麻省理工学院出版社。ISBN 0-262-13393-8。OCLC 45835532。
\item 米尔曼,S.(1977年)。《作为拉比早期分子束实验室学生的回忆》。纽约科学院汇刊,38:87–105。doi:10.1111/j.2164-0947.1977.tb02951.x。
\item Rabi, I.I.(1927年1月)。《晶体的主要磁化率研究》。《物理评论》,29(1):174–185。Bibcode: 1927PhRv...29..174R。doi:10.1103/PhysRev.29.174。
\item Rabi, I.I.(1929年2月2日)。“分子束的折射”。《自然》杂志,123(3092):163–164。Bibcode: 1929Natur.123..163R。doi:10.1038/123163b0。S2CID 4113129。
\item Rabi, I.I.(1929年3月)。“分子束偏转方法”(德文)《物理学杂志》,54(第3–4期):190–197。Bibcode: 1929ZPhy...54..190R。doi:10.1007/BF01339837。S2CID 123202872。
\item Rabi, I.I.; Kellogg, J.M.; Zacharias, J.R.(1934年a)。“质子的磁矩”。《物理评论》,46(3):157–163。Bibcode: 1934PhRv...46..157R。doi:10.1103/PhysRev.46.157。
\item Rabi, I.I.; Kellogg, J.M.; Zacharias, J.R.(1934年b)。“氘核的磁矩”。《物理评论》,46(3):163–165。Bibcode: 1934PhRv...46..163R。doi:10.1103/PhysRev.46.163。
\item Rabi, I.I.; Zacharias, J.R.; Millman, S.; Kusch, P.(1938年)。“一种测量核磁矩的新方法”。《物理评论》,53(4):318。Bibcode: 1938PhRv...53..318R。doi:10.1103/PhysRev.53.318。
\item Rabi, I.I.; Millman, S.; Kusch, P.; Zacharias, J.R.(1939)。“用于测量核磁矩的分子束共振方法:³Li⁶、³Li⁷ 与 ⁹F¹⁹ 的磁矩”。《物理评论》,55(6):526–535。Bibcode: 1939PhRv...55..526R。doi:10.1103/PhysRev.55.526。S2CID 27209454。
\item Rabi, I.I.; Zacharias, J.R.; Millman, S.; Kusch, P.(1992)。“磁共振的里程碑:‘一种测量核磁矩的新方法’,1938年”。《磁共振成像杂志》,2(2):131–133。doi:10.1002/jmri.1880020203。PMID 1562763。S2CID 73238886。
\item Ramsey, Norman(1993)。“I. I. Rabi 1898–1988”。《传记回忆录》,第62卷。华盛顿特区:美国国家科学院出版社。ISBN 0-585-14673-X。OCLC 45729831。
\item Rhodes, Richard(1986)。《原子弹的制造》。纽约:西蒙与舒斯特出版社。ISBN 0-671-44133-7。OCLC 13793436。
\item Rigden, John S.(1987)。《拉比:科学家与公民》。斯隆基金会系列。纽约:基础图书出版社。ISBN 0-465-06792-1。OCLC 14931559。
\item Toennies, J.P.;Schmidt-Böcking, H.;Friedrich, B.;Lower, J.C.A.(2011)。“奥托·斯特恩(1888–1969):实验原子物理的奠基人”。《物理年鉴》,523(12):1045–1070。arXiv:1109.4864。Bibcode: 2011AnP...523.1045T。doi:10.1002/andp.201100228。S2CID 119204397。
\item “对伊西多·艾萨克·拉比的采访”。《核时代的战争与和平》,1986年。检索日期:2016年3月22日。
\item 1980年2月11日,斯蒂芬·怀特对伊西多·艾萨克·拉比的采访,尼尔斯·玻尔图书馆与档案馆,美国物理学会。
\item Find a Grave 上的伊西多·艾萨克·拉比(可在维基数据中编辑)
\item 数学世系计划中的伊西多·艾萨克·拉比
\item 诺贝尔奖官方网站上的伊西多·艾萨克·拉比(可在维基数据中编辑)

\end{itemize}