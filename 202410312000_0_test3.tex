% test3
% license Usr
% type Test

\begin{itemize}
\item \verb`Nth`: number of threads
\item \verb`grid_type`: \verb`[s]` squre, \verb`[r]` rectangular
\item \verb`in_path`: input path (for params.matb and "data.matb")
\item \verb`out_path`: analysis output path

\item \verb`P12_start`, \verb`P12_end`, \verb`P12_step`: $P(r_1, r_2)$ (\autoref{eq_HeAnal_21} ) and partial wave probabilities \verb`P_pw`

\item \verb`Imat_start`, \verb`Imat_end`, \verb`Imat_step`: $I_\lambda(k_1, k_2) = \bra{F_{l_1k_1}}\braket{F_{l_2k_2}}{\Psi_\lambda(r_1,r_2)}$, for double ionization energy. spectrum; output Imat.matb (eq_HeAnal_13), also output \verb`Pk1k2` matrix and total ionization rate to \verb`Pk1k2.matb` (eq_HeAnal_8); in the output, total double ionization probability (eq_HeAnal_14) will be print to output and saved to \verb`Pk1k2.matb`
\item \verb`Imat_kmin`, \verb`Imat_kmax`, \verb`Imat_Nk`

\item \verb`jad_start`, \verb`jad_end`, \verb`jad_step`: joint angular spectrum (\autoref{eq_HeAnal_22} )
\item jad_kmin, jad_kmax, jad_Nk: jad momentum range

\item \verb`tdcs_start`, \verb`tdcs_end`, \verb`tdcs_step`: triple differential cross section (\autoref{eq_HeAnal_23} )
\item \verb`tdcs_th1` [deg], \verb`tdcs_k1`: electron with fixed angle and momentum
\item \verb`tdcs_k2min`, \verb`tdcs_k2max`, \verb`tdcs_Nk2`: integral range for the second electron

\item \verb`sis_start`, \verb`sis_end`, \verb`sis_step`: for single ionization momentum spectrum (project 1st electron to He+ bound states) (\autoref{eq_HeAnal_10} )
\item \verb`sis_kmin`, \verb`sis_kmax`, \verb`sis_Nk`: sis momentum range
\item \verb`Nn1`: number of n's for sis bound states, not used for TAE approximation
\item \verb`sis_r0`: cut from r2 < sis_r0 part from wave function for SIS
\item \verb`bound_r0`: use [0, bound_r0] to calculate bound states for TAE approximation, will be rounded up to integer FE, use 0 to turn off and use He+ bound states.
\item \verb`l_max`, \verb`Nbound`: for TAE approximation, l_max is maximum l of bound states, Nbound is number of lowest bound states to use. Note that bound state energy will depend on (n, l).

\item \verb`single_prob1_start`, \verb`single_prob1_end`, \verb`single_prob1_step`: probability distribution P(\vec r1) (\autoref{eq_HeAnal_25} )
\item \verb`Nth1`, \verb`ph1` [deg]

\item \verb`single_prob2_start`, \verb`single_prob2_end`, \verb`single_prob2_step` (\autoref{eq_HeAnal_25} )
\item \verb`Nth2`, \verb`ph2` [deg]

\item \verb`dipole_start`, \verb`dipole_end`, \verb`dipole_step` (dipole distribution)
\item \verb`dipole_th2` [deg], \verb`dipole_ph2` [deg]
\item \verb`dipole_k2min`, \verb`dipole_k2max`, \verb`dipole_Nk2`

\item \verb`dipole_r_start`, \verb`dipole_r_end`, \verb`dipole_r_step` (dipole distribution, fixed r2) (\autoref{eq_HeAnal_4} )
\item \verb`dipole_r_th2` [deg], \verb`dipole_r_ph2` [deg]
\item \verb`dipole_r_r2min`, \verb`dipole_r_r2max`, \verb`dipole_r_ir2_step`
\end{itemize}
