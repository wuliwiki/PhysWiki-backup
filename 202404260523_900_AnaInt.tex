% 牛顿—莱布尼兹公式(复变函数)
% keys 复变函数|矢量场|线积分|托克斯定理|柯西—黎曼|梯度定理|原函数
% license Xiao
% type Tutor

\begin{issues}
\issueDraft
\end{issues}

\pentry{柯西—黎曼条件\nref{nod_CauRie}, 复变函数的积分\nref{nod_CpxInt}}{nod_a58b}

我们来回顾\autoref{the_CpxInt_1}~\upref{CpxInt}: 复积分的实部和虚部可以分别表示为两个二维矢量场的线积分
\begin{equation}
\int_{C} f(z) \dd{z} = \int_C \bvec f(\bvec r) \vdot \dd{\bvec r} + \I \int_C \bvec g(\bvec r) \vdot \dd{\bvec r}~,
\end{equation}
其中
\begin{equation}
\bvec f(\bvec r) = u\uvec x - v\uvec y
\qquad
\bvec g(\bvec r) = v\uvec x + u\uvec y~.
\end{equation}
什么情况下, 该积分只和积分曲线的起点和终点有关呢? 根据斯托克斯定理\upref{Stokes}, 充分必要条件是
\addTODO{?}

如果其中的函数在某区域 $D$ 解析, 即处处满足柯西—黎曼条件, 那么


\begin{theorem}{}
令复平面的实轴为 $x$ 轴, 虚轴为 $y$ 轴, 用两个二元实函数 $u, v$ 来表示 $f(z)$
\begin{equation}
f (z) = u(x, y) + \mathrm iv(x, y)~.
\end{equation}
令两个矢量场为($\bvec r$ 为 $xy$ 平面上的位置矢量)

那么积分 $\int_{C} f(z) \mathrm{d} z$ 的实部和虚部分别可以看作两个矢量场 $\bvec f(\bvec r), \bvec g(\bvec r)$ 在曲线 $C$ 上的线积分
\begin{equation}
\begin{aligned}
\int_{C} f(z) \dd{z} &= \int_C \bvec f(\bvec r) \vdot \dd{\bvec r} + \I \int_C \bvec g(\bvec r) \vdot \dd{\bvec r}\\
&= \int_{C} u \mathrm{d} x-v \mathrm{d} y+\mathrm{i} \int_{C} v \mathrm{d} x+u \mathrm{d} y~.
\end{aligned}
\end{equation}
\end{theorem}

而柯西—黎曼条件\upref{CauRie}恰好规定了这两个矢量场的旋度为零, 所以如果 $f(z)$ 在考虑的区域上解析, 线积分结果只和起点和终点有关, 与路径无关(所有的路径必须在解析的区域内)。 于是可以得到类似于牛顿—莱布尼兹公式\upref{NLeib}, 令 $\bvec f_R, \bvec f_I$ 的势函数分别为 $F_R, F_I$, 即
\begin{equation}
\grad F_R = f_R~,
\qquad
\grad F_I = f_I~.
\end{equation}
再令
\begin{equation}
F(\bvec r) = F_R(\bvec r) + \I F_I(\bvec r)~,
\end{equation}
有
\begin{equation}\label{eq_AnaInt_2}
\int_{z_1}^{z_2} f(z) \dd{z} = F(z_2) - F(z_1)~.
\end{equation}
对于任意路径成立。 容易证明 $F(z)$ 就是 $f(z)$ 的原函数, 即
\begin{equation}
F'(z) = f(z)~.
\end{equation}
这相当于对\autoref{eq_AnaInt_2} 实部和虚部分别使用梯度定理\upref{Grad}。
