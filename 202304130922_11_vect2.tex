% 四维矢量
% 相对论
\subsection{概念}
设$K$与$K'$为两个坐标原点重合的惯性系。在开始计时后,$K'$相对$K$有沿着$x$轴的相对速度。从$0$时刻开始,光信号沿着x轴运动,设其在两个参考系的时空坐标分别为$(t,x,y,z)$与$(t',x',y',z')$,由光速不变,我们有:

$$ds^2=ct^2-x^2-y^2-z^2=ct'^2-x'^2-y'^2-z'^2$$

从直观上,这看起来是矢量$(t,x,y,z)$“长度”的平方不随惯性系的改变而改变。准确来说,是逆变矢量和协变矢量的对偶内积。形式化这个运算,该闵氏时空下的度规为$\eta_{\mu\nu}=diag(+1,-1,-1,-1) $ ,设光速为1,我们有
\begin{equation}
x^\mu x_\mu =\eta_{\mu\nu}x^\mu x^\nu=\eta_{\rho \sigma}x'^\rho x'^\sigma   
\end{equation}
这意味着改变惯性系相当于对原惯性系进行保距变换,即正交线性变换,我们把这个正交线性变换称之为洛伦兹变换。


