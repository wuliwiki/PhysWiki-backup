% 光速不变原理
% 只谈光速不变原理的由来
\pentry{公理系统\upref{axioms},麦克斯韦方程组\upref{MWEq}}

“光速在任意参考系下都不变”这一理念,和广为流传的误解不同,并不是实验中得来的.部分书籍会简单粗暴地告诉你,Michelson-Morley(迈克尔逊-莫雷)实验(下称\textbf{MM实验})是为了寻找以太存在的证据而进行的,实验结果表明在误差范围内光速在任何参考系都是一样的,因此提出了“光速不变原理”,也就是狭义相对论的两个公理之一.然而,这是对历史的错误描述.一些物理课本使用这样的描述是有其教育意义的,因为用这样的误解很容易引入狭义相对论,而不需要学生有扎实的电动力学基础.

本书秉承准确、翔实的原则,将当代物理史学界对“光速不变原理”的由来阐释如下.

\subsection{关于Michelson-Morley实验的误解}

\subsubsection{误解1:MM实验的目的是寻找以太吗?}

答案是否定的.

MM实验是在1887年进行的.科学家们曾经就“光到底是粒子还是波”争论不休,早年由于牛顿的绝对权威,几乎所有人都顺从他而坚持光的粒子说.在1800年至1850年之间,波动说才逐渐占据了上风.但是在当时的常识看来,波动是介质运动的结果,也就是说,有波动就必须有介质.光既然是波,那么它就一定有介质,物理学家们将其称为\textbf{以太(ether)}.

很明显,根据定义,以太会和光作用.但是以太会不会和其它物质作用呢?这个问题没有直接的答案.光学界的泰斗杨(Young)和菲涅尔(Fresnel)认为以太是不会和普通物质相互作用的;而斯托克斯(Stokes)认为以太应该和普通物质作用,从而会被普通物质拖拽而产生运动.

杨的观点来自于一个天文学现象:恒星的\textbf{像差(aberration)}\footnote{“Upon considering the phenomena
of the aberration of the stars I am disposed to believe, that the luminiferous ether pervades the substance of all material bodies with little or no resistance, as freely perhaps as the wind passes through a grove of trees.” Thomas Young于1804年所说.笔者翻译其原话为:“考虑恒星的像差现象,我自然而然地相信,传播光的以太在任何物质中弥散,只有很少甚至没有阻力,就好像风吹过树林一般.”}.而斯托克斯的观点是,由于光有偏振,意味着光是横波;即使是可见光也有极高的频率\footnote{比如说,589纳米波长的光在我们眼中是黄色的,其频率高达$\nu=c/\lambda=5\times10^{17}\opn{Hz}$.其中$c$为光速,$\lambda=5.89\times10^{-10}\opn{m}$为波长},因此以太必须坚如磐石,怎么可能“像风吹过树林一般”?所以以太必须和普通物质相互拖拽\footnote{笔者对此论述有疑义:如果以太和普通物质不相互作用,那它坚如磐石也和普通物质无关啊.那个时代的物理学家普遍有很多来自日常经验的先入为主思想,在今天看来逻辑不严谨其实并不奇怪.}.%秉承百科原则,可能需要解释恒星像差现象?

MM实验由此产生,其目的是检验以太究竟会不会和普通物质作用.实验的基本思想是,如果以太不和普通物质作用,那么以太可以看成是一种绝对静止的参考系,地球在公转、自转等运动中一定会相对以太而运动,否则就过于凑巧了.但是如果以太和普通物质作用,那么地球就会拖拽以太,在地面上以太应该近乎静止.如果我们在地面检测光在两个相互垂直的方向上的传播速度,那么杨的理论预言光速会有不同,而斯通克斯的理论则语言光速不会变化.

当然了,实验结果是光速在两个方向上的差别非常小,因此当时人们认为MM实验证明了以太和普通物质有相互作用.


