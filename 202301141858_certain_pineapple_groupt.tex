% 群乘法表及重拍定理
% 群乘法表|重排定理

\pentry{群\upref{Sample}}

群按照其元素个数可以大致分为两个类型,拥有有限个元素的离散群和拥有无限个元素的连续群,描述不同群的乘法规则是一个极为重要的事情,在有限群中通常使用群乘法表的概念,而连续群中则使用结合函数来描述群元之间的乘法关系。

在介绍群的乘法表之前先引入重排定理的概念,这个概念将对我们讲乘法表写出来起很大作用。

\subsection{重排定理}

\begin{theorem}{重排定理}
若$a$为群$G$中任意一元素,那么在相同的乘法规则下以下三个群与G是同一个群:

$aG=\{ag_1,~~ag_2,~~ag_3...\}$

$Ga=\{g_1a,~~g_2a,~~g_3a...\}$

$G^{-1}=\{g_1^{-1},~~g_2^{-1},~~g_3^{-1}...\}$
\end{theorem}

证明:
首先考虑第三个群$G^{-1}=\{g_1^{-1},~~g_2^{-1},~~g_3^{-1}...\}$,由于所有元素的逆元是唯一的,所以不同元素的逆元互不相同,切$g^{-1}\in G$,则群$G^{-1}$与群$G$相同,给出的是群$G$的一个重新排列。

考虑$aG=\{ag_1,~~ag_2,~~ag_3...\}$,任意群元$g$都可以写成$g=a(a^{-1}g)$的形式,而$a^{-1}g$是群$G$中的元素,则$aG=\{ag_1,~~ag_2,~~ag_3...\}$与群$G$相同。

证毕

\subsection{群乘法表}

描述有限群的乘法关系的表叫做群乘法表,考虑较为简单的循环群$C3$,其有三个群元,分别为$e,d,f$。其中$e$为单位元。有其群乘法表为:

\begin{table}[ht]
\centering
\caption{$C3$群乘法表}\label{groupt_tab1}
\begin{tabular}{|c|c|c|c|}
\hline
$C3$ & $~e~$ & $~d~$ & $~f~$ \\
\hline
$e$ & $e$ & $d$ & $f$ \\
\hline
$d$ & $d$ & $f$ & $e$ \\
\hline
$f$ & $f$ & $e$ & $d$ \\
\hline
\end{tabular}
\end{table}

表中第三行的第一个元素为$d$,第四列的第一个元素为$f$,第三行第四列的元素为$e$,则表示有一下乘法关系:$df=e$。

由重排定理可以看到,不同的元素乘相同的元素后一定得到不同的元素,则可知群乘法表的每行每列除了第一个元素外均为不重复的元素,不同行列之间的差距仅仅为排列顺序的差别。这一点使得在填写群乘法表时会有类似填数独一样的技巧性。但这带出另一个问题,对于群元数目一定的群的群乘法表不唯一,这也给了我们一个通过群乘法表对有限群进行分类的方法,三阶群只有一个,也就是我们所给出的$C3$群,但四阶群存在两种:$C4$群和$V4$群,他们的群乘法表分别如下:

\begin{table}[ht]
\centering
\caption{$C4$群乘法表}\label{groupt_tab2}
\begin{tabular}{|c|c|c|c|c|}
\hline
$C4$ & $~e~$ & $~d~$ & $~f~$ & $~g~$ \\
\hline
$e$ & $e$ & $d$ & $f$ & $g$ \\
\hline
$d$ & $d$ & $f$ & $g$ & $e$ \\
\hline
$f$ & $f$ & $g$ & $e$ & $d$ \\
\hline
$g$ & $g$ & $e$ & $d$ & $f$ \\
\hline
\end{tabular}
\end{table}

\begin{table}[ht]
\centering
\caption{$V4$群乘法表}\label{groupt_tab3}
\begin{tabular}{|c|c|c|c|c|}
\hline
$V4$ & $~e~$ & $~a~$ & $~b~$ & $~c~$ \\
\hline
$e$ & $e$ & $a$ & $b$ & $c$ \\
\hline
$a$ & $a$ & $e$ & $c$ & $b$ \\
\hline
$b$ & $b$ & $c$ & $e$ & $a$ \\
\hline
$c$ & $c$ & $b$ & $a$ & $e$ \\
\hline
\end{tabular}
\end{table}

注:正常的群乘法表并不一定有关于主对角线的对称性,前几个例子之所以存在对称性是由于所选取的群是阿贝尔群,$g_1g_2=g_2g_1$。

