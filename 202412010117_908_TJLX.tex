% 统计力学(综述)
% license CCBYSA3
% type Wiki

本文根据 CC-BY-SA 协议转载翻译自维基百科\href{https://en.wikipedia.org/wiki/Statistical_mechanics}{相关文章}。

在物理学中,\textbf{统计力学}是一种数学框架,将统计方法和概率理论应用于大量微观实体的集合。有时也称为\textbf{统计物理}或\textbf{统计热力学},其应用包括物理学、生物学、化学、神经科学、计算机科学、信息理论和社会学等多个领域。其主要目的是通过研究原子运动所遵循的物理规律,阐明物质在宏观集合状态下的性质。

统计力学源于经典热力学的发展,成功地解释了宏观物理属性(如温度、压力和热容量),将其与微观参数联系起来。这些微观参数围绕平均值波动,并以概率分布为特征。

虽然经典热力学主要关注\textbf{热力学平衡},但统计力学在\textbf{非平衡统计力学}中得到了广泛应用,用于微观建模不可逆过程的速度,这些过程由不平衡驱动。例如,化学反应以及粒子和热的流动。\textbf{涨落-耗散定理}是将非平衡统计力学应用于研究多粒子系统中最简单的非平衡状态(即稳态电流流动)时获得的基本理论。
\subsection{历史}
1738年,瑞士物理学家兼数学家丹尼尔·伯努利发表了《流体动力学》(*Hydrodynamica*),奠定了气体动理论的基础。在这项工作中,伯努利提出了一个至今仍在使用的观点:气体由大量分子组成,这些分子向各个方向运动,它们对表面的撞击导致了我们感受到的气体压力,而我们感受到的热量只是它们运动的动能。[9]

统计力学领域的创立通常归功于以下三位物理学家:
\begin{enumerate}
\item 路德维希·玻尔兹曼,他发展了关于熵的微观状态集合的基本解释;
\item 詹姆斯·克拉克·麦克斯韦,他建立了微观状态概率分布的模型;
\item 乔赛亚·威拉德·吉布斯,他在1884年首次命名了这一领域。
\end{enumerate}

1859年,在阅读鲁道夫·克劳修斯关于分子扩散的论文后,苏格兰物理学家詹姆斯·克拉克·麦克斯韦提出了麦克斯韦分子速率分布,描述了具有特定速度范围的分子比例。这是物理学中的首个统计规律。[10][11] 麦克斯韦还首次从力学角度论证了分子碰撞会导致温度的均匀化,从而趋于平衡。[12] 五年后,即1864年,年轻的维也纳学生**路德维希·玻尔兹曼**阅读了麦克斯韦的论文,并在其后大部分生涯中进一步发展了这一领域。

统计力学在19世纪70年代由玻尔兹曼的研究正式开创,他的大部分研究成果于1896年的《气体理论讲义》(Lectures on Gas Theory)中发表。[13] 玻尔兹曼关于热力学统计解释的原创论文,包括H定理、输运理论、热平衡、气体状态方程等主题,分布在维也纳科学院和其他学会的约2,000页的论文中。他引入了平衡统计系综的概念,并首次研究了非平衡统计力学,并提出了H定理。
\begin{figure}[ht]
\centering
\includegraphics[width=6cm]{./figures/3904934c348528af.png}
\caption{吉布斯统计力学著作的封面} \label{fig_TJLX_1}
\end{figure}
统计力学”这一术语由美国数学物理学家J. 威拉德·吉布斯于1884年创造。[14] 据吉布斯所述,“统计”一词在力学(即统计力学)中的使用最早可以追溯到苏格兰物理学家**詹姆斯·克拉克·麦克斯韦于1871年的表述:

“在处理大量物质时,由于我们无法察觉单个分子,不得不采用我所描述的统计计算方法,并放弃严格的动力学方法,即通过微积分追踪每个运动。”

—— J. 克拉克·麦克斯韦[15]  

今天,“概率力学”可能会被认为是一个更恰当的术语,但“统计力学”已经牢牢确立下来。[16] 

在去世前不久,吉布斯于1902年出版了《统计力学的基本原理》(Elementary Principles in Statistical Mechanics),这本书将统计力学形式化为一种通用方法,能够处理所有力学系统——无论是宏观还是微观,是气态还是非气态系统。[17] 吉布斯的方法最初是在经典力学框架下推导出来的,但由于其高度的普适性,这些方法后来被发现可以轻松适配于量子力学,并且至今仍是统计力学的基础。[18]
\subsection{原理:力学与系综} 
在物理学中,通常研究两种类型的力学:经典力学和量子力学。对于这两种力学,标准的数学方法包括以下两个概念:
\begin{enumerate}
\item 机械系统在某一特定时刻的完整状态,在数学上表示为相点(经典力学)或纯量子态矢量(量子力学)。  
\item 将状态随时间推演的运动方程:哈密顿方程(经典力学)或薛定谔方程(量子力学)。
\end{enumerate}
通过这两个概念,可以原则上计算出系统在任何其他时间(过去或未来)的状态。然而,这些定律与日常生活经验之间存在脱节,因为我们在宏观尺度上进行操作时(例如进行化学反应),既不需要(甚至理论上也不可能)精确地知道每个分子在微观水平上的同时位置和速度。统计力学通过引入系统状态的不确定性,弥合了力学定律与实际经验之间的这种脱节。

普通力学只研究单一状态的行为,而统计力学引入了统计系综的概念,即系统在各种状态下的大量虚拟、独立副本的集合。统计系综是系统所有可能状态的概率分布。在经典统计力学中,系综是相点的概率分布(与普通力学中的单一相点相对),通常表示为具有规范坐标轴的相空间中的分布。在量子统计力学中,系综是纯态的概率分布,可以用密度矩阵简洁地表示。

与概率的常规处理方式类似,统计系综可以通过以下不同方式进行解释:[17]  
\begin{enumerate}
\item 系综可以被理解为单一系统可能处于的各种状态(认知概率,一种知识形式)。  
\item 系综的成员也可以被看作是独立系统在重复实验中经历的状态,这些系统是在相似但控制不完全的条件下准备的(经验概率),在无限次试验的极限下成立。 
\end{enumerate} 
对于许多目的,这两种解释是等价的,在本文中会交替使用。

无论如何解释概率,系综中的每个状态都会依据运动方程随时间演化。因此,系综本身(即状态上的概率分布)也会演化,因为系综中的虚拟系统会不断地从一个状态转移到另一个状态。系综的演化由刘维尔方程(经典力学)或冯诺依曼方程(量子力学)描述。这些方程通过对系综中每个虚拟系统单独应用力学运动方程得出,并假设虚拟系统的概率在从一个状态演化到另一个状态的过程中保持守恒。

一种特殊的系综类型是那些随时间不演化的系综,称为平衡系综,其条件称为统计平衡。统计平衡发生的条件是,对于系综中的每个状态,系综还包含其未来和过去的所有状态,并且这些状态的概率等于该状态的概率。(相比之下,机械平衡是指力的平衡状态,其已停止演化。)对孤立系统的平衡系综的研究是统计热力学的核心。非平衡统计力学则研究更一般的情形,包括随时间变化的系综和非孤立系统的系综。
\subsection{统计热力学} 
统计热力学(也称为**平衡统计力学**)的主要目标是根据材料组成粒子的属性及其相互作用,从微观角度推导材料的经典热力学性质。换句话说,统计热力学建立了材料在热力学平衡条件下的宏观性质与其内部微观行为和运动之间的联系。

虽然统计力学本身涉及动力学,但统计热力学主要关注统计平衡(稳定状态)。统计平衡并不意味着粒子停止运动(即机械平衡),而是指系综的整体状态不再随时间演化。
\subsubsection{基本假设} 
统计平衡的一个充分(但非必要)条件是:对于一个孤立系统,其概率分布仅依赖于守恒量(如总能量、总粒子数等)的函数。[17] 可以考虑许多不同的平衡系综,但只有其中的一部分与热力学对应。[17] 因此,需要额外的假设来说明为什么给定系统的系综应该采取某种特定形式。

一种常见的方法是采用等概率假设,这在许多教科书中都有提到。[18] 该假设指出:

对于一个具有确知能量和确知组成的孤立系统,系统可以以相等的概率处于任何与这些信息一致的微观状态。

因此,等概率假设为下面描述的微正则系综提供了动机。支持等概率假设的理由包括:
\begin{itemize}
\item 遍历假设:一个遍历系统是指能够随时间演化以探索“所有可达”状态的系统,这些状态具有相同的能量和组成。在遍历系统中,微正则系综是唯一可能的具有固定能量的平衡系综。然而,该方法的适用性有限,因为大多数系统并非遍历的。
\item 无差别原理:在缺乏任何进一步信息的情况下,我们只能对每种可能的情况分配相等的概率。
\item 最大信息熵:无差别原理的一个更复杂版本指出,正确的系综是与已知信息兼容且具有最大吉布斯熵(信息熵)的系综。[19] 
\end{itemize}
统计力学中还提出了其他的基本假设。[9][20][21] 例如,最近的研究表明,统计力学理论可以在没有等概率假设的前提下建立。[20][21] 一种这样的形式化框架基于基本热力学关系以及以下假设:[20]  
\begin{enumerate}
\item 概率密度函数与某种关于系综参数和随机变量的函数成正比。  
\item 热力学状态函数由随机变量的系综平均值描述。  
\item 根据吉布斯熵公式定义的熵,与经典热力学中定义的熵一致。 
\end{enumerate} 
其中,第三条假设可以被以下假设替代:[21]  

3. 在无限高温下,所有的微观状态具有相同的概率。  
\subsubsection{三种热力学系综}
主要文章:系综(数学物理学)、微正则系综、正则系综、大正则系综  

对于任何有限体积内的孤立系统,可以定义三种形式简单的平衡系综。[17] 这些是统计热力学中最常讨论的系综。在宏观极限(定义见下文)下,它们都与经典热力学相对应。  

\textbf{微正则系综(Microcanonical ensemble)}

描述具有精确定义的能量和固定组成(固定粒子数)的系统。微正则系综以相同的概率包含所有与该能量和组成一致的可能状态。  

\textbf{正则系综(Canonical ensemble})

描述与一个具有精确温度的热浴处于热平衡状态的固定组成系统。正则系综包含能量变化的状态,但组成相同;系综中的不同状态根据其总能量赋予不同的概率。  

\textbf{大正则系综(Grand canonical ensemble)} 

描述一个组成不固定(粒子数不确定)并与一个热力学储存器处于热平衡和化学平衡的系统。储存器具有精确的温度和各种类型粒子的精确化学势。大正则系综包含能量和粒子数均变化的状态;系综中的不同状态根据其总能量和总粒子数赋予不同的概率。  

对于包含大量粒子的系统(热力学极限),上述三种系综表现出一致的行为。在这种情况下,仅根据数学便利性选择使用哪种系综。[22] 吉布斯定理关于系综等价性的研究[23],后来发展为测度集中现象理论,[24] 它在科学的许多领域都有应用,从泛函分析到人工智能和大数据技术的方法。[25]  

热力学系综结果不一致的重要情况包括:
\begin{itemize}
\item 微观系统。  
\item 处于相变状态的大系统。  
\item 存在长程相互作用的大系统。
\end{itemize}  
在这些情况下,必须选择正确的热力学系综,因为这些系综之间的差异不仅体现在涨落的大小上,还体现在平均量(例如粒子分布)上。正确的系综是与系统的准备方式和特性相对应的系综,换句话说,是反映该系统已知信息的系综。[18]
\begin{figure}[ht]
\centering
\includegraphics[width=14.25cm]{./figures/56ac94278567d7d2.png}
\caption{} \label{fig_TJLX_2}
\end{figure}
\subsubsection{计算方法}
一旦计算出某个系统的系综特征状态函数,该系统便被“求解”了(可以从特征状态函数中提取宏观可观测量)。然而,计算热力学系综的特征状态函数并非总是简单的任务,因为这需要考虑系统的每一种可能状态。尽管一些假想系统可以被精确求解,但最一般(也是最现实)的情况过于复杂,无法得到精确解。为此,存在多种方法用来逼近真实系综并计算平均量。

\textbf{精确解}

某些情况下可以得到精确解:
\begin{itemize}
\item 对于非常小的微观系统,可以通过枚举系统的所有可能状态直接计算系综(在量子力学中使用精确对角化,或在经典力学中对整个相空间进行积分)。
\item 一些大型系统由许多可分离的微观子系统组成,可以独立分析每个子系统。特别是,理想化的非相互作用粒子气体具有这一特性,从而可以精确推导出麦克斯韦–玻尔兹曼统计、费米–狄拉克统计和玻色–爱因斯坦统计。[18]
\item 少数具有相互作用的大型系统已被求解。通过使用巧妙的数学方法,已经为少数“玩具模型”找到精确解。[26] 一些示例包括贝特假设(Bethe Ansatz)、零场下的方格点阵伊辛模型(square-lattice Ising model in zero field)、硬六边形模型(hard hexagon model)。
\end{itemize}

\textbf{蒙特卡罗方法}

主条目:统计力学中的蒙特卡罗方法

尽管统计物理中的某些问题可以通过近似和展开公式解析求解,但当前的大多数研究依赖现代计算机的强大处理能力来模拟或近似求解。蒙特卡罗模拟是一种常见的处理统计问题的方法,用于揭示复杂系统的性质。蒙特卡罗方法在计算物理、物理化学及相关领域中非常重要,且有广泛的应用,例如在医学物理中被用于模拟辐射传输以进行辐射剂量学计算。[27][28][29]

蒙特卡罗方法仅检查系统的一部分可能状态,这些状态是随机选择的(并按合理权重选取)。只要这些状态能够代表系统所有可能状态的一个典型样本,就可以获得系统的特征函数的近似值。随着随机样本数量的增加,误差可以降低到任意小的水平。

一些经典的蒙特卡罗方法包括:
\begin{itemize}
\item Metropolis–Hastings算法:一种经典的蒙特卡罗方法,最初用于采样正则系综。
\item 路径积分蒙特卡罗方法:也用于采样正则系综。
\end{itemize}
\textbf{其他方法}
\begin{itemize}
\item 对于稀薄的非理想气体,使用例如簇展开的方法,通过微扰理论考虑弱相互作用的影响,从而得到维里展开式。[30]  
\item 对于高密度流体,另一种近似方法基于约化分布函数,特别是径向分布函数。[30]  
\item 分子动力学计算机模拟可以用于计算遍历系统中的微正则系综平均值。通过引入与随机热浴的连接,还可以模拟正则系综和大正则系综条件下的行为。  
\item 混合方法:结合非平衡统计力学的结果(见下文)可能会有所帮助。
\end{itemize}
\subsection{非平衡统计力学}
许多物理现象涉及准热力学的非平衡过程,例如:  
\begin{itemize}
\item 热传导:由材料内部运动引起,由温度不平衡驱动;  
\item 电流:由导体中电荷的运动产生,由电压不平衡驱动;  
\item 自发化学反应:由自由能减少驱动;  
\item 摩擦、耗散和量子退相干;  
\item 外力驱动的系统:如光泵浦等;  
\item 以及一般的不可逆过程。  
\end{itemize}
所有这些过程都以特定的速率随时间发生,这些速率在工程中非常重要。非平衡统计力学领域致力于从微观层面理解这些非平衡过程。(统计热力学只能用于计算最终结果,即当外部不平衡被消除并且系统回归平衡后达到的状态。)

原则上,非平衡统计力学可以是数学上精确的:孤立系统的系综随时间演化,遵循确定性方程,如刘维尔方程或其量子等效形式——冯·诺依曼方程。这些方程是通过将力学运动方程独立应用于系综中的每个状态得出的。这些系综演化方程继承了基础力学运动的复杂性,因此精确解非常难以获得。此外,这些方程是完全可逆的,不会破坏信息(系综的吉布斯熵保持不变)。为了推进不可逆过程的建模,有必要考虑概率和可逆力学之外的附加因素。

因此,非平衡力学是一个活跃的理论研究领域,这些附加假设的适用范围正在不断探索。以下小节将描述一些研究方法。
\subsubsection{随机方法}
非平衡统计力学的一种方法是将**随机行为**(随机性)引入系统中。随机行为破坏了系综中包含的信息。虽然从技术上讲,这种方法并不完全准确(除非是涉及黑洞的假设情况,否则系统本身不会导致信息丢失),但引入随机性是为了反映以下事实:有趣的信息随着时间的推移会被转化为系统内部的微妙关联,或者系统与环境之间的关联。这些关联表现为对关注变量的混乱或伪随机影响。通过用真正的随机性替代这些关联,可以大大简化计算。

\begin{itemize}
\item 玻尔兹曼输运方程:早期的一种随机力学形式出现在“统计力学”这一术语被创造之前,即对动力学理论的研究中。詹姆斯·克拉克·麦克斯韦已证明分子碰撞会导致气体内部的显著混乱运动。随后,路德维希·玻尔兹曼证明,如果将这种分子混乱视为完全随机化,则气体中粒子的运动将遵循一个简单的玻尔兹曼输运方程,并快速使气体恢复到平衡状态(见H定理)。

  玻尔兹曼输运方程及相关方法在非平衡统计力学中因其极度简单性而成为重要工具。这些近似方法在“有趣”的信息在一次碰撞后即刻被转化为微妙关联的系统中效果很好,这基本上将其限制在稀薄气体中。玻尔兹曼输运方程已被证明在轻掺杂半导体(如晶体管中电子输运的模拟)中非常有用,因为这些电子的行为与稀薄气体类似。

 与此主题相关的量子技术是随机相位近似。

\item BBGKY层级:在液体和致密气体中,在一次碰撞后立即丢弃粒子间的关联是不合理的。BBGKY层级(Bogoliubov–Born–Green–Kirkwood–Yvon层级)提供了一种方法,可以推导出玻尔兹曼类型的方程,并将其扩展到非稀薄气体的情况,以包含几次碰撞后的关联。
\item 凯尔迪什形式主义(即NEGF——非平衡格林函数):一种量子方法,能够将随机动力学纳入计算中。这种方法通常用于电子量子输运的计算中。
\item 随机刘维尔方程:这是一种进一步将随机性纳入非平衡统计描述的形式。
\end{itemize}
\subsubsection{近平衡方法}
非平衡统计力学模型中的另一个重要类别处理的是仅稍微偏离平衡状态的系统。在只有很小扰动的情况下,可以使用线性响应理论对系统的响应进行分析。一个显著的结果是涨落-耗散定理,它表明,系统在接近平衡状态时的响应,与系统在完全平衡时的涨落密切相关。本质上,一个稍微偏离平衡状态的系统(无论是由于外力作用还是由于自身的涨落)都会以相同的方式恢复平衡,因为系统无法区分或“感知”自己是如何偏离平衡的。[^30]:664

这为通过平衡统计力学间接获取某些参数(如欧姆电导率和热导率)提供了一条途径。由于平衡统计力学在数学上定义明确且(在某些情况下)更易于计算,涨落-耗散之间的关系成为了近平衡统计力学计算中的一种便捷捷径。

以下是用于建立这种联系的一些理论工具:
\begin{itemize}
\item 涨落-耗散定理(Fluctuation–dissipation theorem)
\item 昂萨格互易关系(Onsager reciprocal relations)
\item 格林–久保关系(Green–Kubo relations)
\item 兰道尔–比特克形式主义(Landauer–Büttiker formalism)
\item 莫里–茨万齐格形式主义(Mori–Zwanzig formalism)
\item 通用非平衡反演理论(GENERIC formalism)
\end{itemize}
\subsubsection{混合方法}
一种高级方法结合了随机方法和线性响应理论。例如,用于计算电子系统导电性中的量子相干效应(如弱局域化、导电波动)的一种方法是利用格林–久保关系(Green–Kubo relations),并通过凯尔迪什方法(Keldysh method)引入由电子间相互作用引起的随机去相干效应来实现。[31][32]
\subsection{应用}
集合形式主义可以用于分析在对系统状态了解存在不确定性的情况下的一般力学系统。集合方法还应用于以下领域:
\begin{itemize}
\item 不确定性的时间传播,[17]
\item 引力轨道的回归分析,
\item 天气的集合预测,
\item 神经网络的动力学,
\item 博弈论和经济学中的有限理性势博弈。
\end{itemize}
统计物理学可以解释并定量描述以下现象:超导、超流、湍流、固体与等离子体中的集体现象,以及液体的结构特性。它也是现代天体物理学的基础。在固体物理学中,统计物理学有助于研究液晶、相变和临界现象。许多关于物质的实验研究完全基于系统的统计描述,例如冷中子、X射线和可见光的散射实验等。统计物理学还在材料科学、核物理、天体物理学、化学、生物学和医学(如研究传染病的传播)中发挥作用。[需要引用]

从无序系统统计物理学中派生的分析和计算技术可以扩展到大规模问题,例如机器学习中分析深度神经网络的权重空间。[33] 因此,统计物理学也在医学诊断领域找到应用。[34]
\subsubsection{量子统计力学}
量子统计力学是将统计力学应用于量子力学系统。在量子力学中,统计集合(可能量子态上的概率分布)由密度算符 \( S \) 描述。该算符是定义在描述量子系统的希尔伯特空间 \( H \) 上的非负、自伴、迹类算符,其迹为 1。这一点可以通过量子力学的各种数学形式化方法证明,其中一种形式化方法由量子逻辑提供。
\subsection{参见} 
\begin{itemize}
\item 量子统计力学  
\item 热力学与统计力学教材列表  
\item 拉普拉斯变换  
\end{itemize}
