% 共轭空间中的强拓扑
% keys 强拓扑|完备
% license Usr
% type Tutor
\pentry{共轭空间与代数共轭空间\nref{nod_ConSpa},线性算子的范数\nref{nod_ONorm}}{nod_0a2f}

\subsection{赋范空间的强拓扑}

由\autoref{ex_tvs_1} 可知,赋范空间是一个拓扑线性空间,因此其上自然由\enref{线性连续泛函}{LinCon}的定义。而赋范空间上根据
\begin{equation}\label{eq_STop_1}
\norm{f}:=\sup_{x\neq0}\frac{\abs{f(x)}}{\norm{x}}~
\end{equation}
可引入线性连续泛函的范数,证明\autoref{eq_STop_1} 满足范数的定义和\autoref{ex_ONorm_1} 的证明完全一样。

