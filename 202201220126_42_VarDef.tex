% 变分的基本定理(变分学)
% keys 变分|基本定理

\begin{issues}
\issueTODO
\end{issues}

\pentry{变分的变换\upref{VarCha}}
本节给出的两个定理,将有助于引出欧拉方程.其中一个定理属于拉格朗日,另一个属于黎曼.由拉格朗日推演的欧拉方程是不精确的,这可从拉格朗日变换\upref{VarCha}的定义中看到.而由黎曼推演可精确的推出欧拉方程,并且得到在极值曲线的正规点上,$y''$存在,而这是事先并未假定的.本节将证明推演欧拉方程用到的这两个定理.

\begin{theorem}{(拉格朗日)}\label{VarDef_the1}
若对任意的属于 $C_1$ 类(\autoref{Varia_sub1}~\upref{Varia})的函数 $\eta(x)$,并且其满足 $\eta(a)=\eta(b)=0$,属于 $C_1$ 类的函数 $M(x)$ 都有
\begin{equation}
\int_a^b M(x)\eta(x)\dd x=0
\end{equation}
则对于一切的 $a\leq x\leq b,M(x)=0$.
\end{theorem}
\begin{theorem}{(黎曼)}\label{VarDef_the2}
若对任意的属于 $C_1$ 类的函数 $\eta(x)$,并且其满足 $\eta(a)=\eta(b)=0$,属于 $C_1$ 类的函数 $M(x)$ 都有
\begin{equation}
\int_a^b M(x)\eta'(x)\dd x=0
\end{equation}
则对于一切的 $a\leq x\leq b$,$M(x)$ 为常数.
\end{theorem}
\subsection{证明}
以下将用反证法证明.
\subsubsection{\autoref{VarDef_the1} 的证明}
设在区间 $[a,b]$  上某点 $c$,$M(c)\neq0$.例如 $M(c)>0$ ,由于 $M(x)$ 的连续性,取充分大的 $n$,可以得到包含在 $[a,b]$ 内的区间 $[x_0,x_0+\frac{\pi}{n}]$ ,它包含着 $c$ 点,并且在它上面 $M(x)$ 大于某一正数 $m$.
定义
\begin{equation}
\eta_0(x)=\begin{vmatrix}
sin^2[n(x-x_0)]\\
0
\end{vmatrix}
\end{equation}

\subsubsection{\autoref{VarDef_the2} 的证明}