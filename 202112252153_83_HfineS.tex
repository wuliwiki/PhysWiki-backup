% 氢原子的精细能级结构

\begin{issues}
\issueDraft
\end{issues}
讨论氢原子时,我们将哈密顿量取为:\begin{equation}
H=-\frac{h^2}{2m}\laplacian-\frac{e^2}{4\pi\epsilon_0 r}
\end{equation}

但是电子的动能加库仑势能之和并不是完整的内容.我们讨论过了对原子核运动的修正,也就是把$m$替换成约化质量.在我们研究氢原子时,还有个极为重要的现象,那就是由相对论效应修正和自旋-轨道耦合所带来的精细结构.比起数量级为$\alpha^2 mc^2$的波尔能量,精细结构是一个非常微小的扰动,其数量级为$\alpha^4 mc^2$,其中$\alpha$就是精细结构常数.
\begin{equation}
\alpha = \frac{e^2}{4\pi\epsilon_0\hbar c} \approx 0.0072973525693(11) \approx \frac{1}{137.036}
\end{equation}

\subsection{相对论修正}
哈密顿量的首项为动能:
\begin{equation}\label{HfineS_eq1}
T=\frac{1}{2}mv^2=\frac{p^2}{2m}
\end{equation}
动量$\mathbf p$的正则替换(canonical substitution)为$-\I\hbar\Nabla$,由此可得动能算符:
\begin{equation}
T=-\frac{\hbar^2}{2m}\laplacian
\end{equation}
不过,注意到\autoref{HfineS_eq1} 为经典动能的表达式;现在我们考虑相对论表达式:
\begin{equation}
T=\frac{mc^2}{\sqrt{1-(v/c)^2}}-mc^2
\end{equation}
其中的第一项为总的相对论能量,第二项为静能.那么两项的差就是动能.这里我们需要用到相对论的动量代替速度来表示动能$\mathbf T$
\begin{equation}
p=\frac{mv}{\sqrt{1-(v/c)^2}}
\end{equation}
由于
\begin{equation}
p^2c^2+m^2c^4=\frac{m^2v^2c^2+m^2c^4[1-(v/c)^2]}{1-(v/c)^2}=\frac{m^2c^4}{1-(v/c)^2}=(T+mc^2)^2
\end{equation}
因此
\begin{equation}
T=\sqrt{p^2c^2+m^2c^4}-mc^2
\end{equation}
我们将其从$p/mc$级数展开得到近似:
\begin{align}
T &= mc^2\left[\sqrt{1+\left(\frac{p}{mc}\right)^2}\right]\\ 
&=mc^2\left[1+\left(\frac{p}{mc}\right)^2-\frac{1}{8}\left(\frac{p}{mc}\right)^4\cdots -1\right]\\
&=\frac{p^2}{2m}-\frac{p^4}{8m^3c^2}+\cdots
\end{align}
因此我们对哈密顿量的最低阶相对论修正为:
\begin{equation}
H'_r=-\frac{p^4}{8m^3c^2}
\end{equation}
在一阶近似微扰理论中对$E_n$的修正是由$H'$在无微扰态\autoref{TIPT_eq6}~\upref{TIPT}中的期待值:
\begin{equation}
E_r^1=\langle H'_r\rangle=-\frac{1}{8m^3c^2}\langle\psi|p^4\psi\rangle=-\frac{1}{8m^3c^2}\langle p^2\psi|p^2\psi\rangle
\end{equation}
由无微扰态的薛定谔方程得出:
\begin{equation}
T=E-V, \ p^2\psi = 2m(E-V)\psi
\end{equation}
因此我们可以得出:
\begin{equation}
E_r^1=-\frac{1}{2mc^2}(E-V)^2=-\frac{1}{2mc^2}[E^2-2E\langle V\rangle+\langle V^2\rangle]
\end{equation}
以上是一般情况下对$E_n$的修正,接下来我们重点考虑氢原子作为一个运用的实例.根据库伦定律我们可以得到氢原子的势能为$V(r)=-\frac{e^2}{4\pi\epsilon_0}\frac{1}{r}$,因此:
\begin{equation}
E_r^1=-\frac{1}{2mc^2}\left[E_n^2+2E_n\frac{e^2}{4\pi\epsilon_0}\left\langle \frac{1}{r}\right\rangle+\left(\frac{e^2}{4\pi\epsilon_0}\right)^2\left\langle \frac{1}{r^2}\right\rangle\right]
\end{equation}
其中$E_n$为无微扰的波尔能级.为了得到最终的结果,我们还需要再无微扰态$\psi_{nlm}$下求得$1/r$和$1/r^2$的期待值,首先$1/r$的平均值的计算比较简单(见\autoref{HfineS_exe1} ):
\begin{equation}
\left\langle\frac{1}{r}\right\rangle = \frac{1}{n^2a}
\end{equation}
其中的$a$为玻尔半径:
\begin{equation}
a=\frac{4\pi\epsilon_0\hbar^2}{me^2}=0.529\times 10^{-10}\rm{m}
\end{equation}
下一个$1/r^2$的期待值(见\autoref{HfineS_exe2} ):
\begin{equation}
\left\langle \frac{1}{r^2}\right\rangle = \frac{1}{(l+1/2)n^3a^2}
\end{equation}
这样我们就得到了:
\begin{equation}
E_r^1=-\frac{1}{2mc^2}\left[E_n^2+2E_n\frac{e^2}{4\pi\epsilon_0}\frac{1}{n^2a}+\left(\frac{e^2}{4\pi\epsilon_0}\right)^2\frac{1}{(l+1/2)n^3a^2}\right]
\end{equation}
进而我们还可以用消去$a$得到:
\begin{equation}
E_r^1=-\frac{(E_n)^2}{2mc^2}\left[\frac{4n}{l+1/2}-3\right]
\end{equation}
其中允许能级$E_n$有著名的玻尔公式:
\begin{equation}
E_{n} =-\left[\frac {m_e}{2\hbar^{2}} \left(\frac {e^ {2}}{4\pi e0}\right)^ {2}\right]  \frac {1}{n^ {2}}  =  \frac {E_ {1}}{n^ {2}}, \ \  n=1,2,3, \cdots 
\end{equation}
由此,我们可以看出相对论修正的$E^1_n$小于$E_n$,其比例系数约为$E_n/(mc^2)=2\times 10^{-5}$.

\subsection{自旋-轨道耦合}
电子围绕原子核做轨道运动,相对的以电子作为参考系,质子围绕着电子做轨道运动.在假设电子为静止的坐标系下,这样一个围绕电子做圆周运动的带正电荷的质子就会产生一个磁场$\bvec B$. 因此就会产生一个作用于有自旋的电子的力矩,使得其自旋方向趋同于磁场的方向.那么就有哈密顿量为:
\begin{equation}
H=-\bvec\mu \cdot \bvec B
\end{equation}
因此我们需要找到质子的磁场和电子自旋的磁矩.

首先对于\textbf{质子的磁场}来说,若是我们从电子静止的角度看质子的运动,并将其看作一个连续的圆环线电流,那么根据电动力学中的毕奥-萨伐尔定律
\begin{exercise}{}\label{HfineS_exe1}
利用维里定理证明计算并得出等式:
\begin{equation}
\left\langle\frac{1}{r}\right\rangle = \frac{1}{n^2a}
\end{equation}
\end{exercise}
\begin{exercise}{}\label{HfineS_exe2}
计算并得出等式:
\begin{equation}
\left\langle \frac{1}{r^2}\right\rangle = \frac{1}{(l+1/2)n^3a^2}
\end{equation}
\end{exercise}
微扰项为
\begin{equation}
H'_{so} = \frac{e^2}{8\pi \epsilon_0 m^2 c^2} \frac{\bvec S \vdot \bvec L}{r^3}
\end{equation}
