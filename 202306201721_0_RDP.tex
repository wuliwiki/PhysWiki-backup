% RDP 远程桌面笔记

\begin{issues}
\issueDraft
\end{issues}

\begin{itemize}
\item \textbf{远程桌面协议(Remote Desktop Protocol, RDP)}
\item Windows 自带的\textbf{远程连接(Remote Desktop Connection)}应用
\item 默认端口为 3389
\item 并不自带内网穿透, 如果没有代理服务器, 只能在局域网内连接。 设置代理服务器见 “FRP 内网穿透笔记\upref{NATthr}”。 至于 WSL1 如何开机启动 frp 客户端, 参考 “WSL 笔记\upref{WSLnt}”
\item Windows 一次只能登录一个用户(包括远程),如果电脑上已经登录一个用户, 远程连接登录另一个, 则当前用户会被 logout,进程全部结束
\item 如果电脑上登录一个用户,远程也登录这个用户,则当前用户被锁屏,进程继续运行
\end{itemize}
