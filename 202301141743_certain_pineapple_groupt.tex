% 群乘法表
% 群乘法表|重排定理

\pentry{群\upref{Sample}}

群按照其元素个数可以大致分为两个类型,拥有有限个元素的离散群和拥有无限个元素的连续群,描述不同群的乘法规则是一个极为重要的事情,在有限群中通常使用群乘法表的概念,而连续群中则使用结合函数来描述群元之间的乘法关系。

在介绍群的乘法表之前先引入重排定理的概念,这个概念将对我们讲乘法表写出来起很大作用。

\subsection{重排定理}

\begin{theorem}{重排定理}
若$g$为群$G$中任意一元素,那么在相同的乘法规则下以下三个群与G是同一个群:

$gG=\{gg_1,~~gg_2,~~gg_3...\}$

$Gg=\{g_1g,~~g_2g,~~g_3g...\}$

$G^{-1}=\{g_1^{-1},~~g_2^{-1},~~g_3^{-1}...\}$
\end{theorem}

证明:
首先考虑第三个群$G^{-1}=\{g_1^{-1},~~g_2^{-1},~~g_3^{-1}...\}$,由于所有元素的逆元是唯一的,所以不同元素的逆元互不相同,切$g^{-1}\in G$
