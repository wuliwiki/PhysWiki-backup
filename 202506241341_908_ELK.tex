% 恩里科·费米(综述)
% license CCBYSA3
% type Wiki

本文根据 CC-BY-SA 协议转载翻译自维基百科 \href{https://en.wikipedia.org/wiki/Enrico_Fermi}{相关文章}。

恩里科·费米(意大利语:[enˈriːko ˈfermi],1901年9月29日-1954年11月28日)是一位意大利裔、后归化为美国公民的物理学家,以建造世界上第一座人工核反应堆——芝加哥一号堆而闻名,并曾是曼哈顿计划的重要成员。他被誉为“核时代的建筑师”以及“原子弹之父”。\(^\text{[1]}\)他是极少数在理论物理和实验物理两个领域都卓有成就的物理学家之一。费米因其在中子轰击引发放射性方面的研究以及对超铀元素的发现而获得1938年诺贝尔物理学奖。他与同事们共同申请了多项与核能应用相关的专利,所有这些专利最终都被美国政府接管。他在统计力学、量子理论、核物理和粒子物理的发展中都作出了重要贡献。

费米的第一个重大贡献是在统计力学领域。1925年,沃尔夫冈·泡利提出了著名的泡利不相容原理,随后费米发表了一篇论文,将该原理应用于理想气体,发展出一种统计方法,如今被称为费米–狄拉克统计。今天,那些遵守不相容原理的粒子被称为“费米子”。后来,泡利为了解释β衰变中能量守恒的问题,提出了在电子发射的同时还会发射一种不带电的不可见粒子这一假设。费米接纳了这个想法,并构建了一个理论模型,纳入了这一假想粒子,并将其命名为“中微子”。他的这一理论后来被称为“费米相互作用”,现今称为“弱相互作用”,是自然界四种基本相互作用之一。在用新发现的中子进行诱导放射性实验时,费米发现慢中子比快中子更容易被原子核俘获,并据此发展出描述该过程的“费米年龄方程”。在用慢中子轰击钍和铀的实验中,费米认为自己合成了新的元素。尽管他因这一发现获得了诺贝尔奖,但后来证实这些“新元素”其实是核裂变的产物。1938年,为了躲避影响其犹太妻子劳拉·卡蓬的意大利新种族法,费米离开意大利,移民美国。在第二次世界大战期间,他参与了“曼哈顿计划”。在芝加哥大学,费米领导的团队设计并建造了“芝加哥堆-1”,该堆于1942年12月2日首次实现了人类制造的、自持的核链式反应。他还在田纳西州橡树岭的X-10石墨反应堆于1943年达到临界状态时在场,次年又见证了华盛顿州汉福德基地的B反应堆启动。在洛斯阿拉莫斯国家实验室,他领导F部门,其中一部分致力于爱德华·泰勒的热核“超级炸弹”项目。他还亲历了1945年7月16日的“特立尼蒂试验”,即首次核弹爆炸测试,并使用著名的“费米估算法”评估了炸弹的当量。

战后,费米协助创建了芝加哥的核研究所,并在J·罗伯特·奥本海默担任主席的总顾问委员会中任职,为美国原子能委员会提供核事务建议。1949年8月苏联成功引爆第一颗裂变原子弹后,费米从道德和技术两方面都强烈反对研制氢弹。1954年,在导致奥本海默失去安全许可的听证会上,费米也是为奥本海默作证的科学家之一。

费米在粒子物理领域也做出了重要贡献,尤其是在与介子(如π介子和μ子)相关的研究方面。他还推测宇宙射线的产生是由于星际空间中的磁场加速物质所致。许多奖项、概念和机构都以费米的名字命名,包括费米一号(快中子增殖反应堆)、恩里科·费米核发电站、恩里科·费米奖、恩里科·费米研究所、费米国家加速器实验室、费米伽马射线太空望远镜、“费米悖论”,以及人造元素“镄”,这使他成为仅有的十六位拥有化学元素以自己命名的科学家之一。
\subsection{早年生活}
\begin{figure}[ht]
\centering
\includegraphics[width=6cm]{./figures/635b80f64a6090ee.png}
\caption{费米出生于罗马盖塔街19号。} \label{fig_ELK_1}
\end{figure}
恩里科·费米于1901年9月29日出生在意大利罗马。\(^\text{[3]}\)他是铁路部司局长阿尔贝托·费米与小学教师伊达·德·加蒂斯的第三个孩子。\(^\text{[3][4][5]}\)他的姐姐玛丽亚比他大两岁,哥哥朱利奥大他一岁。两位男孩幼年时被送到乡下由乳母抚养,直到恩里科两岁半时才返回罗马与家人团聚。\(^\text{[6]}\)尽管按照祖父母的意愿他接受了天主教洗礼,但他的家庭并不虔诚;恩里科成年后一直是无神论者。\(^\text{[7]}\)童年时期,他和哥哥朱利奥有着相同的兴趣爱好,比如制作电动机、玩电动和机械玩具。\(^\text{[8]}\)1915年,朱利奥因喉部脓肿手术不幸去世;玛丽亚则于1959年在米兰附近的一场空难中遇难。\(^\text{[9][10]}\)

在罗马的鲜花广场集市上,费米发现了一本物理书,名为《Elementorum physicae mathematicae》,全书900页,由耶稣会士、罗马学院教授安德烈亚·卡拉法神父用拉丁文编写。书中介绍了当时(1840年出版)对数学、经典力学、天文学、光学和声学的理解。\(^\text{[11][12]}\)在一位同样对科学感兴趣的朋友恩里科·佩尔西科的陪伴下,\(^\text{[13]}\)费米开始了诸如制作陀螺仪、测量地球重力加速度等实验项目。\(^\text{[14]}\)

1914年,费米常常在父亲下班后到办公室门口与其会合。有一次,他遇见了父亲的一位同事阿道夫·阿米代伊。恩里科得知阿道夫对数学和物理感兴趣,便趁机向他请教几何问题。阿道夫意识到小费米问的是射影几何,随后送给他一本由特奥多尔·雷耶所著的相关书籍。两个月后,费米将书还给了阿道夫,声称自己已完成了书后所有的习题,其中一些题目连阿道夫都认为很难。阿道夫核实后惊叹道费米“至少在几何方面是个天才”,并开始更加系统地指导他,给他提供更多关于物理与数学的书籍。阿道夫还指出,费米的记忆力极好,读完书后就能记住全部内容,因此通常读完便归还书籍。\(^\text{[15]}\)
\subsection{比萨高等师范学院}
\begin{figure}[ht]
\centering
\includegraphics[width=6cm]{./figures/7f8ade509d759b3a.png}
\caption{} \label{fig_ELK_2}
\end{figure}
费米于1918年7月高中毕业,他跳过了第三学年。在阿米代伊的敦促下,费米学习了德语,以便阅读当时大量以德语发表的科学论文,并申请了比萨高等师范学院。阿米代伊认为,这所学校能为费米的发展提供比当时的罗马萨皮恩扎大学更好的条件。费米的父母在失去一个儿子后,勉强同意他在学校的住宿区生活四年,远离罗马。\(^\text{[16][17]}\)费米在难度极高的入学考试中获得第一名,其中包括一篇题为“声音的特征”的作文;17岁的费米选择使用傅里叶分析来导出并求解振动棒的偏微分方程。考官在面试他之后断言,费米将成为一位杰出的物理学家。\(^\text{[16][18]}\)

在比萨高等师范学院,费米与同学弗兰科·拉塞蒂一起恶作剧,两人成为了亲密的朋友和合作伙伴。物理实验室主任路易吉·普恰蒂是费米的导师,他曾表示自己几乎教不了费米什么,反而常常请费米教他一些东西。\(^\text{[19]}\)费米在量子物理方面的知识如此扎实,以至于普恰蒂请他主持有关该主题的研讨会。在此期间,费米学习了张量计算法,这是一种在广义相对论中至关重要的数学工具。\(^\text{[20]}\)费米最初选择数学作为主修专业,但很快转为物理学。他在很大程度上是自学成才,系统学习了广义相对论、量子力学和原子物理学。\(^\text{[21]}\)

1920年9月,费米被正式录取进入物理系。由于系里只有三名学生——费米、拉塞蒂和内洛·卡拉拉——导师普恰蒂便允许他们自由使用实验室,进行任何他们感兴趣的研究。\(^\text{[22]}\)费米决定他们应该研究X射线晶体学,于是三人开始合作拍摄劳厄照片——即晶体的X射线照片。1921年,费米在大学三年级期间,在意大利期刊《新物理杂志》上发表了他的第一批科学论文。第一篇题为《平移运动中电荷刚体系统的动力学》(意大利语原文:Sulla dinamica di un sistema rigido di cariche elettriche in moto traslatorio)。这篇论文展现了费米未来研究方向的端倪:他将质量表达为张量——这是一个常用于描述三维空间中运动和变化的数学结构。在经典力学中,质量是一个标量,但在相对论中,它随着速度变化。第二篇论文是《在均匀引力场中电磁电荷的静电学以及电磁电荷的重量》。在这篇论文中,费米使用广义相对论证明,一个电荷的重量等于其系统的静电能量$U$除以光速$c$的平方,即 $U/c^2$。\(^\text{[21]}\)

第一篇论文似乎指出了电动力学理论与相对论之间在电磁质量计算方面的矛盾:电动力学预测的电磁质量为 $\frac{4}{3} \frac{U}{c^2}$,而相对论则给出 $\frac{U}{c^2}$。次年,费米在论文《论电动力学与相对论在电磁质量问题上的一个矛盾》中解决了这个问题,他指出这种表面上的矛盾实际上是相对论自身的一个推论。这篇论文得到了高度评价,并于1922年被翻译成德文,发表在德国科学期刊《物理学杂志》上。\(^\text{[23]}\)同年,费米向意大利期刊《林琴学会会报》提交了题为《在时线附近发生的现象》(意大利语原文:Sopra i fenomeni che avvengono in vicinanza di una linea oraria)的论文。在这篇论文中,他研究了等效原理,并引入了后被称为“费米坐标”的概念。他证明,在接近某条时线的世界线上,空间的行为可以被视为欧几里得空间。\(^\text{[24][25]}\)
\begin{figure}[ht]
\centering
\includegraphics[width=8cm]{./figures/b24ae5308fe0c01a.png}
\caption{光锥是时空中从某一点出发或到达该点的所有可能光线所构成的三维曲面。在图示中,压缩了一个空间维度。时间线为垂直轴。} \label{fig_ELK_3}
\end{figure}
费米于1922年7月向比萨高等师范学校提交了他的论文《一个概率定理及其若干应用》(意大利语:Un teorema di calcolo delle probabilità ed alcune sue applicazioni),并在异常年轻的20岁时获得学位。论文主题是X射线衍射图像。当时理论物理在意大利尚未被视为一个正式学科,唯一能被接受的毕业论文是实验物理类。因此,意大利物理学界在接受来自德国的新思想(如相对论)方面进展缓慢。由于费米在实验室工作中如鱼得水,这一限制对他而言并非不可逾越的障碍。\(^\text{[25]}\)

1923年,在为奥古斯特·科普夫所著《爱因斯坦相对论基础》意大利语版撰写附录时,费米是第一个指出爱因斯坦公式($E = mc^2$)中隐藏着巨大的核能可供开发的人。\(^\text{[26]}\)他写道:“至少在可预见的将来,似乎不可能找到释放这些可怕能量的方法——这也好,因为如此惊人能量一旦爆炸,首先被炸得粉碎的就是那个不幸找到释放方法的物理学家。”\(^\text{[25]}\)

1924年,费米加入了意大利东方大会下属的“阿德里亚诺·莱米”共济会支部。\(^\text{[27]}\)

在1923至1924年间,费米在哥廷根大学跟随马克斯·玻恩学习了一个学期,并在那里结识了维尔纳·海森堡和帕斯夸尔·约尔当。随后,他在1924年9月至12月期间,凭借数学家维托·沃尔特拉的推荐,获得洛克菲勒基金会的奖学金,在莱顿大学与保罗·埃伦费斯特一同学习。在那里,费米遇到了亨德里克·洛伦兹和阿尔伯特·爱因斯坦,并与塞缪尔·古兹密特和扬·廷贝亨成为朋友。

从1925年1月到1926年末,费米在佛罗伦萨大学教授数学物理和理论力学课程,并与拉塞蒂合作,开展了一系列关于磁场对汞蒸气影响的实验。他还在罗马大学参与学术研讨会,讲授量子力学和固体物理。\(^\text{[28]}\)在讲授基于薛定谔方程的、能极为准确地进行预测的新量子力学时,费米常半带惊讶地说:“它居然吻合得这么好,真是不可思议!”\(^\text{[29]}\)

1925年,沃尔夫冈·泡利提出泡利不相容原理后,费米迅速作出回应,发表论文《关于理想单原子气体的量子化》(意大利语:Sulla quantizzazione del gas perfetto monoatomico),将该原理应用于理想气体。这篇论文最具特色之处在于费米提出了一种统计表达形式,描述了大量服从不相容原理的同类粒子在系统中的分布。英国物理学家保罗·狄拉克随后也独立发展了这一统计理论,并指出它与玻色–爱因斯坦统计的关系。因而,该统计方法如今被称为费米–狄拉克统计。\(^\text{[30]}\)自狄拉克之后,遵循不相容原理的粒子被称为“费米子”,而不遵循该原理的粒子则被称为“玻色子”。\(^\text{[31]}\)
\subsection{罗马的教授}
\begin{figure}[ht]
\centering
\includegraphics[width=6cm]{./figures/904220e3d5812f55.png}
\caption{费米和他的研究小组(称为“潘尼斯佩尔纳小组”)在罗马大学物理学研究所的院子里,约1934年。从左到右:奥斯卡·达戈斯蒂诺、埃米利奥·塞格雷、埃多瓦尔多·阿马尔迪、弗朗科·拉塞蒂和费米。} \label{fig_ELK_4}
\end{figure}
意大利的教授职位是通过竞争来授予的,申请人根据其发表的作品,由一委员会的教授进行评定。费米曾申请了萨丁岛卡利亚里大学的数学物理学教授职位,但由于吉奥万尼·乔尔吉的竞争,他未能成功当选。\(^\text{[32]}\)1926年,24岁的费米申请了罗马大学的一个教授职位。这个职位是新设立的,是意大利首批三个位于理论物理学领域的教授职位之一,该职位是应实验物理学教授、物理学研究所所长以及贝尼托·墨索里尼内阁成员奥尔索·马里奥·科尔比诺教授的倡议,由教育部长创建的。科尔比诺教授同时也主持了评选委员会,他希望这个新职位能够提升意大利物理学的标准和声誉。\(^\text{[33]}\)评选委员会选择了费米,而非恩里科·佩尔西科和阿尔多·庞特雷莫利。\(^\text{[34]}\)科尔比诺帮助费米招募了他的团队,团队很快吸引了埃多瓦尔多·阿马尔迪、布鲁诺·庞特科沃、埃托雷·马约拉纳、埃米利奥·塞格雷等杰出学生的加入,并且费米任命了弗朗科·拉塞蒂为他的助手。\(^\text{[35]}\)他们很快就被昵称为“潘尼斯佩尔纳小组”,以纪念物理学研究所所在的街道——潘尼斯佩尔纳街。\(^\text{[36]}\)

费米于1928年7月19日与大学的科学专业学生劳拉·卡彭结婚。\(^\text{[37]}\)他们有两个孩子:内拉,生于1931年1月,朱利奥,生于1936年2月。\(^\text{[38]}\)1929年3月18日,费米被墨索里尼任命为意大利皇家学会成员,4月27日加入法西斯党。当1938年墨索里尼出台种族法案,使意大利法西斯主义在意识形态上更加接近德国纳粹主义时,费米反对法西斯主义。这些法律威胁到了犹太人身份的劳拉,并使费米的许多研究助手失业。\(^\text{[39][40][41][42][43]}\)

在罗马期间,费米和他的团队对物理学的许多实际和理论方面做出了重要贡献。1928年,他出版了《原子物理学导论》,为意大利大学生提供了一本最新的、易于理解的教材。费米还举办了公开讲座,并为科学家和教师撰写了普及性文章,以尽可能广泛地传播新物理学的知识。\(^\text{[44]}\)他的一部分教学方法是,在一天的工作结束后,召集同事和研究生们一起讨论问题,通常是来自他自己研究中的问题。\(^\text{[44][45]}\)成功的标志之一是,外国学生开始来意大利学习。最著名的学生之一是德国物理学家汉斯·贝特(Hans Bethe),他作为洛克菲勒基金会的研究员来到罗马,并与费米合作撰写了1932年的论文《两电子之间的相互作用》(德语:Über die Wechselwirkung von Zwei Elektronen)。\(^\text{[47][44]}\)

在这个时候,物理学家们对β衰变感到困惑,在β衰变中,一个电子从原子核中发射出来。为了满足能量守恒定律,保利假设存在一个无电荷且质量极小或没有质量的不可见粒子,它在同一时间也被发射出来。费米接受了这一想法,并在1933年写下了初步的论文,接着在第二年写了一篇更长的论文,纳入了这一假设的粒子,费米称之为“中微子”。\(^\text{[48][49][50]}\)他的理论,后来被称为费米相互作用,进一步被称为弱相互作用理论,描述了自然界的四种基本相互作用之一。中微子在费米去世后被发现,而他的相互作用理论则揭示了为什么中微子如此难以被探测到。当费米将他的论文提交给英国期刊《自然》时,期刊的编辑拒绝了它,因为它包含了“与物理现实相距太远的推测,不足以引起读者的兴趣”。\(^\text{[49]}\)根据费米的传记作者David N. Schwartz的说法,至少可以说,费米认真要求该期刊出版他的论文是非常奇怪的,因为当时《自然》期刊只会发表这种类型文章的简短笔记,并不适合发表任何新的物理理论。如果要找更合适的期刊,可能《伦敦皇家学会会刊》会更合适。他同意一些学者的假设,即英国期刊的拒绝让费米的年轻同事们(其中一些是犹太人和左翼分子)放弃了对德国科学期刊的抵制,尤其是在希特勒于1933年1月上台后。\(^\text{[51]}\)因此,费米看到了该理论首先在意大利语和德语中出版,然后才在英语中发表。\(^\text{[35]}\)

在1968年英文版的序言中,物理学家Fred L. Wilson指出:

费米的理论,除了支持保利提出的中微子假设外,在现代物理学史上具有特殊意义。必须记住,在理论提出时,仅知道天然发生的$\beta$发射源。后来,当发现正电子衰变时,这一过程很容易被纳入费米的原始框架中。根据他的理论,预测了轨道电子被原子核捕获的现象,并最终得到了观测。随着时间的推移,实验数据显著积累。尽管在$\beta$衰变中多次观察到一些异常现象,费米的理论始终能够应对挑战。

费米理论的影响是广泛的。例如,$\beta$光谱学被确立为研究核结构的强有力工具。但或许费米工作的最具影响力的方面是,他特定形式的$\beta$相互作用确立了一个模式,这一模式适用于研究其他类型的相互作用。这是第一个成功的物质粒子创造与湮灭的理论。在此之前,只有光子被知晓是可以创造和消灭的。\(^\text{[50]}\)

1934年1月,伊雷娜·约里奥-居里和弗雷德里克·约里奥宣布,他们用α粒子轰击元素并使其产生了放射性。\(^\text{[52][53]}\)到3月,费米的助手吉安-卡洛·威克用费米的$\beta$衰变理论提供了理论解释。费米决定转向实验物理,使用1932年詹姆斯·查德威克发现的中子。\(^\text{[54]}\)1934年3月,费米希望通过拉塞蒂的钋铍中子源来看看是否能引发放射性。中子没有电荷,因此不会被带正电的原子核偏转。这意味着它们穿透原子核所需的能量远低于带电粒子,因此不需要粒子加速器,而“Via Panisperna boys”并没有粒子加速器。\(^\text{[55][56]}\)
\begin{figure}[ht]
\centering
\includegraphics[width=8cm]{./figures/b5362410b9f2f7c9.png}
\caption{恩里科·费米(Enrico Fermi)与弗朗科·拉塞蒂(Franco Rasetti,左)和埃米利奥·塞格雷(Emilio Segrè)穿着学术服装合影。} \label{fig_ELK_5}
\end{figure}
费米想到用氡-铍中子源替代钋-铍中子源,他通过将铍粉末填充到玻璃瓶中,抽出空气,然后加入50毫居里的氡气(由朱利奥·切萨雷·特拉巴基提供)来制造这种源。\(^\text{[57][58]}\)这创造了一个更强的中子源,但其有效性随着氡的3.8天半衰期而衰减。他知道这个源还会发射伽马射线,但基于他的理论,他认为这不会影响实验结果。他开始用铂金属进行轰击,铂是一种具有较高原子序数且易于获得的元素,但没有成功。接着他转向铝,铝发射出一个$\alpha$粒子并生成钠,钠通过$\beta$粒子发射衰变为镁。他又尝试了铅,但没有成功,然后尝试了氟,形式是氟化钙,氟发射一个$\alpha$粒子并生成氮,氮通过$\beta$粒子发射衰变为氧。总的来说,他在22种不同的元素中诱发了放射性。\(^\text{[59]}\)费米迅速在1934年3月25日的意大利期刊《La Ricerca Scientifica》上报告了中子诱发放射性的发现。\(^\text{[58][60][61]}\)

由于钍和铀的天然放射性,使得很难确定当这些元素被中子轰击时发生了什么。但在正确排除了比铀轻但比铅重的元素的存在后,费米得出结论,他们已经创造了新元素,他称之为奥斯尼姆和赫斯佩里姆。\(^\text{[62][56]}\)化学家伊达·诺达克建议,某些实验可能产生了比铅更轻的元素,而不是新产生的、更重的元素。当时,她的建议并未受到重视,因为她的团队既没有进行铀的实验,也没有为这一可能性建立理论基础。当时,裂变被认为在理论上是不可行的。如果不是不可能的话。尽管物理学家们预计,通过中子轰击轻元素会形成原子序数更高的元素,但没有人预料到中子会有足够的能量以诺达克所建议的方式,将较重的原子分裂成两个轻元素碎片。\(^\text{[63][62]}\)
\begin{figure}[ht]
\centering
\includegraphics[width=6cm]{./figures/0d154bb9cd18a6d5.png}
\caption{$\beta$衰变。一颗中子衰变为一个质子,并发射出一个电子。为了保持系统的总能量不变,保利和费米假设也会发射一个反电子中微子(${\displaystyle {\bar {\nu }}_{e}}$)。} \label{fig_ELK_6}
\end{figure}
“Via Panisperna的男孩们还注意到一些无法解释的现象。实验似乎在木桌上比在大理石桌面上效果更好。费米记得Joliot-Curie和Chadwick曾指出石蜡在减速中子方面非常有效,所以他决定试一试。当中子通过石蜡时,它们在银上诱发的放射性是没有石蜡时的100倍。费米猜测这是因为石蜡中的氢原子,木头中的氢原子也解释了木桌面与大理石桌面之间的差异。通过重复用水做实验,证明了这一点。他得出结论,氢原子与中子的碰撞使得中子减速。[64][56] 中子与其碰撞的原子核的原子序数越低,每次碰撞损失的能量越多,因此,减少中子的速度所需的碰撞次数就越少。[65] 费米意识到,这会引发更多的放射性,因为慢中子比快中子更容易被捕获。他开发了一个扩散方程来描述这一现象,这个方程后来被称为费米年龄方程。[64][56]

1938年,费米因其“通过中子辐照证明了新放射性元素的存在,并因此发现了由慢中子引起的核反应”而获得了诺贝尔物理学奖,时年37岁。[66] 在斯德哥尔摩获得诺贝尔奖后,费米没有回到意大利,而是与家人一起于1938年12月继续前往纽约市,并在那里申请了永久居留权。决定移居美国并成为美国公民,主要是由于意大利的种族法。[39][67]”
