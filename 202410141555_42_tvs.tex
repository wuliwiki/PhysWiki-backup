% 拓扑向量空间
% keys 拓扑向量空间|局部凸拓扑向量空间|局部凸空间
% license Xiao
% type Wiki
\pentry{拓扑空间\nref{nod_Topol},向量空间\nref{nod_LSpace}}{nod_3053}

\subsection{拓扑向量空间}

\begin{definition}{拓扑向量空间}
一个 $\mathbb{F}$-拓扑向量空间 $V$ 是一个实(或复)向量空间同时也是一个拓扑空间,使得加法 $+: V \times V \to V$ 和数乘 $\cdot: \mathbb{F} \times V \to V$ 都是连续函数。

在常见的的语境下我们要求拓扑向量空间是Hasdorff空间(\enref{见分离性}{Topo5}),此时拓扑向量空间是一个\enref{拓扑(加法)群}{TopGrp}。
\end{definition}

\begin{example}{}
\enref{赋范空间}{NormV}、\enref{内积空间}{InerPd}都是拓扑向量空间。

事实上:赋范空间和内积空间都可以通过定义度量成为度量空间,而度量空间是拓扑空间,对赋范空间(范数为 $\norm{\cdot}$),自然的度量定义为 $d(x,y):=\norm{x-y}$,对内积空间(内积为 $(\cdot,\cdot)$)则为 $d(x,y):=\sqrt{(x-y,x-y)}$。因此它们都是拓扑空间。由于它们本身就是在线性空间上定义的,因而都是线性空间。所以只需验证加法和数乘的连续性。

既然它们的拓扑由度量空间来保证,那么可以用度量定义的拓扑来验证。任意 $z_0=x_0+y_0$,设 $O_{z_0}$ 是 $z_0$ 邻域,于是其包含某一开球  $B(z_0,\epsilon)$, 选取 $x_0$ 的邻域 $B(x_0,\epsilon/2)$ 和 $y_0$ 的邻域 $B(y_0,\epsilon/2)$,于是任意 $x\in B(x_0,\epsilon/2),y\in B(y_0,\epsilon/2)$,对赋范空间和内积空间分别成立 
\begin{equation}
\begin{aligned}
d(x+y,z_0)=&\norm{x+y-(x_0+y_0)}\\
\leq& \norm{x-x_0}+\norm{y-y_0}=d(x,x_0)+d(y,y_0)< \epsilon;\\
d(x+y,z_0)=&\sqrt{(x+y-z_0,x+y-z_0)}\\
=&\sqrt{(x-x_0|x-x_0)+(y-y_0|y-y_0)+2(x-x_0,y-y_0)}\\
\leq&\sqrt{(x-x_0|x-x_0)+(y-y_0|y-y_0)+2\sqrt{(x-x_0,x-x_0)}\sqrt{(y-y_0,y-y_0)}}\\
=& \sqrt{(x-x_0,x-x_0)}+\sqrt{(y-y_0,y-y_0)};\\
=&d(x,x_0)+d(y,y_0)< \epsilon;\\
\end{aligned}~
\end{equation}
即对赋范空间和内积空间,都成立 $B(x_0,\epsilon/2)+B(y_0,\epsilon/2)\subset O_{z_0}$。即加法的连续性得证!

由于
\begin{equation}
\begin{aligned}
\norm{\lambda x-\lambda_0x_0}=&\norm{\lambda(x-x_0)+(\lambda-\lambda_0)x_0}\\
\leq&\abs{\lambda}\norm{x-x_0}+\abs{\lambda-\lambda_0}\norm{x_0},\\
\end{aligned}~
\end{equation}
则在 $x_0\neq0$ 时,只需令取满足 $\delta<\frac{\epsilon}{2\norm{x_0}},\abs{\lambda_0\pm \delta}\epsilon_1<\frac{\epsilon}{2}$ 的 $\delta$ 和 $\epsilon_1$ 分别作为 $\lambda_0,x_0$ 的邻域,就有 $\norm{\lambda x-\lambda_0x_0}<\epsilon$。而在 $x_0=0$ 时,只需任取 $\lambda>0$ 作为 $\lambda_0$ 的邻域,而取满足 $\abs{\lambda_0\pm\delta}\epsilon_1<\epsilon$ 的正数 $\epsilon_1$ 作为 $x_0$ 的球邻域即可。
\end{example}

\begin{theorem}{}\label{the_tvs_1}
拓扑向量空间的任一点的邻域都可由某一零邻域(零向量0的邻域)确定。具体的,若 $U$ 是 $x$ 的邻域,则 $U-x$ 是 $0$ 的邻域。
\end{theorem}

\textbf{证明:}由向量拓扑空间加法的连续性,设 $f_x:U-x\rightarrow U,f_x(v)=v+x,v\in U-x$,即 $f_x$ 是加法运算,那么 $f_x$ 是连续的,于是对 $f_x(0)=x$ 的每一邻域 $U$,$f^{-1}(U)=U-x$ 是 $0$ 的邻域。

\textbf{证毕!}

注意开集必须邻域,于是,\autoref{the_tvs_1} 表明所有的开集都由包含0的开集确定,从而拓扑向量空间的拓扑由零邻域系\footnote{零矢量0的所有邻域}所确定。

\begin{theorem}{}
设 $E$ 是拓扑向量空间,则对任意 $x\in E$ 及不包含 $x$ 的闭集 $F$,都有包含它们的不相交的邻域。即存在开集 $x\in V_1,F\subset V_2$,使得 $V_1\cap V_2=\emptyset$。
\end{theorem}

\textbf{证明:}只需考虑 $x=0$ 和不包含它的任意闭集 $F$ 就够了。事实上,$F$ 不包含 $x$,当且将当 $F-x$ 不包含 0。如若不然,则 $0\in F-x$,那么 $x=0+x\in F-x+x=F$,而这是不可能的。

令 $U=E\backslash F$,则由加法的连续性,存在零邻域 $W_1,W_2$,使得 $W_1-W_2\subset U$,取 $W=W_1\cap W_2$,则存在零邻域 $W$,使得 $W-W\subset U$。设 $y\in [W]$,则点

\textbf{证毕!}




\addTODO{有限维度实向量空间有唯一确定的(Hausdorff)拓扑结构}
% Giacomo:可以参考 https://kconrad.math.uconn.edu/blurbs/topology/finite-dim-TVS.pdf

\subsection{局部凸拓扑向量空间/局部凸空间}
任意拓扑向量空间可能具有的性质与欧氏空间或赋范空间熟知的性质有显著的差异,比赋范空间更一般但保持赋范空间众多性质的重要一类空间构成所谓局部凸空间。


\begin{definition}{局部凸}
向量拓扑空间叫做\textbf{局部凸的},若其中每一非空开集含有非空\enref{凸开子集}{ConSet}。
\end{definition}

\begin{theorem}{}
若空间 $E$ 是局部凸的,则对任一点 $x\in E$ 及其任一邻域 $U$,存在 $x$ 的凸邻域 $V$,使得 $V\subset U$。
\end{theorem}

\textbf{证明:}由\autoref{the_tvs_1} ,只需要证明命题对 $x=0$ 成立即可。设 $U$ 是任一零邻域,则由加法的连续性,可以找到邻域 $V$,使得 $V-V\subset U$。因为 $E$ 是局部凸的,所以存在非空凸开集 $V'\subset V$。设 $y\in V'$,则 $V'-y$ 是含于 $U$ 中的凸零邻域。


\textbf{证毕!}





\addTODO{待续}
% Giacomo:希望有专家来续写这部分。

% \begin{definition}{局部凸拓扑向量空间/局部凸空间}
% 一个拓扑向量空间 $V$ 被称为\textbf{局部凸(locally convex)}的,如果它满足

% \end{definition}


