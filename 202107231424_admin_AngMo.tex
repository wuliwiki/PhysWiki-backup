% 角动量
% 角动量

\begin{issues}
\issueTODO
\end{issues}

从“角动量”的词义上看,角动量可理解成转动一定角度产生的动量,即角动量首先是一个动量,其次这个动量是由物体的转动产生的.我们知道,物体的动量是由于物体的运动产生的,而描述运动的物理量是速度.在物理当中,常常用物理量的英文单词首字母来表示某个物理量.比如用 $\bvec v$ 表示物体运动的速度.“动量是由物体的运动产生的”表达了动量必须包含两个特性:物体和物体的运动.而运动是由速度 $\bvec v$ 来描述的,那动量必须包含的另一特性“物体”该怎么描述呢?当说到一个物体,不考虑它的几何特性,那你会想到物体的什么属性?在力学当中,所有物体都有个重要的物理量,就是质量,用 $m$表示.所以我们可用质量 $m$ 来描述“物体”这个概念.有了 $m$ 和 $\bvec v$,要由这两部件构成动量,该怎么组合呢?首先我们会把它们放在一起构成 $m\bvec v$,注意这里只是单纯的把它们放在一起,并不代表任何运算.但是,考虑到质量 $m$ 是一个标量,速度 $\bvec v$ 是一个矢量,数和矢量放在一起,你想到啥?是不是很自然的想到我们熟悉的数乘.所以我们也就直接用数乘 $m\bvec v$代表动量了.

在得到了动量 $m\bvec v$后,我们来看由于物体转动产生的动量——角动量.我们知道,速度是极短的时间内物体发生的位移与这段时间的比值,而物体的转动产生的位移不就是转动半径 $r$ 乘以 极短时间t内物体转动的角度 $\theta$ 嘛,那么转动产生的速度不就是 $r\theta/t$嘛!而这极短时间内物体转动的角度 $\theta$ 与极短时间 $t$ 的比值 $\theta/t$ 叫作角速度 $\omega$.所以转动产生的速度也就是 $r\omega$,写成矢量形式不就是 $\bvec \omega\times \bvec r$,所以角动量就自然表示成 $m\bvec \omega\times \bvec r$. 我们以后都用 $\bvec L$ 表示角动量,而用 $\bvec p$ 表示动量.
\addTODO{角动量性质、转动惯量、添加例子}