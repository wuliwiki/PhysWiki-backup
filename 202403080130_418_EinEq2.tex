% 爱因斯坦场方程(含宇宙项)
% keys 广义相对论|宇宙常数|爱因斯坦场方程
% license Usr
% type Tutor

\pentry{爱因斯坦场方程\nref{nod_EinEqn},第二 Bianchi 恒等式(微分)\nref{nod_RicciC}}{nod_88fb}

在爱因斯坦推出场方程的年代(约 $1916$ 年),人们一般认为宇宙是静止不动的。但根据当时的爱因斯坦场方程 $ G_{\mu \nu} = 8 \pi G  T_{\mu \nu}$,人们意识到这使得宇宙是正在膨胀或收缩的。(实际上银河系以外存在其他星系都是在 $1924$ 年被埃德温·哈勃首次发现的。)

为此,由第二 Bianchi 恒等式(微分)不难想到,为了仍保证能量守恒,只能对原本的场方程修正。原本的场方程只剩下一个额外自由度,为此,爱因斯坦在场方程左边加上了一个常数 $\Lambda$ 与度规 $g$ 的乘积,得到了宇宙场方程。

\begin{theorem}{爱因斯坦宇宙场方程}
$$G_{\mu \nu} + \Lambda g_{\mu \nu} = 8\pi G T_{\mu \nu} ~.$$
物理意义是宇宙真空场,修正项 $\Lambda g_{\mu \nu}$ 称为\textbf{宇宙项}。
\end{theorem}

在修正添加的 $12$ 年后,也就是 $1929$ 年,哈勃发现所有星系都有远离我们的趋势,这使得宇宙是正在膨胀的大概率是正确的,几乎印证了宇宙场方程的正确性。

$1998$ 年,通过观测在爆炸中的白矮星(Ia 型超新星),发现宇宙正在加速膨胀,再次印证了宇宙场方程的正确性。

在这种情况下,可以改写\textbf{能量-动量张量}:
\begin{equation}
{{T_{\nu}'}^{\mu}} = g_{\mu \lambda} T_{\lambda \nu}' = T_{\nu}^{\mu} + \frac{\Lambda}{8 \pi G} \delta_{\nu}^{\mu} ~.
\end{equation}
从而有对应于宇宙学常数的能量密度与ya'qian