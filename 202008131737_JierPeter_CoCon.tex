% 协变和逆变

\pentry{过渡矩阵\upref{TransM}}

协变和逆变的概念在物理学中极为普遍,它描述的是物理量随着参考系变化等变换而变换的特点.在现代物理学语言中,常常使用各种各样的线性空间来描述物理系统,其中一个向量表示一种状态,一组基底表示一种看待此系统的视角,比如参考系、表象等等,而向量的坐标则意味着在给定视角(基底)下的物理量.因此,协变和逆变实际上描述的是各种向量的坐标随着基底变换的变换特征.

简单来说,协变就是指坐标的变换矩阵和基底变换的过渡矩阵相同,而逆变就是指坐标的变换矩阵和过渡矩阵互逆.

\subsection{协变向量和逆变向量}

协变和逆变的区分,最朴素的理解可以是这样的:如果把线性空间的基向量都变成原来的两倍长,结果$\bvec{v}$的坐标分量都变成了原来的一般,那么$\bvec{v}$就是一个逆变的向量;如果$\bvec{v}$的坐标分量都变成了原来的两倍长,那么$\bvec{v}$就是一个协变的向量.由过渡矩阵\upref{TransM}的结论可知,同一个线性空间中的向量,其坐标总是按照过渡矩阵的逆矩阵变换,也就是说,总是逆变的.因此,协变和逆变的区分只有在多个不同的空间之间才有意义,这是容易混淆的点,要注意理解区别.

学生常常下意识地把同构的不同线性空间看成是同一个空间,这造成了对不同空间的数学本质的理解困难.比如说,位移和速度实际上是两个同构但不相同的空间,它们的基底确实可以进行一一对应,但这种对应也是人为设定的,并没有天然的对应逻辑.

\begin{definition}{向量的协变和逆变}
给定两个同构的线性空间$V_1$和$V_2$,它们的基底相互对应,对应方式根据实际情况来定.如果在这种关联下,当$V_1$的基底按照过渡矩阵$T$变换时,$V_2$中的某个向量按照$T$变换,那么我们称$V_2$中的向量对于$V_1$的变换是协变的;如果$V_2$中的向量按照$T^{-1}$变换,那么我们称$V_2$中的向量对于$V_1$的变换是逆变的.
\end{definition}

\begin{example}{逆变的例子}
如果$V_1$和$V_2$的\textbf{基底}始终\textbf{协变},就是说当$V_1$的过渡矩阵是$T$时,$V_2$的过渡矩阵也是$T$,那么$V_2$的\textbf{向量}对于$V_1$是\textbf{逆变}的.
\begin{itemize}
\item 任何线性空间$V$自身的向量,对于$V$本身是逆变的.
\item 设$V_1$是一维位移空间,$V_2$是一维速度空间,它们的基底之间的关联是:$x \Si{m}$永远对应$x \Si{m/s}$,那么对$V_1$的基底进行任何变换,$V_2$的过渡矩阵总和它一致,即$V_2$的基底和$V_1$的基底协变;因此$V_2$中的向量对于$V_1$是逆变的.
\end{itemize}
\end{example}

\begin{example}{协变的例子}
如果$V_1$和$V_2$的\textbf{基底}始终\textbf{逆变},就是说当$V_1$的过渡矩阵是$T$时,$V_2$的过渡矩阵也是$T$,那么$V_2$的\textbf{向量}对于$V_1$是\textbf{协变}的.
\end{example}







