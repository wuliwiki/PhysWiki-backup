% 无限深势阱中的高斯波包模拟(Matlab)
% keys 无限深势阱|高斯波包|反射|积分|波函数
% license Xiao
% type Tutor

\pentry{无限深势阱\upref{ISW}, 高斯波包\upref{GausPk}, 薛定谔方程\upref{TDSE}}

本文中我们来计算无限深势阱\upref{ISW}中一个高斯波包\upref{GausPk}的运动。 定性来说, 波包会一边移动一边扩散(变宽变矮), 且在两个势阱壁之间来回反弹。 反弹的过程中会发生干涉。

结果如\autoref{fig_wvISW_2} 和\autoref{fig_wvISW_1}, 动画见\href{https://wuli.wiki/apps/wvISW.html}{百科互动演示}。

\begin{figure}[ht]
\centering
\includegraphics[width=12cm]{./figures/7bf303d1a45addcd.png}
\caption{束缚态概率分布, $x$ 轴为束缚态的 $n$, $y$ 轴为概率, 求和约等于 1} \label{fig_wvISW_2}
\end{figure}

\begin{figure}[ht]
\centering
\includegraphics[width=14.25cm]{./figures/538fa8b732981b14.png}
\caption{波包遇到势阱壁后发生反弹, 过程中发生干涉} \label{fig_wvISW_1}
\end{figure}

\subsection{计算}

本文使用原子单位制\upref{AU}, 并假设粒子质量为 1。 假设无限深势阱的区间为 $[0, L]$, 能量的本征波函数(本征态)为(\autoref{eq_ISW_1}~\upref{ISW})
\begin{equation}
\psi _n(x) = \sqrt{\frac{2}{L}} \sin(k_n x)~,
\end{equation}
\begin{equation}
k_n = \frac{n\pi }{L}~.
\end{equation}
能量的本征值为
\begin{equation}
E_n = \frac{k_n^2}{2}~.
\end{equation}
要计算波函数接下来的变化, 用能量本征态展开波函数即可(\autoref{eq_TDSE11_4}~\upref{TDSE11})。 初始时波函数为高斯波包(\autoref{eq_GausWP_1}~\upref{GausWP})
\begin{equation}
\psi (x,0) = \frac{1}{(2\pi \sigma _x ^2)^{1/4}} \E^{-(x - x_0)^2/(2\sigma _x)^2} \E^{\I k_0x}~.
\end{equation}


第一步是把初始波函数投影到能量本征态上
\begin{equation}\label{eq_wvISW_1}
C_n = \int_0^L \psi _n^*(x) \psi (x,0) \dd{x}
= \int_0^L \sqrt{\frac{2}{L}} \sin(k_n x) \psi (x,0) \dd{x}~.
\end{equation}
那么接下来, 波函数的演化可以表示为
\begin{equation}\label{eq_wvISW_2}
\psi (x,t) = \sum _{i=0}^N C_i \E^{-\I E_n t} \psi _n(x)~.
\end{equation}
注意在数值计算中 $N$ 不能取 $+\infty$, 而是一个有限值, 这将会带来\textbf{截断误差}。 为了保证结果正确, 需要使被截去的部分在总波函数中的概率忽略不计, 即 $\sum_{i=N+1}^\infty \abs{C_i}^2 \ll 1$。 在\autoref{fig_wvISW_2} 中可以看到 $N$ 至少要大于 80 左右才能不产生过大的的截断误差。 由于高斯波包的频谱是指数衰减的, 所以随着 $N$ 继续增大, 截断误差也会指数减小。 我们把计算结果随 $N$ 变大而趋于固定的过程叫做计算的\textbf{收敛}。

若我们假设初始波包宽度足够小, 使得波函数在势阱外的函数值可以忽略不计, 则\autoref{eq_wvISW_1} 的定积分可以拓展到无穷区间, 即傅里叶变换。 我们已经知道初始高斯波包(指数)傅里叶变换的结果为\autoref{eq_GausWP_2}~\upref{GausWP}
\begin{equation}
g(k) = \frac{1}{(2\pi \sigma _p^2)^{1/4}} \E^{-(k - k_0)^2/(2\sigma _k)^2} \E^{-\I x_0(k - k_0)}~.
\end{equation}
将正弦函数记为指数的形式为
\begin{equation}
\sqrt{\frac{2}{L}} \sin(k_n x) = \I \sqrt{\frac{\pi }{L}} \qty(\frac{\E^{-\I k_n x}}{\sqrt{2\pi }} - \frac{\E^{\I k_n x}}{\sqrt{2\pi }})~,
\end{equation}
所以\autoref{eq_wvISW_1} 变为
\begin{equation}
C_n = \I \sqrt{\frac{\pi }{L}} [g(k_n) - g(-k_n)]~.
\end{equation}
代入\autoref{eq_wvISW_2} 即可。
