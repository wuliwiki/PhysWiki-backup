% 罗伯特·奥本海默(综述)
% license CCBYSA3
% type Wiki

本文根据 CC-BY-SA 协议转载翻译自维基百科\href{https://en.wikipedia.org/wiki/J._Robert_Oppenheimer}{相关文章}。

\begin{figure}[ht]
\centering
\includegraphics[width=6cm]{./figures/0ac14d0320e1b5df.png}
\caption{} \label{fig_ABHM_1}
\end{figure}
J·罗伯特·奥本海默(出生名朱利叶斯·罗伯特·奥本海默,/ˈɒpənhaɪmər/,1904年4月22日-1967年2月18日),是一位美国理论物理学家,曾在第二次世界大战期间担任曼哈顿计划洛斯阿拉莫斯实验室的主任。他因在监督第一批核武器的研发中所扮演的角色而常被称为“原子弹之父”。

奥本海默出生于纽约市,1925年在哈佛大学获得化学学位,1927年在德国哥廷根大学师从马克斯·玻恩获得物理学博士学位。在其他机构从事研究后,他加入加利福尼亚大学伯克利分校物理系,并于1936年成为正教授。奥本海默在量子力学和核物理领域作出了重要贡献,包括提出用于分子波函数的玻恩–奥本海默近似;在正电子理论、量子电动力学和量子场论方面的工作;以及核聚变中的奥本海默–菲利普斯过程。他与学生们还在天体物理领域做出了重要贡献,包括宇宙射线簇射理论,以及中子星和黑洞理论。

1942年,奥本海默被招募参与曼哈顿计划,并于1943年被任命为该计划位于新墨西哥州的洛斯阿拉莫斯实验室主任,负责研发第一批核武器。他的领导能力和科学专长对计划的成功起到了关键作用。1945年7月16日,他出席了代号“三位一体”的首次原子弹试爆。1945年8月,这些核武器在广岛和长崎的原子弹轰炸中被用于对日本作战,这也是迄今为止核武器在战争中唯一一次被使用。

1947年,奥本海默被任命为新泽西州普林斯顿高等研究院院长,并出任新成立的美国原子能委员会(AEC)总顾问委员会主席。他主张对核能和核武器实行国际管控,以避免与苏联陷入军备竞赛,后来又出于部分道德原因反对氢弹的研发。在第二次红色恐慌期间,他的这些立场,加上他过去与美国共产党有过的联系,导致1954年美国原子能委员会对其进行安全听证,并最终撤销了他的安全许可。此后,他继续从事物理学领域的讲学、写作和研究工作,并于1963年因其对理论物理的贡献获得恩里科·费米奖。2022年12月16日,美国能源部长詹妮弗·格兰霍姆撤销了1954年的决定,称当时的决定是“有缺陷的程序”的结果,并确认奥本海默一直是忠诚的。
\subsection{早年生活}
\subsubsection{童年与教育}
奥本海默于1904年4月22日出生在纽约市的一个不虔诚犹太家庭,出生名朱利叶斯·罗伯特·奥本海默\(^\text{[note 1]}\)。母亲埃拉(娘家姓弗里德曼)是一位画家,父亲朱利叶斯·塞利格曼·奥本海默是一位成功的纺织品进口商。\(^\text{[5][6]}\)罗伯特有一个弟弟弗兰克,后来也成为物理学家。\(^\text{[7]}\)他的父亲出生在普鲁士王国黑森-拿骚省仍属于哈瑙时,1888年青少年时期只身前往美国,身无分文,没有高等教育,甚至不会英语。他被一家纺织公司雇用,并在十年内成为公司高管,最终积累了财富。\(^\text{[8]}\)1912年,家族搬到纽约哈德逊高地西88街附近的河滨大道的一套公寓,那一带以豪华的府邸和联排别墅闻名。\(^\text{[6]}\)他们的艺术收藏包括巴勃罗·毕加索、爱德华·维亚尔和文森特·梵高的作品。\(^\text{[9]}\)

奥本海默最初在阿尔奎因预备学校接受教育。1911年,他进入由费利克斯·阿德勒创办的伦理文化学会学校,\(^\text{[10]}\)该校以伦理运动为基础进行教育,其校训是“行为重于信仰”。奥本海默的父亲多年是该学会成员,并担任董事会成员。\(^\text{[11]}\)奥本海默是一名兴趣广泛的学生,热衷于英语和法语文学,特别喜欢矿物学。\(^\text{[12]}\)他在一年内完成了三、四年级课程,并跳过了八年级的一半。\(^\text{[10]}\)他还向著名法国长笛演奏家乔治·巴雷尔私下学习音乐。在学业最后一年,奥本海默开始对化学产生兴趣。\(^\text{[13]}\)1921年毕业,但在捷克斯洛伐克的家族度假期间,他在雅希莫夫探矿时感染了结肠炎,因此耽误了一年继续深造。他在新墨西哥州康复,并在那里爱上了骑马和美国西南部地区。\(^\text{[14]}\)

1922年,18岁的奥本海默进入哈佛学院。他主修化学;哈佛还要求学习历史、文学、哲学或数学。为了弥补因疾病耽误的时间,他每学期修六门课程,而非通常的四门。他被接纳为本科荣誉学会Phi Beta Kappa成员,并因独立学习被授予物理学研究生资格,使他能够跳过基础课程直接修读高级课程。他因珀西·布里奇曼教授的热力学课程而被实验物理学吸引。奥本海默仅用三年时间,于1925年以优等生身份从哈佛获得文学士学位毕业。\(^\text{[15]}\)
\subsubsection{在欧洲的求学经历}
\begin{figure}[ht]
\centering
\includegraphics[width=6cm]{./figures/bfcfea0693f0bc9c.png}
\caption{1927年7月,荷兰莱顿,海克·卡末林·昂内斯实验室。奥本海默位于中排,从左数第二位} \label{fig_ABHM_2}
\end{figure}
1924年,奥本海默被剑桥大学基督学院录取后,曾致信欧内斯特·卢瑟福,请求允许他在卡文迪许实验室工作,尽管布里奇曼在推荐信中指出,奥本海默在实验室的笨拙表明,相较于实验物理,他更适合从事理论物理研究。卢瑟福对此并不感兴趣,但奥本海默仍然前往剑桥求学;\(^\text{[16]}\)最终,J.J. 汤姆森同意接收他,但条件是他必须完成基础实验课程。\(^\text{[17]}\)

奥本海默在剑桥期间非常不开心,他曾写信给一位朋友说:“我现在过得相当糟糕。实验室的工作无聊透顶,而我做得又极差,以至于完全感觉不到自己学到了任何东西。”\(^\text{[18]}\)他与导师帕特里克·布莱基特(未来的诺贝尔奖得主)关系紧张。根据奥本海默的朋友弗朗西斯·弗格森的说法,奥本海默曾坦白说他曾在布莱基特的办公桌上放过一个涂了毒药的苹果,后来奥本海默的父母说服校方不将他开除。虽然并没有关于投毒事件或留校察看的官方记录,但奥本海默曾定期在伦敦哈雷街接受精神科医生的治疗。\(^\text{[19][20][21][22]}\)

奥本海默身材高瘦,是个抽烟成瘾的人,\(^\text{[23]}\)在专注时常常忘记进食。许多朋友都说他有自我毁灭的倾向。弗格森曾试图通过讲述自己向女友弗朗西丝·基利求婚的故事,分散奥本海默明显的抑郁情绪,但奥本海默突然跳向弗格森,试图掐死他。奥本海默一生都受到抑郁症的困扰,\(^\text{[24][25]}\)曾对弟弟说:“我需要物理,甚于朋友。”\(^\text{[26]}\)

1926年,奥本海默离开剑桥,前往哥廷根大学师从马克斯·玻恩学习;当时哥廷根是全球理论物理学的中心之一。奥本海默在此结识了后来取得巨大成就的朋友,包括维尔纳·海森堡、帕斯卡尔·约尔当、沃尔夫冈·泡利、保罗·狄拉克、恩里科·费米和爱德华·泰勒。他在讨论中非常热情,以至于有时会主导整个讨论。\(^\text{[27]}\)玛丽亚·格佩特曾向玻恩递交了一份由她和其他人签署的请愿书,威胁如果玻恩不让奥本海默安静下来,他们将抵制这门课。玻恩把请愿书放在桌上,让奥本海默看见,却一句话都没说,这一举动收到了预期效果。\(^\text{[28]}\)

1927年3月,年仅23岁的奥本海默在玻恩的指导下获得了哲学博士学位。\(^\text{[29][30]}\)据说在口试结束后,主持口试的詹姆斯·弗兰克教授说道:“我很高兴终于结束了,他差点开始反过来考我。”\(^\text{[31]}\)奥本海默在欧洲期间发表了十余篇论文,包括在量子力学这一新兴领域中的许多重要贡献。他与玻恩共同发表了一篇著名论文,提出了玻恩–奥本海默近似,将分子数学处理中核的运动与电子运动分离,使得在计算时可以忽略核的运动以简化计算。这篇论文至今仍是他引用次数最多的学术成果。\(^\text{[32]}\)
\subsection{早期职业生涯}
\subsubsection{教学工作}
\begin{figure}[ht]
\centering
\includegraphics[width=8cm]{./figures/71ea0980a3ef4c69.png}
\caption{} \label{fig_ABHM_3}
\end{figure}
1927年9月,奥本海默获得了美国国家研究委员会颁发的奖学金,前往加州理工学院从事研究。布里奇曼也希望他能留在哈佛,因此达成了一个折中方案,即在1927–1928学年中,他将奖学金时间分为两段,1927年在哈佛,1928年前往加州理工学院。\(^\text{[33]}\)

在加州理工期间,奥本海默与莱纳斯·鲍林建立了密切的友谊;他们计划联合攻克化学键本质这一领域,鲍林是该领域的先驱,计划由奥本海默提供数学支持,鲍林负责解释结果。然而,这一合作以及两人的友谊在奥本海默邀请鲍林的妻子艾娃·海伦·鲍林一起前往墨西哥幽会后宣告结束。\(^\text{[34]}\)

后来,奥本海默邀请鲍林出任曼哈顿计划化学部主任,但鲍林以自己是和平主义者为由拒绝了邀请。\(^\text{[35]}\)

1928年秋,奥本海默访问了荷兰莱顿大学保罗·厄恩费斯特的研究所,他在几乎没有语言经验的情况下,用荷兰语讲课给大家留下了深刻印象。在那里,他获得了“Opje”的绰号,\(^\text{[36]}\)后来他的学生将其英文化为“Oppie”。\(^\text{[37]}\)从莱顿,他继续前往瑞士联邦理工学院(ETH Zürich),与沃尔夫冈·泡利一起从事量子力学和连续谱的研究。奥本海默尊敬并喜爱泡利,可能还模仿了泡利的个人风格以及他对问题的批判性处理方式。\(^\text{[38]}\)

回到美国后,奥本海默接受了加利福尼亚大学伯克利分校的副教授职位,雷蒙德·塞耶·伯吉极力邀请他,甚至表示愿意与加州理工学院共享他。\(^\text{[35]}\)

在开始伯克利的教授职务之前,奥本海默被诊断出患有轻度肺结核,并在新墨西哥州的一处牧场上与弟弟弗兰克一起度过了几周时间,他租下了这片牧场,并最终购买了它。当他得知牧场可供租赁时,他兴奋地喊道:“热狗!”后来,他将其命名为Perro Caliente(西班牙语中的“热狗”)。\(^\text{[39]}\)后来他常说,“物理学和沙漠”是他“两个伟大的爱”。\(^\text{[40]}\)他从肺结核中恢复后返回伯克利,在那里作为顾问和合作者,与一代物理学家共同发展,这些物理学家钦佩他卓越的才智和广泛的兴趣。他的学生和同事认为他具有迷人的魅力:在私人互动中具有催眠般的吸引力,但在更公开的场合中则常显得冷淡。他的同事们对他有两种看法:一种认为他是一个冷漠且令人敬畏的天才和美学家,另一种认为他是一个自命不凡且不安的装腔作势者。\(^\text{[41]}\)他的学生几乎总是认为他是前者,模仿他的步伐、言语以及其他举止,甚至他倾向于以原始语言阅读完整的文本。\(^\text{[42]}\)汉斯·贝特曾这样评价他:

他带到教学中最重要的因素可能就是他那高雅的品味。他总是知道什么是重要的问题,这从他选择的课题中可以看出来。他真正与这些问题同在,为解决这些问题而努力,并把他的关注传达给了学生群体。在他最鼎盛时期,他的研究小组大约有八到十个研究生和六名博士后。他每天都会在办公室与这些人见面,逐一讨论学生研究问题的进展。他对一切都感兴趣,某个下午他们可能会讨论量子电动力学、宇宙射线、电子对产生和核物理等问题。\(^\text{[43]}\)

奥本海默与诺贝尔奖得主、实验物理学家欧内斯特·劳伦斯以及他的一些回旋加速器开创者紧密合作,帮助他们理解他们的机器在伯克利辐射实验室产生的数据,这些数据最终发展成了今天的劳伦斯伯克利国家实验室。\(^\text{[44]}\)1936年,伯克利大学将奥本海默晋升为正教授,年薪为3300美元(相当于2024年的75,000美元)。作为交换,伯克利要求他减少在加州理工学院的教学时间,因此达成了一个折中的方案,即伯克利每年允许他有六周时间外出,这足以让他在加州理工学院教一个学期的课程。\(^\text{[45]}\)

奥本海默曾多次试图为罗伯特·瑟伯争取伯克利的职位,但遭到了伯吉的阻挠,伯吉认为“系里一个犹太人已经够了”。\(^\text{[46]}\)
\subsubsection{科学研究}
奥本海默在理论天文学(尤其是与广义相对论和核理论相关的领域)、核物理学、光谱学和量子场论方面做出了重要研究,包括其在量子电动力学中的扩展。他也对相对论性量子力学的形式数学产生了兴趣,尽管他对其有效性表示怀疑。他的工作预测了许多后来的发现,包括中子、介子和中子星。\(^\text{[47]}\)

最初,他的主要兴趣是连续谱的理论。他的第一篇公开发表的论文发表于1926年,内容是分子带状光谱的量子理论。他开发了一种方法来计算其跃迁几率。他还计算了氢和X射线的光电效应,得出了K边的吸收系数。他的计算结果与对太阳的X射线吸收观测结果一致,但与氦的观测结果不符。多年后,人们意识到太阳主要由氢组成,他的计算是正确的。\(^\text{[48][49]}\)

奥本海默在宇宙射线簇射理论方面做出了重要贡献。他还研究了场电子发射问题。\(^\text{[50][51]}\)这项工作为量子隧穿效应的概念发展做出了贡献。\(^\text{[52]}\)1931年,他与学生哈维·霍尔共同撰写了一篇论文《光电效应的相对论理论》,\(^\text{[53]}\)在该文中,基于实验证据,他正确地反驳了保罗·狄拉克关于氢原子两能级具有相同能量的说法。随后,他的一位博士生威利斯·兰姆确定这一现象是后来被称为兰姆位移的结果,兰姆因此获得了1955年诺贝尔物理学奖。\(^\text{[47]}\)

与梅尔巴·菲利普斯(奥本海默的第一位博士生)合作,奥本海默研究了氘离子轰击下的人工放射性计算。欧内斯特·劳伦斯和埃德温·麦克米兰用氘离子轰击原子核,发现结果与乔治·伽莫夫的预测非常吻合,但当涉及到更高的能量和更重的原子核时,结果则不符合预期。1935年,奥本海默与菲利普斯共同提出了一个理论——后来称为奥本海默–菲利普斯过程——以解释这些结果。这个理论至今仍在使用。\(^\text{[55][note 3]}\)

早在1930年,奥本海默就写了一篇论文,基本上预测了正电子的存在。这是在狄拉克提出电子既可能具有正电荷也可能具有负能量的论文之后。狄拉克的论文引入了后来被称为狄拉克方程的方程式,统一了量子力学、特殊相对论和当时的新概念——电子自旋,以解释泽曼效应。\(^\text{[57]}\)在大量实验数据的支持下,奥本海默拒绝了将预测中的正电荷电子解释为质子的观点。他认为这些粒子必须与电子具有相同的质量,而实验表明质子要比电子重得多。两年后,卡尔·大卫·安德森发现了正电子,因此获得了1936年诺贝尔物理学奖。\(^\text{[58]}\)

在1930年代末,奥本海默开始对天体物理学产生兴趣,这很可能是通过与理查德·托尔曼的友谊促成的,之后他发表了一系列相关论文。在其中的第一篇论文《恒星中子核的稳定性》(1938年),\(^\text{[59]}\)他与瑟伯共同探讨了白矮星的性质。接着,他与其中一位学生乔治·沃尔科夫共同撰写了《大质量中子核的研究》,\(^\text{[60]}\)该论文证明了恒星质量有一个上限,称为托尔曼–奥本海默–沃尔科夫极限,超出该极限的恒星将无法保持中子星的稳定性,最终会发生引力崩塌。1939年,奥本海默与他的另一位学生哈特兰·斯奈德共同发表了论文《持续引力收缩》,\(^\text{[61]}\)该论文预测了后来被称为黑洞的天体的存在。继玻恩–奥本海默近似论文之后,这些论文仍是他被引用最多的作品,并且在1950年代美国天体物理学研究复兴中起到了关键作用,主要是由约翰·A·惠勒推动的。\(^\text{[62]}\)

奥本海默的论文即使按照他擅长的抽象主题的标准,也被认为很难理解。他喜欢使用优雅但极为复杂的数学技巧来展示物理原理,尽管有时因急于求成而犯数学错误,受到批评。“他的物理学很好,”他的学生斯奈德说,“但他的算术糟糕透了。”\(^\text{[47]}\)

第二次世界大战后,奥本海默仅发表了五篇科学论文,其中一篇是生物物理学方面的,且在1950年之后再也没有发表过论文。后来的诺贝尔奖得主、曾作为访问学者在1951年与他一起在高等研究院工作的穆雷·盖尔曼对此表达了这样的看法:

他没有“坐得住”,即“坐在椅子上的肉”,我知道他从未写过长篇论文或做过长时间的计算之类的事。他没有耐心去做这些;他自己的工作由一些简短的见解构成,但都是相当精彩的。不过,他激励了其他人去做事情,他的影响力是巨大的。\(^\text{[63]}\)
\subsection{私人生活与政治生涯}
\begin{figure}[ht]
\centering
\includegraphics[width=6cm]{./figures/bbe07c2cafdc3619.png}
\caption{1946年的奥本海默} \label{fig_ABHM_4}
\end{figure}
奥本海默的母亲于1931年去世,此后他与父亲的关系更加亲密。尽管父亲仍住在纽约,但他经常前往加利福尼亚探访。\(^\text{[64]}\)1937年,父亲去世,遗留下392,602美元(相当于2024年的860万美元),这笔遗产将在奥本海默和他的弟弟弗兰克之间分配。奥本海默立即立下遗嘱,将自己的遗产捐赠给加利福尼亚大学,用于研究生奖学金。\(^\text{[65]}\)
\subsubsection{政治}

在1920年代,奥本海默对世界事务保持不关注。他声称自己不看报纸或流行杂志,直到1929年华尔街股灾发生六个月后,他和欧内斯特·劳伦斯散步时才得知此事。\(^\text{[66][67]}\) 他曾说过,在1936年总统选举之前,他从未投过票。从1934年起,他对政治和国际事务变得越来越关注。1934年,他将年薪的3\%——大约100美元(相当于2024年的2400美元)——作为两年的资助,用于支持逃离纳粹德国的德国物理学家。\(^\text{[68]}\)在1934年西海岸码头工人罢工期间,他与一些学生,包括梅尔巴·菲利普斯和瑟伯,参加了码头工人的集会。\(^\text{[46]}\)

1936年西班牙内战爆发后,奥本海默为西班牙共和党事业主持募捐活动。1939年,他加入了美国民主与知识自由委员会,积极参与反对纳粹德国对犹太科学家的迫害。像那个时代的大多数自由派团体一样,该委员会后来被贴上了共产主义阵线的标签。\(^\text{[68]}\)

奥本海默许多最亲近的同事在1930年代或1940年代都曾活跃于共产党,包括他的弟弟弗兰克、弗兰克的妻子杰基、基蒂、简·塔特洛克、他的房东玛丽·埃伦·沃什本以及伯克利的几位研究生。\(^\text{[69][70][71][72]}\)奥本海默是否是共产党成员一直存在争议。卡西迪指出,他从未公开加入美国共产党(CPUSA),\(^\text{[68]}\)但海恩斯、克莱尔和瓦西里耶夫则认为,他“实际上在1930年代末是美国共产党一个隐秘成员”。\(^\text{[73]}\)从1937年到1942年,奥本海默在伯克利参与了一个他称之为“讨论小组”的组织,后来小组成员哈孔·谢瓦利耶和戈登·格里菲斯表示,这个小组是伯克利教职员工中的一个“封闭” (秘密)单位,是美国共产党的一部分。\(^\text{[76]}\)

1941年3月,联邦调查局(FBI)为奥本海默建立了档案。档案记录了他参加了1940年12月在谢瓦利耶家中举行的会议,会议上还在场的有加州共产党州秘书威廉·施奈德曼和财务主管艾萨克·福尔科夫。FBI注意到奥本海默是美国公民自由联盟的执行委员会成员,而该组织被认为是一个共产主义阵线组织。不久后,FBI将奥本海默列入其“拘留名单”,以便在国家紧急状态时进行逮捕。\(^\text{[77]}\)

1942年,奥本海默加入曼哈顿计划时,他在个人安全调查问卷中写道,他曾是“几乎所有西海岸共产主义阵线组织的成员。”\(^\text{[78]}\)许多年后,他声称不记得写过这句话,称其不真实,如果他写过类似的内容,那只是“半开玩笑的夸张”。\(^\text{[79]}\)他曾是《人民世界》的订阅者,\(^\text{[80]}\)这是共产党的一份机关报,他在1954年作证时表示:“我曾与共产主义运动有过联系。”\(^\text{[81]}\)

1953年,奥本海默是由反共文化组织文化自由大会主办的“科学与自由”会议的赞助委员会成员。\(^\text{[82]}\)

在1954年的安全许可听证会上,奥本海默否认自己是共产党员,但他认同自己是“共同行动者”,并将这一身份定义为一个同意共产主义许多目标,但不愿意盲目听从任何共产党组织命令的人。\(^\text{[83]}\)根据传记作家雷·蒙克的说法:“从一个非常实际和真实的角度来看,他是共产党的支持者。此外,考虑到他在党活动中投入的时间、精力和金钱,他是一个非常坚定的支持者。”\(^\text{[84]}\)
\subsubsection{关系与子女}
1936年,奥本海默与简·塔特洛克建立了关系,简是伯克利大学一位文学教授的女儿,也是斯坦福大学医学院的学生。两人有着相似的政治观点;她为《西方工人报》写稿,这是美国共产党的一份报纸。\(^\text{[85]}\)1939年,在一段风波不断的关系后,塔特洛克与奥本海默分手。同年8月,他遇到了凯瑟琳(“基蒂”)普宁,一位曾经的共产党成员。基蒂的第一次婚姻仅持续了几个月。她与第二任同居丈夫乔·达利特的关系从1934年持续到1937年,乔·达利特是共产党的活跃成员,并在1937年西班牙内战中丧生。\(^\text{[86]}\)

基蒂从欧洲返回美国,在宾夕法尼亚大学获得了植物学学士学位。1938年,她与医生兼医学研究员理查德·哈里森结婚,并于1939年6月随他搬到加利福尼亚州帕萨迪纳,在那里理查德成为一家当地医院的放射科主任,而她则在加利福尼亚大学洛杉矶分校注册为研究生。她和奥本海默在托尔曼的一次聚会后发生了短暂的丑闻,他们在一起过夜。1940年夏天,她在奥本海默的新墨西哥州牧场与他同住。当她怀孕时,基蒂向理查德要求离婚,理查德同意了。1940年11月1日,她在内华达州雷诺市迅速办理了离婚手续,并与奥本海默结婚。\(^\text{[87][88]}\)

他们的第一个孩子彼得于1941年5月出生,第二个孩子凯瑟琳(“托尼”)于1944年12月7日在新墨西哥州洛斯阿拉莫斯出生。\(^\text{[89]}\)在婚姻期间,奥本海默重新点燃了与塔特洛克的恋情。\(^\text{[90]}\)后来,由于塔特洛克与共产主义的关联,这段持续的联系成为了奥本海默安全审查听证会中的一个问题。\(^\text{[91]}\)

在原子弹的开发过程中,奥本海默因其过去的左翼关系而受到联邦调查局和曼哈顿计划内部安全部门的调查。1943年6月,他前往加利福尼亚探望正在遭受抑郁症困扰的塔特洛克时,受到军方安全人员的跟踪。奥本海默在她的公寓里过夜。\(^\text{[92]}\)1944年1月4日,塔特洛克自杀,奥本海默为此深感悲痛。\(^\text{[93]}\)

在洛斯阿拉莫斯,奥本海默与心理学家、朋友理查德·托尔曼的妻子鲁思·托尔曼发生了感情纠葛。此段关系在奥本海默回到东部担任高等研究院院长后结束,但在理查德于1948年8月去世后,他们重新建立了联系,直到鲁思于1957年去世,两人偶尔见面。虽然他们的信件大部分已丢失,但现存的信件反映了他们之间亲密而充满感情的关系,奥本海默称她为“我的爱”。\(^\text{[94][95]}\)
\subsubsection{神秘主义}
奥本海默的广泛兴趣有时会干扰他对科学的专注。他喜欢那些困难的事情,因为许多科学工作对他来说似乎很简单,他开始对神秘学和隐秘的事物产生兴趣。\(^\text{[97]}\)在哈佛大学学习后,他通过英文翻译开始接触经典的印度教经典。\(^\text{[98]}\)他还对学习语言感兴趣,并在1933年在伯克利大学向亚瑟·W·赖德学习梵语。\(^\text{[100][101]}\)他最终阅读了《博伽梵歌》和《梅格杜塔》等文学作品,并深入思考它们。他后来提到《博伽梵歌》是最深刻影响他人生哲学的书籍之一。\(^\text{[102][103]}\)他写信给弟弟说,《博伽梵歌》“非常简单,且十分神奇”。他后来称它为“任何已知语言中最美丽的哲学歌曲”,并将其复印本作为礼物送给朋友,还将一本个人使用的、破旧的复印本放在他桌子旁的书架上。他在指导洛斯阿拉莫斯实验室时不断引用它,并在弗兰克林·罗斯福总统的纪念服务上引用了《博伽梵歌》中的一段话。\(^\text{[104][101]}\)他给自己的车起名为“迦楼罗”,这是印度教神祇毗湿奴的坐骑鸟。\(^\text{[105]}\)

奥本海默从未在传统意义上成为印度教徒;他没有加入任何寺庙,也没有向任何神祈祷。\(^\text{[106][107]}\)他的弟弟说:“他真的被《博伽梵歌》的魅力和其中的一般智慧吸引了。”\(^\text{[106]}\)有人推测,奥本海默对印度教思想的兴趣始于他早期与尼尔斯·玻尔的交往。玻尔和奥本海默都对古代印度神话故事及其中蕴含的形而上学进行了深入的分析和批判。在一次与大卫·霍金斯的对话中,战争前,奥本海默谈论古希腊文学时说道:“我读过希腊人;我发现印度人更深刻。”\(^\text{[108]}\)奥本海默曾是《世界视野》书系的编辑委员会成员,该书系出版了多种关于哲学的书籍。\(^\text{[109]}\)在1930年代,当他在伯克利教书时,奥本海默成为了湾区一群人的一部分,这个小组由心理学家齐格弗里德·伯恩费尔德召集,讨论精神分析。\(^\text{[110]}\)

他的亲密朋友和同事伊西多尔·艾萨克·拉比曾看到奥本海默度过伯克利、洛斯阿拉莫斯和普林斯顿的岁月,他曾困惑地想:“为什么像奥本海默这样的天才人物不会发现所有值得发现的东西?”\(^\text{[111]}\)拉比反思道:

奥本海默在那些科学传统之外的领域受到了过多的教育,比如他对宗教,尤其是印度教的兴趣,这使得他对宇宙的神秘感有了深刻的体悟,这种感觉几乎像雾霾一样笼罩着他。他看待物理学非常清晰,关注已做的工作,但在边界处,他倾向于觉得存在比实际更多神秘和新奇的东西……[他转向] 远离理论物理学的艰难、粗糙的方法,进入了一个广泛直觉的神秘领域……在奥本海默身上,现实感的元素是薄弱的。然而,正是这种精神品质,正如他在言语和举止中的精致表现,是他个人魅力的基础。他从未完全表达自己,总是给人一种感觉,似乎有尚未揭示的深度敏感性和洞察力。这些可能是天生领导者的品质,似乎拥有未被利用的力量储备。\(^\text{[112]}\)

尽管如此,物理学家路易斯·阿尔瓦雷兹和杰里米·伯恩斯坦等观察者曾建议,如果奥本海默能活到足够长的时间,以便看到他的预测通过实验得到了验证,他可能会因他在引力塌缩方面的工作(涉及中子星和黑洞)而获得诺贝尔奖。\(^\text{[113][114][115]}\)回顾过去,一些物理学家和历史学家认为这是他最重要的贡献,尽管在他生前,其他科学家并未深入研究这一领域。\(^\text{[116]}\)物理学家和历史学家亚伯拉罕·佩斯曾问奥本海默他认为自己最重要的科学贡献是什么——奥本海默提到了他在电子和正电子方面的工作,而不是他在引力收缩方面的研究。\(^\text{[117]}\)奥本海默四次被提名诺贝尔物理学奖,分别是在1946年、1951年、1955年和1967年,但他从未获奖。\(^\text{[118][119]}\)
\subsection{曼哈顿计划}
\subsubsection{洛斯阿拉莫斯}
1941年10月9日,即美国加入第二次世界大战前两个月,富兰克林·D·罗斯福总统批准了紧急实施的原子弹研发计划。10月21日,欧内斯特·劳伦斯将奥本海默带入了后来被称为曼哈顿计划的项目中。亚瑟·康普顿在冶金实验室指派奥本海默接手该项目具体的原子弹设计研究工作。\(^\text{[120]}\)

1942年5月18日,\(^\text{[121]}\)国家国防研究委员会主席詹姆斯·B·康纳特(曾是奥本海默在哈佛的讲师)请求奥本海默接手快中子计算的工作,奥本海默全力投入这一任务。他获得了“快速破裂协调员”的头衔;“快速破裂”是一个技术术语,指的是原子弹中快中子链式反应的传播过程。他的第一项举措之一是在伯克利举办了一期原子弹理论暑期班。欧洲物理学家与他的学生们(包括瑟伯、埃米尔·科诺平斯基、费利克斯·布洛赫、汉斯·贝特和爱德华·泰勒)一起忙于计算制造原子弹所需完成的任务以及这些任务的先后顺序。\(^\text{[122][123]}\)
\begin{figure}[ht]
\centering
\includegraphics[width=6cm]{./figures/60e3f7ce7ff305db.png}
\caption{奥本海默在洛斯阿拉莫斯实验室的身份证照片} \label{fig_ABHM_5}
\end{figure}
1942年6月,美国陆军成立了曼哈顿工程区,以负责原子弹项目中属于军方的部分,开始将项目的责任从科学研究与开发办公室转移到军方手中。\(^\text{[124]}\)同年9月,莱斯利·R·格罗夫斯准将被任命为后来被称为“曼哈顿计划”的项目总负责人。\(^\text{[125]}\)到1942年10月12日,格罗夫斯和奥本海默已决定,为了安全和团队凝聚力,他们需要在一个偏远地区建立一个集中、保密的研究实验室。\(^\text{[126]}\)

格罗夫斯选择奥本海默来领导项目的秘密武器实验室,尽管具体做出这一决定的时间并不明确。\(^\text{[127]}\)这一决定令许多人感到意外,因为奥本海默有左翼政治观点,也没有领导大型项目的经验。格罗夫斯曾担心,由于奥本海默没有获得诺贝尔奖,可能没有足够的威望来指挥其他科学家,\(^\text{[128]}\)但格罗夫斯对奥本海默对项目实际方面的独到理解和他广博的知识印象深刻。作为一名军事工程师,格罗夫斯清楚,在这样一个涉及物理、化学、冶金、军械和工程的跨学科项目中,这种能力至关重要。格罗夫斯还在奥本海默身上察觉到许多人没有发现的东西——一种“过度的雄心”,\(^\text{[129]}\)格罗夫斯认为这种雄心将为推动项目顺利完成提供必要的动力。\(^\text{[129]}\)

奥本海默过去的政治关联并未被忽视,但在1943年7月20日,格罗夫斯下令让奥本海默“立即获得安全许可,无论你们掌握了有关奥本海默先生的何种信息。他对项目绝对是不可或缺的。”\(^\text{[130]}\)拉比认为格罗夫斯任命奥本海默是“一次真正的天才之举,而格罗夫斯通常并不被认为是天才”。\(^\text{[131]}\)

奥本海默倾向于将实验室选址在新墨西哥州,离他的牧场不远。1942年11月16日,他与格罗夫斯及其他人一起考察了一个备选地点。奥本海默担心周围高耸的悬崖会让人感到幽闭,并且有人担心可能发生洪水。随后,他建议了一个他非常熟悉的地点:新墨西哥州圣塔菲附近的一片平坦高地,这里曾是一所私立男校——洛斯阿拉莫斯牧场学校的所在地。工程师们虽然担心通往此地的道路条件差、水源有限,但总体认为这里是理想地点。\(^\text{[132]}\)

洛斯阿拉莫斯实验室最终建在这所学校的原址上,接管了学校的一些建筑,同时又匆忙新建了许多建筑。在该实验室中,奥本海默召集了当时最顶尖的一批物理学家,他称他们为“群星”。\(^\text{[133]}\)

洛斯阿拉莫斯最初计划作为一座军事实验室,奥本海默和其他研究人员本应被编入陆军服役。他甚至为自己订制了一套中校军装并参加了陆军体检,但未能通过。军医认为他体重过轻,仅有128磅(58公斤),并诊断出他长期咳嗽是由于肺结核,同时对他长期的腰骶关节疼痛表示担忧。\(^\text{[134]}\)当拉比和罗伯特·巴彻反对这一入伍计划时,征召科学家的计划最终流产。康纳特、格罗夫斯和奥本海默商定了一个折中方案,由加州大学与战争部签订合同,代为运营该实验室。\(^\text{[135]}\)

很快,事实证明奥本海默大大低估了项目的规模:洛斯阿拉莫斯的人口从1943年的几百人迅速增长到1945年的超过6000人。\(^\text{[134]}\)

起初,奥本海默在组织和管理大型团队方面遇到了一些困难,但在他永久入住洛斯阿拉莫斯后,很快学会了大规模管理的艺术。他因全面掌握项目的所有科学细节,以及努力调和科学家与军方之间不可避免的文化冲突而闻名。维克多·魏斯科普夫写道:

奥本海默在真正意义上指导了这些理论和实验研究。他在迅速把握任何主题要点方面的非凡能力是一个决定性因素;他能够熟悉工作中每个部分的基本细节。

他并非从办公室里进行指挥。他在每一个关键步骤中都在智力上和身体上亲自参与。当新的效应被测量出来时,当新的想法被提出时,他就在实验室或研讨室里。他并不是提出了许多想法或建议,虽然他有时也会这样做,但他主要的影响来自其他方面。是他持续而强烈的在场感,让我们所有人都产生了直接参与的感觉;正是这种感觉,营造出了一种独特的热情和挑战氛围,贯穿了那个地方的整个时期。\(^\text{[136]}\)
\subsubsection{原子弹设计}
\begin{figure}[ht]
\centering
\includegraphics[width=6cm]{./figures/ae94fe7ecc673fc6.png}
\caption{} \label{fig_ABHM_6}
\end{figure}
在战争的这一阶段,科学家们非常担心德国核武器计划的进展可能比曼哈顿计划更快。\(^\text{[137][138]}\)1943年5月25日,奥本海默在一封回信中回应了费米提出的使用放射性物质毒害德国食品供应的建议。奥本海默询问费米是否能够在不让太多人知情的情况下生产出足够的锶。奥本海默继续说道:“我认为除非我们能毒害足够多的食物以致杀死50万人,否则我们不应尝试实施这一计划。”\(^\text{[139]}\)

1943年,研发工作集中在一种名为“瘦子”(Thin Man)的钚枪式裂变武器上。关于钚性质的最初研究使用的是回旋加速器产生的钚-239,这种钚非常纯净,但只能以微量产生。1944年4月,当洛斯阿拉莫斯从X-10石墨反应堆收到第一批钚样品时,发现了一个问题:反应堆生产的钚中钚-240的浓度比“回旋加速器”生产的钚高出五倍,使其不适合用于枪式武器。\(^\text{[140]}\)

1944年7月,奥本海默放弃了“瘦子”的枪式设计,转而支持内爆式武器;“瘦子”的缩小版本后来成为了“小男孩”。利用化学炸药透镜,可以将未达到临界质量的可裂变材料球体压缩成更小、更密集的形态。金属仅需移动很短的距离,便能在极短时间内完成临界质量的组装。\(^\text{[141]}\)1944年8月,奥本海默对洛斯阿拉莫斯实验室进行了全面重组,以集中精力研究内爆技术。\(^\text{[142]}\)他将研发工作集中在枪式装置上,但现在的设计更简单,仅需使用高浓缩铀,由单一小组负责开发。该装置于1945年2月被命名为“小男孩”。\(^\text{[143]}\)经过大量研究工作,复杂的内爆装置设计在奥本海默的另一位学生罗伯特·克里斯蒂的推动下最终成型,并以“克里斯蒂装置”命名,\(^\text{[144]}\)并于1945年2月28日在奥本海默办公室的会议上被最终确定为“胖子”。\(^\text{[145]}\)

1945年5月,一个临时委员会成立,用于就战时和战后核能使用政策提供建议并进行报告。临时委员会成立了一个科学顾问小组,由奥本海默、亚瑟·康普顿、费米和劳伦斯组成,为其提供科学问题的咨询。在向临时委员会提交的报告中,该小组不仅提供了关于原子弹可能造成的物理影响的意见,还提供了对其军事和政治影响的看法。\(^\text{[146]}\)这其中包括了关于是否应在对日本使用原子弹前提前告知苏联等敏感问题的讨论意见。\(^\text{[147]}\)
\subsubsection{三位一体核试验}
\begin{figure}[ht]
\centering
\includegraphics[width=8cm]{./figures/acb33303482d0ffb.png}
\caption{三位一体试验是第一次核装置的引爆。[148]} \label{fig_ABHM_7}
\end{figure}
1945年7月16日凌晨,在新墨西哥州阿拉莫戈多附近,洛斯阿拉莫斯的工作最终以世界上第一颗核武器的试验告一段落。奥本海默在1944年中期将该试验场代号命名为“三位一体(Trinity)”,后来他表示,这个名字来源于约翰·多恩的《神圣十四行诗》;他是在1930年代由简·塔特洛克介绍给他多恩的作品的,而塔特洛克已于1944年1月自杀身亡。\(^\text{[149][150]}\)

与奥本海默一起在控制掩体中的托马斯·法瑞尔准将回忆道:

“奥本海默博士肩负着沉重的责任,随着最后几秒的流逝,他变得越来越紧张。他几乎屏住了呼吸,抓住一根柱子以保持身体稳定。在最后几秒钟里,他直视前方,当播报员喊出‘现在!’后,随之而来的是一阵巨大的闪光,不久之后传来了爆炸的低沉轰鸣声,这时他的脸上终于露出了极度释然的表情。”\(^\text{[151]}\)

奥本海默的弟弟弗兰克回忆说,奥本海默当时说的第一句话是:“我猜,它成功了。”\(^\text{[152][153]}\)
\begin{figure}[ht]
\centering
\includegraphics[width=6cm]{./figures/9c69f47b76d3236c.png}
\caption{奥本海默与格罗夫斯站在“三位一体”核试验塔的残骸前。奥本海默戴着他标志性的宽檐帽;白色防护鞋用于防止放射性尘埃污染。\(^\text{[154]}\)} \label{fig_ABHM_8}
\end{figure}
据1949年的一篇杂志报道,在目睹爆炸时,奥本海默想起了《博伽梵歌》中的诗句:“如果千个太阳的光辉同时在天空中爆发,那将如同那位伟大的存在的辉煌……如今我已化作死亡,世界的毁灭者。”\(^\text{[155]}\)1965年,他这样回忆那一刻:

“我们知道,世界将不再相同。有几个人笑了,有几个人哭了,大多数人保持沉默。我想起了印度教经典《博伽梵歌》中的一句话;毗湿奴正试图说服王子履行自己的职责,并为了给他留下深刻印象,显现出他的多臂形态,说道:‘如今我已化作死亡,世界的毁灭者。’我想我们都或多或少地有过这样的想法。”\(^\text{[156][注5]}\)

拉比后来描述看到奥本海默时说道:“我永远不会忘记他走路的样子……就像《正午》电影里的场景……那种昂首阔步的姿态。他做到了。”\(^\text{[163]}\)尽管许多科学家反对对日本使用原子弹,康普顿、费米和奥本海默都认为试爆并不能说服日本投降。\(^\text{[164]}\)1945年8月6日广岛原子弹爆炸当晚,在洛斯阿拉莫斯的一次集会上,奥本海默走上讲台,双手合握在一起,“就像一个获胜的拳击手”,台下观众欢呼。他表达了遗憾,表示武器准备得太晚,无法用于对付纳粹德国。\(^\text{[165]}\)

然而,8月17日,奥本海默前往华盛顿,亲手将一封信交给战争部长亨利·L·史汀生,表达了他对核武器的憎恶以及希望看到核武器被禁止的愿望。\(^\text{[166]}\)同年10月,他会见了哈里·S·杜鲁门总统,但杜鲁门对奥本海默关于可能与苏联展开军备竞赛的担忧,以及关于原子能应当由国际控制的观点不以为然。当奥本海默说出“总统先生,我觉得我的手上沾满了鲜血”时,杜鲁门勃然大怒,回应说使用原子弹对付日本的决定完全由他(杜鲁门)负责,并且后来表示:“我再也不想在办公室里见到那个狗娘养的。”\(^\text{[167][168]}\)

因担任洛斯阿拉莫斯主任的贡献,奥本海默于1946年被杜鲁门授予“优异奖章”。\(^\text{[169]}\)
\subsection{战后活动}
在广岛和长崎遭受轰炸后,曼哈顿计划公之于众,奥本海默——这位“原子弹之父”突然成为家喻户晓的人物,成为科学的全国性代言人,象征着一种新型的技术官僚权力;\(^\text{[93][170][171]}\)他登上了《生活》和《时代》杂志的封面。\(^\text{[172][173]}\)随着各国意识到原子武器所赋予的战略和政治力量,核物理学成为一股强大的力量。和他那一代的许多科学家一样,奥本海默认为,要想获得原子弹威胁下的安全,唯一途径是依靠诸如新成立的联合国这样的跨国组织,通过实施项目来遏制核军备竞赛。\(^\text{[174]}\)
\subsubsection{高等研究院}
\begin{figure}[ht]
\centering
\includegraphics[width=8cm]{./figures/e5c33c15f220cc26.png}
\caption{奥本海默与阿尔伯特·爱因斯坦曾是同事,并且彼此关系融洽。约摄于1950年。} \label{fig_ABHM_9}
\end{figure}
1945年11月,奥本海默离开洛斯阿拉莫斯返回加州理工学院,\(^\text{[175]}\)但很快发现自己已无心再继续教学工作。\(^\text{[176]}\)1947年,他接受了刘易斯·施特劳斯的邀请,担任新泽西州普林斯顿高等研究院院长。这意味着他需要返回美国东部,同时离开鲁思·托尔曼(他的朋友理查德·托尔曼的妻子),在离开洛斯阿拉莫斯后,他曾与鲁思有过一段感情关系。\(^\text{[177]}\)这份工作年薪为20,000美元,另外还附带院长公馆的免费住宿使用权,该公馆是一栋17世纪的庄园,配有厨师和园丁,周围环绕着265英亩(107公顷)的林地。\(^\text{[178]}\)他收藏了欧洲家具,以及法国后印象派和野兽派的艺术品。他的艺术收藏包括塞尚、德兰、德斯皮奥、弗拉曼克、毕加索、伦勃朗、雷诺阿、梵高和维亚尔的作品。\(^\text{[179]}\)

奥本海默汇聚了处于巅峰状态、来自各个学科领域的知识分子,共同探讨当时最重要的问题。他指导并鼓励了许多著名科学家的研究,包括弗里曼·戴森,以及因发现宇称不守恒而获得诺贝尔奖的杨振宁和李政道。此外,他还为人文学者设立了临时会员名额,例如T.S.艾略特和乔治·F·凯南。然而,这些活动引起了少数数学系成员的不满,他们希望研究院能够保持纯粹科学研究的堡垒。亚伯拉罕·佩斯曾说,奥本海默自己也认为,他在高等研究院的一项失败就是未能成功将自然科学与人文学科的学者真正融合在一起。\(^\text{[180]}\)

在纽约举行的一系列会议中——1947年的庇护岛会议、1948年的波科诺会议以及1949年的奥德斯通会议——物理学家们从战时研究重新转回到理论问题的探讨。在奥本海默的领导下,物理学家们着手解决战前最大尚未解决的问题:量子电动力学中关于基本粒子的无穷、发散且看似无意义的表达式问题。朱利安·施温格、理查德·费曼和朝永振一郎共同攻克了重整化问题,发展出了后来被称为重整化的技术。弗里曼·戴森证明了他们的方法可以得到相似的结果。同时,介子吸收问题和汤川秀树关于介子作为强核力传递粒子的理论也被进一步探讨。奥本海默提出的深入问题促使罗伯特·马尔沙克提出了创新性的“两介子假说”:即实际上存在两种类型的介子,即π介子和μ介子。这直接促成了塞西尔·弗兰克·鲍威尔的重大突破,并因发现π介子获得了诺贝尔奖。\(^\text{[181][注6]}\)

奥本海默一直担任高等研究院院长至1966年,后因健康状况恶化而辞去该职务。\(^\text{[183]}\)截至2023年,他仍是高等研究院任职时间最长的院长。\(^\text{[184]}\)
\subsubsection{原子能委员会}
\begin{figure}[ht]
\centering
\includegraphics[width=8cm]{./figures/eb6518ba82bc8235.png}
\caption{} \label{fig_ABHM_10}
\end{figure}
作为杜鲁门任命的委员会顾问委员会成员,奥本海默对1946年《艾奇逊–李连塔尔报告》产生了重要影响。在这份报告中,委员会主张成立一个国际原子能开发局(Atomic Development Authority),由其拥有所有可裂变材料及其生产手段,如矿山、实验室,以及用于和平能源生产的核电站。伯纳德·巴鲁克被任命负责将该报告转化为提交给联合国的提案,从而形成了1946年的“巴鲁克计划”。“巴鲁克计划”增加了许多关于执行的额外条款,尤其要求检查苏联的铀资源。这被视为试图维持美国核垄断的举措,遭到苏联拒绝。至此,奥本海默清楚地意识到,由于美国和苏联之间的相互猜疑,\(^\text{[185]}\)(甚至连奥本海默自己也开始对苏联失去信任),\(^\text{[186]}\)军备竞赛已不可避免。

1947年,原子能委员会(AEC)作为负责核研究和核武器事务的民用机构成立后,奥本海默被任命为其总顾问委员会(GAC)主席。在这一职位上,他就多个与核相关的问题提供了建议,包括项目资金、实验室建设甚至国际政策,尽管GAC的建议并不总是被采纳。\(^\text{[187]}\)作为GAC主席,奥本海默积极游说推动国际军备控制和基础科学研究资金的支持,并试图将政策引导至避免激烈军备竞赛的方向。\(^\text{[188]}\)

1949年8月,苏联首次原子弹试验成功,比美国预期的时间更早。在接下来的几个月里,美国政府、军方和科学界内部围绕是否应继续研制威力更强、基于核聚变的氢弹(当时称为“超级炸弹”)展开了激烈辩论。\(^\text{[189]}\)自曼哈顿计划时期起,奥本海默就已意识到热核武器的可能性,并在当时安排了少量理论研究来探讨这一可能性,但考虑到开发裂变武器的紧迫需求,未进行更多研究。\(^\text{[190]}\)战争刚结束时,奥本海默曾反对当时继续研制“超级炸弹”,理由是既没有必要,又会在使用中造成巨大的人类伤亡。\(^\text{[191][192]}\)

到了1949年10月,奥本海默和总顾问委员会(GAC)建议不要研发“超级炸弹”(氢弹)。\(^\text{[193]}\)他和其他GAC成员部分出于道德考虑,认为这种武器只能用于战略性使用,将导致数百万人的死亡:“因此,它的使用在消灭平民人口的政策上,比原子弹本身更进一步。”\(^\text{[194]}\)他们也有实际方面的顾虑,因为当时尚无可行的氢弹设计方案。\(^\text{[195]}\)关于苏联可能研制热核武器的可能性,GAC认为美国可以储备足够数量的原子弹,以便在遭受热核打击时进行报复。\(^\text{[196]}\)在这一背景下,奥本海默和其他成员担心,如果将核反应堆从生产原子弹所需材料的任务中转用于生产氢弹所需的氚等材料,将会造成机会成本的损失。\(^\text{[197][198]}\)

随后,原子能委员会(AEC)的大多数成员支持了GAC的建议,奥本海默认为反对“超级炸弹”的斗争将会取得胜利,但支持研发氢弹的倡导者在白宫进行了积极的游说。\(^\text{[199]}\)1950年1月31日,杜鲁门正式决定推进氢弹的研发,他本身也倾向于做出这一决定。\(^\text{[200]}\)奥本海默和其他GAC中反对该项目的成员,特别是詹姆斯·康纳特,感到非常失望,并考虑辞去委员会职务。\(^\text{[201]}\)但他们最终仍选择继续留任,尽管他们对氢弹的反对立场已众所周知。\(^\text{[202]}\)

1951年,泰勒与数学家斯坦尼斯瓦夫·乌拉姆共同提出了氢弹的“泰勒–乌拉姆设计”。\(^\text{[203]}\)这一新设计在技术上看起来是可行的,奥本海默随后正式同意了该武器的研发,\(^\text{[204]}\)尽管他仍在寻找质疑其测试、部署或使用的途径。\(^\text{[205]}\)他后来回忆道:

“我们在1949年的研发计划是一个令人痛苦的项目,你完全可以认为在技术上并不十分合理。因此,也可以认为,即使你能够拥有它,你也不需要它。而1951年的计划在技术上是如此‘甜美’,以至于在这方面已无可辩驳。真正的问题变成了纯粹的军事、政治和人道主义问题,即在你拥有它之后,打算如何处理它。”\(^\text{[206]}\)

1952年8月,奥本海默、康纳特和另一位曾反对氢弹决策的成员李·杜布里奇,在任期届满后一同离开了总顾问委员会(GAC)。\(^\text{[207]}\)杜鲁门拒绝再任命他们,因为他希望委员会中有更多支持氢弹研发的新声音。\(^\text{[208]}\)此外,奥本海默的多位反对者曾向杜鲁门表达过希望奥本海默离开委员会的愿望。\(^\text{[209]}\)
\subsubsection{顾问小组与研究组}
\begin{figure}[ht]
\centering
\includegraphics[width=8cm]{./figures/b92124213a1cc93d.png}
\caption{1946年洛斯阿拉莫斯关于“超级炸弹”(氢弹)的讨论会。前排坐着诺里斯·布拉德伯里、约翰·曼利、恩里科·费米和J.M.B.凯洛格。曼利身后是奥本海默(穿夹克和领带),他左侧是理查德·费曼。最左侧的陆军上校是奥利弗·海伍德。在海伍德和奥本海默之间的第三排是爱德华·泰勒。} \label{fig_ABHM_11}
\end{figure}
在1940年代后期和1950年代初期,奥本海默曾参与多个政府顾问小组和研究项目,其中一些项目将他卷入争议和权力斗争之中。[210]

1948年,奥本海默担任国防部“远程目标小组”主席,该小组由原子能委员会联络员唐纳德·F·卡朋特(Donald F. Carpenter)创建。[211] 该小组研究了核武器的军事用途,包括如何投送核武器的问题。[212] 经过一年的研究后,奥本海默在1952年春撰写了“加百列项目(Project GABRIEL)”的报告草稿,研究了核辐射尘的危害。[213] 奥本海默还是国防动员办公室科学顾问委员会的成员。[214]

1951年,奥本海默参与了“查尔斯项目”,该项目研究了建立有效防御美国本土免受原子弹袭击的空中防御系统的可能性;1952年,他又参与了后续的“东河项目”,在奥本海默的建议下,该项目建议建立一个预警系统,以便在美国城市遭受即将到来的原子弹袭击前提前一小时发出警报。[213] 这两个项目推动了1952年“林肯项目”的启动,这是一项大型研究项目,奥本海默是其中的高级科学家之一。[213] 该项目在麻省理工学院新成立的林肯实验室进行,旨在研究空中防御问题,随后催生了“林肯夏季研究小组”,奥本海默成为其中的核心人物。[215]

奥本海默和其他科学家主张将资源优先用于空中防御,而非用于大规模报复性打击能力,这立即引发了美国空军的反对,[216] 随后双方围绕奥本海默及其支持科学家,还是美国空军是否坚持一种僵化的“马奇诺防线”式哲学展开了辩论。[217] 不论如何,夏季研究小组的工作最终促成了“远程预警线”的建设。[218]

泰勒在战争期间对洛斯阿拉莫斯原子弹研发工作并不感兴趣,以至于奥本海默让他有时间自行研究氢弹项目,[219] 泰勒于1951年离开洛斯阿拉莫斯,并在1952年帮助创办了第二个实验室,即后来的劳伦斯利弗莫尔国家实验室。[220] 奥本海默曾为洛斯阿拉莫斯的历史性研究成果辩护,并反对建立第二个实验室。[221]

“远景计划)”旨在研究提升美国战术作战能力。[213] 奥本海默于1951年晚些时候加入该项目,但撰写了报告中关键的一章,挑战了战略轰炸的理论,主张使用更适合在有限战区冲突中对付敌方部队的小型战术核武器。[222] 由远程喷气式轰炸机投送的战略热核武器必然由美国空军控制,而“远景计划”的结论则建议增加美国陆军和海军在核武器领域的作用。[223] 美国空军对此立即表示强烈反对,[224] 并成功使“远景计划”报告被封存。[225]

1952年,奥本海默担任国务院裁军五人顾问小组主席,[226] 该小组最初敦促美国推迟计划中的首次氢弹试验,并寻求与苏联达成热核试验禁令,理由是避免试验可能会阻止灾难性新武器的发展,并为两国达成新的军备协议打开大门。[227] 然而,该小组在华盛顿缺乏政治盟友,氢弹“常春藤迈克”试验仍按计划进行。[226] 此后,该小组于1953年1月发布最终报告,该报告在奥本海默深刻信念的影响下,呈现出对未来的悲观展望,认为美国和苏联都无法建立有效的核优势,但双方都能对彼此造成可怕的破坏。[228]

该顾问小组的一项建议(奥本海默认为尤为重要),[229] 是建议美国政府减少保密,向美国人民更公开地说明核力量平衡的现实以及核战争的危险。[228] 这一理念在新的艾森豪威尔政府中获得了共鸣,并促成了“坦诚行动(Operation Candor)”的启动。[230] 随后,奥本海默在1953年6月发表于《外交事务》杂志(Foreign Affairs)上的一篇文章中,向美国公众阐述了关于无限扩大核武库缺乏实际用途的观点,[231][232] 并引起了美国主要报纸的关注。[233]

因此,到1953年,奥本海默再次达到了影响力的巅峰,参与了多个政府职务和项目,并接触到关键的战略计划和军力部署情况。[117] 但与此同时,他也成为战略轰炸支持者的敌人,这些人对他反对氢弹以及随后在各个领域持续表达的立场既痛恨又不信任。[234] 这种看法还伴随着他们的担忧,认为奥本海默的声望和说服力使他在政府、军方和科学界中具有危险的影响力。[235]
\subsubsection{安全审查听证会}
\begin{figure}[ht]
\centering
\includegraphics[width=8cm]{./figures/05d26b3cf5439823.png}
\caption{} \label{fig_ABHM_12}
\end{figure}
J·埃德加·胡佛领导下的联邦调查局(FBI)在战争之前就开始监视奥本海默,当时他在伯克利任教时表现出共产主义同情倾向,并且与共产党成员关系密切,包括他的妻子和弟弟。他们强烈怀疑他本人就是共产党员,这一怀疑基于窃听中党员对他的称呼或似乎将他称作共产党员的言论,以及来自党内线人的报告。[236] 自1940年代初以来,他一直受到严密监视,他的住所和办公室被安装了窃听器,电话被监听,邮件被拆封检查。[237]

1943年8月,奥本海默告诉曼哈顿计划的安全人员,有一名他不认识的乔治·埃尔滕顿曾代表苏联向洛斯阿拉莫斯的三名男子索要核机密。当在随后的采访中被追问此事时,奥本海默承认唯一向他提及此事的人是他的朋友哈孔·谢瓦利耶,一位伯克利法国语言文学教授,谢瓦利耶曾在奥本海默家中吃晚饭时私下提到过此事。[238]

联邦调查局(FBI)向奥本海默的政治对手提供了暗示其与共产党有联系的证据。这些对手包括施特劳斯(Lewis Strauss),他是原子能委员会的一名委员,长期以来因奥本海默反对氢弹的立场以及几年前在国会当众让施特劳斯难堪而对他怀有怨恨。施特劳斯曾表示反对向其他国家出口放射性同位素,而奥本海默曾回应说,这些同位素“比电子设备不重要,但又比,比如说,维生素要重要”。[239]

1949年6月7日,奥本海默在众议院非美活动调查委员会(HUAC)作证时承认,他在1930年代与美国共产党有过来往。[240] 他作证说,他的一些学生,包括大卫·玻姆、乔瓦尼·罗西·洛马尼茨、菲利普·莫里森、伯纳德·彼得斯和约瑟夫·温伯格,在与他在伯克利共事期间曾是共产党员。奥本海默的弟弟弗兰克及其妻子杰基也在HUAC作证,承认曾是美国共产党的成员。此后,弗兰克被明尼苏达大学解雇,多年无法在物理领域找到工作,被迫在科罗拉多州从事养牛业。后来,他成为一名高中物理教师,并创办了旧金山探索馆(San Francisco Exploratorium)。[72][241]

导致安全审查听证会启动的直接事件发生在1953年11月7日,[242] 当时威廉·利斯库姆·博登(William Liscum Borden),即该年早些时候之前担任美国国会原子能联合委员会执行主任的人,给胡佛写信称,“J·罗伯特·奥本海默极有可能是苏联的特工”。[243] 艾森豪威尔并未完全相信信中指控的内容,但仍觉得有必要推进调查,[244] 并于12月3日下令在奥本海默与任何政府或军事机密之间设立一堵“隔离墙”。[245]

1953年12月21日,施特劳斯告诉奥本海默,他的安全许可已被暂停,需等待一封信中列出的多项指控得到解决,同时与他讨论了通过申请终止与原子能委员会(AEC)顾问合同的方式辞职事宜。[246] 奥本海默选择不辞职,而是请求举行听证会。[247] 这些指控由AEC总经理肯尼斯·D·尼科尔斯(Kenneth D. Nichols)在信中概述。[248][249] 尼科尔斯曾高度评价奥本海默在“远程目标小组”中的工作,[211] 并表示“尽管(奥本海默)的记录如此,他依然对美国忠诚”。[250] 尽管如此,他仍起草了这封信,但后来写道,他“对信中提及奥本海默反对氢弹研发的内容并不满意”。[251]

随后于1954年4月至5月举行的听证会(秘密进行)重点关注奥本海默过去的共产党联系以及他在曼哈顿计划期间与被怀疑不忠或是共产党员的科学家的交往。[252] 随后,听证会继续审查了奥本海默反对氢弹以及在后续项目和研究小组中的立场。[253] 听证会的文字记录于1954年6月公布,[254] 其中部分内容经过删节。2014年,美国能源部公开了完整的听证会记录。[255][256]
\begin{figure}[ht]
\centering
\includegraphics[width=6cm]{./figures/3f0faaded3953c3a.png}
\caption{} \label{fig_ABHM_13}
\end{figure}
此次听证会的关键内容之一是奥本海默最初关于乔治·埃尔滕顿接触多位洛斯阿拉莫斯科学家的证词,奥本海默后来承认,他编造这一说法是为了保护他的朋友哈孔·谢瓦利耶。奥本海默并不知道,他十年前接受讯问时的两个版本都被记录了下来。在听证会上,他被出示了这些他从未有机会审阅过的记录,感到十分惊讶。实际上,奥本海默从未告诉谢瓦利耶他最终将其供出,而这一证词导致谢瓦利耶失去了工作。谢瓦利耶和埃尔滕顿都确认曾提及他们有办法将信息传递给苏联,埃尔滕顿承认曾对谢瓦利耶说过此话,谢瓦利耶也承认曾向奥本海默提过此事,但他们都表示这只是闲谈性质,否认曾有任何叛国或间谍的想法或暗示,无论是计划上还是实际行动中。两人都从未因任何罪名被定罪。[258]

爱德华·泰勒在作证时表示,他认为奥本海默对美国政府是忠诚的,但他也说:

“在许多情况下,我看到奥本海默博士的行为——或者说我理解到的奥本海默博士的行为——让我极其难以理解。我在很多问题上与他意见完全相左,他的行为在我看来坦率地说显得混乱和复杂。在这一点上,我觉得我希望国家的重大利益掌握在我更能理解、因此更能信任的人手中。从这个非常有限的意义上来说,我想表达这样一种感受:如果公共事务掌握在其他人手中,我个人会感到更安心。”[259]

泰勒的证词激怒了科学界,他几乎被学术界孤立。[260] 欧内斯特·劳伦斯以溃疡性结肠炎发作为由拒绝出庭作证,但一份他谴责奥本海默的访谈被作为证据提交。[261]

许多顶尖科学家以及政府和军方人士都出面为奥本海默作证辩护。物理学家伊西多·艾萨克·拉比表示,吊销安全许可完全没有必要:“他只是个顾问,如果你不想咨询这个人,那就不咨询他,仅此而已。”[262] 但格罗夫斯作证说,按照1954年实施的更严格的安全标准,“如果是今天,我不会批准奥本海默博士的安全许可。”[263]

在听证会结束时,委员会以2比1的投票结果撤销了奥本海默的安全许可。[264] 委员会一致认定他并不不忠于美国,但多数成员认为指控他的24项罪名中有20项属实或基本属实,并认为奥本海默构成安全风险。[265] 随后在1954年6月29日,原子能委员会(AEC)以4比1的投票结果支持了人事安全委员会的结论,施特劳斯撰写了多数意见书。[266] 在该意见书中,他强调奥本海默“性格上的缺陷”、“虚假、闪烁其词和误导性陈述”,以及他过去与共产党员和亲近共产党员人士的交往,是做出这一决定的主要原因。他没有对奥本海默的忠诚度发表评论。[267]