% 纯滚动
% 刚体|滚动|平面平行运动

% Hi,谢谢您的创作,为了更好地交流协作,百科需要申请称为志愿者才可以编辑,请先通过 wuli.wiki 申请

\pentry{刚体定轴转动\upref{RigRot} 转动惯量\upref{RigRot}}

\subsection{纯滚动}
接触面之间没有相对滑动的滚动称为\textbf{纯滚动},或称\textbf{无滑动滚动}.
对于纯滚动的刚体,与接触面相接触的点保持静止.以质量分布均匀的球在水平面上运动为例,如\autoref{Pscrol_fig1}
\begin{figure}[ht]
\centering
\includegraphics[width=7cm]{./figures/Pscrol_1.png}
\caption{纯滚动} \label{Pscrol_fig1}
\end{figure}
应该满足$v_P=\bvec 0$,并且P点处的切向加速度为0.

\subsection{纯滚动的运动学判据}
对于纯滚动的物体,有一定的约束条件.(这里讨论的是如球、圆柱之类的圆形刚体)
\begin{equation}
\begin{aligned}
v_C&=R\omega\\
a_C&=R\beta
\end{aligned}
\end{equation}
其中,$v_C,a_C$
分别是质心的速度和加速度大小,
$R$是半径,
$\omega,\beta$分别是在质心参考系下的转动角速度和角加速度.

\subsection{纯滚动中摩擦力不做功}
假设\autoref{Pscrol_fig1} 中的刚体向前位移了$\mathrm{d}\bvec x$,角位移为$\mathrm{d} \bvec \theta$,刚体所受的摩擦力为$\bvec f$,那么摩擦力做的功可以分为两部分:一部分是$\bvec f$沿$\mathrm{d} \bvec x$ 做的功,另一部分是$\bvec f$的力矩沿$\bvec \theta$做的功
\begin{equation}
\begin{aligned}
W&=\bvec f\cdot\mathrm{d} \bvec x+(\bvec R\times \bvec f)\cdot\mathrm{d} \bvec \theta \notag\\
&=-f\mathrm{d} x+fR\mathrm{d} \theta
\end{aligned}
\end{equation}
注意到$\mathrm{d} x=R\mathrm{d} \theta$

于是\begin{equation}
W=f(-\mathrm{d} x+R\mathrm{d} \theta)=0
\end{equation}
