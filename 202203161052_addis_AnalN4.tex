% Rudin 实分析与复分析笔记 1

\subsection{Chap 1. 抽象积分}
\begin{itemize}
\item (a) 对每一个复数 $z, e^{z} \neq 0$. (b) $\exp$ 的导数是它自己. (c) $\exp$ 限制在实轴上是单调增加的正函数, 且当 $x \rightarrow \infty$ 时, $\mathrm{e}^{x} \rightarrow \infty$; 当 $x \rightarrow-\infty$ 时, $\mathrm{e}^{x} \rightarrow 0$.
(d) 存在一个正数 $\pi$ 使得 $\mathrm{e}^{\I\pi / 2}=\mathrm{i}$, 并使得 $\mathrm{e}^{z}=1$ 当且仅当 $z /(2 \pi \mathrm{i})$ 是整数.(e) $\exp$ 是周期函数,其周期是 $2 \pi \mathrm{i}$. (f) 映射 $t \rightarrow \mathrm{e}^{\mathrm{i}}$ 将实轴映到单位圆上. (g)若 $w$ 是复数且 $w \neq 0$, 则存在某个 $z$ 使 $w=\mathrm{e}^{z}$.

\item 1.2 (a) 集 $X$ 的子集族 $\tau$ 称为 $X$ 上的一个\textbf{拓扑(topology)}, 若 $\tau$ 具有如下三个性质:(i) $\varnothing \in \tau$ 及 $X \in \tau$. (ii) 若 $V_{1} \in \tau, i=1, \cdots, n$, 则 $V_{1} \cap V_{2} \cap \cdots \cap V_{n} \in \tau$. (iii) 若 $\left\{V_{a}\right\}$ 是由 $\tau$ 的元素构成的集族 (有限、可数或不可数), 则 $\bigcup_{a} V_{a} \in \tau$. (b) 若 $\tau$ 是 $X$ 上的拓扑, 则称 $X$ 为一个\textbf{拓扑空间(topological space)}, 且 $\tau$ 的元素称为 $X$ 的开集. (c) 若 $X$ 和 $Y$ 为拓扑空间, 且 $f$ 是 $X$ 到 $Y$ 内的映射, 而对 $Y$ 的每一个开集 $V, f^{-1}(V)$ 是 $X$ 的开集, 则称 $f$ 为\textbf{连续的}.

\item 1.3 (a) 集 $X$ 的子集族 $\mathfrak{M}$ 称为 $X$ 的一个 \textbf{$\sigma$-代数}, 若 $\mathfrak{M}$ 具有如下性质: (i) $X \in \mathfrak{M}$. (ii) 若 $A \in \mathfrak{M}$, 则 $A^{c} \in \mathfrak{M}$, 其中 $A^{c}$ 是 $A$ 关于 $X$ 的余集. (iii) 若 $A=\bigcup_{n=1}^{\infty} A_{n}$ 且 $A_{n} \in \mathfrak{M}, n=1,2,3, \cdots$, 则 $A \in \mathfrak{M}$. (b) 若 $\mathfrak{M}$ 是 $X$ 的 $\sigma$-代数, 则称 $X$ 为一个\textbf{可测空间(measurable space)}, 且 $\mathfrak{M}$ 的元素称为 $X$ 的\textbf{可测集(measurable set)}. (c) 若 $X$ 是可测空间, $Y$ 是拓扑空间, $f$ 是 $X$ 到 $Y$ 内的映射,而对 $Y$ 的每一个开集 $V$, $f^{-1}(V)$ 是 $X$ 的可测集, 则 $f$ 称为\textbf{可测的}.

\item 若对 $f\left(x_{0}\right)$ 的每一个邻域 $V$, 对应有 $x_{0}$ 的一个邻域 $W$, 使得 $f(W) \subset V$, 则称 $X$ 到 $Y$ 内的映射 $f$ 在点 $x_{0} \in X$ \textbf{连续}.

\item 1.5 设 $X$ 和 $Y$ 是拓扑空间, $f$ 是 $X$ 到 $Y$ 内的映射. 当且仅当 $f$ 在 $X$ 的每一点连续时, 映射 $f$ 是\textbf{连续的}.

\item 1.6 (a) 空集可测, (c) 可测集的可数交可测. (d) 由于 $A-B = B^C\bigcup A$, 若 $A,B \in \mathfrak{M}$, 那么 $A-B\in \mathfrak{M}$.

\item 若 $\sigma$-代数的定义中, 只要求对有限并封闭, 则称为一个\textbf{代数}.

\item 1.7 简言之, 连续函数的连续函数是连续的; 可测函数的连续函数是可测的. 设 $Y$ 和 $Z$ 为拓扑空间, 且 $g: Y \rightarrow Z$ 是连续的. (a) 若 $X$ 是拓扑空间, $f: X \rightarrow Y$ 是连续的, 且 $h=g \circ f$, 则 $h: X \rightarrow Z$ 是连续的. (b) 若 $X$ 是可测空间, $f: X \rightarrow Y$ 是可测的, 且 $h=g \circ f$, 则 $h: X \rightarrow Z$ 是可测的. 

\item 1.8 设 $u$ 和 $v$ 是可测空间 $X$ 上的实可测函数, 设 $\Phi$ 是平面到拓扑空间 $Y$ 内的连续映射, 且对 $x \in X$ 定义 $h(x)=\Phi(u(x), v(x))$, 则 $h: X \rightarrow Y$ 是可测的.

\item 1.9 设 $X$ 是可测空间. 下面的命题是定理 1.7 和定理 1.8 的推论:
(a) 若 $u$ 和 $v$ 是 $X$ 上的实可测函数, $f=u+\mathrm{i} v$, 则 $f$ 为 $X$ 上的复可测函数. 
(b) 若 $f=u+\mathrm{i} v$ 是 $X$ 上的复可测函数, 则 $u, v,|f|$ 都是 $X$ 上的实可测函数. 
(c) 若 $f$ 及 $g$ 是 $\mathrm{X}$ 上的复可测函数, 则 $f+g$ 及 $f g$ 亦然.
(d) 若 $E$ 是 $X$ 上的可测集, 且 $\chi_{E}(x)= \begin{cases}1 & \text { 当 } x \in E, \\ 0 & \text { 当 } x \notin E,\end{cases}$, 则 $\chi_{E}$ 是可测函数. 我们称 $\chi_{E}$ 为集 $E$ 的\textbf{特征函数(characteristic function)}. 整本书中, 字母 $\chi$ 将专用于表示特征函数.
(e) 若 $f$ 为 $X$ 上的复可测函数, 则存在 $X$ 上的复可测函数 $\alpha$, 使得 $|\alpha|=1$, 且 $f=$ $\alpha|f|$.

\item 1.10 若 $\mathscr{F}$ 为 $X$ 的任意子集族, 则在 $X$ 内存在一个最小的 $\sigma$-代数 $\mathfrak{M}^{*}$, 使得 $M \subset \mathfrak{M}^{*}$. $\mathfrak{M}^{*}$ 有时称为由 $\mathscr{F}$ \textbf{生成}的 $\sigma$-代数.

\item 1.11 设 $X$ 为拓扑空间. 由定理 1.10, 在 $X$ 内存在一个最小的 $\sigma$-代数 $\mathscr B$, 使得 $X$ 内每一个开集都属于 $\mathscr B$. 就称 $\mathscr B$ 的元素为 $X$ 的\textbf{博雷尔集(Borel set)}.

\item 特别地, 每个闭集是博雷尔集 (由定义, 它是开集的余集), 且闭集的一切可数并 $F_{\sigma}$ 及开集的一切可数交 $G_{\delta}$ 都是博雷尔集. 该记号来源于豪斯多夫 (Hausdorff). 字母 $F$ 及 $G$ 分别用于闭集及开集, 而 $\sigma$ 用于并 (Summe), $\delta$ 用于交 (Durchschnitt). 例如,每个半开区间 $[a, b)$ 是 $R^{1}$ 内的一个 $G_{\delta}$ 集和一个 $F_{\sigma}$ 集.

\item 因为 $\mathscr{B}$ 是 $\sigma$-代数, 我们现在可以把 $X$ 看做一个可测空间, 而博雷尔集则起着可测集的作用. 更简洁地, 考虑可测空间 $(X, \mathscr{B})$. 若 $f: X \rightarrow Y$ 是 $X$ 的连续映射, 其中 $Y$ 为任意拓扑空间, 由定义, 显然对 $Y$ 内每个开集 $V, f^{-1}(V) \in \mathscr{B}$. 换句话说, $X$ 的每个连续映射是博雷尔可测的. 博雷尔可测映射通常称为\textbf{博雷尔映射}或\textbf{博雷尔函数}.

\item 1.12 假设 $\mathfrak{M}$ 是 $X$ 内的 $\sigma$-代数, $Y$ 为拓扑空间, $f$ 是从 $\mathrm{X}$ 到 $\mathrm{Y}$ 的一个映射.
(a) 若 $\Omega$ 为所有集 $E \subset Y$ 使得 $f^{-1}(E) \in \mathfrak{M}$ 的集族, 则 $\Omega$ 为 $Y$ 内的 $\sigma$-代数.
(b) 若 $f$ 可测且 $E$ 为 $Y$ 内的博雷尔集, 则 $f^{-1}(E) \in \mathfrak{M}$.
(c) 若 $Y=[-\infty, \infty]$, 且对每一个实数 $\alpha, f^{-1}((\alpha, \infty]) \in \mathfrak{M}$ ,则 $f$ 可测.
(d) 若 $f$ 可测, $Z$ 为拓扑空间, $g: Y \rightarrow Z$ 为博雷尔映射, 且 $h=g \circ f$, 则 $h: X \rightarrow Z$ 可测.

\item 1.16 在可测空间 $X$ 上. 值域仅由有限个点组成的复函数 $s$ 称为\textbf{简单函数}. 这里是指非负简单函数, 其值域为 $[0, \infty)$ 的有限子集. 注意, 我们从简单函数的值中明显地排除了 $\infty$. 如果 $\alpha_{1}, \cdots, \alpha_{n}$ 为简单函数 $s$ 不同的值, 且令 $A_{i}=\left\{x: s(x)=\alpha_{i}\right\}$, 显然 $s=\sum_{i=1}^{n} \alpha_{i} \chi_{A_{i}}$, 在这里 $\chi_{A_{i}}$ 按照 1.9 (d) 所定义, 是 $A_{i}$ 的特征函数.

\item 1.17 设 $f:X \rightarrow[0, \infty]$ 可测, 则存在 $X$ 上的简单可测函数 $s_{n}$ 使得 (a) $0 \leqslant s_{1} \leqslant s_{2} \leqslant \cdots \leqslant f$. (b) 对每个 $x \in X$, 当 $n \rightarrow \infty$ 时, $s_{n}(x) \rightarrow f(x)$.

\item 1.18 (a) \textbf{正测度}为一个定义在 $\sigma$-代数 $\mathfrak{M}$ 上的函数 $\mu$, 其值域在 $[0, \infty]$ 内, 并且是可数可加的. 即若 $\left\{A_{i}\right\}$ 为 $\mathfrak{M}$ 中互不相交的可数集族, 则 $\mu\left(\bigcup_{i=1}^{\infty} A_{i}\right)=\sum_{i=1}^{\infty} \mu\left(A_{i}\right)$. 为避免麻烦, 我们假设至少对一个 $A \in \mathfrak{M}, \mu(A)<\infty$.
(b) \textbf{测度空间(measure space)}是一个可测空间, 具有定义在其可测集的 $\sigma$-代数上的正测度.
(c) \textbf{复测度}是定义在一个 $\sigma$-代数上的复值可数可加函数.

\item 1.19 设 $\mu$ 为 $\sigma$-代数 $\mathfrak{M}$ 上的正测度, 则
(a) $\mu(\varnothing)=0$.
(b) 若 $A_{1}, \cdots, A_{n}$ 均为 $\mathfrak{M}$ 的两两不相交的元素, 则
$\mu\left(A_{1} \cup \cdots \cup A_{n}\right)=\mu\left(A_{1}\right)+\cdots+\mu\left(A_{n}\right) .$
(c) 若 $A, B \in \mathfrak{M}$, 则 $A \subset B$ 蕴含着 $\mu(A) \leqslant \mu(B)$.
(d) 若 $A=\bigcup_{n=1}^{\infty} A_{n}, A_{n} \in \mathfrak{M}$, 且 $A_{1} \subset A_{2} \subset A_{3} \subset \cdots$, 则当 $n \rightarrow \infty$ 时, $\mu\left(A_{n}\right) \rightarrow \mu(A)$.
(e) 若 $A=\bigcap_{n=1}^{\infty} A_{n}, A_{n} \in \mathfrak{M}$, $A_{1} \supset A_{2} \supset A_{3} \supset \cdots$, 且 $\mu\left(A_{1}\right)$ 有限, 则当 $n \rightarrow \infty$ 时, $\mu\left(A_{n}\right) \rightarrow \mu(A)$.
这些性质除 (c) 外, 对复测度也成立; (b) 称为\textbf{有限可加性}; (c) 称为\textbf{单调性}.

\item 1.20 测度空间的简单例子.
(a) $X$ 为任意集,对任意 $E \subset X$, 若 $E$ 为无穷集, 定义 $\mu(E)=\infty$, 当 $E$ 为有限集时, 令 $\mu(E)$ 为 $E$ 的点数. 这个 $\mu$ 称为 $X$ 上的\textbf{计数测度}.
(b) 固定 $x_{0} \in X$, 对任意 $E \subset X$, 若 $x_{0} \in E$, 定义 $\mu(E)=1$; 若 $x_{0} \notin E$, 定义 $\mu(E)=0$. 这 个 $\mu$ 称为集中在 $x_{0}$ 的\textbf{单位质量}.
(c) 令 $\mu$ 是集 $\{1,2,3, \cdots\}$ 上的计数测度, 令 $A_{n}=\{n, n+1, n+2, \cdots\}$, 则 $\cap A_{n}=\varnothing$, 但对 $n=1,2,3, \cdots, \mu\left(A_{n}\right)=\infty$. 这表明定理 1.19 (e) 中的假设并非多余的.

\item 1.30 我们定义 $L^{1}(\mu)$ 是所有使得 $\int_{X}|f| \mathrm{d} \mu<\infty$ 的、 $X$ 上的复可测函数 $f$ 的集族.

\item 1.31 若 $f=u+\mathrm{i} v$, 这里 $u$ 和 $v$ 是 $X$ 上的实可测函数, 且 $f \in L^{1}(\mu)$, 则对每一个可测集 $E$ 定义 $\int_{E} f \mathrm{~d} \mu=\int_{E} u^{+} \mathrm{d} \mu-\int_{E} u^{-} \mathrm{d} \mu+\mathrm{i} \int_{E} v^{+} \mathrm{d} \mu-\mathrm{i} \int_{E} v^{-} \mathrm{d} \mu$

\item 1.32 设 $f$ 和 $g \in L^{1}(\mu)$, 且 $\alpha$ 和 $\beta$ 是复数, 则 $\alpha f+\beta g \in L^{1}(\mu)$, 且 $\int_{x}(\alpha f+\beta g) \mathrm{d} \mu=\alpha \int_{x} f \mathrm{~d} \mu+\beta \int_{x} g \mathrm{~d} \mu$

\item 1.34 \textbf{勒贝格控制收敛定理}:设 $\left\{f_{n}\right\}$ 是 $X$ 上的复可测函数序列, 使得 $f(x)=\lim _{n \rightarrow \infty} f_{n}(x)$ 对每一个 $x \in X$ 成立. 若存在一个函数 $g \in L^{1}(\mu)$ 使得 $\left|f_{n}(x)\right| \leqslant g(x) \quad(n=1,2,3, \cdots ; x \in X)$, 则 $f \in L^{1}(\mu)$, $\lim _{n \rightarrow \infty} \int_{x}\left|f_{n}-f\right| \mathrm{d} \mu=0$, 并且 $\lim _{n \rightarrow \infty} \int_{X} f_{n} \mathrm{~d} \mu=\int_{X} f \mathrm{~d} \mu$.

\item 1.35 设 $P$ 是对于点 $x$ 可以具有或者不具有的一种性质. 譬如, 若 $f$ 是一个给定的函数, $P$ 可以是性质“ $f(x)>0$ ”, 若 $\left\{f_{n}\right\}$ 是给定的函数序列, $P$ 可以是性质“ $\left\{f_{n}(x)\right\}$ 收敛”. 如果 $\mu$ 是一个 $\sigma$-代数 $\mathfrak{M}$ 上的测度, $E \in \mathfrak{M}$, “$P$ 在 $E$ 上几乎处处成立” (简记为 “ $P$ 在 $E$ 上 a. e. 成立") 这句话意味着 : 存在一个 $N \in \mathfrak{M}$, 使得 $\mu(N)=0, N \subset E$, 并且 $P$ 在 $E-N$ 的每一点上成立. 当然 “几乎处处” 这个概念非常强烈地依赖于所给定的测度. 当明确要求指出测度的时侯, 我们将记作 “a. e. $[\mu]$”.

\item 1.36 设 $(X, \mathfrak{M}, \mu)$ 是一个测度空间, $\mathfrak{M}^*$  是所有这样的 $E \subset X$ 的集族, 对于 $E$ 存在集 $A$ 和 $B \in \mathfrak{M}$, 使得 $A \subset E \subset B$, 且 $\mu(B-A)=0$, 在这种情况下, 定义 $\mu(E)=\mu(A)$, 则 $\mathfrak{M}^{*}$ 是一个 $\sigma$-代数, 且 $\mu$ 是 $\mathfrak{M}^{*}$ 上的一个测度.

\item 1.38 设 $\left\{f_{n}\right\}$ 是一个在 $X$ 上几乎处处有定义的复可测函数序列, 满足 $\sum_{n=1}^{\infty} \int_{X}\left|f_{n}\right| \mathrm{d} \mu<\infty$, 则级数 $f(x)=\sum_{n=1}^{\infty} f_{n}(x)$ 对几乎所有的 $x$ 收敛, $f \in L^{1}(\mu)$, 并且 $\int_{X} f \mathrm{~d} \mu=\sum_{n=1}^{\infty} \int_{X} f_{n} \mathrm{~d} \mu$

\item 1.41 设 $\left\{E_{k}\right\}$ 是在 $X$ 内的可测集序列, 满足 $\sum_{k=1}^{\infty} \mu\left(E_{k}\right)<\infty$, 则几乎所有的 $x \in X$, 至多属于有限个集 $E_{k}$.
\end{itemize}

\subsection{Chap 2. 正博雷尔测度}
\begin{itemize}
\item 2.1 \textbf{复向量空间}, 线性空间到标量域的线性变换叫做\textbf{线性泛函}

\item 2.2 作为线性泛函的积分: 1.32 说明 $L^1(\mu)$ 对任何正测度 $\mu$ 都是一个向量空间, 而映射 $f \rightarrow \int_{x} f \mathrm{~d} \mu$ 是 $L^1(\mu)$ 上的一个线性泛函. 类似地, 如果 $g$ 是任何有界可测函数, 则映射 $f \rightarrow \int_{x} f g \mathrm{~d} \mu$ 也是 $L^1(\mu)$ 上的一个线性泛函. 某种意义上, 这是 $L^1(\mu)$ 上我们唯一感兴趣的一种泛函.

\item 另一个例子:设 $C$ 是单位区间 $I=[0,1]$ 上一切连续复函数的集. 两个连续函数的和是连续的,一个连续函数的任何标量积也是连续的. 因此, $C$ 是一个向量空间. 如果 $\Lambda f=\int_{0}^{1} f(x) \mathrm{d} x \quad(f \in C)$ 是通常的黎曼积分, 则 $\Lambda$ 显然是 $C$ 上的线性泛函; $\Lambda$ 有一个附带的有趣的性质: 它是一个\textbf{正线性泛函}. 也就是说, 当 $f \geqslant 0$ 时, 有 $\Lambda f \geqslant 0$.

\item 2.3 设 $X$ 是拓扑空间. (a) 集 $E \subset X$ 是闭的, 如果它的余集 $E^{c}$ 是开的. (因此, $\varnothing$ 和 $X$ 是闭集, 闭集的有限并是闭集,闭集的任意交是闭集. )(b)集 $E \subset X$ 的\textbf{闭包} $\bar{E}$ 是 $X$ 中包含 $E$ 的最小闭集. (下述推理证明了 $\bar{E}$ 存在: $X$ 中包含 $E$ 的所有闭子集的集族 $\Omega$ 是非空的, 因为 $X \in \Omega$. 令 $\bar{E}$ 是 $\Omega$ 的一切元素的交.) (c) 集 $K \subset X$ 是\textbf{紧的}, 如果 $K$ 的每个开覆盖包含有限子覆盖. 特别地, 如果 $X$ 本身是紧的, 则 $X$ 称为\textbf{紧空间}. (d) 点 $p \in X$ 的一个\textbf{邻域}是 $X$ 的任意一个包含 $p$ 的开子集. (使用这个词并不是十分标准; 有些人对任意包含一个含 $p$ 的开集的集使用“ $p$ 的邻域” 这个词. ) (e) $X$ 是一个\textbf{豪斯多夫空间(Hausdorff space)}, 如果其中任意不同两点有不相交的邻域. (f) $X$ 是\textbf{局部紧的(locally compact)}, 如果 $X$ 的每一点有一个邻域, 它的闭包是紧的. 每个紧空间是局部紧的.

\item 2.5 设 $X$ 是豪斯多夫空间. $K \subset X, K$ 是紧的, 且 $p \in K^{c}$, 则存在开集 $U$ 和 $W$, 使得 $p \in U, K \subset W$, 并且 $U \cap W=\varnothing$.

\item 2.6 设 $\left\{K_{a}\right\}$ 是豪斯多夫空间紧子集的集族, $\cap_{a} K_{a}=\varnothing$, 则 $\left\{K_{a}\right\}$ 族中存在有限个集, 它们的交也是空的.

\item 2.8 设 $f$ 是拓扑空间上的实函数 (或广义实函数). 如果对每个实数 $\alpha,\{x: f(x)>$ $\alpha\}$ 是开的, 则称 $f$ 为\textbf{下半连续的}. 如果对每个实数 $\alpha,\{x: f(x)<\alpha\}$ 是开的, 则称 $f$ 为\textbf{上半连续的}. 显然, 一个实函数是连续的, 当且仅当它同时是上半连续和下半连续的.

\item 2.9 拓扑空间 $X$ 上的复函数 $f$ 的\textbf{支集}是集 $\{x: f(x) \neq 0\}$ 的闭包. 记 $X$ 上支集是紧的所有连续复函数的集为 $C_{c}(X)$.

\item 2.10 设 $X$ 和 $Y$ 是拓扑空间, 并设 $f: X \rightarrow Y$ 是连续的. 若 $K$ 是 $X$ 的紧子集, 则 $f(K)$ 是紧的.

\item 2.11 在本章, 将使用下述约定. 记号 $K \prec f$ 表示 $K$ 是 $X$ 的紧子集, $f \in C_{c}(X)$, 对一切 $x \in X, 0 \leqslant f(x) \leqslant 1$, 并对一切 $x \in K, f(x)=1$. 记号 $f \prec V$ 表示 $V$ 是开集, $f \in C_{c}(X), 0 \leqslant f \leqslant 1$, 并且 $f$ 的支集含于 $V$. 记号 $\mathrm{K}\prec\mathrm{f}\prec\mathrm{V}$ 表示二者都成立.

\item 2.12 设 $X$ 是局部紧的豪斯多夫空间, $V$ 是 $X$ 中的开集, $K \subset V$, 而 $K$ 是紧集, 则存在一个 $f \in C_{c}(X)$, 使得 $K \prec f \prec V$

\item 2.14 \textbf{里斯表示定理(Riesz representation theorem)}: 设 $X$ 是局部紧的豪斯多夫空间, $\Lambda$ 是 $C_{c}(X)$ 上的正线性泛函, 则在 $X$ 内存在一个包含 $X$ 的全体博雷尔集的 $\sigma$-代数 $\mathfrak{M}$, 并存在 $\mathfrak{M}$ 上的唯一一个正测度 $\mu, \mu$ 在下述意义上表示了 $\Lambda$: 
(a) 对每个 $f \in C_{c}(X), \Lambda f=\int_{X} f \mathrm{~d} \mu$.
并有下述的附加性质:
(b) 对每个紧集 $K \subset X, \mu(K)<\infty$.
(c) 对每个 $E \in \mathfrak{M}$, 有 $\mu(E)=\inf \{\mu(V): E \subset V, V$ 是开集 $\} .$
(d) 对每个开集 $E$ 或每个 $E \in \mathfrak{M}$ 而 $\mu(E)<\infty$, 有 $\mu(E)=\sup \{\mu(K): K \subset E, K$ 是紧集 $\} .$
(e) 若 $E \in \mathfrak{M}, A \subset E$, 并且 $\mu(E)=0$, 则 $A \in \mathfrak{M}$.
为清晣起见, 让我们进一步明确一下假设中“正”字的含义: $\Lambda$ 被假定是复向是空间 $C_{c}(X)$ 上的一个线性泛函, 具有如下的附加性质, 即对每一个取值为非负实数的函数 $f, \Lambda f$ 也是非负实数. 简言之, 若 $f(X) \subset[0, \infty)$, 则 $\Lambda f \in[0, \infty)$.

\item 2.15 定义在局部紧的豪斯多夫空间 $X$ 的全体博雷尔集组成的 $\sigma$-代数上的测度 $\mu$ 称为 $X$ 上的\textbf{博雷尔测度}. 如果 $\mu$ 是正的, 并且一个博雷尔集 $E \subset X$ 具有定理 2.14 的性质 (c) 或 (d), 我们就分别称 $E$ 为\textbf{外正则}或\textbf{内正则}的. 如果 $X$ 内的每个博雷尔集同时是外正则和内正则的, 则称 $\mu$ 为\textbf{正则的}.

\item 2.16 拓扑空间中的一个集 $E$ 称为 \textbf{$\sigma$-紧的}, 如果 $E$ 是紧集的可数并. 对于测度空间 (测度为 $\mu$) 中的一个集 $E$, 如果 $E$ 是集 $E_{i}$ 的可数并, 而 $\mu\left(E_{i}\right)<\infty$, 那么称 $E$ 有 \textbf{$\sigma$-有限测度}. 例如在定理 2.14 的情况下, 每个 $\sigma$-紧集有 $\sigma$-有限测度. 另外容易看出, 如果 $E \in\mathfrak{M}$, $E$ 有 $\sigma$-有限测度, 则 $E$ 是内正则的.

\item 2.17 设 $X$ 是局部紧、 $\sigma$-紧的豪斯多夫空间. 若 $\mathfrak{M}$ 和 $\mu$ 都像 2.14 所叙述的那样, 则 $\mathfrak{M}$ 和 $\mu$ 有下述性质:(a)若 $E \in \mathfrak{M}$ 和 $\varepsilon>0$, 则存在闭集 $F$ 和开集 $V$ 使得 $F \subset E \subset V$ 且 $\mu(V-F)<\varepsilon$. (b) $\mu$ 是 $X$ 上的一个正则博雷尔测度. $\mu(B-A)=0$. (c) 若 $E \in \mathfrak{M}$, 则存在集 $A$ 和 $B$ 使得 $A$ 是一个 $F_{\sigma}$ 集, $B$ 是一个 $G_{\delta}$ 集, $A \subset E \subset B$ 且 $\mu(B-A)=0$. 作为 (c) 的推论, 我们看出每个 $E \in \mathfrak{M}$ 是一个 $F_\sigma$ 集和一个测度为 0 的集的并.

\item 2.18 设 $X$ 是局部紧的豪斯多夫空间. 其每个开集是 $\sigma$-紧的. 设 $\lambda$ 是 $X$ 上的任一个正博雷尔测度, 对每个紧集 $K$, 有 $\lambda(K)<\infty$. 则 $\lambda$ 是正则的. 注意, 每个欧氏空间 $R^{k}$ 满足现在的假设, 因为 $R^{k}$ 中的每个开集是闭球的可数并.

\item 2.19 欧氏空间

\item 2.22 若 $A \subset R^{1}$, 并且 $A$ 的每个子集都是勒贝格可测的, 则 $m(A)=0$. 推论:每个正测度集都有不可测的子集.

\item 2.24 \textbf{鲁金定理}: 设 $f$ 是 $X$ 上的复可测函数, $\mu(A)<\infty$, 若 $x \notin A$ 时, $f(x)=0$, 并且 $\varepsilon>0$, 则存在一个 $g \in C_{c}(X)$, 使得 $\mu(\{x: f(x) \neq g(x)\})<\varepsilon$. 并且, 还可以做到 $\sup _{x \in X}|g(x)| \leqslant \sup _{x \in X}|f(x)|$

\item 2.25 \textbf{维塔利-卡拉泰奥多里定理}: 设 $f \in L^{1}(\mu), f$ 是实值的, 并且 $\varepsilon>0$. 则在 $X$ 上存在 函数 $u$ 和 $v, u$ 是上半连续且有上界的, $v$ 是下半连续且有下界的, 使得 $u \leqslant f \leqslant v$, 且 $\int_{X}(v-u) \mathrm{d} \mu<\varepsilon$

\end{itemize}

\subsection{Chap 3. $L^p$-空间}

\begin{itemize}
\item 3.1 设 $\varphi$ 是定义在开区间 $(a, b)$ 上的实函数, 其中 $-\infty \leqslant a<b \leqslant \infty$. 如果对任意 $a<x<b, a<y<b$ 和 $0 \leqslant \lambda \leqslant 1$, 恒有不等式 $\varphi((1-\lambda) x+\lambda y) \leqslant(1-\lambda) \varphi(x)+\lambda \varphi(y)$, 那么称 $\varphi$ 是凸的.

\item 3.2 若 $\varphi$ 在 $(a, b)$ 上是凸的, 则 $\varphi$ 在 $(a, b)$ 上连续.

\item 在这一节里, $X$ 是任意一个具有正测度 $\mu$ 的测度空间.

\item 3.4 若 $p$ 和 $q$ 都是正实数, 使得 $p+q=p q$, 或等价地 $\frac{1}{p}+\frac{1}{q}=1$, 则称 $p$ 和 $q$ 为一对\textbf{共轭指数}. 显然有 $1<p<\infty$ 和 $1<q<\infty$. 一个重要的特殊情况是 $p=q=2$. 当 $p \rightarrow 1$ 时, $q \rightarrow \infty$. 因而也可以把 1 和 $\infty$ 看成是一对共轭指数. 许多分析学家通常不明确指出而把共轭指数记为 $p$ 和 $p^{\prime}$.

\item 3.6 若 $0<p<\infty$ 且 $f$ 是 $X$ 上的一个复可测函数, 定义 $\|f\|_{p}=\left\{\int_{\mathrm{x}}|f|^{p} \mathrm{~d} \mu\right\}^{1 / p}$, 且设 $L^{p}(\mu)$ 空间由所有满足 $\|f\|_{p}<\infty$ 的 $f$ 组成, 称 $\|f\|_{p}$ 为 $f$ 的 \textbf{$L^{p}$-范数}.

\item 若 $\mu$ 是 $R^{k}$ 上的勒贝格测度, 记 $L^{p}\left(R^{k}\right)$ 以代替 $L^{p}(\mu)$. 若 $\mu$ 是集 $A$ 上的计数测度, 习惯上用 $\ell^{p}(A)$ 记对应的 $L^{p}$-空间. 当 $A$ 可数时, 便简记为 $\ell^{p}$. $\ell^{p}$ 的一个元素可以看成是一个复序列 $x=\left\{\xi_{n}\right\}$, 并且 $\|x\|_{p}=\left\{\sum_{n=1}^{\infty}\left|\xi_{n}\right|^{p}\right\}^{1 / p}$

\item 3.7 设 $g: X \rightarrow[0, \infty]$ 是可测的, 且 $S$ 是所有使得 $\mu\left(g^{-1}((\alpha, \infty])\right)=0$ 的实数 $\alpha$ 的集. 若 $S=\varnothing$, 令 $\beta=\infty$. 若 $S \neq \varnothing$, 令 $\beta=\inf S$. 因为 $g^{-1}((\beta, \infty])=\bigcup_{n=1}^{\infty} g^{-1}\left(\left(\beta+\frac{1}{n}, \infty\right]\right)$ 并且因为零测度集的可数并有测度 0 , 可以看出 $\beta \in S$. 我们称 $\beta$ 为 $g$ 的\textbf{本性上确界}. 如果 $f$ 是 $X$ 上的复可测函数, 定义 $\|f\|_{\infty}$ 为 $|f|$ 的本性上确界, 并设 $L^{\infty}(\mu)$ 由所有满足 $\|f\|_{\infty}<\infty$ 的 $f$ 所组成. $L^{\infty}(\mu)$ 的元素有时称为 $X$ 上的\textbf{本性有界可测函数}.

\item 3.8 若 $p$ 和 $q$ 是共轭指数, $1 \leqslant p \leqslant \infty, f \in L^{p}(\mu)$ 和 $g \in L^{q}(\mu)$, 则 $f g \in L^{1}(\mu)$, 并且 $\|f g\|_{1} \leqslant\|f\|_{p}\|g\|_{q}$ 成立.

\item 3.9 假定 $1 \leqslant p \leqslant \infty$, 并且 $f, g \in L^{p}(\mu)$, 则 $f+g \in L^{p}(\mu)$, 且 $\|f+g\|_{p} \leqslant\|f\|_{p}+\|g\|_{p}$ 成立.

\item 3.11 对于 $1 \leqslant p \leqslant \infty$ 和每一个正测度 $\mu$, $L^{p}(\mu)$ 是一个完备的度量空间.

\item 3.12 若 $1 \leqslant p \leqslant \infty,\left\{f_{n}\right\}$ 是在 $L^{p}(\mu)$ 内的柯西序列, 它的极限为 $f$, 则 $\left\{f_{n}\right\}$ 存在一 个子序列, 它几乎处处点态收敛于 $f(x)$.

\item 3.13 设 $S$ 是所有使得 $\mu(\{x: s(x) \neq 0\})<\infty$ 的 $X$ 上的复可测简单函数 $s$ 的类. 若 $1 \leqslant p<\infty$, 则 $S$ 在 $L^{p}(\mu)$ 中是稠密的.

\item 3.14 对 $1 \leqslant p<\infty$, $C_{c}(X)$ 在 $L^{p}(\mu)$ 中稠密.

\item 3.16 一个局部紧豪斯多夫空间 $X$ 上的复函数 $f$, 若对每一个 $\varepsilon>0$, 存在一个紧集 $K \subset X$, 使得对所有不在 $K$ 内的 $x$, $|f(x)|<\varepsilon$ 成立, 则称 $f$ 在\textbf{无穷远点为 0}. 所有在无穷远点为 0 的 $X$ 上的连续函数的类称为 $C_{0}(X)$. 显然 $C_{c}(X) \subset C_{0}(X)$, 并且若 $X$ 是紧的, 这两个类就是重合的, 在这种情况下对它们中的任何一个记为 $C(X)$.

\item 3.17 若 $X$ 是一个局部紧豪斯多夫空间, 则 $C_{0}(X)$ 是 $C_{c}(X)$ 相对于由上确界范数 $\|f\|=\sup _{x \in X}|f(x)|$ 所定义的度量的完备化.
\end{itemize}

\subsection{Chap 4. 希尔伯特空间的初等理论}

\begin{itemize}
\item 4.1 复向量空间 $H$ 称为\textbf{内积空间} (或 $U$ 空间), 如果向量 $x$ 和 $y \in H$ 的每个序对都对应一个复数 $(x, y)$, 即所谓 $x$ 和 $y$ 的内积 (或标量积), 使得下面的法则成立. (a) $(y, x)=\overline{(x, y)}$ (横线表示取共轭复数). (b) 当 $x, y$ 和 $z \in H$ 时, $(x+y, z)=(x, z)+(y, z)$. (c) 当 $x$ 和 $y \in H, \alpha$ 是标量时, $(\alpha x, y)=\alpha(x, y)$. (d) 对所有 $x \in H,(x, x) \geqslant 0$. (e) $(x, x)=0$ 当且仅当 $x=0$. (f) $\|x\|^{2}=(x, x)$.

\item 4.2 性质 4.1 (a)-(d) 蕴含着 $|(x, y)| \leqslant\|x\|\|y\|$ 对所有 $x, y \in H$ 成立.

\item 4.3 对任意 $x$ 和 $y \in H$, 都有 $\|x+y\| \leqslant\|x\|+\|y\|$.

\item 4.4 ……就是说,$H$ 中的每一个柯西序列都收敛,那么 $H$ 就称为\textbf{希尔伯特空间}.

\item 4.5 例子 (a) 对任一固定的 $n$, 所有 $n$ 元序组 $x=\left(\xi_{1}, \cdots, \xi_{n}\right)$ 的集 $C^{n}$, 这里 $\xi_{1}, \cdots, \xi_{n}$ 是复数. 当我们像通常一样按分量定义加法和标量乘法, 并定义 $(x, y)=\sum_{j=1}^{n} \xi_{i} \bar{\eta}_{j} \quad\left(y=\left(\eta_{1}, \cdots, \eta_{n}\right)\right)$ 就成为一个希尔伯特空间.

\item (b) 如果 $\mu$ 是一个正测度, $L^{2}(\mu)$ 就是一个具有内积 $(f, g)=\int_{x} f \bar{g} \mathrm{~d} \mu$ 的希尔伯特空间. 这里右端的被积函数由定理 3.8, 是属于 $L^1(\mu)$ 的. 因此 $(f, g)$ 有确定的意义. 注意 $\|f\|=(f, f)^{1 / 2}=\left\{\int_{x}|f|^{2} \mathrm{~d} \mu\right\}^{1 / 2}=\|f\|_{2}$. $L^{2}(\mu)$ 的完备性(3.11) 表明 $L^{2}(\mu)$ 的确是一个希尔伯特空间 (要记住 $L^{2}(\mu)$ 是看做函数的等价类的空间的; 比较 3.10 节的讨论). 当 $H=L^{2}(\mu)$ 时, 不等式 4.2 和 4.3 转化为霍尔德和闵可夫斯基不等式的特例. 注意, 例 (a) 是例 (b) 的特殊情况. 在 (a) 中的测度是什么?

\item (c) $[0,1]$ 上全体连续复函数的向量空间是一个内积空间, 如果 $(f, g)=\int_{0}^{1} f(t) \overline{g(t)} \mathrm{d} t$ 但它不是希尔伯特空间.

\item 4.6 对任意固定的 $y \in H$, 映射 $x \rightarrow(x, y),\ x \rightarrow(y, x),\ x \rightarrow\|x\|$ 都是 $H$ 上的连续函数.

\item 4.8 向量空间 $V$ 中的集 $E$ 称为\textbf{凸的}, 如果它有下面的几何性质:当 $x \in E, y \in E$, 且 $0<t<1$ 时, 点 $z_{t}=(1-t) x+t y$ 也属于 $E$. 当 $t$ 从 0 变到 1 时, 我们可以设想点 $z_{t}$ 在 $V$ 中描出一条从 $x$ 到 $y$ 的直线段. 凸性要求 $E$ 包含它的任意两点之间的线段. 显然, $V$ 的每个子空间都是凸的. 同样, 如果 $E$ 是凸的, 那么它的每一个平移 $E+x=\{y+x: y \in E\}$ 也一定是凸的.

\item 4.9 若对某个 $x,y \in H,(x, y)=0$, 我们就说 $x$ 正交于 $y$, 有时记作 $x \perp y$. 由于 $(x, y)=0$ 蕴含着 $(y, x)=0$, 关系“$\perp$”是对称的. 设 $x^{\perp}$ 表示所有正交于 $x$ 的 $y \in H$ 的集;而当 $M$ 是 $H$ 的子空间时,设 $M^{\perp}$ 是所有正交于每 个 $x \in M$ 的那些 $y \in H$ 的集. 注意 $x^{\perp}$ 是 $H$ 的一个子空间, 因为 $x \perp y$ 和 $x \perp y^{\prime}$ 蕴涵着 $x \perp\left(y+y^{\prime}\right)$ 和 $x \perp \alpha y . x^{\perp}$ 也恰好是那些使连续函数 $y \rightarrow(x, y)$ 等于 0 的那些点的集. 因此, $x^{\perp}$ 是 $H$ 的闭子空间. 由于 $M^{\perp}=\bigcap_{x \in M} x^{\perp}$, $M^{\perp}$ 是闭子空间的交, 于是得知 $M^{\perp}$ 是 $H$ 的闭子空间.

\item 4.10 在希尔伯特空间 $H$ 中,每一个非空闭凸集 $E$ 都包含唯一的一个具有最小范数的元素.

\item 4.11 设 $M$ 是希尔伯特空间 $H$ 的闭子空间. (a) 对每个 $x \in H$, 有唯一的分解 $x=P x+Q x$ 其中 $P x \in M, Q x \in M^{\perp}$. (b) $P x$ 和 $Q x$ 分别是 $M$ 和 $M^{\perp}$ 中距 $x$ 最近的点. (c) 映射 $P: H \rightarrow M$ 和 $Q: H \rightarrow M^{\perp}$ 都是线性的. (d) $\|x\|^{2}=\|P x\|^{2}+\|Q x\|^{2}$.

\item 4.12 若 $L$ 是 $H$ 上的一个连续线性泛函, 则存在唯一的一个 $y \in H$, 使得 $L x=(x, y)\ \ (x \in H)$.

\item 4.13 \textbf{线性组合}指有限个向量的, \textbf{独立} 即线性无关, \textbf{张成空间} $[S]$, \textbf{规范正交}即正交归一

\item 如果 $\left\{u_{\alpha}: \alpha \in A\right\}$ 是规范正交的, 对于每个 $x \in H$, 我们都对应于指标 $A$ 上的一个复函数 $\hat{x}$, 它定义为 $\hat{x}(\alpha)=\left(x, u_{\alpha}\right) \quad(\alpha \in A)$. 有时, 称这些数 $\hat{x}(\alpha)$ 为关于集 $\left\{u_{\alpha}\right\}$ 的\textbf{傅里叶系数}.

\item 4.14 设 $\left\{u_{\alpha}: \alpha \in A\right\}$ 是 $H$ 中的规范正交集, 且 $F$ 是 $A$ 的一个有限子集, 设 $M_{F}$ 是由 $\left\{u_{\alpha}: \alpha \in F\right\}$ 张成的空间.
(a)若 $\varphi$ 是 $A$ 上的一个复函数且在 $F$ 外取值为 0 , 则存在一个向量 $y \in M_{F}$, 即 $y =\sum_{\alpha \in F} \varphi(\alpha) u_{\alpha}$ 使得 $\hat y(\alpha)=\varphi(\alpha)$ 对每个 $\alpha \in A$ 成立. 同时 $\|y\|^{2} =\sum_{\alpha \in F}|\varphi(\alpha)|^{2}$.
(b) 若 $x \in H$ 且 $s_{F}(x)=\sum_{\alpha \in F} \hat{x}(\alpha) u_{\alpha}$, 则 $\left\|x-s_{F}(x)\right\|<\|x-s\|$ 对除 $s=s_{F}(x)$ 外的所有 $s \in M_{F}$ 成立, 且 $\sum_{a \in F}|\hat{x}(\alpha)|^{2} \leqslant\|x\|^{2}$

\item \textbf{等距映射(isometry)}是保持距离的一种简单映射. 对所有 $X_{0}$ 中的点 $x_{1}, x_{2}, f\left(x_{1}\right)$ 与 $f\left(x_{2}\right)$ 在 $Y$ 中的距离刚好等于 $x_{1}$ 和 $x_{2}$ 在 $X$ 中的距离.

\item 4.16 设 (a) $X$ 和 $Y$ 都是度量空间, 且 $X$ 是完备的. (b) $f: X \rightarrow Y$ 是连续的. (c) $f$ 在 $X$ 的一个稠密子集 $X_{0}$ 上是等距映射. (d) $f\left(X_{0}\right)$ 在 $Y$ 上稠密. 则 $f$ 是从 $X$ 到 $Y$ 上的等距映射. 此结论最重要的内容是 $f$ 把 $X$ 映射到整个 $Y$ 上.

\item 4.17 设 $\left\{u_{\alpha}: \alpha \in A\right\}$ 是 $H$ 中的一个规范正交集, $P$ 是向量 $u_{\alpha}$ 的所有有限线性组合构成的空间. 不等式 $\sum_{\alpha \in A}|\hat{x}(\alpha)|^{2} \leqslant\|x\|^{2}$ 对所有的 $x \in H$ 都成立, 且 $x \rightarrow \hat{x}$ 是 $H$ 到 $\ell^{2}(A)$ 上的连续线性映射, 其限制于 $P$ 的闭包 $\bar{P}$ 是 $\bar{P}$ 到 $\ell^{2}(A)$ 上的等距映射.

\item 4.18 设 $\left\{u_{\alpha}: \alpha \in A\right\}$ 是 $H$ 中的规范正交集. 下面关于 $\left\{u_{\alpha}\right\}$ 的四个条件中的每一个都蕴含着另外三个: (i) $\left\{u_{\alpha}\right\}$ 是 $H$ 的极大规范正交集. (ii) $\left\{u_{\alpha}\right\}$ 的全体有限线性组合 $P$ 在 $H$ 中稠密. (iii) 对每个 $x \in H$, 有 $\sum_{\alpha \in A}|\hat{x}(\alpha)|^{2}=\|x\|^{2}$. (iv) 若 $x \in H, y \in H$, 则 $\sum_{\alpha \in A} \hat{x}(\alpha) \overline{\hat{y}(\alpha)}=(x, y)$. 最后一个公式就是熟知的\textbf{帕塞瓦尔恒等式(Parseval's identity)}. 注意 $\hat{x},\hat{y} \in \ell^{2}(A)$, 所以 $\hat x\bar{\hat y} \in \ell^1(A)$. 从而 (iv) 中的和式的意义是明确的. 当然, (iii) 是(iv)当 $x=y$ 时的特例. 极大规范正交集通常叫做\textbf{完备规范正交集(complete orthonormal set)}或者\textbf{规范正交基(orthonormal bases)}.

\item 4.20 \textbf{偏序集(partially ordered sets)}: 集 $\mathscr{P}$ 称为由二元关系 " $\leqslant$ ” 所\textbf{偏序化(partially ordered by)}, 如果 (a) $a \leqslant b$ 和 $b \leqslant c$ 蕴含着 $a \leqslant c$. (b) 对每个 $a \in \mathscr{P}, a \leqslant a$. (c) $a \leqslant b$ 和 $b \leqslant a$ 蕴涵着 $a=b$. 偏序集 $\mathscr{P}$ 的子集 $\mathscr{Q}$ 称为\textbf{全序的(totally ordered)} 或\textbf{线性序的(linearly ordered)}, 如果每一对 $a, b \in \mathscr{Q}$ 满足 $a \leqslant b$ 或者 $b \leqslant a$.

\item 4.21 \textbf{豪斯多夫极大性定理}: 每一个非空偏序集都含有极大的全序子集.

\item 4.22 希尔伯特空间 $H$ 中的每一个规范正交集 $B$ 都包含在 $H$ 的一个极大规范正交集中.

\item 4.23 $T$ 为单位圆周, 对于 $1 \leqslant p<\infty$ 定义 $L^{p}(T)$ 为所有 $R^{1}$ 上的、勒贝格可测的、周期为 $2 \pi$ 的, 并使得范数 $\|f\|_{p}=\left\{\frac{1}{2 \pi} \int_{-\pi}^{\pi}|f(t)|^{p} \mathrm{~d} t\right\}^{1 / p}$ 为有限的复函数类.

\item $L^{\infty}(T)$ 是所有 $L^{\infty}\left(R^{1}\right)$ 中的周期为 $2 \pi$ 的元素的类, 以其本性上确界为范数. 而 $C(T)$ 由 $T$ 上全 体连续复函数组成 (或者, 等价地, 由全体 $R^{1}$ 上的周期为 $2 \pi$ 的连续复函数组成), 并具有范数 $\|f\|_{\infty}=\sup _{t}|f(t)|$

\item \textbf{三角多项式}是形如 $f(t)=a_{0}+\sum_{n=1}^{N}\left(a_{n} \cos n t+b_{n} \sin n t\right) \quad\left(t \in R^{1}\right)$ 的有限和式, 这里 $a_{0}, a_{1}, \cdots, a_{N}$ 和 $b_{1}, \cdots, b_{N}$ 是复数. 根据欧拉恒等式, 也可以写成 $f(t)=\sum_{n=-N}^{N} c_{n} \mathrm{e}^{\mathrm{i} n t}$ 的形式, 这对大多数目的来说显得更方便些. 很清楚, 每一个三角多项式都有周期 $2 \pi$. 我们用 $Z$ 表示全体整数集, 并且令 $u_{n}(t)=\mathrm{e}^{\mathrm{i} n t} \quad(n \in Z)$ 如果在 $L^{2}(T)$ 中用 $(f, g)=\frac{1}{2 \pi} \int_{-\pi}^{\pi} f(t) \overline{g(t)} \mathrm{d} t$ 来定义内积. 那么, 简单的计算表明 $\left\{u_{n} ; n \in Z\right\}$ 是 $L^{2}(T)$ 的一个规范正交集, 通常称之为\textbf{三角系}.

\item 4.25 若 $f \in C(T)$ 且 $\varepsilon>0$, 则存在一个三角多项式 $P$, 对每个实的 $t$, 有 $|f(t)-P(t)|<\varepsilon$. 一个更为精细的结果曾由 Fejér (1904) 所证明: 任何 $f \in C(T)$ 的傅里叶级数部分和的算术 平均值都一致收敛于 $f$.

\item 4.26 \textbf{傅里叶级}: 数对任意 $f \in L^{1}(T)$, 我们用公式 $\bar{f}(n)=\frac{1}{2 \pi} \int_{-\pi}^{\pi} f(t) \mathrm{e}^{-\mathrm{i} n t} \mathrm{~d} t \quad(n \in Z)$ 定义 $f$ 的\textbf{傅里叶系数}. 这样对每个 $f \in L^{1}(T)$, 都对应了 $Z$ 上的一个函数 $\hat{f}$ . $f$ 的傅里叶级数就是 $\sum_{-\infty}^{\infty} \hat{f}(n) \mathrm{e}^{\mathrm{i} n t}$ 而它的部分和是 $s_{N}(t)=\sum_{-N}^{N} \hat{f}(n) \mathrm{e}^{\mathrm{i} n t} \quad(N=0,1,2, \cdots)$. 因为 $L^{2}(T) \subset L^{1}(T)$, (1)也可以应用于每个 $f \in L^{2}(T)$. 重新叙述定理 4.17 和定理 4.18:
\textbf{里斯-费希尔}定理断言: 当 $\left\{c_{n}\right\}$ 是一个复数序列, 使 $\sum_{n=-\infty}^{\infty} \abs{c_n}^2 < \infty$ 时, 则存在一个 $f \in L^{2}(T)$ 使 $c_{n}=\frac{1}{2 \pi} \int_{-\pi}^{\pi} f(t) \mathrm{e}^{-i n t} \mathrm{~d} t \quad(n \in Z)$. 帕塞瓦尔定理断言: 当 $f \in L^{2}(T)$ 和 $g \in L^{2}(T)$ 时, $\sum_{n=-\infty}^{\infty} \hat{f}(n) \overline{\hat{g}(n)}=\frac{1}{2 \pi} \int_{-\pi}^{\pi} f(t) \overline{g(t)} \mathrm{d} t$. 
左边的级数绝对收敛, 并 $\lim _{N \rightarrow \infty}\left\|f-s_{N}\right\|_{2}=0$. 一个特殊情形是 $\left\|f-s_{N}\right\|_{2}^{2}=\sum_{|n|>N}|\hat{f}(n)|^{2}$.
\end{itemize}


\subsection{Chap 5. 巴拿赫空间技巧的例子}

\begin{itemize}
\item 5.4 对于从一个赋范线性空间 $X$ 到赋范线性空间 $Y$ 内的一个线性变换 $\Lambda$, 下面三个条件的每一个都蕴涵着另外两个: (a) $\Lambda$ 是有界的. (b) $\Lambda$ 是连续的. (c) $\Lambda$ 在 $X$ 的某个点连续.

\item 5.6 \textbf{贝尔定理(Baire's theorem)}: 若 $X$ 是完备度量空间, 则每个由 $X$ 的稠密开子集组成的可数族, 其交在 $X$ 中稠密.

\item 5.9 设 $U$ 和 $V$ 是巴拿赫空间 $X$ 和 $Y$ 的开单位球. 对 $X$ 到 $Y$ 上的每一个有界线性变换 $\Lambda$, 相应地有一个 $\delta>0$, 使 $\Lambda(U) \supset \delta V$. 请注意假设中“到 $Y$ 上”这个词!符号 $\delta V$ 表示集 $\{\delta y, y \in V\}$, 即所有 $\|y\|<\delta$ 的 $y \in$ $Y$ 的集.

\item 5.10 若 $X$ 和 $Y$ 都是巴拿赫空间, $\Lambda$ 是 $X$ 到 $Y$ 上的一一的有界线性变换, 则存在 $\delta>0$, 使 $\|\Lambda x\| \geqslant \delta\|x\| \quad(x \in X)$. 换句话说, $\Lambda^{-1}$ 是 $Y$ 到 $X$ 上的一个有界线性变换.

\item 5.11 一个收敛问题: 对每个 $f \in C(T)$, 是否有 $f$ 的傅里叶级数在每一点 $x$ 都收敛于 $f(x)$ ? 让我们回忆一下, $f$ 的傅里叶级数在点 $x$ 的第 $n$ 个部分和由 $s_{n}(f ; x)=\frac{1}{2 \pi} \int_{-\pi}^{\pi} f(t) D_{n}(x-t) \mathrm{d} t \quad(n=0,1,2, \cdots)$ 给出, 这里 $D_{n}(t)=\sum_{k=-n}^{n} \mathrm{e}^{\mathrm{i} k t}$. 问题在于判定 $\lim _{n \rightarrow \infty} s_{n}(f ; x)=f(x)$ 是否对每个 $f \in C(T)$ 和每个实数 $x$ 成立. 我们在 4.26 看到部分和按 $L^{2}$ 范数收敛于 $f$, 因此定理 3.12 蕴涵着每个 $f \in L^{2}(T)$ (从而对每个 $f \in C(T)$ 亦然) 都是整个部分和序列的某个子序列的 a. e. 点态极限. 但这并没有回答现在提出的问题. 我们将会看到, 巴拿赫一斯坦因豪斯定理否定地回答了这个问题.

\item 5.13 在不存在孤立点的完备度量空间中, 没有一个可数稠密集是 $G_{\delta}$ 集.

\item 5.17 设 $V$ 是复向量空间. (a) 若 $u$ 是 $V$ 上的复线性泛函 $f$ 的实部, 则 $f(x)=u(x)-\mathrm{i} u(\mathrm{i} x) \quad(x \in V)$ (b) 若 $u$ 是 $V$ 上的一个实线性泛函, 而 $f$ 由上式定义, 则 $f$ 是 $V$ 上的一个复线性泛函. (c) 若 $V$ 是赋范线性空间, 而 $f$ 与 $u$ 有上式的关系, 则 $\|f\|=\|u\|$.

\item 5.19 设 $M$ 是赋范线性空间 $X$ 的一个线性子空间, $x_{0} \in X$. 则 $x_{0}$ 属于 $M$ 的闭包 $\bar{M}$ 当且仅当不存在 $X$ 上的有界线性泛函 $f$ 使所有 $x \in M, f(x)=0$ 而 $f\left(x_{0}\right) \neq 0$.

\item 5.20 若 $X$ 是一个赋范线性空间, 且 $x_{0} \in X, x_{0} \neq 0$, 则存在一个 $X$ 上的范数为 1 的有界线性泛函 $f$ 使 $f\left(x_{0}\right)=\left\|x_{0}\right\|$.

\item 5.22 设 $K$ 是一个紧豪斯多夫空间, $H$ 是 $K$ 的紧子集, 并设 $A$ 是 $C(K)$ 的一个子空间, 使 $1 \in A$ ($1$ 表示对每个 $x \in X$ 对应于数 $1$ 的函数 ), 并且使 $\|f\|_{K}=\|f\|_{H} \quad(f \in A)$ 这里我们采用记号 $\|f\|_{E}=\sup \{|f(x)|: x \in E\}$. 由于 5.23 节讨论过的例子, $H$ 有时称为对应于空间 $A$ 的 $K$ 的边界.

\item 5.23 设 $U=\{z:|z|<1\}$ 是复平面上的开单位圆盘. 令 $K=\bar{U}$ (闭单位圆盘), 并取 $H$ 为 $U$ 的边界 $T$. 我们断言每个多项式 $f$, 即每个形如 $f(z)=\sum_{n=0}^{N} a_{n} z^{n}$ 的函数, 这里 $a_{0}, \cdots, a_{N}$ 是复数, 都满足关系式 $\|f\|_{\mathrm{U}}=\|f\|_{\mathrm{T}}$

\item 5.25 设 $A$ 是闭单位圆盘 $\bar{U}$ 上的连续复函数的一个向量空间. 若 $A$ 包含全体多项式, 并且对每个 $f \in A$ 有 $\sup_{z \in U}|f(z)|=\sup _{z \in T} |f(z)|$ (这里 $T$ 是单位图周, 即 $U$ 的边界), 则\textbf{泊松积分}表达式 $f(z)=\frac{1}{2 \pi} \int_{-\pi}^{\pi} \frac{1-r^{2}}{1-2 r \cos (\theta-t)+r^{2}} f\left(\mathrm{e}^{\mathrm{i} t}\right) \mathrm{d} t \quad\left(z=r \mathrm{e}^{\mathrm{i} \theta}\right)$ 对每个 $f \in A, z \in U$ 都成立.

\end{itemize}


\subsection{Chap 6. 复测度}

