% 对数与对数函数(高中)
% keys 对数|对数函数
% license Xiao
% type Tutor
\begin{issues}
\issueDraft
\end{issues}
\pentry{函数\nref{nod_functi},函数的性质\nref{nod_HsFunC},指数函数\nref{nod_HsExpF}}{nod_c094}
\subsection{引入对数运算}

与早已烂熟于胸的加、减、乘、除和乘方、开方,对数是一个陌生的运算。很多教材会从指数运算的前提下去介绍,说他是指数运算的逆运算。但对数运算本身是远早于指数运算出现的,直接从指数去介绍或许会降低陌生的感觉,但很容易造成理解上的困惑。下面会从对数运算诞生的思路去介绍它。

对数运算最早是为了算两个数的乘法而创造出来的。那时还没有计算器,小一点的数当然可以用乘法口诀,但如果想要计算两个位数特别多的数相乘,比如两个四位数相乘,列竖式的方法需要计算四次乘法和四次加法,这显然太麻烦,而且太容易出错了。

John Napier 考虑了这样一个场景:一个点沿着直线在匀速运动,另一个点在线段上运动,它运动的速度与剩下到终点的距离正相关。这样两个点是一一对应的,假设与点 $n$ 对应的另一个点是 $L_n$,线段的长度是$L$,认为在单位时间内点是匀速的,也就是单位时间内运动的长度树枝上等于这段时间内的速度$v_n$,即将到来的这一段的速度与此时剩下的距离成正比$\displaystyle v_{n}=\frac{L_{n-1}}{L}$,从而有:

\begin{equation}
L_{n}=L_{n-1}-v_{n}=L_{n-1}(1-\frac{1}{L})\implies\frac{L_{n}}{L_{n-1}}= \frac{L-1}{L}~.
\end{equation}
这个比例和$n$无关,也就是说每一段已走过的长度与上一段都是等比例变化的。设初始长度为$L_{0}=L$,从而根据现在熟悉的幂运算有:

\begin{equation}\label{eq_Ln_1}
L_n={L_{n}\over L_{n-1}}\cdot{L_{n}\over L_{n-1}}\cdots{L_{2}\over L_{1}}\cdot{L_{1}\over L_{0}}\cdot L_0=L\cdot(1 - \frac{1}{L})^n~.
\end{equation}

实际使用时,比如要计算两个数$L_n,L_m$的乘积,只需要根据他的表格找到对应的$n,m$,求和后再反查找到对应的$L_{n+m}$就可以得到计算的积了,根据指数运算法则,也就是:

\begin{equation}\label{eq_Ln_2}
L_nL_m=L^2(1 - \frac{1}{L})^{n+m}~.
\end{equation}
注意右侧的表达式,$L^2$使得右侧并不符合$L_{n+m}$的表达式$\displaystyle L(1 - \frac{1}{L})^{n+m}$。John Napier根据自己的思路提出了他的对数表,顺带还正式使用了小数点。在制作表时,为了使精度提高,他选择取$L=10^7$,得到了六位数乘法的结果,也避免了\autoref{eq_Ln_2} 中$L^2$带来的问题,因为此时只是小数点位置的区别而非数值上的区别。另外,根据\autoref{eq_Ln_1} 可以得到:

\begin{equation}
{L_n\over L}=((1-{1\over L})^{-L})^{-{n\over L}}~.
\end{equation}
选择取$L=10^7$也无意中使得,$(1-{1\over L})^{-L}$的数值非常接近$\E$,进而让他的对数表几乎就是自然对数表。而“对数”其实是“比例对数”的简称,指的就是“按照\textbf{比例}制作的,\textbf{对}着查表的\textbf{数}”。也就是上面右边的指数$n$是等式左边的数$L_n$的对数,指数运算中,结果称为真数,或者说下面介绍的对数运算中的真数,之所以称为真数,也是因为它是原本要进行乘法计算的“\textbf{真}实的\textbf{数}字”。当然,Napier实际制作表的考量很多,这里给出的只是主要思想。

\subsection{对数运算}

相信上面的例子已经让你察觉了,指数运算中的指数是真数的对数。而现代数学中的对数也不再是如当初一样查表来定义,现在对数运算的定义是基于幂运算的。

\begin{definition}{对数运算}
形如
\begin{equation}
b=\log_ac~.
\end{equation}
的运算称为对数运算。$\log$称为对数运算符。这里的变量与\autoref{def_power_1} 处是一样的。其中$a$称为\textbf{底数},$b$称为\textbf{指数},而$c$称为\textbf{真数}。另外:
\begin{itemize}
\item $\E$ 为底的对数 $\log_{\E} c$ 称为\textbf{自然对数},记作$\ln c$\footnote{在实际使用时,尽管它不叫常用对数,但它其实是理论推导时最常用的,以至于在很多大学教材或论文中如果不写底数则默认是指自然对数,即$\log c$一般是指$\ln c$};
\item $10$ 为底的对数 $\log_{10} c$ 称为\textbf{常用对数},记作$\lg c$\footnote{生活中常见的分贝、地震级别等都使用常用对数};
\item $2$ 为底的对数 $\log_{2} c$在计算机科学中经常使用 。
\end{itemize}
\end{definition}

对数符号的写法与已有的所有写法都不太像,需要习惯一下。由于上面定义中的$a,b,c$与指数运算对应,将对数代入$c=a^b$的指数之后可以发现:
\begin{equation}
a^{\log_a c}=c\qquad \log_a a^b=b~.
\end{equation}

这说明对数运算和指数运算互为逆运算。如果感觉一会儿是$c$,一会儿是$b$,与以往的逆运算有所不同,有点怪。其实,那是因为加法和乘法有交换律,而对数和指数运算没有交换律(即交换底数和真数后,运算结果不会保持不变)。可以这样理解:对于乘法和除法,$m$乘以$a$再除以$a$,原本的数$m$不变,因为两者互为逆运算,因子$a$被“抵消”了。而这里,$m$对一个数$a$做指数运算,再取$a$的对数,结果仍然是$m$,因为指数和对数互为逆运算,“消掉”了底数$a$。反过来,先对$m$取$\log a$,再作为$a$的指数运算,结果也是$m$。这说明对数是指数的逆运算,而非幂的逆运算,运算中抵消的部分是底数,计算始终围绕着指数和真数。

定义要求$a\in\mathbb (0,1)\cup(1,+\infty),c>0$,这样的要求是因为:
\begin{itemize}
\item 当底数  $a = 1$  时,指数运算$a^x$会始终等于 $1$,无法满足对数的是指数的逆运算的需求。
\item 如果底数为负数,乘方的结果在实数范围内可能会得到负数甚至复数,这样的运算结果无法在实数范围内定义。
\item 指数运算的结果总是正数,因此真数  $c$  也必须是正数。
\end{itemize}
总的说来,当前所有的限定都是为了在实数的范畴内能满足指数与对数的互逆关系。

根据指数函数可以得到,对数运算有两个特殊的计算结果:$\log_a a=1,\log_a1=0$。下面给出对数运算的法则。仍然,不需要考虑怎么证明\footnote{在高中阶段他们的证明都是基于指数运算的。},实际使用时,可以将某个变量代入幂运算的结果,可以得到倒数、减法等形式:

\begin{theorem}{对数运算法则}
\begin{itemize}
\item 乘法法则$\log_a(xy)=\log_ax+\log_ay$
\item 换底公式:$\displaystyle \log_a b=\frac{\log_cb}{\log_ca}$
\item 幂法则:$\displaystyle \frac{m}{n}\log_a b=\log_{a^n} b^m$
\end{itemize}
\end{theorem}

这三个公式非常重要,几乎所有的对数运算都离不开它们,请务必熟练使用。第一个公式就是对数内的乘法可以拆成对数外的加法。第二个公式通常会选择某一个特定的底数(如:$\ln$等)来实现,这样只要计算某个底数的就可以计算任意底数的对数了。第三个公式则意味着对数内部的幂次,可以拿到对数外面称为乘法的因子,底数在分母、真数在分子。第三个公式也可以从换底公式的角度来理解,毕竟换底后,底数和真数的位置和这里的位置相同。

\subsection{自然常数$\E$}

到这里就需要引入一个新的常数——自然常数$\E$了。他是继小学接触的$\pi$之后,在数学科目中第二个出现的常数,它们之间有很多相似之处,也有很紧密的关系。

\subsubsection{引入}

自然常数一般会通过下面这个例子来引入。假设在银行存入$1$元,银行承诺年利率为$100\%$,利息的计算公式是“$\text{利息}=\text{本金}\times\text{年利率}\times\text{存款年数(时间)}$”。下面的计算不要关注每次计算的得到的数值,而是要关注计算过程的变化整合。

最简单的情况是银行一年只结算一次利息,这时年末得到的收入就是$1\times100\%\times1=1$。这样,一年后1元会变成$1+1\times100\%=1\times(1+100\%)^1=2$元。

如果要求银行“每半年结算一次利息” ,这样计算的好处是,第一次结算之后的利息会作为本金参与到第二次的计算中。于是第一次结算时,利息为$\displaystyle1\times100\%\times\frac{1}{2}=0.5$。第二次计息时的本金变成了$\displaystyle1+1\times100\%\times\frac{1}{2}=1.5$。第二次的利息就是$\displaystyle(1+1\times100\%\times\frac{1}{2})\times100\%\times\frac{1}{2}=\frac{1.5}{2}=0.75$。于是最终的收入变成了$\displaystyle1+1\times100\%\times\frac{1}{2}+(1+1\times100\%\times\frac{1}{2})\times100\%\times\frac{1}{2}=\frac{1.5}{2}=2.25$。相比第一次多出来的那$0.25$就是因为计息次数变多带来的\footnote{因此银行也会调低短期存储的利率,以降低短期储蓄的收入来保证长期投资的权益。}。整理一下其实就是$1\times(1+\frac{100\%}{2})^2=2.25$。

可以自己试一试“每三个月一次”和“每个月一次”,整理后从银行取出的金额$A(n)$都会变成如下的形式:
\begin{equation}
A(n)=a\times\left(1+\frac{1}{n}\right)^n~.
\end{equation}
其中$a$是本金,$n$是一年计息的次数。

上面算过$A(1)=2,A(2)=2.25$,另外上面提到的“每三个月一次”和“每个月一次”分别对应$A(4)\approx 2.4414$、$A(12)\approx 2.6130$。可以看出$A(n)$似乎是不断递增的,于是不禁要问,这个函数会一直递增吗?有没有最大值?通过计算器计算可知每天一次、小时一次和每秒一次分别是$A(365)\approx 2.7146$、$A(8760)\approx2.7181$、$A(31536000)\approx 2.718266$。看上去好像逐渐停止在小于$2.72$的某个数字了,但如果仅凭这样的直觉,可能会出错。

数学家们证明了,这个函数是一直递增的,而且有一个上限,称这个函数的上限\footnote{确切的说叫上确界。}为\textbf{自然常数(Natural Constant)},也称作\textbf{欧拉数(Euler's number)},记作$\E$,定义为:
\begin{equation}
\E = \lim_{n \to \infty} \left( 1 + \frac{1}{n} \right)^n~.
\end{equation}
如果看不懂这个表达方式没问题,只需要理解它是上面那个存款过程的最终结果,是一个常数,就可以。它描述了增长速度的极限,不仅与利息有关,它还出现在很多自然现象中,简单来说,$\E$代表着在不受限制的情况下,某种东西增长到最快时能达到的程度。

\subsubsection{对比}

自然常数$\E \approx 2.71828$,它和早已在小学时就接触过的$\pi$有许多相似点。

他们都是无理数,这意味着它们不能表示为两个整数的比值。它们的小数部分是无限且不循环的,也就是说,在任何整数进制中它们都永远不会终止或重复。

他们也都是超越数,意思是它们不能作为任何\aref{有理方程的解}{sub_SolEqI_1}。换句话说,这比无理数的要求更加严格。它们不仅不能表示为整数之比,也不能通过各阶的根式表示。$\E$的超越性由查尔斯·埃尔米特(Charles Hermite)在1873年证明,$\pi$的超越性由费迪南德·冯·林德曼(Ferdinand von Lindemann)在1882年证明。

二者都可以用无穷展开的方式来表示,下面给出两个常见的展开方式\footnote{关于求和符号可以参考\enref{求和符号(高中)}{SumSym},关于阶乘可以参考\enref{阶乘(高中)。}{factor}}:
\begin{equation}
\pi=4\sum_{n=0}^\infty\frac{(-1)^i}{2i+1}~.
\end{equation}
\begin{equation}
\E=\sum_{n=0}^\infty\frac{1}{i!}~.
\end{equation}

$\E$的定义有很多种方式,除了之前提到的广为了解的极限定义。下面将给出另一个定义:$\E$ 是使得
\begin{equation}
f'(x) = f(x)~.
\end{equation}
成立的指数函数的底数,这意味着以 $\E$ 为底的指数函数是唯一的能够保持自身增长速度不变的函数。
\subsection{对数函数}

函数图如\autoref{fig_Ln_2}。
\begin{figure}[ht]
\centering
\includegraphics[width=7cm]{./figures/ce690bcbd8c28a93.png}
\caption{几种不同底的对数函数} \label{fig_Ln_2}
\end{figure}


\subsection{指数函数与对数函数的相似性}

\pentry{函数的变换(高中)\nref{nod_FunTra},导数(高中)\nref{nod_HsDerv}}{nod_a54a}

根据\aref{幂运算}{the_power_1}和对数运算的法则,任意$f(x)=a^x$都可以变形,得到:

\begin{equation}
f(x)=e^{x\ln a}~.
\end{equation}

即,所有的指数函数都可以由$\E^x$通过在$x$方向上伸缩或关于$y$轴对称($\ln a<0$时)得到,或者所有的指数函数$a^x$都可认为是$f(x)=e^x$与$g(x)=x\ln a$复合得到的$f(g(x))$。

同理,根据对数运算的性质,任意$f(x)=\log_ax$都可以变形,得到:

\begin{equation}
f(x)=\frac{1}{\ln a}\ln x\iff f(x)\ln a=\ln x~.
\end{equation}

即,所有的对数函数都可以由$\ln x$通过在$y$方向上伸缩或关于$x$轴对称($\ln a<0$时)得到,或者所有的对数函数$\log_ax$都可认为是$f(x)=\ln x$与$\displaystyle g(x)=\frac{x}{\ln a}$复合得到的$g(f(x))$。同时,这个结论也可由指数函数和对数函数在参数相同时互为反函数验证。

综上,所有的指数函数、对数函数之间都是相似的。根据上述关系,已知$\left(\E^x\right)'=\E^x,\left(\ln x\right)'=\frac{1}{x}$,根据复合函数求导法则,可知:
\begin{equation}
\left(a^x\right)'=\ln a\cdot\E^x~,\left(\log_a x\right)'=\frac{1}{x\ln a}.
\end{equation}
