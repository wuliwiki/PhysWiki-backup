% 太阳系外行星
% license CCBYSA3
% type Wiki

(本文根据 CC-BY-SA 协议转载自原搜狗科学百科对英文维基百科的翻译)

太阳系外行星,[1]又称系外行星,是位于太阳系之外的行星。太阳系外行星的第一个证据是在1917年发现的,但是并没有被验证。 第一次对太阳系外行星的科学探测是在1988年,这颗行星在2013年被确认为系外行星。第一次确定的探测发生在1992年。截止至2019年12月,总共有在2971个星系中的3976个行星被确认,其中653个星系有多于一个的行星。

探测系外行星的方法有许多。目前,通过过境测光法和多普勒光谱学法发现的行星最多,但这些方法有明显的观测偏差,更偏向于探测恒星附近的行星。因此,85\%探测到的外行星位于潮汐锁定区内。[2]在一些情况下,一颗恒星附近可以观察到多个行星。大约每五个类太阳恒星 中,有一个是“地球大小的”可居住区的行星。[3]假设银河系中有两千亿颗恒星,那么就有一百一十亿颗潜在的可居住的地球大小的行星,如果计算中包括围绕众多红矮星运行的行星,这个数字将达到四百亿。[4]

已知质量最小的行星是德拉格尔(也称为PSR B1257+12 A或PSR B1257+12 b),它的质量大约是月球质量的两倍。美国宇航局太阳系外行星档案中最大的行星在公元前2562年被记载,[5][6]它的质量大约是木星质量的30倍,尽管根据某些行星的定义(基于氘的核聚变[7]),它太大所以不可能是一颗行星,而是一颗褐矮星。有些行星离它们的恒星非常近,以至于它们只需要几个小时就能绕轨道运行,还有一些行星离它们非常远,以至于它们需要几千年才能绕轨道运行。有些距离恒星太远的行星,很难判断它们是否还受引力束缚。迄今为止,几乎所有被探测到的行星都在银河系内。尽管如此,有证据表明,银河系外的行星,即银河系以外更远的星系中的系外行星,可能存在。[8][9]最近的外行星是比邻星b,距地球4.2光年(1.3秒差距),它围绕着离太阳最近的恒星比邻星运行。[10]

太阳系外行星的发现增强了人们对寻找外星生命的兴趣。人们对运行在恒星的可居住区周围的行星特别感兴趣,因为这些区域可能存在液态水,而液态水是地球表面生命存在的先决条件。行星的可居住性研究还参考了其他一系列因素,来决定行星是否适合容纳生命。[11]

除了系外行星,宇宙中还存在不围绕任何恒星运行的流浪行星。这些行星往往被认为是一个独立的类别,尤其如果它们是气态巨行星,在这种情况下,它们通常被视为亚棕矮星,如WISE 0855-0714。[12]银河系中的流浪行星可能有数十亿颗,甚至更多。[13][14]

\subsection{ 命名系统}
\begin{figure}[ht]
\centering
\includegraphics[width=6cm]{./figures/8f5d7050f19a2955.png}
\caption{系外行星 HIP 65426b 是第一个在恒星 HIP 65426附近被发现的行星.[1]} \label{fig_TYXWXX_1}
\end{figure}
《指定系外行星公约》是国际天文学联盟(IAU)采用的命名多星系统的延伸。对于围绕一颗恒星运行的系外行星来说,这个名称通常是通过取其母星的名字,或者更常见的是,加上一个小写字母来形成的。[15]星系中发现的第一颗行星被命名为“$b$”(母星被认为是“$a$”),之后的行星被命名为后续字母。如果同一系统中的几颗行星同时被发现,离恒星最近的一颗会得到下一个字母,然后是按轨道大小顺序来排列的其他行星。临时的IAU认可的标准存在,以适应环双星的命名。少数系外行星有IAU认可的专有名称。除此之外还有其他命名系统。

\subsection{探测历史}
几个世纪以来,科学家、哲学家和科幻作家都怀疑太阳系外行星的存在,[16]但是没有办法探测到它们,或者知道它们的频率,也无法了解到它们与太阳系的行星有多相似。天文学家否决了十九世纪提出的各种探测主张。早在1917年就发现了太阳系外行星(范·马嫩2号)的第一个证据,但是并没有被验证。[17]在1988年,太阳系外行星疑似首次被科学家探测到。不久之后,首个探测在1992年被确认,同时发现了几颗围绕脉冲星PSR B1257+12运行的地球质量的行星。[17]围绕主序列恒星运行的系外行星的首次确认是在1995年,当时发现了一颗巨大的行星,围绕在51 Pegasi星的四天轨道附近。一些系外行星已经通过望远镜直接成像,但是绝大多数行星是被间接地探测到的,例如利用过境测光法和径向速度法。在2018年2月,研究人员利用钱德拉x光天文台,结合名为微透镜的行星探测技术,发现了遥远星系中行星存在的证据,称“其中一些系外行星相对而言和月球一样小,而另一些则和木星一样大。与地球不同,大多数系外行星与恒星没有紧密的联系,所以它们实际上是在太空中漫游或者在恒星之间松散地轨道运行。我们可以估计这个遥远的星系中的行星数量超过一万亿颗。[18]
\subsubsection{2.1 早期推测}
十六世纪意大利哲学家乔尔丹诺·布鲁诺是哥白尼太阳中心主义理论的早期支持者,此理论主张地球和其他行星围绕太阳运行,乔尔丹诺提出了恒星与太阳相似并且同样被行星伴随的观点。

在十八世纪,艾萨克·牛顿在《学者总论》中也提到了同样的可能性原理。在与太阳的行星进行比较时,他写道,“如果固定恒星是相似系统的中心,恒星都将按照相似的布局建造,并受制于其中心。"[19]

1952年,也就是第一颗热木星被发现的40多年前,奥托·斯特鲁维写道,没有令人信服的理由说明为什么行星不能比太阳系更靠近它们的母星,并提出利用多普勒光谱学和过境测光方法可以探测短轨道的超级木星。[20]
\subsubsection{2.2 不可信的主张}
自十九世纪以来,就有人声称探测到了太阳系外行星。一些最早的主张涉及双星70蛇夫座。1855年,东印度公司马德拉斯天文台的威廉·斯蒂芬·雅各布报告称,轨道异常使得这个系统中极有可能存在行星体。[21]19世纪90年代,芝加哥大学和美国海军天文台的托马斯·塞伊指出,轨道异常证明了70蛇夫座系统中存在暗体,围绕其中一颗恒星的运行周期是36年。[22]然而,林雷·莫尔顿发表了一篇论文,证明一个具有这些轨道参数的三体系统是高度不稳定的。[23]在20世纪50年代和60年代,斯沃斯莫尔学院的彼得·范·德·坎普提出了另一系列著名的探测主张,这次是针对围绕巴纳德星运行的行星。[24]天文学家现在普遍认为所有早期的探测报告都是错误的。[25]

1991年,安德鲁·琳恩、贝利斯和谢马尔声称他们利用了用脉冲星时间差,在PSR 1829-10的轨道上发现了一颗脉冲星行星。[26]这一说法一度受到强烈关注,但琳恩和他的团队很快收回了这一说法。[27]
\subsubsection{2.3 被证实的发现}
\begin{figure}[ht]
\centering
\includegraphics[width=6cm]{./figures/49c9e255382e572f.png}
\caption{哈尔望远镜拍摄到的HR8799恒星的三颗已知行星。来自中央恒星的光被矢量涡旋日冕遮住了。} \label{fig_TYXWXX_2}
\end{figure}
截止至2019年12月,共3433个已确认的系外行星在太阳系外行星百科全书被列出,包括几个自20世纪80年代末以来备受争议的主张被确认。1988年,来自维多利亚大学及不列颠哥伦比亚大学的加拿大天文学家布鲁斯·坎贝尔、沃克和杨斯蒂芬森,首次发表了他们的发现,并获得了后续的证实。[28]尽管他们对声称有行星探测持谨慎态度,但是他们的径向速度观测表明,一颗行星围绕伽马星Cephei运行。天文学家多年后仍对这一观测和其他类似的观测结果持疑态度,部分原因是当时这些观测处于仪器能力的极限。人们认为一些明显的行星可能是棕矮星,质量介于行星和恒星之间的物体。1990年发表了更多的观测结果,支持了绕伽马星轨道运行的行星的存在,[29]但是1992年的后续工作再次引起了严重的怀疑。[30]最后,在2003年,改进的科技使得行星的存在得以证实。[31]
\begin{figure}[ht]
\centering
\includegraphics[width=8cm]{./figures/091cdb0cb026632b.png}
\caption{2MASS J044144 是一颗褐矮星,伴星的质量大约是木星的5到10倍。目前还不清楚这颗伴星是一颗亚褐矮星还是一颗行星。} \label{fig_TYXWXX_3}
\end{figure}
1992年1月9日,射电天文学家亚历山大·沃尔兹森和戴尔·弗莱奥宣布发现两颗围绕脉冲星PSR 1257+12运行的行星。[17]这一发现得到了证实,并被普遍认为是第一次对系外行星的明确探测。后续的观察巩固了这些结果,1994年对第三颗行星的确认在大众媒体上重新引发了这个话题。[32]这些脉冲星行星被认为是在第二轮行星形成中,由产生脉冲星的超新星的不寻常的残余物形成的,或者其他是气态巨行星剩余的岩石核心,它们以某种方式幸存于超新星,然后衰变到它们当前的轨道。
\begin{figure}[ht]
\centering
\includegraphics[width=6cm]{./figures/fab3522817ef3ed6.png}
\caption{AB Pictoris的日冕图像显示了一个伴星(左下角),它是一颗褐矮星,或是一颗巨大的行星。数据是在2003年3月16日在VLT上使用NACO,在AB Pictoris上使用1.4角秒的明暗光掩模获得的。} \label{fig_TYXWXX_4}
\end{figure}
1995年10月6日,日内瓦大学的米歇尔·马约尔和迪迪埃·奎罗兹宣布首次明确探测到一颗绕主序列恒星运行的系外行星,即是附近的G型恒星51 Pegasi。[33][34]这一发现是在上普罗旺斯天文台做出的,开启了外行星探索的新时代。技术的进步,尤其是高分辨率光谱学的进步,促使了许多新的系外行星被更快地探测:天文学家可以通过测量行星对宿主恒星运动的重力影响来间接探测系外行星。更多太阳系外的行星随后通过观察一颗绕轨道运行的行星在恒星前方经过时,恒星的视星等变化量而被发现。

最初,大多数已知的系外行星都是质量很大的行星,其轨道非常靠近它们的母星。天文学家对这些“热木星”感到惊讶,因为行星形成的理论表明,巨型行星应该只在离恒星很远的地方形成。但是最终发现了更多其他种类的行星,现在人们明白热木星只构成了系外行星的少数。1999年,乌斯隆·仙女座成为第一颗已知有多个行星的主序恒星。[35]开普勒-16号包含了第一颗被发现围绕双星系统运行的行星。[36]

2014年2月26日,美国国家航空航天局宣布发现了715颗被新验证的系外行星,利用开普勒太空望远镜。这些系外行星是用一种叫做“多重性验证”的统计技术来检查的。[37][38][39]在这些结果之前,大多数被证实的行星都是气体巨行星,其大小与木星相当或更大,因为它们更容易被探测到,但是开普勒行星的尺寸大多介于海王星和地球之间。[37]

2015年7月23日,美国国家航空航天局公布了开普勒-452b,一颗接近地球大小的行星,围绕G2型恒星的可居住区运行。[40]

2018年9月6日,美国国家航空航天局在处女座发现了一颗距地球约145光年的外行星。[41]这颗系外行星沃尔夫503b是地球的两倍大,被发现时围绕着一种被称为“橙色矮星”的恒星。沃尔夫503b由于离恒星很近,在短短六天内完成了一个轨道运行。这颗系外行星相对靠近地球,它的主星闪耀着极其明亮的光芒。在所谓的富尔顿间隙附近,沃尔夫503b是唯一能被发现的非常大的系外行星。富尔顿间隙在2017年首次被注意到的,即在一定质量范围内行星是很难被观测到的。[42][41]

研究系外行星的天文学家已经在我们的星系中发现了数以千计的系外行星。沃尔夫503b至关重要,因为它离地球很近,便于通过开普勒太空望远镜做更多的拓展研究。沃尔夫503b绕轨道运行的“橙色矮星”是一颗明亮的恒星。科学家称橙色矮星寿命比太阳长三倍。沃尔夫503b对它的橙色矮星有很大的影响。由于沃尔夫503b的尺寸很大,沃尔夫503b对它的主星有引力影响。富尔顿间隙研究为天文学家开辟了一个新的领域,天文学家仍在研究富尔顿间隙中被发现的行星是气态的还是岩石状的。[41]
\subsubsection{2.4 候选的发现}
截至2017年6月,美国国家航空航天局的开普勒任务已经识别了5000多颗行星候选,[43]其中一些接近地球大小,位于可居住区,一些在类太阳恒星周围。[44][45][46]

系外行星种群 – 2017年6月[47][48]
\begin{figure}[ht]
\centering
\includegraphics[width=8cm]{./figures/17c5dfb1ab032182.png}
\caption{系外行星种群} \label{fig_TYXWXX_5}
\end{figure}
\begin{figure}[ht]
\centering
\includegraphics[width=8cm]{./figures/91c17703bc0241b4.png}
\caption{两种尺寸的小行星} \label{fig_TYXWXX_6}
\end{figure}