% 伪标量
% license Xiao
% type Tutor

注:本文参考Jie Peter《代数学基础》

几何代数$\mathcal G(V,q)$有一个非常特殊的元素,通常称之为定向的体积元
(oriented volume element)或者体积形式(volume form),即某个标准正交基 $\{e_i\}$ 中的元素构成的积:
$$\mathrm {vol:=e_1e_2e_3...e_n}~,$$
该体积形式又可被叫作伪标量单位,称其平方的结果为伪标量。

设$\{e_i\}$ 和$\{\theta_i\}$ 为线性空间$V$中的两组标准正交基,且$\theta_j=A^i_j e_i$,$A^i_j$是过渡矩阵的元素。那么由Clifford积的分配律,我们有:
\begin{equation}
\begin{aligned}
\theta_1\theta_2...\theta_n&=(A^{j_1}_1 e_{j_1})(A^{j_2}_2 e_{j_2})(A^{j_n}_n e_{j_n})\\
&=\sum_{\sigma\in S_n} sgn\sigma(A^{\sigma (1)}_1)e_1e_2...e_n\\
&=det A^i_je_1e_2...e_n~,
\end{aligned}
\end{equation}
该过渡矩阵是正交矩阵,因而体积元只相差正负号,可以表明在新的标准基下,体积的“定向性”是否发生改变。

伪标量可能为正负1,或1,取决于该空间的二次型。如果二次型是退化的,那么$I^2=0$。如果不是退化的,比如$R^{s,t}$,我们有
\begin{equation}
I^2=(-1)^{\frac{n(n-1}{2})+t}~,
\end{equation}
其中,n为该空间维度。$(-1)^{\frac{n(n-1)}{2}}$表示将$I^2$重排的逆序数。比如从$e_1e_2e_3e_1e_2e_3$到$e_1e_1e_2e_2e_3e_3$。

$I$可以用来构造新的代数,伪标量可以用来判断简单的代数同构。

\begin{example}{}
$\mathcal G(\mathbb R^3)$的伪标量单位$I=e_1e_2e_3,I^2=-1$。因而{1,I}可以张成子代数$\mathbb C$。同理,$\mathcal G(\mathbb R^2)$的伪标量单位和1也能张成复数代数
\end{example}
\begin{example}{}
给定几何代数$\mathcal G(\mathbb R^{0,1})$,由于$e_1^2=-1$。因而该代数就是复数代数。
\end{example}
\begin{example}{}
可以建立从$\mathcal G(\mathbb R^{0,2})$到四元数的同构。同构需要从单位元映射到单位元。此外,基之间的映射可以为:$e_1\mapsto i,e_2\mapsto j,e_1e_2\mapsto k$。或者是$e_1\mapsto i,e_2 e_1\mapsto j,e_2\mapsto k$。可以验证该映射保代数同态。
\end{example}
