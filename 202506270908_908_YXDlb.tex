% 伊西多·拉比(综述)
% license CCBYSA3
% type Wiki

本文根据 CC-BY-SA 协议转载翻译自维基百科 \href{https://en.wikipedia.org/wiki/Isidor_Isaac_Rabi}{相关文章}。

\begin{figure}[ht]
\centering
\includegraphics[width=6cm]{./figures/6a4d8eb51e8bbdb5.png}
\caption{} \label{fig_YXDlb_1}
\end{figure}
以色列·“伊西多”·艾萨克·拉比(Israel "Isidor" Isaac Rabi,/ˈrɑːbi/;意第绪语:איזידאָר יצחק ראַבי,转写:Izidor Yitzkhok Rabi;1898年7月29日-1988年1月11日)是一位美国核物理学家,因“发明用于记录原子核磁性特性的共振方法”而获得1944年诺贝尔物理学奖。他也是美国最早研究腔体磁控管的科学家之一,该装置广泛用于微波雷达和微波炉中。

拉比出生于加利西亚赖马努夫的一个传统波兰犹太家庭,婴儿时期随家人移民至美国,在纽约下东区长大。1916年,他以电气工程专业身份进入康奈尔大学学习,但不久后转向化学,后来又对物理学产生了兴趣。他在哥伦比亚大学继续深造,并因研究某些晶体的磁化率而获得博士学位。1927年,他前往欧洲,与当时许多顶尖物理学家会面并共事。

1929年,拉比返回美国,哥伦比亚大学为他提供了教职。与格雷戈里·布赖特合作时,他发展出了布赖特–拉比方程,并预测斯特恩–盖拉赫实验可以通过改进来验证原子核的某些特性。他利用核磁共振技术测定原子的磁矩和核自旋的研究,使他获得了1944年诺贝尔物理学奖。核磁共振技术随后成为核物理和化学中的重要工具,并进一步发展出磁共振成像(MRI)技术,使其在医学领域也具有重要意义。二战期间,拉比在麻省理工学院的辐射实验室从事雷达研究,同时也参与了曼哈顿计划。战后,他担任美国原子能委员会下属的一般顾问委员会(GAC)成员,并于1952年至1956年间出任委员会主席。他还参与了国防动员办公室和陆军弹道研究实验室的科学顾问委员会,并曾担任总统德怀特·艾森豪威尔的科学顾问。

拉比参与创建了布鲁克海文国家实验室,并于1946年推动其成立。作为美国教科文组织代表,他还参与了1952年欧洲核子研究中心(CERN)的创建。1964年哥伦比亚大学设立“大学教授”职位时,拉比是首位获得此殊荣的人。1985年,该校还以他的名字命名了一个特别讲席。他于1967年退休,退出教学工作,但仍积极参与系内事务,并一直保留“荣誉大学教授”与“特别讲席教授”头衔直至去世。
\subsection{早年经历}
以色列·艾萨克·拉比于1898年7月29日出生在奥匈帝国时期的加利西亚地区赖马努夫,该地现属波兰。他出生于一个波兰犹太正统家庭。不久之后,他的父亲戴维·拉比移民到了美国。几个月后,年幼的拉比和母亲谢因德尔也前往美国与父亲团聚,全家搬进了曼哈顿下东区一套两居室的公寓。在家中,他们讲意第绪语。

拉比上学时,母亲告诉校方他的名字是“Izzy”,校方人员误以为是“Isidor”的昵称,于是将“Isidor”登记为他的正式名字。从此,这个名字便成为他的官方姓名。后来,为了应对反犹主义,他开始将自己的名字写作“Isidor Isaac Rabi”,并在职业场合以“I.I. Rabi”为人所知。对他的大多数亲友来说,包括1903年出生的妹妹格特鲁德,他通常被简称为“Rabi”。

1907年,全家搬到布鲁克林的布朗斯维尔,并在那里经营一家杂货店。
