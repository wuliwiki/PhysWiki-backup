% 冯·诺伊曼-博内斯-哥德尔集合论(综述)
% license CCBYSA3
% type Wiki

本文根据 CC-BY-SA 协议转载翻译自维基百科\href{https://en.wikipedia.org/wiki/Von_Neumann\%E2\%80\%93Bernays\%E2\%80\%93G\%C3\%B6del_set_theory}{相关文章}。

在数学基础中,冯·诺依曼–伯奈斯–哥德尔集合论(NBG)是一种公理化集合论,是泽梅洛–弗兰克尔–选择公理集合论(ZFC)的保守扩展。NBG 引入了“类”的概念,类是由公式定义的集合,其量词仅对集合进行量化。NBG 可以定义比集合更大的类,例如所有集合的类和所有序数的类。摩尔斯–凯利集合论(MK)允许通过量词对类进行量化的公式来定义类。NBG 是有限公理化的,而 ZFC 和 MK 则不是。

NBG 的一个关键定理是类存在定理,它声明,对于每个量词仅对集合进行量化的公式,都存在一个类,该类包含满足该公式的集合。这个类是通过用类逐步构造公式来构建的。由于所有集合论公式都是由两种原子公式(成员关系和相等性)和有限多的逻辑符号构成,因此只需要有限多的公理来构建满足这些公式的类。这就是为什么 NBG 是有限公理化的原因。类还用于其他构造、处理集合论悖论,并用于表述全局选择公理,该公理比 ZFC 的选择公理要强。

约翰·冯·诺依曼在 1925 年将类引入集合论。他的理论的原始概念是函数和参数。利用这些概念,他定义了类和集合。\(^\text{[1]}\)保罗·伯奈斯通过将类和集合作为原始概念重新表述了冯·诺依曼的理论。\(^\text{[2]}\)库尔特·哥德尔简化了伯奈斯的理论,用于他对选择公理和广义连续统假设相对一致性的证明。\(^\text{[3]}\)
\subsection{集合论中的类} 
\subsubsection{类的用途} 
在 NBG 中,类有几个用途:
\begin{itemize}
\item 它们产生了集合论的有限公理化。\(^\text{[4]}\)  
\item 它们用于表述“非常强的选择公理”\(^\text{[5]}\)——即全局选择公理:存在一个定义在所有非空集合类上的全局选择函数\( G \),使得对于每个非空集合\( x \),都有 \( G(x) \in x \)。  
   这比 ZFC 的选择公理要强:对于每个非空集合的集合\( s \),存在一个选择函数 \( f \),定义在\( s \)上,使得对于所有\( x \in s \),都有\( f(x) \in x \)。  
\item 通过认识到某些类不能是集合,集合论悖论得以解决。例如,假设所有序数的类 \( \text{Ord} \)是一个集合。那么\( \text{Ord} \)是一个按\( \in \)良序的传递集合。所以,根据定义,\( \text{Ord} \)是一个序数。因此,\( \text{Ord} \in \text{Ord} \),这与\( \in \)是\( \text{Ord} \)的良序性相矛盾。因此,\( \text{Ord} \)不是一个集合。不能是集合的类称为适当类;\( \text{Ord} \)是一个适当类。\(^\text{[6]}\)
\item 适当类在构造中很有用。在他证明全局选择公理和广义连续统假设的相对一致性时,哥德尔使用适当类来构建构造宇宙。他在所有序数的类上构造了一个函数,对于每个序数,通过对先前构造的集合应用集合构建操作来构造一个构造集。构造宇宙就是这个函数的像。\(^\text{[7]}\)
\end{itemize}