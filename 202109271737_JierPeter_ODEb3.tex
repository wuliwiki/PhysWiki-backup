% 一阶常系数线性微分方程组
% 常微分方程|ODE|ordinary differential equation|方程组|线性变换|矩阵|矩阵指数

\pentry{矩阵指数\upref{MatExp},常系数线性微分方程\upref{ODEb2}}

一阶常系数线性微分方程组形如
\begin{equation}\label{ODEb3_eq1}
\leftgroup{
    \frac{\dd}{\dd x}x_1&=a_{11}x_1+a_{12}x_2+\cdots+a_{1n}x_n\\
    \frac{\dd}{\dd x}x_2&=a_{21}x_1+a_{22}x_2+\cdots+a_{2n}x_n\\
    &\vdots\\
    \frac{\dd}{\dd x}x_n&=a_{n1}x_1+a_{n2}x_2+\cdots+a_{nn}x_n
}
\end{equation}
其中各$x_i$是关于$t$的未知函数,各$a_{ij}$是已知常数.我们要研究的是如何解出这个方程组中的各未知函数.

别看这个方程有那么多变量$x_i(t)$,实际上我们可以把它们放到一起,构成一个$n$维向量$\bvec{x}(t)=(x_1(t), x_2(t), \cdots, x_n(t))$,这样就可以理解为还是只有一个自变量,只不过自变量从标量变成向量了.

以上述向量理解的方式来看,\autoref{ODEb3_eq1} 右边部分就是一个线性变换$\mat{M}\bvec{x}(t)$,其中
\begin{equation}
\mat{M}=\pmat{
    a_{11}, a_{12}, \cdots, a_{1n}\\
    a_{21}, a_{22}, \cdots, a_{2n}\\
    \vdots, \vdots, \ddots, \vdots\\
    }
\end{equation}





























