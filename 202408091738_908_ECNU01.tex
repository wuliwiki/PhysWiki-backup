% 华东师范大学 2001 年 考研 量子力学
% license Usr
% type Note

\textbf{声明}:“该内容来源于网络公开资料,不保证真实性,如有侵权请联系管理员”

\subsection{(15 分)}
证明$i\left(\hat{P}_x^2\hat{X} - \hat{X}\hat{P}_x^2\right)$是厄米算符。
\subsection{(20 分)}
已知一维系统解出其定态薛定谔方程 $\hat{H} \psi_n(x) = E_n \psi_n(x), \psi_n(x)$为实数,在 $t=0$ 时刻,对系统进行测量,发现$\frac{1}{2}$的几率得能量$E_1$有$\frac{1}{3}$的几率得能量$E_2$,有$\frac{1}{6}$的几率得能量$E_3$。请:

(a)按实系数写出$t=0$ 时刻波函数$\psi(x, t=0)$

(b)求出$t$时刻波函数$\psi(x, t)$

(c)求出$t$时刻能量的测量值

(d)求出$t$时刻位置$X$的期待值。(提示:可用  $X_{ij} = \int \psi_i^* x_j dx$ 表示)。
\subsection{(20 分)}
设泡利矩阵$\hat{\sigma}_x, \hat{\sigma}_Y$满足 $\hat{\sigma}_x^2 = \hat{\sigma}_Y^2 = 1$, 且$\hat{\sigma}_x \hat{\sigma}_Y + \hat{\sigma}_Y \hat{\sigma}_x = 0,$求:

\begin{enumerate}
    \item 在 $\hat{\sigma}_Y $表象中,算符 $\hat{\sigma}_x \cdot \hat{\sigma}_Y$ 的矩阵表示。
    \item 在 $\hat{\sigma}_Y$  表象中,算符$\hat{\sigma_x }$ 的本征值和本征函数。
\end{enumerate}
\subsection{(15 分)}