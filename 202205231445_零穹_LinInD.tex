% 线性无关判别法
% 线性无关|判别法

\pentry{对偶空间\upref{DualSp}}
在矢量空间中,找出线性无关\upref{linDpe}的向量往往是一个基本的任务,这可以从基底张成矢量空间看出(\autoref{VecSpn_def1}~\upref{VecSpn}).再有了对偶空间 \upref{DualSp}$V^*$ 空间的知识后,可以便洁的给出矢量空间 $V$ 中矢量线性无关性的各种判别法.
\begin{lemma}{}\label{LinInD_lem1}
若 $\bvec a_1,\cdots,\bvec a_m$ 是 $V$ 中线性相关的向量,而 $f_1,\cdots,f_m$ 是 $V$ 上任意的线性函数\autoref{DualSp_def3}~\upref{DualSp},那么
\begin{equation}
\det(f_i(\bvec a_j))=0,\quad 1\leq i,j\leq m
\end{equation}
($i$ 是行指标,$j$ 是列指标)\footnote{矩阵的指标往往遵从“左行右列,上行下列”的原则,因为人们习惯“从上到下,从左到右”.在一维情形,“上”和“左”,“下”和“右”并无区别,只是表明某种方向性.}
\end{lemma}
\textbf{证明:}由于 $\bvec a_1,\cdots,\bvec a_m$ 线性相关,必有一个矢量是其余矢量的线性组合,不失一般性,设这个矢量就是 $\bvec a_m$,则
\begin{equation}
\bvec a_m=\sum_{i\neq m}\alpha_i\bvec a_i
\end{equation}
在行列式 $\det(f_i(\bvec a_j))$ 中,从最后一列减去第一列乘 $\alpha_1$,$\cdots$,第 $m-1$ 列乘 $\alpha_{m-1}$ ,于是最后一列变为
\begin{equation}
\begin{aligned}
&f_i(\bvec a_m)-\sum_{j\neq m}\alpha_j f_i(\bvec a_j)=f_i(\bvec a_m-\sum_{j\neq m}\alpha_{j}\bvec a_j)\\
&=f_i(\bvec 0)=0,\quad i=1,\cdots ,m
\end{aligned}
\end{equation}
所以行列式为0.

\textbf{证毕!}

\begin{lemma}{}
如果 $(f_1,\cdots,f_n)$ 是矢量空间 $V$ 的 对偶空间$V^*$ 的一个基底,那么,矢量 $\bvec a_1,\cdots,\bvec a_n\in V$ 线性无关的充要条件为
\begin{equation}\label{LinInD_eq1}
\det(f_i(\bvec a_j))\neq 0
\end{equation}
\end{lemma}
\textbf{证明:} \begin{enumerate}
\item \textbf{充分性:}由\autoref{LinInD_lem1} 直接得证!
\item \textbf{必要性:}矢量 $\bvec a_1,\cdots,\bvec a_n\in V$ 线性无关,意味着 $V=\langle\bvec a_1,\cdots,\bvec a_n\rangle$ (\autoref{VecSpn_def1}~\upref{VecSpn} ).用 $(\bvec e_1,\cdots,\bvec e_n)$ 代表 $V$ 的对偶于 $(f_1,\cdots,f_n)$ 的基底(\autoref{DualSp_sub1}~\upref{DualSp} ),而用 $\alpha_{1j},\cdots,\alpha_{nj}$ 代表矢量 $\bvec a_j$ 在这个基底下的坐标.那么
\begin{pmatrix}
\alpha_{11}&\cdots&\alpha_{1n}\\
\vdots&\vdots&\vdots\\
\alpha_{n1}&\cdots&\alpha_{nn}
\end{pmatrix}
就是由基底 $(\bvec e_1,\cdots,\bvec e_n)$ 到基底 $(\bvec a_1,\bvec a_n)$ 的过渡矩阵\upref{TransM}.由\autoref{TransM_ex1}~\upref{TransM},它是可逆的,从而 $\det(\alpha_{ij}))\neq0$ .但 $\alpha_{ij}=f_i(\bvec a_j)$ ,故而\autoref{LinInD_eq1} 成立.

\textbf{证毕!} 
\end{enumerate}
