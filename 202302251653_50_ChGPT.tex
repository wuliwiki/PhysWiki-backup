% ChatGPT
% ChatGPT 人工智能 自然语言处理 生成模型 Transformer 对话 OpenAI

\textbf{ChatGPT}(Chat Generative Pre-trained Transformer)是一款由OpenAI组织推出的,通过多种语言大数据训练的基于Transformer(转换器)的语言生成模型。该模型的主要功能是与人进行实时对话。

ChatGPT模型可以对人类用户的提问进行回答,也可以接着用户的陈述,做进一步表述。在一次会话当中,该模型还可以记住之前的对话内容,并且对用户的追问和修正建议做适当反应。官方还宣称该模型会拒绝回答一些不适合的问题。与此同时,模型也存在一些局限性:很有可能产生错误的信息,有可能产生有害的建议或者有偏见的内容,以及对尚未训练过的知识了解有限。图1是官方提供的一个对话的例子。

\begin{figure}[ht]
\centering
\includegraphics[width=14cm]{./figures/ChGPT_2.png}
\caption{对话例子} \label{ChGPT_fig2}
\end{figure}

OpenA目前尚未公开ChatGPT模型的原始论文和源程序。根据ChatGPT官方网站(见图2),该模型的训练流程主要是:(1)收集大规模语言数据,训练监督策略;(2)收集比较数据,训练奖励模型;(3)用一种被称为"近端策略优化"的强化学习算法来进一步优化奖励模型。当前的ChatGPT版本是在原来的GPT-3.5模型基础上通过精调(fine-tuning)得来的。训练设备采用的是Azure人工智能计算架构。
\begin{figure}[ht]
\centering
\includegraphics[width=14.25cm]{./figures/ChGPT_1.png}
\caption{ChatGPT模型的训练过程} \label{ChGPT_fig1}
\end{figure}




参考文献:
\begin{enumerate}
\item https://openai.com/blog/chatgpt/
\end{enumerate}