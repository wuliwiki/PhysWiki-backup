% 柯西序列

\pentry{范数\upref{NormV}}

\begin{definition}{柯西序列}
给定某个集合中的序列 $\qty{u_i}$, 当满足以下条件时, 他就叫做柯西序列(Cauchy sequence)

对任意 $\varepsilon > 0$, 存在 $N$, 当 $n, m \geqslant N$ 时就满足 $\norm{u_n - u_m} < \varepsilon$
\end{definition}

形象地说, 柯西序列要求随着 $i$ 增大, 数列中项与项之间的差距逐渐变小.

实数域 $\mathbb R$ 上的柯西序列必定是收敛的, 这是我们可以通过判断数列是否为柯西序列从而判断该序列是否收敛. 但如果我们从 $\mathbb R$ 中把收敛的那点挖走, 那么这个柯西序列在这个集合中就不收敛. 所以柯西序列是否收敛取决于它所属于的集合. % 我想表达的意思很明确, 只是说法可能不够严谨

\subsection{完备性}
如果赋范空间中任意柯西序列\upref{cauchy}都有极限, 那么该赋范空间就是\textbf{完备(complete)}的. 完备的赋范空间常称为\textbf{巴拿赫空间(Banach space)}. 其中, 完备的内积空间特别称作\textbf{希尔伯特空间 (Hilbert space)}.

“完备” 可以形象理解为空间中没有 “漏洞”. 有限维空间都是完备的. 可数维空间都是不完备的. 例如多项式组成的空间就是不完备的(柯西序列的极限可以是 $\E^x$, 但是 $\E^x$ 并不属于该空间).
