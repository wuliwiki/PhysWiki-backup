% 表示的约化
% 等价表示|可约表示|不可约表示|完全可约表示

\begin{issues}
\issueDraft
\issueMissDepend
\end{issues}

\subsection{等价表示}
\begin{definition}{等价表示}
若$D(g)$与$D'(g)$同为群$G$的$n$维表示,且存在可逆的$n$维矩阵$C$满足于$\forall g\in G$,$C^{-1}D(g)C=D'(g)$,那么称$D(g)$与$D'(g)$为等价表示,记作$D(g)\cong D'(g)$。
\end{definition}

等价表示在本质上是对群空间中基的不同选取,定义中提到的变换矩阵$C$则恰好是不同基底之间的过渡矩阵矩阵。

\subsection{可约表示与不可约表示}

在给出可约表示的具体定义之前,先给出一个从形式上的理解:可约表示就是指那那些存在某个等价表示有如下分块矩阵形式的表示:

\begin{equation}
D(g)=\begin{pmatrix}
 D^1(g) & M\\
 0 & D^2(g)
\end{pmatrix}
\end{equation}

上式中的M可以为0,对于M等于0的情况我们称其为完全可约表示。

验证乘法规则:

\begin{align}
D(g_1)D(g_2)&=
\begin{pmatrix}
 D^1(g_1) & M(g_1)\\
 0 & D^2(g_1)
\end{pmatrix}
\begin{pmatrix}
 D^1(g_2) & M(g_2)\\
 0 & D^2(g_2)
\end{pmatrix} \\
&=\begin{pmatrix}
 D^1(g_1)D^1(g_2) & D^1(g_1)M(g_2)+M(g_1)D^2(g_2)\\
 0 & D^2(g_1)D^2(g_2)
\end{pmatrix} \\
&=\begin{pmatrix}
 D^1(g_1g_2) & D^1(g_1)M(g_2)+M(g_1)D^2(g_2)\\
 0 & D^2(g_1g_2)
\end{pmatrix}
\end{align}

容易看出对角线上的“块”满足群乘法规则。
可以证明的一点是,对于有限群来说,可约表示一定完全可约,由于结论较为简洁但证明过程需牵扯一些未在此处定义的概念所以暂不给出证明。

我们可以给出完全可约表示的定义:
\begin{definition}{完全可约表示}
对于矩阵的$n$维表示$D(g)$,若存在$n$维可逆矩阵$X$满足:
\begin{equation}
\forall g_\alpha\in G ,X^{-1}D(g_\alpha)X=
\begin{pmatrix}
 D^1(g_\alpha) & 0\\
 0 & D^2(g_\alpha)
\end{pmatrix}
\end{equation}

也就是说存在一个可逆矩阵将所有群元的表示矩阵相似变换成相同形式的分块对角矩阵的形式,那么该表示称为完全可约表示。
\end{definition}

从前文中可以看到分出的小块也构成群的一个表示,这样通过可约表示得到多个表示的过程称为表示的约化,这也是可约表示名字的由来。

在得到了可约表示的形式后,我们观察其形式可以发现,其将整个线性空间分化为了两个空间。原本的线性空间为两个子空间的直和。
