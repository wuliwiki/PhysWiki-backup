% 阿基米德(综述)
% license CCBYSA3
% type Wiki

本文根据 CC-BY-SA 协议转载翻译自维基百科\href{https://en.wikipedia.org/wiki/Archimedes}{相关文章}。

阿基米德(约公元前287年 – 约公元前212年)是古希腊的数学家、物理学家、工程师、天文学家和发明家,来自西西里岛古城锡拉库萨。[3] 虽然他的生平细节不多,但他被认为是古典时代的领先科学家之一。阿基米德被誉为古代历史上最伟大的数学家,也是所有时代最伟大的数学家之一。[4] 阿基米德通过应用“无穷小”概念和“穷尽法”,为现代微积分和分析学奠定了基础,推导并严格证明了一系列几何定理。[5][6][7] 这些定理包括圆的面积、球的表面积和体积、椭圆的面积、抛物线下的面积、旋转抛物面段的体积、旋转双曲面段的体积,以及螺旋的面积。[8][9]

阿基米德的其他数学成就包括推导出π(圆周率)的近似值、定义并研究了阿基米德螺旋,并设计了一种使用指数法表示非常大数字的系统。他还是最早将数学应用于物理现象的人之一,致力于静力学和流体静力学的研究。阿基米德在这一领域的成就包括杠杆定律的证明,[10] 重心概念的广泛应用,[11] 以及著名的阿基米德原理,即浮力定律的表述。[12] 他还被认为设计了许多创新的机械装置,如螺旋泵、复合滑轮和用于保护家乡锡拉库萨免受入侵的防御性战争机器。

与他的发明不同,阿基米德的数学著作在古代鲜为人知。亚历山大的数学家曾阅读并引用过他的作品,但直到大约公元530年,米利都的伊西多尔在拜占庭的君士坦丁堡才首次做出了全面的汇编,而尤托修斯在同世纪对阿基米德著作的注释,第一次使这些著作获得了更广泛的读者群体。那些在中世纪幸存下来的阿基米德著作的少数副本,成为了文艺复兴时期和17世纪科学家思想的重要来源,[13][14] 而1906年在阿基米德古抄本中发现的失落作品,提供了新的洞见,揭示了他是如何获得数学成果的。[15][16][17][18]
\subsection{传记}  
\subsubsection{早年生活}
\begin{figure}[ht]
\centering
\includegraphics[width=8cm]{./figures/b835f7aaef6d5bdd.png}
\caption{本杰明·韦斯特(Benjamin West)创作的《西塞罗发现阿基米德的墓》(1805)} \label{fig_Archim_1}
\end{figure}
阿基米德大约于公元前287年出生在锡拉库萨的港口城市,该城市当时是意大利南部大希腊地区的一个自治殖民地。出生日期基于拜占庭希腊学者约翰·特泽茨(John Tzetzes)的一段话,他提到阿基米德在公元前212年去世时活了75年。[9] 普鲁塔克在《希腊罗马英杰传》中写道,阿基米德与锡拉库萨国王赫罗二世有亲戚关系,尽管西塞罗则认为他出身卑微。[19][20] 在《沙数计》中,阿基米德提到父亲的名字是菲迪亚斯(Phidias),他是一位天文学家,其他关于他的事迹没有记载。[20][21] 阿基米德的朋友赫拉克利德斯(Heracleides)曾为他写过传记,但这部作品已遗失,使得他的一生仍然笼罩在谜团中。例如,至今无法确认他是否结过婚、有过子女,或者年轻时是否曾访问过埃及的亚历山大。[22] 从他现存的著作中可以看出,他与当时在亚历山大的学者保持着良好的学术关系,包括他的朋友萨摩斯的科农(Conon of Samos)和塞内的馆长埃拉托斯特尼(Eratosthenes of Cyrene)。[b]
\subsubsection{职业生涯}  
阿基米德的生活标准版本是由古希腊和古罗马历史学家在他去世后很久才写成的。关于阿基米德的最早记载出现在波利比乌斯(约公元前200–118年)的《历史》中,写于他去世约70年后。这篇记载对阿基米德作为一个人的描写甚少,主要聚焦于他为防御罗马人而制造的战争机器。波利比乌斯提到,在第二次布匿战争期间,锡拉库萨从罗马转向迦太基,导致由马库斯·克劳狄乌斯·马塞勒斯和阿皮乌斯·克劳狄乌斯·普尔库尔指挥的军事行动,于公元前213年至212年围攻该城。他指出,罗马人低估了锡拉库萨的防御,并提到阿基米德设计的几种机器,包括改进型的投石机、可以弯曲旋转的起重机以及其他投石器。尽管罗马最终攻占了锡拉库萨,但他们因阿基米德的创新而遭受了重大损失。

西塞罗(公元前106–43年)在他的几部作品中提到阿基米德。在担任西西里的财务官时,西塞罗发现了被认为是阿基米德墓的遗址,位于锡拉库萨的阿格里金特门附近,墓地被忽视,四周长满了灌木丛。西塞罗将墓地清理一番,并看到了雕刻并能读出一些作为铭文的诗句。墓碑上有一座雕塑,描绘了阿基米德最喜欢的数学证明:球体的体积和表面积是包含其底面的圆柱体的三分之二。他还提到,马塞勒斯曾将阿基米德建造的两个天文仪器带到罗马。罗马历史学家李维(公元前59年–公元17年)在其著作中重述了波利比乌斯关于锡拉库萨被攻陷以及阿基米德在其中作用的故事。
\subsubsection{死亡}
\begin{figure}[ht]
\centering
\includegraphics[width=6cm]{./figures/d9449557ab0db727.png}
\caption{《阿基米德之死》(1815年),托马斯·德乔治(Thomas Degeorge)画作[29]} \label{fig_Archim_2}
\end{figure}
普鲁塔克(公元45–119年)提供了关于阿基米德在锡拉库萨沦陷后如何死去的至少两种说法。根据最流行的说法,阿基米德正在思考一个数学图形时,城市被攻陷。一名罗马士兵命令他前去见马塞勒斯,但阿基米德拒绝了,称他必须先完成问题的研究。这激怒了士兵,士兵用剑杀死了阿基米德。另一个故事则说,阿基米德携带着数学工具,后来被杀,因为一名士兵认为这些物品是珍贵的。马塞勒斯听闻阿基米德死讯后 reportedly 感到愤怒,因为他认为阿基米德是一个宝贵的科学资产(他曾称阿基米德为“几何学的百臂巨人”),并命令不允许伤害他。

据说阿基米德的遗言是“不要打扰我的圆圈”(拉丁语:Noli turbare circulos meos;希腊语:μὴ μου τοὺς κύκλους τάραττε),这指的是他在被罗马士兵打扰时所研究的数学图形。没有可靠的证据表明阿基米德曾说过这句话,这句话也没有出现在普鲁塔克的记载中。一个类似的引述出现在瓦莱里乌斯·马克西穆斯(约公元30年)所著的《名言和事迹》中,他写道:“...但用手保护着灰尘时说‘我求你,不要打扰这东西’”。