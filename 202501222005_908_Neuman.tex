% 约翰·冯诺依曼(综述)
% license CCBYSA3
% type Wiki

本文根据 CC-BY-SA 协议转载翻译自维基百科\href{https://en.wikipedia.org/wiki/John_von_Neumann}{相关文章}。

\begin{figure}[ht]
\centering
\includegraphics[width=6cm]{./figures/c4c9226c223e913e.png}
\caption{冯·诺依曼在1940年代} \label{fig_Neuman_1}
\end{figure}
约翰·冯·诺依曼(John von Neumann,1903年12月28日—1957年2月8日)是一位匈牙利裔美国数学家、物理学家、计算机科学家和工程师。冯·诺依曼可能是他那个时代涵盖面最广泛的数学家之一,他将纯粹科学和应用科学相结合,并对许多领域作出了重要贡献,包括数学、物理学、经济学、计算机学和统计学。他是量子物理学数学框架建设的先驱,在泛函分析和博弈论的发展中也做出了突出贡献,提出或规范了包括细胞自动机、通用构造器和数字计算机等概念。他对自我复制结构的分析,早于DNA结构的发现。

在第二次世界大战期间,冯·诺依曼参与了曼哈顿计划,他开发了用于爆炸透镜的数学模型,这些透镜在内爆型核武器中起到了重要作用。战前和战后,他为许多组织提供咨询服务,包括科学研究与发展办公室、陆军弹道研究实验室、武装部队特殊武器计划和橡树岭国家实验室等。在1950年代的巅峰时期,他主持了多个国防部委员会,包括战略导弹评估委员会和洲际弹道导弹科学顾问委员会。他还是负责全国所有原子能开发的影响力巨大的原子能委员会的成员。在与伯纳德·施里弗和特雷弗·加德纳的合作中,他在美国首个洲际弹道导弹(ICBM)项目的设计和开发中扮演了关键角色。那时,他被认为是美国核武器方面的顶尖专家,也是美国国防部的首席防御科学家。

冯·诺依曼的贡献和智力能力得到了物理学、数学及其他领域同事的高度赞扬。他所获得的荣誉包括自由勋章以及以他名字命名的月球陨石坑。

**\subsection{生活与教育}  
**\subsubsection{家庭背景}  
冯·诺依曼于1903年12月28日出生在匈牙利王国的布达佩斯(当时是奥匈帝国的一部分),出生于一个富裕的、不信教的犹太家庭。他的出生名为Neumann János Lajos。在匈牙利语中,姓氏排在前面,而他的名字相当于英语中的John Louis。

他是家中三个兄弟中的长子,两个弟弟分别是米哈伊(Mihály,迈克尔)和尼克拉斯(Miklós,尼古拉斯)。他的父亲Neumann Miksa(Max von Neumann)是一位银行家,拥有法学博士学位。父亲于1880年代末从佩奇(Pécs)搬到布达佩斯。Miksa的父亲和祖父均出生于匈牙利北部的翁德(Ond,现在是泽伦茨(Szerencs)的一部分),位于泽普伦(Zemplén)县。冯·诺依曼的母亲是Kann Margit(玛格丽特·坎);她的父母是Kann Jákab和Meisels Katalin,来自Meisels家族。坎家三代人居住在布达佩斯坎-赫勒事务所上方的宽敞公寓里;冯·诺依曼的家庭占据了顶层的18个房间。

1913年2月20日,弗朗茨·约瑟夫皇帝因冯·诺依曼的父亲为奥匈帝国的服务,将他提升为匈牙利贵族。因此,诺依曼家族获得了“Margittai”这一世袭称号,意为“来自马吉塔”(如今的罗马尼亚马尔吉塔)。家族与该镇并无实际联系,这个称号是为了纪念玛格丽特而选择的,他们的家族徽章也描绘了三朵雏菊。冯·诺依曼János成为了“margittai Neumann János”(约翰·诺依曼·德·马吉塔),后来他将其更改为德语的“Johann von Neumann”。