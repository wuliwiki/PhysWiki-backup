% 复旦大学 2015 年考研普通物理
% keys 复旦|考研|普通物理

一、图示系统处于同一铅垂平面内, 细长直杆 $A B$ 的长度为 $r$, 以匀角速度 $\omega$ 绕轴 $A$ 作顺时针转动, 通过其 $B$ 端与半径为 $R=2 r$ 的扇形板 $D G H$ 的圆弧边缘 $(D$ 为其圆心)接触, 从而带动扇形板绕轴 $O$ 转动, 且 $O D=\sqrt{3} R / 3$, 试求图示瞬时 ( $A B$ 处于水平位置, $D G$ 处于铅垂位置) 扇形板的角速度和角加速度.(25 分)

二、真空中, 有一平行板电容器, 两块极板均为半径为 $a$ 的圆板, 将它连接到一个交变电源上, 使极板上的电荷按规律 $Q=Q_{0} \sin \omega t$ 随时间 $t$ 变化 (式中 $Q_{0}$ 和 $\omega$ 均为常量 $)$ .在略去边缘效应的条件下,试求两极板间任一点的磁场强度 $\vec{H}$.(15 分)

三、如图所示, 细圆环管在相连部件带动下沿水平直线轨道纯滚动, 管内有一壁虎,相对于环管爬行,壁虎可被视为一点, 在图中以小球 $B$ 代替.图示瞬间, 壁虎与环管的中心处于同一水平线上, 壁虎相对环管的速率为 $u$, 相对速度的方向朝下,相对速度大小的改变率等于 $0$, 环管中心 $O$ 点的速度向右, 速度大小也为 $u$, 加速度为 $0$.环管的中心半径等于 $R$ .求在此瞬时:
(15 分)
(1)壁虎相对地面的速度大小;
(2)壁虎相对地面的加速度大小;
(3)壁虎在相对地面的运动轨迹上所处位置点的曲率半径的大小.