% 洛伦兹群
% 洛伦兹变换|度规张量|四矢量|李群|李代数

\pentry{群\upref{Group}, 李群, 算符对易与共同本征函数\upref{Commut}, 闵可夫斯基空间\upref{MinSpa},协变和逆变\upref{CoCon}}

\subsection{洛伦兹变换}
物质运动可以看成一连串事件的发展过程.在四维时空里,每个事件都有相应的坐标.对于同一事件,在惯性系 $A$ 里用 $(t, \bvec{x})$ 表示,在惯性系 $A'$ 里用 $(t', \bvec{x'})$ 表示.这就是四维时空的四矢量,洛伦兹变换就是作用在一个惯性系的四维矢量上,使之变为另一个惯性系的四矢量.洛伦兹变换必须是线性的,譬如一个惯性系中的匀速运动,换了参考系也依然是匀速运动.从 $(t, \bvec{x})$ 到 $(t', \bvec{x'})$ 的线性变换是保证一切物理定律在不同惯性系下形式不变的基本要求.

接下来我们考虑两个特殊事件.设惯性系 $A$ 和 $A'$ 在 $t=t'=0$ 时在原点重合,此时原点发出一股光信号,该事件为 $(0,0,0,0)$.在此后的某个时刻,P点接收一光信号.在惯性系 $A$ 里该事件为为 $(t,x,y,z)$,在惯性系 $A'$ 里该事件为 $(t',x',y',z')$.采取自然单位制 $c=1$.那么我们有
\begin{equation}\label{qed1_eq6}
\begin{array}{c}
t^{2}-(x^{2}+y^{2}+z^{2})=0 \\
t^{\prime 2}-(x^{\prime 2}+y^{\prime 2}+z^{\prime 2})=0
\end{array}
\end{equation}

对于一般的两个事件,它们之间可能并不是用光信号联系,甚至可以没有联系.所以以上两个二次式的结果取值是任意的.通过线性变换,可以把上述方程里参考系 $A'$ 中的二次式变为关于 $(t,x,y,z)$ 的二次式.然而由于初始事件在原点重合,这个线性变换只能是
\begin{equation}
t^{2}-(x^{2}+y^{2}+z^{2})=C\left[(t^{\prime 2}-(x^{\prime 2}+y^{\prime 2}+z^{\prime 2})\right]
\end{equation}
其中 $C$ 是常数因子.由于空间中不存在特殊方向,所以 $C$ 只与两惯性系的相对速度大小有关.反过来,我们也有
\begin{equation}
t'^{2}-(x'^{2}+y'^{2}+z'^{2})=C\left[(t^{\prime 2}-(x^{\prime 2}+y^{\prime 2}+z^{\prime 2})\right]
\end{equation}
解得 $C^2=1$,由变换的连续性我们取 $C=1$.
称\autoref{qed1_eq6} 的二次式为\textbf{时空间隔},则洛伦兹变换是不改变时空间隔的线性变换,即
\begin{equation}\label{qed1_eq7}
s^2=t'^{2}-(x'^{2}+y'^{2}+z'^{2})=t^{2}-(x^{2}+y^{2}+z^{2})
\end{equation}

\subsection{洛伦兹群}
洛伦兹变换的集合形成\textbf{洛伦兹群(Lorentz Group)}.采取闵可夫斯基空间的度规张量 $g_{\mu \nu}=\operatorname{diag}(1,-1,-1,-1)$,引入时空坐标四矢量标记
\begin{equation}
\begin{aligned}
x^{\mu} &=\left(x^{0}, x^{i}\right) \quad(i=1,2,3)\\
&=(t, \bvec{x})
\end{aligned}
\end{equation}
时空间隔可以表示为
\begin{equation}\label{qed1_eq1}\begin{aligned}
S^{2} &=x^{0} x^{0}-x^{i} x^{i}=x^{\mu} x^{\nu} g_{\mu \nu} \\
&=x^{\prime 0} x^{\prime 0}-x^{\prime i} x^{\prime i}=x^{\prime \mu} x^{\prime \nu} g_{\mu \nu}
\end{aligned}\end{equation}
设联系两个惯性系的洛伦兹矩阵为 $\Lambda$,矩阵元为 $\Lambda_{\nu}^{\mu}\quad(\mu,\nu=0,1,2,3)$.洛伦兹变换可以表示为
\begin{equation}\label{qed1_eq2}x^{\prime \mu}=\Lambda_{\nu}^{\mu} x^{\nu}=\Lambda_{0}^{\mu} x^{0}+\Lambda_{i}^{\mu} x^{i}\end{equation}
代入\autoref{qed1_eq1} 我们有
\begin{align}
x^{\mu} x^{\nu} g_{\mu \nu}=x^{\rho} x^{\sigma} g_{\rho \sigma}
\end{align}
保持时空距离不变的要求导致
\begin{equation}\label{qed1_eq3}g_{\rho \sigma}=g_{\mu \nu} \Lambda_{\rho}^{\mu} \Lambda_{\sigma}^{\nu}\end{equation}
考虑一个无穷小变换
\begin{equation}\Lambda_{\nu}^{\mu}=\delta_{\nu}^{\mu}+\omega_{\nu}^{\mu}\end{equation}
代入\autoref{qed1_eq3} 得到 $\omega_{\mu \nu}=-\omega_{\nu \mu}$
所以洛伦兹群有六个独立参元素,分别是三个转动自由度和三个平动自由度.

下面我们来探讨洛伦兹群的性质.
四维矢量的矢量积与时空坐标四矢量一样,在洛伦兹变换下不变.设任意四维矢量 $X$,这意味着
\begin{equation}
X^TX=(\Lambda X)^T(\Lambda X)=X^T(\Lambda^T\Lambda)X
\end{equation}
可见,洛伦兹群是正交群.将洛伦兹群记为 $\mathcal{O}(3,1)$,其中 $\mathcal{O}$ 为“$Orthogonal$”正交的意思.
\addTODO{正交群引用}

\subsection{分类}
设洛伦兹矩阵用 $\Lambda$ 表示,可以证明洛伦兹矩阵的行列式 $\det \Lambda=\pm1$.

取洛伦兹矩阵的00分量,根据\autoref{qed1_eq3} 有
\begin{equation}
1=g_{\mu \nu} \Lambda_{\rho}^{0} \Lambda_{\sigma}^{0}= (\Lambda_{0}^{0})^2- (\Lambda_{i}^{0})^2\\
\end{equation}
所以 $\left|\Lambda_{0}^{0} \right|\geqslant1$. 根据 $\Lambda_{0}^{0}$ 的范围($\Lambda_{0}^{0}\geqslant1$ 是orthochronous的洛伦兹变换)和 $\Lambda$ ($\det\Lambda=1$ 是proper的变换)的行列式可以将洛伦兹变换分为四类
\begin{enumerate}
\item proper orthochronous($\Lambda_{+}^{\uparrow}$),对应 $\det\Lambda=1\quad \Lambda_{0}^{0}\geqslant1$.
\item proper non-orthochronous($\Lambda_{+}^{\downarrow}$),对应 $\det\Lambda=1\quad \Lambda_{0}^{0}\leqslant-1$
\item improper orthochronous($\Lambda_{-}^{\uparrow}$),对应 $\det\Lambda=-1 \quad\Lambda_{0}^{0}\geqslant1$
\item improper non-orthochronous($\Lambda_{+}^{\downarrow}$),对应 $\det\Lambda=-1 \quad\Lambda_{0}^{0}\leqslant-1$
\end{enumerate}

\subsection{推导}
采取矩阵记法,设两个惯性系的坐标矢量分别为 $X$ 和 $X'$,度规矩阵为 $g$,洛伦兹矩阵为 $\Lambda$.则\autoref{qed1_eq1},\autoref{qed1_eq2},\autoref{qed1_eq3} 可以分别写为
\begin{equation}
S^{2}=X^{\mathrm{T}} g X
\end{equation}
\begin{equation}
X'=\Lambda X
\end{equation}
\begin{equation}\label{qed1_eq4}
g=\Lambda^{\mathrm{T}} g \Lambda
\end{equation}
将\autoref{qed1_eq4} 两边取行列式,有
\begin{equation}
\det g=\det\Lambda^{\mathrm{T}}\cdot \det g\cdot \det\Lambda
\end{equation}
命题得证.
\subsection{举例}
\begin{enumerate}
\item \textbf{转动}: 纯粹的空间转动下,$x^{\prime 0}=x^{0}, x^{\prime i}=a^{i j} x^{j}$,这时洛伦兹矩阵写为
\begin{equation}       %开始数学环境
\Lambda=\left(                 %左括号
  \begin{array}{cccc}   %该矩阵一共3列,每一列都居中放置
   1& 0 & 0 & 0\\  %第一行元素
   0& a^{11} &  a^{12} &  a^{13}\\  %第二行元素
   0& a^{21} &  a^{22} &  a^{23}\\  %第三行元素
   0& a^{31} &  a^{32} &  a^{33}\\  %第四行元素
  \end{array}
\right)                 %右括号
\end{equation}
由于 $a^{ij}$ 构成的子矩阵为正交矩阵,则此时 $\det\Lambda=\pm1$,所以 $\Lambda$ 可能是 $\Lambda_{+}^{\uparrow}$ 或 $\Lambda_{-}^{\uparrow}$.
\item \textbf{平动(boost)}: 对于沿着x轴的平动,变换为
\begin{equation}\begin{array}{c}
x^{\prime 0}=x^{0} \cosh \eta-x^{1} \sinh \eta \\
x^{\prime 1}=-x^{0} \sinh \eta+x^{1} \cosh \eta \\
x^{\prime 2}=x^{2}, \quad x^{\prime 3}=x^{3}
\end{array}\end{equation}
参数定义为
\begin{align}
 \frac{v}{c} &=\tanh\eta \\
\left(1-(\frac{v}{c})^2 \right)^{-\frac{1}{2}} &=\cosh\eta\\
\left   (1-(\frac{v}{c})^2 \right)^{-\frac{1}{2}}\cdot\frac{v}{c}&=\sinh\eta 
\end{align}
那么对应的洛伦兹矩阵为
\begin{equation}\Lambda=\left(\begin{array}{cccc}
\cosh \eta & -\sinh \eta & 0 & 0 \\
-\sinh \eta & \cosh \eta & 0 & 0 \\
0 & 0 & 1 & 0 \\
0 & 0 & 0 & 1
\end{array}\right)\end{equation}
并有
\begin{equation}\begin{aligned}
&\det\Lambda=\cosh ^{2} \eta-\sinh ^{2} \eta=1\\
&\Lambda_{0}^{0}=\cosh \eta \geqslant 1
\end{aligned}\end{equation}
所以 $\Lambda$ 属于 $\Lambda_{+}^{\uparrow}$
\item \textbf{时间反演}
此时 $\Lambda=diag(-1,1,1,1)$,明显的,$\Lambda$ 属于 $\Lambda_{-}^{\downarrow}$.
\item \textbf{空间反演}
此时洛伦兹矩阵为闵可夫斯基度规,即 $\Lambda=diag(1,-1,-1,-1)$.明显的,此时 $\Lambda$ 属于 $\Lambda_{+}^{\downarrow}$
\end{enumerate}
可以根据群性质证明,所有的变换都可以写成 $\Lambda_{+}^{\uparrow}$ 与时间反演变换 $ \Lambda_{T}$ 以及宇称变换 $\Lambda_{P}$ 的乘积.在一些书上,会把洛伦兹群 $\Lambda_{+}^{\uparrow}$ 写成 $\mathcal{S O}(3,1)^{\uparrow}$,表明这是\textbf{Special Orthogonal groups},\textbf{Special}指的就是行列式为1这个性质.所以$$\mathcal{O}(3,1)=\left\{\mathcal{S O}(3,1)^{\uparrow}, \Lambda_{P} \mathcal{S O}(3,1)^{\uparrow}, \Lambda_{T} \mathcal{S O}(3,1)^{\uparrow}, \Lambda_{P} \Lambda_{T} \mathcal{S O}(3,1)^{\uparrow}\right\}$$所以只需要对 $\mathcal{S O}(3,1)^{\uparrow}$,即 $\Lambda_{+}^{\uparrow}$ 进行研究即可.下述的洛伦兹变换一律指正规正时洛伦兹变换.

\subsection{洛伦兹变换的群元形式}
上述小节已证明,洛伦兹群有6个参数,是反对称矩阵 $\omega_{\mu\nu}$ 的六个独立元素,对应着六个独立生成元.把生成元表示为 $J^{\mu\nu}$,因为有反对称指标 $(\mu,\nu)$ 则有 $J^{\mu\nu}=-J^{\nu\mu}$.根据李群理论,一个广义的洛伦兹群元 $\Lambda$ 可以写为
\begin{equation}
\Lambda=e^{-\frac{\I}{2} \omega_{\mu\nu}J^{\mu\nu}}
\end{equation}

\subsection{四矢量表象下的生成元}
在四矢量表象下,生成元由 4×4 矩阵 $(J^{\mu\nu})_\sigma^\rho$ 表示.
可以证明,
\begin{equation}\left(J^{\mu \nu}\right)_{\sigma}^{\rho}=\I\left(g^{\mu \rho} \delta_{\sigma}^{\nu}-g^{\nu \rho} \delta_{\sigma}^{\mu}\right)\end{equation}
\subsection{洛伦兹群的李代数}
可以证明,洛伦兹群的李代数为:
\begin{equation}\left[J^{\mu \nu}, J^{\rho \sigma}\right]=\I\left(g^{\nu \rho} J^{\mu \sigma}-g^{\mu \rho} J^{\nu \sigma}-g^{\nu \sigma} J^{\mu \rho}+g^{\mu \sigma} J^{\nu \rho}\right)\end{equation}
通过重组 $J^{\mu\nu}$,可以得到两类空间矢量
\begin{equation}\label{qed1_eq5}J^{i}=\frac{1}{2} \epsilon^{i j k} J^{j k}, \quad K^{i}=J^{i 0}\end{equation}
在这个角度上, 可以写为
\begin{equation}\begin{aligned}
\left[J^{i}, J^{j}\right] &=\I \epsilon^{i j k} J^{k} \\
\left[J^{i}, K^{j}\right] &=\I \epsilon^{i j k} K^{k} \\
\left[K^{i}, K^{j}\right] &=-\I \epsilon^{i j k} J^{k}
\end{aligned}\end{equation}
第一个对易子是 $SU(2)$ 的李代数,这意味着\autoref{qed1_eq5} 中 $J^i$ 是角动量. 第二个对易子则表明了 $\bvec K$ 是平动矢量.
