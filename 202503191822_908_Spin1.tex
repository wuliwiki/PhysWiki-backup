% 自旋(综述)
% license CCBYSA3
% type Wiki

本文根据 CC-BY-SA 协议转载翻译自维基百科\href{https://en.wikipedia.org/wiki/Spin_(physics)}{相关文章}。

自旋是基本粒子所具有的一种内禀角动量形式,因此,诸如强子、原子核和原子等复合粒子也具有自旋。[1][2]: 183–184  自旋是量子化的,并且描述自旋相互作用的准确模型需要相对论量子力学或量子场论。

电子自旋角动量的存在是通过实验推断出来的,例如斯特恩-盖拉赫实验(Stern–Gerlach experiment),其中银原子被观察到具有两个可能的离散角动量状态,尽管它们没有轨道角动量。[3] 相对论性的自旋-统计定理(spin–statistics theorem)将电子自旋的量子化与泡利不相容原理(Pauli exclusion principle)联系起来:不相容性的观测结果意味着自旋为半整数,而半整数自旋的观测结果又意味着不相容性。

在数学上,自旋可以用向量来描述(例如光子),也可以用旋量(spinor)或双旋量(bispinor)来描述(例如电子)。旋量和双旋量在某些方面与向量类似:它们具有确定的大小,并且在旋转下发生变化;然而,它们的“方向”采用了一种非传统的方式。所有同类的基本粒子具有相同大小的自旋角动量,尽管其方向可以变化。这些特性通过赋予粒子一个自旋量子数来表示。[2]: 183–184 

自旋的国际单位制(SI)单位与经典角动量相同(即 N·m·s,J·s 或 kg·m²·s⁻¹)。在量子力学中,角动量和自旋角动量具有离散的值,并且它们的大小与普朗克常数成比例。在实际应用中,自旋通常通过将自旋角动量除以约化普朗克常数 ħ 来表示为无量纲的自旋量子数。通常,“自旋量子数”也直接被称为“自旋”。