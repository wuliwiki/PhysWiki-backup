% 列夫·朗道(综述)
% license CCBYSA3
% type Wiki

本文根据 CC-BY-SA 协议转载翻译自维基百科 \href{https://en.wikipedia.org/wiki/Lev_Landau}{相关文章}。

列夫·达维多维奇·朗道(俄语:Лев Дави́дович Ланда́у,1908年1月22日-1968年4月1日)是一位苏联物理学家,在理论物理的诸多领域作出了基础性的贡献。\(^\text{[1][2][3]}\)他被认为是最后一批在物理学各个分支都造诣深厚并做出开创性贡献的科学家之一。\(^\text{[4]}\)他被誉为20世纪凝聚态物理学的奠基人,\(^\text{[5]}\)同时也被广泛认为是苏联最杰出的理论物理学家。\(^\text{[6]}\)
\subsection{生平}
\subsubsection{早年时期}
\begin{figure}[ht]
\centering
\includegraphics[width=6cm]{./figures/ce294ae7da64364f.png}
\caption{朗道一家,1910年} \label{fig_LFLD_1}
\end{figure}
朗道于1908年1月22日出生在俄罗斯帝国的巴库(今属阿塞拜疆),父母是犹太人\(^\text{[11][12][13][14]}\)。他父亲达维德·列沃维奇·朗道是一位从事当地石油工业的工程师,母亲柳博芙·维尼亚米诺芙娜·加尔卡维-朗道是一名医生。两人都来自莫吉廖夫,并毕业于当地的文理中学\(^\text{[15][16]}\)。朗道12岁学习微分学,13岁学习积分学,并在1920年13岁时从中学毕业。由于父母认为他年龄太小,不适合直接升入大学,他先在巴库经济技术学校学习了一年。
1922年,年仅14岁的朗道进入巴库国立大学,同时注册了两个系:物理与数学系以及化学系。后来他中止了化学的学习,但终其一生对化学始终保有兴趣。
\subsubsection{列宁格勒与欧洲时期}
\begin{figure}[ht]
\centering
\includegraphics[width=6cm]{./figures/5c374c1bae8ad643.png}
\caption{1914年的少年朗道} \label{fig_LFLD_2}
\end{figure}
1924年,朗道前往当时苏联物理学的主要中心——列宁格勒国立大学物理系,专注于理论物理的学习,并于1927年毕业。此后,他进入列宁格勒物理技术研究所攻读研究生,并最终于1934年获得物理-数学科学博士学位。\(^\text{[17]}\)1929年至1931年,朗道首次获得出国机会,依靠苏联政府(教育人民委员部)提供的出国奖学金,同时也得到了洛克菲勒基金会的资助。在这段时间里,他已能流利地使用德语和法语,并能以英语交流。\(^\text{[18]}\)后来,他进一步提高了英语水平,并学习了丹麦语。\(^\text{[19]}\)

朗道曾短暂访问哥廷根和莱比锡,随后于1930年4月8日前往哥本哈根,在尼尔斯·玻尔理论物理研究所工作,直到同年5月3日离开。这次访问之后,朗道始终视自己为尼尔斯·玻尔的学生,而他的物理研究方法也受到玻尔深刻的影响。离开哥本哈根后,朗道于1930年中期访问剑桥,与保罗·狄拉克合作研究,\(^\text{[20]}\)同年9月至11月他再次回到哥本哈根,\(^\text{[21]}\)接着于1930年12月至1931年1月在苏黎世与沃尔夫冈·泡利共事。\(^\text{[20]}\)从苏黎世出发后,他第三次前往哥本哈根,\(^\text{[22]}\)并于1931年2月25日至3月19日再次在那里停留,然后于同年返回列宁格勒。\(^\text{[23]}\)
\subsubsection{乌克兰哈尔科夫:国家科学中心哈尔科夫物理技术研究所}
1932年至1937年间,朗道担任国家科学中心哈尔科夫物理技术研究所理论物理系主任,并在哈尔科夫大学和哈尔科夫理工学院讲授课程。除理论研究外,朗道还是乌克兰哈尔科夫理论物理学传统的主要奠基人,这一学派有时被称为“朗道学派”。在哈尔科夫,他与朋友兼前学生叶甫根尼·利夫希茨开始撰写著名的《理论物理教程》,这一套涵盖理论物理全部领域的十卷巨著,至今仍被广泛用作研究生阶段的物理教材。在大清洗期间,朗道因涉及哈尔科夫的“UPTI 案”受到调查,但他设法离开哈尔科夫,前往莫斯科接受新职。\(^\text{[3]}\)

朗道制定了一项著名的综合考试,被称为“理论最低限”,学生只有通过这项考试后才能正式进入他的学派学习。该考试涵盖理论物理的各个方面。从1934年到1961年,仅有43人通过,但这些通过者后来都成为了非常杰出的理论物理学家。\(^\text{[24][25]}\)

1932年,朗道计算出了“钱德拉塞卡极限”;\(^\text{[26]}\)然而,他当时并未将其应用于白矮星。\(^\text{[27]}\)
\subsubsection{莫斯科物理问题研究所}
\begin{figure}[ht]
\centering
\includegraphics[width=8cm]{./figures/cfc5b79c45f1b664.png}
\caption{1938–1939年狱中照片} \label{fig_LFLD_3}
\end{figure}
\begin{figure}[ht]
\centering
\includegraphics[width=8cm]{./figures/a920820c2683ad17.png}
\caption{1934年在哈尔科夫研究所} \label{fig_LFLD_4}
\end{figure}
自1937年至1962年,朗道担任莫斯科物理问题研究所理论部主任。\(^\text{[28]}\)

1938年4月27日,朗道因持有一份传单而被捕,该传单将斯大林主义与德国纳粹主义和意大利法西斯主义相提并论。\(^\text{[3][29]}\)他被关押在内务人民委员部卢比扬卡监狱,直到1939年4月29日才获释。这次获释是由于彼得·卡皮察(著名的低温实验物理学家、该研究所创始人及所长)和尼尔斯·玻尔向约瑟夫·斯大林写信为他求情。\(^\text{[30][31]}\)卡皮察亲自担保朗道的品行,并威胁若不释放朗道将辞职离所。\(^\text{[32]}\)获释后,朗道提出了解释卡皮察发现的超流现象的理论,使用声子(声波激发)与一种新的激发形式——旋子。\(^\text{[3]}\)

朗道还带领一支数学家团队,支持苏联原子弹与氢弹的研制。他曾计算出首枚苏联热核武器的动力学过程,并预测了其当量。由于这项工作,朗道在1949年与1953年两度获得“斯大林奖”,并于1954年被授予“社会主义劳动英雄”称号。\(^\text{[3]}\)

朗道的学生包括:列夫·皮塔耶夫斯基、阿列克谢·阿布里科索夫、亚历山大·阿希耶泽尔、伊戈尔·贾洛辛斯基、叶甫根尼·利夫希茨、列夫·戈尔科夫、伊萨克·哈拉托尼科夫、罗阿尔德·萨格杰耶夫和伊萨克·波梅朗丘克。
\subsubsection{科学成就}
朗道的成就包括:密度矩阵方法在量子力学中的独立共同发现(与约翰·冯·诺依曼同时提出),量子力学的抗磁性理论,超流体理论,二级相变理论,金兹堡–朗道超导理论,费米液体理论,对等离子体物理中朗道阻尼现象的解释,量子电动力学中的朗道极点,中微子二分量理论,对火焰不稳定性的解释(即达里厄–朗道不稳定性),以及用于描述 S 矩阵奇点的朗道方程。

由于他建立了一个能解释液态氦II在低于2.17K(−270.98°C)时性质的超流体数学理论,朗道于1962年荣获诺贝尔物理学奖。\(^\text{[33]}\)
\subsubsection{个人生活与观点}
1937年,朗道与来自哈尔科夫的科拉·T·德罗班泽娃结婚。\(^\text{[34]}\)他们的儿子伊戈尔后来也成为一位理论物理学家。朗道主张“自由恋爱”而非一夫一妻制,他鼓励妻子和自己的学生实践“自由恋爱”。然而,他的妻子对此并不热衷。\(^\text{[3]}\)

朗道一般被描述为无神论者,\(^\text{[35][36][37]}\)然而当苏联电影导演安德烈·塔可夫斯基问他是否相信上帝的存在时,朗道沉默了三分钟,最终回答说:“我想是的。”\(^\text{[38]}\)1957年,克格勃向苏共中央提交了一份详细报告,记录了朗道对1956年匈牙利起义、弗拉基米尔·列宁以及他所称的“红色法西斯主义”的看法。\(^\text{[39]}\)物理学家亨德里克·卡西米尔回忆说,朗道是一个充满激情的共产主义者,并受其革命意识形态所激励。朗道在建立苏联科学体系方面的热情,部分来源于他对社会主义的忠诚。1935年,朗道在苏联《消息报》上发表文章《资产阶级与当代物理学》,批判宗教迷信和资本主导地位,他认为这两者都是资产阶级的倾向。他在文中强调:“党和政府为我国物理学的发展提供了前所未有的机遇。”\(^\text{[3]}\)
\subsubsection{晚年}
1962年1月7日,朗道乘坐的汽车与一辆迎面而来的卡车相撞,他伤势严重,昏迷了两个月。尽管他在许多方面恢复了健康,但他的科学创造力遭到了彻底破坏,\(^\text{[28]}\)此后再也未能完全回归科研工作。他的伤情也使他无法亲自前往领奖,接受1962年诺贝尔物理学奖。\(^\text{[40]}\)

朗道以其犀利幽默闻名一生,以下是他在事故恢复期间与心理学家亚历山大·鲁里亚之间的一段经典对话,\(^\text{[19][41]}\)鲁里亚当时正试图评估他是否有脑部损伤:

鲁里亚:“请给我画一个圆。”
朗道画了一个十字。
鲁里亚:“嗯,那请给我画一个十字。”
朗道画了一个圆。
鲁里亚:“朗道,你为什么不按照我说的做?”
朗道:“如果我照做了,你也许会以为我变傻了。”

1965年,朗道的几位学生和同事在莫斯科郊外的切尔诺戈洛夫卡创立了朗道理论物理研究所,该所在随后三十年由伊萨克·哈拉托尼科夫领导。

同年6月,朗道与叶夫谢·利伯曼在《纽约时报》上发表了一封公开信,声明他们作为苏联犹太人反对美国介入“苏联犹太学生运动”。\(^\text{[42]}\)然而,有学者质疑这封信是否真的由朗道亲自撰写。\(^\text{[43]}\)
\subsubsection{去世}
朗道于1968年4月1日去世,享年60岁,死因是六年前车祸所致伤病的并发症。他安葬于新圣女公墓。\(^\text{[44]}\)
\subsection{研究贡献领域}
\begin{itemize}
\item DLVO理论
\item 费米液体理论
\item 准粒子理论
\item 伊万年科–朗道–凯勒方程
\item 朗道阻尼
\item 朗道分布
\item 朗道规范
\item 朗道动理学方程
\item 朗道极点
\item 朗道磁化率
\item 朗道势能
\item 朗道量子化
\item 朗道理论
\item 朗道–斯奎尔射流
\item 朗道–列维奇问题
\item 朗道–霍普湍流理论
\item 金兹堡–朗道理论
\item 达里厄–朗道不稳定性
\item 朗道–李夫希茨气动声学方程
\item 朗道–雷查杜里方程
\item 朗道–曾纳公式
\item 朗道–李夫希茨模型
\item 朗道–李夫希茨赝张量
\item 朗道–李夫希茨–吉尔伯特方程
\item 朗道–波梅朗丘克–米格达尔效应
\item 朗道–杨定理
\item 朗道原理
\item 斯图尔特–朗道方程
\item 超流性
\item 超导性
\end{itemize}
\subsubsection{教育贡献}
\begin{itemize}
\item 《理论物理教程》
\end{itemize}
\subsection{#遗产}
\begin{figure}[ht]
\centering
\includegraphics[width=6cm]{./figures/86688521fccf2708.png}
\caption{} \label{fig_LFLD_5}
\end{figure}
为纪念朗道,有两个天体以他的名字命名:
\begin{itemize}
\item 小行星 2142 Landau。[45]
\item 月球上的朗道环形山。
\end{itemize}
俄罗斯科学院授予理论物理领域的最高奖项也以他命名:
\begin{itemize}
\item 朗道金质奖章
\end{itemize}
2019年1月22日,谷歌以涂鸦形式庆祝朗道诞辰111周年。[46]

此外,由美国物理学会设立、旨在表彰对等离子体物理的杰出贡献以及欧美合作成就的奖项:朗道–斯皮策奖也部分以他的名字命名。[47]
\subsection{朗道对物理学家的评级}
\begin{figure}[ht]
\centering
\includegraphics[width=6cm]{./figures/05c41d75b2e99f43.png}
\caption{} \label{fig_LFLD_6}
\end{figure}
朗道曾列出一份物理学家名单,并按照他们在创造力、天赋和科研成果等方面的表现,以对数等级制为基础,将他们划分为从 0 到 5 的等级。[48][49][50]等级 0 是最高级,仅授予艾萨克·牛顿;阿尔伯特·爱因斯坦被评为 0.5;等级 1 授予了量子力学的奠基人们,如尼尔斯·玻尔、维尔纳·海森堡、萨蒂扬德拉·纳特·玻色、保罗·狄拉克和埃尔温·薛定谔等人;等级 5 的人被他称为“病理学家”,意指他们在科学上的天赋和创造力极低。[51]朗道最初将自己评为 2.5 级,后来将自己提升为 2 级。物理学家 N·戴维·默明在撰写关于朗道的文章《我与朗道的生活:一个4.5级向2级致敬》中引用了这套评级体系,并幽默地称自己为第4级半。[52][53]

此外,朗道还设计了一个鲜为人知的评级图示体系,以图形方式衡量科学家的天赋。他将科学家分为四类图形,其中第一类是一个简单的三角形,象征那些最具原创性和天赋的科学家,例如狄拉克和爱因斯坦。图形由**两条平行线**组成:下方的线表示“**毅力**”,上方的线表示“天赋与原创性”。[54]
