% 介值定理(综述)
% license CCBYSA3
% type Wiki

本文根据 CC-BY-SA 协议转载翻译自维基百科\href{https://en.wikipedia.org/wiki/Intermediate_value_theorem}{相关文章}。

\begin{figure}[ht]
\centering
\includegraphics[width=8cm]{./figures/17b13a0f3803ea4f.png}
\caption{} \label{fig_JZDL_1}
\end{figure}
在数学分析中,中值定理指出:如果函数 $f$ 是一个连续函数,且其定义域包含区间 $[a, b]$,那么对于任意介于 $f(a)$ 与 $f(b)$ 之间的值,函数 $f$ 在该区间内至少有一个点取到这个值。

该定理有两个重要的推论:
\begin{enumerate}
\item 如果一个连续函数在某个区间内的两个端点处函数值符号相反,那么它在这个区间内至少有一个零点(即有解)——这被称为波尔查诺定理[1][2]。
\item 一个连续函数在一个区间上的值域本身也是一个区间。
\end{enumerate}
\subsection{动机}
