% 自旋与旋转
% license Usr
% type Tutor
\pentry{升降算符\upref{RLop},自旋角动量\upref{Spin},四元数\upref{Quat},四元数与旋转矩阵\upref{QuatN}}

\begin{issues}
\issueDraft
1.从期望值的变化看旋转效应。2.通过同构
\end{issues}

\subsubsection{自旋对态矢作用}
 通过对角动量理论的学习,我们已经知道,在向量空间中,$\mathrm e^{-\mathrm i\hat J_i\phi}$可以描述经典向量绕生成元$\hat J_i$对应的轴旋转$\phi$。但对于可以选择任意表象的态矢而言,这种绕固定轴的转动还可以影响别的观测结果,即期望值。

三维空间中任意态矢的期望值为$(\bar \hat S_x,\bar S_y,\bar S_z)$
\subsubsection{从四元数推导出自旋矩阵}
