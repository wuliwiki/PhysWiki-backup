% 几何向量的内积
% keys 线性代数|几何向量|内积|dot product|点积|点乘|标量积|scalar product|内积|inner product|正交|归一化|正交归一基|交换律|分配律
% license Xiao
% type Tutor

\pentry{几何向量的运算\upref{GVecOp}}{nod_d2b3}
% Giacomo:要不要放弃“内积”这个称呼,就叫点乘或者标量积(数量积)?类似的,只叫垂直不叫正交,把“内积”、“正交”留给线代?
\subsection{几何定义}
我们先来看内积的几何定义。 注意该定义不需要任何坐标系的概念。
\begin{figure}[th]
\centering
\includegraphics[width=6cm]{./figures/e92182f1c525fa29.pdf}
\caption{内积的几何定义}\label{fig_Dot_1}
\end{figure}

两个几何向量的\textbf{内积(inner product)}也叫\textbf{点乘(dot product)}、\textbf{点积}或者\textbf{标量积(scalar product)}。 如\autoref{fig_Dot_1},一般用一个实心圆点表示几何向量的内积(不可省略)。 内积就是把两向量的模长相乘, 再乘以它们的夹角\footnote{对于空间向量来说,我们需要先截取这两个向量所在的平面(如果它们不平行则平面唯一确定)。} $\theta$ 的余弦值:
\begin{equation}\label{eq_Dot_1}
\bvec A \vdot \bvec B = \abs{\bvec A} \abs{\bvec B} \cos\theta ~,
\end{equation}
注意两个向量内积得到的是一个标量。 几何定义中(\autoref{fig_Dot_1}),既可以把内积理解为 $\bvec A$ 投影在 $\bvec B$ 上的模长乘以 $\bvec B$ 的模长,也可以理解为 $\bvec B$ 投影在 $\bvec A$ 上的模长乘以 $\bvec A$ 的模长。 在这种理解下,若量向量的夹角为钝角,投影长度取负值。 可见当两向量模长不变时,若方向相同,内积取最大值 $\abs{\bvec A}\abs{\bvec B}$;若方向相反,内积取最小值 $-\abs{\bvec A}\abs{\bvec B}$;若相互垂直,则内积为 0。

我们说两个内积为 0 的向量互相\textbf{垂直(perpendicular)}, 或者说\textbf{正交(orthogonal)}。 几何向量与自身内积可得该向量模长的平方。 单位向量与自己的内积等于 1。 

% 把一个向量除以自身模长得到同方向单位向量的过程叫做向量的\textbf{归一化(normalization)}。
% Giacomo:在几何向量的加法与数乘\upref{GVecOp}里定义过了。

\subsection{坐标定义}
若已知 $\bvec A, \bvec B$ 在平面直角坐标系 $xy$ 中坐标分别为 $(A_x, A_y)$ 和  $(B_x, B_y)$,那么如何用坐标表示内积运算的结果呢?我们把它定义为
%未完成,向量空间里面要介绍基底及坐标的唯一性。
% Giacomo:驳回,高中数学你还想引用线性代数!下面的内容已经被我用更初等的语言重新描述了。
% \begin{equation}
% \bvec A = A_x \,\uvec x + A_y \,\uvec y ~,\qquad \bvec B = B_x \,\uvec x + B_y \,\uvec y~.
% \end{equation}
% 所以
% \begin{equation}
% \bvec A \vdot \bvec B = (A_x \,\uvec x + A_y \,\uvec y) \vdot (B_x \,\uvec x + B_y \,\uvec y)~.
% \end{equation}
% 根据分配律\autoref{eq_Dot_3},我们可以把两个括号拆开,变为 4 个内积之和。 
% \begin{equation}
% \bvec A \vdot \bvec B = A_x B_x \,\uvec x \vdot \uvec x + A_y B_y \,\uvec y \vdot \uvec y + A_x B_y \,\uvec x \vdot \uvec y + A_y B_x \,\uvec y \vdot \uvec x~.
% \end{equation}
% 其中 $\uvec x \vdot \uvec y = \uvec y \vdot \uvec x = 0$ (相互垂直), 而 $\uvec x \vdot \uvec x = \uvec y \vdot \uvec y = 1$ (相互平行且模长都为1)。 所以最后结果为
\begin{equation}\label{eq_Dot_5}
\bvec A \vdot \bvec B = A_x B_x + A_y B_y~.
\end{equation}
同理,空间向量内积的坐标定义为
% 可以在三维直角坐标系 $xyz$ 中把内积结果用坐标表示
\begin{equation}\label{eq_Dot_4}
\bvec A \vdot \bvec B = A_x B_x + A_y B_y + A_z B_z	~.
\end{equation}

可以想象,如果我们有“更高维度的向量”,我们也可以用坐标定义内积:
\begin{equation}
A_1 B_1 + \dots + A_n B_n ~.
\end{equation}

% Giacomo:高中数学不应该引入这么复杂的东西。
% 注意内积的代数定义也可以拓展到更高维的情况甚至复数的情况, 即对于复数域的 $u_1, u_2, \dots, u_N$ 和 $v_1, v_2, \dots, v_N$,
% \begin{equation}
% \bvec u \vdot \bvec v = \sum_k u_k v_k~.
% \end{equation}

% 注意虽然上式中的坐标取决于正交归一基底的选取, 但内积的结果却与基底的选取无关。 这是因为内积的几何定义是两个几何向量间的几何性质, 与基底无关。

\subsection{两种定义的等价性}

在平面直角坐标系 $xy$ 中,有两个特殊的单位向量
\begin{equation}
\hat{i}: = 
\end{equation}


\subsection{内积的性质}

内积满足交换率:
\begin{equation}\label{eq_Dot_2}
\bvec A \vdot \bvec B = \bvec B \vdot \bvec A~.
\end{equation}
由几何定义,易证。

内积满足分配律:
\begin{equation}\label{eq_Dot_3}
\bvec A \vdot (\bvec B + \bvec C) = \bvec A \vdot \bvec B + \bvec A \vdot \bvec C~,
\end{equation}
考虑坐标定义,我们有
\begin{equation}
\begin{aligned}
\bvec A \vdot (\bvec B + \bvec C) &= A_1 (B_1 + C_1) + A_2 (B_2 + C_2) \\
&= A_1 B_1 + A_1 C_1 + A_2 B_2 + A_2 C_2 \\
&= (A_1 B_1 + A_2 B_2) + (A_1 C_1 + A_2 C_2) \\
&= \bvec A \vdot \bvec B + \bvec A \vdot \bvec C
\end{aligned}~
\end{equation}
空间向量的证明类似。

注意内积不满足结合律,即
\begin{equation}
(\bvec A \vdot \bvec B) \bvec C \ne  \bvec A (\bvec B \vdot \bvec C)~.
\end{equation}
前者是 $\bvec C$ 方向的向量,后者是 $\bvec A$ 方向的向量,显然不一定相等。

\subsection{证明内积的分配律}
\begin{figure}[ht]
\centering
\includegraphics[width=10.5cm]{./figures/474fd3eb555c4ce3.pdf}
\caption{内积分配律的证明} \label{fig_Dot_2}
\end{figure}

如\autoref{fig_Dot_2}, 令 $\bvec D \equiv \bvec B + \bvec C$, 把 $\bvec A \vdot \bvec B$,  $\bvec A \vdot \bvec C$,  $\bvec A \vdot \bvec D$ 分别用几何定义理解为 $\bvec B$,  $\bvec C$,  $\bvec D$ 在 $\bvec A$ 上的投影乘 $\abs{\bvec A}$, 且令投影长度分别为 $L_B, L_C, L_D$。 那么要证明 $\bvec A \vdot (\bvec B + \bvec C) = \bvec A \vdot \bvec D = \bvec A \vdot \bvec B + \bvec A \vdot \bvec C$, 只需证明 $L_D = L_B + L_C$ 即可。现在把 $\bvec B$ 平移使其起点与 $\bvec C$ 的终点对接(投影长度不变)。 从图中立即得出 $L_D = L_B + L_C$。  
