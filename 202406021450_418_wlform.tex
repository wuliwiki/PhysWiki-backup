% 公式(数理逻辑)
% license Usr
% type Tutor

\pentry{命题\nref{nod_prop},命题的连接词\nref{nod_propco}}{nod_587f}
\begin{definition}{原子公式}
单个命题常量或命题变元成为一个\textbf{原子公式}。
\end{definition}
\begin{definition}{(合式)公式}
\textbf{(合式)公式}按以下规则递归地定义:
\begin{enumerate}
\item 原子公式是合式公式。
\item 若 $A$ 是一个合式公式,则 $\neq A$ 也是一个合式公式。
\item 若 $A$ 与 $B$ 都是合式公式,则 $(A \land B)$、$(A \lor B)$、$(A \to B)$ 与 $(A \leftrightarrow B)$ 都是合式公式。
\item \textbf{有限次}的 $1$、$2$、$3$ 组合出来的都是合式公式。
\end{enumerate}

\end{definition}
\begin{definition}{公式的类型}
\begin{enumerate}
\item \textbf{矛盾式(永假式)}:公式在各种的可能情况下均为假,均不成立。
\item \textbf{重言式(永真式)}:公式在各种的可能情况下均为真,均成立。
\item \textbf{可满足式}:公式存在至少一种可能的解释使得公式成立。
\end{enumerate}
\end{definition}

\begin{definition}{公式的等价}
两个公式在所有可能解释的情况下真值均相同,就称这两个公式\textbf{等价},用符号 $\Leftrightarrow$ 或 $=$。
\end{definition}