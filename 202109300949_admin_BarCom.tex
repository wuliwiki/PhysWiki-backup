% 隔板法(排列组合)

\pentry{组合\upref{combin}}

\footnote{参考 Wikipedia \href{https://en.wikipedia.org/wiki/Stars_and_bars_(combinatorics)}{相关页面}.}在排列组合问题中, \textbf{隔板法(stars and bars)}常用解决以下问题:

把 $n$ ($n = 1,2,\dots$)个不加区分的小球放进 $m$($1\leqslant m\leqslant n$) 个有编号的盒子, 每个盒子至少有一个小球, 有多少种不同的方法?
\addTODO{图}

我们可以想象这些小球排成一列后被 $m-1$ 个隔板隔开, 每一组被隔开的小球就相当于装进一个盒子中. 小球之间一共有 $N-1$ 个空隙可以插入隔板, 一个空隙最多插一个隔板, 所以不同的情况数就是组合 $C_{N-1}^{n-1}$.

\subsection{盒子可以为空}
\pentry{范德蒙恒等式\upref{ChExpn}}

把 $n$ ($n=1,2\dots$)个不加区分的小球放进 $m$ ($m=1,2\dots$)个有编号的盒子, 盒子可为空, 有多少种不同的方法?

把所有的情况根据非空盒子的个数分类. 非空盒子个数可能为 $i=1$ 个( $n$ 个小球都在里面), $i=2$ 个, 一直到 $i=\min\qty{m,n}$ 个(若 $n\leqslant m$ 则每个小球都在不同的盒子). 首先从 $m$ 个盒子里面选择 $i$ 个非空盒子会有 $C_n^i$ 种情况. 再考虑这 $i$ 个有编号盒子装 $n$ 个小球(不为空)又有几种情况: 用隔板法得到共有 $C_{n-1}^{i-1}$ 种. 所以一个 $i$ 对应 $C_m^i C_{n-1}^{i-1}$ 种情况. 最后把所有不同 $i$ 的情况数加在一起, 得出所有情况的总数为
\begin{equation}
\sum_{i = 1}^{\min\qty{m,n}} C_m^i C_{n-1}^{i-1} = \sum_{i=1}^{\min\qty{m,n}}  C_m^i C_{n-1}^{n-i}
\end{equation}
这里使用了 $C_a^b = C_a^{a-b}$ (\autoref{combin_eq4}~\upref{combin}). 又由范德蒙恒等式\upref{ChExpn}, 有
\begin{equation}
\sum_{i=1}^{\min\qty{m,n}}  C_m^i C_{n-1}^{n-i} = C_{m+n-1}^n = \frac{(m+n-1)!}{n!(m - 1)!}
\end{equation}
这就是最后的答案.

