% 可测函数
% 叶葛洛夫定理|Egoroff|Lusin|测度|Lebesgue积分|实变函数|广义实数

为了引入新的积分思想,我们首先要讨论可测函数的相关概念.

\subsection{广义实数}

为了方便将来的讨论,我们将实数扩充为“广义实数”,即在实数集合中再添加$\pm\infty$这两个元素.

$\pm\infty$的运算规则简述如下:

\begin{equation}
(\pm\infty)+( \pm \infty)=\pm \infty
\end{equation}

\begin{equation}
(\pm\infty)(\pm\infty)=+\infty
\end{equation}

\begin{equation}
(\pm\infty)(\mp\infty)=-\infty
\end{equation}

设$a, b, c$都是实数,且$b<0<c$,则还有:

\begin{equation}
a+(\pm\infty)=\pm\infty
\end{equation}

\begin{equation}
b(\pm\infty)=\mp\infty
\end{equation}

\begin{equation}
c(\pm\infty)=\pm\infty
\end{equation}

而$(\pm\infty)+(\mp\infty)$是不允许的运算,多数情况下$0\cdot(\pm\infty)$也是不允许的.

最后,还要再补充一个关系:对于任意实数$a$,都有
\begin{equation}
-\infty<a<+\infty
\end{equation}


这样一来,我们就可以不再局限于实数集,而是在广义实数集上定义函数.



\subsection{可测函数}

可测函数是构造Lebesgue积分思想的砖块,就像“柱子的面积”是Riemann积分思想的砖块一样.

\begin{definition}{可测函数}\label{MsbFun_def1}
设$f$是$E\subseteq\mathbb{R}^n$上的函数(值域为广义实数集),其定义域$E$为一个可测集.如果对于任意实数$a$,都有$\{x\in E|f(x)\geq a\}$是可测集,那么称$f$是$E$上的\textbf{可测函数(measurable function)}.
\end{definition}


可测函数要保持的性质是,在对值域进行分划的时候,各分划区间通过逆映射对应的“底面”是可测集,这样才能满足可加性.\autoref{MsbFun_def1} 中的限制看起来没有“任意分划得到的底面都是可测集”,但实际上定义到这个程度已经够了,这可以从下面的\autoref{MsbFun_the1} 看出:


\begin{theorem}{}\label{MsbFun_the1}
设$E$是$\mathbb{R}^n$上的可测集,$f$是$E$上的一个函数,则以下命题等价:
\begin{enumerate}
\item $f$是可测函数;
\item 对于任意实数$a$,$\{x\in E|f(x)>a\}$是可测集;
\item 对于任意实数$a$,$\{x\in E|f(x)\leq a\}$是可测集;
\item 对于任意实数$a$,$\{x\in E|f(x)<a\}$是可测集.
\end{enumerate}
\end{theorem}

\textbf{证明}:

为方便讨论,记$\{x\in E|f(x)\geq a\}=E_a$.

于是$\{x\in E|f(x)>a\}=\bigcup_{k=1}^\infty E_{a+1/k}$,因此它也是可测集.

而$\{x\in E|f(x)<a\}=E-E_a$,故它也是可测集.

最后,$\{x\in E|f(x)\leq a\}=E-\{x\in E|f(x)>a\}$,因此也是可测集.

类似地,用这四句话中任意一个作为可测集的定义,都可以通过上述方式证明其它几句话所描述的也是可测集.

\textbf{证毕}.





\begin{corollary}{}
设$f$为$E$上的可测函数,则$E_{+\infty}=\{x|f(x)=+\infty\}$和$E_{-\infty}=\{x|f(x)=-\infty\}$都是可测集.
\end{corollary}



\begin{definition}{几乎处处}
“几乎处处”,指“除了一个零测集以外都”.比如说,如果说两个函数几乎处处相等,那就是说它们不相等的地方是一个零测集.
\end{definition}

注意,空集也是零测集,所以“几乎处处”也包含了“处处”的情况.

根据可测集的性质,我们容易得到以下\autoref{MsbFun_the2} 和\autoref{MsbFun_the3} ,证明留作习题.

\begin{theorem}{}\label{MsbFun_the2}
设$E\subseteq\mathbb{R}^n$是可测集,$f$和$g$是在$E$上定义的函数.如果两个函数几乎处处相等,那么$f$可测当且仅当$g$可测.
\end{theorem}



\begin{theorem}{}\label{MsbFun_the3}
如果各$E_i\subseteq\mathbb{R}^n$都是可测集,且$f$是每个$E_i$上的可测函数,那么$f$也是$\bigcup_{i=1}^\infty$上的可测函数.
\end{theorem}

根据\autoref{MsbFun_the1} ,我们只需要关注函数几乎处处的性质即可.

\begin{theorem}{}
设$f$和$g$都是可测集$E\subseteq\mathbb{R}^n$上的可测函数,且都几乎处处有意义,那么$af+bg$和$f\cdot g$也都是可测函数
\end{theorem}




