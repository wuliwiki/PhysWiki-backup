% 几何代数(综合)
% license CCBYSA3
% type Wiki

本文根据 CC-BY-SA 协议转载翻译自维基百科\href{https://en.wikipedia.org/wiki/Geometric_algebra}{相关文章}。

在数学中,几何代数(geometric algebra,亦称为Clifford 代数)是一种能够表示和操作几何对象(如向量)的代数。几何代数建立在两个基本运算之上:加法与几何乘积。向量的乘法会生成更高维的对象,称为多向量。与其他用于处理几何对象的形式体系相比,几何代数的一个显著特征是它支持向量的除法(尽管通常不能对所有元素都定义),并且允许不同维度对象之间的加法。

几何乘积最早由赫尔曼·格拉斯曼简要提出[1],他当时主要致力于发展与之密切相关的外代数。1878 年,威廉·金顿·克利福德在格拉斯曼工作的基础上大幅扩展,形成了如今通常被称为Clifford 代数的体系(尽管克利福德本人选择称其为“几何代数”)。克利福德将 Clifford 代数及其乘法定义为格拉斯曼代数与哈密顿四元数代数的统一体。进一步引入格拉斯曼外积的对偶,可以得到Grassmann–Cayley 代数。在 1990 年代末期,基于平面的几何代数以及共形几何代数(CGA)分别为欧几里得几何和经典几何提供了框架[2]。在实际应用中,这些运算及其若干衍生操作,使得代数中的元素、子空间与运算可以自然地对应到几何解释。然而,数十年来几何代数一度被忽视,其发展被当时新兴的向量分析(主要用于描述电磁学)所掩盖。直到 1960 年代,大卫·赫斯特尼斯再次推广“几何代数”这一术语,并强调它在相对论物理学中的重要性[3]。

标量和向量在几何代数中具有其通常的解释,并且构成了代数中的不同子空间。双向量提供了一种比三维向量分析中的伪向量更自然的表示方式。伪向量通常通过叉乘得到,用来描述有向面积、有向旋转角、力矩、角动量以及磁场等物理量。三向量可以表示有向体积,依此类推。几何代数中的一种元素称为刀片,它可用于表示子空间以及到该子空间的正交投影。旋转与反射也能作为代数的元素来表示。与传统的向量代数不同,几何代数能够自然而然地适应任意维度以及任意二次型,例如相对论中的情况。

在物理学中,几何代数的应用实例包括时空代数(以及较少使用的物理空间代数)。几何分析是几何代数的扩展,它结合了微分与积分,可以用来建立其他理论,例如复分析与微分几何——其方法是使用 Clifford 代数来取代传统的微分形式。

几何代数的推广者中,最著名的是大卫·赫斯特尼斯[4]和克里斯·多兰[5],他们认为几何代数应当成为物理学的首选数学框架。支持者声称,它在许多领域中(包括经典力学、量子力学、电磁理论以及相对论)都能提供简洁而直观的描述[6]。此外,几何代数还在计算机图形学[7]与机器人学中找到了计算工具的用途。

