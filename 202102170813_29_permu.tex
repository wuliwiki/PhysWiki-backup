% 对称群与置换的奇偶性
% keys 置换,轮换,对换

  在上一节中,我们简单的介绍了由一个固定的有限集合$\{1, 2, ..., n\}$上所有置换组成的集合连同映射的符合所构成的群:对称群.在这一节我们将更加深入的讨论对称群的一些基本结构和性质.

\subsection{置换的表示方式}
一个置换实际上就是一个函数,要把一个函数表示出来要么给出表达式(一个通用的对应关系),要么把所有的点对应的函数值一一列举出来.对于置换来说我们通常采用后者.
\begin{example}{}
在$S_4$中设一个置换$\sigma=\begin{pmatrix}
1\ 2\ 3\ 4\\ 2\ 1\ 4\ 3
\end{pmatrix}$
\end{example}
这个写法的意思是,作为一个置换.$\sigma$把1对应到了2,2对应到1,3对应到4,4对应到3. 值得注意的是,第一行虽然习惯于按照大小来排列从1到n这些数,但是实际上只要每个数字都出现一次就够了,不按照大小顺序来排列同样也可以表达出一个函数所包含的全部信息.另外,把两行换位置所得到的置换
$\begin{pmatrix}
2\ 1\ 4\ 3\\ 1\ 2\ 3\ 4
\end{pmatrix}$
就是$\sigma$的逆元.


对于一些比较特殊的置换,我们有更简便的表示方法.
\begin{example}{}
仍然考虑$S_4$,设$\tau=\begin{pmatrix}
1\ 2\ 3\ 4\\ 2\ 3\ 1\ 4
\end{pmatrix}$ 在这个置换下,1,2,3进行了“轮换",即每个数字都被对应到它们中的下一个.4被保持不动.



\end{example}
