% 麦克斯韦方程组(外微分形式)
% keys Maxwell|外代数|外微分|exterior product|exterior algebra|Grassmann algebra|exterior derivative|微分形式|differential form|外导数|楔积|wedge product|霍奇算子|霍奇星算子|Hodge|Hodge star operator

\addTODO{本词条处于草稿阶段。}

\subsection{前两个方程}
\pentry{外导数\upref{ExtDer}}

\addTODO{百科中尚未创建“规范场论”部分。使用引用\cite{KnotsVol4}。}


在 $\mathbb{R}^3$ 中考虑电磁场,三个空间轴分别为 $x, y, z$ 轴。

考虑麦克斯韦方程组中的两个方程,$\nabla\cdot\bvec{B}=0$ 和 $\nabla\times\bvec{E}=-\partial_t\bvec{B}$。为了尝试用外代数来表达这两个式子,我们就要把 $\bvec{B}$
表示成一个2-形式 $B=B_z\dd x\wedge\dd y+B_x\dd y\wedge\dd z+B_y\dd z\wedge\dd x$,把 $\bvec{E}$ 表示成一个1-形式 $E=E_x\dd x+E_y\dd y+E_z\dd z$,这样以上两个方程的左边就都可以写成外导数的形式,从而有:

\begin{equation}\label{eq_MWEq2_3}
\dd B=0~
\end{equation}
和
\begin{equation}\label{eq_MWEq2_4}
\dd E=\partial_tB~.
\end{equation}

其中 $\mathbb{R}^4$ 可以写成三维空间和一维时间的乘积:$\mathbb{R}^3\times\mathbb{R}$。这个四维欧几里得空间中的时间轴记为 $x^0$ 轴,空间轴则记为 $x^1, x^2, x^3$ 轴。要注意在这种表示下,$\partial_tB$ 就成了 $\partial_0B$。

现在考虑用一个统一的2-形式 $F=B+E\wedge\dd x^0$ 来表示电磁场\footnote{电场外积一个 $\dd x^0$ 是为了凑成合适的2-形式。},也就是

\begin{equation}
\ali{
    F = {}&B_z\dd x^1\wedge\dd x^2+B_x\dd x^2\wedge\dd x^3+B_y\dd x^3\wedge\dd x^1\\
    &+E_x\dd x^1\wedge \dd x^0+E_y\dd x^2\wedge \dd x^0+E_z\dd x^3\wedge\dd x^0~.
}
\end{equation}
这个\textbf{电磁场形式}的外导数计算如下,我们把结果分成三个部分来方便阅读:
\begin{equation}\label{eq_MWEq2_1}
\ali{
    \dd F ={}& \partial_0 B_z\dd x^0\wedge \dd x^1\wedge \dd x^2 +\\& \partial_0 B_x \dd x^0\wedge \dd x^2\wedge \dd x^3 +\\& \partial_0 B_y \dd x^0\wedge \dd x^3\wedge \dd x^1\\
    &+\\&(\partial_1 B_x+\partial_2 B_y+\partial_3 B_z)\dd x^1\wedge \dd x^2\wedge \dd x^3\\
    &+\\&(\partial_2 E_z-\partial_3 E_y)\dd x^2\wedge \dd x^3\wedge\dd x^0+\\&(\partial_3 E_x-\partial_1 E_z)\dd x^3\wedge \dd x^1\wedge\dd x^0 +\\&(\partial_1 E_y-\partial_2 E_x)\dd x^1\wedge \dd x^2\wedge\dd x^0 ~.\\
}
\end{equation}

\autoref{eq_MWEq2_1} 右边第一个部分和第三个部分是同类项,应该相加,而第二个部分和它们都无关。第一个部分对应$\partial_0\bvec{B}$,第三个部分对应$\nabla\times\bvec{E}$,第二个部分对应$\nabla\cdot\bvec{B}$,由此易得,\autoref{eq_MWEq2_3} 和\autoref{eq_MWEq2_4} 可以统一用一个式子来表达:
\begin{equation}
\dd F = 0~.
\end{equation}






\subsection{后两个方程}

\pentry{霍奇星算子\upref{HodgeO}}

% \begin{equation}
% \prod_{n\in N }(-\frac{1}{n},\frac{1}{n})~.
% \end{equation}


\footnote{该小节节选自《代数学基础》,故所用符号与上一小节有所不同。}接下来看剩下的两个方程:
\begin{equation}
    \left\{
    \begin{aligned}
        \nabla\cdot \bvec{E} ={}& \rho\\
        \nabla\times \bvec{B} ={}& \bvec{j}+\partial_0\bvec{E}~.
    \end{aligned}
    \right
\end{equation}



形式上,这两个方程和前面两个方程有两个区别。第一,电场和磁场的位置调换了,现在电场求的是散度,磁场求的是旋度;第二,方程右边多了一项常数$(\rho, \bvec{J})$。

\textbf{第一个区别}提示我们要使用Hodge对偶,这样\textbf{从}$\mathbb{R}^{0,3}$\textbf{的视角看来}
\footnote{$\mathbb{R}^{s, t}$指配备了一个二次型的实线性空间$\mathbb{R}^{s+t}$,且$s, t$分别为这个二次型的正号、负号之数量。}
,电场从$1$-形式转化为$2$-形式,求外微分就是求其散度;而磁场从$2$-形式转化为$1$-形式,求外微分就是求其旋度。

定义Hodge星算子需要一个非退化双线性形式,我们直接用Minkowski度规——下面会讨论为什么不用欧几里得度量。

于是,$\dd\star F=0$就等同于
\begin{equation}
    \left\{
        \begin{aligned}
            \nabla\times \bvec{B} ={}& \partial_0\bvec{E}\\
            \nabla\cdot \bvec{E} ={}& 0~.
        \end{aligned}
    \right
\end{equation}
注意和$\nabla\times\bvec{E}=-\partial_0\bvec{B}$区分,为什么会有这个差异?


计算一下就知道了:
\begin{equation}
    \begin{aligned}
        \star E ={}& \star (E_1\dd x^1\wedge \dd x^0 + E_2\dd x^2\wedge \dd x^0 + E_3\dd x^3\wedge \dd x^0)\\
        ={}& E_1\dd x^2\wedge \dd x^3 + E_2\dd x^3\wedge \dd x^1 + E_3\dd x^1\wedge \dd x^2~.
    \end{aligned}
\end{equation}
而
\begin{equation}
    \begin{aligned}
        \star B ={}& B_1\dd x^2\wedge \dd x^3+B_2\dd x^3\wedge \dd x^1 + B_3\dd x^1\wedge \dd x^2\\
        ={}& -(B_1\dd x^1\wedge \dd x^0 + B_2\dd x^2\wedge \dd x^0 + B_3\dd x^3\wedge \dd x^0)~.
    \end{aligned}
\end{equation}
所以差异是因为,$\star F$相比$F$,在形式上是把$B$替换为$E$,而把$E$替换为$-B$,多且仅仅多了一个负号。而仔细看计算过程会发现,这个负号是来自Minkowski度规的,如果换用欧几里得度量就会出现两个负号,负负得正,导致$\dd \star F=0$对应的方程形式上就是$\dd F=0$对应的方程中直接调换$\bvec{E}$和$\bvec{B}$的位置而已。


\textbf{第二个区别}提示我们,式子的右端不再是$0$了。方程右边多出的这一项,在传统的向量分析里被认为是\textbf{切向量场},即$\bvec{J}=\rho\partial_0+j^1\partial_1+j^2\partial_2+j^3\partial_3$。但是我们把电磁场$F$表示为$2$-形式,所以起码要考虑改用\textbf{余切向量场},即$\bvec{J}$的对偶
\footnote{就像量子力学里一样,虽然有两个互为对偶的空间(左矢空间和右矢空间),但它们都是量子态的表示空间,同一个态有左矢和右矢两种表示。}
,记为$J$。同样,我们用Minkowski度规来定义切丛与余切丛之间的同构。



综上,再考虑到$J$是$1$-形式,我们还需要额外添加一个星算子,把$3$-形式$\dd \star F$化为$1$-形式$\star \dd \star F$。

现在,我们已经从散度、旋度、$k$-形式的概念直观\textbf{猜}出来式子的两端应该分别有$\star \dd \star F$和$J$,接下来就要通过计算来进一步确定式子的形态:


\begin{equation}
    \begin{aligned}
        \star \dd \star F ={}& \star \dd \star (E_1\dd x^1\wedge \dd x^0 + E_2\dd x^2\wedge \dd x^0 + E_3\dd x^3\wedge \dd x^0)+\\
        & \star \dd \star (B_1\dd x^2\wedge \dd x^3+B_2\dd x^3\wedge \dd x^1 + B_3\dd x^1\wedge \dd x^2)\\
        ={}
        & \star \dd (E_1\dd x^2\wedge \dd x^3 + E_2\dd x^3\wedge \dd x^1 + E_3\dd x^1\wedge \dd x^2)-\\
        & \star \dd (B_1\dd x^1\wedge \dd x^0 + B_2\dd x^2\wedge \dd x^0 + B_3\dd x^3\wedge \dd x^0)\\
        ={}
        & \star (\partial_1E_1+\partial_2E_2+\partial_3E_3)\dd x^1\wedge \dd x^2\wedge \dd x^3+\\
        & \star \partial_0E_1\dd x^0\wedge \dd x^2\wedge \dd x^3 +\\
        & \star \partial_0E_2\dd x^0\wedge \dd x^3\wedge \dd x^1 +\\
        & \star \partial_0E_3\dd x^0\wedge \dd x^1\wedge \dd x^2+\\
        & \star (\partial_2B_1-\partial_1B_2)\dd x^1\wedge \dd x^2 \wedge \dd x^0+\\
        & \star (\partial_3B_2-\partial_2B_3)\dd x^2\wedge \dd x^3 \wedge \dd x^0+\\
        & \star (\partial_1B_3-\partial_3B_1)\dd x^3\wedge \dd x^1 \wedge \dd x^0\\
        ={}
        & (\partial_1E_1+\partial_2E_2+\partial_3E_3)\dd x^0+\\
        & (\partial_0 E_1\dd x^1+\partial_0 E_2\dd x^2+\partial_0 E_3\dd x^3)+\\
        & (\partial_2B_1-\partial_1B_2)\dd x^3+\\
        & (\partial_3B_2-\partial_2B_3)\dd x^1+\\
        & (\partial_1B_3-\partial_3B_1)\dd x^2~.
    \end{aligned}
\end{equation}

由于$J$是$\bvec{J}=\rho\partial_0+j^1\partial_1+j^2\partial_2+j^3\partial_3$关于Minkowski度规的对偶,故
\begin{equation}
J = \rho\dd x^0 - j^1\dd x^1 - j^2\dd x^2 - j^3\dd x^3~.
\end{equation}

比较各$\dd x^i$前的系数可得,$\star \dd \star F = J$等价于
\begin{equation}
\left\{
\begin{aligned}
\partial_1E_1+\partial_2E_2+\partial_3E_3 = \rho&\implies \nabla\cdot \bvec{E}=\rho\\
\left.
\begin{aligned}
(\partial_3B_2-\partial_2B_3)+\partial_0E_1 ={}& -j^1\\
(\partial_1B_3-\partial_3B_1)+\partial_0E_2 ={}& -j^2\\
(\partial_2B_1-\partial_1B_2)+\partial_0E_3 ={}& -j^3
\end{aligned}
\right\}&
\implies -\nabla\times \bvec{B}+\partial_0\bvec{E}=-\bvec{j}~.
\end{aligned}
\right.
\end{equation}



因此,剩下两个方程可以写为
\begin{equation}
    \star \dd \star F = J~.
\end{equation}







% \subsection{符号问题}



% 电磁场理论中有两个因素会影响Hodge算子的符号:Minkowski度规的号差和时空的定向。上述计算默认使用的号差是$-2$,时空的定向为$\dd x^0\wedge \dd x^1\wedge \dd x^2\wedge \dd x^3$。

% 对于$\dd F = 0$,由于该方程只涉及外微分,故不受上述两个因素影响。但是$\star \dd \star F = -J$就不同了。

% 如果号差不变,但时空的定向改为$\dd x^1\wedge \dd x^2\wedge \dd x^3\wedge \dd x^0$,那么每一步

















