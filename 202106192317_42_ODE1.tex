% 一阶线性微分方程
% 微积分|微分方程|常微分方程|一阶线性微分方程|齐次|常数易变法

\begin{issues}
\issueNeedCite
\issueOther{需要补充例子}
\end{issues}

\pentry{常微分方程\upref{ODE}}

\footnote{本文参考 Wikipedia \href{https://en.wikipedia.org/wiki/Ordinary_differential_equation}{相关页面}和 \cite{同济高}}具有以下形式的微分方程叫做\textbf{一阶线性微分方程(first order linear differential equation)}
\begin{equation}\label{ODE1_eq1}
\dv{y}{x} + p(x)y = f(x)
\end{equation}
其中函数 $p(x)$ 和 $f(x)$ 是已知的, 函数 $y(x)$ 是未知的, 称为方程的解. 这三个函数的自变量都是实数, 函数值可以是实数或者复数. 若 $f(x) \equiv 0$, 则称方程为\textbf{齐次的(homogeneous)}, 否则就是\textbf{非齐次的(inhomogeneous)}. 所以 $f(x)$ 叫做方程的\textbf{非齐次项(inhomogeneous term)}.

\subsection{线性}
若干函数 $y_1(x), y_2(x), \dots$ 的线性组合\footnote{线性组合是线性代数中的概念, 线性代数中, 线性空间都可以做线性组合, 例如几何矢量(\autoref{GVecOp_eq1}~\upref{GVecOp}). 特定函数的集合可以看作线性空间(\autoref{LSpace_ex2}~\upref{LSpace}).}定义为他们分别乘以一个常数再相加
\begin{equation}
c_1 y_1(x) + c_2 y_2(x) + \dots
\end{equation}
由于求导操作是线性(\autoref{DerRul_eq1}~\upref{DerRul})的, 容易证明若 $y_i(x)$ ($i = 1, 2,\dots$)都是其次方程的解, 那么它们的线性组合也必定是齐次方程的解. 微分方程的这种性质就叫做\textbf{线性(linear)}.

\subsection{一阶常系数齐次线性微分方程}
当\autoref{ODE1_eq1} 中的 $p(x) \equiv b$ 是常数时, 我们说\autoref{ODE1_eq1} 的方程是常系数的. 先来看齐次的情况, 方程为
\begin{equation}\label{ODE1_eq3}
\dv{y}{x} + by = 0
\end{equation}
也就是说我们想要求一个函数 $y(x)$, 使得它的导数和 $y(x)$ 自己成正比. 一个最显然的解就是 $y(x) \equiv 0$, 但它太简单了没什么意义, 我们说这个解是\textbf{平凡的(trivial)}. 对于非零解, 一个直接的猜测就是自然指数函数 $y(x) = \E^{ax}$ 其中 $a$ 为常数. 把它代入\autoref{ODE1_eq3} 得
\begin{equation}
ay(x) + by(x) = 0
\end{equation}
所以消去 $y(x)$ 得 $b = -a$. 再考虑到方程的线性, 所以我们把这个解乘以任意常数 $C$ 同样是方程的解, 所以\autoref{ODE1_eq3} 的通解为
\begin{equation}\label{ODE1_eq5}
y(x) = C\E^{-bx}
\end{equation}

\begin{example}{稀释酒精}
一个容器中装有质量为 $M$ (千克)的酒精, 从某时刻开始以速率 $I$ (千克每秒)加水, 并以同样的速率从底部排水, 保持容器中液体的总质量不变. 假设容器中有某种快速搅拌装置, 使两种液体时刻保持充分混合, 求容器中酒精浓度 $x$(酒精质量比总质量)随时间的变化关系.

解:令容器中酒精的质量为 $m_1$, 水的质量为 $M - m_1$. 流出容器的液体速率可以分为两部分: 流出酒精的速率为 $xI$, 流出水的速率为 $I - xI$. 所以容器中酒精质量的变化率为
\begin{equation}
m'_1 = -xI
\end{equation}
酒精浓度定义为 $x = m_1/M$, $M$ 是常数, 所以
\begin{equation}
x' = \qty(\frac{m_1}{M})' = \frac{m'_1}{M} = -\frac{I}{M}x
\end{equation}
该微分方程符合\autoref{ODE1_eq3} 的结构, 可以直接写出方程的解为
\begin{equation}
x(t) = C\E^{-It/M}
\end{equation}
但这里的 $C$ 到底是什么呢? 如果我们令开始的时间为 $t = 0$, 那么根据题目条件必须满足 $x(0) = 1$, 代入得 $C = 1$.
\end{example}

\subsection{一阶常系数非齐次线性微分方程}
\addTODO{非齐次, 常数变易法.}
现在,来看 $p(x) \equiv b$ 是常数时非齐次的情况,方程为
\begin{equation}\label{ODE1_eq4}
\dv{y}{x} + by = f(x)
\end{equation}

我们用所谓的常数变易法来求常系数非齐次线性方程\autoref{ODE1_eq4} 的通解.这方法是把\autoref{ODE1_eq5} 的通解中的 $C$ 换为 $x$ 的未知函数 $u(x)$,即作变换
\begin{equation}
y(x)=ue^{-bx}
\end{equation}
于是
\begin{equation}

\end{equation}

\subsection{一般情况}
我们先来看\autoref{ODE1_eq1} 对应的齐次方程
\begin{equation}\label{ODE1_eq2}
\dv{y}{x} + p(x)y = 0
\end{equation}
这是一个可分离变量的方程% 未完成:什么是分离变量?
, 分离变量得
\begin{equation}
\frac{\dd{y}}{y} = -p(x) \dd{x}
\end{equation}
两边积分得
\begin{equation}
\ln\abs{y} = -\int p(x) dx + C
\end{equation}
两边取自然指数得
\begin{equation}
y = \pm \E^C \E^{-\int p(x) \dd{x}}
\end{equation}
把 $\pm \E^C $ 整体看做一个任意常数 $C$, 上式变为.
\begin{equation}\label{ODE1_eq6}
y = C \E^{-\int p(x) \dd{x}}
\end{equation}
这就是一阶线性齐次微分方程\autoref{ODE1_eq2} 的通解, 也叫\autoref{ODE1_eq1} 的\textbf{齐次解}.

\subsection{常数变易法}

现在我们用\textbf{常数变易法}来解非齐次方程\autoref{ODE1_eq1}. 为书写方便, \autoref{ODE1_eq6} 中令 $y_0(x) = \exp(-\int p(x) \dd{x})$. 假设上式中的 $C$ 是一个函数 $C(x)$ 而不是常数, 代入\autoref{ODE1_eq1} 得
\begin{equation}
C'y_0 + C[y_0' + p(x)y_0] = f(x)
\end{equation}
由于 $y_0$ 是齐次解, 上式方括号中求和为 0, 分离变量得
\begin{equation}
\dd{C}= \frac{f(x)}{y_0} \dd{x}
\end{equation}
两边积分得
\begin{equation}
C(x) = \int \frac{f(x)}{y_0} \dd{x}
\end{equation}
所以一阶线性非齐次微分方程的通解为
\begin{equation}\label{ODE1_eq10}
y = y_0  \int \frac{f(x)}{y_0} \dd{x}
\end{equation}
其中
\begin{equation}\label{ODE1_eq11}
y_0(x) = \E^{-\int p(x) \dd{x}}
\end{equation}
注意待定常数包含在\autoref{ODE1_eq10} 的不定积分中, \autoref{ODE1_eq11} 中的不定积分产生的待定常数在代入\autoref{ODE1_eq10} 后可消去.
