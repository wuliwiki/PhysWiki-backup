% 双共轭梯度法解线性方程组(BiCG)
% license Xiao
% type Tutor

\begin{issues}
\issueDraft
\end{issues}

\pentry{多元函数的极值\upref{MulPlo},线性方程组解的结构\upref{LinEq}, 厄米矩阵、自伴矩阵\upref{HerMat}}

\subsection{共轭梯度法}

\footnote{本文参考 \cite{NR3}。}要求解\textbf{对称正定矩阵(symmetric and positively defined, SPD)} $\mat A$, 的线性方程组
\begin{equation}\label{eq_ConGra_1}
\mat A \bvec x = \bvec b~.
\end{equation}
只需要令
\begin{equation}\label{eq_ConGra_2}
f(\bvec x) = \frac{1}{2}\bvec x\Tr \mat A\bvec x - \bvec b\Tr \vdot \bvec x~.
\end{equation}
容易证明 $\grad f = \mat A \bvec x - \bvec b$。 注意 $f(x)$ 是一个凹二次函数, 所以取最小值当且仅当梯度为零。 这样, 解方程组的问题就转化为求函数极小值问题。 我们可以用梯度法(链接未完成)来求最小值, 即从出发点 $\bvec x_0$ 开始, 在梯度方向搜索函数最小值的位置 $\bvec x_1$, 再在其梯度方向搜索最小值的位置 $\bvec x_2$……

同理,若 $\mat A$ 是一个厄米矩阵\upref{HerMat}且正定,把\autoref{eq_ConGra_2} 中的转置改为厄米共轭\upref{HerMat}, 同样可以证明 $f$ 取最小值时\autoref{eq_ConGra_1} 成立。 注意此时梯度不太好定义,那么我们可以分别令
\begin{equation}
\pdv{f}{\Re[x_i]} = 0~, \qquad  \pdv{f}{\Im[x_i]} = 0~
\end{equation}
即可证明。

要找现成的程序, Eigen\upref{Eigen} 提供 \href{https://eigen.tuxfamily.org/dox/classEigen_1_1ConjugateGradient.html}{Eigen::ConjugateGradient},适用于对称或者厄米矩阵。 密矩阵\upref{MatSto}和稀疏矩阵\upref{SprMat}都可以。

\subsection{双共轭梯度法}

梯度法可以拓展为\textbf{双共轭梯度法}, 以适用于任意方矩阵的线性方程组。 该方法的优势在于用户只需要向解算器提供矩阵乘矢量的函数, 而不需要提供矩阵本身。 这样, 矩阵可以具有任意的数据结构, 例如各种稀疏矩阵\upref{SprMat}。

当矩阵 $\mat A$ 接近于单位矩阵时, 该方法收敛更快, 因此, 我们可以选择不直接求解\autoref{eq_ConGra_1} 而是求解
\begin{equation}
(\tilde {\mat A}^{-1}\mat A) \bvec x = \tilde {\mat A}^{-1}\bvec b~.
\end{equation}
其中 $\tilde {\mat A}$ 和 $\mat A$ 接近, 但更易于求解。 这样就有 $\tilde {\mat A}^{-1}\mat A \approx \mat I$。 这里 $\tilde {\mat A}$ 通常称为 preconditioner。 该方法称为 \textbf{preconditioned biconjugate gradient method (PBCG)}。 如果你找不到更好的 preconditioner, 通常可以用 $\mat A$ 的对角线充当。 若选择不使用 preconditioner, 也可以直接令 $\tilde {\mat A}$ 为单位矩阵。

PBCG 的 C++ 代码见 \cite{NR3} 的 \verb|linbcg.h|。 另外 C++ 的 Eigen 库\upref{Eigen}也提供 \verb|Eigen::BiCGSTAB| 算法。 \verb|Matlab| 也有 \verb|x = bicgstab(A,b)| 函数(支持复数)。 其中 \verb|STAB| 意思是 \verb|stablized|, 单纯的 \verb|BiCG| 据说并不稳定, 见 Wikipedia \href{https://en.wikipedia.org/wiki/Biconjugate_gradient_stabilized_method}{相关页面}(Matlab 的 \verb|bicgstab| 函数和这里给出的伪代码非常相似)。
