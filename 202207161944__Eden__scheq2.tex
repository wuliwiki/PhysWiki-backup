% 自旋 1/2 粒子的非相对论波函数
% 自旋|薛定谔方程|泡利方程

\pentry{薛定谔方程(单粒子多维)\upref{QMndim},自旋角动量\upref{Spin},自旋角动量矩阵\upref{spinMt}}
\subsection{历史简介}
在量子力学发展的早期,薛定谔首先提出了 Klein Gordon 方程 $\partial^\mu \partial_\mu \phi+m^2\phi=0$,企图描绘遵从相对论变换的电子波动方程,但却遭遇失败.之后薛定谔退而求其次,转而求它的非相对论近似下的方程,得到了著名的薛定谔方程:
\begin{equation}
i\hbar \frac{\partial }{\partial t}\psi = \hat H\psi=\qty[\frac{ 1}{2m}\qty(-i\hbar\nabla)^2+V(x)]\psi
\end{equation}
换言之,粒子波动方程的能量由 $\hat H=(\hat{\bvec p}^2/2m+V)$ 给出,而 $\hat{\bvec p}=-i\nabla/\hbar$ 为动量算符.

虽然非相对论性的薛定谔方程能很好描绘电子的波粒二象性,但却没有给出电子的\textbf{内禀性质},也就是说,电子是个自旋为 $1/2$ 的粒子,则波函数一定是多分量的,而非单分量的.当空间发生旋转的时候,波函数的分量会随着参考系的旋转而发生变换.直到 Pauli (泡利)给出经典电磁场中的自旋 $1/2$ 电子的 Pauli 方程,人们才终于得到了描绘电子的携带自旋信息的非相对论性方程.Pauli 方程为
\begin{equation}
i\hbar\pdv{t} \psi=\qty[\frac{1}{2m}(\hat{\bvec p}-e\bvec A)^2+e\phi-\bvec \mu\cdot \bvec B]\psi = 0
\end{equation}
其中 $e=-q_e$ 为电子的电荷.若用上述方程描述自旋 $1/2$ 的其他粒子,则需要将 $e$ 用相应的电荷代入.其中电子磁矩 $\bvec \mu$ 为
\begin{equation}
\bvec \mu = \frac{e\hbar}{m}\frac{\bvec \sigma}{2}=\frac{e}{m}\bvec S=-g_{\rm spin}\mu_B \bvec S
\end{equation}
$\sigma$ 是 Pauli 矩阵(\autoref{Spin_eq7}~\upref{Spin}),$\bvec S$ 是自旋角动量算符.$\mu_B=|e\hbar/2m|$ 是 Bohr 磁子,$g_{\rm spin}=2$ 被称为自旋朗德(Lande)g 因子.

第一个提出自旋 1/2 粒子的相对论性方程的是 Dirac(狄拉克).Dirac 注意到 Klein Gordon 场中的负能量和负概率问题的原因是


\subsection{自旋 $1/2$ 粒子的非相对论波函数}
\begin{equation}
\psi(x)=\pmat{\phi(x)\\\chi(x)}
\end{equation}

\subsection{Pauli 方程的解}
由于自旋算符是 $2\times 2$ 的矩阵,所以电子波函数的分量有 $2$ 个.