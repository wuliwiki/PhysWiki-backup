% 接收者操作特征曲线
% 接收 操作 特徵曲線 信号检测论 分类

\textbf{接收者操作特征曲线}(Receiver operating characteristic curve, ROC)是一种用于评价二分类器的分类性能的图表。该方法来源于信号检测论,是在第二次世界大战中,由电子信号工程师发明的。在心理学领域中也有广泛应用。

曲线中点的坐标是\textbf{真阳率}(True positive rate, TPR)和\textbf{假阳率}(False positive rate, FPR)。随着分类器阈值的变化,真阳率和假阳率分别随之改变,由此产生一系列的点,然后将相邻两点连接起来,即构成接收者操作特征曲线。