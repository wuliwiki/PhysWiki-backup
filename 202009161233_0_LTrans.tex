% 线性变换
% keys 线性代数|平面旋转变换|线性变换|逆变换|矩阵

% 未完成: 应该重新写, 从几何矢量出发, 给出 “线性” 的定义, 给出一些平面线性变换的例子(包括投影变换), 然后再总结出一般的代数形式, 给出矩阵表示

\begin{issues}
\issueDraft
\issueOther{本词条需要重新创作和整合,融入章节逻辑体系.}
\end{issues}

\pentry{平面旋转变换\upref{Rot2DT}, 映射\upref{map}}

\subsection{代数理解}
从代数的角度来说,对于给出几个数,把它们分别与一些常数相乘再把积相加,得到另外几个数的的过程就叫\textbf{线性变换}. 例如,在“平面旋转变换\upref{Rot2DT}” 中,直角坐标系中任意一点 $P$ 的坐标 $(x,y)$ 绕远点旋转角 $\alpha $ 以后的坐标为
\begin{equation}\label{LTrans_eq1}
\begin{cases}
x' = (\cos\alpha) x + (-\sin\alpha)y\\
y' = (\sin \alpha)x + (\cos\alpha)y
\end{cases}
\end{equation}
这就是一个常见的线性变换,任意给出两个实数 $x,y$, 通过与常数相乘再相加的方法得到两个新的实数  $x',y'$. 

有些线性变换是一一对应的,例如上面的例子中,任何一组 $x,y$, 有且仅有一组 $x',y'$ 与之对应,反之亦然.在这种情况下,这个变换存在\textbf{逆变换}.

\subsection{线性变换的矩阵表示}

由 $n$ 个数 $x_1 \ldots x_n$ 变换到 $m$ 个数 $y_1 \ldots y_n$ 的线性变换的一般形式为
\begin{equation}\label{LTrans_eq2}
\begin{cases}
y_1 = a_{11} x_1 + a_{12} x_2 + \ldots + a_{1n} x_n\\
y_2 = a_{21} x_1 + a_{22} x_2 + \ldots + a_{2n} x_n\\
\quad\; \vdots \\
y_m = a_{m1} x_1 + a_{m2} x_2 + \ldots + a_{mn} x_n
\end{cases}
\end{equation} 
这里一共有 $m \times n$ 个系数,每个系数的下标由两个数组成, $a_{ij}$ 是计算 $y_i$ 时 $x_j$ 前面的系数.为了书写方便,把这些系数写成一个 $m$ 行 $n$ 列的数表,用圆括号括起来,就是表示该变换的\textbf{矩阵}\upref{Mat}.
\begin{equation}\begin{pmatrix}
a_{11} & a_{12} & \ldots & a_{1n}\\
a_{21} & a_{22} & \ldots & a_{2n}\\
 \vdots & \vdots & \ddots & \vdots \\
a_{m1} & a_{m2} & \ldots & a_{mn}
\end{pmatrix}\end{equation} 

\subsection{几何理解}
\pentry{几何矢量\upref{GVec}}

我们假设\autoref{LTrans_eq2} 中的数都是实数, 那么我们可以把 $(x_1, \dots, x_n)$ 和 $n$ 维空间中的几何矢量意义对应, 即把它们看作是 $X$ 空间中矢量 $\bvec x$ 的 $n$ 个坐标. 同理我们把 $(y_1, \dots, y_n)$ 看作是 $m$ 维空间 $Y$ 中的几何矢量 $\bvec y$ 的坐标.

\textbf{变换}是指, 给出 $X$ 空间中的任意 $\bvec x$, 都能通过某种规则\textbf{映射}(即对应)到 $Y$ 空间中的某个矢量 $\bvec y$. 注意变换不能是一个 $\bvec x$ 对应到多个 $\bvec y$. 变换的概念与函数类似, 若将某变换记为 $\Q T$, 那么可以记
\begin{equation}
\bvec y = \Q T\bvec x
\end{equation}

\textbf{线性变换}是一种满足特定性质的变换. 指给出 $X$ 空间中的任意两个矢量 $\bvec x_1$ 和 $\bvec x_2$ 以及两个实数 $c_1$ 和 $c_2$, 这个变换 $\Q L$ 满足
\begin{equation}
\Q L(c_1 \bvec x_1 + c_2 \bvec x_2) = c_1 \Q L \bvec x_1 + c_2 \Q L \bvec x_2
\end{equation}

% 未完成: 二维平面的线性变换. 例如平移就不是一个线性变换, 缩放是, 倾斜也是, 如果确定了基底的线性变换, 就可以确定任意矢量的线性变换.
