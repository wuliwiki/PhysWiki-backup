% 刚体定轴转动的力矩做功、动能、动能定理

\begin{issues}
\issueDraft
\end{issues}

\pentry{动能 动能定理(单个质点)\upref{KELaw1}, 刚体定轴转动 转动惯量\upref{RigRot}}

\subsection{定轴转动的动能、动能定理}
\addTODO{直接用质点组计算刚体绕轴转动的动能 $I\omega^2/2$, 对比质点运动的能量.}
\addTODO{为什么定轴转动时力矩\upref{Torque}对刚体做功\upref{Fwork}等于 $\tau \theta$?功率等于 $\tau\omega$}

\begin{example}{物理摆的角速度}\label{RigEng_ex1}
\addTODO{在上面两个例题中计算给定夹角 $\theta$ 的角速度, 使用动能定理计算, 说明刚体的势能就是质心的势能.}
\end{example}
