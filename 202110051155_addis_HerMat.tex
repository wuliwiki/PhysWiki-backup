% 厄米矩阵 自伴矩阵
% 矩阵|线性代数|复共轭|转置|厄米共轭

\pentry{对称矩阵\upref{SymMat}}

\subsection{厄米共轭}
我们把矩阵元可以取复数的矩阵叫做\textbf{复数矩阵}. % 未完成: 这句话是不是应该在前面提一下.
与矩阵转置(\autoref{Mat_eq3}~\upref{Mat})类似, 可以定义厄米共轭(Hermitian conjugate)操作:

\begin{definition}{厄米共轭}
要对一个复数矩阵做\textbf{厄米共轭}, 就先将其做转置, 再对每个矩阵元取复共轭. 矩阵 $\mat A$ 的\textbf{厄米共轭矩阵}记为 $\mat A\Her$, 其第 $i$ 行第 $j$ 列的矩阵元为
\begin{equation}
A\Her_{i,j} = A_{j,i}^*
\end{equation}
注意当矩阵元都是实数时, 厄米共轭就是转置. $\mat A\Her$ 也称为 $\mat A$ 的\textbf{伴随矩阵(adjoint matrix)}.
\end{definition}

\subsection{厄米共轭的基本性质}
任意常数乘以厄米矩阵再取共轭, 等于该常数的复共轭乘以矩阵的厄米共轭
\begin{equation}\label{HerMat_eq1}
(c \mat A)\Her = c^* \mat A\Her
\end{equation}

类比转置,% 引用未完成
矩阵相乘的厄米共轭等于分别做厄米共轭, 逆序排列, 再相乘
\begin{equation}\label{HerMat_eq2}
(\mat A \mat B \dots \mat C)\Her  = \mat C\Her \dots \mat B\Her \mat A\Her
\end{equation}

\begin{exercise}{证明}
请根据相关定义证明\autoref{HerMat_eq1} 和\autoref{HerMat_eq2}.
\end{exercise}

\subsection{线性映射的厄米共轭}
\pentry{矩阵与线性映射\upref{MatLS}}
既然矩阵可以用来表示线性映射, 那么其厄米共轭(伴随矩阵)也有对应的算符. 上述 $\mat A$ 和 $\mat A\Her$ 的算符分别记为 $A:X\to Y$, $A\Her: Y\to X$, 其中 $Y = A(X)$ 是 $A$ 的值域空间. 为什么$A\Her$ 要调换 $X, Y$ 呢? 考虑长方形矩阵 $\mat A$ 的厄米共轭矩阵就会发现转置后定义域和值域的维数互换了, 所以 $A\Her:Y\to X$ 是更自然的定义.
\addTODO{对任意线性映射有 $\braket{u}{Av} = \braket{A\Her u}{v} = \braket{v}{A\Her u}$, 即 $\bvec u\Her \mat A \bvec v = \bvec v\Her \mat A\Her \bvec u$ (\autoref{HerMat_eq2}). 这可以看作厄米算符的等效定义, 在泛函分析中更有用.}

\subsection{厄米矩阵\ 自伴矩阵}
若 $\mat A$ 的厄米共轭是其本身, 即
\begin{equation}
\mat A\Her = \mat A
\end{equation}
那么我们称其为\textbf{厄米矩阵}. 厄米矩阵关于对角线对称的任意两个矩阵元互为复共轭
\begin{equation}
A_{i,j} = A_{j,i}^*
\end{equation}
特殊地, 对角线上的矩阵元的复共轭等于本身 ($A_{i,i} = A_{i,i}^*$), 所以厄米矩阵对角线上的矩阵元都是实数.

事实上, 厄米矩阵也可以等效定义为满足
\begin{equation}
\bvec v_i\Her (\mat A \bvec v_j) = (\mat A \bvec v_i)\Her \bvec v_j
\end{equation}
对任意列矢量 $\bvec v_i$ 和 $\bvec v_j$ 都成立的矩阵. 这可以用\autoref{HerMat_eq2} 证明.