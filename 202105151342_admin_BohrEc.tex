% Bohr-Sommerfeld 氢原子模型

\begin{issues}
\issueDraft
\end{issues}

\pentry{玻尔原子模型\upref{BohrMd}}

\footnote{参考 Wikipedia \href{https://en.wikipedia.org/wiki/Old_quantum_theory}{相关页面}
.}本文使用原子单位制\upref{AU}. Sommerfeld 模型和玻尔原子模型一样属于量子力学发展早期的半经典原子模型, 它对玻尔模型进行了改进, 能更好地符合一些实验结果(如塞曼效应). 玻尔模型假设电子以圆形轨道绕原子核旋转, 而 Sommerfeld 模型允许椭圆轨道. 为了简单我们同样先假设原子核固定不动, 要考虑原子核运动使用相对坐标和约化质量\upref{TwoBD}即可.

Sommerfeld 模型中, 轨道量子化的条件有两个
\begin{equation}\label{BohrEc_eq4}
L = l
\end{equation}
\begin{equation}\label{BohrEc_eq3}
\oint m\dot r \dd{r} = kh
\end{equation}
其中 $L$ 是轨道角动量的模长, $l, k$ 是正整数, $h$ 是普朗克常数, $\oint$ 表示对轨道的一个周期积分. 

根据这两个量子化条件, 可以由量子数 $k, l$ 唯一地确定电子的轨道, 经过计算可得轨道总能量为
\begin{equation}
E = -\frac{Z^2}{2n^2}
\end{equation}
其中
\begin{equation}\label{BohrEc_eq2}
n = l + k
\end{equation}

\subsection{数值验证}

把\autoref{CenFrc_eq10}~\upref{CenFrc} 和\autoref{BohrEc_eq4}  代入\autoref{BohrEc_eq3} 得
\begin{equation}\label{BohrEc_eq1}
2\int_{a-c}^{a+c} \sqrt{2m\qty(E - \frac{k}{r} - \frac{l^2}{2mr^2})} \dd{r} = kh
\end{equation}
其中环积分被替换为两个相等的积分, 即从椭圆轨道的近日点 $a-c$ 积分到远日点 $a+c$, 以及逆过程.

开普勒问题(\autoref{CelBd_eq7}~\upref{CelBd},\autoref{CelBd_eq8}~\upref{CelBd})中,
\begin{equation}
e = \sqrt{1 + \frac{2EL^2}{mk^2}}
\qquad
a = \frac{k}{2\abs{E}}
\end{equation}

===============================

, 类氢原子 $n,l,m$ 束缚态中, 轨道角动量模长为 $L = \sqrt{l(l+1)}$, 能量为 $E = -Z^2/(2n^2)$.

Sommerfeld 模型中取 $L = l$, 也就是说当 $m = l$ 时角动量指向 $z$ 方向.

==========================

\subsubsection{Matlab 数值积分}
以下程序用于数值验证\autoref{BohrEc_eq2}. 注意原子单位中 $h = 2\pi$.
\begin{lstlisting}[language=matlab]
n = 2; l = 1;
Z = 1; k = -Z;
E = -Z^2/(2*n^2);
L = l;
me = 1;
a = abs(k)/(2*abs(E)); b = L/sqrt(2*me*abs(E));
c = sqrt(a^2 - b^2);
f = @(r)real(sqrt(2*me*(E - k./r - L^2./(2*me*r.^2))));
r = linspace(a-c, a+c, 500); figure; plot(r, f(r));
k = 2*integral(f, a - c, a + c) / (2*pi);
disp(['k = ' num2str(k)]);
\end{lstlisting}
