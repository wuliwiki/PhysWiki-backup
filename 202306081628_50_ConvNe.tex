% 卷积神经网络
% 深度学习 卷积 神经网络

\textbf{卷积神经网络}(简称卷积网络,Convolutional Neural Networks, CNN)是一类基本的神经网络,网络的主要结构是以卷积运算为核心操作的神经元所组成的\textbf{卷积层}(Convolutional Layer)。

卷积神经网络与以往的传统神经网络,比如全连接网络,有很多相似的地方。它们都有数据输入和输出,通常具有中间隐含层。在深度学习中,卷积网络的隐含层数量往往较多。卷积操作主要考虑相邻神经元之间的关系。由于使用卷积操作,卷积网络与全连接网络相比,参数数量大大减少。而此一特性恰好可以适用于图像处理。因为,在图像中,通常距离接近的像素之间具有较强的关系,而距离较远的像素之间可能没有较大的关系。

由于在实际应用中,大多数图像是二维数据,因为二维卷积网络最为常用。二维卷积层有多个超参数。一个二维卷积层的输入为一个矩阵,其宽为$W_i$、高为$H_i$、通道数$C_i$。卷积核个数为$K$,步长为$S$。补零的数量为$P$。经过卷积层的操作之后,输出一个矩阵,该矩阵的规格为:宽$W_o$、高$H_o$、通道数$C_o$。

输入和输出的超参数之间有如下关系: \\
\begin{itemize}
\item 输出的通道数取决于卷积核的个数,即:$C_o=K$。
\item 输出的宽取决于输入的宽、卷积步长和补零个数,即:$W_o=(W_i+2P)/S+1$
\item 输出的高取决于输入的高、卷积步长和补零个数,即:$H_o=(H_i+2P)/S+1$
\end{itemize}

有些应用场景,所须要处理的数据是三维图像块,此时可以使用三维卷积网络。三维卷积层的超参数如下:
