% 陕西师范大学 2005 年 考研 量子力学
% license Usr
% type Note

\textbf{声明}:“该内容来源于网络公开资料,不保证真实性,如有侵权请联系管理员”

\subsection{(15 分)}
粒子在一维势场$U(x)$中运动,证明属于同一能级的两个東缚定态波函数$\psi_1$与$\psi_2$,只相
差一常数,即$\psi_1=C\psi_2$,$C$为一常数。
\subsection{(20 分)}
已知一维谱振子的哈帟顿算符为$$\hat{H} = -\frac{\hbar^2}{2\mu} \frac{d^2}{dx^2} + \frac{1}{2}\mu \omega^2 x^2~$$处于$\psi(x) = C(2\alpha^2x^2 - 1)e^{-\frac{1}{2}\alpha^2x^2}$状态中,式中$\alpha = \left( \frac{\mu \omega}{\hbar} \right)^{\frac{1}{2}}$,$C$为一常数。
\begin{enumerate}
\item 证明它处于能量的本征态,本征值是多少?
\item 它的动是是否具有确定值,为什么?
\end{enumerate}
\subsection{(15 分)}
什么是厄密算符?以下算符是否为厄密算符,要说明详细理由?
\begin{equation}
\begin{aligned}
(a) & \quad \frac{d}{dx}; \\\\
(b) & \quad \begin{pmatrix}
0 & -i \\\\
i & 0
\end{pmatrix}.
\end{aligned}~
\end{equation}
\subsection{(30 分)}
 氢原子基态的波函数为 $\psi_{100}(r, \theta, \varphi) = Ce^{-r/a_0}$,求:
 \begin{enumerate}
 \item 归一化常数 $C$;
 \item 最可几半径;
 \item 势能 $U = -\frac{1}{4 \pi \epsilon_0} \frac{e^2}{r}$ 的平均值。
 \end{enumerate}
 \subsection{(30 分)}
 粒子在力学量 $\hat{Q}$ 的两个本征态所张成的态空间中运动。在该表象中,其哈密顿量有 $\hat{H}$ 和一力学算符 $\hat{A}$ 的形式为
$$
\hat{H} = b \begin{pmatrix}
2 & 0 & 0 \\
0 & 1 & 0 \\
0 & 0 & 0 
\end{pmatrix}, \quad \hat{A} = a \begin{pmatrix}
1 & 0 & 0 \\
0 & 2 & 0 \\
0 & 0 & 2 
\end{pmatrix}~, \quad a, b \text{为实数}.
$$

(a) 求 $\hat{H}$ 的本征值和相应的本征矢;

(b) 若 $t = 0$ 时,粒子处于
$$
\psi(0) = \frac{1}{\sqrt{2}} \begin{pmatrix}
1 \\
0 \\
1
\end{pmatrix}~
$$
所描述的状态,求$t$时刻粒子的状态,它是否是能量的本征态?

(c) 求 $t = 0$ 时,$\hat{A}$ 的平均值 $\overline{A}$,并讨论$\overline{A}$随时间变化的规律。
\subsection{(20 分)}
已知 $\hat{L}^2$ 和 $\hat{L}_x, \hat{L}_y, \hat{L}_z$ 为角动量算符,$\hat{L}_\pm$定义为 $\hat{L}_\pm = \hat{L}_x \pm i\hat{L}_y$, $\phi = \hat{L}_+ Y_{lm}$, $\psi = \hat{L}_- Y_{lm}$.

证明:$\phi, \psi$ 为 $\hat{L}^2$ 和 $\hat{L}_z$ 的共同本征函数,本征值为多少?
\subsection{(20 分)}
转动惯最为$I$的刚性转子绕一固定点转动,哈密顿算符为 $H_0 = \frac{L^2}{2I}$,可作为双原子分子的模型,

(a) 写出它的哈密顿算符的本征值及本征函数,能级是几度兼并的:

$$
H' = -D \dot{\theta} = -D \dot{\theta} \cos \theta~
$$

讨论 $H$ 的改变对系统状态方程的影响。

(b) 如果有阻尼项,则系统角动量守恒方程的物理性质及本质的改变是什么?给出 $H = H_0 + H'$ 的完整表达形式。

进一步讨论如何通过实验或理论计算来验证阻尼对系统的影响。

$$\cos \phi = \frac{1}{\sqrt{3}} \frac{L}{L_0}~$$