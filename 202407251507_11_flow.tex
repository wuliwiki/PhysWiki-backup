% 流
% license Usr
% type Tutor
我们知道,对切向量场积分可以得到一族曲线,比如地球的一系列纬线。如果加上初始条件,便可以确定具体是哪一条曲线。因此,流形$M$上的积分曲线实际上有两个含参变量,设其为$\theta(t,p),\forall p\in M$。我们可以固定$p$点为曲线的初始位置,得到一条曲线:$\theta_p(t)$,亦可以固定$t$,得到$\theta_t(p):M\rightarrow M$。
\subsection{全局流}
令$p\in M$为曲线的初始位置,即$\theta(0,p)=p$。若$t\in\mathbb R$,我们称曲线族$\theta(t,p):\mathbb R\times M\rightarrow M$为\textbf{全局流(global flow)}。

可以证明\footnote{translation lemma proved in \textsl{introduction to smooth maniflod}},对于$\theta_t(p)$,如果$q=\theta_p(s)$,则$\theta^{(q)}(t)=\theta^{(p)}(t+s)$,所以该光滑映射满足:
\begin{equation}
\begin{aligned}
\theta(t, \theta(s, p)) & =\theta(t+s, p) \\
\theta(0, p) & =p~.
\end{aligned}
\end{equation}
上式的结合律也可以表示为$\theta_t \circ \theta_s(p)=\theta_{t+s}(p)$。因此$\theta$可看作加法群$\mathbb R$对流形$M$的作用。当我们固定$p$时,$\theta_p(t)$便是$\mathbb R$对$M$作用的轨道;而当我们固定$t$时,
\subsection{局域流}
\subsection{流的基本定理}
