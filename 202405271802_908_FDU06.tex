% 复旦大学 2006 量子真题
% license Usr
% type Note


\begin{issues}
\issueTODO
\end{issues}

\textbf{声明}:“该内容来源于网络公开资料,不保证真实性,如有侵权请联系管理员”

复旦大学 2006年招收攻读硕士学位研究生入学考试试题

1. 一维谐振子处在第一激发态 $( \psi (x) = 2 \alpha x e^{-\frac{\alpha x^2}{2}} $) 中,式中 $( \alpha = \sqrt{\frac{m \omega}{\hbar}}$),$( m )$ 是振子质量,$(\omega)$ 是振动频率,求:

(1) 动能的平均值和势能的平均值

(2) 第一激发态几率最大的x位置 (30 分)

2. 考虑自旋 $(\bar{S} = 1/2)$ 的体系

(1) 在 $((S^2, S_z))$ 表象中求算符 $(\hat{O} = A S_y + B S_z^2 )$ 的本征值和归一化本征函数,其中 $( A, B )$ 为实常数

(2) 假定此体系正处在 $(\hat{O})$ 算符的一个本征态中,求测量 $( S_y )$ 得到结果为 $(frac{\hbar}{2})$ 的几率 (30 分)

3. 设能量为 $( E )$ 的粒子在边入射到深度为 $( V_0 )$ 的势垒,求障壁处的反射系数 (30 分)
4. 距离表面 $x$ 处的一个电子受到势场
$[U{x}]=\begin{cases}

\end{cases}\end{cases}} -\frac{k}{x} & x > 0 \\\\\\infty & x \leq 0\end{cases}$]
作用,略去自旋的影响,求
(1) 基态能级和波函数
(2) 现加入一个电场强度为 $\\epsilon$ 的弱电场,用微扰论来基态能量一级修正

(30 分)

5. 两个质量为 $m$ 的无相互作用的粒子处在宽度为 $a$ 的一维无穷深方势阱中运动
(1) 写出体系四个最低能级的能量(不必排序)
(2) 分别求下面三种情况下的能级:
a) 自旋为 \$1/2$ 全同粒子;
b) 自旋为 \$1/2$ 非全同粒子;
c) 自旋为 \$1$ 的全同粒子

求出最低的四个能级的简并度(需说明理由)

(30 分)