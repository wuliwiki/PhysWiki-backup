% 序列的极限

\pentry{极限\upref{Lim}, 实数集的拓扑\upref{ReTop}}

\subsection{基本定义解析}

序列的极限是分析数学中最基本的定义. 词条 数列的极限(简明微积分)\upref{Lim0} 和 极限\upref{Lim} 已经给出了一些序列极限的例子, 它的形式定义以及背后的直观解释. 为完整起见, 这里再重复一次序列极限的定义:

\begin{definition}{数列的极限}
考虑数列 $\{a_n\}$.若存在一个实数 $A$,使得对于\textbf{任意}给定的\textbf{正实数} $\varepsilon > 0$(无论它有多么小),总存在正整数 $N_\epsilon$, 使得对于所有编号 $n>N_\varepsilon$ ,都有 $\abs{a_n - A} < \varepsilon$ 成立,那么数列 $a_n$ 的极限就是 $A$.

将“数列 $\{a_n\}$ 的极限是 $A$”表示为 $\lim\limits_{n\to\infty}a_n=A$.
\end{definition}

正如之前两个词条所解释的, 等式 $\lim\limits_{n\to\infty}a_n=A$ 所表达的含义是"序列 $a_n$ 随着 $n$ 的增大将可以任意地接近 $A$". 或者说, 对于序列 $\{a_n\}$ 进行极限运算, 就是要找到"序列 $a_n$ 越来越接近的那个数". 这种运算显然跟实数的四则运算不一样.

有极限的序列常常称为收敛 (convergent) 的. 如果没有极限, 则序列称为发散 (divergent) 的.
\begin{figure}[ht]
\centering
\includegraphics[width=14cm]{./figures/SeqLim_1.png}
\caption{序列极限的示意图} \label{SeqLim_fig1}
\end{figure}
\begin{example}{求基本极限}
证明$$\lim\limits_{n\to\infty}\frac{1}{2^n}=0.$$

在给出严格证明之前, 首先来看看序列 $\{2^{-n}\}$ 到底能够多么接近零. 直观上, 我们知道它衰减的速度非常快, 例如第四项 $2^{-4}=0.0625$, 而第八项已经是 $2^{-8}=0.00390625$. 相比之下, 倒数序列 $\{1/n\}$ 的第四项只是 $1/4=0.25$, 第八项只是 $1/8=0.125$. 因此, 即便不借助对数运算, 也可以说明序列 $\{2^{-n}\}$ 会逐渐接近于零.

转向严格证明. 首先注意到初等的不等式 $2^n>n$ 对于任何整数 $n\geq1$ 都成立; 这可以使用数学归纳法得到. 因此, 给定一个误差 $\varepsilon>0$ 之后, 要使得 $2^{-n}$ 同零的误差不大于 $\varepsilon$, 只需要 $1/n$ 同零的误差不大于 $\varepsilon$ 就够了, 而为了达到这一点, 只要 $n>1/\varepsilon$ 就够了. 因此, 只要取脚码
$$
N_\varepsilon=\left[\frac{1}{\varepsilon}\right]+1,
$$
即可保证当 $n>N_\varepsilon$ 时有 $2^{-n}<\varepsilon$.
\end{example}

当然, 直观上容易看出, 序列 $\{2^{-n}\}$ 衰减得比倒数序列 $\{1/n\}$ 要快多了. 上面的证明当然远远不是最精确的. 为了刻画一个有极限的序列 $\{a_n\}$ 收敛的速度, 可以考虑如下问题: 给定了一个误差 $\varepsilon>0$ 之后, 为了使得 $|a_n-A|<\varepsilon$ 能够一直成立, 脚码 $n$ 至少得是多大? 与此相关的概念正是无穷小的阶. 

\subsection{基本性质}
序列的极限运算有如下基本性质:

\begin{theorem}{极限的基本性质}
\begin{itemize}
\item 序列的极限若存在, 则必定是唯一的.
\item 极限运算保持序关系: 如果 $\lim\limits_{n\to\infty}a_n=A$, $\lim\limits_{n\to\infty}b_n=B$, 而且从某个 $n$ 开始有 $a_n\geq b_n$, 那么必然有 $A\geq B$.
\item 设 $\lim\limits_{n\to\infty}a_n=A$, $\lim\limits_{n\to\infty}b_n=B$, 则序列 $\{a_n\pm b_n\}$ 和 $\{a_n b_n\}$ 都有极限, 且 $\lim\limits_{n\to\infty}a_n\pm b_n=A\pm B$, $\lim\limits_{n\to\infty}a_nb_n=AB$.
\item 设 $\lim\limits_{n\to\infty}a_n=A$, $\lim\limits_{n\to\infty}b_n=B\neq0$, 则
$$
\lim\limits_{n\to\infty}\frac{a_n}{b_n}=\frac{A}{B}.
$$
\end{itemize}
\end{theorem}

\addTODO{以序列之和的极限为例进行说明.} 

\begin{definition}{有界性}
设 $\{x_n\}$ 是一个序列.若 $\exists M>0$,$\forall n$,有 $|x_n|\leq M$ 成立,则称 $\{x_n\}$ 是\textbf{有界的}.

显然以上定义等价于数集 $\{x_n\}$ 是一个有界集.

若一个序列 $\{x_n\}$ 是有界的,则记为 $x_n=O(1)\ (n\rightarrow \infty)$.若存在 $M_2>M_1>0$ 和正整数 $N$,使得当 $n>N$ 时,有 $M_1<|x_n|<M_2$,则以 $x_n=O_o(1)$ 表示之.
\end{definition}
  
\begin{theorem}{收敛序列的有界性}
\begin{enumerate}
\item 收敛序列是有界的.
\item 收敛序列的极限是唯一的.
\end{enumerate}
\end{theorem}

以序列极限的唯一性为例,它可以用反证法来证明.假设收敛序列存在两个极限 $a$ 和 $b$,即 $\lim\limits_{n\rightarrow \infty}x_n=a$ 且 $\lim\limits_{n\rightarrow \infty} x_n=b$ 且 $a\neq b$.不失一般性,不妨设 $a<b$.现在取 $\epsilon_0=(b-a)/2$,则由极限定义知,存在正整数 $N_1$ 和 $N_2$,使得:
\begin{equation}
  \begin{aligned}
  |x_n-a|<\epsilon_0,\ \ \ \ \ \  &\forall n>N_1\\
  |x_n-b|<\epsilon_0,\ \ \ \ \ \  &\forall n>N_2
  \end{aligned}
\end{equation}
  令 $N=\max\{N_1,N_2\}$,则当 $n>N$ 时,上述两个不等式都成立.那么
\begin{equation}
\begin{aligned}
  x_{N+1} <a+\epsilon_0=(a+b)/2\\
  x_{N+2} >b-\epsilon_0=(a+b)/2
\end{aligned}
\end{equation}
  导致矛盾.所以原命题成立.

\begin{theorem}{保序性}
给定两个序列 $\{x_n\}$ 和 $\{y_n\}$,并且假定
\begin{equation}
  \lim\limits_{n\rightarrow \infty} x_n=a,\ \ \ \ \lim\limits_{n\rightarrow \infty} y_n = b,
\end{equation}
  则有:
\begin{enumerate}
\item  若 $a<b$,则对任意给定的 $c\in (a,b)$,$\exists N_0>0$,使得当 $n>N_0$ 时,有 $x_n<c<y_n$;
\item  若 $\exists N_0>0$,当 $n>N_0$ 时,有 $x_n\leq y_n$,则 $a\leq b$.(注意逆命题不一定成立)
\end{enumerate}
\end{theorem}

\begin{exercise}{}

  构造两个序列 $\{x_n\},\{y_n\}$,使得 $\lim\limits_{n\rightarrow \infty}x_n\le \lim\limits_{n\rightarrow \infty} y_n$,但 $\forall n>0,x_n>y_n$.
\end{exercise}
\begin{theorem}{极限的四则运算}
  设 $\lim\limits_{n\rightarrow \infty} x_n=a,\ \lim\limits_{n\rightarrow \infty} y_n=b$,则

  1. $\lim\limits_{n\rightarrow \infty}(x_n+y_n)=a+b,\ \ \lim\limits_{n\rightarrow \infty}(x_n-y_n)=a-b$;

  2. $\lim\limits_{n\rightarrow \infty}(x_ny_n)=ab$;

  3. $\lim\limits_{n\rightarrow \infty}(x_n/y_n)=a/b$,其中 $b\neq 0,\ y_n\neq 0$.

\end{theorem}

\begin{exercise}{}
  若 $\lim\limits_{n\rightarrow \infty} x_n=1$;序列 $\{y_n\}$ 的元素是整数.那么序列 $\{x_n^{y_n}\}$  的极限是否一定存在?请举一个 $\{x_n^{y_n}\}$ 的极限存在且不为 $1$ 的例子.
\end{exercise}

\begin{theorem}{夹逼收敛定理}
  设序列 $\{x_n\}$, $\{y_n\}$ 和 $\{z_n\}$ 满足 $x_n\leq z_n\leq y_n,\ \forall n>N_0$.
若 $\lim\limits_{n\rightarrow \infty}x_n=\lim\limits_{n\rightarrow \infty}y_n=a$,则 $\lim\limits_{n\rightarrow \infty}z_n=a$.
\end{theorem}