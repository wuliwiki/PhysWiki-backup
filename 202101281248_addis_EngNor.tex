% 一维散射态的归一化
% 波函数|归一化|delta 函数|完备性|薛定谔方程

\begin{issues}
\issueDraft
\end{issues}

\pentry{一维自由粒子(量子)\upref{FreeP1}}

本文使用原子单位制\upref{AU}. 以平面波 $\exp(\I kx)$ 为例, 在动量归一化()中, 我们把波函数(包括归一化系数)看成是以动量(即波数) $k$ 为参数的, 满足连续基底的正交归一关系% 链接未完成
\begin{equation}\label{EngNor_eq1}
\int_{-\infty}^{+\infty} \psi_{k'}(x)^* \psi_{k}(x) \dd{x} = \delta(k' - k) \qquad (k \in \mathbb R)
\end{equation}
其中 $*$ 表示复共轭, 满足该式的 $\psi_k$ 就是动量归一化的:% 链接未完成
\begin{equation}
\psi_k(x) = \frac{\E^{\I kx}}{\sqrt{2\pi}}
\end{equation}
只有满足\autoref{EngNor_eq1} , 才能把任意波函数分解为许多 $\psi_{k}$ 的叠加(积分)
\begin{equation}
\psi(x) = \int_{-\infty}^{+\infty} C(k)\psi_{k}(x) \dd{k}
\end{equation}
使得
\begin{equation}
C(k) = \int_{-\infty}^{+\infty} \psi_{k}(x)^* \psi(x) \dd{k}
\end{equation}

有时候我们也希望用能量 $E = k^2/(2m)$ 作为平面波的参数($E > 0$), 即把任意波函数分解为
\begin{equation}
\psi(x) = \int_{0}^{+\infty} A(E)\psi_E(x) \dd{k}
\end{equation}
的形式. 其中
\begin{equation}
A(E) = \int_{0}^{+\infty} \psi_E(x)^* \psi(x) \dd{k}
\end{equation}
这就要求
\begin{equation}
\int_{-\infty}^{+\infty} \psi_E'(x)^* \psi_E(x) \dd{x} = \delta(E - E')
\end{equation}


我们希望找到平面波的一种归一化系数 $C(k)$ 使其满足
\begin{equation}
\int C(k')^* \E^{-\I k' x} \cdot C(k) \E^{\I kx} \dd{x} = 
\end{equation}


================ 回收 ====================

\subsection{一维散射} % 未完成

想要能量归一化, 需要
\begin{equation}
\int_{-\infty}^{+\infty} \psi_{E'}^*(x) \psi_E(x) \dd{x}  = \delta (E - E')
\end{equation}
能不能在动量归一化的波函数基础上修改, 得到能量归一化的本征函数呢?
动量归一化的要求是
\begin{equation}
\int_{-\infty}^{+\infty} \psi_{k'}^*(x) \psi_k(x) \dd{x}  = \delta(k - k')
\end{equation}
满足
\begin{equation}\ali{
\int_{-\infty}^{+\infty} \psi_k^*(x) \psi(x) \dd{x}
= \int_{-\infty}^{+\infty} \psi_k^*(x) \qty(\int_{-\infty}^{+\infty} c(k')\psi_{k'}(x) \dd{k'}) \dd{x}
= c(k)
}\end{equation}
另外注意 $E = \hbar ^2 k^2/(2m)$, $E' = \hbar^2k'^2/(2m)$. 
\begin{equation}
\delta \qty[ \frac{\hbar ^2}{2m}(k^2 - k'^2)]
\end{equation}
根据 $\delta $ 函数的性质, 若 $x_0$ 是 $f(x)$ 的一个零点
\begin{equation}
\delta[f(x)] = \frac{1}{f'(x)}\delta (x - x_0)
\end{equation}
所以
\begin{equation}
\begin{aligned}
\delta (E - E') & = \delta \qty[\frac{\hbar^2}{2m}(k^2 - k'^2)] = \frac{m}{\hbar ^2 k}\delta (k - k') \\
&= \frac{m}{\hbar^2 k}\int_{-\infty }^{+\infty } \psi_k^*(x) \psi_k(x)\dd{x} 
\end{aligned}
\end{equation}
可得
\begin{equation}
\psi_E (x) = \frac{1}{\hbar} \sqrt{\frac{m}{k}} \psi_k(x)
\end{equation}


=============== 草稿 ================

\subsection{假设 $V(x)$ 关于原点对称}
$\psi_{k,o}(x) = \sin(kx + \phi)$, $\phi$ 是 $k$ 的函数. 

首先
\begin{equation}
\int_{0}^{+\infty} \sin(k'x)\sin(kx)\dd{x} = \frac{\pi}{2}\delta(k'-k)
\end{equation}
现在添加相位 $\phi(k)$, 为方便书写把 $\phi(k),\phi(k')$ 分别记为 $\phi, \phi'$(Wolfram Alph)
\begin{equation}
\int \sin(k'x+\phi')\sin(kx+\phi) \dd{x} = \frac{\sin[(k'-k)x + (\phi'-\phi)]}{2(k'-k)}
- \frac{\sin[(k'+k)x+(\phi'+\phi)]}{2(k'+k)}
\end{equation}
取极限后发现额外的随 $k$ 变化的相位并不影响正交归一
\begin{equation}
\int_{0}^{+\infty} \sin(k'x+\phi')\sin(kx+\phi) \dd{x} = \frac{\pi}{2}\delta(k'-k)
\end{equation}
如果中间有 $V(x) \ne 0$, 导致相移 $\phi(k)$ 需要添加该区间中的修正项
\begin{equation}
\braket{\psi_{k'}}{\psi_k} = \int_{0}^{+\infty} \sin(k'x+\phi')\sin(kx+\phi) \dd{x} + I(k,k')
= \frac{\pi}{2}\delta(k'-k) + I(k,k')
\end{equation}
\begin{equation}
I(k,k') = \int_{0}^{L} \sin(k'x+\phi')\sin(kx+\phi) \dd{x} 
= \frac{\pi}{2}\delta(k'-k)
\end{equation}

\subsection{杂}
\begin{\alpha quati\alpha on}
\int_0^\alpha\alpha \E^{\I kx} \dd{x} = \frac{1}{\I k} (\E^{\I k\alpha} - 1)
= \frac{2}{ k} \sin(k\alpha/2) \E^{\I k\alpha/2}
\end{equation}
这并不是一个 delta 函数.
