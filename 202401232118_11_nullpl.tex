% 零化多项式
% license Usr
% type Tutor

\begin{definition}{}
设$f(x)$是以域$\mathbb F$中元素为系数的一元多项式,$A$是线性空间$V$上的线性变换。若$f(A)=0$(即零变换),则称$f$是$A$的一个\textbf{零化多项式}(null polynomial)。
\end{definition}
可以验证,若$f$是线性变换$A$的零化多项式,那么也是其任意基下矩阵所对应的零化多项式,即无论选取什么基底,代入多项式的最终结果为零矩阵。这是因为若$A=Q^{-1}BQ$,我们有$f(A)=Q^{-1}f(B)Q$。


下面一条定理及其推论表明了寻找零化多项式的重要意义。
\begin{theorem}{}
若$h=fg$,且$f,g$是互素多项式,则有:
\begin{equation}
\opn{ker}h=\opn{ker}f\oplus\opn{ker}g~.
\end{equation}
\end{theorem}
Proof.

首先证明$\opn{ker}f\cap \opn{ker}g=0$。由互素得,存在多项式$u,v$使得$uf+vg=I$。设$\bvec x\in \opn{ker}f\cap \opn{ker}g$,则$(uf+vg)\bvec x=0=\bvec x$。矛盾,因而$f,g$无交集,和为直和。

下证$\opn{ker}h=\opn{ker}f\oplus\opn{ker}g$。
第一步,先证$\opn{ker}h\subset\opn{ker}f\oplus\opn{ker}g$。设$\{\bvec x_i\}$和$\{\bvec y_i\}$分别是$\opn{ker}f$和$\opn{ker}g$的基,则$fg(a^i\bvec x_i+b^i \bvec y_i)=a^igf(\bvec x_i)+b^ifg(\bvec y_i)=0$,第一步得证。

第二步证明$\opn{ker}h\supset\opn{ker}f\oplus\opn{ker}g=\opn{ker}f+\opn{ker}g$。设$\bvec x\in \opn{ker}h$,由之前的证明过程知:$\bvec x=(uf+vg)\bvec x$,只要把这两项分配给$f,g$的核即可。显然,$g(uf\bvec x)=f(vg\bvec x) =0$,得证。

若$h(A)=0$,则$\opn{ker} h=$