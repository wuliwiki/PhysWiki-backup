% 电介质摘要

假设我们知道物质是由带正、负电荷的粒子组成的。在外加电场下,物质中的正、负电荷将对外加电场做出响应。在金属中,存在大量自由的电荷,可以在整个物质内重新分布;而在其余介质中,电荷之间的束缚很强,因此电荷只能在小范围内重新分布。

\begin{figure}[ht]
\centering
\includegraphics[width=5cm]{./figures/DLT_1.pdf}
\caption{请添加图片描述} \label{DLT_fig1}
\end{figure}
\begin{itemize}
\item 在外加电场下,介质中产生大量电偶极子\upref{Dielec},这种过程称为介质的极化。这使介质具有宏观可感的电偶密度(电极化强度)$\bvec P = \lim_{\Delta V \to 0} \frac{\sum \bvec p_i}{\Delta V}$ \upref{ElecPo}
\item 电偶极子导致了极化电荷 $\bvec \rho_P = - \div \bvec P$。\upref{ElePAP}
\item 极化电荷产生了额外的极化场,也就是说,电介质的存在改变了电场的分布。电荷感受到的电场是外加电场与极化场的和。\upref{EFIDE} $\bvec E = \bvec E_0 + \bvec E'$, $\div \bvec E = \frac{\rho_f+\rho_P}{\epsilon_0}$
\item 为了简化极化电荷的影响,引入电位移矢量 \upref{EFIDE}:$\bvec D = \epsilon_0 \bvec E + \bvec P$,并有电位移矢量的高斯定律 $\div \bvec D = \rho_f$
\item 在线性电介质中,电偶密度与合电场成线性关系\upref{EFIDE}: $\mathbf P=\chi_{\mathrm e} \varepsilon_{0} \mathbf E$。此时电位移矢量还可以写为 $\bvec D = \epsilon_0 \epsilon_r \bvec E = \epsilon \bvec E $, 其中$\epsilon_r = 1+\chi_{\mathrm e}, \epsilon = \epsilon_0 \epsilon_r$。
\end{itemize}
