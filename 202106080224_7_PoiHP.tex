% 庞加莱半平面(微分几何计算实例)
% 微分几何|联络|Poincare Half-plane|度量|黎曼联络|黎曼度量|Riemannian connection|Riemannian metric

庞加莱半平面是历史上非常重要的一个模型.众所周知,欧几里得几何学中有五条公理,其中第五条“过直线外一点有且仅有一条直线与已知直线垂直”非常冗长而且绕口,因此历史上一直有不少数学家致力于通过其它四条来推出第五条,也就是将第五公理变成一个定理.在GTM 275\cite{GTM275}中将这种尝试评价为“英雄式”的(heroic).这是出于早期数学家们的一种朴素的直觉,即几何就应该是欧几里得空间那样子的,所以第五公理必须成立,哪怕只是作为定理.后来的人们逐渐意识到第五公理并不能被前四条所证明,并逐渐发展出了符合前四条但违反第五条的几何学,也就是所谓的\textbf{非欧几何学}.庞加莱半平面就是一个典型的例子,在本节的\textbf{测地线}小节我们会简单讨论这一点.

本节的主要目的是以庞加莱半平面为例,演示如何进行具体的计算.

\subsection{庞加莱半平面的定义}

\begin{definition}{庞加莱半平面}
设$\mathbb{H}=\{(x, y)\in \mathbb{R}^2|y>0\}$,即二维实平面的上半平面(不包含$x$轴).在$\mathbb{H}$上定义ying'she
\end{definition}






















