% Python 解释器
% Python|解释器

\begin{issues}
\issueTODO
\issueDraft
\end{issues}

\pentry{计算机语言 — 脚本语言(解释型语言)\upref{CpLgg}}

Python解释器由编译器和虚拟机构成,编译器将源代码转换成字节码,然后再通Python虚拟机来逐行执行这些字节码.

\subsubsection{python程序执行过程:}

\begin{enumerate}
\item 执行 .py 文件,就会启动python解释器

\item 编译器将源文件解释成字节码

\item 虚拟机将字节码转化成机器语言,与操作系统交互

\item 程序运行结束后,将字节码存到pyc文件,便于后续直接执行
\end{enumerate}

\subsubsection{python解释器有很多种:}

\begin{itemize}
\item \textbf{CPython:}\textbf{C语言开发},使用最广,默认的解释器,新的语言特性通常在此率先添加.

\item \textbf{IPython:}\textbf{基于CPython之上}的交互式解释器

\item \textbf{PyPy:}完全使用 \textbf{Python 语言编写}的 Python 实现.它支持多个其他实现所没有的高级特性,例如非栈式支持和 JIT 编译器等.此项目的目标之一是通过允许方便地修改解释器 (因为它是用 Python 编写的),鼓励该对语言本身进行试验.想了解详情可访问 PyPy 项目主页.

\item \textbf{Jython:}以 \textbf{Java 语言编写}的 Python 实现.此实现可以作为 Java 应用的一个脚本语言,或者可以用来创建需要 Java 类库支持的应用.想了解更多信息可访问 Jython 网站.

\item \textbf{IronPython:}另一个 \textbf{.NET} 的 Python 实现,与 Python.NET 不同点在于它是生成 IL 的完全 Python 实现,并且将 Python 代码直接编译为 .NET 程序集.它的创造者就是当初创造 Jython 的 Jim Hugunin.想了解详情可访问 IronPython 网站.

\item \textbf{Python for .NET:}此实现实际上使用了 \textbf{CPython} 实现,但是属于 .NET 托管应用并且可以引入 .NET 类库.它的创造者是 Brian Lloyd.想了解详情可访问 Python for .NET 主页.
\end{itemize}

\subsection{2.1. 调用解释器}
Python 解释器在其被使用的机器上通常安装为 /usr/local/bin/python3.8;将 /usr/local/bin 加入你的 Unix 终端的搜索路径就可以通过键入以下命令来启动它:

\begin{lstlisting}[language=bash]
python3.8
\end{lstlisting}

就能运行了\footnote{在Unix系统中,Python 3.x解释器默认安装后的执行文件并不叫作 python,这样才不会与同时安装的Python 2.x冲突.}.安装时可以选择安装目录,所以解释器也可能在别的地方;可以问问你身边的 Python 大牛,或者你的系统管理员.(比如 /usr/local/python 也是比较常用的备选路径)

在 Windows 机器上当你从 Microsoft Store 安装 Python 之后,python3.8 命令将可使用. 如果你安装了 py.exe 启动器,你将可以使用 py 命令. 参阅 附录:设置环境变量 了解其他启动 Python 的方式.

在主提示符中输入文件结束字符(在 Unix 系统中是 Control-D,Windows 系统中是 Control-Z)就退出解释器并返回退出状态为0.如果这样不管用,你还可以写这个命令退出:quit().

解释器的行编辑功能在支持 GNU Readline 库的系统中也包括交互式编辑,历史替换和代码补全等. 检测是否支持行编辑最快速的方式是在首次出现 Python 提示符时输入 Control-P. 如果听到“哔”提示音,就说明支持行编辑;请参阅附录 交互式编辑和编辑历史 了解有关功能键的介绍. 如果什么都没发生,或是回显了 ^P,说明不支持行编辑;你只能用退格键从当前行中删除字符.

解释器运行的时候有点像 Unix 命令行:在一个标准输入 tty 设备上调用,它能交互式地读取和执行命令;调用时提供文件名参数,或者有个文件重定向到标准输入的话,它就会读取和执行文件中的 脚本.

另一种启动解释器的方式是 python -c command [arg] ...,其中 command 要换成想执行的指令,就像命令行的 -c 选项.由于 Python 代码中经常会包含对终端来说比较特殊的字符,通常情况下都建议用英文单引号把 command 括起来.

有些 Python 模块也可以作为脚本使用.可以这样输入:python -m module [arg] ...,这会执行 module 的源文件,就跟你在命令行把路径写全了一样.

在运行脚本的时候,有时可能也会需要在运行后进入交互模式.这种时候在文件参数前,加上选项 -i 就可以了.

关于所有的命令行选项,请参考 命令行与环境.

\subsubsection{2.1.1. 传入参数}
如果可能的话,解释器会读取命令行参数,转化为字符串列表存入 sys 模块中的 argv 变量中.执行命令 import sys 你可以导入这个模块并访问这个列表.这个列表最少也会有一个元素;如果没有给定输入参数,sys.argv[0] 就是个空字符串.如果给定的脚本名是 '-' (表示标准输入),sys.argv[0] 就是 '-'.使用 -c command 时,sys.argv[0] 就会是 '-c'.如果使用选项 -m module,sys.argv[0] 就是包含目录的模块全名.在 -c command 或 -m module 之后的选项不会被解释器处理,而会直接留在 sys.argv 中给命令或模块来处理.

\subsubsection{2.1.2. 交互模式}

在终端(tty)输入并执行指令时,我们说解释器是运行在 交互模式(interactive mode).在这种模式中,它会显示 主提示符(primary prompt),提示输入下一条指令,通常用三个大于号(>>>)表示;连续输入行的时候,它会显示 次要提示符,默认是三个点(...).进入解释器时,它会先显示欢迎信息、版本信息、版权声明,然后就会出现提示符:

\begin{lstlisting}[language=pythonC]
$ python3.8
Python 3.8 (default, Sep 16 2015, 09:25:04)
[GCC 4.8.2] on linux
Type "help", "copyright", "credits" or "license" for more information.
>>>
\end{lstlisting}
多行指令需要在连续的多行中输入.比如,以 if 为例:
\begin{lstlisting}[language=pythonC]
>>>
>>> the_world_is_flat = True
>>> if the_world_is_flat:
...     print("Be careful not to fall off!")
...
Be careful not to fall off!
\end{lstlisting}
有关交互模式的更多内容,请见 交互模式.

\subsection{2.2. 解释器的运行环境}

\subsubsection{2.2.1. 源文件的字符编码}

默认情况下,Python 源码文件以 UTF-8 编码方式处理.在这种编码方式中,世界上大多数语言的字符都可以同时用于字符串字面值、变量或函数名称以及注释中——尽管标准库中只用常规的 ASCII 字符作为变量或函数名,而且任何可移植的代码都应该遵守此约定.要正确显示这些字符,你的编辑器必须能识别 UTF-8 编码,而且必须使用能支持打开的文件中所有字符的字体.

如果不使用默认编码,要声明文件所使用的编码,文件的 第一 行要写成特殊的注释.语法如下所示:

\begin{lstlisting}[language=python]
# -*- coding: encoding -*-
\end{lstlisting}
其中 encoding 可以是 Python 支持的任意一种 codecs.

比如,要声明使用 Windows-1252 编码,你的源码文件要写成:

\begin{lstlisting}[language=python]
# -*- coding: cp1252 -*-
\end{lstlisting}
关于 第一行 规则的一种例外情况是,源码以 UNIX "shebang" 行 开头.这种情况下,编码声明就要写在文件的第二行.例如:

\begin{lstlisting}[language=python]
#!/usr/bin/env python3
# -*- coding: cp1252 -*-
\end{lstlisting}


