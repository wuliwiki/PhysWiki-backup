% 单向量
% license Xiao
% type Tutor


\begin{issues}
\issueTODO
\end{issues}

本文参考Jie Peter的《代数学基础》,以下默认讨论的域的特征不为2。

\begin{definition}{}
在几何代数$\mathcal G(V,q)$上,称由$V$中向量做外积得到的元素为\textbf{简单多重向量(simple multivector)},或简称为\textbf{单向量}。英文常简称之为\textbf{blade}。

以下是一些常用符号和简称的说明。
\begin{itemize}
\item 由k个向量做外积得到的为k-单向量(k-blade),其集合记为$\mathcal B_k$
\item 记$\mathcal B=\bigcup \limits ^{\infty}_{k=0}\mathcal B_k$
为全体单向量的集合。
\item记$\mathcal B^{\bullet}-\{0\}$为全体非0单向量的集合。
\item 记$\mathcal B^{\times}=\{B\in\mathcal B|B^2\neq0\}$为全体平方不为0的单向量集合。
\item 记$E$为$V$上的一组正交基,$\mathcal B_E$为$E$中元素做外积得到的单向量的集合,其元素成为关于$E$的正交基单向量。
\end{itemize}
\end{definition}
k-向量与k-单向量不是等同的概念。比如2-向量$2e_1e_2+3e_3e_4$就不能写为两个向量的外积。2-单向量$(2e_1+3e_3)\wedge(3e_1+2e_4)$则是$\mathcal G^1$与$\mathcal G^2$的元素之和。
接下来介绍一条看似自然,也非常重要的性质,我们可以简述为“在非退化的子空间里取一组正交基,总能扩展为全空间的正交基”。之所以说看似自然,是因为我们总是默认欧几里得空间自然成立。该条性质将其拓展到非退化的任意子空间。
\begin{theorem}{}
给定几何代数$\mathcal G(V,q)$。对于给定的子空间$W$,如果$q|_W$非退化,则$W$存在关于$q$的正交补空间$W^{\bot}$,使得$V=W\oplus W^{\bot}$
\end{theorem}

proof.

取$W$的正交基$\{e_1,e_2,...,e_k\}$。将其拓展为$V$的一组基
\begin{equation}
\{e_1,e_2,...,e_k\}\cup\{v_{k+1},v_{k+2}...v_{dim \,V}\}~,
\end{equation}
则由于$q|_W$非退化,类似于施密特正交化,我们总可以取
\begin{equation}
u_r=v_r-\sum ^{k}_{i=1}\beta_q(v_r,e_i)q(e_i)^{-1}e_i~,
\end{equation}
可以验证$\{e_1,e_2,...,e_k\}\cup\{u_{k+1},...,u_{dim\,V}\}$构成全空间的一组正交基。且显然,该正交化能实施的前提在于子空间的非退化性($q(e_i)$作为分母不可为$0$)。

我们可以构建子空间退化的反例。例如令$\{e_1,e_2,e_3\}$为三维线性空间$V$的一组基。这组基对应的二次型是:
\begin{equation}
q_{ij}=((1\,0\,0)\quad (0\,0\,1)\quad(0\,1\,1))~,
\end{equation}

回忆二次型的表示方式,该双下标二次型表示为行矩阵的行矩阵。由上式得:$$q(e_1,e_2)=q(e_1,e_3)=0,q(e_1,e_1)=q(e_2,e_3)=q(e_3,e_3)=1,q(e_2,e_2)=0~.$$
$e_1$和$e_2$是一组正交基,但其张成的空间无法扩展为全空间的正交基。这是由子空间的退化性导致的。反设存在$v\in V$并与$e_1,e_2$构成一组正交基,且$v=ae_1+be_2+ce_3$,由正交性解得$a=c=0$,则该向量与$e_2$平行,所以假设不成立。

用单向量可以表示$V$的子空间。比如上例,$e_1$与$e_2$张成的子空间可以表示为$e_1\wedge e_2$。
\begin{definition}{}
给定几何代数$\mathcal G(V,q)$。对于\textbf{非零}单向量$A=v_1\wedge v_2\wedge ...\wedge  v_k$,记$\bar{A}=Span\{v_1,v_2\dots,v_k\}$。
\end{definition}
这个定义是合理的。非零单向量意味着$\{v_i\}$是线性无关组,对应的子空间是唯一的,取$\bar A$上的任意一组基做外积,结果必然是$kA,k\in\mathbb F$,$k$为渡矩阵相应的行列式。
\begin{theorem}{}
取几何代数$\mathcal G(V,q)$。如果$A=v_1\wedge v_2\wedge...\wedge v_k \neq 0 $是其上一个\textbf{非零}单向量,则对于任意$u\in V$有
\begin{equation}
u\wedge A=0\Longleftrightarrow  u\in Span\{v_1,v_2,...,v_k\}~.
\end{equation}
\end{theorem}
Proof.把$u$写为基的线性组合,由外积的幂零性可知,必要性成立。下面证明充分条件。

实对称矩阵总可以对角化。因此总可以把$q$化为正交基$\{o_i\}^n_i=1$对角矩阵,把该矩阵下为0的对角元写为$1$,令其为新的二次型$p$。

建立\textbf{线性映射}$\mathcal G(V,q)\rightarrow \mathcal G(V,p)$,使得$f(o_i)=o_i$,且有$f(A\wedge B)=f(A)\wedge f(B),\forall A,B\in\mathcal G(V,q)$。

利用这组正交基计算可得:$A\wedge B=0\Longleftrightarrow f(A)\wedge f(B)=0$。
取$\bar A$上一组正交基$\{e_i\}^k_{i=1}$,并扩张成全空间的正交基,非退化性使得这样的扩张总存在。利用这组基计算,我们有:$f(u)\wedge f(A)=0\Rightarrow f(u)\in\overline{f(A)}=f(\bar{A})$
综合以上讨论便有

\begin{equation}
\begin{aligned}
u\wedge A=0 &\Longleftrightarrow f(u)\wedge f(A)=0\\
&\Longrightarrow f(u)\in\overline{f(A)}=f(\bar{A})\\
&\Longleftrightarrow u\in\bar{A}~.
\end{aligned}
\end{equation}
证明这条定理的关键在于需要一组从子空间$\bar A$扩张的正交基,因而需要额外讨论退化er的情况。通过改变二次型和


下面介绍一条联系外积和Clifford积的定理。
\begin{theorem}{}
每个\textbf{非零}k-单向量都是k个向量的Clifford积。
\end{theorem}
给定几何代数$\mathcal G(V,q)$,取$\bar A$上关于$q|_{\bar A}$上的一组正交基$\{e_i\}^k_{i=1}$,并由外积定义得:
\begin{equation}
A\propto e_1\wedge e_2\wedge...\wedge e_k=e_1e_2...e_k~,
\end{equation}
准确而言,$A$的每个向量在写为这组基的线性组合后有:

\begin{equation}
A=\bigwedge ^k_{i=1}\left(\sum^k_{j=1}a^j_ie_j\right)=\mathrm{det}\,a^j_ie_1 e_2...e_k~,
\end{equation}

由以上定理易得:
\begin{corollary}{}
对于任意单向量$A$,$A^2$为标量。
\end{corollary}
\begin{corollary}{}
对于任意单向量$A$,若$A^2\neq 0$,则$A$有逆元
\begin{equation}
A^{-1}=\frac{A}{A^2}~.
\end{equation}
\end{corollary}