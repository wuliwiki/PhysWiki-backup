% Julia 的函数

\begin{issues}
\issueDraft
\end{issues}

\begin{itemize}
\item 函数名字中可以包含 \verb|!|, 但不能在首字母. 也可以在任何位置包含 unicode
\item 快速定义函数: \verb|f(x, y) = x + y;| (assignment form)
\item 通常来说, 函数名后面加 \verb|!| 表示函数参数会被改变. 例如 \verb|v = [3,2,1]; sort(v)| 返回排好的数组, 但 \verb|v| 不改变. \verb|sort!(v)| 直接改变 \verb|v|.
\item operator 只是特殊的函数, \verb|1 + 2 + 3 + ...| 相当于 \verb|+(1, 2, 3, + ...)|
\end{itemize}


算符都是函数
\begin{table}[ht]
\centering
\caption{算符和对应的函数}\label{JuFunc_tab1}
\begin{tabular}{|c|c|}
\hline
Expression & Calls \\
\hline
\verb|[A B C ...]| & \verb|hcat| \\
\hline
\verb|[A; B; C; ...]| & \verb|vcat| \\
\hline
\verb|[A B; C D; ...]| & \verb|hvcat| \\
\hline
\verb|A'| & \verb|adjoint| \\
\hline
\verb|A[i]| & \verb|getindex| \\
\hline
\verb|A[i] = x| & \verb|setindex!| \\
\hline
\verb|A.n| & \verb|getproperty| \\
\hline
\verb|A.n = x| & \verb|setproperty!| \\
\hline
\end{tabular}
\end{table}

\begin{itemize}
\item \verb|[1,2,3]'| 返回的是 \verb|LinearAlgebra.Adjoint{Int64, Vector{Int64}}|, Julia 真的牛逼.
\end{itemize}
