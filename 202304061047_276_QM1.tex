% 量子力学的算符和本征问题
% 算符|本征值|测量理论|厄密算符|厄密矩阵|动量算符

% 未完成: 考虑将 “量子力学” 词条中与基本假设无关的内容分到另一个词条中。

\pentry{概率密度函数\upref{RandF}, 狄拉克 delta 函数\upref{Delta}, 平面简谐波\upref{PWave}, 量子力学与矩阵\upref{QMmat}}

“量子力学\upref{QM0}” 中我们已经简单介绍了量子力学的基本假设。 这里我们来进行更详细的说明, 注意我们仍然只讨论做一维直线运动的单个微观粒子。 首先来做一个阶段性总结
\begin{itemize}
\item 粒子的状态由\textbf{波函数}描述。
\item 波函数按照\textbf{薛定谔方程}随时间变化。
\item 某时刻若对粒子测量一个物理量, 则先找到该物理量的\textbf{本征函数}(也叫\textbf{本征态}或\textbf{本征矢}) $\Psi_1, \Psi_2, \dots$ 以及对应的\textbf{本征值}, 把粒子在该时刻的波函数 $\Psi(x, t)$ 表示成这些本征函数的\textbf{线性叠加} $\Psi = C_1 \Psi_1 + C_2 \Psi_2\dots$, 测得第 $i$ 个本征值的概率就是系数 $C_i$ 的模长平方。
\item 测量完后, 粒子的波函数\textbf{坍缩}为第 $i$ 个本征函数。
\end{itemize}

\subsection{位置、动量算符(一维)}
首先要注意的是在讨论量子力学的一维问题时, 我们不能完全假设粒子(质点)在三维空间中延某条直线运动。 如果假设粒子延 $x$ 轴运动, 那么在 $y$ 和 $z$ 方向的动量和坐标就可以同时确定, 而这是违反不确定性原理的。 所以一种理解方法是只关心 $x$ 方向的运动而对 $y,z$ 方向的运动不做任何假设, 另一种理解方法是抽象地认为空间中只存在一个维度。

之前提到, 位置的本征函数是一些无穷窄的函数, 叫做狄拉克 $\delta$ 函数\upref{Delta}, 对应的位置本征值则是这些 $\delta$ 函数所在的位置。 动量的本征函数是一些平面波, 对应的动量本征值就是平面波的空间频率乘以一个常数。 然而我们并没有说明某个物理量的本征波函数是怎么得到的, 以下将进一步介绍。

在量子力学中, 每个可测量的物理量都可以对应一个\textbf{算符(operator)}, 算符可以想象为对波函数的一种操作, 算符作用在波函数上可以得到一个新的波函数。 例如某时刻波函数为 $\sin x$, 求导算符 $\dv*{x}$ 作用在 $\sin x$ 上就得到一个新的波函数 $\cos x$。 又例如坐标 $x$ 也可以作为一个算符, 我们定义将其作用在任意波函数 $\Psi(x, t)$ 上, 就是将其相乘, 即 $x\Psi(x, t)$。 又例如任意函数 $f(x)$ 也可以是一个算符, 我们定义将其作用在 $\Psi(x, t)$ 上得 $f(x)\Psi(x, t)$。

在书写习惯上, 我们将某物理量 $Q$ 的算符用 $\Q Q$ 表示, 如位置的算符用 $\Q x$ 表示, 动量的算符用 $\Q p$ 表示。 当我们熟练以后, 为了书写简洁往往将 “$\hat{\phantom{x}}$” 符号省略。

要得到某个物理量的本征函数, 我们需要解\textbf{本征方程}, 注意本征方程中的波函数不含时间
\begin{equation}
\hat Q \psi(x) = \lambda \psi(x)~.
\end{equation}
其中 $\lambda$ 是本征值, $\psi(x)$ 是本征函数, 二者都是未知的。 如果本征值是离散的(如束缚态的能量), 我们就可以用整数角标来 $i$ 来区分不同的本征值和本征函数, 将它们分别记为 $\lambda_i$ 和 $\psi_i(x)$, 如果本征值是连续的(如位置,动量), 我们可以把角标 $i$ 替换为一个实数参数, 比如记为 $\alpha$。 用 $\lambda(\alpha)$ 区分不同本征值, 将对应的波函数记为 $\psi_\alpha(x)$。 注意本征函数不含时间变量 $t$。 另外, 用于物理量的算符, 其本征值必定是实数。 我们把这类算符叫做\textbf{厄米算符(Hermitian operator)}\footnote{数学上也叫\textbf{自伴算符(self-adjoint operator)}}, 以后会具体介绍。

我们姑且认为\footnote{事实上各算符的定义是完全从经典力学的对应概念导出的,具体请参考\textbf{量子力学中的基本算符}\upref{OprQM}。}, 量子力学的基本假设规定\footnote{严格来说这并不是基本假设的一部分, 但初学时这么认为并没有大碍。}一维情况下\textbf{位置的算符} $\Q x$ 是 $x$, \textbf{动量的算符} $\Q p$ 是(偏)微分算符 $-\I\hbar \pdv*{x}$。 其中 $\I$ 是虚数单位, $\hbar$ 是一个常数, 叫做\textbf{约化普朗克常数}, 即\textbf{普朗克常数}\upref{Consts} $h$ 除以 $2\pi$。

事实上, 同一个算符在不同的\textbf{表象(representation)}下具有不同的形式, 这可以类比同一个矢量用不同的基底可以得到不同的坐标\upref{Gvec2}, 同一个线性算符也表示为不同的矩阵。% 链接未完成
这里使用的是最常见的\textbf{位置表象}, 另外有\textbf{动量表象}\upref{moTDSE}, 现在先不用担心。

可以证明位于坐标 $x_0$ 处的狄拉克 $\delta$ 函数\upref{Delta}是 $\Q x$ 的广义上的本征函数, 且本征值为 $x_0$。 位于原点处的 $\delta$ 函数\upref{Delta}记为 $\delta(x)$, 那么 $x_0$ 处的 $\delta$ 函数就是 $\delta (x - x_0)$。 将本征函数和本征值代入本征方程, 得
\begin{equation}
x \delta(x - x_0) = x_0 \delta(x - x_0)~.
\end{equation}
我们可以从函数图像上对该式做一个定性说明: 由于 $\delta(x - x_0)$ 通常只在 $x_0$ 处一个无穷窄的区间不为零, 所以将 $\delta$ 函数乘以 $x$, 就相当于在这个不为零的无穷窄区间乘以 $x_0$。

动量的本征方程为
\begin{equation}
\Q p \psi(x) = -\I\hbar \pdv{x} \psi(x) = p \psi(x)~.
\end{equation}
代入即可证明本征函数为\footnote{也可以通过解微分方程得到。} $\psi_p(x) = \exp(\I k x)$, 对应的动量本征值为 $p(k) = \hbar k$。 $\exp(\I k x)$ 是一维简谐波\upref{PWave}的复数形式, 再次提醒量子力学中习惯把简谐波或平面简谐波叫做平面波, 即使讨论的是一维问题。 量子力学中的平面波总是指 $\exp(\I k x)$ 而不是 $\sin(kx)$ 或 $\cos(kx)$。

利用波长和波数的关系(\autoref{eq_PWave_2}~\upref{PWave})
$\lambda = 2\pi/k$, 以及 $\hbar = h/2\pi$, 我们就可以得到著名的\textbf{德布罗意公式}
\begin{equation}
\lambda = \frac{h}{p}~.
\end{equation}
所以德布罗意公式描述的是动量本征值和本征函数(平面波)的波长之间的关系, 即动量和波长成反比。 所以平面波的波长 $\lambda$ 也叫\textbf{德布罗意波长}。

\subsection{连续本征值的测量}
我们之前在讨论测量理论是都是假设本征值是离散的, 但在上文我们看到位置和动量的本征值都是连续的, 那么如何从离散拓展到连续呢? 首先, 把本征值 $\lambda$ 对应的本征波函数记为 $\psi_\lambda(x)$, 那么某时刻任何波函数 $\psi(x)$ 仍然可以表示为本征波函数的线性组合, 但要把求和变为定积分, 积分范围是 $\lambda$ 所有可能取值的区间
\begin{equation}
\psi(x) = \int C(\lambda)\psi_\lambda(x) \dd{\lambda}~,
\end{equation}
这时线性组合的系数由离散的 $C_i$ 变为 $\lambda$ 的函数 $C(\lambda)$。

现在, 测量结果的概率就很自然地从离散的概率 $\abs{C_i}$ 变为概率密度函数\upref{RandF} $\abs{C(\lambda)}^2$ 了。 也就是说, 测量值落在某个区间 $\lambda \in [a, b]$ 的概率为
\begin{equation}
P_{ab} = \int_a^b \abs{C(\lambda)}^2 \dd{\lambda}~.
\end{equation}
例如任何波函数分解成无穷多个不同的位置本征函数 $\delta(x-x_0)$ 的线性组合得
\begin{equation}
\psi(x) = \int_{-\infty}^{+\infty} C(x_0) \delta(x - x_0) \dd{x_0}~.
\end{equation}
根据 $\delta$ 函数的性质得 $C(x) = \psi(x)$, 所以在 $x \in [a, b]$ 区间发现粒子的概率为
\begin{equation}
P_{ab} = \int_a^b \abs{C(x)}^2\dd{x} = \int_a^b \abs{\psi(x)}^2\dd{x}~,
\end{equation}

又例如把波函数分解成平面波
\begin{equation}\label{eq_QM1_2}
\psi(x) = \int_{-\infty}^{+\infty} C(p)\frac{\E^{\I p x/\hbar}}{\sqrt{2\pi\hbar}} \dd{p}~.
\end{equation}
这十分类似傅里叶变换(\autoref{eq_FTExp_1}~\upref{FTExp}), 其中 $1/\sqrt{2\pi\hbar}$ 是平面波的归一化系数。 容易证明
\begin{equation}\label{eq_QM1_3}
C(p) = \int_{-\infty}^{+\infty} \psi(x) \frac{\E^{-\I p x/\hbar}}{\sqrt{2\pi\hbar}} \dd{x}~.
\end{equation}
如果测量粒子的动量, 结果落在 $p \in [a, b]$ 的概率为
\begin{equation}
P_{ab} = \int_a^b \abs{C(p)}^2 \dd{p}~.
\end{equation}
事实上, 若使用原子单位, 则\autoref{eq_QM1_2} \autoref{eq_QM1_3} 将和傅里叶变换完全一样, 见\autoref{ex_AU_2}~\upref{AU}。

\subsection{动能、势能、哈密顿算符(一维)}
量子力学的基本假设规定, 其他所有\textbf{可观测量的算符都可以通过位置和动量算符拼凑而成}, 其形式与经典力学中对应物理量的形式相同。 例如, 经典力学中的动能为 $p^2/(2m)$, 那么量子力学中的动能算符就是
\begin{equation}\label{eq_QM1_1}
\Q T = \frac{\Q p^2}{2m}~.
\end{equation}

要理解算符运算并不难, 这里的 $\Q p^2$ 也可以记为 $\Q p \Q p$, 即两个动量算符\textbf{相乘}。 两个算符相乘的定义是, 将右边的算符先作用在波函数上, 再将左边的算符作用在波函数上。 所以 $\Q p^2$ 作用在波函数 $\psi(x)$ 上, 就是
\begin{equation}
\Q p^2 \psi = -\I \hbar\pdv{x} \qty(-\I \hbar\pdv{x} \psi)~.
\end{equation}
由于常数可以提到求导算符的外面, 这就相当于关于 $x$ 求二阶偏导, 然后在乘以 $(-\I \hbar)^2 = -\hbar^2$。
\begin{equation}
\Q p^2 \psi(x) =  -\hbar^2 \pdv[2]{x} \psi(x)~,
\end{equation}
所以动能算符为
\begin{equation}
\Q T = \frac{\Q p^2}{2m} = -\frac{\hbar^2}{2m} \pdv[2]{x}~.
\end{equation}

有了动能的算符, 我们就可以列出动能的本征方程并解出其本征函数和本征值 $E_k$(角标 $k$ 代表 kinetic)
\begin{equation}
\Q T \psi(x) = -\frac{\hbar^2}{2m} \pdv[2]{x}\psi(x) = E_k \psi(x)~.
\end{equation}
巧的是, 解出动能的本征函数与动量的本征函数相同, 也是 $\exp(-\I k x)$, 但对应的本征值不同, 是 $E_k = \hbar^2 k^2/(2m)$。 这事实上并不是巧合, 因为 $\Q T$  正比于 $\Q p^2$, 当第一个 $\Q p$ 作用在动量的本征函数上, 得到的是标量 $p$ 乘以该本征函数, 再经第二个 $\Q p$ 作用, 同样相当于乘以一个 $p$:
\begin{equation}
\ali{
-\I \hbar \pdv{x} \qty(-\I \hbar \pdv{x} \E^{\I k x}) &= -\I \hbar \pdv{x} \qty(\hbar k \E^{\I k x}) =  \hbar k \qty(-\I \hbar \pdv{x} \E^{\I k x})\\
&=  \hbar^2 k^2\E^{\I k x} = p^2 \E^{\I k x}~.
}\end{equation}
两边除以 $2m$, 就验证了动能的本征方程, 并得到本征值。 需要注意的是, 同一个动能本征值 $E_k$ 对应两个互为相反数的波数 $k$, 即存在两个线性无关的本征态(分别是向左和向右的平面波), 且这两个本征态的任意线性组合都是同一个能量本征值的本征函数。 我们把这种一个本征值对应多个本征函数的情况叫做\textbf{简并}, 如果最多由 $N$ 个本征函数, 就有 \textbf{$N$ 重简并}。 所以一维情况下, 动能具有二重简并。

再看势能算符, 经典力学中一维势能函数记为 $V(x)$ 是位置 $x$ 的函数。 那么量子力学中为了变成算符就把 $x$ 替换为位置算符 $\Q x$ 即可。 这和\autoref{eq_QM1_1} 中把动量替换成算符是一个道理。 例如当 $V(x)$ 可以泰勒展开为 $\sum_n c_n x^n$ 时, 势能算符就是 $\sum_n c_n \Q x^n$。 但在位置表象下 $\Q x$ 就是 $x$, 所以势能算符仍然是 $V(x)$。 把势能算符作用在波函数上, 就是把它和波函数相乘。

有了动能和势能算符, 我们就可以把它们相加得到(一维)能量算符, 也就是哈密顿算符
\begin{equation}
\Q H = \Q T + \Q V = -\frac{\hbar^2}{2m} \pdv[2]{x} + V(x)
\end{equation}
哈密顿算符的本征方程, 也就是能量的本征方程, 就叫\textbf{定态薛定谔方程(time independent Schrodinger equation, 缩写 TISE)}
\begin{equation}
-\frac{\hbar^2}{2m} \pdv[2]{x}\psi(x) + V(x)\psi(x) = E\psi(x)
\end{equation}
该方程的解取决于 $V(x)$ 的形式, 我们留到 “定态薛定谔方程(单粒子一维)\upref{SchEq}” 详细讨论。
