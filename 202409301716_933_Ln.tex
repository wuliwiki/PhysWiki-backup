% 对数与对数函数(高中)
% keys 对数|对数函数
% license Xiao
% type Tutor
\begin{issues}
\issueDraft
\end{issues}

\subsection{对数运算}


\subsection{自然对数函数}
以 $\E$ 为底的对数函数 $\log_{\E} x$ 叫做\textbf{自然对数}, 通常记为
\begin{equation}
\ln x \qquad \text{或} \qquad \log x~.
\end{equation}
函数图如\autoref{fig_Ln_2}。
\begin{figure}[ht]
\centering
\includegraphics[width=7cm]{./figures/ce690bcbd8c28a93.png}
\caption{几种不同底的对数函数} \label{fig_Ln_2}
\end{figure}
回忆对数函数的性质
\begin{equation}
\log_a x = \frac{\log_c x}{\log_c a}~.
\end{equation}
所以不同底的对数函数是成比例的。



\subsection{指数函数与对数函数的相似性}

\pentry{函数的变换(高中)\nref{nod_FunTra}}{nod_a54a}

根据幂运算和对数运算的性质,任意$f(x)=a^x$都可以变形,得到:

\begin{equation}
f(x)=e^{x\ln a}~.
\end{equation}

即,所有的指数函数都可以由$e^x$通过在$x$方向上伸缩或关于$y$轴对称($\ln a<0$时)得到,或者所有的指数函数$a^x$都可认为是$f(x)=e^x$与$g(x)=x\ln a$复合得到的$f(g(x))$。

同理根据对数运算的性质,任意$f(x)=\log_ax$都可以变形,得到:

\begin{equation}
f(x)=\frac{1}{\ln a}\ln x\iff f(x)\ln a=\ln x~.
\end{equation}

即,所有的对数函数都可以由$\ln x$通过在$y$方向上伸缩或关于$x$轴对称($\ln a<0$时)得到,或者所有的对数函数$\log_ax$都可认为是$f(x)=\$与$g(x)=x\ln a$复合得到的$f(g(x))$。