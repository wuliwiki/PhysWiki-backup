% 能量守恒定律(综述)
% license CCBYSA3
% type Wiki

本文根据 CC-BY-SA 协议转载翻译自维基百科\href{https://en.wikipedia.org/wiki/Conservation_of_energy}{相关文章}。

能量守恒定律表明,孤立系统的总能量保持不变;即它随着时间的推移是守恒的。[1] 对于封闭系统,该原理表明系统内的总能量只能通过能量的进出而改变。能量既不能被创造,也不能被销毁;它只能从一种形式转化或转移为另一种形式。例如,当一根炸药爆炸时,化学能转化为动能。如果把爆炸中释放的所有形式的能量相加,如碎片的动能和势能以及热和声,就可以得到炸药燃烧过程中化学能的精确减少量。

在经典物理中,能量守恒与质量守恒是不同的。然而,狭义相对论表明质量和能量是相互关联的,反之亦然,其关系由方程 \( E = mc^2 \) 表示,即质能等价,现在科学认为整体的质能是守恒的。从理论上讲,这意味着任何具有质量的物体本身可以转化为纯能量,反之亦然。然而,人们认为这只有在最极端的物理条件下才可能发生,例如宇宙大爆炸后的极短时间内,或黑洞发射霍金辐射时。

根据驻行动原理,通过诺特定理可以严格证明能量守恒是连续时间平移对称性的结果;即物理定律不随时间改变这一事实。

能量守恒定律的一个结果是第一类永动机不可能存在;也就是说,没有外部能量供应的系统不能向其周围环境提供无限的能量。[2] 根据能量的定义,能量守恒在宇宙尺度上可以被认为在广义相对论中被违背。[3]
\subsection{历史}
早在公元前550年左右,米利都的泰勒斯等古代哲学家就隐约意识到某种构成万物的基础物质的守恒。然而,他们的理论与我们今天所知的“质能”并无直接关联(例如,泰勒斯认为这一基础物质是水)。恩培多克勒(公元前490–430年)写道,在他由四种根本元素(土、气、水、火)构成的宇宙体系中,“没有事物生成或毁灭”;[4] 相反,这些元素经历不断的重新排列。另一方面,伊壁鸠鲁(约公元前350年)认为宇宙中的一切都由不可分割的物质单元构成——即原子的古老先驱——他也有一些关于守恒的观点,认为“万物的总和一直是现在的样子,并将永远保持如此”。[5]

1605年,佛兰德科学家西蒙·斯蒂文基于永动机不可能存在的原则,解决了若干静力学问题。

1639年,伽利略发表了对几种情况的分析——包括著名的“中断摆”——可以用现代语言描述为势能与动能的守恒转换。他指出,运动物体上升的高度等于其下降的高度,并据此推断出惯性的概念。这个观察的显著之处在于,在无摩擦表面上,运动物体上升的高度与表面的形状无关。

1669年,克里斯蒂安·惠更斯发表了他的碰撞定律。在他列出的碰撞前后保持不变的量中,包括物体线性动量的总和和动能的总和。然而,当时尚未理解弹性碰撞和非弹性碰撞之间的差别。这导致后来的研究者在争论哪种守恒量更为基础。在他的著作《摆钟》中,他更清晰地阐述了运动物体上升高度的概念,并将此观点与永动机不可能性联系起来。惠更斯对摆运动动力学的研究基于一个简单的原则:重物的重心无法自我提升。
\begin{figure}[ht]
\centering
\includegraphics[width=6cm]{./figures/3d8704027bff5214.png}
\caption{戈特弗里德·莱布尼茨} \label{fig_NLSH_1}
\end{figure}
在1676年至1689年间,戈特弗里德·莱布尼茨首次尝试对与运动(动能)相关的能量进行数学表述。基于惠更斯的碰撞研究,莱布尼茨注意到在许多机械系统中(由若干质量为 \( m_i \) 的物体组成,每个物体的速度为 \( v_i \)),
\[
\sum _{i}m_{i}v_{i}^{2}~
\]
在各质量不相互作用的情况下是守恒的。他称这个量为系统的“活力”或“生命力”。该原理精确地描述了无摩擦情况下动能近似守恒的情形。当时许多物理学家,包括艾萨克·牛顿在内,认为动量守恒(在有摩擦的系统中也成立)是定义的活力:
\[
\sum _{i}m_{i}v_{i}~
\]
后来证明,在适当条件下(如弹性碰撞),这两个量可以同时守恒。

1687年,艾萨克·牛顿发表了《自然哲学的数学原理》,其中提出了他的运动定律。该书围绕力和动量的概念展开。然而,研究者们很快认识到,书中提出的原则虽然适用于点质量,但不足以解决刚体和流体的运动问题,还需要其他原则。

到1690年代,莱布尼茨主张活力守恒和动量守恒削弱了当时流行的交互主义二元论哲学学说。(在19世纪,随着人们对能量守恒的理解加深,莱布尼茨的基本论点得到了广泛认可。一些现代学者继续特别通过守恒理论来批判二元论,而另一些学者则将这一论点纳入关于因果封闭性的更普遍论证中。)[6]
\begin{figure}[ht]
\centering
\includegraphics[width=6cm]{./figures/ab34f1710db0b732.png}
\caption{丹尼尔·伯努利} \label{fig_NLSH_2}
\end{figure}
活力守恒定律由父子二人约翰·伯努利和丹尼尔·伯努利所倡导。约翰在1715年全面阐明了静力学中虚功原理的普遍性,而丹尼尔在他1738年出版的《流体动力学》一书中,基于这一单一的活力守恒原则。丹尼尔对流水活力损失的研究使他提出了伯努利原理,该原理断言损失与流体动力压力变化成正比。丹尼尔还为水力机械提出了功和效率的概念;他提出了气体的动理论,并将气体分子动能与气体温度联系起来。

大陆物理学家对活力的关注最终导致了控制力学的驻定性原理的发现,如达朗贝尔原理、拉格朗日力学和哈密顿力学的表述。
\begin{figure}[ht]
\centering
\includegraphics[width=6cm]{./figures/48adde9092d4cb36.png}
\caption{夏特莱侯爵夫人} \label{fig_NLSH_3}
\end{figure}
埃米莉·杜·夏特莱(Émilie du Châtelet,1706–1749)提出并验证了总能量守恒的假设,这不同于动量守恒。受戈特弗里德·莱布尼茨理论的启发,她重复并宣传了威廉·格拉夫桑德(Willem 's Gravesande)在1722年设计的一个实验:将球从不同高度释放,使其落入一张软泥板中。实验表明,每个球的动能(通过所移动的材料量来表示)与速度的平方成正比。发现泥土的变形与球下落的高度成正比,相当于初始的势能。而包括牛顿和伏尔泰在内的一些早期研究者认为,“能量”与动量并无区别,因此应与速度成正比。根据这种理解,泥土的变形应与球下落高度的平方根成正比。在经典物理中,正确的公式为\(E_k = \frac{1}{2}mv^2\)其中 \( E_k \) 表示物体的动能,\( m \) 为其质量,\( v \) 为其速度。基于此,杜·夏特莱提出能量在任何形式中必须具有相同的维度,以便能够在不同形式(动能、势能、热能等)之间进行比较。[7][8]

约翰·斯米顿(John Smeaton)、彼得·尤尔特(Peter Ewart)、卡尔·霍尔茨曼(Carl Holtzmann)、古斯塔夫·阿道夫·赫恩(Gustave-Adolphe Hirn)和马克·塞金(Marc Seguin)等工程师认识到仅靠动量守恒不足以进行实际计算,并使用了莱布尼茨的原则。该原则也受到一些化学家的支持,如威廉·海德·沃拉斯顿(William Hyde Wollaston)。约翰·普莱费尔(John Playfair)等学者很快指出,动能显然不是守恒的。这对基于热力学第二定律的现代分析来说显而易见,但在18和19世纪,失去的能量的去向仍然未知。

渐渐地,人们开始怀疑摩擦下运动不可避免产生的热量是另一种形式的活力。1783年,安托万·拉瓦锡(Antoine Lavoisier)和皮埃尔-西蒙·拉普拉斯(Pierre-Simon Laplace)对活力理论和热质说这两种相互竞争的理论进行了回顾。[9][10] 拉姆福德伯爵(Count Rumford)在1798年观察到钻孔制造大炮时产生的热量,进一步支持了机械运动可以转化为热量的观点,并强调了这种转化具有定量性且可以预测(从而允许在动能和热量之间建立一个通用的转换常数)。随后,活力开始被称为能量,这是托马斯·杨(Thomas Young)于1807年首次以这个意义使用该词后的结果。
\begin{figure}[ht]
\centering
\includegraphics[width=6cm]{./figures/1e62a28fbd913dc3.png}
\caption{加斯帕尔-古斯塔夫·科里奥利} \label{fig_NLSH_4}
\end{figure}
对活力重新校准为
\[
\frac{1}{2} \sum _{i}m_{i}v_{i}^{2}~
\]
(可以理解为将动能转化为功)主要是加斯帕尔-古斯塔夫·科里奥利(Gaspard-Gustave Coriolis)和让-维克多·庞斯莱特(Jean-Victor Poncelet)在1819年至1839年间的研究成果。前者称之为“工作量”(quantité de travail),后者称之为“机械功”(travail mécanique),并且二人都支持在工程计算中使用这一概念。

1837年,卡尔·弗里德里希·默尔(Karl Friedrich Mohr)在《物理学杂志》上发表的论文《论热的本质》(Über die Natur der Wärme)中提出了能量守恒学说的早期表述之一:“除了已知的54种化学元素外,物理世界中只有一种作用因子,这就是所谓的 Kraft(能量或功)。它可以根据情况表现为运动、化学亲和力、凝聚力、电、光和磁;并且从其中一种形式可以转化为任何其他形式。
\subsubsection{热的机械当量}
现代守恒原理发展的关键阶段是热的机械当量的演示。热质说认为热既不能被创造也不能被销毁,而能量守恒则涉及相反的原则,即热和机械功是可以互换的。

在十八世纪中叶,俄国科学家米哈伊尔·罗蒙诺索夫提出了他的微粒动热理论,否定了热质的概念。通过实验证据,罗蒙诺索夫得出结论,热量并不是通过热质流体的粒子传递的。

1798年,拉姆福德伯爵(本杰明·汤普森)对钻孔制造大炮时产生的摩擦热进行了测量,并提出了热是一种动能的观点;他的测量结果驳斥了热质说,但由于精确性不足,仍留有疑问。

机械等效原理最早以现代形式由德国外科医生尤利乌斯·罗伯特·冯·迈尔于1842年提出。[11] 迈尔在前往荷属东印度的航程中得出这一结论,他发现患者的血液呈现更深的红色,因为在较热的气候下,他们消耗较少的氧气,从而消耗较少的能量来维持体温。他发现热量和机械功都是能量的形式,并在1845年提升了物理学知识后,发表了一篇专著,提出了它们之间的定量关系。[12]
\begin{figure}[ht]
\centering
\includegraphics[width=6cm]{./figures/e249874acfbf6672.png}
\caption{詹姆斯·焦耳} \label{fig_NLSH_6}
\end{figure}
与此同时,在1843年,詹姆斯·普雷斯科特·焦耳通过一系列实验独立发现了机械当量。在其中一个实验中,也就是如今称为“焦耳装置”的实验中,一个系在绳子上的下落重物使浸入水中的桨叶旋转。他证明了重物下落时失去的重力势能等于水通过与桨叶摩擦所获得的内能。

在1840至1843年期间,工程师路德维希·A·科尔丁进行了类似的工作,但他的研究在丹麦以外鲜为人知。

焦耳和迈尔的研究都遭遇了抵制和忽视,但最终焦耳的研究得到了更广泛的认可。

关于焦耳和迈尔在优先权上的争论,请参见“热的机械当量:优先权”。
\begin{figure}[ht]
\centering
\includegraphics[width=6cm]{./figures/119dd2c1792d8908.png}
\caption{焦耳用于测量热的机械当量的装置。一个系在绳子上的下落重物使浸入水中的桨叶旋转。} \label{fig_NLSH_7}
\end{figure}
1844年,威尔士科学家威廉·罗伯特·格罗夫提出了力学、热、光、电和磁之间的关系,认为它们都是单一“力”(在现代术语中即能量)的表现形式。1846年,格罗夫在他的著作《物理力的相关性》中发表了他的理论。[13] 1847年,赫尔曼·冯·亥姆霍兹借鉴了焦耳、萨迪·卡诺和埃米尔·克拉佩龙的早期研究,得出了与格罗夫相似的结论,并在他的著作《论力的守恒》(Über die Erhaltung der Kraft, 1847)中发表了这一理论。[14] 这一出版物标志着该原理在现代的普遍接受。

1850年,苏格兰数学家威廉·兰金首次使用“能量守恒定律”一词来描述这一原则。[15]

1877年,彼得·格思里·泰特声称该原理起源于艾萨克·牛顿,基于他对《自然哲学的数学原理》第40和41条的创造性解读。这现今被认为是辉格式历史的一个例子。[16]
\subsubsection{质能等价}
物质由原子及其构成部分组成,具有固有的或静止质量。在19世纪有限的已知经验范围内,人们发现这种静止质量是守恒的。爱因斯坦1905年的狭义相对论表明,静止质量对应着等价的静止能量。这意味着静止质量可以转化为等价量的(非物质的)能量形式,如动能、势能和电磁辐射能。根据20世纪的经验,当这种转化发生时,静止质量不再守恒,但总质量或总能量仍然守恒。所有形式的能量都对总质量和总能量作出贡献。

例如,电子和正电子各自具有静止质量。它们可以相互湮灭,将其总静止能量转化为光子,光子具有电磁辐射能但没有静止质量。如果这一过程发生在一个不向外界释放光子或其能量的孤立系统中,则系统的总质量和总能量不会改变。所产生的电磁辐射能对系统的惯性(和任何重量)的贡献与湮灭前电子和正电子的静止质量贡献相同。同样,非物质的能量形式也可以转化为具有静止质量的物质。

因此,能量守恒(总量,包括物质或静止能量)和质量守恒(总量,不仅限于静止质量)是一个(等价的)定律。在18世纪,这些定律曾表现为两个看似独立的定律。
\subsubsection{β衰变中的能量守恒}  
主条目:β衰变 § 中微子  
1911年发现,在β衰变中发射的电子具有连续而非离散的能谱,这似乎与能量守恒相矛盾,当时认为β衰变只是一个原子核发射电子的简单过程。[17][18] 这个问题最终在1933年由恩里科·费米解决,他提出了对β衰变的正确描述,即同时发射电子和反中微子,反中微子携带了表面上缺失的能量。[19][20]
\subsection{热力学第一定律} 
对于封闭的热力学系统,热力学第一定律可以表示为:
\[
\delta Q = \mathrm{d}U + \delta W~
\]
或等价地,
\[
\mathrm{d}U = \delta Q - \delta W~
\]
其中,\( \delta Q \) 是通过加热过程传递到系统中的能量,\( \delta W \) 是系统对其环境做功而损失的能量,\( \mathrm{d}U \) 是系统内部能量的变化。

在热量和功的项之前使用 \( \delta \) 表示这些能量增量需要以不同方式解释,而不是 \( \mathrm{d}U \) 形式的内部能量增量(参见非精确微分)。功和热指的是给系统增加或减少能量的过程类型,而内部能量 \( U \) 是系统在不变的热力学平衡状态下的一个属性。因此,“热能”一词用于 \( \delta Q \) 表示“作为加热结果添加的那部分能量”,而不是指某种特定形式的能量。同样,“功能”一词用于 \( \delta W \) 表示“作为做功结果损失的那部分能量”。因此,可以描述处于特定状态的热力学系统所拥有的内部能量的数量,但仅凭当前状态,无法确定过去有多少能量通过加热或冷却流入或流出系统,也无法确定因系统做功或被做功而流入或流出的能量数量。

熵是系统状态的一个函数,用于描述将热转换为功的可能性限制。

对于一个简单的可压缩系统,系统所做的功可以写成:
\[
\delta W = P \, \mathrm{d}V~
\]
其中,\( P \) 是压力,\( \mathrm{d}V \) 是系统体积的微小变化,它们都是系统的变量。在一个理想化并极其缓慢的过程(称为准静态过程)中,如果该过程被视为可逆的,并且热量是从温度略高于系统温度的热源传递来的,则热量可以写成
\[
\delta Q = T \, \mathrm{d}S~
\]
其中,\( T \) 是温度,\( \mathrm{d}S \) 是系统熵的微小变化。温度和熵是系统状态的变量。

对于一个开放系统(可以与环境交换质量),如果系统有多个壁使得质量传递通过刚性壁,热量和功的传递分别通过其他壁进行,那么第一定律可以写成:[21]
\[
\mathrm{d}U = \delta Q - \delta W + \sum_{i} h_{i} \, dM_{i}~
\]
其中,\( dM_{i} \) 是物质种类 \( i \) 的增加质量,\( h_{i} \) 是相应的单位质量焓。注意一般情况下 \( dS \neq \delta Q / T \),因为物质携带自身的熵。而是
\[
dS = \frac{\delta Q}{T} + \sum_{i} s_{i} \, dM_{i}~
\]
其中 \( s_{i} \) 是类型 \( i \) 的单位质量熵,由此可得出基本的热力学关系
\[
\mathrm{d}U = T \, dS - P \, dV + \sum_{i} \mu_{i} \, dN_{i}~
\]
因为化学势 \( \mu_{i} \) 是种类 \( i \) 的偏摩尔吉布斯自由能,而吉布斯自由能定义为
\[
G \equiv H - TS~
\]
\subsection{诺特定理}
\begin{figure}[ht]
\centering
\includegraphics[width=6cm]{./figures/1eed4146266a9f9f.png}
\caption{埃米·诺特(Emmy Noether,1882-1935)是一位具有影响力的数学家,以其在抽象代数和理论物理方面的开创性贡献而闻名。} \label{fig_NLSH_8}
\end{figure}
能量守恒是许多物理理论中的共同特征。从数学角度来看,它被理解为诺特定理的结果。该定理由埃米·诺特于1915年提出,并于1918年首次发表。根据这一定理,在任何遵循驻行动原理的物理理论中,每个连续对称性都对应一个守恒量;如果理论的对称性是时间不变性,则对应的守恒量被称为“能量”。[22] 能量守恒定律是时间平移对称性的结果;能量守恒由物理定律不随时间本身变化的经验事实所暗示。从哲学上讲,这可以表述为“没有事物依赖于时间本身”。换句话说,如果物理系统在时间平移的连续对称性下保持不变,则其能量(即时间的规范共轭量)是守恒的。相反,不在时间平移下保持不变的系统(例如具有时间相关势能的系统)不表现出能量守恒——除非我们认为它们与另一个外部系统交换能量,从而使扩展系统的理论再次成为时间不变的。在平坦时空中的物理理论(如狭义相对论和量子理论,包括量子电动力学)中,有限系统的能量守恒是有效的。
\subsubsection{狭义相对论}
随着亨利·庞加莱和阿尔伯特·爱因斯坦发现狭义相对论,能量被认为是能量-动量四维矢量的一个分量。对于任何封闭系统,从任何给定的惯性参考系来看,该矢量的四个分量(一个能量和三个动量分量)在时间上分别守恒。该矢量的长度(闵可夫斯基范数)也保持不变,对于单个粒子,它是静止质量,对于粒子系统,它是系统的不变质量(在计算长度之前分别对动量和能量求和)。

单个有质量粒子的相对论能量包含一个与其静止质量相关的项,此外还有其运动的动能项。在有质量粒子的零动能极限(即静止系)或在物体或系统保持动能的动量中心系中,粒子或物体的总能量(包括系统中的内部动能)与静止质量或不变质量成正比,正如公式所描述的:
\[
E = mc^{2}~
\]
因此,在狭义相对论中,随着时间的推移能量守恒定律仍然成立,前提是观察者的参考系保持不变。这适用于系统的总能量,尽管不同的观察者对能量值的看法有所不同。还保持守恒并对所有观察者不变的是不变质量,这是任何观察者所能看到的最小系统质量和能量,并由能量-动量关系定义。
\subsubsection{广义相对论}
广义相对论引入了新现象。在膨胀的宇宙中,光子会自发红移,系绳会自发增加张力;如果真空能量为正,则随着空间体积的增大,宇宙的总真空能量似乎也会自发增加。一些学者认为,能量不再以任何可识别的形式有意义地守恒。[23][24]

约翰·贝兹(John Baez)认为,除某些特殊情况外,能量-动量守恒并不明确。能量-动量通常借助于应力-能量-动量伪张量来表达。然而,由于伪张量不是张量,它们在参考系之间不能平滑转换。如果考虑的度规是静态的(即不随时间变化)或渐近平坦的(即在无限远处时空看起来是空的),那么能量守恒可以成立而不会有太大的问题。实际上,一些度规(特别是似乎主导宇宙的弗里德曼-勒梅特-罗伯逊-沃克度规)不满足这些约束条件,因此能量守恒并不明确。[25] 除了依赖于坐标系外,伪张量能量还依赖于所使用的伪张量类型;例如,当使用默勒(Møller)伪张量计算克尔-纽曼黑洞外的能量时,该能量比使用爱因斯坦伪张量计算的结果大一倍。[26]

对于渐近平坦的宇宙,爱因斯坦和其他科学家通过引入特定的全局引力势能来弥补由时空膨胀或收缩引起的质能变化,从而挽救了能量守恒。这种全局能量没有明确定义的密度,技术上也无法应用于非渐近平坦的宇宙;然而,在实际应用中,这可以通过调整来处理,因此在这种观点下,能量在我们的宇宙中是守恒的。[27][3] 艾伦·古斯(Alan Guth)曾表示,宇宙可能是“终极的免费午餐”,并推测在考虑引力势能后,宇宙的净能量为零。[28]
\subsubsection{量子理论}
在量子力学中,量子系统的能量由称为哈密顿算符的自伴(或厄米)算符描述,该算符作用于系统的希尔伯特空间(或波函数空间)。如果哈密顿算符是时间无关的,则系统在演化过程中测量结果的出现概率不会随时间变化。因此,能量的期望值也是时间不变的。在量子场论中,局部能量守恒由能量-动量张量算符的量子诺特定理保证。因此,量子系统在正常的幺正演化中能量是守恒的。

然而,当应用非幺正的波恩规则时,如果系统不处于能量本征态,测量的系统能量可能低于或高于期望值。(对于宏观系统,这种效应通常小到无法测量。)这种能量差的去向尚不完全清楚;大多数物理学家认为,测量过程中能量被转移到宏观环境中或从中获得,[29] 而另一些物理学家认为观测到的能量只是“平均守恒”。[30][31][32] 尚未有实验被确认作为量子力学中违反能量守恒定律的确凿证据,但这并不排除某些新的实验可能会找到量子力学中违反能量守恒定律的证据。[31]
\subsubsection{现状}
在类似Orbo的永动机背景下,埃里克·阿什教授在BBC上表示:“否认[能量守恒]不仅会破坏一些小科学知识,整个科学大厦都会崩塌。我们建构现代世界的所有技术都将化为废墟”。正因为能量守恒,“我们知道——而不必研究特定装置的细节——Orbo不可能运作”。[33]

能量守恒作为一个基本物理原理已有大约两百年。从现代广义相对论的角度来看,实验室环境可以很好地近似为闵可夫斯基时空,在此时空中能量完全守恒。整个地球可以很好地用史瓦西度规近似,在此度规下能量同样完全守恒。基于所有实验证据,任何新的理论(如量子引力)如果要成功,必须解释为什么在地面实验中能量似乎总是完全守恒的。[34] 在一些推测性理论中,量子力学的修正过小,难以在现有的粒子加速器达到的TeV级别附近检测到。双重狭义相对论模型可能主张在足够高能粒子的情况下,能量-动量守恒会失效;然而,这些模型受到观测约束,因宇宙射线似乎能够历经数十亿年而没有显示出异常的非守恒行为。[35] 一些量子力学的解释认为,在应用波恩规则时,由于波函数局域化,观察到的能量趋向增加。如果属实,物体可能会自发升温;因此,这类模型受到大型、冷却的天文观测以及(通常是超冷的)实验室实验的约束。[36]

米尔顿·A·罗斯曼指出,能量守恒定律已在核物理实验中被验证到十亿分之一(\(10^{15}\))的精确度。他将这种精度定义为“对于所有实际用途来说是完美的”。[37]
\subsection{另见}
\begin{itemize}
\item 能量质量
\item 能量转化
\item 拉格朗日力学
\item 热力学定律
\item 零能量宇宙
\end{itemize}
\subsection{参考文献}\\
1. Richard Feynman (1970). \textbf{费曼物理学讲义} 第一卷. Addison Wesley. ISBN 978-0-201-02115-8.\\
2. Planck, M. (1923/1927). \textbf{热力学论述},第三版英文版,由A. Ogg从第七版德文版翻译,Longmans, Green & Co., 伦敦,第40页。\\
3. "能量不是守恒的"。\textbf{Discover Magazine}. 2010年。2022年9月25日检索。\\
4. Janko, Richard (2004). "恩培多克勒,《论自然》" (PDF). \textbf{纸莎草学和碑铭学杂志},150: 1–26。\\
5. Laertius, Diogenes. 著\textbf{名哲学家生平:伊壁鸠鲁}。[永久死链] 此段引自第欧根尼引用的伊壁鸠鲁亲笔信件,其中概述了他的哲学信条。\\