% 角动量的叠加原理

\subsection{两个粒子自旋角动量的叠加}
假设氢原子处于基态(这样就不用考虑轨道角动量),其电子和质子都是自旋为$1/2$的粒子.这两个粒子都可以自旋向上或者自旋向下,也就是说由四种自旋的可能性:
\begin{equation}
\chi^{(1)}_+\chi^{(2)}_+\equiv\uparrow\uparrow;\ \chi^{(1)}_+\chi^{(2)}_-\equiv\uparrow\downarrow;\ \chi^{(1)}_-\chi^{(2)}_+\equiv\downarrow\uparrow;\ \chi^{(1)}_-\chi^{(2)}_-\equiv\downarrow\downarrow
\end{equation}
更严格地讲,每一个粒子是处在上自旋和下自旋线性组合的状态,两个粒子构成的体系是上面
四个态的线性组合态.你应当对我们接下来的操作和推导产生怀疑,不过在介绍严格可靠的形式化之前,我们先用没有严格数学依据和不优雅的箭头做一个初步的介绍.其中$\chi^{(1)}$也就是第一个箭头代表电子自旋,第二个$\chi^{(2)}$代表质子自旋.

我们将这个氢原子的总角动量定义为:
\begin{equation}
\bvec S \equiv \bvec S^{(1)}+\bvec S^{(2)}
\end{equation}
上面的四个组合态都是$S_z$的本征态,也就有:
\begin{equation}
S_z\chi_1\chi_2=(S_z^{(1)}+S_z^{(2)})\chi_1\chi_2=(S_z^{(1)}\chi_1)\chi_2+\chi_1(S_z^{(2)}\chi_2)
\end{equation}
\textbf{注意:}$S_z^{(1)}$仅作用于$\chi_1$;$S_z^{(2)}$仅作用于$\chi_2$.那么我们就得到了该系统$S_z$的本征值:$m\hbar$中的$m=m_1+m_2$:
\begin{equation}
m=1: \ \uparrow\uparrow;\quad m=0: \ \uparrow\downarrow \rm{or} \downarrow\uparrow;\quad m=-1:\ \downarrow\downarrow
\end{equation}


\subsection{自旋角动量和轨道角动量的叠加}
等等.你应该要感到疑惑.对于给定的这两个粒子,其自旋角动量与空间波函数无关,而其轨道角动量与空间波函数无关.
