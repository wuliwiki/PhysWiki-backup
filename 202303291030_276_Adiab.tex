% 绝热过程
% 绝热过程|体积|压强|状态方程|做功

\pentry{热容量\upref{ThCapa}}

在系统状态变化过程中, 如果和外界没有热量和粒子交换, 这个过程就叫做\textbf{绝热过程(adiabatic process)}
\begin{equation}
\Delta Q = 0~.
\end{equation}
根据熵增(\autoref{Entrop_eq1}~\upref{Entrop}) 的定义, 绝热过程是一个等熵过程。

由热力学第一定律\autoref{Th1Law_eq1}~\upref{Th1Law},
\begin{equation}\label{Adiab_eq1}
W + \Delta E = 0
\end{equation}
即系统对外做功和内能增加之和为零。

理想气体的绝热过程中压强体积曲线为
\begin{equation}\label{Adiab_eq5}
P V^\gamma = C
\end{equation}
其中 $C$ 为常数, $\gamma$ 为\textbf{绝热指数} ($i$ 是气体分子)
\begin{equation}
\gamma = \frac{i+2}{i}
\end{equation}
绝热指数也可以用等体热容(\autoref{ThCapa_eq1}~\upref{ThCapa})和等压热容(\autoref{ThCapa_eq2}~\upref{ThCapa})表示为
\begin{equation}
\gamma = \frac{C_P}{C_V}
\end{equation}

\subsection{推导}
考虑一个极短的过程, \autoref{Adiab_eq1} 变为微分形式
\begin{equation}\label{Adiab_eq4}
\dd{W} + \dd{E} = 0
\end{equation}
其中(\autoref{PVgraf_eq1}~\upref{PVgraf})
\begin{equation}\label{Adiab_eq2}
\dd{W} = P\dd{V}
\end{equation}
将理想气体状态方程(\autoref{PVnRT_eq1}~\upref{PVnRT})两边微分得
\begin{equation}
\dd{P}V + P\dd{V} = nRdT
\end{equation}
将气体的内能公式(\autoref{IdgEng_eq1}~\upref{IdgEng})两边微分得
\begin{equation}\label{Adiab_eq3}
\dd{E} = \frac{i}{2}n R\dd{T} = \frac{i}{2} (V\dd{P} + P\dd{V})
\end{equation}


$i$ 是气体分子自由度。 把\autoref{Adiab_eq2} 和\autoref{Adiab_eq3} 代入\autoref{Adiab_eq4} 得 $P$ 和 $V$ 之间得微分方程
\begin{equation}
\gamma P\dd{V} + V \dd{P} = 0
\end{equation}
其中 $\gamma$ 为\textbf{绝热指数}
\begin{equation}
\gamma = \frac{i+2}{i}
\end{equation}
绝热

\begin{equation}
P V^\gamma = C
\end{equation}

\begin{example}{声速}
推导声波在空气中传播的速度。

空气的热导率很小,膨胀和收缩的过程可以认为是绝热过程。可以推出声波的传播速度为(注意绝热过程就是等熵过程)
\begin{equation}
a=\sqrt{\left(\frac{\partial p}{\partial \rho}\right)_S}
\end{equation}
将空气近似地看成理想气体,再代入相关数据,就可以得到和实验测量结果非常接近的速度值。
\addTODO{具体计算过程}
\end{example}
\begin{example}{绝热大气模型}
假设大气是理想气体,其热导率很小,所以大气的对流过程可以近似考虑成绝热过程(实验表明随着高度的增加大气温度下降,这说明不宜用等温大气模型),即
\begin{equation}
\begin{aligned}
&\begin{cases}
&PV_m^\gamma=C\\
&V_m=\frac{RT}{P},P=\frac{\rho R T}{\mu}
\end{cases}
\\
&\Rightarrow \rho^{1-\gamma}T=C'\\
&\Rightarrow (1-\gamma)T\dd \rho+\rho\dd T=0  
\end{aligned}
\end{equation}

那么\autoref{atmDen_eq5}~\upref{atmDen}变为
\begin{equation}
\frac{\gamma}{\gamma-1}T'(z)=-\frac{\mu g}{R}
\end{equation}
大气温度 $T$ 总是 $>0$ 的, 且大气的绝热指数 $\gamma>1$,上式右侧 $<0$,左侧 $<0$,这意味着温度随高度的增加而降低。 再根据气体热容量公式\autoref{ThCapa_eq4}~\upref{ThCapa}约去 $\gamma$,积分得:
\begin{equation}\label{Adiab_eq6}
T=T_0-\int_{z_0}\frac{\mu g}{c_{p,m}} \dd z 
\end{equation}

大气的摩尔质量为 $29 \rm{g\cdot mol^{-1}}$,摩尔定压热容约为 $29 \rm{J\cdot mol^{-1}K^{-1}}$,因此计算得
\begin{equation}
T\approx T_0-z\cdot 10 \rm{K/km}
\end{equation}
即每升高一千米,温度降低约 $10$ 摄氏度。该数值称为干绝热递减率。
\end{example}
