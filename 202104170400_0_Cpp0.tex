% C++ 基础
% keys c++|cpp|语法

Matlab 和 Python 等动态语言虽然用起来方便, 但缺点是运行较慢, 对于一些计算量大的项目不适合. 目前在高性能计算中广泛使用的只有两种语言即 C++ 和 Fortran. 虽然 Fortran 普遍被认为是一个过时的语言, 但在计算物理中, 许多人仍然在使用, 一是因为以前遗留下的 Fortran 代码比较多, 二是一些年纪较大的学者只会 Fortran.

一本在数值算法中很有名的书是 Numerical Recipes, 这本书第三版以前都使用 Fortran 或 C, 而第三版却只有 C++, 这也是本书选择介绍 C++ 而不是 Fortran 的原因之一. 本书将从 Numerical Recipes 中借鉴许多代码上的风格和算法.

C++ 的特征实在多不胜数, 事实上无论是什么语言, 做计算物理的研究者大多会倾向于只选择一些最简单的语法来使用.
我们在这里列出本书使用的 C++ 特性.

\subsection{基础语法}
\begin{itemize}
\item 基本类型 (\verb|bool|, \verb|char|, \verb|int|, \verb|long|, \verb|long| \verb|long|, \verb|float|, \verb|double|, \verb|long double|)
\item 基本算符 (\verb|=, +, -, *, /, %, ++, --, +=, -=, *=, /=, ?:|) 以及优先级
\item scope (scope 内的名字在 scope 外没有定义, scope 内可以定义与 scope 外相同的名字并覆盖, 类的 destructor 会在 scope 结束时自动调用)
\item 判断(\verb|if|, \verb|else if|, \verb|else|)
\item 循环 (\verb|for|, \verb|while|, \verb|do while|)
\item 函数(函数名重载, 变量默认值, 算符函数, inline 函数)
\item \verb|const|
\item 指针
\item 引用
\item 头文件机制 (相当于原地插入头文件中的代码)
\item 多文件编译
\item one definition rule (ODR)
\item 栈(stack)和堆(heap)
\item \verb|typedef|
\item 数组
\item 动态内存管理 \verb|new|, \verb|delete|.
\item 异常处理 \verb|throw|, \verb|try|, \verb|catch|
\end{itemize}

\subsection{标准库}
\begin{itemize}
\item \verb|cmath|
\item \verb|complex|
\item \verb|vector|
\item \verb|string|, \verb|string32|
\item \verb|iostream| (\verb|cin|, \verb|cout|, \verb|<<| 算符, \verb|>>| 算符)
\item \verb|fstream|
\end{itemize}

\subsection{较高级的语法}
\begin{itemize}
\item \verb|constexpr|
\item \verb|namespace|
\item 函数模板
\item 类(constructor, destructor, 定义 operator)
\item 类的继承
\item 类模板
\item 宏 (\verb|#include|, \verb|#define|, \verb|#if|, \verb|#ifdef|, \verb|#ifndef|, \verb|#else|, \verb|#endif|)
\end{itemize}
