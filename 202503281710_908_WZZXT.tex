% 微正则系统(综述)
% license CCBYSA3
% type Wiki

本文根据 CC-BY-SA 协议转载翻译自维基百科\href{https://en.wikipedia.org/wiki/Microcanonical_ensemble}{相关文章}。

在统计力学中,微正则系综是一个统计系综,表示总能量被精确指定的机械系统的可能状态。\(^\text{[1]}\)假设系统是孤立的,即它不能与环境交换能量或粒子,因此(根据能量守恒)系统的能量随着时间不发生变化。

微正则系综的主要宏观变量是系统中的总粒子数(符号:\( N \))、系统的体积(符号:\( V \))以及系统中的总能量(符号:\( E \))。这些变量在系综中被假设为常数。因此,微正则系综有时被称为\(NVE\)系综。

简单来说,微正则系综通过为每个能量落在以\( E \)为中心的范围内的微观状态分配相等的概率来定义。所有其他微观状态的概率为零。由于概率总和必须为 1,因此概率 \( P \)是能量范围内微观状态数\( W \)的倒数,
\[
P = 1/W,~
\]
然后能量范围的宽度被逐渐缩小,直到它变得无限窄,仍然以\( E \)为中心。在这个过程中,当宽度趋于零时,得到微正则系综。\(^\text{[1]}\)
\subsection{适用性}  
由于与平衡统计力学的基本假设(特别是先验相等概率的假设)相关,微正则系综是该理论中的一个重要概念性基石。\(^\text{[2]}\)它有时被认为是平衡统计力学的基本分布。它在一些数值应用中也很有用,例如分子动力学。\(^\text{[3][4]}\)另一方面,大多数非平凡系统在微正则系综中的数学描述繁琐,而且在熵和温度的定义上也存在一些模糊性。因此,在理论计算中,通常更倾向于使用其他系综。\(^\text{[2][5][6]}\)

微正则系综对现实世界系统的适用性取决于能量波动的重要性,这些波动可能源于系统与环境之间的相互作用以及在准备系统时的不可控因素。通常,如果系统在宏观上很大,或者系统的能量已被精确地确定并且之后几乎与环境隔离,那么波动可以忽略不计。\(^\text{[7]}\)在这种情况下,微正则系综是适用的。否则,其他系综更为合适——例如正则系综(能量波动)或巨正则系综(能量和粒子数波动)。
\subsection{性质}  
\subsubsection{热力学量}  
微正则系综的基本热力学势是熵。熵有至少三种可能的定义,每种定义都通过相空间体积函数 \( v(E) \) 表示。在经典力学中,\( v(E) \) 是相空间中能量小于 \( E \) 的区域的体积。在量子力学中,\( v(E) \) 大致是能量小于 \( E \) 的能量本征态的数量;然而,这必须进行平滑处理,以便我们能够求其导数(有关如何处理的详细信息,请参见“精确表达式”部分)。微正则熵的定义如下:
\begin{itemize}
\item 玻尔兹曼熵 \( S_B \):
\[
S_B = k \log W = k \log \left( \omega \frac{dv}{dE} \right)~
\]
玻尔兹曼熵依赖于所谓的“能量宽度” \( \omega \),这是一个具有能量单位的任意量,通常取为小值,引入它是为了使我们对一个无量纲量取对数,因为 \( \frac{dv}{dE} \) 的单位是 1/能量。
\item ‘体积熵’:
\[
S_v = k \log v~
\]
\item ‘表面熵’:
\[
S_s = k \log \frac{dv}{dE} = S_B - k \log \omega~
\]
\end{itemize}
在表面熵中,我们对具有逆能量单位的量取对数,因此改变能量单位会通过加性常数改变该量。玻尔兹曼熵可以看作是表面熵的一种变体,避免了这个问题。

在微正则系综中,温度是一个派生量,而不是外部控制参数。它被定义为所选熵关于能量的导数。[8] 例如,可以定义“温度” \( T_v \)和\( T_s \)如下:
\[
1/T_v= dS_v/dE,~
\]
\[
1/T_s= dS_s/dE=dS_B/dE.~
\]
像熵一样,在微正则系综中有多种方式理解温度。更一般地说,这些基于系综的定义与它们的热力学对应物之间的关系并不完美,尤其对于有限系统。

微正则系综中的压强和化学势由以下公式给出:\(^\text{[9]}\)
\[
\frac{p}{T} = \frac{\partial S}{\partial V}; \qquad \frac{\mu}{T} = -\frac{\partial S}{\partial N}.~
\]
\subsubsection{相变}  
根据严格的定义,相变对应于热力学势或其导数中的非解析行为。\(^\text{[10]}\) 使用这个定义,微正则系综中的相变可以发生在任何大小的系统中。这与正则系综和巨正则系综形成对比,后者的相变只能在热力学极限中发生——即,在具有无限多自由度的系统中。\(^\text{[10][11]}\)粗略来说,定义正则系综或巨正则系综的热库引入了波动,这些波动“平滑”了有限系统中自由能的任何非解析行为。对于宏观系统来说,这种平滑效应通常可以忽略不计,因为这些系统足够大,能够非常好地近似自由能的非解析行为。然而,在小系统的理论分析中,系综的技术性差异可能是重要的。\(^\text{[11]}\)
\subsubsection{信息熵}  
对于给定的机械系统(固定\( N \)、\( V \))和给定的能量范围,微观状态上概率 \( P \)的均匀分布(如在微正则系综中)最大化了系综平均值\( -\langle \log P \rangle \)。\(^\text{[1]}\)
\subsection{热力学类比}  
路德维希·玻尔兹曼在统计力学方面的早期工作导致了他为具有给定总能量的系统提出的同名熵方程\( S = k \log W \),其中 \( W \)是系统在该能量下可达的不同状态的数量。玻尔兹曼并未深入阐述到底什么构成了一个系统的不同状态集合,除了理想气体的特殊情况。这个话题由乔赛亚·威拉德·吉布斯进行了彻底的研究,他为任意机械系统发展了广义统计力学,并定义了本文描述的微正则系综。\(^\text{[1]}\)吉布斯仔细研究了微正则系综与热力学之间的类比,特别是在自由度较少的系统中它们是如何失效的。他引入了两个不依赖于\( \omega \)的微正则熵的进一步定义——上述的体积熵和表面熵。(请注意,表面熵与玻尔兹曼熵的不同之处仅在于\( \omega \)-依赖的偏移量。)

体积熵\( S_v \)和相关温度\( T_v \)与热力学熵和温度密切类比。可以准确地证明
\[
dE = T_v dS_v - \langle P \rangle dV,~
\]
(\( \langle P \rangle \)是系综平均压强),这与热力学第一定律一致。对于表面熵\( S_s \)(或玻尔兹曼熵\( S_B \))及其相关温度\( T_s \),也可以得到类似的方程,然而,这个方程中的“压强”是一个复杂的量,与平均压强无关。\(^\text{[1]}\)
微正则温度\( T_v \)和\( T_s \)在与使用正则系综定义的温度类比时,并不完全令人满意。在热力学极限之外,出现了一些伪影。
\begin{itemize}
\item 组合两个系统的非平凡结果:两个系统,每个系统都由独立的微正则系综描述,可以被带入热接触并允许它们在一个合成系统中达到平衡,该合成系统也由微正则系综描述。不幸的是,两个系统之间的能量流动不能仅根据初始的\( T_s \)预测。即使初始的\( T_s \)相等,仍然可能发生能量传递。此外,合成系统的温度与初始值不同。这与温度应该是一个强度量的直觉相矛盾,即两个相同温度的系统在热接触时不应受到影响。\(^\text{[1]}\)
\item 少粒子系统的奇怪行为:许多结果,如微正则系综的能量均分定理,在用\( T_s \)表示时会产生一个一度或二度自由度的偏移。对于小系统,这种偏移是显著的,因此如果我们将\( S_s \)作为熵的类比,则对于只有一个或两个自由度的系统需要做出几个例外处理。\(^\text{[1]}\)
\item 伪负温度:每当状态密度随能量递减时,就会出现负的\( T_s \)。在某些系统中,状态密度不是能量的单调函数,因此随着能量的增加,\( T_s \)可以多次改变符号。\(^\text{[12][13]}\)
\end{itemize}
解决这些问题的首选方法是避免使用微正则系综。在许多实际情况下,系统与热库保持热平衡,因此能量并不精确已知。在这种情况下,更准确的描述是正则系综或巨正则系综,这两者与热力学有完全的对应关系。\(^\text{[14]}\)
\subsection{系综的精确表达式}  
统计系综的精确数学表达式取决于所考虑的力学类型——量子力学或经典力学——因为在这两种情况下,“微观状态”的概念有很大的不同。在量子力学中,对角化提供了一组具有特定能量的离散微观状态。经典力学则涉及在正则相空间上进行积分,相空间中微观状态的大小可以根据需要进行某种程度的任意选择。

为了构建微正则系综,在两种类型的力学中,首先都需要指定一个能量范围。在下面的表达式中,函数\( f \left( \frac{H - E}{\omega} \right) \)(\(H\)的函数,在能量\( E \)处达到峰值,宽度为\( \omega \))将用于表示包含状态的能量范围。这个函数的一个例子是\(^\text{[1]}\)
\[
f(x) = 
\begin{cases} 
1, & \text{if} \ |x| < \frac{1}{2}, \\
0, & \text{otherwise.} 
\end{cases}~
\]
或者,更平滑地,
\[
f(x) = e^{-\pi x^2}.~
\]
\subsubsection{量子力学的}
量子力学中的统计系综由密度矩阵表示,记作\( \hat{\rho} \)。微正则系综可以使用布拉-凯特表示法写出,涉及系统的能量本征态和能量本征值。给定一个完整的能量本征态基\( |\psi_i\rangle \),由\( i \)索引,微正则系综为:
\[
\hat{\rho} = \frac{1}{W} \sum_i f \left( \frac{H_i - E}{\omega} \right) |\psi_i \rangle \langle \psi_i |,~
\]
其中\( H_i \) 是由 \( \hat{H} |\psi_i\rangle = H_i |\psi_i\rangle \) 确定的能量本征值(此处 \( \hat{H} \) 是系统的总能量算符,即哈密顿算符)。\( W \) 的值由要求 \( \hat{\rho} \) 是一个归一化的密度矩阵来确定,因此
\[
W = \sum_i f \left( \frac{H_i - E}{\omega} \right).~
\]
状态体积函数(用于计算熵)由以下公式给出:
\[
v(E) = \sum_{H_i < E} 1.~
\]
微正则系综通过将密度矩阵的能量宽度趋于零的极限来定义,然而,一旦能量宽度变小到低于能级间隔时,会出现问题。对于非常小的能量宽度,在大多数\( E \)值下,系综根本不存在,因为没有状态落在这个范围内。当系综存在时,它通常只包含一个(或两个)状态,因为在复杂系统中,能级通常只有偶然相等(有关这一点的更多讨论,请参见随机矩阵理论)。此外,状态体积函数也只能在离散的增量中增加,因此它的导数永远只有无限大或零,这使得定义状态密度变得困难。这个问题可以通过不将能量范围完全缩小到零并平滑状态体积函数来解决,然而,这使得系综的定义更加复杂,因为这时除了其他变量外,还必须指定能量范围(共同形成一个\(NVE\omega\)系综)。
\subsubsection{经典力学的}
在经典力学中,一个集合由一个联合概率密度函数\(\rho(p_1, ... p_n, q_1, ... q_n)\)表示,该函数定义在系统的相空间上。相空间具有\(n\)个广义坐标,称为\(q_1, ... q_n\),以及\(n\)个相关的经典动量,称为\(p_1, ... p_n\)。

微正则集合的概率密度函数为:
\[
\rho = \frac{1}{h^n C} \frac{1}{W} f\left(\frac{H - E}{\omega}\right)~
\]

其中:

\begin{itemize}
\item \(H\)是系统的总能量(哈密顿量),它是相空间坐标(\(p_1, ... p_n\))的函数;  
\item \(h\)是一个任意但预定的常数,单位为能量×时间,用于设定一个微观状态的范围,并为\(\rho\)提供正确的维度。[注1]  
\item \(C\)是一个过度计数修正因子,通常用于粒子系统,其中相同的粒子能够互换位置。[注2]  
\end{itemize}
\(W\)的值通过要求\(\rho\)是一个归一化的概率密度函数来确定:
\[
W = \int \ldots \int \frac{1}{h^n C} f\left(\frac{H - E}{\omega}\right) \, dp_1 \ldots dq_n~
\]
这个积分是在整个相空间上进行的。状态体积函数(用于计算熵)定义为:
\[
v(E) = \int \ldots \int_{H < E} \frac{1}{h^n C} \, dp_1 \ldots dq_n~
\]
当能量宽度ω趋近于零时,\(W\)的值与ω成比例减小,表现为\( W = \omega \left(dv/dE\right)\)。

根据上述定义,微正则集合可以被视为相空间中一个无限薄的壳,中心位于一个恒定能量的表面上。尽管微正则集合被限制在这个表面上,但它不一定在该表面上均匀分布:如果相空间中的能量梯度变化,那么微正则集合在该表面的一些部分会“更厚”(更集中)而在其他部分则较薄。这个特性是要求微正则集合为稳态集合的不可避免的结果。
\subsection{例子}
\subsubsection{理想气体}
“微正则系综中的基本量是\( W(E, V, N) \),它等于与给定的\( (E, V, N) \)兼容的相空间体积。通过\( W \),可以计算出所有的热力学量。对于理想气体,能量与粒子的位置无关,因此粒子的位置贡献了一个\( V^N \)的因子到\( W \)。相反,动量受到限制,位于一个\( 3N \)-维的(超)球壳上,半径为 \( \sqrt{2mE} \);它们的贡献等于这个球壳的表面积。最终得到的\( W \)的表达式为\(^\text{[15]}\):
\[
W = \frac{V^N}{N!} \cdot \frac{2\pi^{3N/2}}{\Gamma(3N/2)} \left( 2mE \right)^{(3N-1)/2}~
\]
其中,\( \Gamma(.) \)是伽马函数,因子\( N! \)被包括在内,以考虑粒子不可区分性(参见吉布斯悖论)。在大\( N \)极限下,玻尔兹曼熵\( S = k_B \log W \)为:
\[
S = k_B N \log \left[ \frac{V}{N} \left( \frac{4 \pi m}{3} \frac{E}{N} \right)^{3/2} \right] + \frac{5}{2} k_B N + O(\log N)~
\]
这也被称为萨克尔-特特罗德方程。

温度由以下公式给出:
\[
\frac{1}{T} \equiv \frac{\partial S}{\partial E} = \frac{3}{2} \frac{N k_B}{E}~
\]
这与气体动力学理论中的类似结果一致。计算压力得到理想气体状态方程:
\[
\frac{p}{T} \equiv \frac{\partial S}{\partial V} = \frac{N k_B}{V} \quad \rightarrow \quad pV = N k_B T~
\]
最后,化学势\(\mu\)为:
\[
\mu \equiv -T \frac{\partial S}{\partial N} = k_B T \log \left[ \frac{V}{N} \left( \frac{4\pi m E}{3N} \right)^{3/2} \right]~
\]
\subsubsection{均匀引力场中的理想气体}  
理想气体在均匀引力场中的微正则相体积也可以显式地计算。\(^\text{[16]}\)

以下是对一个三维理想气体的结果,其中包含\(N\)个粒子,每个粒子的质量为\( m \),这些粒子被限制在一个热隔离的容器中,容器在\( z \)-方向上是无限长的,且具有恒定的横截面积\(A\)。假设引力场在负\(z\)-方向上作用,强度为\( g \)。相体积\(W(E, N)\) 为:
\[
W(E, N) = \frac{(2\pi)^{3N/2} A^N m^{N/2}}{g^N \Gamma(5N/2)} E^{(\frac{5N}{2}) - 1}~
\]
其中\(E\)是总能量,包括动能和引力能。

气体密度\(\rho(z)\)作为高度\(z\)的函数可以通过对相体积坐标进行积分得到。结果为:
\[
\rho(z) = \left( \frac{5N}{2} - 1 \right) \frac{mg}{E} \left( 1 - \frac{mgz}{E} \right)^{\frac{5N}{2} - 2}~
\]
类似地,速度大小\( |\vec{v}| \)的分布(对所有高度的平均值)为:
\[
f(|\vec{v}|) = \frac{\Gamma(5N/2)}{\Gamma(3/2)\Gamma(5N/2 - 3/2)} \times \frac{m^{3/2} |\vec{v}|^2}{2^{1/2} E^{3/2}} \times \left( 1 - \frac{m|\vec{v}|^2}{2E} \right)^{\frac{5(N-1)}{2}}~
\]
这些方程在经典系综中的类似公式分别是气压公式和麦克斯韦–玻尔兹曼分布。在极限 \( N \to \infty \)时,微正则系综和经典系综的表达式一致;然而,在有限\( N \)时它们不同。特别地,在微正则系综中,位置和速度不是统计独立的。因此,定义为给定体积\( A \, dz \)中的平均动能的动能温度是容器内非均匀的:
\[
T_{\mathrm{kinetic}} = \frac{3E}{5N-2} \left( 1 - \frac{mgz}{E} \right)~
\]
相比之下,在经典系综中,温度对于任何\(N\)都是均匀的。\(^\text{[17]}\)
\subsection{另见}  
\begin{itemize}
\item 孤立系统  
\item 遍历假设  
\item 洛施密特悖论  
\item 正则系综  
\item 大正则系综
\end{itemize}
\subsection{注释}\\  
a.(历史性说明)吉布斯的原始系综设定中实际令\(h = 1\)[能量单位]×[时间单位],这导致熵和化学势等部分热力学量的数值依赖于所选单位。自量子力学诞生后,为建立与量子力学的半经典对应关系,\(h\)常被设定为普朗克常数。\\  
b.在由N个全同粒子构成的系统中,\(C = N!\)(\(N\)的阶乘)。该因子用于修正相空间中的重复计数问题——由于全同物理态可能出现在相空间的多个不同位置。关于这种重复计数的更多说明,请参阅统计系综条目。\\  

