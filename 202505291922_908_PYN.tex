% 皮亚诺公理(综述)
% license CCBYSA3
% type Wiki

本文根据 CC-BY-SA 协议转载翻译自维基百科 \href{https://en.wikipedia.org/wiki/Peano_axioms}{相关文章}。

在数学逻辑中,皮亚诺公理(Peano axioms,/piˈɑːnoʊ/,[peˈaːno]),也称为德德金–皮亚诺公理或皮亚诺假设,是一组用于描述自然数的公理体系,由19世纪意大利数学家朱塞佩·皮亚诺提出。这些公理在多项元数学研究中几乎未加改动地被广泛采用,其中包括关于数论是否一致与完备等基本问题的研究。

由皮亚诺公理所提供的算术公理化体系,通常被称为皮亚诺算术。

将算术进行形式化的重要性,在赫尔曼·格拉斯曼的工作之前并未被广泛重视。格拉斯曼在19世纪60年代指出,算术中的许多事实可以从关于后继运算和数学归纳的更基本事实中推导出来。1881年,查尔斯·桑德斯·皮尔士提出了一种自然数算术的公理化体系。1888年,理查德·德德金又提出了另一种自然数的公理体系,而皮亚诺则在1889年出版的著作《以新方法阐述的算术原理》(拉丁文:Arithmetices principia, nova methodo exposita)中,将其加以简化并作为一组公理发表。

皮亚诺公理共包括三类陈述:1. 第一条公理断言自然数集合中至少存在一个元素;2. 接下来的四条是关于等号的一般性陈述,在现代的处理方式中,这些往往被看作是“基础逻辑”的一部分,而非皮亚诺公理本身;3. 再接下来的三条公理是关于自然数及其后继运算的基本性质的一阶逻辑陈述;4. 第九条、也是最后一条公理则是一个二阶逻辑陈述,它表达了对自然数进行数学归纳法的原理,正是这一点使得原始皮亚诺公理体系接近二阶算术。如果显式引入加法和乘法两个运算符号,并将第九条二阶归纳公理替换为一阶公理模式,就可以得到一个较弱的一阶系统。术语“皮亚诺算术”有时特指这一限制后的系统。
\subsection{历史上的二阶表述}
当皮亚诺提出他的公理时,数理逻辑的语言还处于起步阶段。他为了表达这些公理而创造的逻辑符号体系并未广泛流行,尽管它成为现代集合成员关系符号(∈,源自皮亚诺的 ε)的起点。皮亚诺明确区分数学符号和逻辑符号,这在当时的数学中还不常见;这种区分最早由哥特洛布·弗雷格在其 1879 年出版的《概念文字》中引入。\(^\text{[7]}\)然而,皮亚诺并不知道弗雷格的工作,而是独立地基于布尔和施罗德的研究重建了自己的逻辑体系。\(^\text{[8]}\)

皮亚诺公理定义了自然数的算术性质,通常表示为集合 $N$ 或 $\mathbb{N}$。这些公理中的非逻辑符号包括一个常数符号 0 和一个一元函数符号 S(表示“后继”)。

第一条公理声明常数 0 是一个自然数:

1.0 是一个自然数。

皮亚诺最初在其公理表述中使用 1 而非 0 作为“第一个”自然数,\(^\text{[9]}\)而他在《数学公式集》中的公理则包括了零。\(^\text{[10]}\)

接下来的四条公理描述了等同性关系。由于这些内容在带有等号的一阶逻辑中是逻辑有效的,因此在现代的数学处理中,它们通常不被视为“皮亚诺公理”的一部分。\(^\text{[8]}\)\\\\
1.对于每一个自然数 $x$,有 $x = x$。即:等同性是自反的。\\
2.对于所有自然数 $x$ 和 $y$,如果 $x = y$,那么 $y = x$。即:等同性是对称的。\\
3.对于所有自然数 $x$、$y$ 和 $z$,如果 $x = y$ 且 $y = z$,那么 $x = z$。即:等同性是传递的。\\
4.对于所有 $a$ 和 $b$,如果 $b$ 是自然数并且 $a = b$,那么 $a$ 也是自然数。即:自然数在等同性下是封闭的。\\

其余的公理定义了自然数的算术性质。假设自然数在一个单值“后继”函数 $S$ 下是封闭的:\\\\
5.对于每一个自然数 $n$,$S(n)$ 也是自然数。即:自然数在后继函数 $S$ 下是封闭的。\\
6.对于所有自然数 $m$ 和 $n$,如果 $S(m) = S(n)$,那么 $m = n$。即:$S$ 是一个单射(注入函数)。\\
7.对于每一个自然数 $n$,命题 $S(n) = 0$ 是假的。即:不存在一个自然数,其后继是 0。
\begin{figure}[ht]
\centering
\includegraphics[width=10cm]{./figures/d58cef55c1c7abc7.png}
\caption{右图中从最近的一块开始的一连串浅色多米诺骨牌,可以用来表示自然数集合 $\mathbb{N}$。\(^\text{[1][11][12]}\)然而,公理 1–8 同样也被“所有多米诺骨牌的集合”(无论是浅色还是深色)所满足。\(^\text{[2]}\)第九条公理(归纳公理)将 $\mathbb{N}$ 限定为那一串浅色骨牌(即“无赘物”原则),因为只有浅色骨牌会在最前一块被推倒时依次倒下。\(^\text{[13]}\)指用继承关系(S)一块一块地排列这些骨牌。指如果不加第九公理,也可能包含不属于自然数的“赘余元素”。} \label{fig_PYN_1}
\end{figure}
公理 1、6、7 和 8 定义了自然数直观概念的一种一元表示方式:数字 1 可以被定义为 $S(0)$,2 为 $S(S(0))$,以此类推。然而,如果将自然数的概念完全建立在这些公理之上,则公理 1、6、7、8 并不蕴含后继函数能够生成所有非零的自然数。

我们直观上认为每一个自然数都可以通过对 0 反复应用后继函数而得到,这一想法需要额外添加一条公理来保证,这条额外的公理通常被称为归纳公理。

9.如果 $K$ 是一个集合,满足以下条件:

\begin{itemize}
\item $0 \in K$,并且
\item 对于每一个自然数 $n$,若 $n \in K$,则 $S(n) \in K$,
\end{itemize}

那么 $K$ 包含所有的自然数。

归纳公理有时也以如下形式表述:

9.如果 $\varphi$ 是一个一元谓词,满足:

\begin{itemize}
\item $\varphi(0)$ 为真,且
\item 对于每一个自然数 $n$,若 $\varphi(n)$ 为真,则 $\varphi(S(n))$ 也为真,
\end{itemize}
那么对于每一个自然数 $n$,$\varphi(n)$ 都为真。

在皮亚诺的最初表述中,归纳公理是一个二阶公理。而现在通常用一个较弱的一阶归纳公理模式来替代这个二阶原理。二阶与一阶的表述之间存在重要区别,这些区别将在下文《皮亚诺算术作为一阶理论》部分中进一步讨论。
\subsubsection{定义算术运算与关系}
如果使用二阶归纳公理,那么可以直接利用皮亚诺公理在自然数集合 $\mathbb{N}$ 上定义加法、乘法以及全序(线性序)关系。然而,在使用一阶归纳的情况下,这种定义是不可行的,因此加法和乘法通常被作为额外的公理加入。这些运算和关系的对应函数通常在集合论或二阶逻辑中构造,并且可以借助皮亚诺公理证明它们的唯一性。

\textbf{加法}

加法是一个函数,它将两个自然数(即 $\mathbb{N}$ 中的两个元素)映射为另一个自然数。它的定义是递归式的:
$$
\begin{aligned}
a + 0 &= a, \quad \text{(1)} \\
a + S(b) &= S(a + b). \quad \text{(2)}
\end{aligned}~
$$
\begin{figure}[ht]
\centering
\includegraphics[width=14.25cm]{./figures/efd1402d1fe12e3c.png}
\caption{} \label{fig_PYN_3}
\end{figure}
为了证明加法的交换律,首先通过对 $b$ 的数学归纳法证明:$0 + b = b \quad \text{和} \quad S(a) + b = S(a + b)$接着使用这两个结果,再通过对 $b$ 的归纳法证明:$a + b = b + a$因此,结构 $(\mathbb{N}, +)$ 构成一个以 0 为单位元的交换幺半群(commutative monoid)。$(\mathbb{N}, +)$ 同时也是一个可消去 magma,因此它可以嵌入一个群中。嵌入 $\mathbb{N}$ 的最小群是整数集 $\mathbb{Z}$。

\textbf{乘法}

类似地,乘法是一个将两个自然数映射为另一个自然数的函数。在加法已定义的前提下,乘法被递归地定义为:

$a \cdot 0 = 0$

$a \cdot S(b) = a + (a \cdot b)$

显然,$S(0)$(即 1)是乘法的右单位元:
$$
a \cdot S(0) = a + (a \cdot 0) = a + 0 = a~
$$
要证明 $S(0)$ 也是**乘法的左单位元**,则需要使用数学归纳公理,因为乘法是以这种方式定义的:
\begin{itemize}
\item $S(0) \cdot 0 = 0$,即 $S(0)$ 是 0 的左单位元;
\item 如果 $S(0) \cdot a = a$,那么我们要证明 $S(0)$ 也是 $S(a)$ 的左单位元:
$$
S(0) \cdot S(a) = S(0) + S(0) \cdot a = S(0) + a = a + S(0) = S(a + 0) = S(a)~
$$
(上述等式使用了加法的交换律。)
\end{itemize}
因此,根据归纳公理,$S(0)$ 是所有自然数的乘法左单位元。此外,可以进一步证明,乘法满足交换律和对加法的分配律:
$$
a \cdot (b + c) = (a \cdot b) + (a \cdot c)~
$$
因此,$(\mathbb{N}, +, 0, \cdot, S(0))$ 构成一个交换半环。

\textbf{不等式}

在自然数集合中,通常的全序关系 ≤ 可以如下定义(假设 0 是自然数):

对所有 $a, b \in \mathbb{N}$,当且仅当存在某个 $c \in \mathbb{N}$ 使得 $a + c = b$,我们称 $a \leq b$。

该关系在加法和乘法下是稳定的:对所有 $a, b, c \in \mathbb{N}$,如果 $a \leq b$,则有:
\begin{itemize}
\item $a + c \leq b + c$
\item $a \cdot c \leq b \cdot c$
\end{itemize}
因此,结构 $(\mathbb{N}, +, \cdot, 1, 0, \leq)$ 构成一个有序半环;并且由于在 0 和 1 之间不存在其他自然数,它是一个离散有序半环。

归纳公理有时也可以用如下形式表述,这种形式使用了更强的假设,并借助了“≤”这个序关系:

对于任意谓词 $\varphi$,如果满足:
\begin{itemize}
\item $\varphi(0)$ 为真,并且
\item 对于每一个 $n \in \mathbb{N}$,如果对所有满足 $k \leq n$ 的 $k \in \mathbb{N}$ 都有 $\varphi(k)$ 为真,那么 $\varphi(S(n))$ 也为真,
\item 那么对于每个 $n \in \mathbb{N}$,都有 $\varphi(n)$ 为真。
\end{itemize}
这种归纳公理的形式被称为强归纳法,它是标准归纳形式的一个推论,但在涉及“≤”这个序关系的推理中通常更为合适。例如,若要证明自然数集合是良序的(即:自然数的任意非空子集都有最小元素),可以如下推理:

设 $X \subseteq \mathbb{N}$ 是一个非空集合,且假设 $X$ 没有最小元素。
\begin{itemize}
\item 由于 0 是自然数中最小的元素,因此必须有 $0 \notin X$。
\item 对于任意 $n \in \mathbb{N}$,假设对于所有满足 $k \leq n$ 的 $k$,都有 $k \notin X$。那么也可以推出 $S(n) \notin X$,否则它将成为 $X$ 的最小元素。
\end{itemize}
因此,根据强归纳原理,对所有 $n \in \mathbb{N}$ 都有 $n \notin X$,也即 $X \cap \mathbb{N} = \varnothing$,这与 $X$ 是自然数的非空子集矛盾。所以,$X$ 必然有最小元素。
\subsubsection{模型}
皮亚诺公理的一个模型是一个三元组 $(N, 0, S)$,其中:$N$ 是一个(必然是无限的)集合,$0 \in N$,$S: N \to N$ 是一个函数,满足上述公理。德德金(Dedekind)在他1888年的著作《数的本质与意义》(德文原名 Was sind und was sollen die Zahlen?,意即“数是什么,它们有何用?”)中证明:任何两个满足皮亚诺公理(包括二阶归纳公理)的模型都是同构的。具体来说,设有两个皮亚诺模型 $(N_A, 0_A, S_A)$ 和 $(N_B, 0_B, S_B)$,则存在一个唯一的同态映射 $f: N_A \to N_B$,满足:
$$
\begin{aligned}
f(0_A) &= 0_B \\
f(S_A(n)) &= S_B(f(n))
\end{aligned}~
$$
并且这个映射是双射(即一一对应)。这意味着:二阶皮亚诺公理是范畴性的,即它的模型在结构上是唯一的(同构的)。(而下面将提到的任何一阶形式的皮亚诺公理,则不具备这一特性。)

\textbf{集合论模型}

皮亚诺公理可以从集合论对自然数的构造以及集合论公理(如 ZF 公理)中导出。\(^\text{[15]}\)标准的自然数构造方式由约翰·冯·诺依曼提出,其出发点是将 0 定义为空集** $\emptyset$,并引入一个集合上的运算符 $s$,定义如下:
$$
s(a) = a \cup \{a\}~
$$
自然数集 $\mathbf{N}$ 被定义为:所有包含空集且在 $s$ 运算下封闭的集合的交集。在这个定义下,每一个自然数(作为集合)等于所有比它小的自然数的集合:
$$
\begin{aligned}
0 &= \emptyset \\
1 &= s(0) = s(\emptyset) = \emptyset \cup \{\emptyset\} = \{\emptyset\} = \{0\} \\
2 &= s(1) = s(\{0\}) = \{0\} \cup \{\{0\}\} = \{0, \{0\}\} = \{0, 1\} \\
3 &= s(2) = s(\{0,1\}) = \{0,1\} \cup \{\{0,1\}\} = \{0,1,\{0,1\}\} = \{0,1,2\}
\end{aligned}~
$$
以此类推。由集合 $\mathbf{N}$、元素 $0$、以及后继函数 $s: \mathbf{N} \to \mathbf{N}$ 构成的结构满足皮亚诺公理。

皮亚诺算术与几个较弱的集合论系统是相同一致性强度的。\(^\text{[16]}\)其中一种系统是在 ZFC 公理体系中将“无穷公理”替换为其否定。另一种系统由“通用集合论”(包括外延性、公设空集的存在、以及附加公理)构成,并增加了一个公理模式:如果某性质对空集成立,且对所有“附加”结构都成立,那么它就对所有集合成立。

\textbf{范畴论中的解释}

皮亚诺公理也可以通过范畴论来理解。设 $\mathcal{C}$ 是一个具有终对象 $1_\mathcal{C}$ 的范畴,定义带基点的一元系统范畴$\mathsf{US}_1(\mathcal{C})$ 如下:
\begin{itemize}
\item $\mathsf{US}_1(\mathcal{C})$ 的对象是三元组 $(X, 0_X, S_X)$,其中$X$ 是 $\mathcal{C}$ 中的一个对象,$0_X : 1_\mathcal{C} \to X$ 是从终对象到 $X$ 的态射,$S_X : X \to X$ 是 $\mathcal{C}$ 中的一个自映射(态射)。
\item 一个从 $(X, 0_X, S_X)\to (Y, 0_Y, S_Y)$ 的态射 $\varphi$ 是 $\mathcal{C}$ 中的一个态射 $\varphi : X \to Y$,满足:$\varphi \circ 0_X = 0_Y$和$\varphi \circ S_X = S_Y \circ \varphi$
\end{itemize}
若 $\mathsf{US}_1(\mathcal{C})$ 存在一个初始对象,则称范畴 $\mathcal{C}$ 满足Dedekind–Peano 公理。这个初始对象称为 $\mathcal{C}$ 中的一个自然数对象。如果 $(N, 0, S)$ 是该初始对象,而 $(X, 0_X, S_X)$ 是任意一个其他对象,则存在唯一的态射 $u : (N, 0, S) \to (X, 0_X, S_X)$,使得:
$$
\begin{aligned}
u(0) &= 0_X \\
u(Sx) &= S_X(u(x))
\end{aligned}~
$$
这正是对 $0_X$ 和 $S_X$ 的递归定义。
\subsubsection{一致性}
当皮亚诺公理最初被提出时,伯特兰·罗素等人认为这些公理隐含地定义了“自然数”这一概念。\(^\text{[17]}\)但亨利·庞加莱则更加谨慎,他指出:只有当这些公理是一致的时,它们才能定义自然数;如果能够从这些公理出发推导出诸如 $0 = 1$ 这样的矛盾,那么这些公理就是不一致的,也就什么都无法定义。\(^\text{[18]}\)1900年,大卫·希尔伯特将“使用仅限有限方法证明皮亚诺公理的一致性”列为他著名的23个问题中的第二个问题。\(^\text{[19]}\)1931年,库尔特·哥德尔证明了他著名的第二不完备性定理,该定理指出:如果皮亚诺算术是一致的,那么它自身无法形式化地证明自身的一致性。\(^\text{[20]}\)

虽然人们普遍认为哥德尔定理否定了使用有限方法证明皮亚诺算术一致性的可能性,但这其实依赖于人们对“有限证明”一词具体含义的理解。哥德尔本人曾指出,可以使用一些不能在皮亚诺算术内部形式化的有限方法,对皮亚诺算术或更强系统给出一致性证明。1958年,哥德尔发表了一种基于类型论的一致性证明方法。\(^\text{[21]}\)1936年,格哈德·根岑给出了一种皮亚诺公理系统一致性的证明,使用了一个称为 $\varepsilon_0$ 的序数的超限归纳法。。\(^\text{[22]}\)根岑解释道:“本文的目的是证明初等数论的一致性,或者说,将一致性问题归约为某些基本原理。”根岑的证明常被认为是有限性的,因为 $\varepsilon_0$ 可以用有限对象来编码(例如,作为一个图灵机描述某种整数的序结构,或者更抽象地,表示为某种有限树的线性有序结构)。然而,根岑的证明是否符合希尔伯特设想的“有限性”要求仍不明确:目前并没有一个被广泛接受的、精确定义的“有限证明”的概念,而希尔伯特本人也从未给出这个概念的明确定义。

当代绝大多数数学家都相信皮亚诺公理是一致的,这种相信或基于直觉,或基于对某种一致性证明(如根岑的证明)的接受。

但也有少数哲学家和数学家(其中一些人还主张超有限主义)拒绝接受皮亚诺公理,因为接受这些公理就意味着接受自然数的无限集合。尤其是,加法(包括后继函数)和乘法被假设为全定义的运算(即对所有自然数都适用)。

有趣的是,存在一些自我验证理论,它们与皮亚诺算术类似,但使用减法和除法代替加法和乘法,并通过特定方式加以公理化,从而避免证明那些等价于加法和乘法“全定义性”的命题。尽管如此,这些理论仍能证明皮亚诺算术中所有真实的 $\Pi_1$ 命题,而且还能扩展为一个一致的理论,该理论能证明自身的一致性(即以“0 ≠ 1”不可由希尔伯特风格证明为形式表述的一致性命题)。\(^\text{[23]}\)
\subsection{一阶理论中的皮亚诺算术}
除第九条公理(归纳公理)外,皮亚诺公理的其余部分都是一阶逻辑中的命题。\(^\text{[24]}\)加法、乘法和序关系这些算术运算,也可以通过一阶公理来定义。上面所述的归纳公理是一个二阶公理,因为它是对谓词进行量化的(也可以理解为对自然数的集合进行量化,而不仅仅是对自然数本身)。作为替代,可以采用一种一阶归纳公理模式。该模式包含对每一个可以在皮亚诺算术的一阶语言中定义的谓词所对应的一个归纳公理。因此,这种模式比原始的二阶归纳公理更弱。\(^\text{[25]}\)它之所以更弱,是因为在一阶语言中可定义的谓词数量是可数的,而自然数集合的总数是不可数的。因此,存在许多集合是无法在一阶语言中描述的(实际上,大多数集合都具备这一性质)。

此外,皮亚诺算术在一阶公理化中还有一个技术限制:在二阶逻辑中,可以从“后继函数”出发推导出加法和乘法运算,但在更受限的一阶逻辑中无法做到这一点。因此,在皮亚诺算术的语言符号中,加法和乘法必须被直接包含进来,并且要明确给出一组连接三种运算之间关系的公理。

以下这组公理(再加上通常的等式公理),包含了罗宾逊算术七条公理中的六条,对于皮亚诺算术的一阶公理化是足够的:\(^\text{[26]}\)
\begin{itemize}
\item 对所有 $x$:
      $\forall x\ (0 \ne S(x))$
     零不是任何数的后继。
\item 对所有 $x, y$:
      $\forall x, y\ (S(x) = S(y) \Rightarrow x = y)$
      后继函数是单射。
\item 对所有 $x$
      $\forall x\ (x + 0 = x)$
      加法对零的单位性。
\item 对所有 $x, y$
      $\forall x, y\ (x + S(y) = S(x + y))$
      加法的递归定义。
\item 对所有 $x$:
      $\forall x\ (x \cdot 0 = 0)$
      乘法对零的定义。
\item 对所有 $x, y$:
      $\forall x, y\ (x \cdot S(y) = x \cdot y + x)$
      乘法的递归定义。
\end{itemize}
除了上述关于数的基本公理之外,皮亚诺算术还包含归纳公理模式,该模式由一组递归可枚举、甚至是可判定的公理组成。对于皮亚诺算术语言中的任意公式 $\varphi(x, y_1, \ldots, y_k)$,其对应的一阶归纳公理为:
$$
\forall \bar{y} \left( \left( \varphi(0, \bar{y}) \land \forall x\left( \varphi(x, \bar{y}) \Rightarrow \varphi(S(x), \bar{y}) \right) \right) \Rightarrow \forall x \varphi(x, \bar{y}) \right)~
$$
其中 $\bar{y}$ 是 $y_1, \ldots, y_k$ 的缩写。一阶归纳公理模式包括了所有这种形式的实例,也就是说,它包含了对每一个一阶公式 $\varphi$ 所对应的归纳公理。
\subsubsection{等价的公理化形式}
上述的皮亚诺算术公理化采用的是一种语言符号系统,其中只包含“零”、以及“后继函数”、“加法”和“乘法”运算符号。然而,除了这种形式,还有许多不同但等价的公理化方式。

其中一种替代表述形式\(^\text{[27]}\),使用的是序关系符号代替后继函数,并基于离散有序半环的语言进行公理化。该系统的公理划分如下:第1–7条:关于半环结构的公理;第8–10条:关于序关系的公理;第11–13条:关于运算与序关系的兼容性;第14–15条:关于离散性的规定。
\begin{enumerate}
\item 对所有 $x, y, z$:
$(x + y) + z = x + (y + z)$  即:加法满足结合律。
\item 对所有 $x, y$:
$x + y = y + x$  即:加法满足交换律。
\item 对所有 $x, y, z$:
$(x \cdot y) \cdot z = x \cdot (y \cdot z)$  即:乘法满足结合律。
\item 对所有 $x, y$:
$x \cdot y = y \cdot x$  即:乘法满足交换律。
\item 对所有 $x, y, z$:
$x \cdot (y + z) = (x \cdot y) + (x \cdot z)$  即:乘法对加法满足分配律。
\item 对所有 $x$:
$x + 0 = x \land x \cdot 0 = 0$  即:0 是加法的单位元,是乘法的吸收元(实际上是多余的\(^\text{[note 3]}\))。
\item 对所有 $x$:
$x \cdot 1 = x$  即:1 是乘法的单位元。
\item 对所有 $x, y, z$:
  若 $x < y \land y < z$,则 $x < z$  即:小于关系满足传递性。
\item 对所有 $x$:
$\lnot(x < x)$  即:小于关系具备反自反性。
\item 对所有 $x, y$:
$x < y \lor x = y \lor y < x$  即:任意两个元素之间满足三歧性。
\item 对所有 $x, y, z$:
  若 $x < y$,则 $x + z < y + z$  即:加上相同的元素保持大小关系。
\item 对所有 $x, y, z$:
  若 $0 < z \land x < y$,则 $x \cdot z < y \cdot z$  即:与同一个正数相乘保持大小关系**。
\item 对所有 $x, y$:
  若 $x < y$,则存在某个 $z$,使得 $x + z = y$  即:任意两个不等的元素之间存在差值。
\item $0 < 1 \land \forall x\ (x > 0 \Rightarrow x \geq 1)$  即:0 和 1 是不同的元素,且它们之间没有中间值。换言之,1 是 0 的后继,表示数是离散的。
\item 对所有 $x$:
$x \geq 0$  即:0 是最小的自然数元素。
\end{enumerate}