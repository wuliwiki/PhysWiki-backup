% 路易·德布罗意(综述)
% license CCBYSA3
% type Wiki

本文根据 CC-BY-SA 协议转载翻译自维基百科\href{https://en.wikipedia.org/wiki/Louis_de_Broglie}{相关文章}。

\begin{figure}[ht]
\centering
\includegraphics[width=6cm]{./figures/34cce1e349a4d7bf.png}
\caption{德布罗意在1929年} \label{fig_Brogli_1}
\end{figure}
路易·维克托·皮埃尔·雷蒙德,第七代布罗意公爵(法语:[də bʁɔj] 或 [də bʁœj],1892年8月15日-1987年3月19日)是法国物理学家和贵族,他对量子理论做出了开创性贡献。在他1924年的博士论文中,他假设了电子的波动性质,并提出所有物质都有波动特性。这个概念被称为德布罗意假设,是波粒二象性的一个例子,并成为量子力学理论的核心部分。

德布罗意于1929年获得诺贝尔物理学奖,因为物质的波动行为在1927年首次得到了实验验证。

德布罗意发现的粒子波动行为被厄尔温·薛定谔用在他提出的波动力学中。德布罗意的导波概念于1927年在索尔维会议上提出,随后被放弃,转而支持量子力学,直到1952年被大卫·玻姆重新发现并加以完善。

路易·德布罗意于1944年当选为法兰西学院第16位成员,担任法兰西科学院的终身秘书。德布罗意是第一位呼吁建立多国实验室的高级科学家,这一提议最终促成了欧洲核子研究组织(CERN)的成立。
\subsection{传记}  
\subsubsection{家庭与教育}
路易·德布罗意出身于著名的布罗意贵族家族,几百年来,该家族的成员在法国担任重要的军事和政治职务。未来物理学家的父亲路易-阿尔方斯-维克多,第五代布罗意公爵,娶了波琳·达尔梅伊尔,她是拿破仑时代将军菲利普·保尔·塞吉尔伯爵的孙女,而塞吉尔伯爵的妻子是传记作家玛丽·塞勒斯廷·阿梅丽·达尔梅伊尔。他们有五个孩子,除了路易,还有:阿尔贝蒂娜(1872–1946),后来成为卢佩侯爵夫人;莫里斯(1875–1960),后成为著名的实验物理学家;菲利普(1881–1890),在路易出生前两年去世;波琳,潘日女伯爵(1888–1972),后成为著名作家。

路易·德·布罗意出生于法国塞纳-马尔姆地区的迪耶普。作为家中的小儿子,路易在相对孤独的环境中长大,阅读了大量书籍,并且特别喜欢历史,尤其是政治历史。从小他记忆力极好,能准确地背诵剧本中的片段,或者列出法兰西第三共和国的所有内阁部长。因此,人们预测他将来会成为一位伟大的政治家。

德·布罗意原本打算从事人文学科的职业,并获得了历史学的学士学位。此后,他转向数学和物理学,并获得了物理学的学位。第一次世界大战爆发后,他主动为军队提供服务,参与了无线电通讯的开发。
\subsubsection{军服务}
毕业后,路易·德·布罗意加入了工程部队,开始了强制性服役。服役开始于蒙·瓦莱里安堡,但不久后,在他哥哥的提议下,他被调到无线电通讯服务,并在埃菲尔铁塔工作,那里有无线电发射机。路易·德·布罗意在第一次世界大战期间一直服役,主要处理技术性问题。特别是,他与莱昂·布里渥因(Léon Brillouin)和哥哥莫里斯一起,参与了与潜艇的无线通信建设。路易·德·布罗意于1919年8月退役,晋升为上士。后来,这位科学家遗憾地表示,他不得不离开自己真正感兴趣的基础科学问题约六年之久。
\subsubsection{科学与教学生涯}  
他的1924年论文《Recherches sur la théorie des quanta》(《量子理论研究》)提出了他的电子波理论。这个理论包括物质的波粒二象性理论,基于马克斯·普朗克和阿尔伯特·爱因斯坦关于光的研究。这项研究最终得出了德布罗意假说,指出任何运动的粒子或物体都有一个相关的波。德布罗意由此创造了物理学的新领域——波动力学(mécanique ondulatoire),将能量(波)和物质(粒子)的物理学统一在一起。他因发现电子的波动性质而获得了1929年诺贝尔物理学奖。

在他后来的职业生涯中,德布罗意致力于发展波动力学的因果解释,反对主导量子力学理论的完全概率模型;该理论在1950年代由大卫·玻姆进一步完善。此理论后来被称为德布罗意–玻姆理论。

除了严格的科学工作,德布罗意还思考并写作关于科学哲学的内容,包括现代科学发现的价值。1930年,他创办了由埃尔曼出版社出版的书籍系列《Actualités scientifiques et industrielles》(《科学与工业新闻》)。

德布罗意于1933年成为法国科学院院士,并于1942年起担任该院的永久秘书。他曾被邀请加入法国天主教科学家联合会(Le Conseil de l'Union Catholique des Scientifiques Francais),但因他无宗教信仰而拒绝了邀请。1941年,他成为维希法国的国家委员会成员。1944年10月12日,他当选为法国科学院院士,接替数学家埃米尔·皮卡尔。由于在占领期间科学院成员的死亡和监禁以及战争带来的其他影响,学院无法达到选举所需的20名成员法定人数;然而,鉴于特殊情况,17名到场成员的一致选举被接受。在这一事件中,历史上唯一一次,由他自己的兄弟莫里斯(1934年当选)接待他成为新成员。

联合国教科文组织于1952年授予他首届卡林加奖,以表彰他在普及科学知识方面的贡献,1953年4月23日,他被选为皇家学会外籍会员。

1960年,随着他的长兄莫里斯(第六代德布罗意公爵)去世且无继承人,路易成为第七代德布罗意公爵,继承了家族的头衔。

1961年,他被授予法国荣誉军团大十字勋章骑士称号。德布罗意于1945年被任命为法国原子能高级委员会顾问,以表彰他为促进工业与科学的紧密合作所做的努力。他在亨利·庞加莱研究所建立了一个应用力学中心,开展光学、控制论和原子能方面的研究。他激励了国际量子分子科学学会的成立,并成为其早期成员。

路易一生未婚。1987年3月19日,他在卢韦西安斯去世,享年94岁。他的公爵头衔由远房表亲维克多-弗朗索瓦(第八代德布罗意公爵)继承。他的葬礼于1987年3月23日在圣皮埃尔-德-诺伊耶教堂举行。
\subsection{科学活动}
\subsubsection{X射线和光电效应的物理学}
路易·德布罗意的第一批研究(20世纪20年代初)是在他哥哥莫里斯的实验室进行的,涉及光电效应的特点和X射线的性质。这些研究探讨了X射线的吸收现象,并使用玻尔理论对这一现象进行了描述,应用量子原理解释了光电子光谱,并给出了X射线光谱的系统分类。这些X射线光谱的研究对于阐明原子内电子壳层的结构非常重要(光学光谱由外层壳层决定)。因此,与亚历山大·多维利耶(Alexandre Dauvillier)共同进行的实验结果揭示了现有电子分布模型的不足;这些困难后来由埃德蒙·斯通(Edmund Stoner)解决。另一个结果是阐明了索末菲公式在确定X射线光谱中线的位置时的不足;这种差异在发现电子自旋后得以解决。1925年和1926年,列宁格勒物理学家奥列斯特·赫沃尔松(Orest Khvolson)提名德布罗意兄弟因其在X射线领域的工作而获得诺贝尔奖。
\subsubsection{物质与波粒二象性}  
在研究X射线辐射的性质并与哥哥莫里斯讨论这些射线的特性时,莫里斯认为这些射线是一种波和粒子的结合,这促使路易·德布罗意意识到需要建立一个将粒子和波的表现形式联系起来的理论。此外,他还了解了马塞尔·布里渊(Marcel Brillouin)在1919到1922年间提出的原子水动力学模型,并试图将其与玻尔理论的结果联系起来。路易·德布罗意工作的起点是爱因斯坦关于光量子的观点。在他1922年发表的第一篇关于这一主题的文章中,法国科学家将黑体辐射视为光量子气体,并通过经典统计力学,在这种表述框架下推导出了维恩辐射定律。在他的下一篇出版物中,他试图将光量子的概念与干涉和衍射现象结合起来,得出结论认为必须将某种周期性与量子联系起来。在这种情况下,光量子被他解释为具有非常小质量的相对论粒子。[25]

路易·德布罗意认为,接下来需要将波动的概念扩展到任何有质量的粒子。在1923年夏天,他取得了决定性的突破。德布罗意在一篇简短的论文《波与量子》(法文:Ondes et quanta)中阐述了他的思想,该论文于1923年9月10日提交给巴黎科学院会议,并标志着波动力学创立的开始。在这篇论文及他随后的博士论文中,[16] 德布罗意提出,具有能量E和速度v的运动粒子,由某种内部周期过程特征化,其频率为 \(E/h\)(后来称为康普顿频率),其中 \(h\) 是普朗克常数。为了将这些基于量子原理的思考与相对论的思想相协调,德布罗意将他所称之为“相位波”的波与运动的物体联系起来,这种波以相位速度 \(c^{2}/v\) 传播。这样一种波,后来被称为物质波或德布罗意波,在物体运动过程中始终与内部周期过程保持相位一致。然后,他研究了电子在闭合轨道上的运动,表明相位匹配的要求直接导致了量子化的玻尔-索末菲条件,也就是说,角动量是量子化的。在接下来的两篇笔记中(分别在9月24日和10月8日的会议上报告),德布罗意得出结论,粒子速度等于相位波的群速度,粒子沿着等相位面法线的方向运动。在一般情况下,粒子的轨迹可以使用费马原理(对于波)或最小作用原理(对于粒子)来确定,这表明几何光学与经典力学之间存在联系。[27]

这一理论为波动力学奠定了基础。它得到了爱因斯坦的支持,通过G·P·汤姆森以及戴维森和杰尔梅的电子衍射实验得到了证实,并通过厄尔温·薛定谔的工作得到了推广。

从哲学角度来看,物质波理论大大破坏了过去的原子论。最初,德布罗意认为真实的波(即具有直接物理解释的波)与粒子相关联。事实上,物质的波动性是通过薛定谔方程定义的波函数来形式化的,波函数是一个纯粹的数学实体,具有概率性解释,缺乏真实物理元素的支持。这个波函数赋予物质波动行为的外观,却没有使真实的物理波出现。然而,直到他去世之前,德布罗意一直回归到物质波的直接和真实物理解释,这一过程是受到了大卫·玻姆工作的影响。
\subsubsection{电子内部时钟的假设}  
在他1924年的博士论文中,德布罗意假设电子具有一个内部时钟,这个时钟是引导粒子运动的“引导波”机制的一部分。[28] 随后,大卫·赫斯特尼斯提出了与薛定谔所建议的“振动运动”(zitterbewegung)之间的联系。[29]

尽管迄今为止验证内部时钟假设和测量时钟频率的尝试并未得出决定性结论,[30] 但最新的实验数据至少与德布罗意的假设相符。[31]
\subsubsection{质量的非零性和可变性}  
根据德布罗意的观点,中微子和光子具有非零的静质量,尽管它们的质量非常低。光子不是完全没有质量,这一点是他理论的内在要求。顺便提一下,正是这种拒绝无质量光子的假设使得他对宇宙膨胀假设产生了怀疑。

此外,他认为粒子的真实质量不是恒定的,而是可变的,并且每个粒子都可以被看作是一个热力学机器,相当于一个循环的作用积分。
\subsubsection{最小作用原理的推广}  
在他1924年论文的第二部分中,德布罗意利用最小作用原理的机械等价性与费马光学原理: “费马原理应用于相位波,与毛佩图斯原理应用于运动物体是相同的;运动物体的可能动态轨迹与波的可能光线是相同的。”这一等价性早在一个世纪前就被威廉·罗文·哈密尔顿指出,并在1830年左右由他发表,适用于光的情况。
\subsubsection{自然法则的二象性}  
德布罗意并没有像马克斯·玻恩所认为的那样,通过统计方法“使矛盾消失”,他将波粒二象性扩展到所有粒子(以及揭示衍射效应的晶体),并将二象性原理扩展到自然法则。

他最后的工作使热力学和力学这两个大的体系的法则形成一个统一的系统:

“当玻尔兹曼及其继承者发展出热力学的统计解释时,人们可以认为热力学是动力学的一个复杂分支。但根据我现在的想法,似乎是动力学才是热力学的简化分支。我认为,在过去几年里,我在量子理论中引入的所有思想中,这一思想无疑是最重要和最深刻的。”

这个想法似乎与连续-离散二象性相契合,因为它的动力学可以是热力学的极限,当假设过渡到连续极限时。它也接近戈特弗里德·威廉·莱布尼茨的观点,莱布尼茨提出,必须有“建筑原理”来完善机械法则的系统。

然而,他认为,这里并不是对立的二象性,而是合成(一个是另一个的极限),并且他认为合成的努力是持续不断的,就像他的第一个公式中那样,其中第一项属于力学,第二项属于光学:
\[ mc^{2} = h\nu ~\]
\subsubsection{光的中微子理论}   
这个理论起源于1934年,提出了光子等同于两个狄拉克中微子融合的观点。[32] 1938年,这一概念因不具有旋转不变性而受到挑战,关于该概念的研究大多被中止。[33]
\subsubsection{隐含热力学}  
德布罗意的最终想法是孤立粒子的隐含热力学。这是试图将物理学中三个最为远离的原理结合起来:费马原理、莫普图理想和卡诺原理。

在这项工作中,作用变成了一种与熵相对的概念,通过一个方程将这两个唯一的普适维度联系起来,形式如下:
\[
\frac{\text{作用}}{h} = -\frac{\text{熵}}{k}~
\]
由于其重大影响,这一理论将不确定性原理带回到作用极值附近的距离,这些距离对应于熵的减少。
\subsection{荣誉与奖项}  
\begin{itemize}
\item 1929年 诺贝尔物理学奖  
\item 1929年 亨利·庞加莱奖章  
\item 1932年 摩纳哥阿尔贝一世奖  
\item 1938年 马克斯·普朗克奖章  
\item 1938年 瑞典皇家科学院院士  
\item 1939年 美国哲学学会国际会员[34]  
\item 1944年 法国科学院院士  
\item 1948年 美国国家科学院国际会员[35]  
\item 1952年 卡林加奖  
\item 1953年 伦敦皇家学会院士[36]  
\item 1958年 美国艺术与科学学院国际荣誉会员[37]
\end{itemize}
\subsection{出版物}
\begin{itemize}
\item 量子理论研究(Recherches sur la théorie des quanta),论文,巴黎,1924年,《物理年鉴》(Ann. de Physique)(10) 3,22(1925年)。  
\item X射线和伽马射线物理学导论(Introduction à la physique des rayons X et gamma),与莫里斯·德·布罗意共同著作,戈尔捷-维拉尔出版社,1928年。  
\item 波动与运动(Ondes et mouvements),巴黎:戈尔捷-维拉尔出版社,1926年。  
\item 第五届索尔维物理学大会报告(Rapport au 5ème Conseil de Physique Solvay),布鲁塞尔,1927年。  
\item 波动力学(Mecanique ondulatoire),巴黎:戈尔捷-维拉尔出版社,1928年。  
\item 波动与粒子研究集(Recueil d'exposés sur les ondes et corpuscules),巴黎:赫尔曼科学书店,1930年。  
\item 物质与光(Matière et lumière),巴黎:阿尔班·米歇尔出版社,1937年。  
\item 新物理学与量子(La Physique nouvelle et les quanta),弗拉马里昂出版社,1937年。  
\item 现代物理中的连续与不连续(Continu et discontinu en physique moderne),巴黎:阿尔班·米歇尔出版社,1941年。  
\item 波动、粒子与波动力学(Ondes, corpuscules, mécanique ondulatoire),巴黎:阿尔班·米歇尔出版社,1945年。  
\item 物理学与微物理学(Physique et microphysique),阿尔班·米歇尔出版社,1947年。  
\item 保罗·朗之万的生平与著作(Vie et œuvre de Paul Langevin),法国科学院,1947年。  
\item 电子与粒子光学(Optique électronique et corpusculaire),赫尔曼出版社,1950年。  
\item 科学家与发现(Savants et découvertes),巴黎:阿尔班·米歇尔出版社,1951年。  
\item 波动力学的因果和非线性解释尝试:双重解理论(Une tentative d'interprétation causale et non linéaire de la mécanique ondulatoire: la théorie de la double solution),巴黎:戈尔捷-维拉尔出版社,1956年。  
\item 英文版:非线性波动力学:因果解释(Non-linear Wave Mechanics: A Causal Interpretation),阿姆斯特丹:爱思唯尔出版社,1960年。
\item 《微物理学的新前景》(Nouvelles perspectives en microphysique),Albin Michel,1956。
\item 《在科学的道路上》(Sur les sentiers de la science),巴黎:Albin Michel,1960。
\item 《M. Jean-Pierre Vigier及其合作者的新粒子理论简介》(Introduction à la nouvelle théorie des particules de M. Jean-Pierre Vigier et de ses collaborateurs),巴黎:Gauthier-Villars,1961。巴黎:Albin Michel,1960。
  英文翻译:Introduction to the Vigier Theory of Elementary Particles,阿姆斯特丹:Elsevier,1963。
\item 《对当前波动力学解释基础的批判性研究》(Étude critique des bases de l'interprétation actuelle de la mécanique ondulatoire),巴黎:Gauthier-Villars,1963。
  英文翻译:The Current Interpretation of Wave Mechanics: A Critical Study,阿姆斯特丹:Elsevier,1964。
\item 《科学的确实性与不确定性》(Certitudes et incertitudes de la science),巴黎:Albin Michel,1966。
\item 与Louis Armand、Pierre Henri Simon等合著,《阿尔伯特·爱因斯坦》(Albert Einstein),巴黎:Hachette,1966。
  英文翻译:Einstein,Peebles Press,1979。
\item 《半个世纪的研究》(Recherches d'un demi-siècle),Albin Michel,1976。
\item 《海森堡不确定性与波动力学的概率解释》(Les incertitudes d'Heisenberg et l'interprétation probabiliste de la mécanique ondulatoire),Gauthier-Villars,1982。
\end{itemize}
\subsection{参考文献}
\begin{enumerate}
\item "de Broglie, Louis-Victor". Lexico UK English Dictionary. Oxford University Press. 归档于 2020年12月4日。
\item "de Broglie". The American Heritage Dictionary of the English Language (第5版). HarperCollins. 检索于2019年8月10日。
\item "De Broglie". Collins English Dictionary. HarperCollins. 检索于2019年8月10日。
\item Léon Warnant (1987). 《法语发音词典:当代规范》(Dictionnaire de la prononciation française dans sa norme actuelle)(第3版)。吉布尔:J. Duculot,S. A. ISBN 978-2-8011-0581-8。
\item Jean-Marie Pierret (1994). 《法语历史语音学与普通语音学概念》(Phonétique historique du français et notions de phonétique générale)。鲁汶:Peeters。第102页。ISBN 978-9-0683-1608-7。
\item Leroy, Francis (2003). 《诺贝尔奖获得者百年:化学、物理与医学》(A Century of Nobel Prize Recipients: Chemistry, Physics, and Medicine)(插图版)。CRC Press。第141页。ISBN 0-8247-0876-8。第141页摘录。
\item Whittaker, Edmund T. (1989). 《以太与电的理论史。第2卷:现代理论,1900 - 1926》(A history of the theories of aether & electricity. 2: The modern theories, 1900 - 1926)(再版)。纽约:Dover Publ. ISBN 978-0-486-26126-3。
\item Antony Valentini: 《经典、量子与亚量子物理学的引导波理论》,博士论文,ISAS,的里雅斯特,1992年。
\item "de Broglie vs Bohm". 1960年出版的Elsevier出版社书籍摘录。检索于2015年6月30日。
\item O'Connor, John J.; Robertson, Edmund F., "Louis de Broglie", MacTutor 数学历史档案,圣安德鲁斯大学。
\item "History of International Academy of Quantum Molecular Science". IAQMS. 检索于2010年3月8日。
\item "Louis de Broglie". Soylent Communications. 检索于2015年6月12日。
\item M. J. Nye. (1997). "Aristocratic Culture and the Pursuit of Science: The De Broglies in Modern France". 《Isis》. 88 (3): 397–421. doi:10.1086/383768. JSTOR 236150. S2CID 143439041.
\item A. Abragam. (1988). "Louis Victor Pierre Raymond de Broglie". 《Biographical Memoirs of Fellows of the Royal Society》 34: 22–41. doi:10.1098/rsbm.1988.0002.
\item J. Lacki. (2008). "Louis de Broglie". 《New Dictionary of Scientific Biography》 1. Detroit: Charles Scribner's Sons: 409–415.
\item de Broglie, Louis Victor. "On the Theory of Quanta" (PDF). Foundation of Louis de Broglie (English translation by A.F. Kracklauer, 2004). 检索于2020年1月2日。
\item "The Nobel Prize in Physics 1929". Nobel Foundation. 归档于2008年10月24日。检索于2008年10月9日。
\item Recherche (PPN) 01331081X: 《Actualités scientifiques et industrielles》,sudoc.fr. 检索于2021年12月11日。
\item Evans, James; Thorndike, Alan S. (2007). 《Quantum Mechanics at the Crossroads: New Perspectives From History, Philosophy And Physics》. Springer. 第71页。ISBN 9783540326632。
\item Kimball, John (2015). 《Physics Curiosities, Oddities, and Novelties》。CRC Press. 第323页。ISBN 978-1-4665-7636-0。
\end{enumerate}