% 自旋与进动
% license Xiao
% type Tutor

\begin{issues}
\issueMissDepend
\issueTODO
\end{issues}



以三维空间中自旋$1/2$的粒子为例,其自旋期望值为$(\overline{\hat S_x},\overline{\hat S_y},\overline{\hat S_z})$。设该粒子的初始态矢为$\ket{a}$,态矢绕$z$轴“转动”后变为$\mathrm e^{- \I\omega \hat S_zt}\ket{a}$。

则期望值变化为:

\begin{equation}
\bra{a}\hat S_i\ket{a}\rightarrow \bra{a}\mathrm e^{ \I\omega \hat S_zt}\hat S_i\mathrm e^{- \I\omega \hat S_zt}\ket{a}~.
\end{equation}
在$\hat S_z$表象下计算$\mathrm e^{ \I\omega \hat S_zt}\hat S_x\mathrm e^{- \I\omega \hat S_zt}$得:

\begin{equation}
\begin{aligned}
\mathrm e^{ \I\omega \hat S_zt}\hat S_x\mathrm e^{- \I\omega \hat S_zt}&=\mathrm e^{ \I\omega \hat S_zt}\left(\frac{1}{2}(\ket{-}\bra{+}+\ket{+}\bra{-})\right)\mathrm e^{- \I\omega \hat S_zt}\\
 &=\frac{1}{2}\left(\mathrm e^{-\mathrm i t}\ket{-}\bra{+}+\ket{+}\bra{-}\mathrm e^{\mathrm i t}\right)\\
 &=\frac{1}{2}\left[\cos(\omega t)(\ket{-}\bra{+}+\ket{+}\bra{-})+\mathrm i\sin(\omega t)(\ket{+}\bra{-}-\ket{-}\bra{+})\right]\\
 &=\cos(\omega t) \hat S_x-\sin(\omega t) \hat S_y~.
\end{aligned}
\end{equation}
因此,$\hat S_x$的期望值变化为:
\begin{equation}
\overline{\hat S_x}\rightarrow  \overline{\hat S_x}\cos(\omega t)-\overline{\hat S_y}\sin(\omega t)~.
\end{equation}
同理可以计算出其他分量的期望值变化:
\begin{equation}
\overline{\hat S_y}\rightarrow \overline{\hat S_y}\cos(\omega t)+\overline{\hat S_x}\sin(\omega t)~,
\end{equation}
\begin{equation}
\overline{\hat S_z}\rightarrow \overline{\hat S_z}~.
\end{equation}
因此,自旋期望值可看作经典矢量,态矢绕自旋$z$分量“旋转”相当于该矢量绕自旋$z$分量“旋转”:
\begin{equation}
\begin{pmatrix}
 \cos(\omega t) &-\sin(\omega t)  &0 \\
  \sin(\omega t) & \cos(\omega t)  & 0\\
  0& 0 &1
\end{pmatrix}
\begin{pmatrix}
 \overline{\hat S_x}\\
  \overline{\hat S_y}\\
 \overline{\hat S_z}
\end{pmatrix}
=
\begin{pmatrix}
  \overline{\hat S'_x}\\
  \overline{\hat S'_y}\\
 \overline{\hat S'_z}
\end{pmatrix}~.
\end{equation}
可以利用贝克-豪斯多夫(Baker-Hausdorff)公式计算$\mathrm e^{ \I\omega \hat S_zt}\hat S_x\mathrm e^{- \I\omega \hat S_zt}$。
计算过程表明自旋期望值的变化适用于任意角动量期望值的变化(即也适用于轨道角动量算子期望值)。



