% 玻尔兹曼方程
% 玻尔兹曼方程

\begin{issues}

\issueDraft
推导未完成
\end{issues}

\pentry{刘维尔定理\upref{LiouTh}}
\footnote{朗道.物理动理学.北京:高等教育出版社,2008.}玻尔兹曼方程是动理学理论的奠基者路德维希 $\cdot$ 玻尔兹曼于1872年首先推导出来的,其可表示为下面的积分微分方程的形式:
\begin{equation}\label{BolzEQ_eq1}
\pdv{f}{t}+\bvec{v}\vdot\nabla f=\int\omega\qty(f'f_1'-ff_1)\dd\Gamma_1\dd\Gamma'\dd\Gamma_1'
\end{equation}
式中,我们用$\Gamma$ 表示分布函数所依赖的变量中除分子质心坐标(和时间 $t$ )以外的一切变量总体. $f,f'$ 是气体分子在其相空间的分布函数 $f(t,\bvec r,\Gamma)$,本文规定函数 $f$ 的附标均对应于其变量 $\Gamma$ 的附标,即$f=f(t,\bvec r,\Gamma),f'=f(t,\bvec r,\Gamma')$,等等.$\omega=\omega(\Gamma',\Gamma_1';\Gamma,\Gamma_1)$是其所有变量的函数,其对应两分子初值为 $\Gamma$ 和 $\Gamma_1$ 而结果为 $\Gamma'$ 和 $\Gamma_1'$ 的碰撞(该碰撞简记为 $\Gamma,\Gamma_1\rightarrow\Gamma',\Gamma_1'$ ). 
\subsection{推导}
首先声明,为书写方便,我们这里的分布函数 $f(t,\bvec r,\Gamma)$ 代表相空间中单位体积元内的平均分子数,它等于通常的分布函数 $\rho(t,\bvec r,\Gamma)$(分子处于相空间中$(\bvec r,\Gamma)$ 附件单位体积元的概率) 乘以总分子数 $N$,这并不影响我们推导玻尔兹曼方程,这也可从 $f=N\rho$ 代入\autoref{BolzEQ_eq1} 和原方程等价看出\footnote{其实主要目的是为了贴合朗道的说法}.

