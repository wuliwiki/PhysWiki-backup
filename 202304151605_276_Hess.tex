% 盖斯定律与设计路径
% keys 赫斯 赫士 盖斯 Hess 路径 状态量变化

\pentry{态函数\upref{statef}}

\footnote{参考了朱文涛的《简明物理化学》,安宇等的《大学物理》课程,李俊杰的《物理化学》课程,刘俊吉等的《物理化学》}

\subsection{盖斯定律}
\begin{theorem}{盖斯定律}
(等压或等容过程中)一个反应,不管是一步完成的还是多步完成的,其热效应总是相同。
\end{theorem}
盖斯定律最早源于实验观察,是能量守恒在热力学中的另一种体现。

\begin{example}{碳的氧化}
\begin{figure}[ht]
\centering
\includegraphics[width=8cm]{./figures/ba8fa2b0324a8d6a.pdf}
\caption{C的氧化} \label{fig_Hess_2}
\end{figure}
一定量的C被氧化为CO2。无论是直接被氧化为CO2,还是先被不完全氧化生成CO、再由CO被氧化为CO2,二者放出的总热量应相同。
\end{example}

\subsection{盖斯定律的物理内涵}
事实上,盖斯定律有着更为基本、深刻的物理含义。\textbf{我们知道,根据状态量的特性,系统状态量的改变与路径无关\upref{StaPro}\upref{statef}}。也就是说,无论C是直接被氧化为CO2,还是先被不完全氧化生成CO、再由CO被氧化为CO2,由于系统的始末状态一致,因此系统始、末状态的焓H分别相同,因此系统的焓变也相同$$\Delta H=H_2-H_1=Constant$$,又因为等压,$$Q=\Delta H$$,系统的放热自然相同。

可见,盖斯定律正是状态量性质的一种体现。

\subsection{“设计路径”}
\begin{figure}[ht]
\centering
\includegraphics[width=12cm]{./figures/2f588186e1fc44aa.pdf}
\caption{设计路径} \label{fig_Hess_1}
\end{figure}

根据系统状态量的改变与路径无关的特性,我们可以\textbf{在两个状态间人为设计一条路径,以求解任意过程前后状态量的变化}。这可以理解为“广义盖斯定律”。

设计的路径一般为等压、等容或等温的可逆(准静态)路径;且可以包括多个步骤,每一步只改变一个变量。这样方便运用现成结论。

此处的状态量不再局限于焓。只要能找到适合的路径,原则上可以计算任意状态量(U,S,G,H...)的变化。然而,由于过程量(Q,W)的变化与路径有关,一般不可设计路径计算过程量的变化\footnote{除了少数 条件特殊 的情况}。

\begin{example}{烧水热力学}
\begin{figure}[ht]
\centering
\includegraphics[width=10cm]{./figures/c0a5de0975af0146.pdf}
\caption{烧水} \label{fig_Hess_3}
\end{figure}
一大气压下,使用500K的恒温烤炉加热一杯1mol的水,使其从室温273K加热至沸点373K(先不考虑相变)。求该过程中水的熵变。

这个问题粗看无从下手。热力学第二定律\upref{Td2Law}阐明的$dS=\delta q/T$只适用于可逆(准静态)过程。但是,视水为系统,根据系统状态量的改变与路径无关的特性,熵变应只和水的初始温度与末温度有关\footnote{别忘了,压力恒为一个大气压}。这启发我们设计一条可逆(准静态)路径,即令热水器的温度与水的温度时时相同。

\begin{figure}[ht]
\centering
\includegraphics[width=10cm]{./figures/5912d80e9e33285f.pdf}
\caption{准静态烧水} \label{fig_Hess_4}
\end{figure}
现在,我们可以运用热力学第二定律计算熵变了。
$$\Delta S=\int \delta q/T$$

如果忽略热容随温度的变化,则 
$$\Delta S=C_p \ln \frac{T}{T_0}$$

那么,我们可以计算这个系统的吸热吗?原则上,Q是过程量,不能设计路径;但这个问题有点特殊。由于这是一个等压过程,\textsl{刚好}有$Q=\Delta H$. $\Delta H$是很典型的可设计路径求解的量,$\Delta H = \int C_p \dd T$。
\end{example}

\subsection{更数学的表述}
\pentry{热容\upref{ThCapa},热力学关系式\upref{MWRel}}
我们尝试使用更数学的方式说明“设计路径”的含义。假设我们已经知道某一状态下系统的内能,(别忘了“状态公理”\upref{statef})
$$U_0=U(T_0,V_0)$$
那么怎么求解另一状态下系统的内能
$$U_1=U(T_1,V_1)$$
?(不考虑相变或者化学反应)

\begin{figure}[ht]
\centering
\includegraphics[width=12 cm]{./figures/efa0125f054f107f.pdf}
\caption{积分形式} \label{fig_Hess_5}
\end{figure}

纯粹运用多元微积分,我们将其写为积分的形式:
\begin{equation}\label{eq_Hess_2}
U(T_1,V_1)=U(T_0,V_0) + \int_{T_0}^{T_1} \left(\pdv{U}{T}\right)_V \dd T + \int_{V_0}^{V_1} \left(\pdv{U}{V}\right)_T \dd V 
\end{equation}

这两个偏导数就隐含了“设计路径”的思路:先固定体积不变,将温度从$T_0$升高至$T_1$;再控制温度不变,将体积从$V_0$升高至$V_1$。

在上一步我们似乎只玩了一个数学游戏。\textsl{道理我都懂,但 $\left(\pdv{U}{T}\right)_V$与$ \left(\pdv{U}{V}\right)_T $ 到底是什么?} 这就需要化用热容\upref{ThCapa}与热力学关系式\upref{MWRel}。

根据以上定理,$$\left(\pdv{U}{T}\right)_V = C_V$$,就是热容;$$ \left(\pdv{U}{V}\right)_T  = PT\beta-P $$,其中$\beta$是定容压强系数。

这些都是方便实验测量(或查表得知)的量,可将他们代入 \autoref{eq_Hess_2} 。因此,尽管我们不知道如何直接测量内能,但\textsl{阴差阳错之间}我们居然能计算内能了。
\begin{equation}\label{eq_Hess_1}
U(T_1,V_1)=U(T_0,V_0) + \int_{T_0}^{T_1} C_V \dd T + \int_{V_0}^{V_1} P(T\beta-1) \dd V 
\end{equation}

% 如果是\textbf{对外界变动不敏感的物质(如固体、液体,但一般不适用于气体),且变化不大},那么\autoref{eq_Hess_1} 可以简化为线性方程
% $$
% U(T_1,V_1)=U(T_0,V_0) + C_V \Delta T +  P(T\beta-1) \Delta V 
% $$


那么你可能想问,那么我们如何确定$U(T_0,V_0)$?事实上我们\textsl{不需要确定},因为在热力学中我们更关心的是状态量的变化,而非他的绝对数值。因此,可将\autoref{eq_Hess_2} 改写为更为实用的变化形式:
\begin{equation}
\Delta U = \int_{T_0}^{T_1} \left(\pdv{U}{T}\right)_V \dd T + \int_{V_0}^{V_1} \left(\pdv{U}{V}\right)_T \dd V 
\end{equation}
以及微分形式(可以理解为$T,V$轻微变化后$U$的变化)
\begin{equation}
\dd U =  \left(\pdv{U}{T}\right)_V \dd T +\left(\pdv{U}{V}\right)_T \dd V 
\end{equation}

还有一种等效的说法:类似于选取势能零点,人为选定一个$U(T_0,V_0)$作为内能零点(即令$U(T_0,V_0)=0$),那么就可以表达其余情况的$U$了 \footnote{你可能觉得选取$U(T = 0,V = 0)=0$是一个好主意。但不要忘了,在温度趋近绝对零度时物质的性质会变得复杂微妙起来;同时,你也没法将有限的物质压缩进无限小的体积中!}

\begin{figure}[ht]
\centering
\includegraphics[width=8cm]{./figures/3e57e1af184b8b0a.pdf}
\caption{选取一个内能零点} \label{fig_Hess_6}
\end{figure}

同理,熵也可以写出类似的表达式\upref{MacroS}。即使我们不知道什么是熵,但我们也能写出熵的变化量!
\begin{equation}
S=\int \left(\frac{C_p}{T}\dd T-\left(\frac{\partial V}{\partial T}\right)_p\dd p\right)+S_0
\end{equation}