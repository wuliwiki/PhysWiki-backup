% 正则化
% 正则化

\textbf{正则化}(Regularization)是机器学习中用于减少泛化误差(测试误差),从而缓解过度拟合的设计策略。当使用正则化策略减少泛化误差时,可能会增大训练误差。

\textbf{参数范数惩罚}是最常用的正则化策略之一。传统机器学习方法就有很多使用,而在当今的深度学习中也应用广泛。参数范数惩罚的主要思想是给目标函数$J$添加一个参数范数惩罚项$\Omega(\theta)$,限制模型的学习能力,从而减少过度拟合的发生。设$J'$为正则化后的目标函数,则有:
\begin{equation}
J'(\bvec \theta;\bvec X,y)=J(\theta;\bvec X,y)+\alpha\Omega(\bvec \theta)
\end{equation}
其中,$\alpha\in[0,\infty)$是权衡范数惩罚项$\Omega$和标准目标函数$J(\bvec X;\bvec \theta)$相对贡献的超参数。将$\alpha$设为$0$表示没有正则化。

\subsection{$L^2$参数正则化}

$L^2$参数正则化指的是用参数的$2$范数作为惩罚项,即把公式$(1)$中的$\Omega(\theta)$写成参数的$2$范数形式。数学表示是:$\Omega(\theta)=\frac{1}{2}||\bvec w||_2^2$.
在优化代价函数时,会使得$\Omega(\bvec \theta)$有极小化的趋势,因此,该正则化策略会使得权值趋向于原点。

采用了$L_2$正则化策略的代价函数表示为:
\begin{equation}
J'(\bvec \theta;\bvec X,y)=J(\bvec \theta;\bvec X,y)+\alpha\frac{1}{2}||\bvec w||_2^2~
\end{equation}

有的文献称$L^2$正则化为\textbf{岭回归}。


\subsection{$L^1$参数正则化}

$L_1$参数正则化也是一种常用的参数范数惩罚策略.顾名思义,就是采用参数的$1$范数作为惩罚项,即$\Omega(\theta)=||\bvec w||_1=\sum_i|\bvec w_i|$.

采用了$L_1$正则化策略的代价函数表示为:
\begin{equation}
J'(\bvec \theta;\bvec X,y)=J(\bvec \theta;\bvec X,y)+\alpha||\bvec w||_1~
\end{equation}