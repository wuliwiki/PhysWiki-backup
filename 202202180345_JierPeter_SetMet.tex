% 集合的测度(实变函数)
% 测度|实变函数|measure|集合|广义函数|泛函|可测集|Borel集

\pentry{集合\upref{Set},微积分或数学分析}

%放在实变函数还是广义函数里?或者广义函数本身作为实变函数的一个部分?

本节为实变函数的一部分,因此讨论中涉及的集合均为实数集合$\mathbb{R}^n$或者其子集.


\subsection{从Riemann积分到Lebesgue积分}

\pentry{黎曼积分与勒贝格积分\upref{Rieman}}

在微积分或数学分析中,介绍积分的时候通常都是指Riemann积分.Riemann积分的思路是\textbf{对定义域作分划},在每个分划区间里取一个代表的函数值作为“高度”,分划区间作为“底部”,构成许多“柱子”,计算这些柱子的“体积”并求和.这种思路的好处是逻辑上容易处理,只需要有极限的概念就能讨论清楚怎么积分.但是它有很多局限性,比如严重依赖函数的连续性,导致理论不完备,无法处理很多例外情况.比较典型的例子有Riemman函数和Dirichlet函数.

Lebesgue积分就是换了一种积分的思路,反过来\textbf{对值域作分划},计算各函数值对应的自变量集合的“面积”,以此来计算“柱子体积”并求和.上面提到的Riemann积分处理不了的病态函数,就可以用Lebesgue积分来处理,并且对于Riemann积分能处理的函数,两种积分算出来的结果是一样的.

考虑Riemann函数,函数值为整数$q$的点是$[0, 1]$上的全体形如$p/q$的既约真分数,记它们构成的集合为$A_q$.现在知道柱子的高度是$q$了,计算柱子的体积还需要底面积,也就是$A_q$的“面积”.从这里就能看出Lebesgue积分的特别之处,即需要考虑非开区间的集合的“面积”.

集合的“面积”,被称为“\textbf{测度(measure)}”.本节我们就来讨论测度的概念.



\subsection{点集的外测度}

\subsubsection{开集的体积}

\begin{definition}{“开方块”的体积}\label{SetMet_def1}

设$I=I_1\times I_2\times I_3\times\cdots\times I_n$是$\mathbb{R}^n$上\textbf{开区间}的笛卡尔积,其中$I_i=(x_i, y_i)$,那么定义$I$的体积为$\abs{I} = \prod_{i=1}^n \abs{y_i-x_i}$.

\end{definition}

$\mathbb{R}^n$的任意开集都能表示为形如\autoref{SetMet_def1} 中$I$的“开方块”的\textbf{不交并},因此开集的体积就可以直接定义为这些开方块的体积之和.

\begin{theorem}{}\label{SetMet_the2}
开集的体积具有如下性质:

\begin{enumerate}
\item \textbf{正定性}:对于任意\textbf{开集}$U$有$\abs{U}\geq 0$,且等号只在$U=\varnothing$时成立;\\
\item \textbf{单调性}:如果两个\textbf{开集}满足$U_1\subseteq U_2$,那么$\abs{U_1}\leq \abs{U_2}$;\\
\item \textbf{可数次可加性}:对于任意可数多个\textbf{开集}$U_i$,有$\abs{U_1\cup U_2\cup U_3\cup \cdots}\leq \abs{U_1}+\abs{U_2}+\abs{U_3}+\cdots$;\\
\item \textbf{完全可加性}:对于至多可数个\textbf{两两不交}的\textbf{开集}$U_n$,有$\abs{\bigcup U_n}=\sum\abs{U_n}$.
\end{enumerate}
\end{theorem}




\subsubsection{任意点集的外测度}

有了开集的体积概念,我们就可以定义一种测度了.

\begin{definition}{外测度}
设$A$是$\mathbb{R}^n$上的\textbf{任意点集},定义$\opn{m^*}(A)=\inf\{\abs{U}: U\text{是}\mathbb{R}^n\text{的开集}\}$为$A$的\textbf{外测度},有时候也表示为$\opn{m^*}A$.
\end{definition}

外测度可以这样理解:要测量一个物体的体积,就用不同形状和大小的容器去容纳它,整理出所有能容纳它的容器,那这些容器的体积的\textbf{下确界}就是该物体的体积.

这就是为什么我们要先定义开集的测度,因为它最直观,最容易理解,用开集来做容器再合适不过.

可以类比,如果先定义闭集的体积,就可以定义任意点集的\textbf{内测度},即反过来用给定的任意点集做容器,看它能容纳的闭集体积的上确界是什么.使用闭集建立内测度理论稍麻烦些,我们这里只用外测度.

\subsubsection{外测度的性质}

\begin{theorem}{}\label{SetMet_the1}
开集的外测度,就是开集的体积.
\end{theorem}

利用开集体积的性质和外测度的定义,很容易证明该定理,故在此省略证明.用简练的数学语言,可以把\autoref{SetMet_the1} 写成:$\opn{m^*}(U)=\abs{U}$,其中$U$是一个$\mathbb{R}^n$上的开集.

显然,外测度是体积的推广,也继承了体积的一些性质:

\begin{theorem}{}\label{SetMet_the3}
\begin{enumerate}
\item \textbf{非负性}:对于任意\textbf{集合}$E$有$\opn{m^*}{E}\geq 0$,且$\opn{m^*}\varnothing=0$\footnote{注意,这不是正定性,即非空集合的测度也可以为零,比如只有一个点的集合.};\\
\item \textbf{单调性}:如果两个\textbf{集合}满足$E_1\subseteq E_2$,那么$\opn{m^*}{E_1}\leq \opn{m^*}{E_2}$;\\
\item \textbf{可数次可加性}:对于任意可数多个\textbf{集合}$E_i$,有$\opn{m^*}({E_1\cup E_2\cup E_3\cup \cdots})\leq \opn{m^*}{E_1}+\opn{m^*}{E_2}+\opn{m^*}{E_3}$;\\
\item \textbf{分离条件下的可数可加性}:对于至多可数个\textbf{集合}$E_i$,如果存在对应的开集$U_i$,使得$E_i\subseteq U_i$,且各$U_i$两两不相交,那么就有$\opn{m^*}{\bigcup U_n}=\sum\opn{m^*}{U_n}$.
\end{enumerate}
\end{theorem}

注意比较\autoref{SetMet_the2} 和\autoref{SetMet_the1} 的异同.\autoref{SetMet_the3} 的第1和第2条很容易证明,留作习题.下面证明后两条.

\textbf{证明}:

\textbf{可数次可加性}:

应用开集体积的可数次可加性以及外测度的定义即可.任取开集$U_i$使得各$E_i\subseteq U_i$,那么有

\begin{equation}
E_1\cup E_2\cup E_3\cup \cdots \subseteq U_1\cup U_2\cup U_3\cup \cdots
\end{equation}

又因为
\begin{equation}\label{SetMet_eq1}
\abs{U_1\cup U_2\cup U_3\cup \cdots} \leq \abs{U_1}+\abs{U_2}+\abs{U_3}\cdots
\end{equation}

任何能容纳$E_1\cup E_2\cup E_3\cup \cdots $的开集,总可以写成$U_1\cup U_2\cup U_3\cup \cdots$的形式,其中$U_i$是任取的能容纳$E_i$的开集.因此由\autoref{SetMet_eq1} 可得,能容纳$E_1\cup E_2\cup E_3\cup \cdots $的开集的体积,小于等于能容纳各$E_i$的开集的体积之和——换句话说,即可数可加性.

\textbf{分离条件下的可数可加性}:

应用开集体积的完全可加性以及外测度的定义即可.

取开集$V_i$使得$E_i\subseteq V_i\subseteq U_i$,那么由开集体积的完全可加性,$\abs{V_1\cup V_2\cup V_3\cup \cdots}=\abs{V_1}+\abs{V_2}+\abs{V_3}+\cdots$.因此,每个能容纳$E_1\cup E_2\cup E_3\cup\cdots$的开集的体积,都等于一组能容纳各$E_i$的开集的体积之和——换句话说,即$\opn{m^*}(E_1\cup E_2\cup E_3)\geq\opn{m^*}E_1+\opn{m^*}E_2+\opn{m^*}E_3+\cdots$.

再结合$\opn{m^*}({E_1\cup E_2\cup E_3\cup \cdots})\leq \opn{m^*}{E_1}+\opn{m^*}{E_2}+\opn{m^*}{E_3}$,可知
\begin{equation}
\opn{m^*}({E_1\cup E_2\cup E_3\cup \cdots})\leq \opn{m^*}{E_1}+\opn{m^*}{E_2}+\opn{m^*}{E_3}
\end{equation}
得证.

\textbf{证毕}.

\subsection{可加性与可测性}

可加性是定义Lebesgue积分时极为基础的性质,它是说,任取两个不相交的集合,则它们的并集的测度等于各自测度的和.如果这个性质成立,那么Lebesgue积分就可以很自然地定义了.但是很可惜的是,它并不成立,我们用以下反例来说明:

\begin{example}{不具有可加性的集合}

取区间$(0, 1)$上的任意一点$x$,定义与该点相关的集合$Q_x=\{r\in(0, 1)|r-x\in\mathbb{Q}\}$.由于有理数相加减仍是有理数,故$Q_x$可以看作对$(0, 1)$区间进行的等价划分,即各$Q_x$两两不相交.

从各$Q_x$中分别取一个元素,构成一个新的集合,记为$S$.把区间$(-1, 1)$中的有理数进行排列(编号)\footnote{注意,不是按大小顺序编号(这是不可能做到的),而是用任意的编号方式.其中一种方式可以参见康托尔的三角编号法.给有理数编号而不是直接说取有理数,是为了方便之后构造并集.}:
\begin{equation}
q_1, q_2, q_3, \cdots 
\end{equation}

构造集合$S_i=\{s+q_i|s\in S\}$\footnote{即$S_i$是$S$平移的结果.使用区间$(-1, 1)$是为了保证全体$S_i$的并能包含$(0, 1)$.}.显然,各$S_i$互不相交.并且由于都是$S$平移的结果,各$S_i$的外测度也相等.

对于任意$Q_x$,从中取任何一个元素$y$,都有$Q_y=Q_x$,且全体$Q_x$的并集就是$(0, 1)$.因此,$\bigcup_{i=1}^\infty S_i\supseteq (0, 1)$.同时,又易证$\bigcup_{i=1}^\infty S_i\subseteq (-1, 2)$.综合这两个属于关系,可以得到
\begin{equation}
1\leq\opn{m^*}(\bigcup_{i=1}^\infty S_i)\leq 3
\end{equation}

可见,我们构造出了一类集合$S_i$,它们各自的测度都相等,彼此不相交,但是可数无穷多个它们的并集,测度却是个非负的有限数.有了这些信息就很容易得到不具有可加性的集合了:

假设对于任意编号$k$,都有$\opn{m^*}(\bigcup_{i=1}^(k+1) S_i)=\opn{m^*}(\bigcup_{i=1}^(k) S_i)+\opn{m^*}S_{k+1}$





\end{example}




















