% 全连接网络
% keys 全连接
% license Xiao
% type Tutor

\dr

\pentry{神经网络\upref{NN}}

\textbf{全连接网络}(Fully-connected neural network, FCNN)是由一系列全连接层组成的深度神经网络,是深度学习中的基本架构。全连接层的特点是相邻两层的任意两个神经元之间均有连接。可以说全连接网络是多层感知机的加深版本。

\begin{figure}[ht]
\centering
\includegraphics[width=6cm]{./figures/e2d45dcbed13526d.png}
\caption{全连接层 [1]} \label{fig_FCNN_1}
\end{figure}
\autoref{fig_FCNN_1} 表示的是一个全连接层。

\begin{figure}[ht]
\centering
\includegraphics[width=8cm]{./figures/6205d3a76aa14e17.png}
\caption{全连接网络 [1]} \label{fig_FCNN_2}
\end{figure}
\autoref{fig_FCNN_2} 表示的是一个全连接网络,其结构是由全连接层堆叠而成。

全连接网络可以用于处理一维数据,也可以处理二维甚至高维数据。比如,在图像处理中就有不少应用。一张图像可以视为一个二维数据。假设一组图像数据。



\textbf{参考文献}
\begin{enumerate}
\item https://www.oreilly.com/library/view/tensorflow-for-deep/9781491980446/ch04.html
\end{enumerate}