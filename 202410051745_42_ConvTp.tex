% 拓扑空间的收敛序列
% keys 收敛序列|拓扑空间
% license Usr
% type Tutor

\pentry{拓扑空间\nref{nod_Topol}}{nod_a6e9}

度量空间中,收敛序列起着奠基性的作用,例如数列的极限就是重要的例子,而数列的极限是微积分的基石。收敛序列的概念很容易搬到拓扑空间中来,但是它在拓扑空间中的地位不像度量空间中那样。主要的原因在于在度量空间中,序列收敛到 $x$ 是 $x$ 是接触点的充要条件,而在拓扑空间中,这一论断失效了。

\subsection{定义}
拓扑空间 $\mathcal T$ 中序列的定义本质是不变的,仅仅在于每一点 $x_n$ 都属于 $\mathcal T$。
\begin{definition}{收敛序列}
设 $\{x_n\}$ 是拓扑空间 $\mathcal T$ 上的点列,若 $x\in\mathcal T$ 的任一邻域都包含从某一 $N$ 开始的所有点 $x_n,n\geq N$,则称 $\{x_n\}$ \textbf{收敛}于 $x$,而 $\{x_n\}$ 则称为\textbf{收敛的}。
\end{definition}

 $M\subset\mathcal T$ 的\textbf{接触点} $x$ 是指 $x$ 的任一邻域都含有 $M$ 的点(即 $x\in[M]$\footnote{$[M]$ 指 $M$ 的闭包})。
 
 \begin{lemma}{}
 取 $[0,1]$ 为基本集合构造如下拓扑 $\mathcal T$:$O$ 是 $\mathcal T$ 的开集,当且将当 $O$ 是从 $[0,1]$ 中去掉任意有限或可数个点得到的。那么 $\mathcal T$ 的序列 $\{x_n\}$ 是收敛的,当且仅当 $\{x_n\}$ 是定常序列,即从某一下标开时,所有元素相同,即当且将当存在整数 $N>0$,使得$x_n=x_{n+1}=\ldots,n\geq N$。
 \end{lemma}
 \textbf{证明:}
设 $\{x_n\}$ 是 $\mathcal T$ 中任意这样的一个序列:对任意整数 $N>0$,都有 $n,m\geq N$,使得 $x_n\neq x_m$。下面用反证法证明。 

假设 $\{x_n\}$ 收敛到 $x\in\mathcal T$,则对任一 $x$ 的邻域 $O$,存在整数 $N>0$,使得 $x_n\in O,n\geq N$。然而,$[0,1]$ 中删掉了一切点 $x_i\neq x$ 得到的子集 $O'$ 满足 $\mathcal T$ 上开集的定义,因此 $O'$ 是 $x$ 的邻域。然而任一 $N>0$,存在 $x_i\neq x,i\geq N$,因此邻域 $O'$ 不能包含从某一项开始的序列中的点。  这一矛盾证明了定理。

 \textbf{证毕!}

\begin{lemma}{}

\end{lemma}


 




