% 阿贝尔微分方程恒等式
% 朗斯基行列式|Wronski行列式|Wronskian|线性微分方程|Abel's Identity|Abel's Formula|阿贝尔公式

本文翻译自WikiPedia的\href{https://en.wikipedia.org/wiki/Abel\%27s_identity}{Abel's Identity}.





In mathematics, Abel's identity (also called Abel's formula[1] or Abel's differential equation identity) is an equation that expresses the Wronskian of two solutions of a homogeneous second-order linear ordinary differential equation in terms of a coefficient of the original differential equation. The relation can be generalised to nth-order linear ordinary differential equations. The identity is named after the Norwegian mathematician Niels Henrik Abel.

Since Abel's identity relates the different linearly independent solutions of the differential equation, it can be used to find one solution from the other. It provides useful identities relating the solutions, and is also useful as a part of other techniques such as the method of variation of parameters. It is especially useful for equations such as Bessel's equation where the solutions do not have a simple analytical form, because in such cases the Wronskian is difficult to compute directly.

A generalisation to first-order systems of homogeneous linear differential equations is given by Liouville's formula.







