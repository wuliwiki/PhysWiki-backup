% SSH 笔记

\subsection{密码连接 ssh}
\begin{itemize}
\item ssh 连接服务器: \verb|ssh name@111.222.333.444 -p 端口号|。 不指定端口的话默认 22
\item \verb`sudo apt-get install ssh` 会安装 ssh server 用于远程控制。 确保 port22 是打开的
\item 检查本机 ip 地址, \verb`hostname -I`, 或者 \verb`ip addr show`, 或者 \verb|ifconfig|.
\item 服务器端安装 openssh-server \verb`sudo apt install openssh-server`
\item 修改 sshd 设置文件 \verb`/etc/ssh/sshd_config` (注意不是 \verb`ssh_config`): 将 \verb`PasswordAuthentication` 改为 \verb`yes`, 最后一行加上 \verb`AllowUsers 用户名`, 用户名是 linux 的用户名
\item 改完设置文件后 \verb`sudo systemctl restart sshd` 才会生效
\item 查看 service 是否在运行用  \verb`sudo systemctl status sshd`
\item 可以试试自己连接自己, 即 \verb`ssh localhost`, 如果成功, 就可以了
\item \verb|sudo service sshd --full-restart| 也可以重启
\end{itemize}

\subsection{putty}
\begin{itemize}
\item 用于在 windows 上用 ssh 连接 linux 电脑
\item 不能自动储存密码, 用升级版: kitty! 更高级: MobaXterm!
\item 注意网络断开会丢失未保存的信息, 确保电脑不要自动待机, 离开前千万要保存!
\item putty 是其他电脑上的远程控制程序。
\item 双击打开出现 setting 界面。 要连接, Host Name 是 IP 地址或者 URL, port 是 22, Connection Type 是 SSH.
\item 所有的设置完成以后可以用 Saved Sessions 保存。
\item putty 中, 右键标题栏可以选择 change settings, 与开始时的 setting 不完全相同。
\item 字体与颜色: 蓝色注释不明显, 在 setting 里面的 colours 选项卡中的 ansi blue 改成 (100, 100, 255)
\item 字体大小可以在 appearance 中修改。
\item 右键标题栏还可以选择新建多个同样设置的窗口, 一个用于编辑程序, 一个用于运行程序等。 用 duplicate session.
\item session timeout : 学校的 linux ssh 设置了5分钟的自动 timeout, 为了防止自动断开, 在 connection 选项卡里面上方的 seconds between keepalives 文本框里面输入 240 (s) 即可每四分钟更新一次。
\item 如果网络断开, 用 restart session 选项即可。
\item connection -> data 里面有 Auto-login username 可以填 linux 的用户名
\item 在命令行中, 用鼠标选中的内容会自动复制到剪切板, 鼠标右键用于粘贴! 命令行的任何能输入的地方都有效! 所以在 Vim 中根本不需要用 visual 来粘贴! 除非不想用鼠标。
\item 要保存密码, 给 putty.exe 创建快捷方式, 右键属性, 把 target 改成 \verb|"...\putty.exe" user@server.com -pw password| 即可(不安全)。
\item 其实 putty 的自动登录是要使用 key 文件的, 见下一节。 把生成的 \verb|id_rsa| 私钥文件用 \verb|puttygen.exe|, 选择菜单 Conversions->Import key 然后 save private key 就可以用这个 private key 设置 putty 自动登录了。
\end{itemize}

\subsection{使用 key 连接 ssh}
\begin{itemize}
\item 参考\href{https://www.digitalocean.com/community/tutorials/ssh-essentials-working-with-ssh-servers-clients-and-keys}{这个教程}
\item 用 \verb`ssh-keygen` 生成 key, 使用默认路径 \verb`~/.ssh/`, 会生成 \verb`id_rsa` 和 \verb`id_rsa.pub` 两个文件。 后者用于加密而不能解密, 是 public key, 可以给任何人。 前者是 private key, 用于解密, 绝对不能给别人。
\item 要先登录 host, 要把 public key (只有一行内容) append 到服务器中的 \verb`~/.ssh/authorized_keys` 文件。
\item 如果使用 \verb|sudo ssh|, 那么设置文件就在 \verb|/root/.ssh/| 下, 可以把 \verb|~/.ssh| 中的文件 copy 过去。
\item 一般来说如果用 key 的话 server 的 \verb`/etc/ssh/sshd_config` 是不需要设置的
\item 其中 append 的操作也可以用 \verb`ssh-copy-id -i /path/to/key.pub SERVERNAME` 会更方便, 如果事先允许密码登录的话。 省略 \verb|-i /path/to/key.pub| 的话默认就用 \verb|~/.ssh/id_rsa.pub|
\item ssh 用 key 来登录的原理大概就是, client 告诉 server 我要用哪个用户名登录, server 去用户名的 \verb`~/.ssh/authorized_keys` 中寻找 public key, 然后用其加密一个随机字符串给 client, client 用 private key 解密, 将字符串的 MD5 hash 发给 server, server 一对照 MD5 如果吻合, 开始连接。  (其实我猜 host 还需要发送一个密钥用于 ssh 通讯, 要不然 client 发给 host 的东西就无法加密了)
\item 现在 ssh 的话应该就不需要密码了
\item 如果 key 怎么都不成功,很可能是文件夹和文件权限的问题, 确保 \verb|用户文件夹|, \verb|.ssh| 文件夹和里面的所有文件具有恰当的(不要太开放)权限, 具体自行 google。
\item 如果想不写 ip 而是写名字, 那么在 \verb|/etc/hosts| 中添加 ip 和对应的名字即可。
\item 如果没有管理员权限, 那么可以在用户文件夹创建 \verb|~/.hosts|, 然后在 \verb|~/.bashrc| 修改环境变量 \verb|export HOSTALIASES=~/.hosts|
\item 另一种方法是, 在 client 上可以写一个设置文件 \verb`~/.ssh/config`, 设置 host 的名字, 就可以用名字而不是 ip 登录了
\begin{lstlisting}[language=bash]
Host 别名
Hostname 网址或者ip或者/etc/hosts中的名字
User 登录的用户名
Port 端口号
PubKeyAuthentication yes
IdentityFile path/to/private_key # ssh-keygen 生成的 key 如 ~/.ssh/id_rsa

# 简单的设置例如(默认项可以忽略,例如端口 22, 当前用户名等)
Host myserver1
Hostname 10.0.2.101
User addis
\end{lstlisting}
\verb|PubKeyAuthentication| 强制使用 key 登录而不用密码。

\item 如果遇到错误 \verb|Unable to negotiate with 40.74.28.9 port 22: no matching host key type found. Their offer: ssh-rsa|, 就在 config 中加上两行 \verb|HostkeyAlgorithms +ssh-rsa| 和 \verb|PubkeyAcceptedAlgorithms +ssh-rsa|。

\item 如果要设置 tunnel (也就是 ssh 到一个机器再 ssh 到另一个), 就先在 \verb|.ssh/config| 中设置好第一个机器的自动登录,再设置第二个机器如下(参考\href{https://askubuntu.com/questions/311447/how-do-i-ssh-to-machine-a-via-b-in-one-command}{这里})
\begin{lstlisting}[language=bash]
Host server1
  Hostname server1.example.com
  User 用户名
  IdentityFile ~/.ssh/id_rsa

Host server2
  Hostname server2.example.com
  User 用户名
  IdentityFile ~/.ssh/id_rsa
  ProxyJump server1
\end{lstlisting}
注意第二个 \verb|IdentityFile| 应该也是本机的, 如果服务器不认识仍然会索取密码。 用 \verb|ssh-copy-id server2| 即可。

\item 如果说用 key \verb`ssh` 的时候说 \verb`permission id_rsa too open`, 用 \verb`chmod 700 id_rsa*` 即可, 其中 7 变为二进制是 \verb`111` 对应 \verb`rwx` 的三个开关都打开, 0 对应 \verb`000` 没有任何权限, 所以 700 就是 owner 有所有权限, 其他用户没有任何权限
\item 虽然 public key 中最后有 \verb`用户名@主机`, 其实这个好像没有任何影响, private key 和 public key 都可以复制到许多不同的机器中, 让这些机器两两之间互相 ssh, 而它们的 \verb`~/.ssh/authorized_keys` 中只需要有一条
\item ssh 登录一个新主机时, 如果 \verb`known_host` 没有, 就会让确认, 确认以后就会把新主机的 id 加入该文件, 如果这个 id 以后变了(例如主机重装了 \verb`openssh-server`) 再登录就会出错(为了安全), 这时只需要将 \verb`known_host` 中该主机的记录删除即可。 也可以直接把整个文件删除, 但这样就所有 ssh 登录都要再确认一次了。
\item \verb|ssh-keygen -R [ip 或者 alias]| 可以删除该 server 在 \verb|known_host| 中的记录。
\end{itemize}

\subsection{ssh 和 sshd 的 authentication 调试}
\begin{itemize}
\item 如果想重置设置文件, 就重装 \verb|openssh-server|。
\item 另外参考\href{https://medium.com/ci-cd-devops/ssh-receive-packet-type-51-154288e46609}{这篇文章}。
\item 首先可以用 \verb|ssh -vvv ...| 来输出非常多的 debug 信息, 看具体哪一步出的问题
\item 其次可以在 \verb|sudo vi /etc/ssh/sshd_config| 中打开 sshd 的 log 功能, \verb|SyslogFacility AUTH|, \verb|LogLevel DEBUG|
\item 重启 \verb`sudo systemctl restart sshd`, 尝试重新链接 ssh
\item 失败后用 \verb|tail -l /var/log/auth.log| 查看日志看看具体问题
\end{itemize}

\subsection{ssh 语法}
\begin{itemize}
\item \verb`ssh xxx@xxx "命令"` 可以直接将某个命令在 host 上执行并退出, 而不需要先进入 host 的 shell, 执行命令再 exit。 用 \verb`&&` 可以将多个命令写在一起依次执行。
\item \verb`ssh xxx@xxx "cat > remote_file" < local_file` 可以直接将本地文件用 ssh 直接传到 host, 而不需要用 sftp。 也可以用 piping, 例如 \verb`cat local_file | ssh xxx@xxx "cat > remote_file"`, 效果一样
\item \verb`ssh xxx@xxx "cat /path/to/remote_file" > /path/to/local_file` 可以将 host 的文件传到本地
\end{itemize}

\subsection{端口转发}
\begin{itemize}
\item 现在就可以测试 \verb|-L| 转发了, 格式为 \verb`ssh -L 本地端口号:目标地址:目标端口号 用户名@转发服务器` 其中 \verb`本地端口号` 永远都是输入这个命令的机器的端口号。 \verb`目标地址` 既可以和 \verb`转发服务器` 相同也可以不同。 其实这个命令就是在通常的 \verb`ssh` 命令中增加了 \verb`-L 本地端口号:目标地址:目标端口号`
\item 可以先在本地做实验, 确定本地装了 open-ssh server 和 client, 确保可以 \verb`ssh localhost`
\item 输入 \verb`ssh -L 6600:120.79.212.166:80 localhost`, 这样就把本地端口 6600 连接到了 wuli.wiki 的 80 端口
\item 执行以后, 结果相当于 \verb`ssh localhost`, 如果这个 session 结束, 那么转发也就结束了。
\item 执行以后可以在浏览器输入 \verb`localhost:6600`, 就相当于输入了 \verb`120.79.212.166:80`
\item 注意一些网站禁止 ip 访问, 或者 ip 访问和域名访问有不同的结果

\item 作为第二个测试, 我们需要一台中转服务器, 例如要从本地通过中转 ssh 到 \verb|120.79.212.166:80| (wuli.wiki), 就在本地执行 \verb`ssh -L 6500:wuli.wiki:22 用户名@中转服务器`
\item 再打开另一个 shell, 执行 \verb`ssh -p 6500 root@localhost`, 其中 \verb`-p` 选项用于指定端口, 就相当于先 ssh 到中转服务器再 ssh 到 wuli.wiki

\item 转发后, 如果要使用 git, 可以用 \verb`:` 指定端口, 注意前面要加上 \verb`ssh://`。  例如 \verb`git clone ssh://root@localhost:6500/root/github/PhysWikiScan`
\end{itemize}

\subsubsection{Remote Forwarding}
\begin{itemize}
\item 用途: 例如校内网的服务器想让外网访问, 那就把外网一台有公共 ip 的转发服务器的某个端口和内网的服务器的某个端口链接
\item \verb`ssh -R 转发服务器端口号:localhost:本地端口号 用户名@转发服务器`
\end{itemize}
