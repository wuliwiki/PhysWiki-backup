% 泛函分析笔记 5
% 泛函分析|数学分析|空间|Banach 空间|希尔伯特空间

\subsection{5.1 Extensions and Embeddings}
\begin{itemize}
\item 线性空间 $X$ 到 $Y$ 的算符 $A$ 和 $B$ 记为 $B \subseteq A$ 当且仅当定义域 $D(B) \subseteq D(A)$ 且它们在 $D(B)$ 上是同一算符. 这时 $A$ 是 $B$ 的 \textbf{extension}

\item $A = B$ 当且仅当 $A\subseteq B$ 且 $B\subseteq A$

\item \textbf{embedding} $X \subseteq Y$ 是\textbf{连续的}当且仅当存他们之间存在线性,单射,连续的算符

\item \textbf{embedding} $X \subseteq Y$ 是\textbf{紧的}当且仅当存他们之间存在线性,单射, 紧的算符
\end{itemize}

\subsection{5.2 Self-Adjoint Operators}
\begin{itemize}
\item 令线性算符 $A: D(A) \subseteq X \to X$ 的定义域 $D(A)$ 在希尔伯特空间 $X$ 上稠密. 定义 $v \in D(A^*)$ 当且仅当存在 $w\in X$ 使 $(v|Au) = (w|u)$ 对任意 $u\in D(A)$ 都成立. 令 $A^*v := w$, 就得到了\textbf{伴随(adjoint)}算符 $A^*: D(A^*) \subseteq X \to X$. 所以 $D(A^*)$ 是满足定义的最大集合

\item 伴随算符: (1) 是线性的 (2) $(\alpha A)^* = \bar \alpha A^*$ (3) $A \subseteq B$ 意味着 $B^* \subseteq A^*$

\item 如果 $D(A^*)$ 在 $X$ 上稠密, 那么存在 $(A^*)^*$, 记为 $A^{**}$

\item 希尔伯特空间 $X$ 上的线性算符 $A$ 是\textbf{对称的(symmetric)} 当且仅当 $A \subseteq A^*$, 即 $(Au|v)=(u|Av)$ 对所有 $u, v\in D(A)$ 成立

\item 希尔伯特空间 $X$ 上的线性算符 $A$ 是\textbf{自伴的(self-adjoint)} 当且仅当 $A = A^*$, 注意 $D(A) = D(A^*)$. 自伴算符都是对称算符

\item 希尔伯特空间 $X$ 上的线性算符 $A$ 是 \textbf{skew-symmetric} 当且仅当 $A \subseteq -A^*$

\item 希尔伯特空间 $X$ 上的线性算符 $A$ 是 \textbf{斜自伴(skew-adjoint)} 当且仅当 $A = -A^*$

\item 希尔伯特空间 $X$ 上的线性连续算符 $A$ 的伴随算符也是线性连续的, 且 $\norm{A} = \norm{A^*}$ 以及 $A^{**} = A$

\item 令 $f:[a, b]\times[a,b]\to\mathbb R$ 为连续函数且 $-\infty< a < b < \infty$. 定义算符 $A$ 为 $(Au)(x) := \int_a^b f(x, y)u(y)\dd{y}$, 令 $X := L_2(a, b)$. 那么 (1) $A: X\to X$ 是线性紧算符, (2) $(A^*u)(x) = \int_a^b f(y, x) u(y)\dd{y}$, $A^*$ 也是线性紧算符. (3) 如果 $f(x, y) = f(y, x)$ 对任意 $x, y\in[a, b]$ 成立, 那么算符 $A$ 是自伴算符

\item 希尔伯特空间 $X$ 上的任意的线性自伴算符 $A$ 都是 \textbf{maximally symmetric}. 也就是说, 如果 $S$ 是对称算符且 $A \subseteq S$, 那么 $A = S$

\item 令 $X := L_2^{\mathbb C}(\mathbb R)$, 且 $(Au)(x) := u'(x)$ ($x \in \mathbb R$), $D(A) := {u\in X: u'\in X}$, $u'$ 为广义导数, 那么 (1) 算符 $A$ 是斜自伴算符, (2) $\I A$ 是自伴算符

\item 令 $B$ 为上一条中的 $A$ 改成 “非广义” 的求导算符, 且定义域为 $X \bigcap C^1(\mathbb R)$, 那么 $B$ 是 skew-symmetric 算符且 $\I B$ 是对称算符

\item 令 $X := L_2^{\mathbb C}(\mathbb R)$, 定义 $(Mu)(x) := xu(x)$ ($\forall x\in R$), $D(M) := \qty{u\in X: Mu\in X}$. 那么算符 $M$ 是自伴算符

\item 回忆:令 $M$ 为 Hilbert 空间 $X$ 的闭线性子空间, 任何 $u\in X$ 存在唯一的分解 $u = v+w$($v\in M$, $w \in M^\bot$). \textbf{正交投影算符} $P:X\to M$ 定义为 $Pu := v$

\item 正交投影算符 $P$ 是线性的, 连续的, 自伴的. 且 $P^2 = P$. 如果 $M$ 为非空, 那么 $\norm{P} = 1$

\item 如果 $P:X\to X$ 是一个线性连续的自伴算符且 $P^2 = P$, 那么 $P$ 就是正交投影算符

\item 令 $A:D(A)\subseteq X\to X$ 为 Hilbert 空间 $X$ 上的线性对称算符, 且 $R(A)$ 在 $X$ 上稠密, 那么 $(A^{-1})^* = (A^*)^{-1}$. 如果这个 $A$ 是自伴算符, 那么 $A^{-1}$ 也是

\item 对 Hilbert 空间 $X$ 上的线性算符 $U:X\to X$, 以下条件等价. (1) $U$ 是 unitary 算符, (2) $UU^* = U^*U = I$, (3) $U$ 是双射且 $U^{-1} = U^*$, (4) $U$ 是满射且 $\norm{Uv} = \norm{v}$ 对所有 $v\in X$ 成立
\end{itemize}

\subsection{5.9 Semigroups, One-Parameter Groups, and Teir Physical Relevance}
\begin{itemize}
\item 令 $X$ 为 Banach 空间. $X$ 上的 \textbf{semigroup} $\qty{S(t)}_{t\ge0}$ 包含一系列算符 $S(t):X\to X$ ($\forall t\ge0$) 使得: $S(t+s) = S(t)S(s)$ ($\forall t, s\ge 0$) 以及 $S(0) = I$

\item Semigroup $\qty{S(t)}_{t\ge0}$ 的 \textbf{generator} $A:D(A) \subseteq X\to X$ 定义为 $Au: = \lim_{t\to+0} [S(t)-I]t^{-1} u$, $u\in D(A)$ 当且仅当这个极限存在

\item 令 $\mathcal S_+$ 为 semigroup, $u(t) := S(t)u_0$ ($\forall t \ge 0$). 那么 $\mathcal S_+$ 是\textbf{强连续的\textbf{strongly continuous}} 当且仅当对任意 $u_0\in X$, 函数 $u:[0,\infty[ \to\mathbb R$ 是连续的

\item $\mathcal S_+$ 被称为 \textbf{nonexpansive} 当且仅当 $\mathcal S_+$ 是强连续的且 $\norm{S(t)u_0 - S(t)v_0} \le \norm{u_0 - v_0}$ ($\forall u_0, v_0 \in X$, 对每个 $t\ge0$)

\item $\mathcal S_+$ 被称为\textbf{线性的(linear)} 当且仅当 $S(t): X\to X$ ($\forall u_0, v_0\in X$)是线性连续的

\item Banach 空间 $X$ 上的 \textbf{one-parameter group} $\qty{S(t)}_{t\in\mathbb R}$ 包含一系列算符 $S(t):X\to X$ ($\forall t\in\mathbb R$), 使得: $S(t+s) = S(t)S(s)$ ($\forall t, s\in \mathbb R$) 以及 $S(0) = I$

\item One-parameter group 的 \textbf{generator} $A:D(A) \subseteq X\to X$ 定义为 $Au: = \lim_{t\to0} [S(t)-I]t^{-1} u$, $u\in D(A)$ 当且仅当这个极限存在

\item one-parameter group 的强连续和线性类比 semigroup

\item $\mathcal S$ 叫做 \textbf{一致连续(uniformly continuous)} 当且仅当 $\mathcal S$ 是线性的且 $\norm{S(t+h)-S(t)}\to0$ 当 $h\to0$ ($\forall t\in\mathbb R$)

\item 一致连续的 $\mathcal S$ 也是强连续的

\item 令 $A:X\to X$ 为 Banach 空间 $X$ 上的线性连续算符, 令 $S(t) := e^{tA}$($\forall t\in\mathbb R$). 那么 (1) $\mathcal S = \qty{S(t)}$ 是 $X$ 上的线性 one-parameter group, generator 为 $A$ (2) $\mathcal S$ 是一致连续(强连续)的. (3) 给出 $u_0 \in X$, 令 $u(t) := e^{tA} u_0$ ($\forall t\in\mathbb R$), 那么 $u = u(t)$ 是微分方程 $u'(t) = Au(t)$ ($-\infty<t<\infty$), $u(0) = u_0$ 的唯一解

\item 如果上面的 $A$ 是线性连续的, 那么上面的微分方程无法用于描述自然中的不可逆过程

\item \textbf{one-parameter unitary group}: 强连续, one-parameter group, 每个 $S(t)$ 都是 unitary 的

\item 令 $A$ 为复 Hilbert 空间 $X$ 中的线性连续的自伴算符, 令 $S(t) := e^{iAt}$($\forall t\in\mathbb R$), 那么 $\qty{S(t)}$ 是 one-parameter unitary group, generator 是 $iA$
\end{itemize}

\subsection{5.13 Applications to the Schrodinger Equation}
\begin{itemize}
\item $u'(t) = -iAu(t)$ ($-\infty<t<\infty$), $u(0) = u_0$

\item 令 $A:D(A) \subseteq X\to X$ 为复 Hilbert 空间 $X$ 的自伴算符, 
\end{itemize}


\subsection{5.14 Applications to Quantum Mechanics}
\begin{itemize}
\item Hilbert 空间中单位矢量 $\psi\in X$ ($(\psi|\psi)=1$)叫做\textbf{态(state)}

\item Hilbert 空间中的自伴算符 $A$ 叫做可观测量

\item 测量

\item 不确定原理

\item \textbf{动量算符} $A:D(A)\subseteq X\to X$ 定义为 $(A\phi)(x) := -i\hbar \dv*{\phi(x)}{x}$ ($\forall x\in\mathbb R$). 其中 $D(A) := \qty{\phi\in X: \phi'\in X}$, $\phi'$ 是广义导数

\item $(B\phi)(x) := x\phi(x)$ ($x\in\mathbb R$) 叫做\textbf{位置算符}, 其中 $D(B) := {\phi\in X: B\phi\in X}$

\item 位置算符和动量算符都是自伴算符(见上文)
\end{itemize}

\subsection{5.15 Generalized Eigenfunctions}
\begin{itemize}
\item 令 $X := L_2^{\mathbb C}(\mathbb R)$, $A:D(A) \subseteq X\to X$ 为对称算符, $\mathcal S\subseteq D(A)$ 那么非零的 tempered distribution $T \in\mathcal S'$ 叫做 $A$ 的 \textbf{广义本征函数(generalized eigenfunction)} (具有实数本征值) 当且仅当 $T(A\phi) = \lambda T(\phi)$ 对所有 $\phi\in\mathcal S$ 成立

\item 广义本征函数系 $\qty{T_\alpha}_{\alpha\in\mathcal A}$ 称为\textbf{完备的(complete)} 当且仅当 “$T_\alpha(\phi) = 0$ 对所有 $\alpha\in\mathcal A$ 和固定的 $\phi\in\mathcal S$ 成立” 意味着 $\phi = 0$

\item 令 $T(\phi) := (\psi|\phi) = \int_{-\infty}^\infty \overline{\psi(x)} \phi(x) \dd{x}$ ($\forall \phi\in\mathcal S$), 那么 (1) 对每个 $\psi\in X$, $T\in\mathcal S'$, (2) 映射 $\psi\mapsto T$ 是从 $X$ 到 $\mathcal S'$ 的线性双射算符

\item 上面 $A$ 中的每个本征函数 $\psi\in D(A)$ 可以映射到一个广义本征函数

\item 令 $p\in\mathbb R$, $T_p(\phi) := \int_{-\infty}^\infty \overline{\phi_p(x)}\phi(x) \dd{x}$ ($\forall \phi\in\mathcal S$) 其中 $\phi_p(x) := \exp(ipx/\hbar)$. $T_p \in \mathcal S'$

\item $\qty{\phi_p}_{p\in\mathbb R}$ (对应的广义函数 $T_p(\phi)$)是动量算符的完备广义本征函数

\item $\qty{\delta_y}_{y\in\mathbb R}$ 是位置算符 $B$ 的完备广义本征函数. 即 $\delta_y(B\phi) = y\delta_y(\phi)$ ($\forall \phi\in\mathcal S$)
\end{itemize}

\subsection{5.20 A Look at Scattering Theory}
\begin{itemize}
\item 实际运动: $\psi(t) := \exp(-itH/\hbar)\psi(0)$, 自由运动: $\psi_0(t) := \exp(-itH_0/\hbar)\psi_0(0)$
\item 运动 $\psi = \psi(t)$ 被称为 \textbf{asymptotically free} 当 $t\to+\infty$

\item $\psi(0)$ 是 asymptotically free motion 的初态当且仅当 $\psi(0)$ 与 $H$ 的所有本征矢(bound states)正交

\item 散射态 $\phi_p$ 是(对应) $H_0$ 的广义本征函数 $T_p(\phi)$. $T_p(H_0 \phi) = p^2/(2m) T_p(\phi)$ ($\forall \phi\in\mathcal S, p\in\mathbb R$)
\end{itemize}

\subsection{5.21 The Language of Physicists in Quantum Physics and the Justification of the Dirac Calculus}
\begin{itemize}
\item 令 $\phi_k(x) := (2\pi)^{-1/2} e^{ikx}$ $x, k\in\mathbb R$
\end{itemize}

\subsection{Appendix}\label{FnalN5_sub1}
\subsubsection{The Lebesgue Measure}
\begin{itemize}
\item $N$-cuboid: $C := \qty{(\xi_1, \dots, \xi_2) \in \mathbb R^N: a_j < \xi_j < b_j}$ ($j = 1, \dots, N$)

\item $N$-cuboid 的体积为 $\opn{vol}(C) := \prod_{j=1}^N(b_j - a_j)$

\item 勒贝格测度将 $\mathbb R^N$ 中的经典的体积从足够有规律的集合拓展到一些 “无规律” 的集合. 令 $\mathcal A$ 为 $\mathbb R^N$ 中可测子集的集合

\item (1) $\mathbb R^N$ 中任意开集和闭集属于 $\mathcal A$. (2) 如果 $A, B \in \mathcal A$, 那么 $A\cup B \in \mathcal A$, $A\cap B \in \mathcal A$, $A - B \in \mathcal A$. (3) $A_n \in \mathcal A$ 的无穷交集和无穷并集仍然属于 $\mathcal A$. (4) 每个 $A \in\mathcal A$ 可以赋予一个数 $0 \le \mu(A) \le \infty$, 叫做 $N$ 维\textbf{测度(measure)}, $A\in\mathcal A$ 叫做\textbf{可测的(measurable)}. (5) 如果 $A\cap B = \emptyset$, 那么 $\mu(A\cup B) = \mu(A) + \mu(B)$. 这对无穷个 $A_n$ 仍然适用. (6) 如果 $C$ 是一个 $N$-cuboid, 那么 $C\in\mathcal A$ 以及 $\mu(C) = \opn{vol}(C)$. (7) $A$ 的测度为零当且仅当对任意 $\varepsilon>0$, 存在可数个 $N$-cuboid $C_1, C_2, \dots$ 使得 $A \subseteq \bigcup_{j=1}^\infty C_j$ 且 $\sum_{j=1}^\infty \mu(C_j) < \varepsilon$.(8) 如果 $A$ 的 $N$ 维测度为零且 $B\subseteq A$, 那么 $B$ 的 $N$ 维测度也是零. (9) $\mathcal A$ 叫做\textbf{最小的(minimal)}, 如果 $\mathcal A'$ 满足以上条件, 就有 $\mathcal A \subseteq \mathcal A'$

\item 测度 $\mu$ 在 $\mathcal A$ 上唯一. $\mu$ 叫做勒贝格测度, 也记为 $\opn{meas}(A) := \mu(A)$ ($\forall A\in\mathcal A$)

\item $\mathbb R^N$ 中有限个或可数个点的 $N$ 维测度为零

\item 一个性质 $P$ \textbf{几乎到处(almost everywhere)}都成立, 当且仅当 $P$ 对 $\mathbb R^N$ 中除了测度为零的集合的点都成立

\item \textbf{几乎所有(almost all)}也类似. 例如几乎所有的实数都是无理数

\item 例如令 $M\subseteq\mathbb R^N$, $u(x) = \lim_{n\to\infty} u_n(x)$ (对几乎所有 $x\in M$)

\item $\mu(M)\le\mu(G)$ 当 $M\subseteq G$, $\mu(\mathbb R^N) = +\infty$ 且 $\mu(\emptyset) = 0$
\end{itemize}

\subsubsection{Step Functions}
\begin{itemize}
\item \textbf{阶梯函数(step function)}: $u:M\subseteq\mathbb R^N \to\mathbb K$, piecewise constant. $M$ 是可测的, 存在有限个两两不相交的 $M_j \subseteq M$, 使得他们的测度为有限且 $u(x) = a_j$ ($\forall x\in M_j$ 和 $\forall j$), $u(x) = 0$(otherwise)

\item 定义阶梯函数的积分为 $\int_M u\dd{x} := \sum_j \opn{meas}(M_j)a_j$
\end{itemize}

\subsubsection{Measurable Functions}
\begin{itemize}
\item \textbf{可测函数(Measurable Function)}: $u: M\subseteq\mathbb R^N \to\mathbb K$ (1) $M$ 是可测的 (2) 存在阶梯函数的序列 $u(x) = \lim_{n\to\infty} u_n(x)$ (对几乎所有 $x\in M$)

\item \textbf{Luzin 定理}: 令 $M$ 为 $\mathbb R^N$ 的可测子集. 函数 $u: M\to\mathbb K$ 是可测的当且仅当对任意 $\delta$, 它在除测度为 $\delta$ 的集合外是连续的

\item 函数 $f:M\subseteq\mathbb R^N \to \mathbb K$ 是可测的如果他在可测集 $M$ 上几乎到处连续

\item 可测函数的线性组合和极限也是可测的

\item 可测函数在定义域中测度为 0 的子集上改变了, 那么新的函数仍然是可测的
\end{itemize}

\subsubsection{Lebesgue Integral}
\begin{itemize}
\item \textbf{勒贝格积分(Lebesgue Integral)}: 函数 $u :M\subseteq\mathbb R^N\to\mathbb K$ 是\textbf{可积(integrable)}的当且仅当两个条件成立: (1) 存在由阶梯函数构成的函数列 $u_n$ 使得 $u(x) = \lim_{n\to\infty} u_n(x)$ 对几乎所有 $x\in M$ 成立 (2) 对任意 $\varepsilon> 0$, 存在 $n_0(\varepsilon)$ 使得 $\int_M \abs{u_n(x) - u_m(x)} \dd{x} < \varepsilon$ 对所有 $n, m\ge n_0(\varepsilon)$ 成立

\item 如果 $u$ 是可积的, 那么定义 $\int_M u\dd{x} := \lim_{n\to\infty} \int_{M} u_n \dd{x}$

\item 显然, 可积函数必定可测
\end{itemize}
