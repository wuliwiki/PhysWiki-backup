% 级数(极简微积分)
% keys series|级数
% license Xiao
% type Tutor

\begin{issues}
\issueDraft
\end{issues}

\pentry{数列的极限(极简微积分)\nref{nod_Lim0}}{nod_a0d0}

对于一个数列 $a_1,a_2,\dots$,形式表达式
\begin{equation}\label{eq_SerCal_1}
\sum_{n=1}^\infty a_n~
\end{equation}
称为以 $a_n$ 为一般项的\textbf{(无穷)级数 (series)}。 其中前 $N$ 项的和
\begin{equation}
S_N:=\sum_{n=1}^N a_n~
\end{equation}
称为级数 $\sum_{n=1}^\infty a_n$ 的\textbf{部分和 (partial sum)}. 如果部分和序列 $S_1,S_2,\dots$ 有极限, 也就是说存在实数 $S$ 使得
\begin{equation}
\lim_{N\to\infty}\sum_{n=1}^N a_n=S~,
\end{equation}
则称级数\autoref{eq_SerCal_1} 收敛到 $S$ \textbf{(converges to $S$)}, $S$ 称为它的\textbf{和(sum)}。 如果部分和序列不存在极限, 则称级数\textbf{不收敛}或\textbf{发散(divergent)}。

一个简单的收敛的例子是
\begin{example}{几何级数}
当$q > 1$的时候,几何级数
$$
\sum_{n=0}^\infty \frac1{q^n} = 1 + \frac1{q} + \frac1{q^2} + \cdots~,
$$
按照等比数列的求和公式,它的部分和为
$$
\sum_{n=0}^N \frac1{q^n} =\frac{1 - \frac1{q^{N+1}}}{1 - \frac1{q}}~.
$$
由此可见它收敛到 $\frac{q}{q - 1}$。
\end{example}

级数收敛的有一个很简单的必要条件:
\begin{theorem}{}
如果级数 $\sum_{n=1}^\infty a_n$ 收敛,那么数列 $a_1, a_2, \dots$ 收敛到 $0$。
\end{theorem}
如果数列 $a_n$ 不收敛,或者收敛到一个非零的数字,就会导致它们的部分和“震荡”。
% Giacomo:我也不知道怎么简单的表述这个现象,难道直接说显然吗?

这个定理不是充分的,我们有反例:
\begin{example}{调和级数}
调和级数
$$
\sum_{n=1}^\infty\frac{1}{n} = 1 + \frac12 + \frac13 + \cdots~
$$
不收敛,具体证明可以参考\autoref{ex_Series_1}~\upref{Series}。
\end{example}

另一方面我们也可以给出两个充分条件:
\begin{theorem}{}
如果级数 $\sum_{n=1}^\infty |a_n|$ 收敛,那么级数 $\sum_{n=1}^\infty a_n$ 收敛,而且我们有
$$
\left| \sum_{n=1}^\infty a_n \right| \leq \sum_{n=1}^\infty |a_n| ~.
$$
\end{theorem}
这是因为我们有
$$
\left|\sum_{n=1}^N a_n\right| \leq \sum_{n=1}^N |a_n|~.
$$

这个条件不是必要的,反例是 $\sum_{n = 1}^\infty (-1)^n\frac1{n}$ 收敛到 $-\ln(2)$。
\addTODO{在分析或者微积分板块证明它。}

另外一个是
\begin{theorem}{}\label{the_SerCal_1}
已知级数 $\sum_{n=1}^\infty b_n$ 收敛,如果我们有 $|a_n| \leq b_n$,那么级数 $\sum_{n=1}^\infty a_n$ 收敛,而且我们有
$$
\sum_{n=1}^\infty a_n \leq \sum_{n=1}^\infty b_n ~.
$$
\end{theorem}



\addTODO{可以看看\enref{级数(分析)}{Series}里还有什么东西可以搬过来}

\subsection{级数的应用}

% Giacomo:为啥要引用这个?
% 小角极限(极简微积分)\nref{nod_LimArc}, 复数\nref{nod_CplxNo}

\pentry{自然对数底(极简微积分)\nref{nod_E}}{nod_f32c}

自然常数 $\E$ 有一个等价定义为
\begin{equation}
\sum_{n=0}^\infty \frac{1}{n!} = 1 + 1 + \frac{1}{2} + \frac{1}{6} + \frac{1}{24} + \dots~,
\end{equation}
这是一个收敛的级数,因为根据\autoref{the_SerCal_1} 
$$
\begin{aligned}
\phantom{=}& 1 + 1 + \frac{1}{2} + \frac{1}{6} + \frac{1}{24} + \dots \\
=& 1 + 1 + \frac{1}{2} + \frac{1}{2 \cdot 3} + \frac{1}{3! \cdot 4} + \frac{1}{4! \cdot 5} + \dots \\
=& 1 + 1 + (\frac{1}{2} + \frac{1}{2 \cdot 3}) + (\frac{1}{3! \cdot 4} + \frac{1}{4! \cdot 5} + \frac{1}{5! \cdot 6} + \frac{1}{5! \cdot 6} + \dots
\end{aligned}~
$$
