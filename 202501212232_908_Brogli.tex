% 路易·德布罗意(综述)
% license CCBYSA3
% type Wiki

本文根据 CC-BY-SA 协议转载翻译自维基百科\href{https://en.wikipedia.org/wiki/Louis_de_Broglie}{相关文章}。

\begin{figure}[ht]
\centering
\includegraphics[width=6cm]{./figures/34cce1e349a4d7bf.png}
\caption{德布罗意在1929年} \label{fig_Brogli_1}
\end{figure}
路易·维克托·皮埃尔·雷蒙德,第七代布罗意公爵(法语:[də bʁɔj] 或 [də bʁœj],1892年8月15日-1987年3月19日)是法国物理学家和贵族,他对量子理论做出了开创性贡献。在他1924年的博士论文中,他假设了电子的波动性质,并提出所有物质都有波动特性。这个概念被称为德布罗意假设,是波粒二象性的一个例子,并成为量子力学理论的核心部分。

德布罗意于1929年获得诺贝尔物理学奖,因为物质的波动行为在1927年首次得到了实验验证。

德布罗意发现的粒子波动行为被厄尔温·薛定谔用在他提出的波动力学中。德布罗意的导波概念于1927年在索尔维会议上提出,随后被放弃,转而支持量子力学,直到1952年被大卫·玻姆重新发现并加以完善。

路易·德布罗意于1944年当选为法兰西学院第16位成员,担任法兰西科学院的终身秘书。德布罗意是第一位呼吁建立多国实验室的高级科学家,这一提议最终促成了欧洲核子研究组织(CERN)的成立。
\subsection{传记}  
\subsubsection{家庭与教育}

路易·德布罗意出身于著名的布罗意贵族家族,几百年来,该家族的成员在法国担任重要的军事和政治职务。未来物理学家的父亲路易-阿尔方斯-维克多,第五代布罗意公爵,娶了波琳·达尔梅伊尔,她是拿破仑时代将军菲利普·保尔·塞吉尔伯爵的孙女,而塞吉尔伯爵的妻子是传记作家玛丽·塞勒斯廷·阿梅丽·达尔梅伊尔。他们有五个孩子,除了路易,还有:阿尔贝蒂娜(1872–1946),后来成为卢佩侯爵夫人;莫里斯(1875–1960),后成为著名的实验物理学家;菲利普(1881–1890),在路易出生前两年去世;波琳,潘日女伯爵(1888–1972),后成为著名作家。