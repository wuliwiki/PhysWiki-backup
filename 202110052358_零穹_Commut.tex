% 对易厄米矩阵与共同本征矢
% 算符|本征矢|本征矢|量子力学|对易

% 和 “算符对易与共同本征函矢” 重复了! 保留这个

\pentry{厄米矩阵的本征问题\upref{HerEig}}
%未完成, 要引入希尔伯特子空间的概念, 说明子空间中的任意一个函数都是本征矢, n重简并的子空间是n维的, 即可以有n个线性无关的本征矢张成. 一般选取两两正交的波函数作为基底, 基底有无穷多种选法, 任何基底经过任意幺正变换以后仍然是子空间的基底. 类比一下

\subsection{对易与交换子}
本文讨论 $N = 2, \dots$ 维矢量空间 $X$ 上的任意两个厄米算符 $A: X \to X$ 和 $A: X \to X$. 当我们在 $X$ 中确定一组正交归一基底后, 它们可以分别表示为 $N \times N$ 的厄米矩阵 $\mat A$ 和 $\mat B$.

我们知道一般来说两个矩阵(线性算符)% 链接未完成
的乘法不满足交换律 $AB = BA$, 即\textbf{不对易(non-commutative)}, 但也存在满足交换律的特殊算符对, 即\textbf{对易的(commutative)}.

定义算符 $A,B$ 的\textbf{对易子(commutator)}为
\begin{equation}
[A, B] = AB - BA
\end{equation}
那么对易可以表示为 $[A,B] = 0$, 反之 $[A,B]\ne 0$.

\begin{theorem}{}
有限维矢量空间 $V$ 中两个厄米算符 $A$ 和 $B$, 以下两个命题互为充分必要条件
\begin{enumerate}
\item $A,B$ 对易($[A,B] = 0$).
\item $V$ 中存在一组正交归一基底, 同时是 $A, B$ 的本征矢.
\end{enumerate}
\end{theorem}


我们可以分为简并和非简并的情况讨论. 当 $A$ 和 $B$ 的本征问题都不存在简并, 那么这组共同正交归一基底是唯一的. 若简并, $A$ 的每个本征子空间内, 都能找到 $B$ 本征矢构成的正交归一基底.
\addTODO{本征子空间需要链接}

\subsection{证明条件 $2 \to 1$}
设算符 $A$ 和 $B$ 有一组共同的本征矢 $v_i$,  则它们同时满足 $A$ 和 $B$ 的本征方程
\begin{equation}
\begin{cases}
A v_i = a_i v_i\\
B v_i = b_i v_i
\end{cases}
\end{equation}
对任何 $v_i$,  都有
\begin{equation}
A (B v_i) = A (b_i v_i) = b_iA v_i = a_i b_i v_i
\end{equation}
\begin{equation}
B (A v_i) = B (a_i v_i) = a_i B v_i = a_i b_i v_i
\end{equation}
所以 $AB v_i = BA v_i$ 即
\begin{equation}
\comm*{A}{B} = AB - BA = 0
\end{equation}
即两算符对易.证毕.

\subsection{证明条件 $1 \to 2$}
要证明 $1 \to 2$,  只需证明 $A$ 的一套本征矢都满足 $B$ 的本征方程即可.

\subsubsection{算符 $A$ 非简并情况( $B$ 是否简并没关系)}
先解出算符 $A$ 的本征方程 $A v_i = a_i v_i$,  如果 $A$ 算符不发生简并(见本征矢的简并%未完成链接
  )那么本征值各不相同, 且给定一个本征值 $a_i$ 其解只可能是 $v_i$ 或者 $v_i$ 乘以一个任意复常数(注释:其实也可以再相乘一个算符 $A$ 不涉及的物理量的函数, 例如总能量算符 $H$ 的本征矢还可以再成一个时间因子 $\E^{\I\omega t}$ ).

因为算符对易, 有
\begin{equation}
A (B v_i) = B (A v_i) = a_i (B v_i)
\end{equation}
把式中的 $B v_i$ 看成一个新的波函数, 上式说明 $B v_i$ 是算符 $A$ 和本征值 $a_i$ 的另一个本征矢. 根据以上分析, $B v_i$ 必定是 $v_i$ 乘以某个复常数(命名为 $b_i$ ), 即
\begin{equation}
B v_i = b_i v_i
\end{equation}
而这正是 $B$ 的本征方程(而 $B$ 也是厄米矩阵, 所以作为本征值 $b_i$ 的数域从复数缩小到实数). 证毕.

\subsubsection{算符 $A$ 简并情况}
假设算符 $A$ 的所有本征值为 $a_i$ (各不相同), 任意一个 $a_i$ 有 $n_i$ 重简并. 若 $n_i = 1$,  对应唯一一个 $v_i$,  那么根据上文对非简并情况的推理, $v_i$ 就已经是 $B$ 的本征矢了. 若 $n_i > 1$,  存在一个 $n_i$ 维希尔伯特子空间, 里面任何一个函数都是 $a_i$ 对应的本征矢, 所以要在子空间中寻找共同本征矢, 只需在子空间中寻找 $B$ 的本征矢即可. 令 $\phi_i$ 为本征值为 $a_i$ 的子空间中的任意函数, 利用对易关系
\begin{equation}
A (B \phi_i) = B (A \phi_i) = a_i (B \phi_i)
\end{equation}
这条式子说明 $B \phi_i$ 是 $A$ 和 $a_i$ 的一个本征矢, 即 $B \phi_i$ 仍然在 $a_i$ 的简并子空间中.所以 $B$ 对子空间来说是一个闭合的厄米算符, 所以必有 $N$ 个线性无关的本征矢. 证毕.%(厄米算符性质 $x$,  未完成)

以下的内容应该归到厄米算符里面讲(厄米算符在希尔伯特空间中是无穷维的矩阵, 但是如果一个厄米算符在一个子空间中闭合,那么就可以通过以下方法找到N个线性无关的本征矢.%厄米算符在
先在空间中任意选取 $n_i$ 个线性无关的正交本征矢 $v_{i1}, v_{i2}\dots v_{i n_i}$ 作为子空间的基底(本征矢的简并%未完成链接
)),并可以用基底 $v_{i1}, v_{i2}\dots v_{i n_i}$ 展开.

令 $B v_{ij} = \sum_{k=1}^{n_i} W_{jk}v_{ik}$ ( $W_{jk} = \bra{v_{ij}} B \ket{v_{ik}}$, 可以是复数), 则 $B$ 在该子空间可以表示成一个 $n_i$ 维的方形矩阵(记为 $W$ ).

以 $v_{i1}, v_{i2} \dots v_{i n_i}$ 为子空间的基底, 子空间内任意函数 $\phi  = x_1 v_{i1} + x_2 v_{i2}\dots$ 可以记为 $\ket{\phi} = (x_1, x_2, \dots, x_{n_i})\Tr$. 根据算符的矩阵表示\upref{OpMat}
, $B$ 在子空间的矩阵元就是系数 $W_{jk}$, 
\begin{equation}
W = \begin{pmatrix}
W_{11} & \ldots & W_{1 n_i}\\
\vdots & \ddots & \vdots \\
W_{n_i 1} & \ldots & W_{n_i n_i}
\end{pmatrix}
\end{equation}
所以 $B$ 在子空间范围内的本征方程的矩阵形式就是
\begin{equation}
W \ket{\phi_k} = b_{ik} \ket{\phi_k}
\end{equation}
所以 $B$ 在子空间的本征值就是 $W$ 的本征值, 本征矢就是 $W$ 的本征矢对应的波函数.

最后要证明的就是 $W$ 矩阵必然存在 $n_i$ 个本征矢.由于 $B$ 是厄米算符,  $W$ 必然是厄米矩阵, 而 $n_i$ 维的厄米矩阵必然存在 $n_i$ 个两两正交的复数本征矢和实数本征值(厄米接矩阵%链接未完成
).

综上所述, 对每一个 $n_i$ 重简并的 $a_i$,  都存在 $n_i$ 个两两正交的本征矢作为 $A$,  $B$ 算符的共同本征矢.证毕.

\subsection{具体计算}
若给出两个对易的厄米矩阵 $\mat A, \mat B$, 计算共同本征矢的方法如下:
\begin{enumerate}
\item 求 $\mat A$ 的正交归一化的本征列矢量, 组成酉矩阵 $\mat P$, 把本征值相同的列放在一起.
\item 计算 $\mat B_1 = \mat P \mat B\mat P^{-1}$ 的块对角矩阵.
\item 解 $\mat B_1$ 每一对角块的本征列矢量矩阵, 用相同顺序排列成块对角矩阵\footnote{要求块对角矩阵的本征矢矩阵\autoref{HerEig_sub2}~\upref{HerEig}, 只要对每个对角块分别解本征矢量矩阵, 再拼成块对角矩阵即可.} $\mat Q$.
\item $\mat Q\mat P$ 的每一列就是 $\mat A, \mat B$ 的一个共同本征矢.
\end{enumerate}

\addTODO{具体例子. 注意第 3 步的注释}
\begin{example}{计算对易厄米矩阵的共同本征矢}
已知两个对易的厄米矩阵 $\bvec{A,B}$ 如下
\begin{equation}
\bvec{A} = \begin{pmatrix}
1 & 0 & 0&0&0\\
0&1&2&0&0\\
0&2&1&0&0\\
0&0&0&0&-\I\\
0&0&0&\I&0
\end{pmatrix}
,
\bvec{B} = \begin{pmatrix}
2 & 0 & 0&0&0\\
0&1&0&0&0\\
0&0&1&0&0\\
0&0&0&2&-\I\\
0&0&0&\I&2
\end{pmatrix}
\end{equation}
试计算它们的共同本征矢.
\end{example}
\textbf{解:}我们按照上面的具体计算步骤来演示:

1.求 $\bvec{A}$ 的正交归一化本征列矢量:矩阵 $\bvec{A}$ 的本征方程为
\begin{equation}\label{Commut_eq1}
\vmat{\bvec{A}-\lambda}=
\vmat{
1-\lambda & 0 & 0&0&0\\
0&1-\lambda&2&0&0\\
0&2&1-\lambda&0&0\\
0&0&0&0-\lambda&-\I\\
0&0&0&\I&0-\lambda
}=0
\end{equation}

由于 $\bvec{A}$ 为分块矩阵 $\bvec{A}=\pmat[\bvec{A_1},\bvec{A_2},\bvec{A_3}]$,其中
\begin{equation}
\bvec{A_1}=\pmat{1}\quad ,\bvec{A_2}=\pmat{1&2\\2&1}\quad
\bvec{A_3}=\pmat{0&-\I\\\I&0}
\end{equation}
由分块矩阵的计算法则,\autoref{Commut_eq1} 可写成
\begin{equation}
\vmat{\bvec{A_1}-\lambda\bvec{E}}\cdot \vmat{\bvec{A_2}-\lambda\bvec{E}}\cdot\vmat{\bvec{A_3}-\lambda\bvec{E}}=0
\end{equation}


解得 $\bvec{A}$ 的本征值为: $\lambda_1=\lambda_2=1,\lambda_3=\lambda_4=-1,\lambda_5=3$

a.对 $\lambda_1=\lambda_2=1$ ,解 $(\bvec{A}-2\bvec{E})\bvec{x}=\bvec{0}$,得基础解系:
\begin{equation}
\bvec{\xi_1}=\pmat{1\\0\\0\\0\\0},\quad
\bvec{\xi_2}=\pmat{0\\0\\0\\-\I\\1}
\end{equation}
经正交归一化得
\begin{equation}
\bvec{\eta_1}=\frac{1}{\sqrt{2}}\pmat{\sqrt{2}\\0\\0\\0\\0},\quad
\bvec{\eta_2}=\frac{1}{\sqrt{2}}\pmat{0\\0\\0\\-\I\\1}
\end{equation}

b.对 $\lambda_3=\lambda_4=-1$ ,解 $(\bvec{A}-2\bvec{E})\bvec{x}=\bvec{0}$,得基础解系:
\begin{equation}
\bvec{\xi_3}=\pmat{0\\1\\-1\\0\\0},\quad
\bvec{\xi_4}=\pmat{0\\0\\0\\\I\\1}
\end{equation}
经正交归一化得
\begin{equation}
\bvec{\eta_3}=\frac{1}{\sqrt{2}}\pmat{0\\1\\-1\\0\\0},\quad
\bvec{\eta_4}=\frac{1}{\sqrt{2}}\pmat{0\\0\\0\\\I\\1}
\end{equation}
c.对 $\lambda_5=3$ ,解 $(\bvec{A}-2\bvec{E})\bvec{x}=\bvec{0}$,得基础解系:
\begin{equation}
\bvec{\xi_5}=\pmat{0\\1\\1\\0\\0}
\end{equation}
经正交归一化得
\begin{equation}
\bvec{\eta_5}=\frac{1}{\sqrt{2}}\pmat{0\\1\\1\\0\\0}
\end{equation}

组成酉矩阵 $\mat P$, 把本征值相同的列放在一起:
\begin{equation}
\bvec{P}=\frac{1}{\sqrt{2}}\begin{pmatrix}
\sqrt{2}&0&0&0&0\\0&0&1&0&1\\0&0
&-1&0&1\\0&-\I&0&\I&0\\0&1&0&1&0
\end{pmatrix}
\end{equation}

2.计算 $\mat B_1 = \mat P^{-1} \mat B\mat P$ 的块对角矩阵:
\begin{equation}
\bvec{P^{-1}}=\frac{1}{\sqrt{2}}\left(
\begin{array}{ccccc}
 \sqrt{2} & 0 & 0 & 0 & 0 \\
 0 & 0 & 1 & 0 & 1 \\
 0 & 0 & -1 & 0 & 1 \\
 0 & -\I & 0 & \I & 0 \\
 0 & 1 & 0 & 1 & 0 \\
\end{array}
\right)
\end{equation}

\begin{equation}
\mat B_1=\mat P^{-1} \mat B\mat P=\left(
\begin{array}{ccccc}
 2 & 0 & 0 & 0 & 0 \\
 0 & 3 & 0 & 0& 0 \\
 0 & 0 & 1 & 0 & 0 \\
 0 & 0 & 0 & 1 & 0 \\
 0 & 0 & 0 & 0 & 1 \\
\end{array}
\right)
\end{equation}
显然,$\bvec{B}$ 已经由酉矩阵 $\bvec{P}$ 对角化,即 $\bvec{P}$ 的每一列是 $\bvec{A,B}$ 的共同吧本征矢. 

\addTODO{结合上文说明和证明}
改用 $\mat P$ 中的基底后, 算符 $A$ 变为对角矩阵 $\mat P \mat A\mat P^{-1}$, 算符 $B$ 变为块对角矩阵 $\mat B_1$. $\mat B_1$ 中每个块的维数就是对应本征子空间的维数. 使 $\mat A$ 对角化的 $\mat P$ 的选择不止一种, 对本征值相同的所有列进行幺正变换同样能使 $\mat A$ 对角化, 而 $\mat Q\mat P$ 就是一个这样的变换.

% 组成的矩阵可以排成 $\mat R$, 每个块都是酉矩阵. 此时 $\mat B$ 对角化后为
% \begin{equation}
% \mat R\mat P \mat B\mat P^{-1}\mat R^{-1} = (\mat R\mat P) \mat B (\mat R\mat P)^{-1}
% \end{equation}
