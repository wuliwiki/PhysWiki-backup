% 丘成桐(综述)
% license CCBYSA3
% type Wiki

本文根据 CC-BY-SA 协议转载翻译自维基百科\href{https://en.wikipedia.org/wiki/Shing-Tung_Yau}{相关文章}。

\begin{figure}[ht]
\centering
\includegraphics[width=6cm]{./figures/fb16385d5284d04b.png}
\caption{} \label{fig_QCT_1}
\end{figure}
丘成桐(Shing-Tung Yau,发音:/jaʊ/;中文:丘成桐;拼音:Qiū Chéngtóng;1949年4月4日出生)是一位中美籍数学家。他是清华大学丘成桐数学科学中心的主任,同时是哈佛大学的名誉教授。直到2022年,丘成桐一直担任哈佛大学威廉·卡斯帕·格劳斯坦数学教授,之后他移居清华大学。

丘成桐1949年出生于汕头,年幼时移居英国香港,1969年移居美国。他因在偏微分方程、卡拉比猜想、正能量定理和蒙热–安培方程等方面的贡献而于1982年获得菲尔兹奖。丘成桐被认为是现代微分几何和几何分析发展的主要贡献者之一。他的工作在凸几何、代数几何、计数几何、镜像对称、广义相对论、弦理论等数学和物理领域产生了深远的影响,同时他的研究也涉及到应用数学、工程学和数值分析等领域。
\subsection{传记}  
丘成桐1949年出生于中华民国广东省汕头市,父母为客家人。[YN19] 他的祖籍是中国嘉应县。[YN19] 他的母亲梁玉兰来自中国梅县区;父亲丘镇英(Chen Ying Chiu)是中华民国国民党学者,涉猎哲学、历史、文学和经济学。[YN19] 他是家中八个孩子中的第五个。[4]

在中国大陆发生共产主义接管时,丘成桐还只有几个月大,他的家人移居到英国香港,并在那里接受教育(除了英语课外),他的学业完全用粤语,而不是父母的母语客家话。[YN19] 他直到1979年,才在华罗庚的邀请下回到大陆,那时中国大陆进入改革开放时代。[YN19] 他们最初住在元朗,1954年搬到沙田。[YN19] 由于失去了所有财产,他们家经济拮据,而他的父亲和第二个姐姐在他十三岁时相继去世。[YN19] 丘成桐开始阅读并欣赏父亲的书籍,变得更加专注于学业。完成培正中学学业后,他于1966至1969年间在香港中文大学学习数学,由于提前毕业,他未获得学位。[YN19] 他将课本留给了他的弟弟丘成栋,后者也决定主修数学。

丘成桐于1969年秋季前往加利福尼亚大学伯克利分校攻读数学博士学位。在寒假期间,他阅读了《微分几何学报》的第一期,并深受约翰·米尔诺(John Milnor)关于几何群体理论的论文启发。[5][YN19] 随后,他提出了普雷斯曼定理的一个推广,并在接下来的学期与布莱恩·劳森(Blaine Lawson)共同进一步发展了这一思想。[6] 基于这项工作,他于1971年获得博士学位,导师是陈省身(Shiing-Shen Chern)。[7]

他在普林斯顿高级研究院度过了一年后,于1972年加入石溪大学担任助理教授。1974年,他成为斯坦福大学的副教授。[8] 1976年,他在加州大学洛杉矶分校(UCLA)担任访问教授,并与物理学家郭玉云结婚,郭玉云是他在伯克利大学攻读研究生时认识的。[8] 1979年,他返回普林斯顿高级研究院,并于1980年成为该院教授。[8] 1984年,他接受了加州大学圣地亚哥分校的讲席教授职位。[9] 1987年,他搬到哈佛大学。[8][10] 2022年4月,丘成桐从哈佛大学退休,担任哈佛大学威廉·卡斯帕·格劳斯坦数学教授名誉教授。[8] 同年,他移居清华大学,担任数学教授。[8][2]

根据丘成桐的自传,他在1978年因英国领事馆撤销了他的香港居留权,而成为“无国籍人”,原因是他已拥有美国永久居留身份。[11][12] 关于1982年获得菲尔兹奖时的身份,丘成桐表示:“我很自豪地说,当我获得数学菲尔兹奖时,我没有任何国家的护照,应该被视为中国人。”[13] 丘成桐一直是“无国籍人”,直到1990年获得美国国籍。[11][14]

丘成桐与科学记者史蒂夫·纳迪斯(Steve Nadis)合作,写了几本书,包括一本非技术性的《卡拉比-丘流形与弦理论》介绍,[YN10][15] 一本哈佛大学数学系的历史,[NY13] 一本支持在中国建设环形电子正电子对撞机的书,[NY15][16][17] 一本自传,[YN19][18] 以及一本关于几何与物理关系的书。[NY24]
\subsection{学术活动}  
丘成桐对现代微分几何和几何分析的发展做出了重要贡献。正如威廉·瑟斯顿(William Thurston)在1981年所说:[19]

“我们很少有机会目睹一位数学家的工作在短短几年内影响整个研究领域的方向。在几何学领域,过去十年中最为显著的此类例子之一,就是丘成桐的贡献。”

他最广为人知的成果包括与郑守远共同解决的蒙热–安培方程的边值问题、与理查德·肖恩(Richard Schoen)共同在广义相对论的数学分析中取得的正能量定理、解决卡拉比猜想、最小曲面的拓扑理论(与威廉·米克斯(William Meeks)共同完成)、唐纳森–乌伦贝克–丘定理(与凯伦·乌伦贝克(Karen Uhlenbeck)共同完成)、以及与郑守远和李大潜(Peter Li)共同提出的丘–郑梯度估计和李–丘梯度估计(用于偏微分方程)。丘成桐的许多成果(除了其他人的成果)被编写成教科书,并与肖恩共同出版。[SY94][SY97]

除了他的研究工作,丘成桐还是多个数学研究所的创始人和主任,这些研究所大多位于中国。约翰·科茨(John Coates)评论道:“没有任何一位当代数学家能够与丘成桐在中国大陆和香港为数学活动筹款方面的成功相提并论。”[6] 在台湾国立清华大学的学术休假期间,丘成桐应高锟(Charles Kao)的邀请,在香港中文大学创建数学研究所。在几年筹款努力后,丘成桐于1993年建立了多学科的数学科学研究所,并邀请他常合作的作者郑守远担任副主任。1995年,丘成桐协助吕永祥(Yongxiang Lu)从罗尼·陈(Ronnie Chan)和陈戌源(Gerald Chan)的晨兴集团筹集资金,支持新成立的中国科学院晨兴数学中心。丘成桐还参与了浙江大学数学科学中心、清华大学数学科学中心、国立台湾大学数学中心以及三亚的数学研究所的建设。[20][21][22][23] 最近,在2014年,丘成桐筹集资金建立了哈佛大学的数学科学与应用中心(他是主任)、绿色建筑与城市中心以及免疫学研究中心。[24]

丘成桐以李政道和杨振宁曾组织的物理学会议为模型,提出了国际华人数学家大会,该大会现每三年举行一次。第一次大会于1998年12月12日至18日在晨兴数学中心举办。他还共同组织了《微分几何学报》和《数学的当前发展》年会。丘成桐是《微分几何学报》[25]、《亚洲数学杂志》[26]和《理论与数学物理进展》[27]的主编。截至2021年,他已指导超过七十位博士生。[7]

在香港,丘成桐在陈戌源的支持下设立了恒隆奖,奖励高中生。他还组织并参与了面向高中生和大学生的会议,例如2004年7月在杭州举行的“为什么学数学?向大师请教!”座谈会,以及2004年12月在香港举行的“数学的奇妙”讲座。丘成桐还共同发起了一系列关于大众数学的书籍《数学与数学人物》。

在2002年和2003年,格里戈里·佩雷尔曼(Grigori Perelman)向arXiv发布了预印本,声称证明了瑟斯顿几何化猜想,并作为特例,证明了著名的庞加莱猜想。尽管他的工作包含了许多新思想和结果,但他的证明在一些技术性论证上缺乏详细的推导。[28] 在接下来的几年里,几位数学家投入了大量时间,填补这些细节,并向数学界提供了佩雷尔曼工作的阐述。[29] 2006年8月,西尔维亚·纳萨尔(Sylvia Nasar)和大卫·格鲁伯(David Gruber)在《纽约客》上发表的一篇著名文章,将涉及丘成桐的几场职业争议公之于众。[13][14]
\begin{itemize}
\item 亚历山大·吉文塔尔(Alexander Givental)指控梁邦、刘克峰和丘成桐不正当地窃取了他在镜像对称领域解决一个著名猜想的功劳。尽管梁−刘−丘的文章出现在吉文塔尔的文章之后,这三人声称吉文塔尔的工作存在漏洞,只有在他们自己发表的文章中才填补了这些漏洞;而吉文塔尔则认为他的原始工作是完整的。纳萨尔和格鲁伯引用了一位匿名数学家的话,表示同意吉文塔尔的观点。[30]  
\item 在1980年代,丘成桐的同事萧钰堂(Yum-Tong Siu)指控丘成桐的博士生田刚(Gang Tian)抄袭了他的一些工作。当时,丘成桐为田刚辩护,否认了萧的指控。[YN19] 进入2000年代后,丘成桐开始加强对萧的指控,表示他认为田刚在普林斯顿大学和北京大学的双重职位是不道德的,尤其是因为田刚从北京大学获得的高薪,而与那些为大学作出更多贡献的教授和学生相比,显得尤为不公。[31][YN19] 《科学》杂志报道了中国此类职位的广泛现象,其中田刚和丘成桐是核心人物。[32]  
\item 纳萨尔和格鲁伯表示,丘成桐自1980年代中期以来 allegedly 未有显著学术成就,他试图通过声称朱希平(Xi-Ping Zhu)和丘成桐的前学生曹怀东(Huai-Dong Cao)解决了瑟斯顿和庞加莱猜想,从而重新获得关注,尽管这部分仅基于佩雷尔曼的一些思想。纳萨尔和格鲁伯引用丘成桐的言论,丘成桐同意他数学中心的代理主任在新闻发布会上将曹和朱各自的功劳定为三成,而佩雷尔曼只有二十五分(剩下的归理查德·汉密尔顿)。几个月后,美国国家公共广播电台(NPR)的《一切尽在讨论》中对此事件的报道回顾了新闻发布会的音频录音,发现丘成桐和代理主任并未作出类似的声明。[33]
\end{itemize}
丘成桐声称纳萨尔和格鲁伯的文章具有诽谤性,并且包含了若干不实之词,且他们没有给予他机会来陈述自己在争议中的立场。他曾考虑对《纽约客》杂志提起诉讼,声称自己遭受了职业损害,但他表示,自己最终决定,这样的行动并不明确会带来什么结果。[YN19] 他建立了一个公关网站,发布了几位数学家,包括他自己和文章中引用的另外两位数学家的回信,回应《纽约客》文章。[34]

在他的自传中,丘成桐表示,他在2006年的一些言论,比如“曹怀东和朱希平给出了庞加莱猜想证明的‘第一个完整且详细的阐述’”,应该更小心地表述。尽管他确实认为曹和朱的工作是对佩雷尔曼工作最详细和最严谨的阐述,但他表示自己应该澄清,他们“并未在任何方面超越佩雷尔曼的工作”。[YN19] 他还一直保持这样的看法(截至2019年),即佩雷尔曼证明的最后部分应该更好地为数学界所理解,同时也有可能存在一些尚未被注意到的错误。
\subsection{数学的技术贡献}
丘成桐在数学领域做出了许多重要的研究贡献,主要集中在微分几何及其在其他数学和科学领域中的应用。除了他的研究工作,丘成桐还编制了在微分几何领域具有影响力的开放问题集,这些问题集包括了既有的著名旧猜想及新的提案和问题。丘成桐在1980年代发布的两份问题清单,已经在2014年更新,并附有有关进展的注释。[35] 特别著名的是关于最小超曲面存在性的猜想和关于最小超曲面的谱几何问题。
\subsubsection{卡拉比猜想}
1978年,丘成桐通过研究复数Monge–Ampère方程,解决了1954年尤金尼奥·卡拉比提出的卡拉比猜想。[Y78a] 作为一个特例,这证明了对于任何第一陈类非正的闭合Kähler流形,存在Kähler-Einstein度量。丘成桐的方法借鉴了卡拉比、于尔根·莫泽(Jürgen Moser)和阿列克谢·波戈列洛夫(Aleksei Pogorelov)早期的工作,这些工作最初是为准线性椭圆型偏微分方程和实Monge–Ampère方程发展起来的,并将其应用于复Monge–Ampère方程的情境。[36][37][38][39]
\begin{itemize}
\item 在微分几何中,丘成桐定理在证明具有特殊全异群的闭合流形的一般存在性方面具有重要意义;根据安布罗斯-辛格定理,任何单连通的闭合Kähler流形,如果其Ricci曲率为零,那么它的全异群必定包含在特殊酉群中。[40] 多米尼克·乔伊斯(Dominic Joyce)和彼得·克罗纳海默(Peter Kronheimer)已发现其他具有特殊全异群的紧致黎曼流形的例子,尽管在其他群的情况下,尚未成功提出类似于卡拉比猜想的普遍存在性结果。[37]  
\item 在代数几何中,卡拉比提出的规范度量的存在性使得通过微分形式可以给出特征类的同样规范的代表。由于丘成桐最初通过证明卡拉比猜想会在此类背景下引发矛盾,成功推导出了对卡拉比猜想本身的引人注目的推论。[Y77] 特别地,卡拉比猜想暗示了宫冈-丘不等式(Miyaoka–Yau inequality),该不等式涉及到曲面上的陈数,并且为复射影平面及二维复单位圆盘商的复结构提供了同伦类型的表征。[36][40]  
\item 卡拉比猜想的一个特例断言,对于任何第一陈类为零的Kähler流形,必定存在一个Ricci曲率为零的Kähler度量。[36] 在弦理论中,1985年,菲利普·坎德拉斯(Philip Candelas)、加里·霍洛威茨(Gary Horowitz)、安德鲁·斯特罗明格(Andrew Strominger)和爱德华·威滕(Edward Witten)发现,由于其特殊的全异群,这些卡拉比-丘流形是超弦的适当配置空间。因此,丘成桐解决卡拉比猜想被认为在现代弦理论中具有根本性的重要性。[41][42][43]
\end{itemize}
在非紧致情形下,卡拉比猜想的理解较为不确定。甘廷和丘成桐将丘成桐对复Monge-Ampère方程的分析扩展到非紧致情形,在这种情形下,使用截断函数和相应的积分估计需要在无穷远处假设某些受控几何条件。[TY90] 这将问题简化为具有这种渐近性质的Kähler度量的存在性问题;他们为某些光滑的拟射影复流形得到了这样的度量。后来,他们扩展了他们的工作,允许存在orbifold奇点。[TY91]  

与布莱恩·格林(Brian Greene)、阿尔弗雷德·夏佩尔(Alfred Shapere)和库姆伦·瓦法(Cumrun Vafa)合作,丘成桐为某些满射全纯映射的正则点集合引入了一个Kähler度量的假设,其中Ricci曲率大致为零。[G+90] 他们能够应用甘廷-丘成桐存在性定理构造一个精确的Ricci平坦的Kähler度量。格林-夏佩尔-瓦法-丘成桐假设及其自然推广,现在被称为半平坦度量,已在Kähler几何学的若干问题分析中变得非常重要。[44][45]
\subsubsection{标量曲率与广义相对论}
正能量定理,由Yau与他的前博士生Richard Schoen合作获得,可以用物理术语描述如下:

在爱因斯坦的广义相对论中,孤立物理系统的引力能量是非负的。

然而,这实际上是微分几何和几何分析中的一个精确定理,其中物理系统通过具有某种广义标量曲率非负性的黎曼流形来建模。因此,Schoen和Yau的方法起源于他们对具有正标量曲率的黎曼流形的研究,这本身就是一个值得关注的话题。Schoen和Yau分析的起点是他们发现了一种简单而新颖的方法,将高斯-科达兹方程插入到三维黎曼流形上稳定最小超曲面的面积二次变分公式中。然后,高斯-博内定理在流形具有正标量曲率时,极大地约束了这种曲面的可能拓扑结构。[SY79a][46][47]

Schoen和Yau通过发现具有各种受控性质的稳定最小超曲面的新型构造,利用了这一观察结果。[SY79a] 他们的一些存在性结果与Jonathan Sacks和Karen Uhlenbeck的类似结果几乎同时发展,尽管采用了不同的技术。他们的基本结果是关于具有预定拓扑行为的最小浸入的存在性。作为他们与高斯-博内定理计算的结果,他们能够得出结论:某些拓扑上独特的三维流形不能具有任何标量曲率非负的黎曼度量。[48][49]

随后,Schoen和Yau将他们的工作应用于广义相对论中某些黎曼渐近平坦初始数据集的情形。他们证明了质量的负性会允许使用Plateau问题构造稳定的最小曲面,这些曲面是测地完全的。然后,他们与高斯-博内定理的计算的非紧类比提供了质量负性与逻辑矛盾。因此,他们能够在其黎曼初始数据集的特殊情况下证明正质量定理。[SY79c][50]

Schoen和Yau通过研究Pong-Soo Jang提出的偏微分方程,将这一结果扩展到正质量定理的完整洛伦兹形式。他们证明了Jang方程的解在黑洞的表面视界之外存在,在这些地方解可能会发散到无穷大。[SY81] 通过将洛伦兹初始数据集的几何与该Jang方程解的图形几何关联起来,并将后者解释为黎曼初始数据集,Schoen和Yau证明了完整的正能量定理。[50] 此外,通过逆向推导他们对Jang方程的分析,他们还能够证明,广义相对论中任何足够集中的能量必定伴随着一个表面视界的存在。[SY83]

由于使用了Gauss–Bonnet定理,这些结果最初仅限于三维黎曼流形和四维洛伦兹流形的情况。Schoen和Yau通过在具有正标量曲率的黎曼流形的最小超曲面上构造正标量曲率的黎曼度量,建立了对维数的归纳法。[SY79b] 这些最小超曲面是通过Frederick Almgren和Herbert Federer利用几何测度理论构造的,通常在大维度下并不光滑,因此这些方法仅直接适用于维度小于八的黎曼流形。在没有任何维度限制的情况下,Schoen和Yau证明了局部共形平坦流形类中的正质量定理。[SY88][36] 2017年,Schoen和Yau发布了一篇预印本,声称解决了这些困难,从而在没有维度限制的情况下证明了归纳法,并验证了任意维度的黎曼正质量定理。

Gerhard Huisken和Yau进一步研究了具有严格正质量的黎曼流形的渐近区域。Huisken早期曾发起了欧几里得空间超曲面体积保持均值曲率流的研究。[51] Huisken和Yau将他的工作适应于黎曼流形的情形,证明了该流的长期存在性和收敛性定理。作为推论,他们确立了正质量流形的新几何特征,即它们的渐近区域被常均值曲率的超曲面所覆盖。[HY96]
\subsubsection{Omori−Yau 最大值原理}  
传统上,最大值原理技术仅直接应用于紧致空间,因为在这种情况下最大值是必然存在的。1967年,Hideki Omori提出了一个新型的最大值原理,适用于那些截面曲率有下界的非紧黎曼流形。可以证明近似最大值是存在的;Omori进一步证明了在适当控制梯度和二阶导数的情况下,近似最大值的存在。Yau部分扩展了Omori的结果,只需要Ricci曲率的下界;这个结果被称为Omori−Yau最大值原理。[Y75b]这种普适性是有用的,因为Ricci曲率出现在Bochner公式中,在代数运算中通常也会使用下界。除了给出最大值原理本身的一个非常简单的证明,Shiu-Yuen Cheng和Yau还证明了在Omori−Yau最大值原理中,Ricci曲率的假设可以被某些具有可控几何性质的切割函数的存在假设所替代。[CY75][36][52][53][54]

Yau能够直接应用Omori−Yau原理来推广复分析中的经典Schwarz−Pick引理。Lars Ahlfors等人曾经将该引理推广到黎曼曲面设置中。通过他的研究方法,Yau能够考虑从一个完备的Kähler流形(具有Ricci曲率下界)映射到一个上有界的Hermitian流形的情况,该流形的全纯双截面曲率由一个负数上界所限制。[Y78b][40][54]

Cheng和Yau广泛使用他们的Omori−Yau原理变体来寻找非紧Kähler流形上的Kähler−Einstein度量,采用了Charles Fefferman提出的假设。由于Cheng和Yau只考虑具有负标量曲率的Kähler−Einstein度量,因此与Yau在Calabi猜想中的早期工作相比,连续性方法中涉及的估计并不那么困难。更微妙的问题,Fefferman早期工作变得重要,涉及到测地线完备性。特别地,Cheng和Yau能够在任何有界、平滑且严格伪凸的复欧几里得空间子集上找到具有负标量曲率的完备Kähler−Einstein度量。[CY80]这些可以被认为是双曲空间Poincaré球模型的复几何类比。[40][55]
\subsubsection{微分哈内克不等式}
姚期智最初将Omori−Yau最大值原理应用于建立若干二阶椭圆偏微分方程的梯度估计。[Y75b] 给定一个满足与拉普拉斯算子、函数和梯度值相关的各种条件的完整且平滑的黎曼流形上的函数,姚期智将最大值原理应用于各种复杂的复合表达式,以控制梯度的大小。尽管所涉及的代数运算很复杂,但姚期智证明的概念形式却异常简单。[56][52]

姚期智的新颖梯度估计被称为“微分哈内克不等式”,因为它们可以沿着任意路径积分,从而恢复出类似于经典哈内克不等式的形式,直接比较微分方程在两个不同输入点上的解值。通过利用Calabi对黎曼流形上距离函数的研究,姚期智和程修远给出了姚期智梯度估计的强大局部化,采用相同的方法简化了Omori−Yau最大值原理的证明。[CY75] 这些估计在黎曼流形上的调和函数的特定情况下被广泛引用,尽管姚期智和程修远−姚期智的原始结果涵盖了更一般的情形。[56][52]

1986年,姚期智和彼得·李利用相同的方法研究了黎曼流形上的抛物型偏微分方程。[LY86][52] 理查德·汉密尔顿将他们的结果在某些几何背景下推广到了矩阵不等式。Li−Yau不等式和Hamilton−Li−Yau不等式的类比在Ricci流的理论中具有重要意义,其中汉密尔顿证明了某些Ricci流的曲率算子的矩阵微分哈内克不等式,格里高利·佩雷尔曼证明了一个与Ricci流耦合的反向热方程解的微分哈内克不等式。[57][56]

程修远和姚期智能够利用他们的微分哈内克估计,证明在某些几何条件下,完整黎曼或伪黎曼空间的闭子流形本身是完整的。例如,他们证明了如果M是闵可夫斯基空间中的一类时空超曲面,且其拓扑是封闭的并且具有常数平均曲率,那么在M上诱导的黎曼度量是完整的。[CY76a] 类似地,他们证明了如果M是仿射空间中的一类仿射超球面,且其拓扑是封闭的,那么在M上诱导的仿射度量是完整的。[CY86] 这些结果是通过推导到给定点的(平方)距离函数的微分哈内克不等式,并沿着内在定义的路径进行积分来实现的。
\subsubsection{唐纳森−乌伦贝克−姚定理}  
1985年,西蒙·唐纳森证明了,在一个非奇异的二维复射影簇上,一个全纯向量束当且仅当该束是稳定的时,存在赫尔米特杨−米尔斯联络。姚期智和凯伦·乌伦贝克的结果将唐纳森的结论推广到允许任何维度的紧Kähler流形。[UY86] 乌伦贝克−姚方法依赖于椭圆型偏微分方程,而唐纳森的方法则使用抛物型偏微分方程,基本上与Eells和Sampson对调和映射的开创性工作平行。唐纳森和乌伦贝克−姚的结果后来被其他作者扩展。乌伦贝克和姚的文章之所以重要,是因为它清楚地说明了全纯向量束的稳定性与构造赫尔米特杨−米尔斯联络时使用的分析方法之间的关系。其基本机制是,如果赫尔米特联络的逼近序列未能收敛到所需的杨−米尔斯联络,则可以通过重新缩放使其收敛到一个子束,并且通过陈−魏尔理论可以验证该子束是不稳定的。[38][58]

像Calabi–Yau定理一样,唐纳森–乌伦贝克–姚定理在理论物理中也具有重要意义。[42] 为了适当地推广超对称的理论,安德鲁·斯特罗明格将赫尔米特杨–米尔斯条件作为他提出的斯特罗明格系统的一部分,旨在将Calabi−Yau条件扩展到非Kähler流形上。[41] 许向飞和姚期智为解决斯特罗明格系统在某些三维复流形上的问题提出了一个假设,将问题简化为复Monge−Ampère方程,并且他们成功求解了该方程。[FY08]

姚期智解决Calabi猜想的工作为如何将具有非正第一Chern类的紧复流形上的Kähler度量变形为Kähler–Einstein度量提供了一个相对完整的答案。[Y78a] 扶贵晃(Akito Futaki)证明了全纯向量场的存在可能成为将这些结果直接扩展到复流形具有正第一Chern类时的障碍。[40] Calabi曾提出过一个建议,认为在任何紧Kähler流形上,如果它们没有全纯向量场,则存在Kähler–Einstein度量。[Y82b] 在1980年代,姚期智和其他人逐渐意识到这一标准并不足够。受到唐纳森−乌伦贝克−姚定理的启发,姚期智提出Kähler–Einstein度量的存在必须与复流形在几何不变理论意义上的稳定性相关联,并且提出研究全纯向量场沿着射影嵌入的变化,而不是直接研究流形上的全纯向量场。[Y93][Y14a] 随后的研究由田刚(Gang Tian)和西蒙·唐纳森(Simon Donaldson)细化了这一猜想,这一猜想后来被称为姚–田–唐纳森猜想,涉及Kähler–Einstein度量和K-稳定性。2019年,陈秀雄、唐纳森和宋孙因解决该猜想获得了奥斯瓦尔德·维布伦奖。[59]
\subsubsection{几何变分问题}  
1982年,李和姚解决了非嵌入情形下的威尔莫尔猜想。[LY82] 更准确地说,他们证明了,给定任意平滑的闭合曲面在3-球体中的浸入,如果该浸入不是嵌入,则威尔莫尔能量下界为\(8\pi\)。这个结果得到了2012年费尔南多·马尔凯斯(Fernando Marques)和安德烈·内维斯(André Neves)的补充,他们证明了在2维圆环\(S^1 \times S^1\)的平滑嵌入的替代情况下,威尔莫尔能量下界为\(2\pi^2\)。[60] 这些结果共同构成了托马斯·威尔莫尔(Thomas Willmore)1965年提出的完整威尔莫尔猜想。尽管它们的假设和结论非常相似,但李−姚和马尔凯斯−内维斯的方法有所不同。尽管如此,它们都依赖于结构上类似的极小化方法。马尔凯斯和内维斯新颖地利用了几何测度理论中的阿尔姆格伦−皮茨(Almgren–Pitts)最小化-极大化理论;而李和姚的方法则依赖于他们新的“共形不变量”,这是一个基于狄里克雷能量的最小化-极大化量。文章的主要工作是将他们的共形不变量与其他几何量联系起来。

威廉·米克斯(William Meeks)和姚期智在三维流形中的最小曲面问题上提出了一些基础性结果,重新审视了杰西·道格拉斯(Jesse Douglas)和查尔斯·莫雷(Charles Morrey)早期工作的未解之处。[MY82][46] 在这些基础之上,米克斯、莱昂·西蒙(Leon Simon)和姚期智提出了许多关于三维黎曼流形中最小化其同调类内面积的曲面的基础性结果。[MSY82] 他们能够给出一些显著的应用。例如,他们证明了,如果\(M\)是一个可定向的三维流形,使得每个平滑的2球面嵌入都可以扩展为单位球体的平滑嵌入,那么\(M\)的任意覆盖空间也满足这一条件。有趣的是,米克斯-西蒙-姚的论文和哈密尔顿关于Ricci流的基础性论文(同年发表)有一个共同的结果,尽管它们使用了截然不同的方法:任何具有正Ricci曲率的简单连通紧三维黎曼流形都与三维球体同胚。
\subsubsection{几何刚性定理}  
在子流形的几何中,外在几何和内在几何都非常重要。它们分别通过内在黎曼度量和第二基本形式来体现。许多几何学家研究了通过限制这些数据的某种形式的常数性所引发的现象。这包括了最小曲面、常曲率和度量具有常标量曲率的子流形等问题,作为特殊情况。
\begin{itemize}
\item 这类问题的典型例子是伯恩斯坦问题,这个问题在詹姆斯·西蒙斯(James Simons)、恩里科·博比埃里(Enrico Bombieri)、恩尼奥·德·乔治(Ennio De Giorgi)和恩里科·朱斯蒂(Enrico Giusti)在1960年代的著名工作中得到了完全解决。他们的工作表明,作为欧几里得空间上的图形的最小超曲面,在低维情况下必须是平面,而在高维情况下则存在反例。[61] 证明平面性关键点是欧几里得空间低维情况下不存在圆锥形和非平面稳定最小超曲面;这一点由理查德·肖恩(Richard Schoen)、莱昂·西蒙(Leon Simon)和丘成桐(Yau)给出了简单的证明。[SSY75] 他们通过将西蒙不等式与面积二次变分公式结合的技术随后在文献中得到了多次应用。[46][62]
\item 考虑到标准伯恩斯坦问题中的“阈值”维度现象,丘成桐和邓兆辉(Shiu-Yuen Cheng)提出一个令人惊讶的事实:在洛伦兹空间的类似问题中没有维度限制:任何在多维闵可夫斯基空间中,作为欧几里得空间图形且具有零均匀曲率的时空超曲面,必须是平面。[CY76a] 他们的证明使用了他们之前用来证明微分哈纳克估计的极大值原理技术。[CY75] 后来,他们利用类似技术给出了完整抛物线或椭圆型仿射超球面的分类的新证明。[CY86]
\item 在他最早的论文之一中,丘成桐考虑了将常均曲率条件扩展到更高的余维度,其中该条件可以解释为均曲率作为法向丛的截面平行,或作为均曲率长度的常数性。在前一种解释下,他完全表征了黎曼空间形式中二维曲面的情形,并在(较弱的)第二种解释下找到了部分结果。[Y74] 其中一些结果被陈邦炎(Bang-Yen Chen)独立发现。[63]
\item 通过扩展菲利普·哈特曼(Philip Hartman)和路易斯·尼伦伯格(Louis Nirenberg)在欧几里得空间内在平坦超曲面的早期工作,丘成桐和邓兆辉考虑了具有常标量曲率的空间形式超曲面。[64] 他们分析中的关键工具是赫尔曼·魏尔(Hermann Weyl)用来解决魏尔等距嵌入问题的微分恒等式的扩展。[CY77b]
\end{itemize}
在子流形刚性问题的背景之外,丘成桐能够借鉴于尔根·莫泽(Jürgen Moser)证明Caccioppoli不等式的方法,从而证明了完备黎曼流形上函数的新刚性结果。他的一个特别著名的结果是:一个次调和函数如果既是正的又是Lp可积的,除非它是常数。[Y76][52][65] 类似地,在完备的Kähler流形上,一个全纯函数如果是Lp可积的,除非它是常数。[Y76]
\subsubsection{闵可夫斯基问题与蒙日–安佩尔方程}  
经典微分几何中的闵可夫斯基问题可以视为指定高斯曲率的问题。在1950年代,路易·尼伦伯格(Louis Nirenberg)和阿列克谢·波戈雷洛夫(Aleksei Pogorelov)解决了二维曲面的闵可夫斯基问题,利用了当时在二维领域上蒙日–安佩尔方程的最新进展。到1970年代,对于蒙日–安佩尔方程的高维理解仍然不足。1976年,丘成桐和杨卫东通过连续法解决了高维情况下的闵可夫斯基问题,采用了完全几何的估计方法,而非依赖蒙日–安佩尔方程的理论。[CY76b][66]  

作为他们解决闵可夫斯基问题的结果,丘成桐和杨卫东在理解蒙日–安佩尔方程方面取得了进展。[CY77a] 关键的观察是,蒙日–安佩尔方程的解的勒让德变换,其图形的高斯曲率可以通过一个简单的公式来指定,这个公式依赖于蒙日–安佩尔方程的“右侧”项。因此,他们能够证明蒙日–安佩尔方程的狄里赫雷问题的一般可解性,当时这个问题在二维领域之外仍然是一个重大未解问题。[66]

丘成桐和杨卫东的论文继承了波戈雷洛夫(Pogorelov)在1971年提出的一些思想,尽管他在丘成桐和杨卫东的工作发表时,公开发表的相关著作缺乏一些重要细节。[67] 波戈雷洛夫后来也发表了他的原始思想的更详细版本,问题的解决通常归功于丘成桐–杨卫东和波戈雷洛夫。[68][66] 丘成桐–杨卫东和波戈雷洛夫的方法在蒙日–安佩尔方程的文献中不再常见,因为其他作者,特别是路易斯·卡法雷利(Luis Caffarelli)、尼伦伯格(Nirenberg)和乔尔·斯普鲁克(Joel Spruck)等,已经发展出直接的技术,这些技术得出了更强有力的结果,并且不需要辅助使用闵可夫斯基问题。[68]

仿射球面自然可以通过某些蒙日–安佩尔方程的解来描述,因此它们的全面理解比欧几里得球面更加复杂,后者并不依赖于偏微分方程。在抛物线情形下,仿射球面通过孔拉德·约根斯(Konrad Jörgens)、尤金尼奥·卡拉比(Eugenio Calabi)和波戈雷洛夫的连续工作被完全分类为抛物面。椭圆型仿射球面被卡拉比识别为椭球体。双曲型仿射球面则表现出更复杂的现象。丘成桐和杨卫东证明了它们在渐近上与凸锥体相切,反过来,每个(均匀)凸锥体都对应于某个双曲型仿射球面。[CY86] 他们还能够提供卡拉比和约根斯–卡拉比–波戈雷洛夫先前分类的全新证明。[66][69]
\subsubsection{镜像对称性}   
卡拉比–尤 manifold 是一个紧致的 Kähler 流形,其 Ricci 曲率为零;作为尤的卡拉比猜想验证的特例,这类流形被证明是存在的。[Y78a] 镜像对称性是一个由理论物理学家在1980年代末提出的猜想,假设复杂维度为三的卡拉比–尤流形可以分成对,这些流形共享某些特性,如欧拉数和霍奇数。基于这一猜想的图景,物理学家菲利普·坎德拉斯(Philip Candelas)、谢尼亚·德拉·奥萨(Xenia de la Ossa)、保罗·格林(Paul Green)和琳达·帕克斯(Linda Parkes)提出了一个枚举几何的公式,该公式编码了四维复射影空间中的一般五次超曲面上任何固定度数的有理曲线的数量。梁邦(Bong Lian)、刘克峰(Kefeng Liu)和尤(Yau)给出了该公式成立的严格证明。[LLY97] 一年之前,亚历山大·吉文塔尔(Alexander Givental)发表了镜像公式的证明;根据梁邦、刘克峰和尤的说法,吉文塔尔的证明的细节只有在他们的工作发表后才得以完全补充。[30] 吉文塔尔和梁–刘–尤的证明有一些重叠,但它们是解决该问题的不同方法,并且每个方法至今都已在教科书中有详细的阐述。[70][71]

吉文塔尔和梁–刘–尤的工作证实了更基本的镜像对称性猜想中的一个预测,即三维卡拉比–尤流形如何成对配对。然而,他们的工作并不依赖于该猜想本身,因此对其有效性没有直接影响。与安德鲁·斯特罗明格(Andrew Strominger)和埃里克·扎斯洛(Eric Zaslow)一起,尤提出了一个几何图像,说明镜像对称性如何能够系统地理解并证明为真。[SYZ96] 他们的观点是,一个三维复杂度的卡拉比–尤流形应该由特殊拉格朗日圆环组成,这些圆环是六维黎曼流形(作为卡拉比–尤结构的基础)中的某些类型的三维最小子流形。镜像流形则通过具有对偶叶层的方式来表征。在1996年以来,斯特罗明格–尤–扎斯洛(SYZ)提议已被以各种方式修改和发展。它提供的概念图像对镜像对称性的研究产生了重要影响,目前关于其各个方面的研究仍是一个活跃的领域。它可以与马科西姆·孔采维奇(Maxim Kontsevich)提出的替代理论——同调镜像对称性猜想进行对比。SYZ猜想的观点侧重于卡拉比–尤空间中的几何现象,而孔采维奇的猜想则将问题抽象为处理纯代数结构和范畴理论的方式。[37][44][70][71]
\subsubsection{比较几何}
在尤与布莱恩·劳森(Blaine Lawson)合著的早期论文中,他们发现了一些关于具有非正曲率的闭黎曼流形的拓扑学的基本结果。[LY72] 他们的平坦圆环定理通过基本群的代数特征来刻画平坦且完全测地浸入的圆环的存在。分裂定理则指出,基本群作为一个最大非交换直接积的分裂意味着流形本身的等距分裂。德特列夫·格罗莫尔(Detlef Gromoll)和约瑟夫·沃尔夫(Joseph Wolf)在同一时期也得出了类似的结果。[72][73] 他们的结果已扩展到具有非正曲率的度量空间上等距群作用的更广泛背景中。[74]

杰夫·奇格(Jeff Cheeger)和尤研究了黎曼流形上的热核。他们建立了黎曼度量的特殊情形,其中测地球面具有常数平均曲率,并证明这种度量由热核的径向对称性来表征。[CY81] 在旋转对称度量的特化情况下,他们使用指数映射将热核迁移到一般黎曼流形上的一个测地球中。假设对称“模型”空间低估了流形本身的里奇曲率,他们进行了直接计算,证明结果函数是热方程的下解。因此,他们得到了关于一般黎曼流形热核的下估计,基于其里奇曲率的下界。[75][76] 在非负里奇曲率的特殊情况下,彼得·李(Peter Li)和尤利用他们的梯度估计来放大并改进了奇格−尤估计。[LY86][52]

尤的一个著名结果,由卡拉比独立得到,表明任何非紧黎曼流形,如果具有非负里奇曲率,则其体积增长率至少为线性增长。[Y76][52] 后来,奇格、米哈伊尔·格罗莫夫(Mikhael Gromov)和迈克尔·泰勒(Michael Taylor)使用毕晓普–格罗莫夫不等式而非函数理论,找到了第二个证明。
\subsubsection{谱几何}
给定一个光滑的紧黎曼流形(有无边界),谱几何研究拉普拉斯–贝尔特拉米算子的特征值。如果流形有边界,则特征值与边界条件的选择相关,通常是迪里希雷条件或诺伊曼条件。保罗·杨(Paul Yang)和尤证明,在封闭二维流形的情况下,第一特征值由一个显式公式上界,该公式仅依赖于流形的属数和体积。[YY80][46] 更早的时候,尤修改了杰夫·奇格(Jeff Cheeger)对奇格常数的分析,从而能够根据几何数据估计第一特征值的下界。[Y75a][77]

在1910年代,赫尔曼·魏尔(Hermann Weyl)证明,在平面上一个光滑的有界开集上,若施加迪里希雷边界条件,则特征值的渐近行为完全由区域所包含的面积决定。这个结果被称为魏尔法则(Weyl's law)。1960年,乔治·波利亚(George Pólya)猜测魏尔法则实际上不仅控制了特征值的渐近分布,还控制了每一个特征值。李和尤证明了波利亚猜想的一个弱版本,获得了对特征值平均值的控制,通过魏尔法则中的表达式来实现。[LY83][78]

1980年,李和尤发现了一些新的拉普拉斯–贝尔特拉米特征值不等式,所有这些不等式都基于最大值原理和五年前由尤和程–尤提出的微分哈纳克估计。[LY80] 他们基于几何数据的下界结果尤其著名,[79][56][52] 并且是首个无需任何条件假设的类似结果。[80] 大约在同一时期,米哈伊尔·格罗莫夫(Mikhael Gromov)通过等周方法获得了类似的不等式,尽管他的结果比李和尤的弱。[75] 在与伊萨多尔·辛格(Isadore Singer)、温·邦(Bun Wong)和邵兴同(Shing-Toung Yau)的合作中,尤使用李–尤的方法,建立了前两特征函数比值的梯度估计。[S+85] 类似于尤通过梯度估计的积分找到哈纳克不等式的方法,他们能够通过积分梯度估计来控制基本间隙,即前两特征值之间的差异。辛格–温–尤–尤的工作启动了一系列研究,其中不同作者发现并改进了关于基本间隙的新估计。[81]

1982年,尤确定了光谱几何学中十四个感兴趣的问题,包括上述波利亚猜想。[Y82b] 尤的一个特别猜想,关于通过特征值的大小控制特征函数的水平集大小,已由亚历山大·洛古诺夫(Alexander Logunov)和尤金尼娅·马利尼科娃(Eugenia Malinnikova)解决,并且由于他们的工作,他们在2017年获得了克雷研究奖的一部分。[82]
\subsubsection{离散与计算几何学}
顾贤峰和尤考虑了二维流形之间共形映射的数值计算(以离散网格的形式呈现),特别是均匀化映射的计算,这与均匀化定理的预测相一致。在零基数表面情况下,映射是共形的,当且仅当它是调和的,因此,顾和尤通过直接最小化离散化的狄里希雷能量来计算共形映射。[GY02] 在更高基数的情况下,均匀化映射通过其梯度进行计算,这些梯度是通过霍奇理论的闭合和调和1-形式来确定的。[GY02] 主要工作是确定经典理论的数值有效离散化方法。他们的方法足够灵活,可以处理带边界的一般表面。[GY03][83] 与Tony Chan、Paul Thompson和Yalin Wang合作,顾和尤将他们的工作应用于匹配两个人脑表面的问题,这是医学影像中的一个重要问题。在最相关的零基数情况下,共形映射在莫比乌斯群的作用下才是良定义的。通过进一步优化测量脑部标志物(如中央沟)不匹配的狄里希雷类型能量,他们获得了由这些神经特征良定义的映射。[G+04]

在图论领域,Fan Chung 和尤广泛发展了黎曼几何中的概念和结果的类比。这些关于微分哈恩不等式、索博列夫不等式和热核分析的结果,部分是在与 Ronald Graham 和 Alexander Grigor'yan 合作的过程中取得的,后来以教科书的形式写入了她著名的《谱图理论》一书的最后几章。[84] 此后,他们为图引入了一个格林函数,等同于图拉普拉斯算子的伪逆。[CY00] 他们的工作自然适用于研究随机游走的击中时间和相关话题。[85][86]

为了找到其结果的一般图论背景,Chung 和 Yau 引入了图的 Ricci 平坦性的概念。[84] 一个更灵活的 Ricci 曲率概念,处理度量空间上的马尔可夫链,后来由 Yann Ollivier 引入。Yong Lin、Linyuan Lu 和 Yau 在图论的特殊背景下发展了 Ollivier 定义的一些基础理论,例如考虑 Erdős–Rényi 随机图的 Ricci 曲率。[LLY11] Lin 和 Yau 还考虑了 Dominique Bakry 和 Michel Émery 早期引入的曲率–维度不等式,并将其与 Ollivier 曲率及 Chung–Yau 的 Ricci 平坦性概念进行了关联。[LY10] 他们还能够证明在局部有限图的情况下,Bakry–Émery 和 Ollivier 曲率的普遍下界。[87]