% 纠缠
% keys 纠缠|entanglement|施密特秩

\pentry{约化密度矩阵\upref{partra}}

我们研究一个二元纯态 $\ket{\psi}_{AB}$ 的子系统 $A$,假设整个大的孤立系统的 Hilbert 空间可以表示为两个子 Hilbert 空间的张量积:$\mathcal{H}_A\otimes \mathcal{H}_B$,其中 $\mathcal A$ 是待研究的子系统的 Hilbert 空间,设约化密度算符为 $\rho_A=\text{tr}_B \ket{\psi}\bra{\psi}$.在约化密度矩阵\upref{partra}词条中,我们证明了 $A,B$ 处于纠缠态的一个判据 \autoref{partra_eq2}~\upref{partra}.

\begin{equation}
\text{tr} \rho^2 <1
\end{equation}

它意味着约化密度算符在某个正交完备基下对角化以后,表现为 $\mathcal{H}_A$ 中若干个纯态(大于一个)组成的系综,系综中每个纯态有 $p_a<1$ 的概率出现.这也意味着\textbf{当且仅当} $\rho_A$ 的\textbf{施密特秩}\footnote{类似于矩阵的秩\upref{MatRnk},可以将这一概念推广到任意线性算符.}大于 $1$, $A,B$ 处于纠缠态.在此处我们考察的约化密度算符是正定算符,因此施密特秩等于正的特征值的个数.若特征值个数 $>1$,体系处于纠缠态,我们称 $A,B$ 之间具有\textbf{量子相关性}.

如果施密特秩为 $1$,那么约化密度算符可以表示为 ${}_A\ket{\varphi}\bra{\varphi}_A,\varphi_A\in \mathcal{H}$,此时 $A$ 与 $B$ 之间是不纠缠的,或者被称为\textbf{可分的}(seperable).此时二元纯态 $\ket{\psi}$ 实际上可以表示为两个子系统的量子态的张量积:
\begin{equation}
\ket{\psi}_{AB}=\ket{\varphi}_A\otimes \ket{\varphi}_B
\end{equation}
