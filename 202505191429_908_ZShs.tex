% 指数函数(综述)
% license CCBYSA3
% type Wiki

本文根据 CC-BY-SA 协议转载翻译自维基百科\href{https://en.wikipedia.org/wiki/Exponential_function}{相关文章}。

在数学中,\textbf{指数函数}是唯一一个将零映射为一且其导数在所有点上都等于自身的实函数。变量 $x$ 的指数函数记作 $\exp x$ 或 $e^x$,这两种记法可以互换使用。之所以称其为“指数”,是因为其自变量可以看作是某个常数 $e \approx 2.718$(即底数)的幂指数。指数函数有多种定义方式,尽管它们的形式迥异,但在数学意义上是等价的。

指数函数可以将加法转化为乘法:它将加法的单位元 0 映射为乘法的单位元 1,并且满足加法转乘法的性质,即$\exp(x + y) = \exp x \cdot \exp y$。它的反函数是自然对数函数,记作 $\ln$ 或 $\log$,它则将乘法转化为加法:$\ln(x \cdot y) = \ln x + \ln y$。

指数函数有时被称为自然指数函数,以对应自然对数的名称,用以区别于其他也常被称作“指数函数”的一些函数。这些函数包括形如$f(x) = b^x$的函数,即以固定底数 $b$ 的幂函数。更一般地,尤其在实际应用中,形如$f(x) = a b^x$的函数也被称为指数函数。这些函数被称为“指数增长”或“指数衰减”,是因为当 $x$ 增加时,函数 $f(x)$ 的变化速率与它当前的取值成正比。

指数函数可以推广到接受复数作为自变量。这一推广揭示了复数乘法、复平面中的旋转以及三角函数之间的内在联系。欧拉公式$\exp(i\theta) = \cos\theta + i\sin\theta$正是这些关系的集中表达和总结。

指数函数甚至还可以进一步推广到其他类型的自变量,比如矩阵以及李代数中的元素。
\subsection{图像}
函数 $y = e^x$ 的图像是一个向上增长的曲线,并且其增长速度快于任何幂函数 $x^n$ 的增长速度。\(^\text{[1]}\)该图像始终位于 $x$ 轴之上,但在 $x$ 取非常大的负值时会无限接近于 $x$ 轴,因此 $x$ 轴是其一条水平渐近线。方程$\frac{d}{dx}e^x = e^x$意味着该图像上任一点的切线斜率,等于该点的函数值(即该点的 $y$ 坐标)。
\subsection{定义与基本性质}
指数函数有多种形式迥异但在数学上等价的定义方式。
\subsubsection{微分方程}
\begin{figure}[ht]
\centering
\includegraphics[width=6cm]{./figures/ad33dd2da36528ba.png}
\caption{指数函数的导数等于其函数值。由于导数表示切线的斜率,这意味着图中的所有绿色直角三角形的底边长度为 1。} \label{fig_ZShs_1}
\end{figure}
最简单的定义之一是:指数函数是唯一一个可导函数,其导数等于自身,并且在变量为 0 时取值为 1。

这个“概念性”定义需要证明存在性和唯一性,但它使得指数函数的主要性质可以很容易地推导出来。

唯一性:若 $f(x)$ 和 $g(x)$ 是两个满足上述定义的函数,那么根据商法则,函数 $f/g$ 的导数在所有点上都为零。这意味着 $f/g$ 是一个常数函数;又因为 $f(0) = g(0) = 1$,所以这个常数为 1,即 $f(x) = g(x)$。

存在性将在接下来的两个小节中分别加以证明。
\subsubsection{自然对数的反函数}
指数函数是自然对数函数的反函数。根据反函数定理,自然对数函数存在反函数,而这个反函数正好满足前述指数函数的定义,这构成了指数函数存在性的第一种证明方式。因此,有如下恒等式:
$$
\ln(\exp x) = x \\
\exp(\ln y) = y~
$$
对所有实数 $x$ 和所有正实数 $y$ 都成立。
\subsubsection{幂级数}
指数函数可以表示为以下幂级数的和:\(^\text{[2][3]}\)
$$
\exp(x) = 1 + x + \frac{x^2}{2!} + \frac{x^3}{3!} + \cdots = \sum_{n=0}^{\infty} \frac{x^n}{n!}~
$$
\begin{figure}[ht]
\centering
\includegraphics[width=6cm]{./figures/751f7b528a566372.png}
\caption{指数函数(蓝色曲线)及其幂级数前 $n + 1$ 项的部分和(红色曲线)} \label{fig_ZShs_2}
\end{figure}
其中 $n!$ 表示$n$的阶乘(即前$n$个正整数的乘积)。根据比值判别法,该级数在任意 $x$ 上都绝对收敛。因此,可以逐项求导,从而得出该级数的和满足之前给出的指数函数定义。

这就构成了指数函数存在性的第二种证明,并且作为推论,还说明了指数函数在任意 $x$ 上都有定义,且在每一点都等于其 Maclaurin 级数的和。
\subsubsection{函数方程}
指数函数满足如下函数方程:
$$
\exp(x + y) = \exp(x) \cdot \exp(y)~
$$
这个结论源于指数函数的唯一性,以及函数$f(x) = \exp(x + y)/\exp(y)$也满足前面提到的指数函数定义。

可以证明,任何满足该函数方程的函数,如果它是连续的或单调的,则一定具有如下形式:$x \mapsto \exp(c x)$其中 $c$ 为常数。这样的函数必然是可导的,并且当其在 0 处的导数为 1 时,就与标准的指数函数相等。
\subsubsection{整数幂的极限}
指数函数可以表示为当整数 $n \to \infty$ 时的极限:\(^\text{[4][3]}\)
$$
\exp(x) = \lim_{n \to +\infty} \left(1 + \frac{x}{n}\right)^n~
$$
利用对数函数的连续性,可以通过取对数来证明这一点,方法是证明:
$$
x = \lim_{n \to \infty} \ln\left( \left(1 + \frac{x}{n} \right)^n \right) = \lim_{n \to \infty} n \ln\left(1 + \frac{x}{n} \right)~
$$
例如可以通过泰勒定理来证明该极限关系。
\subsubsection{性质}
倒数:由函数方程可得$e^x \cdot e^{-x} = 1$因此,对任意实数 $x$,都有e^x \neq 0,$$\frac{1}{e^x} = e^{-x}~$$正值性:对所有实数 $x$,都有$e^x > 0$这一点可以由介值定理推出。由于 $e^0 = 1$,如果存在某个 $x$ 满足 $e^x < 0$,那么在区间 $[0, x]$ 上将存在某个点 $y$ 使得 $e^y = 0$。但因为指数函数恒等于其导数,它是严格单调递增的,不可能在某一点取得零值,从而得出矛盾,因此 $e^x$ 必为正值。

将幂运算扩展到任意正实数底数:设 $b$ 为一个正实数。由于指数函数与自然对数互为反函数,有
$b = \exp(\ln b)$若 $n$ 是整数,由对数的函数方程可得:
$$
b^n = \exp(\ln b^n) = \exp(n \ln b)~
$$
由于最右边的表达式在 $n$ 为任意实数时都有定义,这就允许我们对任意正实数 $b$ 和任意实数 $x$ 定义
$$
b^x = \exp(x \ln b)~
$$
特别地,当底数 $b$ 取自然常数 $e = \exp(1)$ 时,有$\ln e = 1$因此
$$
e^x = \exp(x)~
$$
这说明 $e^x$ 与 $\exp(x)$ 这两种指数函数的记法是等价的。
\subsection{一般指数函数}
一个函数若具有如下形式$x \mapsto b^x$即通过固定底数 $b$,让指数 $x$ 变化而得到的函数,通常被称为一个指数函数(注意使用不定冠词 “an exponential function”)。

更一般地,特别是在应用场景中,“指数函数”一词通常指形如$f(x) = a b^x$的函数。这样定义的合理性在于:如果函数的取值代表某种物理量,那么单位的变更会改变常数 $a$ 的数值,因此强行规定 $a = 1$ 是不合理的。

这些最一般形式的指数函数是满足以下等价刻画的可导函数:
\begin{itemize}
\item 存在常数 $a$ 和 $b > 0$,使得对任意 $x$ 有:$f(x) = a b^x$
\item 存在常数 $a$ 和 $k$,使得对任意 $x$ 有:$f(x) = a e^{k x}$
\item 函数导数与自身之比$f'(x)/f(x)$与 $x$ 无关,即该比值是一个常数。
\item 对于任意实数 $d$,比值$f(x + d)/f(x)$与 $x$ 无关;也就是说,对所有 $x, y^\text{[5]}$ 都有:
$$
\frac{f(x + d)}{f(x)} = \frac{f(y + d)}{f(y)}~
$$
\end{itemize}
\begin{figure}[ht]
\centering
\includegraphics[width=6cm]{./figures/37013190f33ecbf0.png}
\caption{以底数 2 和 $1/2$ 的指数函数} \label{fig_ZShs_3}
\end{figure}
指数函数的底数,是将其写成形式$x \to a b^x$时幂运算中的底数,即 $b$。\(^\text{[6]}\)在其他几种刻画中,底数分别为:第二种形式中的底数为 $e^k$;第三种刻画中为$\exp\left( \frac{f'(x)}{f(x)} \right) $;最后一种刻画中为$\left( \frac{f(x+d)}{f(x)} \right)^{1/d}$。
\subsubsection{在实际应用中}
最后一种刻画方式在经验科学中非常重要,因为它可以直接通过实验检验一个函数是否为指数函数。

所谓的指数增长或指数衰减,即变量的变化速率与变量的当前值成正比,通常使用指数函数建模。例如:无限制的人口增长可能导致马尔萨斯灾难、连续复利计算、放射性衰变等现象。

若用于建模的函数具有如下形式$x \mapsto a e^{kx}$或者等价地,是微分方程$y' = k y$的解,那么常数 $k$ 会根据具体语境被称为衰减常数、崩解常数\(^\text{[7]}\)、速率常数\(^\text{[8]}\),或转化常数\(^\text{[9]}\)。
\subsubsection{等价性的证明}
为了证明上述各性质之间的等价性,可以按如下方式进行:

前两种刻画是等价的,因为如果$b = e^k \quad \text{且} \quad k = \ln b$那么就有
$$
e^{k x} = (e^k)^x = b^x~
$$
因此,形式 $a b^x$ 与 $a e^{k x}$ 是完全等价的。

而指数函数的基本性质(即导数等于自身、以及满足函数方程)则直接推出第三种和最后一种刻画。

假设第三个条件成立,令$k = \frac{f'(x)}{f(x)}$为一个常数。由于$\frac{\partial}{\partial x} e^{k x} = k e^{k x}$
根据求导的商法则,有
$$
\frac{\partial}{\partial x} \left( \frac{f(x)}{e^{k x}} \right) = 0~
$$
因此存在某个常数 $a$,使得$f(x) = a e^{k x}$

如果最后一个条件成立,令$\varphi(d) = f(x + d)/f(x)$该表达式与 $x$ 无关。由于$\varphi(0) = 1$我们可以写成:
$$
\frac{f(x + d) - f(x)}{d} = f(x)\frac{\varphi(d) - \varphi(0)}{d}~
$$
当 $d \to 0$ 时,右边趋于$f(x) \cdot \varphi'(0)$因此导数存在,且$\frac{f'(x)}{f(x)} = \varphi'(0) = k$这就验证了第三个条件,且得出$f(x) = a e^{k x}$对于某个常数 $a$,并且$\varphi(d) = e^{k d}$进一步推导可得:
$$
\left( \frac{f(x + d)}{f(x)} \right)^{1/d} = e^k~
$$
此表达式不依赖于 $x$ 和 $d$。
\subsection{复利}
指数函数最早出现在雅各布·伯努利1683 年对复利的研究中。\(^\text{[10]}\)正是在这项研究中,伯努利引入了如下极限表达式:
$$
\lim_{n \to \infty} \left(1 + \frac{1}{n}\right)^n~
$$
这一极限数后来被称为欧拉常数,记作 $e$。

在计算连续复利时,指数函数如下地参与其中。

如果本金为 1,年利率为 $x$,且按每月复利计算,则每月所获利息为当前金额的 $\frac{x}{12}$,因此每月总金额都会乘以 $(1 + \frac{x}{12})$,一年结束时的总金额为$(1 + \frac{x}{12})^{12}$如果改为按日复利,则总金额为$(1 + \frac{x}{365})^{365}$当一年中的计息周期数趋于无限多时,就得到指数函数的极限定义:
$$
\exp x = \lim_{n \to \infty} \left(1 + \frac{x}{n} \right)^n~
$$
该极限表达式最早由莱昂哈德·欧拉给出。\(^\text{[4]}\)
