% 电阻、欧姆定律、电阻率、电导率
% keys 电阻|导体|电流|电压|欧姆定律

\pentry{电流\upref{I}, 电压\upref{Voltag}}

\footnote{参考 Wikipedia 的 \href{https://en.wikipedia.org/wiki/Resistor}{Resistor} 和 \href{https://en.wikipedia.org/wiki/Ohm's_law}{Ohm's law} 页面。}\textbf{电阻器(resistor)}是电路中常见的元件。 该元件通常由一个参数, \textbf{电阻(resistance)}, 来衡量。 在不至于混淆的情况下, 也可以把电阻器简称为电阻。

理想情况下, 电阻器符合\textbf{欧姆定律(Ohm's law)}
\begin{equation}\label{eq_Resist_5}
U = IR~,
\end{equation}
其中 $U$ 是电阻器两端的电压(电流流入的一端电势较高), $I$ 是流经电阻器的电流, $R$ 是电阻器的电阻($R > 0$)。

\begin{figure}[ht]
\centering
\includegraphics[width=8cm]{./figures/bd6c34d08b32315d.pdf}
\caption{电阻器在电路中的符号} \label{fig_Resist_3}
\end{figure}

\subsection{被动符号规定}\label{sub_Resist_1}
在电路中, 一种常见的符号规定是给电路中的每一条支路预先定义一个\textbf{正方向}, 若电流延正方向流动, 那么电流为正, 反之为负。 若电势延正方向降低, 则电压为正, 反之为负。 这种规定叫做\textbf{被动符号规定(passive sign convention)}, 因为当功率 $W = IU$ 为正时, 元件从电路中获取能量。

欧姆定律的\autoref{eq_Resist_5} 可以按被动符号规定来理解: 如\autoref{fig_Resist_2}, 如果规定电阻所在的电路支路从左到右为正, 那么当电流从左向右流时, $I > 0$, 从右到左时 $I < 0$; 电压 $U_{AB} = U_A - U_B$, 即当电势左高右低时取正值, 反之取负值。

\begin{figure}[ht]
\centering
\includegraphics[width=5cm]{./figures/8df3f9b9e242896b.pdf}
\caption{符号规范} \label{fig_Resist_2}
\end{figure}

由于 $R > 0$, 所以根据\autoref{eq_Resist_5}, $I, U$ 的符号始终相同, 所以电阻从电路中获得的功率为 $W = IU > 0$。

\subsection{电阻率}
作为一个理想模型, 柱形(例如长方体, 圆柱体)电阻器的电阻为
\begin{equation}
R = \frac{\varrho L}{S} 
\end{equation}
其中 $S$ 为电阻的横截面积, $L$ 为电阻的长度, $\varrho$ 为材料的\textbf{电阻率(electrical resistivity)}。 该公式假设内部电流密度处处相等且于柱体平行。 电阻率体现了材料电阻能力, 和材料性质有关, 也可能会随温度,压强,光照(例如光敏电阻),等环境因素变化。 我们也通常把电阻率的倒数 $1/\varrho$ 叫做\textbf{电导率(electrical conductivity)}。

微观形式的欧姆定律为
\begin{equation}
\bvec E = \varrho\bvec j
\end{equation}
这个公式告诉我们, 电阻材料中某点的电流密度与电场成正比。

\subsection{电阻的简单模型}
\begin{figure}[ht]
\centering
\includegraphics[width=3.4cm]{./figures/4feed6580efe6426.pdf}
\caption{电阻的简单模型} \label{fig_Resist_1}
\end{figure}

我们这里用一个简单的经典力学模型推导上文中的概念和公式, 但严格来说, 这个推导需要使用量子力学和半导体理论。 假设一段电阻中,自由电子的电荷密度为 $-\rho$ ($\rho > 0$) 为定值。

当我们在电阻两端施加电压时, 内部会产生匀强电场,
\begin{equation}\label{eq_Resist_6}
E = \frac UL
\end{equation}
使电子受到电场力
\begin{equation}
\bvec F = -e\bvec E
\end{equation}
同时,电子还受到一个源于电子与导体缺陷碰撞的阻力,其大小与电子速度成正比,即
\begin{equation}
\bvec f = -\alpha \bvec v
\end{equation}
$\alpha$即比例系数,总有$\alpha>0$

\begin{figure}[ht]
\centering
\includegraphics[width=6cm]{./figures/78bc011b429d130b.pdf}
\caption{经典理论认为电阻源自于导体中电子与缺陷等的频繁相撞。电子先在外电场下加速,然后发生碰撞,这使电子的定向动能转换为热能;然后电子继续在外电场下加速,再次相撞...这个过程相当于导体对电阻施加阻力。} \label{fig_Resist_4}
\end{figure}

电子在该电场力下加速(由于电子质量很小, 加速过程很快, 可以假设是一瞬间完成的), 直到阻力等于电场力时加速停止, 进行匀速运动。 于是有
\begin{equation}
-e\bvec E - \alpha \bvec v = \bvec 0 \Rightarrow \bvec v = -\frac{e}{\alpha}\bvec E
\end{equation}
所以电阻内电流密度大小为
\begin{equation}\label{eq_Resist_4}
\bvec j = -\rho \bvec v = \frac{e\rho}{\alpha}\bvec E
\end{equation}
电流大小为
\begin{equation}
I = jS = \frac{\rho EeS}{\alpha}
\end{equation}
代入\autoref{eq_Resist_6} 
\begin{equation}\label{eq_Resist_3}
U = I \frac{\alpha L}{\rho eS}
\end{equation}
我们定义\textbf{电阻率}为
\begin{equation}\label{eq_Resist_1}
\varrho = \frac{\alpha}{\rho e}
\end{equation}
然后再根据电阻率定义\textbf{电阻}为
\begin{equation}\label{eq_Resist_2}
R = \frac{\varrho L}{S}
\end{equation}
可见它与长度 $L$ 成正比, 与横截面成反比。 将\autoref{eq_Resist_1} 代入\autoref{eq_Resist_2} 再代入\autoref{eq_Resist_3}, 可得\textbf{欧姆定律}
\begin{equation}
U = IR
\end{equation}
\autoref{eq_Resist_4} 也可以使用电阻率记为
\begin{equation}
\bvec E = \varrho\bvec j
\end{equation}
这相当于欧姆定律的微观形式。
