% 弦的作用量原理
% license Usr
% type Tutor


\subsection{点粒子的作用量}
对于一个 $0$ 维对象(粒子),可以用一个 $(D-1)$ 维的对象(函数)以及一个参数时间来刻画运动:$\vec X(X^0)$。协变地,进行参数化,引入参数 $\tau$ 而写为 $X^\mu(\tau)$。物理真实对于参数 $\tau$ 的选取应该是不依赖的,即
\begin{equation}
	X^\mu(\tau) = X'~^\mu(\tau') ~.
\end{equation}

遵循庞加莱群的庞加莱对称性,作用量也要不依赖于参数选取,最简单的是选取作用量是粒子\textbf{世界线}的长度\footnote{即在 $X^0-\vec X$ 图像中运动的长度,曲线用 $\tau$ 参数化。},即 
\begin{equation}
	S_{point ~particle} = S_{pp} = -m \int \dd \tau \sqrt{- \dot X^\mu \dot X_\mu} ~,
\end{equation}
其中 $-m$ 是为了保证作用量量纲正确。

Then,定义协变速度(已归一化) $u^\mu = \dot X^\mu \sqrt{-\dot X^\nu \dot X_\nu}$。而对作用量变分的结果是 
\begin{equation}
	\delta S_{pp} = -m \int \dd \tau \dot u_{\mu} \delta X^\mu ~.
\end{equation}

现在,引入一个世界线上的度规 $\gamma_{\tau \tau}(\tau)$\footnote{这与世界线是无关的。},则会有标架 $\eta(\tau) = \sqrt{- \gamma_{\tau \tau}(\tau)}$。而作用量变为
\begin{equation}
	S_{pp}' = \frac{1}{2} \int (\eta^{-1} \dot X^\mu \dot X_\mu - \eta m^2) \dd \tau ~.
\end{equation}

此时作用量仍然有:世界线重参数化不变性,和庞加莱对称性。$\eta$ 换系符合 $\eta'(\tau') \dd \tau' = \eta(\tau) \dd \tau$。而此时运动方程通过变分作用量\footnote{结果是 \\ $\delta S_{pp} = \frac{1}{2} \int \dd \tau (- \dot X^\mu \dot X_\mu/\eta^2 - m^2) \delta \eta$。}变为 $\eta^2 = -\dot X^\mu \dot X_\mu / m^2$。

虽然 $S_{pp}$ 与 $S_{pp}'$ 等级啊,但应用上,$S_{pp}'$ 是导数的二次型,更好操作,故路径积分理论中一般使用 $S_{pp}'$。

\subsection{弦的 Nambu-Goto 作用量}

下面考虑一个 $1$ 维的对象的运动,将扫出一个面,这就是\textbf{世界面}(world-sheet)。用 $\sigma$ 参数衡量运动的对象的内部坐标,例如 $\sigma \in [0, l]$;而仍用 $\tau$ 作为世界面的参数。

2-dim world-sheet 用 $X^\mu(\tau, \sigma)$ 记录。考虑一个 \textbf{induced metric}\footnote{induced metric:诱导度规。}:$h_{ab}$,$ab$ 在 $(\tau, \sigma)$ 取遍: $h_{ab} = \partial_a X^\mu \partial_b X_\mu$。这样就会给出 \textbf{Nambu-Goto 作用量}。\footnote{实际上类比点粒子,弦的作用量是 world-sheet 的面积。}
\begin{theorem}{}
	\begin{equation}
		\begin{aligned}
			S_{NG} &= \int \dd \tau \dd \sigma \mathcal L_{NG} \\
			\mathcal L_{NG} &= -\frac{1}{2 \pi \alpha'} \sqrt{-\det h_{ab}} ~.
		\end{aligned}
	\end{equation}
\end{theorem}
其中,$\alpha '$ 是 \textbf{Regge 斜率},给出弦的张力 $T = \frac{1}{2\pi \alpha'}$。事实上,一般取 $\sigma^a = (\sigma^0, \sigma^a) = (\tau, \sigma)$,则写出 
\begin{lemma}{}
	\begin{equation}
		h_{\alpha \beta} = \left(\begin{matrix}
			\dot X^2 & \dot X \cdot X' \\
			\dot X \cdot X' & X'~^2 
		\end{matrix}\right) ~, ~ \det(h_{\alpha \beta}) = (\dot X)^2 (X')^2 - (\dot X \cdot X')^2 ~,
	\end{equation}
	其中,$\dot X = \partial_\tau X$,$X' = \partial_\sigma X$。
\end{lemma}

\begin{theorem}{}
	弦的 Nambu-Goto 作用量有以下两个对称性:
	\begin{enumerate}
		\item 庞加莱对称性:$X'~^\mu(\tau, \sigma) = \Lambda_\nu^\mu X^\mu(\tau, \sigma) + a^\mu$。
		\item 微分同胚的参考系变换,称为 diff invariance,$X'~^\mu(\tau', \sigma') = X^\mu(\tau, \sigma)$。
	\end{enumerate}
\end{theorem}

\subsection{弦的 Polyakov 作用量}
\begin{theorem}{}
	现在取一个有 Lorentz 符号的度规 $\gamma_{ab}(\tau, \sigma) = \text{diag}(-, +)$ 于 world-sheet 上,作用量成为 \textbf{Polyakov 作用量}:
	\begin{equation}
		S_P [X, \gamma] = -\frac{1}{4 \pi \alpha'} \int \dd \tau \dd \sigma \sqrt{-\gamma} \gamma^{ab} h_{ab}, ~ \gamma = \det \gamma_{ab} ~.
	\end{equation}
\end{theorem}

这与 Einstein-Hilbert 作用量是十分相似的:
\begin{equation}
	S_{EH}[g_{\alpha \beta}] = \int \dd^4 x \sqrt{-g} R = \int \dd^4 x \sqrt{-g} g^{\mu\nu} R_{\mu\nu} ~,
\end{equation}
替换 $\gamma^{ab} \leftrightarrow g^{\mu\nu}$,$R_{\mu\nu} \leftrightarrow h_{ab}$。这指出:\footnote{这里 $R_{\mu\nu}$ 是一个描述空间弯曲情况的内禀张量,仅与空间的拓扑性质有关。熟悉引力理论的人应当都知道这就是 Ricci 曲率张量。}
	\textbf{Polyakov 作用量描述弦理论是一个与场 $X^\mu$ 耦合的二维引力理论。}

\subsection{弦作用量间的关系}
对 Polyakov 作用量变分:
\begin{equation}
	\delta_{\gamma} S_P = -\frac{1}{4\pi \alpha'} \int \dd \tau \dd \sigma \sqrt{-\gamma} \delta \gamma^{ab} \left(h_{ab} - \frac{1}{2} \gamma_{ab} \gamma^{cd} h_{cd}\right) ~.
\end{equation}
\footnote{变分过程中利用到 $\delta \gamma = \gamma \gamma^{ab} \delta \gamma_{ab} = -\gamma \gamma_{ab} \delta \gamma^{ab}$。}可以发现这与 Einstein 场方程是十分类似的。

由最小作用量原理,运动方程是 $h_{ab} = \frac{1}{2} \gamma_{ab} \gamma^{cd} h_{cd}$,对这式两侧同时除以行列式的相反数的平方根,得到 $\sqrt{-h} h_{ab} = \sqrt{-\gamma} \gamma_{ab}$,指出两个度规成正比,可以得到 $S_P = S_{NG}$:
\begin{equation}
	\begin{aligned}
		S_P &= -\frac{1}{4 \pi \alpha'} \int \dd \sigma \dd \tau \gamma_{ab} \gamma^{ab} \sqrt{-h} \\ 
		&= -\frac{1}{2\pi \alpha'} \int \dd \sigma \dd\tau \sqrt{-h} \\ 
		&= -\frac{1}{4\pi\alpha'} \int \dd \sigma \dd \tau \sqrt{-\gamma} \gamma^{ab} h_{ab} \\
		&= S_{NG} ~.
	\end{aligned}
\end{equation}

\subsection{Polyakov 作用量的对称性}
\begin{theorem}{} 
	Polyakov 作用量多一个几何的对称性,即 $2$-dim Weyl 不变性:
	\begin{itemize}
		\item 庞加莱对称性:$X'~ ^\mu(\tau, \sigma) = \Lambda_\nu^\mu X^\nu(\tau, \sigma) + a^\mu$, $\gamma'_{ab}(\tau, \sigma) = \gamma_{ab}(\tau, \sigma)$。
		\item Diff-不变性:$X'~^\mu(\tau', \sigma') = X^\mu(\tau, \sigma)$,$\gamma'_{ab}(\tau', \sigma')  \pdv{\sigma'~ ^a}{\sigma^c} \pdv{\sigma'~^b}{\sigma^d} = \gamma_{cd}(\tau, \sigma)$。
		\item Weyl-不变性:$X'~^\mu(\tau, \sigma) = X^\mu(\tau, \sigma)$,$\gamma'_{ab}(\tau, \sigma) = \exp\left(2 \omega(\tau, \sigma) \right) \gamma_{ab}(\tau, \sigma)$。其中 $\omega(\tau, \sigma)$ 任意取。
	\end{itemize}
	Weyl 不变性体现了世界面度规的定域(可以理解为局域的,locally的)缩放不变性。
\end{theorem}