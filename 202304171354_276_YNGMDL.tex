% 杨氏浸润模型
% 杨氏 Young 浸润 表面张力

\begin{issues}
\issueDraft
\end{issues}

\footnote{本文原载于 \href{https://www.phycat.cn/archives/230/}{phycat},基于CC-BY-NC-SA协议}
\pentry{表面张力\upref{sftens}}

\begin{figure}[ht]
\centering
\includegraphics[width=12cm]{./figures/ba97f37e6fa7319d.png}
\caption{杨氏浸润模型,\href{https://www.phycat.cn/archives/230/}{一个可交互的演示}(站外链接)} \label{fig_YNGMDL_1}
\end{figure}
杨氏浸润模型认为,一个位于光滑平面的液滴的形状由三个界面(固-液SL,液-气LV,固-气SV)的表面张力共同决定。即三界面交界点处的"原子"处于三个表面张力的受力平衡状态。
\begin{equation}
\gamma_{SL}+\gamma_{LV}\cos\theta=\gamma_{SV}~.
\end{equation}
