% 厦门大学 2012 年硕士入学物理考试试题
% keys 厦门大学|考研|物理|2012年
% license Copy
% type Tutor


\textbf{声明}:“该内容来源于网络公开资料,不保证真实性,如有侵权请联系管理员”

\begin{enumerate}
\item 一热气球以$1m.s^{-1}$的速率从地面匀速上升,由于风的影响,气球的水平速度随上升的高度而增大,其关系为$v_x=2y$。以上升计时开始时气球所在的位置为坐标原点,水平向右为$x$轴正方向,竖直向上为$y$轴正方向,求:\\
(1)气球的运动方程;\\
(2)气球的轨道方程;\\
(3)任意时刻气球轨道上的法向加速度与切向加速度。
\item 一轻质光滑圆环,半径为$R$,用细线悬挂在支点上。环上套有两个质量均为$m$的小环,它们可以在大圆环上无摩擦地滑动,如图所示,现让两小圆环从大圆环顶部同时由静止向两边下滑。试求:\\
(1)小圆环滑到何处(用$\theta$表示)时大圆环将会上升?\\
(2)若大圆环有质量(假设为质量为$M$)时,则结果又是如何?
\begin{figure}[ht]
\centering
\includegraphics[width=6cm]{./figures/b26ed944830ffe0e.png}
\caption{} \label{fig_XD12_1}
\end{figure}
\item 一平面简谐波沿$X$轴的正方向传播,从波疏媒质垂直入射到波密媒质上$B$点反射,形成反射波,逆$X$轴传播,已知波源位于坐标原点 O,O 点到反射点$B$的距离 $L=1.75(m)$,入射波的振幅$A=0.20(m)$,频率$v=2.0(Hz)$,波长$\lambda=1.4(m)$,当$t=0(s)$时,波源位于离平衡位置$\sqrt{2}A/2$处,且向负方向运动,试求:\\
(1)入射波的波函数;\\
(2)反射波的波函数;\\
(3)$OB$间因入射波与反射波干涉而静止的点的位置。
\begin{figure}[ht]
\centering
\includegraphics[width=8cm]{./figures/f6c7b1f95e9d5d5e.png}
\caption{} \label{fig_XD12_2}
\end{figure}
\item 如图所示,一半径为$R$的均匀带电球体,电荷体密度为$\rho$($\rho$为常数),现在球内挖去以O'为圆心、半径为$r$的球体空腔。设O'与原球心O之间的距离为 $a$,且满足$(a+r)<R$。求:\\
(1)挖去的空腔内任意一点的电场强度;\\
(2)OO'延长线上腔壁两点$P_1,P_2$之间的电势差。
\begin{figure}[ht]
\centering
\includegraphics[width=6cm]{./figures/acf3fe2dc1ebd1b7.png}
\caption{} \label{fig_XD12_3}
\end{figure}
\item 在一半径为$R$的无限长半圆柱面金属薄片中,自下而上通有电流儿,如图所示。试求:
(1)柱面轴线上任一点P处的磁场强度;(2)若在轴线上放一无限长、载有电流乙的导线,
电流方向与柱面相同,求载流导线单位长度所受的安培力。
\end{enumerate}