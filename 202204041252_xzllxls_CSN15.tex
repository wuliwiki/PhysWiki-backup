% 2015 年计算机学科专业基础综合全国联考卷
% 2015 计算机 考研 真题 全国卷

\subsection{一、单项选择题}
\textbf{1~40小题,每小题2分,共80分.下列每题给出的四个选项中.只有一个选项符合题目要求.}

1.已知程序如下: \\
\begin{lstlisting}[language=cpp]
int S(int n)
{
    return(n<=0)?0:s(n-1)+n;
}

void main()
{ 
    cout<<S(1);
}
\end{lstlisting}
程序运行时使用栈来保存调用过程的信息,自栈底到栈顶保存的信息依次对应的是 \\
A.main( )→S(1)→S(0) $\quad$ B.S(0)→S(1)→main( ) \\
C.main( )→S(0)→S(1) $\quad$ D.S(1)→S(0)→main( )

2.先序序列为a,b,c,d的不同二叉树的个数是 \\
A.13 $\quad$ B.14 $\quad$ C.15 $\quad$ D.16

3.下列选项给出的是从根分别到达两个叶结点路径上的权值序列,能属于同一棵哈夫曼树的是 \\
A.24,10,5和24,10,7 $\quad$ B.24,10,5和24,12,7 \\
C.24,10,10和24,14,11 $\quad$ D.24,10,5和24,14,6

4.现有一棵无重复关键字的平衡二叉树(AVL树),对其进行中序遍历可得到一个降序序列.下列关于该平衡二叉树的叙述中,正确的是 \\
A.根结点的度一定为2 $\quad$ B.树中最小元素一定是叶结点 \\
C.最后插入的元素一定是叶结点 $\quad$ D.树中最大元素一定无左子树

5.设有向图G=(V,E),顶点集V={v0,v1,v2,v3},边集E:{<v0,v1>,<v0,v2>,<v0,v3>,<v1,v3>}.若从顶点v0.开始对图进行深度优先遍历,则可能得到的不同遍历序列个数是 \\
A.2 $\quad$ B.3 $\quad$ C.4 $\quad$ D.5

6.求下面带权图的最小(代价)生成树时,可能是克鲁斯卡尔(Kruskal)算法第2次选中但不.是普里姆(Prim)算法(从v4开始)第2次选中的边是
\begin{figure}[ht]
\centering
\includegraphics[width=5cm]{./figures/CSN15_1.png}
\caption{请添加图片描述} \label{CSN15_fig1}
\end{figure}