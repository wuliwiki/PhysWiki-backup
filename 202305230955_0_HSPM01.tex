% 机械运动基础(高中)
% 高中物理|位移|速度|加速度

\addTODO{图像分析示例}

\subsection{基本概念}

\subsubsection{机械运动}
物体在空间中所处的位置随时间发生的变化,叫做\textbf{机械运动}。机械运动是最简单的运动形式。

\subsubsection{质点}
某些情况下,物体的大小和形状对研究的问题没有影响或影响可以忽略,而只需突出“物体具有质量”这个要素,我们可以把这个物体简化为一个具有质量的物质点,这样的点称为\textbf{质点}。

\subsubsection{参考系}
判断一个物体是运动还是静止,总要选取一个物体作为标准,被选作标准的物体叫\textbf{参照物}。
为了描述一个物体在空间中位置随时间的变化,我们要在参照物上建立一套\textbf{坐标系},并在同一坐标系的各处都配置同步的时钟,这便组成了一个\textbf{参考系}。

请注意,参考系分\textbf{惯性系}和\textbf{非惯性系},在非惯性系下惯性定律不成立。我们可以思考一个简单的案例,我们在静止火车的光滑桌面上放一个球,当火车开始加速时,以火车为参考系,球在不受力的情况下做加速运动。高中物理中充分理解参考系是十分重要的。

一般地,我们不严格区分“参照物”和“参考系”,而是强调\textbf{坐标系是参考系的数学抽象}。

\subsubsection{时刻和时间}
\begin{figure}[ht]
\centering
\includegraphics[width=8cm]{./figures/0445422aaf3e02e4.pdf}
\caption{时刻与时间(间隔)} \label{fig_HSPM01_2}
\end{figure}
\begin{itemize}
\item \textbf{时刻}:某一瞬时,是时间轴上的一点。常见描述为“第 $n$ 秒初(末)”。
\item \textbf{时间间隔}:两个时刻之间的间隔,是时间轴上的一段。
\end{itemize}

设 $t_1$ 和 $t_2$ 分别为先后的两个时刻,$\Delta t$ 表示这两个时刻之间的时间间隔,则 $\Delta t = t_2 - t_1$。

\subsubsection{位置,路程,位移}
\begin{figure}[ht]
\centering
\includegraphics[width=8cm]{./figures/aa951335f592579f.pdf}
\caption{位置、路程、位移} \label{fig_HSPM01_3}
\end{figure}
\begin{itemize}
\item \textbf{位置}:质点相对于参考点(常为坐标系原点)的距离和方向,在坐标系中是一个点。\footnote{“位置”可以用一个从原点指向质点的坐标的矢量描述,也称“位矢”。在大学物理中,位置常使用$\bvec r$, 位移常使用$\Delta \bvec r$. 基于位矢,位移很容易被定义。$\Delta \bvec r = \bvec r_1 - \bvec r_0$}
\item \textbf{路程}:质点运动轨迹的长度。存在局限性,不能反应运动的某些本质,描述不够精确。常用符号为 $s$,在强调“变化”时可用$\Delta s$,是一个\textbf{标量}。
\item \textbf{位移}:初位置指向末位置的有向线段,是描述质点位置变化的物理量,是一个\textbf{矢量}。常用符号为 $\bvec x$,在强调“变化”时可用$\Delta \bvec x$,单位为 $\mathrm{m}$.
\end{itemize}

\subsubsection{速度}
\begin{itemize}
\item 定义:位移与发生这段位移所用时间之比,表示物体运动的快慢。表达式为
\begin{equation}\label{eq_HSPM01_3}
\bvec{v} = \frac{\Delta \bvec x}{\Delta t}~.
\end{equation}
其中,$\bvec v$ 是速度矢量, $\Delta \bvec x$ 是 $\Delta t$ 时间内的位移矢量。 速度一般是指瞬时速度, 此时 $\Delta t$ 是一段极短的时间。
\item 在直线运动的情况下, 可以把直线作为 $x$ 轴, 使用一个坐标(标量) $x$ 表示位置, 用坐标之差(标量) $\Delta x$ 表示位移, 当 $\Delta x > 0$, 就代表位移矢量指向 $x$ 轴正方向, $\Delta x < 0$ 则指向负方向。 用标量代替矢量是为了方便书写和计算, 但在概念上位移仍然是矢量。
\item 单位:米每秒,符号是 $\mathrm{m/s}$,常用的单位还有 $\mathrm{km/h}$,$1\mathrm{m/s}=3.6\mathrm{km/h}$。
\item 方向:速度是矢量,其方向与物体的运动方向相同。对于直线运动来说,如果我们选定某一个方向为正方向,则速度方向就可以用正、负号来表示。
\item \textbf{平均速度}:物体的位移与发生位移所用时间之比,描述物体位置变化的快慢,其方向与 $\Delta x$ 一致。
\item \textbf{瞬时速度}:物体在某一时刻或某一位置的速度,描述物体在某一时刻或经过某一位置时运动的快慢,其方向与该物体在这个时刻或经过这个位置时的运动方向一致。瞬时速度可以用极短时间 $\Delta t$ 内的平均速度来计算。另外,在匀速直线运动中,瞬时速度始终和平均速度相同。
\item \textbf{平均速率}:路程与时间之比,是一个标量。一般情况下,平均速度只表示位置变化的平均快慢,而平均速率才能表示通常意义的物体运动的平均快慢。一般情况下,平均速度的大小不等于平均速率的大小,只有在单方向直线运动中才相等,但也不能描述为平均速率就是平均速度。
\end{itemize}

\subsubsection{加速度}
\begin{itemize}
\item 定义:速度矢量的变化量 $\Delta \bvec v$ 与发生这一变化所用时间 $\Delta t$ 之比。
\item 公式:
\begin{equation}
\bvec a=\frac{\Delta \bvec v}{\Delta t}~.
\end{equation}
和速度的定义类似, 加速度一般指瞬时加速度, 此时 $\Delta t$ 是一个极短的时间。 在直线运动的情况下, 同样可以用标量 $a$ 和标量 $\Delta v$ 代替矢量, 正负号决定矢量的方向。
\item 物理意义:描述物体运动速度变化的快慢。
\item 单位:在国际单位制中,加速度的单位是 $\mathrm{m/s^2}$,读作米每平方秒。
\item 方向:加速度的方向总是与速度变化量的方向相同,与速度方向无关。
\end{itemize}

\subsection{匀变速直线运动}
速度均匀变化的直线运动,即加速度不变的直线运动。
\begin{figure}[ht]
\centering
\includegraphics[width=14cm]{./figures/3db7a022ae88deab.pdf}
\caption{匀加速运动与匀速运动的小球。蓝色小球的加速度为1,初速度为0;红色小球做速度为1的匀速运动。一个关于匀加速运动更生动的\href{https://www.geogebra.org/m/kX9C9yDw}{动态演示}(站外链接)} \label{fig_HSPM01_1}
\end{figure}
\subsubsection{速度—时间公式}
\begin{equation}\label{eq_HSPM01_1}
v= v_0+ a t~.
\end{equation}

$v_0$ 为初速度,$a$ 为加速度,$t$ 为时间,$v$ 为 $t$ 时刻的瞬时速度。当初速度 $v_0=0$ 时,有 $v=a t$.

物理意义:末速度等于初速度和速度变化量的矢量和。

若 $a$ 与 $v_0$ 同向,则物体做匀加速直线运动,$v$ 逐渐增大。若 $a$ 与 $v_0$ 异向,则物体做匀减速直线运动,$v$ 逐渐减小。

\subsubsection{位移—时间公式}
\begin{equation}\label{eq_HSPM01_2}
x=v_0 t+\frac12 at^2~.
\end{equation}

$v_0$ 为初速度,$a$ 为加速度,$t$ 为时间,$x$ 为 $t$ 时刻的位移。

\subsubsection{速度—位移公式}
联立\autoref{eq_HSPM01_1} 和\autoref{eq_HSPM01_2} 消去时间 $t$ 可得
\begin{equation}\label{eq_HSPM01_4}
v^2-v_0^2=2ax~.
\end{equation}
该式由匀变速直线运动的两个基本公式推导出来,便于解决不含时间的问题。

\subsubsection{二级结论}
(1)做匀变速直线运动的物体某段时间里的平均速度等于这段时间内中间时刻的瞬时速度,且等于初、末速度矢量和的一半,由\autoref{eq_HSPM01_3} \autoref{eq_HSPM01_2} 可证:
\begin{equation}
\bar v=\frac xt=\frac{v_0t+\frac 12at^2}{t}=v_0+a\frac t2=v_{\frac t2}~.
\end{equation}
联立\autoref{eq_HSPM01_1} 消去 $a$ 可得:
\begin{equation}
\bar v=\frac{v_0+v}{2}~.
\end{equation}

(2)某段位移内中间位置的瞬时速度与这段位移的初速度 $v_0$ 和末速度 $v$ 之间满足关系式:
\begin{equation}
v_\frac x2=\sqrt \frac{v_0^2+v^2}2~.
\end{equation}
由\autoref{eq_HSPM01_4} 可知:
\begin{equation}
v_{\frac x2}^2-v_0^2=2a\frac x2~,
\end{equation}
联立\autoref{eq_HSPM01_4} 消去 $ax$ 即可得证。

(3)在连续相等的时间间隔 $T$ 里的位移之差 $\Delta x$ 为恒定值。

在第一个时间 $T$ 内位移为
\begin{equation}
x_1=v_0T+\frac12aT^2~.
\end{equation}
在时间 $2T$ 内位移为
\begin{equation}
x_{2T}=v_0\cdot2T+\frac12a(2T)^2
\end{equation}
在第二个时间 $T$ 内位移为
\begin{equation}
x_2=x_{2T}-x_1=v_0T+\frac32aT^2~.
\end{equation}
在连续相等的时间间隔 $T$ 里的位移之差为
\begin{equation}
\Delta x=x_2-x_1=aT^2~.
\end{equation}

\subsubsection{初速度为0的匀加速直线运动的几个比例式}
设物体做初速度 $v_0=0$ 的匀加速直线运动,以 $t=0$ 开始计时。

(1)$T$ 秒末、$2T$ 秒末、$3T$ 秒末、…、$nT$ 秒末的瞬时速度之比 $v_1:v_2:v_3:\dots:v_n~,$
\begin{equation}
v_1:v_2:v_3:\dots:v_n=1:2:3:\dots:n~.
\end{equation}

(2)前 $T$ 秒、前 $2T$ 秒、前 $3T$ 秒、…、前 $nT$ 秒的位移之比 $x_1:x_2:x_3:\dots:x_n~,$
\begin{equation}
x_1:x_2:x_3:\dots:x_n=1^2:2^2:3^2:\dots:n^2~.
\end{equation}

(3)$T$ 秒内、$2T$ 秒内、$3T$ 秒内、…、$nT$ 秒内的位移之比 $x_1:x_2:x_3:\dots:x_n~,$
\begin{equation}
x_1:x_2:x_3:\dots:x_n=1:3:5:\dots:(2n-1)~.
\end{equation}

(4)通过 $x$、前 $2x$、前 $3x$、…、前 $nx$ 的位移所用时间之比 $t_1:t_2:t_3:\dots:t_n~,$
\begin{equation}
t_1:t_2:t_3:\dots:t_n=1:\sqrt{2}:\sqrt{3}:\dots:\sqrt{n}~.
\end{equation}

(5)通过第一个 $x$、第二个 $x$、第三个 $x$、…、第 $n$ 个 $x$ 的位移所用时间之比 $t_1:t_2:t_3:\dots:t_n~,$
\begin{equation}
t_1:t_2:t_3:\dots:t_n=1:\sqrt{2}-1:\sqrt{3}-\sqrt{2}:\dots:\sqrt{n}-\sqrt{n-1}~.
\end{equation}

\subsection{常见问题}
\subsubsection{相遇和追及问题}
当两个物体在同一直线上运动时,由于各自的运动情况不一,两物体之间的距离会不断变化,在研究这两个物体的运动时,就涉及到相遇和追及的问题。相遇和追及问题实质上就是研究两个物体是否会在同一时刻到达同一位置的问题。研究方法有临界条件法、判断法和图像法等。

\subsubsection{自由落体运动}
条件:物体在只受重力的情况下(理想化模型),从静止即 $v_0=0$ 开始下落。忽略下落高度变化导致的重力变化。

实质:初速度 $v_0=0$,加速度 $a=g$ 的匀加速直线运动。

自由落体的加速度:也就是当地的重力加速度。重力加速度方向为竖直向下,其大小随纬度的增大而增大。由于忽略了下落高度变化导致的重力变化,我们可以认为在同一地区的重力加速度是恒定的,通常默认取 $g=9.8\mathrm{m/s^2}$,有时为了计算方便会注明取 $g=10\mathrm{m/s^2}$.

自由落体运动是匀变速直线运动的特例,根据匀变速直线运动的规律,易知在自由落体运动中:
\begin{equation}
v=gt~,
\end{equation}
\begin{equation}
h=\frac12gt^2~,
\end{equation}
\begin{equation}
v^2=2gh~.
\end{equation}

\subsubsection{竖直上抛运动}
把物体以一定的初速度沿着竖直方向向上抛出,仅在重力作用下,物体所做的运动叫做竖直下抛运动。

实质:初速度 $v_0\neq0$ 且方向竖直向上,加速度 $a=g$ 的匀加速直线运动。

基本公式(规定初速度 $v_0$ 的方向为正方向):
\begin{equation}
v=v_0-gt~,
\end{equation}
\begin{equation}
x=v_0t-\frac12gt^2~,
\end{equation}
\begin{equation}
v^2-v_0^2=-2gx~.
\end{equation}

推论:

上升到最高点的时间 $t_1$ 和从最高点落回到抛出点的时间 $t_2$ 相等,且有
\begin{equation}
t_1=t_2=\frac{v_0}{g}~.
\end{equation}

从抛出到落回抛出点的总时间为
\begin{equation}
t=t_1+t_2=\frac{2v_0}{g}~.
\end{equation}

上升的最大高度为
\begin{equation}
H=\frac{v_0^2}{2g}~.
\end{equation}

竖直上抛运动的物体在上升和下落阶段会经过同一位置,其运动具有对称性:上升和下落阶段经过同一位置时速度等大反向(\textbf{速度对称})。上升和下落阶段经过同一段竖直距离所用的时间相等(\textbf{时间对称})。上升阶段为减速到0的过程,下落阶段为自由落体运动回到抛出点,上升阶段可以看作下落阶段的逆过程。

\subsubsection{竖直下抛运动}
把物体以一定的初速度沿着竖直方向向下抛出,仅在重力作用下,物体所做的运动叫做竖直下抛运动。

实质:初速度 $v_0\neq0$ 且方向竖直向下,加速度 $a=g$ 的匀加速直线运动。

基本公式(规定初速度 $v_0$ 的方向为正方向):
\begin{equation}
v=v_0+gt~,
\end{equation}
\begin{equation}
x=v_0t+\frac12gt^2~,
\end{equation}
\begin{equation}
v^2-v_0^2=2gx~.
\end{equation}

从运动的合成和分解来看,竖直下抛运动可以看成是\textbf{竖直向下的匀速直线运动}和\textbf{自由落体运动}合成的。