% 康托尔集(综述)
% license CCBYSA3
% type Wiki

本文根据 CC-BY-SA 协议转载翻译自维基百科\href{https://en.wikipedia.org/wiki/Cantor_set}{相关文章}。

在数学中,康托尔集是一个位于同一直线线段上的点集,它具有许多违反直觉的性质。该集合最早由亨利·约翰·斯蒂芬·史密斯于1874年发现\(^\text{[1][2][3][4]}\),并在1883年被德国数学家格奥尔格·康托尔提及\(^\text{[5][6]}\)。

通过对该集合的研究,康托尔及其他数学家为现代点集拓扑奠定了基础。最常见的构造是康托尔三分集,它通过不断从一条线段中去除中间三分之一,并对剩余的每一段重复该过程来构造。康托尔在其论文中仅顺带提到了这种三分构造,作为一个“完美但稠密度为零”的集合的例子\(^\text{[5]}\)。

更一般地,在拓扑学中,\textbf{康托尔空间}是指与康托尔三分集同胚的拓扑空间(配备其子空间拓扑)。康托尔集在自然意义上同胚于离散二点空间
\(2^\mathbb{N}\)(即离散二元集合的可数笛卡尔积)。根据L.E.J.布劳威尔的一个定理,这等价于以下五个条件的同时满足:完美(无孤立点)、非空、紧致、可度量且零维\(^\text{[7]}\)。
\subsection{三分集的构造与公式}
康托尔三分集(记作 $\mathcal{C}$)是通过对一组线段不断地删除中间的开放三分之一区间而构造出来的。最开始,从区间 $[0, 1]$ 中删除中间的开放三分之一区间 $\left(\frac{1}{3}, \frac{2}{3}\right)$,留下两个闭合的线段:$\left[0, \frac{1}{3}\right] \cup \left[\frac{2}{3}, 1\right]$接下来,对这两个剩余的线段分别删除它们的中间三分之一,得到四个线段:$\left[0, \frac{1}{9}\right] \cup \left[\frac{2}{9}, \frac{1}{3}\right] \cup \left[\frac{2}{3}, \frac{7}{9}\right] \cup \left[\frac{8}{9}, 1\right]$如此重复下去,康托尔三分集就是在这个无限过程中的所有步骤中都未被删除的点所组成的集合,仍然属于区间 $[0, 1]$。这个构造过程也可以用递归方式来描述。令:
$$
C_0 := [0, 1]~
$$
接着定义:
$$
C_n := \frac{C_{n-1}}{3} \cup \left(\frac{2}{3} +\frac{C_{n-1}}{3}\right) = \frac{1}{3} \left(C_{n-1} \cup (2 + C_{n-1})\right)
\quad \text{对所有 } n \geq 1~
$$
于是,康托尔三分集可以表示为:
$$
\mathcal{C} := \lim_{n \to \infty} C_n = \bigcap_{n=0}^{\infty} C_n = \bigcap_{n=m}^{\infty} C_n \quad \text{对任意 } m \geq 0~
$$
这个构造过程的前六步如下图所示(原文提及有图示,此处略去)。
\begin{figure}[ht]
\centering
\includegraphics[width=14.25cm]{./figures/5a6934f361e2bc93.png}
\caption{} \label{fig_KTRJ_1}
\end{figure}
利用自相似变换的思想设$T_L(x) = x/3$、$T_R(x) = (2 + x)/3$,则递归定义为:$C_n = T_L(C_{n-1}) \cup T_R(C_{n-1})$于是,可以写出康托尔集的显式闭式表达如下 \(^\text{[8]}\):
$$
\mathcal{C} = [0, 1] \setminus \bigcup_{n=0}^{\infty} \bigcup_{k=0}^{3^n - 1} \left( \frac{3k+1}{3^{n+1}}, \frac{3k+2}{3^{n+1}} \right)~
$$
其中每一个被移除的中间三分之一是从闭区间$\left[ \frac{3k+0}{3^{n+1}}, \frac{3k+3}{3^{n+1}} \right] = \left[ \frac{k+0}{3^n}, \frac{k+1}{3^n} \right]$中去除的开区间$\left( \frac{3k+1}{3^{n+1}}, \frac{3k+2}{3^{n+1}} \right)$。或者,康托尔集也可以表示为:
$$
\mathcal{C} = \bigcap_{n=1}^{\infty} \bigcup_{k=0}^{3^{n-1}-1} \left( \left[ \frac{3k+0}{3^n}, \frac{3k+1}{3^n} \right] \cup \left[ \frac{3k+2}{3^n}, \frac{3k+3}{3^n} \right] \right)~
$$
这里,每个闭区间$\left[\frac{k+0}{3^{n-1}}, \frac{k+1}{3^{n-1}}\right] = \left[\frac{3k+0}{3^n}, \frac{3k+3}{3^n}\right]$中的中间三分之一区间$\left(\frac{3k+1}{3^n}, \frac{3k+2}{3^n}\right)$
通过与两端子区间$\left[\frac{3k+0}{3^n}, \frac{3k+1}{3^n}\right] \cup \left[\frac{3k+2}{3^n}, \frac{3k+3}{3^n}\right]$的交集被“移除”。

这个移除中间三分之一的过程是一个有限细分规则的简单示例。康托尔三分集的补集是分形弦的一个例子。
\begin{figure}[ht]
\centering
\includegraphics[width=14.25cm]{./figures/c1c347db78f52e5c.png}
\caption{} \label{fig_KTRJ_2}
\end{figure}
从算术角度来看,康托尔集由所有不需要数字1来表示的三进制(以3为底)小数组成,这些数都属于区间 $[0, 1]$。

正如上面的图示所示,康托尔集中的每一个点都可以通过一棵无限深的二叉树中的一条路径唯一确定。这条路径在每一层都会向左或向右分叉,取决于该点位于被删去线段的哪一侧。将每次向左转用0表示,向右转用2表示,便可得到该点对应的三进制小数。
\subsubsection{曼德博对“凝结”式构造的描述}
在《自然的分形几何》中,数学家伯努瓦·曼德博提出了一个富有想象力的思想实验,来帮助非数学读者设想集合 𝒞 的构造。他的叙述从想象一根金属棒(也许是轻质金属)开始,其中棒的物质通过迭代地向两端转移而“凝结”。随着棒的各段变得越来越小,它们变成细长而致密的小块,最终变得太小太暗而无法看见。

凝结:康托尔棒的构造源自我称之为“凝结”的过程。它从一根圆柱棒开始,最好将其想象为密度极低的材料。然后,棒的物质从中间三分之一“凝结”到两端的三分之一部分中,使得两端部分的位置保持不变。接着,每个端三分之一的中间三分之一部分的物质再次凝结到其各自的两端三分之一部分中,如此反复无穷,直到最终剩下无限多个、无限细而密度无限高的小块。这些小块沿直线以一种由生成过程精确决定的方式分布。在这个图示中,“凝结”最终甚至需要“锤打”才能继续,但当印刷机和我们的肉眼都无法再分辨时,过程便终止了:最后一行与倒数第二行已无法区分——它的每一个最终部分都被看作是一个灰色的小块,而非两个并列的黑色小块。\(^\text{[9]}\)
\subsection{构成}
由于康托尔集是由未被排除的点构成的,因此单位区间中剩余部分的比例(即测度)可以通过计算被移除的总长度来确定。这个总长度构成一个几何级数:
$$
\sum_{n=0}^{\infty} \frac{2^n}{3^{n+1}} = \frac{1}{3} + \frac{2}{9} + \frac{4}{27} + \frac{8}{81} + \cdots = \frac{1}{3} \left( \frac{1}{1 - \frac{2}{3}} \right) = 1.~
$$
因此,剩下的比例为 $1 - 1 = 0$。

这个计算表明,康托尔集不可能包含任何非零长度的区间。也许令人惊讶的是,竟然还有东西剩下——毕竟,被移除区间的总长度正好等于原始区间的长度。然而,仔细观察这个构造过程可以发现,必定有东西被保留下来,因为每次移除“中间三分之一”实际上是移除开区间(即不包含端点的集合)。比如,从原始区间 $[0, 1]$ 中移除线段 $(\frac{1}{3}, \frac{2}{3})$,实际上保留了端点 $\frac{1}{3}$ 和 $\frac{2}{3}$。后续的每一步也不会移除这些端点或其他端点,因为每次被移除的区间始终是当前保留区间的内部部分。因此,康托尔集并非空集,而且实际上包含不可数无限多个点(这可由前文中关于无穷二叉树路径的描述推得)。

看起来好像只剩下了构造过程中那些被保留下来的端点,但实际上并非如此。例如,数 $\frac{1}{4}$ 有唯一的三进制形式 $0.020202\ldots = 0.\overline{0}_2$。它处于最底部的三分之一,再处于该部分上方三分之一的部分,再处于那部分的底部三分之一,如此反复。由于它从未出现在任何一个中间三分之一段中,因此它从未被移除。但它也不是任何中间段的端点,因为它不是 $1/3$ 的任何幂的整数倍。\(^\text{[10]}\)

所有这些被移除区间的端点都是终止的三进制小数,并且属于集合:
$$
\left\{ x \in [0,1] \mid \exists i \in \mathbb{N}_0: x \cdot 3^i \in \mathbb{Z} \right\} \quad \left( \subset \mathbb{N}_0 \cdot 3^{-\mathbb{N}_0} \right)~
$$
这是一个可数无限的集合。而就基数而言,康托尔集中的绝大多数元素既不是端点,也不是像 $1/4$ 这样的有理数点。整个康托尔集事实上是不可数的。
\subsection{性质}
\subsubsection{基数}
可以证明,在康托尔集的构造过程中,最终保留下来的点与最初一样多,因此康托尔集是不可数的。为说明这一点,我们展示一个函数 $f$ 从康托尔集 $\mathcal{C}$ 映射到闭区间 $[0,1]$,使得 $f$ 是满射(即 $f$ 将 $\mathcal{C}$ 映射到整个 $[0,1]$),从而说明 $\mathcal{C}$ 的基数不少于 $[0,1]$ 的基数。由于 $\mathcal{C} \subseteq [0,1]$,其基数也不会更大,因此根据康托–伯恩施坦–施罗德定理,两者的基数实际上是相等的。

为构造这个函数,我们考虑 $[0,1]$ 区间内的点在三进制(base 3)表示中的形式。回想一下,某些三进制小数,特别是集合$\left( \mathbb{Z} \setminus \{0\} \right) \cdot 3^{-\mathbb{N}_0}$中的元素,存在不止一种表示方法。例如:$\frac{1}{3}$ 可以写作 $0.13 = 0.1\overline{0}_3$,也可以写作 $0.0222\ldots_3 = 0.0\overline{2}_3$;$\frac{2}{3}$ 可以写作 $0.2_3 = 0.2\overline{0}_3$,也可以写作 $0.1222\ldots_3 = 0.1\overline{2}_3$。\(^\text{[11]}\)在构造康托尔集的第一步中,被移除的是所有三进制形式为 $0.1xxxxx\ldots_3$ 的数字,其中 $xxxxx\ldots_3$ 是介于 $00000\ldots_3$ 与 $22222\ldots_3$ 之间的任意数字序列。因此,第一步之后剩下的数字形式为:
\begin{itemize}
\item 形如 $0.0xxxxx\ldots_3$ 的数(包括 $0.022222\ldots_3 = 1/3$);
\item 形如 $0.2xxxxx\ldots_3$ 的数(包括 $0.222222\ldots_3 = 1$)。
\end{itemize}
简而言之,第一步后留下的点就是那些三进制表示中小数点后第一位不是 1的数。

