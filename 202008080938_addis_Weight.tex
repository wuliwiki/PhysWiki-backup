% 重力 重量
% 万有引力|重力

\pentry{万有引力\upref{Gravty}, 圆周运动的向心力\upref{CentrF}}

一般来说, 重力的定义并不明确. 有的地方直接把一个质点在某点的\textbf{万有引力(gravity)}定义为重力. 但另一些情况下把弹簧秤的度数定义为重力(例如在地面参考系). 为了区分, 我们建议不要使用 “重力”, 而是直接将前者称为(万有)引力, 后者称为\textbf{重量(weight)}或者\textbf{视重}. 英语中, “重力” 并没有单独对应的词汇.

什么情况下引力会和重量不同? 答案是非惯性系中. 最常见的例子就是地球表面虽然可以近似为惯性系, 但严格来说却不是. 如果只考虑地球的自转, 那么地表任意一个相对静止的点在惯性系中都会做圆周运动. 为了便于分析, 我们先选取惯性系. 当一个相对地表静止的质点挂在弹簧秤上达到平衡时, 它所受的引力 $\bvec G$ 和弹簧对它的拉力 $\bvec T$(或者台秤的支持力) 的合力提供圆周运动的向心力\upref{CentrF}.
\begin{equation}
\bvec F_c = \bvec G + \bvec T
\end{equation}
所以不妨定义重量矢量为弹簧拉力的逆矢量
\begin{equation}
\bvec F_w = -\bvec T = \bvec G - \bvec F_c
\end{equation}
那么重量, 也就是弹簧秤的示数大小就是模长 $F_w = \abs{\bvec F_w}$.

(例题未完成)

% 计算一下重力加速度
