% 【导航】高中数学
% keys 高中|数学|概述
% license Xiao
% type Map

\begin{issues}
\issueDraft
\end{issues}

% Giacomo:首先要确定目标
小时百科面向的主要群体是对物理、数学、计算机等感兴趣或在科研上存在需求的本科生及研究生,但事实上,有很多的知识是在高中阶段(甚至小学或初中)就有涉及。另外有很多的读者反映,当前高中数学学习过程中存在一些问题或困惑,比如:
\begin{itemize}
\item 由于教学目标、进度或其他原因导致的,学生不能深入或透彻理解当前教材或课堂上的教学内容,最后变成“听不懂、记不住、做不会”;
\item 高考改革,不仅要求在当前缩减的大纲下保证知识体系的构建和联通,还需要能够快速理解一些陌生概念与已有概念的关系。这要求学生在高中阶段的学习中:一方面,提前去接触一些陌生的新的概念,避免畏惧;另一方面,知识体系稳固,分析、理解新概念与已有知识的关系;
\item 由于课纲修改造成的部分重要知识的调整,使得尽管高中自成一派,但与大学的知识衔接不上,造成本科时学习的障碍。很多知识如果没有高中阶段没有接触过,产生一些比较直观的理解或印象,那么,在后期的深入接触中,由于知识体系存在缺位,导致可能会存在一些理解上的障碍。
\end{itemize}

因此,高中数学的主要目标包括三点:
\begin{enumerate}
\item 应对高中生特点,在语言、记号、图片、例题上给予更多的形象解释;
\item 立足高中内容,参考高中数学课本,力求建立稳固的知识体系;
\item 增加一定的深度,尽可能展现一些当前高中教材中不涉及的,但又对高中学生理解而言不过于艰难的内容,尽量为学生开拓视野。
\end{enumerate}
以期帮助高中学生或具有初中水平的数学爱好者以一种更便捷和容易的方式学习、备考,并培养重要的自学能力。

\subsection{初中回顾}

相信有不少现在已经是高中生的同学,因为初中的某些知识掌握不牢或理解不透,造成自己在高中的学习过程中,一步一道坎。因此下面会对初中的一些比较重要的内容进行回顾和提示,注意这里不会讲的太细,更多的是提醒你有哪些知识应用不畅的话,会成为高中阶段学习的绊脚石。好消息是,有不少初中知识在高中阶段是不会再考察的了,因此,不要担心初中的噩梦重来。主要涉及的内容包括:

\begin{itemize}
\item \enref{因式分解与一元二次方程}{quasol}:单项式与多项式、因式分解、十字相乘法、求根公式、韦达定理
\item \enref{数字与函数回顾}{HsFunB}:自然数、整数、有理数、实数、坐标系、正比例函数、一次函数、反比例函数、二次函数
\item \enref{几何与解析几何初步}{HsGeBa}:点、线、角、三角形、三角函数、圆、坐标系、直线的常见表达式
\end{itemize}


\subsection{集合与命题}

我们每天接触到的事物,其实可以被视为一个个“集合”。无论是你喜欢的书籍,还是你最爱吃的水果,它们都可以被整理成一个集合。集合是数学中的一种基础概念,它就像是一个“大袋子”,能够装下各式各样的元素。这些元素可能是数字、字母,甚至可以是更多更抽象的东西。整座数学大厦建立在“集合”的基础之上,后面学习的内容也自然以此为基础,所有的概念归根到底都是在集合的视角下进行的。通过学习集合,你会发现数学其实充满了分类与整理的乐趣,就像一个把混乱变为有序的魔法师。高中开头第一步就是了解集合的语言和思考方法。涉及的内容包括:

\begin{itemize}
\item \enref{集合}{HsSet}
\item \enref{集合的基本运算}{HsSeOp}
\end{itemize}

数学不仅仅是关于数字的学问,还是一门关于逻辑的艺术。数学家们用命题来表达一个清晰的观点或陈述。命题要么是真要么是假,而判断命题的真假,就像是在解一道逻辑谜题。想象一下,就像你在一个神秘的房间中,面前有几个开关和灯泡,你需要通过观察来判断哪些开关能点亮灯泡。命题的学习就像是解开这些谜题的过程,你会在其中锻炼出严密的逻辑思维,并且会发现,逻辑的世界也是如此奇妙和引人入胜。而令人苦恼的考试,就是由一个个命题构成的。涉及的内容包括:
\begin{itemize}
\item \enref{命题与推理}{HsLogi}
\item \enref{复合命题}{HsCoPr}
\end{itemize}

\subsection{函数、等式与不等式}

在数学中,函数是一个重要的概念,它帮助我们描述和理解两个变量之间的关系。无论是在代数、几何,还是物理学中,函数都扮演着不可或缺的角色。函数不仅仅是将一个数映射到另一个数的工具,它实际上是我们用来探索和描述各种现象的强大工具。通过学习函数,你会发现,很多看似复杂的问题,其背后其实都隐藏着一个简单的规律。掌握了函数的概念,你将能够更轻松地理解和解决各种数学问题。

\begin{itemize}
\item \enref{函数}{functi}
\item \enref{函数的性质}{HsFunC}
\item \enref{函数的变换}{FunTra}
\item \enref{幂函数}{power}
\item \enref{指数与指数函数}{HsExpF}
\item \enref{对数与对数函数}{Ln}
\end{itemize}

导数是函数的一个拓展部分,也是高中数学中的一个难点。

\begin{itemize}
\item \enref{导数}{HsDerv}
\item \enref{导数的计算}{HsDerB}
\item \enref{导数的性质与构造}{HsDerC}
\end{itemize}

等式和不等式是表达关系的两种核心形式。它们帮助我们明确地说明一个量与另一个量之间是相等还是有所差异。等式讲究精确,不等式则关注范围。它们既是数学语言的基础,也是解决问题的关键工具。理解这些概念,你会发现很多复杂的情景其实只是在等式与不等式之间切换思路。在此基础上,你还将学会区分方程与恒等式——前者寻找特定解,后者则在任意条件下都成立,以及各种不等式的类型和它们的应用。这些细微而重要的区别,将帮助你在数学问题中更加精准地分析与推理。

\begin{itemize}
\item \enref{等式与不等式}{HsEquN}
\item \enref{解方程与解不等式}{SolEqI}
\item \enref{解方程组}{LinEqu}
\item \enref{恒等式与恒成立不等式}{HsIden}
\end{itemize}

\subsection{数列}

数列是数学中的一类特殊“数字队列”,它们按照某种规律排列,让我们能够从中发现规律。比如,等差数列中,每个数字与前一个数字的差是固定的;而在等比数列中,每个数字与前一个数字的比是固定的。这些规律不仅让我们可以预测数列中的后续项,还帮助我们更好地理解函数的本质。实际上,数列与函数之间有着密切的联系——数列可以看作是函数在整数上的取值,从而让我们在分析数列问题时,能够运用函数的工具。通过学习数列,你将发现数学中的许多问题都可以通过这种有序的数字排列来解答,从而为你打开通往更深层次数学研究的大门。

\begin{itemize}
\item \enref{数列}{HsSeFu}
\item \enref{等差数列}{HsAmPg}
\item \enref{等比数列}{HsGmPg}
\item \enref{数学归纳法}{MathId}
\end{itemize}
\subsection{平面向量}

平面向量不仅仅是数学中的一根“有向线段”,它其实是理解空间中运动、力量和变化的有力工具。向量既有长度,又有方向,这让它可以用来描述物体的速度、力的作用方向,甚至是图形的平移和旋转。在平面几何和物理问题中,向量提供了一种简洁而有效的解决方案。学习平面向量,你会发现,很多看似复杂的运动和变化,背后都隐藏着简单的向量运算,这将让你在数学和科学的探索中如虎添翼。


\subsection{三角函数}

三角函数是数学中的一个重要工具,帮我们解决许多不同的问题。它不仅帮助我们解三角形,准确地计算三角形的边长和角度;还在时间与频率的分析中扮演了关键角色,因为它们具有周期性的特征,能够描述像声音和光这样的波动现象。无论是在几何中找到准确的答案,还是在理解自然界的周期变化,三角函数都发挥着重要作用。通过学习三角函数,你会发现它不仅仅是纷繁复杂的公式,而是一个可以解决实际问题的有力工具。

\begin{itemize}
\item \enref{弧度制与任意角}{HsAngl}
\end{itemize}
\subsection{解析几何}

解析几何是一个将代数与几何结合的强大工具,它用坐标和方程的语言来研究图形和空间关系,赋予了几何图形新的表达方式。然而,与初中几何通过逻辑推导解决问题不同,解析几何对计算能力提出了更高的要求。推导过程中的创造性被部分替代为对精确运算和解方程的严格需求。尽管如此,这种方式赋予我们更系统、更通用的解决问题方法。通过解析几何,我们能够以一种统一的视角处理复杂的几何问题,发现它们背后深刻的联系,而这正是它不可替代的独特魅力。


\subsection{立体几何}

在立体几何中,我们不仅要掌握线与面之间的关系,还要学会如何处理它们所形成的角度。这个领域不仅延续了传统几何的一些基础概念和手段,还融入了类似平面向量的强大工具——空间向量。二者的结合帮助我们轻松解决空间中的问题。更让人放心的是,由于向量的加持,立体几何在高考中的难度并不高,只要你理解了基本概念,就能应付自如。

\subsection{计数原理与概率统计}

生活中充满了各种不确定性,而计数原理与概率统计正是帮助我们理解和处理这些不确定性的数学工具。通过计数原理,我们可以找出所有可能的情况,比如排列组合问题中的各种选项;而通过概率统计,我们能够计算出每种情况发生的可能性,比如抽奖中中奖的几率。这两个工具结合起来,帮助我们在面对随机事件时做出更明智的决策。不论是预测天气、设计实验,还是简单的游戏策略,学习计数原理和概率统计都会让你更好地应对生活中的各种随机挑战。

最后,相信通过阅读上面的介绍,你已经对高中数学的图景成竹在胸。现在,就让我们一步一步地从基础开始,稳固知识,开阔视野。打好根基之后,再从习题的练习中对这些知识加深理解、熟练运用。相信我,高中数学的挑战一定难不倒你!顺便祝各位备考顺利!

% \addTODO{移动下方内容至其他页面。}

% \subsection{几何向量}

% 线性代数的研究对象是向量和矩阵,而我们最早认识的向量就是\textbf{几何向量},这里我们回顾几何向量的相关概念。

% 几何向量的存在与坐标系无关,它是一些有长度有方向的箭头。我们把(二维)平面中的向量称为平面向量,(三维)空间中的向量称为空间向量;在高中数学的语境下,我们把(一维)直线上的向量称为标量,但这是不严谨的。

% % 对于讨论问题的不同,我们有时仅需要处于同一平面(\textbf{二维空间})的所有几何向量,有时需要\textbf{三维空间}中的所有几何向量,最简单的情况下只需要沿某条线(\textbf{一维空间})的所有几何向量(这时我们可以规定一个正方向,且仅使用几何向量的模长加正负号来表示几何向量以简化书写)。

% \addTODO{高中数学中平面向量和空间向量的链接}
% % Giacomo:是不是应该把高中数学/物理,改成中学数学/物理?

% 几何向量有起点(箭尾)和终点(箭头),但我们对几何向量的绝对位置不感兴趣,我们只在乎起点和终点的相对位置,即两个几何向量如果有相同的方向和长度就被视为同一个向量。

% \enref{几何向量的一些基本运算}{GVec} 同样不需要有任何坐标系的概念,\textbf{几何向量相加}按照三角形法则或平行四边形法则即可。
% \textbf{几何向量数乘}就是把几何向量的模长乘以一个实数,若乘以正数,方向不变,若乘以负数,取相反方向。 \textbf{几何向量的线性组合}是把若干几何向量分别乘以一个实数再相加得到新的几何向量。

% 几何向量的\enref{内积}{Dot}等于一个几何向量在另一个几何向量上的投影长度乘以另一个几何向量的模长得到一个实数,几何向量的\textbf{模长}等于几何向量与自身内积再开方,把几何向量除以自身模长使模长变为单位长度的过程叫做\textbf{归一化}。若两几何向量内积为零,这两个几何向量相互\textbf{正交}\footnote{对于几何向量,正交就是方向垂直,不加区分。}。

% 三维欧几里得空间中,两几何向量\enref{叉乘}{Cross}得到的几何向量垂直于两几何向量,模长为一个几何向量在另一个几何向量垂直方向的投影长度乘以另一个几何向量的模长。

% 为了方便描述几何向量之间的关系,我们选取一些\textbf{线性无关}的几何向量作为所有几何向量的\textbf{基底},使空间中的任何几何向量可以用这些基底的唯一一种线性组合来表示,$N$ 维空间需要 $N$ 个基底向量。一般来说,基底不必互相正交。我们先把这些基底排序,任意几何向量表示成它们的线性组合时,把式中的 $N$ 个系数按照顺序排列,就是该几何向量的\textbf{坐标},通常用列几何向量表示。由于线性组合的唯一性,每个几何向量的坐标是唯一的。

% 为了方便计算任意几何向量的坐标,往往取\enref{正交归一的基底}{OrNrB}(所有基底模长为1,任意两基底互相正交)。这样,任意向量的坐标都可以通过与基底的内积得到。

% \addTODO{添加相关文章的链接}

% \subsubsection{几何向量的线性变换}
% \addTODO{科普版本的线性映射}

% 我们可以设计一种规则把某个空间的任意几何向量\textbf{变换}(\textbf{映射})到另一个空间的几何向量;如果任意几何向量线性组合的变换等于这些几何向量分别变换再线性组合,这个变换就被称为\enref{线性变换}{LTrans}。

% \addTODO{具体例子旋转变换}



% \addTODO{链接待处理,文本待处理}

