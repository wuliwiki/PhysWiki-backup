% 电阻
% 电阻|导体|电流|电压|欧姆定律

\pentry{电流\upref{I}, 电压\upref{Voltag}}

欧姆定律为
\begin{equation}
U = IR
\end{equation}
柱形电阻器的电阻于电阻率的关系
\begin{equation}
R = \frac{\varrho L}{S}
\end{equation}


\subsection{电阻的简单模型}
我们这里用一个简单的经典力学模型推导电阻的性质及欧姆定律, 但严格来说, 这个推导需要使用量子力学和半导体理论. 平行板电容器, 中间有某种均匀的导电材料, 该材料中自由电子的电荷密度 $\rho < 0$ 为定值, 每个电子受到的阻力与电子速度成正比, 比例常数 $\alpha > 0$. 即
\begin{equation}
f = -\alpha v
\end{equation}
当我们在电容器两板上施加电压时, 内部会产生匀强电场, 使电子受到电场力
\begin{equation}
F = -Ee
\end{equation}
电子在该电场力下加速(由于电子质量很小, 这个过程很快可以认为是一瞬间完成的), 直到阻力等于电场力时加速停止, 进行匀速运动. 于是有
\begin{equation}
Ee = \alpha v
\end{equation}
所以电阻内电流密度大小为
\begin{equation}\label{Resist_eq4}
\bvec j = \rho \bvec v = \frac{e\rho}{\alpha}\bvec E
\end{equation}
电流为
\begin{equation}
I = jS = \frac{\rho EeS}{\alpha}
\end{equation}
电阻两端电压为
\begin{equation}
U = EL
\end{equation}
带入上式得
\begin{equation}\label{Resist_eq3}
U = I \frac{\alpha L}{\rho eS}
\end{equation}
我们定义\textbf{电阻率}为
\begin{equation}\label{Resist_eq1}
\varrho = \frac{\alpha}{\rho e}
\end{equation}
电阻率和材料性质有关, 可能会随温度,压强,光照,等环境因素变化. 定义电阻为
\begin{equation}\label{Resist_eq2}
R = \frac{\varrho L}{S}
\end{equation}
可见它与长度 $L$ 成正比, 于横截面成反比. 将\autoref{Resist_eq1} 带入\autoref{Resist_eq2} 再带入\autoref{Resist_eq3}, 可得欧姆定律
\begin{equation}
U = IR
\end{equation}
\autoref{Resist_eq4} 也可以使用电阻率记为
\begin{equation}
\bvec E = \varrho\bvec j
\end{equation}
这相当于欧姆定律的微观形式.
