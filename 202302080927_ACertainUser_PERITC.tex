% 包晶相图

\footnote{本文参考了刘智恩的《材料科学基础》}
\pentry{匀晶相图\upref{ISOMOR}}
包晶转变:由一个液相与一个固相生成另一种固相。 $L + \alpha \rightarrow \beta$

在热力学中,往往只关心相的变化;但由于动力学因素,实际冷却时,各相往往形成一定的有序组织结构。本文一并简要讨论,以Pt-Ag合金的平衡冷却为例。

\subsection{包晶相图}

\begin{figure}[ht]
\centering
\includegraphics[width=10cm]{./figures/PERITC_4.pdf}
\caption{典型的包晶系合金$Pt-Ag$相图.数据来源:刘智恩《材料科学基础》。仅供示意,未按实际比例绘制。} \label{PERITC_fig4}
\end{figure}

\subsubsection{$10.5\%<wAg<42.4\%$的合金} 
\begin{figure}[ht]
\centering
\includegraphics[width=14cm]{./figures/PERITC_2.pdf}
\caption{$10.5\%<wAg<42.4\%$的合金。} \label{PERITC_fig2}
\end{figure}
*:图中脱熔相的$\alpha_I$是原书写法,新版已更新为$\alpha_{II}$,下同

全程的相转变:$L \rightarrow \alpha+\beta$

全程的组织转变:$L \rightarrow (\alpha+\beta)_{Peritectic} + \alpha_{II} + \beta_{II}$

1点以上,先发生匀晶转变 $L \rightarrow \alpha$

2点处, 发生包晶转变 $L+ \alpha \rightarrow \beta$。此处相变完成后,此前匀晶反应形成的α没有被完全消耗,而L被完全消耗,系统由α与β组成。由于生成β需要原子的扩散,β往往围绕α相而生成,体现明显的“包晶”特点。
\begin{itemize}
\item 包晶转变是恒成分转变,即包晶转变中,先后结晶的部分的成分一致
\item 包晶转变是恒温转变: f=2-3+1=0,系统没有自由度
\end{itemize}

2点以下,发生脱熔转变。$\alpha \rightarrow \beta_{II}, \beta \rightarrow \alpha_{II}$

% \subsubsection{$wAg=42.4\%$的合金} 

% \begin{figure}[ht]
% \centering
% \includegraphics[width=14cm]{./figures/PERITC_1.pdf}
% \caption{$w_{Ag}=42.4\%$的合金} \label{PERITC_fig1}
% \end{figure}

% 全程的相转变:$L \rightarrow \alpha+\beta$

% 全程的组织转变:$L \rightarrow \beta + \alpha_{II}$

% D点以上,先发生匀晶转变 $L \rightarrow \alpha$

% D点处,发生包晶转变,由固相α与液相L生成一个固相β, $L + \alpha \rightarrow \beta$。此处相变完成后,L与α均被完全消耗,系统由β组成。

% D点以下,发生脱熔转变。$\beta \rightarrow \alpha_{II}$

\subsubsection{$42.4\%<wAg<66.3\%$的合金}
\begin{figure}[ht]
\centering
\includegraphics[width=14cm]{./figures/PERITC_5.pdf}
\caption{$42.4\%<wAg<66.3\%$的合金} \label{PERITC_fig5}
\end{figure}
全程的相转变:$L \rightarrow \alpha+\beta$

全程的组织转变:$L \rightarrow \beta + \alpha_{II}$

2点以上,先发生匀晶转变 $L \rightarrow \alpha$

3点处,发生包晶转变 $L + \alpha \rightarrow \beta$.此处,相变完成后,此前匀晶反应形成的α被完全消耗,但L没有被完全消耗,系统由β+L组成。

3点以下,再发生匀晶转变,生成β $L \rightarrow \beta$。

4点处,L被完全消耗,系统完全由β组成

4点以下,发生脱溶转变 $\beta \rightarrow \alpha_{II}$

\subsubsection{其余成分的合金}
仅发生简单的匀晶-脱熔转变,不再列举。

