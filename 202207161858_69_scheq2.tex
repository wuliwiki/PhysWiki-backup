% 自旋 1/2 粒子的非相对论波函数
% 自旋|薛定谔方程|泡利方程

\pentry{薛定谔方程(单粒子多维)\upref{QMndim},自旋角动量\upref{Spin},自旋角动量矩阵\upref{spinMt}}

在量子力学发展的早期,薛定谔首先提出了 Klein Gordon 方程,企图描绘遵从相对论变换的电子波动方程,但却遭遇失败.之后薛定谔退而求其次,转而求它的非相对论近似下的方程,得到了著名的薛定谔方程:
\begin{equation}
i\hbar \frac{\partial }{\partial t}\psi = \hat H\psi=\qty[\frac{ 1}{2m}\qty(-\frac{i}{\hbar}\nabla)^2+V(x)]\psi
\end{equation}
换言之,粒子波动方程的能量由 $\hat H=(\hat P^2/2m+V)$ 给出,而 $\hat P=-i\nabla/\hbar$ 为动量算符.

虽然非相对论性的薛定谔方程能很好描绘电子的波粒二象性,但却没有给出电子的\textbf{内禀性质},也就是说,电子是个自旋为 $1/2$ 的粒子,则波函数一定是多分量的,而非单分量的.当空间发生旋转的时候,波函数的分量会随着参考系的旋转而发生变换.直到 Pauli 给出经典电磁场中的自旋 $1/2$ 电子的 Pauli 方程,人们才终于得到了描绘电子的携带自旋信息的非相对论性方程.Pauli 方程为
\begin{equation}
i\hbar\pdv{t} \psi=\qty[\frac{1}{2m}(\bvec p-e\bvec A)^2]
\end{equation}
