% 线性算子的行列式
% license Xiao
% type Tutor

\begin{issues}
\issueDraft
\end{issues}

\pentry{线性映射的张量积\upref{linSW},行列式\upref{Deter},对偶空间\upref{DualSp},向量空间的对称/反对称幂\upref{vecSAS}}
% Giacomo:帮我看看到底是哪篇文章引用了对偶空间。。。

\subsection{体积形式}

% Giacomo:这篇文章用不到这部分结论
% \begin{theorem}{}
% 全体反对称 $k$-形式(参考对称/反对称多线性映射\upref{SASmap})构成的向量空间与 ${\large \wedge}^k V^*$ 自然同构(不依赖基的选取)。
% \end{theorem}

% \addTODO{在合适的文章中添加这个证明}
% \textbf{证明:}全体反对称 $k$-形式构成的向量空间自然同构于 $({\large \wedge}^k V)^*$\autoref{the_vecSAS_1}~\upref{vecSAS},它自然同构于 ${\large \wedge}^k V^*$(TODO:证明)。

% 因此,我们可以定义反对称 $k$-形式之间的反对称积(外积)。

我们把 $n$ 维向量空间 $V$ 上的反对称 $n$-形式称为\textbf{体积形式}。

\subsection{行列式}

$f$ 的 $k$ 阶张量幂$f^{\otimes k}: V^{\otimes k} \to W^{\otimes k}$ 保留了 $V^{\otimes k}$ 元素的对称/反对称性,
\begin{equation}
\begin{aligned}
f(v_1 \cdots v_k) &= \sum_{\sigma \in S_n} f(v_{\sigma(1)}) \otimes \cdots \otimes f(v_{\sigma(k)}) \\
&= f(v_1) \cdots f(v_k)~,
\end{aligned}~
\end{equation}

\begin{equation}
\begin{aligned}
f(v_1 \wedge \cdots \wedge v_k) &= \sum_{\sigma \in S_n} \opn{sign}(\sigma) f(v_{\sigma(1)}) \otimes \cdots \otimes f(v_{\sigma(k)}) \\
&= f(v_1) \wedge \cdots \wedge f(v_k)~.
\end{aligned}~
\end{equation}

因此可以定义 ${\large \wedge}^k f: {\large \wedge}^k V \to {\large \wedge}^k W$ 和 $\opn{Sym}^k f: \opn{Sym}^k V \to \opn{Sym}^k W$

如果 $f$ 是一个线性算子,即 $f: V \to V$,我们发现 $k$ 阶反对称幂子空间 ${\large \wedge}^k V$ 是 $f^{\otimes k}$ 的不变子空间\upref{InvSP}。



