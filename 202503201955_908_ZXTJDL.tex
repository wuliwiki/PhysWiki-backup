% 自旋统计定理(综述)
% license CCBYSA3
% type Wiki

本文根据 CC-BY-SA 协议转载翻译自维基百科\href{https://en.wikipedia.org/wiki/Spin\%E2\%80\%93statistics_theorem}{相关文章}。

自旋-统计定理证明了粒子的内禀自旋(不源于轨道运动的角动量)与该类粒子集合的量子统计性质之间的关系是量子力学数学的必然结果。在以约化普朗克常数\( \hbar \)为单位的描述下,所有在三维空间中运动的粒子具有以下特性:整数自旋的粒子服从玻色-爱因斯坦统计;半整数自旋的粒子服从费米-狄拉克统计\(^\text{[1][2]}\)。
\subsection{自旋-统计关系} 
所有已知粒子都遵循费米-狄拉克统计或玻色-爱因斯坦统计。粒子的内禀自旋总是能够预测该类粒子集合的统计性质,反之亦然 \(^\text{[3]}\):  
\begin{itemize}
\item 整数自旋的粒子是玻色子,遵循玻色-爱因斯坦统计;  
\item 半整数自旋的粒子是费米子,遵循费米-狄拉克统计。  
\end{itemize}
自旋-统计定理证明了量子力学的数学逻辑预测或解释了这一物理结果\(^\text{[4]}\)。  

对于不可区分粒子的统计性质,其影响是最基本的物理效应之一。例如:泡利不相容原理 —— 规定每个占据的量子态中最多只能容纳一个费米子,决定了物质的形成。物质的基本组成部分,如质子、中子和电子,都是费米子。另一方面,介导物质粒子之间相互作用的粒子,如光子,都是玻色子\([\text{citation needed}]\)。自旋-统计定理试图解释这一基本的二分性的起源\(^\text{[5]: 4}\)。
\subsection{背景}
从直观上看,自旋作为粒子的内禀角动量属性,似乎与该类粒子集合的基本性质无关。然而,这些粒子是不可区分的,因此涉及多个不可区分粒子的任何物理预测在交换这些粒子时都不应发生变化。
\subsubsection{量子态与不可区分粒子}
在量子系统中,物理态由态矢量描述。如果两个态矢量之间仅相差一个整体相因子(忽略其他相互作用),则它们在物理上是等效的。  

对于一对不可区分粒子,它们只有一个物理态。这意味着:如果交换粒子的位置(即进行一个排列变换),不会产生新的物理态,而是得到与原始物理态相匹配的态。实际上,无法区分交换前后的哪个粒子处于哪个位置。  

尽管物理态在粒子交换后不变,但态矢量本身可能会因交换而改变符号。由于这种符号变化只是一个整体相因子,因此它不会影响物理态。  

证明自旋-统计关系的核心因素是相对论,即:物理定律在洛伦兹变换下保持不变。场算符在洛伦兹变换下的变换方式取决于它们所创造的粒子的自旋。  

此外,还需要引入一个关键假设:微因果性假设:类空分离的场算符要么对易,要么反对易。这一假设仅适用于相对论性理论(具有时间方向的理论),否则类空的概念将失去意义。然而,证明过程中需要采用欧几里得时空的方法,在该方法中,时间方向被视为一个空间方向,具体如下:  


洛伦兹变换包括:三维旋转 洛伦兹推进 

**洛伦兹推进(boosts)** 将参考系转换为**不同速度的惯性系**,其数学表现类似于**时间方向上的旋转**。  

在**量子场论的关联函数(correlation functions)** 的解析延拓(analytic continuation)中,**时间坐标可以变为虚数**。此时:  
- **洛伦兹推进(boosts)会变为旋转(rotations)**。  
- 这样得到的新“时空”**仅具有空间方向**,称为 **欧几里得时空(Euclidean spacetime)**。