% 丘成桐(综述)
% license CCBYSA3
% type Wiki

本文根据 CC-BY-SA 协议转载翻译自维基百科\href{https://en.wikipedia.org/wiki/Shing-Tung_Yau}{相关文章}。

\begin{figure}[ht]
\centering
\includegraphics[width=6cm]{./figures/fb16385d5284d04b.png}
\caption{} \label{fig_QCT_1}
\end{figure}
丘成桐(Shing-Tung Yau,发音:/jaʊ/;中文:丘成桐;拼音:Qiū Chéngtóng;1949年4月4日出生)是一位中美籍数学家。他是清华大学丘成桐数学科学中心的主任,同时是哈佛大学的名誉教授。直到2022年,丘成桐一直担任哈佛大学威廉·卡斯帕·格劳斯坦数学教授,之后他移居清华大学。

丘成桐1949年出生于汕头,年幼时移居英国香港,1969年移居美国。他因在偏微分方程、卡拉比猜想、正能量定理和蒙热–安培方程等方面的贡献而于1982年获得菲尔兹奖。丘成桐被认为是现代微分几何和几何分析发展的主要贡献者之一。他的工作在凸几何、代数几何、计数几何、镜像对称、广义相对论、弦理论等数学和物理领域产生了深远的影响,同时他的研究也涉及到应用数学、工程学和数值分析等领域。
\subsection{传记}  
丘成桐1949年出生于中华民国广东省汕头市,父母为客家人。[YN19] 他的祖籍是中国嘉应县。[YN19] 他的母亲梁玉兰来自中国梅县区;父亲丘镇英(Chen Ying Chiu)是中华民国国民党学者,涉猎哲学、历史、文学和经济学。[YN19] 他是家中八个孩子中的第五个。[4]


