% 浮力、阿基米德原理(初中)
% keys 重力|液体|密度|浮力|等效法
% license Xiao
% type Tutor

\pentry{液体中的压强、连通器} % 词条未完成

\subsection{物体完全浸没的浮力}

\footnote{本文参考 Wikipedia \href{https://en.wikipedia.org/wiki/Archimedes'_principle}{相关页面}。}我们先用一个简单易懂的方式解释浮力。 假设在重力加速度为 $g$ 的环境中, 容器中密度为 $\rho_0$ 的液体完全静止。 这时令液体内部有一任意形状的闭合曲面, 体积为 $V_0$。
\begin{figure}[ht]
\centering
\includegraphics[width=6cm]{./figures/37c45d2588804df1.pdf}
\caption{浮力就是液体(或其他流体)对物体表面的压力的合力} \label{fig_Buoy_1}
\end{figure}
把曲面内部的液体作为一个整体做受力分析, 其质量为 $m = \rho_0 V_0$, 所受重力为 $mg = \rho_0 V_0 g$。 由于曲面中液体保持静止, 说明曲面外的液体对曲面内的液体施加的合力与它的重力等大反向,这个力就是\textbf{浮力(buoyancy)}。 现在我们如果把曲面内的液体替换为一个表面光滑的其他物体, 由于曲面形状不改变, 外界液体对该物体的浮力仍然为
\begin{equation}\label{eq_Buoy_1}
F = \rho_0 V_0 g~.
\end{equation}
这就是所谓的 “物体所受的浮力等于它排开液体的重力”。 为什么要说 $V_0$ 是 “排开水的体积” 而不直接说是 “物体的体积” 呢? 当物体完全浸没在水中时,二者的确是相等的,但下文我们还会讨论物体只有部分浸没的情况,此时二者就不相等。

虽然我们上面习惯性地使用 “水” 或 “液体” 这个词, 但任\textbf{何流体中的物体都可能受到浮力}。 例如氢气球或热气球上升就是因为在空气中受到了浮力。此时 $\rho_0$ 就是空气的密度。

为什么把水替换成物体后表面各点受到的压力不变呢? 物体表面任意一点受到的水压仅取决于深度,水是 “看不见” 物体内部有什么的。

既然浮力与物体的材料无关,为什么同样形状的物体有的会在水中上浮而有的却下沉呢? 这是因为物体的上浮或下沉取决于物体受到的合力,也就是物体受到的所有力的叠加。在这里物体除了受浮力还受重力,而重力是物体的材料和体积共同决定的。当重力大于浮力,合力向下,所以下沉。反之则上浮。所以再次强调\textbf{浮力不包括重力,与物体的材质无关}。

若物体的密度恒定,那么它所受的重力大小就是
\begin{equation}
G = mg = \rho V g~.
\end{equation}
其中 $m, V, \rho$ 分别是物体的质量、体积和密度。物体完全浸没时有 $V = V_0$。 对比\autoref{eq_Buoy_1} 我们会发现当物体密度等于水的密度时($\rho = \rho_0$),$G = F$,也就是合力为零,这代表物体可以静止在水中。 当 $\rho < \rho_0$ 时 $G < F$, 合力向上, 物体会上浮。 当 $\rho > \rho_0$ 时 $G > F$, 合力向下, 物体下沉。

\subsection{浮力的本质}

\subsection{物体露出水面的浮力}
\begin{figure}[ht]
\centering
\includegraphics[width=14.25cm]{./figures/d9a53f212729468e.pdf}
\caption{物体露出水面时的浮力,左右两个物体浮力相等} \label{fig_Buoy_2}
\end{figure}
如果物体只有部分在水中,则情况会更复杂一些。 若水的外部为真空, 那么物体表面在水面之外的部分不受任何压强。于是我们可以等效认为物体露出水面的部分不存在。这样就可以使用上文的等效方法计算浮力了。 所以使用\autoref{eq_Buoy_1} 计算该物体浮力时, $V_0$ 应该仍然是物体排开水的体积而不是物体自身的体积。


但一般来说,物体表面在水面之外的部分仍然会受到大气压强,气体同样

\begin{example}{柱体的浮力}
柱体的浮力容易通过简单的压强计算获得。 例如一个边长为 $L$ 的立方体完全浸泡在某种液体中, 其四个侧面受到的压强都是沿水平方向的, 且互相抵消, 对浮力贡献为零。 下表面受到向上的压强为 $P_1 = \rho g h_1$ 对浮力的贡献为 $F_1 = L^2 P_1$。 同理, 上表面受到向下的压强为 $P_2 = \rho g h_2$, 对浮力的贡献为 $F_2 = -L^2 P_2$。 所以总浮力为
\begin{equation}
F = F_1 + F_2 = \rho g L^2 (h_1 - h_2) = \rho g L^3 = \rho g V~,
\end{equation}
其中 $\rho$ 是液体的体积。 对于其他柱体(如圆柱,三棱柱等), 若竖直放置, 同样可以通过以上方法得到\autoref{eq_Buoy_1}。
\end{example}

\subsection{散度法}
\pentry{牛顿—莱布尼兹公式的高维拓展\upref{NLext}}

现在我们用面积分的方法表示浮力。 令 $z$ 轴竖直向上, 且水面处 $z = 0$, 则水面下压强为
\begin{equation}
P = -\rho_0 g z~.
\end{equation}
现在把上述的闭合曲面划分为许多个微面元, 第 $i$ 个面元用矢量 $\Delta \bvec s_i$, 表示, 其中模长为面元的面积, 方向为从内向外的法向。 这个面元受到外界液体的压力为
\begin{equation}
\Delta \bvec F_i = -P\Delta \bvec s_i = \rho_0 g z \Delta \bvec s_i~.
\end{equation}
现在把所有面元所受的压力求和, 并用曲面积分\upref{SurInt}表示为
\begin{equation}
\bvec F = \oint \rho_0 g z \dd{\bvec s}~,
\end{equation}
这就是物体所受的浮力。 使用\autoref{eq_NLext_3}~\upref{NLext} 得
\begin{equation}
\bvec F = \int \grad(\rho_0 g z) \dd{V} = \rho_0 g V_0 \uvec z~,
\end{equation}
可见该结论与“等效法”中得出的一致。
