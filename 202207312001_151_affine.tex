% 仿射变换在解析几何中的应用
% keys 仿射变换 圆锥曲线
\begin{definition}{仿射变换}
设椭圆\,\(\frac{x^2}{a^2}+\frac{y^2}{b^2}=1\),其中\,\(a>b>0\),置变换:
$$x'=\frac{x}{a},y'=\frac{y}{b}$$
则椭圆化为单位圆\,\(C:x'^2+y'^2=1\)
\end{definition}
届时,我们可以就可以抛开繁琐的代数计算,运用几何性质解决问题.此前,我们先介绍仿射变换的几个性质.
\begin{lemma}{}
变换后,平面内任意一条直线的斜率变为原来的\,\(\frac{a}{b}\)
\end{lemma}
\begin{lemma}{}
变换后,平面上任意区域的面积变为原来的\,\(\frac1{ab}\)
\end{lemma}
\begin{lemma}{}
变换后,线段中点依然是线段中点;关于坐标轴对称的元素依然关于坐标轴对称;平面区域的重心保持不变
\end{lemma}
\begin{lemma}{}
变换前后,平行关系保持不变
\end{lemma}
\begin{lemma}{}
变换前后,平行线段的长度比保持不变
\end{lemma}
\begin{corollary}{}
设椭圆 \(\frac{x^2}{a^2}+\frac{y^2}{b^2}\),直线 \(l\) 交椭圆于点 \(A\) 和\,\(B\),点 \(P(x_0,y_0)\) 为线段 \(AB\) 的中点,求直线斜率
\begin{figure}[ht]
\centering
\includegraphics[width=10cm]{./figures/affine_1.png}
\caption{请添加图片描述} \label{affine_fig1}
\end{figure}

\textbf{解:}作变换 \(x'=\frac{x}{a},y'=\frac{y}{b}\) ,则椭圆化为单位圆 \(C:x'^2+y'^2=1\),且 \(P'\left(\frac{x_0}{a},\frac{y_0}{b}\right)\)

故 \(k_{OP'}=\frac{ay_0}{bx_0}\)

由性质3,\(P'\) 为 \(A'B'\) 的中点

在圆中,由垂径定理,\(OP'\bot A'B'\)

而 \(k_{A'B'}\cdot k_{OP'}=-1\) ,得 \(k_{A'B'}=-\frac{bx_0}{ay_0}\)

由上述性质 1, \(k_l=\frac{b}{a}\cdot k_{A'B'}=-\frac{b^2x_0}{a^2y_0}\)
\end{corollary}
\begin{corollary}{}
设椭圆   \(\Gamma:\frac{x^2}{a^2}+\frac{y^2}{b^2}=1(a>b>0)\),直线 \(l\) 切椭圆于点 \(P(x_0,y_0)\),求直线斜率 

\textbf{解:}作变换 \(x'=\frac{x}{a},y'=\frac{y}{b}\) ,则椭圆化为单位圆 \(C:x'^2+y'^2=1\),\(P'\left(\frac{x_0}{a},\frac{y_0}{b}\right)\) 

故  \(k_{OP'}=\frac{ay_0}{bx_0}\) 

在圆中,由切线定理, \(OP'\bot l'\) 

而  \(k_{l'}\cdot k_{OP'}=-1\) ,得 \(k_{l'}=-\frac{bx_0}{ay_0}\) 

由上述性质 1, \(k_l=\frac{b}{a}\cdot k_{l'}=-\frac{b^2x_0}{a^2y_0}\) 

\end{corollary}

\begin{corollary}{}
设椭圆 \(\Gamma:\frac{x^2}{a^2}+\frac{y^2}{b^2}=1(a>b>0)\), \(A\) 、\(B\) 和 \(M\) 为椭圆上的点,点 \(A\) 和 \(B\) 关于原点对称,求证: \(k_{MA}\cdot k_{MB}\) 为一定值
​
\begin{figure}[ht]
\centering
\includegraphics[width=10cm]{./figures/affine_2.png}
\caption{请添加图片描述} \label{affine_fig2}
\end{figure}

\textbf{证明:}作变换 \(x'=\frac{x}{a},y'=\frac{y}{b}\) ,则椭圆化为单位圆 \(C:x'^2+y'^2=1\)

则 \(A'B'\) 为圆 \(C\) 的直径,所以 \(M'A'\bot M'B'\) ,\(k_{M'A'}\cdot k_{M'B'}=-1\)

由性质1, \(k_{MA}\cdot k_{MB}=-1\cdot \frac{b}{a}\cdot \frac{b}{a}=-\frac{b^2}{a^2}\)

\end{corollary}

\begin{corollary}{}
设椭圆 \(\Gamma:\frac{x^2}{a^2}+\frac{y^2}{b^2}=1(a>b>0)\), \(A\) 和 \(B\) 为椭圆上的点,点 \(M\) 为 \(AB\) 的中点,求证: \(k_{AB}\cdot k_{OM}\) 为一定值

\begin{figure}[ht]
\centering
\includegraphics[width=10cm]{./figures/affine_3.png}
\caption{} \label{affine_fig3}
\end{figure}

证明:作变换 \(x'=\frac{x}{a},y'=\frac{y}{b}\) ,则椭圆化为单位圆 \(C:x'^2+y'^2=1\)

由性质3, \(M'\) 为 \(A'B'\) 的中点,由垂径定理, \(O'M'\bot A'B'\) , \(k_{OM'}\cdot k_{A'B'}=-1\)

由性质1, \(k_{OM}\cdot k_{AB}=-1\cdot \frac{b}{a}\cdot \frac{b}{a}=-\frac{b^2}{a^2}\)

\end{corollary} 

\begin{corollary}{等角定理}
设椭圆 \(\Gamma:\frac{x^2}{a^2}+\frac{y^2}{b^2}=1(a>b>0)\), \(A\) 和 \(B\) 为椭圆上的点,直线 \(AB\) 交 \(x\) 轴于点 \(P(x_0,0)\) ,在 \(x\) 轴上求一点 \(G\) ,使 \(x\) 轴平分 \(\angle AGB\) 

\begin{figure}[ht]
\centering
\includegraphics[width=14.25cm]{./figures/affine_4.png}
\caption{请添加图片描述} \label{affine_fig4}
\end{figure}
\textbf{解:}作变换 \(x'=\frac{x}{a},y'=\frac{y}{b}\) ,则椭圆化为单位圆 \(C:x'^2+y'^2=1\), \(P'\left(\frac{x_0}{a},0 \right)\) 

由性质3, \(\angle A'G'B'\) 依然被 \(x\) 轴平分,记 \(B'G'\) 交圆 \(O\) 于另一点 \(D'\)

易见 \(\angle A'OG'=\frac{1}{2}\cdot\angle A'OD'=\angle A'B'D'\) 

又 \(\angle A'P'O=\angle B'P'G'\)

所以 \(\angle OA'P'=\angle OG'B'\)

又 \(\angle A'G'O=\angle B'G'O\)

故 \(\angle OA'B'=\angle A'G'O\)

又 \(\angle A'OG'=\angle A'OG'\)

故 \(\triangle A'OG'\sim \triangle P'OA'\)

进而 \(\frac{OP'}{OA'}=\frac{OA'}{OG'}\)

故 \(OP'\cdot OG'=OA'^2=1\) 

即 \(G'\left(\frac{a}{x_0},0\right)\)

由性质1, \(G\left( \frac{a^2}{x_0},0\right)\) 

\end{corollary}

\begin{corollary}{}
过椭圆 \(\Gamma:\frac{x^2}{a^2}+\frac{y^2}{b^2}=1\) 上任一点 \(A(x_0,y_0)\) 作两条倾斜角互补的直线交椭圆于点 \(P\),\(Q\) ,求证: \(k_{PQ}\) 为一定值
\begin{figure}[ht]
\centering
\includegraphics[width=14.25cm]{./figures/affine_5.png}
\caption{请添加图片描述} \label{affine_fig5}
\end{figure}
\textbf{证明:}作变换 \(x'=\frac{x}{a},y'=\frac{y}{b}\) ,则椭圆化为单位圆 \(C:x'^2+y'^2=1\) , \(A\left(\frac{x_0}{a},\frac{y_0}{b}\right)\) 

由性质3, \(A'P'\),\(A'Q'\) 仍然关于铅垂线对称,故 \(\angle P'A'B'=\angle OA'B'\)

同弧所对的圆周角等于圆心角的一半,故 \(\angle P'OB'=\angle Q'OB'\)

又 \(OP'=OQ'\) ,等腰三角形三线合一,所以 \(OB'\bot P'Q' \), \(k_{P'Q'}\cdot k_{OB'}=-1\)

又因为 \(k_{OA'}=-k_{OB'}\) ,所以 \(k_{OA'}\cdot k_{P'Q'}=1\)

而 \(k_{OA'}=\frac{a\cdot y_0}{b\cdot x_0}\) ,因此 \(k_{P'Q'}=\frac{b\cdot x_0}{a\cdot y_0}\)

由性质1, \(k_{PQ}=\frac{b^2\cdot x_0}{a^2 \cdot y_0}\) 
\end{corollary}
\begin{corollary}{}
求椭圆 \(\frac{x^2}{a^2}+\frac{y^2}{b^2}=1\) 内接三角形的最大面积和外切三角形的最小面积
\begin{figure}[ht]
\centering
\includegraphics[width=14.25cm]{./figures/affine_6.png}
\caption{请添加图片描述} \label{affine_fig6}
\end{figure}
\textbf{解:}作变换 \(x'=\frac{x}{a},y'=\frac{y}{b}\) ,则椭圆化为单位圆 \(C:x'^2+y'^2=1\)

由琴生不等式:
$$S'_{\text{ins}}\leq \frac{3}{2}\cdot \sin\frac{\pi}{3}=\frac{3\sqrt{3}}{4}, S'_{\text{ext}}\geq 3\cot\frac{\pi}{6}=3\sqrt{3}$$ 

由性质2: 
$$S_{\text{ins}\max}=\frac{3\sqrt{3}}{4}\cdot ab, S_{\text{ext}\max}=3\sqrt{3}\cdot ab$$ 

\end{corollary}
\begin{corollary}{}

\end{corollary}
