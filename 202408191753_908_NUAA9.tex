% 南京航空航天大学 2004 量子真题答案
% license Usr
% type Note

\textbf{声明}:“该内容来源于网络公开资料,不保证真实性,如有侵权请联系管理员”


\subsection{一}
1. 解:\\
(1) 因为 $\hat{p} \psi = p\left[e^{\frac{i}{\hbar}(PX - Et)} - e^{-\frac{i}{\hbar}(PX + Et)}\right] \neq p\psi$\\
所以 $\psi$不是动量算符$\hat{p}$的本征函数。\\\\
(2) 因为$\hat{p}^2 \psi = p^2\left[e^{\frac{i}{\hbar}(PX - Et)} + e^{-\frac{i}{\\hbar}(PX + Et)}\right] = p^2\psi$ \\
所以 $\psi$是动量平方算符$\hat{p}^2$ 的本征值函数。 

2.解:\\
$(\hat{p} + \hat{x})\psi = \lambda \psi$\\
即:$(-i\hbar\frac{\partial}{\partial x} + x)\psi = \lambda \psi \quad \frac{\hbar}{i} \frac{d\psi}{dx} = (\lambda - x)\psi$ \\
所以$\frac{\hbar}{i} \frac{d\psi}{\psi} = (\lambda - x)dx$\\
附注: 本征函数 $\psi = Ae^{\frac{1}{\hbar}(\lambda x - \frac{x^2}{2})}$ \\
      本征值 $\lambda$ 为所有实数。

3.解:
\begin{align}
\overline{V}(r) &= -\frac{e^2}{4 \pi \epsilon_0} \int_{-\infty}^{+\infty} \psi^* \frac{1}{r} \psi \, d\tau 
= -\frac{e^2}{4 \pi \epsilon_0} \int_0^\infty \frac{1}{r} |\psi|^2 \, d\tau\\
&= \left( -\frac{e^2}{4 \pi \epsilon_0} \right) \left( \frac{1}{\pi a_0^3} \right) \int_0^\pi \sin \theta \, d\theta \int_0^{2\pi} \, d\phi \int_0^\infty \frac{1}{r} r^2 e^{-\frac{2}{a_0}^2} \, dr\\
&= -\frac{1}{4 \pi \epsilon_0} \frac{e^2}{a_0}~
\end{align}
