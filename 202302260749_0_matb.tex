% SLISC 的 matt/matb 文件格式
% Matlab|文件储存|文本文件|二进制文件

\begin{issues}
\issueTODO
\end{issues}

\pentry{SLISC 简介}

SLISC 库提供了两种特殊的文件格式用于储存各种变量和矩阵, 每个文件可以存如多个不同的变量(包括矩阵)并指定名称, 读取时可以指定要读取的变量名。 两种文件分别使用拓展名 \verb|.matt| 和 \verb|.matb|。 其中 \verb|mat| 表示 matrix, \verb|t| 表示 text, \verb|b| 表示 binary。 顾名思义, 第一种是文本文件, 第二种是二进制文件。 二进制文件具有占用硬盘空间少(大约是文本文件的 1/3), 读写速度快的优点。 文件格式简单, 在任何其他语言中都可以轻易地写出读写程序。

以下给一个 matb 文件的读写例子, 也可以把例子中的 matb 换成 matt。

\begin{lstlisting}[language=cpp]
#include "SLISC/matb.h"
using namespace slisc;

int main()
{
    // 随便初始化一些变量, 目前支持 SLISC 库中的绝大部分标量和密矩阵
    Int i = 1; Doub d = 3.1; Comp c(1, 2);
    VecInt v(3); linspace(3, 1, 3); CmatInt a(2,2); linspace(a);

    Matb matb("test.matb", "w"); // 打开 matb 文件, 使用写入模式
    // 相当于 Matb matb; matb.open("test.matb", "w");
    save(i, "i", matb); // 保存变量, 指定变量名, 可以是任意字符串
    save(d, "d", matb); // 注意同个文件中保存的变量名不能重复
    save(c, "c", matb);
    save(v, "c", matb);
    save(a, "a", matb);
    matb.close(); // 关闭文件。
    // 在 destructor 中会自动调用, open() 时也会自动调用

    Int i1; Doub d1; Comp c1; VecInt v1(3); CmatInt a1;
    matb.open("test.matb", "w"); // 重新用读取模式打开文件
    // 读取变量, 可以按照任何顺序, 不需要全部读取
    // 矩阵会被自动 resize()
    load(i1, "i", matb); 
    load(d1, "d", matb);
    load(c1, "c", matb);
    load(v1, "c", matb);
    load(a1, "a", matb);
    matb.close();

    // 此时可以对比读取的变量是否和保存的变量一致
}
\end{lstlisting}

\subsection{文件格式}
我们先介绍二进制格式 matb。 matb 文件使用 little endian 储存(链接未完成, 通常的 x86 处理器都满足)。 所有整数都用 \verb|long long| (8 字节有符号整型)储存,  每个变量的前 8 个字节是变量名的长度 \verb|Nname|, 然后是 \verb|Nname| 字节的变量名, 紧接着是 8 个字节的矩阵元类型编号(对照表未完成, 见 \verb|SLISC0/preprocessor/type_num.m|), 8 个字节的矩阵维度 \verb|Ndim| (标量的维数是 0, 向量维度是 1) 接着 \verb|8*Ndim| 字节记录每个维度的长度(标量不需要)。 最后, 再记录 \verb|N * sizeof(元素类型)|, 其中 \verb|N| 是元素个数, 等于各个维度的长度相乘。

在文件的最后, 若一共有 \verb|Nvar| 个变量, 那么就用 \verb|Nvar*8| 字节记录下每个变量的第一个字节在文件中的字节数(第一个变量总是从第 0 个字节开始)(未完成:目前是倒序的)(另外这些信息只对 matt 有必要), 然后再用 8 字节记录下 \verb|Nvar|, 这是为了读取时查找方便。 最后为了验证文件的完整性, 用 16 字节写入字符串 \verb|"Matb_End_of_File"|。 该字符串用于读取时验证文件的完整性, 若不存在则说明文件写入没有完成。 事实上 \verb|matb| 不需要记录该信息, 因为逐个变量跳查已经足够快了。 \verb|matt| 使用该信息则可以加快查找变量的速度(因为每个矩阵元在文件中的长度是不等的)。

对于文本格式 \verb|matt|, 每个整数或浮点数直接转换成字符串写入文本, 每两个数之间使用空格隔开, 复数则写成两个相邻的数(未完成:现在代码并不是这么做的, 而是类似 \verb|1+2i|)。

\subsection{Matb 类}
\begin{lstlisting}[language=cpp]
// all non-data intergers in file are Llong
class Matb {
public:
    Matb();
    Matb(Str_I fname, Char_I rw);
    Char m_rw; // 模式
    ifstream m_in;
    ofstream m_out;
    Llong m_filesize; // 文件大小
    Str m_fname; // 文件名
    vecStr m_name; // 变量名
    vecLlong m_type; // 变量类型
    vector<vecLlong> m_size; // 矩阵尺寸
    vecStr m_data; // 矩阵元二进制数据(仅 "m" 模式使用)
    vecLlong m_ind; // 变量位置

    void open(Str_I fname, Char_I rw); // 打开文件
    Bool isopen(); // 文件是否打开

    // 内部函数
    void read_data(); // 读取到 m_data
    void write_data(Str_I fname); // 写入到 m_data
    Long size() const; // 变量个数
    Long data_size(Long_I i) const; // 第 i 个变量的 m_data 长度
    void get_profile(); // 读取变量信息
    Long data_pos(Long_I var_ind) const; // 第 i 个变量在文件中的位置
    Long search(Str_I name); // 搜索变量编号, 如果不存在返回 -1
    ~Matb();
};
\end{lstlisting}

\subsection{修改模式}
打开 matb 的模式除了 \verb|"r", "w"| 外还可以是 \verb|"m"|, 表示 modify。 在该模式下, 打开文件后可以对变量进行增加, 删除, 修改等操作。 该功能尚未完善。

\subsection{其他工具}
可执行文件 \verb|SLISC/util/matbinfo| 可以显示 matb 文件中的所有变量信息, 用法是 \verb|matbinfo 文件名.matb|。

可执行文件 \verb|SLISC/util/matt2matb| 和 \verb|matb2matt| 可以把 matt 文件和 matb 文件互相转换, 例如 \verb|matb2matt -d -r 文件名.matb| 会生成 \verb|文件名.matt|。 其中 \verb|-d| 选项在转换后把原文件删除, 如果 \verb|文件名.matt| 已经存在, \verb|-r| 选项会重新转换并将其替换。

可执行文件 \verb|SLISC/util/matb_q2d| 可以把 matb 文件中所有四精度数值转换为双精度(包括 \verb|Qdoub| 和 \verb|Qcomp|)。 这样可以方便在不支持四精度的语言中读取(例如 Matlab)。

Matlab 程序 \verb|SLISC/util/mattload.m| 和 \verb|matbload.m| 分别可以在 Matlab 中读取 \verb|matt| 和 \verb|matb| 文件。
