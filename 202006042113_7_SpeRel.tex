% 光速不变原理
% 尺缩|钟慢|狭义相对论|洛伦兹变换|伽利略变换

光速不变问题.

许多实验(如迈克尔逊—莫雷实验)% 未完成
发现, 真空中的光速在不同参考系下都具有相同的值, 我们把它定义为\footnote{按照最新的国际单位标准, 先人为定义光速的数值, 再由原子钟定义时间单位 “秒”, 再由二者定义 “米” 的长度. 但在此前, 是先定义 “米” 和 “秒”, 再由实验测出光速, 但这种方式并没有新的定义准确.}
\begin{equation}
c = 299792458\Si{m/s}
\end{equation}

这个实验事实不符合牛顿的时空观, 即伽利略变换. 因为

% 未完成: 应该把 “洛伦兹变换” 中的相关内容放到 “狭义相对论” 中
