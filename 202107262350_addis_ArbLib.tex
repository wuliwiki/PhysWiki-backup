% Arb 任意精度计算库

\pentry{C++ 基础\upref{Cpp0}}

Arb 是一款支持任意精度计算的 C/C++ 程序, 支持对计算进行严谨的误差估计, 即每个任意精度浮点数 $z$ 都会伴随一个误差半径 $r$, 使得精确结果必定落在复平面上以 $z$ 为圆心半径为 $r$ 的圆盘中. 通过增加浮点数的精度, 就可以用数值方法无限逼近理论值. Arb 还提供了许多特殊函数的计算, 例如 $\Gamma$ 函数\upref{Gamma}, 超几何函数\upref{HypGeo} 等, 以及常用的线性代数功能和离散傅里叶变换\upref{DFT}等. 官方主页 \href{https://arblib.org/}{arblib.org} 包含详细的文档.

\subsection{安装}
以 Ubuntu 为例, 最方便的安装方式就是使用 \verb|apt| 安装. 首先安装 dependency
\begin{equation}
sudo apt install libmpfr-dev libflint-dev
\end{equation}
其中 \verb|MPFR| 和 \verb|flint| 两个包分别用于任意精度浮点数以及数论. 然后安装 Arb
\begin{lstlisting}[language=bash]
sudo apt install libflint-arb-dev libflint-arb2
\end{lstlisting}

但目前这并不是 Arb 的最新版本(例如没有实现库仑函数\upref{CulmF}). 要获得最新版本, 可以直接从 \href{https://github.com/fredrik-johansson/arb/}{GitHub} 下载源码编译即可(默认使用 gcc 编译器). 目前笔者使用的版本是 release 2.19.0.

可以用 \verb|./configure --help| 查看编译选项, 若所有的包都安装在默认目录则不需要编译选项.
\begin{lstlisting}[language=bash]
./configure [编译选项];
make -j4;
sudo make install;
\end{lstlisting}
其中 \verb|-j4| 是使用 4 线程进行编译, 也可以改成其他数字.

\subsection{编译}
在 Ubuntu 中如果你使用 \verb|apt| 安装, 在编译程序是需要加 \verb|-lflint-arb| 选项. 如果你直接从源码编译, 则需要加 \verb|-larb| 选项. liru
