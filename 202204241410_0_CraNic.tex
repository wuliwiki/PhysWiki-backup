% Crank-Nicolson 算法解一维含时薛定谔方程
% keys 算法|薛定谔方程|数值解|偏微分方程

\begin{issues}
\issueDraft
\end{issues}

\pentry{薛定谔方程\upref{TDSE}}

\footnote{参考 \cite{NR3}.}薛定谔方程为
\begin{equation}
-\frac12 \pdv[2]{\psi}{x} + V\psi = \I \pdv{\psi}{t}
\end{equation}
用 Crank-Nicolson 或 Caley scheme\footnote{二者是一回事, 见 \cite{NR3} 19.2 节.} 得到的结果是
\begin{equation}\label{CraNic_eq2}
\qty(1+\frac{\I}{2}\mat H\Delta t)\bvec\psi_{n+1} = \qty(1-\frac{\I}{2}\mat H\Delta t)\bvec\psi_n
\end{equation}
其中 $\bvec\psi_n$ 是时刻 $t_n$ 的波函数列矢量(已知), $\bvec\psi_{n+1}$ 为时刻 $t_{n+1}$ 的未知矢量.

但事实上, 还可以继续减少计算量. 将\autoref{CraNic_eq2} 整理后得
\begin{equation}\label{CraNic_eq5}
\qty(\frac12 + \frac{\I}{4}\mat H\Delta t)\qty(\bvec\psi_{n+1}+\bvec\psi_n) = \bvec\psi_n
\end{equation}
解这个方程, 再减去 $\bvec \psi_n$ 即可.

\subsection{等间距网格}
其中二阶导数用三点差分计算, 令 $\psi_{i,n} = \psi(x_i, t_n)$, $V_{i,n} = V(x_i, t_n)$ 得
\begin{equation}
\ali{
\psi_{i,n+1} - \psi_{i,n} &= \frac{\I\Delta t}{4\Delta x^2} (\psi_{i-1,n} - 2\psi_{i,n} + \psi_{i+1, n} + \psi_{i-1,n+1} - 2\psi_{i,n+1} + \psi_{i+1,n+1})\\
&\qquad\qquad - \frac{\I\Delta t}{2}(V_{i,n}\psi_{i,n} + V_{i,n+1}\psi_{i,n+1})
}\end{equation}
令 $\alpha = \I\Delta t/(4\Delta x^2), \beta = \I\Delta t/2$, 整理可得
\begin{equation}\label{CraNic_eq4}
\ali{
&\quad -\alpha\psi_{i-1,n+1} + (1+2\alpha + \beta V_{i,n+1})\psi_{i,n+1} - \alpha \psi_{i+1,n+1}\\
&= \alpha\psi_{i-1,n} + (1 - 2\alpha - \beta V_{i,n})\psi_{i,n} + \alpha \psi_{i+1,n}
}\end{equation}

我们把一个区间划分成 $N_x - 1$ 段等长的区间, 并令 $N_x$ 个格点为 $x_1\dots x_{N_x}$. 最简单的边界条件是取 $\psi(x_1) = \psi(x_{N_x}) = 0$. 这样\autoref{CraNic_eq4} 中的 $i$ 可以取 $i = 2\dots N_x - 1$, 得到 $N_x - 2$ 条式子, 其中只有 $\psi_{2,n+1}\dots \psi_{N_x-1,n+1}$ 这 $N_x - 2$ 个未知量, 每条式子最多包含连续 3 个未知量. 将线性方程用矩阵表示, 就可以得到一个三对角矩阵(第一行和最后一行只有两个系数).



\subsection{虚时间}
使用虚时间后, \autoref{CraNic_eq2} 和\autoref{CraNic_eq5} 分别变为
\begin{equation}
\qty(1+\frac12\mat H\Delta t)\psi_{n+1} = \qty(1-\frac12\mat H\Delta t)\psi_n
\end{equation}
\begin{equation}
\qty(\frac12 + \frac14\mat H\Delta t)\qty(\psi_{n+1}+\psi_n) = \psi_n
\end{equation}

(代码未完成)
