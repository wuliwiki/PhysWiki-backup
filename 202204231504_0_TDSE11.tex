% 含时薛定谔方程(单粒子一维)


当哈密顿算符 $H$ 不随时间变化时, 我们说这个系统能量守恒\upref{QMcons}. 这时我们可以用分离变量法,令
\begin{equation}
\Psi(\bvec r, t) = \psi(\bvec r) T(t)
\end{equation}
 
\begin{equation}\label{TDSE11_eq1}
H\psi = E\psi
\end{equation}
以及
\begin{equation}\label{TDSE11_eq2}
\I \hbar\pdv{t}T = ET
\end{equation}
其中\autoref{TDSE11_eq1} 就是定态薛定谔方程\upref{SchEq}, 即哈密顿算符的本征方程. \autoref{TDSE11_eq2} 有简单的解
\begin{equation}
T(t) = \E^{-\I E t/\hbar}
\end{equation}

根据 $H$ 的不同, 本征值 $E$ 可以取离散或连续的值. 先看只取离散值的简单情况(如无限深势阱\upref{ISW}), 令能级为 $E_n$ ($n = 1, 2, \dots$), 那么含时薛定谔方程的通解为
\begin{equation}\label{TDSE11_eq5}
\Psi(\bvec r, t) = \sum_n C_n \psi_n(\bvec r) \E^{-\I E_n t/\hbar}
\end{equation}
其中 $C_n$ 为待定系数, 由初始条件决定. 一个简单的例子见 “无限深势阱中的高斯波包\upref{wvISW}”.

如果 $E$ 只在某个区间内取连续值, 我们同样可以使用分离变量法, 只是求和变为积分, 系数变为能量的函数
\begin{equation}
\Psi(\bvec r, t) = \int C(E) \psi_E(\bvec r) \E^{-\I E t/\hbar} \dd{E}
\end{equation}
若能量是简并的, 即定态薛定谔方程中一个能量 $E$ 由多个线性无关的解 $\psi_{E,i}(x)$ ($i=1,2\dots$), 那么上式变为
\begin{equation}\label{TDSE11_eq4}
\Psi(\bvec r, t) = \int \sum_i C_i(E) \psi_{E,i}(\bvec r) \E^{-\I E t/\hbar} \dd{E}
\end{equation}

一个经典的例子见 “一维自由粒子(量子)\upref{FreeP1}” .
