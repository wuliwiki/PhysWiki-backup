% 乔治·比德尔·艾里(综述)
% license CCBYSA3
% type Wiki

本文根据 CC-BY-SA 协议转载翻译自维基百科 \href{https://en.wikipedia.org/wiki/George_Biddell_Airy}{相关文章}。

\begin{figure}[ht]
\centering
\includegraphics[width=6cm]{./figures/f2f2a06f3fcb14e2.png}
\caption{} \label{fig_AL_1}
\end{figure}
乔治·比德尔·艾里爵士(Sir George Biddell Airy,/ˈɛəri/;1801年7月27日-1892年1月2日)是一位英国数学家和天文学家,曾于1826年至1828年担任卢卡斯数学教授,并于1835年至1881年担任第七任皇家天文学家。他的诸多成就包括对行星轨道的研究、测定地球平均密度、提出二维固体力学问题的解法,以及在担任皇家天文学家期间,确立格林尼治作为本初子午线的所在地。
\subsection{传记}
艾里出生于诺森伯兰郡的安尼克,出身于一个世代相传的艾里家族,其家族可追溯至14世纪居住在西摩兰郡肯特米尔的同姓家族。他所属的这一支系因英国内战而遭受打击,遂迁至林肯郡并以务农为生。艾里的早期教育先是在赫里福德的初等学校完成,随后进入科尔切斯特皇家文法学校就读。\(^\text{[1]}\)艾里是个性格内向的孩子,但因擅长制作豌豆射手而在同学中颇受欢迎。\(^\text{[2]}\)

从13岁起,艾里经常住在他位于萨福克郡普莱福德的叔叔亚瑟·比德尔家中。比德尔将艾里介绍给了他在普莱福德庄园的朋友、废奴主义者托马斯·克拉克森。克拉克森是剑桥大学的数学硕士,先是考察了艾里的古典学水平,随后又请剑桥三一学院的一位研究员对他进行数学能力的考核。\(^\text{[3][4]}\)结果,艾里于1819年以“资助生”的身份进入三一学院学习,这意味着他缴纳较低的学费,但必须通过承担部分服务工作来抵消差额。\(^\text{[5]}\)在那里,他的学术表现十分出色,几乎立刻就被认为是当届最优秀的学生。1822年他当选为三一学院奖学金获得者,次年以“高级状元”身份毕业,并获得史密斯一等奖。

1824年10月1日,他被选为三一学院研究员;1826年12月,接替托马斯·特顿(Thomas Turton)担任卢卡斯数学教授。此职位他仅担任了一年多,随后于1828年2月当选为普卢米安天文学教授,并出任新成立的剑桥天文台台长。\(^\text{[1]}\)1836年,他当选为皇家学会院士;1840年当选为瑞典皇家科学院外籍院士;1859年又成为荷兰皇家艺术与科学学院外籍院士。\(^\text{[6]}\)
