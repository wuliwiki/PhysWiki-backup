% 东南大学 2013 年 考研 量子力学
% license Usr
% type Note

\textbf{声明}:“该内容来源于网络公开资料,不保证真实性,如有侵权请联系管理员”

\textbf{1.(15 分)}以下叙述是否正确:

(1) 若厄密算符 $\hat{A}$ 与 $\hat{B}$ 对易,则它们必有共同本征态;

(2) 仅当体系处在定态时,守恒量的平均值才不随时间变化;

(3) 一维谐振子的能量本征态既有束缚态,也有散射态;

(4) 厄密算符的本征值必为正数;

(5) 空间平移对称性导致动量守恒。

\textbf{2.(15 分)}质量为 $m$ 的粒子处于 $\delta$ 势阱中,$V(x) = -\gamma \delta(x)$, $(\gamma > 0)$。

(1) 试根据 Schrödinger 方程证明 $x=0$ 处波函数的跃变条件为
$$\psi'(0^+) - \psi'(0^-) = -\left(2m\gamma/\hbar^2\right)\psi(0);~$$

(2) 试求束缚态能级和相应的归一化能量本征函数。

\textbf{3.(15 分)}一质量为 $m$ 的粒子以能量 $E$ 从左往右入射,受到以下势场的散射
$$V(x) =\begin{cases} V_0, & (x < 0) \\\\0, & (x > 0) \end{cases}\quad (V_0 > 0)~$$

在以下两种情况下计算反射系数和透射系数:
(1) $E > V_0$;
(2) $0 < E < V_0$。

提示:一维几率流密度公式为 
$$j(x) = \left(\hbar/i2m\right)\left(\psi^* \partial \psi/\partial x - \psi \partial \psi^*/\partial x\right)~$$

\textbf{4.(15 分)}试证 Bloch 函数
$$\psi_k(r) = \exp(ik \cdot r)\phi_k(r), \quad \phi_k(r) = \phi_k(r + a),~$$
是平移算符 $\hat{D}(a) = \exp\left(-ia \cdot \hat{p}/\hbar\right)$ 的本征态,相应的本征值为 $\exp\left(-ik \cdot a\right)$。

\textbf{5.(15 分)}设体系的 2 个粒子可处于 3 个单粒子态 $\phi_i(q), \phi_j(q), \phi_k(q)$。在以下三种情况下求体系可能的量子态数目:
(1) 不同粒子;
(2) 全同 Bose 粒子;
(3) 全同 Fermi 粒子。

\textbf{6.(15 分)}一质量为 $m$ 带电量为 $q$ 的粒子在均匀电场 $E = (0, \epsilon, 0)$ 和均匀磁场 $B = (0, 0, B)$ 中运动,磁场的矢量势选为 $A = (-By, 0, 0)$。

(1) 写出粒子的哈密顿算符 $\hat{H}$,并证明动量 $\hat{p}_x$ 和 $\hat{p}_z$ 均为守恒量;

(2) 求守恒量完全集 $(\hat{H}, \hat{p}_x, \hat{p}_z)$ 的共同本征函数及相应的本征值。

\textbf{7.(15 分)}两个电子的总角动量为 $\hat{S} = \hat{s}_1 + \hat{s}_2$,令 $\hat{P}_{12} = (1 + \hat{\sigma}_1 \cdot \hat{\sigma}_2)/2$,试求:

(1) $\hat{P}_{12}^2$;

(2) $\hat{P}_{12}^2 - \hat{S}^2/\hbar^2$;

(3) $\hat{P}_{12} \ket{SM}$,其中 $\ket{SM}$ 为 $\hat{S}^2$ 和 $\hat{S}_z$ 的共同本征态。

\textbf{8.(15 分)}某一维简单晶格的原子总数为 $N$,相邻原子间距离为 $a$,在简谐近似下的哈密顿算符可表为
$$\hat{H} = \sum_k \left(\hat a_k^\dagger \hat a_k + \frac{1}{2}\right)\hbar \overline{\omega}_k~$$
其中 $\hat a_k$ 为声子的消灭算符,$[\hat a_k, \hat a_k^\dagger] = \delta_{kk'}$。已知声子的色散关系为
$$\overline{\omega}_k = \overline{\omega}_0 |\sin(ka/2)|, \quad (-\pi/a \leq k < \pi/a)~$$
试求:
(1) 体系的基态能 $E_0$;
(2) 声子的态密度 $g(\omega)$。

\textbf{9.(15 分)}设一维谐振子的能量本征态为 $\ket{n}, (n = 0, 1, 2, \dots)$,微扰哈密顿算符为 $\hat{H}' = -k\hat{x}$。试用微扰论求能级的修正(准确到二级近似)。提示:非简并微扰论的能级修正公式为:
$$\hat{E}_k = E_k^{(0)} + \langle k|\hat{H}'|k\rangle + \sum_{m\neq k} \frac{\langle k|\hat{H}'|m\rangle \langle m|\hat{H}'|k\rangle}{E_k^{(0)} - E_m^{(0)}}~$$
$$\langle n'|\hat{x}|n \rangle = \sqrt{\frac{\hbar}{m\omega}} \left( \sqrt{\frac{n+1}{2}} \delta_{n',n+1} + \sqrt{\frac{n}{2}} \delta_{n',n-1} \right)~$$ 

\textbf{10.(15 分)}一量子体系哈密顿算符为 $\hat{H}_0$,$\hat{H}_0|n\rangle = E_n |n\rangle$,$\langle n'|n\rangle = \delta_{nn'}$。

$$\sum_n |\langle n|\psi(t=1)\rangle|^2 = 0~$$
时刻体系处于 $|k\rangle$ 态,受到以下微扰作用

$$\hat{H}'(t) = \begin{cases} 0, & t < 0 \\\\\hat{H}', & t > 0 \end{cases}~$$

试在一级近似下求 $t \rightarrow \infty$ 时的跃迁速率 $\omega_{kk'}$。

提示:量子跃迁几率公式
$$P_{kk'}(t) = \frac{1}{\hbar^2} \left| \int_0^t dt' \langle k|\hat{H}'|k'\rangle e^{i\omega_{kk'}t'} \right|^2~$$
$$\omega_{kk'} = (E_k - E_{k'})/\hbar~$$

$$\frac{\sin^2(\alpha x)}{x^2} \xrightarrow{\alpha \rightarrow \infty} \pi \alpha \delta(x),~$$