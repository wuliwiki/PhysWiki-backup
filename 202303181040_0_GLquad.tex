% Gauss-Lobatto 积分
% 高斯积分|数值积分|多项式

\begin{issues}
\issueAbstract
\end{issues}

\pentry{定积分\upref{DefInt}, 换元积分法\upref{IntCV}}

\footnote{参考 Wikipedia \href{https://en.wikipedia.org/wiki/Gaussian_quadrature}{相关页面}。}\textbf{高斯积分(Gauss quadrature)}可以用求和来近似表示积分
\begin{equation}
\int_{-1}^1 f(x) \dd{x} \approx \sum_i w_i f(x_i)~.
\end{equation}
对于某区间的一组 $x_i, w_i$($i = 1,\dots,N$), 那么当 $f(x)$ 是小于等于 $2N-1$ 阶的多项式时, 上式取等号。

我们可以用换元积分法把上述定积分的区间变为任意 $[a,b]$。 令
\begin{equation}
t = \frac{b-a}{2}x + \frac{b+a}{2}
\end{equation}
那么换元得(过程略)
\begin{equation}
\int_a^b g(t) \dd{t} \approx \sum_i w'_i g(t_i)~.
\end{equation}
其中
\begin{equation}
g(t) = \frac{2}{b-a} f\qty(\frac{2t - b - a}{b - a}) = \frac{2}{b-a} f(x)~,
\end{equation}
\begin{equation}
t_i = \frac{b-a}{2}x_i + \frac{b+a}{2}
\end{equation}
\begin{equation}
\omega'_i = \frac{b-a}{2}\omega~.
\end{equation}


\subsection{Gauss-Lobatto 积分}
Gauss-Lobatto 积分中令采样点(包括两个端点)的个数为 $N$, 如果 $f(x)$ 是 $2N-3$ 阶多项式($f(x) = x^{2N-3} + \dots$), 则积分没有误差。

注意 Gauss-Lobatto 积分是对称的
\begin{equation}\label{GLquad_eq5}
x_i = -x_{N-i+1}, \qquad w_{i} = w_{N-i+1}~.
\end{equation}

其中 $x_0 = -1, x_N = 1$, 剩下的 $x_i$ ($1 < i < N$)是 $P'_{N-1}(x)$ 的根\footnote{$P_N(x)$ 在 $[-1,1]$ 有 $l$ 个根。 $P'_N(x)$ 有 $N-1$ 个根, 所以 $P'_{N-1}(x)$ 有 $N-2$ 个根, 加上两个端点就是 $N$ 个。}, $P_N(x)$ 是勒让德多项式\upref{Legen}。 另见\autoref{Legen_eq6}~\upref{Legen}。
\begin{equation} % 已数值验证
w_i = \frac{2}{N(N-1)[P_{N-1}(x_i)]^2}
\end{equation}

低阶情况下 $x_i$ 可以表示为带根号的表达式, 在 Mathematica\upref{Mma} 中求解解析式和任意精度数值解如\begin{lstlisting}[language=mma, caption=gauss\_lobatto.nb]
(*节点数*) NN = 6;
(*有效数字*) digits = 38;
sol = Solve[D[LegendreP[NN - 1, x], x] == 0, x];
sol = Append[sol, {x -> -1}];
sol = Append[sol, {x -> 1}];
N[sol, digits]
w = N[2/(NN (NN - 1) (LegendreP[NN - 1, x])^2) /. sol, digits]
\end{lstlisting}
其中 \verb|sol| 得到的是 list of rule, \verb|/.| 用于把这些 rule 作用到前面的表达式上面。

\subsection{正交归一基底}
每个基底都是 $N-1$ 阶多项式, 由于阶数和零点数一样, 多项式可以唯一确定, 即拉格朗日插值多项式
\begin{equation}\label{GLquad_eq2}
\ali{
p_n(x) &= \prod_{i=1}^{n-1} \frac{x-x_i}{x_n-x_i} \prod_{i=n+1}^{N} \frac{x-x_i}{x_n-x_i}\\
&= \frac{x-x_1}{x_n-x_1} \times\dots\times\frac{x-x_{n-1}}{x_n-x_{n-1}}\frac{x-x-{n+1}}{x_n-x_{n+1}} \dots \frac{x-x_N}{x_n-x_N}~,
}\end{equation}
\begin{equation}
p_n(x_{n'}) = \delta_{n, n'}~.
\end{equation}

由于任意两个基底乘积只是 $2N-2$ 阶的多项式, 所以可以用求和精确代替其积分。
\begin{equation}
\int_{-1}^1 p_i(x) p_j(x) \dd{x} = \sum_k w_k p_i(x_k) p_j(x_k) = w_i \delta_{ij}
\end{equation}

所以 $N$ 个正交归一基底就是
\begin{equation}\label{GLquad_eq3}
f_n(x) = \frac{1}{\sqrt{w_n}} p_n(x)
\end{equation}
满足
\begin{equation}
f_i(x_j) = \frac{1}{\sqrt{w_i}} \delta_{ij}
\end{equation}

另一种等效的表示方法是利用 $N$ 阶 Gauss-Lobatto 数值积分对应的多项式, $N-1$ 阶勒让德多项式的导数, $P'_{N-1}(x)$,  来构建满足条件的多项式。 根据定义,其 $N-2$ 个零点分别为 $x_2, x_3\dots x_{N-1}$, 为了加入 $x_1=-1$ 与 $x_n=1$ 这两个零点,将其变为 $N$ 阶多项式
\begin{equation}\label{GLquad_eq10}
(1-x^2)P'_{N-1}(x)
\end{equation}
然而,\autoref{GLquad_eq5} 要求 $p_n(x_n)=1$, 所以我们将\autoref{GLquad_eq10} 除以它自己在 $x_n$ 处的切线,在 $x=x_n$ 处形成极限类型 $0/0=1$ 即可得到 阶多项式 $u_n(x)$。 
\begin{equation}\label{GLquad_eq11}
u_n(x) = \frac{(1-x^2)P'_{N-1}(x)}{[(1-x^2)P'_{N-1}(x)]'_x} = x_n (x-x_n)
\end{equation}
该式与\autoref{GLquad_eq2} 事实上是完全相同的多项式,因为所有具有 $N-1$ 个零点的 $N-1$ 阶多项式都可以因式分解成\autoref{GLquad_eq2} 的形式乘以一个待定常数。用\autoref{GLquad_eq11} 便于快速地展开多项式(因为勒让德多项式的系数可以直接通过公式计算)。
