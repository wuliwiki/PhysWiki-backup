% 共轭与共轭类
% 共轭|共轭类|共轭元素|等价关系|不变子群|conjugate

\pentry{群\upref{Group} 群乘法表及重排定\upref{groupt} 正规子群和商群\upref{NormSG}}

\subsection{共轭与共轭类}

\begin{definition}{共轭}
若有群元$d,f\in G$,且$\exists g\in G$使得$gdg^{-1}=f$,则称群元$d$与群元$f$共轭。记作$d$~$f$。
\end{definition}

共轭关系是一个等价关系,满足自反率、对称率和传递率:\\
自反率:$g=ggg^{-1}$,则$g$~$g$。 \\
对称率:若$d$~$f$,则$\exists g\in G$使得$gdg^{-1}=f$,那么$d=g^{-1}fg=
g^{-1}f(g^{-1})^{-1}$,则$f$~$d$ \\
传递率:若$d$~$f$,$f$~$h$,则$\exists g_1,g_2\in G$,使$h=g_2fg_2^{-1}=
g_2g_1dg_1^{-1}g_2^{-1}$ $=g_2g_1d(g_2g_1)^{-1}$,则有$d$~$h$。

这一概念与矩阵中的相似矩阵类似。

\begin{definition}{共轭类}
群中所有相互共轭的元素的集合称为群的一个共轭类。
\end{definition}

\begin{corollary}{共轭元素的阶相同}
相互共轭的元素的阶数相同
\end{corollary}
证明: 

考虑$gdg^{-1}=f$,若f的阶数为$m$,有$f^m=e$。

那么$d^m=g^{-1}(gdg^{-1})^mg=g^{-1}f^mg=g^{-1}eg=e$,证毕。

这条推论会为我们找共轭类提供极大便利,我们只需要去验证那些阶数相等的群元,由此我们也可以得到另一个推论:
\begin{corollary}{}
单位元自成一类
\end{corollary}
\begin{corollary}{等价轴}
对于多面体对称群,绕等价轴(可以通过群元对应的对称变换相互转变的转轴轴)旋转相同角度的群元素互为共轭元素。
\end{corollary}

这个定理的证明涉及到连续群表示论中有关$SO(3)$群的一个结论:具有相同旋转角度的转动互为共轭类,证明的思想就是先将$a$轴转到$b$轴上,在$b$轴上完成旋转后在转回到$a$轴上,绕$a$轴和$b$轴旋转相同角度的操作即为互为共轭的两个元素。

例如,对于一个正方体,我们可以通过绕面心对称轴的旋转将一个对角线变到另一个对角线上,这样这两个对角线便成为了“等价轴”,又有绕对角线旋转$\frac{2\pi}{3}$为正方体对称群的一个群元,则绕不同对角线旋转$\frac{2\pi}{3}$随对应的群元互为共轭元素,一起组成了一个共轭类。

\begin{example}{求$D3$群的共轭类}
首先列出$D3$群的乘法表:
\begin{table}[ht]
\centering
\caption{$D3$群乘法表}\label{gpcon_tab1}
\begin{tabular}{|c|c|c|c|c|c|c|}
\hline
        $D3$ & $~e~$ & $~d~$ & $~f~$ & $~a~$ & $~b~$ & $~c~$ \\ \hline
        $e$ & $e$ & $d$ & $f$ & $a$ & $b$ & $c$ \\ \hline
        $d$ & $d$ & $f$ & $e$ & $b$ & $c$ & $a$ \\ \hline
        $f$ & $f$ & $e$ & $d$ & $c$ & $a$ & $b$ \\ \hline
        $a$ & $a$ & $c$ & $b$ & $e$ & $f$ & $d$ \\ \hline
        $b$ & $b$ & $a$ & $c$ & $d$ & $e$ & $f$ \\ \hline
        $c$ & $c$ & $b$ & $a$ & $f$ & $d$ & $e$ \\ \hline
\end{tabular}
\end{table}

从图中可以看出有以下关系:$a^{-1}da=f$,$d^{-1}ad=b$,$d^{-1}bd=c$\\
则有$D3$群的共轭类为:$\{e\}$,$\{d,f\}$,$\{a,b,c\}$。
\end{example}

\subsection{与正规子群的关系}
在正规子群最初的定义中,我们曾要求其生成的左右陪集相等,即$\forall g \in G$ \\
$gH=Hg$,既$g^{-1}Hg=H$,用本节的语言便是$H$中包含完整的共轭类。

则由此可以引出找正规子群的方法,我们可以先写出所有的共轭类,在通过将几个共轭类组合起来,找到运算封闭的共轭类组合便可以得到正规子群。

值得注意的一个性质是子群的阶数是群$G$的因数,这与前几条条件组合起来一般可以非常快速的找到正规子群。

\begin{example}{找到$D3$群的非平庸的不变子群}
由前一例题中结果可知$D3$群中的非平庸的不变子群为$\{e,d,f\}$

注:非平庸是值排除显然的正规子群$\{e\}$与群$G$本身。
\end{example}