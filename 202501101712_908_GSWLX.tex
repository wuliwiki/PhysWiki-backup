% 干涉 (物理学)(综述)
% license CCBYSA3
% type Wiki

本文根据 CC-BY-SA 协议转载翻译自维基百科\href{https://en.wikipedia.org/wiki/Wave_interference}{相关文章}。

对于无线电通信中的干扰,请参阅《干扰(通信)》。
“干涉图样”在此重定向。有关莫尔条纹,请参阅《莫尔条纹》。对于医学术语,请参阅《干涉图样(肌电图)》
\begin{figure}[ht]
\centering
\includegraphics[width=10cm]{./figures/1379a86de5dd6781.png}
\caption{两波的干涉。相位相同:两个较低的波组合(左侧面板), resulting in a wave of added amplitude(建设性干涉)。相位相反:(这里是180度),两个较低的波组合(右侧面板), resulting in a wave of zero amplitude(破坏性干涉)。} \label{fig_GSWLX_1}
\end{figure}
在物理学中,干涉是指两种相干波通过考虑它们的相位差,将它们的强度或位移相加的现象。 如果两波处于相位相同或相反的状态,所产生的波可能具有更大的强度(建设性干涉)或较小的振幅(破坏性干涉)。干涉效应可以在所有类型的波中观察到,例如光波、无线电波、声波、表面水波、引力波或物质波,以及扬声器中的电波。
