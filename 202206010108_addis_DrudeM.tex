% 德鲁德模型

汤姆逊(J.J.Thomsom)在1987年发现了电子,这对物质结构的物理理论产生了直接且深刻的影响.显然的也就引出了金属导电性和其内部自由电子的存在息息相关.

在汤姆逊的发现三年之后,德鲁德(Drude)较为成功的借鉴了理想气体动力学理论的思想和假设,并且将其运用在对金属的研究上.

德鲁德的自由电子气体模型简单地将金属看作是由\textbf{价电子(Valence electron)}所构成的基本均匀的电子气体.\footnote{参考 Wikipedia \href{https://en.wikipedia.org/wiki/Drude_model}{相关页面}.}
\begin{figure}[ht]
\centering
\includegraphics[width=5cm]{./figures/DrudeM_1.pdf}
\caption{德鲁德模型电子(此处以蓝色显示)不断在较重的静止晶体离子(以红色显示)之间反弹.} \label{DrudeM_fig1}
\end{figure}
\subsection{假设}
\begin{enumerate}
\item \textbf{独立电子近似(Independent electron approximation)}:电子之间不会相遇,不存在任何相互作用.
\item \textbf{自由电子近似}:1.电子和离子之间不会相遇;2.电子在每次碰撞前后都是沿着直线运动;
\item \textbf{Jellium 近似}:正电荷,也就是原子核被假定均匀分布在空间中;电子密度在空间中也是一个均匀的量.由于正电荷的均匀分布,因此其对电子施加的电场为零
\item 德鲁德模型中的碰撞和经典动力学理论的碰撞是一样的,是在一个瞬间对电子的速度产生的变化的原因.碰撞如\autoref{DrudeM_fig1} 所示.
\item 类似于理想气体的碰撞:假设电子仅仅只需通过碰撞就能够达到和其周围环境的热力学平衡.电子会“忘记”碰撞前的速度,也就是电子碰撞前后的速度互不相关.电子碰撞之后的速度由能量(温度)决定:
\begin{equation}\label{DrudeM_eq7}
\frac{3}{2}k_BT = \frac{1}{2}mv_0^2
\end{equation}
\item 假设电子在单位时间碰撞的概率为 $1/\tau $,也就是说电子在一段无穷小的时间段 $dt$ 内碰撞的概率是 $dt/\tau$.其中 $\tau$ 是弛豫时间,或者称之为平均自由时间.
\end{enumerate}
\subsection{模型的建立}
首先,我们先来直接看到电子密度 $n$ 的公式为:
\begin{equation}
n=N_A\frac{Z\rho_m}{A}
\end{equation}

其中 $N_A=6.022\times 10^{23}$(阿伏伽德罗常数)表示每摩尔金属元素的原子个数,单位是 $\rm{atoms/mole}$.$\rho_m$ 是元素的质量密度,单位是 $\rm{g/cm^3}$.$A$ 是元素的相对原子量,单位是 $\rm{g/mole}$.不难看出单位体积下物质的数量为 $\rho_m/A$.由于 $Z$ 是每个原子所提供的价电子数量,因此我们也就有了上述电子密度的公式.

接下来我们将每个电子平均的体积近似的看作成一个半径为 $r_s$ 的球体,并且用其来表示电子密度的大小,也就是电子之间平均间隔的大小:

\begin{equation}
\frac{V}{N}=\frac{1}{n}=\frac{4}{3}\pi r_s^3 \Rightarrow r_s=\left(\frac{3}{4\pi n}\right)^\frac{1}{3}
\end{equation}
其中 $V$ 是金属体积,$N$ 是总体导电的电子数量.由于 $r_s$ 的大小大约在 $0.1\rm{nm}$ 左右,因此我们习惯用Ångström单位 $\buildrel _{\circ} \over {\mathrm{A}}=0.1\rm{nm}$,或者波尔半径 $a_0=0.529\times 10^{-1}\rm{nm}$ 用作长度计量单位.

上述的模型不难让我直接就联想到了这似乎能够揭示金属电阻的某些性质.

\subsection{金属DC电路的导电率}

根据欧姆定律,穿过导线的电流 $I$ 和电势 $V$ 成正比:$V=IR$,其中的电阻就是 $R$.德鲁德模型就能够解释这一现象并且不通过实验测量估计出电阻的大小.

在考虑电阻的时候,一般我们不考虑导线的形状,而是定性的考虑导线的金属构成.我们将导线材质本身的电阻率 $\rho$ 定义为一个在电场 $\bvec E$ 和电流密度 $\bvec j$ 的某一点上的正比常数:
\begin{equation}\label{DrudeM_eq1}
\bvec E =\rho \bvec j
\end{equation}

其中电流密度 $\bvec j$ 是与电荷流动方向平行的向量,它的大小是单位时间穿过垂直于流动单位平面的电荷数量.因此,对于一个穿过长度为 $L$ 并且横截面为 $A$ 的导线上的均匀电流 $I$,它的电流密度是 $j=I/A$.由于穿过导线的电势为:$V=EL$,结合\autoref{DrudeM_eq1} 可得:
\begin{equation}
V=\frac{I\rho L}{A}; R = \frac{\rho L}{A}
\end{equation}

现在从之前电子密度的角度出发,我们考虑单位体积内的电子数量 $n$ 一直以速度 $\bvec v$ 运动,那么电流密度是平行于 $\bvec v$ 的.接下来,在一段时间 $dt$ 内的电子将会在 $\bvec v$ 的方向上向前移动 $v dt$ 的距离,使得有 $n(v dt)A$ 个电子在 $dt$ 时间内穿过垂直于流动方向的截面 $A$.由于单个电子的电荷为 $-e$,因此我们就得到了电流密度为:
\begin{equation}\label{DrudeM_eq2}
\bvec j = -ne\bvec v
\end{equation}

由于金属中任意点的电子总是沿着不同方向运动且有着不同的热能,因此净电流密度也是\autoref{DrudeM_eq2} 这里的 $\bvec v$ 是平均电子速度.在没有电场影响的情况下,一个电子运动方向更倾向于和其他电子的运动方向不同,因此显然平均的 $\bvec v$ 是零,净电流密度也是零.在电场 $E$ 的作用下,电子的平均速度显然和电场是相反的,这就引出了后面的计算.

考虑一个典型的电子最后一次碰撞后经过的时间为 $t$,它在碰撞之后的速度为:
\begin{equation}
\bvec v = \bvec v_0 + (-e\bvec E t/m)
\end{equation}

其中,$v_0$ 是碰撞后一瞬间的速度.$-e\bvec E t/m$ 是电场贡献的,根据 $ma = F = -e\bvec E$.由于我们假设一个电子可以是在任意方向上发生碰撞的,因此每个电子的 $\bvec v_0$ 对平均电子速度没有贡献.由于两次碰撞之间的平均时间 $t$ 是弛豫时间 $\tau$,因此:
\begin{equation}
\bvec v_{avg} = -\frac{e\bvec E \tau}{m}; \ \bvec j =\left(\frac{ne^2\tau}{m}\right)\bvec E
\end{equation}
这一结论带来和电阻率 $\rho$ 对立的导电率 $\sigma = 1/\rho$:
\begin{equation}\label{DrudeM_eq4}
\bvec j = \sigma \bvec E; \ \sigma = \frac{ne^2\tau}{m}
\end{equation}
接下来我们也就可以用电阻率估计出弛豫时间的大小:
\begin{equation}
\tau \frac{m}{\rho n e^2}
\end{equation}
我们可以通过测量导电率计算出弛豫时间,进而得到平均自由程的大小:
\begin{equation}
l=v_0\tau
\end{equation}
其中 $v_0$ 是平均电子速度,$l$ 是电子两次撞击之间平均移动的距离.

在任意 $t$ 时刻下,平均电子的速度 $\bvec=\bvec p(t)/m$,其中 $\bvec p$ 是单位电子的总动量,也就是说总的动能=电子数 $\times$ 单位电子的总动量,因此:
\begin{equation}
\bvec j= -\frac{ne\bvec p(t)}{m}
\end{equation}

设单位电子的总动量经过一段无穷小时间 $dt$ 后为:$\bvec p (t+dt)$,那么在 $t$ 时刻选取的一个电子在 $t+dt$ 时间发生碰撞的概率将会是 $dt/\tau$. 因此,电子在 $t+dt$ 时间内不发生碰撞的概率为 $1-dt/\tau$.

尽管电子没有发生碰撞,它仍然会受到来自均匀电场或者/和磁场的力 $\bvec f(t)$,使得我们需要附加的动量 $\bvec f(t)dt - O(dt)^2$.

因此在不考虑在 $t$ 到 $t+dt$ 时间段内碰撞过程中的动量的贡献,我们有:

\begin{equation}
\bvec p(t+dt)=\left(1-\frac{dt}{\tau}\right)\left[\bvec p(t)+\bvec f(t)dt - O(dt)^2\right]
\end{equation}
注意到 $(dt/\tau)\bvec f(t)dt$ 并不会影响线性项,因此:
\begin{equation}
\bvec p(t+dt)-\bvec p(t)=-\left(\frac{dt}{\tau}\right)\bvec p(t)+\bvec f(t)dt - O(dt)^2
\end{equation}
两边同时除以 $dt$,并且取 $dt\rightarrow 0$ 的极限可得:
\begin{equation}\label{DrudeM_eq3}
\dv{\bvec p(t)}{t}= -\frac{\bvec p(t)}{\tau}+\bvec f(t)
\end{equation}

也就是说我们在单位电子的总动量的运动方程中引入了一个分数阶带阻尼项来描述电子碰撞的带来的衰减影响.
\subsection{霍尔效应和磁阻}
一些预备的介绍请阅读霍尔效应\upref{Hallef},接下来我们将用德鲁德模型去解释霍尔效应.首先,我们先来看到霍尔效应实验的结构如下图所示,
\begin{figure}[ht]
\centering
\includegraphics[width=12cm]{./figures/DrudeM_2.pdf}
\caption{霍尔效应实验的基本设置简化图} \label{DrudeM_fig2}
\end{figure}
在导线的 $x$ 轴方向上施加电场 $E_x$ 并且穿过导线有着同样沿着 $x$ 轴的电流密度 $j_x$.此外,我们在 $z$ 轴方向上有一个磁场 $\bvec H$,使得有洛伦兹力:
\begin{equation}
-\frac{e}{c}\bvec v \cross H
\end{equation}
注意电子运动的方向和电流的方向正好\textbf{相反},导致电子会向 $y$ 轴的负方向偏移.随着电子不断在导线一侧聚集,导线的 $y$ 轴方向上就会形成一个transverse 横向电场,也称作Hall场 $E_y$,使得其与洛伦兹力互相平衡,最终电流还是仅仅只在 $x$ 轴方向上流动.接下来我们考虑在外磁场 $H$ 作用下,电阻率的变化,也就是横向磁阻:
\begin{equation}
\rho(H)=\frac{E_x}{j_x}
\end{equation}
由于横向电场 $E_y$ 平衡了洛伦兹力,因此磁场 $\bvec H$ 和电流密度 $j_x$ 应当和 $E_y$ 成正比,那么我们就称这一系数为\textbf{霍尔系数}:
\begin{equation}\label{DrudeM_eq5}
R_H = \frac{E_y}{j_xH}
\end{equation}
注意到这里的霍尔系数是负的,因为霍尔场 $E_y$ 是沿着 $y$ 轴的负方向.

在计算霍尔系数和磁阻之前,我们需要先找到电流密度 $j_x,j_y$ 和电场的分量 $E_x,E_y$,以及施加磁场 $\bvec H$.我们考虑在每一个电子上的力都是和位置无关的:
\begin{equation}
\bvec f = -e\left(\bvec E +\bvec v\cross \frac{\bvec H}{c}\right)
\end{equation}
因此,根据\autoref{DrudeM_eq3} 可得:
\begin{equation}
\dv{\bvec p}{t}=-e\left(\bvec E +\frac{\bvec P}{mc}\cross \bvec H\right)-\frac{\bvec p}{\tau}
\end{equation}
在稳态时的电流是与时间无关的,因此 $p_x,p_y$ 满足:
\begin{align}
0=-eE_x-\omega_c p_y -\frac{p_x}{\tau}\\
0=-eE_y+\omega_c p_y -\frac{p_y}{\tau}
\end{align}
其中的
\begin{equation}
\omega_c = \frac{eH}{mc}
\end{equation}
是\textbf{回旋频率(cyclotron frequency)}.我们将等式的两侧同时乘以 $-ne\tau/m$ 并且引入之前讨论的电流密度\autoref{DrudeM_eq2} 
\begin{align}
\sigma_0 E_x = \omega_c \tau j_y+j_x\\
\sigma_0 E_y = \omega_c \tau j_x+j_y
\end{align}
其中的 $\sigma_0$ 是我们在DC的讨论中的导电率\autoref{DrudeM_eq4}.根据我们之前的要求,当达到平衡态时,电流仅仅只在 $x$ 方向流动,因此 $j_y = 0$,那么显然就有:
\begin{equation}
 E_y = \frac{\omega_c \tau }{\sigma_0}j_x=\frac{H}{nec}j_x
\end{equation}
带回到\autoref{DrudeM_eq5} 可得:
\begin{equation}
R_H=-\frac{1}{nec}
\end{equation}

\subsection{金属AC电路的导电率}
这部分也揭示了德鲁德模型的自由电子气体所具有的光学性质.金属中的电荷被与时间相关的电场诱导,该电场为:
\begin{equation}
\bvec E(t)=\Re(\bvec E(\omega)\E^{-\I\omega t})
\end{equation}
那么\autoref{DrudeM_eq3} 中的 $\bvec f(t)=-e\bvec E$ 也就是:
\begin{equation}
\dv{\bvec p}{t}=-\frac{\bvec p}{\tau}-e\bvec E
\end{equation}
我们希望得到一个稳定态的解以如下形式:
\begin{equation}
\bvec p(t)=\Re(\bvec p(\omega)\E^{-\I\omega t})
\end{equation}
将复数的 $\bvec p$ 和 $\bvec E$ 带回到:
\begin{equation}
-\I\omega\dv{\bvec p}{\omega}=-\frac{\bvec p(\tau)}{\tau}-e\bvec E(\omega)
\end{equation}
由于电流密度:
\begin{equation}
\bvec j = \frac{-ne\bvec p}{m}
\end{equation}
可得:
\begin{align}
\bvec j(t)&=\Re(\bvec j(\omega)\E^{-\I\omega t})\\
\bvec j(\omega)&=\frac{-ne\bvec p}{m}=\frac{ne^2/m\bvec E(\omega)}{(1/\tau)-\I\omega}
\end{align}
我们可以将其写做:
\begin{equation}
\bvec j(\omega)=\sigma(\omega)\bvec E(\omega)
\end{equation}
其中的 $\sigma(\omega)$ 我们称作频率相关电导率或者AC电导率:
\begin{equation}
\sigma(\omega)=\frac{\sigma_0}{1-\I\omega\tau},\ \sigma_0=\frac{ne^2\tau}{m}
\end{equation}
注意到在频率为零的情况下,该公式和之前的\autoref{DrudeM_eq4} 是相符合的.这一结论最为重要的应用是在金属中的电磁辐射的传播.

这一结论看起来似乎并不能解释电磁波,因为其电场的垂直方向上有磁场 $\bvec H$.此外,还要电磁波是随着时间和空间变化的,然而\autoref{DrudeM_eq4} 假定的是一个空间上均匀的力.

首先,磁场产生的力 $-e\bvec p/mc\cross \bvec H$ 远远小于电场,是其 $\tau/c$ 倍数的大小,因此可以忽略不记.
尽管电流密度完全是由从最后一次在 $\bvec r$ 碰撞开始,电场对电子的贡献决定的,但是在最后一次碰撞后电子仅仅只会移动极少个平均自由程.因此,我们仍然可以正确的计算 $\bvec r$ 点的电流密度 $\bvec j(\bvec r,\Omega)$:
\begin{equation}
\bvec j(r,\omega)=\sigma(\omega)\bvec E(\bvec r,\omega)
\end{equation}
因此只要当波长 $\lambda$ 大于平均自由程这一结论都是有效的.接下来考虑特定电流密度 $\bvec j$,我们可以将麦克斯韦方程写做如下:
\begin{equation}
\div \bvec E =0;\ \div \bvec H = 0;\ \curl \bvec E =-\frac{1}{c}\pdv{ \bvec H}{t};\ \curl H = \frac{4\pi}{c}\bvec j+\frac{1}{c}\pdv{ \bvec E}{t}
\end{equation}

我们希望找到含时间的解,注意到在金属中我们可以将电流密度可得:
\begin{equation}
\curl(\curl\bvec E)=-\laplacian \bvec E =\frac{\I\omega}{c}\curl\bvec H=\frac{\I\omega}{c}\left(\frac{4\pi\sigma}{c}\bvec E-\frac{\I\omega}{c}\bvec E\right)
\end{equation}
化简可得:
\begin{equation}
-\laplacian \bvec E =\frac{\omega^2}{c^2}\left(1+\frac{4\pi\I\sigma}{\omega}\bvec E\right)
\end{equation}
该结论和普通的电磁波有着同样的形式:
\begin{equation}\label{DrudeM_eq6}
-\laplacian \bvec E = \frac{\omega^2}{c^2}\epsilon(\omega)\bvec E
\end{equation}
其中的介电常数为常数:
\begin{equation}
\epsilon(\omega)=1+\frac{4\pi\I\sigma}{\omega}
\end{equation}
如果我们的频率足够高其满足:
\begin{equation}
\omega\tau\gg 1
\end{equation}
那么根据近似和之前的结论就可以得到:
\begin{equation}
\sigma(\omega)=-\frac{\omega_p^2}{\omega^2}
\end{equation}
其中的 $\omega_p$ 称作等离子(plasma)频率:
\begin{equation}
\omega_p^2=\frac{4\pi ne^2}{m}
\end{equation}
显然我们看到当 $\sigma<0$ 并且是实数的时候($\omega<\omega_p$),\autoref{DrudeM_eq6} 的解会在空间中随指数衰减,因此电磁波无法传播.不过,当 $\sigma>0$ 并且为实数时($\sigma>\sigma_p$)解会以正弦波的形式震荡,因此电磁波得以传播,金属变得可通过(transparent).这一结论当然只有在高频率的假设下成立.

\subsection{金属导热率}
德鲁德模型还能够成功的解释Wiedemann 和 Franz 1853 的经验定理.维德曼-夫兰兹定理是由大量实验事实发现,它描述了金属电导率 $\sigma$ 和热导率 $\rho$ 之间的关系:\footnote{参考 hyperphysics \href{http://hyperphysics.phy-astr.gsu.edu/hbase/thermo/thercond.html}{相关页面}.}
\begin{equation}
\frac{\kappa}{\sigma}=LT
\end{equation}
其中 $L$ 称为洛伦兹系数,理论上:
\begin{equation}\label{DrudeM_eq47}
L=\frac{\kappa}{\sigma T}=\frac{\pi^2}{3}\left(\frac{k_B}{e}\right)^2 = 2.44\times10^{-8}\rm{W\Omega K^{-2}}
\end{equation}
$L$ 和温度无关.为了理解上面的公式,我们考虑电场中的电子受力为 $e\bvec E$,金属的电导率和 $e^2$ 成正比.金属热容和 $k_B T$ 成正比,电子受到的力和 $k_B\grad T$ 成正比,因此 $\kappa/(\sigma T)$ 和 $k_B^2/e^2$ 成正比.

为了定义和估计导热率,我们考虑一个温度正在缓慢变化的金属条,其一段较为热另一端较为冷.假设金属条上没有热源或者热陷(sources or sinks of heat)那么其两端会慢慢变温.也就是说,热量的流动是从温度较高流向温度较低的.数学上看待就是热量流动的方向和温度的梯度相反.接下来我们定义热量流动密度 $\bvec j_q$ 的大小为单位时间内穿过单位截面(垂直于热流方向)的热能,热量流动密度的方向平行于热量流动的方向.那么我们就得到了傅里叶定理(Fourier's law):
\begin{equation}
\bvec j_q =-\kappa\grad T
\end{equation}
正比常数 $\kappa$ 为材料的导热率,公式中符号的物理意义为:热流动的方向和温度梯度方向时相反的,也就是热总是从高温流向低温的.

回到德鲁德模型,它假设金属中大量的热量是由传到电子所承载的.这一假设是源自于我们经验观察到金属的导热性往往远优于绝缘体.也就是离子对导热性的作用在金属上远远小于其传导电子.

接下来我们将讨论如何运用德鲁德模型得到\autoref{DrudeM_eq47}.首先,我们简化地考虑一维情况下的模型,也就是能量只能在 $x$ 轴上变化.假设 $\epsilon(T)$ 为金属中单位电子在温度 $T$ 时的热能,那么一个电子最后一次在 $x'$ 碰撞的平均热能为 $\epsilon(T[x'])$.

电子从\textbf{高温部分}到达 $x$ 的最后一次碰撞平均会在 $x-v\tau$ 处,并且每个携带的热能为 $\epsilon(T[x-v\tau])$.因此,高温部分对热流密度的贡献为 $(n/2)v\epsilon(T[x-v\tau])$,类似的低温部分有 $(n/2)-v\epsilon(T[x+v\tau])$.最后我们就得到了热流密度为:
\begin{equation}
j^q = \frac{nv}{2}[\epsilon(T[x-v\tau]-\epsilon(T[x+v\tau]]
\end{equation}
由于,一个平均自由程上的温度变化十分小,使得我们可以将其沿 $x$ 点展开得到:
\begin{equation}
j^q = nv^2\tau\dv{\epsilon}{T}\left(-\dv{T}{x}\right)
\end{equation}
现在我们考虑三维 $\bvec v$,根据
\begin{equation}
\langle v_x^2\rangle = \langle v_y^2\rangle=\langle v_z^2 \rangle = \frac{1}{3}v^2
\end{equation}
由于
\begin{equation}
n\dv{\epsilon}{T}=\left(\frac{N}{V}\right)\dv{\epsilon}{T}=\left(\dv{E}{T}\right)V=c_v
\end{equation}
其中 $c_v$ 为电子比热(electroni specific heat),我们可以得到:
\begin{equation}
\bvec j^q =\frac{1}{3}v^2\tau c_v(-\grad T)
\end{equation}
也就有导热率为:
\begin{equation}
\kappa=\frac{1}{3}v^2\tau c_v =\frac{1}{3}lvc_v
\end{equation}
其中的 $v^2$ 为电子的均方速度.两边同时除以导热率 $\sigma$ 以消除弛豫时间 $\tau$ 可得:
\begin{equation}
\frac{\kappa}{\sigma}=\frac{\frac{1}{3}mv^2\tau c_v}{ne^2}
\end{equation}
之前我们的假设已经提到电子碰撞之后的速度由能量(温度)决定,结合\autoref{DrudeM_eq7} 可得:
\begin{equation}
L=\frac{\kappa}{\sigma T}=\frac{3}{2}\left(\frac{k_B}{e}\right)^2 = 1.11\times10^{-8}\rm{W\Omega K^{-2}}
\end{equation}
不过我们注意到该结果还是和\autoref{DrudeM_eq47} 有差异的.德鲁德模型还在室温附近将电子比热 $c_v$ 估算大了两个数量级,不过巧合的是 $v^2$ 又估算小了两个数量级,错上加错结果补偿抵消了.后续我们会在德鲁德-索默菲尔德模型(金属自由电子气体模型)% 未完成
介绍一个更加精确的描述.

\subsection{德鲁德模型的局限}

\begin{example}{}
(a). 证明随机选取的电子在时间间隔 $t$ 内没有碰撞的概率为 $e^{t/\tau}$.

(b). 证明在接下来的时间间隔 $t$ 内,它没有碰撞的概率是相同的.
\end{example}
\begin{example}{}
\textbf{计算在电子两次碰撞在离子之间的时间段 $t$ 的平均损耗的能量:}

首先,我们假设碰撞前的平均速度为 $\bvec v_0$,可得第二次碰撞之后的平均速度的为 $\bvec v_f=\bvec v_0 -\frac{e\bvec E t}{m}$,那么就有动能的变化为:
\begin{align}
\Delta \bvec T &= \bvec T_i-\bvec T_f=\frac{1}{2}m\left(\left(\bvec v_0 -\frac{e\bvec E t}{m}\right)^2-\bvec v_0^2\right)\\
&=\frac{1}{2}m\left(-2\bvec v_0\frac{e\bvec E \tau}{m}+\left(\frac{e\bvec E t}{m}\right)^2\right)\\
\end{align}
由于在碰撞前的速度 $\bvec v_0$ 的方向是“混乱”的,因此 $\langle \bvec v_0\rangle = 0$,那么不考虑势能就有:
\begin{equation}
\langle \Delta \bvec T \rangle = \frac{(e\bvec E t)^2}{2m}
\end{equation}
\end{example}
\begin{exercise}{}
紧接着上一个例题,现在我们试图(a).证明每个电子在每次碰撞所损耗的平均能量为:$(eE\tau)^2/m$,并且(b).每秒每立方厘米上的平均能量损耗为:
\begin{equation}
\left(\frac{ne^2\tau}{m}\right)\bvec E^2=\sigma \bvec E^2
\end{equation}
(c).推导出长度为 $L$、截面为 $A$ 的导线的功率损耗为 $I^2R$,其中 $I$ 是电流 $R$ 是导线的电阻.

\textbf{(a).}之前我们讨论了在时间段 $[t, t + dt]$ 内两次连续碰撞的概率
为 $p(t)=(dt/\tau)e^{- t/\tau}$,那么每次碰撞所损耗的平均能量是对碰撞概率对于每次损耗的积分:
\begin{equation}
\int_0^\infty p(t) \langle\Delta\bvec T\rangle
\end{equation}
带入并化简可得:
\begin{equation}
\frac{(e\bvec E)^2}{2m\tau}\int_0^\infty t^2 e^{- t/\tau}dt
\end{equation}
接下来我们就直接套用下面的特殊积分公式:
\begin{equation}
\int_0^\infty x^n e^{-ax^p}dx = \frac{\Gamma\left(\frac{n+1}{p}\right)}{pa^{\left(\frac{n+1}{p}\right)}}
\end{equation}
可得:
\begin{equation}
\int_0^\infty t^2 e^{- t/\tau}dt=\frac{\Gamma(3)}{(\tau^{-1})^3}=2\tau^3;\ \Gamma(n)=(n-1)!
\end{equation}
最后我们就得到了每一次碰撞中,每个电子在离子上的平均能量的损失为:
\begin{equation}
2\tau^3\frac{(e\bvec E)^2}{2m\tau}=\frac{(e\bvec E\tau)^2}{m}
\end{equation}
(b).单位体积内的电子数量 $n$(电子数量/体积)乘以每个电子在离子上每次碰撞的平均能量的损失(平均能量损失/(电子数量 $\times$ 碰撞)可以得到单位体积内所有电子每一次碰撞平均能量损失的总和.现在我们还要将每次的碰撞每一时刻,显然需要除以弛豫时间 $\tau$(每次碰撞之间的时间)(时间/碰撞),下面我们以一种更加清晰明了的形式表示:
\begin{align}
\frac{\mbox{电子数量}}{\mbox{体积}}\times \frac{\mbox{平均能量损失}}{\mbox{电子数量 }\times \mbox{碰撞}}\times\frac{\mbox{碰撞}}{\mbox{时间}}=\frac{\mbox{平均能量损失}}{\mbox{体积}\times \mbox{时间}}
\end{align}

那么我们带入推导可以得到:
\begin{equation}
n \frac{(e\bvec E\tau)^2}{m} \frac{1}{\tau} = \frac{ne^2\tau}{m}\bvec E^2 = \sigma \bvec E^2
\end{equation}
(c).注意到在(b)小问中我们得到的就是单位体积下的功率损耗.现在我们有 $V=LA$,因此有功率损耗:
\begin{equation}
P = LA\sigma \bvec E^2
\end{equation}
之前我们讨论过:
\begin{equation}
R = \frac{L}{\sigma A},\ \bvec I = \bvec j A=\sigma\bvec E A
\end{equation}
显然有:
\begin{equation}
\bvec I^2 R=\sigma^2\bvec E^2 A^2 \frac{L}{\sigma A} = LA\sigma \bvec E^2 = P
\end{equation}
\end{exercise}




