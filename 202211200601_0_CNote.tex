% C 语言笔记

% C 语言笔记

\begin{issues}
\issueDraft
\end{issues}

\subsection{常用的字符串处理}
\begin{lstlisting}[language=cpp]
#include "string.h"
char s1[20] = "some string1";
char s2[20] = "some string2";
strlen(s1) // 字符串长度
strcpy(s1, s2) // 复制(s2 可以是 literal)
strcmp(s1, s2) // 比较
strcat(s1, s2) // 连接
strrev(s1) // 反转
\end{lstlisting}

\subsection{printf 和 scanf}
\begin{lstlisting}[language=cpp]
#include <stdio.h>
void main( )
{
  int x,y,z;
  scanf("%d+\n,\n=%d",&x,&y);
  z=x*y;
  printf("x=%d,y=%d\n",x,y);
  printf("xy=%d\n",z);
}
\end{lstlisting}
引号内除了特殊字符,其它都需要输入一摸一样的,否则会出错.但是,1.变量前面可以多打任意多个空格和回车,2.任意多个空格、回车相连等效.
 
\verb|%d,%c,%f|, 都行 \verb|%s| 输入字符串, \verb|%lld| 输入 \verb|long long| 类型.

\begin{lstlisting}[language=cpp]
#include<stdio.h>

void main()
{
char s[20];
scanf("%s",&s);
printf("%s\n",s);
}
\end{lstlisting}
具体的格式代码见\href{https://www.tutorialspoint.com/c_standard_library/c_function_printf.htm}{这里}.

\begin{lstlisting}[language=cpp]
int i = 1234;
printf("=== int ===\n");
printf("%6.5d\n", i); // " 01234"

printf("=== string ===\n");
string s = "abcdABCD";
printf("%s\n", s.c_str()); // "abcdABCD"

double x = 12.345;
printf("=== double 1 ===\n");
printf("%8.3f\n", x); // "  12.345"
printf("%10.3e\n", x); // " 1.235e+01"
printf("%g\n", x); // "12.345"
printf("%5.3g\n", x); // " 12.3"

double y = 0.000012345678;
printf("=== double 2 ===\n");
printf("%g\n", y); // "1.23457e-05"
printf("%10.3g\n", y); // "  1.23e-05"
\end{lstlisting}

如果不想输出到命令行而是输出到字符串, 就用 \verb|int sprintf(char *str, const char *format, ...)|. 返回写入的字符数. 该函数返回写入到 \verb|str| 的总字符数.

\subsection{getchar}
\begin{lstlisting}[language=cpp]
#include <stdio.h>
#include <string.h>
void main()
{
 int i=1;
 char str[5]={0};
    while(i<=5)
   {str[i]=getchar();i++;}

 i=1;
 while(i<=5)
 {printf("%d  ",str[i]);i++;}
 printf("\n");

 i=1;
 while(i<=5)
 {printf("%c",str[i]);i++;}
 printf("\n");
}
\end{lstlisting}

\subsection{struct 和 union}

\begin{lstlisting}[language=cpp]
struct 名称 {} // 声明变量: struct 名称 变量
typedef struct {} 名称 // 声明变量: 名称 变量
typedef enum {} 名称 // 同理
typedef union {} 名称 // 同理
\end{lstlisting}

\begin{lstlisting}[language=cpp]
union {} // 和 struct 类似, 但变量共用内存
\end{lstlisting}

\subsection{函数指针}
\begin{lstlisting}[language=cpp]
// 函数指针
double(*p1)(double) = sin;
// 定义函数指针类
typedef double(*Tpf)(double);
Tpf p3 = sin;
// 定义函数类
typedef double(Tf)(double);
Tf *p2 = sin;
// 作为类型参数
vector<double(*)(double)> arr {sin, cos} arr;
// 调用函数
cout << "sin(1) = " << arr[0](1) << endl;
\end{lstlisting}

\subsection{文件处理}
\begin{itemize}
\item \verb|fopen| 以及选项 \verb|a, a+, w, w+| 创建新文件.
\item \verb|fopen| 打开已有文件.
\item \verb|fscanf, fgets| 读取文件
\item \verb|fprintf| 写文件
\item \verb|fseek, rewind| 移动到文件的某个位置.
\item \verb|fclose| 关闭文件
\end{itemize}
