% 复流形
% keys 复流形|流形|微分几何

\pentry{流形\upref{Manif}}

\subsection{复流形}

将光滑流形\autoref{Manif_def3}~\upref{Manif}定义中的 $\mathbb{R}^n$ 替换为 $\mathbb{C}^n$,图册中的"光滑映射"替换为"复解析映射", 我们就得到复流形的定义.

\begin{definition}{复图和复图册}
$N$ 是一个 $n$ 维拓扑流形,如果存在开集 $U \in \mathcal{T}_N$ 和同胚映射 $\varphi: U \rightarrow \mathbb{C}^n$,那么称 $(U,\varphi)$ 是 $N$ 上的一张\textbf{复图}.如果图的一个集合 $\mathcal{A}=\{(U_\alpha, \varphi_\alpha)\}$ 覆盖了 $N$,即 $\bigcup\{U_\alpha\}=N$,那么称这个集合 $\mathcal{A}$ 是一个\textbf{复图册}.
\end{definition}

\addTODO{$\mathbb{C}^n \to \mathbb{C}^m$ 的全纯函数}

\begin{definition}{全纯相容}
考虑一个拓扑流形 $N$ 的两个复图 $(U, \varphi)$ 和 $(V, \phi)$.如果 $U \cap V \neq \varnothing$,且 $\varphi \circ \phi^{-1}: \phi(V) \rightarrow \varphi(U)$ 和 $\phi \circ \varphi^{-1}: \varphi(U) \rightarrow \phi(V)$ 都是全纯(复解析)映射,那么我们称这两个图是\textbf{全纯相容的(compatible)}.
\end{definition}

\begin{definition}{复流形}
一个拓扑流形 $N$ 和加上其上一组全纯相容的复图册 $\mathcal{A}$,被称为一个\textbf{复流形(complex manifold)} $(N, \mathcal{A})$.
\end{definition}

\addTODO{复切向量的三种定义和复化切向量}
\addTODO{复1-形式和复化1-形式}

\subsection{近复流形}

\begin{definition}{复结构(向量空间)}
一个实向量空间 $V$ 上的一个\textbf{近复结构}是一个线性算符 $J: V \to V$ 满足 $J \circ J = - \text{id}_V$.
\end{definition}

\begin{theorem}{复向量空间的近复结构}
一个 $n$ 维复向量空间
\end{theorem}


