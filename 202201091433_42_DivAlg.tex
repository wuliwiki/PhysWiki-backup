% 带余除法
% keys 余式|商式

\pentry{一元多项式\upref{OnePol}}
在一元多项式\upref{OnePol}的最后提到,系数在数域 $\mathbb{F}$ 上的全体一元多项式构成一元多项式环 $\mathbb{F}[x]$.在环 $\mathbb{F}[x]$ 中,可以做加、减、乘三种运算,运算的结果还是该多项式环 $\mathbb{F}[x]$ 中的元素(一元多项式).自然,我们会像在整数里那样,考虑除法运算.你将看到,和整数一样,在一元多项式里,任意两个多项式做除法运算的结果并不一定还是一个多项式,取而代之的是带余除法.
\begin{theorem}{带余除法}
设 $f(x)$ 与 $g(x)$ 为 $\mathbb{F}[x]$ 中的两个多项式,并且 $g(x)\neq 0$,则存在唯一的 $\mathbb{F}[x]$ 中的多项式 $q(x),r(x)$,使得
\begin{equation}
f(x)=q(x)g(x)+r(x)
\end{equation}
其中 $\mathrm{deg}\;r(x)<\mathrm{deg}\;g(x)$, $q(x),r(x)$ 分别称为 $g(x)$ 除 $f(x)$ 的\textbf{商式}和\textbf{除式}.并将这种算法称为\textbf{带余除法},有时称为\textbf{长除法}.
\end{theorem}
\subsection{证明}
1.首先证明 $q(x)$,$r(x)$ 的存在性.

当 $\mathrm{deg}\;f(x)<\mathrm{deg}\;g(x)$ 时,取 $q(x)=0,r(x)=f(x)$ 即可;

假设当 $\mathrm{deg}\;f(x)-\mathrm{deg}\;g(x)\leq k$ 时,$q(x),r(x)$ 存在,那么当 $\mathrm{deg}\;f(x)-\mathrm{deg}\;g(x)=k+1$ 时,设
\begin{equation}
f(x)=\sum_{i=0}^n a_i x^i\quad g(x)=\sum_{i=0}^m b_ix^i
\end{equation}
取
\begin{equation}
f_1(x)=f(x)-\frac{a_n}{b_m}x^{n-m}g(x)
\end{equation}
注意 $f_1(x)$ 的 $m$ 次项系数为0,有
\begin{equation}
\mathrm{deg}\;f_1(x)-\mathrm{deg}\;g(x)\leq f(x)-1-\mathrm{deg}\;g(x)=k
\end{equation}
由数学归纳法