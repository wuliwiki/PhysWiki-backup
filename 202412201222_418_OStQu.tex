% 开弦量子化
% license Usr
% type Tutor



\subsection{开弦量子化}
在现在的边界条件\footnote{$\partial_\sigma X^i = 0$,在 $\sigma = 0, l$。}下,波动的解可以展开为 

\begin{equation}
	X^i(\tau, \sigma) = x^i + \frac{p^i}{p^+} \tau + 
    \mathbf{i} \sqrt{2 \alpha '} \sum_{n = -\infty, n \neq 0}^{\infty} \frac{1}{n} \alpha_n^i \exp\left(-\frac{\pi \mathbf{i} nc\tau}{l}\right) \cos \frac{\pi n \sigma}{l} ~.
\end{equation}

$X^i$ 是实的,要求 $\alpha_{-n}^i = (\alpha_n^i)^\dagger$。此外,定义质心变量
\begin{equation}
	\begin{aligned}
		x^i(\tau) &= \frac{1}{l} \int_0^l \dd \sigma X^i(\tau, \sigma)~, \\
		p^i(\tau) &= \int_0^l \dd \sigma \Pi^i(\tau, \sigma) = \frac{p^+}{l} \int_0^l \dd \sigma \partial_\tau X^i(\tau, \sigma) ~.
	\end{aligned}
\end{equation}
而波动的解的展开中的算符 $x^i \equiv x^i(0)$ 和 $p^i \equiv p^i(0)$ 是薛定谔算符。

正则量子化,附加对易关系
\begin{equation}
	[x^-, p^+] = -\mathbf{i} ~,
\end{equation}
\begin{equation}
	[X^i(\sigma), \Pi^j(\sigma')] = \mathbf{i} \delta^{ij} \delta(\sigma - \sigma') ~.
\end{equation}

或考虑为傅里叶分量的的形式:
\begin{equation}
	[x^i, p^j] = \mathbf{i} \delta^ij ~,
\end{equation}
\begin{equation}
	[\alpha_m^i, \alpha_n^j] = m \delta^{ij} \delta_{m, -n} ~.
\end{equation}

对于每个 $m$ 和 $i$,模式满足一个谐振子代数:
\begin{equation}
	\alpha_m^i \sim \sqrt m a, ~ \alpha_{-m}^i \sim \sqrt{m} a^\dagger, ~ m > 0, \text{where} ~[a, a^\dagger] = 1 ~.
\end{equation}

态 $\ket{0; k}$\footnote{其中 $k = (k^+, k^i)$。} 可以被下降算符湮灭,是质心动量的本征态:
\begin{equation}
	p^+\ket{0; k} = k^+\ket{0; k}, ~ p^i\ket{0; k} = k^i \ket{0; k} ~,
\end{equation}
\begin{equation}
	\alpha_m^i \ket{0; k} = 0, ~ m > 0 ~.
\end{equation}
一般的态由上升算符作用在 $\ket{0; k}$ 上得到:
\begin{equation}
	\ket{N; k} = \left[\prod_{i=2}^{D-1} \prod_{n=1}^\infty \frac{(\alpha_{-n}^i)^{N_{i, n}}}{\sqrt{n^{N_{i, n}} N_{i, n}!}}\right] \ket{0; k} ~.
\end{equation}
独立的态可以视为质心动量 $k$ 和模式占有数 $N_{i, n}$ 标记。质心动量正是点粒子的自由度,而振子则代表有无穷多内部自由度。从时空观点看,每个占有数的选择对应着不同的粒子或自旋。上面这个 $\ket{K; k}$ 的式子构成单个开弦的 Hilbert 空间 $\mathcal H$。特别的,$\ket{0; 0}$ 是动量为 $0$ 的单个弦的基态,并不是无弦的真真空态 $\ket{\text{Vecuum}}$。对于 $n$-弦 Hilbert 空间 $\mathcal H_n$ 由 $n$ 个空间 $\ket{N; k}$ 的乘积组成。实际上,在自由极限下,弦论的整个 Hilbert 空间是 
\begin{equation}
	\mathcal H = \ket{\text{Vacuum}} \oplus \mathcal H_1 \oplus \mathcal H_2 \oplus \cdots ~~
\end{equation}

现在,代入展开进哈密顿量表达式得到 
\begin{equation}
	H = \frac{p^i p^i}{2 p^+} + \frac{1}{2 p^+ \alpha' } \left(\sum_{n=1}^\infty \alpha_{-n}^i \alpha_n^i + A\right) ~.
\end{equation}
其中 $A = (D-2) \sum_{n=1}^\infty n/2$。由于算符的次序是反常的,将下降算符放在右边,上升算符放在左边,常数 $A$ 就来源于此处产生的对易子。

\subsection{时空维数分析}

首先,进行重整化可以得到 $A = (2-D)/24$,这是删除表观发散量后的有限量,是 Casimir 能量的例子,源于弦长有限。对于点粒子 $p^- = H$,从而 
\begin{equation}
	m^2 = 2 p^+ H - p^i p^i = \frac{1}{\alpha'} \left(N + \frac{2-D}{24}\right) ~,
\end{equation}
其中 $N$ 是能级:
\begin{equation}
	N = \sum_{i=2}^{D-1} \sum_{n=1}^\infty n N_{i, n} ~.
\end{equation}
每个态的质量由激发态的能级决定。

考虑一些轻的弦,例如最轻的是 
\begin{equation}
	\ket{0; k}, ~ m^2 = \frac{2-D}{24 \alpha'} ~.
\end{equation}
若 $D>2$,$m^2 < 0$,此时成为\textbf{快子(tachyon)}。这意味着真空是不稳定的。现在不考虑这个情况,而再激发一次。

弦的最低激发态是激发 $n=1$ 的模式 $1$ 次:
\begin{equation}
	\alpha_{-1}^i \ket{0; k}, ~ m^2 = \frac{26-D}{24 \alpha'} ~,
\end{equation}
回顾小群分类。对于无质量粒子和有质量粒子,自旋的分析是不同的。有质量粒子标准动量 $p^\mu = (m, 0, \cdots ,0)$,内部态是旋转群 $\text{SO}(D-1)$ 的一个表示。而对于无质量粒子,标准动量 $p^\mu = (E, E, 0, \dots, 0)$,仅能使 $\text{SO}(D-2)$ 作用在横向上使 $p^\mu$ 不变,同样内部的态构成群表示\footnote{对于 $D=4$ 是熟知的,也就是在 Wigner 小群分类讨论过的情况。}。

Lorentz 不变仅产生这些需求,但 CPT 对称性,也就是拓扑考虑会要求粒子对螺旋度的关系仅与螺旋度的绝对值有关。

一个有质量粒子有 $D-1$ 个自旋态、一个无质量粒子则有 $D-2$ 个态。在第一激发态我们仅能找到 $D-2$ 个态 $\alpha_{-1}^i \ket{0; k}$,故必定是无质量的,这指出一定有 $A=-1, D=26$,即所谓弦论是 $26$ 维时空。

光锥量子化挑出了两个纵向方向,留下 $\text{SO}(D-2)$ 作用在横向上,横向的自旋生成元仍然是:
\begin{equation}
	S^{ij} = -\mathbf{i} \sum_{n=1}^\infty \frac{1}{n} (\alpha_{-n}^i \alpha_n^j - \alpha_{-n}^j \alpha_n^i) ~.
\end{equation}