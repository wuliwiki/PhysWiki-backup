% 流 流密度
% 矢量场|电流密度|能流密度|速度

\pentry{矢量场\upref{Vfield}}

\subsection{流}

\subsection{流密度}
\textbf{流密度}可以用于描述某时刻流体在的空间流动的速率. 我们以水流为例, 在一条河流或管道中, 某时刻三维空间中任意一点 $\bvec r$ 处, 都对应一个水流速度 $\bvec v$, 如果我们在该点放置一个垂直于速度的微小截面, 令其面积为 $\Delta S$, 在一段微小时间 $\Delta t$ 内流经截面的质量为 $\Delta m$, 那么质量流密度可以用极限定义为
\begin{equation}\label{CrnDen_eq1}
\bvec j(\bvec r) = \lim_{\Delta S, \Delta t \to 0} \frac{\Delta m}{\Delta S \Delta t}
\end{equation}
这是一个关于位置的矢量函数, 即矢量场, 也可以称为\textbf{流密度场}. 广义来说, \autoref{CrnDen_eq1} 中的分子可以是不同的物理量. 若分子是质量则称流密度为\textbf{质量流密度}, 若是能量则称为\textbf{能流密度}, 若是粒子数则称为\textbf{粒子流密度}, 若是电荷量则称为\textbf{电流密度}\upref{Idens}, 等等. 

我们也可以根据密度和速度来定义流密度
\begin{equation}\label{CrnDen_eq2}
\bvec j(\bvec r) = \rho(\bvec r) \bvec v(\bvec r)
\end{equation}
其中密度定义为(分母同样可以替换成其他物理量)
\begin{equation}\label{CrnDen_eq3}
\rho(\bvec r) = \lim_{\Delta V} \frac{\Delta m}{\Delta V}
\end{equation}
\autoref{CrnDen_eq2} 的定义和\autoref{CrnDen_eq1} 是等效的, 因为时间 $\Delta t$ 内流经 $\Delta S$ 的体积为 $\Delta V = \Delta S \cdot v  \Delta t$, 代入\autoref{CrnDen_eq3} 再代入\autoref{CrnDen_eq2} 就得到了\autoref{CrnDen_eq1}.

\subsection{流密度与流}
