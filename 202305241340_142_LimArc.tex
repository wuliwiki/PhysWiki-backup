% 小角极限(简明微积分)
% 微积分|极限|弧

\pentry{函数的极限(简明微积分)\upref{FunLim}}

\footnote{参考 Wikipedia \href{https://en.wikipedia.org/wiki/Small-angle_approximation}{相关页面}。}这里要介绍的是一个简单的几何问题,然而它在微积分和物理中却非常频繁地出现。

如\autoref{fig_LimArc_1}, 令平面上 $O$ 点为圆心,以 $R$ 作为半径画圆。取一段的圆心角为 $\theta $ 的圆弧 $AB$ (令长为 $l$),并作线段 $AB$。 我们定义弧长和线段长度的相对误差为
\begin{equation}
E = \frac{l - AB}{l}~.
\end{equation}

\begin{figure}[ht]
\centering
\includegraphics[width=10cm]{./figures/119b0bb51213fda1.pdf}
\caption{单位圆中,随着角度 $\theta$ 不断减小,弧长与线段长度的相对误差也不断减小}\label{fig_LimArc_1}
\end{figure}

由弧长公式得
\begin{equation} \label{eq_LimArc_1}
l = R\theta ~.
\end{equation}
线段 $AB$ 的长度为
\begin{equation}\label{eq_LimArc_2}
AB = 2R\sin \frac{\theta }{2}~.
\end{equation}
显然弧长 $l$ 大于线段长度$AB$ (两点之间直线最短),但从图中可以看出随着 $\theta $ 越来越小,二者的相对误差($E$)越来越小。用极限\upref{FunLim}的语言来说,就是当 $\theta $ \textbf{趋近于 $0$ } 时,它们的比值\textbf{趋近于1}。 注意这只是一个经验上的总结, 我们暂时不证明。
% \addTODO{引用证明}

所以有
\begin{equation}
1=\lim_{\theta\to 0} \frac{AB}{l} = \lim_{\theta\to 0} \frac{2R\sin (\theta/2)}{R\theta} 
= \lim_{\theta\to 0}\frac{\sin (\theta/2)}{\theta/2}~.
\end{equation}
令 $x = \theta/2$, 有
\begin{equation}
\lim_{x\to 0} \frac{\sin x}{x} = 1~.
\end{equation}

这是一个非常重要的极限。 在物理中, 我们常常会就某个小角使用近似 $\sin x \approx x$, 例如 “单摆\upref{Pend}” 以及 “双缝干涉\upref{Young}”。 具体来说, 这是一个\textbf{一阶近似},  以后会看到 $x$ 就是 $\sin x$ 的泰勒展开(\autoref{eq_Taylor_3}~\upref{Taylor})的第一项。

\addTODO{类似地还有 $\tan\theta$}