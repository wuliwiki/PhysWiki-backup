% Termux 笔记(安卓系统运行 Linux)
% license Usr
% type Note


\begin{issues}
\issueDraft
\end{issues}

一些信息
\begin{itemize}
\item 不能直接在安卓目录中创建 symlink。 得用 \verb`~` 目录管理 repo(一个 work around 就是用 \verb`~` 管理 \verb`.git` 文件夹, 安卓目录存对应的 \verb`tree`
\item \verb`termux` 中根目录的文件在于安卓的 \verb`/Android/data/` 中,然而这个目录有些手机不能访问(如三星)
\item 可以运行 \verb`g++` 编译器, \verb`python`, \verb`ffmpeg` 等
\item 只能开一个窗口, 可以用 \verb`tmux` 命令模拟多个窗口。
\item 校园网连不上,但是手机流量可以,大概是校园网禁止了外部 DNS。
\item 一些 POSIX 功能如 \verb`<unistd>` 里面的以及 address sanitizer 用不了。 除此之外 SLISC 不依赖任何三方库的测试成功
\end{itemize}

安装
\begin{itemize}
\item 测试机型:Samsung s21 ultra (2023/12/2 最新系统)
\item 下一个 Andronix,按照指引做就行
\item 中途可能提示 termux 不是最新版, 下个 F-droid 商店,把 termux 更新到最新
\item Andronix 会复制一个命令到剪切板, 需要手动粘贴到 termux 中运行 \verb`pkg` 等命令安装 Linux distro (笔者用的 ubuntu22)
\item 运行完后在 \verb`~` 目录会有一个 \verb`start-ubuntu22.sh` 文件, 执行即可以 \verb`root` 身份进入 ubuntu。
\item 现在就可以愉快地玩耍了。
\item 不要彻底 \verb`exit`, 会卡住,要重启手机才行
\end{itemize}


文件目录
\begin{itemize}
\item \verb`sdcard` 里面是安卓机的根目录
\end{itemize}

