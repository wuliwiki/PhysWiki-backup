% 最小生成树
% Pirm|最小生成树|算法|C++

\begin{issues}
\issueMissDepend
\issueDraft
\end{issues}

生成树的定义:是指在一个带权的无向联通图中选择 $n$ 个点和 $n - 1$ 条边构成的无向联通子图.

最小生成树的定义即为边权最小的生成树.

求最小生成树最常用的两种算法为:Prim 和 Kruskal.Prim 常用于稠密图,Kruskal 则相反.

\subsection{Prim 算法}

Prim 算法的思路与 Dijkstra 算法非常相似.
定义 $S$ 为当前已经确定了属于最小生成树的结点,$T$ 为集合外的结点.使用 \verb|dist| 数组存储每个结点到 $S$ 集合的距离,距离定义为如果有多个结点指向 $S$ 集合,则距离最短的边为这个结点到 $S$ 集合的距离.最开始初始化所有结点到 $S$ 集合的距离为 $+\infty$,$1$ 号点到 $S$ 集合的距离为 $0$.一共进行 $n$ 次迭代,每次找到 $T$ 集合中距离 $S$ 集合距离最短的结点 $t$,然后用 $t$ 结点更新其他点(与 $t$ 相连的结点)到 $S$ 集合的距离,然后把 $t$ 从 $T$ 集合中删去,加入到 $S$ 集合中,则 $t$ 结点为当前已经确定了属于最小生成树的结点.

具体的做法是用一个布尔数组来标记一个点是否属于 $S$ 集合,\verb|st[i]| 为 \verb|true| 则结点 $i$ 属于 $S$ 集合,反之不属于.每次从未标记的结点中选择一个 \verb|dist| 值最小的结点,把这个结点加入到 $S$ 集合中,并把这个结点的权值加到答案里,然后更新所有出边.

朴素 Prim 的时间复杂度为:$\mathcal{O}(n^2)$,使用堆优化可以达到 $\mathcal{O}(m \log_2 n)$,但是使用堆优化的 Prim 算法代码太长,不如直接用 Kruskal.

\begin{lstlisting}[language=cpp]
const int N = 510, M = 1e5 + 10, INF = 0x3f3f3f3f;
int n, m, dist[M], st[M], g[N][N];  // g 数组用邻接矩阵存储图

int prim()
{
    memset(dist, 0x3f, sizeof dist);
    dist[1] = 0;
    
    int res = 0;
    for (int i = 0; i < n; i ++ )
    {
        int t = -1; // t 表示不在 S 集合中距离 S 集合距离最短的结点,-1 表示还没找到
        for (int j = 1; j <= n; j ++ )
            if (!st[j] && (t == -1 || dist[j] < dist[t]))  
                t = j;  
        
        st[t] = true;
        if (dist[t] == INF) return INF; // 找到一个点与图是不联通的
        res += dist[t]; // 这个结点到 S 集合的边权为答案
        
        for (int j = 1; j <= n; j ++ )
            dist[j] = min(dist[j], g[t][j]); // 用 t 更新其他点的距离
    }
     
    return res;
}
\end{lstlisting}

\subsection{Kruskal 算法}
