% 尤金·维格纳(综述)
% license CCBYSA3
% type Wiki

本文根据 CC-BY-SA 协议转载翻译自维基百科\href{https://en.wikipedia.org/wiki/Eugene_Wigner}{相关文章}。

\begin{figure}[ht]
\centering
\includegraphics[width=6cm]{./figures/d45abc1dfbc9adc2.png}
\caption{} \label{fig_YJwgn_1}
\end{figure}
尤金·保罗·维格纳(匈牙利语:Wigner Jenő Pál,发音:[ˈviɡnɛr ˈjɛnøː ˈpaːl];1902年11月17日-1995年1月1日)是匈牙利裔美国理论物理学家,同时也对数学物理作出了贡献。他因“在原子核和基本粒子理论上的贡献,特别是发现并应用基本对称性原理”获得1963年诺贝尔物理学奖。[1]

维格纳毕业于柏林工业高等学校(今柏林工业大学),曾在柏林的威廉皇帝研究所担任卡尔·魏森伯格和理查德·贝克尔的助理,并在哥廷根大学跟随大卫·希尔伯特工作。维格纳与赫尔曼·外尔共同将群论引入物理学,尤其是引入物理对称性理论。在此过程中,他在纯数学领域也做出了开创性工作,发表了多篇重要数学定理,特别是“维格纳定理”,该定理是量子力学数学表述的基石。他还因对原子核结构的研究而闻名。1930年,普林斯顿大学招募了维格纳和约翰·冯·诺伊曼,他随即移居美国,并于1937年获得美国公民身份。

维格纳曾与利奥·西拉德和阿尔伯特·爱因斯坦共同参加会议,促成了著名的“爱因斯坦-西拉德信”,促使时任总统富兰克林·罗斯福批准成立铀顾问委员会,研究核武器的可行性。维格纳担心德国核武器项目会率先研制出原子弹。在曼哈顿计划期间,他领导团队设计核反应堆,将铀转化为武器级钚。当时,核反应堆还仅存在于纸面上,还没有任何反应堆实现临界。维格纳对杜邦公司不仅承担建造任务,还获得了反应堆详细设计权感到失望。1946年初,他成为克林顿实验室(今橡树岭国家实验室)研发负责人,但由于对美国原子能委员会官僚干预感到沮丧,重返普林斯顿。

战后,维格纳曾在多个政府机构任职,包括1947年至1951年在国家标准局任职,1951年至1954年在国家研究委员会数学小组任职,曾参与国家科学基金会物理小组工作,并于1952年至1957年及1959年至1964年两次担任美国原子能委员会有影响力的总咨询委员会成员。晚年时期,他更加关注哲学领域,并发表了《数学在自然科学中非凡有效性的非理性》一文,这篇文章成为他在技术数学和物理学以外最著名的作品。
\subsection{早年生活与教育}
