% 引力波和测试质量的相互作用

\begin{issues}
\issueNeedCite
\issueMissDepend
\end{issues}

这一节我们讲引力波和探测器的相互作用.在理想情况下,我们把探测器看作一系列测试质量.在广义相对论里面,数学上选择一个规范对应于物理上选择一个特定的观测者.在前面的学习中我们已经知道了在TT规范下,引力波具有非常简单的形式,所以现在我们想要懂得TT规范对应的是什么参考系.我们也将学习到,在探测器共动参考系下,描述探测器上的引力波将更加直观.因此,在我们讨论引力波和探测器的相互作用的时候,知道我们使用的是哪个参考系是非常重要的事情.

理解一个特定的规范下的物理意义,我们需要使用两个工具:测地线方程和测地线偏移方程.

\subsection{测地线方程和测地线偏移}
考虑一个含参数$\lambda$的曲线$x^\mu(\lambda)$.两个相距$d\lambda$的点的时空间隔是
\begin{equation}
\begin{aligned}
ds^2 & = g_{\mu\nu}  dx^\mu dx^\nu \\
& = g_{\mu\nu } \frac{dx^\mu}{d\lambda} \frac{dx^\nu}{d\lambda} d \lambda^2~.
\end{aligned}
\end{equation}
沿着类空曲线,我们有$ds^2>0$,所以我们可以直接开平方
\begin{equation}
ds = (g_{\mu\nu } dx^\mu dx^\nu )^{1/2}~. 
\end{equation}
这是曲线固有距离的一个量度.一个类时的曲线有$ds^2<0$.在这种情况下,我们可以定义固有时间$\tau$
\begin{equation}\label{intera_eq1}
c^2 d \tau^2  = - ds^2 = - g_{\mu\nu} dx^\mu dx^\nu ~. 
\end{equation}
我们可以直接使用$\tau$来作为参数$\lambda$. 因此$x^\mu = x^\mu(\tau)$. 从\autoref{intera_eq1} 我们可以导出
\begin{equation}\label{intera_eq2}
g_{\mu\nu} \frac{dx^\mu}{d\tau} \frac{dx^\nu}{d\tau} = - c^2~. 
\end{equation}
我们可以定义四动量如下
\begin{equation}
u^\mu = \frac{d x^\mu}{d \tau} ~,
\end{equation}
\autoref{intera_eq2} 可以写作
\begin{equation}
g_{\mu\nu} u^\mu u^\nu = - c^2~.
\end{equation}
考虑所有可能的满足$x^\mu(\tau_A)=x^\mu_{A}$和$x^\mu(\tau_B) = x^\mu_{B}$的类时曲线.对如下的作用量变分可以得到经典路径
\begin{equation}
S = - m \int_{\tau_A}^{\tau_B} d \tau~. 
\end{equation}
这给出了测地线方程
\begin{equation}\label{intera_eq3}
\frac{d^2 x^\mu}{d \tau^2} + \Gamma^\mu_{\nu\rho} (x) \frac{dx^\nu}{d \tau} \frac{dx^\rho}{d\tau} = 0~. 
\end{equation}
这是一个测试质量在不受外力的情况下在弯曲的背景下的经典运动方程.用四速度的话,测地线方程变成如下的形式
\begin{equation}
\frac{du^\mu}{d\tau} + \Gamma^\mu_{\nu\rho} u^\nu u^\rho = 0~.
\end{equation}
现在考虑隔得很近的两个测地线,一个是$x^\mu(\tau)$,另一个是$x^\mu(\tau)+\xi^\mu(\tau)$. $x^\mu(\tau)$满足\autoref{intera_eq3} ,$x^\mu(\tau) + \xi^\mu(\tau)$满足
\begin{equation}\label{intera_eq4}
\frac{d^2(x^\mu+\xi^\mu)}{d\tau^2} + \Gamma^\mu_{\nu\rho} (x+\xi) \frac{d(x^\nu+\xi^\nu)}{d\tau} \frac{d x^\rho+\xi^\rho}{d \tau} = 0~.
\end{equation}
\autoref{intera_eq4} - \autoref{intera_eq3} 并且展开到$\xi$的一阶,我们有
\begin{equation}\label{intera_eq5}
\frac{d^2 \xi^\mu}{d \tau^2} + 2 \Gamma^\mu_{\nu\rho} (x) \frac{dx^\nu}{d\tau} \frac{d\xi^\rho}{d\tau} + \xi^\sigma \partial_\sigma \Gamma^\mu_{\nu\rho} (x) \frac{dx^\nu}{d\tau} \frac{dx^\rho}{d\tau} = 0 ~. 
\end{equation}
引入协变导数
\begin{equation}
\frac{D V^\mu}{d\tau} \equiv \frac{d V^\mu}{d\tau} + \Gamma^\mu_{\nu\rho} V^\nu \frac{dx^\rho}{d\tau} ~.
\end{equation}
于是\autoref{intera_eq5} 可以被写作
\begin{equation}
\frac{D^2 \xi^\mu}{D \tau^2} = - R^\mu_{\nu\rho\sigma} \xi^\rho \frac{dx^\nu}{d\tau} \frac{dx^\sigma}{d\tau} ~,
\end{equation}
用四速度表示出来,我们有
\begin{equation}
\frac{D^2 \xi^\mu}{D\tau^2} = - R^\mu_{\nu\rho\sigma} \xi^\rho u^\nu u^\sigma ~. 
\end{equation}
这个方程告诉我们两个相邻的类时测地线会受到潮汐引力.潮汐引力是由黎曼张量决定的.

\subsection{局域惯性系和自由落体参考系}
在讨论TT参考系和探测器参考系之前,我们可以回顾一下局域惯性系和自由落体参考系.在广义相对论里,总可以存在一个坐标变换,令给定的坐标点$P$上的所有的Christoffel符号都为零.
\begin{equation}
\Gamma^\mu_{\nu\rho} (P) = 0 ~.
\end{equation}
在这个系下,$P$点的测地线方程为
\begin{equation}
\frac{d^2 x^\mu}{d\tau^2} \bigg|_{P} = 0 ~.
\end{equation}
在这个参考系下,测试粒子是自由下落的,虽然只是在时空上的一个点而已.这样的参考系被称为“局域惯性系”.

这样的参考系可以按照如下的步骤来构造.在点P我们可以选择一组四矢量$e_\alpha$.$\alpha = 0,\ldots,3$. 让他们满足$\eta_{\mu\nu}e^\mu_\alpha e^\nu_\beta = \eta_{\alpha\beta}$. 现在考虑从P点出发的测地线.($n^0$,$n^1$,$n^2$,$n^3$)是基矢$e_\alpha$的四个分量.用这个方法,我们可以走到时空中的所有点.

在一个足够小的空间里面,测地线不会相交.所以每个点只会碰到有且仅有一条测地线.因此,这个方法可以让我们给P点附近足够小的区域里面的所有点给定坐标.这样的坐标系统被成为Riemann normal coordinates.我们现在来验证Riemann normal coordiante确实是局域惯性系的一种具体实现.

首先,从$\{e_{\alpha}\}$的定义我们可以自然推导出$g_{\mu\nu}(P) = \eta_{\mu\nu}$这样一个事实.为了证明$\Gamma^\mu_{\nu\rho}(P)$也成立,考虑测地线方程\autoref{intera_eq3} .依照定义,坐标对于固有时是线性的,这说明$d^2 x^\mu/d \tau^2 = 0$, $dx^\mu/d\tau = n^\nu$. 因此测地线方程变成了如下形式
\begin{equation}
\Gamma^\mu_{\nu\rho} (P) n^\nu n^\rho = 0 ~.
\end{equation}
上式对所有$n^\mu$都成立,我们得出如下结论
\begin{equation}
\Gamma^\mu_{\nu\rho} (P) = 0~.
\end{equation}
Riemann normal coordinates因此为局域惯性系提供了一个具体的例子.

在局域惯性系里面,一个测试粒子只能在时空中的一点自由运动.但我们可以做得比这个好得多.我们可以建造一个测试质量在整个测地线都是自由落体运动的参考系.一个自由旋转的物体沿着$x^\mu(\tau)$运动会满足如下方程
\begin{equation}
\frac{ds^\mu}{d\tau} + \Gamma^\mu_{\nu\rho} s^\nu \frac{dx^\rho}{d\tau} = 0~.
\end{equation}
其中$s^\mu$是自旋四矢量.在静止系下,这个四矢量可以约化到$s^\mu = (0,\mathbf s)$. 这个方程是$ds^\mu/d\tau=0$的协变形式.$ds^\mu/d\tau=0$说明了平坦时空下的角动量守恒.我们从P点开始构造局域惯性系.我们使用三个陀螺仪来标记空间坐标.然后我们把这个参考系沿着测地线移动.从定义可以看出,在这个参考系下,陀螺仪不会随着坐标轴旋转,因为它们定义了轴的方向.于是,在整个类时测地线上,我们都有$ds^\mu/d\tau = 0$并且$\Gamma^\mu_{\nu\rho}$在整个类时测地线上都为0. 这样的参考系被称作自由落体参考系(Riemann normal coordinates的坐标轴用陀螺仪来标记).它的坐标被成为Fermi normal coordinates.一个自由下落参考系因此是整个测地线上的一个局域的惯性参考系.这样一个参考系是自由在空间中移动的卫星所使用的参考系.

\subsection{TT参考系和proper detector参考系}
\subsubsection{TT参考系}
在TT参考系下,引力波可以呈现出特别简单的形式.从\autoref{intera_eq3} 可以看出
\begin{equation}
\frac{d^2 x^i}{d\tau^2} \bigg|_{\tau = 0} = - \bigg[ \Gamma^i_{\nu\rho} (x) \frac{dx^\nu}{d\tau} \frac{dx^\rho}{d\tau} \bigg]_{\tau = 0} = - \bigg[ \Gamma^i_{00} \bigg( \frac{dx^0}{d\tau} \bigg)^2 \bigg]_{\tau = 0}~.
\end{equation}
推导第二个等号的时候我们用到了在$\tau = 0$的时候$dx^i/d\tau = 0$. 这是因为我们令测试质量一开始处于静止状态.现在我们把度规进行展开$g_{\mu\nu} = \eta_{\mu\nu} + h_{\mu\nu}$然后按照$h_{\mu\nu}$进行小量展开,克里斯托夫联络变成了
\begin{equation}
\Gamma^\mu_{\nu\rho} = \frac{1}{2} \eta^{\mu\sigma} (\partial_\nu h_{\rho\sigma} + \partial_\rho h_{\nu\sigma} - \partial_{\sigma} h_{\nu\rho}  )~. 
\end{equation}
因此
\begin{equation}
\Gamma^i_{00} = \frac{1}{2} (2\partial_0 h_{0i} - \partial_i h_{00}) ~.
\end{equation}
在TT规范下,这个量为零,因为规范条件会令$h_{00} = h_{0i} = 0$.因此,如果在$\tau = 0$时刻$dx^i/d\tau$为0,在TT规范下它的导数$d^2 x^i/d\tau^2$同样为0,因此,$dx^i/d\tau$总是为0.这表明,在TT规范下,如果一个粒子在波传来之前是静止的,那么波传过去之后也会同样保持静止.

在测试粒子之间的距离很小的时候,测地线偏离方程是适用的.在TT规范下,两个本来静止的测试粒子之间的距离$\xi^i$同样也不变.我们使用测地线偏离方程的第$i$分量.因为在$\tau = 0$,$dx^i/d\tau = 0$, $dx^0/d\tau = c$,所以我们有
\begin{equation}
\frac{d^2\xi^i}{d\tau^2} \bigg|_{\tau = 0} = - \bigg[ 2 c \Gamma^i_{0\rho} \frac{d\xi^\rho}{d\tau} + c^2 \xi^\sigma \partial_\sigma \Gamma^i_{00} \bigg]_{\tau = 0}~.
\end{equation}
在TT规范下,这个方程可以化简为
\begin{equation}
\frac{d^2\xi^i}{d\tau^2}\bigg|_{\tau = 0} = - \bigg[ \dot{h}_{ij} \frac{d\xi^i}{d\tau} \bigg]_{\tau = 0}~. 
\end{equation}
因此,在$\tau = 0$,我们有$d\xi^i/d\tau = 0$,于是我们同样有$d^2\xi^i/d\tau^2 = 0$.因此$\xi^i$会一直保持不变.在TT规范下,原本静止的测试质量测到的原时$\tau$和坐标时$t$是相等的.

TT规范表明了,在广义相对论里面,物理的效应并不依赖于坐标系,所以这个理论是对于坐标变换不变的.在TT规范下测试质量的位置保持不变并不说明引力波对测试质量不起作用,只不过是我们通过测试质量来定义这样一个坐标系,使得引力波经过的时候,测试质量保持静止.如果我们需要看出物理的变化,我们可以看proper distance或者proper time.

比方说考虑这两个事件$(t,x_1,0,0)$和$(t,x_2,0,0)$. 在TT规范下,坐标距离$x_2-x_1=L$保持不变,即使是引力波沿着z轴传播也是如此.但是两个点之间的proper distance是
\begin{equation}
s = (x_2-x_1)[1+h_+\cos(\omega t)]^{1/2} \simeq L[1+\frac{1}{2} h_+ \cos(\omega t) ] ~.
\end{equation}




