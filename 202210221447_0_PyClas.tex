% Python 的类
% keys Python|编程|数值计算

\begin{issues}
\issueDraft
\end{issues}

\addTODO{类和继承参考\href{https://www.w3schools.com/python/python_inheritance.asp}{这里}, 算符重载参考\href{https://www.geeksforgeeks.org/operator-overloading-in-python/}{这里}.}

\subsection{基础}
\begin{lstlisting}[language=python]
# 定义 person 类
class person(): # 括号可以省略
    def __init__(self, name, age): # 构造函数(只能有一个)
        self.name = name 
        self.age = age 
    def show(self): 
        print("name is", self.name ) 
        print("age is", self.age ) 

# 生成对象
p1 = person("jason", "30") 
p2 = person("justin", "28")

# 调用成员函数
p1.show();
p2.show();
\end{lstlisting}
运行结果
\begin{lstlisting}
name is jason
age is 30
name is justin
age is 28
\end{lstlisting}

来定义一个平面点类
\begin{lstlisting}[language=python]
class point:
    """这里是 point.__doc__ 的内容"""
    def __init__(self, x, y): 
        self.x = x 
        self.y = y 
    def __str__(self): # 用于 print()
        return "({0}, {1})".format(self.x, self.y)
    def __add__(self, other): # 算符 +, self 必须是第一个变量
        if isinstance(other, point):
            return point(self.x + other.x, self.y + other.y)
        else:
            return point(self.x + other, self.y + other)
\end{lstlisting}
\begin{itemize}
\item 用 \verb|dir(类名)| 或者 \verb|dir(obj)| 查看对象的所有性质(attributes)
\item 用 \verb|vars(obj)| 查看对象的所有数据成员, 等效地也可以用 \verb|teddy.__dict__|
\item \verb|obj.__module__| 查看对象的模块, 如果不在模块中定义的, 就是 \verb|'__main__'|. 如果是内建类型, 就是 \verb|'builtins'|
\end{itemize}

\subsection{算符重载}
\begin{itemize}
\item 算符对应的函数名: \verb|__add__|(\verb|+|), \verb|__sub__|(\verb|-|), \verb|__mul__|(\verb|*|), \verb|__truediv__|(\verb|/|), \verb|__floordiv__|(\verb|//|), \verb|__mod__|(\verb|%|), \verb|__lshift__|(\verb|<<|), \verb|__rshift__|(\verb|>>|), \verb|__and__|(\verb|&|), \verb|__or__|(\verb`|`), \verb|__xor__|(\verb|^|), \verb|__invert__|(\verb|~|), \verb|__lt__|(\verb|<|), \verb|__le__|(\verb|<=|),\verb|__eq__|(\verb|==|),\verb|__ne__|(\verb|!=|),\verb|__gt__|(\verb|>|),\verb|__ge__|(\verb|>=|), \verb|__getitem__(self, i)|(\verb|v[i]|), \verb|__setitem__(self, i, val)|(\verb|v[i] = val|), \verb|__iter__(self)| (迭代器 \verb|iter()|), \verb|__next__(self)|(迭代器 \verb|next()|)
\item \verb|__radd__(self, other)| 可以定义 \verb|other + self| (当 \verb|other| 具有不同类型时会被调用).
\item \verb|x[start:end:step]| 中, \verb|start:end:step| 等效于 \verb|slice(start, end, step)|, 类型是 \verb|slice|. 如果三个中哪个被省略了, 哪个就是 \verb|None|. 注意 \verb|start:end:step| 不能单独使用.
\end{itemize}

\subsection{继承}
\begin{itemize}
\item \verb|class 类名(基类名):| 定义派生类, \verb|类名.__base__| 查看基类. 最底层的基类都是 \verb|object|
\end{itemize}
