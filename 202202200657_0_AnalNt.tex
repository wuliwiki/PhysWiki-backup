% 数学分析笔记
% rudin|数学分析|实数|复数|集合论|集合

\begin{issues}
\issueMissDepend
\issueOther{这是一个总结, 应该放到所有相应内容之后, 以及给出详细词条的链接}
\end{issues}

本文参考 \cite{Rudin}.

\subsection{Chap 1. 实数系和复数系}

\begin{itemize}
\item \textbf{有理数(rational number)}记为 $Q$, 实数记为 $R$

\item 虽然任意两个不同的有理数间还有一个有理数, 但是有理数集中还是会有 “间隙”, 而实数集填补了这些间隙.

\item \textbf{集合(set)}:\textbf{属于(in)} $x \in A$, \textbf{不属于(not in)} $x \notin A$

\item \textbf{空集(empty set)}, \textbf{非空(none empty)},\textbf{子集(subset)} $A \subseteq B$,\textbf{超集(superset)} $B \supseteq A$, \textbf{真子集(proper subset)}

\item \textbf{有序集(ordered set)}, 任意不相等的两个元可以比较大小

\item \textbf{有上界(bounded above)}: 任意元小于等于超集中的某个元. \textbf{上界(upper bound)}

\item \textbf{最小上界(least upper bound, supremem)} : $\alpha = \sup E$; \textbf{最大下界(greatest lower bound, infimum)}: $\alpha = \inf E$

\item 有上界未必有最小上界. 例如有理数集合中,小于 $\sqrt{2}$ 的子集不存在最小上界.

\item 如果对于任意非空有上界的 $E \subset S$, 都有 $\sup E \in S$, 那 $S$ 就具有 \textbf{upper bound property}

\item \textbf{域(field)} 集合 $F$ 定义了\textbf{加法}和\textbf{乘法}. 加法满足: 闭合性, 交换律, 结合律, 存在 0 元, 存在逆元. 乘法满足: 闭合性, 交换律, 结合律, 存在单位元, 存在倒数. 加法和乘法满足分配律.

\item 有理数集是一个域

\item \textbf{有序域(ordered field)}

\item 存在一个有序域 $R$ 具有 upper bound property, 且有理数集 $Q$ 是其子集. $R$ 就是实数.

\item 实数的\textbf{阿基米德性质}: 存在整数 $n$ 使 $nx > y$ ($x > 0$)

\item $x \in R$, $x > 0$, $n$ 为整数, 存在实数 $y$ 使 $y^n = x$

\item \textbf{稠密(dense)}: 两个不同的实数间必有一个有理数

\item \textbf{extended real number system} 是在实数集基础上加入 $\pm\infty$ 两个符号. 对任何实数有 $-\infty < x < +\infty$. 所有非空子集都有最小上界和最大下界. 相比于无穷, 实数集中的元被称为 \textbf{finite}.

\item 复数是一对有序实数 $(a, b)$, 定义了加法和乘法后, 就变成了一个域. 定义 $\I = (0, 1)$.

\item 对正整数 $k$, $R^k$ 定义为所有 $k$ 个有序实数的集合 $\bvec x = (x_1, \dots, x_k)$, 其中 $x_i$ 叫做坐标.

\item 定义 $R^k$ 中的内积为 $\bvec x \vdot \bvec y = \sum_{i = 1}^k x_i y_i$

\item 定义模长为 $\abs{x} = (\bvec x \vdot \bvec x)^{1/2}$

\item 定义了内积和模长的 $R^k$ 被称为\textbf{欧几里得 $k$ 空间(euclidean k-space)}. 这也是一个\textbf{度量空间(metric space)}(见下文). 通常 $R^1$ 叫做线, $R^2$ 叫做面
\end{itemize}

\subsection{Chap 2. 基本拓扑}

\begin{itemize}
\item \textbf{函数}是两个集合 $A$, $B$ 之间的\textbf{映射}; 定义域, 值域, 值. $f(A)$ 就是集合 $A$ 的 \textbf{image}; $f(A) \subset B$. 如果 $f(A) = B$, 那么 $f$ 把 $A$ 映射到(onto) $B$. $f^{-1}(E)$ \textbf{inverse image}. \textbf{1-1 映射}

\item 如果 $A$ 到 $B$ 存在 1-1 映射, 记为 $A \sim B$: \textbf{reflexive} $A \sim A$, \textbf{symmetric} $A \sim B \to B \sim A$, \textbf{transitive} $A \sim B, B \sim C \to A \sim C$. 此时称 $A$ 和 $B$ 等效, 他们具有相同的\textbf{基数(cardinal number)}即元素个数.

\item 定义 $J_n$ 为集合 $1,2,\dots, n$, 定义 $J$ 为 $1, 2, \dots$

\item $A \sim J_n$ 为\textbf{有限(finite)}, 不是有限就是\textbf{无限(infinite)}, $A \sim J$ 为\textbf{可数(countable)}\footnote{也叫 enumerable 或者 denumerable}. \textbf{至多可数(at most countable)} 就是有限或者可数

\item 对无限集来说, “含有同样多个元素” 变得很模糊, 但是 1-1 映射的定义仍然有效(只要写出一个表达式)

\item 有限集不可能与其真子集等效, 而无限集可以

\item 数列是 $J$ 的映射 $f(n) = x_n$, 记为 $\{x_n\}$. $x_n$ 叫做一项. 如果 $x_n \in A$, 那该序列就叫 $A$ 中(元素)的序列.

\item 可数集的无穷子集仍然是可数的

\item \textbf{Sequence of set} $\{E_\alpha\}$, 每个 $\alpha \in A$ 都对应一个 $E_\alpha$(其实就是 set of sets, 但不这么叫)

\item \textbf{并集(Union)}: $\bigcup\limits_{m = 1}^n E_m$, \textbf{交集(intersection)}: $\bigcap\limits_{m = 1}^n E_m$

\item 并集和交集的混合运算法则与加法和乘法差不多

\item 如果 $E_n\ (n = 1, 2\dots)$ (无穷多个)是可数的, 那么它们的并集仍然是可数的

\item 如果 $A$ 是至多可数, 且对 $\alpha \in A$, $B_\alpha$ 也是至多可数, 那么 $T = \bigcup_{\alpha \in A} B_\alpha$ 也是至多可数

\item 如果 $A$ 可数, $a_i \in A$,  而 $B_n$ ($n$ 为固定的正整数)是所有 $(a_1, \dots, a_n)$ 的集合, 那么 $B_n$ 也是可数的

\item 自然数和有理数(可以看作两个有序整数)都可数, 无理数不可数

\item 如果集合 $X$ 中的元素可以叫做\textbf{点(point)}, 如果一个值为实数的函数 $d(p, q), \ p \in X,\ q \in X$ 满足: 当 $p = q$, $d(p, q) = 0$, 当 $p \ne q$, $d(p, q) > 0$, $d(p, q) = d(q, p)$, $d(p, q) \leqslant d(p, r) + d(r, q)$, $r \in X$, 我们就说这是一个\textbf{度量空间(metrix space)}, 函数 $d$ 叫做\textbf{距离函数(distance function)}, 或者\textbf{度规(metric)}. 度读 du 第四声.

\item \textbf{区间(segment)} $(a, b)$ 是所有 $a < x < b$ 的实数, \textbf{interval} $[a, b]$ 是所有 $a \leqslant x \leqslant b$ 的实数.

\item interval 也叫 1-方格, 类似地, $R^k$ 中可以定义 \textbf{k-方格(k-cell)}, 2-方格是长方形

\item 类似地, $R^k$ 空间也可以定义 \textbf{开/闭球(open/closed ball)}

\item \textbf{凸(convex)}: $E \subset R^k$ 对任意 $0 < \lambda < 1$ 和 $\bvec x, \bvec y \in E$ 满足 $\lambda \bvec x + (1 - \lambda) \bvec y \in E$. 例如, ball 和 k-方格都是 convex 的.

\item 度量空间中, \textbf{邻域(neighborhood)} $N_r$: 到某点距离小于 $r$ 的集合($r > 0$)

\item 度量空间中, \textbf{极限点(limit point)} $p$: 所有邻域存在一个与 $p$ 不同的点(无论半径有多小)

\item 如果不是极限点, 那就是 \textbf{孤立点(isolated point)}

\item 对度量空间的子集 $E$, 如果所有极限点都属于 $E$, $E$ 就是\textbf{闭(closed)}的.

\item 对度量空间的子集 $E$, 如果点 $p \in E$ 的在度量空间中某个邻域是 $E$ 的子集, $p$ 就是 $E$ 的\textbf{内点(interior point)}

\item 对度量空间的子集 $E$, 如果 $E$ 中的任意一点都是内点, $E$ 就是 \textbf{开(open)} 的.

\item 特例: 在孤立点构成的度量空间中, 任何子集都既开又闭.

\item 注意度量空间 $X$ 的子集 $E$ 的开或闭取决于 $X$ 的选取. 如果 $E$ 就是 $X$ 本身, 那么 $E$ 既开又闭.

\item \textbf{补集(complement)}

\item 如果一个闭集合中每一点都是它的极限点, 那么该集合就是\textbf{完全(perfect)} 的

\item 如果集合中任意一点都在某个 $r$ 为实数的邻域内, 这个集合就是\textbf{有界的(bounded)}

\item (私货)对一个度量空间 $X$, 若存在 $r\in R$ 使任意两点 $p,q\in X$ 都满足 $d(p,q) \leqslant r$, 那么 $X$ 就是\textbf{有界的(bounded)}

\item 集合 $E$ 在集合 $X$ 上\textbf{稠密(dense)}: $X$ 中任意一点都是 $E$ 的一个极限点或者 $E$ 中的一点. (例如有理数在实数上稠密)

\item 任何邻域都是开的

\item 如果 $p$ 是一个极限点, 那么它的任何邻域都有无限多个点

\item 有限个点的集合中没有极限点

\item $(\bigcup_\alpha E_\alpha)^c = \bigcap_\alpha (E_\alpha^c)$ 其中 $c$ 代表补集

\item 集合 $E$ 是开的当且仅当它的补集是闭的. $E$ 是闭的当且仅当它的补集是开的.

\item 空集和全集既开又闭

\item 任意多开集合的并集仍然是开的, 任意多闭集合的交集仍然是闭的

\item 有限个开集合的交集仍然是开的, 有限个闭集合的并集仍然是闭的

\item 设 $X$ 是度量空间, 如果 $E \subset X$, $E'$ 表示 $E$ 在 $X$ 中所有极限点组成的集. 那么, 把 $\bar E = E \cup E'$ 叫做 $E$ 的\textbf{闭包(closure)}

\item 设 $X$ 是度量空间,而 $E \subset X$, 那么 (a) $\bar E$ 是闭的, (b) $E = \bar E$ 当且仅当 $E$ 闭, (c) 如果闭集 $F \subset X$ 且 $E \subset F$, 那么 $\bar E \subset F$. 由 (a) 和 (c), $\bar E$ 是 $X$ 中包含 $E$ 的最小闭子集

\item 设 $E$ 是一个非空实数集, 上有界, 令 $y = \sup E$ ,那么 $y \in \bar E$.
因此, 如果 $E$ 闭, 那么 $y \in E$.

\item 令 $Y \subset X$, $E \subset Y$, $E$ 相对 $Y$ 是开的当且仅当 $E = Y \bigcap G$, 对某个开的 $G \subset X$

\item 若 $X$ 的一组开子集 $\{G_\alpha\}$ 使 $E \subset \bigcup_\alpha G_\alpha$, 那么 $\{G_\alpha\}$ 就是 $E$ 的\textbf{开覆盖(open cover)}

\item \textbf{紧集(compact set)}: 如果 $\{G_\alpha\}$ 是 $K$ 的开覆盖, 那么存在有限个 $\alpha_1,\dots, \alpha_n$ 使得 $K \subset G_{\alpha_1} \bigcup \dots \bigcup G_{\alpha_n}$. 即任何开覆盖都存在有限的子覆盖. 紧集是分析中的非常重要概念.

\item 有限集都是紧集

\item 如果 $E \subset Y \subset X$, 那么 $E$ 可能是 $Y$ 中的开集而不是 $X$ 的开集. 闭集也同理.

\item 2.33 假设 $K \subset Y \subset X$. 那么 $K$ 在 $X$ 中是紧的当且仅当它在 $Y$ 中也是紧的.

\item 2.34 度量空间的紧子集是闭的

\item (私货)一个例子: 证明开区间 $(0,1)$ 不是一个紧集: 易知它的一组无穷开覆盖为 $\bigcup_{n=0}^{\infty}((2/3)^n/3, (2/3)^n)$, 且不存在有限子覆盖. 反之 $[0,1]$ 则不存在类似的问题. 另外注意这性质与所考虑区间的父集无关.

\item (私货)度量空间的紧集都是有界的.

\item 2.35 紧集的闭子集也是紧的

\item 闭集和紧集的交集是紧的

\item 2.36 如果 $\{K_\alpha\}$ 是度量空间 $X$ 的一组紧子集且任意有限个 $\{K_\alpha\}$ 的交集为非空, 那么 $\bigcap K_\alpha$ 也是非空的

\item 2.37 如果 $E$ 是紧集 $K$ 的无穷子集, 那么 $E$ 在 $K$ 中存在极限点

\item 2.38 如果 $\{I_n\}$ 是 $R^1$ 中的一组闭区间序列, 且 $I_n \supset I_{n+1} (n = 1, 2, 3,\dots)$, 那么 $\bigcap_1^\infty I_n$ 非空

\item 2.40 $k$-方格是紧的

\item 2.41 对 $R^k$ 中的集合 $E$, 这三个条件等价: (a) $E$ 闭且有界. (b) $E$ 是紧的. (c) $E$ 中的任意无限集在 $E$ 中存在极限点

\item 2.42 Weierstrass 定理: $R^k$ 中任何有界的无限集在 $R^k$ 中有(至少)一个极限点

\item 2.43 令 $P$ 为 $R^k$ 内的非空完全集. 那么 $P$ 是不可数的

\item 2.44 \textbf{Cantor 集}说明 $R^1$ 中存在没有区间的完全集.

\item 2.45 度量空间 $X$ 的两个子集 $A$ 和 $B$ 被称为\textbf{分离的(separated)} 如果 $A \cap \bar B$ 和 $\bar A \cap B$ 都是空集.

\item 如果 $E \subset X$ 不是两个非空分离集的并, 就说 $E$ 是连通(connected)集.

\item 2.46 分离的两个集是不相交的, 但不相交的集合不一定是分离的. 例如 $[0,1]$ 和 $(1,2)$ 不是分离的.

\item 2.47 实数集 $R^1$ 的子集 $E$ 是连通的, 当且仅当: 如果 $x\in E$, $y\in E$, 且 $x < z < y$, 那么 $z \in E$.
\end{itemize}

\subsection{Chap 3. 数列与级数}

\begin{itemize}
\item 3.1 度量空间 $X$ 中的序列 $\{p_n\}$ 叫做\textbf{收敛的(converged)}, 如果有一个有下述性质的点 $p\in X$: 对每个 $\epsilon > 0$, 有一个正整数 $N$, 使得 $n \geqslant N$ 时, $d(p_n,p) < \epsilon$. 这时候也说 $\{p_n\}$ 收敛于 $p$, 或者说 $p$ 是 $\{p_n\}$ 的极限, 写作 $p_n\to p$ 或 $\lim_{n\to \infty} p_n = p$. 如果不收敛, 就说它发散.

\item 收敛的定义不仅依赖于数列还依赖于 $X$, 例如 $\{1/n\}$ 在 $R^1$ 中收敛于 $0$, 但在正实数集合中不收敛. 所以要强调 “在 $X$ 中” 收敛. (这与小时百科的\autoref{cauchy_def1}~\upref{cauchy} 不同)

\item 一切点 $p_n$ 的集合是 $\{p_n\}$ 的\textbf{值域(range)}, 序列的值域可以是有限的, 也可以是无限的. 如果值域是有界的, 就说序列是有界的.

\item 3.2 度量空间 $X$ 中的序列 $\{p_n\}$: (a) $\{p_n\}$ 收敛于 $p\in X$, 当且仅当 $p$ 的每个邻域, 能包含除了有限项以外的一切项. (b) 如果数列同时收敛于 $p, p'$, 那么 $p' = p$. (c) 数列收敛则必有界. (d) 如果 $E \subset X$, 而 $p$ 是 $E$ 的极限点, 那么在 $E$ 中有一个序列收敛到 $p$.

\item 3.3 假定 $\{s_n\}, \{t_n\}$ 是复序列,且极限为 $s, t$ 那么 (a) $\lim_{n\to\infty} (s_n+t_n) = s+t$, (b) 对任何数 $c$, $\lim_{n\to\infty}cs_n = cs$; (c) $\lim_{n\to\infty}(s_n t_n) = st$; (d) $\lim_{n\to\infty} (c+s_n) = c+s$

\item 3.4 (a) 假定 $\bvec x_n \in R^k$ ($n=1,2,...$) 而 $\bvec x_n = (a_{1,n}, ...,a_{k,n})$ 那么序列收敛于 $(a_1,...,a_k)$ 当且仅当 $\lim_{n\to\infty} a_{j,n} = a_j$; (b) 假定 $\{\bvec x_n\}, \{\bvec y_n\}$ 是 $R^k$ 的序列, $\{\beta_n\}$ 是实数序列, 并且 $\bvec x_n\to \bvec x, \bvec y_n\to \bvec y, \bvec \beta_n\to\bvec \beta$. 那么 $\lim_{n\to\infty} (\bvec x_n+\bvec y_n) = \bvec x+\bvec y$, $\lim_{n\to\infty} \bvec x_n \vdot \bvec y_n = \bvec x \vdot \bvec y$, $\lim_{n\to\infty} \beta_n \bvec x_n = \beta \bvec x$.

\item 3.5 有序列 $\{p_n\}$, 取正整数序列 $\{n_k\}$, 使 $n_1<n_2<...$ 那么序列 $\{p_{n_i}\}$ 便叫做 $\{p_n\}$ 的\textbf{子序列(subsequence)}, 如果 $\{p_{n_i}\}$ 收敛, 就把它的极限叫做 $\{p_n\}$ 的\textbf{部分极限(subsequential limit)}. 序列收敛于 $p$ 当且仅当它的任何子序列收敛于 $p$.

\item 3.6 如果 $\{p_n\}$ 是紧度量空间 $X$ 中的序列, 那么 $\{p_n\}$ 有某个子序列收敛到 $X$ 中的某个点. (b) $R^k$ 中的每个有界序列含有收敛的子序列.

\item 3.7 度量空间 $X$ 里的序列 $\{p_n\}$ 的部分极限组成 $X$ 的闭子集.

\item 3.8 度量空间 $X$ 中的序列 $\{p_n\}$ 叫做\textbf{柯西序列(Cauchy)}, 如果对于任何 $\epsilon>0$ 存在着正整数 $N$, 只要 $m,n\geqslant N$ 就有 $d(p_n, p_m)<\epsilon$.

\item 3.9 设 $E$ 是度量空间 $X$ 的非空子集, 又设 $S$ 是一切形式为 $d(p,q)$ 的实数集, $p,q\in E$. $\sup S$ 叫做 $E$ 的直径, 记为 $\opn{diam} E$.

\item 3.10 (a) 如果 $E$ 是度量空间 $X$ 中的集, 那么闭包满足 $\opn{diam}\bar E = \opn{diam} E$. (b) 如果 $\{K_n\}$ 是 $X$ 中的紧集的序列, 且 $K_n \supseteq K_{n+1}$ 又若 $\lim_{n\to\infty} \opn{diam} K_n = 0$, 那么 $\bigcap_1^\infty K_n$ 由一个点组成.

\item 3.11 (a) 在度量空间中, 收敛序列是柯西序列. (b) 如果 $X$ 是紧度量空间, 并且如果 $\{p_n\}$ 是 $X$ 中的柯西序列, 那么序列收敛于 $X$ 的某个点\footnote{闭空间中的柯西序列都收敛, 度量空间的紧子集是闭的 (2.34).}. (c) 在 $R^k$ 中, 每个柯西序列收敛.

\item 3.12 如果度量空间 $X$ 的每个柯西序列都在 $X$ 中收敛, 就说该空间是完备的.

\item 因此所有紧度量空间以及所有欧氏空间都是完备的. 还说明度量空间 $X$ 的闭子集是完备的.

\item 3.13 实数序列的\textbf{单调递增}($s_n\leqslant s_{n+1}$) 和\textbf{单调递减}.

\item 3.14 单调序列收敛, 当且仅当它是有界的.

\item 3.15 $s_n\to \pm \infty$ 的定义

\item 3.16 设 $\{s_n\}$ 是实数序列. $E$ 是所有可能的子序列的极限组成的集(可能含有 $\pm\infty$). $s^* = \sup E$, $s_* = \inf E$ 这两个数叫做序列的\textbf{上极限(upper limit)}和\textbf{下极限(lower limit)}. 记为 $\lim_{n\to\infty} \sup s_n = s^*$, $\lim_{n\to\infty} \inf s_n = s_*$.

\item 3.17 

\item 3.18 例: (a) 给出一个包含一切有理数的序列, 那么每个实数是它的部分极限, 且 $\lim_{n\to\infty}\sup s_n = +\infty$, $\lim_{n\to\infty}\inf s_n = -\infty$. (b) 设 $s_n = (-1)^n/[1+(1/n)]$, 则上下极限为 $1,-1$. (c) 实数序列的极限为 $s$ 当且仅当上下极限都等于 $s$.

\item 3.19 如果 $N$ 是固定的正整数, 当 $n\geqslant N$ 时 $s_n \leqslant t_n$, 那么 $\lim_{n\to\infty}\inf s_n \leqslant \lim_{n\to\infty}\inf t_n$, $\lim_{n\to\infty}\sup s_n \leqslant \lim_{n\to\infty}\sup t_n$.

\item 3.20 (a) $p > 0$ 时 $\lim_{n\to\infty}1/n^p = 0$. (b) $p>0$ 时 $\lim_{n\to\infty}{}\sqrt[n]{p} = 1$. (c) $\lim_{n\to\infty} \sqrt[n]{n} = 1$. (d) $p>0$, 且 $\alpha$ 是实数时 $\lim_{n\to\infty} n^\alpha/(1+p)^n = 0$. (e) $\abs{x}<1$ 时 $\lim_{n\to\infty} x^n = 0$.

\item 3.21 对序列 ${a_n}$, 令 $s_n = \sum_{k=1}^n a_k$ 为\textbf{部分和}, $\sum_{n=1}^\infty a_n$ 叫做无穷级数, 简称级数. 如果 $s_n$ 收敛, 就说级数收敛, 并记为 $\sum_{n=1}^\infty a_n = s$. $s$ 叫做级数的和, 是 $s_n$ 的极限. 如果 $s_n$ 发散, 就说级数发散.

\item 柯西准则(3.11)可以重新表述为, $\sum a_n$ 收敛, 当且仅当, 对于任意 $\epsilon>0$, 存在整数 $N$, 使得 $m \geqslant n \geqslant N$ 时 $\abs{\sum_{k=n}^m a_k} \leqslant \epsilon$. 特别地,当 $m=n$ 时 $\abs{a_n}\leqslant\epsilon$($n\geqslant N$)

\item 3.23 如果 $\sum a_n$ 收敛, 则 $\lim_{n\to\infty} a_n = 0$.

\item 3.24 各项为非负的级数收敛, 当且仅当其部分和构成有界数列.

\item 3.25 (a) 如果 $N_0$ 是某个固定的正整数. $n \geqslant N_0$ 时 $\abs{a_n}\leqslant c_n$ 而且 $\sum c_n$ 收敛, 那么 $\sum a_n$ 也收敛. (b) 如果当 $n\geqslant N_0$ 时 $a_n\geqslant d_n\geqslant 0$ 而且 $\sum d_n$ 发散, 那么 $\sum a_n$ 也发散.

\item 如果 $f$ 在 $p$ 有极限, 那么它是唯一的

\item $\lim_{x\to p}f(x)=A$, $\lim_{x\to p}g(x)=B$, 那么 (a) $\lim_{x\to p}(f+g)(x)=A+B$, (b) $\lim_{x\to p}(fg)(x)=AB$, (c) $\lim_{x\to p}(f/g)(x)=A/B$(若 $B\ne 0$).
\end{itemize}

\addTODO{以下所有内容}

\subsection{Chap 4. 连续性}

\begin{itemize}
\item 4.1 \textbf{函数的极限}:令 $f:E\subset X\to Y$, $X,Y$ 为度量空间, 且 $p$ 是 $E$ 的极限点. 凡是我们写当 $x\to p$ 时 $f(x)\to q$, 或 $\lim_{x\to p}f(x)=q$, 就是存在 $q\in Y$ 具有以下性质: 对每个 $\epsilon>0$, 存在 $\delta>0$, 使 $d_{Y}(f(x),q)<\epsilon$ 对满足 $0<d_X(x,p)<\delta$ 的一切点 $x\in E$ 成立.

\item 4.2 $\lim_{x\to p}f(x)=q$ 当且仅当 $\lim_{n\to\infty} f(p_n)=q$ 对 $E$ 中满足 $p_n\ne p$, $\lim_{n\to\infty} p_n=p$ 的每个序列 $\{p_n\}$ 成立.

\item 4.6 连续性: $\lim_{x\to p}f(x)=f(p)$.

\item 4.7 $h(x) = g(f(x))$. 如果 $f$ 在点 $p$ 连续, 且 $g$ 在点 $f(p)$ 连续, 那么 $h$ 在点 $p$ 连续.

\item 4.8 定理: 度量空间 $X,Y$ 的函数 $f:X\to Y$ 连续, 当且仅当对 $Y$ 的每个开集 $V$, $f^{-1}(V)$ 是 $X$ 中的开集. 这是连续性的一个极有用的特征.

\item 推论: 度量空间 $X,Y$ 的函数 $f:X\to Y$ 连续, 当且仅当对 $Y$ 的每个闭集 $C$, $f^{-1}(C)$ 是 $X$ 中的闭集.

\item 4.9 设 $f,g$ 是度量空间 $X$ 上的复连续函数, 那么 $f+g$, $fg$ 与 $f/g$ 在 $X$ 上连续. 在最后一个中, 假定对一切 $x\in X$, $g(x)\ne 0$.

\item (a) 度量空间上的$\bvec f:X\to R^k$ 连续当且仅当每个分量函数都连续. (b) 如果 $\bvec f, \bvec g:X\to R^k$ 连续, 那么 $\bvec f+\bvec g$ 与 $\bvec f\vdot \bvec g$ 都在 $X$ 上连续.

\item 4.13 有界: $\abs{\bvec f(x)}\leqslant M$.

\item 4.14 若度量空间 $X,Y$ 的函数 $f:X\to Y$ 是连续的, 那么 $f(X)$ 是紧的.

\item 4.15 如果 $\bvec f$ 是把紧度量空间 $X$ 映入 $R^k$ 内的连续映射, 那么 $\bvec f(X)$ 是闭的和有界的. 因此 $\bvec f$ 是有界的.

\item 4.16 如果 $\bvec f$ 是紧度量空间 $X$ 上的连续实函数, 且 $M = \sup_{p\in X} f(p)$, $m=\inf_{p\in X} f(p)$, 那么一定存在 $r,s\in X$ 使 $f(r)=M$ 以及 $f(x)=m$.

\item 4.17 设 $f$ 是把紧度量空间 $X$ 映满度量空间 $Y$ 的连续 1-1 映射, 那么逆映射 $f^{-1}$ 是 $Y$ 映满 $X$ 的连续映射.

\item 4.18 对度量空间 $X,Y$ 的函数 $f:X\to Y$, 称 $f$ 在 $X$ 上一致连续, 若对每个 $\epsilon>0$ 总存在 $\delta >0$ 对一切满足 $d_X(p,q)<\delta$ 的 $p,q\in X$ 都能使 $d_\gamma(f(p),f(q))<\epsilon$.

\item 4.19 设 $f$ 是把紧度量空间 $X$ 映入度量空间 $Y$ 的连续映射. 那么 $f$ 在 $X$ 上一致连续.

\item 4.20 设 $E$ 是 $R^1$ 中的非紧集, 那么 (a) 有在 $E$ 上连续却无界的函数. (b) 有在 $E$ 上连续且有界, 却没有最大值的函数. (c) 如果 $E$ 是有界的, 有在 $E$ 上连续却不一致连续的函数.

\item 4.22 设 $f$ 是把连通的度量空间 $X$ 映入度量空间 $Y$ 内的连续映射, $E$ 是 $X$ 的连通子集, 那么 $f(E)$ 是连通的.

\item 4.25 设 $f$ 定义在 $(a,b)$ 上, 定义一点的\textbf{左极限}和\textbf{右极限}…… 显然极限存在当且仅当左极限和右极限存在且相等. 如果函数在一点不连续, 就说在这点\textbf{间断(discontinuous)}.

\item 4.26 设 $f$ 定义在 $(a,b)$ 上, 如果 $f$ 在一点 $x$ 间断, 并且如果 $f(x+)$ 和 $f(x-)$ 都存在, 就说 $f$ 在 $x$ 发生了\textbf{第一类间断}. 其他间断称为\textbf{第二类间断}.

\item 4.28 设 $f:(a,b)\to R$, 若 $a<x<y<b$ 时有 $f(x)\leqslant f(y)$, 就说 $f$ 在 $(a,b)$ 上\textbf{单调递增}; 若有 $f(x)\geqslant f(y)$ 就是\textbf{单调递减}. 二者统称为\textbf{单调函数}.

\item 4.30 设 $f$ 在 $(a,b)$ 上单调, 那么 $(a,b)$ 中使 $f$ 间断的点最多是可数的.

\item 4.31 间断点不一定是孤立点. 

\item 4.32 对任意 $c\in R$, 集合 ${x|x>c}$ 叫做 $+\infty$ 的一个邻域, 记为 $(c,+\infty)$. 类似地, $(-\infty, c)$ 是 $-\infty$ 的一个邻域.

\item 4.33 把函数的极限用领域的语言拓展到了广义实数系.
\end{itemize}

\subsection{Chap 5. 微分法}
\begin{itemize}
\item 5.1 \textbf{导数(导函数)}: 定义在 $[a,b]$ 上的实值函数, …… $f'(x) = \lim_{t\to x} [f(t)-f(x)]/(t-x)$.

\item 如果 $f'$ 在点 $x$ 有定义, 就说 $f$ 在 $x$ 点\textbf{可微}或\textbf{可导}. 如果在 $E\subset [a,b]$ 的每一点有定义, 就说 $f$ 在 $E$ 上可微.

\item 5.2 …… 若 $f$ 在 $x\in [a,b]$ 可微(可导), 那么它在 $x$ 点连续.

\item 5.5 …… $h'(x) = g'(f(x))f'(x)$.

\item 设 $f$ 是定义在度量空间 $X$ 上的实函数, 说 $f$ 在点 $p\in X$ 取得\textbf{局部极大值}, 如果存在 $\delta>0$, 对任意 $q\in X$ 且 $d(p,q)<\delta$ 有 $f(q)\leqslant f(p)$. 局部极小值的定义类似.

\item 5.8 设 $f$ 定义在 $[a,b]$ 上; $x\in [a,b]$, 如果 $f$ 在点 $x$ 取得局部极大值而且 $f'(x)$ 存在, 那么 $f'(x) = 0$.

\item 
\end{itemize}


\subsection{Chap 6. Riemann-Stieltjes 积分}
\begin{itemize}
\item 6.2 函数 $f$ 关于单调递增函数 $\alpha$ 在 Riemann 意义上可积, 记为 $f\in \mathcal{R}(\alpha)$: $\int_a^b f(x) \dd\alpha(x)$ 或 $\int_a^b f\dd{\alpha}$.

\item 6.6 在 $[a,b]$ 上 $f\in\mathcal{R}(\alpha)$ 当且仅当对任意的 $\epsilon>0$, 存在一个分法 $P$ 使 $U(P,f,\alpha)-L(P,f,\alpha)<\epsilon$.

\item 6.8 如果 $f$ 在 $[a,b]$ 上连续, 那么在 $[a,b]$ 上 $f\in \mathcal{R}(\alpha)$.

\item 6.9 如果 $f$ 在 $[a,b]$ 上单调, $\alpha$ 在 $[a,b]$ 上连续, 那么 $f\in \mathcal{R}(\alpha)$.

\item 6.10 假设 $f$ 在 $[a,b]$ 上有界, 只有有限个间断点. $\alpha$ 在 $f$ 的每个间断点上连续, 那么 $f\in \mathcal{R}(\alpha)$.

\item 6.11 假设在 $[a,b]$ 上 $f\in \mathcal{R}(\alpha)$, $m\leqslant f\leqslant M$. $\phi$ 在 $[m, M]$ 上连续, 并且在 $[a,b]$ 上 $h(x) = \phi(f(x))$. 那么在 $[a,b]$ 上 $h\in \mathcal{R}(\alpha)$.

\item 6.12 (a) 如果在 $f_1,f_2 \in \mathcal{R}(\alpha)$, 那么 $f_1+f_2 \in \mathcal{R}(\alpha)$. 对任意常数 $c$, $cf\in \mathcal{R}(\alpha)$, 并且 $\int_a^b (f_1+f_2)\dd{\alpha} = \int_a^bf_1\dd{\alpha} + \int_a^bf_2\dd{\alpha}$. $\int_a^b cf\dd{\alpha} = c\int_a^b f\dd{\alpha}$.

\item  6.12 (b) 如果在 $[a,b]$ 上 $f_1\leqslant f_2$, 那么 $\int_a^bf_1\dd{\alpha} \leqslant \int_a^bf_2\dd{\alpha}$.

\item  6.12 (c) 如果在 $[a,b]$ 上 $f\in \mathcal{R}(\alpha)$, 并且 $a<c< b$, 那么在 $[a,c]$ 及 $[c,b]$ 上 $f\in \mathcal{R}(\alpha)$, 并且 $\int_a^cf\dd{\alpha}+\int_c^bf\dd{\alpha} = \int_a^bf\dd{\alpha}$

\item 6.12 (d) 如果在 $[a,b]$ 上 $f\in \mathcal{R}(\alpha)$ 并且 $[a,b]$ 上 $\abs{f(x)}\leqslant M$, 那么 $\abs{\int_a^bf\dd{\alpha}} \leqslant M[\alpha(b)-\alpha(a)]$.

\item 6.12 (e) 如果 $f\in \mathcal{R}(\alpha_1)$ 并且 $f\in \mathcal{R}(\alpha_2)$, 那么 $f\in \mathcal{R}(\alpha_1+\alpha_2)$ 并且 $\int_a^bf\dd{(\alpha_1+\alpha_2)} = \int_a^bf\dd{\alpha_1}+ \int_a^bf\dd{\alpha_2}$. 如果 $f\in \mathcal{R}(\alpha)$ 且 $c$ 是正常数, 那么 $f\in \mathcal{R}(c\alpha)$ 而且 $\int_a^bf\dd{(c\alpha)} = c\int_a^bf\dd{\alpha}$

\item 如果在 $[a,b]$ 上 $f,g\in \mathcal{R}(\alpha)$ 那么 (a) $fg\in \mathcal{R}(\alpha)$; (b) $\abs{f}\in \mathcal{R}(\alpha)$ 而且 $\abs{\int_a^bf\dd{\alpha}} \leqslant \int_a^b\abs{f}\dd{\alpha}$

\item 6.17 $\int_a^b f\dd{\alpha} = \int_a^b f(x)\alpha'(x)\dd{x}$.

\item 6.20 设在 $[a,b]$ 上 $f\in \mathcal{R}$, 对于 $a\leqslant x\leqslant b$, 令 $F(x) = \int_a^x f(t)\dd{t}$. 那么 $F$ 在 $[a,b]$ 上连续; 如果 $f$ 又在 $[a,b]$ 的 $x_0$ 点连续, 那么 $F$ 在 $x_0$ 可微, 并且 $F'(x_0) = f(x_0)$.

\item 6.21 微积分基本定理: 如果在 $[a,b]$ 上 $f\in \mathcal{R}$. 在 $[a,b]$ 上又有可微函数 $F$ 满足 $F' = f$, 那么 $\int_a^b f(x)\dd{x} = F(b)-F(a)$.

\item 6.22 分部积分:……
\end{itemize}

\subsection{Chap 7. 函数序列与函数项级数}

\begin{itemize}
\item 7.1 假设 $n=1,2,...$, $\{f_n\}$ 是一个定义在集 $E$ 上的函数序列, 再假设数列 $\{f_n(x)\}$ 对每个 $x\in E$ 收敛. 我们便可以由 $f(x) = \lim_{n\to\infty} f_n(x)$ ($x\in E$) 确定一个函数 $f$. 这时我们说 $\{f_n\}$ 在 $E$ 上收敛. $f$ 是 $\{f_n\}$ 的极限或\textbf{极限函数}.

\item 类似地, 如果对每个 $x\in E$, $\sum f_n(x)$ 收敛, 如果定义 $f(x) = \sum_{n=1}^\infty f_n(x)$ ($x\in E$), 就说函数 $f$ 是级数 $\sum f_n$ 的和.

\item …… 所以, 积分的极限和极限的积分, 即使两者都是有限的, 也未必相等.

\item 7.7 如果对每一个 $\epsilon >0$, 有一个整数 $N$, 使得 $n\geqslant N$ 时, 对一切 $x\in E$, 有 $\abs{f_n(x)-f(x)} \leqslant \epsilon$, 我们就说函数序列在 $E$ 上\textbf{一致收敛}于函数 $f$. 一致收敛必定\textbf{逐点收敛}.

\end{itemize}


\subsection{Chap 8. 一些特殊函数}

\subsection{Chap 9. 多元函数}

\begin{itemize}
\item 9.1 (a) 向量空间 (b) 线性组合; 若 $S \subset R^n$, $E$ 是 $S$ 内元素的所有线性组合的集, 就说 $S$ \textbf{生成} $E$. (c) 线性无关 (d) 维度 (e) 基; 坐标

\item 9.3 基底和线性无关向量的一些定理

\item 9.4 线性变换(线性算子)

\item 9.5 线性算子是 1-1 的当且仅当值域是定义域

\item 9.6 (a) $L(X,Y)$ 代表所有 $X$ 到 $Y$ 的线性变换构成的集. $L(X,X)$ 写成 $L(X)$. 线性变换的线性组合. (b) 线性变换的乘积. (c) 范数 $\norm{A}$ 为所有数 $\abs{A\bvec x}$ 的最小上界, 其中 $x$ 取遍 $R^n$ 中所有 $\abs{\bvec x}\leqslant 1$ 的向量. 对 $x\in R^n$ 有不等式 $\abs{A\bvec x}\leqslant \norm{A}\abs{\bvec x}$

\item 9.7 (a) $A\in L(R^n,R^m)$, 则 $\norm{A}<\infty$ 且 $A$ 是一致连续映射. (b) $\norm{A+B}\leqslant \norm{A}+\norm{B}$, $\norm{cA} = \abs{c}\norm{A}$. 以 $\norm{A-B}$ 作为距离, 那么 $L(R^n,R^m)$ 就是一个度量空间. (c) $\norm{BA} \leqslant \norm{B}\norm{A}$

\item 9.8 设 $\Omega$ 为 $R^n$ 上所有可逆线性算子的集合. (a) 若 $A\in\Omega$, $B\in L(R^n)$, 而且 $\norm{B-A}\cdot\norm{A^{-1}}<1$, 则 $B\in \Omega$. (b) $\Omega$ 是 $L(R^n)$ 的开子集, 映射 $A\to A^{-1}$ 在 $\Omega$ 上是连续的.

\item 9.9 \textbf{矩阵}: $A\bvec x_j = \sum_{i=1}^m a_{ij}\bvec y_i$ ($1\leqslant j\leqslant n$); $\norm{A}\leqslant \qty{\sum_{i,j}a_{ij}^2}^{1/2}$

\item 9.11 设 $E$ 是 $R^n$ 中的开集, $\bvec f: E\to R^m$, $x\in E$. 如果存在把 $R^n$ 映入 $R^m$ 的线性变换 $A$, 使得 $\lim_{\bvec h\to 0} \abs{\bvec f(\bvec x+\bvec h) - \bvec f(\bvec x) - A\bvec h}/{\abs{\bvec h}} = 0$, 就说 $\bvec f$ 在 $\bvec x$ 处\textbf{可微}, 并写成\footnote{$A$ 的矩阵就是雅可比矩阵} $\bvec f'(\bvec x) = A$. (注: $A_{ij} = \pdv*{f_i}{x_j}$)

\item 一元函数的可导和可微等价.

\item 9.13 上面的极限能被写成 $\bvec f(\bvec x+\bvec h) - \bvec f(\bvec x) = \bvec f'(\bvec x)\bvec h + \bvec r(\bvec h)$, 其中余项 $\bvec r(\bvec h)$ 满足 $\lim_{\bvec h\to0} \abs{\bvec r(\bvec h)}/\abs{\bvec h} = 0$.

\item 9.16 \textbf{偏导数}

\item 9.17 若 $\bvec f$ 在点 $x\in E$ 可微, 那么偏导数 $D_jf_i(x)$ 存在, 且 $\bvec f'(\bvec x)\bvec e_j = \sum_{i=1}^m (D_jf_i)(\bvec x)\bvec u_i$ ($1\leqslant j\leqslant n$)

\item 9.18 例: 设 $\gamma$ 是把开区间 $(a,b)\subset R^1$ 映入开集 $E\subset R^n$ 内的可微映射, 即 $\gamma$ 是 $E$ 内的可微曲线. 令 $f$ 为域 $E$ 上的实值可微函数. 于是 $f$ 是从 $E$ 到 $R^1$ 内的可微映射. 定义 $g(t) = f(\gamma(t))$ ($a< t<b$). 于是由链式法则得到 $g'(t) = \bvec f'(\gamma(t))\gamma'(t)$ ($a<t<b$).

\item \textbf{梯度} $(\grad f)(\bvec x) = \sum_{i=1}^n (D_i f)(\bvec x)\bvec e_i$

\item 9.20 设 $\bvec f$ 是开集 $E\subset R^n$ 到 $R^m$ 内的可微映射. 如果 $\bvec f'$ 是把 $E$ 映入 $L(R^n,R^m)$ 的连续映射, 就说 $\bvec f$ 是在 $E$ 内\textbf{连续可微}的. 更明确地, 它要求对每个 $x\in E$ 以及 $\epsilon > 0$, 存在 $\delta >0$, 使当 $y\in E$ 以及 $\abs{\bvec x-\bvec y}<\delta$ 时, $\norm{\bvec f'(\bvec y)-\bvec f'(\bvec x)} < \epsilon$. 我们也说 $\bvec f$ 是 $\mathscr C'$ 映射, 或者 $\bvec f\in \mathscr'(E)$.

\item 9.21 设 $\bvec f$ 把开集 $E\subset R^n$ 映入 $R^m$ 内. 那么当且仅当 $\bvec f$ 的所有偏导数 $D_jf_i$ ($i\leqslant i\leqslant m, 1\leqslant j\leqslant n$)在 $E$ 上都存在并且连续时, $\bvec f\in \mathscr C'(E)$.

\item 9.22 设 $X$ 是度量为 $d$ 的度量空间. 如果 $\varphi:X\to X$, 并且存在 $c<1$, 对一切 $x,y\in X$, 使得 $d(\varphi(x),\varphi(y))\leqslant cd(x,y)$, 那么, 就说 $\varphi$ 是 $X$ 到 $X$ 内的一个\textbf{凝缩函数}.

\item 9.23 如果 $X$ 是完备度量空间, $\varphi$ 是 $X$ 到 $X$ 内的凝缩函数, 那么存在唯一满足 $\varphi(x)=x$ 的 $x\in X$. 也就是说 $\varphi$ 有唯一的不动点.

\item 粗略地说, 反函数定理说的是, 一个连续可微映射 $\bvec f$, 在使线性变换 $\bvec f'(\bvec x)$ 可逆的点 $\bvec x$ 的邻域内是可逆的. (且反函数也是连续可微的)

\item 9.24 设 $\bvec f$ 把开集 $E\subset R^n$ 映入 $R^m$ 内的 $\mathscr C'$ 映射, 对某个 $a\in E$, $\bvec f'(\bvec a)$ 可逆, 且 $\bvec b = \bvec f(\bvec a)$. 那么 (a) 在 $R^n$ 内存在开集 $U$ 和 $V$, 使得 $a\in U, b\in V$ , $\bvec f$ 在 $U$ 上是 1-1 的, 并且 $f(U) = V$; (b) 若 $\bvec g$ 是 $\bvec f$ 的逆, 他在 $V$ 内由 $\bvec g(f(\bvec x)) = x$ ($\bvec x\in U$) 确定, 那么 $\bvec g\in \mathscr C'(V)$.

\item 9.28 \textbf{隐函数定理}: 

\item 9.38 如果 $\bvec f$ 把开集 $E\subset R^n$ 映入 $R^n$ 内, 且 $\bvec f$ 在点 $x\in E$ 可微, $\bvec f'(x)$ 的行列式就叫做 $\bvec f$ 在 $x$ 的\textbf{函数行列式}. 记为 $J_f(\bvec x) = \det f'(\bvec x)$. 如果 $(y_1,\dots,y_n)=\bvec f(x_1,\dots, x_n)$, 我们又把 $J_f(\bvec x)$ 记为 $\pdv*{(y_1,\dots,y_n)}{(x_1,\dots, x_n)}$.
\end{itemize}

\subsection{Chap 10. 微分形式的积分}

\begin{itemize}
\item 10.1 设 $I^k$ 是 $R^k$ 中的 $k$-方格, 它由满足 $a_i\leqslant x_i \leqslant b_i$($i=1,\dots,k$) 的一切 $\bvec x=(x_1,\dots,x_k)$ 组成, $I^j$ 是 $R^j$ 中的 $j$-方格, 它由前 $j$ 个不等式来确定, $f$ 是 $I^k$ 上的连续函数. 令 $f = f_k$, 而用下式定义 $I^{k-1}$ 上的函数 $f_{k-1}$: $f_{k-1}(x_1,\dots,x_{k-1}) = \int_{a_k}^{b_k} f_k(x_1,\dots,x_{k-1},x_k)\dd{x_k}$. $f_k$ 在 $I^k$ 上的一致连续性表明 $f_{k-1}$ 在 $I^{k-1}$ 上连续. 因此, 我们能够重复应用这种手续, 得到 $I^j$ 上的连续函数 $f_j$,…… $k$ 步以后, 就能得到一个数 $f_0$, 我们就把这个数叫做 $f$ 在 $I^k$ 上的积分, 并写成 $\int_{I^k} f(x)\dd{x}$ 或者 $\int_{I^k} f$.

\item 10.2 对每个 $f\in \mathscr C(I^k)$, 积分结果与顺序无关.

\item 10.3 $R^k$ 上一个(实或复)函数 $f$ 的\textbf{支集(support)}, 是使 $f(x)\ne 0$ 的一切点集的闭包. 如果 $f$ 是带有紧支集的连续函数, 令 $I^k$ 是含有 $f$ 的支集的任意 $k$-方格, 并定义 $\int_{R^k}f = \int_{I^k}f$. 这样定义的积分显然与 $I^k$ 的选择无关.

\item 10.4 令 $Q^k$ 是由 $R^k$ 中符合 $x_1+\dots+x_k$ ($i=1,\dots,k$) 的一切点 $\bvec x=(x_1,\dots,x_k)$ 组成的 $k$-单形. 

\item 10.5 \textbf{本源映射}: $\bvec G(\bvec x) = \bvec x + [g(\bvec x)-x_m]\bvec e_m$.

\item 10.6 在 $R^n$ 上, 只把标准基的某一对成员交换, 而其他成员不变的线性算子 $B$ 叫做\textbf{对换}. $B$ 也可以看成是交换两个坐标, 而基不变.

\item 10.10 设 $E$ 是 $R^n$ 中的开集. $E$ 中的 \textbf{$k$-曲面($k$-surface)}是从紧集 $D\subset R^k$ 到 $E$ 内的 $\mathscr C'$ 映射 $\Phi$. $D$ 叫做 $\Phi$ 的\textbf{参数域(parameter domain)}. $D$ 中的点记为 $\bvec u = (u_1,\dots, u_k)$.

\item 10.11 设 $E$ 是 $R^n$ 中的开集. $E$ 中的 $k\geqslant 1$ 次\textbf{微分形式}(简称为 $E$ 中的 $k$-\textbf{形式})是一个用 $\omega = \sum a_{i_1\dots i_k}(\bvec x)\dd{x_{i_1}} \wedge \dots \wedge \dd{x_{i_k}}$ (指标 $i_1,\dots,i_k$ 各自从 $1$ 到 $n$ 独立变化) 作符号表示的函数, 它给 $E$ 中的每个 $k$-曲面 $\Phi$, 按照规则……

\item 10.26 对向量空间 $X,Y$, 令 $f:X\to Y$. 如果 $\bvec f-\bvec f(\bvec 0)$ 是线性的, 就称 $\bvec f$ 为\textbf{仿射的(affine)}. 也就是说要求存在某个 $A\in L(X,Y)$ 使得 $\bvec f(\bvec x) = \bvec f(0)+A\bvec x$.

\item 定义\textbf{标准单形(standard simplex)} $Q^k$ 为由形如 $\bvec u = \sum_{i=1}^k \alpha_k\bvec e_i$ 使 $\alpha_i\geqslant 0$($i=1,\dots,k$)并且 $\sum \alpha_i\leqslant 1$ 的一切 $\bvec u\in R^k$ 组成的集.

\item \textbf{有向仿射 $k$-单形(oriented affine $k$-simplex)} $\sigma = [\bvec p_0,\bvec p_1,\dots,\bvec p_k]$ 是用 $Q^k$ 作参数域, 由仿射映射 $\sigma(\alpha_1\bvec e_1+\dots+\alpha_k\bvec e_k) = \bvec p_0 + \sum_{i=1}^k \alpha_i(\bvec p_i-\bvec p_0)$. 给出的 $R^n$ 中的 $k$-曲面. 注意……
\end{itemize}


\subsection{Chap 11. Lebesgue 理论}
