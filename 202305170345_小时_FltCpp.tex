% 双精度和四精度浮点数笔记(C++)

\begin{issues}
\issueDraft
\end{issues}

另见 “双精度和变精度浮点数测试(Matlab)\upref{FltMat}”。

\subsubsection{double}
\begin{itemize}
\item 根据 IEEE 754\footnote{参考 Wikipedia \href{https://en.wikipedia.org/wiki/IEEE_754}{相关页面}。} 标准, 双精度的格式是 1 bit 符号 + 11 bit 指数 + 52 bit 小数(mantissa)。
\item 符号位也用于区分 \verb|0| (普通的 \verb|0| 就是 \verb|+0|)和 \verb|-0|(或者 \verb|-1*0|, 例如 \verb|1/0, 1/(-0)| 分别得到 \verb|inf, -inf|, 反之亦然。 另外一些函数也对 \verb|-0| 敏感, 例如 \verb|atan2|。
\item \verb|0 == -0|, \verb|inf == inf|, \verb|-inf == -inf|, \verb|inf * 2 == inf|, \verb|inf * 0 == nan|。
\item 无论哪种 \verb|nan| 和自身都不相等, 可以用这种方法来判断一个浮点数是否是 \verb|nan|。 也可以用 \verb|isnan()|。
\end{itemize}

C++ 浮点相关函数:
\begin{itemize}
\item \verb|bool signbit(x)| 获取符号位(对于 \verb|-0| 来说, 这不同于 \verb|(x < 0)|)
\item \verb|ldexp(x, n)| 计算 $x \times 2^n$; 
\item \verb|double man = frexp(x, &exp2)| 可以拆分 mantissa 和 exp。 其中 \verb|0.5 <= man < 1|; 
\item \verb|int fpclassify(x)| 可以用来给浮点数归类(返回 \verb|FP_NORMAL, FP_SUBNORMAL, FP_ZERO, FP_INFINITE, FP_NAN| 之一)。 
\item \verb|bool isfinite(x)| 检查是否是 \verb|FP_NORMAL, FP_SUBNORMAL, FP_ZERO| 之一。
\item \verb|bool isnormal(x)| 检查 \verb|fpclassify| 是否返回 \verb|FP_NORMAL|。 
\end{itemize}


\subsubsection{ULP}
\textbf{unit in the last place (ULP)} 就是 mantissa 增加 1 bit 时表示的数增加多少。 这等于 $2^{-52}\times 2^{e}$。 $e$ 是双精度的指数(以 2 为底)。 其中 $2^{-52}$ 一般记为 \verb|eps|。

所以双精度的 mantissa 表示 1 时, 相对精度就是 \verb|eps|, 表示 \verb|1-eps| 时, 相对精度就非常接近 \verb|eps/2|(\textbf{最小相对误差})。 所以一般来说相对精度在二者之间。 但一个例外就是当绝对值非常小时(小于 \verb|1e-1022| 也就是 $2.225\e{-308}$), 也就是 \textbf{subnormal}, mantissa 的第一 bit 可能并不是默认的 1, 此时相对误差可能会变得很大。

\subsubsection{需要多少位十进制数}
需要多少位十进制数可以通过四舍五入精确表示任意的双精度数呢? $N$ 位有效数字的十进制数的相对误差介于 $10^{-N}$ (前几位是 999... 时)和 $10^{-N+1}$ (前几位是 1000... 时)之间。 所以为了满足 \verb|eps/2|($1.11\e{-16}$) 的相对误差, \textbf{最多需要 17 位有效数字}。

那么一个 double 能不能精确地用十进制表示呢? 可以的, 因为 \verb|eps| 可以, 但可能需要非常多的位数尤其是 2 的指数非常大或者非常小的时候可能会需要几百位, 没有太大必要。

\subsubsection{long double}
在 x86 机器上, \verb|long double| 一般是 1 bit 符号 + 15 bit 指数 + 64 bit 小数 = 80 bit, 十进制大约是 18-19 位有效数字\footnote{Visual Studio 中据说 long double 和 double 是一样的。}。 虽然 \verb|sizeof| 是 16 字节, 但实际上只使用了 10 字节\footnote{详见\href{https://en.wikipedia.org/wiki/Extended_precision}{这里} 的 x86 extended precision format 一节}。 \verb|long double| 的运算是硬件支持的, 所以比较快。

\begin{lstlisting}[language=none]
sizeof(Doub) = 8
sizeof(Ldoub) = 16

std::numeric_limits<Doub>::max() = 1.797693135e+308
std::numeric_limits<Ldoub>::max() = 1.189731495e+4932

std::numeric_limits<Doub>::min() = 2.225073859e-308
std::numeric_limits<Ldoub>::min() = 3.362103143e-4932

std::numeric_limits<Doub>::denorm_min() = 4.940656458e-324
std::numeric_limits<Ldoub>::denorm_min() = 3.645199532e-4951

std::numeric_limits<Doub>::epsilon() = 2.220446049e-16
std::numeric_limits<Ldoub>::epsilon() = 1.084202172e-19 // 2^(-63)

std::numeric_limits<Doub>::digits10 = 15
std::numeric_limits<Ldoub>::digits10 = 18
\end{lstlisting}

然而 \verb|sqrt| 函数似乎并不会提高精度(仍然是 16 位):
\begin{lstlisting}[language=none]
sqrt(2.)       = 1.4142135623730951454
sqrt(Ldoub(2)) = 1.4142135623730951454
precise val    = 1.4142135623730950488...
\end{lstlisting}

另外测试一下双精度的二进制表示(\autoref{fig_CmArit_3}~\upref{CmArit}), 例如双精度的 \verb|10| 表示为
\begin{lstlisting}[language=none]
01000000 00100100 (后面省略 48 个 0)
\end{lstlisting}
第一位 0 说明是正数, 1 是负数。 接下来 11 bit 先看成 unsigned, 然后再减 $2^{10}-1 = 1023$, 这里指数 \verb|10000000010| 表示 $1026-1023 = 3$, 于是指数项为 $2^{3} = 8$。 所以要表示 10, 小数项应该是 $10/8 = 1+1/4$。 第 13 bit 表示 1/2, 14 bit 表示 1/4, 所以第 14 bit 是 1, 后面的小数全是零。 另外, 由于 intel 的 x86 和 x86-64\upref{x86} 都使用 little endian, 所以在内存中实际上 byte 的顺序是相反的。

理论上指数部分可以表示 $-1023$ 到 $1024$ 之间的数, 但首尾两个数有特殊用途, 所以只能表示 $-1022$ 到 $1023$。 当指数为 $-1023$ 时, 仍然相当于指数为 $-1022$ 但小数部分前面不加 1(如果小数也都是零, 那就表示 0, 所以 \verb|0| 除了第一 bit 其他都必须是 0)。 当指数为 $1024$ 时, 表示 \verb|inf|(要求小数部分都是零, 正负号仍然有效)或者 \verb|nan| (\verb|0b1...| 表示安静的 \verb|qNaN|, \verb|0b0..1..| 表示会发错误信号的 \verb|sNaN|)。

双精度类型可以表示 \verb|-2e53| 到 \verb|2e53| 的每一个整数, 更小或更大的整数有的也可以表示(事实上此时只可以表示整数), 只是不连续。

\begin{lstlisting}[language=none]
std::numeric_limits<double>::max() = 1.797693135e+308
01111111 11101111 11111111 11111111 11111111 11111111 11111111 11111111
std::numeric_limits<double>::min() = 2.225073859e-308 = 2^-1022
00000000 00010000 00000000 00000000 00000000 00000000 00000000 00000000
std::numeric_limits<double>::epsilon() = 2.22e-16 = 2^-52
00111100 10110000 00000000 00000000 00000000 00000000 00000000 00000000
std::numeric_limits<double>::denorm_min() = 4.940656458e-324 = 2^-1074
00000000 00000000 00000000 00000000 00000000 00000000 00000000 00000001
std::numeric_limits<double>::denorm_min()*11
00000000 00000000 00000000 00000000 00000000 00000000 00000000 00001011
double 1/0.:
01111111 11110000 00000000 00000000 00000000 00000000 00000000 00000000
double max()*2 (正无穷, 同 1/0.):
01111111 11110000 00000000 00000000 00000000 00000000 00000000 00000000
double -max()*2 (负无穷):
11111111 11110000 00000000 00000000 00000000 00000000 00000000 00000000
double sqrt(-1.) (NaN):
11111111 11111000 00000000 00000000 00000000 00000000 00000000 00000000
\end{lstlisting}
其中 \verb|denorm_min()| 比较特殊, 当指数 bit 全为 0 的时候, 指数为 \verb|-1022| 而不是 \verb|-1023|, 且小数部分前面不加 1。 这叫做 \verb|denominize|。

\subsection{四精度}
\footnote{参考 Wikipedia \href{https://en.wikipedia.org/wiki/Quadruple-precision_floating-point_format}{相关页面}。}使用四精度比使用任意精度类型(如 MPFR 或 Flint 库)要快得多, 因为任意精度浮点数是动态分配内存的。

如果想完全利用 16 字节, 见 GCC/g++ 编译器的 \verb|__float128| 类型(\href{https://gcc.gnu.org/onlinedocs/gcc-8.1.0/libquadmath.pdf}{文档})。 相当于 FORTRAN 的 \verb|QUAD real|。 有 112 bit 小数, 15 bit 指数和 1 bit 正负号(\autoref{fig_FltCpp_1} )。 实测的 bit 表示和上面二进制的同理(也存在两个奇怪的地方)。

\begin{figure}[ht]
\centering
\includegraphics[width=10cm]{./figures/35dada58827a8506.png}
\caption{见 Wikipedia} \label{fig_FltCpp_1}
\end{figure}


加上头文件 \verb|#include <quadmath.h>|, 使用 \verb|g++| 编译, 编译时加上选项 \verb|-lquadmath| 和 \verb|-fext-numeric-literals| 即可。

\verb|__complex128| 是对应的四精度复数类型, 相当于两个 \verb|__float128|。

\begin{lstlisting}[language=none]
FLT128_MAX = 1.1897e+4932
FLT128_MIN = 3.3621e-4932
FLT128_EPSILON = 1.9259e-34 // 2e-112
FLT128_DENORM_MIN = 6.4752e-4966
FLT128_MANT_DIG = 113 // 包括正负位
FLT128_MAX_EXP = 16384 // 2e14
FLT128_DIG = 33 // 33-34 位十进制有效数字
M_PIq = 3.141592653589793238462643383279503 // 精确到最后一位
sqrtq(3) = 1.732050807568877293527446341505872 // 精确到最后一位
\end{lstlisting}

给 \verb|__complex128| 的实部和虚部分别赋值:
\begin{lstlisting}[language=cpp]
__complex128 y;
__real__ y = 实部;
__imag__ y = 虚部;
\end{lstlisting}
看来还是用 \verb|std::complex<__real128>| 比较好。

另见 “C++ 高精度计算资源列表\upref{CppMp}”。
