% 数列(高中)
% keys 高中|数列的概念|数列的函数特性
% license Usr
% type Tutor

\begin{issues}
\issueDraft
\end{issues}
\pentry{函数\nref{nod_functi}}{nod_7191}

在之前的学习中,已经介绍过函数的概念。函数是一种描述变量之间关系的数学工具,高中阶段接触到的函数,其定义域通常是连续的,比如实数集 $\mathbb{R}$ 或区间,因此函数的图像往往是一条连续的曲线。然而,数学世界中并非所有的关系都具有连续性。一些仅定义在自然数集合 ${1, 2, 3, \dots}$ 或其子集上的特殊函数,被称为数列。最初,数列是人们将观测到的对象按顺序排列而形成的一种表示方式,最初的研究者甚至没有意识到其与函数的联系。

尽管数列作为独立的数学领域有着丰富的研究内容和独特的性质,但从函数的视角观察数列会发现它实际上继承了函数的许多特点。在学习数列时,既要关注它作为离散排列的独特性,也要结合函数的视角来探索它的规律。与函数类似,数列通过离散点的形式揭示了事物的内在结构和发展规律,是进一步理解数学体系的重要工具。


\subsection{定义与相关概念}

一般地,按一定次序排列的一列数叫做\textbf{数列(sequence)},数列中的每一个数叫作这个数列的\textbf{项}。数列的一般形式可以写成
\begin{equation}
a_1,a_2,a_3,\cdots,a_n,\cdots~
\end{equation}
简记为数列 $\begin{Bmatrix} a_n \end{Bmatrix}$,其中数列的第1项 $a_1$,也称\textbf{首项};$a_n$ 是数列的第 $n$ 项,也叫数列的\textbf{通项}。

项数有限的数列,称为\textbf{有穷数列};项数无限的数列,称为\textbf{无穷数列}。

如果数列 $\begin{Bmatrix} a_n \end{Bmatrix}$ 的第 $n$ 项 $a_n$ 与 $n$ 之间的函数关系可以用一个式子表示成 $a_n = f(n)$,那么这个式子就叫作这个数列的\textbf{通项公式},数列的通项公式就是相应函数的解析式。

\textsl{注意:不是所有数列都能写出通项公式。}

\subsection{函数特性}
一般地,一个数列 $\begin{Bmatrix} a_n \end{Bmatrix}$,如果从第2项起,每一项都大于前一项,即 $a_{n+1}>a_n$,那么这个数列叫作\textbf{递增数列}。

如果从第2项起,每一项都小于前一项,即 $a_{n+1}<a_n$,那么这个数列叫作\textbf{递减数列}。

如果数列 $\begin{Bmatrix} a_n \end{Bmatrix}$ 的各项都相等,那么这个数列叫作\textbf{常数列}。

如果数列 $\begin{Bmatrix} a_n \end{Bmatrix}$ 从第2项起,有些项大于它的前一项,有些项小于它的前一项,这样的数列叫\textbf{摆动数列}。
