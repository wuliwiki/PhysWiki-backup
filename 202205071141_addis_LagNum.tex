% 拉格朗日方程的数值解

\begin{issues}
\issueDraft
\end{issues}

\pentry{哈密顿正则方程\upref{HamCan}, 偏导与差分\upref{ParDf}}

若程序中给出拉格朗日量拉格朗日方程\upref{Lagrng}的数值函数 $L(q, \dot q, t)$, 输入和输出均为数值(例如双精度实数), 那么如何数值求解运动方程呢?

问题的关键是如何列出一个一阶常微分方程组. 如果可以给出系统哈密顿量的数值函数, 那么数值解哈密顿正则方程是首选的做法, 因为它本身已经是一阶常微分方程组, 可以直接用解算器进行数值解. 但对于较复杂的系统, 哈密顿量比拉格朗日量难算得多, 甚至可能没有解析表达式.

相比之下, 拉格朗日量虽然容易写出, 但方程的数值解比哈密顿正则方程要难一些, 且如果用差分法计算微分会引入一定的数值误差.

由拉格朗日方程(\autoref{LagEqQ_eq1}~\upref{LagEqQ})
\begin{equation}
\dv{t} \pdv{L}{\dot q_i} = \pdv{L}{q_i} + Q_i
\quad (i=1,\dots,N)
\end{equation}
根据全微分\upref{TDiff}, 左边有
\begin{equation}
\dv{t} \pdv{L}{\dot q_i} = \sum_j\pdv{\dot q_j}\pdv{L}{\dot q_i}\ddot q_j + \sum_j\pdv{q_j}\pdv{L}{\dot q_i}\dot q_j + \pdv{t}\pdv{L}{\dot q_i} \quad (i,j=1,\dots,N)
\end{equation}
代入得
\begin{equation}
\sum_j\pdv{\dot q_j}\pdv{L}{\dot q_i}\ddot q_j = \pdv{L}{q_i} - \sum_j\pdv{q_j}\pdv{L}{\dot q_i}\dot q_j - \pdv{t}\pdv{L}{\dot q_i} + Q_i
\end{equation}
这样, 二阶导数只存在于左边的 $\ddot q_j$, 其他项都只是 $q,\dot q, t$ 的函数. 该式中的偏微最好计算出解析表达式, 但也可以通过差分来计算\upref{ParDf}(会引入更多误差). 数值解线性方程\footnote{如果 $N$ 不大可以直接用克拉默法则\upref{kramer}, 甚至写出解析解.}, 就可以得到(令 $v_i = \dot q_i$)
\begin{equation}
\leftgroup{
&\dot v_i = f_i(q, v_i, t)\\
&\dot q_i = v_i
}\qquad (i = 1,\dots,N)
\end{equation}
就得到了 $2N$ 条方程组成的一阶常微分方程组, 变量一共有 $2N$ 个. 可以使用四阶龙格库塔法\upref{OdeRK4} 等方法求解.

\addTODO{代码未完成.}
\begin{lstlisting}[language=matlab, caption=lagr\_solve.m]
% 数值解拉格朗日方程
% L(qq, t), qq = [q, dq]
% dL(i,qq,t), d2L(i,qq,t), dLdL(i,j,qq,t) 可选
function lagr_solve(L, qq0, RelTol, dL, d2L, dLdL)
h = 1e-6;
if isempty(d2L)
    d2L = @(i,qq,t)d2L_dif(L,i,qq,t,h);
end
if isempty(dLdL)
    dLdL = @(i,j,qq,t)dLdL_dif(L,i,j,qq,t,h);
end
end

% D^2 L / D(qq_i)^2
function ret = d2L_dif(L, i, qq, t, h)
L2 = L(qq,t);
qq(i) = qq(i) + h;
L3 = L(qq,t);
qq(i) = qq(i) - 2*h;
L1 = L(qq,t);
ret = (L1 + L3 - 2*L2)/(2*h)^2;
end

% D^2 L / D(qq_i) D(qq_j)
function ret = dLdL_dif(L, i, j, qq, t, h)
qq(i) = qq(i) - h; qq(j) = qq(j) - h;
L1 = L(qq,t);
qq(i) = qq(i) + 2*h;
L2 = L(qq,t);
qq(j) = qq(j) + 2*h;
L3 = L(qq,t);
qq(i) = qq(i) - 2*h;
L4 = L(qq,t);
ret = (L1 - L2 + L3 - L4)/(2*h)^2;
end
\end{lstlisting}
