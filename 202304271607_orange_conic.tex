% 圆锥曲线(总结)

\begin{issues}
\issueDraft
\end{issues}
%感觉不够循序渐进,会不会不大适合初学者?

\subsection{圆锥曲线的定义}
\footnote{本文参考自Wikipedia的Conic Section、圆锥曲线词条。本文适用于CC-BY-SA。}\textbf{圆锥曲线(conic section)}的一般定义:曲线上任意一点至一定点(焦点)的距离,与它至一直线(准线)的垂直距离始终成固定比例(离心率):
\begin{equation}
\left \{P \vert \frac{\abs{PF}}{\abs{PL}}=e \right \}~.
\end{equation}
其中$P$为曲线上一点,$F$为定点,$L$为定直线,$e$为该固定比例,$\abs{PL}$表示点到直线的垂直距离。

根据$e$的取值,可将圆锥曲线分为三类。相应的图例请参考下文。
\begin{table}[ht]
\centering
\caption{圆锥曲线的分类}\label{tab_conic_2}
\begin{tabular}{|c|c|}
\hline
离心率 & 名称\\
\hline
$0<e<1$ & 椭圆\\
\hline
$e=1$ & 抛物线\\
\hline
$e>1$ & 双曲线\\
\hline
\end{tabular}
\end{table}

\subsection{圆锥曲线方程}
本节使用的术语请参考下文。
\begin{figure}[ht]
\centering
\includegraphics[width=8cm]{./figures/e1e09183f1ed2a51.pdf}
\caption{圆锥曲线方程} \label{fig_conic_1}
\end{figure}

以极坐标表示圆锥曲线时,原点为圆锥曲线的(一个)焦点。其中 $e$ 是离心率, $l$ 是半通径,极角 $\theta$ 的取值范围是所有使 $r>0$ 的值。 
具体的推导与说明请参考圆锥曲线的极坐标方程\upref{Cone}。

以直角坐标表示圆锥曲线时,原点为该图形的几何中心(对于抛物线,则是他的顶点),这与极坐标是不同的。有时,相对于抽象的极坐标方程,直角坐标表示的圆锥曲线更为直观。

\subsection{圆锥曲线常用术语}

\begin{figure}[ht]
\centering
\includegraphics[width=10cm]{./figures/7fce973eb866b9f5.pdf}
\caption{椭圆} \label{fig_conic_2}
\end{figure}

\begin{figure}[ht]
\centering
\includegraphics[width=10cm]{./figures/3684fb1030e06835.pdf}
\caption{抛物线} \label{fig_conic_3}
\end{figure}

\begin{figure}[ht]
\centering
\includegraphics[width=10cm]{./figures/4b2cec14439c36b1.pdf}
\caption{双曲线} \label{fig_conic_4}
\end{figure}

一些术语:
\begin{itemize}
\item 焦距:两焦点的距离
\item 焦准距:焦点至与之对应的准线的距离
\item 通径:过焦点做准线的平行线交曲线于两点,这两点所确定的直线段(的长度)
\item 离心率:半焦距与半长轴之比;或如定义所述,曲线上一点至焦点与准线的距离之比。
\end{itemize}

\begin{table}[ht]
\centering
\caption{圆锥曲线术语及定义}\label{tab_conic_1}
\begin{tabular}{|c|c|c|c|c|c|c|c|c|}
\hline
名称 & 直角坐标方程 & 半焦距 Linear Eccentricity $c$ & 离心率 Eccentricity $e = \frac{c}{a}$ & 半通径 Semi Latus Rectum $l=\frac{b^2}{a}$ & 焦准距 Focal Parameter$p=\frac{b^2}{c}$ & 焦点坐标&准线方程 &备注\\
\hline
% (圆)Circle & $x^2+y^2=a^2$ & 0 & 0 & $a$ & \ & 一般不认为是圆锥曲线;在思考“离心率、半焦距时”可以不严谨地认为圆是两焦点重合于圆心、准线在无穷远处的椭圆\\
% \hline
椭圆 Ellipse & $\frac{x^2}{a^2} + \frac{y^2}{b^2} = 1$ & $\sqrt{a^2-b^2}$, $b^2+c^2=a^2$ & $\sqrt{1-\frac{b^2}{a^2}} < 1$ & $\frac{b^2}{a}$ & $\frac{b^2}{\sqrt{a^2-b^2}}$ & $(\pm c,0)$ & $x=\pm a^2/c$& \\
\hline
抛物线 Parabola & $y^2=4ax$ & \ & 1 & $2a$ & $2a$ & $(a,0)$ & $x=-a$ & 只有一条准线和一个焦点\\
\hline
双曲线 Hyperbola & $\frac{x^2}{a^2} - \frac{y^2}{b^2} = 1$ & $\sqrt{a^2+b^2}$,$a^2+b^2=c^2$ & $\sqrt{1+\frac{b^2}{a^2}}$ > 1 & $\frac{b^2}{a}$ & $\frac{b^2}{\sqrt{a^2+b^2}}$ & $(\pm c,0)$ & $x=\pm a^2/c$ & 分为互不相连的两支 \\
\hline
\end{tabular}
\end{table}

常用恒等式:
\begin{itemize}
\item $c=ae$
\item $l=pe$
\end{itemize}
