% 经典力学(综述)
% license CCBYSA3
% type Wiki

本文根据 CC-BY-SA 协议转载翻译自维基百科\href{https://en.wikipedia.org/wiki/Classical_mechanics}{相关文章}。

\begin{figure}[ht]
\centering
\includegraphics[width=8cm]{./figures/c0ac5b53deabd6dc.png}
\caption{显示卫星绕地球轨道运动的示意图,其中展示了垂直的速度和加速度(力)矢量,通过经典力学的解释进行表示。} \label{fig_JDLX_1}
\end{figure}
经典力学是一种物理理论,用于描述物体的运动,例如抛射物、机械部件、航天器、行星、恒星和星系等。经典力学的发展涉及物理学方法和哲学的重大变革。[1] “经典”这一限定词将这种力学类型与20世纪初物理学革命之后发展起来的物理学区分开来,这些现代物理理论揭示了经典力学的局限性。[2]

经典力学最早的形式通常被称为牛顿力学。它基于17世纪以艾萨克·牛顿爵士为代表的奠基性工作的物理概念,以及牛顿、戈特弗里德·威廉·莱布尼茨、莱昂哈德·欧拉等人发明的数学方法,用来描述物体在力的作用下的运动。后来,基于能量的方法由欧拉、约瑟夫-路易·拉格朗日、威廉·罗恩·哈密顿等人发展,最终形成了分析力学(包括拉格朗日力学和哈密顿力学)。这些进步主要发生在18世纪和19世纪,超越了早期的工作;经过一定的修正,它们被应用于现代物理学的各个领域。

如果一个遵循经典力学定律的物体的当前状态已知,就可以确定它将来的运动方式以及过去的运动轨迹。然而,混沌理论表明,经典力学对长期预测并不可靠。在研究非极端质量且速度未接近光速的物体时,经典力学能够提供精确的结果。当研究的对象接近原子直径大小时,就需要使用量子力学;而描述接近光速的速度时,则需要使用狭义相对论。当物体质量极大时,广义相对论则变得适用。一些现代文献将相对论力学归入经典物理学范畴,认为它是该领域最成熟和精确的形式。
\subsection{分支}  
\subsubsection{传统划分}  
经典力学传统上分为三个主要分支。\textbf{静力学}是经典力学的一个分支,研究作用在不加速的物理系统上的力和力矩的分析,这些系统与其环境处于平衡状态。[3] \textbf{运动学}描述点、物体以及物体系统(物体群)的运动,而不考虑引起这些运动的力。[4][5][3] 作为一个研究领域,运动学常被称为“运动的几何学”,有时甚至被视为数学的一个分支。[6][7][8] \textbf{动力学}不仅仅描述物体的行为,还考虑解释这些行为的力。一些作者(例如Taylor (2005)[9] 和 Greenwood (1997)[10])将狭义相对论包含在经典动力学中。
\subsubsection{力与能量}  
另一种划分方法基于数学形式主义的选择。经典力学可以用多种不同的数学方式来表述。这些不同形式的物理内容是相同的,但它们提供了不同的洞察力,并有助于进行不同类型的计算。虽然“牛顿力学”一词有时被用作非相对论经典物理的同义词,但它也可以特指一种基于牛顿运动定律的特定形式主义。在这种意义上,牛顿力学强调力作为矢量量的重要性。[11]  

与之相对的是,\textbf{解析力学}利用描述系统整体运动的标量属性——通常是动能和势能。运动方程通过某种关于标量变化的基本原理从标量量推导出来。解析力学的两个主要分支是:  \textbf{拉格朗日力学},它使用广义坐标和对应的广义速度,在配置空间中进行分析;\textbf{哈密顿力学},它使用坐标和对应的广义动量,在相空间中进行分析。  

这两种表述通过对广义坐标、速度和动量进行\textbf{勒让德变换}是等价的;因此,两者包含描述系统动力学所需的相同信息。此外,还有其他表述方法,例如\textbf{哈密顿-雅可比理论}、\textbf{鲁斯力学}和\textbf{阿佩尔运动方程}。

所有描述粒子和场运动的方程,无论采用何种形式主义,都可以从一种广泛适用的结果——\textbf{最小作用量原理}推导出来。一个重要的结果是\textbf{诺特定理},它将守恒定律与它们所关联的对称性联系起来。
按应用领域划分  
另一种划分方式是根据应用领域:  

- **天体力学**:涉及恒星、行星和其他天体的运动。  
- **连续介质力学**:用于描述作为连续体的材料,例如固体和流体(包括液体和气体)。  
- **相对论力学**:包括狭义和广义相对论,用于研究速度接近光速的物体的运动。  
- **统计力学**:为将单个原子和分子的微观性质与材料的宏观或整体热力学性质联系起来提供了理论框架。