% 华东师范大学 2015 年硕士研究生物理考试试题
% keys 华东师范大学|物理|考研|2015年
% license Copy
% type Tutor

\textbf{声明}:“该内容来源于网络公开资料,不保证真实性,如有侵权请联系管理员”


普适气体常量  $R=8.31 J/(mol.k)$ ,玻尔兹曼常量 $k=1.38*10^{-23}$  , 电子质量  $m_4=9.11*10^{-31}$, 真空介电常量 $\varepsilon_0=8.85*10^{-12}C^2.N^{-1}$ , 普朗克常量 $h=6.63*10^{-34}$
\begin{enumerate}
\item 如图所示,一圆盘绕通过其中心且垂直于盘面的转轴,以角速度 $\omega$ 作定轴转动,$A,B,C$  三点与中心的距离均为$r$。则图示 $A,B$  点速度差 $\bar v_{AB}=\bar v_A-\bar v_B$ 与 $A,C$ 点的速度差 $\bar v_{AC}=\bar v_A-\bar v_C$  的关系为$(\qquad)$\\
(A)$\bar v_{AB}>\bar v_{AC}$\\
(B)$\bar v_{AB}<\bar v_{AC}$\\
(C)$\bar v_{AB}=\bar v_{AC}$\\
(D)$\abs{\bar v_{AB}}=\abs{\bar v_{AC}}$
\begin{figure}[ht]
\centering
\includegraphics[width=6cm]{./figures/22a12d6c47f71e14.png}
\caption{} \label{fig_HSD15_1}
\end{figure}
\item 题1中,如果圆盘的质量为$M$,在$A$点处有一个质量为$m$的人。刚开始人和圆盘相对于地面均处于静止状态。当人沿着圆盘走一圈时,圆盘相对于地面转过的角度为$(\qquad)$\\
(A)$\displaystyle \frac{4\pi m}{2m+M}$\\
(B)$\displaystyle \frac{2\pi m}{2m+M}$\\
(C)$\displaystyle \frac{2\pi m}{m+M}$\\
(D)$0$
\item 令电子的速率为$v$,则电子的动能 $E_x$ 对于比值v/c的图线可用下列图中哪一个图表示?(c表示真空中光速)\\
\begin{figure}[ht]
\centering
\includegraphics[width=12cm]{./figures/327338a7924728b9.png}
\caption{} \label{fig_HSD15_2}
\end{figure}
\item 一质点沿螺旋线状的曲线自外向内运动,如图所示、已知其走过的弧长与时间的一次方成正比。则该运动为$(\qquad)$\\
(A)加速度值越来越小的匀速率曲线运动\\
(B)加速度值越来越大的匀速率曲线运动\\
(C)加速度值越来越小的变速率曲线运动\\
(D)加速度值越来越大的变速率曲线运动
\begin{figure}[ht]
\centering
\includegraphics[width=6cm]{./figures/29c8f8a1e0dc637f.png}
\caption{} \label{fig_HSD15_5}
\end{figure}
\item 质量分别为 $m_A$ 和$m_B (m_A>m_B)$、速度分别为和的两质点$A$和$B$,受到相同的冲量作用,则$(\qquad)$\\
(A)$A$的动量增量的绝对值比$B$ 的小\\
(B)$A$的动量增量的绝对值比 $B$的大\\
(C)$A,B$的动盘增量相等\\
(D)$A,B$的速度增量相等
\item 老师和学生各带一个钟。他们在长沙站对好钟(显示是同一时刻),学生就坐上火车往武汉去。如果考虑到相对论效应,当学生到达武汉后,他的钟与老师的钟相比,哪一个会变慢?设同一参考系的钟都是同步的。
\item 如图所示,一圆盘形工件K套装在一根可绕自身轴线转动的固定轴$A$上,圆盘K的中心线与固定轴A的中轴线互相重合,圆盘的内外直径分别为$D$和$D_1$,该工件在外力矩作用下获得角速度$\omega_0$这时撤掉外力矩,工件在轴所受的阻力矩作用下最后停止转动,其间经过了时间t,则轴所受的平均阻力为$(\qquad)$,这里圆盘工件绕其中心轴转动的转动惯量为$m(D^2+D^2_1)/8$,$m$为圆盘的质量,轴的转动惯量忽略不计。
\begin{figure}[ht]
\centering
\includegraphics[width=6cm]{./figures/d131a424c9c72b61.png}
\caption{} \label{fig_HSD15_4}
\end{figure}
\item 一质点在二恒力共同作用下,位移为$\Delta F=3\bar i +8\bar j \quad$ (SI):在此过程中,动能增量为$24J$,已知其中一恒力$\bar F_1$=$12 \bar i-3 \bar j$(SI),则另一恒力所作的功为$(\qquad)$
\item 如图所示,$x$轴沿水平方向,$y$轴竖直向下,$z$轴垂直于纸面向内,在$t=0$时刻将质量为$m$的质点以初速度$\bar v_0$ 从参考点平抛出去,则在任意时刻$t$ 质点所受的对原点$O$的力矩为$(\qquad)$。
\begin{figure}[ht]
\centering
\includegraphics[width=6cm]{./figures/b81e3d7d8dd13817.png}
\caption{} \label{fig_HSD15_3}
\end{figure}
\item 质心参考系的特点是$(\qquad)$。
\item 质量一定的理想气体,从相同状态出发,分别经历等温过程,等压过程和绝热过程,使其体积增加一倍,那么气体温度的改变(绝对值) $(\qquad)$\\
(A)绝热过程中最大,等压过程中最小\\
(B)绝热过程中最大,等温过程中最小\\
(C)等压过程中最大,绝热过程中最小\\
(D)等压过程中最大,等温过程中最小
\item 一物质系统从外界吸收一定的热量,则$(\qquad)$。
(A)系统的内能一定增加
(B)系统的内能一定减少
(C)系统的内能一定保持不变
(D)系统的内能可能增加,也可能减少或保持不变
\item 刚性三原子分子理想气体的压强为$P$,体积为$V$,则它的内能$(\qquad)$\\
(A) $2PV$
(B) $PV$
(C)$3PV$
(C)$4.5PV$


\item 一瓶氢气和一瓶氦气气体密度相同,分子的平均平动能相同,而且它们都处于平衡状态,则它们$(\qquad)$\\
(A)温度相同,压强相同\\
(B)温度,压强都不相同\\
(C)温度相同 ,但氦气的压强大于氢气的压强\\
(D)温度相同,但氢气的压強大于氦气的压强
\item 设高温热源的热力学温度是低温热源的热力学温度的n倍,则理想气体在一次卡诺循环中,传给低温热源的热最是从高温热源吸取热量的$(\qquad)$\\
(A)$n$倍\\
(B)$n-1$倍\\
(C) $\frac{1}{n}$倍\\
(C)$\frac{n+1}{n}$倍
\item 用一个隔板把绝热容器分成体积为$V_1$和$V_2$两部分,两部分初始温度均为T,初始压强均为P.但所盛气体种类不同,若将隔板抽开,让气体均匀混合,求混合前后系统的熵变$\Delta S$ = $(\qquad)$
\item 一定量的某种理想气体在等压过程中对外做功为$200J$,若此种气体为单原子分子气体,则该过程中需吸收的热量是$(\qquad)$,若此种气体为双原子分子气体,则该过程中吸收的热盘是$(\qquad)$。
\item 水蒸汽分解为同温度的氢气和氧气 。气体的内能增加百分比为$(\qquad)$。
\item 一飞机在地面时机舱中的压力指示为$1.01*10^5 Pa$,到高空后压强降为$8.11*10^4 Pa$,设大气温度均为27度,问此时飞机距地面的高度为$(\qquad)$(设空气的摩尔质量为$2.89*10^{-2}Kg/mol$)。
\item 一绝热密封容器体积为$V=10^{-2}m^3$,以速度$v=100m/s$作匀速直线运动。容器中有$100g$氢气,当容器突然停止时,氢气的温度增加量$\Delta T=(\qquad)$。
\item 自然光入射光强为I,,通过两个尼科耳棱镜。当两个尼科耳棱镜主截面的夹角由 45°减小为 30°时,试问透射光强的变化$(\qquad)$\\
(A)增大\\
(B)减小\\
(C)不变\\
(D)不能确定
\item 菲涅耳圆孔衍射实验中,对轴上某参考点圆孔刚好露出$\frac{3}{2}$个半波带时,该点的光强度与自由传播时的之比为$(\qquad)$\\
(C)$1:2$\\
(D)$2:1$\\
(A)$1:\sqrt{2}$\\
(B)$\sqrt{2}:1$
\item 在白光照射下,夫琅禾费衍射的零级斑正中心是什么颜色的$(\qquad)$\\
(A)白色
(B)红白色
(C)蓝白色
(D)彩色
\end{enumerate}
