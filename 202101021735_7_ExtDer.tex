% 外导数
% 外代数|外导数|外微分|叉乘|梯度|散度|旋度

\pentry{外微分\upref{ExtDif}}

\subsection{外导数的定义}

外导数是一种流形上的\textbf{外微分代数}上的映射,其术语分为两部分,“外”和“导数”.“外”,指的是它把各$\Omega^k(M)$中的元素映射到$\Omega^k(M)$\textbf{之外};“导数”,指的是它具有和求导类似的性质.实际上,矢量分析中的求导就是外导数的一个特例——你可能会问,求导并不具有“外”的特点,怎么就是特例了呢?我们会在本节中解释这一点.

\begin{definition}{外导数}
给定流形$M$,其外微分代数是$\Omega (M)$.定义映射$\dd:\Omega (M)\rightarrow\Omega (M)$,满足:
\begin{itemize}
\item $\forall \omega\in\Omega^k(M)$,有$\dd \omega\in\Omega^{k+1}(M)$.
\item 对于光滑函数$f\in C^\infty(M)$,$\dd f$就是$f$的方向导数(1-形式).
\item \textbf{线性性}:任取$a, b\in \mathbb{R}$和$\omega, \mu\in\Omega(M)$,有$\dd(a\omega+b\mu)=a\dd\omega+b\dd\mu$.
\item \textbf{Leibniz性}:任取$\omega, \mu\in\Omega^1(M)$,有$\dd(\omega\wedge\mu)=\omega\wedge\dd\mu-\dd\omega\wedge\mu$\footnote{注意有个负号,以及$\omega, \mu$都特指1-形式.}.
\end{itemize}
称这个映射为$\Omega (M)$或者说$M$上的一个\textbf{外导数}.
\end{definition}

定义中





