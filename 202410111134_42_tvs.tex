% 拓扑向量空间
% keys 拓扑向量空间|局部凸拓扑向量空间|局部凸空间
% license Xiao
% type Wiki
\pentry{拓扑空间\nref{nod_Topol},向量空间\nref{nod_LSpace}}{nod_3053}

\subsection{拓扑向量空间}

\begin{definition}{拓扑向量空间}
一个 $\mathbb{F}$-拓扑向量空间 $V$ 是一个实(或复)向量空间同时也是一个拓扑空间,使得加法 $+: V \times V \to V$ 和数乘 $\cdot: \mathbb{F} \times V \to V$ 都是连续函数。

在常见的的语境下我们要求拓扑向量空间是Hasdorff空间(\enref{见分离性}{Topo5}),此时拓扑向量空间是一个\enref{拓扑(加法)群}{TopGrp}。
\end{definition}

\begin{example}{}
\enref{赋范空间}{NormV}、\enref{内积空间}{InerPd}都是拓扑向量空间。

事实上赋范空间和内积空间都可以通过定义度量成为度量空间,而度量空间是拓扑空间,对赋范空间(范数为 $\norm{\cdot}$),自然的度量定义为 $d(x,y):=\norm{x-y}$,对内积空间(内积为 $(\cdot,\cdot)$)则为 $d(x,y):=\sqrt{(x-y,x-y)}$。因此它们都是拓扑空间。由于它们本身就是在线性空间上定义的,因而都是线性空间。所以只需验证加法和数乘的连续性。
\end{example}

\addTODO{有限维度实向量空间有唯一确定的(Hausdorff)拓扑结构}
% Giacomo:可以参考 https://kconrad.math.uconn.edu/blurbs/topology/finite-dim-TVS.pdf

\subsection{局部凸拓扑向量空间/局部凸空间}

\addTODO{待续}
% Giacomo:希望有专家来续写这部分。

% \begin{definition}{局部凸拓扑向量空间/局部凸空间}
% 一个拓扑向量空间 $V$ 被称为\textbf{局部凸(locally convex)}的,如果它满足

% \end{definition}


