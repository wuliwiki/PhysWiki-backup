% 整数模与裴蜀定理
% keys 整数模|最大公约数|裴蜀定理|算术基本定理
% license Usr
% type Tutor

\begin{definition}{数的模}
数的\textbf{模(module)}是指,对于一个数的集合 $S$,若任意两个 $S$ 中的元素 $x, y$,他们的和与差 $(x\pm y)$ 都是 $S$ 中的元素,即 $\forall x, y \in S$,$(x \pm y) \in S$。
\end{definition}
显然单独一个数 $0$ 也构成一个模,称为\textbf{零模(null module)}。

下面我们考虑整数的模,由定义可知,若 $a \in S$,则 $(a-a)=0 \in S$,即 $0$ 总在数的模中