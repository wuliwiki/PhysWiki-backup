% 卡尔·弗里德里希·高斯(综述)
% license CCBYSA3
% type Wiki

本文根据 CC-BY-SA 协议转载翻译自维基百科\href{https://en.wikipedia.org/wiki/Carl_Friedrich_Gauss}{相关文章}。

约翰·卡尔·弗里德里希·高斯(德语:Gauß [kaʁl ˈfʁiːdʁɪç ˈɡaʊs];拉丁语:Carolus Fridericus Gauss;1777年4月30日–1855年2月23日)是德国数学家、天文学家、测量学家和物理学家,对数学和科学的多个领域做出了重要贡献。他自1807年起成为哥廷根天文台台长和天文学教授,直至1855年去世。高斯被广泛认为是历史上最伟大的数学家之一。

在哥廷根大学学习期间,他提出了多个数学定理。高斯以私人学者的身份完成了他的代表作《算术研究》和《天体运动理论》。他给出了代数基本定理的第二个和第三个完整证明,对数论作出了贡献,并发展了二次和三次二次型的理论。

高斯在发现冥王星作为矮行星的工作中发挥了重要作用。他关于受到大行星影响的行星状物体的运动的研究,导致了高斯引力常数和最小二乘法的引入,而高斯在Adrien-Marie Legendre发表之前就已发现了这一方法。高斯与其他学者一起负责了1820年至1844年期间对汉诺威王国的大规模地理测量和弧长测量项目;他是地球物理学的创始人之一,并提出了磁学的基本原理。他实际工作的成果包括1821年发明了太阳能标,1833年发明了磁力计,以及与威廉·爱德华·韦伯一起于1833年发明了第一台电磁电报机。

高斯是第一个发现并研究非欧几里得几何的人,并且他自己创造了这个术语。他还在约160年前就发展了快速傅里叶变换,比约翰·图基和詹姆斯·库利提前了数十年。

高斯拒绝发表未完成的工作,留下了多部未完成的作品,交由后人编辑。他认为学习的过程,而非拥有知识本身,才是最令人享受的。高斯曾坦言自己不喜欢教学,但他的一些学生后来成为了有影响力的数学家,如理查德·德德金德和伯恩哈德·黎曼。