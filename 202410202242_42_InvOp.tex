% 逆算子
% keys 逆算子|可逆性
% license Usr
% type Tutor

\pentry{拓扑线性空间中的线性算子\nref{nod_TLinO}}{nod_d7fd}

逆算子是双射的算子的逆。当算子可逆时,在某些情形下逆算子和算子有很多相同的性质,比如,线性算子的逆算子时线性的,而完备赋范空间之间的线性有界算子的逆是有界的。


\begin{definition}{逆算子}
设 $D_A$ 是拓扑线性空间的子集,$A:D_A\rightarrow E_1$ 是其上的算子(映射),$\Im A$ 是 $A$ 的象。其对任意 $y\in\Im A$,方程
\begin{equation}
Ax=y~
\end{equation}
 有唯一解,则称算子 $A$ \textbf{可逆}。此时映射
 \begin{equation}
 \Im A\rightarrow E,Ax\mapsto x~  
 \end{equation}
 称为 $A$ 的\textbf{逆算子},记作 $A^{-1}$。

\end{definition}

\subsection{性质}
下面定理表明,线性算子的逆算子是线性的。

\begin{theorem}{}
\enref{线性算子}{TLinO} $A$ 的逆算子 $A^{-1}$ 是线性的。
\end{theorem}

\textbf{证明:}
任意 $\Im A$ 中的 $y_1=A x_1,y_2=Ax_2$,因为
\begin{equation}
A(\alpha x_1+\beta x_2)=\alpha Ax_1+\beta Ax_2=\alpha y_1+\beta y_2.~
\end{equation}
所以
\begin{equation}
A^{-1}(\alpha y_1+\beta y_2)=\alpha x_1+\beta x_2=\alpha A^{-1}y_1+\beta A^{-1}y_2.~
\end{equation}


\textbf{证毕!}

下面定理表明,在完备赋范空间(Banach空间)之间的线性有界算子的逆算子是有界的。

\begin{theorem}{逆算子的Banach定理}\label{the_InvOp_1}
设 $E,E_1$ 是\enref{完备赋范空间}{banach},$A$ 是 $E$ 到 $E_1$ 的可逆线性算子,则其逆算子 $A^{-1}$ \enref{有界}{BoundO}。
\end{theorem}

为证明它,需要如下引理。
\begin{lemma}{}\label{lem_InvOp_1}
设 $M$ 是Banach空间 $E$ 中的\enref{处处稠密集}{MaDen}。则任意非零元 $y\in E$ 可展开成级数:
\begin{equation}
y=y_1+\cdots+y_n+\cdots,~
\end{equation}
其中 $y_k\in M$ 且 $\norm{y_k}\leq 3\norm{y}/2^k$。
\end{lemma}
\textbf{证明:}下面用逐次构造 $y_k$ 进行证明。选择 $y_1$ 使得
\begin{equation}
\norm{y-y_1}\leq\norm{y}/2.~
\end{equation}
这是一个半径为 $\norm{y}/2$ 中心在 $y$ 的球,由于 $M$ 在 $E$ 中处处稠密,因此在该球内任一点的邻域都有 $M$ 的点,因此这样的选择是可能的。

同样由于 $M$ 在 $E$ 中处处稠密,可选 $y_n$ 使得 $\norm{y-y_1-\cdots-y_n}\leq\norm{y}/2^n$。根据 $y^k$ 的选择,
\begin{equation}
\norm{y-\sum_{k=1}^n y_k}\rightarrow0.~
\end{equation}
即级数 $\sum_{k=1}^\infty y_k$ 收敛于 $y$。此外
\begin{equation}
\begin{aligned}
\norm{y_n}=&\norm{y_n+y_{n-1}+\cdots+y_1-y+y-y_1-\cdots-y_{n-1}}\\
\leq&\norm{y-y_1-\cdots-y_{n}}+\norm{y-y_1\cdots-y_{n-1}}\\
\leq&\norm{y}/2^n+\norm{y}/2^{n-1}=3\norm{y}/2^n.
\end{aligned}~
\end{equation}

\textbf{证毕!}

\textbf{\autoref{the_InvOp_1} 的证明:}
证明的思路是这样的:通过证明 $E_1$ 中存在稠密集,从而可由\autoref{lem_InvOp_1} 展开 $E_1$ 的任意非零元 $y$,这一级数在逆算子作用下对应 $E$ 中的级数,并且通过构造这一级数收敛点某个 $x$,进而由 $A$ 的线性和连续性可知 $Ax=y$,且 $x$ 的范数恒小于某个常数乘以对应的 $y$ 的范数,从而表明 $A$ 有界。

\textbf{稠密集的构造:}\enref{Baire定理}{Baire}断定完备赋范空间不能表为可数无处稠密集的并,因此若能构造 $E_1$ 为可数个集的并,则这其中一定有一个集不是无处稠密的,即有一个集在某一球上稠密。由于 $\frac{\norm{y}}{\norm{A^{-1}y}}$ 对具体的 $y\in E_1$ 必定小于某个正整数,因此若令 $M_k$ 为所有满足 $\norm{A^{-1}y}\leq k\norm{y}$ 的 $y\in E$ 的全体,则 $E_1=\bigcup\limits_{k=1}^\infty M_k$。即 $E_1$ 可表为可数集的并,因为可设其中的 $M_n$ 在某个球 $B$ 中稠密。

在 $B$ 中选择中心在 $M_n$ 中的球层 $P$:满足不等式 $\beta<\norm{z-y_0}<\alpha$ 的 $z$ 的全体,其中 $0<\beta<\alpha,y_0\in M_n$。若把该球层中心移到原点便得球层 $P_0=\{z|0<\beta<\norm{z}<\alpha\}$。下面将表明,有某个 $M_N$ 在 $P_0$ 中稠密,且在 $E_1$ 中稠密:设 $z\in P\cap M_n$,则 $z-y_0\in P_0$,且
\begin{equation}
\begin{aligned}
\norm{A^{-1}(z-y_0)}\leq&\norm{A^{-1}z}+\norm{A^{-1}y_0}\leq n(\norm{z}+\norm{y_0})\\
\leq&n(\norm{z-y_0}+2\norm{y_0})\\
=&n\norm{z-y_0}\qty(1+\frac{2\norm{y_0}}{\norm{z-y_0}})\\
\leq& n\norm{z-y_0}(1+2\norm{y_0}/\beta).
\end{aligned}~
\end{equation}
量 $n(1+2\norm{y_0}/\beta)$ 不依赖于 $z$。令 $N:=1+n[1+2\norm{y_0}/\beta]$(其中 $[\cdot]$ 是高斯记号,表示一个数的整数部分)。因而由我们的构造 $z-y_0\in M_N$。进而由 $M_n$ 在 $P$ 中稠密知 $M_N$ 在 $P_0$ 中稠密($P\subset[M_n]\Rightarrow P_0=P-y_0\subset[M_n]-y_0\subset[M_N]$)。

任意非零元 $y\in E_1$,总有 $\lambda$ ,使得 $\beta<\norm{\lambda y}<\alpha$,即 $\lambda<P_0$





\textbf{证毕!}



