% Redis 简介
% keys 数据库 | 缓存 | 中间件 | NoSQL
% license Usr
% type Tutor

\begin{issues}
\issueDraft
\issueTODO
\end{issues}

\subsection{Redis概述}

Redis(Remote dictionary server)是一个开源的,基于内存的数据存储系统,它常用于数据库,缓存,消息队列等各种场景,是目前最热门的NoSQL数据库之一。

早期,互联网公司的数据库系统大多是类似MySQL这种传统数据库来对外提供服务。随着互联网快速发展,外部访问量越来越大,数据库的瓶颈越来越明显(主要是由于机械硬盘IO导致的)。

\subsubsection{数据结构 }
Redis支持多种数据结构,包括5种基础数据类型和5种高级数据类型。

\textbf{基本类型:}

\begin{itemize}
\item 字符串 String
\item 列表 List
\item 集合 Set
\item 有序集合 SortedSet
\item 哈希 Hash
\end{itemize}

\textbf{高级类型:}

\begin{itemize}
\item 消息队列 Stream
\item 地理空间 Geospatial
\item HyperLoglog
\item 位图 Bitmap
\item 位域 Bitfield
\end{itemize}

\subsubsection{支持的使用方法}
\begin{itemize}
\item CLI, 命令行
\item GUI, 图形界面
\item API
\end{itemize}

\subsubsection{优势}
\begin{itemize}
\item 性能极高
\item 数据类型丰富, 单键值对最大支持512M大小的数据
\item 简单易用,支持多有主流编程语言
\item 支持数据持久化,主从复制,哨兵模式等高可用特性
\end{itemize}

\subsection{安装redis}
\addTODO{MAC安装redis}
\subsubsection{Windows安装redis}
\begin{enumerate}
\item WSL安装(同Linux安装,推荐)
\item Docker 安装
\begin{lstlisting}[language=bash]
docker search redis # 搜索redis
docker pull redis # 拉取redis镜像
\end{lstlisting}
\item exe 安装

\end{enumerate}
\subsubsection{Linux安装redis}
