% 欧几里得(综述)
% license CCBYSA3
% type Wiki

本文根据 CC-BY-SA 协议转载翻译自维基百科\href{https://en.wikipedia.org/wiki/Euclid}{相关文章}。

\begin{figure}[ht]
\centering
\includegraphics[width=6cm]{./figures/7848a3bb491b282e.png}
\caption{《欧几里得》,由朱塞佩·德·里贝拉(Jusepe de Ribera)创作,约1630–1635年。} \label{fig_Euclid_1}
\end{figure}
欧几里得(/ˈjuːklɪd/;古希腊语:Εὐκλείδης;约公元前300年活跃)是一位古希腊数学家,主要从事几何学和逻辑学的研究。被称为“几何学之父”,他最著名的成就是《几何原本》这部著作,它奠定了几何学的基础,直到19世纪初期,这些基础一直主导着该领域。他的体系,现被称为欧几里得几何学,结合了之前希腊数学家的理论创新与综合,包括厄多克索斯(Eudoxus of Cnidus)、希波克拉底(Hippocrates of Chios)、泰勒斯(Thales)和提阿托斯(Theaetetus)等人的理论。与阿基米德和佩尔加的阿波罗尼斯一起,欧几里得通常被认为是古代最伟大的数学家之一,也是数学史上最具影响力的人物之一。

关于欧几里得的生平知之甚少,绝大多数资料来源于几百年后学者普罗克洛斯(Proclus)和帕普斯(Pappus)的记载。中世纪的伊斯兰数学家创造了一个富有幻想色彩的生平,而中世纪拜占庭学者和早期文艺复兴学者则误将他与早期哲学家梅加拉的欧几里得混淆。现在普遍认为,欧几里得大部分时间都在亚历山大里亚度过,约生活在公元前300年,介于柏拉图的学生与阿基米德之间。也有一些猜测认为,欧几里得曾在柏拉图学园学习,后来在穆萨尤姆教授数学;他被认为是将早期的柏拉图学派传统与后来的亚历山大里亚学派传统连接起来的桥梁。

在《几何原本》中,欧几里得通过少数公理推导出定理。他还撰写了关于透视学、圆锥曲线、球面几何、数论和数学严谨性的著作。除了《几何原本》,欧几里得还写了光学领域的基础性著作《光学》,以及其他较不为人知的作品,如《数据》和《现象》。关于《几何分割》和《镜面反射学》是否为欧几里得所作,学术界仍有争议。欧几里得还被认为撰写了许多已经失传的作品。
\subsection{生命}  
\subsubsection{传统叙述}
\begin{figure}[ht]
\centering
\includegraphics[width=8cm]{./figures/0d7a20b6f91e0fa9.png}
\caption{拉斐尔在《雅典学派》(1509–1511)中的欧几里得形象细节,展示了他在教授学生。} \label{fig_Euclid_2}
\end{figure}
“Euclid”这个英语名字是古希腊名字Eukleídes(Εὐκλείδης)的英文化版本。[4][a] 它来源于“eu-”(εὖ;意为“好”)和“klês”(-κλῆς;意为“名声”),意思是“著名的,荣耀的”。[6] 在英语中,"Euclid"通过转喻有时指代他最著名的作品《几何原本》,或其副本,[5] 有时也被当作“几何”的同义词。[2]

与许多古希腊数学家一样,关于欧几里得的生平细节大多未知。[7] 他被认为是四部大部分存世的著作的作者——《几何原本》、 《光学》、 《数据》、 《现象》——但除此之外,关于他的确切信息几乎没有。[8][b] 传统的叙述主要依赖于公元5世纪普罗克洛斯在其《欧几里得《几何原本》第一卷注释》中的记载,以及公元4世纪初亚历山大的帕普斯的一些轶事。[4][c]

根据普罗克洛斯的说法,欧几里得生活在柏拉图(公元前347年去世)的几位追随者之后,并且在数学家阿基米德(公元前287年–公元前212年)之前;[d] 具体来说,普罗克洛斯将欧几里得置于托勒密一世统治时期(公元前305/304–282年)。[7][8][e] 欧几里得的出生日期不详;一些学者估计大约在公元前330年[11][12] 或公元前325年,[2][13] 但其他学者则避免做出推测。[14] 假设他是希腊血统,[11] 但他的出生地未知。[15][f] 普罗克洛斯认为欧几里得遵循柏拉图的传统,但没有确凿的证据可以证实这一点。[17] 他不太可能与柏拉图同代,因此通常推测他是柏拉图的学生,在雅典的柏拉图学园接受教育。[18] 历史学家托马斯·希思支持这一理论,指出大多数有能力的几何学家都生活在雅典,包括许多欧几里得依赖的前人的工作;[19] 历史学家米哈利斯·西亚拉罗斯认为这只是一个猜测。[4][20] 无论如何,欧几里得的著作内容表明他熟悉柏拉图几何学的传统。[11]

在《帕普斯集》中,帕普斯提到阿波罗尼乌斯曾与欧几里得的学生一起在亚历山大学习,这表明欧几里得曾在那里工作并创立了一个数学传统。[8][21][19] 该城市由亚历山大大帝于公元前331年建立,[22] 托勒密一世自公元前306年起统治,使其在亚历山大帝国分裂后的混乱战争中拥有相对的稳定性。[23] 托勒密开始了希腊化进程,并委托建造许多建筑,建立了庞大的穆塞翁学术机构,这是当时的一个领先教育中心。[15][g] 推测欧几里得是穆塞翁最早的学者之一。[22] 欧几里得的死亡日期不详;有学者推测他大约在公元前270年去世。[22]
\subsubsection{身份与历史性}
\begin{figure}[ht]
\centering
\includegraphics[width=8cm]{./figures/9a263cfe27f2006e.png}
\caption{多梅尼科·马罗利(Domenico Maroli)在1650年代的画作《厄尔克里德·梅加拉(Euclid of Megara Dressing as a Woman to Hear Socrates Teach in Athens)》描绘了厄尔克里德(Euclid)以女性装扮前往雅典聆听苏格拉底的讲授。当时,哲学家厄尔克里德与数学家厄尔克里德被错误地认为是同一个人,因此这幅画中桌子上放置了数学物品。[25]} \label{fig_Euclid_3}
\end{figure}
欧几里得通常被称为“亚历山大城的欧几里得”,以区分他与早期的哲学家梅加拉的欧几里得(苏格拉底的弟子,曾出现在柏拉图的对话录中),两者在历史上常常被混淆。[4][14] 公元1世纪的罗马编年史家瓦勒留斯·马西穆斯(Valerius Maximus)错误地将欧几里得的名字与欧多克索斯(公元前4世纪的数学家)互换,误将他作为柏拉图派遣给询问如何立方倍增的人们的数学家。[26] 由于这一提到大约百年之前的数学欧几里得,欧几里得与梅加拉的欧几里得在中世纪拜占庭文献中(现在已失传)混淆,最终导致欧几里得这位数学家被附上了两位人物生平的细节,并被描述为“梅加拉人”(Megarensis)[4][28]。拜占庭学者西奥多·梅托基特斯(约1300年)明确地将这两位欧几里得混为一谈,印刷商埃尔哈德·拉特多尔特(Erhard Ratdolt)也在其1482年版的《元素》拉丁语版中沿用了这种说法。[27] 在数学家巴托洛梅奥·赞贝尔蒂(Bartolomeo Zamberti)于1505年翻译《元素》时,将有关两位欧几里得的现存生平片段附在了前言中,此后相关出版物都沿用了这一辨识方法。[27] 后来的文艺复兴学者,尤其是彼得·拉姆斯(Peter Ramus),重新评估了这一观点,并通过年代学问题和早期文献中的矛盾证明其错误。[27]

中世纪阿拉伯文献提供了大量关于欧几里得生平的信息,但这些资料完全无法验证。[4] 大多数学者认为这些资料的真实性存疑;[8] 赫斯特别指出,这些虚构的内容是为了加强这位受人尊敬的数学家与阿拉伯世界的联系。[17] 也有许多关于欧几里得的轶事,虽然它们的历史性尚不确定,这些故事“将他描绘为一位和蔼可亲、温和的老人”。[29] 最著名的故事是普罗克卢斯(Proclus)讲述的关于托勒密问欧几里得是否有比读《元素》更快捷的几何学习方式,欧几里得回答说:“没有王道可以走捷径。”[29] 这一轶事值得怀疑,因为在斯托巴乌斯(Stobaeus)中也记录了梅内克莫斯(Menaechmus)与亚历山大大帝之间非常相似的互动。[30] 这两段记载均写于公元5世纪,且没有注明来源,且都未出现在古希腊文献中。[31]

关于欧几里得约公元前300年的活动的任何确定性日期都受到当时缺乏直接提及的质疑。[4] 欧几里得的最早原始提及出现在阿波罗尼乌斯(Apollonius)写给《圆锥曲线》序言中的信中(公元前2世纪初):“《圆锥曲线》第三卷包含许多令人惊讶的定理,这些定理对于解方程和求解空间位置的解的数目非常有用。其中大多数,特别是最精妙的,是新的。当我们发现这些定理时,我们意识到欧几里得并没有完成三线和四线的定位,只完成了一个偶然的片段,而且即使那个片段也做得不太好。”[26] 《元素》推测至少部分在公元前3世纪就已流传,因为阿基米德和阿波罗尼乌斯都理所当然地使用了其中的一些命题;[4] 然而,阿基米德采用了与《元素》不同的比例理论的早期版本。[8] 《元素》中所含材料的最古老物理副本可追溯至公元100年左右,在埃及奥克里恩库斯的废墟中发现的纸草纸片上。[26] 直接引用《元素》且日期明确的最早文献出现在公元2世纪,由盖伦(Galen)和阿弗罗狄西亚的亚历山大(Alexander of Aphrodisias)提及;到那时,《元素》已经成为标准的学校教材。[26] 一些古希腊数学家提到欧几里得时,通常称他为“ὁ στοιχειώτης”(“《元素》的作者”)。[32] 在中世纪,一些学者认为欧几里得不是历史人物,而他的名字来源于希腊数学术语的误传。[33]
\subsection{作品}  
\subsubsection{《几何原本》}
\begin{figure}[ht]
\centering
\includegraphics[width=8cm]{./figures/592c992974e918fd.png}
\caption{一块公元约75–125年的《几何原本》纸莎草 fragment,发现于奥克斯里欣库斯(Oxyrhynchus),该图表伴随于第二卷第五命题。[34]} \label{fig_Euclid_4}
\end{figure}
欧几里得以他的十三卷著作《几何原本》最为人知(古希腊文:Στοιχεῖα;Stoicheia),被认为是他的代表作。[3][35] 其中大部分内容来源于早期的数学家,包括欧多克索斯、希奥斯的希波克拉底、塔勒斯和西阿图斯,而其他一些定理则被柏拉图和亚里士多德提及。[36] 由于《几何原本》本质上取代了许多早期并且现已失传的希腊数学,因此很难区分欧几里得的工作与其前辈的工作。[37][h] 古典学者马克斯·阿斯珀(Markus Asper)总结道:“显然,欧几里得的成就在于将公认的数学知识整理成一个有条理的结构,并增加新的证明以填补空白”,而历史学家塞拉菲娜·库莫(Serafina Cuomo)则将其描述为一个“结果的宝库”。[38][36] 尽管如此,西阿拉罗斯(Sialaros)进一步指出:“《几何原本》令人瞩目的紧密结构展示了超越单纯编辑的作者控制。”[9]

《几何原本》并不仅仅讨论几何学,正如有时人们所认为的那样。[37] 它通常被分为三个主题:平面几何(第1–6卷)、基础数论(第7–10卷)和立体几何(第11–13卷)——尽管第5卷(关于比例)和第10卷(关于无理线)并不完全符合这一划分。[39][40] 该书的核心是散布在其中的定理。[35] 使用亚里士多德的术语,这些定理大致可以分为两类:“第一原理”和“第二原理”。[41] 第一类包括被标记为“定义”(古希腊文:ὅρος 或 ὁρισμός)、“公设”(αἴτημα)或“共同概念”(κοινὴ ἔννοια)的陈述;[41][42] 只有第一卷包括公设——后来的公理——和共同概念。[37][i] 第二类由命题组成,呈现时伴有数学证明和图示。[41] 不清楚欧几里得是否将《几何原本》视为一本教科书,但其呈现方式使其成为一种自然的教科书选择。[9] 整体而言,作者的语气保持一般性和非个人化。[36]