% 库仑定律(综述)
% license CCBYSA3
% type Wiki

本文根据 CC-BY-SA 协议转载翻译自维基百科\href{https://en.wikipedia.org/wiki/Coulomb\%27s_law}{相关文章}。

\begin{figure}[ht]
\centering
\includegraphics[width=6cm]{./figures/e58a5f04bd7289ad.png}
\caption{两个点电荷 q1 和 q2 之间的静电力 F 的大小与它们电荷大小的乘积成正比,且与它们之间距离的平方成反比。相同电荷相互排斥,异性电荷相互吸引。} \label{fig_KL_1}
\end{figure}
库仑的反比平方定律,简称库仑定律,是一条物理学的实验定律[1],用于计算静止的两个带电粒子之间的相互作用力。这种电力通常称为静电力或库仑力[2]。尽管这一定律早有认识,但它是在1785年由法国物理学家查尔-奥古斯丁·库仑首次公布的。库仑定律对于电磁学理论的发展至关重要,甚至可能是其起点[1],因为它使得有意义的讨论粒子电荷量成为可能[3]。

该定律指出,两个点电荷之间的吸引或排斥静电力的大小(或绝对值)与它们电荷大小的乘积成正比,且与它们之间距离的平方成反比[4]。库仑发现,相同电荷的物体会相互排斥:

因此,从这三个实验中可以得出结论:两个带有相同类型电荷的球体相互排斥的力,遵循与距离的平方成反比的规律[5]。