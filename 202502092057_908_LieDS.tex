% 李代数(综述)
% license CCBYSA3
% type Wiki

本文根据 CC-BY-SA 协议转载翻译自维基百科\href{https://en.wikipedia.org/wiki/Lie_algebra}{相关文章}。

在数学中,李代数(发音为 /liː/,LEE)是一个向量空间 \( \mathfrak{g} \),配有一个名为李括号的运算,它是一个交替双线性映射 \(\mathfrak{g}\times \mathfrak{g}\to \mathfrak{g}\),满足雅可比恒等式。换句话说,李代数是一个定义在域上的代数,其中的乘法运算(称为李括号)是交替的,并且满足雅可比恒等式。两个向量 \(x\)和 \(y\) 的李括号记作 \(xy\)。李代数通常是一个非结合代数。然而,每个结合代数都可以生成一个李代数,该李代数由相同的向量空间构成,且使用交换子李括号,即 \([x,y]=xy-yx\)。

李代数与李群密切相关,李群是既是群又是光滑流形的群:每个李群都会生成一个李代数,该李代数是单位元处的切空间。(在这种情况下,李括号衡量了李群不满足交换律的程度。)反过来,任何定义在实数或复数上的有限维李代数,都有一个对应的连通李群,且唯一性仅限于覆盖空间(李的第三定理)。这种对应关系使得我们能够通过李代数这一线性代数的简化对象来研究李群的结构和分类。

更详细地说:对于任何李群,单位元1附近的乘法操作在一阶近似下是交换的。换句话说,每个李群 \( G \) 在一阶近似下大致是一个实向量空间,即 \( G \) 在单位元处的切空间 \( \mathfrak{g} \)。在二阶近似下,群操作可能是非交换的,描述 \( G \) 在单位元附近不交换性的二阶项赋予了 \( \mathfrak{g} \) 李代数的结构。一个显著的事实是,这些二阶项(李代数)完全决定了 \( G \) 在单位元附近的群结构。它们甚至决定了 \( G \) 的全局结构,直到覆盖空间为止。

在物理学中,李群作为物理系统的对称群出现,它们的李代数(单位元附近的切向量)可以被看作是无穷小的对称运动。因此,李代数及其表示在物理学中被广泛使用,特别是在量子力学和粒子物理学中。

一个基础的例子(并非直接来自一个结合代数)是三维空间 \(\mathfrak{g} = \mathbb{R}^3\),其李括号由叉积定义:\([x, y] = x \times y.\)这是反对称的,因为\(x \times y = -y \times x\),并且它不满足结合性,而是满足雅可比恒等式:
\[x \times (y \times z) + y \times (z \times x) + z \times (x \times y) = 0.~\]
这是空间旋转群的李代数,每个向量 \(v \in \mathbb{R}^3\) 可以被视为绕轴 \(v\) 的无穷小旋转,角速度等于 \(v\) 的大小。李括号是衡量两个旋转之间不交换性的工具。由于旋转与自身交换,因此具有交替性质:\([x, x] = x \times x = 0\).
\subsection{历史}  
李代数是由索福斯·李(Sophus Lie)在1870年代提出的,用于研究无穷小变换的概念,[1] 并在1880年代由威廉·基林(Wilhelm Killing)独立发现。[2] "李代数" 这一名称由赫尔曼·外尔(Hermann Weyl)在1930年代提出;在早期的文献中,常用“无穷小群”(infinitesimal group)这一术语。