% 高能物理
% license CCBYSA3
% type Wiki

(本文根据 CC-BY-SA 协议转载自原搜狗科学百科对英文维基百科的翻译)

粒子物理(也称为高能物理学)是物理学的一个分支,主要研究构成物质和辐射的粒子的性质。虽然“粒子”一词可以指代各种类型的微小物体(例如质子、气体粒子,甚至家里的灰尘),但粒子物理通常研究不可再分割的可探测的最小粒子(“基本粒子”)及其相互作用。根据目前的理解,这些“基本粒子”都是量子场的激发,它们之间的相互作用也受到量子场的支配。而解释这些“基本粒子”,量子场及其动力学的主流理论叫做标准模型。现代粒子物理通常研究标准模型及其各种可能的扩展,这些扩展通常会涉及到最新发现的希格斯玻色子,甚至众所皆知的力场——引力。[1][2]

\subsection{亚原子粒子}
\begin{figure}[ht]
\centering
\includegraphics[width=10cm]{./figures/2cba813a245660be.png}
\caption{物理学标准模型的粒子含量} \label{fig_GNWL_1}
\end{figure}

现代粒子物理研究的重点是亚原子粒子,包括原子的组分如电子、质子和中子(质子和中子是复合粒子,称为重子,由夸克组成)等。这些亚原子粒子(如光子、中微子、μ子,以及各种奇异粒子)一般通过辐射,放射或散射过程产生。粒子的动力学受到量子力学的支配,具有波粒二象性,例如它们在某些实验条件下表现出类似于粒子的行为,而在另一些实验条件下则表现出类似于波的行为。用更专业的术语来说,这些粒子的状态是用希尔伯特空间中的量子态向量来描写的,当然在量子场论中也是如此。按照粒子物理学家们的惯例,“基本粒子”一词仅适用于那些就目前所知不可再分割的,不是由其它粒子构成的粒子[3]。

\begin{table}[ht]
\centering
\caption\label{tab_DXYL_2}
\begin{tabular}{|c|c|c|c|c|c}
\hline
\qquad &\textbf{类型}&\textbf{世代}&\textbf{反粒子}\\
\hline
\textbf{夸克}& 2 & 3 & 一对 & 3