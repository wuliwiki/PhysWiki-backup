% 密度矩阵
% keys 量子力学|纯态|混合态|密度矩阵|测量
% license Xiao
% type Tutor

\pentry{矩阵的迹\upref{trace}, 投影算符\upref{projOp}, 量子力学的基本假设\upref{QMPos}}

\footnote{参考 Shankar, Principles of Quantum Mechanics 2ed, 以及 Wikipedia}若一个系综中的 $N$ 个系统中, 有 $n_i$ ($i = 1,2,\dots,k$) 个在状态 $\ket{i}$ (这里假设 $\ket{i}$ 是正交归一的)。 那么这个系综可以用\textbf{密度矩阵(density matrix)}(或算符)描述
\begin{equation}\label{eq_denMat_1}
\rho = \sum_i p_i\ket{i}\bra{i}   ~.
\end{equation}
其中 $p_i = n_i/N$ 是随机选一个系统, 处于状态 $\ket{i}$ 的概率。 若所有系统都处于同一个 $\ket{i}$, 那么这个系综就是\textbf{纯的(pure)}, 否则就是\textbf{混合的(mixed)}。

\subsection{非纯态的等效}
纯态 $\ket{\psi}$ 可以唯一地表示为密度矩阵 $\rho = \ket{\psi}\bra{\psi}$, 它的意义也相当明确。

然而对于非纯态, 我们可以使用不同的正交归一的(纯态)基底的概率组合来表示, 测量结果却是一样的。 因为测量并不能区分量子概率和经典概率。

例如有一束电子。 若这束电子是纯态的, 我们就可以通过多次测量得到这个纯态(除了一个总体相位因子)。 例如通过测量 $\ket{y+}$ 和 $\ket{y-}$ 的概率, 我们可以确定 $\ket{x+} + c\ket{x-}$ 中的复数 $c$。

但若这束电子的自旋方向是随机的, 我们既可将其等效为随机的一半 $\ket{x+}$ 和 $\ket{x-}$ 构成的, 也可以等效为随机的一半 $\ket{y+}$ 和 $\ket{y-}$ 构成的。 虽然 “实际” 上它们是不一样的, 但任何测量都无法区分这两种情况。 所以密度矩阵可以是 $\rho = (\ket{x+}\bra{x+} + \ket{x-}\bra{x-})/2$ 也可以是 $\rho = (\ket{y+}\bra{y+} + \ket{y-}\bra{y-})/2$。 事实上, 它们都是单位矩阵的一半。 注意如果概率不是均分的, 如 $\rho = (\ket{x+}\bra{x+}/3 + 2\ket{x-}\bra{x-})/3$ 就无法用 $\ket{y\pm}$ 基底表示(一般的对角矩阵经过相似变换后不是对角矩阵)。

纯态和非纯态都可以对应一个唯一密度矩阵。

\subsection{密度矩阵的性质}
\pentry{半正定矩阵\upref{DefMat}}
\begin{itemize}
\item 密度矩阵算符是厄米算符(自伴算符)

\item 密度矩阵的迹为 1

\item 密度矩阵是半正定矩阵。可以从\autoref{eq_denMat_1} 看到,对任意态 $\ket\varphi$,总是有 
    $\bra\varphi \rho \ket\varphi\ge 0$
\item 对于给定力学量算符$\Omega$,求取$\Omega$的期望值
    是$\left\langle \Omega \right\rangle = \opn{tr}\left(\rho\Omega\right)$
    
    对于某个物理量对应的算符 $\Omega$, 它的\textbf{系综平均值(ensemble average)}为
    \begin{equation}
    \ev{\bar\Omega} = \sum_i p_i \mel{i}{\Omega}{i}~.
    \end{equation}
    这个平均值既包含了每个 $\ket{i}$ 的平均, 又包含了对每个系统的平均。

    系综平均也可以用迹表示为 $\opn{tr}(\Omega\rho)$。 根据迹的定义,
    \begin{equation}\label{eq_denMat_3}
    \opn{tr}(\Omega\rho) = \sum_j \mel{j}{\Omega\rho}{j} = \sum_{i,j} p_i\mel{j}{\Omega}{i} \braket{i}{j} = \sum_{i} p_i\mel{i}{\Omega}{i} = \ev{\bar\Omega}~.
    \end{equation}
    证毕。

    对于纯态, 获得测量值 $\omega$ 的概率可以看作投影算符 $\mathbb P_\omega$ 的平均值
    (满足 $\mathbb P_\omega^2 = \mathbb P_\omega$)
    \begin{equation}
    P(\omega) = \abs{\braket{\omega}{\psi}}^2 = \braket{\mathbb P_\omega \psi}{\mathbb P_\omega \psi} = \mel{\psi}{\mathbb P_\omega}{\psi}~,
    \end{equation}
    所以对于混合态, 测量值 $\omega$ 的概率为
    \begin{equation}
\overline{P(\omega)} = \opn{tr}(\mathbb P_\omega\rho)~.
\end{equation}

\item 密度矩阵厄米半正定的性质使得我们总是将它在一组正交完备基 $\ket{\psi_1},\ket{\psi_2},\cdots$ 下对角化:
    \begin{equation}
    \rho = \sum\limits_i p_k\ket{\psi_k} \bra{\psi_k}~.
    \end{equation}
    这表明系统处于 $\ket{\psi_k}$ 状态的概率为 $p_k$。

    考虑薛定谔方程
\end{itemize}
\subsection{密度矩阵矩阵元}


回顾在基矢量$\left\{ \ket{u_n}\right\}$下,密度矩阵$\rho$的表达形式:
$$\rho = \sum\limits_k p_k\rho_k = \sum\limits_{m,n}\rho_{mn}\ket{u_m}\bra{u_n} ~,$$
上式中$\rho_k = \ket{\psi_k}\bra{\psi_k}$,$\rho_{m,n}$是密度矩阵的矩阵元。

\subsubsection{密度矩阵的对角元}

我们首先考虑对角元,即$\rho_{nn}$,显然有
$$\rho_{nn} = \sum\limits_{k} p_k \left[\rho_k\right]_{nn}~.$$
其中$\left[\rho_k\right]_{nn}$表示密度矩阵$\rho_k = \ket{\psi_k}\bra{\psi_k}$的第$n$个对角元。

假设:
$$\ket{\psi_k} = \sum\limits_n C_n^{k}\ket{u_n} = \sum\limits_n\ket{u_n}\braket{u_n}{\psi_k}~,$$
那么有:
$$\left[\rho_k\right]_{nn} = \left|\braket{u_n}{\psi_k}\right|^2 = \left| C_n^{k}\right|^2~.$$
则可以得到$\rho_{nn} = \sum\limits_k p_k \left|C_n^k\right|^2$,由此可以看到密度矩阵的对角元应当是一个非负实数。

现在考虑其物理意义,可以发现计算式中$\left| C_n^k \right|^2$表示的是量子态$\ket{\psi_k}$,在沿着基$\left\{\ket{u_n}\right\}$进行测量时,塌缩到$\ket{u_n}$的概率,经过经典概率$p_k$的叠加,$\rho_{nn}$表示的实际上是密度矩阵在沿着基$\left\{\ket{u_n}\right\}$测量时得到态$\ket{u_n}$的概率,因此我们称对角元$\rho_{nn}$为$\ket{u_n}$的\textbf{布局数}。

而且可以注意到,当且仅当所有$\psi_k$对应的$C_n^k$均为$0$时,$\rho_{nn} = 0$。

\subsubsection{密度矩阵的非对角元}

接下来我们讨论非对角元的情况,首先列出非对角元的计算式:
$$\rho_{pq} = \sum\limits_k p_k \braket{u_p}{\psi_k}\braket{\psi_k}{u_q} = \sum\limits_k p_k C_p^k \left(C_q^k\right)^*~.$$

注意到非对角元计算式中$C_p^k\left(C_q^k\right)^*$不再是模方,也就不一定为正实数,因此即使$C_p^k \left(C_q^k\right)^*$不全为$0$,$\rho_{pq}$也可能为$0$

考虑非对角项的物理意义,$C_p^k\left(C_q^k\right)^* = \braket{u_p}{\psi_k}\braket{\psi_k}{u_q}$当且仅当$\braket{u_p}{\psi_k}$和$\braket{u_q}{\psi_k}$均不为$0$的时候才非零,那么换句话说就是当且仅当$\ket{\psi_k}$中存在$\ket{u_p}$和$\ket{u_q}$的线性叠加时才非$0$。所以我们可以这样说:至少对于纯态,非对角元非零就意味着存在相干。反之也是一样,但对于混态则不同,我们举一个例子:

\begin{example}{}
考虑$\ket{+} = \frac{1}{\sqrt{2}}\left(\ket{\uparrow} + \ket{\downarrow}\right)$,$\ket{-} = \frac{1}{\sqrt{2}}\left(\ket{\uparrow} - \ket{\downarrow}\right)$。

可以看出$\ket{+}$和$\ket{-}$各自是存在相干的。同样我们写出其密度矩阵也可以看出其密度矩阵非对角元存在非$0$。
\begin{equation}
\rho_+ =  \frac{1}{2}\begin{pmatrix}
1&1 \\
1&1 \\
\end{pmatrix}~~~~,
\rho_- =  \frac{1}{2}\begin{pmatrix}
1&-1 \\
-1&1 \\
\end{pmatrix}~.
\end{equation}

但现在我们以各自$\frac{1}{2}$的概率混合起来,则有:
$$\rho = \frac{1}{2}\rho_+ + \frac{1}{2}\rho_- = \frac{1}{2}\begin{pmatrix}
1&0 \\
0&1 \\
\end{pmatrix}~.$$

可以看到其中非对角元的元素均为$0$。
\end{example}

对于这种情况,我们可以说两个态的相干性通过求取经典平均的这一过程抵消了。

但另一方面,如果非对角元非零,则一定意味着相干,因为非对角元非零代表着至少有一个$C_p^k\left(C_q^k\right)^*$非零,且并未因为经典平均抵消。因此我们称非对角元为\textbf{相干项}。

\subsubsection{矩阵元的演化}

前文中已经提到了薛定谔绘景下密度矩阵对事件的演化sssss,我们由此讨论矩阵元的演化。选取哈密顿量的本征基$\left\{\ket{u_n}\right\}$作为密度矩阵的基:

\begin{equation}
\begin{aligned}
&H \ket{u_n} = E_n \ket{u_n}~, \\ 
&\rho\left( t \right)_{mn} = \bra{u_m} \rho\left( t \right) \ket{u_n}~.
\end{aligned}~
\end{equation}

则对于对角元:
\begin{equation}
\begin{aligned}
\rho\left( t \right)_{nn} &= \bra{u_n}\rho\left( t \right)\ket{u_n}\\
&= \bra{u_n}\mathrm{e}^{-\mathrm{i}H t}\rho\left( 0 \right)\mathrm{e}^{\mathrm{i}Ht}\ket{u_n} \\
&=\bra{u_n}\mathrm{e}^{-\mathrm{i}E_n t}\rho\left( 0 \right)\mathrm{e}^{\mathrm{i}E_nt}\ket{u_n} \\
&=\bra{u_n}\rho\left( 0 \right)\ket{u_n} \\
&=\rho\left( 0 \right)_{nn}~.
\end{aligned}~
\end{equation}

对于非对角元:

\begin{equation}
\begin{aligned}
\rho \left( t \right)_{pq} &= \bra{u_p}\rho\left(t\right)\ket{u_q} \\
&= \bra{u_p}\mathrm{e}^{-\mathrm{i}Ht}\rho\left(0\right)\mathrm{e}^{\mathrm{i}Ht}\ket{u_q} \\
&= \bra{u_p}\mathrm{e}^{-\mathrm{i}E_pt}\rho\left( 0 \right)\mathrm{e}^{\mathrm{i}E_qt}\ket{u_q} \\
&=\mathrm{e}^{\mathrm{i}\left( E_q - E_p \right)t}\bra{u_p}\rho\left( 0 \right)\ket{u_q} \\
&= \rho \left( 0 \right)_{pq}\mathrm{e}^{\mathrm{i}\left(E_q-E_p\right)t}~.
\end{aligned}~
\end{equation}

这说明了在能量表象下,\textbf{密度矩阵$\rho$的布局数不变,而相干项以体系的 Bohr 频率震荡}。






\subsection{统计系综与平衡态}
从 ssssss 中可以看到,如果密度算符与哈密顿量 $H$ 对易,那么 $\rho(t)=\rho(0)$,密度算符不随时间发生变化,系综中每个状态发生的概率不变,根据\autoref{eq_denMat_3} ,各个物理量的系综平均值也是不变的。系统处于平衡态。

因此统计系综处于平衡态等价于 $[\rho,H]=0$。由于两个算符对易,它们实际上可以被同时对角化。存在一组基底 $\ket{1},\cdots,\ket{n}$,满足 $\rho \ket{i} = p_i \ket{i}$,且 $H\ket{i}=E_i \ket{i}$。也就是说密度矩阵在哈密顿量的一组正交的本征态基底下可对角化:
\begin{equation}
\rho = \sum_i p_i \ket{i}\bra{i},\quad H\ket{i}=E_i \ket{i}~.
\end{equation}

