% 微分形式(简明微积分)

\begin{issues}
\issueDraft
\issueOther{微分形式应当形如$\dd{x} \wedge \dd{y}$}
\end{issues}

\pentry{平面旋转变换\upref{Rot2DT}, 斯托克斯定理(简明微积分)\upref{Stokes}, 雅可比矩阵和行列式\upref{JcbDet}}

考虑二维的散度定理\upref{Divgnc}
\begin{equation}\label{eq_DForm_1}
\oint \bvec f \vdot \uvec n \dd{l} = \int \div \bvec f \dd{s}~.
\end{equation}
其中 $\dd{l}$ 是曲线上的一个小线段, 曲线逆时针为证。 $\uvec n$ 是该线段的法向量, 向外为正。

具体应该如何计算呢? 假设曲线的参数方程为 $x(t), y(t)$, 于是 $\dd{\bvec l} = (x'(t)\dd{t}, y'(t)\dd{t})$, 旋转 $90^\circ$ \upref{Rot2DT}后变为 $\uvec n \dd{l} = (y'(t)\dd{t}, -x'(t)\dd{t})$。 令矢量场为 $\bvec f = (f_x(x,y), f_y(x,y))$, 于是\autoref{eq_DForm_1} 变为
\begin{equation}
\oint [f_x(x(t),y(t)) y'(t) - f_y(x(t),y(t)) x'(t)]\dd{t}
= \int \qty(\pdv{f_x}{x} + \pdv{f_y}{y}) \dd{x}\dd{y}~,
\end{equation}
等式左边也可以简单地记为
\begin{equation}
\oint f_x(x,y) \dd{y} - f_y(x,y) \dd{x} = \oint f_x(x,y) \dd{y} - \oint f_y(x,y) \dd{x}~,
\end{equation}
其中 $f_x(x,y) \dd{y} - f_y(x,y) \dd{x}$ 就是一个\textbf{微分形式}。 注意右边的两个环积分中, 分别可以把曲线分为两部分, 例如 $\oint f_x(x,y) \dd{y}$ 中可以把曲线延正方向 $y$ 坐标增加的部分 $I^+$ 和减小的部分 $I^-$。 此时可以直接用 $y$ 作为参数 $t$, $x = x(y)$, 那么
\begin{equation}
\oint f_x(x,y) \dd{y} = \int_{I^+} f_x(x(y),y) \dd{y} + \int_{I^-} f_x(x(y),y) \dd{y}~.
\end{equation}
在第一个积分中, $\dd{y}$ 是正的, 第二个中 $\dd{y}$ 是负的。
\addTODO{面积分, 每个曲面积分可以分为三个积分, 再分为两个方向的积分。 另外弄一个球坐标曲面积分的例子, 引入雅可比行列式。 此时两个面积分就可以合并为一个, 因为雅可比行列式的正负号会自动处理朝向。 此时要强调 $\dd{x_i}$ 的顺序很重要}

\subsection{微分形式}
\begin{definition}{微分形式}
在 $R^N$ 中, 令 $k < N$, 那么 \textbf{$k$-微分形式($k$-differential form)}, 简称 \textbf{$k$-形式}, 记为
\begin{equation}\label{eq_DForm_2}
\omega = \sum_{i_1,\dots,i_k} f_{i_1,\dots,i_k}(x_1,\dots,x_k) \dd{x_{i_1}}\dots\dd{x_{i_k}}~,
\end{equation}
其中求和中每个 $i$ 从 1 到 $k$ 变化。

定义一个 $k$ 维曲面, 用参数方程表示为 $x_i(u_1,\dots,u_k)$ ($i=1,\dots,N$)。 微分形式 $\omega$ 在该曲面上的积分为
\begin{equation}
\sum_{i_1,\dots,i_k} \oint f_{i_1,\dots,i_k}(x_1(u_1,\dots,u_k),\dots,x_k(u_1,\dots,u_k)) \frac{\partial(x_{i_1},\dots,x_{i_k})}{\partial(u_1,\dots,u_k)} \dd{u_1},\dots,\dd{u_k}~,
\end{equation}
其中 $\frac{\partial(\dots)}{\partial(\dots)}$ 是雅可比行列式\upref{JcbDet}, $\frac{\partial(\dots)}{\partial(\dots)}\dd{u_1},\dots,\dd{u_k}$ 是曲面上的有向表面积。
\end{definition}
根据雅可比行列式的性质, 当 $\dd{x_{i_1}}\dots\dd{x_{i_k}}$ 中有任意两个下标相同, 则对应的积分为零。 交换任意两个 $\dd{x_i}$, 积分取相反数。

\subsection{微分形式的微分}
定义\autoref{eq_DForm_2} 的微分为
\begin{equation}
\dd{\omega} = \sum_{i_1,\dots,i_k}\sum_{j=1}^N \pdv{x_j} f_{i_1,\dots,i_k}(x_1,\dots,x_k) \dd{x_j}\dd{x_{i_1}}\dots\dd{x_{i_k}}~.
\end{equation}

\addTODO{举一个斯托克斯定理的例子}
