% 自治系统解的特点
% keys 自治系统|解的性质
% license Xiao
% type Tutor

\pentry{基本知识(常微分方程)\upref{ODEPr}}
本节将说明自治系统(\autoref{def_ODEPr_2}~\upref{ODEPr})解的两个性质。一般的自治系统的对应的标准方程组(\autoref{def_ODEPr_1}~\upref{ODEPr})可写为
\begin{equation}\label{eq_AuSy_1}
y'_i=f'_i(y_1,\cdots,y_n),\quad i=1,\cdots,n~.
\end{equation}
采用矢量写法为
\begin{equation}
\dv{y}{x}=f(y)~,
\end{equation}
其中 $y=(y_1,\cdots,y_n),f=(f_1,\cdots,f_n)$。与“基本知识(常微分方程)\upref{ODEPr}”中类似,我们总假定 $f_i(y_1,\cdots,y_n)$ 及其一阶偏导数在其定义区间(记为 $\Delta$)上连续。约定:当出现关于指标 $i$ 的表达式而不指出其取值范围时,就代表对每个 $i$ 所能取的值表达式均成立。

\begin{theorem}{}
若 $y_i=\varphi_i(x)$ 是方程组\autoref{eq_AuSy_1} 的解,则 $y_i=\varphi^*_i(t)=\varphi_i(x+c)$ 也是方程组\autoref{eq_AuSy_1} 的解,其中 $c$ 是任意常数。
\end{theorem}
\textbf{证明:}由复合函数求导法则\upref{ChainR},成立
\begin{equation}
\varphi^*_i^'(t)=~.
\end{equation}



\textbf{证毕!}