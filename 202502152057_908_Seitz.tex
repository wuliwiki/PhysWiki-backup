% 弗雷德里克·塞茨(综述)
% license CCBYSA3
% type Wiki

本文根据 CC-BY-SA 协议转载翻译自维基百科\href{https://en.wikipedia.org/wiki/Frederick_Seitz}{相关文章}。

弗雷德里克·塞茨(1911年7月4日-2008年3月2日)是美国物理学家,固态物理学的先驱,也是气候变化否认者。塞茨曾担任洛克菲勒大学第4任校长(1968年-1978年),以及美国国家科学院第17任院长(1962年-1969年)。塞茨获得了国家科学奖章、NASA杰出公共服务奖等多项荣誉。

他在伊利诺伊大学厄本那-香槟分校创立了弗雷德里克·塞茨材料研究实验室,并在美国其他地方创立了多个材料研究实验室。[1][2] 塞茨还是乔治·C·马歇尔研究所的创始主席。[3]
\subsection{背景和个人生活 } 
塞茨于1911年7月4日出生在旧金山。他的母亲也来自旧金山,而以他命名的父亲则出生于德国。[4] 塞茨在高中最后一年中途从利克-威尔默丁高中毕业,随后进入斯坦福大学学习物理,并在三年内获得学士学位,[1] 于1932年毕业。[5] 他于1935年5月18日与伊丽莎白·K·马歇尔结婚。[6]

塞茨于2008年3月2日去世,享年96岁,地点为纽约。[7][8] 他留下了一子、三名孙子和四名曾孙。[7]