% 数学归纳法(高中)
% keys 数学归纳法|递推|归纳法
% license Usr
% type Tutor

\begin{issues}
\issueDraft
\end{issues}

\pentry{数列\nref{nod_HsSeFu}}{nod_aa25}


在数列部分,研究的是从自然数到实数的映射。尤其是数列的通项公式,似乎像一个准则,只要给出一个n就能得到一个结果。如果将数列的通项公式看作一个命题,(解释一下怎么看)。那么只有在n与对应的$a_n$时,才能得到结果是对的。这个判断过程可以抽象出来成为一种证明命题的方式。数学归纳法是一种证明数学命题的重要方法,主要用于证明某些关于正整数的命题。

数学归纳法是一种逻辑性强、步骤清晰的方法,不仅在数列中有广泛应用,还可以用于证明多项式、几何问题等。数学归纳法是一种强有力的数学工具,通过验证基础情况和递推过程,能够帮助我们解决大量规律性问题。它强调逻辑的严谨性和完整性,是高中数学中不可或缺的证明方法之一。

\subsection{从多米诺骨牌开始}

在介绍数学归纳法之前,先来看看一种常见的常见连锁效应——多米诺骨牌。(形容一下多米诺骨牌搭建和推倒的过程。)。设想一下,该如何用最简单的方法来描述这个过程?

一个直接的思路是,搭建的人员在搭建时,要保证每一张骨牌倒下时,都会带动下一个骨牌倒下。然后,在推倒时,保证自己能推倒第一块。这样,第一块倒下就会带动第二块,第二块倒下就会带动第三块,一直这样倒下去,那么所有的骨牌都会倒下。

这种“传递性”的现象,就是数学归纳法背后的原理。

\begin{definition}{数学归纳法}
对某个与自然数相关的命题$P(n)$,利用\textbf{数学归纳法}证明$P(n)$成立的过程分为三步:
\begin{enumerate}
\item 验证:检查命题对某个特定起始点(通常是最小值)是否成立,即命题$P(1)$或$P(0)$成立成立。
\item 假设:假设命题在某个自然数$k$成立,即假设 $n = k$ 时命题$P(k)$成立。
\item 推导:证明命题在归纳假设的前提下,可以推导出$n = k+1$ 时命题$P(k+1)$也成立。
\end{enumerate}
如果以上三步都完成,就可以得出结论:该命题对所有自然数$n$ 都成立。
\end{definition}

可以看到,验证对应的就是第一块能推倒,假设就是指某张骨牌倒下,而推导得出的结论就是下一张骨牌也会倒下。

注意事项与常见误区
	1.	遗漏基础验证
如果未验证初始点 P(1),推导将失去起点支撑。
	2.	归纳推导不严谨
推导步骤必须从归纳假设出发,完整证明 P(k+1)。
	3.	适用范围不明确
在使用归纳法前,需明确命题的适用范围。

尽管通常称为数学归纳法,但本身只是叫做归纳法,“数学”是与其他领域分隔开。虽然数学归纳法名字中有“归纳”,但是数学归纳法并非逻辑上不严谨的\aref{归纳推理法}{sub_HsLogi_1},它属于完全严谨的演绎推理法。数学归纳法是一种从特殊到一般的递归式证明,与反证法等方法相比,更适用于递推关系明显的问题。
数学归纳法是一种公理模式存在,如果满足则判断正确。也就是说它本身是不可以证明的。其实它是定义自然数的公理之一,也就是说,只要有自然数存在的场合,就天然存在数学归纳法。

效果:
\begin{itemize}
\item 保证自然数的完备性:确保性质对所有自然数都成立,不必逐一检查每一个自然数。
\item 建立递归定义的基础:提供严格的证明这些递归定义正确性的方法。
\end{itemize}

\begin{example}{证明:对于任意正整数$n$都有$a^n-b^n=\left(a-b\right)\left(a^{n-1}+a^{n-2}b+\cdots+b^{n-1}\right)$。}
证明:

当 $n = 1$ 时,$S_1 = 1$。公式 $\frac{1(1+1)}{2} = 1$ 成立。假设对于 $n = k$,命题成立,即:
\begin{equation}
S_k = 1 + 2 + 3 + \cdots + k = \frac{k(k+1)}{2}~.
\end{equation}
需要证明 $n = k+1$ 时命题也成立,根据假设,将 $S_k$ 代入可得:
\begin{equation}
\begin{aligned}
S_{k+1} &= S_k + (k+1)\\
&= \frac{k(k+1)}{2} + (k+1)\\
&= (k+1)\left(\frac{k}{2} +1\right)\\
&= \frac{(k+1)(k+2)}{2}~.
\end{aligned}
\end{equation}
这与命题的形式一致。综上,利用数学归纳法,可以证明命题对所有正整数 $n$ 都成立。
\end{example}

\begin{example}{证明 $2^n > n$ 对 $n \geq 1$ 成立。}
证明:

当 $n = 1$ 时,$2^1 = 2 > 1$,成立。假设 $2^k > k$ 成立。需要证明 $n = k+1$ 时命题也成立。

首先,当 $k \geq 1$ 时,$2k \geq k+1$。根据假设,将 $2^k > k$代入可得:
\begin{equation}
\begin{aligned}
2^{k+1} &= 2 \cdot 2^k\\
&> 2k\\
&\geq k+1~.
\end{aligned}
\end{equation}
与命题的形式一致。综上,利用数学归纳法,可以证明命题对所有正整数 $n$ 都成立。
\end{example}

\begin{example}{证明:对任意正整数$n$,$9^{n} - 1$ 能被 $8$ 整除。}
证明:

当 $n = 1$ 时:$9^1 - 1 = 9 - 1 = 8$。显然$8\text{÷}8=1$,命题在 $n = 1$ 时成立。假设当 $n = k$ 时,命题成立,这意味着存在整数 $m$,使得:
\begin{equation}
9^k - 1 = 8m~.
\end{equation}
当 $n = k+1$ 时,代入假设,有:
\begin{equation}
\begin{aligned}
9^{k+1} - 1 &= 9\cdot9^{k} - 1\\
&= 9\cdot(8m+1)- 1\\
&= 8\cdot9m+9- 1\\
&= 8\cdot(9m+1)~.
\end{aligned}
\end{equation}
显然,$9^{k+1} - 1$ 是 $8$ 的倍数,所以能被 $8$ 整除。

由数学归纳法可知,对所有正整数$n$,$9^{n} - 1$ 能被 $8$ 整除。
\end{example}


数学归纳法把复杂的证明过程转化为了对最终结果的猜想。由此,也使得在高中阶段绕过复杂困难的证明,直接使用其它方法得到结果并使用数学归纳法给出证明成为可能。

数学归纳法并不只在做题中有用,通常想要证明一个命题对实数都成立,可以先利用数学归纳法证明它对自然数成立,然后再扩展至有理数,最后利用极限再扩展至实数。因此,数学归纳法是很多命题的源头。

所有与自然数有关的命题都可以通过数学归纳法来证明。读者可以自行尝试一下利用数学归纳法证明“\enref{恒等式与恒成立不等式}{HsIden}”中给出的表达式。

数学归纳法的实际应用

数学归纳法在以下方面有广泛应用:
	1.	数列公式
如等差数列、等比数列的通项公式与求和公式。
	2.	整除性问题
证明某些数字规律,如 n^3 - n 能被 6 整除。
	3.	几何与组合
如多边形内角和公式,或排列组合的递归性质。
	4.	递归算法
验证算法的正确性,确保每一步都符合规律。



\subsection{*数学归纳法的变式}

下面介绍的方法都是可以从数学归纳法推知的其他类型的数学归纳法。因此,可以看作这些归纳法都是在数学归纳法的前提下给出了一些其他的条件来辅助。由于使用了一些条件,使得证明过程更简单。这些内容在高中阶段完全不涉及,仅作开阔视野。

\subsubsection{从$N$开始的数学归纳法}
比如从 n = 2 或 n = 0 起步。

\subsubsection{强归纳法}

:假设命题对多个前面情况成立,再推导出后续。

\subsubsection{逆归纳法}
