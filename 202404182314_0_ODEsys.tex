% 一阶线性常微分方程组(简明微积分)
% keys 多元微积分|微分方程|常微分方程|微分方程组|一阶线性常微分方程组|矩阵|Split Operator
% license Xiao
% type Tutor

\begin{issues}
\issueOther{和一阶常系数线性微分方程组(常微分方程)\upref{ODEb3}内容重复了}
\end{issues}

\pentry{一阶线性微分方程\nref{nod_ODE1}}{nod_d622}

\subsection{常系数情况}
形式为
% \addTODO{一般一阶线性微分方程组如何化成该形式?}
\begin{equation}\label{eq_ODEsys_1}
\dv{\bvec v}{t} = \mat A \bvec v~
\end{equation}
的一阶偏微分方程组的解析解(先假设 $\mat A$ 为常矩阵)为
\begin{equation}\label{eq_ODEsys_2}
\bvec v(t) = \exp(\mat At) \bvec v(0)~.
\end{equation}
其中矩阵的指数函数有类似于泰勒级数\upref{Taylor}定义
\begin{equation}\label{eq_ODEsys_4}
\exp(\mat At) = 1 + \mat At + \frac1{2!} (\mat At)^2 + \dots~
\end{equation}
代入即可验证\autoref{eq_ODEsys_1}。这类似于一阶常系数常微分方程\upref{ODEa1}的解,即 $\mat A$ 是一个常数而不是矩阵的情况。

但\autoref{eq_ODEsys_4} 不方便直接计算。此时如果 $\mat A$ 可以对角化为(链接未完成,$\mat U$ 是本征列向量排成的矩阵, $\mat\Lambda$ 是对应本征值 $\lambda_1, \lambda_2, \dots$ 排成的对角矩阵)
\begin{equation}
\mat A = \mat U\mat\Lambda\mat U^{-1}~,
\end{equation}
代入\autoref{eq_ODEsys_4} 就有
\begin{equation}\ali{
\exp(\mat At) &= \mat U\mat U^{-1} + \mat U\mat\Lambda\mat U^{-1} t + \frac1{2!} (\mat U\mat\Lambda\mat U^{-1} t)^2 + \dots\\
&= \mat U\qty(1 + \mat\Lambda t + \frac1{2!} (\mat\Lambda t)^2 + \dots)\mat U^{-1}\\
&= \mat U\exp(\Lambda t)\mat U^{-1}\\
&= \mat U \pmat{\E^{\lambda_1 t} & 0 & \dots\\ 0 & \E^{\lambda_2 t} &\dots\\ \vdots & \vdots & \ddots} \mat U^{-1}~.
}\end{equation}
代入\autoref{eq_ODEsys_2} 就是方程组的解。

\subsection{含时系数情况的形式解}
当\autoref{eq_ODEsys_1} 中的 $\mat A$ 是 $t$ 的函数 $\mat A(t)$ 时, 我们可以取微小时间步长 $\Delta t$, 在每个 $\Delta t$ 内近似认为 $\mat A(t_i)$ 为常数, 再取极限
\begin{equation}
\bvec v(t) = \lim_{\Delta t \to 0} \prod \exp\qty[\mat A(t_i)\Delta t] \bvec v(0)~,
\end{equation}
如果两个矩阵 $\mat P, \mat Q$ 对易, 就有
\begin{equation}
\exp(\mat P)\exp(\mat Q) = \exp(\mat P + \mat Q)~,
\end{equation}
但一般来说 $\mat A(t_i)$ 之间不对易, 所以我们定义一个时间排序算符 $\Q{\mathcal T}$ 使例如
\begin{equation}
\Q {\mathcal T} [\mat A(t_1) \mat A(t_3) \mat A(k_2)] = \mat A(t_3) \mat A(t_2) \mat A(t_1) \qquad ( t_1 < t_2 < t_3 )~,
\end{equation}
这样通解在形式上就可以记为
\begin{equation}\label{eq_ODEsys_3}
\bvec v(t) = \Q {\mathcal T} \exp(\int_0^ t \mat A(t')\dd t') \bvec v(0)~,
\end{equation}
然而这么做对于数值计算并没有太大意义。

\subsection{厄米矩阵}
如果 $\mat H$ 是厄米矩阵, 那么 $\E^{\I \mat H t}$ ($t\in \mathbb R$)是一个酉矩阵(幺正矩阵)\upref{UniMat}。 所以 $\bvec v(t)$ 的模 $\abs{\bvec v(t)} = \sum \abs{v_i}^2$ 将不随时间变化。

证明: 把\autoref{eq_ODEsys_4}(??) 两边取厄米共轭得
\begin{equation}
(\E^{\I \mat H t})\Her = \E^{-\I \mat H\Her t} = \E^{-\I \mat H t}~,
\end{equation}
由于 $[H,H] = 0$, 有 $(\E^{\I \mat H t})\Her \E^{\I \mat H t} = \E^{-\I \mat H t}\E^{\I \mat H t} = \mat I$。证毕。

\subsection{数值计算}
如果矩阵 $\bvec M$ 是厄米矩阵, 则可以先做对角化 $\mat M = \mat U \mat \Lambda \mat U\Her$, 其中 $\mat \Lambda$ 是对角矩阵, $\mat U$ 是酉矩阵。 这样就有
\begin{equation}
\exp(\mat M) = \mat U \mat U\Her + \mat U\mat \Lambda\mat U\Her + \mat U\frac1{2!} \mat \Lambda^2 \mat U\Her + \dots = \mat U \exp(\mat \Lambda) \mat U\Her~,
\end{equation}
由于对角矩阵相乘等于每个对角元分别相乘, 把 $\mat\Lambda$ 的每个矩阵元求指数函数就可以得到 $\exp(\mat \Lambda)$。 这样做可以减少计算量。

事实上, 以上做法相当于分离变量, 当 $\mat A$ 是厄米矩阵时, 令 $\bvec v(t) = f(t)\bvec u$, 代入方程得 
\begin{equation}
\frac{f'(t)}{f(t)} \bvec u = \mat A \bvec u~,
\end{equation}
由于 $\mat A$ 和 $\bvec u$ 都不含时, 所以可以令
\begin{equation}\ali{
&\mat A \bvec u = \lambda \bvec u~,\\
& f(t)' = \lambda f(t)~.
}\end{equation}
其中第一个方程是 $\mat A$ 的本征方程, 解为 $N$ 个本征矢 $\bvec u_i$ (即 $\mat U$ 的第 $i$ 列) 和 $N$ 个本征值 $\lambda_i$ (即 $\mat \Lambda$ 的第 $i$ 个对角元)。 第二条方程的解为 $f(t) = \exp(\lambda t)$。 所以方程的通解为
\begin{equation}
\bvec v(t) = \sum_{i = 1}^N c_i f_i(t) \bvec u_i = \sum_{i = 1}^N \mat u_i\Her  \bvec v(0) \exp(\lambda_i t) \bvec u_i = \mat U \exp(\Lambda t) \mat U\Her \bvec v(0)~.
\end{equation}

在此基础上使用 Lanczos 算法\upref{Lanc}可以进一步提高效率。


\subsection{算符拆分}
有时候我们希望可以在上述计算中把 $\mat A$ 写成几个矩阵的和的形式(以两个为例) $\mat A = \mat B + \mat C$。 当 $\mat B$ 和 $\mat C$ 对易时显然有
\begin{equation}
\exp(\mat At) = \exp(\mat Bt)\exp(\mat Ct)~.
\end{equation}
在程序中这么做可能可以进一步提高速度\footnote{例如二维波函数的动能算符 $T = T_x + T_y$}。 如果 $\mat B$ 和 $\mat C$ 不对易, 严格来说上式不成立, 但可以证明 $t \to 0$ 时近似成立
\begin{equation}\ali{
& \quad \exp(\mat Bt)\exp(\mat Ct) \\
& = \qty(1+\mat Bt+\frac{1}{2!}\mat B^2t^2 + \dots)\qty(1+\mat Ct+\frac{1}{2!}\mat C^2t^2 + \dots)\\
& = 1 + (\mat B+\mat C)t + \frac{1}{2!}\qty(\mat B^2 + \mat C^2 + 2\mat B\mat C) t^2 + \dots\\
& = \exp(\mat At) + \order{t^2}~,
}\end{equation}
这里的 $\order{t^2}$ 是由于第二个等号后面是 $2\mat B\mat C$ 而不是 $\mat B\mat C + \mat C\mat B$。

% 未完成: 但我一直不明白为什么 $exp(-iH_0t/2)exp(-iH_It)exp(-iH_0t/2)$ 可以达到 $\order(t^3)$ 的精度。
