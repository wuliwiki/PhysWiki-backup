% 点电荷模型与连续电荷模型的缺陷
% 点电荷模型|连续电荷模型|粗粒化|有效场论
\pentry{电场的能量\upref{EEng},电荷守恒、电流连续性方程\upref{ChgCsv}}

\subsection{点电荷模型与发散困难}
在计算电荷的\textbf{固有能}\footnote{参考电场的能量\upref{EEng}}时,我们发现计算结果是发散的:
\begin{equation}\label{elecdb_eq1}
\begin{aligned}
E&=\int \dd[3]{\bvec x}\frac{\epsilon_0}{2} \left(\frac{e}{4\pi\epsilon_0 |\bvec x|^2}\right)^2=\frac{e^2}{2(4\pi)^2\epsilon_0}\int_0^\infty 4\pi r^2\dd r \frac{1}{r^4}\\
&=\frac{e^2}{8\pi\epsilon_0} \int_0^\infty \frac{1}{r^2}\dd r
\end{aligned}
\end{equation}
发散的积分区域是 $r\rightarrow 0$ 处.如果我们取一个积分下限的截断,从 $1/\Lambda$ 积分到 $\infty$,我们将得到
\begin{equation}
E=\frac{e^2}{8\pi\epsilon_0}\int_{1/\Lambda}^{\infty} \frac{1}{r^2}\dd r=\frac{e^2}{8\pi\epsilon_0}\Lambda
\end{equation}
结果关于 $\Lambda$ 是线性发散的.如果\textbf{假设}经典电磁学的公式成立(麦克斯韦方程组,电荷电流守恒方程),那么一定能够得到一个电磁场的能动张量\upref{EMtens},它反映了经典电磁场中能量、动量的流动.因此电荷的固有能实际上是有意义的物理量,它反映了空间中一个电荷所激发的能量,根据相对论的质能公式,\textbf{静止参考系下电荷在全空间所激发的能量的总和实际上就包含在我们所测得的电荷的质量内}.那么为什么会存在发散呢?或者说,我们得到的这一线性发散的结果是否合理呢?

答案是,发散苦难意味着我们的理论(我们的物理模型)是存在缺陷的.我们采取了过多理想化的假设,例如将电子视为点电荷,并且将库仑定律(我们利用库仑定律计算了电场)对任意小距离仍然成立.而这实际上是一个过于自信的假设.人们的观测手段总是有限的,我们没有任何根据去确定一个简单的公式能够对任意小、任意大尺度的物理都成立.也正因此,在上面的计算中,当 $r\rightarrow 0$ 时理论出现了发散,这意味着 $r\rightarrow 0$ 时我们的经典电磁学,或者我们的点电荷模型出了问题.

下面我们来看一个例子,从另一个角度理解点电荷模型的缺陷.
\begin{example}{辐射阻尼}
考察一个垂直于直面的磁场中,作圆周运动的电子.根据带电粒子的辐射\upref{chgrad}中的\textbf{拉莫尔公式},由于电子在做圆周运动时加速度非零,所以电子向外电磁辐射而自身损失了能量.在词条带电粒子的辐射\upref{chgrad}中,我们是通过能量守恒得到了这一结果.然而洛伦兹力公式在这里失效了.是什么力导致了电子减速呢?
\end{example}
垂直于直面的磁场肯定不会使电子减速,因为它总是垂直于电子的运动方向.而如果真正地考虑电子近邻区域的电场与磁场,如果它是发散的,那么洛伦兹公式的合法性就更成了问题.可以猜测,电子减速是因为带电粒子由于自身的辐射对自身运动的影响,这一部分影响被称为\textbf{辐射阻尼}.而在点电荷模型中,粒子自身辐射对自己的影响是一个“未知量”,如果我们试图在点电荷模型的框架内去计算辐射阻尼,会发现它在数学上甚至不是良定义的.正因为 $\delta$-函数的存在,导致经典电磁理论将出现一系列\textbf{发散解}.基于这个层面的考虑,

从上面的讨论中我们看到,一切矛头都指向了点电荷模型,或者说我们对微观尺度的物理一无所知.历史上许多物理学家试图解决这个问题,并提出了各种各样的模型.Max Abraham 和 Lorentz 假定了带电粒子的电荷分布仅仅存在于一个尺度为 $a$ 的一个范围内并且是球对称的;假定了带电粒子是
刚性的,因此它的电流密度为:$J\bvec (\bvec x, t) =\rho(\bvec x, t)\bvec v(t)$.在这样的模型下,电子自能不再发散,电子自身辐射对自己有影响,而经过具体得到的电子受力公式区别于洛伦兹力公式,将出现一个辐射阻尼项:
\begin{\begin{equation}

\end{equation}
}


从更现代的观点上看,经典电磁学实际上是量子电动力学的有效场论,它只在一定的尺度、一定的能标下成立;而量子电动力学(QED)也是一个低能有效场论,在 QED 的计算中,电子的自能是对数发散的,因此我们需要对理论进行正规化,取一个动量截断 $\Lambda$ ,则自能 $\sim \log \Lambda$.一个有趣的结果是,量子电动力学中随着动量的提高,自能发散程度比经典电磁学要慢.

\subsection{连续电荷模型与粗粒化}
在连续电荷模型中,我们采用点荷密度和电流密度分布的函数 $\rho,\bvec J$.这实际上是一个理想化的模型.例如当我们在考虑一个中性的材料,虽然在宏观角度上看它是不带电的(那么根据连续电荷模型,$\rho=0,\bvec J=0$,我们计算得到的电场能量\upref{EEng}应当为 $0$),然而当我们真正地去考察其微观结构,我们会发现材料是由许多带正电的离子、许多被离子束缚的电子、或自由电子组成的.当我们真正计算这些微观层面上电荷间的相互作用,我们会发现他们的累加不为 $0$.这一部分能量通常被我们称为\textbf{化学能},它是材料的一个属性.

因此在连续电荷模型中,我们是从粗粒化的角度去考察我们的物理系统,所以在距离很小的两点荷之间的相互作用的信息并没有被我们考虑进来.在粗粒化的过程中我们实际上\textbf{丢失了那些微观的物理信息},但尽管如此,许多物理量的计算仍然可以采用经典电磁学的公式,电荷守恒、电流连续性方程\upref{ChgCsv}、麦克斯韦方程组(介质)\upref{MWEq1}在\textbf{粗粒化的层面上}仍然成立.也就是说,我们不仅对电荷分布、电流分布粗粒化,还对电场、电极化强度、磁感应强度粗粒化.再考察这些宏观物理量之间的关系式,它们将满足介质中的麦克斯韦方程组,其形式与真空中的麦克斯韦方程组是相当类似的,只有几个系数上有差异.即使我们承认我们不知道那些微观的物理信息,我们仍然能够在宏观层面上得到有意义的物理结果,这正是\textbf{有效场论}的精髓.