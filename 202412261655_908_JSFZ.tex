% 计算复杂性理论(综述)
% license CCBYSA3
% type Wiki

本文根据 CC-BY-SA 协议转载翻译自维基百科\href{https://en.wikipedia.org/wiki/Computational_complexity_theory}{相关文章}。

在理论计算机科学和数学中,计算复杂性理论专注于根据资源使用情况对计算问题进行分类,并探索这些分类之间的关系。计算问题是由计算机解决的任务。一个计算问题可以通过机械地应用数学步骤(如算法)来解决。

如果一个问题的解决需要大量资源,无论使用何种算法,都被视为固有的困难问题。该理论通过引入计算模型来正式化这种直觉,以研究这些问题并量化它们的计算复杂性,即解决问题所需的资源量,如时间和存储空间。还使用其他复杂性度量,如通信量(用于通信复杂性)、电路中的门数(用于电路复杂性)以及处理器数量(用于并行计算)。计算复杂性理论的一个重要作用是确定计算机能够做什么以及不能做什么的实际限制。P与NP问题,作为七大千年奖问题之一,是计算复杂性领域的一部分。

在理论计算机科学中,与计算复杂性紧密相关的领域有算法分析和可计算性理论。算法分析与计算复杂性理论之间的一个关键区别是,前者致力于分析特定算法解决问题所需的资源量,而后者则提出一个更为一般的问题,即所有可能用于解决同一问题的算法。更精确地说,计算复杂性理论试图对能够或不能在适当限制的资源下解决的问题进行分类。反过来,施加对可用资源的限制是计算复杂性与可计算性理论的区别所在:后者理论探讨的是哪些类型的问题原则上可以通过算法解决。


\subsubsection{问题实例}  
一个计算问题可以视为一个无限的实例集合,每个实例都有一组(可能为空)的解。计算问题的输入字符串称为问题实例,不应与问题本身混淆。在计算复杂性理论中,问题指的是待解决的抽象问题。与此相对,问题的一个实例是一个相对具体的表述,可以作为决策问题的输入。例如,考虑素数测试问题。实例是一个数字(例如,15),如果该数字是素数,解答是“是”,否则是“否”(在这种情况下,15不是素数,答案是“否”)。换句话说,实例是问题的特定输入,解答是与该输入对应的输出。

为了进一步突出问题和实例之间的区别,考虑旅行商问题的决策版本实例:是否存在一条最多2000公里的路线,经过德国的15个最大城市?对于这个特定问题实例的定量答案,对解决问题的其他实例帮助不大,例如询问一条在米兰所有景点之间,且总长度不超过10公里的环路。因此,复杂性理论关注的是计算问题,而不是特定的问题实例。
\subsubsection{表示问题实例}  
在考虑计算问题时,问题实例通常是一个由字母表组成的字符串。通常,字母表被视为二进制字母表(即{0, 1}集合),因此这些字符串是位字符串。如同实际计算机一样,必须对除位字符串外的数学对象进行适当的编码。例如,整数可以用二进制表示,图形可以通过其邻接矩阵直接编码,或者通过将其邻接表编码为二进制来表示。

尽管一些复杂性理论定理的证明通常假设某种具体的输入编码选择,但讨论通常保持足够抽象,以独立于编码选择。这可以通过确保不同的表示方法可以高效地相互转换来实现。