% 环和域
% 实数|域|加|乘
% 概念介绍基本完成

\pentry{环\upref{Ring}}

在群\upref{Group}的基础上,我们可以定义更复杂的对象,环和域.简单来说,环和域各自有两个运算,通常称为加法和乘法,其中加法必须构成一个阿贝尔群.由于多了一个运算,我们还需要考虑两个运算之间的复合关系,因此还额外引入了一个性质,即乘法对加法的分配性.

\subsection{比群多了一个运算: 环和域}

借用群的定义,我们可以简单地列出定义环和域所用到的公理,注意对比它们的异同:

\begin{definition}{}
一个\textbf{环(ring)}是一个配备了加法 $+$ 和乘法 $\times$ 两种运算的集合 $R$,记作 $(R, +, \times)$,并且满足以下公理:
\begin{enumerate}
    \item $(R, +)$ 是一个\textbf{交换群}\upref{Group}.其单位元记为 $0$.
    \item 如果记 $R^*$ 为 $R-\{0\}$,那么 $(R, \times)$ 构成一个\textbf{幺半群}.其单位元记为 $1$\footnote{幺半群意味着环的定义里包含乘法单位元,也就是幺元.这是有争议的,有的数学家从形式上定义,不要求环有幺元,而把有幺元的特别称呼为“幺环”;但是有的数学家认为,我们很少研究不含幺元的环,所以不如直接定义环必须有幺元,简化讨论.本书使用后一种传统,但是有时也会为了强调而使用“幺环”的术语,也就是说我们会将“环”和“幺环”视作等价的术语来使用.}.
    \item 乘法对加法满足分配律
\end{enumerate}
\end{definition}

\begin{definition}{}
一个\textbf{域(field)}是一个配备了加法 $+$ 和乘法 $\times$ 两种运算的集合 $\mathbb{F}$,记为 $(\mathbb{F}, +, \times)$,并且满足以下公理:

\begin{enumerate}
    \item $(\mathbb{F}, +)$ 是一个\textbf{交换群}.其单位元记为 $0$.
    \item 如果记 $\mathbb{F}^*$ 为 $\mathbb{F}-\{0\}$,那么 $(\mathbb{F}^*, \times)$ 构成一个\textbf{交换群}.其单位元记为 $1$\footnote{交换群意味着乘法是交换的,这也有争议.少数数学家定义的环不要求乘法交换性,而把有交换性的称为交换域.但是绝大多数文献的定义中都默认域是有乘法交换性的,而把不具有乘法交换性的类似对象称为“体”或者“(可)除环”.}.
    \item 乘法对加法满足分配律
\end{enumerate}
\end{definition}


以上为了简洁考虑,我们直接引用了群和幺半群的概念来定义环和域.但我觉得对于初学者,为了便于消化、记忆环和域的特点,有必要把它们的定义\textbf{重新展开为}以下详细的性质:

\begin{definition}{}
一个\textbf{环(ring)}是一个配备了加法 $+$ 和乘法 $\times$ 两种运算的集合 $R$,记作 $(R, +, \times)$,并且满足以下公理:
\begin{enumerate}
    \item 加法 $+$ 是封闭的.
    \item 加法 $+$ 具有结合性.
    \item 加法 $+$ 存在单位元,记为 $0$.
    \item 任意元素在加法 $+$ 下存在逆元.
    \item 加法 $+$ 是交换的.
    \item 乘法对加法满足分配律:对于任意 $x, y, z\in R$,有 $x\times(y+z)=x\times y+x\times z$.
    \item 乘法 $\times$ 是封闭的.
    \item 乘法 $\times$ 具有结合性.
    \item 乘法 $\times$ 存在单位元,记为 $1$.
\end{enumerate}
\end{definition}

\begin{definition}{}
一个\textbf{域(field)}是一个配备了加法 $+$ 和乘法 $\times$ 两种运算的集合 $\mathbb{F}$,记为 $(\mathbb{F}, +, \times)$,并且满足以下公理:

\begin{enumerate}
    \item 加法 $+$ 是封闭的.
    \item 加法 $+$ 具有结合性.
    \item 加法 $+$ 存在单位元,记为 $0$.
    \item 任意元素在加法 $+$ 下存在逆元.
    \item 加法 $+$ 是交换的.
    \item 乘法对加法满足分配律:对于任意 $x, y, z\in \mathbb{F}$,有 $x\times(y+z)=x\times y+x\times z$.
    \item 乘法 $\times$ 是封闭的.
    \item 乘法 $\times$ 具有结合性.
    \item 乘法 $\times$ 存在单位元,记为 $1$.
    \item 任意元素在乘法 $\times$ 下存在逆元.
    \item 乘法 $\times$ 是交换的.
\end{enumerate}
\end{definition}

比较下来,域的定义只比环的定义多了第10和第11条,一个要求乘法可逆(即可以做除法),一个要求乘法交换;其它部分则完全相同.两个定义中,第1到第5条定义了加法的性质,使得环或域在加法下构成交换群;第5条定义了两个运算间的关系,即分配律;剩下的则定义了乘法的性质.可以看到,域比环更具体,环比域更抽象.当然了,群最抽象,环和域都是群的具体例子.

一般来说,为了方便,我们通常会省去乘法符号,把 $r\times s$ 写为 $rs$.元素 $r$ 的加法逆元记为 $-r$,这样就可以把 $r+(-s)$ 记为 $r-s$.如果元素 $r$ 有乘法逆元,那么我们把它的乘法逆元记为 $r^{-1}$,于是就有 $rr^{-1}=r^{-1}r=1$,像我们在群论初步中看到 $xx^{-1}=x^{-1}x=e$ 一样.

要注意的是,环的乘法只有在去掉加法单位元 $0$ 的时候才能构成幺半群或者群,这是因为任何元素乘以 $0$ 还是 $0$,就像小学就学过的实数乘法一样.这是由\textbf{分配律}导致的,我们把它写为以下定理:

\begin{theorem}{}
设 $R$ 是一个环,$0$ 是其加法单位元.则对于任意 $r\in R$,有 $r\times 0=0\times r=0$.
\end{theorem}
\textbf{证明}:

$r\times 0=r\times (r-r)=r\times r-r\times r=0$.

\textbf{证毕}.



我们可以简单地理解为,\textbf{环}是“能进行\textbf{加减乘}运算的集合”,其中\textbf{乘法还不一定交换};\textbf{域}则是能“进行\textbf{加减乘除}运算的集合”,而且\textbf{加法和乘法都可以交换}.

最常见的环是\textbf{整数环}.全体整数构成的集合,配备通常的加法、乘法运算后,构成一个环,并且还是交换环.实函数也可以用来构成环:如果 $f, g$ 都是从实数到实数的映射,那么对于任意实数 $x$,定义 $(f+g)(x)=f(x)+g(x)$ 以及 $(f\times g)(x)=f(x)g(x)$,这样,\textbf{全体实函数的集合}配备如此定义的 $+$ 和 $\times$ 运算,就构成一个交换环.

非交换环的例子也可以用函数构造出来,方法是把以上定义的函数环中的\textbf{乘法}替换为\textbf{复合}运算:$(f\circ g)(x)=f(g(x))$.这样,由于映射的复合一般不交换\footnote{比如说,对于函数 $f=x+1$ 和函数 $g=x^2$,我们有 $f\circ g=x^2+1$,但 $g\circ f=(x+1)^2=x^2+1+2x\not=f\circ g$.},于是 $(\{\text{全体实函数}\}, +, \circ)$ 就是一个非交换环.除此之外,我们将来会遇到的\textbf{矩阵}等对象也能构成非交换环,但是矩阵乘法的非交换性本质上相当于映射复合的非交换性,因为矩阵可以用来表示线性变换,此时矩阵的乘法对应的是线性变换的复合.不用担心,我们会在线性代数章节里再讨论这些.

最常见的域就是有理数域和实数域.在有理数集合和实数集合上配备通常的加法、乘法运算后,都构成域,称作有理数域和实数域.

从定义很容易看出域一定是环,但上面所举的环的例子中,都有乘法不可逆的元素,因此它们都只是环,不是域.比如说,整数环里 $2$ 这个元素,就不存在整数的乘法逆元;实函数环里 $f(x)=x$ 也不存在乘法逆元(此处取交换的环的那个例子的定义,用实数的乘法导出函数的乘法),因为它有零点 $f(0)=0$,导致不存在实函数 $g$ 使得 $(f\times g)(x)$ 恒为 $1$.




\subsection{体和素域的概念}

\begin{definition}{体}\label{field_def1}
给定一个集合 $H$,如果这个集合中定义了两个运算,加法 “+” 和乘法 “$\times$”,并且 $H$ 对于加法构成一个阿贝尔群,而 $H-\{0\}$ 构成群($0$ 为 $H$ 加法群的单位元),并且乘法对加法满足分配律,即对于任何 $a, b, c\in H$,满足 $a\times(b+c)=a\times b+a\times c$,那么我们称 $(H, +, \times)$ 构成一个\textbf{体(skew field)},或称\textbf{可除环(division ring)}、\textbf{除环}.
\end{definition}

像在环论中省略乘法的符号一样,我们也常常把体中的“$\times$”符号省略,比如说,将分配律表示为 $a(b+c)=ab+ac$.

简单来说,体就是能进行加减乘除的一个集合,其中加法是可交换的,乘法却不一定.由于乘法不一定交换,这就使得除法运算相对复杂,但我们在此不过多展开.

\begin{example}{四元数体}
四元数\upref{Quat}词条中所定义的全体四元数构成的集合,配上所定义的加法和乘法,构成一个体,称为\textbf{四元数体}.
\end{example}

\begin{example}{矩阵体}
某个域上的全体 $n$ 阶可逆矩阵,配上矩阵加法和乘法,构成一个体.
\end{example}

类比子群和子环的定义,我们也可以定义子体.

\begin{definition}{子体}
体 $H$ 的子集 $S$ 如果满足其对已有的加法和乘法仍然构成体,那么称 $S$ 是 $H$ 的一个\textbf{子体}.
\end{definition}
子体的概念引出了以下关键概念.
\begin{definition}{素体与素域}
一个体的子体之交显然还是一个子体,因此每个体都存在唯一的非平凡不可约子体,称为这个体的\textbf{素体(prime field)}或\textbf{素域}.
\end{definition}

素体又被称作素域的原因是,素体必然是 $\mathbb{Z}_p$ 或 $\mathbb{Q}$,其中 $p$ 是素数.这两种体都是域.

那么我们刚才讨论中的域是什么呢?

\begin{definition}{域}
给定一个体 $\mathbb{F}$,如果 $\mathbb{F}$ 的乘法满足交换律,那么称其为一个\textbf{域(field)}.
\end{definition}

一个域的子体总是乘法可交换的,因此也都称为域的\textbf{子域(sub-field)}.

数学界主流将域视作乘法可交换的体,因此当谈到域时,总是认为乘法可交换.少数数学家会把我们以上定义的体称为域,而将我们定义的域称为交换域,但这并不是主流,因此本书使用以上定义.

\begin{example}{数域}
包含整数集的域,称为数域.最重要的数域有三个,有理数域 $\mathbb{Q}$,实数域 $\mathbb{R}$ 和复数域 $\mathbb{C}$,其中 $\mathbb{Q}$ 是最小的数域,也就是说任何数域都包含它;实数域是有理数域的完备化,意味着有理数域中的收敛数列都收敛于某个实数;复数域是最大的数域,也就是说任何数域都是复数域的子域.

注意,$\mathbb{Z}_p$ 并不是 $\mathbb{Z}$ 的子域,因为在 $\mathbb{Z}_p$ 中,$(p-1)+1=0$,而这在 $\mathbb{Z}$ 中是不可能的.
\end{example}

素域的概念对于描述任意的域是关键,以至于我们用素域定义了一个概念,称作域的特征:

\begin{definition}{域的特征}
给定域 $\mathbb{F}$,如果它所包含的素域是 $\mathbb{Z}_p$,那么称 $\mathbb{F}$ 的\textbf{特征(character)}是 $p$;如果它的素域是 $\mathbb{Q}$,那么称它的特征是 $0$.
\end{definition}

由定义可见,特征的值取素数或者 $0$,这个值在很大程度上决定了域的代数性质.








\subsection{}








