% 北太天元插件开发笔记
% license Xiao
% type Note

\begin{issues}
\issueDraft
\end{issues}

北太天元是国产的 Matlab 替代品。 目前还不是很完善,bug 较多。 \href{https://disk.pku.edu.cn/#/link/0EE816030B2E1A296FB029890388CD02?gns=41CD4566C310478980471F9C52E8C51B\%252F5B93BC04B04148DAA3CA90083970528B\%252FE869B8A40C754CE4B4033134D7DE6E22}{北大内部下载链接}。 \href{https://www.baltamatica.com/community/sposts/detail/21af2f4f-9e66-5f90-453f-b3cf26ab3833.html}{开发者预览下载}。

\begin{itemize}
\item 对象临时托管到 BALTAM 中时,必须从 \verb|baltam::extern_obj_base| 继承
\begin{lstlisting}[language=cpp, caption=extern\_obj.h节选]
struct extern_obj_base {
    virtual extern_obj_base* dup() const { return nullptr; }
    virtual ~extern_obj_base() {};
    virtual std::string to_string() const { return ""; }
    virtual std::string classname() const { return "extern object"; };
    int sid; ///< 整数,用于存储内核自动分配的外部类 ID
};
\end{lstlisting}
\end{itemize}

\subsection{插件的 main.cpp}
\begin{itemize}
\item \verb|#define PLUGIN_NAME "vector"| 定义插件名称
\item \verb|bxArray| 相当于 Matlab api 的 \href{https://www.mathworks.com/help/matlab/apiref/mxarray.html}{mxArray}, 基本上就是 Matlab 的矩阵, 什么都能装。
\item 若编译动态库, 就在编译时定义 \verb|BV_USE_DYN_LOADER|
\end{itemize}

\begin{lstlisting}[language=cpp,caption=vector 插件的 main.cpp]
static bexfun_info_t flist[] = {
    {"我的命名空间::函数名1", 函数名1, 函数名1的帮助说明字符串},
    {"我的命名空间::函数名2", 函数名2, 函数名2的帮助说明字符串},
    ...
    {"", NULL, NULL} // 表示结尾
};
// 告诉 baltam 插件包含的函数
// 如果使用了命名空间, 在 Baltam 中调用函数同样需要 \verb|命名空间::函数名| 来调用
bexfun_info_t *bxPluginFunctions() { return flist; }

// [可选] plugin 初始化, 输入 libbex 的句柄, 正常结束返回 0
// int bxPluginInitLib(void *hdl) {...}

// 自定义 class, 必须继承于 extern_obj_base
...
struct bVector : public extern_obj_base {
    BALTAM_LOCAL static int ID; // 定义插件的 ID (会被内核使用)
    std::vector<bxArray*> data;
    // 复制这个对象(duplicate) 用于 baltam 中的 B = A
    extern_obj_base *dup() const override;
    ~bVector() override;
    // 可选:打印对象信息
    std::string to_string() const override;
    // 可选:自定义类型名字
    std::string classname() const override;
};

// 声明自定义类型
int bxPluginInit(int, const bxArray*[]) {
    bxAddCXXClass<自定义类型>(PLUGIN_NAME, bex::__bxAddCXXClass_impl);
    return 0;
}

int bVector::ID = 0;

// bex 就相当于 matlab 的 mex 哈哈哈
extern_obj_base *bVector::dup() const {
    bVector *ret = new bVector(*this);
    for (size_t i = 0; i < this->data.size(); ++i)
        ret->data[i] = bex::__bxDuplicateArray(data[i]);
    return ret;
}

// 把 vector<> 中的每个 bxArray 毁灭
bVector::~bVector(){
    for (auto &i : data)
        bex::__bxDestroyArray(i);
}

// 自定义插件函数的 prototype 为 (nlhs 是输出个数, nrhs 是输入个数)。
// 当 nlhs 为零时, 输出 plhs[0] 可以传给 ans 变量
void 函数名(int nlhs, bxArray *plhs[], int nrhs, const bxArray *prhs
[]);

// 显示 bVector 信息
// bxGetExtObj() 把 bxArray 转换为外部对象
void bv_show(int, bxArray *[], int nrhs, const bxArray *prhs[]){
    bVector *bv = bxGetExtObj<bVector>(prhs[0], bex::__bxGetExtObj_impl);
// 这里用 cout 并不妥, 应该用 bxPrintf, 见文档
    for (size_t i = 0; i < bv->data.size(); ++i){
        std::cout << '[' << i << "]    "
            << bex::__bxTypeCStr(bv->data[i]) << std::endl;
    }
}
\end{lstlisting}

\subsection{文档笔记}
\begin{itemize}
\item 插件目录为 \verb|Baltam/plugin/插件名/|, 插件的 so 文件必须是 \verb|main.so|
\item \verb|int bxPrintf(const char * format, ...);| 用法和 \verb|printf| 一样
\item \verb|void bxErrMsgTxt(const char * str);| 终止当前函数调用并向图形界面或命令行窗口输出错误信息
\item \verb|baInt bxAsInt(const bxArray *ba, int * err);| 尝试将给定 bxArray 对象转化为内置整数
\item \verb|int bxAsCStr(const bxArray *ba, char *buff, baSize size);| 尝试将给定 bxArray 对象转化为 char 类型字符串
\end{itemize}

\subsubsection{算符重载}
\begin{lstlisting}[language=cpp]
// 用于实现重载 + 的声明, 函数名称任意
void my_add(int, bxArray*[], int, const bxArray*[]);

int bxPluginInit(int, const bxArray*[]) {
    bxAddCXXClass<自定义类型1>(PLUGIN_NAME);
    // 向内核注册重载的算符, ID 可以是自定义对象, 也可以是内建对象的
    bxRegisterBinaryOperator(PLUGIN_NAME, "+", 自定义类型1::ID,
        自定义类型2::ID, my_add);
    return 0;
}
\end{lstlisting}
