% 首都师范大学 2005 年硕士考试试题
% keys 首都师范大学|2005年|考研|物理
% license Copy
% type Tutor



\textbf{声明}:“该内容来源于网络公开资料,不保证真实性,如有侵权请联系管理员”

\begin{enumerate}
\item 一质量为 m、长度为乚的均匀杆,在距离其一端点为 0.2L 处钻一小孔,并将孔穿在一光滑水平轴上,使杆可以在竖直平面内自由摆动。求杆做微幅摆动的周期。
\item 一质量为 m、半径为R的圆柱体,放置在一固定斜面上。圆桂体与斜面之间的摩擦系数为μ。将圆柱体放置在斜面上,让其从静止开始向下做无滑滚动。为保证圆柱体不发生滑动,斜面的倾角不能大于多少?
\item 两个滑冰运动员的质量各为M,以速率$v_0$沿两平行线相对滑行,滑行路线间的垂直距离为L。当彼此交错时,两运动员各自抓住长度为L的绳索的一端,并在相对旋转的过程中各自收拢绳索。求,当绳长为0.5L时,他们各自的速率为多少?总动能的变化如何?
\item 一平行板电容器,板面积为s,间距为d,接在电源上以保持板间电压为U。将两板之间距离拉开一倍。求:\\(1)静电能的改变:\\(2)电场对电源所做的功;\\(3)外力对极板所做的功;\\(4)说明此过程中系统的能量守恒与转化关系。
\item 无穷长载流中空圆管置于真空中(见图1),其内、外半径分别为a,b。管的横截面中电流均匀分布,总电流为I。管外又包有绝缘层,其外半径为c。已知圆管和绝缘层的相对磁导率分别为$\mu_{r1}$和$\mu_{r2}$。求:\\
(1)空间各点磁感应$\vec B$的分布;\\
(2)当$\mu_{r1}$≈1,$\mu_{r2}$>>1时磁化面电流的分布。
\begin{figure}[ht]
\centering
\includegraphics[width=8cm]{./figures/6d57156d20f7208f.png}
\caption{} \label{fig_SSD05_2}
\end{figure}
\item 在半径为R的无限长螺线管中通上电流(见图2),当电流随时间发生变化时,在管中产生变化的磁场。若管内磁场的变化率为$\displaystyle \dv{\vec B}{t}$,且$\displaystyle \dv{\vec B}{t}=\vec C$(\vec C为大于零的常矢量)。求:距离管中心轴线为r(r<R)处感生电场的大小。
\begin{figure}[ht]
\centering
\includegraphics[width=8cm]{./figures/050b57cd1ae45d44.png}
\caption{} \label{fig_SSD05_4}
\end{figure}
\item 一长为l,质量为m的导体棒CD,其电阻为R,沿两条平行导电轨道无擦滑下,轨道电阻不计。轨道与水平面之间夹角为$\theta $,整个装置放置在均磁场中,场的磁感应强度为$\vec B_0$(见图3)求:导体下滑时速度随时间的变化规律。
\item 考虑精细结构,氢原子光谱中帕邢系的第一条谱线应由多少个波长成分组成,试画出相应的精细结构能级跃迁图,并标明各能级对应的原子态符号。
\item 某元素原子光谱中的一条谱线为$^3D_1 \to ^3P_1$跃迁产生,试求:\\
(1)在弱磁场B中,这两个能级各分裂为几个能级?分裂后相邻能级间隔各为多少$\mu_B B$?($\mu_B=\frac{eh}{2m_e}$)为玻尔磁子。\\
(2)在弱磁场B中,该谱线会分裂为多少条(在垂直与平行磁场两方向观察)?与原谱线的波数差各为多少$\widetilde L$?($\widetilde L=\frac{eB}{4\pi m_e c}$为洛伦兹单位)\\
(3)画出相应的能级跃迁图。
\begin{figure}[ht]
\centering
\includegraphics[width=8cm]{./figures/c5d1669b7fffe687.png}
\caption{} \label{fig_SSD05_5}
\end{figure}
\item (1)氢原子与中子质量分别为1.007825u和1.008665u,试计算$^12_6C$原子核的结合能与核子平均结合能.\\
(2)从含有1克$^{232}_{90}Th$的薄片中测得每秒放射 4100个α粒子,试计算$^{232}_{90} Th$的半衰期。

\end{enumerate}

