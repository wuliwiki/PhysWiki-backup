% Hartree-Fock 近似
% keys Hartree-Fock近似
\pentry{二次量子化\upref{SecQua}}

Hartree-Fock近似是用来近似描述相互作用电子气的理论,其核心在于平均场思想。

写出库伦相互作用电子气的哈密顿量:

$$H=\sum\limits_{i}\langle i|H_0|i\rangle c_{i,\sigma}^\dagger c_{i}^~+\frac{1}{2}\sum\limits_{\mu ,\nu,\mu', \nu'}V_{\mu \nu,\mu' \nu'}c_{\mu'}^\dagger c_{\nu'}^\dagger c_{\nu}^~ c_{\mu}^~~. $$
上式中的下表$\mu,\nu$等包括了粒子的自旋信息,值得注意的是我们考虑的是库伦相互作用,并不作用于自旋,所以相互作用项中的四算符部分里$\mu'$和$\mu$、$\nu'$和$\nu$表示的是同一个粒子散射前后的状态,其自选应该是相同的。

但四算符的难以处理的,我们使用平均场手段将其近似为二算符。我们需要关注四算符的期待值:
$$
\langle n_\mu, n_\nu |c_{\mu'}^\dagger c_{\nu'}^\dagger c_{\nu}^~ c_{\mu}^~ |n_\mu ,n_\nu \rangle = \langle n_\mu, n_\nu |c_{\mu'}^\dagger c_{\nu'}^\dagger |n_\mu-1 ,n_\nu-1 \rangle~.
$$

keyi

