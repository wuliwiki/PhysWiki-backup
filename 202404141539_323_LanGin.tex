% Landau-Ginzburg 理论
% keys 相变|超导|平均场
% license Usr
% type Wiki

\subsection{背景介绍}
Ginzburg 谈到,最初他想要解决在超导体中出现的温差电现象。当超导体出现温度梯度时,导体中会出现特殊的超导电流,并且涌现出电磁场。当时的伦敦方程仅仅能解释其中的部分现象,而且伦敦方程预测普通金属与超导金属的界面能是负的,因此当时急需一个非电磁场起源的界面能
\footnote{Ginzburg, Vitaly L. "Nobel Lecture: On superconductivity and superfluidity (what I have and have not managed to do) as well as on the “physical minimum” at the beginning of the XXI century." Reviews of Modern Physics 76.3 (2004): 981.}