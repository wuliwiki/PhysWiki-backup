% 卡西米尔效应(综述)
% license CCBYSA3
% type Wiki

本文根据 CC-BY-SA 协议转载翻译自维基百科\href{https://en.wikipedia.org/wiki/Casimir_effect}{相关文章}。

\begin{figure}[ht]
\centering
\includegraphics[width=6cm]{./figures/0b8861a1a0020cbd.png}
\caption{平行板上的卡西米尔力} \label{fig_KMXR_1}
\end{figure}
在量子场论中,\textbf{卡西米尔效应}(或称\textbf{卡西米尔力})是一种作用在受限空间宏观边界上的物理力,它源于场的量子涨落。当该效应以单位面积上的力来描述时,有时会使用“卡西米尔压力”这一术语。[2][3] 这一效应以荷兰物理学家\textbf{亨德里克·卡西米尔}的名字命名,他于1948年预测了电磁系统中的这一效应。

同年,卡西米尔与迪尔克·波尔德一起描述了一个类似的效应,这一效应发生在中性原子靠近宏观界面时,称为卡西米尔–波尔德力。[4] 他们的结果是对伦敦–范德瓦尔斯力的推广,且包括了由于光速有限所导致的时滞。伦敦–范德瓦尔斯力、卡西米尔力和卡西米尔–波尔德力的基本原理可以在同一框架下进行表述。[5][6]

在1997年,史蒂文·K·拉莫罗进行了直接实验,定量测量了卡西米尔力,结果与理论预测值相差不超过5\%。[7]

卡西米尔效应可以通过以下观点理解:宏观物质界面的存在,如电导体和介电体,改变了第二量子化电磁场能量的真空期望值。[8][9] 由于这一能量的值依赖于材料的形状和位置,卡西米尔效应表现为这些物体之间的力。

任何支持振荡的介质都有类似的卡西米尔效应。例如,绳上的珠子[10][11],以及浸入湍流水或气体中的板[12][13]都能说明卡西米尔力。

在现代理论物理中,卡西米尔效应在质子的手征袋模型中起着重要作用;在应用物理中,它在一些新兴的微技术和纳米技术中具有重要意义。[14]
\subsection{物理性质}
典型的例子是两个未带电的导电板,在真空中相距几纳米。在经典描述中,缺乏外部场意味着板间不存在场,也没有力将它们连接在一起。[15] 但是,当使用量子电动力学真空来研究这个场时,可以看到板对构成场的虚拟光子产生影响,并产生一个净力[16]——这种力要么是吸引力,要么是排斥力,取决于板的具体排列。尽管卡西米尔效应可以通过虚拟粒子与物体的相互作用来表达,但最好通过量子化场在物体之间的零点能量来描述,并且更容易计算。这个力已经被测量,并且是一个由第二量子化正式捕捉到的效应的典型例子。[17][18]

在这些计算中,边界条件的处理存在争议。事实上,“卡西米尔的原始目标是计算导电板上可极化分子之间的范德瓦尔斯力。”因此,它可以在没有任何提及量子场的零点能量(真空能量)的情况下进行解释。[19]

由于该力的强度随距离迅速减小,因此只有在物体之间的距离很小时才能测量到。这种力变得非常强大,以至于在亚微米尺度下,它成为未带电导体之间的主导力。实际上,在10纳米的间距下——约为原子典型尺寸的100倍——卡西米尔效应产生相当于约1大气压的压力(精确值依赖于表面几何形状和其他因素)。[17]
\subsection{历史}
荷兰物理学家亨德里克·卡西米尔和迪尔克·波尔德在1947年于飞利浦研究实验室提出了两种可极化原子之间以及这种原子与导电板之间存在一种力;这种特殊形式的力被称为卡西米尔–波尔德力。[4] 在与尼尔斯·玻尔的对话后,玻尔建议这与零点能量有关,卡西米尔单独提出了理论,预测了中性导电板之间的力,这一现象被称为卡西米尔效应。[20]

该力的预测后来扩展到了有限导电性的金属和介电体,而后来的计算则考虑了更一般的几何形状。在1997年前的实验中,卡西米尔力被定性地观察到,并通过测量液态氦薄膜的厚度间接验证了预测的卡西米尔能量。最终,1997年拉莫罗的直接实验定量测量了卡西米尔力,结果与理论预测值相差不超过5\%。[7] 随后的实验则达到了几百分点的精度。
\subsection{可能的原因}
\subsubsection{真空能量}
卡西米尔效应的原因可以通过量子场论来描述,量子场论指出,所有各种基本场(如电磁场)必须在空间中的每一个点上量子化。在简化的视角下,可以将物理学中的“场”想象成空间中充满了相互连接的振动球和弹簧,场的强度可以通过球从其静止位置的位移来可视化。该场中的振动会传播,并受到特定场的适当波动方程的控制。量子场论的二次量子化要求每个这样的球-弹簧组合都要量子化,即场的强度必须在空间中的每一点上量子化。在最基本的层面上,空间中的每个点的场是一个简单的谐振子,而它的量子化意味着每个点都有一个量子谐振子。场的激发对应于粒子物理学中的基本粒子。然而,即使是真空也具有极其复杂的结构,因此所有的量子场论计算必须参考这种真空模型。

真空隐含地拥有粒子可能具有的所有属性:自旋[21](对于光来说是极化)、能量等等。平均而言,这些属性大多相互抵消:从这个意义上来说,真空毕竟是“空的”。一个重要的例外是真空能量,或者说真空的能量期望值。简单谐振子的量子化表明,这样的谐振子可能具有的最低能量或零点能量为:
\[ E = \frac{1}{2} \hbar \omega ~\]
对空间中所有点的所有可能谐振子进行求和会得到一个无穷大的量。由于只有能量差是物理上可测量的(有一个显著的例外是引力,它超出了量子场论的范围),因此这个无穷大可以被视为数学上的一个特性,而不是物理上的特性。这个论点是重正化理论的基础。在这种方式处理无穷大量曾是量子场论学者普遍不安的原因之一,直到1970年代重正化群的出现,这是一种为尺度变换提供自然基础的数学形式化方法。

当物理学的范围扩展到包括引力时,对这种形式上无穷大的量的解释依然存在问题。目前尚没有令人信服的解释说明为什么它不应该导致一个比观测到的宇宙常数大几个数量级的值。[22] 然而,由于我们还没有完全一致的量子引力理论,因此同样也没有令人信服的理由说明它应该导致我们所观测到的宇宙常数值。[23]

对于费米子,卡西米尔效应可以理解为费米子算符(−1)F的谱不对称性,在这种情况下它被称为Witten指数。
\subsubsection{相对论性范德瓦尔斯力}
另一种观点是,麻省理工学院的Robert Jaffe在2005年的一篇论文中指出:“卡西米尔效应可以在不涉及零点能量的情况下进行公式化,卡西米尔力也可以计算出来。它们是带电粒子和电流之间的相对论性量子力。平行板之间的卡西米尔力(单位面积)随着细结构常数α趋近于零而消失,而标准结果(似乎与α无关)对应于α趋近于无穷大的极限。”他还指出:“卡西米尔力只是金属板之间的(相对论性、滞后)范德瓦尔斯力。”[19] 卡西米尔和波尔德的原始论文使用了这种方法来推导卡西米尔-波尔德力。1978年,Schwinger、DeRadd和Milton发布了一种类似的推导,描述了平行板之间的卡西米尔效应。[24] 更近期的研究中,Nikolic从量子电动力学的基本原理出发,证明了卡西米尔力并非源自电磁场的真空能量,[25] 并用简单的语言解释了卡西米尔力的基本微观起源为何来自范德瓦尔斯力。[26]
\subsection{效应}
卡西米尔的观察是,存在于金属或介电体等大体物体中的二次量子化电磁场,必须遵循与经典电磁场相同的边界条件。特别地,这影响了在导体或介电体存在下真空能量的计算。

举个例子,考虑计算金属腔体内电磁场的真空期望值,例如雷达腔体或微波波导。在这种情况下,正确的方式是通过求解腔体中驻波的能量来找到场的零点能量。每一个可能的驻波对应一个能量;假设第n个驻波的能量为En。腔体内电磁场的真空期望值为:
\[
\langle E \rangle = \frac{1}{2} \sum_{n} E_n~
\]
其中求和遍历所有可能的n值,列举所有驻波。因子 \( \frac{1}{2} \) 是因为第n模式的零点能量是 \( \frac{1}{2} E_n \),其中En是第n模式的能量增量(这个 \( \frac{1}{2} \) 与方程 \( E = \frac{1}{2} \hbar \omega \) 中的 \( \frac{1}{2} \) 相同)。以这种方式写出来的和显然是发散的,但它可以用来构造有限的表达式。

具体来说,可以问零点能量如何依赖于腔体形状s。每个能级En依赖于形状,因此应该写为 \( E_n(s) \),而真空期望值则是 \( \langle E(s) \rangle \)。此时,有一个重要的观察:如果壁的形状s在点p处稍微改变(比如通过δs),则壁上的力等于真空能量的变化。即,力为:
\[
F(p) = - \left. \frac{\delta \langle E(s) \rangle}{\delta s} \right|_p~
\]
在许多实际计算中,这个值是有限的。[27]

板间的吸引力可以通过专注于一维情况来理解。假设一个可移动的导电板位于与两个相距较远的板(距离为l)之间的距离a处。当 \( a \ll l \) 时,槽内宽度为a的状态受到严格约束,使得任何一个模式的能量E与下一个模式的能量之间有很大的间隔。与此不同,在大区域l内,存在着大量的状态(约 \( \frac{l}{a} \) 个),这些状态的能量在 \( E \) 和下一个模式之间均匀分布,换句话说,所有的能量都略大于E。现在,若将a缩短一个量da(da为负值),则窄槽中的模式波长缩小,从而使得能量增加,比例为 \( -\frac{da}{a} \);而所有位于大区域内的 \( \frac{l}{a} \) 个状态波长拉长,能量相应减少,比例为 \( -\frac{da}{l} \)(注意分母不同)。这两个效应几乎互相抵消,但净变化略为负值,因为大区域内所有的 \( \frac{l}{a} \) 个模式的能量略大于槽中的单一模式。因此,力是吸引性的:它倾向于使板间的距离变小,板子彼此靠近,穿越狭窄的槽。
\subsection{卡西米尔效应的推导(假设使用ζ-正则化)}
在卡西米尔的原始计算中,他考虑了两个导电金属板之间的空间,板间距离为a。在这种情况下,驻波的计算特别简单,因为电场的横向分量和磁场的法向分量必须在导体表面消失。假设板平行于xy平面,驻波为:
\[
\psi_n(x, y, z; t) = e^{-i\omega_n t} e^{i k_x x + i k_y y} \sin(k_n z)~
\]
其中ψ表示电磁场的电分量,为了简便,忽略了极化和磁分量。在这里,\( k_x \) 和 \( k_y \) 是平行于板的方向上的波数,
\[
k_n = \frac{n\pi}{a}~
\]
是垂直于板的波数。这里,n是整数,源于要求ψ在金属板上消失。该波的频率为:
\[
\omega_n = c \sqrt{{k_x}^2 + {k_y}^2 + \frac{n^2 \pi^2}{a^2}}~
\]
其中c是光速。真空能量是所有可能激发模式的总和。由于板的面积很大,我们可以通过在k空间中对两个维度进行积分来进行求和。假设周期性边界条件,得到:
\[
\langle E \rangle = \frac{\hbar}{2} \cdot 2 \int \frac{A \, dk_x \, dk_y}{(2\pi)^2} \sum_{n=1}^{\infty} \omega_n~
\]
其中A是金属板的面积,并引入了一个因子2来表示波的两种可能极化。这种表达式显然是无穷大的,为了继续计算,方便引入一个调节因子(下文将详细讨论)。这个调节因子将使得表达式变为有限的,最终会去掉。经过ζ-正则化的每单位面积的能量表达式为:
\[
\frac{\langle E(s)\rangle}{A} = \hbar \int \frac{dk_x \, dk_y}{(2\pi)^2} \sum_{n=1}^{\infty} \omega_n \left|\omega_n\right|^{-s}~
\]
最后,要求取极限 \(s \to 0\)。这里的s是一个复数,不要与之前讨论的形状s混淆。对于实数且大于3的s,这个积分和求和是有限的。这个和在 \(s = 3\) 处有一个极点,但可以通过解析延拓到 \(s = 0\),使得表达式变为有限。上述表达式简化为:
\[
\frac{\langle E(s)\rangle}{A} = \frac{\hbar c^{1-s}}{4\pi^2} \sum_{n} \int_0^\infty 2\pi q \, dq \left| q^2 + \frac{\pi^2 n^2}{a^2} \right|^{\frac{1-s}{2}}~
\]
其中引入了极坐标 \(q^2 = k_x^2 + k_y^2\) 将双重积分转化为单重积分。前面的 \(q\) 是雅可比因子,\(2\pi\) 来自角度积分。如果 \(\text{Re}(s) > 3\),则该积分收敛,结果为:
\[
\frac{\langle E(s)\rangle}{A} = - \frac{\hbar c^{1-s} \pi^{2-s}}{2a^{3-s}} \frac{1}{3-s} \sum_{n} |n|^{3-s} = - \frac{\hbar c^{1-s} \pi^{2-s}}{2a^{3-s}(3-s)} \sum_{n} \frac{1}{|n|^{s-3}}.~
\]
在 \(s\) 接近0时,这个和是发散的,但如果假设大频率激发的阻尼对应于黎曼ζ函数的解析延拓至 \(s = 0\) 是在某种意义上物理上合理的,那么有:
\[
\frac{\langle E\rangle}{A} = \lim_{s \to 0} \frac{\langle E(s)\rangle}{A} = -\frac{\hbar c \pi^2}{6a^3} \zeta(-3)~
\]
但是 \(\zeta(-3) = \frac{1}{120}\),所以得到:
\[
\frac{\langle E\rangle}{A} = -\frac{\hbar c \pi^2}{720 a^3}~
\]
解析延拓显然消除了一个加法性的正无穷,巧妙地精确地解释了板间缝隙外部的零点能量(上述未包括),但这种能量会随着板块在封闭系统内的运动而变化。对于理想化的、完美导电的板和它们之间有真空的情况,每单位面积上的 Casimir 力 \( F_c / A \) 为:
\[
\frac{F_{\mathrm{c}}}{A} = -\frac{d}{da} \frac{\langle E \rangle}{A} = -\frac{\hbar c \pi^2}{240 a^4}~
\]
其中:
\begin{itemize}
\item \(\hbar\) 是约化普朗克常数,
\item \(c\) 是光速,
\item \(a\) 是两块板之间的距离。
\end{itemize}

该力是负值,表示力是吸引力:将两块板移动得更近,能量降低。出现 \(\hbar\) 显示了每单位面积上的 Casimir 力 \( F_c / A \) 非常小,并且力本质上来源于量子力学。

通过对上述方程积分,可以计算将两块板分离到无穷远所需的能量:
\[
U_E(a) = \int F(a) \, da = \int -\frac{\hbar c \pi^2}{240 a^4} \, da = \frac{\hbar c \pi^2 A}{720 a^3}~
\]
其中:
\begin{itemize}
\item \(\hbar\) 是约化普朗克常数,
\item \(c\) 是光速,
\item \(A\) 是其中一块板的面积,
\item \(a\) 是两块板之间的距离。
\end{itemize}
在 Casimir 的原始推导中,[20] 一块可移动的导电板被放置在两块相距较远的板之一的短距离 \(a\) 处(距离为 \(L\))。考虑板两侧的零点能量。与上述解析延拓假设不同,使用 Euler–Maclaurin 求和法和正则化函数(例如指数正则化)计算发散的和与积分,而不是像上述的 \(|\omega_n|^{-s}\) 这么异常的情况。[28]”