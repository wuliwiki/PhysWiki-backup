% 阶与原根
% keys 阶|原根
% license Usr
% type Tutor

\pentry{费马小定理与欧拉定理\nref{nod_fermse},线性同余\nref{nod_linmod}}{nod_856e}
\begin{lemma}{}\label{lem_ordprr_1}
对于命题 $P(n)$ 而言,若对于每个 $a$, $b$ 都有 $P(a)$ 与 $P(b)$ 蕴含 $P(a+b)$ 且至少在 $b \le a$ 时蕴含 $P(a-b)$,而 $r$ 是使 $P$ 成立的最小正正数,则:
\begin{enumerate}
\item 对于各个自然数 $k = 1, 2, \dots$,$P(kr)$ 也为真。
\item 任何使得 $P(m)$ 成立的非负整数 $m$ 都是 $r$ 的倍数。
\end{enumerate}
\end{lemma}
\textbf{证明}:对于第一点是显然的,由于 $P(r)$ 与 $P(r)$ 都为真,故 $P(r+r)$ 为真......

现在考虑第二点。根据 $r$ 的定义,$0 < r \le m$,记 $m = qr + s$,从而 $s = m - qr$,其中 $0 \le s < r$ 而 $q \ge 1$。根据第一点,$P(r) \rightarrow P(kr)$,而 $P(a)$ 与 $P(b)$ 蕴含 $P(a \pm b)$,故 $P(m)$ 与 $P(qr)$ 可以推导出 $P(s)$。而由于 $r$ 的定义,$s$ 必定为 $0$。故 $q = kr$。证毕!


\begin{definition}{阶}
记 $d$ 是使得
\begin{equation}
a^x \equiv 1 \pmod m ~~
\end{equation}
的最小正整 $x$,则称 $d$ 为 $a$ 模 $m$ 意义下的\textbf{阶(order)}。
\end{definition}

\begin{definition}{原根}
特别的,若 $d$ 是 $a$ 模 $m$ 的阶,且 $d = \varphi(m)$,则称 $d$ 是 $a$ 的一个\textbf{原根(primitive root)}。
\end{definition}

考虑阶的条件,记
\begin{equation}
P(x) : a^x \equiv 1 \pmod m ~,
\end{equation}
这会使得 $P(x)$ 与 $P(y)$ 显然蕴含 $P(x+y)$,而在 $y \le x$ 时,
\begin{equation}
a^{x-y} \equiv b \pmod m ~,
\end{equation}
则
\begin{equation}
a^x \equiv b a^y \pmod m ~,
\end{equation}
故必定 $b=1$ 而 $P(x)$ 与 $P(y)$ 在 $y \le x$ 时蕴含 $P(x-y)$。这就使得符合\autoref{lem_ordprr_1} 的条件,这就指出 $d | \varphi(m)$。

阶与原根可以理解为数论上的某种本原单位根。
