% 阿尔伯特·爱因斯坦
% license CCBYSA3
% type Wiki

(本文根据 CC-BY-SA 协议转载自原搜狗科学百科对英文维基百科的翻译)

阿尔伯特·爱因斯坦(/ˈaɪnstaɪn/\textbf{EYEN}-styne;[1] 德语:[ˈalbɛɐ̯t ˈʔaɪnʃtaɪn]( 发音);1879年3月14日 ——1955年4月18日)是出生于德国的理论物理学家,[2]他发展了现代物理学的两大支柱之一(与量子力学一起)——相对论。[3][4]他的工作也因其对科学哲学的影响而闻名。[5][6]他最为人所知的质能等价公式 E = mc2被称为“世界上最著名的方程式”。[7]他获得了1921年诺贝尔物理学奖,“基于他对理论物理学的贡献,尤其是因为他发现了光电效应定律”,[8]这是量子理论发展的关键一步。

爱因斯坦在职业生涯初期认为牛顿力学已经不足以调和经典力学和电磁场的定律,这让他在伯尔尼瑞士专利局工作期间(1902-1909年)发展了狭义相对论。同时,他意识到相对论原理也可以扩展到引力场,于是在1916年发表了一篇关于广义相对论的论文,其中包含了他的引力理论。他继续处理统计力学和量子理论的问题,这使粒子理论和分子运动能够被解释。他还研究了光的热性质,这为光的光子理论奠定了基础。1917年,他应用广义相对论来模拟宇宙的结构。[9][10]

除了在布拉格的一年之外,爱因斯坦在1895年至1914年间都生活在瑞士。在此期间,于1896年放弃了德国国籍,于1900年在苏黎世获得了瑞士联邦理工学院(后来的瑞士联邦技术学院)的学术文凭。在无国籍五年多之后,他、于1901年获得了瑞士公民身份,并将其保留终身。1905年,被苏黎世大学授予博士学位。同年,在著名的奇迹年发表了四篇开创性的论文并引起了学术界的注意,此时他才26岁。爱因斯坦于1912年至1914年间在苏黎士教理论物理,然后前往柏林,在那里他被选为普鲁士科学院院士。

1933年,当爱因斯坦访问美国时,阿道夫·希特勒上台执政。由于爱因斯坦的犹太背景,他没有返回德国,[11]在美国定居,并于1940年成为美国公民。[12]在第二次世界大战前夕,他签署了一封致富兰克林·罗斯福总统的信,提醒他注意“新型超强炸弹”的潜在发展,并建议美国开始类似的研究,这最终导致了曼哈顿计划的诞生。爱因斯坦支持同盟国,但他谴责将核裂变作为武器的想法。他与英国哲学家伯特兰·罗素签署了《罗素—爱因斯坦宣言条约》,强调了核武器的危险性。他隶属于新泽西州的普林斯顿高等研究院,直到1955年去世。

爱因斯坦发表了300多篇科学论文和150多篇非科学著作。[9][13]他的智力成就和独创性使“爱因斯坦”这个词成为“天才”的同义词。[14]尤金·维格纳将爱因斯坦与他同时代的人相比写道,“爱因斯坦的理解甚至比扬西·冯·诺伊曼更深刻。他的思想比冯·诺伊曼的更具穿透力和独创性。这是一个非常了不起的声明。”[15]

\subsection{生活和事业}
\subsubsection{1.1 早期生活和教育}
\begin{figure}[ht]
\centering
\includegraphics[width=6cm]{./figures/28110856e7f46d2c.png}
\caption{1882年,爱因斯坦三岁时} \label{fig_AYST_1}
\end{figure}
阿尔伯特·爱因斯坦于1879年3月14日出生在德意志帝国符腾堡王国的乌尔姆。[2]他的父亲是赫尔曼·爱因斯坦,母亲是鲍林·科赫,分别是一名工程师和推销员。1880年,爱因斯坦一家人搬到了慕尼黑,他的父亲和叔叔雅各布在那里创立了这家基于直流电制造电气设备的公司。[2]

爱因斯坦一家是不遵守犹太教义的犹太人,他从5岁开始在慕尼黑的一所天主教小学就读了三年。8岁时,他被转到了路易斯波特体育馆(现称阿尔伯特·爱因斯坦体育馆),在那里接受了高级小学和中学教育,直到7年后离开德意志帝国。[16]
\begin{figure}[ht]
\centering
\includegraphics[width=6cm]{./figures/92a19247301e3ac0.png}
\caption{1893年,爱因斯坦14岁时} \label{fig_AYST_2}
\end{figure}
1894年,由于赫尔曼和雅各布缺乏资金将他们的设备从直流(DC)标准转换为更高效的交流(AC)标准,他们的公司失去了向慕尼黑提供电照明的投标。[17]损失迫使他们将慕尼黑工厂出售。为了寻找生意,爱因斯坦一家搬到了意大利,先是去了米兰,几个月后去了帕维亚。当全家搬到帕维亚时,15岁的爱因斯坦留在慕尼黑,在路易斯波特体育馆完成学业。他的父亲打算让他从事电气工程,但爱因斯坦与当局发生了冲突,并对学校的制度和教学方法表示不满。他后来写道,学习和创造性思维的精神在严格的死记硬背学习中丧失了。1894年12月底,用医生的证明说服学校让他离开后,爱因斯坦去意大利和他在帕维亚的家人团聚。[18]在意大利期间,他写了一篇题为“关于磁场中以太状态的研究”的短论文。[19][20]

爱因斯坦从小就擅长数学和物理,比同龄人早几年达到数学水平。十二岁的爱因斯坦在一个夏天自学了代数和欧几里得几何。同时爱因斯坦在12岁时也独立地发现了毕达哥拉斯定理的原始证明。[21]在给了12岁的爱因斯坦一本几何教科书后,家庭教师马克斯·塔木德说:“爱因斯坦在短时间内就完成了整本书。随后他致力于高等数学...很快,他的数学天赋就飞得如此之高,我都跟不上了。”[22]他对几何和代数的热情使这个12岁的孩子相信自然可以被理解为“数学结构”。[22]爱因斯坦12岁开始自学微积分,14岁时他说自己“掌握了积分和微分”。[23]

13岁时,爱因斯坦被介绍给康德学习纯粹理性批判,康德成为了他最喜欢的哲学家,他的导师说:“当时他还是个孩子,只有十三岁,但是康德的作品,常人无法理解,对他来说似乎很清楚。”[22]
\begin{figure}[ht]
\centering
\includegraphics[width=6cm]{./figures/b0289fde805edc8f.png}
\caption{爱因斯坦17岁时获得的入学证书,显示了他在阿戈维亚州学校的毕业成绩(Aargauische Kantonsschule,分数为1-6分,其中6分是最高分数)。他的分数:德语5分;法语3分;意大利语5分;历史6分;地理4分;代数6分;几何6分;画法几何6分;物理6分;化学5分;自然历史5分;艺术与技术制图4分} \label{fig_AYST_3}
\end{figure}
1895年,16岁的爱因斯坦参加了苏黎世瑞士联邦理工学院(后拉来的瑞士联邦科技学院)的入学考试。他在考试的一般课程上没有达到要求的标准,[24]但是在物理和数学上取得了优异的成绩。[25]根据理工学院校长的建议,他于1895年和1896年在瑞士阿劳的阿尔戈维亚州学校(体育馆)完成了中学教育。在寄宿于约斯特·温特尔教授的家庭时,他爱上了温特尔的女儿玛丽。爱因斯坦的姐姐玛嘉后来嫁给了温特尔的儿子保罗。[26]1896年1月,在他父亲的批准下,爱因斯坦放弃了他在德国符腾堡王国的国籍,以逃避服兵役。[27]1896年9月,他以优异的成绩通过了瑞士马图拉考试,包括物理和数学科目的6级最高分,分数范围为1-6。[28]17岁时,他报名参加了苏黎世理工学院为期四年的数学和物理教学文凭课程。玛丽·温特尔比他大一岁,她搬到了瑞士的奥尔森堡担任教师。

爱因斯坦未来的妻子,一位20岁的塞尔维亚女性米列娃·马利奇,也在那一年就读于理工学院。她是数学和物理教学文凭课程部分六名学生中唯一的女性。在接下来的几年里,爱因斯坦和马利奇的友谊发展成了爱情,他们一起阅读课外物理书籍,爱因斯坦对此越来越感兴趣。1900年,爱因斯坦通过了数学和物理考试,并被授予联邦理工学院教学文凭。[29]有人声称马利奇和爱因斯坦在他1905年的论文上合作过,[30][31]这篇论文被称为奇迹年 论文,但是研究过这个问题的物理历史学家没有发现任何证据表明马利奇做出了任何实质性的贡献。[32][33][34][35]

\subsubsection{1.2 婚姻和儿童}
\begin{figure}[ht]
\centering
\includegraphics[width=6cm]{./figures/c19fe20fe3b3fddf.png}
\caption{1904年,爱因斯坦25岁时} \label{fig_AYST_4}
\end{figure}
爱因斯坦和马利奇的早期通信于1987年被发现并发表,其中披露这对夫妇有一个女儿,名叫“丽莎尔”,1902年初出生在诺维萨德,马利奇和她的父母住在那里。马利奇没有带着孩子回到瑞士,孩子的真实姓名和命运均不明。1903年9月爱因斯坦来信的内容表明,这个女孩要么被送养,要么在婴儿期死于猩红热。[36][37]
\begin{figure}[ht]
\centering
\includegraphics[width=6cm]{./figures/1de872074fbaa68b.png}
\caption{1921年,爱因斯坦和他的第二任妻子埃尔莎} \label{fig_AYST_5}
\end{figure}
爱因斯坦和马利奇于1903年1月结婚。1904年5月,他们的儿子汉斯·爱因斯坦出生在瑞士伯尔尼。他们的小儿子爱德华于1910年7月出生在苏黎世。这对夫妇于1914年4月搬到柏林,但得知爱因斯坦认为他的表妹埃尔莎是最有吸引力的人后,马利奇和他们的儿子们回到了苏黎世。[38]他们分居五年,之后于1919年2月14日离婚。[39]爱德华大约20岁时被诊断患有精神分裂症。[40]他的母亲照顾他,他还被关在精神病院好几个时期,最后在马利奇死后被永久关押在精神病院。[41]

在2015年披露的信件中,爱因斯坦写信给他的初恋玛丽·温特尔,讲述了他的婚姻以及他对她的强烈感情。他在1910年写道,当时他的妻子怀上了他们的第二个孩子,他写道:“我每时每刻都在发自内心地爱着你,我是如此的不快乐,只有男人才能如此”。他谈到他对玛丽的爱是“执迷不悟的爱”和“错过的生活”。[42]

自从1912年和埃尔莎·温特哈尔交往后,爱因斯坦于1919年与她结婚[43][44][45]。[45]他们于1933年移民到美国。埃尔莎于1935年被诊断患有心脏和肾脏疾病,并于1936年12月去世。[46]

\subsubsection{1.3 朋友}
爱因斯坦的著名朋友有米给雷·贝索、保罗·埃伦费斯特、格罗斯曼·马塞尔、查诺斯·普莱西、丹尼尔·波辛、莫里斯·索洛文和斯蒂芬·怀斯。[47]

\subsubsection{1.4 专利局}
\begin{figure}[ht]
\centering
\includegraphics[width=6cm]{./figures/b72398c75f0c0e83.png}
\caption{奥林匹亚学院的创始人:康拉德·哈比奇、莫里斯·索洛文和爱因斯坦} \label{fig_AYST_6}
\end{figure}
1900年毕业后,爱因斯坦花了将近两年时间寻找一个教学职位。他于1901年2月获得瑞士公民身份,[48]但是由于医疗原因没有被征召入伍。在格罗斯曼·马塞尔父亲的帮助下,他在伯尔尼的联邦知识产权局,即专利局找到了一份工作,[49][50]作为三级助理审查员。[51][52]

爱因斯坦评估了各种设备的专利申请,包括砾石分类器和机电打字机。[52]1903年,他在瑞士专利局的职位成为永久职位,他在完全掌握机器技术之前一直被提拔。[53]

他在专利局的大部分工作都与电信号传输和时间的机电同步有关,这两个技术问题在思想实验中显露无疑,最终导致爱因斯坦得出了光的本质和空间与时间之间基本联系的激进结论。[53]

1902年,爱因斯坦和几个在伯尔尼认识的朋友成立了一个小型讨论组,自嘲地称为“奥林匹亚学院”,并会定期开会讨论科学和哲学。他们的读物包括儒勒·昂利·庞加莱、恩斯特·马赫和大卫·休谟的作品,这些著作影响了他的科学和哲学观点。[54]

\textbf{首批科学论文}

\begin{figure}[ht]
\centering
\includegraphics[width=6cm]{./figures/645995350a4ffe29.png}
\caption{1921年,爱因斯坦获得诺贝尔物理学奖后的官方画像} \label{fig_AYST_7}
\end{figure}
1900年,爱因斯坦的论文“毛细现象的结论”发表在杂志《物理学年鉴》上。[55][56]1905年4月30日,爱因斯坦完成了他的论文,[57]并让实验物理教授阿尔弗雷德·克莱纳担任形式顾问。爱因斯坦凭借他的论文”分子尺寸的新测定”获得了苏黎世大学的博士学位。[57][58]

1905年,这被称为爱因斯坦的奇迹年,他发表了四篇开创性的论文,分别是关于光电效应、布朗运动、狭义相对论以及质量和能量的等效性,这些论文使他在26岁时就引起了学术界的注意。

\subsubsection{1.5 学业生涯}
1908年,他被公认为一位杰出的科学家,并被任命为伯尔尼大学的讲师。第二年,在苏黎世大学做了一个关于电动力学和相对论原理的讲座后,阿尔弗雷德·克莱纳推荐他到新成立的理论物理系担任老师。爱因斯坦于1909年被任命为副教授。[59]

爱因斯坦于1911年4月成为布拉格的德国查尔斯-费迪南德大学的正教授,并因此获得奥匈帝国的奥地利公民身份。[60][61]在布拉格逗留期间,他写了11部科学著作,其中5部是关于辐射数学和固体量子理论的。1912年7月,他回到了他在苏黎世的母校。从1912年到1914年,他是苏黎世联邦理工学院担任理论物理教授,在那里他教授分析力学和热力学。他还研究了连续介质力学、分子热理论和万有引力问题,他与数学家兼朋友的马塞尔格罗斯曼一起工作。[62]

1913年7月3日,他在柏林被投票选为普鲁士科学院的成员。马克斯·普朗克和瓦尔特·能斯特第二周在苏黎士拜访了他,劝说他加入该学院,并为他提供了即将成立的凯撒·威廉物理研究所主任的职位。[63](学院的会员资格包括在柏林洪堡大学没有教学任务的带薪工资和教授职位。)他于7月24日正式当选为学院成员,并同意第二年搬到德意志帝国。他搬到柏林的决定也考虑到了可以住在他表妹埃尔莎附近,与她发展了一段浪漫的恋情。他于1914年4月1日加入学院,进而加入柏林大学。[64]随着那一年第一次世界大战的爆发,凯撒·威廉物理研究所的计划流产了。该研究所随后成立于1917年10月1日,由爱因斯坦担任主任。[65]1916年,爱因斯坦当选为德国物理学会主席(1916-1918)。[66]

根据爱因斯坦在1911年关于他的新广义相对论的计算,来自另一颗恒星的光应该被太阳的引力弯曲。1919年,亚瑟·爱丁顿爵士在1919年5月29日日食期间证实了这一预测。这些观察结果发表在国际媒体上,使爱因斯坦闻名于世。1919年11月7日,英国主要报纸 《泰晤士报》 印刷了标题为“科学革命” –新的宇宙理论 –牛顿思想被推翻”的横幅。[67]

1920年,他成为荷兰皇家艺术与科学学院的外籍成员。[68]1922年,他因“对理论物理的贡献,特别是对光电效应定律的发现”获得1921年诺贝尔物理学奖。[8]虽然广义相对论仍被认为是有争议的,但引用文献也没有将引用的光电工作视为说明 ,而是仅仅作为一个定律的发现,因为光子的概念被认为是古怪的,直到1924年安德拉·纳特·博斯推导出普朗克光谱才被普遍接受。爱因斯坦于1921年当选为英国皇家学会的外籍会员。[3]1925年,他还获得了皇家学会颁发的科普利奖章勋章。[3]

\subsubsection{1.6 1921-1922年:出国旅行}
\begin{figure}[ht]
\centering
\includegraphics[width=10cm]{./figures/bc63f7e1b8d3261a.png}
\caption{1922年至1932年,爱因斯坦在国际智力合作委员会(国际联盟)的一次会议上} \label{fig_AYST_8}
\end{figure}
1921年4月2日,爱因斯坦第一次访问纽约市,受到了市长约翰·弗朗西斯·海伦的正式欢迎,随后举行了为期三周的讲座和招待会。他接着在哥伦比亚大学和普林斯顿大学发表了几次演讲,并在华盛顿陪同国家科学院的代表访问了白宫。回到欧洲后,他成为英国政治家和哲学家霍尔丹子爵在伦敦的客人,在那里他会见了几位著名的科学、知识和政治人物,并在伦敦国王学院发表了演讲。[69][70]

1921年7月,他还发表了一篇题为《我对美国的第一印象》的文章,其中他试图简要描述美国人的一些特征,就像亚历西斯·德·托克维尔在1921年发表了自己对美国民主 的印象一样 (1835)。[71]对于他的一些观察,爱因斯坦显然很惊讶:“给游客留下印象的是对生活的快乐、积极的态度...美国人友好、自信、乐观,不嫉妒。”[72]

1922年,作为为期六个月的旅行和演讲之旅的一部分,他去了亚洲,后来又去了巴勒斯坦,访问了新加坡、锡兰和日本,在那里他给成千上万的日本人做了一系列讲座。第一次公开演讲后,他在皇宫会见了皇帝和皇后,成千上万的人前来观看。在给他儿子的一封信中,他描述了自己对日本人的印象,认为日本人谦虚、聪明、体贴,对艺术有真正的感受。[73]在1922-1923年访问亚洲期间,他在自己的旅行日记中表达了对中国人、日本人和印度人的一些看法,这些看法在2018年被重新发现时被描述为仇外心理和种族主义的判决。[74]

由于爱因斯坦的远东之行,他无法亲自在1922年12月的斯德哥尔摩颁奖典礼上接受诺贝尔物理学奖。一位德国外交官代替他发表了宴会演讲,他称赞爱因斯坦不仅是一位科学家,还是一位国际和平缔造者和活动家。[75]

在回程中,他访问了巴勒斯坦12天,这将成为他对该地区的唯一访问。他受到的欢迎就如同他是一位国家元首,而不是物理学家,其中包括抵达英国高级专员赫伯特·塞缪尔爵士家时的礼炮。在一次招待会上,大楼挤满了想看和想听他讲话的人。在爱因斯坦对观众的讲话中,他表达了犹太人开始被认为是世界上一股力量的喜悦。[76]

爱因斯坦在1923年访问了西班牙两个星期,在那里他与圣地亚哥·拉蒙-卡哈尔进行了短暂的会面,并获得了阿方索十三世国王授予他的文凭,任命他为西班牙科学院的成员。[77]

从1922年到1932年,爱因斯坦是日内瓦国际联盟智力合作国际委员会的成员(在1923-1924年间中断了几个月),[78]这是为促进科学家、研究人员、教师、艺术家和知识分子之间的国际交流而成立的机构。[79]秘书长埃里克·德拉蒙德最初被任命为瑞士代表,但天主教活动人士奥斯卡·哈利基和朱塞佩·莫塔说服他成为德国代表,从而让贡扎格·德·雷诺德获得瑞士席位,并借此宣扬传统天主教价值观。[80]爱因斯坦的前物理学教授亨德里克·洛伦兹和法国化学家玛丽·居里也是该委员会的成员。

\subsubsection{1.7 1930-1931年:去美国旅行}
1930年12月,爱因斯坦第二次访问美国,最初打算作为加州理工学院的研究员进行为期两个月的工作访问。在他第一次美国之行受到全国关注后,他和他的安排者们旨在保护他的隐私。尽管到处都是电报和邀请让他接受奖励或公开演讲,他还是拒绝了。[81]

到达纽约市后,爱因斯坦被带到了不同的地方和活动,包括唐人街,与《纽约时报》的编辑共进午餐以及在大都会歌剧院的表演卡门 ,他在到场时受到观众的欢呼。在接下来的几天里,市长吉米·沃克给了他这座城市的钥匙,并会见了哥伦比亚大学的校长,校长将爱因斯坦描述为“思想的统治者”。[82]纽约河畔教堂的牧师哈里·爱默生·福斯迪克带爱因斯坦参观了教堂,并在入口处展示了一尊根据爱因斯坦制作的全尺寸雕像。[82]在纽约逗留期间,他还和聚集在麦迪逊广场花园的15,000人一起参加了光明节的庆祝活动。[82]
\begin{figure}[ht]
\centering
\includegraphics[width=6cm]{./figures/133891c0fd9c3cb4.png}
\caption{1931年1月,爱因斯坦(左)和查理·卓别林在好莱坞城市灯光首映式上} \label{fig_AYST_9}
\end{figure}
爱因斯坦接着去了加利福尼亚,在那里他遇到了加州理工学院院长和诺贝尔奖获得者罗伯特·密立根。他和密立根的友谊比较“尴尬”,因为密立根“有爱国军国主义的倾向”,爱因斯坦是一个明显的和平主义者。[83]在对加州理工学院学生的演讲中,爱因斯坦指出科学往往弊大于利。[84]

这种对战争的厌恶也导致爱因斯坦与作家厄普顿·辛克莱和电影明星查理·卓别林交朋友,两人都以和平主义著称。环球影城的负责人卡尔·拉姆勒带爱因斯坦参观了他的工作室,并把他介绍给卓别林。他们很快就建立了融洽的关系,卓别林邀请爱因斯坦和他的妻子埃尔莎到他家吃饭。卓别林说爱因斯坦外表的平静和温柔似乎隐藏着一种“高度情绪化的气质”,从这种气质中他获得了“非凡的智力”。[85]

卓别林的电影,《城市之光》几天后将在好莱坞首映,卓别林邀请爱因斯坦和埃尔莎作为他的特邀嘉宾加入他的行列。爱因斯坦的传记作家沃尔特·伊萨克森将此描述为“名人新时代最难忘的场景之一”。[84]卓别林在后来的柏林之行中拜访了爱因斯坦的家,并回忆起他的“朴素的小公寓”和他开始写理论的钢琴。卓别林推测它“可能被纳粹分子用作引火柴”。[85]

\subsubsection{1.8 1933年:移民到美国}