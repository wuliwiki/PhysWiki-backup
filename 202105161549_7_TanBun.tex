% 向量丛和切丛
% 向量丛|切向量|截面|光滑截面
\begin{issues}
\issueDraft
\end{issues}
\addTODO{创建“环上的模”词条后,应将其引用为本词条的预备知识}

\pentry{纤维丛\upref{Fibre}}

“切丛”是“切向量丛”的简称.顾名思义,它是切向量构成的向量丛,而这里的切向量是在流形上定义的,因此切丛的底空间是流形.

\begin{definition}{切丛}
给定流形 $M$,以 $M$ 为底空间,把各 $p\in M$ 上的 $T_pM$ 视为该点处的一根纤维,得到的纤维丛就称为流形 $M$ 上的\textbf{切丛(tangent bundle)}.
\end{definition}

一个切向量场可以视为切丛的一种特殊的子集,称为“\textbf{截面(section)}”.使用这个术语是为了强调这种子集的特殊性,它在每一根纤维上都取且仅取一个点,看起来就像是纤维丛的一个截面.同样地,一个光滑切向量场有时也被称作切丛上的一个\textbf{光滑截面}.

局部来看,流形 $M$ 上切丛的每根纤维是一个线性空间;整体来看,每个切向量场本身都可以看成一个向量,构成一个线性空间,记为 $\mathfrak{X}(M)$.$\mathfrak{X}(M)$ 作为线性空间的\textbf{基域}和 $M$ 是一致的,具体来说,实流形 $M$ 上的 $\mathfrak{X}(M)$,其数乘时使用的“数字”就是实数.但是我们常研究另一种情况,即把对 $\mathfrak{X}(M)$ 进行数乘时的“数字”取为 $M$ 上的一个光滑函数.这个时候,$\mathfrak{X}(M)$ 不再能被看成一个线性空间,而应该是一个环 $C^\infty(M)$ 上的模.这里 $C^\infty(M)$ 是 $M$ 上全体光滑函数的集合,它只是一个环,不满足乘法逆元存在性.

从流形整体来看,一个1-形式,或者说余切向量场,本身也可以看成是模 $\mathfrak{X}(M)$ 上的一个线性映射.

流形 $M$ 上也可以定义切丛之外的向量丛,同样有截面的概念,只不过此时的截面不再是切向量场了.向量丛 $E$ 上全体光滑截面的集合,记为 $\Gamma(E)$.由上所述,光滑向量场的集合 $\mathfrak{X}(M)$ 是 $\Gamma(E)$ 的特例,正如切丛是向量丛的特例.

向量丛之间有两类比较重要的映射:

\begin{definition}{点算子}
设 $E$ 和 $F$ 是同一个底空间 $M$ 上的两个\textbf{向量丛},称丛映射 $\varphi:E\rightarrow F$ 是一个\textbf{点算子(point operator)},如果对于任意的光滑截面 $s\in\Gamma(E)$ 和点 $p\in M$,当 $s(p)=0$ 时必有 $\varphi(s)(p)=0$.

换句话说,如果称 $Z_s\{p\in M|s(p)=0\}$ 为 $s$ 的\textbf{零化子(null set)},那么 $\varphi$ 是一个点算子当且仅当 $Z_s\subseteq Z_{\varphi{s}}$.
\end{definition}

\begin{definition}{局部算子}
设 $E$ 和 $F$ 是同一个底空间 $M$ 上的两个\textbf{向量丛},称丛映射 $\varphi:E\rightarrow F$ 是一个\textbf{局部算子(local operator)},如果对于任意的光滑截面 $s\in\Gamma(E)$,\textbf{开集}$U\subseteq E$ 和点 $p\in M$,当 $s(U)=\{0\}$ 时必有 $\varphi(s)(U)=\{0\}$.

换句话说,如果记 $Z_s$ 的\textbf{内部}\footnote{见\textbf{点集的内部、外部和边界}\upref{Topo0}词条.}为 $Z_s^\circ$,那么 $\varphi$ 是一个局部算子当且仅当 $Z_s^\circ\subseteq Z_{\varphi{s}}^\circ$.
\end{definition}

局部算子等价于模 $\mathfrak{X}(M)$ 上的一个线性映射,我把这一点表述为如下定理:

\begin{theorem}{}\label{TanBun_the1}
给定流形 $M$ 上的两个向量丛 $E$ 和 $F$.如果 $\varphi:\Gamma(E)\rightarrow\Gamma(F)$ 是 $C^\infty(M)$-线性的\footnote{即 $\forall s\in\Gamma(E)$ 和 $\forall f\in C^\infty(M)$,都有 $\varphi(fs)=f\varphi(s)$.},那么 $\varphi$ 必为一个局部算子.
\end{theorem}

\begin{exercise}{}
证明\autoref{TanBun_the1} .思路之一可以是利用bump function来构造一个使得 $fs=s$ 的函数 $f$,然后讨论 $s, fs$ 和 $\varphi{s}$ 的零化子的内部之间的关系.
\end{exercise}

\begin{exercise}{\autoref{TanBun_the1} 的反思}
在把 $\mathfrak{X}(M)$ 整体视为一个环 $C^\infty(M)$ 上的模的时候,局部算子的性质有什么几何意义吗?
\end{exercise}

最后,我们再介绍一个概念,\textbf{参考系}.时空都是流形,而在时空理论中我们常讨论不同参考系下的物理定律,因此有必要定义一下流形上的参考系是什么.

\begin{definition}{参考系}
给定流形 $M$ 上的一个 $r$ 维向量丛 $E$.设 $E$ 上一个光滑截面的集合 $Fr=\{\bvec{e}_1, \bvec{e}_2, \cdots, \bvec{e}_r\}$ 满足:对于任意 $p\in M$,$\{\bvec{e}_1(p), \bvec{e}_2(p), \cdots, \bvec{e}_r(p)\}$ 构成纤维 $E_p$ 的一组基,那么称 $Fr$ 是 $E$ 上的一个\textbf{参考系(frame)}.
\end{definition}










