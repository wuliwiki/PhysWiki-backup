% 让·勒朗·达朗贝尔(综述)
% license CCBYSA3
% type Wiki

本文根据 CC-BY-SA 协议转载翻译自维基百科\href{https://en.wikipedia.org/wiki/Jean_le_Rond_d\%27Alembert}{相关文章}。

“‘达朗贝尔’重定向至此。其他用法,参见达朗贝尔 (消歧义)。  
不要与德朗布尔混淆。”
\begin{figure}[ht]
\centering
\includegraphics[width=6cm]{./figures/9e2e02aa6a250ffa.png}
\caption{达朗贝尔的粉彩肖像,由莫里斯·昆汀·德·拉图尔创作,1753年。} \label{fig_BR_1}
\end{figure}
让-巴蒂斯特·勒朗·达朗贝尔[a](/ˌdæləmˈbɛər/ DAL-əm-BAIR;[1] 法语:[ʒɑ̃ batist lə ʁɔ̃ dalɑ̃bɛʁ];1717年11月16日-1783年10月29日)是法国数学家、力学家、物理学家、哲学家和音乐理论家。在1759年之前,他与丹尼斯·狄德罗共同担任《百科全书》的编辑。[2] 用于求解波动方程的达朗贝尔公式以他的名字命名。[3][4][5] 波动方程有时也被称为达朗贝尔方程,代数学基本定理在法语中以达朗贝尔命名。
\subsection{早年生活}
达朗贝尔出生于巴黎,是作家克劳丁·盖林·德·坦森和骑士路易-卡缪·德图什(当时任炮兵军官)的私生子。在他出生时,德图什正在国外。出生几天后,他的母亲将他遗弃在巴黎圣让勒朗教堂的台阶上。根据习俗,他以该教堂的守护圣人命名。达朗贝尔被送到一个孤儿院,但他的父亲找到他并将他安置在一位玻璃工的妻子——鲁索夫人家中,达朗贝尔在这里生活了将近50年。[6] 鲁索夫人对他鼓励甚少。当他向她讲述自己的一些发现或所写的东西时,她通常会回复道:

“你永远只会成为一个哲学家——那是什么?不过是一个一生折腾自己、死后才被人谈论的傻瓜罢了。”[7]

德图什暗中资助了让·勒朗的教育,但并不希望他的亲子关系被正式承认。