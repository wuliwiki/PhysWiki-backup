% Linux 硬盘操作笔记(Gparted, fdisk, resize2fs, Clonezilla)

\begin{issues}
\issueDraft
\end{issues}

\begin{itemize}
\item \verb|Gparted| 是一个 GUI 硬盘分区工具, 在 Ubuntu live CD 中自带
\item \verb|fdisk -l| 查看所有挂载的硬盘
\item \verb|resize2fs -p /dev/sd? ???K| 可以改变 ext4 文件系统的大小, Gparted 用的就是这个命令. 这个命令需要很长时间.
\item \verb|mklabel| 修改分区的 label
\item \verb|mkfs.ext4 /dev/sdx1| 或者 \verb|mke4fs -t ext4 /dev/sdb1| 把某个分区格式化为 \verb|ext4|
\item Clonezilla 的分区备份对 GPT 和 MBR 都是通用的, 如果发生任何问题, 用 live cd 重装系统,确保能正常启动, 然后还原对应的分区就行,无需任何额外设置.
\item 
\end{itemize}
