% 勒内·笛卡尔(综述)
% license CCBYSA3
% type Wiki

本文根据 CC-BY-SA 协议转载翻译自维基百科\href{https://en.wikipedia.org/wiki/Ren\%C3\%A9_Descartes}{相关文章}。

勒内·笛卡尔(/deɪˈkɑːrt/ day-KART 或英国发音:/ˈdeɪkɑːrt/ DAY-kart;法语:[ʁəne dekaʁt] ⓘ;[注3][11] 1596年3月31日 – 1650年2月11日)[12][13]: 58  是法国哲学家、科学家和数学家,被广泛认为是现代哲学和科学兴起的奠基人物之一。数学在他的研究方法中至关重要,他将几何与代数相结合,创立了解析几何。笛卡尔的职业生涯大部分时间是在荷兰共和国度过的,最初在荷兰国军服役,后来成为荷兰黄金时代的核心知识分子。[14] 尽管他服务于一个新教国家,且后来被批评者视为自然神论者,笛卡尔实际上是罗马天主教徒。[15][16]

笛卡尔哲学的许多元素可以在晚期的亚里士多德主义、16世纪复兴的斯多葛主义或更早的哲学家如奥古斯丁的思想中找到前例。在他的自然哲学中,他在两个主要方面不同于当时的学派。首先,他拒绝将有形实质划分为质料和形式;其次,他拒绝在解释自然现象时诉诸于神或自然的终极目的。[17] 在神学中,他坚持神创造行为的绝对自由。笛卡尔拒绝接受前人哲学家的权威,常常将自己的观点与之前的哲学家区分开来。在《灵魂的激情》开篇中,这部早期现代情感论著中,笛卡尔甚至声称他将“仿佛从未有人写过这些问题一样”来论述该主题。他最著名的哲学陈述是“我思故我在”(拉丁语:cogito, ergo sum;法语:Je pense, donc je suis),出现在《方法谈》(1637年,以法语和拉丁语写成,1644年)和《哲学原理》(1644年拉丁语版,1647年法语版)中。[注4] 这一陈述要么被解释为逻辑三段论,要么被视为一种直觉思想。[18]

笛卡尔常被称为现代哲学之父,被广泛认为是17世纪对认识论关注增加的主要推动者。[19][注5] 他为17世纪大陆理性主义奠定了基础,后由斯宾诺莎和莱布尼茨提倡,但后来受到霍布斯、洛克、贝克莱和休谟组成的经验主义学派的反对。早期现代理性主义的兴起——作为历史上首次独立的系统性哲学学派——对现代西方思想产生了广泛影响,诞生了笛卡尔(笛卡尔主义)和斯宾诺莎(斯宾诺莎主义)两个理性主义哲学体系。正是17世纪的理性主义大师如笛卡尔、斯宾诺莎和莱布尼茨赋予了“理性时代”其名称和历史地位。莱布尼茨、斯宾诺莎[20] 和笛卡尔都在数学和哲学方面造诣颇深,笛卡尔和莱布尼茨还在多个科学领域有所贡献。[21] 尽管只有莱布尼茨被广泛认可为博学家,这三位理性主义者都在各自的著作中整合了不同的知识领域。[22]

笛卡尔的《第一哲学沉思》(1641年)至今仍是大多数大学哲学系的标准教材。笛卡尔在数学方面的影响同样显著,以他的名字命名了笛卡尔坐标系。他被誉为解析几何之父,这一数学分支后来用于微积分和数学分析的发现。笛卡尔也是科学革命的关键人物之一。
\subsection{生平}
\subsubsection{早年生活}
\begin{figure}[ht]
\centering
\includegraphics[width=6cm]{./figures/266c186e2fde6001.png}
\caption{笛卡尔家族的族徽。} \label{fig_DKE_1}
\end{figure}
笛卡尔出生的房子位于图赖讷的拉艾

勒内·笛卡尔于1596年3月31日出生在法国图赖讷省的拉艾(现今法国安德尔-卢瓦尔省的笛卡尔镇)。[23] 1597年5月,他的母亲让娜·布罗沙在生下一个死胎后几天去世。[24][23] 笛卡尔的父亲乔阿希姆是雷恩议会的成员。[25]: 22 勒内与祖母和叔祖一起生活。尽管笛卡尔一家是罗马天主教徒,但普瓦图地区当时由新教徒胡格诺派控制。[26] 1607年,由于体弱多病,笛卡尔晚入学,进入位于拉弗莱什的耶稣会皇家亨利-勒格朗学院。[27][28] 在那里,他接触了数学和物理学,包括伽利略的作品。[29][30] 在此期间,笛卡尔首次接触到赫尔墨斯神秘主义。1614年毕业后,他在普瓦捷大学学习了两年(1615-1616年),于1616年获得教会法和民法的学士和执照学位,[29] 这符合其父亲希望他成为律师的愿望。[31] 随后,他前往巴黎。

笛卡尔在普瓦捷大学的毕业注册记录,1616年

在《方法谈》中,笛卡尔回忆道:[32]: 20–21

我完全放弃了对书本的学习,决定不再追求任何除了可以在自己内心或伟大的‘世界之书’中找到的知识。我将余下的青春岁月用于旅行,拜访宫廷和军队,与不同性格和地位的人交往,积累各种经验,在命运带给我的各种情境中考验自己,并始终反思遇到的一切,以从中获得一些收获。
\begin{figure}[ht]
\centering
\includegraphics[width=6cm]{./figures/4849a80f640afc83.png}
\caption{笛卡尔出生于图赖讷的拉艾的房子。} \label{fig_DKE_2}
\end{figure}
\begin{figure}[ht]
\centering
\includegraphics[width=6cm]{./figures/02e60f46db774513.png}
\caption{笛卡尔在普瓦捷大学的毕业登记记录,1616年} \label{fig_DKE_3}
\end{figure}
\subsubsection{军队服役}
出于成为职业军官的志向,1618年笛卡尔以雇佣兵的身份加入了新教的荷兰国军,在拿骚的毛里茨指挥下于布雷达服役,[29] 并正式学习由西蒙·斯蒂文确立的军事工程学。[33] 因此,笛卡尔在布雷达得到了许多鼓励,来提升他的数学知识。[29] 通过这种方式,他结识了多德雷赫特学校的校长艾萨克·贝克曼,[29] 并为他写了《音乐概要》(1618年撰写,1650年出版)。[34]

自1619年起,笛卡尔为天主教的巴伐利亚公爵马克西米利安效力,[35] 并于1620年11月在布拉格附近参加了白山之战。[36][37]

根据阿德里安·巴耶的记述,1619年11月10日至11日夜间(圣马丁节),笛卡尔驻扎在多瑙河畔新堡,将自己锁在一个带有“火炉”的房间里(可能是一个陶瓷炉)[38] 以避寒。在那里,他做了三个梦,[39] 并认为一个神圣的灵感向他揭示了一种新的哲学。然而,有人推测笛卡尔所谓的第二个梦实际上是一次“爆头综合症”的发作。[40] 醒来后,他已经构思出了解析几何以及将数学方法应用于哲学的思想。他从这些异象中得出结论:科学的追求对他而言是通向真正智慧的道路,也是他一生工作的重要部分。[41][42] 笛卡尔还清楚地意识到所有真理彼此关联,因此找到一个基本真理并用逻辑推理就可以打开通往所有科学的大门。他很快就达到了这一基本真理:他著名的“我思故我在”。[43]
\subsubsection{职业生涯}

\textbf{法国}

1620年,笛卡尔离开了军队。他参观了洛雷托的圣殿圣母堂,然后游历了多个国家,之后回到法国,并在接下来的几年里在巴黎度过了一段时间。在巴黎,他撰写了第一篇关于方法的论文:《指导心智的规则》(Regulae ad Directionem Ingenii)。[43] 1623年,他抵达拉艾,将自己的全部财产出售,投资于债券,这为他余生提供了舒适的收入。[44][45]: 94 笛卡尔以观察者的身份参加了1627年黎塞留枢机围攻拉罗谢尔的行动。[45]: 128 在那里,他对黎塞留修建的大堤的物理特性产生了兴趣,并从数学角度研究了围攻期间所见的一切。他还结识了法国数学家吉拉尔·德萨尔格。[46] 同年秋天,他与梅森及其他学者一起前往教廷大使圭迪·迪·巴尼奥的住所,聆听炼金术士尼古拉·德·维利耶(西尔·德·尚杜)关于一种新哲学原则的讲座,[47] 枢机主教贝吕尔建议他在宗教裁判所势力范围之外的某地撰写他的新哲学的论述。[48]

\textbf{荷兰}
\begin{figure}[ht]
\centering
\includegraphics[width=6cm]{./figures/e7a65f692f782526.png}
\caption{在阿姆斯特丹,笛卡尔住在西市场6号(左侧的笛卡尔之家)。} \label{fig_DKE_4}
\end{figure}
笛卡尔于1628年返回荷兰共和国。[39] 1629年4月,他进入弗拉内克尔大学,师从阿德里安·梅蒂乌斯,住在一个天主教家庭或租住在沙尔德马城堡。次年,他以“普瓦图人”之名入读莱顿大学,这是一所新教大学。[49] 他跟随雅各布斯·戈利乌斯学习数学,后者向他介绍了帕普斯的六边形定理,并跟随马丁·霍腾修斯学习天文学。[50] 1630年10月,他与贝克曼发生冲突,指责对方剽窃了他的一些思想。在阿姆斯特丹,笛卡尔与一名女仆海伦娜·扬斯·范德斯特罗姆有过一段关系,并在1635年于代芬特尔生下了女儿弗朗辛。她接受了新教洗礼,[51][52] 但5岁时因猩红热去世。

与当时的许多道德家不同,笛卡尔并不贬低情感,反而为其辩护;[53] 在1640年弗朗辛去世时,他泪流满面。[54] 根据杰森·波特菲尔德最近的一部传记,“笛卡尔认为,成为男子汉并不意味着要抑制泪水。”[55] 拉塞尔·肖托推测,父亲身份的体验以及失去孩子的经历是笛卡尔工作中的一个转折点,使其研究重点从医学转向对普遍答案的追寻。[56]

尽管频繁搬家,[注6] 笛卡尔在荷兰的二十多年间完成了所有主要著作,开创了数学和哲学的革命。[注7] 1633年,伽利略被意大利宗教裁判所谴责,笛卡尔因此放弃了出版《世界论》的计划,这是他前四年的研究成果。然而,在1637年,他将该作品的一部分发表为三篇论文:[57]《气象》(Les Météores)、《屈光学》(La Dioptrique)和《几何学》(La Géométrie),并以他著名的《方法谈》(Discours de la méthode)作为引言。[57] 在书中,笛卡尔提出了四条思维规则,以确保我们的知识建立在坚实的基础上:[58]

首先,永远不要接受任何我不确定为真的事物;也就是说,要小心避免仓促和偏见,并且我的判断中不应包含任何未在我心中清晰而明确呈现且排除一切疑点的事物。

在《几何学》中,笛卡尔运用了他与皮埃尔·费马共同发现的成果。这后来被称为笛卡尔几何。[59]
\begin{figure}[ht]
\centering
\includegraphics[width=6cm]{./figures/36f19911beea68e2.png}
\caption{《哲学原理》(Principia philosophiae)标题页,1656年} \label{fig_DKE_5}
\end{figure}

笛卡尔在余生中继续发表有关数学和哲学的著作。1641年,他出版了一部形而上学论文《第一哲学沉思》(Meditationes de Prima Philosophia),用拉丁文写成,主要面向学术界。1644年,他出版了《哲学原理》(Principia Philosophiae),这是一种对《方法谈》和《第一哲学沉思》的综合。1643年,乌得勒支大学谴责了笛卡尔哲学,笛卡尔被迫逃往海牙,定居于埃蒙德-比宁。

1643年至1649年间,笛卡尔与他的女友住在埃蒙德-比宁的一家旅馆中。[60] 笛卡尔与卑尔根的主人安东尼·斯图德勒·范祖克交好,并参与了他的宅邸和庄园的设计。[61][62][63] 他还结识了数学家兼测量员德克·雷姆布兰茨·范尼罗普。[64] 他对范尼罗普的知识印象深刻,甚至向康斯坦丁·惠更斯和弗朗斯·范斯库滕推荐了他。[65]

克里斯蒂亚·默瑟认为,笛卡尔可能受到了西班牙作家兼天主教修女阿维拉的特蕾莎的影响,后者五十年前出版了《心灵城堡》,探讨了哲学反思在智力成长中的作用。[66][67]

笛卡尔通过荷兰军队中的意大利将军阿方索·波洛蒂与波希米亚公主伊丽莎白展开了长达六年的书信往来,主要涉及道德和心理学主题。[68] 与这段书信往来相关,1649年他出版了《心灵激情》(Les Passions de l'âme),并将其献给公主。由克洛德·皮科特神父翻译的《哲学原理》法语版于1647年出版,同样献给伊丽莎白公主。在法语版的前言中,笛卡尔称赞了通过哲学来获得智慧的途径。他指出了获得智慧的四个通常来源,并最终指出还有第五个更好且更安全的来源,即对第一原因的探寻。[69]

\textbf{瑞典}
\begin{figure}[ht]
\centering
\includegraphics[width=8cm]{./figures/f7b5996a40a40b4c.png}
\caption{笛卡尔与瑞典女王克里斯蒂娜在斯德哥尔摩的谈话} \label{fig_DKE_6}
\end{figure}
到1649年,笛卡尔已成为欧洲最著名的哲学家和科学家之一。[57] 当年,瑞典女王克里斯蒂娜邀请他到宫廷,协助创办一所新的科学学院,并教授她有关爱情的思想。[70] 笛卡尔接受了邀请,在冬季前往瑞典帝国。[71] 克里斯蒂娜对《心灵激情》很感兴趣,并激励笛卡尔将其出版。[72]

他居住在皮埃尔·沙努家中,住址位于斯德哥尔摩的Västerlånggatan,距离三皇冠城堡不到500米。在那里,沙努与笛卡尔使用托里拆利水银气压计进行观测。[70] 笛卡尔还向布莱兹·帕斯卡发起挑战,首次在斯德哥尔摩进行气压读数实验,以观察大气压力是否可以用于天气预报。[73]
\subsubsection{逝世}
笛卡尔安排在女王生日后每周三次在清晨5点给克里斯蒂娜女王授课,地点在她寒冷且通风的城堡中。然而,到了1650年1月15日,女王实际上只与笛卡尔会面了四五次。[70] 很快,双方明显地表现出不喜欢对方;她不感兴趣他的机械哲学,而他也对她对古希腊语言和文学的兴趣不以为然。[70] 1650年2月1日,笛卡尔感染肺炎,并于2月11日在沙努特家中去世。[74]

“昨天早上约凌晨四点,笛卡尔先生在法国大使沙努特先生的住所去世。据我所知,他因患胸膜炎病了几天。但由于他不愿服用药物,似乎出现了高烧。随后,他一天内进行了三次放血,但未失大量血。女王陛下对他的去世深表哀悼,因为他是如此博学之人。他的遗体被制成了蜡像。他并不打算死在这里,因为在去世前不久他决定在机会允许时返回荷兰。”[75]

根据沙努特的说法,笛卡尔死因是肺炎,而克里斯蒂娜的医生约翰·范·沃伦则认为是支气管肺炎,但他未获许可进行放血治疗。[76] (冬季似乎很温和,[77] 除了1月下半月的严寒,笛卡尔本人对此有所描述;然而,“此话可能既是笛卡尔对知识氛围的看法,也是对天气的描述。”)[72]
\begin{figure}[ht]
\centering
\includegraphics[width=7cm]{./figures/630c8d75edc0654b.png}
\caption{笛卡尔的墓(中间,带有铭文细节),位于巴黎圣日耳曼德佩修道院} \label{fig_DKE_7}
\end{figure}
\begin{figure}[ht]
\centering
\includegraphics[width=7cm]{./figures/38f749404f30d034.png}
\caption{笛卡尔纪念碑,建于1720年代,位于阿道夫·弗雷德里克教堂} \label{fig_DKE_8}
\end{figure}
E. Pies 基于医生范·沃伦的一封信对这一说法提出质疑;然而,笛卡尔拒绝了他的治疗,之后也出现了更多质疑这一说法真实性的观点。[78] 在2009年,德国哲学家西奥多·埃贝特提出,笛卡尔可能是被反对其宗教观点的天主教传教士雅克·维奥格毒死的。[79][80] 作为证据,埃贝特指出,笛卡尔的侄女凯瑟琳·笛卡尔在1693年撰写的《笛卡尔哲学家之死报告》中,提到她的叔叔在去世前两天接受“圣餐”时,似乎含蓄地提到了一次投毒。[81]

作为一个身处新教国家的天主教徒,[82][83][84] 笛卡尔被安葬在斯德哥尔摩的阿道夫·弗雷德里克教堂的墓地中,该教堂的墓地主要用于埋葬孤儿。他的手稿归沙努特的妹夫克劳德·克莱尔塞利耶所有,他是一位虔诚的天主教徒,开始通过选择性地删减、添加和发表笛卡尔的信件来将其“塑造成一位圣人”。[85][86]: 137–154  1663年,教皇将笛卡尔的作品列入禁书目录。1666年,在他去世16年后,笛卡尔的遗骸被运回法国,安葬在圣艾蒂安迪蒙教堂。1671年,路易十四禁止了所有笛卡尔主义的讲座。尽管1792年国民公会计划将他的遗骸迁至先贤祠,他最终于1819年被重新安葬在圣日耳曼德佩修道院,但失去了一个手指和头骨。[注8] 他据称的头骨现在在巴黎人类博物馆,[87] 但一些2020年的研究表明,这可能是伪造的。原始头骨可能在瑞典被分成若干部分并赠送给私人收藏者;其中一部分于1691年抵达隆德大学,并至今保存于该校。[88]
\subsection{哲学著作}
\begin{figure}[ht]
\centering
\includegraphics[width=6cm]{./figures/3f1202ed1566e010.png}
\caption{勒内·笛卡尔工作中} \label{fig_DKE_9}
\end{figure}
在《方法谈》中,笛卡尔尝试得出一套可以毫无疑问地被视为真实的基本原则。为此,他使用了一种被称为“极端/形而上学的怀疑”的方法,有时也称为“方法论怀疑”或“笛卡尔怀疑”:他拒绝任何可以被怀疑的想法,然后重新确立它们,以获得真正知识的坚实基础。[89] 笛卡尔从头开始构建他的思想,这在《第一哲学沉思》中得以体现。他将此与建筑进行类比:先将地表土移开,以建立新的建筑或结构。笛卡尔将他的怀疑称为“土壤”,将新的知识称为“建筑”。对笛卡尔而言,亚里士多德的基础主义是不完整的,而他的怀疑方法则增强了基础主义。[90]

起初,笛卡尔仅得出一个首要原则:他思考。这在《方法谈》中用拉丁语表达为“Cogito, ergo sum”(英语:“我思故我在”)。[91] 笛卡尔得出结论,如果他怀疑,那么必定有某物或某人正在进行怀疑;因此,怀疑本身证明了他的存在。“这句话的简单含义是,如果有人对存在持怀疑态度,这本身就是他存在的证明。”[92] 这两个首要原则——“我思考”和“我存在”——后来通过笛卡尔在《第一哲学沉思》第三沉思中的清晰而明确的感知得到了确认:笛卡尔推理道,他对这两个原则的清晰感知确保了它们的不可怀疑性。

笛卡尔得出结论,他可以确定自己存在,因为他在思考。但是什么样的形式?他通过感官感知到自己的身体;然而,感官之前曾是不可靠的。因此,笛卡尔确定唯一不可怀疑的知识是他是一个“思维的存在”。思考是他所做的事,其力量必然来自于他的本质。笛卡尔将“思维”(cogitatio)定义为“在我内发生的、使我对其立即有意识的事情,正如我对它有意识那样”。因此,思考是一个人对其即时有意识的每一项活动。[93] 他给出理由认为清醒的思维是可与梦境区分的,并且一个人的心智不可能被“恶魔”劫持,呈现出一个虚幻的外部世界。[90]

“因此,我以为自己是用眼睛看到的东西,实际上是通过我心智中的判断力来掌握的。”[94]: 109

通过这种方式,笛卡尔开始构建一个知识体系,舍弃了感知作为不可靠的途径,而只接受演绎作为方法。[95]
\subsubsection{心身二元论}
\begin{figure}[ht]
\centering
\includegraphics[width=6cm]{./figures/bc1738493f6191a7.png}
\caption{《人》(1664年)} \label{fig_DKE_10}
\end{figure}
笛卡尔受巴黎附近圣日耳曼昂莱城堡展示的自动机启发,研究了心灵与身体之间的联系及其相互作用。[96] 他心身二元论的主要影响来自神学和物理学。[97] 心身二元论是笛卡尔的标志性学说,并渗透到他提出的其他理论中。笛卡尔的心身分离理论被称为笛卡尔二元论(或心身二元论),对后来的西方哲学产生了影响。[98] 在《第一哲学沉思》中,笛卡尔试图证明上帝的存在以及人类灵魂与身体的区别。人类是心灵与身体的结合体;[99] 因此,笛卡尔的二元论包含了心灵与身体不同但密切相连的观点。尽管许多同时代的读者难以理解心灵与身体的区分,笛卡尔认为这一概念非常直接。他使用“样式”的概念,即物质存在的方式。在《哲学原理》中,笛卡尔解释道:“我们可以清晰地理解一种物质,独立于它的样式,而相反地,我们无法理解脱离物质的样式。”理解样式脱离物质需要智力抽象,[100] 笛卡尔对此解释如下:

“智力抽象在于我将思维从这个更丰富的观念的一部分内容转移开,以便更专注于另一部分。因此,当我考虑某种形状而不想到它所属于的物质或延展时,我在进行一种心理抽象。”[100]

根据笛卡尔的观点,当两个实体可以彼此独立存在时,它们才是真正独立的。因此,笛卡尔推论出上帝与人类是独立的,人类的身体与心灵也是相互独立的。[101] 他认为身体(一个有延展的事物)与心灵(一个无延展的、非物质的事物)之间的巨大差异使二者在本体上截然不同。根据笛卡尔的不可分割性论证,心灵是完全不可分割的,因为“当我考虑心灵,或我自己,只是作为一个思维的存在时,我无法在自己内部区分任何部分;我理解自己是一个完整且单一的存在。”[102]

此外,在《沉思》中,笛卡尔讨论了一块蜡,并阐述了笛卡尔二元论的核心学说:宇宙包含两种截然不同的实体——被定义为思维的心灵或灵魂,和被定义为物质且不具思维的身体。[103] 在笛卡尔时代的亚里士多德哲学中,宇宙被认为是具有目的性或终极目的的。所有发生的事情,不论是星体的运动还是树木的生长,据说都可以通过某种目的、目标或结果来解释。亚里士多德称之为“最终原因”,这些最终原因在解释自然的运作方式时不可或缺。笛卡尔的二元论支持了传统亚里士多德科学与开普勒和伽利略的新科学之间的区别,后者在解释自然时否认了神圣力量和“最终原因”的作用。笛卡尔的二元论为后者提供了哲学依据,通过将最终原因从物质宇宙(或延展之物)中移除,转而交给心灵(或思维之物)。因此,笛卡尔的二元论为现代物理学铺平了道路,同时也为关于灵魂不朽的宗教信仰打开了大门。[104]

笛卡尔的心物二元论暗含了一种关于人类的概念。根据笛卡尔的观点,人类是心灵与身体的复合体。笛卡尔将心灵置于优先地位,认为心灵可以独立于身体存在,但身体无法独立于心灵存在。在《沉思》中,笛卡尔甚至认为,心灵是实体,而身体仅由“偶性”组成。[105] 但他确实主张心灵与身体密切相连:[106]

“自然也通过痛苦、饥饿、口渴等感觉教导我,我并不仅仅是如同船长在船中那样存在于我的身体中,而是非常紧密地与它结合在一起,甚至与之交融,以至于我和身体构成一个整体。如果不是这样,我这个纯粹的思维存在在身体受伤时不会感到疼痛,而是会通过理智感知到损伤,就像水手通过视力察觉到船上的损坏一样。”[106]

笛卡尔关于身体化的讨论引发了他二元论哲学中最令人困惑的问题之一:一个人心灵与身体的结合关系究竟是什么?[106] 因此,笛卡尔的二元论在他去世后为心身问题的哲学讨论设定了议题。[107] 笛卡尔也是一位理性主义者,相信先天观念的力量。[108] 他提出了先天知识的理论,认为所有人类通过上帝的力量在出生时便拥有知识。这一先天知识的理论后来遭到经验主义哲学家约翰·洛克(1632–1704)的反驳。[109] 经验主义主张所有知识都通过经验获得。
\subsubsection{生理学和心理学}
在1649年出版的《灵魂的激情》中,[110] 笛卡尔讨论了当时普遍的观念,即人体内含有“动物灵”。这些动物灵被认为是轻而流动的液体,在神经系统中快速循环,往返于大脑与肌肉之间。人们认为这些动物灵会影响人类的灵魂或灵魂的激情。笛卡尔将激情分为六种基本类型:惊奇、爱、憎恨、欲望、喜悦和悲伤。他认为,这些激情是原始精神的不同组合,并影响灵魂去意图或追求某些行动。例如,他主张恐惧是一种促使灵魂在身体中产生反应的激情。与他关于灵魂和身体分离的二元论教义一致,他推测大脑的某个部分是灵魂与身体之间的连接点,并将松果体确定为这一连接部位。[111] 笛卡尔认为,信号通过动物灵从耳朵和眼睛传递到松果体。因此,松果体的不同运动引发了各种动物灵的活动。他认为松果体中的这些运动基于上帝的意志,人类应渴望和喜欢对自己有益的事物。但他也指出,身体内的动物灵可能会扭曲松果体的指令,因此人类需要学习如何控制自己的激情。[112]

笛卡尔提出了一个关于身体对外界事件自动反应的理论,对19世纪的反射理论产生了影响。他主张外部运动(如触摸和声音)到达神经末端,影响动物灵。例如,火的热量会作用于皮肤上的某个点,引发一连串反应,动物灵通过中枢神经系统到达大脑,随后动物灵传回肌肉,使手移开火源。[112] 通过这一反应链,身体的自动反应不需要思维过程。[108]

最重要的是,他是最早认为灵魂也应被纳入科学研究的科学家之一。他挑战了当时灵魂为神圣的观念,因此宗教权威视其书籍为危险。[113] 笛卡尔的著作成为了情感理论的基础,探讨了认知评估如何转化为情感过程。笛卡尔认为大脑类似于一台运作的机器,数学和力学可以解释其中复杂的过程。[114] 在20世纪,艾伦·图灵基于数学生物学创立了计算机科学,这受到了笛卡尔的启发。他的反射理论在他去世200多年后成为高级生理学理论的基础。生理学家伊万·巴甫洛夫是笛卡尔的热心崇拜者。[115]
\subsubsection{关于动物}
笛卡尔否认动物具备理性或智慧。[116] 他认为动物并非没有感知或知觉,但这些可以用机械原理来解释。[117] 尽管人类有灵魂或心智,能够感到痛苦和焦虑,动物由于没有灵魂则无法感受到痛苦或焦虑。如果动物表现出痛苦的迹象,那只是为了保护身体免受伤害,但它们缺乏必要的内在状态以真正体验痛苦。[118] 虽然笛卡尔的观点并未被普遍接受,但在欧洲和北美广为流行,使人类可以毫无顾忌地对待动物。认为动物完全不同于人类,仅是机器的观点,使虐待动物合法化并得到法律和社会规范的支持,直到19世纪中期才有所改变。[119]: 180–214 随后,查尔斯·达尔文的著作逐渐削弱了笛卡尔关于动物的观点。[120]: 37 达尔文主张人类与其他物种的连续性暗示了动物也可能会经历痛苦。[121]: 177
\subsubsection{道德哲学}
对于笛卡尔而言,伦理学是一门科学,是最高和最完美的科学。与其他科学一样,伦理学的根基在于形而上学。[95] 因此,他论证了上帝的存在,探讨了人类在自然界中的位置,提出了心身二元论,并捍卫自由意志。然而,作为坚定的理性主义者,笛卡尔明确表示理性在追求个人应当追求的美好事物时已足够,而美德在于正确的推理,应当指导人们的行为。然而,这种推理的质量取决于知识和心智状态。为此,他认为完整的道德哲学应包括对身体的研究。[122]: 189 他在与波希米亚公主伊丽莎白的通信中讨论了这一主题,并由此撰写了《灵魂的激情》,其中研究了人的身心过程和反应,着重于情感或激情。[123] 他关于人类激情和情感的作品成为其追随者(参见笛卡尔主义)哲学的基础,对文学和艺术应该是什么以及如何引发情感的观点产生了持久影响。[124]

笛卡尔和芝诺都将至高之善与美德等同。对于伊壁鸠鲁而言,至高之善是快乐,笛卡尔认为这实际上与芝诺的教义并不矛盾,因为美德带来一种精神上的愉悦,这种愉悦优于身体上的享受。对于亚里士多德认为幸福(eudaimonia)依赖于道德美德和适度财富等外在好运,笛卡尔并不否认财富对幸福的贡献,但他指出,这些外在因素在很大程度上超出个人控制范围,而心智则完全在个人控制之下。[123] 笛卡尔的道德作品出现在他人生的最后阶段,但早在《方法谈》中,他就提出了三条格言,以便在将所有观念置于怀疑的同时能够采取行动。这些格言被称为他的“临时道德”。
\subsubsection{宗教}
在第三和第五篇《沉思》中,笛卡尔分别提出了善良的上帝的证明(商标论证和本体论证明)。笛卡尔信任感官所提供的现实,因为他相信上帝赋予了他一个正常运作的心智和感官系统,不会意图欺骗他。然而,基于这一假设,笛卡尔最终确立了通过演绎和感知获得关于世界知识的可能性。因此,在认识论方面,笛卡尔的贡献包括奠基主义的概念以及理性是唯一可靠的获取知识的方法的可能性。然而,笛卡尔非常清楚,实验对于验证和确认理论是必要的。[95]

笛卡尔引用因果充足性原则[125] 来支持他关于上帝存在的商标论证,并引用卢克莱修的话作为支持:“Ex nihilo nihil fit”(“无中不能生有”)。[126] 牛津参考将该论证概括为:“我们对完美的概念与其完美的源头(上帝)相关,就像制造者在工艺品上留下的印记或商标一样。”[127] 在第五篇《沉思》中,笛卡尔提出了一种本体论论证,基于“能够思考一个至善无限存在的概念”的可能性,并认为“在我心中所有的观念中,我对上帝的观念是最真实、最清晰和最明确的。”[128]

笛卡尔自认为是虔诚的天主教徒,[82][83][84] 《沉思》的目的之一就是捍卫天主教信仰。他试图用理性为神学信仰奠定基础,但在当时遭到了强烈反对。帕斯卡尔将笛卡尔视为理性主义者和机械论者,指责他是自然神论者:“我无法原谅笛卡尔;在他的全部哲学中,他尽最大努力不依赖上帝。然而,他无法避免让上帝打个响指启动世界;此后,他再也不需要上帝了。”同时,他的强有力的同时代人马丁·斯库克则指责他持有无神论观点,尽管笛卡尔在《沉思》中对无神论进行了明确的批判。天主教会于1663年将他的书列为禁书。[129][130][131]: 274

笛卡尔还回应了对外部世界的怀疑论。通过这种怀疑方法,他并非为怀疑而怀疑,而是为了获得具体而可靠的信息,也就是确定性。他主张,感官知觉是非自愿地来到他面前的,并非出于他的意志。它们是外在于他的感官的,而笛卡尔认为,这证明了在他的思想之外确实存在某种东西,即外部世界。笛卡尔进一步指出,外部世界中的事物是物质的,因为上帝不会欺骗他关于这些传达的观念,而上帝赋予他“倾向”去相信这些观念是由物质引起的。笛卡尔还认为,实体是无需任何辅助就能存在或运作的东西。他进一步解释,只有上帝才能是真正的“实体”,但心灵也是实体,意味着它只需上帝的帮助就能运作。心灵是思维的实体,思维实体的来源是观念。[132]

笛卡尔避免涉及神学问题,专注于展示他的形而上学与神学正统之间不存在不兼容性。他避免尝试从形而上学上论证神学教义。当有人质疑他仅凭证明灵魂与身体是独立的实体并未确立灵魂不朽性时,他回应说:“我并不认为我有能力用人类理性来解决那些取决于上帝自由意志的事宜。”[133]
\subsection{数学}
\subsubsection{x 表示未知数;指数符号}
笛卡尔“发明了用 x、y 和 z 表示方程中的未知数,并用 a、b 和 c 表示已知数的惯例”。他还“开创了标准符号”的使用,用上标来表示幂或指数;例如,x² 中的 2 表示 x 的平方。[134][135]: 19
\subsubsection{解析几何}
\begin{figure}[ht]
\centering
\includegraphics[width=10cm]{./figures/17c428d841208d60.png}
\caption{使用他发明的 x 轴和 y 轴的笛卡尔坐标图} \label{fig_DKE_11}
\end{figure}
笛卡尔最持久的遗产之一是他对笛卡尔几何或解析几何的贡献,这种几何利用代数来描述几何图形;笛卡尔坐标系也因此以他命名。他首次将代数赋予知识体系中的重要地位,将其作为一种自动化或机械化推理的方法,特别是用于抽象、未知量的推理。[136]: 91–114 在此之前,欧洲数学家视几何为更为基础的数学形式,作为代数的基础。像帕西奥利、卡尔达诺、塔塔利亚和费拉里等数学家用几何证明代数规则。三次以上的方程被认为是“非真实的”,因为三维形式(例如立方体)被视为现实的最大维度。笛卡尔提出,抽象量 \(a^2\) 可以表示长度或面积。这与弗朗索瓦·韦达等数学家的观点相反,他们认为二次方必须表示面积。尽管笛卡尔没有深入研究这一主题,但他在构想更一般的代数科学或“通用数学”方面领先于戈特弗里德·威廉·莱布尼茨,将其作为符号逻辑的前身,用以符号化逻辑原理和方法,并机械化一般推理。[137]: 280–281
\subsubsection{对牛顿数学的影响}
当前流行的观点认为,笛卡尔对年轻的艾萨克·牛顿的影响最大,这可以说是他最重要的贡献之一。然而,笛卡尔的影响并非直接来自他《几何学》的法语原版,而是源于弗朗斯·范·斯库滕扩充的拉丁文第二版。[138]: 100 牛顿延续了笛卡尔对三次方程的研究,将这一学科从希腊视角的束缚中解放出来。最重要的概念是笛卡尔对单变量的现代化处理方式。[139]: 109–129

\textbf{微积分的基础}

笛卡尔的工作为莱布尼茨和牛顿发展的微积分奠定了基础,他们将微积分应用于切线问题,从而促进了现代数学这一分支的演进。[140] 他的符号规则也是一种常用的方法,用于确定多项式的正根和负根的数量。
\subsection{物理学}
\subsubsection{哲学、形而上学与物理学}
笛卡尔通常被视为第一个强调使用理性来发展自然科学的思想家。[141] 对他而言,哲学是一种包含所有知识的思维体系,他在写给一位法国译者的信中表达了这一观点:[95]

“因此,整个哲学如同一棵树,形而上学是树根,物理学是树干,所有其他科学是从树干上生长出来的分支,归结为三大主干,即医学、力学和伦理学。所谓道德科学,我理解为最高和最完美的科学,假设其他科学已被完全掌握,它是智慧的最终境界。”
\subsubsection{力学}

\textbf{机械哲学}

笛卡尔对物理学的兴趣最早归因于他在1618年结识的业余科学家和数学家伊萨克·贝克曼,贝克曼是被称为机械哲学的新思想学派的前沿人物。在这种推理基础上,笛卡尔提出了许多关于机械和几何物理学的理论。[142] 据说他们是在布雷达市场的一个公告前相遇的,公告上列出了一个待解的数学问题。笛卡尔请求贝克曼将该问题从荷兰语翻译成法语。[143] 在随后的会面中,贝克曼向笛卡尔介绍了他关于机械理论的微粒论方法,并说服他将研究重心转向数学化的自然探索。[144][143] 1628年,贝克曼还向笛卡尔介绍了许多伽利略的思想。[144] 他们一起研究了自由落体、链形曲线、圆锥曲线以及流体静力学。两人都认为有必要创建一种将数学和物理紧密联系起来的方法。[43]

\textbf{预见“功”的概念}

尽管物理学中的“功”概念直到1826年才被正式使用,但在此之前已经存在类似的概念。[145] 1637年,笛卡尔写道:[146]

“将100磅重的物体提升一英尺两次,等同于将200磅提升一英尺,或将100磅提升两英尺。”

\textbf{动量守恒}

在1644年的《哲学原理》中,笛卡尔阐述了他对宇宙的看法,并描述了他的三条运动定律。[147] (牛顿后来基于笛卡尔的论述制定了自己的运动定律。)[142] 笛卡尔将“运动量”(拉丁语:quantitas motus)定义为物体大小和速度的乘积,[148] 并主张宇宙中的运动总量是守恒的。[148]

如果x的大小是y的两倍,但速度是y的一半,那么二者的运动量相同。

[上帝]创造了物质以及它的运动……仅通过让事物按照自然运行,他保留了相同数量的运动……就像他最初所设定的一样。

笛卡尔发现了动量守恒定律的早期形式。[149] 他认为运动量应适用于直线运动,而不是伽利略所设想的完美圆周运动。[142][149] 笛卡尔的发现不应被视为现代意义上的动量守恒定律,因为他的概念中没有将质量区分于重量或大小,而且他认为守恒的是速度而非速度的方向(即速度大小)。[150][151][152]

\textbf{行星运动}

笛卡尔关于行星运动的旋涡理论后来被牛顿否定,牛顿提出了他的万有引力定律。牛顿《自然哲学的数学原理》第二卷的大部分内容都是用来反驳笛卡尔的旋涡理论。
\subsubsection{光学}
笛卡尔也对光学领域作出了贡献。他通过几何构造和折射定律(在法国称为笛卡尔定律,更常见的称呼是斯涅尔定律)证明了彩虹的角半径为42度(即彩虹边缘和从太阳通过彩虹中心到眼睛的光线所形成的角度为42°)。[153] 他还独立发现了反射定律,并在他的光学论文中首次公开提到这一定律。[154]
\subsection{气象学}
在《方法谈》中,笛卡尔附加了一篇讨论他关于气象学理论的附录,称为《气象论》(Les Météores)。他首次提出元素由小粒子组成,这些粒子不完美地结合在一起,从而在其间留下小空间。这些空间随后被较小且运动更快的“微妙物质”填充。[155] 这些粒子根据所构成的元素而有所不同,例如,笛卡尔认为水的粒子“像小鳗鱼一样,虽然它们相互缠绕和扭曲,但并不会打结或钩住到无法轻易分开的程度。”[155] 相比之下,构成更坚固材料的粒子则以一种形成不规则形状的方式构造。粒子的大小也很重要;如果粒子较小,不仅速度更快且持续运动,而且更容易受到较大粒子的搅动。不同的特性,例如组合和形状,产生了材料的不同次要性质,如温度。[156] 这一想法构成了笛卡尔气象学理论的基础。

尽管笛卡尔否定了亚里士多德的大部分气象理论,他仍保留了一些亚里士多德使用的术语,如蒸汽和呼气。这些“蒸汽”会被太阳从“地面物质”中吸入天空并生成风。[155] 笛卡尔还提出,下降的云层会将下方的空气移开,从而产生风。下降的云层还可能产生雷声。他推测,当一片云停留在另一片云之上,且上方云层周围的空气很热时,会将上方云层周围的蒸汽凝结,使得粒子下落。当从上方云层下落的粒子与下方云层的粒子碰撞时,就会产生雷声。[156] 他将自己的雷声理论与雪崩理论相比较。笛卡尔认为,雪崩产生的轰鸣声是因为被加热的雪因此变得更重,坠落到下方的雪上。[156] 这一理论得到经验支持,“因此可以理解,为什么冬天雷声比夏天少,因为那时高层云层没有足够的热量来使它们分解。”[156]

笛卡尔提出了关于闪电产生的另一种理论。他认为闪电是由于被困在两片碰撞云层之间的“呼气”引起的。他认为,为了让这些呼气有条件产生闪电,它们必须在炎热干燥的天气中变得“精细且易燃”。[156] 每当云层相撞时,就会引起呼气的燃烧,从而产生闪电;如果上方的云层比下方的云层重,还会产生雷声。

笛卡尔还认为,云由水滴和冰组成,他认为当空气无法再支撑它们时就会降雨。如果空气不够暖,水滴不会融化,则会以雪的形式降落。而冰雹则是在云中的水滴融化后因寒冷空气再次冻结。[155][156]

笛卡尔在气象学理论中并未使用数学或仪器(因为当时还没有这些工具)来支持他的理论,而是使用定性推理来推导出他的假设。[155]
\subsection{历史影响}
\subsubsection{脱离教会教义}
\begin{figure}[ht]
\centering
\includegraphics[width=6cm]{./figures/e129ea0fbf052276.png}
\caption{《沉思录》封面} \label{fig_DKE_12}
\end{figure}
笛卡尔常被称为现代西方哲学之父,其思想方法深刻改变了西方哲学的进程,并为现代性奠定了基础。[19][157] 《第一哲学沉思》的前两章提出了著名的方法性怀疑,代表了笛卡尔作品中对现代思想影响最深远的部分。[158] 有人认为,笛卡尔本人并未意识到这一变革的革命性影响。[159] 将讨论从“什么是真实的”转向“我可以确定什么?”这一转变,实际上将真理的权威担保从上帝转移到人类自身(尽管笛卡尔自己声称他从上帝那里获得启示)——而传统“真理”概念意味着一种外在权威,“确定性”则依赖于个体的判断。

在人类中心的革命中,人类被提升为主体、行动者,一个拥有自主理性的独立存在。这一革命性步骤为现代性奠定了基础,其影响至今未绝:即人类从基督教的启示真理和教会教义中解放出来;人类自行制定法律,独立立场。[160][161][162] 在现代性中,真理的保证者不再是上帝,而是每个人类个体,每个人都是自身现实的“自觉塑造者和保证者”。[163][164] 因此,每个人都变成了理性成人、主体和行动者,而不再是顺服上帝的孩子。这一视角的变化标志了从基督教中世纪向现代时期的转变,这种转变已在其他领域被预见,而现在由笛卡尔在哲学领域加以表述。[163][165]

笛卡尔的这种人类中心主义视角,将人类理性确立为自主的基础,为启蒙时代从上帝和教会中获得解放奠定了基础。根据马丁·海德格尔的说法,笛卡尔作品的视角也为随后的全部人类学研究奠定了基础。[166] 笛卡尔的哲学革命有时被认为激发了现代人类中心主义和主观主义的兴起。[19][167][168][169]
\subsubsection{当代反响}
在商业层面,《方法谈》在笛卡尔有生之年仅发行了一个500册的版本,其中200册为作者留用。同样的命运也降临在《沉思录》唯一的法语版上,截至笛卡尔去世时仍未售罄。然而,与此同时,该书的拉丁文版在欧洲学术界广受追捧,并成为笛卡尔的商业成功之作。[170]: xliii–xliv

尽管笛卡尔在晚年已在学术界声名显赫,但在学校教授其作品却颇具争议。乌得勒支大学医学教授亨利·德·罗伊(Henricus Regius,1598–1679)因教授笛卡尔的物理学课程而被大学校长吉斯伯特·福特(Voetius)谴责。[171]

哲学教授约翰·科廷汉(John Cottingham)称笛卡尔的《第一哲学沉思》是“西方哲学的关键文本之一”。科廷汉还表示,这部作品是“笛卡尔所有著作中研究最广泛的”。[172]: 50

《理性之梦》和《启蒙之梦》作者、《经济学人》前资深编辑安东尼·戈特利布(Anthony Gottlieb)指出,笛卡尔和托马斯·霍布斯之所以在21世纪的第二个十年仍然引发争论,是因为他们的思想在某些问题上依然具有现实意义,比如“科学的进步对我们对自我和上帝观念的理解意味着什么?”以及“政府应如何处理宗教多样性?”[173]

在2018年与泰勒·科恩的采访中,阿格尼丝·卡拉德描述了笛卡尔在《沉思录》中的思想实验,笛卡尔鼓励人们对一切所信的事物进行全面系统的怀疑,以“看看自己会得出什么结论”。她说:“笛卡尔得出的结论是一种他在自己头脑中可以构建的真正的真理。”[174] 卡拉德还认为,哈姆雷特的独白——“对生命与情感本质的沉思”——类似于笛卡尔的思想实验。哈姆雷特/笛卡尔是“脱离世界的”,仿佛被“困在自己头脑中”。[174] 科恩问卡拉德,笛卡尔通过他的思想实验是否真的发现了什么真理,还是仅仅是“现代‘我们生活在模拟中’的早期版本——恶魔是模拟而不是贝叶斯推理?”卡拉德同意这种说法可以追溯到笛卡尔,尽管笛卡尔认为自己已经反驳了它。她解释道,在笛卡尔的推理中,你确实会“回到上帝的思维中”——在一个“上帝所创造的真实世界”中……整个问题在于与现实的连接,而不是虚构的存在。如果你生活在上帝创造的世界中,那么上帝可以创造真实的事物,因此你生活在一个真实的世界里。[174]
\subsubsection{笛卡尔与玫瑰十字会的关联}
关于笛卡尔是否属于玫瑰十字会仍有争议。[175]

他的姓名首字母与玫瑰十字会广泛使用的缩写“R.C.”相关联。[176] 此外,1619年笛卡尔搬到乌尔姆,而该地是国际知名的玫瑰十字会运动中心。[176] 在他前往德国的旅程中,他遇到了约翰内斯·福尔哈贝尔,此人曾表达过加入该会的意愿。[177]

笛卡尔将一部题为《世界公民波利比乌斯的数学宝库》的作品献给“全世界的学者,尤其是德国尊贵的玫瑰十字兄弟会(B.R.C.)”。这部作品并未完成,其出版情况尚不明确。[178]
\subsection{参考文献}
\subsubsection{著作}

\begin{itemize}
\item \textbf{1618年} 《音乐概要》(Musicae Compendium):关于音乐理论和音乐美学的论文,笛卡尔献给早期合作者伊萨克·贝克曼(1618年写成,1650年首次出版,死后出版)。[179]: 127–129 
\item \textbf{1626–1628年} 《指导心灵的规则》(Regulae ad directionem ingenii):未完成。1684年首次以荷兰语出版,1701年在阿姆斯特丹首次以拉丁文出版(《笛卡尔死后发表的物理与数学小品集》)。最佳的批判版本由乔瓦尼·克拉普利编辑(海牙:马提努斯·尼霍夫出版社,1966年)。
\item \textbf{约1630年} 《立体的元素》(De solidorum elementis):关于柏拉图立体和三维图形数的分类。部分学者认为该书预示了欧拉的多面体公式。未发表;1650年在笛卡尔的斯德哥尔摩遗产中发现,运回巴黎途中沉船浸泡在塞纳河中三天,被莱布尼茨在1676年抄录,随后遗失。莱布尼茨的抄本后来也遗失,约1860年在汉诺威重新发现。[180]
\item \textbf{1630–1631年}《通过自然光寻找真理》(La recherche de la vérité par la lumière naturelle):未完成的对话,1701年出版。[181]: 264ff 
\item \textbf{1630–1633年} 《世界》(Le Monde)和《人》(L'Homme):笛卡尔关于自然哲学的首次系统性阐述。《人》于1662年死后以拉丁文翻译出版;《世界》于1664年死后出版。
\item \textbf{1637年} 《方法谈》(Discours de la méthode):作为《随笔集》的导言,包括《屈光学》《气象论》和《几何学》。
\item \textbf{1637年} 《几何学》(La Géométrie):笛卡尔在数学领域的主要作品。英文译本由迈克尔·马霍尼翻译(纽约:多佛出版社,1979年)。
\item \textbf{1641年} 《第一哲学沉思》(Meditationes de prima philosophia),也称《形而上学沉思》。拉丁文版;第二版于次年出版,增加了额外的异议和回复,以及致迪内特的一封信。由吕伊纳公爵翻译的法文版可能未经笛卡尔亲自监督,于1647年出版。包括六篇异议和回复。
\item \textbf{1644年} 《哲学原理》(Principia philosophiae):拉丁文教科书,最初笛卡尔希望用其取代当时大学使用的亚里士多德教科书。1647年,克洛德·皮科特在笛卡尔的监督下完成了法文版《哲学原理》,附有致普法尔茨公主伊丽莎白的前言。
\item \textbf{1647年}《对某传单的评论》(Notae in programma):笛卡尔对其曾经的弟子亨利库斯·雷吉乌斯的回应。
\item \textbf{1648年}《人体描述》(La description du corps humain):克莱塞利耶于1667年笛卡尔去世后发表。
\item \textbf{1648年} 《与伯尔曼的对话》(Responsiones Renati Des Cartes...):记录了1648年4月16日笛卡尔与弗朗斯·伯尔曼之间问答的笔记。1895年重新发现,并于1896年首次出版。1981年让-玛丽·贝萨德编辑出版了注释的双语版(拉丁文和法文),由巴黎大学出版社(PUF)发行。
\item \textbf{1649年} 《灵魂的激情》(Les passions de l'âme):献给普法尔茨的伊丽莎白公主。
\item \textbf{1657年} 《书信集》(Correspondance)(三卷:1657年,1659年,1667年):由笛卡尔的文学执行人克莱塞利耶出版。1667年的第三版是最完整的版本,但克莱塞利耶删除了许多与数学相关的内容。
\end{itemize}
2010年1月,荷兰哲学家埃里克-扬·博斯在浏览谷歌时发现了笛卡尔的一封此前不为人知的信件,日期为1641年5月27日。博斯在宾夕法尼亚州哈弗福德的哈弗福德学院保存的亲笔签名摘要中看到了这封信。该学院并不知道这封信从未被发表过。这是过去25年中发现的第三封笛卡尔的信件。[182][183]
\subsubsection{笛卡尔全集版本}

\begin{itemize}
\item 《笛卡尔作品集》(Oeuvres de Descartes),查尔斯·亚当和保罗·塔内里编辑,巴黎:Léopold Cerf出版社,1897–1913年,13卷;新修订版,巴黎:Vrin-CNRS,1964–1974年,11卷(前五卷包含信件)。[该版本通常以AT(Adam和Tannery的首字母)加罗马数字卷号引用,例如,AT VII指《笛卡尔作品集》第7卷。]
\item 《善意研究:寻找真理及其他早期作品》(Étude du bon sens, La recherche de la vérité et autres écrits de jeunesse,1616–1631年),文森特·卡罗和吉尔·奥利沃编辑,巴黎:PUF出版社,2013年。
\item 《笛卡尔全集》(Descartes, Œuvres complètes),让-玛丽·贝萨德和丹尼斯·坎布舍尔新编版,巴黎:伽利玛出版社,出版卷:
\item 第一卷:早期作品。《心灵指导规则》,2016年。
\item 第三卷:《方法谈及随笔》,2009年。
\item 第八卷第一册:书信集1,简-罗伯特·阿尔莫加特编辑,2013年。
\item 第八卷第二册:书信集2,简-罗伯特·阿尔莫加特编辑,2013年。
\item 《勒内·笛卡尔作品集 1637–1649》,米兰:Bompiani出版社,2009年,2531页。原版全集及意大利对照翻译,由G. 贝尔吉奥索主编,I. 阿戈斯蒂尼、M. 马罗内、M. 萨维尼协作,ISBN 978-88-452-6332-3。
\item 《勒内·笛卡尔作品集 1650–2009》,米兰:Bompiani出版社,2009年,1723页。死后作品全集及意大利对照翻译,由G. 贝尔吉奥索主编,I. 阿戈斯蒂尼、M. 马罗内、M. 萨维尼协作,ISBN 978-88-452-6333-0。
\item 《勒内·笛卡尔:所有信件 1619–1650》,米兰:Bompiani出版社,2009年第二版,3104页。全新书信集及意大利对照翻译,由G. 贝尔吉奥索主编,I. 阿戈斯蒂尼、M. 马罗内、F.A. 梅斯基尼、M. 萨维尼和J.-R. 阿尔莫加特协作,ISBN 978-88-452-3422-4。
\item 《勒内·笛卡尔,伊萨克·贝克曼,马林·梅森:信件集 1619–1648》,米兰:Bompiani出版社,2015年,1696页。全集及意大利对照翻译,由朱利娅·贝尔吉奥索和简-罗伯特·阿尔莫加特主编,ISBN 978-88-452-8071-9。
\end{itemize}
\subsubsection{特定作品的早期版本}
\begin{itemize}
\item 《方法谈》(Discours de la méthode),1637年 [已归档于2016年3月4日]
\item 《笛卡尔哲学原理》(Renati Des-Cartes Principia philosophiæ),1644年 [已归档于2016年4月9日]
\item 《笛卡尔先生的世界,或光的论述》(Le monde de Mr. Descartes ou le traité de la lumière),1664年 [已归档于2016年3月4日]
\item 《几何学》(Geometria),1659年 [已归档于2020年10月24日]
\item 《第一哲学沉思》(Meditationes de prima philosophia),1670年 [已归档于2020年10月24日]
\item 《哲学作品集》(Opera philosophica),1672年 [已归档于2020年10月27日]
\end{itemize}
\subsubsection{英文全集译本}
\begin{itemize}
\item \textbf{1955年} 《哲学作品》(The Philosophical Works),译者:E.S. 哈尔丹和G.R.T. 罗斯,杜佛出版社出版。该版本通常以首字母HR(代表Haldane和Ross)加罗马数字卷号引用,例如,HR II指此版本的第2卷。
\item \textbf{1988年} 《笛卡尔哲学著作》(The Philosophical Writings of Descartes),共3卷,译者:J. 科廷汉姆、R. 斯图托夫、A. 肯尼和D. 默多克,剑桥大学出版社出版。该版本通常以首字母CSM(代表Cottingham, Stoothoff和Murdoch)或CSMK(代表Cottingham, Stoothoff, Murdoch和Kenny)加罗马数字卷号引用,例如,CSM II指此版本的第2卷。
\item \textbf{1998年} 《勒内·笛卡尔:世界及其他著作》(René Descartes: The World and Other Writings),译者及编辑:斯蒂芬·高克罗杰,剑桥大学出版社出版。(此译本主要收录了关于物理学、生物学、天文学、光学等领域的科学著作,这些作品在17和18世纪影响深远,但在现代笛卡尔哲学作品集中通常被忽略或大幅删减。)
\end{itemize}
\subsubsection{单部作品的译本}
\begin{itemize}
\item \textbf{1628年} 《指导天才的规则》(Regulae ad directionem ingenii):《自然智能指导规则》,双语版,编译:G. 赫弗南,阿姆斯特丹/亚特兰大:Rodopi出版社,1998年 [已归档于2021年8月16日]。
\item \textbf{1633年} 《世界,或光的论述》(The World, or Treatise on Light),译者:迈克尔·S·马奥尼 [已归档于2021年6月21日]。
\item \textbf{1633年} 《人论》(Treatise of Man),译者:T. S. 霍尔,马萨诸塞州剑桥:哈佛大学出版社,1972年。

\item \textbf{1637年}《方法谈、光学、几何学和气象学》(Discourse on the Method, Optics, Geometry and Meteorology),译者:P. J. 奥尔斯坎普,修订版(印第安纳波利斯:Hackett出版社,2001年)。
\item \textbf{1637年} 《笛卡尔几何学》(The Geometry of René Descartes),译者:D. E. 史密斯 & 马西亚·拉瑟姆(芝加哥:开放法庭出版社,1925年) [已归档于2021年8月16日]。
\item \textbf{1641年} 《第一哲学沉思》(Meditations on First Philosophy),译者:J. 科廷汉姆,剑桥:剑桥大学出版社,1996年。拉丁文原版。另一个英文标题:形而上学沉思。包含六项异议与答复。次年出版的第二版包括额外的异议和答复及致迪奈的信。 [HTML在线拉丁-法-英版已归档于2006年8月27日]。
\item \textbf{1644年} 《哲学原理》(Principles of Philosophy),译者:V. R. 米勒 & R. P. 米勒:(多德雷赫特/波士顿/伦敦:Kluwer Academic出版社,1982年) [已归档于2020年9月30日]。
\item \textbf{1648年} 《笛卡尔与布尔曼的对话》(Descartes' Conversation with Burman),译者:J. 科廷汉姆,牛津:克拉伦登出版社,1989年。
\item \textbf{1649年} 《灵魂的激情》(Passions of the Soul),译者:S. H. 福斯(印第安纳波利斯:Hackett出版社,1989年)。致波美拉尼亚的伊丽莎白公主 [已归档于2021年8月16日]。
\item \textbf{1619–1648年}《勒内·笛卡尔、伊萨克·贝克曼、马林·梅森:信件集 1619–1648》,编者:朱利娅·贝尔吉奥索和简·罗伯特·阿尔莫加特,米兰:Bompiani出版社,2015年,1696页,ISBN 978-88-452-8071-9。
\end{itemize}
\subsection{参见}
\begin{itemize}
\item 3587号小行星Descartes
\item 水桶论证
\item 笛卡尔圆
\item 笛卡尔唯物主义(并非笛卡尔所持或提出的观点)
\item 笛卡尔平面
\item 笛卡尔积
\item  图的笛卡尔积
\item 笛卡尔剧场
\item 笛卡尔树
\item 笛卡尔陨石坑与月球上的高地(阿波罗16号登陆点)
\item 笛卡尔数
\item 笛卡尔奖
\item 笛卡尔符号法则
\item 笛卡尔-惠更斯奖
\item 笛卡尔定理(四切圆)
\item 笛卡尔关于总角度缺陷的定理
\item 笛卡尔叶形线
\item 以勒内·笛卡尔命名的事物列表
\item 巴黎笛卡尔大学
\end{itemize}
\subsection{注释}
\begin{enumerate}
\item Étienne Gilson在《笛卡尔与神学中的自由》(La Liberté chez Descartes et la Théologie, Alcan, 1913, pp. 132–147)中认为邓斯·司各脱不是笛卡尔意志主义的来源。尽管笛卡尔与司各脱之间存在学说上的差异,“仍可以认为笛卡尔从司各脱的意志主义传统中借鉴了某些思想”。[8]
\item 尽管笛卡尔最具标志性的肖像的作者传统上被认为是弗朗斯·哈尔斯,但没有他们见面的记录。在20世纪,这一假设受到了广泛质疑。[10]
\item 形容词形式:Cartesian /kɑːrˈtiːziən, -ˈtiːʒən/
\item 西班牙哲学家戈麦斯·佩雷拉在一百年前已提出了类似的思想,即“我知道我知道一些东西,任何知道的人都存在,那么我存在”(拉丁语:nosco me aliquid noscere, & quidquid noscit, est, ergo ego sum)。
\begin{itemize}
\item  佩雷拉, 戈麦斯. 1749 [1554]. 《论灵魂不朽》(De Immortalitate Animae)。《安东尼亚珍珠》(Antoniana Margarita)。第277页。
\item  桑托斯·洛佩斯, 莫德斯托. 1986. “戈麦斯·佩雷拉,梅迪纳的医生和哲学家。” 收录于E. L. Sanz编辑的《梅迪纳-德尔坎波及其土地的历史》,卷一:诞生与扩展。
\end{itemize}
\item 参见:认识论转向。
\item 在荷兰期间,他频繁更换住所,居住地点包括多德雷赫特(1628年)、弗拉内克(1629年)、阿姆斯特丹(1629–1630年)、莱顿(1630年)、阿姆斯特丹(1630–1632年)、代芬特尔(1632–1634年)、阿姆斯特丹(1634–1635年)、乌得勒支(1635–1636年)、莱顿(1636年)、埃格蒙德(1636–1638年)、桑特普尔特(1638–1640年)、莱顿(1640–1641年)、恩德赫斯特(靠近奥赫斯特的一座城堡,1641–1643年),最终长期居住在埃格蒙德-比嫩(1643–1649年)。
\item 他在代芬特尔和阿姆斯特丹与亨里克斯·雷内里同住,并曾与康斯坦丁·惠更斯和福庇斯库斯·福尔图纳图斯·普莱普修斯会面;1648年,弗朗斯·布尔曼在埃格蒙德-比嫩对他进行了采访。亨里克斯·雷吉乌斯、扬·斯塔姆皮恩、弗朗斯·范·斯库滕、科门纽斯和吉斯伯特·沃蒂乌斯是他的主要反对者。
\item 然而,现今他的遗骨并未安葬在墓穴中。
\end{enumerate}
\subsection{参考文献}
\subsubsection{引用}
1. Tad M. Schmaltz,《激进的笛卡尔主义:笛卡尔在法国的接受》,剑桥大学出版社,2002年,第257页 [2020年11月15日存档于Wayback Machine]。\\
2. Fumerton, Richard (2000)。"知识论正当化的基础理论"。《斯坦福哲学百科全书》[2018年4月24日存档]。检索于2018年8月19日。\\
3. Bostock, D.,《数学哲学:导论》,Wiley-Blackwell,2009年,第43页 [2020年8月1日存档于Wayback Machine]:"笛卡尔、洛克、贝克莱和休谟都认为数学是我们思想的理论,但他们都没有为这一观念论主张提供任何论据,并显然认为这是无争议的"。\\
4. Gutting, Gary (1999)。《实用自由主义与现代性批判》。剑桥大学出版社。第116页。ISBN 978-0521649735。现代性始于笛卡尔对奥古斯丁主义的转变。泰勒强调"笛卡尔在许多方面深受奥古斯丁主义的影响"。\\
5. Yolton, J. W.,《实在论与表象论:本体论随笔》,剑桥大学出版社,2000年,第136页。\\
6. "真理的对应论" [2014年2月25日存档于Wayback Machine] (斯坦福哲学百科全书)。\\
7. Gaukroger 1995, 第228页。\\
8. John Schuster,《笛卡尔-斗士:物理-数学、方法与粒子机械主义1618–33》,Springer,2012年,第363页 [2020年11月15日存档于Wayback Machine],注释26。\\
9. Gillespie, Michael Allen (1994)。“第一章:笛卡尔与欺骗之神”。《尼采之前的虚无主义》。芝加哥:芝加哥大学出版社。第1–32页,263–64。ISBN 978-0226293479。Caton有力地论证了笛卡尔使用“genius malignus”代替“deus deceptor”以避免亵渎罪的指控。\\
10. Nadler, Steven,《哲学家、牧师和画家:笛卡尔的肖像》 [2020年11月15日存档于Wayback Machine] (普林斯顿,新泽西州:普林斯顿大学出版社,2013年),第174–198页。\\
11. Wells, John (2008)。《朗文发音词典》(第三版)。皮尔森朗文。ISBN 978-1-4058-8118-0。\\
12. "笛卡尔"。《柯林斯英语词典》。哈珀柯林斯。[2017年8月9日存档]。检索于2019年3月12日。\\
13. Colie, Rosalie L. (1957)。《光明与启蒙》。剑桥大学出版社。第58页。\\
14. Nadler, Steven. 2015。《哲学家、牧师和画家:笛卡尔的肖像》。普林斯顿大学出版社。ISBN 978-0-691-16575-2。\\
15. "No. 3151: Descartes". www.uh.edu. 取自2023年3月13日。\\
16. "Rene Descartes | Encyclopedia.com". www.encyclopedia.com. 取自2023年3月13日。
17. Carlson, Neil R. (2001). 《行为生理学》。Needham Heights, Massachusetts: Pearson: Allyn & Bacon。第8页。ISBN 978-0-205-30840-8。\\
18. ""我思故我在"或"我思。因此我在"?笛卡尔的名言"Cogito ergo sum"已在全世界广泛传播。在英语中通常被翻译为"我思故我在""。italki. 取自2023年11月4日。\\
19. Bertrand Russell (2004) 《西方哲学史》 [2021年8月16日存档于Wayback Machine],第511, 516–517页。\\
20. Moorman, R. H. 1943. "数学对斯宾诺莎哲学的影响 [2016年3月3日存档于Wayback Machine]"。《国家数学杂志》 18(3):108–115。\\
21. Grondin, J., 《形而上学导论:从巴门尼德到列维纳斯》,纽约:哥伦比亚大学出版社,2004年,第126页 [2021年8月16日存档于Wayback Machine]。\\
22. Phemister, Pauline (2006). 《理性主义者:笛卡尔、斯宾诺莎和莱布尼茨》。Polity Press。第16页。ISBN 0745627439。\\
23. Bruno, Leonard C. (2003) [1999]。《数学与数学家:全球数学发现的历史;卷1》。Baker, Lawrence W. Detroit, Mich.: U X L。第99页。ISBN 978-0-7876-3813-9。OCLC 41497065。
24. "怀疑主义、经院哲学与笛卡尔哲学的起源"。《怀疑主义、经院哲学与笛卡尔哲学的起源》(第二章)。剑桥大学出版社。2000年。第27–54页。doi:10.1017/CBO9780511487309.004。ISBN 978-0521452915。[2022年8月22日存档]。取自2022年8月22日。\\
25. Rodis-Lewis, Geneviève (1992)。《笛卡尔的生平与其哲学的发展》。在Cottingham, John (编)《剑桥笛卡尔指南》中。剑桥大学出版社。第22页。ISBN 978-0-521-36696-0。[2017年2月1日存档]。取自2016年1月27日。\\
26. "All-history.org"。存档于2015年1月29日。取自2014年12月23日。\\
27. Clarke 2006。\\
28. "笛卡尔,勒奈 | 网络哲学百科全书"。存档于2010年5月28日。取自2021年7月27日。\\
29. Bruno, Leonard C. (2003) [1999]。《数学与数学家:全球数学发现的历史;卷1》。Baker, Lawrence W. Detroit, Mich.: U X L。第100页。ISBN 978-0-7876-3813-9。OCLC 41497065。\\
30. Porter, Roy (1999) [1997]。《新科学》。人类的最大福祉:从古代到现代的医学史。英国:Harper Collins。第217页。ISBN 978-0-00-637454-1。\\
31. Baird, Forrest E.; Kaufmann, Walter (2008)。《从柏拉图到德里达》。新泽西州上萨德尔河:Pearson Prentice Hall。第373–77页。ISBN 978-0-13-158591-1。\\
32. 笛卡尔。[1637] 2011。《方法论》。竹北:Hyweb Technology。第20–21页 [2021年8月16日存档于Wayback Machine]。\\
33. Gaukroger 1995,第66页。\\
34. McQuillan, J. C. 2016。《早期现代美学》。马里兰州兰厄姆:Rowman & Littlefield。第45页 [2020年8月1日存档于Wayback Machine]。\\
35. "勒奈·笛卡尔 – 传记"。数学史。存档于2017年7月19日。取自2020年9月27日。\\
36. Parker, N. Geoffrey. 2007. "白山战役 [2015年5月9日存档于Wayback Machine]"(修订版)。大英百科全书。\\
37. Jeffery, R. 2018。《波希米亚的伊丽莎白公主:哲学公主》。马里兰州兰厄姆:Lexington Books。第68页 [2020年11月8日存档于Wayback Machine]。\\
38. Rothkamm, J.,《Institutio Oratoria: Bacon, Descartes, Hobbes, Spinoza》(莱顿和波士顿:Brill,2009),第40页 [2021年8月16日存档于Wayback Machine]。\\
39. Bruno, Leonard C. (2003) [1999]。《数学与数学家:全球数学发现的历史;卷1》。Baker, Lawrence W. Detroit, Mich.: U X L。第101页。ISBN 978-0-7876-3813-9。OCLC 41497065。\\
40. Otaiku AI (2018)。《笛卡尔是否患有爆炸性头部综合症?》。临床睡眠医学杂志。14 (4): 675–78。doi:10.5664/jcsm.7068。ISSN 1550-9389。PMC 5886445。PMID 29609724。\\
41. Durant, Will; Durant, Ariel (1961)。《文明的故事:第七部分,理性时代的开始》。纽约:Simon and Schuster。第637页。ISBN 978-0-671-01320-2。\\
42. Clarke 2006,第58–59页。\\
43. Durandin, Guy。1970。《哲学的原则》。介绍和注释。巴黎:Librairie Philosophique J. Vrin。\\
44. Gaukroger 1995,第132页。\\
45. Shea, William R. 1991。《数字与运动的魔力》。科学历史出版物。\\
46. Aczel, Amir D. (2006年10月10日)。《笛卡尔的秘密笔记本:数学、神秘主义和探索理解宇宙的真实故事》。Crown。第127页。ISBN 978-0-7679-2034-6。\\
47. Matton, Sylvain, 编。2013年。《关于可饮黄金的书信,附〈关于自然和混合的真正原理的认识〉的论文及库恩拉斯〈永恒智慧圆形剧场〉评论片段》,作者为尼古拉·德·维利耶。巴黎:文森特·卡罗德的序言。\\
48. Moote, A. L. 1989. 《路易十三:正义之君》。奥克兰:加利福尼亚大学出版社。第271–72页 [2021年8月16日存档于Wayback Machine]。\\
49. "鸟瞰图"。于2021年9月21日存档。检索于2022年5月3日。\\
50. Grayling, A. C. 2006. 《笛卡尔:勒内·笛卡尔的生平及其时代的地位》。Simon & Schuster。第151–52页。\\
51. 《我思,故我在:勒内·笛卡尔的生平》。David R. Godine Publisher, 2007. ISBN 978-1567923353。于2022年9月26日存档。检索于2022年5月3日。\\
52. 《启蒙的智慧》。Rowman & Littlefield, 2022. ISBN 978-1633887947。于2022年9月26日存档。检索于2022年5月3日。\\
53. "Descartes, Rene | 网络哲学百科全书"。检索于2023年5月10日。\\
54. Durant, Will; Durant, Ariel (1961). 《文明的故事:第七部分,理性时代的开始》。纽约:Simon and Schuster。第638页。ISBN 978-0-671-01320-2。\\
55. Porterfield, J., 《勒内·笛卡尔》(纽约:Rosen Publishing, 2018),第66页 [2021年8月16日存档于Wayback Machine]。\\
56. Russell Shorto, 《笛卡尔的骨头:信仰与理性冲突的骨架史》ISBN 978-0-385-51753-9(纽约:Random House, 2008)。\\
57. Bruno, Leonard C. (2003) [1999]。《数学与数学家:全球数学发现的历史》,卷1。Baker, Lawrence W. 底特律,密歇根州:U X L。第103页。ISBN 978-0-7876-3813-9。OCLC 41497065。\\
58. 笛卡尔,《方法论的谈话》(竹北:Hyweb Technology, 2011),第88页 [2021年8月16日存档于Wayback Machine]。\\
59."皮埃尔·德·费马 | 传记与事实"。《大英百科全书》。于2017年11月15日存档。检索于2017年11月14日。\\
60."海伦娜·扬斯是谁?"。2015年6月30日。\\
61."安东尼斯·斯图德勒·范·祖尔克,佛兰芒商人兼贝亨领主"。2016年2月11日。\\
62."不平等的友谊 | LeesKost"。\\
63."贝亨的古府"。\\
64.玛莉丝·瑞克斯(编辑)(2012)。《迪克·伦勃朗兹·范·尼尔普(1610-1682)的通信》。海牙:荷兰休更斯研究所出版社。ISBN 9789087592714。\\
65."彼得·范登伯格的新书:鞋匠与哲学家"。2022年6月3日。\\
66.梅瑟,C.,《笛卡尔对阿维拉的特雷莎的负债,或为何我们应研究哲学史上的女性》 [2021年8月16日存档于Wayback Machine],《哲学研究》174,2017年。\\
67.克雷格,D. J.,《她思考,因此我存在》 [2021年8月16日存档于Wayback Machine],《哥伦比亚杂志》,2017年秋季。\\
68.哈斯,E.,《笛卡尔女性:旧制度下理性话语的各种版本和颠覆》,(伊萨卡:康奈尔大学出版社,1992年),第67-77页 [2021年8月16日存档于Wayback Machine]。\\
69.布洛姆,John J.,《笛卡尔,他的道德哲学与心理学》。纽约大学出版社,1978年。ISBN 0-8147-0999-0。\\
70.奥克曼,Susanna (1991)。《克里斯蒂娜与笛卡尔:解构一个神话》。载《瑞典女王克里斯蒂娜及其圈子:17世纪哲学自由主义者的转型》。布里尔知识史研究系列,第21卷。莱顿和波士顿:布里尔出版社。第44–69页。doi:10.1163/9789004246706_004。ISBN 978-90-04-24670-6。ISSN 0920-8607。\\
71.布鲁诺,Leonard C. (2003) [1999]。《数学与数学家:世界数学发现的历史》。Baker, Lawrence W. 底特律,密歇根州:U X L。第103–104页。ISBN 978-0-7876-3813-9。OCLC 41497065。\\
72.史密斯,Kurt(2010年秋季)。"笛卡尔的生平与著作"。《斯坦福哲学百科全书》。于2021年3月23日存档。检索于2005年5月2日。\\
73.《现代气象学:由气象学会主办的六场讲座系列:于1878年和1879年举行》,第73页。\\
74.布鲁诺,Leonard C.(2003年)[1999年]。《数学与数学家:世界数学发现的历史》。Baker, Lawrence W. 底特律,密歇根州:U X L。第104页。ISBN 978-0-7876-3813-9。OCLC 41497065。\\
75.《迪克·伦勃朗兹·范·尼尔普的通信(1610–1682)》第61页,第84页。\\
76.“有证据表明勒内·笛卡尔被谋杀了”。《新观察家》(法语)。2010年2月12日。2019年5月20日存档。检索于2020年9月27日。\\
77.“1529至1990年北波罗的海冬季严寒季节的重建与分析,S. Jevrejeva(2001),第6页,表3”(PDF)。2021年3月9日存档(PDF)。检索于2020年1月27日。\\
78.皮耶斯,E.,《笛卡尔谋杀案》,索林根,1996年,以及埃伯特,T.,《勒内·笛卡尔之谜般的死亡》,阿沙芬堡,Alibri出版社,2009年。法语翻译版:《笛卡尔之死的谜团》,巴黎,埃尔曼出版社,2011年。\\
79.“笛卡尔被天主教神父毒死”——《卫报》,2010年2月14日。《卫报》。2010年2月14日。2013年9月9日存档。检索于2014年10月8日。\\
80.埃伯特,Theodor(2009年)。《勒内·笛卡尔之谜般的死亡》(德语)。Alibri出版社。ISBN 978-3-86569-048-7。2021年8月16日存档。检索于2020年8月11日。\\
81.埃伯特,Theodor(2019年)。“笛卡尔是否因中毒死亡?”《早期科学与医学》24(2): 142–185。doi:10.1163/15733823-00242P02。S2CID 199305288。\\
82.加尔斯坦,奥斯卡(1992年)。《罗马与斯堪的纳维亚的反宗教改革:古斯塔夫·阿道夫和瑞典女王克里斯蒂娜的时代,1622-1656》。布里尔出版社。ISBN 978-90-04-09395-9。2021年8月3日存档。检索于2020年10月3日。\\
83.罗迪斯-路易斯,吉纳维芙(1999年)。《笛卡尔:他的生活与思想》。康奈尔大学出版社。ISBN 978-0-8014-8627-2。2021年7月25日存档。检索于2020年10月3日。\\
84.奥皮,格雷厄姆;特拉卡基斯,N. N.(2014年)。《早期现代宗教哲学:西方宗教哲学史,第3卷》。劳特利奇出版社。ISBN 978-1-317-54645-0。2021年8月16日存档。检索于2020年10月3日。\\
85."Andrefabre.e-monsite.com"。2014年11月5日存档。检索于2014年12月21日。\\
86.沃森,R.,《我思故我在:勒内·笛卡尔的生平》(波士顿:David R. Godine,2002年),第137–154页。2021年7月25日存档。\\
87.“5位头颅被盗的历史人物”。《Strange Remains》。2015年7月23日。2016年12月22日存档。检索于2016年11月29日。\\
88.“(据称)笛卡尔头骨的碎片”。\\
89.科彭哈弗,丽贝卡。“怀疑论的形式”。2005年1月8日存档。检索于2007年8月15日。\\
90.纽曼,莱克斯(2016年)。Zalta,Edward N.(编辑)。《斯坦福哲学百科全书》(2016年冬季版)。斯坦福大学形而上学研究实验室。2021年3月8日存档。检索于2017年2月22日。\\
91“我思故我在 | 哲学”。《大英百科全书》。2015年6月16日存档。检索于2021年7月28日。\\
92.‘十本书:由拉杰·珀斯威德选择’。《英国精神病学杂志》。2003年11月10日存档。检索于2007年11月13日。\\
93.笛卡尔,勒内(1644年)。《哲学原理》。第IX卷。\\
94.巴特,C.,《物质与心灵》(剑桥,麻省:麻省理工学院出版社,2019年),第109页。2021年8月16日存档。\\
95.笛卡尔,勒内。“作者致《哲学原理》法文译者的信,作为前言”。约翰·维奇译。2018年1月17日存档。检索于2011年12月6日。\\
96.奥尔森,理查德(1982年)。《科学的神化与科学的反抗:科学在西方文化中的历史意义》。第2卷。加利福尼亚大学出版社,第33页。\\
97.沃森,理查德·A.(1982年1月)。“是什么驱动心灵:笛卡尔二元论的考察”。《美国哲学季刊》。第19卷,第1期。伊利诺伊大学出版社:73-81。JSTOR 20013943。\\
98.戈贝尔特,R. D.,《心灵与身体的舞台:笛卡尔剧院中的激情与互动》(斯坦福:斯坦福大学出版社,2013年)。2021年8月16日存档。\\
99.大卫·坎宁(2014年)。《剑桥笛卡尔《沉思》的指南》。剑桥大学出版社,第277页。ISBN 978-1-107-72914-8。\\
100.大卫·坎宁(2014年)。《剑桥笛卡尔《沉思》的指南》。剑桥大学出版社,第278页。ISBN 978-1-107-72914-8。\\
101.大卫·坎宁(2014年)。《剑桥笛卡尔《沉思》的指南》。剑桥大学出版社,第279页。ISBN 978-1-107-72914-8。\\
102.大卫·坎宁(2014年)。《剑桥笛卡尔《沉思》的指南》。剑桥大学出版社,第280页。ISBN 978-1-107-72914-8。\\