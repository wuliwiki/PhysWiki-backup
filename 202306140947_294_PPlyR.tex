% 本原多项式(高等代数)
% keys 本原多项式|有理系数多项式|整系数多项式|线性代数|高等代数|多项式|primitive polynomial

\addTODO{词条:多项式(高等代数)}

任何一个有理系数多项式乘以一个整数,总能得到一个整系数多项式,且二者的根完全一样;类似地,任何整系数多项式,如果其系数有公共整数因子,那也可以用这个因子去除该多项式,得到的还是整系数多项式,且根不变。

综上,研究有理系数多项式的根时,可以把焦点完全集中在下述“本原多项式”上。


\subsection{本原多项式及其性质}


\begin{definition}{本原多项式}\label{def_PPlyR_1}

若整系数多项式$f(x)$的各系数之最大公因子是$1$,则称$f(x)$为一个\textbf{本原多项式(primitive polynomial)}。

\end{definition}

显然,每个有理系数多项式都唯一对应一个本原多项式。首项系数为$1$的多项式(简称为首一多项式)必为本原多项式。

下面给出有关本原多项式基本性质的两个引理:

\begin{lemma}{}\label{lem_PPlyR_1}
若$f$和$g$都是本原多项式,则$fg$也是本原多项式。
\end{lemma}

\textbf{证明}:

设$f(x) = \sum_{i=0}^n a_ix^i$,$g(x) = \sum_{j=0}^m b_jx^j$。由\autoref{def_PPlyR_1} ,任取正整数$k>1$,都存在$i_0, j_0$使得$k\not\mid a_{i_0}b_{j_0}$。

考虑
\begin{equation}\label{eq_PPlyR_1}
\begin{aligned}
f(x)g(x) &= a_0b_0 + \sum_{i+j=1}a_ib_jx + \sum_{i+j=2}a_ib_jx^2+\cdots +a_nb_mx^{n+m}~.
\end{aligned}
\end{equation}

对于上面任取的正整数$k$,$k\mid a_0b_0$一共带来两种情况:$k$整除$a_0, b_0$中的一个但不整除另一个,或$k$同时整除$a_0, b_0$。我们看看这两个情况分别导致什么结果:

假设$k\mid a_0$,且$k\not\mid b_0$。则当$k\mid a_1b_0+a_0b_1$时,必有$k\mid a_1$。于是,如果$k\mid\sum_{i+j=d}a_ib_j$对于$d<i_0+j_0$都成立,那么可推出$k\mid a_d$对于$d<i_0+j_0$都成立。

假设$k\mid a_0$且$k\mid b_0$。则当$k\mid a_2b_0+a_1b_1+a_0b_2$时,必有$k\mid a_1b_1$。于是问题又回到原点。

这么一来,如果各$k\mid\sum_{i+j=d}a_ib_j$对于$d<i_0+j_0$都成立,则要么所有满足$i<i_0+j_0-1$的$a_i$都能被$k$整除,要么要么所有满足$j<i_0+j_0-1$的$b_j$都能被$k$整除。

现在考虑\autoref{eq_PPlyR_1} 右边的第$i_0+j_0$项系数。由上一段的结论,该系数求和式中,除去$a_{i_0}b_{j_0}$以外的所有$a_ib_j$都能被$k$整除。由于已知$k\mid a_{i_0}b_{j_0}$,故知$k$不整除这一项。

由$k$的任意性,知\autoref{eq_PPlyR_1} 各项系数的最大公因子是$1$。于是$fg$是本原多项式。



\textbf{证毕}。



\begin{lemma}{}\label{lem_PPlyR_2}
设$f(x)=g(x)h(x)$,$g$是本原多项式。

如果$f$是整系数多项式,那么$h$是整系数多项式。

如果$f$是本原多项式,那么$h$是本原多项式。
\end{lemma}

\textbf{证明}:

设$h(x)$的本原多项式是$\frac{1}{k}h(x)$,则由\autoref{lem_PPlyR_1} ,$\frac{1}{k}f(x)$是本原多项式。

如果$f$是本原多项式,则$k=1$,从而$h$是本原的。

如果$f$是整系数多项式,则$k$是正整数,从而$h$是整系数多项式(因为本原)$\frac{1}{k}h$乘以一个正整数$k$的结果,因而也是整系数的。



\textbf{证毕}。




\subsection{有理系数多项式的根}

\begin{theorem}{}
设$f(x)=\sum_{i=0}^n a_ix^i$是一个\textbf{本原多项式}。如果有理数$\frac{v}{u}$是它的一个根,其中整数$u, v$互素,则
\begin{equation}\label{eq_PPlyR_2}
u\mid a_n, v\mid a_0
\end{equation}
\end{theorem}

\textbf{证明}:

由题设,$(ux-v)\mid f(x)$,且$(ux-v)$是本原的。设$f(x)=(ux-v)h(x)$,则由\autoref{lem_PPlyR_2} ,知$h(x)$是本原多项式。

设$h(x)=\sum_{j=0}^{n-1} b_ix^i$,则$a_0=vb_0$,且$a_n=ub_{n-1}$,于是得证\autoref{eq_PPlyR_2} 。

\textbf{证毕}。














