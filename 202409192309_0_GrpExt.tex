% 群的扩张
% keys 扩张|group extension|正合序列|exact sequence|短正合序列|short exact sequence
% license Xiao
% type Tutor


\pentry{直积和半直积(群)\nref{nod_GrpPrd}}{nod_4306}

注:本节参考《代数学基础》,本节主要编辑即是此书作者。

如果说求商群的过程是在“模糊化”一个群,那么直积可以理解为“精细化”一个群。

具体地,考虑群直积$G\times H$,则$\{e\}\times H$是其一个正规子群,于是求商群$G\times H/\{e\}\times H$的过程就是把第一分量相同的元素$(g, h_1)$和$(g, h_2)$都视为同一个元素,即模糊了它们之间的差别。反之,已知群$G$和$H$,则求直积的过程可以理解为把每个元素$g\in G$细化为集合$\{(g, h)\mid h\in H\}$,从一个点变成更多点。


那么求商群和求直积是不是互为逆运算呢?很可惜,并不是。



\begin{example}{}\label{ex_GrpExt_1}
考虑循环群$\mathbb{Z}/4\mathbb{Z}$和Klein群$K_4=\mathbb{Z}/2\mathbb{Z}\times \mathbb{Z}/2\mathbb{Z}$。二者都含$4$个元素,都有正规子群$\mathbb{Z}/2\mathbb{Z}$,但$\mathbb{Z}/4\mathbb{Z}\neq \mathbb{Z}/2\mathbb{Z}\times \mathbb{Z}/2\mathbb{Z}$,故不能循环群$\mathbb{Z}/4\mathbb{Z}$写成它的商群和正规子群的直积。
\end{example}




如何正确表达求商群的逆运算呢?如果说求商群$G/N$是把正规子群$N$的每个左陪集都看成一个元素,抹去其运算细节,那么反过来,把$G/N$中的每个元素都扩张为一个群,就能得到$G$,我们称这个过程为\textbf{群的扩张}。



\begin{definition}{群的扩张}\label{def_GrpExt_1}
给定群$K$和$N$,如果存在一个群$G$和群同态$f:G\to K$,使得$\opn{ker}f\cong N$,则称$G$为群$K$过群$N$的\textbf{扩张(extension)}。

称$N$为该扩张的\textbf{扩张核},或简称为核。
\end{definition}




群的扩张还可以用一种更简洁的语言来表达。



\begin{definition}{正合序列}

给定群$G_i$,如果存在同态$f_i:G_i\to G_{i+1}$,使得$\opn{Im}f_i=\opn{ker}f_{i+1}$,则称如下的序列
\begin{equation}
    \cdots \xrightarrow{f_{i-2}} G_{i-1} \xrightarrow{f_{i-1}} G_i \xrightarrow{f_i} G_{i+1} \xrightarrow{f_{i+1}}\cdots~
\end{equation}
为一个\textbf{正合序列(exact sequence)}。

不至于混淆时,也可以省略不写箭头上方的同态。

令$1$表示平凡群,则称
\begin{equation}
    1\xrightarrow{}H\xrightarrow{\lambda}G\xrightarrow{\mu}K\xrightarrow{}1~
\end{equation}
为一个\textbf{短正合序列(short exact sequence)}。

\end{definition}


按照正合序列的定义,短正合序列中$\lambda$必是单射,这意味着$H$同构于$G$的某个子群;同时根据\textbf{群同态基本定理}(\autoref{exe_Group2_1}),$K\cong G/H$,这意味着我们可以用短正合序列来定义群的扩张:




\begin{exercise}{群的扩张(另一定义)}
证明:“群$G$是群$K$过群$H$的扩张”当且仅当“存在短正合序列$1\xrightarrow{}H\xrightarrow{\lambda}G\xrightarrow{\mu}K\xrightarrow{}1$”。
\end{exercise}



接下来,我们看一些群扩张的例子,加深体会。



\begin{example}{}

\autoref{ex_GrpExt_1} 中的两个群,分别可以写成以下扩张:
\begin{equation}
\left\{
\begin{aligned}
    &1\xrightarrow{}\mathbb{Z}/2\mathbb{Z}\xrightarrow{\lambda_1}K_4\xrightarrow{\mu_1}\mathbb{Z}/2\mathbb{Z}\xrightarrow{}1, \\
    &1\xrightarrow{}\mathbb{Z}/2\mathbb{Z}\xrightarrow{\lambda_2}K_4\xrightarrow{\mu_2}\mathbb{Z}/2\mathbb{Z}\xrightarrow{}1~. 
\end{aligned}
\right. 
\end{equation}

这个例子说明,同一个群过同一个群的扩张,方式不唯一。

\end{example}




\begin{example}{}
考虑\textbf{满同态}$\mu:\opn{S}_3\to\mathbb{Z}/2\mathbb{Z}$,其中奇置换都映射到$1$,偶置换映射到$0$。考虑$\opn{ker}\mu=\{(1), (1\ 2\ 3), (1\ 3\ 2)\}=\mathbb{Z}/3\mathbb{Z}$,易得短正合序列:
\begin{equation}
1\xrightarrow{}\mathbb{Z}/3\mathbb{Z}\xrightarrow{\lambda}\opn{S}_3\xrightarrow{\mu}\mathbb{Z}/2\mathbb{Z}\xrightarrow{}1~. 
\end{equation}

这个序列说明,结构简单的阿贝尔群也能扩张出结构复杂的非阿贝尔群。
\end{example}


\begin{example}{}
从群扩张的角度,可以清晰地看出正交群和自旋群的结构差异:

\begin{equation}
    1\xrightarrow{}\opn{SO}_3\xrightarrow{}\opn{O}_n\xrightarrow{}\mathbb{Z}/2\mathbb{Z}\xrightarrow{}1~. 
\end{equation}
\begin{equation}
    1\xrightarrow{}\mathbb{Z}/2\mathbb{Z}\xrightarrow{}\opn{Spin}_n\xrightarrow{}\opn{SO}_3\xrightarrow{}1~. 
\end{equation}

直观来说,正交群$\opn{O}_n$可以看成是给特殊正交群整体加了一个“镜像”,自旋群$\opn{Spin}_n$则是给特殊正交群的每个元素加了一个“镜像”。

\end{example}






























