% Matlab 符号计算和变精度计算
% keys 符号计算|Matlab|变精度计算|任意精度计算|求导|积分|特殊函数

\pentry{Matlab 的程序调试及其他功能\upref{MatOtr}}

Matlab 的符号计算需要符号计算工具箱, 取决于你的证书类型, 可能需要额外购买. 另外需要提醒的是, 虽然 Matlab 和 Python 都有符号计算功能, 但在符号计算领域 Mathemtica\upref{Mma} 才更为主流, 其默认界面也更适合符号计算.

\subsection{符号变量和符号表达式}
\begin{itemize}
\item Matlab 中用于储存符号计算表达式的变量类型为 \verb|sym|. 可以用 \verb|syms 变量1 变量2 ...| 声明变量类型为 \verb|sym|. 例如 \verb|syms x y z;|. Matlab 的大部分自带算符和函数支持 \verb|sym| 类型的变量, 例如 \verb|x^2| 就是 \verb|sym| 类型的表达式 $x^2$, $x$ 并不是一个数值而是符号. 若此时令 \verb|expr = x^2|, 那么用 \verb|class(expr)| 可以验证 \verb|expr| 的类型也是 \verb|sym|.

\item 除了使用 \verb|syms| 一次声明几个符号变量, 也可以使用 \verb|sym(字符串)| 其中字符串只能是变量名或数字. 例如 \verb|syms x; expr = x^2;| 得到的 \verb|expr| 和 \verb|expr = sym('x')^2;| 得到的 \verb|expr| 是等效的.

\item \verb|str2sym(字符串)| 可以把字符串转换为符号表达式. 例如 \verb|str2sym('pi')| 或 \verb|str2sym('sqrt(3)/2')|.

\item  对表达式求导如 \verb|diff(sym('x')^2)|. 若要求偏导, \verb|diff(sym('x')^2 * sym('y')^3)| 默认对 \verb|x| 求偏导, 结果是 $2x y^3$. 也可以声明对 \verb|y| 求偏导, 如 \verb|diff(sym('x')^2 * sym('y')^3, sym('y'))|, 或者更简洁地, \verb|diff(sym('x')^2 * sym('y')^3, 'y')|. 这是因为, 如果函数的一个参数需要符号表达式, 但如果输入时使用了字符串或者数值, 那么 Matlab 就会自动将其用 \verb|sym()| 函数进行转换. 建议总是声明对哪个变量求偏导.

\item  要把一个数字作为符号, 可以使用例如 \verb|sym('2')| 或 \verb|sym('7/3')|, 但这仅限于分式, 不允许诸如 \verb|sym('sqrt(2)')| 这样的用法, 应该用 \verb|str2sym('sqrt(2)')|.

\item 另一种方法是使用 \verb|sym(数值)|. 例如 \verb|sqrt(sym(2))| 的结果是表达式 $\sqrt 2$, 而不同于数值计算中的 \verb|sqrt(2) = 1.414...|. 由于符号计算是精确无误差的, 无理数 $\sqrt{2}$ 并不会被自动转换为小数形式. 另一个例子, \verb|d = 3.1; sqrt(sym(d))| 得到的是表达式 $961/100$.

\item  \verb|sym(数值)| 会自动\textbf{猜测} \verb|数值| 所代表的根式, 例如 \verb|sym(0.866025403784439)| 的结果是表达式 $\sqrt{3}/2$, 又例如 \verb|sym(pi)| 的结果是精确的圆周率符号 $\pi$. 注意在较新版的 Matlab 中, \verb|sym('pi')| 将得到名为 \verb|pi| 的普通变量, 而不是圆周率.

\item 如果不希望猜测, 那么可以用 “双精度和变精度浮点数测试(Matlab)\upref{FltMat}” 中的 \verb|num2sym(双精度)|.

\item 若 \verb|sym(...)| 作用在一个数组上(可以是任何上述类型), 那么则逐个元素作用, 并生成 \verb|sym| 类型的数组.

\item 一个 \verb|sym| 类型和一个 double 类型进行 \verb|+, -, *, /, ^| 等运算时, \verb|double| 类型的数会自动被 \verb|sym()| 函数转换为 \verb|sym| 类型. 例如 \verb|sym(1)/3| 和 \verb|sym(1)/sym(3)| 是等效的. 这样可以让输入更简洁.
\end{itemize}

\subsubsection{符号替换}
\begin{itemize}
\item \verb|subs(符号表达式,符号变量, 新表达式)| 可以把表达式中的所有 \verb|符号变量| 替换为 \verb|新表达式|. 例如 \verb|subs(sin(sym('x')), 'x', 'y')| 相当于 \verb|subs(sin(sym('x')), sym('x'), sym('y'))|, 结果是 $\sin y$. 又例如 \verb|subs(sin(sym('x')), 'x', pi/4)| 的结果是 $\sqrt 2/2$.
\item 若 \verb|新表达式| 是一个数组, 则依次对每个元素进行替换, 输入一个 \verb|sym| 类型的数组.
\item 如果要对一个符号表达式求数值近似, 那么用 \verb|double()|, 例如 \verb|double(sqrt(sym(2)))| 结果是 \verb|1.414...|, 是一个双精度数值, 误差就是双精度类型的相对误差 \verb|eps|, 约为 $2.2\e{-16}$.
\end{itemize}

\subsection{变精度计算}
注意变精度计算功能往往和符号计算一同使用, 但这两个功能从实现来看是完全不同的. 变精度计算和双精度一样本质上还是数值浮点计算, 计算过程存在数值误差. 只不过我们可以规定每个变量的有效数字为任意多, 而不是统一使用双精度类型的 15-16 位有效数字.

Matlab 的变精度计算并不自带误差追踪功能, 例如两个十分相近的数相减, 返回的小数位中可能大部分都是错的. 而 Mathematica 的做法是返回更少小数位, 但保证最后一位是对的.

\begin{itemize}
\item 相比于 \verb|double(符号表达式)| 返回双精度结果, \verb|vpa(符号表达式, 位数)| 可以计算符号表达式的任意位有效数字结果, 例如 \verb|vpa(sqrt(sym('2')), 50)| 计算 $\sqrt{2}$ 的 50 位有效数字, 并能保证四舍五入后\textbf{最后一位有效数字正确}. 返回的值是一个 \verb|'sym'| 类型的对象而不是 \verb|'double'| 类型.

\item 虽然 \verb|vpa| 返回的类型也是 \verb|'sym'|, 但它是有误差的, 本质上是一个类似 \verb|double| 的浮点类型. \verb|vpa()| 就是 variable precision arithmetic, 变精度计算. 我们姑且把这种有误差的 \verb|sym| 数字称为\textbf{变精度浮点数}. 我们可以把变精度浮点数当作 \verb|double| 类型的拓展. \verb|double| 在十进制下只有约 15-16 位有效数字, 而变精度浮点数的有效数字位数可以任意指定.

\item 要检查一个 \verb|sym| 类型的对象 \verb|x| 是否有误差, 只需要在命令行中把它显示出来(可以使用 \verb|disp()| 函数, 也可以直接运行 \verb|x| 不加分号). 如果显示中出现了小数点, 那么他就是变精度浮点数. \verb|sym| 类型的整数,无论有多少位,如果没有误差则会把每一位显示出来而不是用科学计数法(因为科学计数法带小数点,表示有误差).

\item 除了 \verb|vpa()| 函数会输出变精度浮点数, 另一种方法如 \verb|sym('12.3')| 或者 \verb|sym('1.23e1')| (用 \verb|vpa| 也一样), 运行后显示结果为 \verb|12.3|, 存在小数点, 所以是变精度浮点数. 相比之下, \verb|sym('123/10')| 结果显示为 $123/10$ 该数无论参与任何计算都是绝对精确的.

\item 变精度浮点数(vpa) 的计算并不会追踪误差, 例如 \verb|sym('1.2345678902234567890323456789042345678905234567890') - sym('1.2345678902234567890323456789042345678905234500000')| 返回的是 \verb|6.789000000000011835767836416814e-45| 而不是 \verb|6.789e-45|. 而 Mathematica 就会返回 \verb|6.789e-45|.

\item 注意 \verb|vpa()| 和 \verb|sym()| 中如果有效数字超过 15-16, 要用引号, 否则会先转换为 double, 超出 15-16 位的部分丢失.

\item \verb|vpa(数值)| 和 \verb|sym(数值)| 一样会自动\textbf{猜测} \verb|数值| 所代表的根式或符号, 结果相当于 \verb|vpa(sym(数值))|. 例如 \verb|vpa(0.866025403784439)| 的结果是表达式 $\sqrt{3}/2$, 又例如 \verb|sym(pi)| 的结果是精确的圆周率符号 $\pi$. 例如 \verb|vpa(0.866025403784439) - vpa(str2sym('sqrt(3)/2'))| 结果为零, 又例如 \verb|vpa(pi, 1000)| 可以得到圆周率的 1000 位有效数字.

\item  若不希望 \verb|vpa| 猜测, 则应该使用引号, 例如 \verb|vpa('0.866025403784439') - vpa(str2sym('sqrt(3)/2'))| 不为零. 使用引号可以在当前 \verb|digits| 设置的精度内最精确地表示引号内的数字. 验证: \verb|vpa('1.866025403784439') - vpa('1.866025403784438')|

\item 猜测功能会忽略双精度最后几位的误差, 例如 \verb|vpa(1+1e-14)| 得 \verb|1.0|, \verb|sym(1+1e-14)| 得 \verb|1|. \verb|sym(pi+1e-14)| 得 \verb|pi|. 这有时候可能反而会造成麻烦.

\item 如果猜测没有发生, 那么 \verb|vpa(双精度)| 会先把 \verb|双精度| 变为 ieee 二进制表示, 然后在后面添零拓展精度. 验证: \verb|num2bin2(vpa(1.234567890223456789))| 和 \verb|num2bin2(1.234567890223456789)| 结果相同.

\item 如果不希望猜测, 那么可以用 “双精度和变精度浮点数测试(Matlab)\upref{FltMat}” 中的 \verb|num2vpa(双精度, 有效数字)|.

\item 符号工具箱自带的函数(如 \verb|gamma|, \verb|hypergeom|)会保证输出结果的最后一位正确(可以对比 Mathematica 结果), 但由于上一条中误差, 自己写的函数则无法保证.

\item \verb|hypergeom| 只有符号工具箱中有, 所以参数直接输入 double 就行, \verb|gamma| 函数有普通的版本(只支持 double 实数), 如果输入 double 复数会出错, 这是只需要用 \verb|gamma(vpa(复数))| 即可.

\item 和双精度变量一样, 若把变精度浮点数进行运算, 则其本身的误差会在运算中传递, 且在计算中会产生新的截断误差. 举例: \verb|1 + sym('1e-40') - 1| 的结果显示为 \verb|0.0|, 而 \verb|1 + sym(10)^(-40) - 1| 的结果显示 \verb|1/10000....|(40 个 0).

\item 变精度浮点数的运算的默认位数可以用 \verb|digits(整数)| 设置, 如果不设置, 则默认是 32 位. 也可以使用不含自变量的 \verb|digits()| 查看当前设置的位数. 实际上的有效位数比设置的要多 8 位, 所以默认是 40 位. 这可以用 “双精度和变精度浮点数测试(Matlab)\upref{FltMat}” 中的 \verb|digits2| 验证.

\item \verb|digits| 函数是控制全局的, 如果设置了 \verb|digits| 以后调用某函数, 那么函数内部的变精度计算也都会采用同样精度.

\item 无论参与变精度运算的数有多少个有效数字, 运算结果取当前的默认位数. 例如设置 \verb|digits(32)| 后, \verb|vpa(sym(pi)/2, 50) + vpa(sym(pi)/2, 100)| 的结果仍然是 32 位有效数字.

\item 设置了 \verb|digits| 以后, 字符串 vpa 如 \verb|vpa('1.2345...')| 相当于 \verb|vpa('1.2345...', 有效数字)| (实际上还要多 8 位, 也就是最后一位后面有 8 个零), 若两个精度高于 \verb|digits| 设置的 \verb|vpa('数字字符串')| 进行运算, 就先把高精度的有效数字减少为低精度的再进行运算. 但如果其中一个 \verb|vpa('数字字符串')| 精度低于 \verb|digits| 的设置, 就先变为 \verb|digits| 设置的精度. 例子: 先设置 \verb|digits(32)|, 则 \verb|vpa('1.2345678902234567890323456789042345678905234567890623456789') - vpa('1.23456789022345678903234567890423456')| 的结果有 8 位有效数字. 另一个例子 \verb|vpa('1.2345678902234567890323456789') - vpa('1.2345678902234567890323456788')| 结果有 12 位有效数字.

\item 例子: 若运行 \verb|digits(50)|, 再运行 \verb|(1 + sym('1e-40')) - 1| 就会得到 \verb|0.000...000999...99938892...|, 这就比默认的 32 位有效数字计算精确多了, 但还是存在误差.

\item 变精度浮点数可以和解析表达式混合使用, 例如 \verb|sym('1.2')*sqrt(2*sym('x'))|, 结果是 $1.2 \sqrt{2x}$, 其中 $1.2$ 是变精度浮点数. 相比之下, \verb|sym(12/10)*sqrt(2*sym('x'))| 得到精确的 $6\sqrt{2x}/5$.
\end{itemize}

\subsection{特殊函数}
由于一些特殊函数若用双精度计算,往往在一些区间产生较大误差. 所以 Matlab 决定只使用任意精度来计算. 以复数域的 $\Gamma$ 函数\upref{Gamma} 为例, 若直接输入 \verb|gamma(1+1i)| 则会出错, 因为 Matlab 中非符号计算版本的 \verb|gamma| 函数只支持 \verb|double| 类型的实数输入. 所以要调用符号计算工具箱提供的 \verb|gamma()|, 就输入一个 \verb|sym| 类型的变量即可. 例如 \verb|gamma(vpa(1+1i))| 返回 \verb|0.49801... - 0.15494...i|, 又例如 \verb|gamma(sym(1+1i))| 返回表达式 \verb|gamma(1 + 1i)|. 这是因为 \verb|vpa(1+1i)| 是变精度浮点数, 可以显示为小数, 而 \verb|sym(1+1i)| 不存在误差, 不能显示为小数. 当然, 我们也可以用例如 \verb|vpa(gamma(sym(1+1i)), 50)| 来求任意位有效数字.
