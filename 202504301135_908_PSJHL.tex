% 朴素集合论(综述)
% license CCBYSA3
% type Wiki

本文根据 CC-BY-SA 协议转载翻译自维基百科\href{https://en.wikipedia.org/wiki/Naive_set_theory}{相关文章}。

朴素集合论是用于讨论数学基础的几种集合论之一。\(^\text{[3]}\)与公理化集合论不同,公理化集合论是使用形式逻辑定义的,而朴素集合论则是以自然语言非正式地定义的。它描述了离散数学中常见的数学集合的各个方面(例如维恩图和关于其布尔代数的符号推理),并且足以满足当代数学中集合论概念的日常使用。\(^\text{[4]}\)

集合在数学中具有重要意义;在现代形式化处理方式中,大多数数学对象(如数字、关系、函数等)都是通过集合来定义的。朴素集合论足以应对许多目的,同时也为更为正式的处理方法提供了一个基础。
\subsection{方法}  
在“朴素集合论”的意义上,朴素理论是一种非形式化的理论,即使用自然语言描述集合及其运算的理论。这种理论将集合视为柏拉图式的绝对对象。诸如“和”、“或”、“如果...那么”、“非”、“存在某个”、“对每个”之类的词汇,按照普通数学中的用法处理。出于方便,即使在更高级的数学中,朴素集合论及其形式化方法仍然占主导地位——包括在集合论本身更为正式的设置中。

集合论的最初发展是朴素集合论。它是在19世纪末由乔治·康托尔创立的,作为他研究无限集合的一部分\(^\text{[5]}\),并由戈特洛布·弗雷格在他的《算术基础》一书中进一步发展。

朴素集合论可能指代几个非常不同的概念。它可以指:
\begin{itemize}
\item 公理化集合论的非正式呈现,例如保罗·哈尔莫斯的《朴素集合论》。
\item 乔治·康托尔早期或后期的理论以及其他非正式系统。
\item 明确不一致的理论(无论是否是公理化的),例如戈特洛布·弗雷格的理论\(^\text{[6]}\),该理论导致了罗素悖论,以及朱塞佩·皮亚诺\(^\text{[7]}\)和理查德·德德金德的理论。
\end{itemize}
\subsubsection{悖论}  
假设任何属性都可以用来形成集合,且没有限制,这会导致悖论。一个常见的例子是罗素悖论:不存在一个由“所有不包含自身的集合”组成的集合。因此,朴素集合论的一致系统必须包含对可以用来形成集合的原则的某些限制。
\subsection{康托尔的理论}  
有些人认为乔治·康托尔的集合论实际上并未涉及集合论悖论(参见Frápolli 1991)。确定这一点的一个困难在于康托尔没有提供他系统的公理化。到1899年,康托尔已经意识到从无约束解释他的理论会导致一些悖论,例如康托尔悖论\(^\text{[8]}\)和布拉利-福尔蒂悖论\(^\text{[9]}\),但他并不认为这些悖论会使他的理论失效。\(^\text{[10]}\)实际上,康托尔悖论可以从上述(错误的)假设中推导出来——即任何属性\(P(x)\)都可以用来形成集合——这里的\(P(x)\)可以是“x是一个基数”。弗雷格明确地对一个理论进行了公理化,该理论中可以解释朴素集合论的形式化版本,正是这个形式化理论在伯特兰·罗素提出悖论时所处理的,而不一定是康托尔所想的理论——如前所述,康托尔意识到几个悖论的存在。
\subsubsection{公理化理论}  
公理化集合论是为了回应这些早期理解集合的尝试而发展起来的,目的是精确定义哪些操作是允许的,以及何时允许。
\subsubsection{一致性}  
朴素集合论不一定是不一致的,只要它正确地规定了可以考虑的集合。这可以通过定义来完成,定义是隐含的公理。所有公理可以明确地陈述,就像哈尔莫斯的《朴素集合论》一样,实际上这是对通常的公理化泽梅洛-弗兰克尔集合论的非正式呈现。它被称为“朴素”的原因是其语言和符号是普通非正式数学中的用法,而且它不涉及公理系统的一致性或完备性。

同样,公理化集合论也不一定是一致的:不一定是没有悖论的。根据哥德尔的不完备性定理,一个足够复杂的一阶逻辑系统(包括大多数常见的公理化集合论)不能在理论内部被证明是一致的——即使它实际上是一致的。然而,通常认为常见的公理化系统是一致的;通过它们的公理,它们确实排除了某些悖论,如罗素悖论。基于哥德尔的定理,我们只是不知道——而且永远也无法知道——这些理论或任何一阶集合论中是否没有悖论。

“朴素集合论”这个术语今天仍然在一些文献中使用\(^\text{[11]}\),指的是弗雷格和康托尔研究的集合论,而不是现代公理化集合论的非正式对应物。
\subsubsection{实用性}  
选择公理化方法与其他方法之间的差异在很大程度上是出于方便。在日常数学中,最佳选择可能是非正式地使用公理化集合论。通常,只有在传统要求时,才会提到特定的公理,例如,选择公理在使用时常常被提到。同样,正式的证明只有在特殊情况下才会出现。这种公理化集合论的非正式使用(取决于符号)可以呈现出与下文所述的朴素集合论相同的外观。与严格的形式化方法相比,它在阅读和写作上更为简便(在大多数陈述、证明和讨论中),并且比严格形式化的方法更不容易出错。
\subsection{集合、成员关系和等式}
在朴素集合论中,集合被描述为一个明确定义的对象集合。这些对象被称为集合的元素或成员。对象可以是任何事物:数字、人、其他集合等。例如,4是所有偶整数集合的成员。显然,偶数集合是无限大的;并没有要求集合必须是有限的。
\begin{figure}[ht]
\centering
\includegraphics[width=6cm]{./figures/a0b50c3cf3b9f84d.png}
\caption{乔治·康托尔原始集合定义的段落} \label{fig_PSJHL_1}
\end{figure}
集合的定义可以追溯到乔治·康托尔。他在1915年的文章《超无限集合论的基础》中写道:

"Unter einer 'Menge' verstehen wir jede Zusammenfassung M von bestimmten wohlunterschiedenen Objekten unserer Anschauung oder unseres Denkens (welche die 'Elemente' von M genannt werden) zu einem Ganzen."

— 乔治·康托尔

集合是我们感知或思维中一组确定且彼此不同的对象的整体,这些对象被称为集合的元素。

— 乔治·康托尔
\subsubsection{关于一致性的说明}  
\begin{figure}[ht]
\centering
\includegraphics[width=6cm]{./figures/47e8e4e33d9a79c7.png}
\caption{吉useppe·皮亚诺在《算术原理新方法展示》(Arithmetices principia nova methodo exposita)一书中首次使用符号ϵ} \label{fig_PSJHL_2}
\end{figure}
从这个定义中并不能推导出集合是如何形成的,以及对集合进行哪些操作会再次产生集合。在“明确定义的对象集合”中的“明确定义”一词本身无法保证集合的确切构成和哪些不是集合的一致性和明确性。尝试实现这一点将属于公理化集合论或公理化类论的领域。

在此背景下,非正式表述的集合论问题在于,它们并未源自(也不隐含)任何特定的公理化理论,因此可能会有多个广泛不同的形式化版本,这些版本有不同的集合和不同的规则来形成新集合,而这些版本都符合原始的非正式定义。例如,康托尔的逐字定义允许在什么构成集合方面有相当大的自由。另一方面,不太可能康托尔特别关心包含猫和狗的集合,而只是关心包含纯数学对象的集合。这样的集合类的一个例子可以是冯·诺依曼宇宙。但即使确定了考虑中的集合类,在不引入悖论的情况下,哪些集合形成规则是允许的,仍然不是很清楚。

为了固定下面的讨论,“明确定义”一词应被解释为一种意图,带有隐式或显式的规则(公理或定义),以排除不一致性。目的是将一致性方面常常深奥且复杂的问题与通常更简单的上下文分开讨论。由于哥德尔第二不完备定理,即使是公理化集合论也无法明确排除所有可能的不一致性(悖论),因此这并不会妨碍朴素集合论在下面考虑的简单上下文中的实用性。它只是简化了讨论。除非特别提及,一致性从此被视为理所当然。
\subsubsection{成员关系} 
如果x是集合A的成员,那么也可以说x属于A,或者x在A中。这用x ∈ A表示。符号∈是从希腊字母epsilon("ε")派生而来,由吉useppe·皮亚诺在1889年引入,它是希腊语单词ἐστί(意思是“是”)的第一个字母。符号∉常用来表示x ∉ A,意思是“x不在A中”。
\subsubsection{等式}  
当两个集合A和B具有完全相同的元素时,它们被定义为相等的;也就是说,如果A的每个元素都是B的元素,并且B的每个元素也是A的元素。(参见外延公理。)因此,集合完全由其元素确定;描述是无关紧要的。例如,元素为2、3和5的集合等于所有小于6的素数集合。如果集合A和B相等,用符号A = B表示(如通常所示)。
\subsubsection{空集合}  
空集合,表示为∅,有时表示为{},是一个没有任何成员的集合。因为集合完全由其元素确定,所以只能有一个空集合。(参见空集合公理。)\(^\text{[12]}\) 虽然空集合没有成员,但它可以是其他集合的成员。因此,∅ ≠ {∅},因为前者没有成员,而后者有一个成员。\(^\text{[13]}\)
\subsection{指定集合}   
描述集合的最简单方式是将其元素列在大括号中(这被称为外延地定义集合)。因此,$\{1, 2\}$表示一个集合,它的唯一元素是1和2。(参见配对公理。)注意以下几点:

\begin{itemize}
\item 元素的顺序无关紧要;例如,$\{1, 2\} = \{2, 1\}$。
\item 元素的重复(多重性)无关紧要;例如,$\{1, 2, 2\} = \{1, 1, 1, 2\} = \{1, 2\}$。
\item (这些是前一节中等式定义的结果。)
\end{itemize}

指定集合  
这种符号可以被非正式地滥用,例如说$\{ \text{dogs} \}$来表示所有狗的集合,但这个例子通常会被数学家读作“包含单一元素'dogs'的集合”。

这个符号的一个极端(但正确)例子是$\{\}$,它表示空集合。

符号$\{x : P(x)\}$,有时也写作$\{x \mid P(x)\}$,用于表示包含所有满足条件$P$的对象的集合(这被称为外延地定义集合)。例如,$\{x \mid x \in \mathbb{R}\}$表示实数集合,$\{x \mid x \text{ has blonde hair}\}$表示所有有金发的事物的集合。

这种符号被称为集合构造符号(或在函数式编程的语境中称为“集合理解”)。集合构造符号的某些变体有:
\begin{itemize}
\item $\{x \in A \mid P(x)\}$表示所有已经是集合$A$的成员,并且满足条件$P$的$x$的集合。例如,如果$Z$是整数集合,那么$\{x \in Z \mid x \text{ is even}\}$表示所有偶整数的集合。(参见选择公理。)
\item $\{F(x) \mid x \in A\}$表示通过将集合$A$的成员代入公式$F$得到的所有对象的集合。例如,$\{2x \mid x \in Z\}$再次表示所有偶整数的集合。(参见替换公理。)
\item $\{F(x) \mid P(x)\}$是集合构造符号的最一般形式。例如,$\{x' \text{ 's owner} \mid x \text{ is a dog}\}$表示所有狗主人组成的集合。
\end{itemize}
\subsection{子集}  
给定两个集合$A$和$B$,如果$A$的每个元素也是$B$的元素,则称$A$是$B$的子集。特别地,每个集合$B$都是它自身的子集;如果$B$的子集不等于$B$,则称其为$B$的真子集。

如果$A$是$B$的子集,那么也可以说$B$是$A$的超集,或者说$A$包含在$B$中,或者说$B$包含$A$。用符号表示,$A \subseteq B$表示$A$是$B$的子集,$B \supseteq A$表示$B$是$A$的超集。一些作者使用符号$\subset$和$\supset$表示子集,另一些作者则仅将这些符号用于真子集。为了明确,您可以显式地使用符号$\subsetneq$和$\supsetneq$来表示不等式。

作为一个例子,设$R$为实数集合,$Z$为整数集合,$O$为奇整数集合,$P$为现任或前任美国总统的集合。那么,$O$是$Z$的子集,$Z$是$R$的子集,(因此)$O$是$R$的子集,其中在所有情况下,子集甚至可以读作真子集。并不是所有集合都可以这样比较。例如,既不是$R$是$P$的子集,也不是$P$是$R$的子集。

根据上述集合等式的定义,立即得出结论:给定两个集合$A$和$B$,$A = B$当且仅当$A \subseteq B$且$B \subseteq A$。事实上,这通常被作为集合等式的定义。通常,在尝试证明两个集合相等时,目标是证明这两个包含关系。空集合是每个集合的子集(空集合的所有元素也是任何集合$A$的成员这一陈述显然为真)。

给定集合$A$的所有子集的集合称为$A$的幂集,记作$2^A$或$P(A)$;"P"有时用手写字体表示:$\wp(A)$。如果集合$A$有$n$个元素,那么$P(A)$将有$2^n$个元素。
\subsection{全集与绝对补集}  
在某些情况下,可能会考虑所有讨论中的集合作为某个给定的全集的子集。例如,在研究实数集合$R$(及其子集)的性质时,$R$可以被视为全集。真正的全集并未包含在标准集合论中(参见下文的悖论部分),但它包含在一些非标准集合论中。

给定一个全集$U$和$U$的一个子集$A$,$A$在$U$中的补集定义为

$$A^C := \{x \in U \mid x \notin A\}.~$$

换句话说,$A^C$("A的补集";有时简写为$A'$,即"A-素")是$U$中所有不是$A$成员的元素的集合。因此,在前面提到的子集部分中,如果$Z$是全集,那么$O^C$就是偶整数的集合;如果$R$是全集,那么$O^C$就是所有实数集合,其中包含偶整数或完全不是整数的数。
\subsection{并集、交集和相对补集}  
给定两个集合$A$和$B$,它们的并集是由所有属于$A$、$B$或两者的对象组成的集合(参见并集公理)。它表示为$A \cup B$。

$A$和$B$的交集是所有既属于$A$又属于$B$的对象的集合。它表示为$A \cap B$。

最后,$B$相对于$A$的相对补集,也称为$A$和$B$的集合理论差集,是所有属于$A$但不属于$B$的对象的集合。它表示为$A \setminus B$或$A - B$。

符号表示如下:
$$A \cup B := \{x \mid (x \in A) \vee (x \in B)\}~$$  
$$A \cap B := \{x \mid (x \in A) \wedge (x \in B)\} = \{x \in A \mid x \in B\} = \{x \in B \mid x \in A\}~$$ 
$$A \setminus B := \{x \mid (x \in A) \wedge \neg (x \in B)\} = \{x \in A \mid \neg (x \in B)\}.~$$
对于$A \setminus B$,集合$B$不必是$A$的子集;这是相对补集与前一节中绝对补集($A^C = U \setminus A$)的区别。

为了说明这些概念,设$A$为左撇子人群的集合,$B$为金发人的集合。那么$A \cap B$是所有左撇子金发人的集合,而$A \cup B$是所有左撇子、金发或两者都有的人群的集合。另一方面,$A \setminus B$是所有左撇子但不是金发的人的集合,而$B \setminus A$是所有金发但不是左撇子的人的集合。

现在,设$E$为所有人类的集合,$F$为所有超过1000岁生物的集合。那么在这种情况下,$E \cap F$是什么?没有任何一个人类活过1000岁,因此$E \cap F$必须是空集合$\{\}$。

对于任何集合$A$,幂集$P(A)$在并集和交集运算下是一个布尔代数。

\subsection{有序对与笛卡尔积 } 
直观地说,一个有序对仅仅是两个对象的集合,其中一个可以被区分为第一个元素,另一个为第二个元素,并具有基本性质:两个有序对相等,当且仅当它们的第一个元素相等且第二个元素相等。

形式上,具有第一个坐标$a$和第二个坐标$b$的有序对,通常表示为$(a, b)$,可以定义为集合
$$\{\{a\}, \{a, b\}\}.~$$
由此可得,两个有序对$(a, b)$和$(c, d)$相等,当且仅当$a = c$且$b = d$。

另外,有序对也可以形式化地看作是一个集合$\{a, b\}$,并且在此集合上定义一个全序。

(符号$(a, b)$也用于表示实数线上的开区间,但上下文应明确指出所指的意义。否则,符号$]a, b[$可以用于表示开区间,而$(a, b)$用于表示有序对)。

如果$A$和$B$是集合,则笛卡尔积(或简称积)定义为:
$$A \times B = \{(a, b) \mid a \in A \text{ 且 } b \in B\}.~$$
即,$A \times B$是所有有序对的集合,其第一个坐标是$A$的元素,第二个坐标是$B$的元素。

这个定义可以扩展到有序三元组的集合$A \times B \times C$,更一般地,可以扩展到任意正整数$n$的有序$n$元组集合。甚至可以定义无限笛卡尔积,但这需要更深奥的积的定义。

笛卡尔积最早由雷内·笛卡尔在解析几何的背景下提出。如果$R$表示所有实数的集合,则$R^2 := R \times R$表示欧几里得平面,而$R^3 := R \times R \times R$表示三维欧几里得空间。
\subsection{一些重要的集合}  
有一些普遍存在的集合,其符号几乎是通用的。以下列出了一些这些集合。在列表中,$a$、$b$和$c$表示自然数,$r$和$s$表示实数。
\begin{enumerate}
\item \textbf{自然数}用于计数。黑板粗体字母$ \mathbb{N} $通常表示这个集合。
\item \textbf{整数}出现在像$x + a = b$这样的方程的解中。黑板粗体字母$ \mathbb{Z} $通常表示这个集合(来自德语"Zahlen",意为数字)。
\item \textbf{有理数}出现在像$a + bx = c$这样的方程的解中。黑板粗体字母$ \mathbb{Q} $通常表示这个集合(来自"quotient",因为$ \mathbb{R} $用于表示实数集合)。
\item \textbf{代数数}出现在多项式方程(带整数系数)的解中,可能涉及根式(包括$i = \sqrt{-1}$)和某些其他的无理数。$ \overline{\mathbb{Q}} $通常表示这个集合。上划线表示代数闭包操作。
\item \textbf{实数}表示"实数线",包括所有可以被有理数逼近的数。这些数可以是有理数或代数数,但也可能是超越数,后者不能作为有理系数多项式方程的解。黑板粗体字母$ \mathbb{R} $通常表示这个集合。
\item \textbf{复数}是实数和虚数的和:$r + si$。这里$r$或$s$(或两者)可以是零;因此,实数集合和严格虚数集合是复数集合的子集,复数集合为实数集合提供了代数闭包,意味着每一个有理系数的多项式都有至少一个根在这个集合中。黑板粗体字母$ \mathbb{C} $通常表示这个集合。请注意,由于$ r + si $可以与平面中的点$(r, s)$对应,$ \mathbb{C} $基本上与笛卡尔积$ \mathbb{R} \times \mathbb{R} $是"相同"的("相同"意味着在一个集合中的任何点都确定另一个集合中的唯一点,并且在计算结果中,使用哪一个集合进行计算并不重要,只要乘法规则适用于$ \mathbb{C} $)。
\end{enumerate}
\subsection{早期集合论中的悖论}  
集合的无限制构造原则,被称为无限制理解公理模式(axiom schema of unrestricted comprehension),

如果$P$是一个属性,则存在一个集合$Y = \{x : P(x)\}$,\(^\text{[14]}\) 

是几个早期悖论的来源:
\begin{itemize}
\item $Y = \{x \mid x \text{ 是一个序数}\}$导致了1897年的布拉利-福尔蒂悖论,这是第一个发布的反论。
\item $Y = \{x \mid x \text{ 是一个基数}\}$在1897年产生了康托尔悖论。\(^\text{[8]}\)
\item $Y = \{x \mid \{\} = \{\}\}$在1899年导致了康托尔的第二个悖论。[10] 这里,属性$P$对所有$x$都为真,无论$x$是什么,因此$Y$将是一个全集,包含所有内容。
\item $Y = \{x \mid x \notin x\}$,即所有不包含自身作为元素的集合,导致了1902年的罗素悖论。
\end{itemize}

如果无限制理解公理模式被削弱为选择公理模式或分离公理模式,

如果$P$是一个属性,则对于任何集合$X$,存在一个集合$Y = \{x \in X : P(x)\}$,\(^\text{[14]}\) 那么所有上述悖论将消失。\(^\text{[14]}\)有一个推论:在选择公理模式作为理论的公理时,作为该理论的定理推导出:

所有集合的集合是不存在的。

或者,更为壮观地说(哈尔莫斯的表述\(^\text{[15]}\)
):没有宇宙。证明:假设它存在,并称之为$U$。现在应用分离公理模式,令$X = U$,并对$P(x)$使用$x \notin x$。这再次导致罗素悖论。因此,$U$在这个理论中不能存在。\(^\text{[14]}\)

与上述构造相关的是集合的形成:
\begin{itemize}
\item $Y = \{x \mid (x \in x) \rightarrow \{\} \neq \{\}\},$ 
\end{itemize} 
其中蕴含的陈述显然是假的。从$Y$的定义出发,使用常规推理规则(以及阅读下文链接文章中的证明时的一些思考),可以得出$Y \in Y \rightarrow \{\} \neq \{\}$和$Y \in Y$成立,因此$\{\} \neq \{\}$。这是库里的悖论。

令人(也许出乎意料地)惊讶的是,问题并不是$x \in x$的可能性。问题再次出在无限制理解公理模式,允许$(x \in x) \rightarrow \{\} \neq \{\}$作为$P(x)$。如果使用选择公理模式而非无限制理解公理模式,结论$Y \in Y$将不成立,因此$\{\} \neq \{\}$就不再是逻辑结果。

然而,通常$x \in x$的可能性会被显式地去除\(^\text{[16]}\),或者,例如在ZFC中,隐式地去除\(^\text{[17]}\),通过要求正则性公理成立。\(^\text{[17]}\)它的一个后果是:

没有集合$X$使得$X \in X$,

换句话说,没有集合是它自己的元素。\(^\text{[18]}\)

分离公理模式过于弱小(而无限制理解公理是一个非常强的公理——对集合论而言太强了),无法发展出上面概述的集合论及其常规运算和构造。\(^\text{[14]}\)正则性公理本身也具有限制性。因此,人们需要提出其他公理来保证存在足够的集合以构建集合论。这些公理中的一些已经在上文中非正式地描述过,另外许多公理也是可能的。并非所有可以想象的公理都能自由地组合成一致的理论。例如,ZFC的选择公理与“每个实数集合都是勒贝格可测的”这一可想象的命题不兼容。前者暗示后者是错误的。
\subsection{另见}
\begin{itemize}
\item 集合的代数  
\item 公理化集合论  
\item 内部集合论  
\item 集合恒等式和关系列表  
\item 集合论  
\item 集合(数学)  
\item 偏序集合
\end{itemize}
\subsection{注释}  
\begin{enumerate}
\item “数学中一些词汇的最早已知用法(S)”。2020年4月14日。  
\item 哈尔莫斯1960年,《朴素集合论》。  
\item 杰夫·米勒写道,朴素集合论(与公理化集合论相对)在1940年代偶尔被使用,并在1950年代成为一个固定的术语。它出现在赫尔曼·韦尔对P. A. Schilpp编辑的《伯特兰·罗素的哲学》(1946年)的评论中。美国数学月刊,第53卷,第4期,第210页,和拉兹洛·卡尔玛(1946年)的评论中,《克利尼与罗斯的悖论》,符号逻辑杂志,第11卷,第4期,第136页。[1] 该术语后来由保罗·哈尔莫斯在一本书中推广。[2]  
\item 麦克莱恩,桑德斯(1971年),“范畴代数与集合论基础”,《公理化集合论》(《纯数学研讨会论文集》,第十三卷,第一部分,加利福尼亚大学洛杉矶分校,1967年),普罗维登斯,RI:美国数学学会,第231-240页,MR 0282791。“实际工作的数学家通常是从朴素集合论的角度思考问题(大概与ZF差不多等价)……任何新的基础系统的实际要求之一是,这个系统应该能够被那些不精通基础研究的数学家‘朴素地’使用”(第236页)。  
\item 康托尔 1874年。  
\item 弗雷格 1893年,第二卷,耶拿,1903年,pp. 253-261,弗雷格在后记中讨论了悖论。  
\item 皮亚诺 1889年,公理52,第四章产生悖论。  
\item 康托尔给大卫·希尔伯特的信,1897年9月26日,Meschkowski & Nilson 1991,第388页。  
\item 康托尔给理查德·德德金德的信,1899年8月3日,Meschkowski & Nilson 1991,第408页。  
\item 康托尔给理查德·德德金德的信,1899年8月3日和1899年8月30日,泽梅洛1932年第448页(所有可思考的类的系统)和Meschkowski & Nilson 1991,第407页。(没有所有集合的集合。)  
\item F. R. Drake,《集合论:大型基数入门》(1974年)。ISBN 0 444 10535 2。  
\item 哈尔莫斯 1974年,第9页。  
\item 哈尔莫斯 1974年,第10页。  
\item 杰赫 2002年,第4页。  
\item 哈尔莫斯 1974年,第二章。  
\item 哈尔莫斯 1974年,参见关于罗素悖论的讨论。  
\item 杰赫 2002年,第1.6节。  
\item 杰赫 2002年,第61页。
\end{enumerate}