% 弧度制与有向角(高中)
% keys 高中|角|弧度|
% license Xiao
% type Tutor

\begin{issues}
\issueMissDepend
\end{issues}

\subsection{为什么采用弧度制?}

传统教科书在介绍弧度时,通常直接抛出弧度与角度之间的转化公式,而没有详细解释引入弧度制的必要性,这让许多人在初次接触弧度制时感到困惑:“为什么需要弧度制?”。另外,日常生活中以“度”作为单位描述角度早已深入人心,且这些角度通常是简单的整数,而弧度值却常以分数乘以$\pi$的形式出现,其表达的复杂性在视觉上与角度形成了鲜明对比,这进一步加剧了初学者对弧度制的抗拒心理。

在日常生活中,角的使用主要集中在两种情境:一是描述两条直线之间的倾斜关系,二是通过圆心角来分割圆周。在表达倾斜的角度时,“度分秒”系统十分方便,它精细且规整,足以满足日常使用的需求。对于圆的分割,由于这一操作在人们的生活中极为常见,通常并不需要精确量取角度值,因而人们往往忽略了这种分割实际上依赖于角度的概念。

早在约公元前2000年,古巴比伦人选择用$360^\circ$来表示圆周的完整度量,这一选择可能与360的独特性质有关——360是最小的、能够被除7以外所有一位数整除的整数。这一性质使得$360^\circ$在处理常见的“将圆分成几份”的问题时极为便利,既能满足分割的精细需求,又避免了度量数值过大而难以操作的问题。但这种基于便利性的选择,并不具有数学上的普遍性。例如,如果某人主张“为了让圆分成7份也能得到整数”,采用$2520^\circ$作为圆周的度量,在理论上也是完全可行的。这种随意性与英制单位、市制单位,甚至曾经的某些公制单位类似,它们都是根据人们的习惯和实际需求设定的。

然而,在数学领域,这种依赖习惯的单位定义可能会带来问题。数学需要一种普适且逻辑严密的方式来定义角的度量,以避免随意性,同时在各种情境下保持简洁。此外,数学计算通常涉及无单位的纯数字,而非带有单位的量,由此问题在公元6世纪时被印度数学家\textbf{阿耶波多(Aryabhata)}发现。他在研究正弦函数时遇到了一个不知如何处理的情况\footnote{\href{https://www.hanspub.org/journal/PaperInformation?paperID=71063&utm_source=chatgpt.com}{马瑞芳, 库在强. 数学史融入弧度制的教学设计研究[J]. 教育进展, 2023, 13(8): 5911-5919.}}:
\begin{equation}
60^\circ+\sin30^\circ~.
\end{equation}
其中,$\sin 30^\circ$是一个无单位的实数,而$60^\circ$则是一个带有角度单位的量。这种单位的不一致使得它们无法直接进行代数运算(例如加法或乘法),类似于“$1\text{元} + 2$”,使人无法理解结果的意义。此外,随着数学研究的深入,人们还需要计算角$\alpha$的平方、立方等幂次以及它们的和。如果继续使用“度”为单位,这将引入诸如$\circ^2$和$\circ^3$等奇怪的单位。这些实际需求促使数学家思考如何定义一种新的角度度量方式,以满足以下要求:
\begin{enumerate}
\item 能够将角度转化为无单位的量,或者单位是“1”的量,从而便于代数运算。
\item 与传统的“度分秒”系统具有明确的换算关系。
\item 定义应独立于历史、习惯,具有普适性和数学上的严密性,并能在复杂分析、物理学等领域中表现出自然的适用性。
\end{enumerate}

在这个问题被发现之前,人们已经有了分割圆的经验。由于生活中容易量取长度,人们常将角度的分割转化为长度的分割,从而降低测量的难度。尽管,现在人们知道圆周的长度通常是一个无理数,这种方法事实上也是一种近似。然而,这一思路为角度测量问题的解决提供了重要启发。

阿耶波多通过选择用同一单位度量圆的半径和圆周,实际上提出了一个与现代弧度制几乎一致的概念。1748年,瑞士数学家\textbf{欧拉(Euler)}在其著作《无穷小分析概论》中,正式定义了以圆的半径作为弧长的度量单位。下面看看这样的定义可以推知些什么。已知圆周长度 $L$ 与半径 $r$ 的比值是一个固定值,由于以圆的半径作为弧长的度量单位,这就得到了算数式:
\begin{equation}
L=2\pi r~.
\end{equation}
而一段圆弧$l$如果是周长的$\displaystyle\frac{n}{m}$,那么,它自然就可以表示为:
\begin{equation}
l=\frac{n}{m}L=\frac{2n}{m}\pi r~.
\end{equation}
因此,弧长 $l$ 与半径 $r$ 之间建立了一一对应的关系。在\aref{圆的相关概念}{sub_HsGeBa_1}中提到,每段弧长都对应着一个圆心角。

由于所有的圆都是相似的,对应同一圆心角的不同弧长之间的差异仅来源于圆的半径 $r$ 的不同。如果希望使新的圆心角度量方法不受半径 $r$ 的影响,需要消除半径 $r$ 的作用。将 $r$ 移到等式的左侧,可以得到:
\begin{equation}\label{eq_HsAngl_1}
\frac{l}{r}=\frac{2n\pi}{m}~.
\end{equation}
这样,对于占圆周相同比例的圆弧,不论圆的半径 $r$ 多大,右侧的值始终保持不变。

如果将右侧的值作为弧对应的圆心角的度量方式,那么通过定义“以圆的半径作为弧长的度量单位”,人们得到了一个新的角度度量方法,这种方法满足之前提出的所有要求:
\begin{enumerate}
\item 它是一个无单位的纯数值。
\item 由于“度”在定义时采用了类似的按比例等分的原则,若规定整圆的圆心角为一个固定值,这种新的方式可以实现与传统的度分秒系统的快速转换。
\item 它仅依赖于常数 $\pi$。
\end{enumerate}
这种新的角度度量方法也就是\textbf{弧度制(radian measure)}。

\subsection{弧度制}

根据\autoref{eq_HsAngl_1} 对于任意圆,其弧长 $l$ 与圆的半径 $r$ 存在固定比例关系,即:

\begin{equation}
l = \theta r~.
\end{equation}
称$\theta$为弧所对圆心角的\textbf{弧度数}。尽管之前说过$\theta$是个数,但通常根据习惯,为了表示他是与角相关的量,会在后面加上单位$\Si{rad}$,读作\textbf{弧度}。注意,这只是为了在物理计算中与其他单位复合,表示它是角,本质上还是无量纲的。

根据圆的周长公式,圆周对应的圆心角为:
\begin{equation}
\frac{L}{r}=2\pi~.
\end{equation}
也就是说,弧度中与$360^\circ$对应的是$2\pi\Si{rad}$。在单位圆中长度为 $1$ 的弧,所对的圆心角称为 \textbf{$1$ 弧度(radian)}角。
例如,一个完整的圆周对应的弧长为 $2\pi r$,因此它的角度用弧度表示为 $2\pi$,而无需再考虑圆周的分割比例。



由于弧度制消除了不同半径带来的影响,因此一般只研究\textbf{单位圆},即半径为1的圆。可以认为,弧度就是圆心角在单位圆上所对的弧长。


。在弧度制中,圆周对应的圆心角为 $2\pi$ 弧度,每 $1$ 弧度对应的弧长等于圆的半径。
它的单位符号是 $\Si{rad}$,读作\textbf{弧度}。

\addTODO{补图}

这种以弧度作为单位来度量角的单位制,叫做\textbf{弧度制}。这种度量方法有效地把角的弧度单位与长度单位统一起来。弧度制确立了角的弧度数与十进制实数间的一一对应关系。

我们知道,圆周率的定义是
\begin{equation}\label{eq_HsAngl_2}
\pi = \frac{C}{d}~.
\end{equation}
在单位圆中,
\begin{equation}\label{eq_HsAngl_3}
\begin{aligned}
r &= 1~,\\
d &= 2~.
\end{aligned}
\end{equation}
由\autoref{eq_HsAngl_2} 和\autoref{eq_HsAngl_3} 可得
\begin{equation}
C = 2\pi~.
\end{equation}
根据弧度数的定义,我们可以得到周角 $\alpha$ 的弧度数
\begin{equation}
\alpha = C / r = 2\pi~.
\end{equation}
同理可以推出$\pi$弧度所对应的角度
\begin{equation}
\pi = 180^\circ~.
\end{equation}
% Giacomo:这个等式并不合理,可能会误导读者。

弧度制的优点在于,我们可以快速得到角、半径和弧长的关系
\begin{equation}
\begin{aligned}
l &= \alpha \cdot r~, \\
\alpha &= \frac{l}{r}~, \\
r &= \frac{l}{\alpha}~.
\end{aligned}
\end{equation}



\subsection{有向角 - 角的概念的推广}

% Giacomo:感觉说的不是很清晰,需要进一步解释。

按逆时针方向旋转形成的角叫做\textbf{正角};按顺时针方向旋转形成的角叫做\textbf{负角};如果一条射线从起始位置没有作任何旋转,或终止位置与起始位置重合,我们称这样的角为\textbf{零度角},又称\textbf{零角},记作 $\alpha = 0$

角的终边(除端点外)在第几象限,我们就说这个角是第几象限角。

一般地,所有与角 $\alpha$ 终边相同的角,连同角 $\alpha$ 在内,可构成一个集合
\begin{equation}
S = \begin{Bmatrix} \beta|\beta=\alpha+2k\pi,k \in Z \end{Bmatrix}~.
\end{equation}

% \textsl{注:$2k\pi$ 是弧度制写法,在下面介绍}

一般地,正角的弧度数是一个正数,负角的弧度数是一个负数,零角的弧度数是 $0$。
