% 弦理论(综述)
% license CCBYSA3
% type Wiki

本文根据 CC-BY-SA 协议转载翻译自维基百科\href{https://en.wikipedia.org/wiki/String_theory}{相关文章}。

在物理学中,弦理论是一种理论框架,其中粒子物理学中的点状粒子被称为弦的一维物体所取代。弦理论描述了这些弦如何在空间中传播并相互作用。在大于弦尺度的距离范围内,弦表现得像一个粒子,其质量、电荷和其他属性由弦的振动状态决定。在弦理论中,弦的众多振动状态之一对应于引力子,一种携带引力的量子力学粒子。因此,弦理论是一种量子引力理论。

弦理论是一个广泛而多样的学科,试图解决许多深刻的基础物理学问题。弦理论为数学物理学做出了诸多贡献,这些贡献已应用于黑洞物理学、早期宇宙宇宙学、核物理学和凝聚态物理学的各种问题,并激发了纯数学领域的一些重大进展。由于弦理论可能提供一个统一的引力和粒子物理学的描述,它成为了万物理论的候选者,即一个自洽的数学模型,能够描述所有基本力和物质形式。尽管在这些问题上进行了大量研究,但目前尚不清楚弦理论在多大程度上描述了真实世界,或者该理论在选择细节时允许多少自由度。

弦理论最早在1960年代末期作为强核力的理论进行研究,但随后由于量子色动力学的兴起而被放弃。随后,人们意识到,正是使弦理论不适合作为核物理学理论的那些特性,使其成为量子引力理论的一个有前景的候选者。弦理论的最早版本是玻色子弦理论,它只包含了被称为玻色子的粒子类别。后来,它发展成了超弦理论,超弦理论假设玻色子和称为费米子的粒子类别之间存在一种叫做超对称的联系。在1990年代中期,人们推测出,超弦理论的五个一致版本其实是一个十一维的单一理论的不同极限情形,这个理论被称为M理论。1997年底,理论物理学家发现了一个重要的关系,叫做反德西特/共形场论对偶性(AdS/CFT对偶性),它将弦理论与另一种物理理论——量子场论——联系了起来。

弦理论的一个挑战是,完整的理论在所有情况下都没有一个令人满意的定义。另一个问题是,该理论被认为描述了一个庞大的可能宇宙的景观,这使得基于弦理论的粒子物理学理论发展变得复杂。这些问题导致物理学界一些人批评这些物理学方法,并质疑继续进行弦理论统一研究的价值。
\subsection{基本原理}
\subsubsection{概述}
\begin{figure}[ht]
\centering
\includegraphics[width=6cm]{./figures/b1d0a7d0dd142f3e.png}
\caption{弦理论的基本物体是开弦和闭弦。} \label{fig_String_1}
\end{figure}  
在20世纪,出现了两种理论框架来阐述物理学的基本法则。第一种是阿尔伯特·爱因斯坦的广义相对论,这一理论解释了引力的作用以及宏观层面上时空的结构。另一种是量子力学,这是一种完全不同的框架,利用已知的概率原理来描述微观层面的物理现象。到1970年代末,这两种框架已被证明足以解释大多数已观察到的宇宙特征,从基本粒子到原子,再到恒星和整个宇宙的演化。[1]

尽管取得了这些成功,但仍有许多问题亟待解决。现代物理学中最深刻的问题之一是量子引力问题。[1]广义相对论是在经典物理框架内 formul化的,而其他基本力则在量子力学框架内描述。为了将广义相对论与量子力学的原则统一起来,需要一个量子引力理论,但当尝试将量子理论的常规法则应用于引力时,便会遇到困难。[2]

弦理论是一种理论框架,试图解决这些问题。

弦理论的起点是这样一个观点:粒子物理学中的点状粒子也可以被建模为一种称为“弦”的一维物体。弦理论描述了弦如何在空间中传播并相互作用。在某个特定版本的弦理论中,只有一种类型的弦,它可能看起来像一个小的环或普通弦的一段,并且可以以不同的方式振动。在大于弦尺度的距离范围内,弦看起来就像是一个普通的粒子,符合非弦模型中的基本粒子,其质量、电荷以及其他特性由弦的振动状态决定。作为量子引力的应用,弦理论提出了一种振动状态,负责产生引力子——一种尚未被证实的量子粒子,理论上它承担引力的作用。[3]

过去几十年中,弦理论的主要发展之一是发现了某些“对偶性”,即将一种物理理论与另一种物理理论联系起来的数学变换。研究弦理论的物理学家发现了不同版本弦理论之间的一些对偶性,这导致了一个猜想:所有一致的弦理论版本都可以被统一在一个单一的框架中,称为\(M\)理论。[4]

弦理论的研究还在黑洞的性质和引力相互作用方面取得了一些成果。当人们试图理解黑洞的量子特性时,出现了一些悖论,弦理论的研究试图澄清这些问题。1997年底,这一领域的研究达到了顶峰,发现了反-德西特/共形场理论对应(AdS/CFT)。[5]这是一个理论结果,将弦理论与其他理论上更为清楚的物理理论联系起来。AdS/CFT对应对于黑洞和量子引力的研究具有重要意义,并且已被应用于其他领域,[6]包括核物理和凝聚态物理。[7][8]

由于弦理论包含了所有基本相互作用,包括引力,许多物理学家希望它最终能够发展到足以完全描述我们的宇宙,从而成为一种“万物理论”。目前弦理论研究的目标之一是找到一个能够再现已观察到的基本粒子谱、具有小的宇宙常数、包含暗物质并提供合理的宇宙膨胀机制的理论解。尽管在这些目标上已有一些进展,但目前尚不清楚弦理论在多大程度上能够描述现实世界,或者该理论在细节选择上允许多少自由度。[9]

弦理论的挑战之一是,完整的理论在所有情况下都没有一个令人满意的定义。弦的散射最直接的定义方法是使用微扰理论的技巧,但通常并不清楚如何非微扰地定义弦理论。[10]此外,是否存在某种原理来选择弦理论的真空态——即决定我们宇宙特性的物理状态——也尚不明确。[11]这些问题使得部分学者批评将物理学统一的这些方法,并质疑继续研究这些问题的价值。[12]
\subsubsection{弦}
\begin{figure}[ht]
\centering
\includegraphics[width=6cm]{./figures/fd6b90c488650f4c.png}
\caption{量子世界中的相互作用:点状粒子的世界线或弦论中闭合弦所扫过的世界面} \label{fig_String_2}
\end{figure}
将量子力学应用于像电磁场这样的在时空中扩展的物理对象,称为量子场论。在粒子物理学中,量子场论是我们理解基本粒子的基础,这些粒子被建模为基本场中的激发。[13]

在量子场论中,通常使用微扰理论的技术来计算各种物理事件的概率。微扰量子场论由理查德·费曼等人在二十世纪上半叶发展起来,使用一种叫做费曼图的特殊图形来组织计算。人们可以想象,这些图形描绘了点状粒子的路径及其相互作用。[13]

弦论的出发点是这样的观点:量子场论中的点状粒子也可以被建模为一维物体,称为弦。[14] 弦的相互作用通过推广在普通量子场论中使用的微扰理论来最直接地定义。在费曼图的层面上,这意味着用一个二维(2D)表面来替代表示点粒子路径的一维图形,从而表示弦的运动。[15] 与量子场论不同,弦论没有完整的非微扰定义,因此许多物理学家希望解答的理论问题仍然无法触及。[16]

在基于弦论的粒子物理学理论中,弦的特征长度尺度被假定为普朗克长度量级,即\(10^{-35}\)米,这是量子引力效应被认为变得显著的尺度。[15] 在更大尺度下,比如物理实验室中可见的尺度,这样的物体将与零维的点粒子无法区分,而弦的振动状态将决定粒子的类型。弦的一个振动状态对应于引力子,一种量子力学粒子,携带引力作用。[3]

原始版本的弦理论是玻色弦理论,但这个版本仅描述了玻色子——一种传递物质粒子之间相互作用力的粒子类别。玻色弦理论最终被称为超弦理论的理论所取代。这些理论既描述玻色子也描述费米子,并且它们包含了一个叫做超对称的理论概念。在具有超对称性的理论中,每个玻色子都有一个对应的费米子,反之亦然。[17]

超弦理论有多个版本:类型I、类型IIA、类型IIB和两种异质弦理论(\(SO(32)\)和\(E_8\times E_8\))。不同的理论允许不同类型的弦,并且在低能量下出现的粒子表现出不同的对称性。例如,类型I理论包含开放弦(具有端点的弦段)和闭合弦(形成闭环的弦),而类型IIA、IIB和异质弦理论仅包含闭合弦。[18]
\subsubsection{额外维度}
\begin{figure}[ht]
\centering
\includegraphics[width=6cm]{./figures/8815ea074c9ece9c.png}
\caption{紧致化的一个例子:在大尺度下,一个具有一个圆形维度的二维表面看起来像是一维的。} \label{fig_String_3}
\end{figure}
在日常生活中,我们熟悉的空间有三个维度(3D):高度、宽度和长度。爱因斯坦的广义相对论将时间视为与三个空间维度平等的维度;在广义相对论中,空间和时间并不是作为独立的实体来描述,而是统一为四维(4D)时空。在这个框架中,重力现象被视为时空几何的结果。[19]

尽管宇宙可以用四维时空很好地描述,但物理学家考虑其他维度的理论有几个原因。在某些情况下,通过以不同维度来建模时空,理论变得在数学上更易处理,从而可以更容易地进行计算并获得一般性的见解。[a] 还有一些情况,在二维或三维时空中的理论对于描述凝聚态物理中的现象非常有用。[13] 最后,存在一些情形,可能时空的维度实际上超过了四维,但这些额外的维度仍然未能被探测到。[20]

弦理论要求额外的时空维度以保持其数学一致性。在玻色弦理论中,时空是26维的,而在超弦理论中是10维的,在M理论中是11维的。因此,为了使用弦理论描述真实的物理现象,必须设想一些场景,其中这些额外的维度在实验中不会被观察到。[21]
\begin{figure}[ht]
\centering
\includegraphics[width=6cm]{./figures/cf09ca86ce3b51c2.png}
\caption{一个五次卡拉比–尤流形的截面} \label{fig_String_4}
\end{figure}
紧致化是修改物理理论中维度数量的一种方式。在紧致化过程中,假设一些额外的维度“自我闭合”,形成圆形。[22] 在这些卷曲的维度变得非常小的极限下,便得到一种有效维度较低的理论。对此的标准类比是考虑一个多维物体,比如花园水管。如果从足够远的地方看,水管似乎只有一个维度,即它的长度。然而,当靠近水管时,会发现它包含第二个维度,即其周长。因此,一只在水管表面爬行的蚂蚁将会在两个维度上移动。

紧致化可以用来构造有效四维时空的模型。然而,并非所有紧致化额外维度的方式都会产生具有正确性质的模型来描述自然界。在一个可行的粒子物理模型中,紧致的额外维度必须呈现卡拉比–尤流形的形状。[22] 卡拉比–尤流形是一种特殊的空间,在弦理论应用中通常被认为是六维的。它以数学家尤金尼奥·卡拉比和丘成桐的名字命名。[23]

减少维度的另一种方法是所谓的‘膜世界’场景。在这种方法中,物理学家假设可观察的宇宙是一个更高维空间的四维子空间。在这种模型中,粒子物理学中的传递力的玻色子来自于端点附着在四维子空间上的开放弦,而引力则来自于通过更大环境空间传播的闭合弦。这个思想在尝试基于弦理论开发现实世界物理学模型的过程中发挥着重要作用,并且为引力相比其他基本力的弱性提供了一个自然的解释。[24]
\subsubsection{对偶性}
\begin{figure}[ht]
\centering
\includegraphics[width=14.25cm]{./figures/58ea06ab2273f446.png}
\caption{弦理论对偶性的图示。蓝色边表示S-对偶性,红色边表示T-对偶性。} \label{fig_String_5}
\end{figure}
弦理论的一个显著特点是,不同版本的理论最终都以非常非平凡的方式相互关联。不同弦理论之间可能存在的一种关系被称为S-对偶性。这种关系表明,在某些情况下,一种理论中强相互作用的粒子集合可以被视为另一种完全不同理论中弱相互作用的粒子集合。粗略来说,如果一组粒子经常结合并衰变,则称其为强相互作用;如果这种过程发生得不频繁,则称其为弱相互作用。I型弦理论通过S-对偶性与\(SO(32)\)的异质弦理论是等价的。同样,IIB型弦理论也通过S-对偶性以非平凡的方式与自身相关联。[25]

不同弦理论之间的另一种关系是T-对偶性。在这里,考虑的是弦在一个圆形额外维度上传播。T-对偶性表明,一条在半径为\(R\)的圆圈上传播的弦,与一条在半径为\(1/R\)的圆圈上传播的弦是等价的,意思是一个描述中的所有可观察量与对偶描述中的量是对应的。例如,一条弦在绕圆圈传播时具有动量,它也可以绕圆圈一圈或多圈。弦绕圆圈缠绕的圈数称为缠绕数。如果在一种描述中,弦的动量为\(p\),缠绕数为\(n\),那么在对偶描述中,它的动量将是\(n\),缠绕数将是\(p\)。例如,IIA型弦理论通过T-对偶性与IIB型弦理论等价,异质弦理论的两个版本也通过T-对偶性相关联。[25]

一般来说,对偶性一词指的是两种看似不同的物理系统以非平凡的方式实际上是等价的情况。通过对偶性相关联的两种理论不一定是弦理论。例如,Montonen–Olive对偶性是量子场论之间S-对偶性关系的一个例子。AdS/CFT对应关系是一个将弦理论与量子场论联系起来的对偶性例子。如果两种理论通过对偶性相关联,意味着一种理论可以以某种方式被转换,使得它最终看起来与另一种理论完全相同。然后,这两种理论就被称为在变换下彼此对偶。换句话说,这两种理论是同一现象的数学上不同的描述。[26]
\subsubsection{膜}
\begin{figure}[ht]
\centering
\includegraphics[width=6cm]{./figures/94cdfc802840194e.png}
\caption{附着在一对D-膜上的开放弦} \label{fig_String_6}
\end{figure}
在弦理论及其他相关理论中,膜是一个物理对象,它将点粒子的概念推广到更高的维度。例如,点粒子可以被看作是一个零维的膜,而弦可以被看作是一个一维的膜。也可以考虑更高维度的膜。在\(p\)维度下,这些被称为p-膜。‘膜’一词来自于‘膜’(membrane)一词,后者指的是二维膜。[27]

膜是动态对象,可以根据量子力学的规则在时空中传播。它们具有质量,并可以具有其他属性,如电荷。一个p-膜在时空中扫过一个\((p+1)\)维的体积,称为它的世界体积。物理学家通常研究类似于电磁场的场,这些场存在于膜的世界体积上。[27]

在弦理论中,D-膜是一个重要的膜类别,当考虑开放弦时会出现。随着开放弦在时空中传播,它的端点必须位于D-膜上。D-膜中字母“\(D\)”指的是系统中一种称为狄利克雷边界条件的特定数学条件。弦理论中对D-膜的研究产生了重要的成果,例如AdS/CFT对应关系,这为量子场论中的许多问题提供了新的见解。[27]

膜经常从纯数学的角度进行研究,它们被描述为某些范畴中的对象,例如复代数簇上的相干层的导出范畴,或辛流形的福卡亚范畴。[28] 膜的物理概念与数学范畴概念之间的联系,导致了代数几何与辛几何[29]以及表示论[30]领域中的重要数学见解。
\subsection{M理论}  
在1995年之前,理论物理学家认为超弦理论有五种一致的版本(I型、IIA型、IIB型以及两种异质弦理论)。这一理解在1995年发生了变化,当时爱德华·威滕提出这五种理论仅仅是一个十一维理论——M理论的特殊极限情形。威滕的猜想基于其他物理学家的一些工作,包括阿肖克·森、克里斯·赫尔、保罗·汤森德和迈克尔·达夫等人。他的这一宣布引发了一场研究热潮,现被称为第二次超弦革命。[31]
\subsubsection{超弦理论的统一}
\begin{figure}[ht]
\centering
\includegraphics[width=10cm]{./figures/d43777704e0c6f56.png}
\caption{这是一张示意图,展示了M理论、五种超弦理论和十一维超引力之间的关系。阴影区域表示在M理论中可能的不同物理情景家族。在某些对应于拐点的极限情况下,使用其中一种标记的六种理论来描述物理现象是自然的。} \label{fig_String_7}
\end{figure}
在1970年代,许多物理学家开始对超引力理论产生兴趣,该理论将广义相对论与超对称结合起来。广义相对论在任何维度下都是合理的,而超引力则对维度的数量设定了上限。[32] 1978年,Werner Nahm的研究表明,能够制定一致的超对称理论的最大时空维度是十一。[33] 同年,来自巴黎高等师范学校的Eugene Cremmer、Bernard Julia和Joël Scherk证明,超引力不仅允许最多十一维,而且在这个最大维度下表现得最为优雅。[34][35]

最初,许多物理学家希望通过将十一维超引力紧凑化,可能构建出符合现实的四维世界模型。人们希望这些模型能够提供四种基本自然力的统一描述:电磁力、强核力、弱核力和引力。然而,随着该方案中各种缺陷的被发现,对十一维超引力的兴趣很快减少了。其中一个问题是物理定律似乎区分顺时针和逆时针旋转,这一现象被称为“手性”。Edward Witten和其他人观察到,这种手性属性无法通过从十一维紧凑化轻易推导出来。[35]

在1984年的第一次超弦革命中,许多物理学家转向弦理论,将其视为粒子物理和量子引力的统一理论。与超引力理论不同,弦理论能够容纳标准模型中的手性,并且提供了一种与量子效应一致的引力理论。[35] 弦理论的另一个特点是其高度的独特性,这也是许多物理学家在1980年代和1990年代被其吸引的原因。在普通的粒子理论中,可以考虑任何由任意拉格朗日量描述其经典行为的基本粒子集合。而在弦理论中,可能性要受到更多限制:到1990年代,物理学家已认为只有五种一致的超对称版本的弦理论。[35]

尽管只有少数几种一致的超弦理论,但为什么没有一个统一的一致性公式依然是一个谜。[35] 然而,随着物理学家开始更仔细地研究弦理论,他们意识到这些理论之间以错综复杂且非平凡的方式相互关联。他们发现,在某些情况下,一个强相互作用的弦系统可以视为一个弱相互作用的弦系统。这种现象被称为S对偶性。Ashoke Sen在四维异托弦的背景下研究了这一现象[36][37],Chris Hull和Paul Townsend则在IIB型理论的背景下进行了研究。[38] 理论物理学家还发现,不同的弦理论可能通过T对偶性相互关联。该对偶性意味着,在完全不同的时空几何上传播的弦可能是物理等价的。[39]

在大约同一时期,当许多物理学家研究弦的性质时,一小群物理学家开始考察更高维物体的可能应用。1987年,Eric Bergshoeff、Ergin Sezgin和Paul Townsend证明了十一维超引力包含了二维膜。[40] 直观上,这些物体看起来像是穿越十一维时空传播的薄片或膜。就在这一发现之后,Michael Duff、Paul Howe、Takeo Inami和Kellogg Stelle考虑了十一维超引力的特定紧凑化方式,其中一个维度卷曲成一个圆。[41] 在这种情境下,可以想象膜绕着圆形维度展开。如果圆的半径足够小,那么这个膜看起来就像是十维时空中的弦。Duff及其合作者证明,这种构造精确地重现了在IIA型超弦理论中出现的弦。[42]

1995年,Edward Witten在一次弦理论会议上提出了一个令人惊讶的观点,认为所有五种超弦理论实际上只是十一维时空中的一个单一理论的不同极限情形。Witten的这一声明将之前关于S对偶性和T对偶性以及弦理论中高维膜的出现的所有研究成果汇集在一起。[43] 在Witten宣布之后的几个月里,成百上千篇新的论文出现在互联网上,确认了他提议的不同部分。[44] 今天,这一系列的工作被称为第二次超弦革命。[45]

最初,一些物理学家建议,这一新理论是一个膜的基本理论,但Witten对膜在该理论中的角色持怀疑态度。在1996年的一篇论文中,Hořava和Witten写道:“虽然有人提出十一维理论是一个超膜理论,但也有一些理由对这种解释表示怀疑,因此我们将其非承诺性地称为M理论,将M与膜的关系留待未来来解答。”[46] 在尚未理解M理论的真正意义和结构的情况下,Witten建议,M可以根据个人喜好代表“魔法”、“神秘”或“膜”,而这一名称的真正意义应当在更基础的理论公式化后再做决定。[47]
\subsubsection{矩阵理论} 
在数学中,矩阵是一个由数字或其他数据构成的矩形数组。在物理学中,矩阵模型是一种特殊类型的物理理论,其数学公式中以矩阵的概念为重要组成部分。矩阵模型描述了一组矩阵在量子力学框架中的行为。[48]

矩阵模型的一个重要例子是1997年Tom Banks、Willy Fischler、Stephen Shenker和Leonard Susskind提出的BFSS矩阵模型。该理论描述了一组九个大矩阵的行为。在他们的原始论文中,这些作者展示了该矩阵模型的低能极限由十一维超引力描述。这些计算使他们提出,BFSS矩阵模型与M理论完全等价。因此,BFSS矩阵模型可以作为M理论正确公式化的原型,并作为在相对简单的环境中研究M理论性质的工具。[48]

矩阵模型公式化M理论的发展使物理学家开始考虑弦理论与一种叫做\textbf{非交换几何}的数学分支之间的各种联系。这个学科是普通几何的推广,数学家通过使用非交换代数的工具来定义新的几何概念。[49] 在1998年的一篇论文中,Alain Connes、Michael R. Douglas和Albert Schwarz证明了矩阵模型和M理论的某些方面可以通过非交换量子场论来描述,非交换量子场论是一种特殊类型的物理理论,其中时空通过使用非交换几何在数学上进行描述。[50] 这建立了矩阵模型和M理论与非交换几何之间的联系。此后,这一发现迅速引发了非交换几何与各种物理理论之间其他重要联系的发现。[51][52]
\subsection{黑洞}  
在广义相对论中,黑洞被定义为时空中的一个区域,其中引力场强大到任何粒子或辐射都无法逃脱。在当前公认的恒星演化模型中,黑洞被认为是在大质量恒星经历引力坍缩时形成的,并且许多银河系被认为在其中心存在超大质量黑洞。黑洞在理论上也具有重要意义,因为它们对试图理解引力的量子性质的理论学家提出了深刻的挑战。弦理论已被证明是研究黑洞理论性质的重要工具,因为它提供了一个框架,理论学家可以在其中研究黑洞的热力学性质。[53]
\subsubsection{贝肯斯坦–霍金公式}  
在物理学的一个分支——统计力学中,熵是衡量物理系统的随机性或无序度的指标。这个概念在19世纪70年代由奥地利物理学家路德维希·玻尔兹曼研究,他证明了气体的热力学性质可以从其许多组成分子的综合性质中推导出来。玻尔兹曼认为,通过对气体中所有不同分子的行为进行平均,可以理解诸如体积、温度和压力等宏观性质。此外,这一观点使他给出了熵的精确定义,即分子(也称为微观状态)数目的自然对数,这些微观状态会导致相同的宏观特性。[54]

在20世纪,物理学家开始将相同的概念应用于黑洞。在大多数系统中,如气体,熵与体积成比例。1970年代,物理学家雅各布·贝肯斯坦提出,黑洞的熵与其事件视界的表面积成正比,事件视界是物质和辐射可能逃脱其引力吸引的边界。[55] 当与物理学家斯蒂芬·霍金的理论结合时,[56] 贝肯斯坦的研究得出了黑洞熵的精确公式。贝肯斯坦–霍金公式表示熵\(S\)为:
\[
S = \frac{c^3 k A}{4 \hbar G}~
\]
其中,\(c\)是光速,\(k\)是玻尔兹曼常数,\(\hbar\)是约化普朗克常数,\(G\)是牛顿引力常数,\(A\)是事件视界的表面积。[57]

像任何物理系统一样,黑洞具有熵,熵是通过不同的微观状态的数量来定义的,这些微观状态会导致相同的宏观特征。贝肯斯坦–霍金熵公式给出了黑洞熵的期望值,但到1990年代,物理学家仍然缺乏通过在量子引力理论中计数微观状态来推导该公式的方法。找到这种推导公式的方法被认为是任何量子引力理论(如弦理论)可行性的一个重要检验。[58]
\subsubsection{弦理论中的推导}  
在1996年的一篇论文中,Andrew Strominger和Cumrun Vafa展示了如何在弦理论中推导贝肯斯坦–霍金公式,针对某些黑洞。[59] 他们的计算基于一个观察,即D-brane——当它们弱相互作用时,看起来像波动的膜——在相互作用强烈时,会变成具有事件视界的致密、巨大的物体。换句话说,在弦理论中,强相互作用的D-brane系统与黑洞是无法区分的。Strominger和Vafa分析了这种D-brane系统,并计算了将D-brane放置在时空中的不同方式,使得它们的合并质量和电荷与结果黑洞的给定质量和电荷相等。他们的计算准确地重现了贝肯斯坦–霍金公式,包括1/4的因子。[60] 随后的Strominger、Vafa及其他人的工作精细化了原始计算,并给出了描述非常小黑洞所需的“量子修正”的精确值。[61][62]

Strominger和Vafa在他们的原始工作中考虑的黑洞与真实的天体物理黑洞有很大不同。一个区别是,Strominger和Vafa仅考虑了极端黑洞,以便使计算变得可处理。这些黑洞被定义为具有与给定电荷兼容的最低可能质量的黑洞。[63] Strominger和Vafa还将注意力限制在五维时空中的黑洞,并且这些黑洞具有非物理的超对称性。[64]

尽管Strominger和Vafa的熵计算最初是在弦理论中这一非常特定且物理上不现实的背景下发展起来的,但这一计算为我们提供了一个定性理解,说明了如何在任何量子引力理论中解释黑洞熵。实际上,1998年,Strominger认为,原始结果可以推广到任意一致的量子引力理论,而不依赖于弦或超对称性。[65] 2010年,他与其他几位作者合作,展示了关于黑洞熵的一些结果可以扩展到非极端的天体物理黑洞。[66][67]
\subsection{AdS/CFT 对应}
反德西特/共形场论(AdS/CFT)对应提供了一种研究弦理论及其性质的方法。这一理论结果表明,在某些情况下,弦理论等效于一个量子场论。除了揭示弦理论的数学结构外,AdS/CFT 对应还为量子场论的许多方面提供了新的见解,尤其是在传统计算方法失效的领域。[6]  

AdS/CFT 对应最初由Juan Maldacena于1997年底提出。[68] Steven Gubser、Igor Klebanov 和 Alexander Markovich Polyakov以及Edward Witten在随后发表的文章中进一步阐述了该对应的重要方面。[69][70] 截至2010年,Maldacena 的论文已被引用超过7000次,成为高能物理领域被引用次数最多的论文。[b]
\subsubsection{对应概述}
\begin{figure}[ht]
\centering
\includegraphics[width=6cm]{./figures/417cf5d53940c5cf.png}
\caption{由三角形和正方形镶嵌的双曲平面} \label{fig_String_8}
\end{figure}
在AdS/CFT 对应中,时空的几何结构由\textbf{爱因斯坦方程}的一种特定真空解描述,这种解被称为反德西特空间(Anti-de Sitter Space, AdS)。[6] 从基本角度来看,反德西特空间是一个时空的数学模型,其中点之间的距离(度量)与普通欧几里得几何中的距离概念不同。它与\textbf{双曲空间}密切相关,后者可以被看作如左图所示的圆盘。[71] 该图显示了一个由三角形和正方形镶嵌的圆盘。可以在这个圆盘上定义一种特殊的距离,使得所有三角形和正方形大小相同,并且圆形外边界与内部的任何点都相距无穷远。[72]  

可以想象一叠\textbf{双曲圆盘},其中每个圆盘代表宇宙在某一时刻的状态。这样形成的几何对象就是\textbf{三维反德西特空间}。[71] 它看起来像一个\textbf{实心圆柱体},其中\textbf{任意横截面}都是双曲圆盘的副本。在这个图像中,时间沿着垂直方向流动。这个圆柱体的表面在AdS/CFT对应中起着重要作用。与双曲平面类似,反德西特空间的曲率特性使得内部的任何点实际上都与边界表面相距无穷远。[72]
\begin{figure}[ht]
\centering
\includegraphics[width=8cm]{./figures/97a18477b1ea79f9.png}
\caption{三维反德西特空间类似于一叠双曲圆盘,每个圆盘代表宇宙在某一时刻的状态。由此形成的时空看起来像一个实心圆柱体。} \label{fig_String_9}
\end{figure}
这种构造描述了一个仅具有两个空间维度和一个时间维度的假想宇宙,但它可以推广到任意维度。事实上,双曲空间可以具有多于两个的维度,并且可以通过“堆叠”多个双曲空间的副本来构建更高维的\textbf{反德西特空间}模型。[71]

反德西特空间的一个重要特性是其\textbf{边界}(在三维反德西特空间的情况下,它看起来像一个圆柱体)。该边界的一个性质是,在边界上围绕任意一点的小区域内,它的结构与\textbf{闵可夫斯基空间}非常相似,而闵可夫斯基空间是\textbf{非引力物理}中使用的时空模型。[73]  

因此,可以考虑一个\textbf{辅助理论},其中“时空”由反德西特空间的边界给出。这一观察结果构成了AdS/CFT 对应的起点,该理论表明,反德西特空间的边界可以被视为一个量子场论的“时空”。该对应关系的核心主张是,这个量子场论与大块反德西特空间中的某种引力理论(例如弦理论)是等价的,其等价性体现在二者之间存在一个“字典”,可以将一个理论中的实体和计算转换为另一个理论中的对应物。例如,引力理论中的\textbf{单个粒子}可能对应于边界理论中的\textbf{某些粒子集合}。此外,这两个理论的预测在数量上完全相同,例如,如果在引力理论中两个粒子有40\%的几率发生碰撞,那么在边界理论中,对应的粒子集合也应有40\%的几率发生碰撞。[74]
\subsubsection{对量子引力的应用}
AdS/CFT 对应的发现是物理学家对\textbf{弦理论和量子引力}理解上的一项重大进展。其重要性体现在以下两个方面:首先,该对应关系提供了一种用量子场论来表述弦理论的方法,而相比之下,量子场论的理论框架已经较为成熟和完善。其次,它为物理学家提供了一个通用框架,在其中可以研究并尝试解决\textbf{黑洞悖论}等问题。[53]

1975年,斯蒂芬·霍金发表了一项计算结果,表明黑洞并非完全黑暗,而是由于事件视界附近的量子效应会发出微弱的辐射,这一现象被称为\textbf{霍金辐射}。[56]  

起初,霍金的研究结果对理论学家提出了挑战,因为它表明黑洞可能会摧毁信息。更准确地说,霍金的计算似乎与量子力学的基本公设之一相冲突,该公设认为物理系统会按照薛定谔方程随时间演化。这个性质通常被称为\textbf{时间演化的幺正性}。霍金的计算结果与量子力学幺正性公设之间的这种\textbf{明显矛盾},被称为\textbf{黑洞信息悖论}。[75]

AdS/CFT 对应在一定程度上解决了黑洞信息悖论,因为它表明,在某些情境下,黑洞可以\textbf{以符合量子力学的方式演化}。事实上,可以在AdS/CFT 对应的框架下研究黑洞,任何这样的黑洞都对应于反德西特空间边界上的一组粒子配置。[76]这些边界上的粒子遵循标准的量子力学规则,特别是\textbf{以幺正方式演化},因此,黑洞也必须\textbf{以幺正方式演化},从而符合量子力学的基本原则。[77]2005年,霍金宣布该悖论已经被 AdS/CFT 对应解决,并支持信息守恒的观点。他还提出了一种\textbf{具体机制},说明黑洞可能如何保留信息。[78]
\subsubsection{对核物理的应用}
除了在\textbf{量子引力的理论问题}上的应用外,AdS/CFT 对应还被用于研究量子场论中的多种问题。其中一个利用AdS/CFT 对应研究的物理系统是夸克-胶子等离子体(quark–gluon plasma),这是一种在粒子加速器中产生的特殊物质状态。这种物质状态在\textbf{重离子碰撞}(如金或铅原子核在高能条件下碰撞)时会短暂出现。这类碰撞会使组成原子核的夸克在约两万亿开尔文(Kelvin)的极端温度下解禁闭,这一条件类似于大爆炸后约\(10^{-11}\)秒时的宇宙环境。[79]

夸克-胶子等离子体的物理由量子色动力学(Quantum Chromodynamics, QCD)描述,但在涉及夸克-胶子等离子体的问题上,QCD 在数学上极为复杂,难以直接求解。[c]2005年,Đàm Thanh Sơn及其合作者发表的一篇论文表明,AdS/CFT 对应可用于研究夸克-胶子等离子体的某些性质,并将其用\textbf{弦理论的语言}加以描述。[80]通过应用 AdS/CFT 对应,Sơn 及其合作者将夸克-胶子等离子体与五维时空中的黑洞联系起来。他们的计算表明,与夸克-胶子等离子体相关的两个物理量——\textbf{剪切粘度}和\textbf{熵的体积密度}之比应大致等于某个普适常数。2008年,布鲁克海文国家实验室的相对论重离子对撞机(RHIC)实验确认了夸克-胶子等离子体的这一预测值。[7][81]
\subsection{对凝聚态物理的应用} 
AdS/CFT 对应也被用于研究\textbf{凝聚态物理}的一些问题。几十年来,实验凝聚态物理学家发现了许多奇异物态,包括超导体(superconductors)和超流体(superfluids)。这些物态通常使用\textbf{量子场论}的形式主义进行描述,但某些现象很难用\textbf{标准场论技术}来解释。一些凝聚态理论学家(如Subir Sachdev)希望AdS/CFT 对应能够将这些系统用弦理论的语言描述,并加深对其行为的理解。[7]  

目前,利用弦理论方法成功描述了超流体向绝缘体的相变。\textbf{超流体}是一种由电中性原子组成的系统,可以无摩擦流动。这种系统通常在实验室中利用液态氦制造,但近年来,实验物理学家开发出了一种新方法:将数万亿个冷原子倒入由交叉激光束形成的光学晶格中。在最初阶段,这些原子表现得像超流体,但当实验人员增加激光的强度时,原子的流动性逐渐减弱,并最终突然转变为绝缘体状态。在这一转变过程中,原子表现出一种异常行为。例如,原子减速停止的速率取决于温度和普朗克常数(Planck constant),而普朗克常数是量\textbf{子力学的基本参数},但在该系统的其他相态描述中并不会出现。这种现象最近得到了更深入的理解,研究者采用了一种对偶描述,即通过高维黑洞的性质来描述超流体。[8]
\subsection{现象学} 
除了作为一个具有重要理论兴趣的概念外,弦理论还提供了一个构建现实世界物理模型的框架,该框架结合了广义相对论和粒子物理。现象学(Phenomenology)是理论物理的一个分支,研究者在其中基于更抽象的理论概念构建符合现实的物理模型。弦现象学(String Phenomenology)是弦理论的一部分,旨在基于弦理论构建现实或准现实的物理模型。  

然而,由于理论和数学上的困难,以及实验上需要极高能量才能进行检验,迄今为止,尚无实验证据能够明确指向任何一个弦理论模型是正确的基本自然描述。这导致物理学界的一部分人批评这些统一理论的方法,并质疑继续研究这些问题的价值。[12]
\subsubsection{粒子物理}
当前被广泛接受的描述基本粒子及其相互作用的理论被称为粒子物理标准模型(Standard Model of Particle Physics)。该理论统一描述了三种基本自然力:电磁力、强核力和弱核力。尽管标准模型在解释各种物理现象方面取得了巨大成功,但它仍然不是对现实世界的完整描述。其局限性包括:无法纳入引力,以及诸如层级问题(hierarchy problem)、无法解释费米子质量的结构或暗物质等未解难题。

弦理论被用于构建超越标准模型的各种粒子物理模型。通常,这些模型基于紧致化(compactification)的概念。物理学家从弦理论或 M 理论的十维或十一维时空出发,并假设额外维度的形状。通过适当地选择这些额外维度的几何结构,他们可以构建大致类似于标准模型的物理模型,同时包含一些尚未发现的额外粒子。[82]一种常见的从弦理论推导现实物理的方法是从十维的异托弦理论(heterotic theory)出发,并假设时空的六个额外维度的形状类似于六维 Calabi–Yau 流形。这种紧致化方法提供了许多从弦理论中提取现实物理的途径。此外,还可以使用其他类似的方法,基于\(M\)理论构建符合现实或准现实的四维世界模型。[83]
\subsubsection{宇宙学}
\begin{figure}[ht]
\centering
\includegraphics[width=8cm]{./figures/9867a4a9cd4f8413.png}
\caption{由威尔金森微波各向异性探测器(Wilkinson Microwave Anisotropy Probe, WMAP)绘制的宇宙微波背景辐射(CMB)图} \label{fig_String_10}
\end{figure} 
大爆炸理论(Big Bang Theory)是目前主流的宇宙学模型,用于描述宇宙从已知的最早时期到后续大尺度演化的过程。尽管该理论成功解释了许多观测到的宇宙特征,例如星系的红移、氢和氦等轻元素的相对丰度,以及宇宙微波背景辐射(CMB)的存在,但仍有几个未解之谜。例如,标准大爆炸模型无法解释:为什么宇宙在各个方向上看起来如此均匀(各向同性问题);为什么在极大尺度上宇宙似乎是平坦的(平坦性问题);为什么某些理论预测的粒子(如磁单极子)在实验中从未被观测到。[84]

目前,超越大爆炸理论的主要候选理论是宇宙暴胀理论(cosmic inflation)。该理论由Alan Guth等人在1980 年代提出,假设在标准大爆炸理论所描述的膨胀阶段之前,宇宙经历了一段极端迅速的加速膨胀时期。\textbf{宇宙暴胀理论}在保留大爆炸理论成功解释的现象的同时,为宇宙中的一些神秘特征提供了自然的解释。[85] 此外,该理论还得到了宇宙微波背景辐射(CMB)的观测支持,CMB 是自大爆炸约38 万年后便充满整个宇宙的辐射信号。[86]

在暴胀理论(inflation theory)中,宇宙最初的快速膨胀是由一种假设的粒子引起的,这种粒子被称为暴胀子(inflaton)。该理论并未固定暴胀子的确切性质,而是认为这些性质最终应从更基础的理论(如弦理论)推导出来。[87]实际上,已经有许多研究尝试在弦理论所描述的粒子谱中寻找暴胀子,并利用弦理论研究宇宙暴胀现象。尽管这些方法可能最终在观测数据中(如宇宙微波背景辐射(CMB)测量)找到支持,但目前弦理论在宇宙学中的应用仍处于早期阶段。[88]
\subsection{与数学的联系}
除了对理论物理研究产生影响外,弦理论还推动了纯数学领域的一系列重大进展。与许多正在发展的理论物理概念一样,弦理论目前尚未建立一个完全严格的数学公式化框架,即尚无法精确定义其所有概念。因此,研究弦理论的物理学家通常依靠物理直觉,猜测在不同数学结构之间可能存在的关系,尽管这些结构表面上看似完全不同,但它们用于描述弦理论的不同部分。随后,这些猜测被数学家证明,从而使弦理论成为纯数学中新思想的重要来源。[89]
\subsubsection{镜像对称}
\begin{figure}[ht]
\centering
\includegraphics[width=6cm]{./figures/cda90144ca5c961b.png}
\caption{Clebsch 三次曲面是一种被称为代数簇(algebraic variety)的几何对象的示例。枚举几何(enumerative geometry)的一个经典结果表明,在该曲面上恰好存在 27 条完全位于其上的直线。} \label{fig_String_11}
\end{figure}
当Calabi–Yau 流形被引入物理学,作为弦理论中紧致化额外维度的一种方式后,许多物理学家开始研究这些流形。在20 世纪 80 年代末,一些物理学家注意到,对于弦理论的一种特定紧致化方式,无法唯一地重构出对应的 Calabi–Yau 流形。[90] 相反,两种不同的弦理论版本——IIA 型弦理论和IIB 型弦理论,可以分别紧致化在完全不同的 Calabi–Yau 流形上,却导致相同的物理现象。在这种情况下,这些Calabi–Yau 流形被称为“镜像流形”(mirror manifolds),而两种物理理论之间的关系被称为镜像对称(mirror symmetry)。[28]

无论弦理论的 Calabi–Yau 紧致化是否能够正确描述自然,不同弦理论之间镜像对偶性的存在都具有重要的数学意义。在弦理论中使用的Calabi–Yau 流形也是纯数学中的研究对象,而镜像对称使数学家能够解决枚举几何(enumerative geometry)中的问题。枚举几何是数学的一个分支,主要研究几何问题的解的数量。[28][91]

\textbf{枚举几何}研究一类被称为代数簇(algebraic varieties)的几何对象,这些对象由多项式的零点集定义。例如,右侧所示的Clebsch 三次曲面是一个代数簇,它由四个变量的某个三次多项式定义。十九世纪数学家Arthur Cayley 和George Salmon取得的一个著名结果表明,在这样的曲面上恰好存在 27 条完全位于其上的直线。[92]  

推广这一问题,我们可以进一步探究:在五次 Calabi–Yau 流形(如上方所示的流形,由一个五次多项式定义)上,可以画出多少条直线?这一问题由十九世纪德国数学家 Hermann Schubert解决,他发现恰好存在 2,875 条这样的直线。 1986 年,几何学家Sheldon Katz证明,在该五次 Calabi–Yau 流形上,完全位于其中并由二次多项式定义的曲线(例如圆)共有 609,250 条。[93]

到1991 年,枚举几何(enumerative geometry)的大多数经典问题已经被解决,对这一领域的研究兴趣开始减弱。[94]然而,在1991 年 5 月,物理学家Philip Candelas、Xenia de la Ossa、Paul Green 和 Linda Parkes证明了镜像对称(mirror symmetry)可以用来将一个 Calabi–Yau 流形上的复杂数学问题转化为其镜像流形上的更简单问题,从而重新激发了枚举几何的研究。[95]特别是,他们利用\textbf{镜像对称} 计算出一个六维 Calabi–Yau 流形中恰好包含 317,206,375 条三次曲线。[94]除了计算三次曲线的数量外,Candelas 及其合作者还得出了更多更一般的关于有理曲线计数的结果,这些结果远超数学家此前的研究进展。[96]
\subsubsection{怪异月光}
\begin{figure}[ht]
\centering
\includegraphics[width=6cm]{./figures/2b5b463763a30cb0.png}
\caption{} \label{fig_String_12}
\end{figure}
群论(Group Theory)是数学的一个分支,专门研究对称性(symmetry)的概念。例如,可以考虑一个几何图形,如正三角形。有多种操作可以作用于这个三角形,而不改变其形状。例如,可以将其旋转 120°、240° 或 360°,或者可以沿着图中标记为\(S_0\)、\(S_1\)或\(S_2\)的轴进行反射。每种这样的操作都称为对称变换,而这些对称变换的集合满足某些特定的数学性质,使其成为数学家所称的群(group)。在这个特定的例子中,该群被称为阶为 6 的二面体群(dihedral group of order 6),因为它包含六个元素。一个一般的群可以描述有限个或无限多个对称操作;如果只包含有限个对称操作,则称为有限群(finite group)。[104]

数学家通常致力于对特定类型的数学对象进行分类(或列举)。然而,通常认为有限群(finite groups)过于多样化,以至于无法得到一个实用的分类方案。一个更温和但仍然极具挑战性的问题是对所有有限单群(finite simple groups)进行分类。有限单群是一种不可再分解的有限群,它们在构造任意有限群时起着类似于素数在构造任意整数中的作用,即通过取乘积来构造更复杂的对象。[e]当代群论的一项重大成就是有限单群的分类(Classification of Finite Simple Groups),这是一个数学定理,它提供了所有可能的有限单群的完整列表。[104]

该分类定理(classification theorem)确定了若干无限族的群(infinite families of groups),以及26 个不属于任何族的额外群。 这些额外的群被称为“散在群”(sporadic groups),每个群的存在都源于一种非凡的数学结构组合。其中最大的散在群,即“怪兽群”(monster group),包含超过\(10^{53}\)个元素,其数量比地球上的原子总数多一千倍。[105]
\begin{figure}[ht]
\centering
\includegraphics[width=6cm]{./figures/a8db3a8af0ee4d94.png}
\caption{} \label{fig_String_13}
\end{figure}
