% 格林函数法解非齐次偏微分方程

\begin{issues}
\issueDraft
\end{issues}

\pentry{狄拉克 delta 函数\upref{Delta}}

弦的方程
\begin{equation}\label{GreenF_eq1}
-\laplacian y = f(x)
\end{equation}

格林函数法, 先令格林函数 $G(x_0, x)$ 满足
\begin{equation}
-\laplacian G(x_0, x) = \delta(x - x_0)
\end{equation}
这相当于弦上只有一点受力.

一个连续的受力分布 $f(x)$ 可以分解为许多不同位置的 $x_0$ 的线性组合(积分)
\begin{equation}
f(x) = \int_a^b f(x_0) \delta(x - x_0) \dd{x_0}
\end{equation}
由于\autoref{GreenF_eq1} 的方程是线性的, 
