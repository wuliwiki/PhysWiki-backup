% 东南大学 2011 年 考研 量子力学
% license Usr
% type Note

\textbf{声明}:“该内容来源于网络公开资料,不保证真实性,如有侵权请联系管理员”

\textbf{1.(15 分)}以下叙述是否正确:

\begin{enumerate}
    \item  仅当体系处在定态时,守恒量的平均值才不随时间变化;
    \item  若厄密算符 $\hat{A}$ 与 $\hat{B}$ 不对易,则它们一定没有共同本征态;
    \item  一维谐振子的量子态空间是无穷维的;
    \item  如体系处于力学量 $\hat{A}$ 的本征态,则测量 $\hat{A}$ 必会得到一个确定值;
    \item  空间平移对称性导致宇称守恒。
\end{enumerate}

\textbf{2.(15 分)}质量为 $m$ 的粒子处于一维无限深方势井中,$V(x) = 0$,$(|x| < a/2)$;$V(x) = \infty$,$(|x| > a/2)$。试求能量本征值和归一化的本征函数。

\textbf{3.(15 分)}一维谐振子的哈密顿量为 
$$\hat{H} = \hat{p}^2/2m + m\omega^2 \hat{x}^2/2,~$$ 
定义
$$\hat{a} \equiv (\alpha \hat{x} + i \hat{p}/\hbar\alpha)/\sqrt{2}, \quad (\alpha = \sqrt{m \omega / \hbar}).~$$,试证明:
\begin{enumerate}
    \item  $[\hat{a}, \hat{a}^\dagger] = 1$;
    \item  $\hat{H} = (\hat{N} + 1/2)\hbar \omega$,其中 $\hat{N}$ 定义为 $\hat{N} \equiv \hat{a}^\dagger \hat{a}$;
    \item  若 $|\hat{n}\rangle$ 为 $\hat{N}$ 的本征态,即 $\hat{N}|\hat{n}\rangle = n|\hat{n}\rangle$,则 $\hat{a}^\dagger|n\rangle$ 也是 $\hat{N}$ 的本征态,并求相应的本征值。
\end{enumerate}

\textbf{4.(15 分)}一维谐振子哈密顿量为 
$$\hat{H} = \frac{\hat{p}^2}{2m} + \frac{m\omega^2 \hat{x}^2}{2},~$$ 
定义 $\hat{X}(t) = e^{i\hat{H}t/\hbar} \hat{x} e^{-i\hat{H}t/\hbar}$,试证明:
\begin{enumerate}
    \item  $d\hat{X}(t)/{dt} = \hat{P}(t)/{m}, d\hat{P}(t)/{dt} = -m\omega^2 \hat{X}(t)$;
    \item  $\hat{X}(t) = \hat{x} \cos(\omega t) + (\hat{p}/{m\omega} )\sin(\omega t)$;
    \item  $\hat{P}(t) = \hat{p} \cos(\omega t) - m\omega \hat{x} \sin(\omega t)$。
\end{enumerate}

\textbf{5.(15 分)}设 $|n\rangle$ 是某体系的一个非简并的能量本征态,试证明 $|n\rangle$ 一定是该体系所有守恒量的共同本征态。

\textbf{6.(15 分)}一质量为 $m$ 带电量为 $q$ 的粒子在均匀磁场 $\mathbf{B} = (0,0,B)$ 中运动,磁场的矢量势选为 $\mathbf{A} = (-By,0,0)$,粒子的哈密顿算符为 $\hat{H} = (1/2m)(\hat{p} - q\mathbf{A}/c)^2$.
    \begin{enumerate}
        \item 动量 $\hat{p}_x, \hat{p}_y, \hat{p}_z$ 中哪些是守恒量?
        \item 求粒子的能量本征值。
    \end{enumerate}

\textbf{7.(15 分)}质量为 $m$ 的粒子以能量 $E$ 从左入射,受势场的散射:$V(x) = 0, (x < 0); V(x) = V_0, (x > 0)$,其中 $E > V_0 > 0$,计算反射系数和透射系数。提示:几率流密度公式为 $j(x) = (\hbar/i2m)(\psi^*\psi' - \psi^{*'}\psi)$.

\textbf{8.(15 分)}2 个全同粒子组成体系的哈密顿量为 $\hat{H} = \hat{h}_1 + \hat{h}_2, \hat{h}_i = \hat{p}_i^2/2m + mw^2 x_i^2/2, (i = 1,2)$. $\hat{h}_i \phi_n(x_i) = \epsilon_n \phi_n(x_i), \epsilon_n = (n + 1/2)\hbar \omega, (n = 0,1,\cdots)$ 分以下两种情况写出体系的基态波函数和基态能量: 
    \begin{enumerate}
        \item  Bose 子;
        \item  Fermi 子。
    \end{enumerate}

 \textbf{9.(15 分)}设 $\hat{\sigma}$ 为泡利算符,$\hat{\sigma}_z$ 的归一化本征态为 $| \pm \rangle$,即 $\hat{\sigma}_z | \pm \rangle = \pm | \pm\rangle$.
    \begin{enumerate}
        \item 利用 $\hat{\sigma}_z \hat{\sigma}_x = -\hat{\sigma}_x \hat{\sigma}_z = i \hat{\sigma}_y$ 和 $\hat{\sigma}_z^2 = 1$ 证明 $\hat{\sigma}_z | + \rangle = e^{i \alpha} | - \rangle$,$(\alpha^* = \alpha)$;
        \item 取 $\alpha = 0$,求 $\hat{\sigma}_z | \pm \rangle$,$\hat{\sigma}_y | \pm \rangle$。
    \end{enumerate}   

 \textbf{10.(15 分)}一维谐振子的能量本征方程为 $\hat{H}|n\rangle = E_n^{(0)} |n\rangle$ 其中 $\hat{H} = \hat{p}^2/2m + mw^2 x^2 / 2$,$E_n^{(0)} = (n + 1/2)\hbar \omega$,$\langle n| \hat{x} | k \rangle =\left( \sqrt{k+1/2} \delta_{n, k+1} + \sqrt{k/2} \delta_{n, k-1} \right)/\alpha$。体系受到微扰哈密顿 $\hat{H}' = \lambda \hat{x}$ 的作用,试用以下公式计算能级 $E_k^{(0)}$ 的修正
$$E_k \approx E_k^{(0)} + \langle k | \hat{H}' | k \rangle + \sum_{n\neq k} \frac{|\langle n | \hat{H}' | k \rangle |^2}{E_k^{(0)} - E_n^{(0)}}.~$$