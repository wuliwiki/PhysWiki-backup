% 函数视角下的三角函数(高中)
% keys 函数|三角函数|周期|性质
% license Usr
% type Tutor

\begin{issues}
\issueDraft
\end{issues}

\pentry{三角函数\nref{nod_HsTrFu},函数\nref{nod_functi},函数的性质\nref{nod_HsFunC},导数\nref{nod_HsDerv}}{nod_5a43}

\subsection{三角函数的性质}
易得,正弦函数和余弦函数的\textbf{定义域为全体实数},正切函数的定义域为 $\begin{Bmatrix}\alpha|\alpha \neq \frac{\pi}{2}+k\pi,k\in Z\end{Bmatrix}~.$
\subsubsection{周期性}
正弦函数、余弦函数、正切函数的都是周期函数,根据定义易得,正弦函数和余弦函数,周期为 $2k\pi(k\in Z,k\neq0)$,正切函数的周期为 $k\pi(k\in Z,k\neq0)$.
\subsubsection{导数}

\subsubsection{图像}

\begin{figure}[ht]
\centering
\includegraphics[width=14.25cm]{./figures/bb986656153a547d.png}
\caption{正弦函数} \label{fig_HsTrFu_1}
\end{figure}
\begin{figure}[ht]
\centering
\includegraphics[width=14.25cm]{./figures/d0ae1e0d9c167f1e.png}
\caption{余弦函数} \label{fig_HsTrFu_2}
\end{figure}
可以看出正弦函数和余弦函数是定义域为 $R$ 值域为 $[-1,1]$ 最小正周期 $T = 2\pi$ 的周期函数。
