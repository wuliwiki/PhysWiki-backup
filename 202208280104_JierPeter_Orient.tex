% 定向
% 取向

本文翻译自Encyclopedia of Mathematics(数学百科)的\href{https://encyclopediaofmath.org/wiki/Volume_form}{Orientation}\footnote{Volume form. Encyclopedia of Mathematics. URL: http://encyclopediaofmath.org/index.php?title=Volume_form&oldid=32331.}词条.

\subsection{一般概念}

在传统数学中,一个\textbf{定向(orientation)}(或译作\textbf{取向})是指一种坐标系的等价划分,如果两个坐标系\textbf{正相关(positively related)}则是等价的.

对于有限实线性空间$\mathbb{R}^n$,一个坐标系由一组基确定,而两组基等价的条件是\textbf{转移矩阵}\upref{TransM}的行列式为正数.这个等价关系划分出两个等价类.对于复数的情况,即$\mathbb{C}^n$,任取其复基$\{e_1, \cdots, e_n\}$,则能导出实基$\{e_1, \cdots, e_n, \I e_1, \cdots, \I e_n\}$,从而可以将其视为$\mathbb{R}^{2n}$.任意两个复基分别导出的实基就是正相关的(也就是说,复结构定义了$\mathbb{R}^{2n}$上的定向).

在一条线、一个面或者更一般的实\textbf{仿射空间}\upref{AfSp}$E^n$上,一个坐标系由一个点(原点)和一组基给定,坐标系的变换由一个平移(改变原点)和一个基变换给定.坐标系的变换是正的,当且仅当基变换的转移矩阵行列式为正数.(举个例子:基向量的偶置换.)两个坐标系定义的定向相同,当且仅当其中一个可以连续地变为另一个,即存在由参数$t\in[0, 1]$给定的一族坐标系$O_t, e_t$关于$t$是连续的,则$O_0, e_0$到$O_1, e_1$的变换就是连续的.在$n-1$维超平面上的\textbf{反射(reflection)}映射能反转定向,即将一个定向中的坐标系映入另一个定向.

坐标系的等价类也能用不同的\textbf{几何体(geometric figures)}\footnote{译注:geometric figures指任何点、线、面等构成的集合,是几何空间的子集.}来定义.如果一个几何体按照某种规则与一个坐标系关联,那么它的镜像在统一规则像



















