% 安德烈-马里·安培(综述)
% license CCBYSA3
% type Wiki

本文根据 CC-BY-SA 协议转载翻译自维基百科\href{https://en.wikipedia.org/wiki/Andr\%C3\%A9-Marie_Amp\%C3\%A8re}{相关文章}。
\begin{figure}[ht]
\centering
\includegraphics[width=6cm]{./figures/91e635e1a8e48ddb.png}
\caption{安培的雕版画} \label{fig_AP_2}
\end{figure}
安德烈-玛丽·安培(André-Marie Ampère ForMemRS,英国发音:/ˈɒ̃pɛər, ˈæmpɛər/,美国发音:/ˈæmpɪər/,[1] 法语:[ɑ̃dʁe maʁi ɑ̃pɛʁ];1775年1月20日-1836年6月10日)[2] 是法国物理学家和数学家,是经典电磁学学科的奠基人之一,他将其称为‘电动力学’。他还发明了众多应用,例如由他命名的螺线管和电报机。作为一位自学成才的科学家,安培是法国科学院的院士,并在巴黎综合理工学院和法兰西学院担任教授。

国际单位制中的电流单位安培(A)以他的名字命名。他的名字还被刻在埃菲尔铁塔上的72个名字之一。‘运动学’(kinematic)一词是他创造的法语‘cinématique’的英语版本,[3] 它来源于希腊语 κίνημα kinema(意为‘运动’),其本身衍生自 κινεῖν kinein(意为‘移动’)。[4][5]
\subsection{传记}
\subsubsection{早年生活} 
安德烈-玛丽·安培于1775年1月20日出生在里昂,他的父亲是成功的商人让-雅克·安培,母亲是让娜·安托瓦内特·德苏蒂耶尔-萨尔塞·安培。他出生于法国启蒙运动的鼎盛时期,童年和少年时期大部分时间都在靠近里昂的家族庄园——波莱米约-蒙多尔(Poleymieux-au-Mont-d'Or)度过。[6] 让-雅克·安培是一位成功的商人,同时也是让-雅克·卢梭哲学的仰慕者。卢梭在其著作《爱弥儿》中阐述的教育理论成为安培教育的基础。卢梭认为,男孩应该避免接受正式学校教育,而是从“自然中直接获得教育”。安培的父亲将这一理念付诸实践,允许儿子在家族藏书丰富的图书馆中自学。因此,像乔治-路易·勒克莱尔(乔治·路易·布封)的《自然史》(自1749年开始撰写)和丹尼斯·狄德罗与让·勒隆·达朗贝尔的《百科全书》(1751年至1772年间新增的卷册)等法国启蒙时期的杰作,成为了安培的“老师”。[需要引用]  

然而,年轻的安培很快重新开始了拉丁语课程,这使他能够深入研究莱昂哈德·欧拉和丹尼尔·伯努利的著作。[7]
\subsubsection{法国大革命}
此外,安培利用接触最新书籍的机会,从12岁起自学高等数学。晚年时,安培声称自己在18岁时对数学和科学的了解已经达到了人生的巅峰。然而,作为一位博学多才的人,他的阅读范围还包括历史、旅行、诗歌、哲学和自然科学。[7] 由于他的母亲是一位虔诚的天主教徒,安培在接触启蒙科学的同时,也接受了天主教信仰的熏陶。法国大革命(1789-1799)在他的少年时代爆发,这对他产生了深远的影响。安培的父亲因革命新政府的召唤而参与公共事务,[8] 并在里昂附近的一个小镇担任地方法官(*juge de paix*)。1792年,雅各宾派掌控革命政府后,他的父亲让-雅克·安培因抵制新的政治浪潮,于1793年11月24日被送上断头台,这是雅各宾派清洗的一部分。

1796年,安培结识了朱莉·卡隆(Julie Carron),并于1799年结婚。同年,安培开始了他的第一份正式工作,担任数学教师,这使他有了经济保障,与卡隆结婚,并在次年迎来了他们的第一个孩子让-雅克(以其父亲命名)。后来,让-雅克·安培因语言学研究而声名鹊起。安培的成长与法国向拿破仑政权的过渡同步,他在拿破仑支持的技术官僚体系中找到了新的成功机会。

1802年,安培被任命为布雷斯堡中央学校(École Centrale)的物理和化学教授,并将生病的妻子和婴儿让-雅克·安托万·安培留在里昂。他利用在布雷斯堡的时间进行数学研究,并于1802年完成了《关于博弈数学理论的思考》(*Considérations sur la théorie mathématique du jeu*)的论文,这是一部关于数学概率的著作。他于1803年将其送交巴黎科学院审议。
\subsubsection{教学生涯}
\begin{figure}[ht]
\centering
\includegraphics[width=6cm]{./figures/23e944b2e0dca4a0.png}
\caption{《科学哲学试论》} \label{fig_AP_1}
\end{figure}
1803年7月,安培的妻子去世后,[9][10] 他搬到了巴黎,并于1804年在新成立的综合理工学院(École Polytechnique)担任导师。尽管他没有正式的学术资格,安培于1809年被任命为该校的数学教授。在1828年之前,他一直在该校担任各种职务。此外,1819年和1820年,安培分别在巴黎大学开设了哲学和天文学课程;1824年,他被选为法国学院(Collège de France)实验物理学著名讲席的教授。1814年,安培被邀请加入新成立的帝国学院数学班,这是改革后的国家科学院的上级机构。

在被选入科学院之前的几年,安培参与了广泛的科学研究,撰写论文并探讨从数学、哲学到化学和天文学的各种主题,这在当时顶尖的科学知识分子中是很常见的。安培曾声称:“在他18岁时,他的人生中有三个至高点:第一次领圣体、阅读安托万·莱昂纳·托马的《笛卡尔的颂辞》,以及攻占巴士底狱的消息。”在他妻子去世的那一天,他写下了《诗篇》中的两句经文,并祈祷:“主啊,慈悲的上帝,请在天上让我与那些您允许我在地上所爱的人团聚。”在困境中,他会通过阅读《圣经》和教会教父的作品寻求慰藉。[11]

作为一名虔诚的平信徒,他曾一度接纳年轻学生弗雷德里克·奥扎南(Frédéric Ozanam,1813–1853)住在家中。奥扎南后来成为慈善会议(Conference of Charity,后称圣文森·德保禄会)的创始人之一。[citation needed] 奥扎南于1998年由教宗若望·保禄二世册封为真福。通过安培的引荐,奥扎南接触到了新天主教运动的领袖,例如弗朗索瓦-勒内·德·夏多布里昂(François-René de Chateaubriand)、让-巴蒂斯特·亨利·拉科代尔(Jean-Baptiste Henri Lacordaire)和夏尔·福布斯·勒内·德·蒙塔朗贝尔(Charles Forbes René de Montalembert)。[citation needed]
\subsubsection{电磁学领域的研究}
1820年9月,安培的朋友、后来的悼词作者弗朗索瓦·阿拉戈向法国科学院的成员展示了丹麦物理学家汉斯·克里斯蒂安·奥斯特德的惊人发现:电流能够使邻近的磁针偏转。安培随即开始发展一种数学和物理理论,以理解电与磁之间的关系。在奥斯特德实验工作的基础上,安培进一步研究发现,平行导线中的电流会相互吸引或排斥,具体取决于电流方向是否相同或相反——这一发现奠定了电动力学的基础。他还通过数学的方法,将这些实验结果推广为普遍的物理定律,其中最重要的就是后来被称为**安培定律**的原理。该定律指出,两段载流导线之间的相互作用力与导线长度及其电流强度成正比。安培还将这一原理应用于磁学,证明了其定律与法国物理学家夏尔·奥古斯丁·库仑的电作用定律之间的和谐一致性。安培对实验技术的专注和精湛技能,使他的科学研究牢牢根植于新兴的实验物理领域。

安培还对电磁关系提供了物理学上的理解,他推测了“电动力分子”(即电子概念的雏形)的存在,这种分子是电与磁的基本组成单元。基于这一物理解释,安培发展出了一套关于电磁现象的物理学理论,这一理论既可通过实验验证,又具有数学上的预测能力。近百年后,在1915年,阿尔伯特·爱因斯坦和万德·约翰内斯·德哈斯通过**爱因斯坦-德哈斯效应**证明了安培假设的正确性。

1827年,安培发表了他的代表作《仅通过实验得出的电动力学现象数学理论的回忆录》(**Mémoire sur la théorie mathématique des phénomènes électrodynamiques uniquement déduite de l'experience**)。这部著作为其新科学——电动力学命名,并成为该领域的奠基之作。

同年,安培被选为英国皇家学会的外籍会员,次年被选为瑞典皇家科学院的外籍会员。[12] 或许,他最崇高的荣誉来自詹姆斯·克拉克·麦克斯韦,后者在《电与磁理论》一书中称安培为“电学领域的牛顿”。
\subsection{荣誉}
\begin{itemize}
\item 1825年10月8日:安培被选为比利时皇家科学、文学与美术学院院士。[13]
\end{itemize}
\subsubsection{遗产}
1881年国际电力博览会签署的一项国际公约,将**安培**(ampere)确立为电流的标准单位之一,以表彰安培在现代电学科学创建中的贡献。同一公约中,还有库仑(coulomb)、伏特(volt)、欧姆(ohm)、瓦特(watt)和法拉(farad)等单位,它们分别以安培同时代的科学家命名:法国的夏尔-奥古斯丁·库仑、意大利的亚历山德罗·伏打、德国的乔治·欧姆、苏格兰的詹姆斯·瓦特和英格兰的迈克尔·法拉第。安培的名字还被刻在埃菲尔铁塔的72个名字之一中。

许多街道和广场以安培的名字命名。此外,还有以他命名的学校、一座里昂地铁站、一种图形处理器微架构、月球上的一座山,以及挪威的一艘电动渡轮。[14]
\subsection{著作}

\begin{itemize}
\item 《关于数学游戏理论的思考》(Considérations sur la théorie mathématique du jeu),Perisse,里昂,巴黎,1802年,可在线阅读于互联网档案馆。
\item 安德烈-玛丽·安培(1822),《电动力学观察集:包含各种论文、通知、书信或科学期刊摘录,涉及两个电流之间的相互作用、磁针或地球与两个磁铁之间的相互作用》(Recueil d'observations électro-dynamiques,法文),出版商:Chez Crochard,2010年9月26日获取。
\item 安德烈-玛丽·安培;雅克·M. 巴比内(1822),《关于电和磁的新发现的展示》(Exposé des nouvelles découvertes sur l'électricité et le magnétisme,德文),出版商:Chez Méquignon-Marvis,2010年9月26日获取。
\item 安德烈-玛丽·安培(1824),《电动力学装置的描述》(Description d'un appareil électro-dynamique,法文),出版商:Chez Crochard ... et Bachelier,2010年9月26日获取。
\item 安德烈-玛丽·安培(1826),《电动力学现象理论,仅从实验中推导》(Théorie des phénomènes électro-dynamiques, uniquement déduite de l'expérience,法文),出版商:Méquignon-Marvis,2010年9月26日获取。
   \item 安德烈-玛丽·安培(1883),《电动力学现象数学理论,仅从实验中推导》(Théorie mathématique des phénomènes électro-dynamiques: uniquement déduite de l'expérience,法文,第2版),出版商:A. Hermann,2010年9月26日获取。

\item 安德烈-玛丽·安培(1834),《科学哲学随笔,或对所有人类知识的自然分类的分析性展示》(Essai sur la philosophie des sciences, ou, Exposition analytique d'une classification naturelle de toutes les connaissances humaines,德文),出版商:Chez Bachelier,2010年9月26日获取。
  \item 安德烈-玛丽·安培(1834),《科学哲学随笔》(Essai sur la philosophie des sciences,德文,第1卷),出版商:Chez Bachelier,2010年9月26日获取。
   \item 安德烈-玛丽·安培(1843),《科学哲学随笔》(Essai sur la philosophie des sciences,德文,第2卷),出版商:Bachelier,2010年9月26日获取。
\end{itemize}
\textbf{部分译本:}
\begin{itemize}
\item Magie, W.M.(1963年)。《物理学资料汇编》(**A Source Book in Physics**)。哈佛大学出版社,剑桥,马萨诸塞州,第446–460页。
\item Lisa M. Dolling;Arthur F. Gianelli;Glenn N. Statile 编(2003年)。《时间的考验:物理理论发展的阅读材料》(**The Tests of Time: Readings in the Development of Physical Theory**)。普林斯顿:普林斯顿大学出版社,第157–162页。ISBN 978-0691090856。
\end{itemize}
\textbf{完整译本:}
\begin{itemize}
\item Ampère, André-Marie(2015年)。André Koch Torres Assis(编)。《安培电动力学:对安培电流元之间作用力的意义与演化分析,并附其代表作〈电动力现象理论〉的完整译本,仅从经验中推导》(Ampère's electrodynamics: analysis of the meaning and evolution of Ampère's force between current elements, together with a complete translation of his masterpiece: Theory of electrodynamic phenomena, uniquely deduced from experience)。J. P. M. C Chaib 译。蒙特利尔:Apeiron。ISBN 978-1-987980-03-5。[原文存档](https://web.archive.org/web/20221009000000/https://www.example.com)(PDF)于2022年10月9日。
\item Ampère, André-Marie(2015年)。《从实验唯一推导出的电动力学现象的数学理论》(Mathematical Theory of Electrodynamic Phenomena, Uniquely Derived from Experiments)。Michael D. Godfrey(斯坦福大学)译。
\end{itemize}
\subsection{参考文献:}\\
1. "Ampère". 《Random House Webster's Unabridged Dictionary》。\\ 
2. 《科学传记词典》(Dictionary of Scientific Biography)。美国:查尔斯·斯克里布纳之子公司(Charles Scribner's Sons)。1970年。ISBN 9780684101149。\\ 
3. Ampère, André-Marie(1834年)。《科学哲学论》(Essai sur la Philosophie des Sciences)。巴黎:Bachelier出版社。\\ 
4. Merz, John(1903年)。《19世纪欧洲思想史》(A History of European Thought in the Nineteenth Century)。伦敦:Blackwood。第52页。\\
5. O. Bottema 和 B. Roth(1990年)。《理论运动学》(Theoretical Kinematics)。纽约:多佛出版公司(Dover Publications)。序言,第5页。ISBN 0-486-66346-9。\\
6. "Andre-Marie Ampere". 《IEEE全球历史网络》(IEEE Global History Network)。IEEE。检索于2011年7月21日。\\
7. 一段或多段引用内容来自公共领域的出版物:Chisholm, Hugh,编。(1911年)。《Ampère, André Marie》。《大英百科全书》(Encyclopaedia Britannica)。第11版。剑桥大学出版社。第878–879页。\\
8. 《安德烈·玛丽·安培传记》。检索于2019年9月3日。\\
9. 安培在深深爱着的第一任妻子去世后再婚,但第二段婚姻非常不幸,并以离婚告终。\\
10. Laidler, Keith J.(1993年)。《点燃这样的蜡烛》(To Light such a Candle)。牛津大学出版社。第128页。\\
11. 《天主教百科全书》(Catholic Encyclopedia)。检索于2007年12月29日。\\
12. 《图书馆和档案目录》(Library and Archive Catalogue)。皇家学会。检索于2012年3月13日。\\
13. 《比利时皇家科学院会员和准会员传记索引》(Index biographique des membres et associés de l'Académie royale de Belgique (1769–2005))。第15页。\\
14. 《电池渡轮克服障碍,现在解决方案已清晰》(Batterifergen har måttet stå over avgangen. Nå er løsningen klar)。Teknisk Ukeblad。2016年11月。检索于2016年11月19日。
\subsection{延伸阅读:}
\begin{itemize}
\item Williams, L. Pearce(1970年)。《安培,安德烈-玛丽》("Ampère, André-Marie")。《科学传记词典》(Dictionary of Scientific Biography)。第1卷。纽约:查尔斯·斯克里布纳之子公司。第139–147页。ISBN 978-0-684-10114-9。
\item Hofmann, James R.(1995年)。《安德烈-玛丽·安培》(André-Marie Ampère)。牛津:布莱克威尔出版社(Blackwell)。ISBN 978-0631178491。
\item Duhem, Pierre Maurice Marie(2018年9月9日)。《安培力定律:现代引介》(Ampère's Force Law: A Modern Introduction)。Alan Aversa(译)。doi:10.13140/RG.2.2.31100.03206/1。检索于2019年7月3日。([EPUB](https://example.com))
\end{itemize}
\subsection{外部链接:}
\begin{itemize}
\item [安培与电学的历史](链接)——一个法语网站,由法国国家科学研究中心(CNRS)编辑,包含安培的书信集(包括全文及批判性版本,附带手稿图片,超过1000封信件)、安培的书目、实验及3D模拟。
\item [安培博物馆](链接)——一个法语网站,展示位于法国里昂附近Poleymieux-au-Mont-d'or的安培博物馆的信息。
\item [安培的电动力学](链接)——包含《电动力现象理论》的完整英文翻译版本(PDF)。
\item [安德烈-玛丽·安培之友协会](链接):一个致力于纪念安德烈-玛丽·安培的法国协会,负责管理安培博物馆。  
\item [约翰·J·奥康纳与埃德蒙·F·罗伯逊,"安德烈-玛丽·安培"](链接):圣安德鲁斯大学数学史档案馆(MacTutor History of Mathematics Archive)。  
\item [埃里克·沃尔夫冈·魏斯坦 (主编),"安培,安德烈(1775-1836)"](链接):ScienceWorld。  
\item [天主教百科全书中的安德烈·玛丽·安培](链接) 
\item [电学单位历史](链接)
\end{itemize}