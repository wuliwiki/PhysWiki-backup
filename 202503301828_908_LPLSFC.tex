% 拉普拉斯方程(综述)
% license CCBYSA3
% type Wiki

本文根据 CC-BY-SA 协议转载翻译自维基百科\href{https://en.wikipedia.org/wiki/Laplace\%27s_equation}{相关文章}。

在数学和物理学中,拉普拉斯方程是一个二阶偏微分方程,得名于皮埃尔-西蒙·拉普拉斯,他在1786年首次研究了其性质。通常写作:
\[
\nabla^2 f = 0~
\]
或
\[
\Delta f = 0~
\]
其中\(\Delta = \nabla \cdot \nabla = \nabla^2\)是拉普拉斯算符,\(^\text{[注1]}\)\(\nabla\cdot\)是散度算符(也表示为“div”),\(\nabla\)是梯度算符(也表示为“grad”),\(f(x, y, z)\)是一个二次可微的实值函数。因此,拉普拉斯算符将标量函数映射到另一个标量函数。

如果右边指定为一个给定函数,\(h(x, y, z)\)则我们有
\[
\Delta f = h~
\]
这被称为泊松方程,是拉普拉斯方程的一种推广。拉普拉斯方程和泊松方程是椭圆型偏微分方程的最简单例子。拉普拉斯方程也是赫尔姆霍兹方程的特例。

拉普拉斯方程解的一般理论称为势理论。拉普拉斯方程的二次连续可微解是调和函数,\(^\text{[1]}\)这些函数在多个物理学分支中非常重要,特别是在静电学、引力学和流体动力学中。在热传导的研究中,拉普拉斯方程是稳态热方程。\(^\text{[2]}\)一般来说,拉普拉斯方程描述了平衡状态或那些不显式依赖于时间的情况。
