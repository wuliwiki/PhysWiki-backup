% 图空间与赋权图
% keys 图空间|赋权图
% license Usr
% type Tutor

\pentry{图\nref{nod_Graph},向量空间\nref{nod_LSpace}}{nod_7029}
图空间是指定义在图的点集和边集上的函数全体,它们刚好满足\enref{向量空间}{LSpace}的定义。定义在点集上的函数全体叫做点空间,而定义在边集上的函数全体叫做边空间。赋权图则是在边空间中选择一个函数和图组成的二元组,所选的函数叫做权函数,权函数在边的值叫做改边的权。

\begin{definition}{点空间,边空间}
设 $D$ 是一个图\upref{Graph},其点集和边集为 $V(D)=\{v_1,\cdots,v_n\},E(D)=\{a_1,\cdots,e_m\}$。则定义在 $V(D)$ 上,值域为 $\mathbb R$ 的函数全体称为 $G$ 的\textbf{点空间}(vertex space),记作 $\mathcal V(G)$。而定义在 $E(D)$ 上,值域为 $\mathbb R$ 的函数全体称为 $G$ 的\textbf{边空间}(edge space),记作 $\mathcal E(G)$。
\end{definition}


\begin{exercise}{}
试证明:点空间 $\mathcal V(G)$ 和边空间 $\mathcal E(G)$ 满足\aref{向量空间}{def_LSpace_2}的定义。它们上加法和数乘定义如下:
\begin{equation}
\begin{aligned}
&(f+g)(x):=f(x)+g(x),\quad &f,g\in\mathcal V(G)\text{或} \mathcal E(G),v_i\in V(G)\text{或} E(G),\\
&(\lambda f)(x):=\lambda f(x),\quad &f\in\mathcal V(G),x\in V(D)\text{或} f\in\mathcal E(G),x\in E(D),\lambda\in \mathbb R.
\end{aligned}~
\end{equation}

\end{exercise}










