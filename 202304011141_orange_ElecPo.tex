% 电极化强度
% 电解质|极化强度|电偶极矩|极化电荷

\begin{issues}
\issueOther{需要与 “电介质的简单模型” 整合}
\end{issues}

\pentry{电介质\upref{Dielec}}

在电介质\upref{Dielec}一节中,我们从分子的电结构出发,说明了两类电介质极化的微观过程虽然不同,但宏观的效果却是相同的,都是在电介质的两个相对表面上出现了异号的极化电荷,在电介质内部有沿电场方向的电偶极矩。因此下面从宏观上描述电介质的极化现象时,就不分两类电介质来讨论了。

在电介质内任取一物理无限小的体积元 $\Delta V$(但其中仍有大量的分子),当没有外电场时,这体积元中所有分子的电偶极矩的矢量和 $\sum \mathbf p$ 等于零。但是,在外电场的影响下,由于电介质的极化,$\sum \mathbf p$ 将不等于零。外电场愈强,被极化的程度愈大,$\sum \mathbf p$ 的值也愈大。因此我们取单位体积内分子电偶极矩的矢量和,即
\begin{equation}
\mathbf P=\frac{\sum \mathbf p}{\Delta V} ~,
\end{equation}
或者写为更加直观的形式
\begin{equation}
\bvec P = N \cdot \bvec p
\end{equation}
其中$\bvec N$为电偶极子的数密度,即单位体积中电偶极子的数量。\footnote{较真地说,这种写法假定了每一个偶极子的极矩都相同。不过,这个公式只是在提醒你$\bvec P$矢量的物理含义,而不是进行定量的计算。}

$\bvec P$作为量度电介质极化程度的基本物理量,称为该点($\Delta V$ 所包围的一点)的电极化强度(electric polarization)或 $\mathbf P$ 矢量。在国际单位制中,电极化强度的单位是 $\rm C/m^2$。
