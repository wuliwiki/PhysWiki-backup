% 有限覆盖与紧性

\pentry{实数集的拓扑\upref{ReTop} 序列的极限\upref{SeqLim}}

\subsection{紧致性的概念}
紧致性 (compactness) 是一个重要的拓扑概念. 在分析学中, 它首次出现于对定义在实数集子集上函数的研究中. 我们试着举一些例子来说明紧致性是怎样一个概念.

从逻辑上说, 我们还没有引入连续函数这一概念, 但这并不妨碍我们从直观上去理解它. 直观上讲, 对于点集$E\subset\mathbb{R}$上的函数$f:E\to\mathbb{R}$, 如果当$x\in E$越来越接近$x_0$时, 函数值$f(x)$也会越来越接近$f$在$x_0$处的值$f(x_0)$, 那么就可以认为它在点$x_0\in E$处是"连续"的. 说$f$在$E$上连续, 也就是它在$E$的每一点处都连续. 

显然连续性是一个局部性质: 函数在一点处是否连续, 只跟它在这一点的某个邻域里的行为有关. 对于一般的集合$E$, 从其上函数的局部性质是无法推出整体性质的. 例如, 在开区间$(0,1)$上, 函数$f_1(x)=1/x$是连续的, 但它在$x\to0$时无界; 函数$f_2(x)=x^2$连续且有界, 但却达不到它的最大和最小值; 函数$f_3(x)=\sin(1/x)$连续且有界, 但在$x\to0$时震荡得越来越厉害, 根本没有极限. 归根结蒂, 这些"不好"的整体性质, 都来自于定义域$(0,1)$的某种"不好的性质": 由于$0$本身不属于开区间$(0,1)$, 所以没办法用$0$的邻域去覆盖到接近$0$的那些点.