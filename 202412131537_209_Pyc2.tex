% Python 第一步
% license Usr
% type Tutor

\subsection{创建一个Python项目}

打开 Pycharm , 新建项目

\begin{figure}[ht]
\centering
\includegraphics[width=14.25cm]{./figures/2ada35917a742b32.png}
\caption{选项选取} \label{fig_Pyc2_2}
\end{figure}

右键项目名——新建——Python 文件

\begin{figure}[ht]
\centering
\includegraphics[width=14.25cm]{./figures/44ea62f4da2d69c8.png}
\caption{新建Python文件} \label{fig_Pyc2_1}
\end{figure}

给文件命名

\begin{figure}[ht]
\centering
\includegraphics[width=14.25cm]{./figures/cd91ca45cc505f0c.png}
\caption{命名} \label{fig_Pyc2_3}
\end{figure}

恭喜你,建立了你的第一个Python文件!

\subsection{输入}

\begin{figure}[ht]
\centering
\includegraphics[width=14.25cm]{./figures/8bb25afdd70aa412.png}
\caption{编写代码位置} \label{fig_Pyc2_4}
\end{figure}

打出 \footnote{单词 print 的中文意思是打印,所以括号里面的内容是打印出来的?}print() ,其中括号里面的内容就是输出内容,例如下图

\begin{lstlisting}[language=python]
print(1234)
\end{lstlisting}

然后右键文件名,选择运行。(操作如下图)

\begin{figure}[ht]
\centering
\includegraphics[width=14.25cm]{./figures/b5c6db6c52a0d894.png}
\caption{运行} \label{fig_Pyc2_5}
\end{figure}

接着会弹出运行框。框内就是结果

\begin{figure}[ht]
\centering
\includegraphics[width=14.25cm]{./figures/9c001a49cf8c70ed.png}
\caption{输出结果} \label{fig_Pyc2_6}
\end{figure}

现在你已经会如何运行编写好的程序了,那我们继续讨论 print。

现在打出

\begin{lstlisting}[language=python]
print(1+2)
\end{lstlisting}

运行,结果如下

\begin{lstlisting}[language=bash]
3
\end{lstlisting}

可以发现,得到的结果是 3,而不是1+2

这就是print的另一个功能,可以计算括号内的结果并输出。

\subsubsection{变量}



大纲:
变量——【】




print 默认输出是换行的,如果要实现不换行需要在变量末尾加上 end="":

实例(Python 3.0+)
#!/usr/bin/python3
 
x="a"
y="b"
# 换行输出
print( x )
print( y )
 
print('---------')
# 不换行输出
print( x, end=" " )
print( y, end=" " )
print()

继续打出

\begin{lstlisting}[language=python]
print(你好)
\end{lstlisting}

运行后,结果如下

\begin{figure}[ht]
\centering
\includegraphics[width=14.25cm]{./figures/c18bf6eea4e75d37.png}
\caption{运行结果} \label{fig_Pyc2_7}
\end{figure}

并没有输出 你好 而是报错了(出现了红字),这是为什么呢?

这就要讲到数字类型与字符串了

\subsection{数字类型与字符串}

\subsubsection{3.1 数字类型}
python中数字有四种类型:整数、布尔型、浮点数和复数。

\begin{itemize}
\item int (整数) 如:1, 只有一种整数类型 int,表示为长整型,没有 python2 中的 Long。
\item bool (布尔), 如 True。
\item float (浮点数), 如 1.23、3E-2
\item complex (复数) - 复数由实部和虚部组成,形式为 a + bj,其中 a 是实部,b 是虚部,j 表示虚数单位。如 1 + 2j、 1.1 + 2.2j
\end{itemize}

字符串(String)
Python 中单引号 ' 和双引号 " 使用完全相同。
使用三引号(''' 或 """)可以指定一个多行字符串。
转义符 \。
反斜杠可以用来转义,使用 r 可以让反斜杠不发生转义。 如 r"this is a line with \n" 则 \n 会显示,并不是换行。
按字面意义级联字符串,如 "this " "is " "string" 会被自动转换为 this is string。
字符串可以用 + 运算符连接在一起,用 * 运算符重复。
Python 中的字符串有两种索引方式,从左往右以 0 开始,从右往左以 -1 开始。
Python 中的字符串不能改变。
Python 没有单独的字符类型,一个字符就是长度为 1 的字符串。
字符串切片 str[start:end],其中 start(包含)是切片开始的索引,end(不包含)是切片结束的索引。
字符串的切片可以加上步长参数 step,语法格式如下:str[start:end:step]

