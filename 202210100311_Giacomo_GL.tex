% 一般线性群
% 一般线性群|矩阵

\begin{issues}
\issueDraft
\end{issues}

\pentry{矢量空间\upref{LSpace},群\upref{Group}}

\begin{definition}{一般线性群}
对于给定域 $\mathbb{F}$ 上的一个向量空间 $V$,$V$上的全体可逆自线性算符构成一个群,记做 $\opn{GL}(V; \mathbb{F})$(或者 $\opn{GL}_{\mathbb{F}}(V)$ ),称为 $V$ (在$\mathbb{F}$上)的\textbf{一般线性群}.

特别的,当 $V = \mathbb{F}^n$ 时, $\opn{GL}(\mathbb{F}^n; \mathbb{F})$ 可以视作可逆矩阵的群,简写为 $\opn{GL}(n; \mathbb{F})$ (或者 $\opn{GL}_n(\mathbb{F})$ ),称为 $\mathbb{F}$ 的 $n$维\textbf{一般线性(矩阵)群}.

当 $\mathbb{F} = \mathbb{R}$ 或者 $\mathbb{C}$ 时,由于 $\mathbb{R}, \mathbb{C}$ 本身是拓扑空间,$\opn{GL}(n; \mathbb{F})$ 可以视作拓扑空间(矩阵空间) $\opn{Mat}(n \times n) \cong \mathbb{F}^{n^2}$的子拓扑空间,因此 $\opn{GL}(n; \mathbb{R}), \opn{GL}(n; \mathbb{C})$ 是拓扑群\autoref{TopGrp_def1}~\upref{TopGrp},更进一步的它们也是李群\autoref{LieGrp_def1}~\upref{LieGrp}.
\end{definition}

对于线性算符 $T \in \opn{GL}(n; \mathbb{F})$(或者更一般的 $\opn{GL}(V; \mathbb{F})$), $T$ 是可逆的当且仅当它的行列式为 $0 \in \mathbb{F}$,即
$$
\opn{GL}(n; \mathbb{F}): = \{T \in \opn{Mat}(n \times n) \mid \det(T) \neq 0\}
$$

\begin{definition}{特殊线性群}
$\mathbb{F}$ 的 $n$维\textbf{特殊线性(矩阵)群},$\opn{SL}(n; \mathbb{F})$,定义为
$$
\{T \in \opn{Mat}(n \times n) \mid \det(T) = 1\}
$$
类似的我们也可以定义\textbf{特殊线性群} $\opn{SL}(V)$.
\end{definition}

因为在域 $\mathbb{F}$ 中, $0 \neq 1$,很容易看出 $\opn{SL}(n; \mathbb{F})$ 是 $\opn{GL}(n; \mathbb{F})$ 的子群.


\subsection{作为李群的一般线性群}

这个章节我们来重点看看 $\opn{GL}(n; \mathbb{R})$ 和 $\opn{GL}(n; \mathbb{C})$ 以及它们的子群.

$\mathbb{R}$ 和 $\mathbb{C}$ 与一般的域的区别在于,
\begin{enumerate}
\item $\mathbb{R}$ 和 $\mathbb{C}$ 都是(Hausdorff\autoref{Topo5_def2}~\upref{Topo5})拓扑空间
\item $\mathbb{R}$ 和 $\mathbb{C}$ 上可以定义绝对值/模长,
$$
\begin{aligned}
\abs{\cdot}: \mathbb{R} &\to \mathbb{R}, x \mapsto \left\{\begin{aligned}
x \text{ if } x \geq 0 \\
-x \text{ if } x < 0 \\
\end{aligned} \right. \\
\abs{\cdot}: \mathbb{C} &\to \mathbb{C}, z \mapsto \sqrt{z \bar{z}}
\end{aligned}
$$
\item 我们有shi fen
\end{enumerate}