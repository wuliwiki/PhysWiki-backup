% 实数的表示

\pentry{完备公理(戴德金分割)\upref{Cmplt}}

我们最熟悉的实数表示方式是十进制小数或者二进制数. 在这里, 我们可以借助实数公理而给予这些表示以严格的逻辑基础. 特别地, 这能够解释一个古老的问题: $0.\dot{9}$ 到底等不等于 $1$?

我们固定一个正整数$q>1$作为进位制的基底. 当然, 以下的推理完全用不到$q$是整数这件事, 但取$q=2$或$q=10$当然是最方便的.

我们从如下非常简单的命题开始. 
\begin{lemma}{}
设实数$x>0$. 则有唯一一个整数$n\in\mathbb{Z}$和唯一一个$k\in\{0,1,2,...,q-1\}$, 使得$kq^n\leq x<(k+1)q^{n}$.
\end{lemma}

