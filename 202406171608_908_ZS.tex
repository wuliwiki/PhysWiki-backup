% 折射
% license CCBYSA3
% type Wiki

(本文根据 CC-BY-SA 协议转载自原搜狗科学百科对英文维基百科的翻译)

\begin{figure}[ht]
\centering
\includegraphics[width=6cm]{./figures/f21bf66165bd4191.png}
\caption{一束光线在塑料块中被折射。} \label{fig_ZS_1}
\end{figure}

在物理学中,\textbf{折射}是波从一种介质传播到另一种介质传播方向的变化,或者是在介质中的传播方向逐渐变化。 光的折射是最常见的折射现象,但声波和水波等其他波也会经历折射。波被折射的程度取决于波速的变化以及波传播相对于速度变化方向的初始方向。

对于光,折射遵循斯涅尔定律,该定律指出,对于给定的一对介质,入射角 $\theta_1$ 和折射角 $\theta_2$ 的正弦之比等于两种介质中的相速度之比 $(v_1 / v_2)$,或者等同于两种介质的折射率之比 $(n_2 / n_1)$。

$\frac{\sin \theta_1}{\sin \theta_2} = \frac{v_1}{v_2} = \frac{n_2}{n_1}$

光学棱镜和透镜利用折射来改变光线的方向,人眼也是如此。材料的折射率随着光的波长而变化,[3] 因此折射角也相应地变化。这被称为色散,并导致棱镜和彩虹将白光分成其组成光谱颜色。[4]

\subsection{常规解释}

\begin{figure}[ht]
\centering
\includegraphics[width=6cm]{./figures/45f5f2affa73a650.png}
\caption{当波进入较慢的介质时,波前被压缩。为了使波阵面在边界保持连续,波必须改变传播方向。} \label{fig_ZS_2}
\end{figure}

想象一个波从一种材料流向另一种传播速度较慢的材料(如图所示)。如果它以某个角度到达两种材料之间的界面,波的一侧将首先到达第二种材料,因此更早减速。随着波的一边走得越来越慢,整个波将向那一边偏折。这就是为什么当一个波进入一个较慢的物质时,它会远离表面或向法线弯曲。在相反情况(即波从一种材料流向传播速度更快的另一种材料)下,波的一侧将加速,并且波将向那一侧的方向偏折。

理解这件事情的另一种方法是考虑界面处波长的变化。当波从一种材料传播到另一种材料,其中波具有不同的速度v,而波的频率f将保持不变,但是波前之间的距离或波长λ=v/f将改变。如果速度降低,如上图的右边所示,波长也会减小。波阵面和界面之间的角度以及波阵面之间距离的变化必须在界面上改变,以保持波阵面完好无损。从以上的考虑,可以推导出两种材料中入射角θ1,、透射角θ2 和波速v1和v2 之间的关系。这是折射定律或斯涅耳定律,可以写成[5]

$\frac{\sin \theta_1}{\sin \theta_2} = \frac{v_1}{v_2}  $

折射现象可以从二维或三维波动方程中以更基本的方式导出。界面处的边界条件将要求界面两侧波矢的切向分量相同。[6] 由于波矢的大小取决于波速,因此波矢的方向需要改变。

上面讨论的相关波速是波的相速度。这通常接近群速度,群速度可以被视为波的真实速度,但是当它们的值不同时,在所有与折射相关的计算中使用相速度是很重要的。

垂直于界面传播的波(即其波前平行于边界),即使波的速度改变,波的传播方向也不会改变。

\subsection{光}

\begin{figure}[ht]
\centering
\includegraphics[width=6cm]{./figures/d44fa1b043f81ec1.png}
\caption{一支笔部分浸没在一碗水中,由于水面的折射而显得弯曲。} \label{fig_ZS_3}
\end{figure}

光的折射在我们日常生活的许多地方都能见到。它使水面下的物体看起来比实际距离更近。它是光学透镜的基础,使得如眼镜、照相机、双筒望远镜、显微镜和人眼等仪器的实现成为可能。折射也是一些自然的光学现象的产生原因,包括彩虹和海市蜃楼。

\subsubsection{2.1 折射定律}


\subsubsection{2.2 水面折射}

\subsubsection{2.3 色散}

\subsubsection{2.4 大气折射}

\subsection{水波}

\subsection{临床意义}

\subsection{声学}

\subsection{隧道效应}

\subsection{参考文献}