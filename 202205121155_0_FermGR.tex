% 费米黄金法则

\pentry{几种含时微扰\upref{TDPEx}}

\footnote{参考 \cite{Brandsen}.}使用含时微扰法, 方形脉冲的跃迁率为
\begin{equation} % 已和 Brandsen 验证
\dv{P}{t} = \frac{2\pi}{\hbar} \abs{W_{fi}}^2\rho(E_f)
\end{equation}
若加上简谐调制, 则需要除以 4 得到每个方向的跃迁概率. 这叫做\textbf{费米黄金法则(Fermi's Golden rule)}.

其中 $\rho$ 是能级密度. 若末态处于连续态 $\ket{\bvec k}$ 中(例如平面波或库伦平面波\upref{CulmWf}), 归一化条件 $\braket{\bvec k'}{\bvec k} = \delta(\bvec k - \bvec k')$, 那么
\begin{equation}
\rho(E_f) = \frac{4\pi k_f}{m} = 4\pi\sqrt{\frac{2E}{m}}
\end{equation}

\subsection{方形脉冲}
方形脉冲(\autoref{TDPEx_eq2}~\upref{TDPEx})
\begin{equation}
\abs{c_i(t)}^2 = \frac{\abs{W_{fi}}^2}{\hbar^2} \Delta t^2 \sinc^2[\omega_{fi}\Delta t/2]
\end{equation}
当 $\Delta t$ 变大的时候, $\sinc^2$ 函数趋近于 $\delta$ 函数(\autoref{sinc_eq1}~\upref{sinc}).
\begin{equation}
\sinc^2[(\omega_{fi}\mp\omega)\Delta t/2] \to \frac{2\pi\hbar}{\Delta t}\delta(E_f - E_i \mp \omega\hbar)
\end{equation}
$E_i=E_f$ 附近的态能量密度为 $\rho(E_f)$, 那么总跃迁概率约等于
\begin{equation}
\Delta P = \rho(E_f)\int_{-\epsilon}^{\epsilon}\abs{c_i(t)}^2 \dd{E_f}
= \frac{2\pi}{\hbar} \abs{W_{fi}}^2\rho(E_f)\Delta t
\end{equation}
\textbf{跃迁率(transition rate)}, 即单位时间的概率为
\begin{equation}
\dv{P}{t} = \frac{2\pi}{\hbar} \abs{W_{fi}}^2\rho(E_f)
\end{equation}

\subsection{方形脉冲中的简谐微扰}
方形脉冲中的简谐微扰(\autoref{TDPEx_eq1}~\upref{TDPEx})
\begin{equation}
\abs{c_i(t)}^2 = \frac{\abs{W_{fi}}^2}{4\hbar^2} \Delta t^2 \{\sinc^2[(\omega_{fi}-\omega)\Delta t/2] + \sinc^2[(\omega_{fi}+\omega)\Delta t/2]\}
\end{equation}

若 $E_{f0} = E_i \pm \omega\hbar$ 附近的态能量密度为 $\rho(E_{f0})$, 那么总跃迁概率约等于
\begin{equation}
\Delta P = \rho(E_{f0})\int_{E_{f0}-\epsilon}^{E_{f0}+\epsilon}\abs{c_i(t)}^2 \dd{E_f}
= \frac{\pi}{2\hbar} \abs{W_{fi}}^2\rho(E_{f0})\Delta t
\end{equation}
每个方向的跃迁率为
\begin{equation}
\dv{P}{t} = \frac{\pi}{2\hbar} \abs{W_{fi}}^2\rho(E_{f})
\end{equation}
