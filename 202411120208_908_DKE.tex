% 勒内·笛卡尔(综述)
% license CCBYSA3
% type Wiki

本文根据 CC-BY-SA 协议转载翻译自维基百科\href{https://en.wikipedia.org/wiki/Ren\%C3\%A9_Descartes}{相关文章}。

勒内·笛卡尔(/deɪˈkɑːrt/ day-KART 或英国发音:/ˈdeɪkɑːrt/ DAY-kart;法语:[ʁəne dekaʁt] ⓘ;[注3][11] 1596年3月31日 – 1650年2月11日)[12][13]: 58  是法国哲学家、科学家和数学家,被广泛认为是现代哲学和科学兴起的奠基人物之一。数学在他的研究方法中至关重要,他将几何与代数相结合,创立了解析几何。笛卡尔的职业生涯大部分时间是在荷兰共和国度过的,最初在荷兰国军服役,后来成为荷兰黄金时代的核心知识分子。[14] 尽管他服务于一个新教国家,且后来被批评者视为自然神论者,笛卡尔实际上是罗马天主教徒。[15][16]

笛卡尔哲学的许多元素可以在晚期的亚里士多德主义、16世纪复兴的斯多葛主义或更早的哲学家如奥古斯丁的思想中找到前例。在他的自然哲学中,他在两个主要方面不同于当时的学派。首先,他拒绝将有形实质划分为质料和形式;其次,他拒绝在解释自然现象时诉诸于神或自然的终极目的。[17] 在神学中,他坚持神创造行为的绝对自由。笛卡尔拒绝接受前人哲学家的权威,常常将自己的观点与之前的哲学家区分开来。在《灵魂的激情》开篇中,这部早期现代情感论著中,笛卡尔甚至声称他将“仿佛从未有人写过这些问题一样”来论述该主题。他最著名的哲学陈述是“我思故我在”(拉丁语:cogito, ergo sum;法语:Je pense, donc je suis),出现在《方法谈》(1637年,以法语和拉丁语写成,1644年)和《哲学原理》(1644年拉丁语版,1647年法语版)中。[注4] 这一陈述要么被解释为逻辑三段论,要么被视为一种直觉思想。[18]