% JavaScript Async 笔记
% license Usr
% type Note

\pentry{JavaScript 入门笔记\upref{JS}}

\subsection{Promise}
\begin{lstlisting}[language=js]
console.log("Before Promise");

let myPromise = new Promise((resolve, reject) => {
    console.log("Inside Promise executor");
    resolve("Operation successful");
});

myPromise.then(
    (value) => { console.log(value); },  // Executed asynchronously
    (error) => { console.log(error); }
);

console.log("After Promise");
\end{lstlisting}
输出:
\begin{lstlisting}[language=none]
Before Promise
Inside Promise executor
After Promise
Operation successful
\end{lstlisting}

\begin{itemize}
\item \verb`Promise` 是内建类型, 生成时会马上执行第一个 arg 提供的函数(叫做 Promise executor), \verb`resolve, reject` 函数也同样由系统提供,用于承诺完成或出错时告知系统。 在这个例子中, 在生成器中, \verb`resolve()` 马上被调用了。
\item 在上例中,我们说 \verb`myPromise` 立即被 \verb`resolve` 成了一个字符串 \verb`"Operation successful"`。
\item \verb`resolve, reject` 也可以延迟调用, 例如:
\end{itemize}
\begin{lstlisting}[language=js]
let myPromise = new Promise((resolve, reject) => {
        setTimeout(() => {
            resolve('Completed!');
        }, milliseconds);
    });
\end{lstlisting}
\begin{itemize}
\item 在该例中, \verb`setTimeout` 马上会执行完毕(浏览器会使用另一个线程打开一个计时器), 然后 \verb`myPromise` 就处于一个没有 resolve 的状态。
\end{itemize}
\begin{itemize}
\item 现在来讨论 \verb`myPromise.then(handler1, handler2)`。 这个命令并不会立即执行 \verb`handler1, handler2`, 只是告诉 \verb`myPromise` 应该怎么处理 resolve 后的结果。 该命令后面的东西会马上继续运行直到结束。 注意即使 \verb`Promise executor` 执行完了, 也需要等到后面的所有代码运行结束了, 进入 event loop 才能执行 handler。
\item 所以无论 \verb`new Promise` 语句中,是否立即调用该了 \verb`resolve` 或 \verb`reject`, handler 都会在当前所有代码执行完以后,且承诺 resolve 了以后才执行。
\end{itemize}

\subsubsection{连续的 .then():}
\begin{itemize}
\item \verb`.then()` 方法返回什么呢? 返回一个新的承诺, resolve 到 \verb`handler` 返回的值! 于是这个新的 promise 等到完成后又可以被另一个 \verb`.then()` 处理。
\item 如果每个 \verb`.then()` 都没有 \verb`handler2`, 那么系统会提供默认的, 它的行为就是把错误信息传给新生成的 promise, 并立即调用 \verb`reject()` 把这个信息放进去。 如果最后yi'ge最后还可以加上 \verb`.catch()`, 因为如果一个 promise 出错了,
\end{itemize}
