% 准静态过程

\subsection{平衡态}

在不受外界影响的条件下,系统的宏观性质不随时间变化的状态,称为\textbf{平衡态(equilibrium state)}, 否则就是\textbf{非平衡态(nonequilibrium state)}.

具体来说,一定质量的气体在一容器内,如果它与外界没有交换能量,没有外场作用,内部也没有任何形式的能量转化,经过一段时间后,气体各部分终将达到相同的密度、相同的温度、压强等,所有的宏观性质都不随时间而变化,这种状态就是平衡态.

在实际情况中,并不存在完全不受外界影响、而且宏观性质绝对保持不变的系统,所以平衡态只是一个理想的模型,它是在一定条件下对实际情况的概括和抽象.

应当指出,平衡态是指系统的宏观性质不随时间变化;从微观方面看,组成系统的分子的热运动是永不停息的,通过分子的热运动的相互碰撞,其总效果在宏观上表现为不随时间变化的,所以平衡态实际上是\textbf{热动平衡状态(thermodynamical equilibrium state)}.

来思考这样一个例子:如果将一根金属棒的两端分别放在沸水和冰水混合物中,经过一段时间后,虽然棒上各处的温度不随时间变化,但这种状态仍不是平衡态,而是\textbf{定常态(steady state)},因为金属棒与外界有能量交换.

\subsection{准静态过程}

当气体的外界条件发生改变时,它的状态就会发生变化.气体从一个状态不断地变化到另一状态,所经历的是一个状态变化的过程.过程进展的速度可以很快,也可以很慢,实际过程通常比较复杂.如果过程进展得十分缓慢,使所经历的一系列中间状态都无限接近平衡状态,这个过程就叫做\textbf{准静态过程(quasi-static process)}或\textbf{平衡过程(equilibrium process)}.准静态过程就是实际过程无限缓慢进行时的极限情况.

显然,准静态过程是个理想的过程,它和实际过程毕竟是有差别的,但在许多情况下,可近似地把实际过程当作准静态过程处理,所以准静态过程是个很有用的理想模型.具体来说,一个系统如果最初处于非平衡态,经过一段时间过渡到一个平衡态,这一过渡时间叫\textbf{弛豫时间(relaxation time)}.在一个实际过程中,如果系统的状态发生一个可以被实验查知的微小变化所需的时间比弛豫时间长得多,那么在任何时刻进行观察时,系统都有充分的时间达到平衡态.这样的过程就可以当成准静态过程处理.

例如,原来气缸内处于平衡态的气体受到压缩后再达到平衡态所需要的时间,即弛豫时间,大约是$10^{-3}\mathrm s$或更小,如果在实验中压缩一次所用的时间是$1\mathrm s$,这时间是上述弛豫时间的$10^3$倍,气体这一压缩过程就可以认为是准静态过程.实际内燃机汽缸气体经历一次压缩时间大约是$10^{-2}\mathrm s$,这个时间也已是上述弛豫时间的$10$倍以上.从理论上对这种压缩过程作初步研究时,也把它当成准静态过程处理.
