% 正交函数系 2
% keys 正交;函数系;贝塞尔不等式
% license Xiao
% type Tutor
\pentry{定积分(简明微积分)\nref{nod_DefInt}}{nod_3bdc}


\footnote{$\Gamma$. M. 菲赫金哥尔茨。微积分学教程\;卷三[M].北京:高等教育出版社,2006:345-349}
\footnote{斯米尔诺夫。斯米尔诺夫高等数学卷四第一分册[M].哈尔滨:哈尔滨工业大学出版社,2018:6-12}
若函数系 
\begin{equation}\label{eq_OFS_1}
\qty{\varphi_n(x)}\quad(n\in \mathbb{N})~
\end{equation}
中各函数及其平方在区间 $\qty[a,b]$ 上皆可积分,且满足
\begin{equation}
\int_a^b\varphi_n(x)\varphi_m(x)\dd x=
\left\{
    \begin{aligned}
    &0\quad &(m\neq n)\\
   &\lambda_n>0\quad &(m=n)~,
    \end{aligned}
\right.
\end{equation}
则称函数系 $\qty{\varphi_n(x)}$ 为\textbf{正交函数系}。当 $\lambda_n=1(n\in\mathbb{N})$ 时,该函数系称为\textbf{正交规范系}(或\textbf{正交标准系})。显然,任意的正交函数系都可化为正交规范系 
\begin{equation}\label{eq_OFS_4}
\qty{\frac{\varphi_n(x)}{\sqrt{\lambda_n}}}\quad(n\in \mathbb{N})~.
\end{equation}

设 $f(x)$ 是任一实函数,在区间 $[a,b]$ 内是连续的,则数值
\begin{equation}
c_k=\frac{1}{\lambda_k}\int_a^b f(x)\varphi_k(x)\dd x\quad (k\in \mathbb{N})~
\end{equation}
称为函数 $f(x)$ 关于函数系\autoref{eq_OFS_1} 的\textbf{傅里叶系数}。由 $c_k$ 的定义,我们有等式
\begin{equation}\label{eq_OFS_2}
\int_a^b\qty[f(x)-\sum_{k=1}^nc_k\varphi_k(x)]^2\dd x=\int_a^b\qty[f(x)]^2\dd x-\sum_{k=1}^nc_k^2\lambda_k~.
\end{equation}

根据\autoref{eq_OFS_2},可得
\begin{equation}\label{eq_OFS_3}
\sum_{k=1}^\infty c_k^2\lambda_k\leq\int_a^b\qty[f(x)]^2\dd x~.
\end{equation}
对于正交规范系, $\lambda_k=1$,此时\autoref{eq_OFS_3} 便是所谓的\textbf{贝塞尔(bessel)不等式}. 

若对任何在区间 $[a,b]$ 定义的连续函数 $f(x)$,\autoref{eq_OFS_3} 中的等号成立,则称函数系 $\qty{\varphi_n(x)}$ 是\textbf{完整的}(或\textbf{完备的})。
%\begin{example}{三角函数系}
%%\begin{equation}
%\qty{1,\cos nx,\sin nx}\quad(n\in \mathbb{Z^{+}})
%%\end{equation}
%是正交函数系。
\subsection{例子}
\begin{enumerate}
\item 在区间 $[-\pi,\pi]$ 上,三角函数系
\begin{equation}
\qty{1,\cos nx,\sin nx}\quad(n\in \mathbb{Z^{+}})~
\end{equation}
是正交函数系。\\

\textbf{证明:}
由于
\begin{equation}
\begin{aligned}
\cos nx\cdot\cos mx&=\frac{1}{2}\qty[\cos(n+m)x+\cos(n-m)x],\\
\cos^2 nx&=\frac{1}{2}[\cos(2nx)+1],\\
\sin nx\cdot\sin mx&=\frac{1}{2}\qty[\cos(n-m)x-\cos(n+m)x],\\
\sin^2 nx&=\frac{1}{2}[1-\cos(2nx)],\\
\sin nx\cdot\cos mx&=\frac{1}{2}\qty[\sin(n+m)x+\sin(n-m)x],\\
\sin nx\cdot\cos nx&=\frac{1}{2}\sin(2nx).
\end{aligned}~
\end{equation}
所以当 $m\neq n$ 时,
\begin{equation}
\begin{aligned}
&\int_{-\pi}^\pi\cos nx \cdot \cos mx\dd x\\
&=\frac{1}{2}\qty[\frac{\sin(n+m)x}{n+m}+\frac{\sin(n-m)x}{n-m}]{\huge |}_{-\pi}^{\pi}\\
&=0,\\
&\int_{-\pi}^\pi\sin nx \cdot \sin mx\dd x\\
&=\frac{1}{2}\qty[\frac{\sin(n-m)x}{n-m}-\frac{\sin(n+m)x}{n+m}]{\huge |}_{-\pi}^{\pi}\\
&=0,\\
&\int_{-\pi}^\pi\sin nx \cdot \cos mx\dd x\\
&=-\frac{1}{2}\qty[\frac{\cos(n+m)x}{n+m}+\frac{\cos(n-m)x}{n-m}]{\huge |}_{-\pi}^{\pi}\\
&=0.\\
\end{aligned}
~
\end{equation}
而 $m=n$ 时,
\begin{equation}
\begin{aligned}
&\int_{-\pi}^\pi\cos^2 nx\dd x=\frac{1}{2}\qty[\frac{\sin(2nx)}{2n}+x]{\huge |}_{-\pi}^{\pi}=\pi,\\
&\int_{-\pi}^\pi\sin^2 nx\dd x=\frac{1}{2}\qty[x-\frac{\sin(2nx)}{2n}]{\huge |}_{-\pi}^{\pi}=\pi,\\
&\int_{-\pi}^\pi\sin nx \cdot \cos nx\dd x=-\frac{\cos(2nx)}{4n}{\huge |}_{-\pi}^{\pi}=0.
\end{aligned}
~
\end{equation}
此外 $\int_{\pi}^{\pi}\dd x=2\pi$。\\

\textbf{证毕!}\\
由\autoref{eq_OFS_4} 和证明可看出,三角正交函数系 $\{1,\cos(nx),\sin(nx)|n\in \mathbb Z^{+}\}$ 的标准系是 
\begin{equation}
\qty{\frac{1}{\sqrt{2\pi}},\frac{\cos(nx)}{\sqrt{\pi}},\frac{\sin(nx)}{\sqrt{\pi}}{\huge|}n\in \mathbb Z^{+}}~.
\end{equation}




\item 由超越方程
\begin{equation}
\tan\xi=c\xi \quad(c\; \mathrm{is\; a\; constant})~
\end{equation}
 的所有正根组成的集合 $\qty{\xi_n\vert n\in\mathbb{Z}^+}$
 构造的函数系
 \begin{equation}
 \sin\frac{\xi_1}{l}x,\ \sin\frac{\xi_2}{l}x,\ \cdots,\ \sin\frac{\xi_n}{l}x,\ \cdots~
 \end{equation}
 在区间 $\qty[0,l]$ 上是正交函数系。\\
 
 同样,以超越方程
\begin{equation}
\cot\xi=c\xi \quad(c\; \mathrm{is\; a\; constant})~
\end{equation}
 的所有正根组成的集合 $\qty{\xi'_n\vert n\in\mathbb{Z}^+}$
 构造的函数系
 \begin{equation}
 \cos\frac{\xi'_1}{l}x,\ \cos\frac{\xi'_2}{l}x,\ \cdots,\ \cos\frac{\xi'_n}{l}x,\ \cdots~
 \end{equation}
 在区间 $\qty[0,l]$ 上也是正交函数系。
 \item 勒让德多项式
\begin{equation}
X_0(x)=1,X_n(x)=\frac{1}{2^nn!}\dv[n]{\qty(x^2-1)^n}{x}\qquad(n=1,2,\cdots)~
\end{equation}
是在区间 $\qty[-1,1]$ 是正交函数系,且 $\lambda_n=\frac{2}{2n+1}$
\item 以贝塞尔函数 $J_0(x)$ 的所有正根构成的集合 $\qty{\xi_n\vert n\in\mathbb{Z}^+}$
构造的函数系 $\qty{\sqrt{x}J_0(\xi_n x)} $ 在区间 $\qty[0,1]$ 上是正交函数系。
\end{enumerate}

\subsection{线性无关函数的正交化}
在线性代数里面,通过斯密特正交化(\autoref{sub_OrNrB_1} )手续,可由 $N$ 个线性无关的向量构造出同样多个两两正交且标准的向量,使原来的向量可由新向量线性表出。这一切对函数来说完全适用。

$N$ 个在区间 $[a,b]$ 上定义的连续函数 $\psi_i(x)\quad (i=1,\cdots ,N)$ 称为在区间 $[a,b]$ 线性无关的,若含常系数 $a_i$ 的关系式
\begin{equation}
\sum_i^N a_i\psi_i(x)=0~,
\end{equation}
仅当 $a_i=0$ 时成立。现在来作在区间 $[a,b]$ 上可由 ${\psi_i(x)}$ 线性表示的正交标准化的函数 ${\varphi_i(x)}$。

记(显然,这里 $\ev{*,*}$ 相当于矢量空间中的内积)
\begin{equation}
\ev{f,F}=\int_a^b f(x)F(x)\dd x~.
\end{equation}

函数 ${\psi_i(x)}$ 的正交化过程可按如下方式进行(为简洁起见,将函数 $f(x)$ 简记为 $f$)
\begin{equation}
\begin{aligned}
&&\varphi_1=\frac{\psi_1}{\sqrt{\ev{\psi_1,\psi_1}}}\\
\chi_2&=\psi_2-\ev{\varphi_1,\psi_2}\varphi_1,\quad&\varphi_2=\frac{\chi_2}{\sqrt{\ev{\chi_2,\chi_2}}}\\
\chi_3&=\psi_3-\ev{\varphi_1,\psi_3}\varphi_1-\ev{\varphi_2,\psi_3}\varphi_2,\quad &\varphi_3=\frac{\chi_3}{\sqrt{\ev{\chi_3,\chi_3}}}\\
&\vdots &\vdots
\\
\chi_N&=\psi_N-\sum_{i=1}^N\ev{\varphi_i,\psi_N}\varphi_i,\quad&\varphi_N=\frac{\chi_N}{\sqrt{\ev{\chi_N,\chi_N}}}~.
\end{aligned}
\end{equation}

在函数 $\chi_i$ 中,没有一个可变为恒等于0,即 $\ev{\chi_i,\chi_i}\neq 0$。因为比如说 $\chi_2=0$,则 $\varphi_1$ 与 $\psi_2$ 线性相关
\begin{equation}
\psi_2-\ev{\varphi_1,\psi_2}\varphi_1=0~.
\end{equation}
这归结为 $\psi_1$ 与 $\psi_2$ 之间线性相关,这与条件 $\psi_1$ 和 $\psi_2$ 的线性无关矛盾。
\subsection{推广到复函数情形}
上面关于实函数的一切,可立即推广到实变量 $x$ 的复函数情形
\begin{equation}
\varphi_n(x)=\rho_n(x)+\I \sigma_n(x)\quad(n\in \mathbb{Z}^+)~.
\end{equation}

这时函数系的正交标准性由下式表达
\begin{equation}
\ev{\varphi_m,\varphi_n}=\int_a^b\varphi_m^*\varphi_n\dd x=\left\{
\begin{aligned}
&0 \quad (p\neq q)\\
&1 \quad (p=q)~.
\end{aligned}\right.
\end{equation}
复函数 $\varphi_n(x)$ 关于函数 $f(x)$ 的傅里叶系数为
\begin{equation}
c_n=\ev{\varphi_n,f}~.
\end{equation}
我们处处以模平方代替原值平方,\autoref{eq_OFS_2} 将是
\begin{equation}
\int_a^b\abs{f-\sum_{k=1}^nc_k\varphi_k}^2\dd x=\int_a^b\abs{f}^2\dd x-\sum_{k=1}^n\abs{c_k}^2~,
\end{equation}
 而贝塞尔不等式将是
 \begin{equation}
 \sum_{k=1}^\infty\abs{c_k}^2\leq\int_a^b\abs{f}^2\dd x~.
 \end{equation}
 
 \begin{example}{}\label{ex_OFS_1}
 $\{e^{\I nx}|n\in\mathbb Z\}$ 是 $[-\pi,\pi]$ 上的正交函数系。\\
 
\textbf{证明:}
\begin{equation}
\begin{aligned}
\int_{-\pi}^\pi e^{\I nx}e^{-\I mx}\dd x&=\int_{-\pi}^\pi e^{\I (n-m)x}\dd x\\&=-\I\frac{e^{\I(n-m)x}}{n-m}{\huge|}_{-\pi}^{\pi}\\
&=0,\quad n\neq m,\\
\int_{-\pi}^\pi e^{\I nx}e^{-\I nx}\dd x&=\int_{-\pi}^\pi \dd x=2\pi.
\end{aligned}~
\end{equation}

\textbf{证毕!}

由证明还可看出,$\qty{\frac{e^{\I nx}}{\sqrt{2\pi}}{\huge|}n\in\mathbb Z}$ 是它的正交标准系。
 \end{example}
 \begin{example}{}\label{ex_OFS_2}
 \autoref{ex_OFS_1} 的函数系是在 $[-\pi,\pi]$ 上的,它刚好是该函数系的函数 $e^{\I x}$ 的最小周期。然而更常见的周期往往不被限定为 $2\pi$,为此,我们希望能将它推广到周期为 $2L$ 的情形,从而积分区间为 $[-L,L]$。若将函数的定义区间 $[-L,L]$ 变到区间 $[-\pi,\pi]$,那么就可以继续使用前面的函数系。设 $f(x)$ 是定义在区间 $[-L,L]$ 上的周期为 $2L$ 的函数,那么 $F(x)=f(\frac{L}{\pi}x)$ 是定义在区间 $[-\pi,\pi]$ 上的周期函数。因此可对 $F(x)$ 使用函数系 $\{e^{\I nx}|n\in\mathbb Z\}$ 展开。若得到了 $F(x)$ 的展开式,那么 $f(x)$ 就可以利用 $f(x)=F(\frac{\pi}{L}x)$ 确定。设 
 \begin{equation}
 F(x)=\sum_{n=-\infty}^\infty c_ne^{\I n x},~
 \end{equation}
 那么
 \begin{equation}\label{eq_OFS_5}
 f(x)=\sum_{n=-\infty}^\infty c_ne^{\I n\frac{\pi}{L} x}.~
 \end{equation}
 因此,我们希望证明 $\{e^{\I n\frac{\pi}{L} x}|n\in\mathbb Z\}$ 是 $[-L.L]$ 上的正交函数系。\\
\textbf{证明:}
\begin{equation}
\begin{aligned}
\int_{-L}^L e^{\I n\frac{\pi}{L}x}e^{-\I m\frac{\pi}{L}x}\dd x&=\int_{-L}^L e^{\I (n-m)\frac{\pi}{L}x}\dd x\\
&=-\I\frac{Le^{\I(n-m)\frac{\pi}{L}x}}{(n-m)\pi}{\huge|}_{-L}^{L}\\
&=0,\quad n\neq m,\\
\int_{-L}^L e^{\I n\frac{\pi}{L}x}e^{-\I n\frac{\pi}{L}x}\dd x&=\int_{-L}^L \dd x=2L.
\end{aligned}~
\end{equation}
\textbf{证毕!}

同样的,这表明 $\{\frac{1}{\sqrt{2L}}e^{\I n\frac{\pi}{L} x}|n\in\mathbb Z\}$ 是 $[-L,L]$ 上的标准正交函数系。
 \end{example}

\begin{example}{傅里叶变换}
我们考察\autoref{ex_OFS_2} 中 $L\rightarrow\infty$ 的展开式\autoref{eq_OFS_5} 的形式。注意到
\[
  \sum_{n=-\infty}^\infty c_ne^{\I n\frac{\pi}{L} x}~  
\]
中若令 $z_n=n\frac{\pi}{L}$,那么 $\Delta z_n:=z_{n+1}-z_n=\frac{\pi}{L}$。并注意 $c_n=\frac{1}{2L}\ev{e^{\I z_n x},f}$ 因此\autoref{eq_OFS_5} 可写为
\begin{equation}
f(x)&=\sum_{n=-\infty}^\infty \frac{Lc_n}{\pi}e^{\I z_n x}\Delta z_n\\
&=.~ 
\end{equation}
这恰好就是定义在区间 $[-\infty,\infty]$ 上的 $z$ 函数 $f(z)=\frac{Lc_n}{\pi}e^{\I z_n x}$ 积分
\begin{equation}

\end{equation}

\end{example}