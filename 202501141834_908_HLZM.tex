% 花拉子米(综述)
% license CCBYSA3
% type Wiki

本文根据 CC-BY-SA 协议转载翻译自维基百科\href{https://en.wikipedia.org/wiki/Al-Khwarizmi}{相关文章}。
\begin{figure}[ht]
\centering
\includegraphics[width=6cm]{./figures/6cab70e0324244c5.png}
\caption{20世纪木刻版画,描绘阿尔-花拉子米} \label{fig_HLZM_1}
\end{figure}
穆罕默德·伊本·穆萨·阿尔-花拉子米(约780年–约850年),或简称阿尔-花拉子米,是一位波斯学者,致力于数学、天文学和地理学的研究,并在阿拉伯语学术领域产生了深远的影响。约在公元820年,他在巴格达的智慧之家工作,该地是阿拔斯王朝的首都。

他在代数领域的代表性著作《阿尔-贾布尔》(《计算的简明书》)编写于813年至833年之间,提出了线性方程和二次方程的系统解法。他在代数方面的成就之一是通过完成平方法来解二次方程,并为此提供了几何证明。由于阿尔-花拉子米是第一个将代数作为独立学科来研究的人,并且引入了“简化”和“平衡”方法(将减去的项转移到方程的另一边,即在方程两边相同项的消去),因此他被誉为代数的奠基人或创始人。英语中的“algebra”一词就来源于他那本著作的简写标题(阿尔-贾布尔,意为“完成”或“重新联合”)。他的名字还衍生出了英语中的“algorism”和“algorithm”以及西班牙语、意大利语和葡萄牙语中的“algoritmo”;以及西班牙语中的“guarismo”和葡萄牙语中的“algarismo”,均表示“数字”。

12世纪时,阿尔-花拉子米关于印度算术的教科书《印度人数字算法》(Algorithmo de Numero Indorum)的拉丁文翻译,规范了印度数字系统,并将基于十进制的位值计数系统引入西方世界。同样,阿尔-贾布尔由英学者罗伯特·查斯特在1145年翻译成拉丁文,并一直作为欧洲大学的主要数学教材,直到16世纪。

阿尔-花拉子米修订了公元2世纪由罗马学者克劳狄乌斯·托勒密撰写的《地理学》,列出了城市和地方的经纬度。他还制作了一套天文表格,并撰写了关于历法的作品,以及关于天文仪器和日晷的研究。阿尔-花拉子米在三角学方面也作出了重要贡献,编制了准确的正弦和余弦表,并制作了第一个正切表。