% 热力学第一定律
% keys 热力学第一定律|能量守恒|做功|传热|内能

\begin{issues}
\issueDraft
\end{issues}

\pentry{压力体积图\upref{PVgraf}, 理想气体内能\upref{IdgEng}}

热力学第一定律是能量守恒在热力学中的形式,外部对系统传递的热量等于系统对外做功加上系统的内能增加:
\begin{equation}\label{Th1Law_eq1}
\Delta Q = W + \Delta E
\end{equation}
我们称压强 $p$ 和体积 $V$ 为系统的力学参量,其中 $p$ 为强度量,$V$ 为广延量.$E$ 为系统的内能,有时也用字母 $U$ 表示.


在力学相互作用过程中,系统和外界之间转移的能量就是功.对于容器中的气体,设其压强为 $p$,体积为 $V$,那么对外做功可以写成积分形式:
\begin{equation}
W = \int p \dd{V}
\end{equation}
热力学第一定律写成微分形式是
\begin{equation}
\dd E=\delta Q-\delta W=\delta Q-p\dd V
\end{equation}
$Q$ 和 $W$ 前用的是 $\delta$ 符号而不是全微分符号,是因为 $Q$ 和 $W$ 和系统变化的过程本身有关.$E$ 前面用的是全微分符号,是因为 $E$ 本身代表气体系统的内能函数,是一个态函数,而 $\Delta E$ 只与初始和最终的系统状态有关.$E$ 是系统状态的函数,我们称它为 \textbf{态函数},而 $W$ 和 $Q$ 并不是系统状态的函数,它们用来描述在一个系统变化过程中功和热量的传递,是一个 \textbf{过程量}\upref{StaPro}.

虽然,$\delta W$ 是过程量,但 $\delta W/p=\dd V$ 是全微分($V$ 是态函数).$\delta Q/T=\dd S$ 也是全微分,其中 $S$ 为热力学熵\upref{Entrop}.

\subsection{内能和态函数}
我们可以用几个宏观的热力学参量来完整地刻画一个热力学平衡系统.例如,对于一个无外场的孤立气体系统,压强 $p$ 和温度 $T$ 足以刻画这个气体系统的一切宏观特征.对于理想气体,有状态方程 $pV=nRT$,压强 $p$ 和温度 $T$ 足以描绘整个理想气体系统.

如果某个函数只和系统的热力学参量有关,也就是只和系统状态有关,我们称它为\textbf{态函数}.热力学研究的就是热力学系统的态函数之间的关系.

对理想气体\upref{Igas}, 令分子自由度为 $i$, 有
\begin{equation}
E = \frac{i}{2}n RT
\end{equation}

\addTODO{需要加一个范德瓦尔斯气体的词条,作为经典例子}