% 初等矩阵与初等变化
%初等矩阵 初等变换 

\begin{issues}
\issueMissDepend
\end{issues}

\footnote{本文参考了Steven J. Leon的 Linear Algebra with Applications与MIT的《线性代数》课程.本文适用于CC-BY-SA.}

\subsection{初等变换 Elementary Operations}
初等变换包括三种类型,以下分别举例说明:

\begin{enumerate}
\item 交换两行(列)

\begin{equation}
\left[
    \begin{array}{ccc}
        a_{11} & a_{12} & a_{13}\\
        a_{21} & a_{22} & a_{23}\\
        a_{31} & a_{32} & a_{33}\\
    \end{array}
\right]
\Rightarrow
\left[
    \begin{array}{ccc}
        a_{11} & a_{12} & a_{13}\\
        a_{31} & a_{32} & a_{33}\\
        a_{21} & a_{22} & a_{23}\\
    \end{array}
\right]
\end{equation}

\item 一行(列)扩大非零常数倍数
\begin{equation}
\left[
    \begin{array}{ccc}
        a_{11} & a_{12} & a_{13}\\
        a_{21} & a_{22} & a_{23}\\
        a_{31} & a_{32} & a_{33}\\
    \end{array}
\right]
\Rightarrow
\left[
    \begin{array}{ccc}
        a_{11} & a_{12} & a_{13}\\
        ca_{21} & ca_{22} & ca_{23}\\
        a_{31} & a_{32} & a_{33}\\
    \end{array}
\right]
\end{equation}
\item 一行(列)加上另一行(列)的常数倍数
\begin{equation}
\left[
    \begin{array}{ccc}
        a_{11} & a_{12} & a_{13}\\
        a_{21} & a_{22} & a_{23}\\
        a_{31} & a_{32} & a_{33}\\
    \end{array}
\right]
\Rightarrow
\left[
    \begin{array}{ccc}
        a_{11} & a_{12} & a_{13}\\
        a_{21}+ca_{31} & a_{21}+ca_{32} & a_{21}+ca_{33}\\
        a_{31} & a_{32} & a_{33}\\
    \end{array}
\right]
\end{equation}
\end{enumerate}

\subsection{初等矩阵 Elementary Matrices}
初等矩阵是单位矩阵$\mat I$只经一次初等变化得到的矩阵,也包括三种类型,常记作 $\mat E$.以下分别举例说明.
\begin{table}[ht]
\centering
\caption{初等矩阵}\label{EleOpt_tab1}
\begin{tabular}{|c|c|c|c|}
\hline
种类 & 效果 & 形式 & 逆矩阵\upref{InvMat} & 行列式\upref{Deter} \\
\hline
1 & 交换两行(列) & 
交换第2,3行
$
\mat E_1=
    \begin{bmatrix}
        1 & 0 & 0\\
        0 & 0 & 1\\
        0 & 1 & 0\\
    \end{bmatrix}
$
& 
$
    \mat E_1^{-1}=\mat E_1=
    \begin{bmatrix}
        1 & 0 & 0\\
        0 & 0 & 1\\
        0 & 1 & 0\\
    \end{bmatrix}
$
& $det (\mat E_1) = -1$ \\
\hline
2 & 一行(列)扩大非零常数倍数 & 
第2行扩大c倍
$
    \mat E_2= \begin{bmatrix}
        1 & 0 & 0\\
        0 & c & 0\\
        0 & 0 & 1\\
    \end{bmatrix}
$
& 
$
    \mat E_2^{-1}= \begin{bmatrix}
        1 & 0 & 0\\
        0 & \frac{1}{c} & 0\\
        0 & 0 & 1\\
    \end{bmatrix}
$
& 
$det (\mat E_2) = c$
 \\
\hline
3 & 一行(列)加上另一行(列)的常数倍数 & 
第2行加上第3行的c倍
$
    \mat E_3 = \begin{bmatrix}
        1 & 0 & 0\\
        0 & 1 & c\\
        0 & 0 & 1\\
    \end{bmatrix}
$
& 
$
    \mat E_3^{-1} = \begin{bmatrix}
        1 & 0 & 0\\
        0 & 1 & -c\\
        0 & 0 & 1\\
    \end{bmatrix}
$
& $det (\mat E_3) = 1$\\
\hline
\end{tabular}
\end{table}
初等矩阵的逆矩阵仍是原类型的初等矩阵.

\begin{example}{}
如何理解初等矩阵的逆矩阵?以交换两行的初等矩阵为例.

根据逆矩阵的定义,$\mat E_1^{-1} \mat E_1 = \mat I$.$\mat E_1$的效果是"交换第2,3行",那么左乘的$\mat E_1^{-1}$的效果也应该是再次"交换第2,3行"使矩阵恢复$\mat I$.可见,$\mat E_1^{-1} =  \mat E_1$

同理,可理解剩余两种变换.
\end{example}

\subsection{初等变化与初等矩阵}
\pentry{矩阵乘法\upref{Mat}}
\begin{theorem}{}
对矩阵进行一次初等行操作⇔相应的初等矩阵E*矩阵 (左乘)

对矩阵进行一次初等列操作⇔矩阵*相应的初等矩阵E(右乘)
\end{theorem}

\begin{example}{}
例如
\begin{equation}
\begin{bmatrix}
    a_{11} & a_{12} & a_{13}\\
    a_{31} & a_{32} & a_{33}\\
    a_{21} & a_{22} & a_{23}\\
\end{bmatrix}
=
    \begin{bmatrix}
        1 & 0 & 0\\
        0 & 0 & 1\\
        0 & 1 & 0\\
    \end{bmatrix}
\begin{bmatrix}
        a_{11} & a_{12} & a_{13}\\
        a_{21} & a_{22} & a_{23}\\
        a_{31} & a_{32} & a_{33}\\
\end{bmatrix}
\end{equation}
\end{example}

运用初等变换的性质,可以很容易地理解一些结论.

\begin{example}{}
例如行列式的性质\upref{DetPro}的\autoref{DetPro_the6}~\upref{DetPro}“将行列式的两列交换,结果取相反数”: 交换两列,相当于对相应矩阵做一次初等列变换.那么自然有,
$$det (\mat A \mat E_1)=det (\mat A) det(\mat E_1)=-det (\mat A)$$
\end{example}
% $$
%     \begin{bmatrix}
%         1 & 0 & 0\\
%         0 & 1 & 0\\
%         0 & 0 & 1\\
%     \end{bmatrix}
% $$

\subsection{行等价 Row Equivalent}
\begin{definition}{行等价}
若$\mat A$经过有限次初等行变换即可得到$\mat B$,则称$\mat A$与$\mat B$行等价.
\begin{equation}
\mat A= ... \mat E_3 \mat E_2 \mat E_1 \mat  B
\end{equation}
\end{definition}

可以证明,所有可逆矩阵\upref{InvMat}都行等价于单位矩阵 $\mat I$.

可以用该思路来理解 高斯消元法求逆矩阵\upref{InvMGs}.
