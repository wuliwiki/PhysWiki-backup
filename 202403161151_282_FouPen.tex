% 傅科摆(科普)
% license Usr
% type Art

\begin{issues}
\issueDraft
\end{issues}

傅科摆科普视频的思路
\begin{itemize}
\item GPT 生成一些历史
\item 如果在北极
\item 如果在赤道
\item 如果在某个纬线的角速度(引出蚂蚁转弯问题,参考知乎)
\item 举例转椅,用多边形的极限来解释转动的角度,如果在北极附近的一个小多边形,每推过一条边,换一个方向继续推。其实本质上是球面上多边形的问题。
\end{itemize}

19 世纪中叶,人们对地球的自转有了初步的理解,但直接证明地球自转仍然是一个挑战。 1851年,莱昂·傅科在巴黎的巴拿赛神殿中挂起了一根长达 67 米的钢丝,其末端悬挂着一个重达 28 公斤的铅球。 当他让铅球沿一个初始方向摆动时,摆动的平面逐渐旋转,从而直接展示了地球自转的影响。这个简单而又直观的实验让傅科摆成为了地球物理学和天文学领域的一个重要里程碑。

傅科摆为什么能说明地球在自传? 最简单的方法是设想我们把傅科摆放在南极或北极, 摆动的平面会如何变化? 此时傅科摆的摆动平面将会在 24 小时内完成一整圈的旋转, 直接反映了地球自转的周期。 这是因为在北极(或南极),水平面与傅科摆的摆动平面垂直。

如果我们将傅科摆放置在赤道上,情况就完全不同了。 无论开始时摆动沿哪个方向,傅科摆的摆动平面实际上不会出现旋转现象。 这也是容易理解的,最简单的两种情况是南北和东西方向的摆动。

傅科摆最有趣的情况在于,如果我们把它放在介于南北极和赤道之间的某个纬度,它摆动方向相对于地面的转动是多块呢?不难猜测,每天的转动介于赤道和极点之间,也就是从几乎不转动到每天转动一圈。 巧妙的是,它每天转动的圈数恰好是维度的正弦值 $\sin\theta$(这其实假设了地球是完美的均匀球体,实际情况和该公式存在微小的偏差)。该公式给出的定量转速既符合力学理论,也符合实验结果,这更强有力说明地球是一个自转的近似球体。 因为如果只是观察到转动,有人可能会认为是

在以上的讨论中,我们要区分两种转动,一种是单摆的位置沿地球纬线划出的圆周运动,在一天当中,该圆周运动必定会完成一整圈的转动,这转速和维度无关。另一种转动是单摆的方向和水平地面之间的相对转动,这才是我们讨论傅科摆时关心的。

傅科摆的摆动平面都会以某种特定的速率旋转,这个速率取决于其所在的纬度。这引出了一个有趣的问题:如果一只蚂蚁在某个纬度线上沿圆形路径行走,是否需要调整其行进方向以保持在这个纬线上?这个问题实际上与傅科摆展示的现象有着直接的关联,它涉及到在不同纬度上由于地球自转引起的相对运动变化。

举例转椅和多边形极限
为了更深入地理解傅科摆的原理,我们可以使用一个转椅作为类比。假设你坐在转椅上,手中持有一个向下挂着重物的绳子。当转椅旋转时,你会观察到重物相对于地面的摆动方向发生变化,这类似于
