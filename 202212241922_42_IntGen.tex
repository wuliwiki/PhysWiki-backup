% 一般积分
% 一般积分|常微分方程
\pentry{基本知识(常微分方程)\upref{ODEPr}}
下面将介绍常微分方程的一般积分,它是常微分方程解的一般形式。物理学中“运动积分”(或“运动方程的积分”)(运动积分\upref{motint})和“循环积分”中的“积分”之所以叫“积分”,就是因为它对应的数学对象正是本词条要介绍的“一般积分”。

“积分”的概念源于微积分,“积分”和“导数”彼此对应:“你”是“我”的导数,“我”就是“你”的积分,反之亦然。微分方程是关于未知函数导数的方程,其主要目的是要求得该未知函数,满足微分方程的未知函数称为该方程的解,所以微分方程的解当然就是微分方程中对应的导数的积分。由于这样的理由,人们也把微分方程的解叫作微分方程的\textbf{积分}。
\subsection{$n$ 阶常微分方程的一般积分}
$n$ 阶常微分方程一般形式为(\autoref{ODEPr_eq6}~\upref{ODEPr}):
\begin{equation}\label{IntGen_eq1}
F \qty(x,y,y',\cdots,y^{(n)}) =0
\end{equation}
或写成(\autoref{GO2SOD_eq2}~\upref{GO2SOD})
\begin{equation}\label{IntGen_eq2}
y^{(n)}=f(x,y,y',\cdots,y^{(n-1)})
\end{equation}
自变量 $x$ 的任何函数,若满足方程\autoref{IntGen_eq1} 或\autoref{IntGen_eq2} ,就叫作这方程的\textbf{解};而微分方程的求解问题也叫作求微分方程的\textbf{积分问题}。

对于 $n$ 阶微分方程\autoref{IntGen_eq1} 或\autoref{IntGen_eq2} ,有存在与唯一定理,它可以叙述为:若函数 $f(x,y,y',\cdots,y^{(n-1)})$ 是 $(x,y,y',\cdots,y^{(n-1)})$ 的单值函数,且在 $(x_0,y_0,y_0',\cdots,y_0^{(n-1)})$ 的一邻域内,$f$ 连续且有对 $y,y',\cdots,y^{(n-1)}$ 的一阶连续偏微商,则满足初始条件
\begin{equation}
y|_{x=x_0}=y_0,y'|_{x=x_0}=y_0',\cdots,y^{(n-1)}|_{x=x_0}=y_0^{(n-1)}
\end{equation}
的解存在且唯一。
\subsection{常微分方程组的一般积分}

