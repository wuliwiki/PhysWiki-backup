% 2012 年考研数学试题(数学一)
% keys 考研|数学
% license Copy
% type Tutor
\subsection{选择题}
\begin{enumerate}
\item 曲线 $\displaystyle y=\frac{x^2+x}{x^2-1}$ 的渐近线的条数为 ($\quad$)\\
(A) $0$\\
(B) $1$\\
(C) $2$\\
(D) $3$
\item 设函数 $f(x)=(e^x-1)(e^{2x}-2)\dots(e^{nx}-n)$ ,其中 $n$ 为正整数,则 $f'(0)$=($\quad$)\\
(A)$(-1)^{n-1}(n-1)!$\\
(B)$(-1)^n(n-1)!$\\
(C)$(-1)^{n-1}n!$\\
(D) $(-1)^n n!$
\item 如果函数 $f(x,y)$ 在点$ (0,0) $处连续,那么下列命题正的是($\quad$)\\
(A)若极限 $\displaystyle \lim_{\substack {x\to0 \\ y\to 0}}\frac{f(x,y)}{\abs{x}+\abs{y}}$ 存在,则 $f(x,y)$ 在点  $(0,0)$ 处可微。\\
(B)若极限 $\displaystyle \lim_{\substack {x\to0 \\ y\to 0}}\frac{f(x,y)}{x^2+y^2}$ 存在,则 $f(x,y)$ 在点  $(0,0)$ 处可微。\\
(C)若 $f(x,y)$ 在点  $(0,0)$ 处可微,则极限 $\displaystyle \lim_{\substack {x\to0 \\ y\to 0}}\frac{f(x,y)}{\abs{x}+\abs{y}}$ 存在,\\
(D)若 $f(x,y)$ 在点  $(0,0)$ 处可微,则极限 $\displaystyle \lim_{\substack {x\to0 \\ y\to 0}}\frac{f(x,y)}{x^2+y^2}$ 存在
\item 设 $\displaystyle I_k=\int_{0}^{k\pi} e^{x^2}\sin x\dd{x}(k=1,2,3) $ ,则有($\quad$)\\
(A)$I_1<I_2<I_3$\\
(B)$I_3<I_2<I_1$\\
(C)$I_2<I_3<I_1$\\
(D)$I_2<I_1<I_3$
\item  设 $\mat \alpha_1=\pmat{0\\0\\c_1},\mat \alpha_2=\pmat{0\\1\\c_2},\mat \alpha_3=\pmat{1\\-1\\c_3},\mat \alpha_1=\pmat{-1\\1\\c_4}$ ,其中 $c_1,c_2,c_3,c_4$ 为任意常数,则下列向量组线性相关的为 ($\quad$)\\
(A)$\mat{ \alpha_1,\alpha_2,\alpha_3}$\\
(B)$\mat{ \alpha_1,\alpha_2,\alpha_4}$\\
(C)$\mat{ \alpha_1,\alpha_3,\alpha_4}$\\
(D)$\mat{ \alpha_2,\alpha_3,\alpha_4}$
\item 设 $\mat A$  为3阶矩阵, $\mat P$ 为3阶可逆矩阵,且 $\mat{P^{-1}AP}=\pmat{1&0&0\\0&1&0\\0&0&2}$  。若 $\mat{p=(\alpha_1,\alpha_2,\alpha_3)},\mat{Q=(\alpha_1+\alpha_2,\alpha_2,\alpha_3)}$,则 $\mat {Q^{-1}AQ}$ =($\quad$)\\
(A)$\pmat{1&0&0\\0&2&0\\0&0&1}$\\ \\
(B)$\pmat{1&0&0\\0&1&0\\0&0&2}$\\ \\
(C)$\pmat{2&0&0\\0&1&0\\0&0&2}$\\ \\
(D)$\pmat{2&0&0\\0&2&0\\0&0&1}$\\
\item 设随机变量 $X$ 与 $Y$ 相互独立,且分别服从参数为1与参数为4的指数分布,则  $P\{X<Y\}$=($\quad$)\\
(A) $\displaystyle \frac{1}{5}$\\
(B) $ \displaystyle \frac{1}{3}$\\
(C) $\displaystyle  \frac{2}{3}$\\
(D) $\displaystyle \frac{4}{5}$
\item  将长度为 $1m$ 的木棒随机地截成两段,则两段长度关系系数为 ($\quad$)\\
(A) $1$\\
(B) $\displaystyle \frac{1}{2}$\\
(C) $\displaystyle -\frac{1}{2}$\\
(D) $-1$
\end{enumerate}
\subsection{填空题}
\begin{enumerate}
\item 若函数 $f(x)$  满足方程 $f''(x)+f'(x)-2f(x)=0$及$f''(x)+f(x)=2e^x$ ,则$f(x)$=($\quad$)
\item $\displaystyle \int_{0}^{2} x\sqrt{2x-x^2}$=($\quad$)
\item $\displaystyle \eval{ grad(xy+\frac{z}{y})}_{(2,1,1)}$=($\quad$)
\item  设 $\Sigma=\{(x,y,z)|x+y+z=1,x\ge 0,y \ge 0,z \ge 0\}$  ,则$\displaystyle \int \int_\Sigma y^2\dd{S}$=($\quad$)
\item 设 $\mat \alpha$ 为3维单位列向量, $\mat E$ 为3阶单位矩阵,则矩阵 $\mat {E-\alpha\alpha \Tr}$ 的秩为($\quad$)
\item 设 $A,B,C$ 是随机事件, $A$ 与 $c$ 互不相容, $P(AB)=\frac{1}{2},(C)$  则
\end{enumerate}
