% gongshi

\section{Problem 2}
\subsection{A}
Prooving Mercer's Theorem.\\
For any valid kernel we have that the function has a feature map $\phi$.
So kernel matrix K is equal to $\phi\big( \overrightarrow{x_i}\big) \cdot \phi\big( \overrightarrow{x_j}\big)$.
In order to be a $n\times n$ array the $\phi\big( \overrightarrow{x_i}\big)$ will be a column vector.
So K=$V^TV$. From the tranverse matrix definition we know that if $V$ has real values then $V^TV$ is 
PSD. As $V$ has real values and also all $c \in \mathbb{R}^n$ then $c^TKc \geq 0$.

\begin{enumerate}[label=\alph*]
\item ) $k_1$ has a feature map $\Phi_1$ and an inner product $\langle\rangle H_{k_1}$.\\
So $ak_1(x,\widetilde{x}) = \langle \sqrt{a}\Phi_1{x}, \sqrt{a}\Phi_1{\widetilde{x}} \rangle H_{k_1}$. \\
The same can be done for $bk_2(x,\widetilde{x})$.\\
We have at the end 
$\langle \sqrt{a}\Phi_1{x}, \sqrt{a}\Phi_1{\widetilde{x}} \rangle H_{k_1} + \langle \sqrt{b}\Phi_2{x}, \sqrt{b}\Phi_2{\widetilde{x}} \rangle H_{k_2}$\\
$= \langle [\sqrt{a}\Phi_1{x},\sqrt{b}\Phi_2{x}], [\sqrt{a}\Phi_1{\widetilde{x}},\sqrt{b}\Phi_2{\widetilde{x}}] \rangle$
Where this on its own is a inner product of a new kernel.
\item ) As the already valid kernels are equal to their respective feature functions.
$k_1(x,\widetilde{x}) = a(x)^T*a(\widetilde{x})\\
k_2(x,\widetilde{x}) = b(x)^T*b(\widetilde{x})$
As this 2 functions have n,m size as arrays respectively the whole system can be written as
\begin{align*}
k(x,\widetilde{x}) &= k_1(x,\widetilde{x})*k_2(x,\widetilde{x})\\
		   &= a(x)^T*a1(\widetilde{x}) * b(x)^T*b(\widetilde{x})\\
		   &= \Big( \sum_{i=1}^{n} a_i(x)*a_i(\widetilde{x} \Big)*\Big( \sum_{j=1}^{m} b_j(x)*b_j(\widetilde{x} \Big)\\
		   &= \sum_{i=1}^{n}\sum_{j=1}^{m}[a_i(x)*b_j(x)] * [a_i(\widetilde{x})*b_j(\widetilde{x})]\\
		   &= \sum_{i=1}^{n}\sum_{j=1}^{m} c_{i_j}(x)*c_{i_j}(\widetilde{x})\\
		   &= c(x)^T*c(\widetilde{x})\\
\end{align*}
So the product can also be represented by its feature function. It is a valid kernel
\item ) This proof comes form the a),b). As the polynomial will be something like $ax^n+b^{n-1}+....+c$.\\
From a) we know that $ax+bx$ is a valid kernel. Also the kernel that are in power $(n,n-1)$ are also valid kernels
because of b). Therefore all the summations are producing a valid kernel.
\item ) As c) is creating a valid kernel basically this follows is as exp is a function with positive coefficient.\\
Moreover $exp(x) = \lim_{i \to \infty} \big( 1+x+ \ldots + \frac{x^i}{i!} \big)$.\\
So d) is a valid kernel.
\end{enumerate}
\subsection{B}
This function is an RBF function with $\sigma = 1$
So the expansion of our formula is 
\begin{align*}
exp\Big( -\frac{1}{2} \lVert{x}\rVert^2\Big) * exp\Big( -\frac{1}{2} \lVert{y}\rVert^2\Big) * \sum_{i=0}^{\infty} \frac{(xy)^i}{i} &= \\
exp\Big( -\frac{1}{2} \lVert{x}\rVert^2\Big) * exp\Big( -\frac{1}{2} \lVert{y}\rVert^2\Big) * (1 + \frac{x*y}{1!} + \frac{{x*y}^2}{2!}) + \ldots) &= \\
exp\Big( -\frac{1}{2} \lVert{x}\rVert^2\Big) * exp\Big( -\frac{1}{2} \lVert{y}\rVert^2\Big) * (1*1 + \frac{x}{\sqrt{1!}}* \frac{y}{\sqrt{1!}} +  \frac{x^2}{\sqrt{2!}}* \frac{y^2}{\sqrt{2!}} + \ldots) &= \\
\phi(x) * \phi(y)\\
\end{align*}

where $\phi(x) = \Big[ 1,\frac{x}{\sqrt{1!}},\frac{x^2}{\sqrt{2!}}, \ldots \Big]$
