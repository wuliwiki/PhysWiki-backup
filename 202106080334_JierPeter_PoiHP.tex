% 庞加莱半平面(微分几何计算实例)
% 微分几何|联络|Poincare Half-plane|度量|黎曼联络|黎曼度量|Riemannian connection|Riemannian metric

庞加莱半平面是历史上非常重要的一个模型.众所周知,欧几里得几何学中有五条公理,其中第五条“过直线外一点有且仅有一条直线与已知直线垂直”非常冗长而且绕口,因此历史上一直有不少数学家致力于通过其它四条来推出第五条,也就是将第五公理变成一个定理.在GTM 275\cite{GTM275}中将这种尝试评价为“英雄式”的(heroic).这是出于早期数学家们的一种朴素的直觉,即几何就应该是欧几里得空间那样子的,所以第五公理必须成立,哪怕只是作为定理.后来的人们逐渐意识到第五公理并不能被前四条所证明,并逐渐发展出了符合前四条但违反第五条的几何学,也就是所谓的\textbf{非欧几何学}.庞加莱半平面就是一个典型的例子,在本节的\textbf{测地线}小节我们会简单讨论这一点.

本节的主要目的是以庞加莱半平面为例,演示如何进行具体的计算.

\subsection{庞加莱半平面的定义}

\begin{definition}{庞加莱半平面}
设$\mathbb{H}^2=\{(x, y)\in \mathbb{R}^2|y>0\}$,即二维实平面的上半平面(不包含$x$轴).在$\mathbb{H}^2$上定义\textbf{黎曼度量}$<*,*>$为,对于任意点$(x, y)\in \mathbb{H}^2$处的\textbf{切向量}$(a_i, b_i)\in T_{(x, y)}\mathbb{H}^2$,有
\begin{equation}
<(a_1, b_1), (a_2, b_2)>=\frac{a_1a_2+b_1b_2}{y^2}
\end{equation}
则$\mathbb{H}^2$配合该度量所得到的\textbf{黎曼流形}称为\textbf{庞加莱半平面(Poincaré half-plane)}.
\end{definition}

对于庞加莱半平面度量的描述,更简洁的表达是:$\frac{\dd x\otimes \dd x+ \dd x\otimes\dd x}{y^2}$.这里的$\dd x$是指$\mathbb{H}^2$上的函数$f(x, y)=x$的微分$\dd f$,同样地$\dd y$是$g(x, y)=y$的微分$\dd g$,它们都是$\mathbb{H}^2$上的$1$-形式,而$\otimes$是它们的张量积.

如果用通常的$(x, y)\in \mathbb{R}^2$作为坐标来描述$\mathbb{H}^2$,那么在$\mathbb{R}^2$中的两个点,在保持$x$坐标不变时同步向$y$的正方向移动,那么它们的距离会缩短,并在$y$坐标趋于正无穷时距离趋于零.反过来,如果两个点的$y$坐标始终相同,$x$坐标不变,那么它们同步趋近于$x$轴时,彼此距离会趋近于正无穷.

\subsection{联络形式}

\pentry{联络形式与结构定理\upref{ConFom}}

本小节先计算$\mathbb{H}^2$上的一个联络形式.

由\textbf{反对称定理}\autoref{ConFom_the1}~\upref{ConFom},计算联络形式时选标准正交基来进行计算最为方便,因为$\mathbb{H}^2$是二维的,标准正交基下的联络形式矩阵是一个反对称的$2$阶方阵,也就是说可以被一个分量唯一确定.

第一步,选择标准正交基:
\begin{equation}
\begin{aligned}
\uvec{e}_1=y\frac{\partial}{\partial x}\\
\uvec{e}_2=y\frac{\partial}{\partial y}
\end{aligned}
\end{equation}

第二步,计算对偶基:

由于$\dd x\cdot \frac{\partial }{\partial x}=\dd y\cdot \frac{\partial}{\partial y}=1$,$\dd x\cdot \frac{\partial }{\partial y}=\dd y\cdot \frac{\partial}{\partial x}=0$,我们可以得到对应的对偶基:

\begin{equation}\label{PoiHP_eq1}
\begin{aligned}
\theta^1=\frac{1}{y}\dd x\\
\theta^2=\frac{1}{y}\dd y
\end{aligned}
\end{equation}

因此我们可以计算出\footnote{注意外微分的幂零性,即$\mathrm{d}^2=0$.}:
\begin{equation}\label{PoiHP_eq2}
\begin{aligned}
\dd\theta^1&=-\frac{1}{y^2}\dd y\wedge\dd x=\frac{1}{y^2}\dd x\wedge \dd y\\
\dd\theta^2&=0
\end{aligned}
\end{equation}

第三步,设挠率为$0$,应用\textbf{结构定理}\autoref{ConFom_the2}~\upref{ConFom}和\textbf{反对称定理}\autoref{ConFom_the1}~\upref{ConFom}得:

\begin{equation}\label{PoiHP_eq3}
\begin{aligned}
&\dd\theta^1=-\omega^1_2\wedge\theta^2\\
&\dd\theta^2=-\omega^2_1\wedge\theta^1=\omega^1_2\wedge\theta^1
\end{aligned}
\end{equation}



最后,联立\autoref{PoiHP_eq1} 、\autoref{PoiHP_eq2} 和\autoref{PoiHP_eq3} ,得到:

\begin{equation}
\begin{aligned}
-\omega^1_2\wedge\theta^2&=\theta^1\wedge\theta^2\\
\omega^1_2\wedge\theta^1&=0
\end{aligned}
\end{equation}

因此易得$\omega^1_2=-\theta^1=-\frac{1}{y}\dd x$.



\subsection{高斯曲率}

\pentry{高斯绝妙定理\upref{Egreg}}

由\textbf{高斯绝妙定理}\upref{Egreg},如果要计算$\mathbb{H}^2$的高斯曲率,我们就要去计算某个基下的曲率形式$\Omega^i_j$,从而有$K=\Omega^1_2(\uvec{e}_1, \uvec{e}_2)$.

由上一小节计算出来的联络形式,结合\textbf{结构定理}\autoref{ConFom_the2}~\upref{ConFom},注意$\omega^1_1=\omega^2_2=0$,我们有:
\begin{equation}
\begin{aligned}
\Omega^1_2&=\dd \omega^1_2+\omega^1_k\wedge\omega^k_2\\
&=\dd(-\frac{1}{y}\dd x)+\omega^1_1\wedge\omega^1_2+\omega^1_2\wedge\omega^2_2\\
&=\frac{1}{y^2}\dd y\wedge\dd x\\
&=-\frac{1}{y^2}\dd x\wedge \dd y
\end{aligned}
\end{equation}

于是
\begin{equation}
\begin{aligned}
K&=\Omega^1_2(y\frac{\partial}{\partial x}, y\frac{\partial}{\partial y})\\
&=-\frac{1}{y^2}\dd x\wedge \dd y(y\frac{\partial}{\partial x}, y\frac{\partial}{\partial y})\\
&=-\dd x\wedge \dd y(\frac{\partial}{\partial x}, \frac{\partial}{\partial y})\\
&=-(\dd x\cdot \frac{\partial}{\partial x})(\dd y\cdot \frac{\partial}{\partial y})+(\dd y\cdot \frac{\partial}{\partial x})(\dd x\cdot \frac{\partial}{\partial y})\\
&=-1+0\\
&=-1
\end{aligned}
\end{equation}

因此庞加莱半平面的高斯曲率处处为$-1$,是典型的罗巴切夫斯基几何.




\subsection{Christoffel符号}
\pentry{Christoffel符号\upref{CrstfS}}

我们计算$\mathbb{H}^2$在通常的$(x, y)\in\mathbb{R}^2$坐标下的Christoffel符号.为方便计,将导子$\frac{\partial}{\partial_a}$简记为$\partial_a$.注意对于任何只依赖于$y$的函数$f(y)$,有$\uvec{e}_1f(y)=y\frac{\partial}{\partial x}f(y)=0$,因此$\nabla_{\uvec{e}_1}f(y)X=f(y)\nabla_{\uvec{e}_1}X$.

\begin{equation}
\begin{aligned}
\nabla_{\partial_x}\partial_x&=\nabla_{\frac{1}{y}\uvec{e}_1}\frac{1}{y}\uvec{e}_1\\
&=\frac{1}{y^2}\nabla_{\uvec{e}_1}\uvec{e}_1\\
&=\frac{1}{y^2}\omega^k_1(\uvec{e}_1)\uvec{e}_k\\
&=\frac{1}{y^2}\omega^2_1(\uvec{e}_1)\uvec{e}_2\\
&=-\frac{1}{y^2}
\end{aligned}
\end{equation}










