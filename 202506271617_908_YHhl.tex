% 约翰·惠勒(综述)
% license CCBYSA3
% type Wiki

本文根据 CC-BY-SA 协议转载翻译自维基百科 \href{https://en.wikipedia.org/wiki/John_Archibald_Wheeler}{相关文章}。

约翰·阿奇博尔德·惠勒(John Archibald Wheeler,1911年7月9日-2008年4月13日)是美国理论物理学家。他在二战后基本上推动了美国对广义相对论研究兴趣的复兴。惠勒还曾与尼尔斯·玻尔合作,解释了核裂变的基本原理。他与格雷戈里·布赖特共同提出了“布赖特–惠勒过程”的概念。他最广为人知的成就是推广了“黑洞”这一术语,[1] 用以描述在20世纪早期就已被预测会因引力坍缩形成的天体,并且他还创造了“量子泡沫”“中子慢化剂”“虫洞”和“从比特中产生物质”等术语,[2] 并提出了“一电子宇宙”假说。斯蒂芬·霍金称惠勒为“黑洞故事的英雄”。[3]

惠勒在21岁时于约翰斯·霍普金斯大学获得博士学位,导师是卡尔·赫茨费尔德。随后,他在国家研究委员会奖学金支持下,师从格雷戈里·布赖特(和尼尔斯·玻尔(学习。1939年,他与玻尔合作,利用液滴模型解释核裂变机制,并发表了一系列论文。二战期间,他在芝加哥的曼哈顿计划冶金实验室工作,帮助设计核反应堆;之后又前往华盛顿州里奇兰的汉福德基地,协助杜邦公司建造核反应堆。战争结束后,他回到普林斯顿任教,但在1950年代初再次进入政府体系,参与氢弹的设计与建造。他和爱德华·泰勒是热核武器的主要平民倡导者。[4]

惠勒的大部分职业生涯都在普林斯顿大学物理系度过,他于1938年加入该校,并一直任职至1976年。在普林斯顿任教期间,他指导了46名博士生,是指导博士生数量最多的物理教授。

65岁时,惠勒离开普林斯顿大学。1976年,他被任命为德克萨斯大学奥斯汀分校理论物理中心主任,并一直任职至1986年退休,随后成为名誉教授。
\subsection{早年与教育}
约翰·阿奇博尔德·惠勒于1911年7月9日出生在佛罗里达州杰克逊维尔,父母是图书馆员约瑟夫·L·惠勒和梅布尔·阿奇博尔德(Mabel Archibald,昵称 Archie)惠勒。[5] 他是四个孩子中最年长的一个。他的弟弟约瑟夫获得了布朗大学的博士学位和哥伦比亚大学的图书馆学硕士学位;弟弟罗伯特(Robert)在哈佛大学获得地质学博士学位,并在多家石油公司及多所大学担任地质学家;妹妹玛丽(Mary)在丹佛大学学习图书馆学,后成为一名图书馆员。[6] 他们在俄亥俄州扬斯敦(Youngstown)长大,但在1921至1922年间曾在佛蒙特州本森的农场居住一年,惠勒曾在那里的一所单间教室学校上学。回到扬斯敦后,他进入了雷恩高中学习。[7]

1926年从巴尔的摩城市学院高中毕业后,[8] 惠勒凭借马里兰州提供的奖学金进入约翰斯·霍普金斯大学。[9] 1930年,他在美国国家标准局暑期工作期间发表了第一篇科学论文。[10] 1933年,他获得博士学位。他在导师卡尔·赫兹费尔德的指导下完成的论文题目是《氦的色散与吸收理论》。[11] 他获得了国家研究委员会奖学金,并于1933至1934年在纽约大学师从格雷戈里·布赖特学习,[12] 随后于1934至1935年前往哥本哈根师从尼尔斯·玻尔学习。[13] 在1934年发表的一篇论文中,布赖特和惠勒提出了“布赖特–惠勒过程”,即光子可能转化为物质(电子–正电子对)的机制。[9][14]
