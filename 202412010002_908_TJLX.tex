% 统计力学(综述)
% license CCBYSA3
% type Wiki

本文根据 CC-BY-SA 协议转载翻译自维基百科\href{https://en.wikipedia.org/wiki/Statistical_mechanics}{相关文章}。

在物理学中,\textbf{统计力学}是一种数学框架,将统计方法和概率理论应用于大量微观实体的集合。有时也称为\textbf{统计物理}或\textbf{统计热力学},其应用包括物理学、生物学、化学、神经科学、计算机科学、信息理论和社会学等多个领域。其主要目的是通过研究原子运动所遵循的物理规律,阐明物质在宏观集合状态下的性质。

统计力学源于经典热力学的发展,成功地解释了宏观物理属性(如温度、压力和热容量),将其与微观参数联系起来。这些微观参数围绕平均值波动,并以概率分布为特征。

虽然经典热力学主要关注\textbf{热力学平衡},但统计力学在\textbf{非平衡统计力学}中得到了广泛应用,用于微观建模不可逆过程的速度,这些过程由不平衡驱动。例如,化学反应以及粒子和热的流动。\textbf{涨落-耗散定理}是将非平衡统计力学应用于研究多粒子系统中最简单的非平衡状态(即稳态电流流动)时获得的基本理论。
\subsection{历史}
1738年,瑞士物理学家兼数学家丹尼尔·伯努利发表了《流体动力学》(*Hydrodynamica*),奠定了气体动理论的基础。在这项工作中,伯努利提出了一个至今仍在使用的观点:气体由大量分子组成,这些分子向各个方向运动,它们对表面的撞击导致了我们感受到的气体压力,而我们感受到的热量只是它们运动的动能。[9]

统计力学领域的创立通常归功于以下三位物理学家:
\begin{enumerate}
\item 路德维希·玻尔兹曼,他发展了关于熵的微观状态集合的基本解释;
\item 詹姆斯·克拉克·麦克斯韦,他建立了微观状态概率分布的模型;
\item 乔赛亚·威拉德·吉布斯,他在1884年首次命名了这一领域。
\end{enumerate}

1859年,在阅读鲁道夫·克劳修斯关于分子扩散的论文后,苏格兰物理学家詹姆斯·克拉克·麦克斯韦提出了麦克斯韦分子速率分布,描述了具有特定速度范围的分子比例。这是物理学中的首个统计规律。[10][11] 麦克斯韦还首次从力学角度论证了分子碰撞会导致温度的均匀化,从而趋于平衡。[12] 五年后,即1864年,年轻的维也纳学生**路德维希·玻尔兹曼**阅读了麦克斯韦的论文,并在其后大部分生涯中进一步发展了这一领域。

统计力学在19世纪70年代由玻尔兹曼的研究正式开创,他的大部分研究成果于1896年的《气体理论讲义》(Lectures on Gas Theory)中发表。[13] 玻尔兹曼关于热力学统计解释的原创论文,包括H定理、输运理论、热平衡、气体状态方程等主题,分布在维也纳科学院和其他学会的约2,000页的论文中。他引入了平衡统计系综的概念,并首次研究了非平衡统计力学,并提出了H定理。
\begin{figure}[ht]
\centering
\includegraphics[width=6cm]{./figures/3904934c348528af.png}
\caption{吉布斯统计力学著作的封面} \label{fig_TJLX_1}
\end{figure}
统计力学”这一术语由美国数学物理学家J. 威拉德·吉布斯于1884年创造。[14] 据吉布斯所述,“统计”一词在力学(即统计力学)中的使用最早可以追溯到苏格兰物理学家**詹姆斯·克拉克·麦克斯韦于1871年的表述:

“在处理大量物质时,由于我们无法察觉单个分子,不得不采用我所描述的统计计算方法,并放弃严格的动力学方法,即通过微积分追踪每个运动。”

—— J. 克拉克·麦克斯韦[15]  

今天,“概率力学”可能会被认为是一个更恰当的术语,但“统计力学”已经牢牢确立下来。[16] 

在去世前不久,吉布斯于1902年出版了《统计力学的基本原理》(Elementary Principles in Statistical Mechanics),这本书将统计力学形式化为一种通用方法,能够处理所有力学系统——无论是宏观还是微观,是气态还是非气态系统。[17] 吉布斯的方法最初是在经典力学框架下推导出来的,但由于其高度的普适性,这些方法后来被发现可以轻松适配于量子力学,并且至今仍是统计力学的基础。[18]
\subsection{原理:力学与系综} 
在物理学中,通常研究两种类型的力学:经典力学和量子力学。对于这两种力学,标准的数学方法包括以下两个概念:
\begin{enumerate}
\item 机械系统在某一特定时刻的完整状态,在数学上表示为相点(经典力学)或纯量子态矢量(量子力学)。  
\item 将状态随时间推演的运动方程:哈密顿方程(经典力学)或薛定谔方程(量子力学)。
\end{enumerate}
通过这两个概念,可以原则上计算出系统在任何其他时间(过去或未来)的状态。然而,这些定律与日常生活经验之间存在脱节,因为我们在宏观尺度上进行操作时(例如进行化学反应),既不需要(甚至理论上也不可能)精确地知道每个分子在微观水平上的同时位置和速度。统计力学通过引入系统状态的不确定性,弥合了力学定律与实际经验之间的这种脱节。

普通力学只研究单一状态的行为,而统计力学引入了统计系综的概念,即系统在各种状态下的大量虚拟、独立副本的集合。统计系综是系统所有可能状态的概率分布。在经典统计力学中,系综是相点的概率分布(与普通力学中的单一相点相对),通常表示为具有规范坐标轴的相空间中的分布。在量子统计力学中,系综是纯态的概率分布,可以用密度矩阵简洁地表示。

与概率的常规处理方式类似,统计系综可以通过以下不同方式进行解释:[17]  
\begin{enumerate}
\item 系综可以被理解为单一系统可能处于的各种状态(认知概率,一种知识形式)。  
\item 系综的成员也可以被看作是独立系统在重复实验中经历的状态,这些系统是在相似但控制不完全的条件下准备的(经验概率),在无限次试验的极限下成立。 
\end{enumerate} 
对于许多目的,这两种解释是等价的,在本文中会交替使用。

无论如何解释概率,系综中的每个状态都会依据运动方程随时间演化。因此,系综本身(即状态上的概率分布)也会演化,因为系综中的虚拟系统会不断地从一个状态转移到另一个状态。系综的演化由刘维尔方程(经典力学)或冯诺依曼方程(量子力学)描述。这些方程通过对系综中每个虚拟系统单独应用力学运动方程得出,并假设虚拟系统的概率在从一个状态演化到另一个状态的过程中保持守恒。

一种特殊的系综类型是那些随时间不演化的系综,称为平衡系综,其条件称为统计平衡。统计平衡发生的条件是,对于系综中的每个状态,系综还包含其未来和过去的所有状态,并且这些状态的概率等于该状态的概率。(相比之下,机械平衡是指力的平衡状态,其已停止演化。)对孤立系统的平衡系综的研究是统计热力学的核心。非平衡统计力学则研究更一般的情形,包括随时间变化的系综和非孤立系统的系综。
\subsection{统计热力学} 
统计热力学(也称为**平衡统计力学**)的主要目标是根据材料组成粒子的属性及其相互作用,从微观角度推导材料的经典热力学性质。换句话说,统计热力学建立了材料在热力学平衡条件下的宏观性质与其内部微观行为和运动之间的联系。

虽然统计力学本身涉及动力学,但统计热力学主要关注统计平衡(稳定状态)。统计平衡并不意味着粒子停止运动(即机械平衡),而是指系综的整体状态不再随时间演化。
\subsubsection{基本假设} 
统计平衡的一个充分(但非必要)条件是:对于一个孤立系统,其概率分布仅依赖于守恒量(如总能量、总粒子数等)的函数。[17] 可以考虑许多不同的平衡系综,但只有其中的一部分与热力学对应。[17] 因此,需要额外的假设来说明为什么给定系统的系综应该采取某种特定形式。

一种常见的方法是采用等概率假设,这在许多教科书中都有提到。[18] 该假设指出:

对于一个具有确知能量和确知组成的孤立系统,系统可以以相等的概率处于任何与这些信息一致的微观状态。

因此,等概率假设为下面描述的微正则系综提供了动机。支持等概率假设的理由包括:
\begin{itemize}
\item 遍历假设:一个遍历系统是指能够随时间演化以探索“所有可达”状态的系统,这些状态具有相同的能量和组成。在遍历系统中,微正则系综是唯一可能的具有固定能量的平衡系综。然而,该方法的适用性有限,因为大多数系统并非遍历的。
\item 无差别原理:在缺乏任何进一步信息的情况下,我们只能对每种可能的情况分配相等的概率。
\item 最大信息熵:无差别原理的一个更复杂版本指出,正确的系综是与已知信息兼容且具有最大吉布斯熵(信息熵)的系综。[19] 
\end{itemize}