% 紧束缚近似
% keys 紧束缚|能带理论

\pentry{金属中的自由电子气体\upref{mfcgas},布洛赫理论\upref{Bloch}}

在固体物理学中,除了用近自由电子近似\upref{egasmd}分析电子能带结构,还有一种方法,被称为\textbf{紧束缚近似}在一些场合的问题下是一个很好的近似,并且在计算和分析能带上能体现出一些优势.

在紧束缚近似模型中,一般假定电子受到了原子实很强的束缚作用,当它靠近一个原子实时,它的势场近似为库伦势.因此电子在原子实附近的行为和原子轨道波函数(类氢原子的束缚态\upref{HWF})有一定的相关性.
\begin{figure}[ht]
\centering
\includegraphics[width=8cm]{./figures/tbappx_2.png}、
\caption{一维晶格的势场} \label{tbappx_fig2}
\end{figure}


由于势场是所有原子实产生的库伦势的叠加,所以在紧束缚近似模型中,假定在原子实附近的电子主要受该原子实的库伦势的影响,而将其他的原子实的库伦势对它的作用视为微扰.原子轨道波函数的能级是分立的,第 $n$ 个轨道对应 $n^2$ 个简并的态(不考虑自旋).但如果引入其他原子实的势场微扰,这些能级就会发生劈裂,如下图所示.

\begin{figure}[ht]
\centering
\includegraphics[width=12cm]{./figures/tbappx_1.png}
\caption{紧束缚近似的能带示意图} \label{tbappx_fig1}
\end{figure}

\subsection{紧束缚近似的计算}

假设 $V(\bvec r-\bvec R_m)$ 为格点 $m$ 处的原子实产生的势场(一般可以近似视为库伦势场),设 $\phi_i(\bvec r)$ 为该势场下电子的束缚态波函数.那么根据核外电子的相关知识,我们知道 $\phi_i(\bvec r)$ 可以是 $s$ 轨道、$p$ 轨道波函数等等,并且可以由 $n,l,m$ 三个量子数来描述(不考虑自旋).这些束缚态波函数满足
\begin{equation}
\left[-\frac{\hbar^2}{2m}\nabla^2+V(\bvec r-\bvec R_m)\right]\phi_i(\bvec r-\bvec R_m)=\epsilon_i
\end{equation}

然而晶格的真正周期性势场 $U(\bvec r)$ 与 $V(\bvec r)$ 是有差距的,正如 \autoref{tbappx_fig2} 所示.我们将 $U(\bvec r)-V(\bvec r-\bvec R_m)$ 视为微扰哈密顿量.我们要求的波函数 $\psi(\bvec r)$ 满足
\begin{equation}\label{tbappx_eq1}
\qty[-\frac{\hbar^2}{2m}\nabla^2+U(\bvec r)]\psi(\bvec r)=E
\end{equation}

围绕不同的格点,在各自的势场 $V(\bvec r-\bvec R_m),m=1,\cdots,n$ 下共有 $n$ 个束缚态波函数 $\phi_i(\bvec r-\bvec R_m),m=1,\cdots,n$.我们假设在周期性势场下,最终的电子波函数是这 $n$ 个束缚态波函数的线性叠加:
\begin{equation}
\psi(\bvec r)=\sum_{m} a_m \phi_i(\bvec r-\bvec R_m)
\end{equation}
代入\autoref{tbappx_eq1} 后可以得到
\begin{equation}
\sum_{m}\qty[\epsilon_i+(U(\bvec r)-V(\bvec r-\bvec R_m)]
\end{equation}
