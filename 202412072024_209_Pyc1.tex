% Python 环境搭建
% keys Python
% license Usr
% type Tutor

\subsection{Python 安装} 

\subsubsection{2.1.Window 平台安装 Python:}
访问 \href{https://www.python.org/downloads/windows/}{Python 官网关于 Windows 下载},一般就下载  Windows installer,其中“Stable Releases”是指稳定版本,推荐下载。
\begin{figure}[ht]
\centering
\includegraphics[width=14.25cm]{./figures/a331410dd57d19b6.png}
\caption{Python 官网关于 Windows} \label{fig_Python_1}
\end{figure}
% 安装以后在开始菜单中搜索程序名如 \verb`Python 3.8 (64-bit)`, 点击后即可打开 Python 命令行。 安装包也会自动安装 \verb`pip3`, 可以在 Powershell 或者 cmd 中输入 \verb`pip3 --version` 查看。
% \addTODO{pip3 是什么?没介绍}

\subsubsection{2.2.MAC 平台安装 Python:}
MAC 系统都自带有 Python2.7 环境,你可以在链接 \href{https://www.python.org/downloads/mac-osx/}{Python 官网关于 mac 下载} 上下载最新版安装 Python 3.x。你也可以参考源码安装的方式来安装。

\subsubsection{2.3.Unix 和 Linux 平台安装 Python:}
你可以访问 \href{https://www.python.org/downloads/source/}{Python 官网关于 Linux 下载},选择适用于 Unix/Linux 的源码压缩包。
\begin{figure}[ht]
\centering
\includegraphics[width=14.25cm]{./figures/83c4a69337735cb6.png}
\caption{Python 官网关于 Linux} \label{fig_Python_2}
\end{figure}

也可以直接用命令行下载
\begin{lstlisting}[language=bash]
wget https://www.python.org/ftp/python/3.7.6/Python-3.7.6.tgz
\end{lstlisting}

创建安装目录(你想放哪就放哪)
\begin{lstlisting}[language=bash]
mkdir -p /usr/local/python3
\end{lstlisting}

解压
\begin{lstlisting}[language=bash]
tar -zxvf Python-3.7.6.tgz
\end{lstlisting}

编译安装
\begin{lstlisting}[language=bash]
# gcc 环境、zlib 库和 readline-devel 包
yum -y install gcc
yum -y install zlib*
yum install readline-devel
# 配置
cd Python-3.7.6
./configure --prefix=/usr/local/python3
# 编译安装
make && make install
\end{lstlisting}

建立软链接
\begin{lstlisting}[language=bash]
ln -s /usr/local/python3/bin/python3.7 /usr/bin/python3
ln -s /usr/local/python3/bin/pip3.7 /usr/bin/pip3
\end{lstlisting}

测试安装
\begin{lstlisting}[language=bash]
# 返回 Python 3.7.6(版本)
python3 --version
# 命令行输出
python3
......
print("你好")
\end{lstlisting}

\subsection{库(包)的简介与Pip} 

\begin{itemize}
\item 库(包)是Python的扩展,是Python基本功能的延拓(具体会在后文解释)
\item Pip 是 Python 包管理工具,该工具提供了对Python 包的查找、下载、安装、卸载的功能。\footnote{注意:Python 2.7.9 + 或 Python 3.4+ 以上版本都自带 pip 工具}
\end{itemize}

你可以通过以下命令来判断是否已安装:

% pip --version     # Python2.x 版本命令

\begin{lstlisting}[language=bash]
pip3 --version    # Python3.x 版本命令
\end{lstlisting}




