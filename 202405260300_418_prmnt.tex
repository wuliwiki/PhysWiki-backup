% 素数与合数
% keys 数论|素数|质数
% license Usr
% type Tutor

\pentry{整除\nref{nod_divisb}}{nod_7083}
\begin{definition}{素数}
若对于一个正整数 $p > 1$,满足 $p$ 除了 $1$ 与 $p$ 自己以外没有其他因子。也就是说:
\begin{equation}
\forall n \in (1,p), n \not{\mid} ~ p ~,
\end{equation}
就称 $p$ 是\textbf{素数(prime)},又称质数。
\end{definition}

\begin{definition}{合数}
大于 $1$ 且不是素数的整数称为\textbf{合数(composite)}。
\end{definition}

由定义可以立即得到推论。
\begin{corollary}{}
对于所有大于 $1$ 的正整数,其要么是质数,要么是合数。
\end{corollary}
\begin{corollary}{}
合数必定有大于 $1$ 且小于其本身的因子。也就是说对于合数 $m > 1$,必定存在 $1 < l < m$,$l | m$。
\end{corollary}



在定义了素数后我们可以引入一个定理。
\begin{theorem}{}\label{the_prmnt_1}
对于所有大于 $1$ 的正整数都是素数的乘积。
\end{theorem}
\textbf{证明}:$n$ 要么自己本身就是素数,否则 $n$ 就有大于 $1$ 且小于 $n$ 的因子。不失一般性的,考虑 $n$ 的大于 $1$ 且小于 $n$ 的最小因子 $a$,则 $a$ 要么是素数,要么 $a$ 是合数。对于 $a$ 是合数的时候,则必定存在 $1 < l < a$ 使得 $l | a$,而 $a | m$,故 $l|m$(整除的传递性)。这使得 $l$ 是更小的一个 $n$ 的大于 $1$ 的因子,矛盾!故 $a$ 必定是素数。

这指出,$n$ 要么本身就是素数,要么可以被一个小于 $n$ 的素数 $p_1$ 整除。而 $n / p_1$ 又是一个数,要么本身是素数,要么可以被一个小于 $n / p_1$ 的素数 $p_2$ 整除,依次类推...... 最终终将得到一个 $p_{m+1} = n/ p_1 / p_2 / \cdots / p_m$ 是一个素数。这就使得 $n$ 可以表示为各个素数 $p_1, p_2, \dots, p_{m+1}$ 的乘积。证毕。

\begin{theorem}{欧几里得第二定理}

\end{theorem}