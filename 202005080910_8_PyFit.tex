% SciPy数值计算库(二): 最小二乘法
\pentry{Python 入门\upref{Python}}

\subsection{最小二乘拟合}
最小二乘法是一种数学优化技术.它通过最小化误差的平方和寻找数据的最佳函数匹配.利用最小二乘法可以简便地求得未知的数据,并使得这些求得的数据与实际数据之间误差的平方和为最小.

通常情况下最小二乘法是这样的:
给定一组数,假设是二维数据.$(x_1,y_1),(x_2,y_2),\cdots,(x_n,y_n)$ ,并且 $x,y$是相关的.假设满足方程,

\begin{equation}
y=f(x)
\end{equation}


但是这个方程我们是不知道具体什么样的.我们找到一个近似函数

[公式]

对于每一个点 [公式] 有 [公式] , [公式] 是残差,这里姑且叫误差吧.那么总的误差平方和为

[公式]