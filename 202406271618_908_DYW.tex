% 大亚湾核反应堆中微子实验
% license CCBYSA3
% type Wiki

(本文根据 CC-BY-SA 协议转载自原搜狗科学百科对英文维基百科的翻译)

\textbf{大亚湾核反应堆中微子实验}总部设在中国,是一个研究中微子的跨国粒子物理项目。此跨国合作包括来自智利、美国、俄罗斯、捷克共和国及中国大陆和台湾地区的研究人员。该项目的美方资金由美国能源部高能物理办公室资助。

大亚湾核反应堆中微子实验位于大亚湾,在香港东北方向约52公里,深圳以东约42公里。其在香港的香港仔隧道粒子物理实验室还有一个附属项目。香港仔实验室用来测量可能会影响大亚湾核反应堆中微子实验的由宇宙射线缪子产生的中子。

大亚湾核反应堆中微子实验由八个反中微子探测器组成,它们分布在距六个核反应堆1.9km (1.2 mi)范围内的三个位置。每个探测器含有被光电倍增管和防护层包围的20吨液体闪烁计数器(掺杂钆的直链烷基苯)。[1]

中国开平市正在开发一个更大的后续项目—江门地下中微子天文台(JUNO),[2]它将使用填充了20000吨液体闪烁器的丙烯酸玻璃球体来探测核反应堆里的反中微子。江门地下中微子天文台已于2015年1月破土动工,预计将在2020年投入运营。[3]

\subsection{中微子振荡}
该实验主要研究中微子振荡,旨在利用大亚湾核电站和岭澳核电站反应堆产生的反中微子来测量交叉混合角 θ13 。除此之外,科学家们也对中微子是否违反Charge-Parity守恒感兴趣。

在2012年3月8号,大亚湾合作组宣布了[4][5][6]一个偏差为5.2σ的新发现:$\theta_{13} \neq 0$,且
$$\sin^2(2 \theta_{13}) = 0.092 \pm 0.016 \\, (\text{stat}) \pm 0.005 \\, (\text{syst})~$$
这一重要结果代表了一种新的出乎意料大的振荡类型。[7]这与早期由T2K、米诺斯(MINOS)和双周(Double Chooz)实验得出的不明显结果是一致的。当$\theta_{13}$拥有如此大的值,NOνA实验对中微子质量等级敏感的概率约为50\%。这些实验也许可以用来探明中微子之间的电荷宇称破坏。

大亚湾合作组于2014年发布了新的数据分析,[8]它使用能谱来改善交叉混合角的界限:
$$\sin^2(2 \theta_{13}) = 0.090^{+0.008}_{-0.009}~$$
另一个测量氛围获得的中子的独立实验结果也发表了:[9]
$$\sin^2(2 \theta_{13}) = 0.083 \pm 0.018~$$
大亚湾实验已经利用它的数据搜索到了一个轻型惰性中微子的信号,因此排除了一些以前未被探测到的中微子质量范围。[10]

在2015年莫里蒙德(Moriond)物理会议上,新的交叉混合角和质量差的最佳拟合结果被提出:[11]