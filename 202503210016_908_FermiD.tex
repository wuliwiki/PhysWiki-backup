% 费米-狄拉克统计(综述)
% license CCBYSA3
% type Wiki

本文根据 CC-BY-SA 协议转载翻译自维基百科\href{https://en.wikipedia.org/wiki/Fermi\%E2\%80\%93Dirac_statistics}{相关文章}。

费米–狄拉克统计是一种适用于由许多不相互作用的相同粒子组成的系统的量子统计,这些粒子遵循泡利不相容原理。其结果是费米–狄拉克分布,描述了粒子在不同能量态上的分布。该统计方式以恩里科·费米和保罗·狄拉克命名,他们分别在 1926 年独立推导出了这一分布。\(^\text{[1][2]}\)费米–狄拉克统计属于统计力学的范畴,并且基于量子力学的基本原理。

费米–狄拉克统计适用于具有半整数自旋(\(1/2\)、\(3/2\) 等)的相同且不可区分的粒子,这些粒子被称为费米子,并处于热力学平衡状态。当粒子之间的相互作用可以忽略时,该系统可以用单粒子能级来描述。其结果是粒子在这些能级上的费米–狄拉克分布,其中任何两个粒子都不能占据相同的状态,这对系统的性质产生了重要影响。费米–狄拉克统计最常应用于电子,电子是一种自旋为\(1/2\)的费米子。

费米–狄拉克统计的对应理论是玻色–爱因斯坦统计,它适用于具有整数自旋(0、1、2 等)的相同且不可区分的粒子,这些粒子被称为玻色子。在经典物理中,麦克斯韦–玻尔兹曼统计用于描述相同但可区分的粒子。与费米–狄拉克统计不同,在玻色–爱因斯坦统计和麦克斯韦–玻尔兹曼统计中,多个粒子可以占据相同的量子态。
\subsection{历史}
