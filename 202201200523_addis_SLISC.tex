% SLISC 库概述
% C++|矩阵

\pentry{C++ 基础\upref{Cpp0}}

相对于 fortran 或者 Matlab 等, 用 C++ 做数值计算的一个缺陷就是语言本身(或标准库)没有矩阵以及高维矩阵类. 但我们可以用第三方库或者自己写一个. 我们选择后者, 原因是为了教学需要我们需要保持代码的简单易读, 避免复杂的 C++ 语法.

本书中我们将大量使用自编的 \textbf{SLISC} 矩阵库, 即 \textbf{Scientific Library in Simple C++}. 代码可以从 \href{https://github.com/MacroUniverse/SLISC0}{GitHub 仓库}下载. 该库的特点是尽量不使用 C++ 的复杂语法(如模板)和复杂的类结构, 使代码便于阅读学习和修改, 同时又保持相对较高的性能. SLISC 使用兼容性较高的 C++11 标准.

SLISC 库在矩阵类的基础上还实现了一些科学计算常用的功能, 如线性代数运算、 插值、 计时\upref{SliTim}、 文件和目录操作\upref{Sfile}、 矩阵文件读写\upref{matb}、 字符串处理、 特殊函数、 任意精度计算、 量子力学计算等. 当然, 一些功能基于其他开源项目如 Intel MKL, Arb\upref{ArbLib}, GSL\upref{GSL}, Eigen, Arpack\upref{Arpkpp}. 在编译时可以通过宏来决定是否使用这些依赖.

在 SLISC 中, 我们希望把矩阵进行一定程度的封装, 但又几乎不损失性能. 许多人使用流行的矩阵库例如 Eigen, 但是其代码复杂, 错误信息不容易懂, 代码修改起来非常困难. 例如 Eigen 的 \verb|MatrixXd| 矩阵类有 6 次继承, 一个 2x2 的矩阵在 gdb 里面调试的时候显示出来的变量信息是这样的:
\begin{lstlisting}[language=cpp]
<Eigen::PlainObjectBase<Eigen::Matrix<double,-1,-1,0,-1,-1>>> = 
{<Eigen::MatrixBase<Eigen::Matrix<double,-1,-1,0,-1,-1>>> = 
{<Eigen::DenseBase<Eigen::Matrix<double,-1,-1,0,-1,-1>>> = 
{<Eigen::DenseCoeffsBase<Eigen::Matrix<double,-1,-1,0,-1,-1>, 3>> = 
{<Eigen::DenseCoeffsBase<Eigen::Matrix<double,-1,-1,0,-1,-1>, 1>> = 
{<Eigen::DenseCoeffsBase<Eigen::Matrix<double,-1,-1,0,-1,-1>, 0>> = 
{<Eigen::EigenBase<Eigen::Matrix<double,-1,-1,0,-1,-1>>> =
{<No data fields>}, <No data fields>}, 
<No data fields>}, <No data fields>}, <No data fields>}, <No data fields>},
m_storage = {m_data = 0x855ceb0, m_rows = 2, m_cols = 2}}, <No data fields>
\end{lstlisting}
实际上这里面的重点只有最后一行, 显示了矩阵在内存中的地址 \verb|m_data|, 行数 \verb|m_rows| 以及列数 \verb|m_cols|. 这样的库只适合直接拿来用, 不适合阅读、学习和修改, 尤其是对非计算机专业的同学来说.

以下列出 SLISC 的一些特性, 我们以后会详细介绍.
\begin{itemize}
\item 全部定义使用 \verb|namespace slisc|, 所有宏以 \verb|SLS_| 开头.
\item 完全不使用模板, 用 Matlab/Octave 生成代码. 初学用户不需要了解代码如何生成, 可以直接阅读或使用生成后的代码.
\item 实现了密矩阵和一些稀疏矩阵.
\item 实现了密矩阵的剪切,并用于函数接口. 这会有少量的额外运算, 但使用起来却比指针方便得多. 用户可以自行选择使用哪种接口.
\item debug 模式实现了详尽的尺寸和指标检查, 可以及时发现指标超出长度等错误.
\item 函数尽量首先使用和 Lapack 类似的指针接口实现, 然后再封装一层更友好的接口.
\item 线性代数运算底层使用 Intel 的 Math Kernel Library (MKL) 来实现, 可以保证性能.
\item 尽量不使用 \verb|unsigned| 类型.
\item 为了保证性能, 函数内部尽量不改变矩阵尺寸, 这是因为内存的动态分配往往耗时较多.
\end{itemize}

\subsection{重新定义类型名称}
在 C++ 标准中, 一些基本类型在内存中的长度在不同计算机上可能会不同. 例如 \verb|long| 有时候和 \verb|int| 一样是 4 字节, 而另一些时候则和 \verb|long long| 一样是 8 字节. 因此, SLISC 仿照 Numerical Recipe 的方式, 定义自己的类型名称. 以后本书统一使用这些定义.
\begin{itemize}
\item \verb|Char| 是 \verb|char|, 即 1 字节字符或整数
\item \verb|Uchar| 是 \verb|unsigned char|, 即 1 字节无符号整数
\item \verb|Short| 是 2 字节整数
\item \verb|Int| 是 4 字节整数
\item \verb|Long| 是 \verb|Int| 或者 \verb|Llong|, 取决于宏 \verb|SLS_USE_INT_AS_LONG| 是否定义. SLISC 库种, \verb|Long| 被用来作为矩阵的索引类型.
\item \verb|Llong| 是 8 字节整数
\item \verb|Float| 是 \verb|float|(4 字节)
\item \verb|Doub| 是 \verb|double|(8 字节)
\item \verb|Ldoub| 是 \verb|long double| (16 字节)
\item \verb|Fcomp| 是 \verb|std::complex<float>| (8 字节)
\item \verb|Comp| 是 \verb|std::complex<double>| (16 字节)
\item \verb|Lcomp| 是 \verb|std::complex<long double>| (32 字节)
\end{itemize}
另外我们还定义一些附加类型, 用于声明函数的参数. 例如 \verb|Doub_I| 是 \verb|const Doub|, 表示函数的输入参数. 又如 \verb|Doub_O| 是 \verb|Doub &|, 表示函数的输出参数. \verb|Doub_IO| 也是 \verb|Doub &|, 但表示既作为输入也作为输出. 以上的标量类型都同理, 在后面加 \verb|_I|, \verb|_O|, \verb|_IO| 分别表示 in, out, in-out.

以后还会看到矢量类和矩阵类, 但和标量不同, 为了避免不必要的复制, 它们都必须 pass by reference 而不是 pass by value. 例如矩阵类 \verb|CmatDoub| 的 \verb|CmatDoub_I| 是 \verb|const CmatDoub &|, \verb|CmatDoub_O| 和 \verb|CmatDoub_IO| 都是 \verb|CmatDoub &|.
