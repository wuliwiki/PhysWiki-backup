% 机器学习(综述)
% license CCBYSA3
% type Wiki

本文根据 CC-BY-SA 协议转载翻译自维基百科\href{https://en.wikipedia.org/wiki/Machine_learning}{相关文章}。

\textbf{机器学习(ML)}是人工智能的一个研究领域,关注于开发和研究能够从数据中学习并对未见过的数据进行泛化的统计算法,从而在没有明确指令的情况下执行任务。[1] 深度学习领域的进展使得神经网络在性能上超越了许多先前的方法。[2]

机器学习应用于许多领域,包括自然语言处理、计算机视觉、语音识别、电子邮件过滤、农业和医学。[3][4] 将机器学习应用于商业问题的领域被称为预测分析。

统计学和数学优化(数学编程)方法构成了机器学习的基础。数据挖掘是一个相关的研究领域,专注于通过无监督学习进行探索性数据分析(EDA)。[6][7]

从理论角度来看,可能大致正确(PAC)学习为描述机器学习提供了一个框架。
\subsection{历史}  
“机器学习”这一术语由IBM员工、计算机游戏和人工智能领域的先驱亚瑟·塞缪尔(Arthur Samuel)于1959年创造。[8][9] 在这一时期,“自我学习计算机”这个同义词也曾被使用。[10][11]

尽管最早的机器学习模型是在1950年代由亚瑟·塞缪尔发明的,该程序用于计算每方在跳棋中的获胜概率,但机器学习的历史可以追溯到几十年来人类对研究人类认知过程的渴望和努力。[12] 1949年,加拿大心理学家唐纳德·赫布(Donald Hebb)出版了《行为的组织》(The Organization of Behavior)一书,在书中他提出了通过神经元之间特定交互形成的理论神经结构。[13] 赫布关于神经元相互作用的模型为人工智能和机器学习算法在节点(或计算机用来传输数据的人工神经元)下如何工作奠定了基础。[12] 其他研究人类认知系统的学者也为现代机器学习技术做出了贡献,包括逻辑学家沃尔特·皮茨(Walter Pitts)和沃伦·麦卡洛克(Warren McCulloch),他们提出了早期的神经网络数学模型,旨在开发模拟人类思维过程的算法。[12]

到1960年代初,雷神公司(Raytheon)开发了一种实验性的“学习机器”,名为Cybertron,它采用打孔带存储,用于分析声纳信号、心电图和语音模式,使用的是基础的强化学习。它通过人工操作员/教师反复“训练”以识别模式,并配备了一个“错误”按钮,用于在做出错误决策时促使其重新评估。[14] 1960年代有关机器学习的代表性书籍之一是尼尔森(Nilsson)的《学习机器》一书,主要讨论了用于模式分类的机器学习。[15] 与模式识别相关的兴趣持续到1970年代,正如Duda和Hart在1973年所描述的那样。[16] 1981年,有报告讨论了使用教学策略,使人工神经网络学习从计算机终端识别40个字符(26个字母、10个数字和4个特殊符号)。[17]

汤姆·M·米切尔(Tom M. Mitchell)提出了机器学习领域算法的广泛引用的正式定义:“如果一个计算机程序在经验E的基础上,针对某些任务类别T,通过性能度量P,在T类别中的任务执行表现有所提高,则该计算机程序可以说是从经验E中学习。”[18] 这个关于机器学习所涉及任务的定义提供了一个基本的操作性定义,而不是从认知角度来定义该领域。这一概念沿袭了阿兰·图灵(Alan Turing)在其论文《计算机器与智能》中的提议,其中“机器能思考吗?”的问题被“机器能做我们(作为思维实体)能够做的事吗?”所取代。[19]

现代机器学习有两个目标。一是根据已开发的模型对数据进行分类;另一目的是根据这些模型对未来的结果进行预测。一个专门用于数据分类的假设性算法,可能会使用计算机视觉技术,结合监督学习来训练算法识别癌变的痣。用于股票交易的机器学习算法可能会向交易员提供未来潜在的预测。[20]
\subsection{与其他领域的关系}  
\subsubsection{人工智能}
\begin{figure}[ht]
\centering
\includegraphics[width=6cm]{./figures/108658dec604aeb3.png}
\caption{机器学习作为人工智能的一个子领域[21]} \label{fig_JQXX_1}
\end{figure}
作为一项科学事业,机器学习源自于对人工智能(AI)的探索。在人工智能作为学术学科的早期,一些研究者希望让机器从数据中学习。他们试图通过各种符号方法来解决这个问题,以及当时被称为“神经网络”的方法;这些方法大多是感知机和其他模型,后来被发现实际上是统计学中广义线性模型的再发明。[22] 概率推理也被应用,特别是在自动化医疗诊断中。[23]: 488

然而,随着对逻辑、知识为基础的方法的日益重视,人工智能与机器学习之间出现了分歧。概率系统面临数据获取和表示的理论和实践问题。[23]: 488 到1980年,专家系统已经主导了人工智能,统计学也不再受到青睐。[24] 虽然符号/知识为基础的学习方法在人工智能中仍然有所研究,推动了归纳逻辑编程(ILP)的发展,但更多的统计学研究现在已经脱离了人工智能的范畴,转向了模式识别和信息检索。[23]: 708–710, 755 神经网络研究在同一时期也被人工智能和计算机科学放弃。这一领域也被从其他学科的研究者所延续,包括约翰·霍普菲尔德、戴维·鲁梅哈特和杰弗里·辛顿。他们的主要成功是在1980年代中期重新发明了反向传播算法。[23]: 25

机器学习(ML)在1990年代重新组织并被认定为独立的学科,开始蓬勃发展。该领域的目标从实现人工智能转向解决具有实际性质的可解问题。它的重点从人工智能继承的符号方法转向了借鉴统计学、模糊逻辑和概率论的方法和模型。[24]