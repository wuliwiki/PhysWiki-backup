% 克利福德代数(综述)
% license CCBYSA3
% type Wiki

本文根据 CC-BY-SA 协议转载翻译自维基百科\href{https://en.wikipedia.org/wiki/Clifford_algebra}{相关文章}。

在数学中,克利福德代数[a] 是由具备二次型的向量空间生成的一类代数。它是一个带有单位元的结合代数,同时具有一个特定子空间这一附加结构。作为 $K$-代数,它们推广了实数、复数、四元数以及若干其他超复数体系。[1][2] 克利福德代数理论与二次型理论及正交变换理论紧密相关。克利福德代数在几何学、理论物理以及数字图像处理等诸多领域中具有重要应用。该名称源自英国数学家威廉·金登·克利福德(William Kingdon Clifford, 1845–1879)。

在所有克利福德代数中,最为常见的是正交克利福德代数,它们亦称为(伪)黎曼克利福德代数,以区别于辛克利福德代数。[b]
