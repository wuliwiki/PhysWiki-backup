% 量子力学的基本原理(量子力学)
% 量子力学|态矢

在介绍量子力学的基本原理之前,我们要对“什么是物理理论”做一个简单分析.这是因为量子力学常因“反直觉”而让初学者迷惘,我希望以下讨论能帮助初学者理清概念,从而自然地接受量子力学的语言.

\subsection{从牛顿理论到量子理论}

从现代科学哲学的视角看,一个物理理论是一个数学模型,在这个模型中有一些概念是有实验对应的.这就是说,一个物理理论首先是一个数学理论,而使它区分于数学、成为物理的因素即是“实验”,可以直观理解为“有能在仪器上看到、用感官观测到”的量,通常称之为“可观测量”.

以牛顿力学为例.牛顿力学可以认为是四维空间中的几何学,其中“点的坐标”这一概念就是可观测量,它可以显示为尺子上的数值.更准确地说,考虑到牛顿力学中时间的绝对性,该理论应该是一维空间上处处沾了一片三维空间的“纤维丛”上的几何学,不同的观察者眼中会有不同的三维空间坐标,但是时间坐标不变.

除了几何学假定以外,牛顿力学还受三大定律的约束.这三大定律定义了一个概念,“力”.力本身不是可观测量,但我们可以借助此概念来描述物体运动的规律.质量为$m$的物体被劲度系数为$k$、原长为$l$的弹簧拉着,做角速度为$\omega$的匀速圆周运动,则规律预言,弹簧的伸长量是$\frac{l\omega^2m}{k-m\omega^2}$.伸长量是可观测量,所以我们可以做实验,看看测出来的伸长量是否是这个值,以此来判断牛顿理论的准确性.
\addTODO{弹簧的长度是心算的,可能有误.核算后再删除此“未完成”.}

牛顿理论中大量概念都是可观测的,因此看起来很符合直觉.哪怕是“力”这一实际上不可观测的量,也可以体现在上述弹簧的规律中,因此人们发明了弹簧秤来测量力——实际上测量的是弹簧长度,只是由牛顿理论和力联系起来了.




























