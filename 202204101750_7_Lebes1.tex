% 非负函数的 Lebesgue 积分
% 实变函数|勒贝格积分

\pentry{可测函数\upref{MsbFun}}

Lebesgue积分的思路是对函数的值域进行分划,以相应值域的逆映射作为“柱底”.归根到底,Lebesuge积分还是要对定义域作分划的,但相比Riemann积分的直接对定义域作分划,Lebesgue积分的分划方式更任意.对于可测函数,Lebesgue积分的分划得到的“柱底”都是可测集.

我们就从将可测集划分为两两不交的可测子集入手,先研究这种分划的性质.

\subsection{可测集的分划}



\begin{definition}{可测分划}
设$E\in\mathbb{R}^n$是可测集.如果有限\textbf{族}$\{E_1, E_2, \cdots, E_n\}$中各$E_i$\textbf{两两不交}、都是$E$的子集、\textbf{可测},且$E=\bigcup^n_{i=1}E_i$,那么称集族$\{E_i\}_{i=1}^n$为可测集$E$的一个\textbf{分划},或者\textbf{可测分划}.
\end{definition}

如果$A=\{E_i\}_{i=1}^n$和$B=\{F_i\}_{i=1}^m$都是$E$的分划,那么易证$C=\{E_i\cap F_j|E_i\in A, F_j\in B\}$也是$E$的分划.称$C$是分划$A$和$B$的\textbf{合并}.

容易看到,$C$中存在每一个$E_i$的分划$\{E_i\cap F_j\}_{j=1}^m$,类似地也存在每一个$F_j$的分划,像是更细一层地进行分划.因此,如果分划$C$是$A$和另一个分划的合并,我们就称$C$是比$A$\textbf{更细}的分划,反过来$A$比$C$\textbf{更粗}.

\begin{definition}{上和与下和}

设$f$是$E$上的非负可测函数,$D=\{E_i\}_{i=1}^n$是$E$的一个可测分划.定义$a_i=\inf_{x\in E_i}f(x)$,$A_i=\sup_{x\in E_i}f(x)$,则称
\begin{equation}
s_D=\sum_{i=1}^n a_i \opn{m}E_i
\end{equation}
为$f$在$E$上关于分划$D$的\textbf{下和},而称
\begin{equation}
S_D=\sum_{i=1}^n A_i \opn{m}E_i
\end{equation}
为$f$在$E$上关于分划$D$的\textbf{上和}.

\end{definition}

如果$A\subseteq B\subseteq E$,那么显然$f$在$A$上的上确界要小于等于在$B$上的上确界,在$A$上的下确界要大于等于在$B$上的下确界,因此容易得出以下引理:

\begin{lemma}{}\label{Lebes1_lem1}

设$f$是可测集$E$上的可测函数,$A$和$B$是$E$的可测分划,且$A$比$B$更细.那么

\begin{equation}
s_B\leq s_A\leq S_A\leq S_B
\end{equation}


\end{lemma}

由此可得一个有用的推论:

\begin{corollary}{}\label{Lebes1_cor1}
设$f$是可测集$E$上的可测函数,$D_1$和$D_2$是$E$的可测分划.那么

\begin{equation}
s_{D_i}\leq S_{D_j}
\end{equation}
对任意$i, j\in\{1, 2\}$成立.

\end{corollary}

就是说,不管怎么求分划,任意两个分划之间,上和一定大于等于下和,不会出现一个分划的下和大于另一个分划的上和这种情况.






\subsection{积分}

有了分划和上下和的概念,我们描述起积分就方便多了.

\begin{definition}{}

设$f$是可测集$E$上的可测函数,$\Lambda$是$E$的一切可能的分划之集合.那么称
\begin{equation}
\overline{\int_E} f(x) \dd x=\inf_{D\in \lambda} \{S_D\}
\end{equation}
为$f$在$E$上的\textbf{上积分},称
\begin{equation}
\underline{\int_E} f(x) \dd x=\sup_{D\in \lambda} \{s_D\}
\end{equation}
为$f$在$E$上的\textbf{下积分},



\end{definition}

简而言之,上积分就是上和的下确界,下积分就是下和的上确界.

显然,简单函数的上积分和下积分总是相等,也就是\textbf{可测函数的结构}\upref{MsbFSt}中\autoref{MsbFSt_eq1}~\upref{MsbFSt}所定义的简单函数的Lebesgue积分.

函数的上下积分有以下重要的性质:

\begin{theorem}{}\label{Lebes1_the1}

设$f$、$g$是可测集$E$上的可测函数,$\{E_1, E_2\}$是$E$的某个分划.

\begin{enumerate}
  \item 若$f\leq g$几乎处处成立,则有
  \begin{equation}
  \underline{\int_E} f(x) \dd x \leq \underline{\int_E} g(x) \dd x \leq \overline{\int_E} f(x) \dd x \leq \overline{\int_E} g(x) \dd x
  \end{equation}
  \item 若$\{E_1, E_2\}$是$E$的某个分划,则
  \begin{equation}\label{Lebes1_eq1}
  \underline{\int_E} f(x) \dd x=\underline{\int_{E_1}} f(x) \dd x+\underline{\int_{E_2}} f(x) \dd x
  \end{equation}
  且
  \begin{equation}\label{Lebes1_eq4}
  \overline{\int_E} f(x) \dd x=\overline{\int_{E_1}} f(x) \dd x+\overline{\int_{E_2}} f(x) \dd x
  \end{equation}
  \item 恒有
  \begin{equation}\label{Lebes1_eq5}
  \underline{\int_E} \qty(f(x)+g(x)) \dd x \geq \underline{\int_E} f(x) \dd x+\underline{\int_E} g(x) \dd x
  \end{equation}
  和
  \begin{equation}\label{Lebes1_eq6}
  \overline{\int_E} \qty(f(x)+g(x)) \dd x \leq \overline{\int_E} f(x) \dd x+\overline{\int_E} g(x) \dd x
  \end{equation}
\end{enumerate}



\end{theorem}

\textbf{证明}:

1.太过显然,在此从略\footnote{若不是那么显然,可留言,笔者视情况补充证明细节.}.

2. 为方便,记$A$为对$E$进行任意分划后求上和的结果的集合;$A'$为先将$E$作分划$\{E_1, E_2\}$后,再分别对这两个$E_i$作分划后求上和,将两个上和相加后,所得值的集合.这样,按定义,\autoref{Lebes1_eq4} 的左边就是$A$的下确界,右边就是$A'$的下确界.

\autoref{Lebes1_eq4} 左边是对$E$作分划,右边则是先将$E$作分划$\{E_1, E_2\}$后,再分别对这两个$E_i$作分划,因此可知$A'\subseteq A$,故必有\footnote{子集的下确界大于等于母集的下确界.}

 \begin{equation}\label{Lebes1_eq2}
  \underline{\int_E} f(x) \dd x \leq \underline{\int_{E_1}} f(x) \dd x+\underline{\int_{E_2}} f(x) \dd x
  \end{equation}

另一方面,由\autoref{Lebes1_lem1} ,可知任取$A$中一个数字$a$,都必有$A'$中的数字$a'$,使得$a'\leq a$.因此又有

\begin{equation}\label{Lebes1_eq3}
\underline{\int_E} f(x) \dd x \geq \underline{\int_{E_1}} f(x) \dd x+\underline{\int_{E_2}} f(x) \dd x
\end{equation}

综合\autoref{Lebes1_eq2} 和\autoref{Lebes1_eq3} 即得\autoref{Lebes1_eq4} .

由上下和与上下积分定义的对偶性,可直接推得\autoref{Lebes1_eq1} .由此得证.

3. 只需证明\autoref{Lebes1_eq5} 即可,之后可由对偶性直接推知\autoref{Lebes1_eq6} .

考虑任意可测集$E_i\subseteq E$上的$f$和$g$,则由加法和下确界的定义直接可得“$f$的下确界加$g$的下确界\textbf{小于等于}$f+g$的下确界”.

于是,对于$E$的任意分划$D$,总存在两个分划$D_1$和$D_2$\footnote{直接取$D_1=D_2=D$就行,更细当然更好.},使得“$f$对于$D_1$计算出来的下和加上$g$对于$D_2$计算出来的下和”,\textbf{大于等于}“$f+g$对于$D$计算出来的下和”.因此,前者的上确界大于等于后者的下确界,也即\autoref{Lebes1_eq5} .

\textbf{证毕}.


\autoref{Lebes1_the1} 所描述的性质是非常符合直觉的.类比Riemann积分的定义过程,我们也希望上下积分相等,从而成为新的积分定义.事实上,可测函数就具有这样优良的性质.

\begin{theorem}{}\label{Lebes1_the2}
设$E\subseteq \mathbb{R}^n$是\textbf{测度有限}的可测集,$f$是其上\textbf{非负有界}函数,那么
\begin{equation}\label{Lebes1_eq7}
\overline{\int_E} f(x) \dd x = \underline{\int_E} f(x) \dd x
\end{equation}
\end{theorem}
的\textbf{充要条件}是$f$为\textbf{可测函数}.


\textbf{证明}:

\textbf{充分性}:

设$f$是\textbf{非负有界}的\textbf{可测}函数.

由\autoref{Lebes1_cor1} ,必有
\begin{equation}
\overline{\int_E} f(x) \dd x \geq \underline{\int_E} f(x) \dd x
\end{equation}

因此,接下来只需要证明:对于任意$\epsilon>0$,总存在一个分划$D$,使得$s_D\geq S_D-\epsilon$.

设$\opn{m}E=c$,由题设知$c<+\infty$.又因为$f$非负有界,不妨设$f(x)\in [0, s)$.

将$[0, s)$拆分为一系列区间$A_{k, i}=[\frac{i}{k}s, \frac{i+1}{k}s)$的不交并,其中$k$是任意给定的正整数,$i$是取值范围为$[0, k)$的整数.

利用区间$A_{k, i}$来对$E$进行分划:$E_{k, i}=\{x\in E|f(x)\in A_{k, i}\}$.显然,固定$k$时,各$E_{k, i}$构成$E$的一组分划.

对于任意固定的$k$,在每个$E_{k, i}$上,$f$的上确界和下确界之差\textbf{小于等于}$s/k$,而各$E_{k, i}$的外测度之和为$c$.因此,该固定的$k$按上述方式决定的分划下,$f$在$E$上的上和与下和之差\textbf{小于等于}$sc/k$.

因此,只需要取$k>sc/\epsilon$,所得分划就是所要的$D$.

\textbf{必要性}:

设$f$\textbf{非负有界},且\autoref{Lebes1_eq7} 式成立.

\addTODO{笔者不明白这里为什么需要必要性.按理说只有可测函数才能定义上和与下和、进而有上下极限的概念啊?没有可测条件谈什么\autoref{Lebes1_eq7} 的存在性?更不用说成立了.}

% 参考定理的位置:江泽坚《实变函数论》P122 定理2


\textbf{证毕}.

\autoref{Lebes1_the2} 告诉我们,对于测度有限的$E$上的非负可测函数$f$,其上下积分时相等的,于是我们就可以把它们统一称为“积分”,记为
\begin{equation}
\int_E f(x) \dd x
\end{equation}
进一步,由\autoref{Lebes1_eq5} 和\autoref{Lebes1_eq6} 可知,对于可测函数$f$,有
\begin{equation}
\int_E [f(x)+g(x)] \dd x=\int_E f(x) \dd x+\int_E g(x) \dd x
\end{equation}

然而,\autoref{Lebes1_the2} 讨论的是“有界”的可测函数,颇有限制.任意的非负函数有没有类似的性质呢?我们没法套用\autoref{Lebes1_the2} 的证明方式,因为失去了有界性就无法用同样的方法对$E$进行\textbf{有限}划分了.不过,回想一下\autoref{MsbFSt_the1}~\upref{MsbFSt}是怎么证明的,你会发现我们可以用同样的思路来从有界推广到无界.

设$f$是测度有限的可测集$E$上的非负可测函数.对于任意正整数$k$,定义一个$E$上的新函数$f_k$如下:$f_k(x)=\min \{k, f(x)\}$.直观来说,$f_k$就像是用一根长棍子去“压”$f$,把$k$以上的部分全都压平到$k$的高度.这样,每个$f_k$都是非负有界的可测函数,它们都是有积分的了.于是,我们可以定义$f$的积分为:
\begin{equation}
\int_E f(x) \dd x = \lim\limits_{k\to\infty} \int_E f_k(x) \dd x
\end{equation}







