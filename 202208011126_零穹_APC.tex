% 布置、排列、组合
% 布置|排列|组合

\begin{issues}
\issueDraft
\end{issues}

\pentry{计数原理\upref{ProfPM}}
组合学的一个重要方面在与计数问题.实际上,在很长一段时间里,大多数数学工作者把组合学与“计数”当作一回事.“计数”往往是指找出进行某个确定的运算方法的个数,与之紧密联系的内容便是排列组合.

“排列”和“组合”这两个术语在高中部分已经熟悉.本节将给予这些概念于一个更完整、更直观的描述,这意味着要建立一些术语,而这些术语给出了这种直观的意义.排列组合在很多方面都有应用,本节的术语将使得这种应用更加直观.
\subsection{布置Arrangement}
排列组合的很多应用在于将一定数量的“物体”放置在一定数量的“房间”(盒子)中,下面的定义使得这种应用更加明显.
\begin{definition}{物体、房间、布置}\label{APC_def2}
设 $X,Y$ 是有限集,把 $X$ 的元素叫做\textbf{物体},把 $Y$ 的元素叫\textbf{房间}.映射 $f:X\rightarrow Y$ 称为\textbf{物体集合 $X$ 到房间集合 $Y$ 的一种\textbf{布置}(arrangement)}.
\end{definition}
物体集合 $X$ 到房间集合 $Y$ 的映射 $f:X\rightarrow Y$ 使得 $y_i$ 与 $X$ 的一些元素构成的集合
\begin{equation}
\{x|x\in X,f(x)=y_i\}
\end{equation}
相对应.称 $f$ 为布置暗示着 $f$ 将 $X$ 中的物体放置或分配到 $Y$ 中的房间内.

和很多命名一样,关于集合的元素是为了方便人们的记忆和应用.比如研究矢量的集合称为矢量空间,集合中的元素称为矢量;研究点集拓扑学的集合称为拓扑空间,其上的元素称为点.对应的,与组合数学对应的集合分别称为物体(定义域)和房间(值域),而物体到房间的映射称为布置.

当 $\abs{X}=n,\abs{Y}=m$ 时,每个函数 $f$ 和一个字符串“$f(x_1)\cdots f(x_n)$” 相对应.往往字符串代表着一个字,比如字 word 是由字母 “w”,“o”,“r”,“d” 构成的, 下面的定义给出了这种直观上的意义.
\begin{definition}{字母,字}\label{APC_def1}
设 $X,Y$ 分别是基数为 $\abs{X}=n,\abs{Y}=m$ 的有限集, $f:X\rightarrow Y$ 是 $X$ 到 $Y$ 上的映射,则称 $Y$ 的元素为\textbf{字母}, $f$ 是由 $Y$ 中的字母形成的长度为 $n$ 的一个\textbf{字} $f(x_1)\cdots f(x_n)$.
\end{definition}
上面定义中,$X$ 可看成给出了字的一个顺序.

通过这些概念,得到以下一些定理.

\begin{theorem}{}\label{APC_the1}
设 $X,Y$ 分别是基数为 $\abs{X}=n,\abs{Y}=m$ 的有限集,则映射 $f:X\rightarrow Y$ 的个数等于 $m^n$.
\end{theorem}
\textbf{证明:}由\autoref{APC_def1} ,$f$ 的个数等于由 $Y$ 中的字母拼成的长度为 $n$ 的字的个数.因为字的第一个字母有 $m$ 种选法,第二个,$\cdots$ ,第 $n$ 字母也有 $m$ 种选法.由乘法原则(\autoref{ProfPM_sub1}~\upref{ProfPM}),结果字的个数等于 $\underbrace{m\cdots m}_{n\text{个}}=m^n$.

\textbf{证毕!}

将上述定理换成\autoref{APC_def2} 的语言,得到:
\begin{theorem}{}
$n$ 个物体的集合在 $m$ 个房间的集合里的布置个数为 $m^n$.
\end{theorem}

\autoref{APC_the1} 就是不管集合 $X,Y$ 是否有限,所有的映射 $f:X\rightarrow Y$ 的集合常记作 $Y^X$ \upref{Topo8}的原因.

\begin{definition}{有序布置}
设物体集合 $X$ 被布置在房间集合 $Y$ 里,且每个房间可容纳 $X$ 中任意个物体,若改变房间中物体的顺序得到的布置不同,则称这样的布置叫作房间里的\textbf{有序布置(ordered arrangement)}.$n$ 个物体在 $m$ 个房间里有序布置的个数记作 $[m]^n$.
\end{definition}
\begin{theorem}{}
$n$ 个物体在 $m$ 个房间里有序布置的个数 $[m]^n$ 为
\begin{equation}
[m]^n=m(m+1)\cdots(m+n-1)
\end{equation}
\end{theorem}
\textbf{证明:}
记 $T_n$ 为 $n$ 个物体在 $m$ 个房间里的所有有序布置的集合.设 $T_{n-1}$ 是前 $n-1$ 个物体在 $m$ 个房间里的所有有序布置的集合,则 $T_n$ 可以这样构造:将第 $n$ 个物体 $x_n$ 插入到 $T_{n-1}$ 的每一有序布置中得到.

由于 $T_{n-1}$ 的每一有序布置都可用 $n-1$ 个物体和 $m-1$ 个分隔符表示,其中 $m-1$ 个分隔符将 $n-1$ 个物体分为 $m$ 段,第 $i$ 段对应第 $i$ 个房间内物体的一个有序布置.因此可以用 $(n-1)+(m-1)=n+m-2$ 个记号的序列来表示 $T_{n-1}$ 的一个布置,这些记号标记了这 $n-1$ 个物体和 $m-1$ 个分隔符. 

将 $x_n$ 加入到 $T_{n-1}$ 的每个有序布置中,其可以放在每个记号标记的物体或房间前后,每一种方法都对应 $T_{n}$ 中的不同有序布置.所以 $T_{n-1}$ 的每个有序布置都可产生 $n+m-2+1=n+m-1$ 个 $T_n$ 的有序布置.于是
\begin{equation}
\abs{T_n}=(m+n-1)\abs{T_{n-1}}
\end{equation}
重复这种关系,可得到
\begin{equation}
\abs{T_n}=(m+n-1)(m+n-2)\cdots (m+1)\abs{T_1}
\end{equation}
由于单个物体在 $m$ 个房间的有序布置个数 $\abs{T_1}=m$,所以
\begin{equation}
[m]^n=\abs{T_n}=m(m+1)\cdots(m+n-1)
\end{equation}

\textbf{证毕!}
\subsection{排列}
\begin{theorem}{}
设 $X,Y$ 分别是基数为 $\abs{X}=n,\abs{Y}=m$ 的有限集,则单射 $f:X\rightarrow Y$ 的个数为 $m(m-1)\cdots(m-n+1)$.
\end{theorem}
\textbf{证明:}函数 $f$ 的单一性由字的字母是两两不同的这个事实表示.因为字的第一个字母有 $m$ 种选法,第二个字母有 $m-1$ 种选法,$\cdots$ ,第 $n$ 个字母有从剩下的 $m-n+1$ 个字母中有 $m-n+1$ 中不同选法.由乘法原则,由 $m$ 元集 $Y$ 的 $n$ 个不同字母构成的字的个数为 $m(m-1)\cdots(m-n+1)$.

\textbf{证毕!}

同样的,上述定理相当于:
\begin{theorem}{}\label{APC_the2}
将 $n$ 个物体的集合在 $m$ 个房间的集合里进行布置,要求每个房间至多包含一个物体的布置个数为 $m(m-1)\cdots(m-n+1)$.
\end{theorem}

\begin{definition}{排列,全排列}
从 $m$ 元集 $Y$ 的 $n$ 个不同字母构成的字叫作从集合 $Y$ 的 $m$ 个物体每次取 $n$ 个的\textbf{排列}(\textbf{permutation})(或 $m$ 元集 $Y$ 的 $n$ 排列),其个数记为 $[m]_n$ (或$P(m,n)$),其中 $m\geq n$.若 $m=n$,则 $[n]_n$ 也记作 $P_n$ 或 $n!=1\cdot2\cdots n$,此时的排列称 $n$元集 $Y$ 的\textbf{全排列}.
\end{definition}
由\autoref{APC_the2} 
\begin{equation}
P(m,n)=m(m-1)\cdots(m-n+1)
\end{equation}

\subsection{组合}

\begin{definition}{增字}
若集合 $Y=\{y_1,y_2,\cdots,y_m\}$ 是个偏序集\autoref{Relat_sub2}~\upref{Relat},那么称由 $Y$ 的 $n$ 个字母构成的字 $y_{i1}y_{i2}\cdots y_{in}$ 为\textbf{增字},如果 $y_{i1}\leq y_{i2}\leq\cdots\leq y_{in}$.若 $y_{i1}<y_{i2}<\cdots<y_{in}$,则称字 $y_{i1}y_{i2}\cdots y_{in}$ 为\textbf{严格增字}.
\end{definition}
\begin{theorem}{}
从 $m$ 个字母中选出 $n$ 个构成的增字个数等于 $\frac{[m]^n}{n!}$.
\end{theorem}
\textbf{证明:}下面将增字和有序排列联系起来:若房间 $y_i$ 里有选自集合 $\{x_1,x_2,\cdots,x_n\}$ 的 $p_i$ 个物体,在构造增字里这个房间 $y_i$ 就写 $p_i$ 次:$\underbrace{y_i\cdots y_i}_{p_i\text{个}}$;
若房间 $y_i$ 不包含物体,$y_i$ 就不出现在增字中.例如(不失一般性设 $y_1\leq y_2\leq\cdots$):
\begin{equation}
\begin{aligned}
&x_2| x_1 x_3 x_6|x_4 x_3|\quad|x_5\cdots\\
&y_1\quad y_2\qquad y_3\quad y_4\quad y_5
\end{aligned}
\end{equation}
对应增字
\begin{equation}
y_1y_2y_2y_2y_3y_3y_5\cdots
\end{equation}

这一构造说明哪些物体放在房间里对增字没有影响,而只有房间里的物体个数有意义.因此若 $n$ 个物体用 $n!$ 不同的方法排列,对应的增字仍相同(因为同一物体对应的房间是不变的, $n$ 个物体重新排列不会新增和减少物体,那么每个房间里的物体数仍不变,或者可这样理解: $n$ 个物体重新排列相当于对应的增字里的字母的重新排列,而打乱字母并不会改变增字).

\textbf{证毕!}