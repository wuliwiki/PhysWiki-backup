% 无理数(数论)
% keys 无理数
% license Usr
% type Tutor

\begin{definition}{有理数与无理数}
能表示为互素的整数 $a$ 与 $b\neq 0$ 的比值 $a/b$ 的数称为\textbf{有理数}。

非有理数,也就是不能表示为互素整数 $a$ 与 $b\neq 0$ 的比值 $a/b$ 的数称为\textbf{无理数}。
\end{definition}

\begin{theorem}{}
$\sqrt{2}$ 无理。
\end{theorem}
\textbf{证明}:假设若 $\sqrt 2$ 有理,则对于 $\sqrt 2 = a/b$,即方程 $a^2=2b^2$ 有整数解,且 $(a, b) = 1$。
而注意到等式右侧是 $2$ 的整数倍,是偶数,故 $a$ 也是偶数,即 $a=2c$,这会使得 $(2c)^2 = 2b^2$,可化为 $2c^2=b^2$,同样是的 $b$ 是偶数而 $a, b$ 就有公约数 $2$,与 $(a, b) = 1$ 矛盾!得证。

\begin{theorem}{}
对于整数 $N$,$\sqrt[m]{N}$ 无理,除非 $N$ 是某整数 $n$ 的 $m$ 次方。
\end{theorem}
\textbf{证明}:仍考虑 $\sqrt[m]{N} = a/b$,从而可化为
\begin{equation}
a^m = N b^m, ~ (a, b) = 1 ~,
\end{equation}
而