% Jordan 引理
% 复变函数|围道积分|约尔当引理|若尔当引理|若当引理|Jordan's lemma

\pentry{留数定理\upref{ResThe}}

Jordan 引理可以结合\textbf{留数定理}\upref{ResThe},利用围道积分来处理一些复杂的实函数定积分.

\begin{theorem}{Jordan 引理}
如果$g(z)$是一个连续的\footnote{更一般地,只要要求存在一个半径$r$,使得$g$在“复平面的上半平面”和“以$r$为半径的圆弧之外的区域”的交集里连续,即可.}函数,且总有$\lim\limits_{\abs{z}\to\infty}g(z)=0$,那么对于任何正实数$a$就有
\begin{equation}
\lim\limits_{R\to \infty}\int_{C(R)}g(z)\E^{az\I}\dd z=0
\end{equation}
其中$C(R)$是半径为$R$的半圆弧路径,圆心为原点,坐落于复平面上半平面,路径方向顺逆时针都一样.


\end{theorem}

\textbf{证明}:

$C(R)$上从$\theta_1$到$\theta_2$的一段圆弧上的积分为:

\begin{equation}
\begin{aligned}
\int_{C(R)|_{\theta_1}^{\theta^2}}g(z)\E^{az\I}\dd z&=\int_{\theta_1}^{\theta_2}g( R\E^{\theta\I} )\E^{aR\E^{\theta\I}\I}\frac{\dd R\E^{\theta\I}}{\dd \theta}\dd\theta\\
&=R\I\int_{\theta_1}^{\theta_2}g(R\E^{\theta\I})\E^{(aR\E^{\theta\I}+\theta)\I}\dd\theta\\
&=R\I\int_{\theta_1}^{\theta_2}g(R\E^{\theta\I})\E^{(aR\cos\theta+\theta)\I-aR\sin\theta}\dd\theta
\end{aligned}
\end{equation}

考虑复变函数的柯西不等式:$\abs{\int_{C(R)}f(z)\dd z} \leq \int_{C(R)}\abs{f(z)}\dd z$,可知

\begin{equation}
\begin{aligned}
\abs{R\I\int_{\theta_1}^{\theta_2}g(z)\E^{(az+\theta)\I}\dd\theta}=R
\end{aligned}
\end{equation}

\textbf{证毕}.














