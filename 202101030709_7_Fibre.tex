% 纤维丛
% keys 纤维丛|丛空间|拓扑学|拓扑|向量丛|乘积拓扑

\begin{issues}
\issueDraft
\end{issues}

\pentry{乘积拓扑\upref{Topo6},矢量空间\upref{LSpace}}

\subsection{定义}

直观来说,纤维丛是指在一个拓扑空间$B$的每一个点都长出来另一个拓扑空间$F$所得到的一个空间.每一个点$x\in B$上的$F$被称为一根\textbf{纤维(fibre)},这些纤维所在的$B$称为\textbf{底空间(base space)},而整个结构$(B, F)$就是一个\textbf{纤维丛(fibre bundle)}.

准确的定义如下所述,其中$E$就是“$B$上每个点都长出一个$F$的丛空间”:

\begin{definition}{纤维丛}
给定拓扑空间$B$和$F$,如果存在一个拓扑空间$E$和一个连续满射$\pi:E\rightarrow B$,使得对于任意的$x\in B$,都有$\pi^{-1}(x)\cong F$,那么称这个结构$(E, F, B, \pi)$为一个\textbf{纤维丛(fibre bundle)},称$E$是这个纤维丛的\textbf{全空间(total space)},$F$是其\textbf{纤维(fibre)},$B$是其\textbf{底空间(base space)},有时也译作\textbf{基空间}.
\end{definition}

如果把$B$想象成一块土地,$F$想象成一棵草,那么$E$就是“土地上长了一片草”这一概念,$E$的每个元素就是某棵草上的一个点.定义中的连续满射$f$的作用是把这样的一个点映射到相应的草所在的地点.

要注意的是,$E$不完全等同于$B\times F$.对于$B\times F$来说,任意给定两个$x_1, x_2\in B$,我们自然可以找到$x_1\times F$和$x_2\times F$上的一一对应关系,这是由集合笛卡尔积的定义决定的.但是纤维丛$E$上,如果上述$x_1\not=x_2$,那么两个地方长出来的纤维是没有天然的双射对应的的\footnote{在微分几何中,我们研究的切丛是纤维丛的一种,而所谓的“联络”实际上就是指定了不同纤维间的双射.}.这就是“纤维丛”这一名称的深意,而乘积空间应该被想象纤维被粘在一起的情况,只是纤维丛的一个定义了额外联系的特例.

两个纤维丛之间可以有映射偶:

\begin{definition}{纤维丛的态射}
设$(E_1, F_1, B_1, f_1)$和$(E_2, F_2, B_2, f_2)$是两个纤维丛,
\end{definition}

\subsection{纤维丛的例子}






