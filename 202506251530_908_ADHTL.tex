% 爱德华·泰勒(综述)
% license CCBYSA3
% type Wiki

本文根据 CC-BY-SA 协议转载翻译自维基百科 \href{https://en.wikipedia.org/wiki/Edward_Teller}{相关文章}。

\begin{figure}[ht]
\centering
\includegraphics[width=6cm]{./figures/aa578cdb3e82db6d.png}
\caption{} \label{fig_ADHTL_1}
\end{figure}
爱德华·泰勒(Edward Teller,匈牙利语:Teller Ede,1908年1月15日-2003年9月9日)是一位匈牙利裔美国理论物理学家和化学工程师,因其在氢弹发展中的关键角色而被俗称为“氢弹之父”。他是基于斯坦尼斯瓦夫·乌拉姆设计提出的“泰勒–乌拉姆构型”的共同发明人之一。

泰勒性格激烈,据称“受到百万吨级爆炸梦想的驱使,有救世主情结,展现出专断的行为风格”。\(^\text{[1]}\)他曾设计过一种名为“闹钟模型”的热核炸弹,其爆炸当量高达1000兆吨(即10亿吨TNT),并建议通过船只或潜艇投送。这种武器将具备焚毁一个大陆的能力。\(^\text{[1]}\)

泰勒于1908年出生于奥匈帝国,20世纪30年代移民至美国,是一批被称为“火星人”的匈牙利科学家移民中的一员。他在核物理、分子物理、光谱学以及表面物理等领域作出了诸多贡献。他对恩里科·费米的β衰变理论的拓展,以“伽莫夫–泰勒跃迁”的形式,为该理论的实际应用奠定了重要基础;而“杨–泰勒效应”和“布鲁瑙尔–埃米特–泰勒理论”(Brunauer–Emmett–Teller theory,简称 BET 理论)至今仍以原始形式被广泛应用,是物理与化学领域的核心理论之一。\(^\text{[2]}\)泰勒惯于以基础物理原理思考问题,常常与同行讨论,以突破难题。这一特点在他与斯坦尼斯瓦夫·乌拉姆共同设计可行的热核聚变炸弹方案时表现得尤为明显。然而,后来他在性格上却否定了乌拉姆所起的关键作用。赫伯特·约克指出,泰勒实际上是利用了乌拉姆提出的“压缩与加热启动热核聚变”的基本思想,绘制出他自己的“超级炸弹”方案草图。\(^\text{[1]}\)在乌拉姆提出其方案之前,泰勒原始设想的“经典超级炸弹”本质上是一个通过加热未压缩液态氘以期引发持续热核燃烧的系统。\(^\text{[1]}\)这个设想虽然简单,源自基本物理原理,但泰勒对其执着追求的程度极为强烈,即使被证明是错误的,或已有人指出无法实现,他仍不放弃。为了获得华盛顿对其“超级武器”计划的支持,泰勒提出在“绿色屋行动”中进行一次热核辐射内爆实验,即所谓的“乔治”试验。\(^\text{[1]}\)

泰勒对托马斯–费米理论也作出了重要贡献,该理论是密度泛函理论的前身——密度泛函理论是现代量子力学处理中复杂分子的标准工具之一。1953年,泰勒与尼古拉斯·梅特罗波利斯、阿里安娜·罗森布鲁斯、马歇尔·罗森布鲁斯以及奥古斯塔·泰勒共同发表了一篇论文,该论文成为蒙特卡洛方法在统计力学中的应用、以及贝叶斯统计中马尔可夫链蒙特卡洛(MCMC)研究文献的重要起点。\(^\text{[3]}\)泰勒早期即参与了“曼哈顿计划”,该计划研发了世界上第一枚原子弹。他积极推动热核武器的研发,但融合(聚变)型武器最终是在二战后才出现。他共同创立了劳伦斯利弗莫尔国家实验室,并曾担任该实验室的主任或副主任。然而,由于他在其前上司、洛斯阿拉莫斯实验室负责人J·罗伯特·奥本海默的安全审查听证会上发表了有争议的反面对证言,泰勒遭到了科学界的排斥。

泰勒继续获得美国政府和军方科研体系的支持,尤其是在他倡导发展核能、保持强大核武库以及推进积极的核试验计划方面。在晚年,他提出了许多颇具争议的技术方案,以解决军事和民用问题,例如“战车计划”——利用热核爆炸在阿拉斯加开凿一个人工港口的设想,以及支持罗纳德·里根提出的“战略防御倡议”。泰勒曾获得恩里科·费米奖和阿尔伯特·爱因斯坦奖。他于2003年去世,享年95岁。
\subsection{早年生活与学术起步}
爱德·泰勒于1908年1月15日出生在布达佩斯,当时属于奥匈帝国的领土,他出身于一个犹太家庭。父亲米克沙·泰勒是一位律师,母亲伊洛娜(Ilona,娘家姓Deutsch)是一位钢琴家。\(^\text{[4][5][6]}\)他在布达佩斯就读于明塔文理中学。\(^\text{[7]}\)泰勒是不可知论者。他后来写道:“宗教在我家从来不是个问题,事实上,几乎从未被提起过。我唯一的宗教教育,是因为明塔中学规定所有学生必须修习自己所属宗教的课程。我们家只过一个节日——赎罪日,那天我们全家会一起禁食。但我父亲仍会在每个安息日和所有犹太节日为他的父母祈祷。我所理解的上帝,是:如果他真的存在,那太好了;我们非常需要他,但几千年来我们从未见过他。”\(^\text{[8]}\)泰勒学说话较晚,但从小就对数字产生了浓厚兴趣,经常以心算大数作为娱乐。\(^\text{[9]}\)
\begin{figure}[ht]
\centering
\includegraphics[width=6cm]{./figures/953820095e9928c9.png}
\caption{泰勒的青年时期照片} \label{fig_ADHTL_2}
\end{figure}
泰勒于1926年离开匈牙利前往德国,部分原因是米克洛什·霍尔蒂政权实施的歧视性“入学限额”政策。青少年时期匈牙利的政治动荡与革命,使他心中长期对共产主义和法西斯主义怀有敌意。\(^\text{[10]}\)

1926年至1928年间,泰勒在卡尔斯鲁厄大学学习数学与化学,并获得化学工程理学学士学位。\(^\text{[11][12]}\)他曾表示,使他转向物理学的关键人物是来访教授赫尔曼·马克(Herman Mark)。\(^\text{[13]}\)在聆听马克关于分子光谱学的讲座后,泰勒意识到,是物理学中的全新思想正在根本性地改变化学的前沿。\(^\text{[14]}\)马克是高分子化学领域的专家,而该领域对于理解生物化学至关重要;他还向泰勒介绍了路易·德布罗意等人提出的量子物理领域的最新突破。正是这些讲座激发了泰勒改学物理的强烈动机。\(^\text{[15]}\)当他将这一决定告诉父亲时,父亲感到十分担忧,特地前往学校拜访他并与教授们面谈。化学工程学位可为毕业生在化工企业提供一条稳定且高薪的职业路径,而物理学则没有如此清晰的职业前景。虽然泰勒并未得知父亲与教授们具体谈了什么,但最终的结果是他获得了父亲的许可,得以追随自己成为物理学家的梦想。\(^\text{[16]}\)

随后,泰勒进入慕尼黑大学,师从著名物理学家阿诺德·索末菲学习物理。1928年,他仍是慕尼黑大学的学生时,不幸被一辆有轨电车撞倒,右脚几乎被截断。此后终生,他走路都一瘸一拐,有时还需佩戴义肢。\(^\text{[17][18]}\)他所服用的止痛药干扰了思考能力,于是他决定停药,转而凭借意志力来对抗疼痛。他甚至利用安慰剂效应来自我暗示——让自己相信喝下的是止痛药而不是水。\(^\text{[19]}\)物理学家维尔纳·海森堡曾评价说,泰勒之所以能很好地应对这场事故,靠的并不是单纯的坚忍,而是他精神上的强韧与顽强。\(^\text{[20]}\)
\begin{figure}[ht]
\centering
\includegraphics[width=6cm]{./figures/cab8a66db84879d1.png}
\caption{泰勒1935年入境美国时携带的匈牙利护照。} \label{fig_ADHTL_3}
\end{figure}
1929年,泰勒转学至莱比锡大学,并于1930年在海森堡的指导下获得物理学博士学位。他的博士论文是关于氢分子离子的量子力学处理,属于该领域最早的精确研究之一。同年,他结识了俄国物理学家乔治·伽莫夫和列夫·朗道。此外,捷克物理学家乔治·普拉切克与泰勒的终生友谊,也对他在科学与哲学方面的发展产生了重要影响。正是普拉切克促成了泰勒于1932年前往罗马与恩里科·费米共度的一个暑期,从而使泰勒的科研方向转向核物理。\(^\text{[21]}\)同样在1930年,泰勒前往哥廷根大学进修。由于马克斯·玻恩和詹姆斯·弗兰克的存在,该校当时是世界物理研究的重镇。\(^\text{[22]}\)但随着1933年1月阿道夫·希特勒出任德国总理,德国对犹太人而言变得不再安全。泰勒在国际救援委员会的协助下离开德国。\(^\text{[23]}\)他短暂前往英国,随后移居哥本哈根一年,期间在尼尔斯·玻尔的指导下工作。\(^\text{[24]}\)1934年2月,他与多年的恋人奥古斯塔·玛丽亚·“米齐”(发音为“米茨”)·哈尔卡尼结婚,米齐是他朋友的妹妹。由于米齐信奉加尔文教,婚礼也在一座加尔文教堂中举行。\(^\text{[20][25]}\)1934年9月,泰勒返回英国。\(^\text{[26][27]}\)

米齐曾在匹兹堡留学,她一直希望能重返美国。这个机会在1935年来临——在乔治·伽莫夫的推荐下,泰勒受邀前往美国,成为乔治·华盛顿大学的物理学教授,自此他与伽莫夫共事至1941年。\(^\text{[28]}\)1937年,泰勒在乔治·华盛顿大学提出了“扬-泰勒效应”,这一效应描述了在某些情况下分子的几何结构会发生扭曲;该效应影响金属的化学反应,尤其是在某些金属染料的颜色变化方面具有重要意义。\(^\text{[29]}\)泰勒与赫尔曼·亚瑟·扬(Hermann Arthur Jahn)对该效应进行了纯粹数学物理层面的分析。与此同时,泰勒还与斯蒂芬·布鲁瑙尔(Stephen Brunauer)和保罗·休·埃米特(Paul Hugh Emmett)合作,在表面物理和化学领域作出了重要贡献——他们共同提出了著名的布鲁瑙尔–埃米特–泰勒(BET)等温吸附模型。\(^\text{[30]}\)1941年3月6日,泰勒与米齐正式归化为美国公民。\(^\text{[31]}\)

在乔治·华盛顿大学任职期间,泰勒与伽莫夫每年共同组织“华盛顿理论物理会议”(1935–1947),吸引了众多顶级物理学家参会。\(^\text{[32]}\)

二战爆发后,泰勒希望为战争贡献力量。在知名航空动力学家、同为匈牙利移民的西奥多·冯·卡门(Theodore von Kármán)建议下,泰勒与好友汉斯·贝特(Hans Bethe)合作,共同研究冲击波传播理论。多年后,他们对冲击波后方气体行为的解释,对导弹再入研究的科学家而言具有重要价值。\(^\text{[33]}\)
\subsection{曼哈顿计划}
\subsubsection{洛斯阿拉莫斯实验室}
\begin{figure}[ht]
\centering
\includegraphics[width=6cm]{./figures/4bac35635142d67a.png}
\caption{泰勒在洛斯阿拉莫斯的身份证件照片} \label{fig_ADHTL_4}
\end{figure}
1942年,泰勒受邀参加由罗伯特·奥本海默在加利福尼亚大学伯克利分校主持的夏季规划研讨会,讨论曼哈顿计划的起始方向——即美国研发第一批核武器的努力。就在几周前,泰勒曾与他的朋友兼同事恩里科·费米会面,探讨核战争的前景。费米漫不经心地提出,也许可以利用核裂变武器引发更大规模的核聚变反应。虽然泰勒最初向费米解释了他认为这种设想行不通的理由,但他很快就被这个可能性所吸引,并对“仅仅制造”一枚原子弹感到无聊——尽管这项工作当时还远未完成。在伯克利的会议上,泰勒将讨论的焦点从裂变武器引向了聚变武器的可能性——他称之为“超级炸弹”,这是氢弹的一种早期设想。\(^\text{[34][35]}\)

芝加哥大学物理系主任阿瑟·康普顿负责协调哥伦比亚大学、普林斯顿大学、芝加哥大学和加州大学伯克利分校的铀研究工作。为了消除分歧和重复,康普顿将科学家们集中调往芝加哥的冶金实验室。\(^\text{[36]}\)尽管泰勒和米茜此时已是美国公民,但由于他们在敌对国家仍有亲属,泰勒最初并未前往芝加哥。\(^\text{[37]}\)1943年初,新墨西哥州洛斯阿拉莫斯实验室开始建设。奥本海默担任该实验室的主任,其任务是设计原子弹。泰勒于1943年3月搬到了那里。\(^\text{[38]}\)在洛斯阿拉莫斯,泰勒因深夜弹钢琴而惹恼了邻居。\(^\text{[39]}\)

泰勒加入了理论(T)部。\(^\text{[40][41]}\)他获得了一个秘密身份,名为“埃德·蒂尔登”(Ed Tilden)。\(^\text{[42]}\)他对自己未被任命为该部门负责人感到不满,该职位最终由汉斯·贝特担任。奥本海默让他研究构建裂变武器的一些非常规方法,比如“自催化”机制,即在核链式反应进行过程中,炸弹的效率会提升,但这一方法被证明并不现实。\(^\text{[41]}\)他还研究了使用氢化铀代替金属铀,但其效率被发现“微不足道甚至更差”。\(^\text{[43]}\)尽管在战争期间氢弹的开发被列为低优先级(因为裂变武器的研制已足够困难),泰勒仍坚持推进自己的聚变武器设想。\(^\text{[40][41]}\)在一次前往纽约的途中,他请玛丽亚·哥珀特-梅耶为他计算“超级炸弹”的相关数据。她证实了泰勒自己的结论:Super 不会奏效。\(^\text{[44]}\)

1944年3月,在泰勒的领导下成立了一个特别小组,负责研究内爆式核武器的数学模型。\(^\text{[45]}\)该小组同样遇到了不少困难。由于泰勒对“超级炸弹”的兴趣,他在内爆计算上的投入不如贝特所期望的那么多。这项任务起初也是低优先级,但随着埃米利奥·塞格雷小组发现钚存在自发裂变现象,内爆式核弹的重要性大大提升。1944年6月,应贝特的请求,奥本海默将泰勒调出T部门,让他直接向自己汇报,领导一个专责于“超级炸弹”的特别小组。他的职位由英国代表团的鲁道夫·佩耶尔斯接替,后者又引入了后来被揭露为苏联间谍的克劳斯·福克斯。\(^\text{[46][44]}\)当恩里科·费米于1944年9月加入洛斯阿拉莫斯实验室时,泰勒的“超级炸弹”小组并入了费米所领导的F部门。\(^\text{[46]}\)该小组成员包括斯坦尼斯拉夫·乌拉姆、简·罗伯格、杰弗里·丘、哈罗德和玛丽·阿戈夫夫妇,\(^\text{[47]}\)以及玛丽亚·哥珀特-梅耶。\(^\text{[48]}\)

泰勒在核弹研究中作出了重要贡献,尤其是在揭示内爆机制方面。他是第一个提出使用实心核心设计(的科学家,该设计最终被证明是成功的。这种设计后来被称为“克里斯蒂核心”,以实现它的物理学家罗伯特·F·克里斯蒂命名。\(^\text{[49][50][51][52]}\)1945年7月的“三位一体”核试验中,泰勒是少数几个戴着护目装备而非背对地面观看核爆的人之一。他后来回忆称,那道核爆闪光“就像我拉开了黑屋的窗帘,阳光顿时洒满房间”。\(^\text{[53]}\)
\subsubsection{投放原子弹的决策}
在首次核武器试验(1945年7月的“三位一体”试验)前后,匈牙利裔物理学家利奥·西拉德发起了一份《西拉德请愿书》,主张应在对日本实际使用原子弹之前进行一次示威性展示,并且认为这种武器不应被用于杀伤人类。针对西拉德的请愿,泰勒向好友罗伯特·奥本海默征询意见。泰勒认为奥本海默天生是个领袖,能够帮助他应对这样棘手的政治问题。奥本海默安慰泰勒说,一个国家的命运应当交由华盛顿那些理性的政治家来决定。在奥本海默的影响下,泰勒最终决定不签署这份请愿书。\(^\text{[54]}\)

因此,泰勒写了一封信作为对西拉德的回应,信中写道:

我并不真正认同你的反对意见。我不认为有任何可能能够禁止某种武器。如果我们还有一丝生存的希望,那就是设法终结战争。武器越具有决定性,它就越肯定会在任何真正的冲突中被使用,而任何协议都无济于事。我们唯一的希望,是将我们的研究成果的真相公之于众。这或许能让所有人认识到下一场战争将是灾难性的。为此目的,实际的作战使用甚至可能是最有效的手段。\(^\text{[55]}\)

在多年后撰写回忆录时反思这封信,泰勒写道:

首先,西拉德是对的。作为那些参与制造原子弹的科学家,我们确实肩负着特殊的责任。其次,奥本海默也是对的。我们对当时的政治局势了解不足,无法形成有效的判断。第三,我们当时本应做、却没有做到的事情是:制定一套技术方案,让原子弹在东京高空(非常高的高度)引爆作为示威用途,并将该方案提交给杜鲁门总统。\(^\text{[56]}\)

泰勒当时并不知道,他的四位同事已被当时仍属秘密的“临时委员会”(1945年5月至6月间)征询意见。正是该组织最终决定了新型武器的最初使用方式。委员会下设一个四人组成的科学顾问小组,由奥本海默领导,该小组的结论是立即在日本进行军事使用是最佳方案:

我们的科学同仁们对于这些武器的初始使用方式并无一致意见:从纯技术性的展示到以军事用途促使日本投降的主张,观点各异……有人强调立即军事使用可挽救美国士兵的生命……我们更倾向于后者的观点;我们无法提出任何技术展示能有效终结战争;我们也看不到直接军事使用之外有可接受的替代方案。\(^\text{[57]}\)

后来,泰勒得知奥本海默曾参与“临时委员会”的决策,并支持了立即将新武器用于军事用途。这与他此前从奥本海默本人那里获得的印象完全相反——当时他就西拉德请愿书向奥本海默请教,奥本海默告诉他国家的命运应交由华盛顿的理性政客决定。泰勒在得知这一事实后,他与奥本海默的师生关系开始恶化。\(^\text{[54]}\)

1990年,历史学家巴顿·伯恩斯坦指出,泰勒声称自己是对使用原子弹“隐秘持异议者”的说法“不具说服力”。\(^\text{[58]}\)在2001年的《回忆录》中,泰勒宣称他确实曾向奥本海默游说反对使用原子弹,但奥本海默劝他不要采取任何行动,并表示科学家应将军事决策交给军方处理;泰勒还声称他当时并不知道奥本海默和其他科学家实际上被征询了关于原子弹使用方式的意见,并暗示奥本海默表现出虚伪的一面。\(^\text{[59]}\)
\subsection{氢弹}
\begin{figure}[ht]
\centering
\includegraphics[width=8cm]{./figures/b92124213a1cc93d.png}
\caption{1946年4月,在曼哈顿工程区赞助的关于“超级炸弹”(氢弹早期概念)的洛斯阿拉莫斯专题讨论会上,几位物理学家合影。前排从左至右为:诺里斯·布拉德伯里、约翰·曼利、恩里科·费米和J.M.B.凯洛格。身穿深色外套、站在曼利身后的是罗伯特·奥本海默;在奥本海默左侧的是理查德·费曼。左侧的军官是奥利弗·海伍德上校。} \label{fig_ADHTL_5}
\end{figure}
尽管诺里斯·布拉德伯里在1945年11月接替奥本海默成为洛斯阿拉莫斯实验室主任后,曾邀请泰勒担任理论(T)部的负责人,泰勒还是于1946年2月1日离开了洛斯阿拉莫斯,回到芝加哥大学担任教授,并成为费米和玛丽亚·格佩特·梅耶的亲密合作者。\(^\text{[60]}\)格佩特·梅耶对元素内部结构的研究最终使她在1963年获得了诺贝尔物理学奖。\(^\text{[61]}\)

1946年4月18日至20日,泰勒参加了在洛斯阿拉莫斯举行的一场会议,会议旨在回顾战时关于“超级炸弹”(Super,即氢弹)计划的研究工作。与会者讨论了热核燃料(如氘)的性质以及氢弹可能的设计方案。会议的结论是:泰勒对氢弹可行性的评估过于乐观。氘的需求量很大,而氘在燃烧过程中的辐射损失也很高,使得其可行性令人怀疑。将昂贵的氚添加进热核混合物可能会降低点火温度,但即便如此,当时没人知道需要多少氚,也不确定即使加入氚是否能促进热量传播。\(^\text{[62][63]}\)

在会议结束时,尽管遭到部分成员(如罗伯特·瑟伯)的反对,泰勒仍提交了一份乐观的报告,称氢弹是可行的,并应鼓励继续开展其研发工作。克劳斯·弗克斯也参加了这次会议,并将相关信息传递给了莫斯科。泰勒还与约翰·冯·诺伊曼一起提出了利用内爆引发“超级炸弹”(Super,即氢弹)的方法。由于泰勒提出的“经典超级炸弹”模型存在极大不确定性,奥本海默后来曾表示,他希望苏联人在研制自己的氢弹时采用这一设计方案,因为这几乎肯定会延缓他们的研发进度。\(^\text{[62]}\)
\begin{figure}[ht]
\centering
\includegraphics[width=10cm]{./figures/ba5f47c69108241a.png}
\caption{泰勒和乌拉姆于1951年3月9日撰写的机密论文:《关于异催化爆轰 I:流体透镜与辐射镜》,他们在文中提出了革命性的新设计——分级内爆,这就是氢弹的核心机密。} \label{fig_ADHTL_6}
\end{figure}
到1949年,苏联支持的政府已经开始在整个东欧掌握政权,并在泰勒的祖国匈牙利成立了傀儡国家“匈牙利人民共和国”(1949年8月20日),当时他的许多家人仍居住在那里。\(^\text{[64]}\)随着苏联在1949年8月29日首次成功引爆原子弹,美国总统哈里·杜鲁门随后宣布启动氢弹的紧急开发计划。\(^\text{[65]}\)

1950年,泰勒重返洛斯阿拉莫斯,参与该项目的研发。他坚持要让更多理论物理学家参与进来,但包括费米和奥本海默在内的许多知名同行坚信氢弹项目在技术上不可行、在政治上也不可取。彼时尚无任何可行的设计方案。\(^\text{[65]}\)然而,参与苏联氢弹研发的苏联科学家宣称他们是独立完成该项目的。\(^\text{[66]}\)
\begin{figure}[ht]
\centering
\includegraphics[width=6cm]{./figures/4ae5010c2ecffdbd.png}
\caption{泰勒–乌拉姆设计将裂变燃料和聚变燃料物理上分开,并利用初级装置产生的X射线通过周围外壳的“反射”来压缩次级装置。} \label{fig_ADHTL_7}
\end{figure}
1950年,波兰数学家斯坦尼斯瓦夫·乌拉姆和其合作者科内利乌斯·埃弗雷特进行的计算,以及费米的验证表明,泰勒此前对氚启动反应所需数量的估计过于保守;即便使用更多氚,聚变过程中的能量损耗也过大,无法维持聚变反应的持续传播。1951年,泰勒和乌拉姆取得突破,并在一篇机密的1951年3月论文《关于异催化爆轰 I:流体动力透镜与辐射反射镜》中提出了一种切实可行的、兆吨级氢弹新设计。这项设计被称为“泰勒–乌拉姆设计”,但关于这一成果中泰勒与乌拉姆各自具体的贡献,公开领域中并无明确结论,自20世纪50年代初以来,在公开与机密讨论中一直存在争议。\(^\text{[67]}\)

在1999年接受《科学美国人》采访时,泰勒对记者说:

“我确实做出了贡献;乌拉姆没有。很抱歉我必须以这样突兀的方式回答这个问题。乌拉姆对旧方法确实感到不满。他带着一个点子的一部分来找我,而那部分内容我早已设想过,却很难让别人听进去。他愿意在那篇论文上署名。但当涉及到为那篇论文辩护、真正投入工作时,他拒绝了。他说,‘我不相信这套方案。’”\(^\text{[10]}\)

这一问题存在争议。贝特早在1952年就认为泰勒对氢弹的发明做出了真正的创新,\(^\text{[68]}\)并在1954年称其工作为“一次天才的闪现”。\(^\text{[69]}\)他在这两次都强调泰勒的作用,目的是指出即便投入更多支持或资金,氢弹的研制也不会因此加快——这一观点遭到泰勒强烈反对。其他与泰勒立场对立的科学家(如J·卡森·马克)则声称,如果没有乌拉姆和其他人的协助,泰勒根本无法接近成功。\(^\text{[70]}\)乌拉姆本人也曾表示,泰勒不过是将他最初的设计“推广化”了而已。\(^\text{[71]}\)
\begin{figure}[ht]
\centering
\includegraphics[width=8cm]{./figures/23b489842b9dc597.png}
\caption{艾维·麦克)“香肠”装置的一景:这是世界上第一枚真正意义上的热核装置,图中可见其仪器设备和低温设备。那些长管道用于测量目的,其作用是在装置引爆瞬间将“初级”和“次级”阶段释放的首波辐射(即“泰勒光”)传输到测量仪器,然后在爆炸中被摧毁。右下角坐着的人用以展示比例。} \label{fig_ADHTL_8}
\end{figure}
这一突破——其细节至今仍属机密——显然在于将核裂变与核聚变部分分离,并利用裂变弹产生的X射线先压缩聚变燃料(通过一种被称为“辐射内爆”的过程),然后再将其点燃。乌拉姆的设想似乎是利用初级裂变弹产生的机械冲击来促使次级聚变,而泰勒很快意识到,来自初级的X射线能以更对称的方式完成这项任务。实验室的一些成员(特别是J·卡森·马克)后来表示,利用X射线的想法最终谁都能想到,泰勒之所以能立即想到这一点,是因为他当时正在为1951年春季的“温室”核试验做准备,实验内容正是研究裂变弹的X射线对氘氚混合物的作用。\(^\text{[67]}\)

普莉西拉·约翰逊·麦克米兰在其著作《J·罗伯特·奥本海默的毁灭:现代军备竞赛的诞生》中写道,泰勒“隐瞒了乌拉姆的作用”,并称只有“辐射内爆”这一部分是泰勒的主意。泰勒甚至拒绝在专利申请上签字,因为那将需要乌拉姆的签字。托马斯·鲍尔斯则写道:“当然,所有的核弹设计者都知道真相,许多人认为泰勒是科学界中最卑劣、最令人鄙视的那种人——窃取功劳者。”\(^\text{[72]}\)

无论所谓的“泰勒–乌拉姆设计”实际包含哪些组成部分,以及各位参与者各自的贡献如何,在该设计提出之后,项目中的科学家们立刻认识到这正是他们长期以来苦苦寻求的答案。那些原本怀疑裂变-聚变武器是否可行的人也转而相信,这仅仅是时间问题,美国和苏联迟早都会研制出多兆吨级的武器。甚至连最初反对该项目的奥本海默也称这个想法为“技术上令人愉悦”。\(^\text{[73]}\)
\begin{figure}[ht]
\centering
\includegraphics[width=8cm]{./figures/7ccbee44e61c945e.png}
\caption{1952年“常春藤·麦克”试验的成功——世界上首次真正意义上的热核爆炸——似乎证明了泰勒长期以来对氢弹主张的正当性。} \label{fig_ADHTL_9}
\end{figure}
尽管泰勒曾参与提出这一设计,并长期倡导这一概念,但他并未被选为氢弹研发项目的负责人(他棘手的性格名声很可能是原因之一)。1952年,他离开洛斯阿拉莫斯,加入了新成立的加州大学辐射实验室利弗莫尔分部,该实验室在很大程度上是因他的推动而设立的。1952年11月1日,在“常春藤麦克”试验中,第一枚采用泰勒–乌拉姆结构的热核武器成功引爆,泰勒因此在媒体上被称为“氢弹之父”。泰勒本人没有前往太平洋试验场观测试验,他称自己在那里“不受欢迎”,于是选择在伯克利通过地震仪观测爆炸结果。\(^\text{[74]}\)

曾有人认为,通过分析此次试验的放射性沉降物,苏联(其氢弹研发由安德烈·萨哈罗夫领导)可能会破译美国的新设计。然而,这一说法后来被苏联的核武研究人员否认。\(^\text{[75]}\)鉴于官方对氢弹发展的高度保密,政府当时并未公开太多信息,媒体常将整个氢弹的设计和研制过程归功于泰勒及其新成立的利弗莫尔实验室,实际上该武器最初是由洛斯阿拉莫斯实验室研制的。\(^\text{[66]}\)

泰勒的许多同事对他似乎乐于将全部功劳归于自己感到不满,尽管他在项目中只是部分贡献者。作为回应,在恩里科·费米的鼓励下,泰勒撰写了一篇题为《众人的工作》的文章,于1955年2月发表在《科学》杂志上,强调氢弹的研制不是他一人之功。后来他在回忆录中写道,自己在1955年的文章中说了一个“善意的谎言”,目的是“安抚被激怒的情绪”,实际上他主张自己才是该发明的真正功臣。\(^\text{[76][77]}\)

泰勒以沉迷于理论上有趣但实际上不可行的项目而闻名(早期“经典”氢弹方案就是其中一个)。\(^\text{[39]}\)关于泰勒在氢弹项目中的工作,汉斯·贝特曾评论道:

“没有人会因为1946年的计算出错而责备泰勒,特别是在洛斯阿拉莫斯当时还没有足够强大的计算机设备。但他在洛斯阿拉莫斯之所以受到指责,是因为他基于极不完整的计算数据,就引导实验室,乃至整个国家,走上了一条冒险的研究之路,而他自己应该很清楚这些计算并不充分。”\(^\text{[78]}\)
\begin{figure}[ht]
\centering
\includegraphics[width=8cm]{./figures/77f959961d09f047.png}
\caption{“麦克试验”的蘑菇云,爆炸当量为1040万吨。} \label{fig_ADHTL_10}
\end{figure}
在曼哈顿计划期间,泰勒主张研发一种使用氢化铀的炸弹,但许多理论物理学同行认为这种方案不太可能奏效。\(^\text{[79]}\)在利弗莫尔实验室,泰勒继续推进氢化铀炸弹的研究,最终的结果是一个哑弹。\(^\text{[80]}\)乌拉姆曾写信给一位同事,提到他和泰勒分享的一个想法时说:“爱德华对此充满热情;这或许正是一个迹象,说明这个想法不会奏效。”\(^\text{[81]}\)费米也曾打趣说,泰勒是他所认识的唯一一个拥有“多重狂热”的“单一狂热者”。\(^\text{[82]}\)

核武器档案网站的卡里·萨布莱特认为,热核武器中“辐射内爆压缩”设计是乌拉姆最早提出的;但另一方面,他也指出,泰勒在1945年首次提出“聚变助爆”的构想却鲜少被认可,而这一技术对当今核武器的小型化与可靠性至关重要,并已在所有现代核武中普遍应用。\(^\text{[83]}\)

20世纪50年代初,爱德华·泰勒在原子能委员会顾问总委员会的一次会议上提出了“日晷计划”,该计划中的炸弹预计爆炸当量为100亿吨TNT,而其对应的“指针计划”则预计为10亿吨TNT。这两种装置最终都没有建造或测试。
\subsection{奥本海默争议}
\begin{figure}[ht]
\centering
\includegraphics[width=6cm]{./figures/9a341eadb1c5d47b.png}
\caption{} \label{fig_ADHTL_11}
\end{figure}
1954年,特勒在奥本海默的安全许可听证会上作证反对奥本海默,因此引发争议。在洛斯阿拉莫斯期间,特勒与奥本海默多次在裂变和聚变研究等问题上发生冲突。在听证会上,特勒是唯一一个表示奥本海默不应获得安全许可的科学界成员。\(^\text{[84]}\)

当原子能委员会(AEC)律师罗杰·罗布在听证会上询问他是否打算“暗示奥本海默博士对美国不忠”时,特勒回答道:

“我并不想暗示那种事情。我知道奥本海默是一个智力非常敏锐且性格复杂的人,我认为如果我试图分析他的动机,那是自以为是且错误的行为。但我一直认为,也现在仍然认为,他对美国是忠诚的。我相信这一点,并且会一直相信,除非有非常确凿的证据证明相反。”\(^\text{[85]}\)

紧接着,他被问及是否认为奥本海默是“安全风险”,对此他作证说:

“在很多情况下,我看到奥本海默博士的行为——或者说,据我理解奥本海默博士的行为方式——对我来说极难理解。我在很多问题上都与他完全意见不合,他的行为在我看来坦率地说是混乱而复杂的。因此,在这一点上,我觉得我更愿意将这个国家的重大利益交托给我能更好理解、因此也更信任的人。从这个非常有限的意义上说,我想表达一种感觉:如果公共事务掌握在其他人手中,我个人会感到更安心。”\(^\text{[69]}\)

特勒还作证说,奥本海默对热核计划的看法似乎更多是基于对武器在科学上是否可行的判断,而非其他任何因素。他还补充说,奥本海默在洛斯阿拉莫斯的领导工作“作为科学家和管理者都是非常杰出的成就”,称赞他“思维极为敏捷”,并且是“一位非常出色而卓越的主任”。\(^\text{[69]}\)

然而,在此之后,他详细说明了自己认为奥本海默在推动积极发展热核武器项目方面是如何阻挠自己的,并且长篇批评奥本海默在其职业生涯中不同时期未能加大对此问题的投入,表示:

“如果这是一个关于智慧和判断力的问题,并且这种智慧和判断力可以通过他自 1945 年以来的行为来体现,那么我会说,最好不要给予安全许可。”\(^\text{[69]}\)

通过将关于氢弹早期工作的判断分歧重新界定为国家安全风险的问题,特勒实质上是在国家安全至关重要的领域中给奥本海默定了罪。他的证词使奥本海默被国会助理指控为苏联间谍提供了口实,最终导致奥本海默的职业生涯被摧毁。\(^\text{[86]}\)

听证会后,奥本海默的安全许可被正式撤销。特勒的大多数前同事都不赞成他的证词,他因此被科学界大范围孤立。\(^\text{[84]}\)此后,特勒始终否认自己有意陷害奥本海默,甚至声称自己是在试图为其辩护。然而,有文献证据表明,情况可能并非如此。在作证前六天,特勒曾与原子能委员会的一位联络官会面,并建议在证词中“加深指控”。\(^\text{[87]}\)

特勒始终坚持认为自己的证词并未对奥本海默造成实质性伤害。2002年,特勒辩称奥本海默并没有被这场安全听证会“摧毁”,而是“只是再也没有被邀请参与政策事务”。他表示,自己的言辞之所以显得反应过度,是因为他刚刚得知奥本海默曾未能及时汇报海康·谢瓦利耶的接触请求——谢瓦利耶曾劝说奥本海默为苏联提供帮助。特勒说,事后看来,如果再让他选择一次,他会以不同的方式回应。\(^\text{[84]}\)

历史学家理查德·罗兹表示,在他看来,由时任原子能委员会主席刘易斯·斯特劳斯主导的听证会在一开始就注定会撤销奥本海默的安全许可,无论特勒作证与否。然而,由于特勒的证词最具杀伤力,他因此成为众矢之的,被认为是听证会裁定的主要责任人,甚至因此失去了像罗伯特·克里斯蒂这样的朋友。在一桩著名事件中,克里斯蒂甚至拒绝与他握手。这种对待方式成为他日后在物理学界被视为“弃子”的象征,最终迫使他不得不与工业界人士结盟。\(^\text{[88]}\)
\subsection{美国政府工作与政治倡议}
