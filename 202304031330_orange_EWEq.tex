% 电场波动方程
% 麦克斯韦方程组|波动方程|电场|介质|电位移矢量

\begin{issues}
\issueDraft
\issueOther{需要增加复数形式}
\end{issues}

\pentry{麦克斯韦方程组(介质)\upref{MWEq1}, 矢量算符运算法则\upref{VopEq}, 平面波\upref{PWave}}

真空中, 由麦克斯韦方程组\upref{MWEq} 得
\begin{equation}
\curl(\curl\bvec E) = -\pdv{t} (\curl\bvec B) = -\epsilon_0\mu_0 \pdv[2]{t} \bvec E~.
\end{equation}
根据\autoref{eq_VopEq_7}~\upref{VopEq} 化简得
\begin{equation}
\laplacian \bvec E - \frac{1}{c^2} \pdv[2]{\bvec E}{t} = 0~,
\end{equation}
这就是电场的波动方程。 所以电场的各个分量分别满足三维波动方程(链接未完成)。 它的解为平面波
\begin{equation}\label{eq_EWEq_1}
\bvec E(\bvec r, t) = \bvec E_0 \cos(\bvec k\vdot \bvec r - \omega t)~,
\end{equation}
其中 $\omega = c\abs{\bvec k} = ck$。 而通解是这些平面波的任意线性组合。 注意如果 $\bvec E_0$ 中存在平行于 $\bvec k$ 的分量, 那么 $\div \bvec E \ne 0$, 所以二者必须垂直, 即 $\bvec E \vdot \bvec k = 0$。 电场的通解可表示为
\begin{equation}
\bvec E(\bvec r, t) = \int \bvec E_0(\bvec k) \cos(\bvec k\vdot \bvec r - \omega k t) \dd[3]{k}~.
\end{equation}

根据 $\curl \bvec E = -\pdv*{\bvec B}{t}$, 可求出\autoref{eq_EWEq_1} 伴随的磁场为
\begin{equation}
\bvec B(\bvec r, t) = \bvec B_0 \cos(\bvec k \vdot \bvec r - \omega t)~.
\end{equation}
其中 $\bvec B_0$ 的模长为 $\abs{\bvec B_0} = \abs{\bvec E_0}/c$, 于 $\bvec E_0$ 垂直, 方向满足 $\uvec E \cross \uvec B = \uvec k$。 可见\textbf{电磁波是横波}。

\subsection{介质中}

非线性光学中一般认为介质具有 $\mu = \mu_0$,且假设 $\div\bvec E = 0$ 仍然成立

介质中没有自由电荷或自由电流。

类似真空情况的推导过程,有
\begin{equation}
\curl(\curl\bvec E) = -\pdv{t} (\curl\bvec B) = -\mu_0 \pdv{t} (\curl\bvec H)
= -\mu_0 \pdv[2]{t} \bvec D
\end{equation}

把电位移矢量的定义 $\bvec D = \epsilon_0\bvec E + \bvec P$ 代入上式,化简为

\begin{equation}
\curl(\curl\bvec E) = -\pdv{t} (\curl\bvec B) = -\mu_0 \pdv{t} (\curl\bvec H)
= -\mu_0 \pdv[2]{t} \bvec D
\end{equation}