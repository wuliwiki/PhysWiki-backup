% 巴拿赫-塔斯基定理(综述)
% license CCBYNCSA3
% type Wiki

本文根据 CC-BY-SA 协议转载翻译自维基百科\href{https://en.wikipedia.org/wiki/Banach\%E2\%80\%93Tarski_paradox}{相关文章}。

\begin{figure}[ht]
\centering
\includegraphics[width=10cm]{./figures/a4f2dc0381172744.png}
\caption{} \label{fig_BTS_1}
\end{figure}
巴拿赫–塔尔斯基悖论是集合论几何中的一个定理,其内容如下:给定三维空间中的一个实心球体,存在一种将该球体分解为有限个不相交子集的方式,这些子集可以以不同的方式重新组合,从而得到两个与原球体完全相同的副本。实际上,重新组合过程仅涉及移动和旋转这些部分,而不改变它们的原始形状。然而,这些部分本身并不是传统意义上的“固体”,而是无穷多个点的散布。重构可以通过最少五个部分来实现。[1]

定理的另一种形式表述为:给定任意两个“合理的”固体物体(例如一个小球和一个巨大球),这两个物体的切割部分可以重新组合成对方。这通常被非正式地表述为“一个豌豆可以被切割并重新组合成太阳”,并被称为“豌豆和太阳悖论”。

该定理是一个真实悖论:它与基本的几何直觉相矛盾,但并不是错误的或自相矛盾的。通过将球体分割成部分并通过旋转和平移来移动它们,而没有任何拉伸、弯曲或添加新点,“将球体翻倍”似乎是不可能的,因为所有这些操作从直觉上讲都应该保持体积不变。这样的操作保持体积的直觉并不是数学上荒谬的,甚至它也包含在体积的正式定义中。然而,这里不适用这种直觉,因为在这种情况下,无法定义所考虑子集的体积。重新组合它们会产生一个具有体积的集合,而这个体积恰好与开始时的体积不同。

