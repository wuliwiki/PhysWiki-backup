% 光电离时间延迟

\begin{issues}
\issueDraft
\end{issues}

\pentry{量子散射的延迟\upref{tDelay}, 含时微扰理论\upref{TDPT}}

本文使用原子单位制\upref{AU}. 例如一个有限深势阱(短程势)中有一个束缚态, 被一个电场波包电离之后, 光电子波包逃出势阱. 那么光电子波包的延迟是多少呢? 我们以下使用一阶含时微扰理论\upref{TDPT} 来分析. 我们假设势阱在坐标原点, 且 $t = 0$ 时电场波包的中心刚好到达原点.

初态和末态能量分别为 $E_0, E$, 令 $\omega = E - E_0$, 不含时的末态记为 $\psi_E(x)$, 并令其为满足边界条件
\begin{equation}\label{HeAna2_eq2}
\psi_E(x \to +\infty) \propto \E^{\I kx}
\end{equation}
其中 $A(k)$ 是实函数. 那么一阶微扰系数为
\begin{equation}
c_E(t) = -\I  \mel{\psi_E}{H'}{\psi_0} \tilde f(-\omega)
\end{equation}
电离波包可以由末态展开:
\begin{equation}
\psi(x, t) = \int c_E(t) \psi_E(x)\E^{-\I E t} \dd{E}
\end{equation}
那么根据\autoref{tDelay_eq4}~\upref{tDelay} 延迟为
\begin{equation}\label{HeAna2_eq3}
\tau = \pdv{E} \arg [c_E]
\end{equation}
对于没有 chirp 的电场波包, $\tilde f(-\omega)$ 相位恒定(想象一个高斯波包乘以正弦函数的傅里叶变换). 所以随 $E$ 变化的相位只有矩阵元, 绝对时间延迟为
\begin{equation}\label{HeAna2_eq1}
\tau = \pdv{E} \arg \mel{\psi_E}{H'}{\psi_0}
\end{equation}
这个延迟是相对与 $t = 0$ 也就是电磁波包到达原点的时间.

\subsection{三维库仑势能光电离延迟}
库仑球面波记为 $\ket{C_{l,m}(k)}$, 含时微扰
\begin{equation}
\ket{\psi(t)} = \sum_{l,m} c_{l,m}(k) \ket{C_{l,m}(k)} \E^{-\I E t} \dd{k}
\end{equation}
其中系数为
\begin{equation}
c_{l,m}(k) = -\I \mel{C_{l,m}}{z}{i} \tilde f(-\omega)
\end{equation}
相位不随 $k$ 或 $\omega$ 变化. 注意对于氢原子束缚态作为初始态, 由于选择定则 $\Delta m = 0$, $\Delta l = \pm 1$, 只有最多两个 $c_{l,m}$ 不为零.

投影到库伦平面波为 $\ket{C(\bvec k)} = \sum_{l,m} a_{l,m}^{(-)}(\bvec k) \ket{C_{l,m}}$, 把 $\ket{\psi(t)}$ 投影到 $\ket{C(\bvec k)}$ 上有
\begin{equation}
\braket{C(\bvec k)}{\psi(t)} = \sum_{l,m} a_{l,m}^{(-)*}(\bvec k) c_{l,m}(k) \E^{-\I E t} 
= \sum_{l,m} \frac{\I^{-l}}{k} \E^{\I\sigma_l} Y_{l,m} (\uvec k) c_{l,m}(k) \E^{-\I E t} 
\end{equation}
所以和\autoref{HeAna2_eq3} 同理, 当求和中的两项中有一项 $l$ 主导时, 该系数产生的延迟仅由库仑相移贡献
\begin{equation}
t_{EWS}^C = \pdv{\sigma_l}{E}
\end{equation}
但是库仑平面波本身还存在不收敛的相移 $-\eta\ln (2kr)$, 这求导以后是一个负的, 会带来一个负延迟, 即提前.

库仑渐进相位为(\autoref{CulmF_eq7}~\upref{CulmF})
\begin{equation}
F_l(k, r) \overset{r\to \infty}{\longrightarrow} \sin\qty[kr - \frac{\pi l}{2} + \frac{Z}{k}\ln(2kr) + \sigma_l(k)] % Z > 0
\end{equation}
\begin{equation}
\sigma_l(k) = \arg[\Gamma(l+1-\I Z/k)] \qquad (Z > 0)
\end{equation}
\begin{figure}[ht]
\centering
\includegraphics[width=8cm]{./figures/HeAna2_1.png}
\caption{库仑相移 $\sigma_l(k)$} \label{HeAna2_fig1}
\end{figure}
其中 $\sigma_l(k)$ 产生的相移与距离无关, 可以单纯看作是非直线轨道带来的延迟(不确定), 并且是 XUV 波包一瞬间带来的. $kr\ln(2kr)$ 只和距离有关, 带来的延迟是库仑力产生的, 而且并不收敛. 那么光电离的总延时就是
\begin{equation}
\tau = \pdv{E} \arg[\sigma_l(k) + \frac{Z}{k}\ln(2kr)] % Z > 0
\end{equation}
第一项是瞬间的, 而第二项是波包移动过程中缓慢积累且不收敛的.

一定要强调光电离延迟 $\tau$ 和 streaking 延迟是不一样的. 在 Streaking 实验中, 前两项都会直接转换为 streaking 延迟(因为电离瞬间产生). 第三项在上式中不收敛, 但 streaking 延迟却总是收敛的. 配合 IR 的扰动就变为了 $t_{CLC}$ 项.

\subsection{变相位的延时}
\addTODO{如果是长程库仑势能, 延迟就会取决于距离而不收敛. 但动量却收敛, 所以使用 streaking 仍然会获得一个延迟, 但这个延迟和上面的是两码事, 然而 Pazourek 仍然使用\autoref{HeAna2_eq1}, 这里面还有更深的奥妙……}

如果\autoref{HeAna2_eq2} 中本来就有额外的取决于能量的相位 $\delta(E)$ 那么同样需要加到\autoref{HeAna2_eq1} 中
\begin{equation}
\tau = \pdv{E} \arg \mel{\psi_E}{H'}{\psi_0} + \pdv{E} \delta(E)
\end{equation}
并且这个相位可能还会取决于距离和其他参数 $\delta(E, x)$. 例如库仑相移中, 这个相位产生的延迟并不随距离收敛.

\addTODO{另一个问题是, 延迟不仅和 He+ 的 $n$ 有关还与 $l$ 有关, 或者说还与 Stark 效应的好量子数有关, 那么 Pazourek 使用的是哪个呢? 还是说取平均呢? 还是要仔细看 Pazo12}
