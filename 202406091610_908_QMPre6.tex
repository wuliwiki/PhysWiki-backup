% 数的观念
% license Usr
% type Art

(本文根据 CC-BY 协议转载自季燕江的《量子序曲》, 进行了重新排版和少量修改)

\subsection{万物皆数}

数是抽象的观念,它并不在自然中对我们现身。当我们说水的时候,我们知道水是什么,当我们说红的时候也知道红对我们意味着什么,但数是什么呢?当我们说一的时候,是什么意思呢?

“一”是个动作,是我用手指向某物,但我为何要做这动作呢?我是做给他者(另外一个我)看的,“瞧,此物”。这就意味着一种整全性,我无法用手指同时指向0-1之间所有的实数,我指向的是一个整全的对象,我并没有指向部分,除非你要求我澄清。

弗雷格在《算术基础》中问1+1=2意味什么?它不可能是两个月亮相加。世界上也不可能找到两个相同的月亮。

那么1、和1是什么呢?

1是用手指,1是再用手指,1就是count这个动作。

据说在某些原始部落,人们没有超过数字3的概念,他们没法像我们这样数数,他们只能这样:

“1,2,3,3,3,……”

但,假如我们去和这些原始部落中的人做交换,我们能糊弄他们吗?

我们用玻璃珠和他们换珍珠,我们能用3颗玻璃珠,换来比如100颗珍珠吗?

这当然不可能。原始人只是缺乏对数字的命名,没有像我们那样定下加法口诀表,但这并不意味着他们没有算术技术。

比如他们可以用一个玻璃珠和一个珍珠配对,当1 vs 1都配好对了,我们拿走珍珠,他们拿走玻璃珠。

当然这是极其原始的算术技术,但把石头、米粒或小木棍摆放在地上确实给数字一个直观的印象。

必须用不同的东西count,或者空间分离,或者在时间的序列上间隔。count就是数数,数数是用不同的东西数,把石头一个个摆开就是在数数,虽然没有命名,但已经在数了。

1+1是count, count 这个动作,我们对count, count的命名是2,这就是1+1=2. 而count, count, count 我们命名为3。这背后的基础是生活,我们过某种合作的生活导致我们发明了count, count, ... 这种计数技术。它可能用于交换,拿走一个果子,我们就摆一块石头,再拿走一个,再摆一块,....

\subsubsection{自然数}

小孩学数学的第一步是背诵,1, 2, 3, 4,…这就是对count, count,...的命名。n+1= n+1 表示count n次后,再count一次。n+1对n而言是唯一确定的,而且n+1不同于之前任何一个count, 这里我们需要不同的命名,如此定义的对象将像“不闭合的珠链”一样无尽伸展出去。

最简单的数是自然数,0,1,2,3……,从学习的角度,我们是这么掌握自然数的:

\begin{enumerate}
\item 

首先是背诵,先是背熟10以内的自然数:

\begin{align}
\text{0,1,2,3,4,5,6,7,8,9,10}~
\end{align}

这可以借助10个手指头。

然后还是背诵,背熟20以内的:

\begin{align}
\text{……11,12,13,14,15,16,17,18,19,20}~
\end{align}

这里已经涉及个位和10位的问题了,好在我们有科学计数法,大于10但小于20的自然数可写为$1n$,这里$n$是0到9中间的一个数:

\begin{equation}
1n  = 1 \times 10 + n~
\end{equation}

严格来说,我们现在还没有定义加法,从0,1,2,3……开始到18,19,20的罗列仅仅是个罗列,是个有方向的一一罗列,好比是20来个好伙伴手拉手列成一队,从左到右,我一一清点他们,记熟他们的名字和位置。

进一步地,我们还可以背熟0到100的数字,这其实就是对一个链式结构的命名,命名并把它们记熟。这个过程是没头的,中国古代有五数,“一、十、百、千、万”,每逢10进1,一而十,十而百……

有“一、十、百、千、万”,日常生活中碰到的数字足够表达了。

\item

把从0到20的自然数背熟,就是在把握里面的数学结构,我们也可以通过定义运算把这里面的数学结构说清楚。

运算就是对数字的操作,比如我们可以定义加法:

$m + n$,先从0开始数,数到$m$,然后$m+1$,$m+2$,……,一直到$m+n$,因为数字已经背熟了,我们发现$m+n$就是我们背熟的数字序列中的某个数。

这个就是所谓“掰手指头”,由三开始再掰两个就是五,记作:

\begin{equation}
3+2 = 5~
\end{equation}