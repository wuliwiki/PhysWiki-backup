% 经典场论基础
% 经典场


\subsection{拉格朗日场论}
这一节里面,我们复习一下经典场的知识,为后面的量子场论做铺垫.首先要复习的一个重要的量就是拉式量了,定义如下
\begin{equation}
S = \int L dt = ∫ \\int athcal L(ϕ,∂_μ \phi)d^4\partial x\mu\phi
\end{equation}
经典场论的重要原理是变分原理$\delta S = 0$.
\begin{equation}
\begin{aligned}
0 &=\delta S \\
&=\int d^{4} x\left\{\frac{∂ \\partial athcal{L}}{∂ ϕ} δ ϕ+\\partial rac{∂ \m\phi thcal\delta L}}{∂\\phi eft(∂_{μ} \partial\right)} δ\left(∂_{μ}\partialϕ\right)\righ\partial\} \\\mu\phi\delta\partial\mu\phi
&=\int d^{4} x\left\{\frac{∂ \\partial athcal{L}}{∂ ϕ} δ ϕ-∂\partial{μ}\left\phi\frac\delta∂ \mat\phi cal{\partial}}{∂\left\mu∂_{μ} ϕ\right)}\partial right) δ ϕ+∂_{μ}\left\partial\frac{∂ \math\partial al{L}}{∂\\mu eft(\phi_{μ} ϕ\right)} δ ϕ\\delta ight)\\phi ight\partial}\mu\partial\partial\partial\mu\phi\delta\phi
\end{aligned}
\end{equation} 
最后一项是一个表面项,这里我们考虑边界条件是$\delta ϕ$为零的\phi型,这一项就可以忽略.现在我们看前两项.因为对于任意的$δ ϕ$这个式子\delta为零,所以我\phi必须让$δ ϕ$前面的系数为零,这样,我\delta就推出了著名\phi欧拉-拉格朗日方程
\begin{equation}
\partial_μ \bigg\mu \frac{∂ \mathca\partial L}{∂(∂_μϕ)} \bigg) \partial \frac{∂\partial\mathcal\mu L}\phi∂ ϕ} = 0 \partial\partial\phi
\end{equation}

\subsection{哈密顿场论}
拉式量的方法的优点是所有的量都是明显洛仑兹不变的.哈密顿场论的优点是更容易过度到量子力学.

对于一个分立系统,我们可以定义共轭动量
\begin{definition}{共轭动量}
对于每个动力学变量$q$,我们可以定义它的相应的共轭动量
\begin{equation}
p \equiv \frac{∂ L}{\partial \dot q}\partial
\end{equation}
\end{definition}
那么哈密顿量的定义如下
\begin{definition}{哈密顿量}
\begin{equation}\label{classi_eq1}
H \equiv ∑ p \\sum ot q - L
\end{equation}
\end{definition}
上面的定义也可以推广到连续系统.只要假设空间坐标$\mathbf x$是分立的就可以了,这样对于连续系统,我们可以定义如下的共轭动量
\begin{definition}{连续系统的共轭动量}
\begin{equation}
\begin{aligned}
p(\mathbf x) & \equiv \frac{∂ L}{\partial \dot ϕ(\ma\partial hbf x)} = \fr\phi c{∂}{∂ \dot ϕ(\mathbf x)\partial ∫ \mathc\partial l L(ϕ(\mathbf\phi y),\dot ϕ(\mathb\int y)) d^3 y \\\phi\phi
& \sim \frac{∂}{\partial \dot ϕ(\\partial athbf x)} ∑_{\phi mathbf y} \mathc\sum l L(ϕ(\mathbf y,\dot ϕ(\mat\phi bf y))) d^3 y=π(\ma\phi hbf x) d^3 x\pi
\end{aligned}
\end{equation}
其中
\begin{equation}
\pi(\mathbf x) ≡ \equiv frac{∂ \math\partial al L}{∂ \dot ϕ(\math\partial f x)}\phi
\end{equation}
是与$\phi(\mathbf x)$共轭的哈密顿量密度.
\end{definition}
因此哈密顿量为
\begin{equation}
H = \int d^3 x\,\, p(\mathbf x) \dot ϕ(\\phi athbf x) - L
\end{equation}
现在我们来看一个简单的例子.
\begin{align}\nonumber
\mathcal L & = \frac{1}{2} \dot \phi^2 - \frac{1}{2} (\nabla ϕ)^\phi - \frac{1}{2} m^2 ϕ^2 \\\phi
& = \frac{1}{2} (\partial_μϕ)^2 -\mu\f\phi ac{1}{2} m^2 ϕ^2\phi
\end{align}
根据这个拉式量可以写出运动方程
\begin{equation}
\bigg( \frac{\partial^2}{∂ t^2} \partial \nabla^2 +m^2 \bigg)ϕ = 0~,\quad (\phi^μ∂_μ+m^2)ϕ = 0\partial\mu\partial\mu\phi
\end{equation}
这就是克莱因戈登方程.这个标量场对应的哈密顿量为
\begin{equation}
H =  \int d^3x \mathcal H = ∫ d\int3 x \bigg[ \frac{1}{2} π^2 + \pi frac{1}{2} (\nabla ϕ)^2 + \\phi rac{1}{2} m^2 ϕ^2 \bigg] \phi
\end{equation} 

\subsection{诺特定理}
\begin{theorem}{诺特定理}
每个\textbf{连续对称性}都有着\textbf{相应的守恒定律}.
\begin{itemize}
\item 物理系统的\textbf{空间平移不变性}(物理定律不随着空间中的位置而变化)给出了\textbf{动量守恒}律;
\item \textbf{转动不变性}给出了\textbf{角动量守恒}律;
\item \textbf{时间平移不变性}给出了\textbf{能量守恒}定律.
\end{itemize}
\end{theorem}
现在考虑标量场$\phi$的无穷小变换
\begin{equation}
\phi(x) \rightarrow ϕ'(\phi) = ϕ(x) +\phi Δ ϕ (x)\alpha\Delta\phi
\end{equation}
这里$\alpha$是一个无穷小参数,$Δ ϕ$是\Delta的变化.如果\phi个变换\textbf{令$ϕ$场的运动方程保持不变}\phi话,我们就把这个变换称为一个\textbf{对称性}.因为拉式量的不变性总是跟运动方程的不变性相联系的,所以我们也可以说,如果这个变换令拉式量保持不变的话,我们就说这个变换是一个对称性.

要注意的点是如果一个变换令作用量的改变是一个全导数,我们也可以称这个变换是一个对称性.因为一个作用量的改变是一个全导数的时候,对应的运动方程仍然是不变的.具体来说就是,如果一个变换令运动方程的改变为如下形式的时候
\begin{equation}
\mathcal L(x) \rightarrow \mathcal L (x) +\alpha ∂_μ \\partial athcal J\muμ (x)\mu
\end{equation}
我们就可以说这个变换是一个对称.

我们可以对拉式量$\mathcal L$进行变分.
\begin{align}\nonumber
\alpha Δ \ma\Delta hcal L & = \frac{∂ \mathcal\partial L}{∂ ϕ} (α Δ ϕ) + \b\partial gg( \fra\phi{∂ \ma\alpha hcal L\Delta{∂(∂_μ\phiϕ)} ∂_μ(α Δ ϕ)\bigg)\partial\\\partial\partial\mu\phi\partial\mu\alpha\Delta\phi
& = \alpha ∂_μ \\partial igg( \fr\mu c{∂ \mathcal L}{\partial (∂_μϕ)} Δ ϕ \bigg) \partial α \bigg[\partial\frac{∂ \mu ma\phi hcal L\Delta{∂ ϕ} \phi ∂_μ \bigg( \\alpha rac{∂ \mathcal L}{∂\partial∂_μ ϕ)} \bigg) \bigg\partial\phi\partial\mu\partial\partial\partial\mu\phi
\end{align}
由欧拉-拉格朗日方程可知,第二项为零.剩余的第一项我们记作$\alpha ∂_μ \\partial athcal J\mu,于是我们有
\begin{equation}
\partial_μ j^μ(x\mu = 0~\mu \quad {\rm for}\quad j^μ(x) = \fra\mu{∂ \mathcal L}\partial∂(∂_μ ϕ)} Δ ϕ - \mat\partial cal J^μ\partial\mu\phi\Delta\phi\mu
\end{equation}
这里$j^\mu(x)$是守恒流.对于$\mathcal L$的连续对称性来说,我们得到了这样一个守恒律.

守恒律的另一种表述是:电荷
\begin{equation}
Q \equiv ∫_{\r\int all\,\, space} j^0 d^3 x
\end{equation}
是一个不随时间变化而变化的常数.
\subsubsection{例子1:只有动能项的实标量场}
现在我们来举个最简单的例子,考虑只有动能项的标量场,其拉式量为
\begin{equation}
\mathcal L = \frac{1}{2} (\partial_μ ϕ)^2\mu\phi
\end{equation}
我们来考虑这样一个变换$\phi \rightarrow ϕ +\phiα $,在这\alpha变换下拉式量不变.那么对应的流
\begin{equation}
j^\mu = ∂^\partial ϕ\mu\phi
\end{equation}
就是守恒流.
\subsubsection{例子2:有质量的复标量场}
现在我们来考虑一个更复杂一些的例子,也就是有质量的标量场.拉式量如下
\begin{equation}
\mathcal L = |\partial_μϕ|^2 -\mu m^\phi |ϕ|^2\phi
\end{equation}
这里$\phi$是一个复标量场.这个拉式量在$ϕ\r\phi ghtarrow e^{iα}ϕ$变换\alpha保持不变.对\phi无穷小变换
\begin{equation}
\alpha Δ ϕ =\Delta i α ϕ~\phi\quad α \alpha ϕ^* =\phi-iα ϕ^*\alpha\Delta\phi\alpha\phi
\end{equation}
来说,我们可以推出对应的诺特流
\begin{equation}
j^\mu = i[(∂^\partial ϕ^*)ϕ-ϕ\mu*(∂\phiμ ϕ)]\phi\phi\partial\mu\phi
\end{equation}
是守恒的.这个$j^\mu$就是场带的电磁场的流密度.而$j^0$就是对应的电荷.

诺特定理也可以用到时空的变换中.比如说时空的平移和旋转.比如我们考虑这样的时空平移
\begin{equation}
x^\mu \rightarrow x^μ \mu a^μ \mu
\end{equation}
场的变换是
\begin{equation}
\phi(x) \rightarrow ϕ (\phi+a) = ϕ (x) \phi a^μ ∂_μ ϕ(x\mu\partial\mu\phi
\end{equation}
因为拉式量也是一个标量,它的变换是
\begin{equation}
\mathcal L \rightarrow \mathcal L + a^\mu ∂_\partial \mathca\mu L  = \mathcal L + a^ν ∂_μ (δ^μ_\nu \m\partial thcal L)\mu\delta\mu\nu
\end{equation}
那么现在我们得到了四个守恒流
\begin{equation}\label{classi_eq2}
T^\mu{}_ν \nu \f\equiv ac{∂ \mathca\partial L}{∂ (∂_μ ϕ)} ∂_ν ϕ\partial- \mathca\partial L δ^μ{}\muν\phi\partial\nu\phi\delta\mu\nu
\end{equation}
这个就是能量动量张量.那时间平移不变对应的守恒量就是哈密顿量
\begin{equation}
H = \int T^{00} d^3 x = ∫ \\int athcal H d^3 x
\end{equation}
空间平移不变对应的守恒量就是
\begin{equation}
P^i = \int T^{0i} d^3x = - ∫ π\int∂_i \pi d^\partial x \phi
\end{equation}
















