% 麦克斯韦方程组
% keys 麦克斯韦方程组|电磁感应|高斯定律|电位移|安培定律

\begin{issues}
\issueDraft
\end{issues}

\footnote{参考 Wikipedia \href{https://en.wikipedia.org/wiki/Maxwell's_equations}{相关页面}.}麦克斯韦方程组描述了经典电磁理论中电荷如何影响电磁场, 以及电磁场变化的规律.它有多种不同的表示方法,本词条中会分别讨论.

\subsection{微分形式}
麦克斯韦方程组共有四条方程
\begin{align}
\label{MWEq_eq1}
&\div \bvec E = \frac{\rho}{\epsilon_0}\\
\label{MWEq_eq2}
&\curl \bvec E = -\pdv{\bvec B}{t}\\
\label{MWEq_eq3}
&\div \bvec B = 0 \\
\label{MWEq_eq4}
&\curl \bvec B = \mu_0 \bvec j + \mu_0\epsilon_0 \pdv{\bvec E}{t}
\end{align}
其中\autoref{MWEq_eq1} 到\autoref{MWEq_eq4} 分别是电场的高斯定律证明\upref{EGausP},法拉第电磁感应定律\upref{FaraEB},磁场的高斯定律\upref{MagGau}, 安培环路定理\upref{AmpLaw}(加位移电流).%链接未完成
注意电场和磁场不是完全对称的, 可以通过引入磁单极子\upref{BMono}的概念使它们完全对称.

\subsection{积分形式}
\begin{align}
\oint \bvec E \vdot \dd{\bvec s} &= \frac{1}{\epsilon_0}\int \rho \dd{V}\\
\oint \bvec E \vdot \dd{\bvec l} &= -\int \pdv{\bvec B}{t} \vdot \dd{\bvec s}\\
\oint \bvec B \vdot \dd{\bvec s} &= 0\\
\oint \bvec B \vdot \dd{\bvec l} &= \mu_0 \int \bvec j \vdot \dd{\bvec s} + \mu_0 \epsilon_0 \int \pdv{\bvec E}{t} \vdot \dd{\bvec s}
\end{align}

\subsubsection{高斯单位制}
高斯单位制\upref{GaussU}中的麦克斯韦方程组具有更为对称的形式
\begin{align}
&\div \bvec E = 4\pi\rho\\
&\curl \bvec E = -\frac{1}{c}\pdv{\bvec B}{t}\\
&\div \bvec B = 0 \\
&\curl \bvec B = \frac{4\pi}{c} \bvec j + \frac{1}{c}\pdv{\bvec E}{t}
\end{align}

% \subsection{外导数形式}
% \pentry{外导数\upref{ExtDer},闵可夫斯基空间\upref{MinSpa},Hodge算子(尚未撰写)}

% 在闵可夫斯基空间 $(\mathbb{R}^4, \eta)$ 中考虑电磁场

% 把 $\bvec{B}$
% 表示成一个2-形式 $B=B_z\dd x\wedge\dd y+B_x\dd y\wedge\dd z+B_y\dd z\wedge\dd x$,把 $\bvec{E}$ 表示成一个1-形式 $E=E_x\dd x+E_y\dd y+E_z\dd z$,再定义一个2-形式 $F=B+E\wedge\dd x^0$,那么麦克斯韦方程组可以表示为:

% \begin{equation}
% \leftgroup{
% \dd F&=0\\\star\dd\star F&=J
% }
% \end{equation}

% 其中 $\mathrm{d}$ 是闵可夫斯基流形上的外导数算子,$\star$ 是Hodge星算子,$J$ 是四维电流密度.




