% 恩里科·费米(综述)
% license CCBYSA3
% type Wiki

本文根据 CC-BY-SA 协议转载翻译自维基百科 \href{https://en.wikipedia.org/wiki/Enrico_Fermi}{相关文章}。

恩里科·费米(意大利语:[enˈriːko ˈfermi],1901年9月29日-1954年11月28日)是一位意大利裔、后归化为美国公民的物理学家,以建造世界上第一座人工核反应堆——芝加哥一号堆而闻名,并曾是曼哈顿计划的重要成员。他被誉为“核时代的建筑师”以及“原子弹之父”。\(^\text{[1]}\)他是极少数在理论物理和实验物理两个领域都卓有成就的物理学家之一。费米因其在中子轰击引发放射性方面的研究以及对超铀元素的发现而获得1938年诺贝尔物理学奖。他与同事们共同申请了多项与核能应用相关的专利,所有这些专利最终都被美国政府接管。他在统计力学、量子理论、核物理和粒子物理的发展中都作出了重要贡献。

费米的第一个重大贡献是在统计力学领域。1925年,沃尔夫冈·泡利提出了著名的泡利不相容原理,随后费米发表了一篇论文,将该原理应用于理想气体,发展出一种统计方法,如今被称为费米–狄拉克统计。今天,那些遵守不相容原理的粒子被称为“费米子”。后来,泡利为了解释β衰变中能量守恒的问题,提出了在电子发射的同时还会发射一种不带电的不可见粒子这一假设。费米接纳了这个想法,并构建了一个理论模型,纳入了这一假想粒子,并将其命名为“中微子”。他的这一理论后来被称为“费米相互作用”,现今称为“弱相互作用”,是自然界四种基本相互作用之一。在用新发现的中子进行诱导放射性实验时,费米发现慢中子比快中子更容易被原子核俘获,并据此发展出描述该过程的“费米年龄方程”。在用慢中子轰击钍和铀的实验中,费米认为自己合成了新的元素。尽管他因这一发现获得了诺贝尔奖,但后来证实这些“新元素”其实是核裂变的产物。1938年,为了躲避影响其犹太妻子劳拉·卡蓬的意大利新种族法,费米离开意大利,移民美国。在第二次世界大战期间,他参与了“曼哈顿计划”。在芝加哥大学,费米领导的团队设计并建造了“芝加哥堆-1”,该堆于1942年12月2日首次实现了人类制造的、自持的核链式反应。他还在田纳西州橡树岭的X-10石墨反应堆于1943年达到临界状态时在场,次年又见证了华盛顿州汉福德基地的B反应堆启动。在洛斯阿拉莫斯国家实验室,他领导F部门,其中一部分致力于爱德华·泰勒的热核“超级炸弹”项目。他还亲历了1945年7月16日的“特立尼蒂试验”,即首次核弹爆炸测试,并使用著名的“费米估算法”评估了炸弹的当量。

战后,费米协助创建了芝加哥的核研究所,并在J·罗伯特·奥本海默担任主席的总顾问委员会中任职,为美国原子能委员会提供核事务建议。1949年8月苏联成功引爆第一颗裂变原子弹后,费米从道德和技术两方面都强烈反对研制氢弹。1954年,在导致奥本海默失去安全许可的听证会上,费米也是为奥本海默作证的科学家之一。

费米在粒子物理领域也做出了重要贡献,尤其是在与介子(如π介子和μ子)相关的研究方面。他还推测宇宙射线的产生是由于星际空间中的磁场加速物质所致。许多奖项、概念和机构都以费米的名字命名,包括费米一号(快中子增殖反应堆)、恩里科·费米核发电站、恩里科·费米奖、恩里科·费米研究所、费米国家加速器实验室、费米伽马射线太空望远镜、“费米悖论”,以及人造元素“镄”,这使他成为仅有的十六位拥有化学元素以自己命名的科学家之一。
\subsection{早年生活}
\begin{figure}[ht]
\centering
\includegraphics[width=6cm]{./figures/635b80f64a6090ee.png}
\caption{费米出生于罗马盖塔街19号。} \label{fig_ELK_1}
\end{figure}
恩里科·费米于1901年9月29日出生在意大利罗马。\(^\text{[3]}\)他是铁路部司局长阿尔贝托·费米与小学教师伊达·德·加蒂斯的第三个孩子。\(^\text{[3][4][5]}\)他的姐姐玛丽亚比他大两岁,哥哥朱利奥大他一岁。两位男孩幼年时被送到乡下由乳母抚养,直到恩里科两岁半时才返回罗马与家人团聚。\(^\text{[6]}\)尽管按照祖父母的意愿他接受了天主教洗礼,但他的家庭并不虔诚;恩里科成年后一直是无神论者。\(^\text{[7]}\)童年时期,他和哥哥朱利奥有着相同的兴趣爱好,比如制作电动机、玩电动和机械玩具。\(^\text{[8]}\)1915年,朱利奥因喉部脓肿手术不幸去世;玛丽亚则于1959年在米兰附近的一场空难中遇难。\(^\text{[9][10]}\)

在罗马的鲜花广场集市上,费米发现了一本物理书,名为《Elementorum physicae mathematicae》,全书900页,由耶稣会士、罗马学院教授安德烈亚·卡拉法神父用拉丁文编写。书中介绍了当时(1840年出版)对数学、经典力学、天文学、光学和声学的理解。\(^\text{[11][12]}\)在一位同样对科学感兴趣的朋友恩里科·佩尔西科的陪伴下,\(^\text{[13]}\)费米开始了诸如制作陀螺仪、测量地球重力加速度等实验项目。\(^\text{[14]}\)

1914年,费米常常在父亲下班后到办公室门口与其会合。有一次,他遇见了父亲的一位同事阿道夫·阿米代伊。恩里科得知阿道夫对数学和物理感兴趣,便趁机向他请教几何问题。阿道夫意识到小费米问的是射影几何,随后送给他一本由特奥多尔·雷耶所著的相关书籍。两个月后,费米将书还给了阿道夫,声称自己已完成了书后所有的习题,其中一些题目连阿道夫都认为很难。阿道夫核实后惊叹道费米“至少在几何方面是个天才”,并开始更加系统地指导他,给他提供更多关于物理与数学的书籍。阿道夫还指出,费米的记忆力极好,读完书后就能记住全部内容,因此通常读完便归还书籍。\(^\text{[15]}\)
\subsection{比萨高等师范学院}
\begin{figure}[ht]
\centering
\includegraphics[width=6cm]{./figures/7f8ade509d759b3a.png}
\caption{} \label{fig_ELK_2}
\end{figure}
费米于1918年7月高中毕业,他跳过了第三学年。在阿米代伊的敦促下,费米学习了德语,以便阅读当时大量以德语发表的科学论文,并申请了比萨高等师范学院。阿米代伊认为,这所学校能为费米的发展提供比当时的罗马萨皮恩扎大学更好的条件。费米的父母在失去一个儿子后,勉强同意他在学校的住宿区生活四年,远离罗马。\(^\text{[16][17]}\)费米在难度极高的入学考试中获得第一名,其中包括一篇题为“声音的特征”的作文;17岁的费米选择使用傅里叶分析来导出并求解振动棒的偏微分方程。考官在面试他之后断言,费米将成为一位杰出的物理学家。\(^\text{[16][18]}\)

在比萨高等师范学院,费米与同学弗兰科·拉塞蒂一起恶作剧,两人成为了亲密的朋友和合作伙伴。物理实验室主任路易吉·普恰蒂是费米的导师,他曾表示自己几乎教不了费米什么,反而常常请费米教他一些东西。\(^\text{[19]}\)费米在量子物理方面的知识如此扎实,以至于普恰蒂请他主持有关该主题的研讨会。在此期间,费米学习了张量计算法,这是一种在广义相对论中至关重要的数学工具。\(^\text{[20]}\)费米最初选择数学作为主修专业,但很快转为物理学。他在很大程度上是自学成才,系统学习了广义相对论、量子力学和原子物理学。\(^\text{[21]}\)

1920年9月,费米被正式录取进入物理系。由于系里只有三名学生——费米、拉塞蒂和内洛·卡拉拉——导师普恰蒂便允许他们自由使用实验室,进行任何他们感兴趣的研究。\(^\text{[22]}\)费米决定他们应该研究X射线晶体学,于是三人开始合作拍摄劳厄照片——即晶体的X射线照片。1921年,费米在大学三年级期间,在意大利期刊《新物理杂志》上发表了他的第一批科学论文。第一篇题为《平移运动中电荷刚体系统的动力学》(意大利语原文:Sulla dinamica di un sistema rigido di cariche elettriche in moto traslatorio)。这篇论文展现了费米未来研究方向的端倪:他将质量表达为张量——这是一个常用于描述三维空间中运动和变化的数学结构。在经典力学中,质量是一个标量,但在相对论中,它随着速度变化。第二篇论文是《在均匀引力场中电磁电荷的静电学以及电磁电荷的重量》。在这篇论文中,费米使用广义相对论证明,一个电荷的重量等于其系统的静电能量$U$除以光速$c$的平方,即 $U/c^2$。\(^\text{[21]}\)
