% 精确对角化
% keys 量子力学
% license Usr
% type Tutor

\begin{issues}
\issueDraft       % 本文处于草稿阶段
\end{issues}
本篇针对科研小白的一个启蒙。

对于量子力学的数值计算,这一篇是基础中的基础,也是必经之路。我们首先需要知道,在所有的模型中哈密顿量是可以写成一个矩阵。
例如一个最简单的无相互作用的toy模型
\begin{equation}\label{ham1}
\hat{H}=  \sum_j^{N}V_j c_j^{\dagger}c_j.~
\end{equation}
$N$ 是系统中的粒子数,$V_j$ 是势能项,$c_j^{\dagger}$ 和 $c_j$ 分别是产生和湮灭算符。产生算符 $c_j^{\dagger}$ 在位置 $j$ 创建一个粒子,湮灭算符 $c_j$ 在位置 $j$ 湮灭一个粒子。$c_j^{\dagger}c_j$是表示$\hat{n}$粒子数算符,其定义为:
\begin{equation}
\hat{n}_j = c_j^{\dagger} c_j~.
\end{equation}
在没有相互作用的情况下,哈密顿量 $\hat{H}$ 是对角化的,这意味着哈密顿量的矩阵形式 $\mathcal{H}$ 是对角矩阵。这样我们就可以直接得到系统的能量本征值,它们对应于对角矩阵 $\hat{H}$ 的对角元素 $V_j$。因此,系统的能量本征值 ${E_j}$ 就是 ${V_j}$。
设N=4,可以在这里哈密顿量可以直接写为一个4*4的对角矩阵:
\begin{equation}
\mathcal{H}=\left(
\begin{matrix}
V_1 & 0& 0&0\\
0 &V_2&0&0\\
0&0&V_3&0\\
0&0&0&V_4\\
\end{matrix}
\right)~.
\end{equation}
