% 集合(综述)
% license CCBYSA3
% type Wiki

本文根据 CC-BY-SA 协议转载翻译自维基百科\href{https://en.wikipedia.org/wiki/Set_(mathematics)}{相关文章}。

\begin{figure}[ht]
\centering
\includegraphics[width=6cm]{./figures/c3177a16637a20f4.png}
\caption{欧拉图中的一组多边形} \label{fig_JHSX_1}
\end{figure}
在数学中,集合是不同事物的集合;这些事物被称为集合的元素或成员,通常是任何类型的数学对象:数字、符号、空间中的点、线条、其他几何形状、变量,甚至是其他集合。集合可以是有限的或无限的,具体取决于其元素的数量是否有限。存在一个没有元素的唯一集合,称为空集;只有一个元素的集合称为单集合。

集合在现代数学中无处不在。实际上,集合理论,特别是泽尔梅洛-弗兰克尔集合理论,自20世纪上半叶以来,已经成为为所有数学分支提供严谨基础的标准方法。
\begin{figure}[ht]
\centering
\includegraphics[width=6cm]{./figures/7cd2858c49f58e9d.png}
\caption{这个集合等于上面所示的集合,因为它们具有完全相同的元素。} \label{fig_JHSX_2}
\end{figure}
\subsection{背景}
在19世纪末之前,集合并没有被专门研究,也没有与数列明确区分开来。大多数数学家认为无穷大是潜在的——意味着它是一个无尽过程的结果——因此他们不愿意考虑无限集合,即那些成员数量不是自然数的集合。具体来说,一条线并没有被看作是其点的集合,而是看作一个点可能位于其中的轨迹。

无限集合的数学研究始于乔治·康托尔(Georg Cantor,1845-1918)。这带来了一些违反直觉的事实和悖论。例如,数轴上有一个无限多个元素,其数量严格大于自然数的无限集合,而任何线段的元素数量与整个空间相同。此外,拉塞尔悖论意味着“所有集合的集合”这一短语是自我矛盾的。

这些违反直觉的结果,加上其他的悖论,导致了数学的基础危机,最终通过广泛采纳泽尔梅洛-弗兰克尔集合理论作为集合论和所有数学的坚实基础得以解决。

与此同时,集合开始在所有数学领域广泛应用。特别是,代数结构和数学空间通常是通过集合来定义的。此外,许多较早的数学成果也以集合的形式重新表述。例如,欧几里得的定理常常被表述为“素数集合是无限的”。大卫·希尔伯特曾预言,集合在数学中的广泛使用:“没有人会把我们从康托尔为我们创造的天堂中赶出去。”

通常,数学中对集合的常见使用并不需要泽尔梅洛-弗兰克尔集合理论的完整能力。在数学实践中,集合可以独立于该理论的逻辑框架进行操作。

本文的目的是总结在数学中常用的集合操作规则和属性,而不涉及任何逻辑框架。对于研究集合的数学分支,请参见集合论;对于对应逻辑框架的非正式介绍,请参见朴素集合论;对于更正式的介绍,请参见公理化集合论和泽尔梅洛-弗兰克尔集合理论。

