% 雅各布·伯努利(综述)
% license CCBYSA3
% type Wiki

本文根据 CC-BY-SA 协议转载翻译自维基百科\href{https://en.wikipedia.org/wiki/Jacob_Bernoulli}{相关文章}。

\begin{figure}[ht]
\centering
\includegraphics[width=6cm]{./figures/2704fab0dd45f809.png}
\caption{} \label{fig_YGBbnl_1}
\end{figure}
雅各布·伯努利\(^\text{[a]}\)(Jacob Bernoulli,也被称为英语中的詹姆斯或法语中的雅克;1655年1月6日[旧历 1654年12月27日]—1705年8月16日)是一位瑞士数学家。在莱布尼茨与牛顿的微积分优先权之争中,他站在戈特弗里德·威廉·莱布尼茨一方,是莱布尼茨微积分法的早期支持者,并为其作出了诸多贡献。作为伯努利家族的一员,他与其兄约翰·伯努利一道,是变分法的奠基人之一。他还发现了基本数学常数 $e$。然而,他最重要的贡献是在概率论领域,在其著作《概率艺术》中首次推导出了大数法则的初步形式。
\subsection{生平简介}
\begin{figure}[ht]
\centering
\includegraphics[width=6cm]{./figures/71d5c1d90164a5d8.png}
\caption{} \label{fig_YGBbnl_2}
\end{figure}
雅各布·伯努利出生于瑞士联邦的巴塞尔,父系是新教香料商人世家,\(^\text{[4][5]}\)母亲则出身于一个从事银行业与城市治理的家庭。\(^\text{[6]}\)

遵从父亲的意愿,他最初学习神学并成为牧师。但与父母的期望相反,\(^\text{[7]}\)他也私下研习数学和天文学。1676年至1682年间,他游历欧洲各地,向当时的著名学者学习最新的数学与科学成果,其中包括约翰内斯·胡德、罗伯特·波义耳以及罗伯特·胡克的研究。在此期间,他还提出了一种关于彗星的理论,但该理论被证明是错误的。下图为1682年《学者通报》刊登的对伯努利彗星新体系尝试的批评:

伯努利返回瑞士后,自1683年起在巴塞尔大学教授力学。他的博士论文《三重问题的解法》完成于1684年,\(^\text{[8]}\)并于1687年正式出版。\(^\text{[9]}\)
