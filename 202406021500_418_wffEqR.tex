% 公式的基本等价关系(数理逻辑)
% license Usr
% type Tutor

\pentry{公式(数理逻辑)\nref{nod_wlform}}{nod_e930}
\begin{theorem}{公式的基本等价关系}
令 $A, B, C$ 都是任意的合式公式,则有以下的关系成立:
\begin{enumerate}
\item 幂等律:
\begin{equation}
A \land A \Leftrightarrow A, A \lor A \Leftrightarrow A~.
\end{equation}
\item 交换律:
\begin{equation}
A \land B \Leftrightarrow B \land A, A \lor B \Leftrightarrow B \lor A ~.
\end{equation}
\item 结合律:
\begin{equation}
A \land (B \land C) \Leftrightarrow (A \land B) \land C, A \lor (B \lor C) \Leftrightarrow (A \lor B) \lor C ~.
\end{equation}
\item 同一律:
\begin{equation}
A \land 1 \Leftrightarrow A, A \lor 0 \Leftrightarrow A ~.
\end{equation}
\item “分配律”:
\begin{equation}
\begin{aligned}
A \lor (B \land C) &\Leftrightarrow (A \lor B) \land (A \lor C) ,\\
A \land (B \lor C) &\Leftrightarrow (A \land B) \lor (A \land C) ~.
\end{aligned}
\end{equation}
\item 逆否(假言易位):
\begin{equation}
(A \to B) \Leftrightarrow ((\neg B) \to (\neg A)) ~.
\end{equation}
\item 
\end{enumerate}
\end{theorem}