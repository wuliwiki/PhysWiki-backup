% 整环
% keys 整环|零因子|整除性
\pentry{环同态\upref{RingHm}}



我们常见的整数环属于一类性质极为良好的环,我们称这个分类为“无零因子交换幺环”,或者“整环”.注意区分“整数环”和“整环”这两个相似的术语,前者特指整数构成的环,后者则指一类环.

\begin{definition}{零因子}
在环$R$中,如果有两个\textbf{非零}的元素$a, b$使得$ab=0$,那么我们称$a$是一个\textbf{左零因子},而$b$是一个\textbf{右零因子}.如果某元素即使左零因子又是右零因子,那么我们称它是一个\textbf{零因子(zero divisor)}.
\end{definition}

简单来说,零因子就是“相乘得到零的非零元素”,其名称的含义就是“零的因子”.整数环里不存在零因子,但是这个概念也不难理解:考虑环$\mathbb{Z}_{12}$,在这个环里,$3\not=0$,$4\not=0$,但是$3\times 4=12=0$.

\begin{definition}{整环}
对于环$R$,如果它的乘法交换并且没有零因子,那么我们称这个环是一个\textbf{整环(domain)}.
\end{definition}

整环的概念可以记为“无零因子交换幺环”,用以指代它的三个关键特点:“无零因子”、“交换”和“有乘法单位元”.最后一个特点在本书的语境下显得冗余,因为我们限定环都是含有乘法单位元的;强调幺环只是为了避免使用其它术语体系时可能的混淆.

\begin{example}{整环的例子}
\begin{itemize}
\item 整数环
\item 多项式环
\item 高斯整数环$\mathbb{Z}[\I]$
\end{itemize}

\end{example}

接下来是一系列非常有用的概念,也是后续的进阶词条的基础.

\begin{definition}{因子和整除性}
在整环$R$中,如果对于$a, b\in R$,存在$r\in R$使得$b=ar$,那么我们称$a$整除$b$,记为$a|b$,同时称$a$是$b$的一个\textbf{因子(factor)}或\textbf{除数(divisor)}.
\end{definition}

\begin{definition}{单位}
对于整环$R$,如果$a\in R$有\textbf{乘法逆元}$a^{-1}$,那么称$a$是$R$的一个\textbf{单位(unit)}.
\end{definition}

\begin{definition}{真因子}
在整环$R$中,如果对于$a, b\in R$,存在$r\in R$使得$b=ar$,并且$r$不是一个单位,那么称$a$是$b$的一个\textbf{真因子(proper factor)}.
\end{definition}

\begin{definition}{素元素}
对于整环中的元素$p\in R$,如果它满足“如果任何$a, b\in R$使得$p|ab$,则必有$p|a$或者$p|a$”,则称它为$R$的一个\textbf{素元素(prime element)}.
\end{definition}





