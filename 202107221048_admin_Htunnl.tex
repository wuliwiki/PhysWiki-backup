% 氢原子隧道电离

\footnote{参考 Wikipedia \href{https://en.wikipedia.org/wiki/Tunnel_ionization}{相关页面}.}本文使用原子单位制\upref{AU}.

氢原子的电离率(单位时间的概率)在 $E \ll E_a$ 时为
\begin{equation}
\omega = \frac{4}{\abs{E}} \exp[-\frac{2}{3\abs{E}}]
\end{equation}


注意若给氢原子的哈密顿算符添加恒定电场项 $\bvec E \vdot \bvec r$, 那么该系统将不存在严格的束缚态. 因为在电场的反方向 $-\bvec E$ 当 $r\to\infty$ 时 $\bvec E \vdot \bvec r \to -\infty$.

\begin{figure}[ht]
\centering
\includegraphics[width=12cm]{./figures/Htunnl_1.png}
\caption{隧道电离示意图} \label{Htunnl_fig1}
\end{figure}

