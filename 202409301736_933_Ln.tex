% 对数与对数函数(高中)
% keys 对数|对数函数
% license Xiao
% type Tutor
\begin{issues}
\issueDraft
\end{issues}
\pentry{函数\nref{nod_functi},函数的性质\nref{nod_HsFunC},指数函数\nref{nod_HsExpF}}{nod_c094}
\subsection{对数运算}


对数运算有两个特殊的计算结果:$\log_a a=1,\log_a1=0$

\begin{theorem}{对数运算法则}
\begin{itemize}
\item 乘法法则$\log_a(xy)=\log_ax+\log_ay$
\item 换底公式:$\displaystyle \log_a b=\frac{\log_cb}{\log_ca}$
\item 幂法则:$\displaystyle \frac{m}{n}\log_a b=\log_{a^n} b^m$
\end{itemize}
\end{theorem}

\subsection{自然对数函数}
以 $\E$ 为底的对数函数 $\log_{\E} x$ 叫做\textbf{自然对数}, 通常记为
\begin{equation}
\ln x \qquad \text{或} \qquad \log x~.
\end{equation}
函数图如\autoref{fig_Ln_2}。
\begin{figure}[ht]
\centering
\includegraphics[width=7cm]{./figures/ce690bcbd8c28a93.png}
\caption{几种不同底的对数函数} \label{fig_Ln_2}
\end{figure}


\subsection{指数函数与对数函数的相似性}

\pentry{函数的变换(高中)\nref{nod_FunTra},导数(高中)\nref{nod_HsDerv}}{nod_a54a}

根据\aref{幂运算}{the_power_1}和对数运算的法则,任意$f(x)=a^x$都可以变形,得到:

\begin{equation}
f(x)=e^{x\ln a}~.
\end{equation}

即,所有的指数函数都可以由$\E^x$通过在$x$方向上伸缩或关于$y$轴对称($\ln a<0$时)得到,或者所有的指数函数$a^x$都可认为是$f(x)=e^x$与$g(x)=x\ln a$复合得到的$f(g(x))$。

同理,根据对数运算的性质,任意$f(x)=\log_ax$都可以变形,得到:

\begin{equation}
f(x)=\frac{1}{\ln a}\ln x\iff f(x)\ln a=\ln x~.
\end{equation}

即,所有的对数函数都可以由$\ln x$通过在$y$方向上伸缩或关于$x$轴对称($\ln a<0$时)得到,或者所有的对数函数$\log_ax$都可认为是$f(x)=\ln x$与$\displaystyle g(x)=\frac{x}{\ln a}$复合得到的$g(f(x))$。同时,这个结论也可由指数函数和对数函数在参数相同时互为反函数验证。

综上,所有的指数函数、对数函数之间都是相似的。根据上述关系,已知$\left(\E^x\right)'=\E^x,\left(\ln x\right)'=\frac{1}{x}$,根据复合函数求导法则,可知:
\begin{equation}
\left(a^x\right)'=\ln a\cdot\E^x~,\left(\log_a x\right)'=\frac{1}{x\ln a}.
\end{equation}
