% Christoffel 符号
% keys 克里斯托费尔|克氏符|克氏|Christoffel|测地线|geodesic|广义相对论|relativity

% \pentry{联络形式与结构定理\upref{ConFom}}
\pentry{黎曼联络\upref{RieCon}}

\addTODO{黎曼联络可能不需要}

流形的特点是局部与我们熟悉的欧几里得空间同胚。尽管我们经常讨论的是流形的内禀性质,不涉及具体的图或者嵌入,但是在实际应用的时候,比如计算广义相对论的现象时,我们却要关心特定图中的数值关系。本节引入的是著名的Christoffel符号,它描述了在特定图中联络的性质。Christoffel的计算实例,请参考\textbf{庞加莱半平面(微分几何计算实例)}\upref{PoiHP}词条。

本节中默认 $(M, \nabla)$ 是一个带仿射联络的流形。


\subsection{Christoffel符号的概念}\label{sub_CrstfS_1}

对于 $M$ 的任意一个图 $(U, \varphi)$,由于 $\varphi(U)$ 是一个欧几里得空间,即实数坐标空间,因此它的光滑向量场集合自带一组标准正交基 $\{\frac{\partial}{\partial x^i}\}$。为方便计,我们可以将每个导子 $\frac{\partial}{\partial x^i}$ 简记为 $\partial_i$。

点 $\varphi(p)=(\varphi(p)_1, \varphi(p)_2, \cdots)\in\varphi(U)$ 处和 $\partial_i$ 相对应的道路,可以取 $c_i:I\to \varphi(U)$ 为代表,其中 $c_i(0)=\varphi(p)_i$,且 $\frac{\dd}{\dd t}c_i(t)|_{t=0}=1$。

回忆\textbf{切空间(欧几里得空间)}\upref{tgSpaE}中的讨论,导子和道路都是“切向量”这一对象的等价描述。上面给出导子后又给了对应的道路,是为了提示你该如何将 $\varphi(U)$ 中的切向量通过 $\varphi^{-1}$ 映射为 $U\subseteq M$ 上的切向量。

每个图唯一对应一个量,称为Christoffel符号,如\autoref{def_CrstfS_1} 所示。

\begin{definition}{Christoffel符号}\label{def_CrstfS_1}

对于 $M$ 的图 $(U, \varphi)$,易知 $\{\partial_i\}$ 是 $\varphi(U)$ 上光滑向量场的基,因此存在一组光滑函数 $\Gamma^k_{ij}$,使得
\begin{equation}
\nabla_{\partial_i}\partial_j=\Gamma^k_{ij}\partial_k~.
\end{equation}

称 $\Gamma^k_{ij}$ 为联络 $\nabla$ 在图 $(U, \varphi)$ 上的\textbf{Christoffel 符号(symbol)},简称\textbf{克氏符}。
\end{definition}

回忆\textbf{爱因斯坦求和约定}\upref{EinSum}的规定,$\Gamma^k_{ij}$ 是由 $\Gamma$ 类型的元素构成的嵌套矩阵,这里每个元素都是一个\textbf{光滑函数}。如果固定 $i$ 和 $j$,那么 $\Gamma^k_{ij}$ 就是一个光滑函数构成的列矩阵,用来表示 $\partial_k$ 线性组合出 $\nabla_{\partial_i}\partial_j$ 的系数。

Christoffel符号的分量由所选择的图来决定,因此并不是流形上不变的量,这就把它和张量场区分开来。张量场的定义不依赖于图的选择,我们讨论的时候也都可以摆脱图来讨论,只不过当选定图了以后,一个张量场总可以表示为光滑函数的嵌套矩阵;但Christoffel符号就是依赖图来定义的,根本就没有脱离图的概念,所以要注意区分\footnote{记住,不是所有有上下标的东西都是张量场的。上下标的表示法,只是对嵌套矩阵的简化表达而已。}。


下面是一个重要的性质。

\begin{theorem}{无挠对称性}\label{the_CrstfS_1}
流形 $(M, \nabla)$ 是无挠的,\textbf{当且仅当}在任意图中,$\Gamma^k_{ij}=\Gamma^k_{ji}$。
\end{theorem}

\textbf{证明}:

$\Rightarrow$:

因为无挠,故 $\nabla_{\partial_i}\partial_j-\nabla_{\partial_j}\partial_i=[\partial_i, \partial_j]$。而由于欧几里得空间中偏微分算子的交换性,$[\partial_i, \partial_j]=0$,故 $\nabla_{\partial_i}\partial_j=\nabla_{\partial_j}\partial_i$。

又因为 $\{\partial_k\}$ 是线性不相关的,因此 $\Gamma^k_{ij}=\Gamma^k_{ji}$。

$\Leftarrow$:

由\autoref{exe_affcon_1}~\upref{affcon}可知,$T(f\partial_i, g\partial_j)=fgT(\partial_i, \partial_j)$。任意向量场都可以表示为 $f^i\partial_i$ 的形式,其中 $f^i$ 的类型是“光滑函数”。

将任意两个光滑向量场分别表示成 $f^i\partial_i$ 和 $g^j\partial_j$,于是有
\begin{equation}
T(f^i\partial_i, g^j\partial_j)=f^ig^jT(\partial_i, \partial_j)=f^ig^j(\Gamma^k_{ij}\partial_k-\Gamma^k_{ji}\partial_k-[\partial_i, \partial_j])~.
\end{equation}

由于 $\Gamma^k_{ij}=\Gamma^k_{ji}$,且偏微分算子交换,故上式为 $0$。

\textbf{证毕}。

由于\autoref{the_CrstfS_1} ,无挠的联络也常被称为\textbf{对称联络(symmetric connection)}。


\subsection{相关计算}

\subsubsection{计算具体坐标系中的联络}

\begin{exercise}{}
设在 $M$ 的某个图(坐标系)中,联络 $\nabla$ 的Christoffel符号为 $\Gamma^k_{ij}$,那么对于任意两个向量场 $x^i\partial_i$ 和 $y^j\partial_j$,证明:
\begin{equation}\label{eq_CrstfS_1}
\nabla_{x^i\partial_i}(y^j\partial_j)=[x^i(\partial_iy^s)+x^iy^j\Gamma^s_{ij}]\partial_s~.
\end{equation}
\end{exercise}

在具体坐标系中,我们往往为了简便,直接用坐标矩阵来代表一个向量,比如说将向量 $x^i\partial_i$ 表示为它的坐标矩阵 $x^i$。用这种坐标语言,\autoref{eq_CrstfS_1} 就写为
\begin{equation}\label{eq_CrstfS_4}
\nabla_{x^i}y^j=x^i(\partial_iy^s)+x^iy^j\Gamma^s_{ij}~.
\end{equation}

\subsubsection{由度规张量导出Christoffel符号}

在任意给定的坐标系中,黎曼度量用一个张量场 $g_{ij}$ 来表示,满足
\begin{equation}
<x^i\partial_i, y^j\partial_j>=x^iy^jg_{ij}~.
\end{equation}

矩阵 $x^i, y^j$ 和嵌套矩阵 $g_{ij}$ 的元素类型是该参考系上的光滑函数,因此符合“光滑向量场内积是光滑函数”的性质。

由\textbf{黎曼联络}词条中的\autoref{cor_RieCon_1}~\upref{RieCon},黎曼度量唯一地确定一个对称联络,即黎曼联络。具体到给定坐标系上的时候,这条推论的含义就变成了“度量张量场 $g_{ij}$ 决定了Christoffel符号 $\Gamma^k_{ij}$”。也就是说,我们应该可以用 $g_{ij}$ 计算出 $\Gamma^k_{ij}$。

考虑到偏微分算子的交换性,即 $[\partial_a, \partial_b]=0$ 恒成立,代入\autoref{eq_RieCon_6}~\upref{RieCon}后有:
\begin{equation}\label{eq_CrstfS_2}
\begin{aligned}
2<\nabla_{\partial_i}\partial_j, \partial_k>&=2<\Gamma^s_{ij}\partial_s, \partial_k>\\
&=2\Gamma^s_{ij}g_{sk}\\
&=\partial_ig_{jk}+\partial_jg_{ki}-\partial_kg_{ij}~.
\end{aligned}
\end{equation}

把\autoref{eq_CrstfS_2} 的最后两行同时乘以 $g^{kr}$
\begin{equation}
2\Gamma^{s}_{ij}g_{sk}g^{kr}=g^{kr}(\partial_ig_{jk}+\partial_jg_{ki}-\partial_kg_{ij})~.
\end{equation}

由指标的升降的约定法则,$g_{sk}g^{kr}=\delta_s^r$,进而有:
\begin{equation}\label{eq_CrstfS_3}
\Gamma^{r}_{ij}=\frac{1}{2}g^{kr}(\partial_ig_{jk}+\partial_jg_{ki}-\partial_kg_{ij})~,
\end{equation}

\autoref{eq_CrstfS_3} 就是“(伪)黎曼度量在导出无挠联络”在具体坐标系中的表示。






