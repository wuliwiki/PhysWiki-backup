% 物理学常数
% 国际单位|基本常数|光速|普朗克常数|元电荷|玻尔兹曼常数|阿伏伽德罗常数

\pentry{国际单位制\upref{SIunit}}

\subsection{精确定义的常数}
\footnote{本词条参考 Wikipedia \href{https://en.wikipedia.org/wiki/Physical_constant}{相关页面}, 以及 NIST 的 2018 CODATA \href{https://physics.nist.gov/cuu/Constants/Table/allascii.txt}{常数表}}2019 年 5 月 20 日生效的国际单位制\upref{SIunit}中精确定义了 7 个基本常数, 每个基本单位的大小都可以由这些常数的定义测量出来. 如果某个数值有测量误差, 我们把最后两位的不确定度写在括号中, 例如 $1.23(45)$ 表示 $1.23 \pm 0.45$.
\begin{table}[ht]
\centering
\caption{精确定义的常数}\label{Consts_tab1}
\begin{tabular}{|c|c|c|}
\hline
符号 & 精确值 & 名称 \\
\hline
$\nu_{Cs}$ & $9,192,631,770\Si{Hz}$ & 铯原子 133 基态超精细能级间的跃迁辐射电磁波频率 \\
\hline
$c$ & $299,792,458\Si{m/s}$ & 真空中的光速 \\
\hline
$h$ & $6.62607015\e{-34}\Si{Js}$ & 普朗克常数 \\
\hline
$e$ & $1.602176634\e{-19}\Si{C} $ & 元电荷 \\
\hline
$k_B$ & $1.380649\e{-23} \Si{J/K}$ & 玻尔兹曼常数 \\
\hline
$N_A$ & $6.02214076\e{23} $ & 阿伏伽德罗常数 \\
\hline
$K_{cd}$ & $683\Si{Im/W}$ & $540\Si{THz}$ 电磁波的照射效率 \\
\hline
\end{tabular}
\end{table}
另外约化普朗克常数定义为 $\hbar = h/(2\pi)$.

\addTODO{Kcd 这个常数有什么用?}

\subsection{力学}
\textbf{万有引力常数}\upref{Gravty}
\begin{equation}
G \approx 6.67430(15)\e{-11} \Si{m^3 kg^{-1} s^{-2}}
\end{equation}

\subsection{电动力学}

\textbf{真空磁导率}\upref{BioSav}
\begin{equation}
\mu_0 \approx 1.25663706212(19)\e{-6} \Si{H/m}
\end{equation}
在 2019 年更新前, 它被定义为 $4\pi\e{-7} \Si{H/m}$. 现在 $\mu_0$ 可以通过测量精细结构常数\upref{FinStr}获得.

\textbf{真空介电常数}\upref{ClbFrc}
\begin{equation}
\epsilon_0 = 1/(\mu_0 c^2) \approx 8.8541878128(13)\e{-12}
\end{equation}
相关词条: 电容\upref{Cpctor}.

\subsection{量子力学}
\textbf{玻尔半径}\upref{BohrMd}
\begin{equation}
a_0 = \frac{4\pi \epsilon_0 \hbar^2}{m_e e^2} = \frac{\hbar}{\alpha m_e c} \approx 5.29177210903(80)\e{-11}\Si{m}
\end{equation}

\textbf{精细结构常数}\upref{FinStr}
\begin{equation}
\alpha = \frac{e^2}{4\pi\epsilon_0\hbar c} \approx 7.2973525693(11)\e{-3}
\end{equation}

\textbf{玻尔磁子}\upref{BohMag}
\begin{equation}
\mu_B = \frac{e\hbar}{2m_e} = 9.2740100783(28)\e{-24} \Si{J/T}
\end{equation}
\textbf{原子质量单位}(仍然由 C12 定义)
\begin{equation}
m_u = 1.66053906660(50)\e{-27} \Si{kg}
\end{equation}
\textbf{电子质量}
\begin{equation}
m_e = 9.1093837015(28)\e{-31}
\end{equation}
\textbf{质子质量}
\begin{equation}
m_p = 1.67262192369(51)\e{-27}\Si{kg}
\end{equation}
\textbf{中子质量}
\begin{equation}
m_n = 1.67492749804(95)\e{-27}\Si{kg}
\end{equation}

\subsection{统计力学}
理想气体常数(精确)\upref{PVnRT}
\begin{equation}
R = k_B N_A = 8.31447165136438 \,\Si{J/K}
\end{equation}
斯特藩—玻尔兹曼常数(精确)\upref{SteBol}
\begin{equation}
\sigma = \frac{2\pi^5k_B^4}{15c^2h^3} = 5.670374419\dots\e{-8} \Si{Wm^{-2}K^{-4}}
\end{equation}
