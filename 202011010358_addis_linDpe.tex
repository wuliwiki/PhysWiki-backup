% 线性相关性

\pentry{几何矢量的运算\upref{GVecOp}}

如果存在至少一组\textbf{不全为零}系数 $c_i$ 使几个矢量的线性组合等于零, 这些矢量就被称为\textbf{线性相关(linearly dependent)}的
\begin{equation}\label{linDpe_eq2}
\sum_i^N c_i \bvec v_i = \bvec 0
\end{equation}
这是因为对于任何一个 $c_j$ 不为零的项, 矢量 $\bvec v_j$ 都可以表示为其他矢量的线性组合. 只需把上式除以 $c_j$ 即可
\begin{equation}\label{linDpe_eq3}
\bvec v_j = -\sum_{i \ne j}\frac{c_i}{c_j} \bvec v_i
\end{equation}
如果不存在满足\autoref{linDpe_eq2} 的非零系数 $c_i$, 这些矢量就是\textbf{线性无关(linearly independent)}的, 即任何矢量都不可能被其他矢量的线性组合表示. 

\begin{exercise}{}
试说明任意两个共线的矢量必然是线性相关的, 平面任意三个矢量必然是线性相关的.
\end{exercise}

如果一个矢量集合中的矢量是线性相关的,那么这个集合被称为一个\textbf{线性相关组};反之,若线性无关,则称为一个\textbf{线性无关组}.

如果一组矢量之间线性相关,那么至少有一个矢量是“冗余”的,也就是说,它可以被其它矢量的线性组合表示出来.这样一来,对于线性相关的矢量组,如果用它们的线性组合来表示其它矢量,那么表示方式都不是唯一的.线性无关的矢量组,最重要的性质就是它们的线性组合表达式是唯一的,由此引入了基底和坐标\upref{Gvec2}等概念.

显然, 给定一个非零的线性相关组, 通过逐个移除这些“冗余”的矢量, 我们总可以得到一个线性无关组.
