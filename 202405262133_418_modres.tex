% 同余与剩余类
% keys 同余|剩余类
% license Usr
% type Tutor

\pentry{整除\nref{nod_divisb}}{nod_e98f}

\begin{definition}{同余与剩余}
若(整数) $m$ 是(整数) $x$ 减去(整数) $a$ 的一个因子,即 $m | (x-a)$,则称 $x$ 与 $a$ 模\footnote{此处的“模”对应英文 $\operatorname{mod}$,与整数模无关。} $m$ \textbf{同余},记作
\begin{equation}
x \equiv a \pmod m ~.
\end{equation}
同时,若 $x \equiv a \pmod m$,就 说 $a$ 是作 $x$ 模 $m$ 的一个\textbf{剩余(residue)}。若还有 $0 \le a < m$,就说 $a$ 是作 $x$ 模 $m$ 的\textbf{最小剩余(least residue)}。
\end{definition}
有时候会忽略整数条件,此时可以说 $\pi \equiv -\pi \pmod{2\pi}$。即只要 $x-a$ 是 $m$ 的整数倍即可。

\begin{definition}{剩余类}
在模 $m$ 的前提下,与给定的某个剩余关于 $m$ 同余的所有数将组成一个集合,称为模 $m$ 的一个\textbf{剩余类(class of residue)}。而这剩余类中的每个元素都叫做这个类的一个\textbf{代表(represent)}。
\end{definition}

\begin{corollary}{}
显然,$m$ 共有 $m$ 个不同的剩余类,分别有代表
\begin{equation}
0, 1, \dots, (m-1) ~.
\end{equation}
\end{corollary}
\begin{definition}{完全剩余系}
对于模 $m$ 的前提下,将会有 $m$ 个不同的剩余系分别有代表 $0, 1, \dots, (m-1)$。对于任意 $m$ 个分别属于这 $m$ 个不同的剩余系的数,这 $m$ 个数组成的集合都称为\textbf{模 $m$ 的一个完全剩余系},简称模 $m$ 的一个\textbf{完系(complete system)}。
\end{definition}

同余式有一些经典的性质:
\begin{itemize}
\item $a \equiv b$ $\Leftrightarrow$ $b \equiv a$;
\item $a \equiv b$, $b \equiv c$ $\rightarrow$ $a \equiv c$;
\item $a \equiv a'$, $b \equiv b'$ $\rightarrow$ $a + a' \equiv b + b'$。
\end{itemize}
此外,在令 $a \equiv a', b \equiv b', c \equiv c', \dots$ 的前提下,还有
\begin{itemize}
\item $k_a a + k_b b + \cdots \equiv k_a a' + k_b b' + \dots$;
\item $a^2 \equiv a'^2$, $a^3 \equiv a'^3$, $\dots$。
\end{itemize}
以及对于 $\varphi(a, b, \dots)$ 是某\textbf{整系数多项式},则
$$\varphi(a, b, \dots) \equiv \varphi(a', b', \dots) ~.$$

\pentry{最大公约数与最小公倍数\nref{nod_gcdlcm}}{nod_6ac6}
\begin{theorem}{}
若 $a \equiv b \pmod n$ 且 $a \equiv b \pmod m$,则 $a \equiv b \pmod{[m, n]}$,其中 $[m, n]$ 是 $n$ 与 $m$ 的最小公倍数。
\end{theorem}
这定理有许多方法证明,这里给出一个证明思路:考虑 $p^c$ 是能整除 $[m, n]$ 的素数 $p$ 的最高次幂,则 $p^c | n$ 或 $p^c | m$,故 $p^c | (a - b)$,这对于 $[m, n]$ 的每个素因子都成立,即可证明。

\begin{definition}{缩系}

\end{definition}