% 暗物质在银河系和宇宙中的分布
% license Usr
% type Tutor



宇宙中的平均暗物质密度非零。然而,暗物质远不止这个平均值。由于结构形成,暗物质在特定系统中的分布与平均宇宙学值有显著差异。特别是,为了解释直接和间接暗物质探测搜索的结果,人们需要了解银河系以及其他星系和各种其他天体系统中的暗物质密度分布ρ(x)和暗物质速度分布f(x, v)。目前,这些量的估计仅能做出有根据的猜测,存在显著的不确定性。在本章中,我们将讨论这些猜测是如何得出的,以及它们的结果。我们首先从两个简单但同时也非常有力的一般性陈述开始:

1.暗物质倾向于在受引力束缚的系统中大致呈球形分布。结构形成的初始条件通常预测近乎球形的分布。正常的重子物质随后在引力井中坍缩,在这一过程中它耗散能量并冷却下来,形成了许多星系所展示的旋转盘。除非暗物质具有同样大的散射截面,否则它具有可忽略的相互作用,从而具有可忽略的耗散。忽略可见物质的引力效应,暗物质因此保持球形分布,因此可以合理假设天体系统中的暗物质密度ρ(r)在很大程度上只是径向坐标r的函数。

2.在大多数感兴趣的系统中,暗物质是非相对论性的。例如,束缚在我们银河系中的暗物质粒子必须具有低于逃逸速度的速度,vesc ≈ 500 km/s。对于其他系统,如矮星系、星系团等,也有类似的考虑,只是适当的vesc不同。

超越上述一般性陈述,任何给定系统的ρ(r)和f(v)原则上可以通过三种概念上独立的途径确定:

(A) 通过精确追踪星系中的恒星运动学(或星系团中的星系运动学),推断出潜在的引力势,从而得出观测到的追踪器运动所负责的物质的密度和速度分布;

(B) 通过理论方法,(半)解析地描述引力坍缩的过程及其最终结果;

(C) 通过N体数值模拟。

实际上,

(A) 太不精确,无法提供完整的ρ(r),但它有助于确定局部暗物质密度;

(B) 面临形式和技术上的困难,这是由于问题的非线性特性以及由于重子物理的存在,然而它提供了对所涉及复杂过程的宝贵见解。

因此,当前最流行的方法主要依赖于(C),辅以其他策略。通常,这是一个两步过程:

1. 基于(通常是)模拟、理论或观测,猜测ρ(r)和f(v)的功能形式,用最少数量的自由参数来表示。
2. 根据对我们银河系或其他星系中暗物质的可靠观测来确定自由参数。

我们将在2.2节和2.3节中分别展示这一过程的结果,分别针对密度和速度分布。最后,在2.4节中,我们将讨论如何进一步改进,超越“暗物质球形奶牛”的限制。然而,在进行之前,值得详细阐述上述三种方法。

\subsection{确定暗物质分布的方法} 

基本观测到银河系旋转曲线趋于平坦,允许推断出在暗物质占主导地位的较大半径r处,ρ(r) ∝ 1/r^2。可以使用相同的想法尝试在较小的r处提取ρ(r):绕银河中心做圆周运动的物体本质上是总引力势的追踪器,因此也是物质分布的追踪器。使用这些追踪器得到的旋转曲线与从星系中可见物质预期的旋转曲线之间的不匹配必须由暗物质来解释。因此,可以通过将适当参数化的函数拟合到总旋转曲线上来得到暗物质密度分布[31]。这通常被称为确定暗物质分布的全局方法(与下面讨论的局部方法相对,后者较为保守,只旨在确定太阳系位置和周围1 kpc内的暗物质密度)。该方法概念上很简单。为了获得旋转曲线的预测,首先假设只有由于恒星和星系间气体中的重子观测到的质量分布作为源,解决泊松方程∇²φ = 4πGρ,考虑到它们的近似柱状(而非球状)几何形状r, z, θ。然后牛顿方程v²/r = −∂φ/∂r预测了在z = 0的银河平面上的旋转速度v(r) = (v_star² + v_gas²)^(1/2),这些速度是测量得到的。3与v(r)的数据比较表明,需要额外的“暗”贡献才能重现观测结果,
\begin{equation}
v^2 = v_{star}^2 + v_{gas}^2 + v_{dark}^2~.
\end{equation}
然后通过从观测值中减去预测值来确定暗物质的贡献。

然而,出现了一些困难。首先,恒星运动学数据仍然存在显著的不确定性,这些不确定性既来自测量本身,也来自其解释(在银河系中也是如此,例如,与太阳到银河中心的精确距离和太阳的圆周速度有关)。其次,上述不匹配拟合的结果在很大程度上取决于对星系可见组分的假设。更具体地说,必须设计一个星系模型,将其作为由球状体、可能的条带、盘、气体晕,甚至可能还有球状体核心的较小盘组成的系统。...,每个部分都使用几种基于观测的配置进行适当建模。气体密度可以从21厘米图谱中提取,而恒星模型则需要将观测到的光度转换为质量密度。尽管最近由于基于光学观测的数据现在正被近红外数据所取代,这些数据只需要进行小幅度的建模修正,但这些重子形态的不确定性仍然很大。最后,由于重子物质在星系的前几kpc(银河系的前约5 kpc)内主导了引力势,因此在该区域内确定暗物质的贡献本质上非常困难。

因此,在实践中,观测方法足以确认星系中暗物质的存在,并对其大规模分布有一个大致的了解,但目前还不足以进行更精确的确定。 

