% 顺磁质的磁化
% 磁化|磁矩|顺磁性|逆磁性

对顺磁质而言,虽然每个原子或分子有一定的磁矩,但由于分子的无规则热运动,各个分子磁矩排列的方向是十分纷乱的,对顺磁质内任何一个体积元来说,其中各分子的分子磁矩的矢量和$\sum \mathbf{m}_{mole}=0$,因而对外界不显示磁效应.在外磁场$\mathbf B_0$的作用下,分子磁矩$\mathbf{m}_{mole}$的大小不改变,但是外磁场$\mathbf B_0$要促使,$\mathbf{m}_{mole}$绕磁场方向进动,并具有一定的能量.这里的能量,是说固有磁矩$\mathbf m_{mole}$在外磁场中的能量为$W_m=-\mathbf m_{mole} \vdot \mathbf B$.由此可见,当$\mathbf m_{mole}$与$\mathbf B_0$方向相同时,能量最低.

我们知道,介质中存在着大量原子或分子,由于这些原子或分子之间的相互作用和碰撞,促使分子磁矩$\mathbf m_{mole}$改变方向,从而改变$\mathbf m_{mole}$在外磁场中的能量状态.一方面,从能量的角度来看,分子磁矩尽可能要处于低的能量状态,即$\mathbf m_{mole}$与外磁场方向一致的状态;另一方面,分子热运动又破坏了分子磁矩沿磁场方向有序排列.当达到热平衡时,原子或分子的能量遵守玻耳兹曼分布律,处在较低能量状态的原子数或分子数比高能量状态的要多,亦即其分子磁矩$\mathbf m_{mole}$靠近外磁场方向的分子数较多.显然,磁场越强,温度越低,分子磁矩$\mathbf m_{mole}$排列也越整齐,这时,在顺磁体内任取一体积元$\Delta V$,其中各分子磁矩的矢量$\sum \mathbf m_{mole}$将有一定的量值,因而在宏观上呈现出一个与外磁场同方向的附加磁场.这就是顺磁性的来源.

应当指出,顺磁质受到外磁场的作用后,其中的原子或分子也会产生抗磁性,但在通常情况下,多数顺磁质分子的附加磁矩$\sum \Delta\mathbf m_{mole}$比$\sum \mathbf m_{mole}$小很多,所以这些磁介质主要显示出顺磁性.