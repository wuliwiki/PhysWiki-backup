% 南京理工大学 2010 年 研究生入学考试试题 普通物理(B)
% license Usr
% type Note

\textbf{声明}:“该内容来源于网络公开资料,不保证真实性,如有侵权请联系管理员”

\subsection{一。填空题(32分,每空2分)}
\begin{enumerate}
    \item 已知一电子的运动方程可表示为 $r = b \cos \omega t\vec{i} + b \sin \omega t\vec{j} + ct\vec{k}$,式中 $a,b$ 为常数,$t$以秒计。$r$以来计,随在$t$时刻,电子的速度为__________,加速度为 ___________。
    \item 一质量为$m$的小球系在长为$L$的细绳的一端,绳的另一端固定于$O$点。先使小球以$v_0$速度做圆周水平匀速运动,然后细绳逐渐缩短,绳始终与运动方向夹角为$\theta$的小球的速度表达式为___________ ,细绳的张力为多大为 ___________。
    \item 设一平面简谐波沿 $x$ 轴正方向传播,已知 $x=0$ 处原点的振动方程为$ y = A \cos(\omega t - \pi/3)$,波速为 $v$。波在 $x=L$ 处终止反射,则$x=x_0$ 处 $(x_0 < L)$ 原点由于反射被引起的振动方程为 ___________ ,$x_0$ 处是波节位置的条件是 $x_0 =$___________。
    \item 如图所示,一沿正 $x$ 方向传播的平面简谐波,波速为 $v = 200 m/s$,波长 $\lambda = 20 m$,则 $x = 0$ 处质点的振动方程为_________,该平面简谐波方程为_________。
\begin{figure}[ht]
\centering
\includegraphics[width=6cm]{./figures/31f0d09795712f3c.png}
\caption{} \label{fig_NJU10_1}
\end{figure}
    \item $2 mol$ 氧气在27°C时的内能等于 ___________,其分子的平均动能是 ___________ ,平均平动动能是 ___________。
    \item 设一个气体分子的密度分布函数为$f(v)$,则单位体积中,$v_1\to v_2$区间内的分子数为 ___________。
    \item 带电量为 $q$ 半径为 $R_1$ 的导体球 $A$,与内、外半径分别为 $R_2$ 和 $R_3$ 接地的同心金属球壳 $B$ 间充满介电常数为 $\varepsilon$ 的介质,构成一球形电容器。则该电容器的电容 $C=$___________。设导体球 $A$ 带电 $q$,则该电容器内任一点 $P$ 处的电场强度 $E=$___________,电容器储存的电能___________。若球壳$B$ 接地,则导体球 $A$ 的电势为 ___________。
\end{enumerate}
\subsection{二、填空题(32分。每空2分)}
1. 半径为 $R$ 的圆环,均匀带电,单位长度的电量为 $\lambda$。 以每秒 $n$ 转绕 通过环心并与环面需直的轴作等速转动。则环的等效磁矩大小为___________,轴线上距坏心为$x$处的任一点$p$的磁感应强度大小为___________。
\begin{figure}[ht]
\centering
\includegraphics[width=6cm]{./figures/9eca76d12caabf5d.png}
\caption{} \label{fig_NJU10_2}
\end{figure}
2. 均匀磁场 $B$ 中置一直角边长为 $a$通有强度为 $I$ 的稳恒电流的等腰直角三角形线圈 $ACD$。 使线圈绕 $AC$ 边匀速转动,线圈平面与磁场方向平行。 如图所示, 现线圈所受的力矩的大小为___________。 在磁力矩作用下,线圈平面绕 $AC$ 边转过 $\pi/3$ 圈, 磁力矩做的功($I$ 在旋转过程中不变)为___________。
\begin{figure}[ht]
\centering
\includegraphics[width=6cm]{./figures/1b69f57a20f285d4.png}
\caption{} \label{fig_NJU10_3}
\end{figure}
3. 在恒真空场的均匀磁场中,长为 $l$ 的导体棒 $ab$ 以$\omega$逻时针绕$a$点勾速转动, 如图, 则切出电动势的大小为___________,且 ___________ 点的电势为。
\begin{figure}[ht]
\centering
\includegraphics[width=6cm]{./figures/bc3edc382260c554.png}
\caption{} \label{fig_NJU10_4}
\end{figure}
4. 在真空中,一平面电磁波的磁场 $B = B_0 = B_0 \cos \left( \omega (t + \frac{z}{c} )\right) \\, (T)$。则该电磁波的传播方向为___________,电场强度为___________。

5.用氮-氖撒光器发出的波长为$632.8nm$的单色光徽牛领环实验,已知所用平凸透镜的鱼率半径为$10.0m$,平面直径为$3.0cm$,则能观察列___________,条暗坏。若把整个装置放入水中$(\eta=1.33)$,总观察到___________,条暗坏:

6、两个偷报化方向相互看宜的偏报片平行放置,组合战正交偏报片。现让光强为$I_0$的一東自然光垂直射入谈正交偏探片,则透射光强为___________。若在阿偏振片之间放入第三块偷报片。其偏化方向与第一块偷振片的偏振化方向央角为 30”,则透射光强为___________。

7、动能$E=1.53MV$的电子运动的德布罗意波长为___________。

8、处于第一灏发态的氢原子的势能为___________,其核外电子绕核运动的动能为___________。

9. 已知一维无限深势井中粒子的波函数为: $\psi_n(x) = \sqrt{\frac{2}{a}} \sin{\frac{n \pi }{a}}x$. 设 $n = 1$ 时, 粒子在 $x = \frac{a}{3}$ 处出现的概率密度为: ___________。
\subsection{三、(12分)}
如图所示,水平桌面上。一长为1-10m:质量为m-3.0的匀质烟杆,一端围定于$O$点,细杆可经过0点竖直轴在水平桌面上标动烟杆与桌面间的摩操系数为$\eta=0.20$,开始时杆静止。现有一质量为,20g:速度w-400ay,沿水平方向以与杆成0=30”常射入杆的中点日留在杆内。来:
(1)擅击后杆开始神动的角速度大小:
(2)子弹射入后,细扦所受摩擦力矩:
(3)烟杆的角加速度
