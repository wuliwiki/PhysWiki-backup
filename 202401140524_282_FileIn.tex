% 文件管理导航
% license Usr
% type Map

\begin{issues}
\issueDraft
\end{issues}

一个储存设备本质上就是一根很长的纸带,划分成许多小格,每个格子可以读或写上 0 或 1。 并且每个读写都可以指定读或写第几个格子。这就是著名的图灵机中的储存模型。

这个模型看起来很简单,但在实际中我们通常对储存设备有一些性能上的要求:
\begin{itemize}
\item 纸带足够长,可以容纳需要保存的数据
\item 读写速度快
\item 写入的数据长时间不会自然损坏
\item 有防误写的机制,例如可以查看纸带在之前某个时间点的数据
\end{itemize}

\addTODO{需要一个硬盘常识的词条}

计算机文件简介\upref{ComFil}

计算机文件备份基础(附 python 多版本增量备份脚本)\upref{SimBac}

如何给文件加密(含 python 加密脚本)\upref{encryp}

用网盘增量备份文件\upref{PanBak}

Git 笔记\upref{Git}

用 Git 备份文件夹\upref{gitBac}

Git-LFS 笔记\upref{gitLFn}

ZFS 文件系统(Zettabyte File System)笔记\upref{ZFS}
