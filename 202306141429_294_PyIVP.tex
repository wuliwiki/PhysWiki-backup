% SciPy 求解常微分方程组的初值问题
% keys 常微分方程组|Python|数值解
\subsection{微分方程组}
这里我们再次以“天体运动的简单数值计算\upref{KPNum0}” 中的问题为例,利用 Python 语言实现微分方程组
\begin{equation}\label{eq_PyIVP_10}
\begin{cases}
x' = v_x\\
y' = v_y\\
v'_x = -GMx/(x^2 + y^2)^{3/2}\\
v'_y = -GMy/(x^2 + y^2)^{3/2}
\end{cases}~
\end{equation}
的数值解,其中参数 $GM=1$。
\subsection{solve\_ivp 求解器介绍}
在对该问题求解之前,我们先简单介绍\verb|Python|中关于微分方程数值求解库极相关函数的用法。一般,最常用的数值计算库为\verb|scipy|,而微分方程求解对应的模块为\verb|scipy.integrate.solve_ivp|.

 \verb|solve_ivp|的一般格式为
 \begin{lstlisting}[language=python]
 solve_ivp(fun, t_span, y0, method='RK45', t_eval=None,max_step=np.inf,
        dense_output=False, events=None, 
        vectorized=False, args=None, **options)
 \end{lstlisting}
 其中, 输入参数分别为
\begin{itemize}
\item \verb|fun|是微分方程(组)的右端;
\item  \verb|t_span=(t0,tf)|代表积分区间为\verb|t0|到\verb|tf|;
\item  \verb|y0|为初始条件;
\item \verb|method|为数值积分方法,这里可以使用的积分方法有\verb|RK45|、\verb|RK23|、\verb|DOP853|(8阶显式龙格库塔法)、\verb|Radau|(5阶Radau IIA族的隐式Runge-Kutta方法)、\verb|BDF|(隐式多步变阶)、\verb|LSODA|(具有自动刚度检测和切换的Adams/BDF方法)等。需要主要的是显式Runge-Kutta方法(\verb|RK23、RK45、DOP853|)应用于非刚性问题,隐式方法(\verb|Radau、BDF|)应用于刚性问题。
\item \verb|t_eval|是可选参数,需要数组类型数据。如果给定就在这些时间点进行求解,否则求解器自己选择时间点进行求解。
\item \verb|max_step|允许的最大步长。默认为\verb|np。inf|,即步长不受约束,仅由解算器确定。如果求解不能满足需求,可以手动改变最大步长已达到精度要求。相应的还有最小步长参数\verb|min_step|.
\item \verb|dense_output|为布尔类型,默认为假,是否计算连续解。
\item \verb|args|为元组(\verb|tuple|)类型的参数,用于向微分方程传递必要的参数
\item 后面还有许多可选参数,很少会使用。
\end{itemize}

函数返回值分别为
\begin{itemize}
\item  \verb|t| 返回计算的时间点数据,是一个 \verb|ndarray| 数据类型,长度为 \verb|(n_points,)|.
\item \verb|y| 大小为 \verb|(n, n_points)| 的 \verb|ndarray| 的微分方程解的数据。
\item 还有许多返回参数,感兴趣读者可以去\href{https://scipy.github.io/devdocs/reference/generated/scipy.integrate.solve_ivp.html#r179348322575-7}{官网}查看。
\end{itemize}


\subsection{代码实现}
基于对\verb|solve_ivp|的使用说明,我们接下来对微分方程组\ref{eq_PyIVP_10} 进行数值计算。具体代码如下:
\begin{lstlisting}[language=python]
# 导入必要的库
from scipy.integrate import solve_ivp
import  numpy as np
import matplotlib.pyplot as plt

# 定义一些参数
GM = 1 # 万有引力常数乘以中心天体质量
x0,y0 = 1,0 # 初始位置
vx0,vy0 = 0,0.7 # 初始速度
tspan = (0, 4) # 总时间和步数
init0=(x0,y0,vx0,vy0)

# 定义微分方程
def odefun(t,z,GM):
    x,y,vx,vy  = z
    temp = -GM /(x**2 + y**2)**(3/2)
    return [vx,vy, temp * x, temp * y]
# 调用求解器求解微分方程
sol = solve_ivp(odefun,tspan,init0,dense_output=True,
          max_step=0.001,args=(GM,))
t = np.linspace(0, 4, 3000)
z = sol.sol(t)


# 微分方程解的可视化

# xy的时序图
plt.plot(t, z.T)
plt.xlabel('t')
plt.legend(['x', 'y','vx','vy'], shadow=True)
plt.show()

# xy的相图
plt.plot(z[0],z[1])
plt.xlabel('t')
plt.xlabel('x')
plt.ylabel('y')
plt.title('Lotka-Volterra System')
plt.show()
\end{lstlisting}

对应的输出如图所示。
\begin{figure}[ht]
\centering
\includegraphics[width=8cm]{./figures/3679b4c98257f195.png}
\caption{位置与速度关于时间的图形。} \label{fig_PyIVP_1}
\end{figure}

\begin{figure}[ht]
\centering
\includegraphics[width=8cm]{./figures/e6a11677ba73dd4d.png}
\caption{$x$ 与 $y$ 的相图。} \label{fig_PyIVP_2}
\end{figure}

我们现在来看代码的第\verb|2-4|行是一些基本必要库的导入。 
\verb|7-9|行为方程的一些参数与初始条件。
\verb|10|行为积分区间。注意它是一个元包数组,也可以写成\verb|[t0,tf]|形式。
\verb|14-17|行为微分方程的定义,这里需要传递一个参数\verb|GM|.
\verb|19|行为调用求解器求解微分方程组,我们使用了\verb|dense_output|参数,这样我们就可以在连续的对时间取值计算对应的\verb|x,y,vx,vy|. 
\verb|28-39|行均为作图部分。
