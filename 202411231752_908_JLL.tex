% 伽利略·伽利莱(综述)
% license CCBYSA3
% type Wiki

本文根据 CC-BY-SA 协议转载翻译自维基百科\href{https://en.wikipedia.org/wiki/Galileo_Galilei}{相关文章}。

\begin{figure}[ht]
\centering
\includegraphics[width=6cm]{./figures/f16d6423773a5687.png}
\caption{约1640年的肖像画} \label{fig_JLL_1}
\end{figure}
伽利略·迪·文琴佐·博纳尤蒂·德·伽利略(\textbf{Galileo di Vincenzo Bonaiuti de' Galilei},1564年2月15日-1642年1月8日),通常简称为伽利略·伽利莱(\textbf{Galileo Galilei},/ˌɡælɪˈleɪoʊ ˌɡælɪˈleɪ/,美国英语中也读作 /ˌɡælɪˈliːoʊ -/;意大利语:[ɡaliˈlɛːo ɡaliˈlɛːi]),或单名称为伽利略(\textbf{Galileo}),是一位佛罗伦萨的天文学家、物理学家和工程师,有时被描述为“博学多才之人”。他出生于当时属于佛罗伦萨公国的比萨市。伽利略被誉为观测天文学之父、现代经典物理学之父、科学方法之父以及现代科学之父。

伽利略研究了速度与加速度、重力与自由落体、相对性原理、惯性、抛体运动等,并在应用科学和技术领域开展了工作,描述了摆的特性和“静水天平”。他是文艺复兴早期温度计(即热测量仪)的开发者之一,还发明了多种军用罗盘。通过他改进的望远镜,伽利略观测到了银河的恒星、金星的位相、木星的四大卫星、土星的光环、月球的陨石坑以及太阳黑子。他还制作了一种早期显微镜。

伽利略对哥白尼日心说的支持遭到了天主教会内部和一些天文学家的反对。1615年,罗马宗教裁判所对这一问题进行了调查,得出结论认为他的观点与当时普遍接受的《圣经》解释相矛盾。[9][10][11]

后来,伽利略在《两大世界体系的对话》(1632年)中为自己的观点辩护,但书中似乎对教皇乌尔班八世进行了攻击和嘲讽,这使伽利略疏远了教皇及之前一直大力支持他的耶稣会士。[9] 他因此受到宗教裁判所的审判,被认定为“严重可疑的异端”,并被迫公开认错。伽利略此后被软禁在家中度过余生。[12][13] 在此期间,他撰写了《两种新科学》(1638年),主要探讨了运动学和材料强度问题。[14]
\subsection{早年生活与家庭}  
伽利略于1564年2月15日出生在比萨(当时属于佛罗伦萨公国),是文琴佐·伽利莱和朱莉亚·阿曼纳蒂的长子。他的父亲文琴佐是著名的鲁特琴演奏家、作曲家和音乐理论家,他的母亲朱莉亚是当地一位显赫商人的女儿。两人于1562年结婚,当时文琴佐42岁,而朱莉亚24岁。伽利略自己也成为了一名出色的鲁特琴演奏家,并可能从父亲那里早早学会了对权威的怀疑态度。[15][16]  

伽利略的五个兄弟姐妹中,有三个在婴儿期幸存下来。他最小的弟弟米开朗基罗(或称米开拉尼奥洛)也成为了一名鲁特琴演奏家和作曲家,但这一生都给伽利略带来了经济负担。[17] 米开朗基罗未能履行父亲承诺的嫁妆分担责任,这导致伽利略的妹夫们试图通过法律途径追讨欠款。此外,米开朗基罗有时还需要向伽利略借钱,以支持他的音乐事业和旅行。这些经济压力可能促使伽利略早年就产生了发明一些能够带来额外收入的装置的想法。[18]  

伽利略八岁时,家人搬到了佛罗伦萨,但他被留在比萨,由穆齐奥·泰达尔迪照顾了两年。十岁时,他离开比萨,与家人团聚在佛罗伦萨,并开始接受雅科波·博尔吉尼的指导。[15] 从1575年至1578年,他在佛罗伦萨东南约30公里的瓦隆布罗萨修道院接受教育,特别是在逻辑方面的学习。[19][20]
\subsubsection{名字}  
伽利略通常只用他的名字来称呼自己。在当时的意大利,姓氏并非必需,而他的名字“伽利略”(Galileo)与他有时使用的家族姓氏“伽利莱”(Galilei)源于同一祖先。无论是他的名字还是姓氏,最终都可追溯到他的祖先伽利略·博纳尤蒂(Galileo Bonaiuti),一位15世纪佛罗伦萨的重要医生、教授和政治家。[21] 伽利略·博纳尤蒂被安葬在佛罗伦萨的圣十字大教堂,这也是大约200年后伽利略·伽利莱的安葬地。[22]  

当伽利略使用多于一个名字时,他有时称自己为“伽利略·伽利莱·林切奥”(Galileo Galilei Linceo),以表示他是林切伊学院(Accademia dei Lincei)的一员。该学院是教皇国成立的一所精英科学组织。在16世纪中期的托斯卡纳,长子通常以父母的姓氏命名为名字。[23] 因此,伽利略·伽利莱的名字未必是专门为了纪念他的祖先伽利略·博纳尤蒂。意大利男性名“伽利略”(Galileo)及其派生的姓氏“伽利莱”(Galilei)来源于拉丁语“Galilaeus”,意为“加利利人”。[24][21]  

伽利略名字和姓氏的圣经渊源后来成为一个著名双关语的主题。[25] 1614年,在伽利略事件期间,伽利略的一位反对者、多米尼加会士托马索·卡奇尼(Tommaso Caccini)发表了一篇颇具争议且影响深远的布道词。他在布道中引用《使徒行传》1:11说道:“加利利人哪,你们为什么站着望天呢?”(可能含有对伽利略的讽刺)。[citation needed]  
\subsubsection{子女}  
\begin{figure}[ht]
\centering
\includegraphics[width=6cm]{./figures/79d02887461111a7.png}
\caption{据认为是伽利略长女维尔吉尼亚的肖像,她对父亲特别忠诚。} \label{fig_JLL_2}
\end{figure}
尽管伽利略是一位虔诚的天主教徒,[26] 他与玛丽娜·甘巴(Marina Gamba)未婚生育了三个孩子。他们育有两个女儿:维尔吉尼亚(Virginia,生于1600年)和利维娅(Livia,生于1601年),以及一个儿子:文琴佐(Vincenzo,生于1606年)。[27]  

由于孩子们是非婚生的,伽利略认为两个女儿难以嫁人,且无法承担高昂的经济支持或嫁妆费用。这些费用可能会使伽利略重蹈帮助两位姐妹嫁人的经济困境。[28] 因此,她们唯一合适的选择就是修道生活。两位女儿都被阿尔切特里的圣马太修道院接纳,并在此度过了余生。[29]  

维尔吉尼亚进入修道院后改名为玛利亚·切莱斯特(Maria Celeste)。她于1634年4月2日去世,并与伽利略合葬在佛罗伦萨的圣十字大教堂。利维娅进入修道院后改名为阿尔坎杰拉修女(Sister Arcangela),大部分时间身体都不好。伽利略的儿子文琴佐后来被确认为合法继承人,并与塞斯蒂莉亚·博基内里(Sestilia Bocchineri)结婚。[30]  
\subsection{职业生涯与首次科学贡献}
伽利略年轻时曾认真考虑成为一名神职人员,但在父亲的敦促下,他于1580年进入比萨大学学习医学。[31] 他受到了佛罗伦萨的吉罗拉莫·博罗(Girolamo Borro)和弗朗切斯科·博纳米奇(Francesco Buonamici)讲座的影响。[20] 1581年,在学习医学时,他注意到一个吊灯在空气流动的作用下以大小不同的弧度摆动。他发现,通过与自己的心跳比较,无论吊灯摆动的幅度多大,摆动一次所需的时间似乎相同。回到家后,他用两根等长的摆进行了实验,一个摆幅较大,另一个摆幅较小,发现它们的摆动时间相同。直到近一百年后,克里斯蒂安·惠更斯(Christiaan Huygens)的研究才将摆的等时性用于制造精准的计时器。[32]  

在此之前,伽利略被刻意远离数学,因为医生的收入高于数学家。然而,在偶然听了一场几何学讲座后,他说服了不情愿的父亲,让他改学数学和自然哲学,而不是医学。[32] 他发明了一种热测量仪(thermoscope),这是温度计的前身。1586年,他出版了一本小册子,介绍他设计的静水天平,这一发明使他首次引起学术界的关注。此外,伽利略还学习了绘画艺术(\textbf{disegno}),1588年在佛罗伦萨的美术学院(\textbf{Accademia delle Arti del Disegno})担任透视法与明暗法的讲师。同年,他应佛罗伦萨学院邀请发表了两场讲座,题为《但丁地狱的形状、位置与大小》,试图提出关于但丁地狱的严格宇宙模型。[33] 受城市艺术传统和文艺复兴艺术家作品的启发,伽利略形成了美学思维。在美术学院任教期间,他与佛罗伦萨画家奇戈利(Cigoli)建立了终生的友谊。[34][35]  

1589年,伽利略被任命为比萨大学的数学教授。1591年,他的父亲去世,他承担起照顾弟弟米开朗基罗的责任。1592年,他转到帕多瓦大学,教授几何学、力学和天文学,直至1610年。[36] 在此期间,伽利略在基础科学(如运动学和天文学)以及应用科学(如材料强度和望远镜的开创性研究)方面取得了重要发现。他的兴趣广泛,还研究了占星术,这在当时是一门与数学、天文学和医学紧密相关的学科。[37][38]
\subsubsection{天文学} 
\textbf{开普勒超新星}  

第谷·布拉赫(Tycho Brahe)及其他人曾观测过1572年的超新星。1605年1月15日,奥塔维奥·布伦佐尼(Ottavio Brenzoni)写信给伽利略,提到了1572年的超新星以及1601年较暗的新星。伽利略在1604年观测并讨论了开普勒超新星。由于这些新星没有显示出可检测的日视差,伽利略得出结论认为它们是遥远的恒星,因此推翻了亚里士多德关于天穹不变性的观点。[39]

\textbf{折射望远镜}
\begin{figure}[ht]
\centering
\includegraphics[width=8cm]{./figures/534f6baa94d041ab.png}
\caption{伽利略的“cannocchiali”望远镜,展于佛罗伦萨伽利略博物馆} \label{fig_JLL_3}
\end{figure}
或许仅仅基于对荷兰人汉斯·李普希(Hans Lippershey)在1608年尝试申请专利的首个实用望远镜的描述,[40] 伽利略在次年制作了一台放大倍率约为3倍的望远镜。他随后改进了设计,制造出放大倍率高达约30倍的版本。[41] 使用伽利略望远镜,观察者可以看到放大且正立的地面影像——这就是通常所说的地面望远镜或单筒望远镜。他还用它来观测天空;一段时间内,他是少数能够制作适合天文观测望远镜的人之一。1609年8月25日,他向威尼斯人展示了一台早期望远镜,放大倍率约为8倍或9倍。  

伽利略的望远镜也成为他的一个盈利副业,他将望远镜卖给商人,后者发现它在海上导航和贸易中都非常有用。伽利略于1610年3月出版了一本简短的论文,题为《星际信使》(**Sidereus Nuncius**),记录了他最初的天文望远镜观测成果。[42]

\textbf{月球}
\begin{figure}[ht]
\centering
\includegraphics[width=6cm]{./figures/7cfb4912b973d48d.png}
\caption{《星际信使》中月球的插图,1610年在威尼斯出版} \label{fig_JLL_4}
\end{figure}
1609年11月30日,伽利略将他的望远镜对准了月球。[43] 虽然他并非第一个通过望远镜观察月球的人(英国数学家托马斯·哈里奥特(Thomas Harriot)早在四个月前就进行了观察,但他仅记录到“奇怪的斑点”),[44] 但伽利略是第一个推测月球表面不规则阴影是由月球上的山脉和陨石坑遮挡光线所致的人。在他的研究中,他还绘制了月球地形图,并估算了山脉的高度。月球并非如亚里士多德所宣称的那样是一颗透明而完美的球体,也不像但丁所描述的那样是“第一颗行星”,一颗“升入天穹的永恒明珠”。  

伽利略有时被认为在1632年发现了月球纬度上的天平动,[45] 尽管托马斯·哈里奥特或威廉·吉尔伯特(William Gilbert)可能早已做出这一发现。[46]  

伽利略的朋友、画家奇戈利(Cigoli)在他的某幅画作中包含了一幅逼真的月球描绘,这可能是基于他通过自己望远镜的观察所得。[34]

\textbf{木星的卫星} 

1610年1月7日,伽利略通过望远镜观测到木星附近有三个他当时描述为“极其微小、完全不可见的小恒星”,它们位于木星附近,并与木星在一条直线上。[47] 随后几晚的观测显示,这些“恒星”相对于木星的位置在不断变化,这种现象无法用它们是真正固定的恒星来解释。1月10日,伽利略注意到其中一个“恒星”消失了,他推测这是因为它被木星遮挡了。在接下来的几天里,他得出结论,这些天体正在围绕木星运行:他发现了木星四大卫星中的三个。[48]  

伽利略在1月13日发现了第四颗卫星。他将这四颗卫星命名为“美第奇之星”(Medicean stars),以此向未来的资助人托斯卡纳大公科西莫二世·德·美第奇(Cosimo II de' Medici)及其三位兄弟致敬。[49] 不过,后来天文学家为了纪念伽利略,将这些卫星改名为“伽利略卫星”。这些卫星实际上也在1610年1月8日被西蒙·马里乌斯(Simon Marius)独立发现,并在1614年出版的《木星的世界》(\textbf{Mundus Iovialis})中被马里乌斯命名为\textbf{木卫一(Io)}、\textbf{木卫二(Europa)}、\textbf{木卫三(Ganymede)} 和 \textbf{木卫四(Callisto)}。[50]
\begin{figure}[ht]
\centering
\includegraphics[width=8cm]{./figures/9e52a2a199660cbe.png}
\caption{1684年绘制的法国地图,展示了早期地图的轮廓(浅色线条)与使用木星卫星作为精确时间参考进行的新测量结果(较粗线条)之间的对比。} \label{fig_JLL_5}
\end{figure}
伽利略对木星卫星的观测在天文学界引发了争议:一颗行星有更小的天体围绕其运行,这与亚里士多德宇宙学的原则不符,后者认为所有天体都应围绕地球运转。[51][52] 起初,许多天文学家和哲学家拒绝相信伽利略能够发现这样的现象。[53][54] 更加复杂的是,其他天文学家很难验证伽利略的观测结果。当伽利略在博洛尼亚展示望远镜时,与会者难以看到木星的卫星。其中一人,马丁·霍尔基(Martin Horky),注意到通过望远镜观察某些固定恒星(如室女座的角宿一)时,它们会显得是双星。他认为这证明了望远镜在观察天体时具有误导性,从而对木星卫星的存在表示怀疑。[55][56]  

然而,罗马的克里斯托弗·克拉维乌斯(Christopher Clavius)天文台确认了伽利略的观测结果。尽管对如何解释这些现象仍存疑问,但当伽利略次年访问罗马时,他受到了英雄般的欢迎。[57] 在接下来的18个月里,伽利略持续观测这些卫星,到1611年年中,他得出了这些卫星轨道周期的极其准确的估计值——这是约翰内斯·开普勒(Johannes Kepler)曾认为不可能完成的壮举。[58][59]  

伽利略还看到了其发现的实际用途。在海上确定船只的东西位置需要船上的时钟与本初子午线的时钟同步。解决这个经度问题对于航行安全至关重要,因此西班牙和后来荷兰都设立了巨额奖励来鼓励这一问题的解决。由于伽利略发现的卫星的掩食现象相对频繁且时间可以被非常准确地预测,它们可以用来校准船上的时钟。伽利略申请了这些奖励。然而,从船上观测这些卫星过于困难,但这种方法被用于陆地测量,包括重新绘制法国的地图。[60]: 15–16 [61]

\begin{figure}[ht]
\centering
\includegraphics[width=14.25cm]{./figures/5532f8a9bd158eb0.png}
\caption{1610年,伽利略·伽利莱通过望远镜观察到金星展现出相位变化,尽管它在地球的天空中始终靠近太阳(第一张图)。这一发现证明了金星围绕太阳运行,而不是围绕地球运行,这与哥白尼的日心模型的预测相符,并否定了当时普遍接受的地心模型(第二张图)。} \label{fig_JLL_6}
\end{figure}
从1610年9月起,伽利略观察到金星呈现出一整套类似月球的相位变化。尼古拉·哥白尼(Nicolaus Copernicus)提出的日心模型预测,金星应该会展现所有相位,因为金星绕太阳的轨道会使其被照亮的半球在金星位于太阳的对侧时面对地球,而在金星位于太阳的地球侧时背向地球。在托勒密的地心模型中,任何行星的轨道都不可能与承载太阳的球壳相交。因此,传统上金星的轨道被完全置于太阳的近侧,在这种情况下金星只能呈现新月和朔相。也可以将其轨道完全置于太阳的远侧,这样金星只能呈现盈凸相和满相。

伽利略通过望远镜观察到金星的弦月相、盈凸相和满相后,托勒密的地心模型变得无法成立。在17世纪早期,由于这一发现,大多数天文学家转而支持某种形式的地-日心混合行星模型,[62][63] 如第谷模型(Tychonic model)、卡佩拉模型(Capellan model)和扩展卡佩拉模型(Extended Capellan model),这些模型有的包含每日自转的地球,有的则没有。这些模型解释了金星的相位变化,同时避免了完全日心模型关于恒星视差的“驳斥”。因此,伽利略对金星相位的发现是从完全地心模型向通过地-日心混合模型过渡到完全日心模型的两阶段转变中最具实证意义的贡献之一。[citation needed]