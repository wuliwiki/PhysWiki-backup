% 光的折射 斯涅尔定律
% 折射|折射定律|光路|惠更斯原理|斯涅尔定律

\pentry{惠更斯原理\upref{Huygen}}
\subsection{折射率}
介质的折射率可以由光在介质中的速度 $v$ 定义, 令 $c$ 为真空中的光速, 则
\begin{equation}
n = \frac{c}{v}
\end{equation}
这说明折射率和速度成反比. 由于真空中的光速是光的最大速度, 所以折射率必然大于等于 1.

\subsection{折射定律}

\textbf{折射定律}也叫\textbf{斯涅尔定律}定律, 在几何光学中描述光从一种介质由光滑的界面入射到另一种介质时角度的变化. 令两种介质的折射率分别为 $n_1$ 和 $n_2$, 光线在两种介质中关于法线的夹角分别为 $\theta_1$ 和 $\theta_2$, 则两介质中的光线共面, 且
\begin{equation}
n_1 \sin\theta_1 = n_2 \sin\theta_2
\end{equation}


\subsection{推导}
要推导斯涅尔定律, 就不得不考虑光的波动性. 我们可以用惠更斯原理%未完成
来解释这个公式

% 图未完成

许多条光线, 等相位面, 先到达界面的光线有更多时间, 球面波, 半径, 另一束光线才到达, 中间的光线...
