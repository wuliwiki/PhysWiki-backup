% 热平衡 热力学第零定律
% keys 热平衡|热力学第零定律|温度|容器

\begin{issues}
\issueDraft
\end{issues}

\subsection{热平衡}
热力学研究的对象是一个由大量微观粒子(分子或其他粒子)组成的一个宏观物质系统(例如一个绝热容器中的气体).经验指出,一个孤立系统(与外界没有物质和能量交换的系统)若放置得足够久,将会达到这样一种状态——系统的各种宏观性质在长时间内部发生任何变化.这称为热力学平衡态.
\addTODO{弛豫时间}
\addTODO{态函数}
\addTODO{}

\subsection{热力学第零定律}
若物体 $A$ 与物体 $C$ 达到热平衡, 且物体 $B$ 与物体 $C$ 也达到热平衡, 那么 $A$ 和 $B$ 之间同样有热平衡.

简单来说, 热平衡意味着温度处处相等. 那么这个定律也就不证自明了.
