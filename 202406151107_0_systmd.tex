% service、systemd、systemctl 笔记
% license Xiao
% type Note

\begin{itemize}
\item 要创建一个 service, 就创建文件 \verb|/etc/systemd/system/名字.service|:
\item 需要 sudo 身份。
\begin{lstlisting}[language=none,caption=名字.service]
[Unit]
Description=服务的描述(允许空格)

[Service]
ExecStart=/路径/命令 参数1 参数2
# 等 15 秒再运行以上命令,防止系统没有加载完成(如网络设置等)
ExecStartPre=/bin/sleep 15

[Install]
WantedBy=multi-user.target
\end{lstlisting}
\item 注意 \verb|ExecStart| 不支持 \verb|>| 和 \verb`|` 等
\item \verb|sudo systemctl daemon-reload| 使 service 文件生效
\item \verb|sudo systemctl enable 名字.service| 就可以设置开机自启服务(通过创建一个 symlink)。 \verb`.service` 可以省略
\item \verb|systemctl disable| 不开机自启
\item \verb|systemctl status| 查看状态和日志的最新几行(\verb`-n 50` 可以指定显示多少行,\verb`-l` 可以显示完整的行而不是截断)
\item \verb|systemctl start| 启动
\item \verb|systemctl stop| 停止
\item \verb|systemctl restart| 重启
\item 也可以在 \verb`ps aux | grep 名字` 查看是否在运行
\end{itemize}

\subsubsection{更多参数}
\begin{lstlisting}[language=none,caption=名字.service]
[Unit]
Description=服务的描述(允许空格)
Documentation=文档的url
After=在哪些服务运行后运行(如果它们没运行也没关系)
Before=在哪些服务之前运行(如果它们没运行也没关系)
Wants=在运行之前试图运行的另一个服务(如果失败也没关系)
Requires=在运行之前试图运行的另一个服务(如果失败就不运行)

[Service]
ExecStart=/路径/启动命令 参数1 参数2
ExecStop=/路径/停止命令 参数1 参数2
WorkingDirectory=/当前路径/
ExecStartPre=/路径/命令 参数1 参数2
Restart=always, on-failure, on-abnormal, on-watchdog, on-abort, no.
Type=simple(默认)|forking|oneshot|dbus, notify and idle.
User=运行服务的用户(默认 root)
Group=运行服务的组(默认 root)
Environment=环境变量
EnvironmentFile=环境变量文件

[Install]
WantedBy=在这个 target 启动该 service,失败也没关系
RequiredBy=失败有关系
\end{lstlisting}

\begin{itemize}
\item 例如我如果有一个脚本 \verb`hello.sh`,内容是 \verb`while true; do echo hello; sleep 3; done`。 那么应该设置 \verb`ExecStop=/bin/kill $MAINPID`,其中 \verb`MAINPID` 是 systemd 的一个变量,即当前服务进程的 PID。
\item \verb`Type=simple` 意思是 \verb`ExecStart` 命令的进程就是服务进程。 该命令一旦运行就算是服务正在运行。
\item \verb`Restart=always` 意思是只要服务进程终止了(无论什么原因),就立刻重新开始运行。
\item \verb`Restart=on-failure` 意思是只有 exit code 不为 0 或者被 signal 终止才 restart。 正常执行完毕(exit code 0)不重启。
\item \verb`WantedBy=multi-user.target` 意思是启动后到了 \verb`multi-user.target` 阶段才开始运行。 后者大概就是支持多用户登陆,但在 GUI 出现之前,经常使用。
\item 如果程序输出到 \verb`stdout` 和 \verb`stderr`,那么可以用 \verb`systemctl status 服务` 查看最后的一点 log, 如果要看更多就用 \verb`sudo journalctl -u 服务.service [-n 50]` 后面的 \verb`-n` 可以指定看最后多少行。
\end{itemize}

\subsection{supervisor}
\begin{itemize}
\item 参考\href{https://www.digitalocean.com/community/tutorials/how-to-install-and-manage-supervisor-on-ubuntu-and-debian-vps}{这里}。
\end{itemize}
