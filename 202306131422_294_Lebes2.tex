% Lebesgue 积分的一些补充性质
% keys Levi定理|Lebesgue基本定理|Fatou引理

\pentry{Lebesgue 积分\upref{Lebes1},子列极限、上极限与下极限\upref{SubLim}}

由于非负可测函数的积分定义为“压平”后的非负有界可测函数的积分之极限,易得以下性质:

\begin{lemma}{}\label{lem_Lebes2_1}
如果 $E$ 是零测集,那么 $E$ 上的任意函数都是可测的,且其积分为 $0$。
\end{lemma}

这里的函数取值可以是广义实数 $\pm\infty$。

由描述Lebesgue积分几何意义的\autoref{the_Lebes1_3}~\upref{Lebes1},容易推得以下定理:

\begin{theorem}{Levi定理}\label{the_Lebes2_1}

设 $\{f_k\}$ 是可测集 $E$ 上的一列\textbf{单调不减}的\textbf{非负可测}函数,则有
\begin{equation}
\int_E  \qty[\lim\limits_{k\to \infty} f_k(x)] \dd x = \lim\limits_{k\to \infty}\int_E f_k(x) \dd x~.
\end{equation}

\end{theorem}

简单来说,Levi定理就是指单调不减的非负可测函数列具有“可极限换序”的性质。

由Levi定理可以直接得到以下推论:

\begin{corollary}{Lebesgue基本定理}

设 $\{f_k\}^\infty_{k=1}$ 是可测集 $E$ 上的一列\textbf{非负可测}函数,则有

\begin{equation}
\int_E  \qty[\sum\limits_{k=1}f_k(x)] \dd x = \sum\limits_{k=1} \int_E f_k(x) \dd x~.
\end{equation}

\end{corollary}

回忆\textbf{上极限}与\textbf{下极限}的\autoref{def_SubLim_1}~\upref{SubLim},我们可以把这个定义推广到函数列上去。为了方便阅读,我们在这里简单解释一下上下极限的含义,已熟悉的读者可直接跳过,需要更详细解释的读者请跳转到\textbf{子列极限、上极限与下极限}\upref{SubLim}。

\begin{definition}{数列的上、下极限}\label{def_Lebes2_2}
给定实数列 $\{a_n\}$。

定义
\begin{equation}
\lim\limits_{n\to\infty} \sup\{a_i\}_{i=n}^\infty = \opn{\overline{\lim}}\limits_{n\to\infty} a_n~
\end{equation}
为 $\{a_n\}$ 的\textbf{上极限(upper limit)}。

类似地,定义
\begin{equation}
\lim\limits_{n\to\infty} \inf\{a_i\}_{i=n}^\infty = \opn{\underline{\lim}}\limits_{n\to\infty} a_n~
\end{equation}
为 $\{a_n\}$ 的\textbf{下极限(lower limit)}。
\end{definition}


简单来说,数列的上极限就是“上确界的极限”,下极限就是“下确界的极限”。当然,由于上确界数列 $\{\sup \{a_i\}_{i=n}^\infty\}$ 是单调不增的,也可以说上极限是“上确界的下确界”,类似地,下极限就是“下确界的上确界”。后面这两种解释更适用于\textbf{子列极限、上极限与下极限}\upref{SubLim}中的讨论。


\begin{definition}{函数列的上、下极限}\label{def_Lebes2_1}

给定一列实函数 $f_k$。定义其上、下极限函数在 $x$ 处的值为数列 $f_k(x)$ 的上、下极限。

\end{definition}

有了上下极限的概念,我们才可以描述清楚以下性质:

\begin{theorem}{Fatou引理}\label{the_Lebes2_2}

设 $f_k$ 是可测集 $E$ 上的一列\textbf{非负可测}函数,则
\begin{equation}\label{eq_Lebes2_1}
\int_E \opn{\underline{\lim}}\limits_{k\to \infty} f_k(x) \dd x \leq \opn{\underline{\lim}}\limits_{k\to \infty}\int_E f_k(x) \dd x~.
\end{equation}

\end{theorem}

\textbf{证明}:

为方便讨论,定义一个工具函数:

\begin{equation}
g_k(x) = \inf\limits_{i\geq 0}\{f_{k+i}(x)\}~.
\end{equation}

则由\autoref{def_Lebes2_1} 可知,$\opn{\underline{\lim}}\limits_{k\to\infty} f_k(x)=\lim\limits_{k\to\infty} g_k(x)$。另外易得,$\{g_k\}$ 是一列 $E$ 上的一列\textbf{单调不减}的\textbf{非负可测}函数。

由确界的性质,可知 $g_k(x)\leq f_k(x)$ 处处成立。因此有
\begin{equation}
\lim_{k\to\infty} \int_E g_k(x) \dd x \leq \lim_{k\to\infty} \int_E f_k(x) \dd x~.
\end{equation}

于是由\autoref{the_Lebes2_1} ,结合极限的性质可得:

\begin{equation}
\begin{aligned}
\int_E \opn{\underline{\lim}}\limits_{k\to\infty}f_k(x) \dd x &=
\int_E \lim_{k\to\infty} g_k(x) \dd x \\
&=\lim_{k\to\infty} \int_E g_k(x) \dd x\\
&\leq \lim_{k\to\infty} \int_E f_k(x) \dd x~.
\end{aligned}
\end{equation}


\textbf{证毕}。

\autoref{the_Lebes2_2} 有没有可能更强一些呢?就是说,能不能把\autoref{eq_Lebes2_1} 的不等号写成等号呢?答案是否定的,用下面这个简单的例子就可以看出来:

\begin{example}{}
考虑区间 $[0, 1]$ 上的函数列 $\{f_k\}$,其定义如下:

当 $k$ 为奇数时,
\begin{equation}
f_k(x)=\leftgroup{
    &1\quad x\in[0, 1/2]\\
    &0\quad x\in[1/2, 0]~.
}
\end{equation}

当 $k$ 为偶数时,
\begin{equation}
f_k(x)=\leftgroup{
    &0\quad x\in[0, 1/2]\\
    &1\quad x\in[1/2, 0]~.
}
\end{equation}

则显然 $\opn{\underline{\lim}}\limits_{k\to \infty}f_k(x)=0$ 处处成立,故
\begin{equation}\label{eq_Lebes2_2}
\int_{[0, 1]} \opn{\underline{\lim}}\limits_{k\to \infty}f_k(x) \dd x = 0~,
\end{equation}

但由于各 $\int_{[0, 1]} f_k(x) \dd x$ 都等于 $1/2$,故
\begin{equation}\label{eq_Lebes2_3}
\opn{\underline{\lim}}\limits_{k\to \infty}\int_{[0, 1]} f_k(x) \dd x = 1/2~.
\end{equation}

比较\autoref{eq_Lebes2_2} 和\autoref{eq_Lebes2_3} 即可知,\autoref{eq_Lebes2_1} 的等号不总是成立。

\end{example}

考虑到 $\abs{\int_E f(x) \dd x}=\abs{\int_E f^+(x) \dd x - \int_E f^-(x) \dd x}$ 和 $\int_E \abs{f(x)} \dd x = \int_E f^+(x) \dd x + \int_E f^-(x) \dd x$,可得\textbf{柯西不等式}:

\begin{equation}\label{eq_Lebes2_4}
\abs{\int_E f(x) \dd x} \leq \int_E \abs{f(x)} \dd x~.
\end{equation}

\begin{theorem}{积分的绝对连续性}

设 $f$ 在 $E$ 上Lebesgue可积且\textbf{几乎处处有限},那么对于任意的 $\epsilon>0$,总存在 $\delta>0$ 使得只要 $A\subseteq E$ 的测度 $\opn{m}A<\delta$,就一定有
\begin{equation}
\abs{\int_A f(x) \dd x} < \epsilon~.
\end{equation}

\end{theorem}

\textbf{证明}:

由柯西不等式\autoref{eq_Lebes2_4} ,可知只需要考虑 $f$ 是非负可测函数的情况即可。又由\autoref{lem_Lebes2_1} ,只需要考虑 $f$ 处处有限的情况即可,即有界。下设 $f$ 是有界的非负可测函数。

设 $f$ 的一个上界为 $s$,那么只需要取 $\delta<\epsilon/s$ 即可得证。

\textbf{证毕}。







