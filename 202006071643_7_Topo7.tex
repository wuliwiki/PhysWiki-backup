% 商拓扑
\pentry{二元关系\upref{Relat}}

\subset{商集}

拓扑空间本身作为一个集合,可以在其上定义某种等价关系,然后我们可以将同一个等价类中的点看成是同一个点.这个过程也可以理解为点之间的粘合.

\begin{example}{烟卷}\label{Topo7_ex1}
在$\mathbb{R}^2$中取一个矩形的闭子集:$A=\{(x,y)\in\mathbb{R}^2:x\in[-1,1],y\in[0,1]\}$.在这个闭子集上定义一个等价关系$\sim$如下:对于$y\in[0,1]$,有$(1, y)\sim(-1, y)$;其它点都只和自己等价.如果把$A$看成一个矩形纸条,那么商集$A/\sim$可以看成是这个纸条的两边对应点粘在一起,卷成一个烟卷的样子.

\begin{figure}[ht]
\centering
\includegraphics[width=8cm]{./figures/Topo7_2.pdf}
\caption{卷纸条的过程示意.左边是矩形纸条$A$,右边是模去等价关系$\sim$(或者说把等价的点粘在一起)以后得到的集合.} \label{Topo7_fig2}
\end{figure}

\end{example}



我们给出纸条$A$的准确坐标只是为了方便描述上述等价关系.实际上,在$\mathbb{R}^2$中选择任意的有界连通闭子集,都可以找到这个闭子集和$A$的同胚映射,在拓扑意义下它们就是相同的集合.取一个任意的四边形、三角形或者不规则图形都可以,你可以直观地想象成揉捏一个橡皮泥,只要不撕裂它,不把本来分开的点粘在一起,那揉出来的任何形状都是同胚的.当然,用矩形纸条描述以上商集最为方便.

\begin{example}{甜甜圈}

还是利用\autoref{Topo7_ex1}中的$A$,这次等价关系$\sim$为:对于$x\in[-1,1], y\in[0,1]$,有$(x,0)\sim(x,1), (1, y)\sim(-1, y)$,其它点只和自己等价.这一次,商集$A/\sim$可以看成把纸条的左右两边粘合,上下两边也粘合.

\end{example}

