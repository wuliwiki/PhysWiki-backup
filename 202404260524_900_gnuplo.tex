% gnuplot 笔记
% license Xiao
% type Note

\begin{issues}
\issueDraft
\end{issues}

\begin{itemize}
\item \href{http://www.gnuplot.info/documentation.html}{文档}
\item 按画图框上的刷新按钮即可更新画图匡, 或者用上箭头翻出 plot 指令再执行一遍
\item 设置 terminal 的类型 (就是输出的目标)
\begin{lstlisting}[language=none]
set term wxt enhanced (wxt 窗口)
set term qt (qt 窗口)
set term postscript eps enhanced "Times-New-Roman" 12 (eps 文件)
\end{lstlisting}
\item 输出文件名/格式 \verb|set output 'example.eps'|
\item 最简单的画函数图 \verb|set plot sin(x)*exp(-x/10)|
\item 由文件画图:
\item 在 ascii 数据文件里面(通常把拓展名写成 .dat) , 两列数据分别代表 x 轴和 y 轴.
\item 简单画图 \verb|plot 'xy1.dat', 'xy2.dat'|
\item 指定颜色 (lc 是 line color)
plot 'xy1.dat' lc rgb 'red', 'xy2.dat' lc rgb 'black'
\item 添加标题等
\begin{lstlisting}[language=none]
set title 'TitleName'
set xlabel 'x'; set ylabel 'y'
\end{lstlisting}
\item 图例
\begin{lstlisting}[language=none]
unset key 关闭图例
set key 恢复图例
plot 'xy1.dat' title 'data1', 'xy2.dat' title 'data2'
\end{lstlisting}
\item 指定范围 \verb|[xmin:xmax] [ymin:ymax]|
\verb|plot [0:xmax] [ymin:ymax] 'sin.dat'| (不改变的范围可以省略)
\item 设置网格 \verb|set grid|
\item 数学常数 \verb|pi|
\item \verb|{/Symbol abcdefg}|
\end{itemize}
