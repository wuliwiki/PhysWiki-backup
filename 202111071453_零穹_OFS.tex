% 正交函数系 2
% 正交;函数系;贝塞尔不等式

\begin{issues}
\issueDraft
\end{issues}

若函数系 
\begin{equation}\label{OFS_eq1}
\qty{\varphi_n(x)}\quad(n\in \mathbb{N})
\end{equation}
中各函数及其平方在区间 $\qty[a,b]$ 上皆可积分,且满足
\begin{equation}
\int_a^b\varphi_n(x)\varphi_m(x)\dd x=
\left\{
    \begin{aligned}
    &0\quad &m\neq n\\
   &\lambda_n>0\quad &m=n
    \end{aligned}
\right.
\end{equation}
则称函数系$\qty{\varphi_n(x)}$ 为\textbf{正交函数系}.当 $\lambda_n=1(n\in\mathbb{N})$ 时,该函数系称为\textbf{正交规范系}(或\textbf{正交标准系}).显然,任意的正交函数系都可化为正交规范系 
\begin{equation}
\qty{\frac{\varphi_n(x)}{\sqrt{\lambda_n}}}\quad(n\in \mathbb{N})
\end{equation}

设 $f(x)$ 是任一实函数,在区间 $[a,b]$ 内是连续的,则数值
\begin{equation}
c_k=\frac{1}{\lambda_k}\int_a^b f(x)\varphi_k(x)\dd x\quad (k\in \mathbb{N})
\end{equation}
称为函数 $f(x)$ 关于函数系\autoref{OFS_eq1} 的\textbf{傅里叶系数}.由 $c_k$ 的定义,我们有等式
\begin{equation}\label{OFS_eq2}
\int_a^b\qty[f(x)-\sum_{k=1}^nc_k\varphi_k(x)]^2\dd x=\int_a^b\qty[f(x)]^2\dd x-\sum_{k=1}^nc_k^2\lambda_k
\end{equation}

根据\autoref{OFS_eq2} ,可得
\begin{equation}\label{OFS_eq3}
\sum_{k=1}^\infty c_k^2\lambda_k\leq\int_a^b\qty[f(x)]^2\dd x
\end{equation}
对于正交规范系, $\lambda_k=1$,此时\autoref{OFS_eq3} 便是所谓的\textbf{贝塞尔(bessel)不等式}. 

若对任何在区间 $[a,b]$ 定义的连续函数 $f(x)$,\autoref{OFS_eq3} 中的等号成立,则称函数系 $\qty{\varphi_n(x)}$ 是\textbf{完整的}(或\textbf{完备的}).
%\begin{example}{三角函数系}
%%\begin{equation}
%\qty{1,\cos nx,\sin nx}\quad(n\in \mathbb{Z^{+}})
%%\end{equation}
%是正交函数系.

可以证明:
\begin{enumerate}
\item 在区间 $[-\pi,\pi]$ 上,三角函数系
\begin{equation}
\qty{1,\cos nx,\sin nx}\quad(n\in \mathbb{Z^{+}})
\end{equation}
是正交函数系.
\item 由超越方程
\begin{equation}
\tan\xi=c\xi \quad(c\; \mathrm{is\; a\; constant})
\end{equation}
 的所有正根
 \begin{equation}
 \xi_1,\xi_2,\cdots,x_n\cdots
 \end{equation}
 构造的函数系
 \begin{equation}
 \sin\frac{\xi_1}{l}x,\sin\frac{\xi_2}{l}s,\cdots,\sin\frac{\xi_n}{l}x,\cdots
 \end{equation}
 在区间 $\qty[0,l]$ 上是正交函数系.
 \item 勒让德多项式
\begin{equation}
X_0(x)=1,X_n(x)=\frac{1}{2^nn!}\dv[n]{\qty(x^2-1)^n}{x}\quad(n=1,2,\cdots)
\end{equation}
是在区间 $\qty[-1,1]$ 是正交函数系,且 $\lambda_n=\frac{2}{2n+1}$
\item 以贝塞尔函数 $J_0(x)$ 的所有正根构成的集合$\qty{\xi_n\vert n\in\mathbb{Z}^+}$
构造的函数系$\qty{\sqrt{x}J_0(\xi_n x)} $在区间 $\qty[0,1]$ 上是正交函数系.
\end{enumerate}


%证明见\autoref{FSTri_sub1}~\upref{FSTri}
%\textbf{证明}:
%\begin{equation}
%\begin{aligned}
%\int_{-\pi}^{\pi}\cos nx\dd x&=\frac{\sin nx}{n}\bigg\vert_{-\pi}^{\pi}=0\quad \\
%\int_{-\pi}^{\pi}\sin nx\dd x&=-\frac{\cos nx}{n}\bigg\vert_{-\pi}^{\pi}=0\\
%\int_{-\pi}^{\pi}\sin nx\cos mx\dd x&=\frac{1}{2}\int_{-\pi}^{\pi}\qty[\sin(n+m)x+\sin(n-m)x]\dd x=0\\
%\int_{-\pi}^{\pi}\cos nx\cos mx\dd x=\frac{1}{2}\int_{-\pi}^{\pi}\qty[\cos(n+m)x+\cos(n-m)x]\dd x=0
%\end{aligned}
%\end{equation}
%\end{example}


