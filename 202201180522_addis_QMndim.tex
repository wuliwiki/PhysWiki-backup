% 单粒子多维薛定谔方程
% 量子力学|梯度|张量积空间|本征值

\begin{issues}
\issueDraft
\issueOther{说明本质上是张量积空间中的矢量.}
\issueOther{讲解可拆分的情况 $H = H_1\otimes I_2 + I_1 \otimes H_2$}
\issueOther{可能发生的简并情况}
\issueOther{讲解什么是投影到子空间(二维无限深势阱为例)}
\end{issues}

\pentry{张量积空间\upref{DirPro}, 拉普拉斯算符\upref{Laplac}}

\subsection{单个粒子在多维空间中的波函数}
本词条中的“多维” 指的是二维和三维. 与牛顿力学一样, 在学习完粒子的一维运动后, 我们希望能了解粒子在平面上的运动(二维), 或者空间中的运动(三维). 在多于一维的情况下, 波函数变为位置矢量 $\bvec r$ 以及时间 $t$ 的函数
\begin{equation}
\Psi(\bvec r, t)
\end{equation}
例如在二维直角坐标系中, $\Psi(\bvec r, t) = \Psi(x, y, t)$, 又例如在三维的球坐标系中, $\Psi(\bvec r, t) = \Psi(r, \theta, \phi, t)$.
% 未完成: 概率

\subsection{矢量算符}
在多维空间中, 位置和动量分别从标量拓展为矢量, 所以对应地, 位置算符和动量算符也分别拓展为\textbf{矢量算符}(我们暂时把这两个算符的定义看作是量子力学的基本假设)
\begin{equation}
\Q{\bvec r} = \bvec r = x\uvec x + y\uvec y + z\uvec z
\end{equation}
\begin{equation}\ali{
&\quad \Q{\bvec p} = -\I\hbar \grad = \Q p_x \uvec x + \Q p_y \uvec y + \Q p_z \uvec z\\
&= \qty(-\I\hbar \pdv{x}) \uvec x + \qty(-\I\hbar \pdv{y}) \uvec y + \qty(-\I\hbar \pdv{z}) \uvec z
}\end{equation}
当矢量算符作用在波函数上后, 得到的函数的自变量仍然是 $(\bvec r, t)$, 而函数值却变为一个复数矢量(矢量的三个分量都是复数). 把位置算符作用在波函数上, 就是把波函数分别乘以 $x, y, z$ 的坐标, 并作为函数值的三个分量. 而把动量算符作用在波函数上, 就先把各个方向的动量算符分别作用, 并作为函数值的三个分量. 所以在矢量算符的本征方程
\begin{equation}
\Q{\bvec Q} \Psi(\bvec r) = \bvec \lambda \Psi(\bvec r)
\end{equation}
中, 我们只有使用矢量本征值才能保证等号两边都是矢量. 矢量算符的本征方程也可以写成三个分量的形式
\begin{equation}
\Q Q_x \Psi(\bvec r) = \lambda_x \Psi(\bvec r) \qquad
\Q Q_y \Psi(\bvec r) = \lambda_y \Psi(\bvec r) \qquad
\Q Q_z \Psi(\bvec r) = \lambda_z \Psi(\bvec r)
\end{equation}

不难验证, 位置的本征函数就是三维 $\delta$ 函数
\begin{equation}
\delta(\bvec r - \bvec r_0) = \delta(x - x_0) \delta(y - y_0) \delta(z - z_0)
\end{equation}
对应的本征值为 $\bvec r_0$. 动量的本征函数就是三维平面波
\begin{equation}
\exp(\I \bvec k \vdot \bvec r) = \exp(\I k_x x) \exp(\I k_y y) \exp(\I k_z z)
\end{equation}
对应的本征值为 $\bvec p = \hbar\bvec k$.
