% 外代数
% keys 外代数|外微分|线性空间|格拉斯曼代数|Grassmann|外积|矢量分析|向量分析|外积|外乘|楔积|楔乘|exterior algebra|exterior derivative|wedge product|反对称
% license Xiao
% type Tutor

\pentry{基(线性代数)\nref{nod_VecSpn}, 域上的代数\nref{nod_AlgFie}}{nod_61f4}
外代数是Clifford代数的一个特例,利用运算结构,其结果被限制在有限维空间上。

\subsection{外代数的概念}

给定线性空间 $V$,任取 $x, y\in V$,定义 $x\wedge y\not\in V$ 是一个新的元素,其中符号 $\wedge$ 称作\textbf{外积(exterior product)},有时也叫做\textbf{楔积(wedge product)},前者是因为这个运算得到的是 $V$ 以外的新元素,后者是由于符号长得像个楔子。注意,为了方便,我们没有使用线性代数中常见的粗体正体符号来表示向量。

利用各 $x\wedge y$ 构造新的线性空间:定义 $x\wedge y=-y\wedge x$ 对所有 $x, y\in V$ 成立,这同时意味着 $x\wedge x=0$。定义一个加法 $+$,使得对于 $x_1, x_2, y\in V$,都有 $(x_1+x_2)\wedge y=x_1\wedge y+x_2\wedge y$;再定义数乘为对于任意基本域中的数字 $a$,都有 $a(x\wedge y)=(ax)\wedge y=x\wedge(ay)$。这样,集合 $\{x\wedge y|x, y\in V\}$ 构成一个线性空间,记为 $\bigwedge^2 V$。同时,为了统一考虑,记 $V=\bigwedge^1 V$。

$\bigwedge^1 V$ 和 $\bigwedge^2 V$ 之间也可以进行楔积,并且满足\textbf{结合律}:$x\wedge(y\wedge z)=(x\wedge y)\wedge z$,由此可以拿掉结合括号,定义 $x\wedge y\wedge z=x\wedge(y\wedge z)=(x\wedge y)\wedge z$。集合 $\{x\wedge y\wedge z|x, y, z\in V\}$ 张成的线性空间,记为 $\bigwedge^3 V$。

同理,我们可以构造出任意阶的 $\bigwedge^k V$。要注意的是,如果 $k>\opn{dim} V$,那么 $\bigwedge^k V=\{0\}$(幂零性使然)。另外,把 $V$ 的基本域 $\mathbb{F}$ 看成一个一维线性空间,记 $\mathbb{F}=\bigwedge^0 V$。

不同线性空间之间可以用直和组合在一起,因此以上这些空间也都可以作直和,得到一个 $\bigwedge V=\bigoplus^{\opn{dim}V}_{k=0}\bigwedge^k V=\mathbb{F}\oplus\bigwedge^1V\oplus\bigwedge^2V\cdots$。这个 $\bigwedge V$,就被称作 $V$ 上的\textbf{外积空间(exterior product space)}或\textbf{楔积空间(wedge product space)}。总结上述限制条件,我们便得到了外代数的运算结构。
\begin{definition}{外代数}
任给域 $\mathbb{F}$ 上的线性空间 $V$,定义向量之间的乘法为外积 $\wedge$ 。对于任意$\bvec x,\bvec y,\bvec z\in V$及任意$a,b,c\in \mathbb F$,外积具有如下性质:
\begin{enumerate}
\item \textbf{结合性:}$\bvec x\wedge (\bvec y\wedge \bvec z)=(\bvec x\wedge \bvec y)\wedge \bvec z$;
\item \textbf{线性性:}$a\bvec x\wedge(b\bvec y+c\bvec z)=ab\bvec x\wedge \bvec y+ac\bvec x\wedge\bvec z$;
\item \textbf{反对称性:} $\bvec x\wedge \bvec y=-\bvec y\wedge\bvec z$;
\item \textbf{非平凡性:} 若$\bvec x\neq\bvec y$,则$x\wedge y\neq \bvec 0 $。
\end{enumerate}

故构成 $\mathbb{F}$ 上的有限维结合代数。称之为 $V$ 上的\textbf{外代数(exterior algebra)}、\textbf{楔积(wedge product)}或\textbf{格拉斯曼代数(Grassmann algebra)}。
\end{definition}


外代数中的元素可以有形象的几何理解。$\bigwedge^1 V$ 中的元素就是 $V$ 中的元素,我们可以想象成箭头。$\bigwedge^2 V$ 中的元素可以看成箭头对,或者是箭头对表示的平行四边形。同样,$\bigwedge^k V$ 中的元素都可以看成是 $k$ 个箭头张成的一个 $k$ 维对象。

外代数有一个重要的性质,我们用\autoref{exe_ExtAlg_1} 和\autoref{exe_ExtAlg_2} 来阐述 :

\begin{exercise}{}\label{exe_ExtAlg_1}
证明:如果 $k>\opn{dim} V$,那么 $\bigwedge^kV=\{0\}$。
\end{exercise}

\begin{exercise}{}\label{exe_ExtAlg_2}
证明:对于 $\opn{dim} V=k$,有 $\opn{dim} \bigwedge V=2^k$。思路提示:考虑各 $\opn{dim}\bigwedge^iV$ 的值,再对比 $(1+1)^k$ 的二项式展开。
\end{exercise}
\begin{example}{张量的外代数}
若令 $\bigwedge^k V=\Lambda^k_0 T$,其中 $\Lambda^k_0(V)$ 是所以 $(0,p)$ 型的斜对称张量(\upref{SIofTe})构成的集合。那么此时得到的外代数便和张量里的外代数\upref{WegofT}一致。而外积 $\wedge$ 由张量的交错化映射定义,它满足这里外积的一切性质。
\end{example}



三维欧几里得空间 $\mathbb{R}^3$ 中的叉乘实际上就是外积。这是因为,$\opn{dim}\mathbb{R}^3=\opn{dim}\bigwedge^2\mathbb{R}^3$,这样一来,如果给定 $\mathbb{R}^3$ 的标准正交基 $\{x, y, z\}$,那么我们可以建立同构 $*: \bigwedge^2\mathbb{R}\rightarrow\mathbb{R}^3$,使得 $*(x\wedge y)=z, *(y\wedge z)=x, *(z\wedge x)=y$,这样就可以通过这个同构来把外积变成 $\mathbb{R}^3$ 内部的向量积。这一映射也是叉乘的“右手定则”的来源,我们也完全可以规定 $*(x\wedge y)=-z, *(y\wedge z)=-x, *(z\wedge x)=-y$,这样定义出来的叉乘就是符合左手定则的了。

三维线性空间是唯一可以构造反交换代数的非平凡空间,就是因为只有三维的 $V$ 才满足 $\opn{dim}V=\opn{dim}\bigwedge^2V$,因而可以建立 $\bigwedge^2 V$ 和 $V$ 之间的同构,从而把楔积变成叉积。相应地,比复数更高维的可除代数只有四元数。

外代数是一个“\textbf{分次线性空间(graded vector space)}”,就是说,它作为一个线性空间,每个向量具有一个“次数”,定义如下:每个 $\bigwedge^kV$ 中的向量,其\textbf{次数(grade)}就是 $k$;对于任意向量 $v\in \bigwedge V$,我们总可以把它拆分成各 $\bigwedge^kV$ 中基向量的线性组合,这些基向量中次数最高的就定义为 $v$ 的次数。


\subsection{对偶空间的外代数}\label{sub_ExtAlg_1}

实流形上极为常见的外代数是微分形式生成的外代数,也就是所谓的外微分。由于一个微分 $k$-形式可以看成是将 $k$ 个向量场变成一个光滑函数的映射,也就是在每个切点处都是一个将 $k$ 个切向量变成一个实数的张量,因此要研究微分形式的外代数,首先就要搞清楚对偶空间的外代数。

对偶空间中的元素,都是主空间中的线性函数。$k$ 个线性函数的外积,被定义为“将 $k$ 个主空间向量映射为一个数”的多重线性映射。我们自然想到将 $k$ 个向量的映射联系到对偶空间外代数中的 $k$ 阶元素。

具体来说,假如我们有两个多重线性映射 $f$ 和 $g$,分别是 $n$ 阶和 $m$ 阶的,那么我们定义 $f\wedge g$ 为如下 $n+m$ 阶多重线性映射:
\begin{equation}
\begin{aligned}
&f\wedge g(x_1, x_2, \cdots, x_{n+m})\\=
&\sum\limits_{\sigma\in S_{n+m}}\opn{sgn}(\sigma) f(x_{\sigma(1)}, x_{\sigma(2)}, \cdots, x_{\sigma(n)})\cdot g(x_{\sigma(n+1)}, \cdots, x_{\sigma(n+m)})
\end{aligned}~.
\end{equation}
其中 $\sigma$ 是一个 $n+m$ 元置换,当 $\sigma$ 为奇变换时 $\opn{sgn}(\sigma)$ 为 $-1$,$\sigma$ 为偶变换时 $\opn{sgn}(\sigma)$ 为 $1$。

最简单的例子,就是两个对偶向量的外积。设 $V^*$ 是线性空间 $V$ 的对偶空间,令 $f, g\in V^*$,那么对于任意 $\bvec{v}, \bvec{u}\in V$,我们有:
\begin{equation}
f\wedge g(\bvec{v}, \bvec{u})=f(\bvec{v})g(\bvec{u})-f(\bvec{u})g(\bvec{v})~.
\end{equation}
其中一共涉及置换群 $S_2$ 中的两个置换,$(1)$ 和 $(1\phantom{2}2)$,分别是偶置换和奇置换。
