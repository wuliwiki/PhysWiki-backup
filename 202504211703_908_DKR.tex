% 笛卡儿积(综述)
% license CCBY4
% type Wiki

本文根据 CC-BY-SA 协议转载翻译自维基百科\href{https://en.wikipedia.org/wiki/Cartesian_product}{相关文章}。

在数学中,特别是集合论中,两个集合\(A\)和\(B\)的笛卡尔积,记作\( A \times B \),是所有有序对\( (a, b) \)的集合,其中\( a \in A \),且\( b \in B \)\(^\text{[1]}\)。用集合构造符号表示为:  
\[
A \times B = \{ (a, b) \mid a \in A \text{ 且 } b \in B \}.^\text{[2][3]}~
\]
可以通过对“行的集合”与“列的集合”取笛卡尔积来创建一个表格。若取笛卡尔积 rows × columns,那么表格的每个单元格就包含一个形如(行值,列值)的有序对。\(^\text{[4]}\)

同样地,也可以定义\(n\)个集合的笛卡尔积,称为 n 重笛卡尔积,它可以表示为一个\(n\)-维数组,其中每个元素是一个\(n\)元组(n-tuple)。有序对是 2 元组(2-tuple)或称为“偶对”。更一般地,还可以定义一个按索引排列的集合族的笛卡尔积。

笛卡尔积的名称来自于勒内·笛卡尔,\(^\text{[5]}\)他对解析几何的建立促成了这一概念的产生,该概念进一步推广后形成了“直积”的形式。
\subsection{集合论中的定义}
要对笛卡尔积进行严格的定义,必须在集合构造符号中指定一个定义域。在这种情况下,定义域必须包含笛卡尔积本身。

对于集合\( A \)和\( B \)的笛卡尔积,若使用典型的 Kuratowski 对的定义,即将有序对\( (a, b) \)定义为\((a, b) = \{\{a\}, \{a, b\}\}\)那么一个合适的定义域是幂集的幂集\(\mathcal{P}(\mathcal{P}(A \cup B))\)其中\( \mathcal{P} \)表示幂集运算符(即某集合所有子集的集合)。

于是,集合\( A \)和\( B \)的笛卡尔积可以定义为:  
\[
A \times B = \{ x \in \mathcal{P}(\mathcal{P}(A \cup B)) \mid \exists a \in A,\ \exists b \in B,\ x = (a, b) \}~
\]  
也就是说,笛卡尔积是所有属于该幂集的集合\( x \),其中\( x \)是某个\( a \in A \)、\( b \in B \)所构成的 Kuratowski 有序对。
\subsection{示例}  
\subsubsection{一副扑克牌}
\begin{figure}[ht]
\centering
\includegraphics[width=14.25cm]{./figures/923db8c9149a95e0.png}
\caption{} \label{fig_DKR_1}
\end{figure}