% 一维自由粒子(量子)
% keys 自由粒子|平面波|波包|归一化|简并
% license Xiao
% type Tutor

\pentry{傅里叶变换与连续正交归一基底\nref{nod_COrNoB}, 薛定谔方程\nref{nod_TDSE}, 原子单位制\nref{nod_AU}}{nod_65c5}

本文使用\enref{原子单位制}{AU}。 当含时\enref{薛定谔方程}{TDSE}中势能函数 $V(x) = 0$ 时, 有
\begin{equation}\label{eq_FreeP1_1}
-\frac{1}{2m}\pdv[2]{x}\Psi(x,t) = \I \pdv{t} \Psi(x,t)~.
\end{equation}
一般用分离变量法解该方程, 通解为\autoref{eq_TDSE11_4}~\upref{TDSE11}。 但首先要解出对应的定态薛定谔方程
\begin{equation}
-\frac{1}{2m}\dv[2]{x}\psi_E(x) = E \psi_E(x)~.
\end{equation}
只有 $E > 0$ 时有可归一化的解, 也就是熟悉的\enref{平面波}{PWave}
\begin{equation}
\psi_E(x) = \E^{\I k x}~.
\end{equation}
其中 $k = \pm\sqrt{2mE}$, 注意定态薛定谔方程具有二重简并, 即一个 $E$ 对应两个线性无关解。 令 $\omega = E = k^2/2$, 则通解为
\begin{equation}\label{eq_FreeP1_2}
\Psi(x,t) = \int_{0}^{+\infty} \qty[A_+(\omega) \E^{\I (\abs{k}x - \omega t)} + A_-(\omega) \E^{\I (-\abs{k}x - \omega t)}] \dd{\omega}~.
\end{equation}
其中第一项是向右移动的平面波, 第二项向左移动。 一种更简洁的表示方法是做变量替换 $\omega(k) = k^2/(2m)$, 把通解记为关于 $k$ 的积分
\begin{equation}\label{eq_FreeP1_3}
\Psi(x,t) = \frac{1}{\sqrt{2\pi}} \int_{-\infty}^{+\infty} C(k) \E^{\I (k x - \omega t)} \dd{k}~.
\end{equation}
直观上来理解, \autoref{eq_FreeP1_2} 和\autoref{eq_FreeP1_3} 都表示许多不同频率的平面波的叠加,所以是等效的, 系数 $A_\pm(\omega)$ 和 $C(k)$ 具有对应关系, 以后我们会看到他们之间如何\enref{转换\}upref{EngNor}。 注意\autoref{eq_FreeP1_3} 恰好是 $C(k)$ 的反傅里叶变换(\autoref{eq_FTExp_1}~\upref{FTExp})。 当 $t = 0$ 时有
\begin{equation}
C(k) = \frac{1}{\sqrt{2\pi}} \int_{-\infty}^{+\infty} \Psi(x,0) \E^{-\I k x} \dd{x}~,
\end{equation}
这样我们就从初始波函数 $\Psi(x,0)$ 得到了系数。

\autoref{eq_FreeP1_3} 相当于把波函数在正交归一的基底 $\psi_k(x) = \E^{\I kx}/\sqrt{2\pi}$ 展开, 满足正交归一条件(\autoref{eq_COrNoB_8}~\upref{COrNoB})
\begin{equation}
\braket{\psi_{k'}}{\psi_k} = \delta(k' - k)~,
\end{equation}
我们把这种归一化叫做\textbf{动量归一化}(原子单位中 $k$ 就是动量); 而\autoref{eq_FreeP1_2} 则把波函数在另一组正交归一基底上展开, 这组基底一般使用\textbf{能量归一化}, 其基底 $\psi_{E,1}, \psi_{E,2}$ (方括号中的两项)满足归一化条件
\begin{equation}
\braket{\psi_{E',i'}}{\psi_{E,i}} = \delta_{i',i}\delta(E' - E)~,
\end{equation}
详见“\enref{动量归一化和能量归一化}{EngNor}”。

一个简单的例子见“\enref{一维高斯波包(量子)}{GausWP}”。

事实上, 无论我们使用什么样的初始波包, 它接下来的演化都是平淡无奇的, 可以定性地划分为匀速移动和扩散。 在熟悉了本文以后, 我们可以开始考虑一个稍微复杂一些的问题: 当势能函数 $V(x)$ 在某个区间不为零时, 波包经过这个区间会如何变化? 这就是一维散射问题\upref{Sca1D}。
