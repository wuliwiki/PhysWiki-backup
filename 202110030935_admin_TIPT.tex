% 不含时微扰理论

\begin{issues}
\issueDraft
\end{issues}

\footnote{参考 \cite{GriffQ} 相关章节.}不含时微扰理论.

\begin{equation}
H = H_0 + \lambda H^1
\end{equation}
\begin{equation}
E_n = E_n^0 + \lambda E_n^1 + \lambda^2 E_n^2 + \dots
\end{equation}
令 $\psi_n^0$ 是 $H_0$ 的任意一组完备正交归一基底.
\begin{equation}
\psi_n = \psi_n^0 + \lambda\psi_n^1 + \lambda^2 \psi_n^2 + \dots
\end{equation}
\begin{equation}
H \psi_n = E_n \psi_n
\end{equation}
令 $\lambda \to 0$, 忽略 $\order{\lambda^2}$. 有
\begin{equation}
H^0\psi_n^1 + H^1 \psi_n^0 = E_n \psi_n^1 + E_n^1 \psi_n^0
\end{equation}
要使其恒成立, 就要求投影到任意 $\psi_m^0$ 上都成立:
\begin{equation}
\mel{\psi_m^0}{H^0}{\psi_n^1} + \mel{\psi_m^0}{H^1}{\psi_n^0} = E_n \braket{\psi_m^0}{\psi_n^1} + E_n^1 \braket{\psi_m^0}{\psi_n^0}
\end{equation}
第一项 

\subsection{非简并情况}
若没有微扰时, 波函数处于非简并的束缚态 $\ket{\psi_n^0}$, 且能量为 $E_n^0$. 加入不含时微扰 $H'$ 后, 一阶近似下该束缚态能量变为
\begin{equation}
E_n = E_n^0 + E_n^1
\end{equation}
波函数变为
\begin{equation}
\psi_n = \psi_n^0 + \psi_n^1
\end{equation}
其中
\begin{equation}
E_n^1 = \mel{\psi_n^0}{H'}{\psi_n^0}
\end{equation}
\begin{equation}
\psi_n^1 = \sum_{m \ne n} \frac{\mel{\psi_m^0}{H'}{\psi_n^0}}{E_n^0 - E_m^0} \psi_m^0
\end{equation}

\subsection{简并情况}
假设 $[H, H'] = 0$, 那么在每个简并子空间中用存在一组共同本征矢. 这就叫做 “好” 量子态. 要求好量子态的本征问题, 令 $H$ 在该子空间中的本征态为 $\psi_n^0$, 那么 $H'_{i,j} = \mel*{\psi_i^0}{H'}{\psi_j^0}$. 解出 $N$ 个本征矢和本征值, 就是能量和波函数的一阶修正.

另一个术语叫做\textbf{好量子数(good quantum number)}, 就是好量子态对应的量子数(本征值的编号). 如果好量子态是一个常见物理量的算符 $A$ 的本征态, 也就是说 $H, H', A$ 两两对易, 且 $A$ 在当前简并子空间中并不简并, 那么 $A$ 的量子数就是好量子数.
