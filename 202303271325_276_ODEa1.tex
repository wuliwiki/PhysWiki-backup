% 一阶常微分方程解法:变量可分离方程
% 常微分方程|ordinary differential equation|ODE|初等解法|变量分离法

\pentry{常微分方程简介\upref{ODEint}}

形如 $\frac{\dd y}{\dd x}=f(x)g(y)$ 的微分方程,称作\textbf{变量可分离}方程。这类方程是最容易解的。

对方程两边进行移项,得到 $\frac{1}{g(y)}\dd y=f(x)\dd x$,这样就把变量分离开了。两边同时求积分,得到等式
\begin{equation}\label{ODEa1_eq1}
\int\frac{1}{g(y)}\dd y=\int f(x)\dd x+C~,
\end{equation}
其中 $C$ 是积分常数。

可见,\autoref{ODEa1_eq1} 的左边是 $y$ 的函数,右边是 $x$ 的函数,只要能把这两个积分写出来,那么微分方程也就解出来了。

\begin{example}{}
考虑方程 $\frac{\dd y}{\dd x}=xy$。

分离变量后有
\begin{equation}
\int\frac{1}{y}\dd y=\int x\dd x+C~.
\end{equation}

算出积分后可得
\begin{equation}
\ln\abs{y}=\frac{1}{2}x^2+C~,
\end{equation}

即
\begin{equation}
\abs{y}=\E^{\frac{1}{2}x^2}\cdot\E^C~.
\end{equation}

将 $\E^C$ 重新记为一个新的常数 $K$,那么该方程的通解最终就可以表示为
\begin{equation}
y=\pm K\E^{\frac{1}{2}x^2}~.
\end{equation}
\end{example}

\begin{exercise}{}
求解方程 $\frac{\dd y}{\dd x}=\frac{x}{y}$。答案是 $\abs{x}=\abs{y}$,也可以写成 $x^2=y^2$。
\end{exercise}

\begin{exercise}{}
求解方程 $\frac{\dd v}{\dd t}=\frac{t^2}{v+b}$。答案是 $3v^2+6bv=2t^3$,也可以写成 $t=(\frac{3}{2}v^2+3bv)^{1/3}$.
\end{exercise}

\subsection{可化为变量可分离形式的方程}

\begin{theorem}{}\label{ODEa1_the1}
形如 $\frac{\dd y}{\dd x}=g(\frac{y}{x})$ 的方程,可以化为变量可分离的形式。
\end{theorem}

\textbf{证明}:

定义一个新的变量 $u=\frac{y}{x}$,于是 $y=ux$,于是 $\dd y=u\dd x+x\dd u$。

于是原方程可写为
\begin{equation}
u+x\frac{\dd u}{\dd x}=g(u)~,
\end{equation}
也就是
\begin{equation}
\frac{\dd u}{\dd x}=\frac{g(u)-u}{x}~,
\end{equation}

这是一个关于 $x$ 和 $u$ 的变量可分离方程。

\textbf{证毕}。

\begin{example}{}
考虑方程 $\frac{\dd y}{\dd x}=\frac{y}{x}+\E^{-\frac{y}{x}}$。

根据\autoref{ODEa1_the1} 的证明结论,取 $u=y/x$,将方程改写为4
\begin{equation}
\frac{\dd u}{\dd x}=\frac{1}{x\E^u}~.
\end{equation}

进行变量分离后,积分得
\begin{equation}
\E^u=\ln\abs{x}+C~,
\end{equation}

这就是其通解。


\end{example}


\begin{corollary}{}\label{ODEa1_cor1}
形如 $\frac{\dd y}{\dd x}=\frac{a_1x+b_1y+c_1}{a_2x+b_2y+c_2}$ 的方程,可以化为变量可分离的形式。
\end{corollary}

\textbf{证明}:

当 $c_1=c_2=0$ 的时候,方程就是 $\frac{\dd y}{\dd x}=\frac{a_1x+b_1y}{a_2x+b_2y}$,右边只需要上下同时除以 $x$ 就可以得到 $\frac{\dd y}{\dd x}=\frac{a_1+b_1\frac{y}{x}}{a_2+b_2\frac{y}{x}}$,这就符合了\autoref{ODEa1_the1} 的情况。

但是当 $c_1$ 和 $c_2$ 不全为零的时候,我们就得想办法化成相同的情况。

$a_1x+b_1y+c_1=0$ 和 $a_2x+b_2y+c_2=0$ 分别表示两条直线。它们的交点在原点,当且仅当 $c_1=c_2=0$,也就是说现在考虑的情况是这两条直线的交点不在原点——那我们就把它挪到原点不就好了?

联立 $a_1x+b_1y+c_1=0$ 和 $a_2x+b_2y+c_2=0$,求出交点 $(x_0, y_0)$。作变量代换(也就是移动整个坐标系,使得原点跑到两直线交点处):
\begin{equation}
\leftgroup{
    X=x-x_0\\
    Y=y-y_0
}~,
\end{equation}

这样就有 $a_1X+b_1Y=0$ 和 $a_2X+b_2Y=0$。

同时由于是平移,换句话说变量代换里 $x_0$ 和 $y_0$ 是常数,因此还有 $\dd X=\dd x$,$\dd Y=\dd y$。

于是原方程化为
\begin{equation}
\frac{\dd Y}{\dd X}=\frac{a_1X+b_1Y}{a_2X+b_2Y}~.
\end{equation}

如前所述,这可以化为变量可分离形式。

\textbf{证毕}。

\begin{corollary}{}
形如 $\frac{\dd y}{\dd x}=g(\frac{a_1x+b_1y+c_1}{a_2x+b_2y+c_2})$ 的方程,都可以化为变量可分离的形式。
\end{corollary}

\textbf{证明}:

用和\autoref{ODEa1_cor1} 相同的方法作变量代换即可。

\textbf{证毕}。





\begin{example}{}
考虑方程 $\frac{\dd y}{\dd x}=\frac{3x+2y-4}{x+3y+1}$。

直线 $3x+2y-4=0$ 和 $x+3y+1=0$ 的交点是 $(2, -1)$,因此我们作变量代换:
\begin{equation}
\leftgroup{
    X=x-2\\
    Y=y+1
}~.
\end{equation}
得到
\begin{equation}\label{ODEa1_eq4}
\frac{\dd Y}{\dd X}=\frac{3X+2Y}{X+3Y}~,
\end{equation}
将\autoref{ODEa1_eq4} 右边上下同除以 $X$ 以后得
\begin{equation}\label{ODEa1_eq5}
\frac{\dd Y}{\dd X}=\frac{3+2Y/X}{1+3Y/X}~,
\end{equation}

按\autoref{ODEa1_the1} 的方式,令 $U=Y/X$,则\autoref{ODEa1_eq5} 又化为
\begin{equation}
\frac{\dd U}{\dd X}=\frac{\frac{3+2U}{1+3U}-U}{X}~,
\end{equation}

移项,整理得
\begin{equation}\label{ODEa1_eq6}
\int\frac{3U+1}{-3U^2+U+3}\dd U=\int\frac{1}{X}\dd X+C~,
\end{equation}

两边积分后即可得解。\autoref{ODEa1_eq6} 左边的积分稍有复杂,但技巧不难。

\end{example}








\begin{example}{}
考虑方程 $\frac{\dd y}{\dd x}=(\frac{2x+2y-6}{x+3y-7})^2$。

直线 $2x+2y-6=0$ 和 $x+3y-7=0$ 的交点是 $(1, 2)$,因此我们作变量代换:
\begin{equation}
\leftgroup{
    X=x-1\\
    Y=y-2
}~,
\end{equation}
得到
\begin{equation}\label{ODEa1_eq2}
\frac{\dd Y}{\dd X}=(\frac{2X+2Y}{X+3Y})^2~.
\end{equation}

将\autoref{ODEa1_eq2} 化为
\begin{equation}\label{ODEa1_eq3}
\frac{\dd Y}{\dd X}=(\frac{2+2Y/X}{1+3Y/X})^2~.
\end{equation}

按\autoref{ODEa1_the1} 的方式,令 $U=Y/X$,则\autoref{ODEa1_eq3} 又化为
\begin{equation}
\frac{\dd U}{\dd X}=\frac{(\frac{2+2U}{1+3U})^2-U}{X}~,
\end{equation}

接下来只需要积分即可。不过说起来容易做起来难,这里关于 $U$ 的积分非常复杂,解到这一步能看出来它可以变量分离即可。

\end{example}









