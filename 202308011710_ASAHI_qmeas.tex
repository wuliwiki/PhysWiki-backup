% 量子测量
% 测量|投影测量|POVM测量

\pentry{量子力学基本原理\upref{QMPrcp}}

在量子力学的学习中,我们了解了测量公设。粗略地来讲,在量子态上进行对某个可观测量的测量,就会让这个量子态坍缩到这个可观测量的某个本征态上,同时返回这个本征态对应的本征值作为测量结果。得到这个本征态的概率由Born定则给出。在本节中,我们将会从量子信息的角度来研究量子测量,首先我们会回顾投影测量的基本结构,然后我们将会介绍von Neumann的测量理论,最后我们将会给出投影测量的推广——POVM测量。

\subsection{投影测量}

在数学上我们知道,可观测量$A$对应的是一个厄米算符,谱分解定理表明$A$的本征向量构成一组Hilbert空间的完备正交基,也即(先假设没有发生本征值的简并)
\begin{equation}
A=\sum_i\lambda_i\ket{i}\bra{i}.~
\end{equation}
对于待测量的物理系统,我们可以用这一组完备正交基来对量子态进行展开,具体来说,对于一个量子态$\ket{\psi}$,在上面测量$A$得到结果$\lambda_i$的概率为$p(\lambda_i)=|\braket{\psi}{i}|^2$。

对于含有简并的情况也可以类似地处理。虽然在这种情况下本征态分解并不唯一,但是总可以先考虑$A$的一个特定分解,然后考虑在这一分解下与$A$具有相同本征态的另一个厄米算符$A'$,但是其本征值$\lambda_i'$不含简并。在这种情况下对$A$进行测量应当理解为“先对量子态进行了$A'$所对应的投影测量,再对测量结果$\lambda_i'$重新标记为$A$再相应本征态$\ket{i}$上所对应的本征值$\lambda_i$”。换句话说,含有简并的情况下的投影测量相当于是对不含简并的情况的测量的一种粗粒化。

现在我们就可以在数学上重新审视什么是投影测量了。在量子力学中,更一般的投影测量$\Pi$被定义为一些投影算符的集合:
$$
\Pi=\left\{\Pi_j: \Pi_i \Pi_j=\delta_{i j} \Pi_i, \sum_j \Pi_j=\mathbb{I}\right\}.~
$$
两个条件分别对应着正交性和完备性。不含简并的测量相当于在说每一个$\Pi_j$都是秩为1的,即可以写作$\Pi_j=\ket{j}\bra{j}$。

由测量公设可以得到,当我们对一个系统进行投影算符$\Pi_j$所对应的投影测量,我们会得到下面的结果:
\begin{enumerate}
\item 如果测量前的量子态处于$\ket{\psi}$,那么测量得到结果$\lambda_i$的概率为\begin{equation}
p(\lambda_i)=\Vert \Pi_i\ket{\psi}\Vert^2=\bra{\psi}\Pi_i\ket{\psi}.~
\end{equation}
\item 如果测量得到结果$\lambda_i$,测量后的系统将会演化为下面的归一化量子态
\begin{equation}
\frac{\Pi_i\ket{\psi}}{\Vert\Pi_i\ket{\psi}\Vert}.~
\end{equation}
\item 测量结果的平均值为\begin{equation}
\langle A\rangle \equiv \sum_i \lambda_i p\left(\lambda_i\right)=\sum_i \lambda_i\left\langle\psi\left|\Pi_i\right| \psi\right\rangle=\langle\psi|A| \psi\rangle.~
\end{equation}
如果是在混态上进行测量的,很容易得到此时的平均值可以表示为
\begin{equation}
\langle A\rangle=\operatorname{Tr}(\rho A).~
\end{equation}

\end{enumerate}

\begin{exercise}{}
如果量子态是混态而不是纯态,那么上面的这些结论应该如何进行改写呢?
\end{exercise}

\subsection{von Neumann测量模型}

von Neumann的测量模型是试图理解量子态测量和坍缩的重要尝试。这一模型的基本想法是,测量仪器会和待测系统因为某种动力学过程而耦合起来,从而可以通过对测量仪器进行观测从而导致对量子系统的观测。考虑一个最简单的例子,即对量子比特态$\ket{\psi}=a\ket{0}+b\ket{1}$进行计算基上的投影测量。那么测量的过程可以被描述为
\begin{enumerate}
\item 初始状态:待测量量子态处于$\ket{\psi}_S=a\ket{0}_S+b\ket{1}_S$,测量设备处于$\ket{0}_M$状态。
\item 测量设备与量子态发生相互作用:整体的量子态演化为$a\ket{0}_S\ket{0}_M+b\ket{1}_S\ket{1}_M$。
\item 坍缩:测量设备处于宏观可区分状态$\ket{0}_M$或$\ket{1}_M$,且整体量子态发生坍缩。整体量子态分别以$|a|^2$和$|b|^2$的概率坍缩到$\ket{0}_S\ket{0}_M$和$\ket{1}_S\ket{1}_M$。
\end{enumerate}

在更加精细的von Neumann测量模型中,还会对这一耦合的动力学过程进行更详细的建模。比如我们想要研究的系统的Hamiltonian为$H$(我们想要去测量这个Hamiltonian的本征谱),测量设备由一个指针来代表,它可以被建模成一个一维自由粒子。在这种情况下,总系统的耦合Hamiltonian由$H\otimes p$来表示。如果这个指针的质量足够大以至于可以忽略其动能项,那么演化算符可以表示为\begin{equation}
e^{-i t H \otimes p}=\sum_a\left[\left|E_a\right\rangle\left\langle E_a\right| \otimes e^{-i t E_a p}\right].~
\end{equation}
其中$\ket{E_a}$是$H$的本征值为$E_a$的本征态。如果初始的指针制备在$\ket{x=0}$,那么上述演化算符便会让系统实现演化
\begin{equation}
\sum_a c_a\ket{E_a}\otimes\ket{x=0}\mapsto \sum_a c_a \ket{E_a}\otimes \ket{x=tE_a}.~
\end{equation}
这就代表着我们可以通过直接对指针进行测量来获得量子系统的信息。

当然了,von Neumann测量仍然没有回答为什么量子态会坍缩这个问题,而是把问题从量子系统本身丢到了测量仪器乃至于观测者上。但是无论怎样,这样的测量模型仍然给了后续研究者以很大启发,比如上世纪80年代提出的弱测量技术便是在von Neumann测量的基础上得以提出的。

\subsection{POVM测量}

在量子系综中我们已经明白了一个重要的哲学:量子信息中的纯态在物理上代表着封闭系统的描述,在信息上代表着我们掌握着系统的全部信息,在概率的角度则代表着系统中没有任何经典概率的成分。而量子信息中的混态在物理上代表着和外界耦合后对原系统的描述,在信息上代表着有一部分信息没有被我们掌握,在概率上代表着存在经典概率的成分。

从这三个出发点:物理,信息和概率的角度,我们都可以将传统的投影测量进行推广。具体来说,我们可以考虑以下三个问题作为出发点:
\begin{enumerate}
\item 如果我们将物理系统和环境进行耦合,然后进行整体的幺正演化,然后再在环境上进行投影测量,那么这个投影测量会对物理系统对应的密度矩阵产生什么样的变化?
\item 如果投影测量本身存在着瑕疵,比如说测量设备的分辨率有限,导致我们无法精确地区分两个相邻的本征值,那么这时的投影测量应该被描述成什么样的操作?
\item 考虑这样一个测量设备,输入一个未知的量子比特态后,它会以50\%的概率进行$Z$基矢测量,以50\%的概率进行$X$基矢测量。测量后会输出$\{\pm1\}$作为测量结果。这个设备对应的测量是什么?
\end{enumerate}

我们将会看到,正如量子系综中,这三个角度都会导向密度矩阵和混态一样,在测量理论中这三个角度都会导向相同的作为投影测量的推广——POVM测量。

在定义POVM测量之前,我们先引入一个推广版本的测量公设:
\begin{theorem}{广义测量公设}
(广义)量子测量操作由一组测量算符$\{M_m\}_m$描述,作用于待测量系统的Hilbert空间上,指标$m$对应着实验上可能的测量结果。如果测量过程刚开始时量子系统处于状态$\rho$,那么结果$m$发生的概率为
\begin{equation}
p(m)=\operatorname{Tr}(M_m\rho M_m^\dagger).~
\end{equation}
测量后相应的系统末态变为
\begin{equation}
\frac{M_m\rho M_m^\dagger}{\operatorname{Tr}(M_m\rho M_m^\dagger)}.~
\end{equation}
\end{theorem}

容易看出,概率的求和归一性可以得到对测量算符的约束
\begin{equation}
\sum_m p(m)=\sum_m\operatorname{Tr}(M_m\rho M_m^\dagger)=\operatorname{Tr}(\sum_m\rho M_m^\dagger M_m)=1=\operatorname{Tr}(\rho).~
\end{equation}
即
\begin{equation}
\sum_m M_m^\dagger M_m=\mathbb{I}.~
\end{equation}
}

为了说明这种测量公设确实是投影测量的推广,只需要注意到当测量算符$M_m$刚好是向$\ket{m}$的投影$M_m=\Pi_m=\ket{m}\bra{m}$时,投影测量的公设就能得到满足。

POVM测量的定义便是基于这一版本的测量公设。具体来说,我们定义一个新的算符\begin{equation}
E_m=M_m^\dagger M_m.~
\end{equation}
它被称为POVM元素,并称集合$\{E_m\}_m$为一个POVM。在上述的定义下不难验证这些POVM元素拥有这些性质:
\begin{enumerate}
\item E_m\ge 0
\item E_m=E_m^\dagger
\item \sum_m E_m=\mathbb{I}
\end{enumerate}



