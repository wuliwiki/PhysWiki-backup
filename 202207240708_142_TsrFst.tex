% 张量(向量与矩阵)
% keys 张量|矩阵

\pentry{矩阵\upref{Mat}}

% 在多元微积分中引入矩阵是为了将一元微积分中导数值的概念.具体而已,对于函数 $f: \mathbb{R} \to \mathbb{R}$,在某个点处(比如 $0$ 点)可以找到一条切线,我们可以用一个实数来表示切线的斜率;类似的,当我们考虑函数 $f: \mathbb{R}^n \to \mathbb{R}^m$ 的

直观上说,矩阵就是把数字排列成长方形.从这个角度来说,矩阵是一个纯粹的“组合数学”概念:不需要学习集合论、微积分、线性代数,只需要理解了数(整数,有理数,实数或者复数都可)的概念和上面的加、减、乘(可以没有除法)我们就可以理解矩阵.
\addTODO{什么是《组合数学》,可能需要一个词条来解释数学的子学科}
\addTODO{可以考虑移动到《矩阵》词条}

不过,长方形只是一个很简单的二维图形,抛开纸张(二维)和长方形的限制我们可以得到很多不同的“数字结构”,其中张量就是其中一种(不那么简单的)推广.

在线性代数中,我们会学到用矩阵来理解(选定基之后的)线性变换\upref{LinMap},用张量来理解多线性变换\upref{MulMap}.

\subsection{列向量和行向量}




