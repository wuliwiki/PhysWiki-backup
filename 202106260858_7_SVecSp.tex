% 超线性空间
% keys super vector space|超空间|超代数|super algebra|超向量空间|超对称性|super symmetry|分次空间|graded space

\pentry{矢量空间\upref{LSpace},环的理想和商环\upref{Ideal}}

超线性空间是一种在现代理论物理中应用的代数结构,用于描述\textbf{超对称性}的各种代数性质.它是普通线性空间的一个简单推广,附加了对其中元素奇偶性的判断.

\begin{definition}{超线性空间}
域$\mathbb{F}$上的\textbf{超线性空间(super vector space)}$V$是一个\textbf{分次(graded)线性空间},它是两个齐次子空间$V_0$和$V_1$的直和:$V=V_0\oplus V_1$,其中$0, 1\in \mathbb{Z}_2$.
\end{definition}


$V_i$中的元素称为\textbf{齐次的(homogeneous)}.

齐次元素的\textbf{奇偶性(parity)}由所属子空间决定:对于$v\in V_i$,定义$\abs{v}=i\in\mathbb{Z}_2$,其中$\abs{v}=0$时称$v$是\textbf{偶(even)}的,否则是\textbf{奇(odd)}的.在理论物理中,偶元素也被称为\textbf{玻色元素(Bose element)},奇元素则被称为\textbf{费米元素(Fermi element)}.

由于超线性空间是两个齐次空间的直和,因此超线性空间中也存在非齐次的元素.

\begin{definition}{维度}
令$V=V_0\oplus V_1$是超线性空间,$V_i$是它的齐次子空间.如果$\opn{dim}V_0=n$,$\opn{dim}V_1=m$,那么称$V$的维度是$n|m$.$V$的坐标空间是$\mathbb{F}^{n+m}$上赋予分次结构的结果,记为$\mathbb{F}^{n|m}$.

约定$\mathbb{F}^{n+m}$的前$n$个坐标表示$V_0$中
\end{definition}














