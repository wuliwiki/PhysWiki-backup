% 玻色子
% license CCBYSA3
% type Wiki

(本文根据 CC-BY-SA 协议转载自原搜狗科学百科对英文维基百科的翻译)

在量子力学中,玻色子boson(/ˈboʊsɒn/,[1] /ˈboʊzɒn/[2])是遵循玻色-爱因斯坦统计的粒子;玻色子是两类粒子之一,另一类是费米子;[3] 玻色子这个名字是保罗·狄拉克(Paul Dirac)[4][5] 为了纪念与阿尔伯特·爱因斯坦(Albert Einstein)一起发展玻色-爱因斯坦统计理论(描述基本粒子性质的理论)[6]的印度物理学家、加尔各答大学和达卡大学的物理教授萨特延德拉·纳特·玻色(Satyendra Nath Bose)[7][8]而命名的。

玻色子包含基本粒子,如光子、胶子、W玻色子和Z玻色子(标准模型的四个传递力的规范玻色子)、最近发现的希格斯玻色子和量子引力理论中的引力子。一些复合粒子也是玻色子,如介子和稳定的质量数为偶数的原子核,如氘(一个质子和一个中子,原子质量数= 2)、氦-4或铅-208 以及一些准粒子(例如库珀对、等离子体激元和声子)。[9]

玻色子的一个重要特征就是玻色统计不限制占据相同量子态的玻色子的数量。氦-4被冷却成超流体时,就是这种特性的例证。[10] 与玻色子不同,两个相同的费米子不能占据同一个量子态。构成物质的基本粒子(即轻子和夸克)是费米子,而基本玻色子是力的传递者,起到将物质粘合在一起的作用。[11] 该属性适用于所有具有整数自旋的粒子($s=0,1,2$等),这是作为自旋统计定理的结果。当玻色子的气体被冷却到绝对零度附近的时候,粒子的动能减少到可以忽略的程度,并且它们凝结成最低能级状态。这种状态被称为玻色-爱因斯坦凝聚。这一性质被认为是超流现象的解释。

\subsection{类型}
玻色子可以是基本粒子,比如光子,也可以是复合粒子,比如介子。

虽然大多数玻色子是复合粒子,但在粒子物理的标准模型中,还是有五个玻色子是基本粒子:
\begin{itemize}
\item 标准模型需要(至少)一个标量玻色子 (自旋=0)
\end{itemize}
$H^0$希格斯玻色子
\begin{itemize}
\item 四个矢量玻色子 (自旋=1) 是标准模型要求的规范玻色子:
\end{itemize}\\
$Y$   光子\\
$g$   胶子(八种不同类型)\\
$Z$   中性弱玻色子\\
$W\pm$   带电弱玻色子(两种类型)\\

可能有第六种张量玻色子(自旋=2), 引力子 (G), 也就是传递引力的粒子。它仍然是一个假设的基本粒子,因为迄今为止所有将引力纳入标准模型的尝试都失败了。 如果引力子确实存在,它一定是玻色子,而且一定是规范玻色子。

复合玻色子, 如氦核,在超流和玻色-爱因斯坦凝聚的其他应用中非常重要。

\subsection{性能}
\begin{figure}[ht]
\centering
\includegraphics[width=6cm]{./figures/73f18e0d96cbc1e6.png}
\caption{无限方阱势中(玻色子)二粒子态的对称波函数。} \label{fig_Boson_1}
\end{figure}
玻色子不同于费米子,费米子服从费米-狄拉克统计。两个或多个相同的费米子不能占据同一的量子态。

由于具有相同能量的玻色子可以在空间中占据相同的位置,玻色子通常是力的传递者,包括介子等复合玻色子。费米子通常与物质有关(尽管在量子力学中这两个概念之间的区别并不明显)。

玻色子是服从玻色-爱因斯坦统计的粒子:交换两个玻色子(相同种类),系统的波函数不变。[12] 另一方面,费米子服从费米-狄拉克统计和泡利不相容原理:两个费米子不能占据同一个量子态, 可以说明了包含费米子的物质的“刚性”或“硬度”。 因此,费米子有时被认为是物质的组成成分,而玻色子被认为是传递相互作用的粒子(力的传递者),或者辐射的组成成分。玻色子的量子场是遵循规范对易关系的玻色子场。

激光、脉泽、 超流氦-4和玻色-爱因斯坦凝聚体的性质都是玻色子统计的结果。另一个结果是热平衡状态下的光子气光谱是普朗克光谱,一个例子就是黑体辐射; 还有一个就是不透明的早期宇宙的热辐射,如今称为微波背景辐射。基本粒子间的相互作用被称为基本相互作用。虚玻色子与实粒子的基本相互作用产生了我们所知道的所有力。

所有已知的基本粒子和复合粒子都是玻色子或费米子,这取决于它们的自旋:自旋为半整数的粒子是费米子;具有整数自旋的粒子是玻色子。在非相对论量子力学的框架下,这是一个纯粹的经验观察结果。 在相对论量子场论中,自旋统计定理表明半整数自旋粒子不能是玻色子,整数自旋粒子不能是费米子。[13]

在大系统中, 当玻色子和费米子的波函数重叠时,他们的统计差异只有在密度较大时才体现。 低密度时,两种统计都可以被麦克斯韦-波尔兹曼统计所良好近似,这就是经典统计。

\subsection{基本玻色子}
目前所有观测到的基本粒子要么是费米子,要么是玻色子。所有观测到的基本玻色子又都是规范玻色子:光子,$W玻色子和Z玻色子,胶子,除了希格斯玻色子是标量玻色子。