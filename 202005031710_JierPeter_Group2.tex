% 群同态

\pentry{映射\upref{map}正规子群\upref{Group1}}

\subsection{同构}

让我们来观察两个群$(\mathbb{Z}, +)$和$(2\mathbb{Z},+)$.如果我们把$2\mathbb{Z}$中的$2$都看成$1$,$4$都看成$2$,以此类推,将$2k$都看成$k$,那么两个群的运算规则是一模一样的.比如说,$2\mathbb{Z}$中有$2+4=6$,对应的是$\mathbb{Z}$中$1+2=3$的等式.

我们研究集合和群的时候,元素叫什么名字并不重要,重要的是元素之间是否相同以及运算规则是怎样的.那么,如果我们真的将$2\mathbb{Z}$中的元素$2k$都重命名为$k$,它就和$\mathbb{Z}$没什么区别了.所以在群的意义上,如果不考虑子群关系,单独把$\mathbb{Z}$和$2\mathbb{Z}$拿出来的时候,我们就认为它们是不可区分的,完全相同的两个群.

如果我们建立一个映射$f:\mathbb{Z}\rightarrow2\mathbb{Z}$,定义为$f(k)=2k$,那么这个$f$就是一个双射

