% 卡西米尔效应(综述)
% license CCBYSA3
% type Wiki

本文根据 CC-BY-SA 协议转载翻译自维基百科\href{https://en.wikipedia.org/wiki/Casimir_effect}{相关文章}。


在量子场论中,\textbf{卡西米尔效应}(或称\textbf{卡西米尔力})是一种作用在受限空间宏观边界上的物理力,它源于场的量子涨落。当该效应以单位面积上的力来描述时,有时会使用“卡西米尔压力”这一术语。[2][3] 这一效应以荷兰物理学家\textbf{亨德里克·卡西米尔}的名字命名,他于1948年预测了电磁系统中的这一效应。

同年,卡西米尔与**迪尔克·波尔德**一起描述了一个类似的效应,这一效应发生在中性原子靠近宏观界面时,称为**卡西米尔–波尔德力**。[4] 他们的结果是对伦敦–范德瓦尔斯力的推广,且包括了由于光速有限所导致的时滞。伦敦–范德瓦尔斯力、卡西米尔力和卡西米尔–波尔德力的基本原理可以在同一框架下进行表述。[5][6]

在1997年,**史蒂文·K·拉莫罗**进行了直接实验,定量测量了卡西米尔力,结果与理论预测值相差不超过5%。[7]

卡西米尔效应可以通过以下观点理解:宏观物质界面的存在,如电导体和介电体,改变了第二量子化电磁场能量的真空期望值。[8][9] 由于这一能量的值依赖于材料的形状和位置,卡西米尔效应表现为这些物体之间的力。

任何支持振荡的介质都有类似的卡西米尔效应。例如,绳上的珠子[10][11],以及浸入湍流水或气体中的板[12][13]都能说明卡西米尔力。

在现代理论物理中,卡西米尔效应在质子的手征袋模型中起着重要作用;在应用物理中,它在一些新兴的微技术和纳米技术中具有重要意义。[14]