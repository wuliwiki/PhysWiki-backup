% 圆锥曲线的配极(高中数学)
% keys 高中数学|圆锥曲线|极点|极线
% license Xiao
% type Art






\begin{definition}{关于圆的极点与极线(极点在圆外)}\label{def_EclPol_1}

给定一个圆$C$,取$C$外一点$P$,如\autoref{fig_EclPol_1} 所示。

过点$P$作圆$C$的切线,切点为$A$和$B$,则称直线$AB$是点$P$\textbf{关于圆}$C$的极线。

反之,取圆$C$的一根弦,与圆的交点为$A$和$B$,过这两个点作圆$C$的切线,称其交点$P$为直线$AB$\textbf{关于圆}$C$的极点。

\begin{figure}[ht]
\centering
\includegraphics[width=8cm]{./figures/542dd0f77d5d5e19.pdf}
\caption{关于圆的极点与极线的示意图。点$P$和直线互为关于给定圆的极点和极线。} \label{fig_EclPol_1}
\end{figure}

\end{definition}




极点和极线总是成对出现,因此一定要强调“关于圆的”。比如说,单独给定一个圆和一个点,不能说这个点就是圆的极点,因为没有“圆的极点”这种说法。

\autoref{def_EclPol_1} 中只局限于关于圆的情况,且$P$点在圆之外。事实上,极点和极线的相对关系可以关于所有圆锥曲线定义,点$P$也可以在平面上任何位置。我们接下来就通过讨论逐步明晰这些概念。




\begin{theorem}{极线的方程}\label{the_EclPol_1}

给定圆$C:x^2+y^2=R^2$和其外一点$P=(x_0, y_0)$,则$P$关于圆$C$的极线为
\begin{equation}
l: x_0x+y_0y = R^2~. 
\end{equation}

\end{theorem}



\textbf{证明}:

如\autoref{fig_EclPol_1} ,极线$l=AB$与直线$PO$正交,而$PO$的斜率为$y_0/x_0$,故$AB$的斜率为$-x_0/y_0$,因此易得其方程为
\begin{equation}
AB: x_0x+y_0y=c~, 
\end{equation}
其中$c$待定,不妨设$c>0$。

如何确定$c$呢?我们绕$O$点旋转整个坐标系,则在此过程中$AB$到$O$的距离保持不变。如\autoref{fig_EclPol_2} ,这个距离$\abs{OH}$满足
\begin{equation}
\abs{OH}\sqrt{\qty(\frac{c}{x_0})^2+\qty(\frac{c}{y_0})^2} = \frac{c}{x_0}\frac{c}{y_0}~, 
\end{equation}
整理后得
\begin{equation}
\abs{OH}=\frac{c}{\sqrt{x_0^2+y_0^2}}~. 
\end{equation}

\begin{figure}[ht]
\centering
\includegraphics[width=10cm]{./figures/b3e2553083f5e708.pdf}
\caption{计算已知直线到原点距离的示意图。如图,绿色直线的方程为$x_0x+y_0y=c$,过$O$作其垂线,垂足为$H$。} \label{fig_EclPol_2}
\end{figure}


对于本定理考虑的极点和极线的情况,旋转过程中$x_0$和$y_0$一直变化,但是$\sqrt{x_0^2+y_0^2}$不变,恒为$\abs{OP}$,因此$c$也保持不变。将$B$旋转到$(R, 0)$的位置,将这个坐标代入直线$AB$的表达式后易得
\begin{equation}
c=R^2~. 
\end{equation}
这就确定了$c$。

综上得证。

\textbf{证毕}。



由\autoref{the_EclPol_1} 可见,给定了圆(相当于给定$R$),再给定点$P$(相当于给定$x_0$和$y_0$),则能唯一确定极线$x_0x+y_0y=R^2$。反之,给定了圆和一条直线,则能唯一确定相对应的极点。更重要的是,这个互相确定的过程不需要限定$P$在圆外,因此可以由此直接推广极点和极线的概念。




\begin{definition}{关于圆的极点与极线}\label{def_EclPol_2}

给定一个圆$C: x^2+y^2=R^2$,取一点$P=(x_0, y_0)$,则称直线$l:x_0x+y_0y=R^2$是点$P$关于圆$C$的极线;反之,点$P$是直线$l$关于圆$C$的极点。

\end{definition}





更一般地,对于任意圆锥曲线$E$,都可以成对定义关于$E$的极点和极线。



\begin{definition}{关于圆锥曲线的极点和极线}

给定圆锥$E:\frac{x^2}{a^2}+\frac{y^2}{b^2}=1$,则点$(x_0, y_0)$关于$E$的极线为$\frac{x_0x}{a^2}+\frac{y_0y}{b^2}=1$。

给定双曲线$H:\frac{x^2}{a^2}-\frac{y^2}{b^2}=1$,则点$(x_0, y_0)$关于$H$的极线为$\frac{x_0x}{a^2}-\frac{y_0y}{b^2}=1$。

给定抛物线$P:y^2=2px$,则点$(x_0, y_0)$关于$P$的极线为$y_0y=p(x+x_0)$。

\end{definition}


$(zx_0, zy_0, z)$关于圆$x^2+y^2=R^2z^2$的极线为$x_0x+y_0y=R^2z$

















