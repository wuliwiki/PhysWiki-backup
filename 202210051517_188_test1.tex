% 分块矩阵

\pentry{矩阵\upref{Mat}}

把一个矩阵分成若干个子矩阵,称为\textbf{矩阵的分块},将矩阵看作是由子矩阵组成的矩阵,这种矩阵称为\textbf{分块矩阵}.
\begin{equation}
\mat M=
\begin{pmatrix}
\mat A & \mat B\\
\mat C & \mat D
\end{pmatrix}
\end{equation}

容易由矩阵的加法运算定义得到分块矩阵的加法运算规律

\begin{equation}
\mat A+\mat B=
\begin{pmatrix}
\mat A_1 & \mat A_2\\
\mat A_3 & \mat A_4
\end{pmatrix}
+
\begin{pmatrix}
\mat B_1 & \mat B_2\\
\mat B_3 & \mat B_4
\end{pmatrix}
=
\begin{pmatrix}
\mat A_1+\mat B_1 & \mat A_2+\mat B_2\\
\mat A_3+\mat B_3 & \mat A_4+\mat B_4
\end{pmatrix}
\end{equation}

由矩阵的数乘定义得到分块矩阵的数乘运算规律

\begin{equation}
k\mat A=
k\begin{pmatrix}
\mat A_1 & \mat A_2\\
\mat A_3 & \mat A_4
\end{pmatrix}
=
\begin{pmatrix}
k\mat A_1 & k\mat A_2\\
k\mat A_3 & k\mat A_4
\end{pmatrix}
\end{equation}
\textsl{其中$\mat A$是定义在数域$K$上的矩阵,$k\in{K}$}

分块矩阵的转置运算为

\begin{equation}
\mat A=
\begin{pmatrix}
\mat A_1 & \mat A_2\\
\mat A_3 & \mat A_4
\end{pmatrix}
,
\mat A'
=
\begin{pmatrix}
\mat A_1' & \mat A_3'\\
\mat A_2' & \mat A_4'
\end{pmatrix}
\end{equation}