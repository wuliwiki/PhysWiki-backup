% 乔治·比德尔·艾里(综述)
% license CCBYSA3
% type Wiki

本文根据 CC-BY-SA 协议转载翻译自维基百科 \href{https://en.wikipedia.org/wiki/George_Biddell_Airy}{相关文章}。

\begin{figure}[ht]
\centering
\includegraphics[width=6cm]{./figures/f2f2a06f3fcb14e2.png}
\caption{} \label{fig_AL_1}
\end{figure}
乔治·比德尔·艾里爵士(Sir George Biddell Airy,/ˈɛəri/;1801年7月27日-1892年1月2日)是一位英国数学家和天文学家,曾于1826年至1828年担任卢卡斯数学教授,并于1835年至1881年担任第七任皇家天文学家。他的诸多成就包括对行星轨道的研究、测定地球平均密度、提出二维固体力学问题的解法,以及在担任皇家天文学家期间,确立格林尼治作为本初子午线的所在地。
\subsection{传记}
艾里出生于诺森伯兰郡的安尼克,出身于一个世代相传的艾里家族,其家族可追溯至14世纪居住在西摩兰郡肯特米尔的同姓家族。他所属的这一支系因英国内战而遭受打击,遂迁至林肯郡并以务农为生。艾里的早期教育先是在赫里福德的初等学校完成,随后进入科尔切斯特皇家文法学校就读。\(^\text{[1]}\)艾里是个性格内向的孩子,但因擅长制作豌豆射手而在同学中颇受欢迎。\(^\text{[2]}\)

从13岁起,艾里经常住在他位于萨福克郡普莱福德的叔叔亚瑟·比德尔家中。比德尔将艾里介绍给了他在普莱福德庄园的朋友、废奴主义者托马斯·克拉克森。克拉克森是剑桥大学的数学硕士,先是考察了艾里的古典学水平,随后又请剑桥三一学院的一位研究员对他进行数学能力的考核。\(^\text{[3][4]}\)结果,艾里于1819年以“资助生”的身份进入三一学院学习,这意味着他缴纳较低的学费,但必须通过承担部分服务工作来抵消差额。\(^\text{[5]}\)在那里,他的学术表现十分出色,几乎立刻就被认为是当届最优秀的学生。1822年他当选为三一学院奖学金获得者,次年以“高级状元”身份毕业,并获得史密斯一等奖。

1824年10月1日,他被选为三一学院研究员;1826年12月,接替托马斯·特顿担任卢卡斯数学教授。此职位他仅担任了一年多,随后于1828年2月当选为普卢米安天文学教授,并出任新成立的剑桥天文台台长。\(^\text{[1]}\)1836年,他当选为皇家学会院士;1840年当选为瑞典皇家科学院外籍院士;1859年又成为荷兰皇家艺术与科学学院外籍院士。\(^\text{[6]}\)
\subsubsection{研究}
\begin{figure}[ht]
\centering
\includegraphics[width=6cm]{./figures/de664d7acdba7a8b.png}
\caption{乔治·比德尔·艾里} \label{fig_AL_2}
\end{figure}
在这些早期岁月中,艾里在数学和物理领域的写作活动之频繁,可从以下事实略窥一二:在他被任命之前,他已为《皇家学会哲学汇刊》撰写了三篇重要论文,为剑桥哲学学会撰写了八篇论文。在剑桥天文台,艾里很快展现出其出色的组织能力。接手时,该天文台唯一的望远镜是过境仪,他便将全部精力投入其上。通过建立一套规范的工作系统,并采取细致的观测数据整理计划,他能够使观测工作始终保持最新,并每年按时发表观测结果,这种准时的作风令当时的同行惊叹不已。不久之后,一台壁圈仪被安装起来,并于1833年开始定期使用进行观测。同年,诺桑伯兰公爵为剑桥天文台捐赠了一块口径为12英寸的优质物镜,该物镜的安装由艾里亲自设计并监督施工,尽管其建成已是他于1835年调任格林尼治之后。\(^\text{[1]}\)

艾里在这一时期的著作主要分为数学物理和天文学两类。前者大多源自他教授课程中与光学理论相关的问题,其中尤为重要的是他关于《圆孔物镜的衍射》一文,以及他对彩虹完整理论的阐述。因这些研究成果,他于1831年获得了皇家学会的科普利奖章。在这一时期的天文学著作中,最重要的包括:他对木星质量的研究、为英国科学促进会撰写的关于19世纪天文学进展的报告,以及他关于《地球和金星运动中一项长期不等项》的研究。\(^\text{[7]}\)

他的这篇精彩而富有启发性的报告中有一节专门探讨了“英国与其他国家在天文学进展方面的比较”,并指出英国在这方面的明显劣势。后来,这一批评在很大程度上因他本人的努力而得以改善。\(^\text{[8]}\)

\textbf{地球平均密度的测定}

艾里最杰出的研究之一是他对地球平均密度的测定。1826年,他萌生了一个想法:通过在一座深井的顶部和底部进行摆钟实验来解决这个问题。他于当年在康沃尔的多尔科斯矿(Dolcoath mine)首次尝试,但因其中一个摆钟发生意外而失败。第二次尝试在1828年,由于矿井被水淹没再次告吹。之后多年未能再次开展实验。直到1854年,他终于在英格兰北部南希尔兹附近的哈顿矿井(Harton pit)进行了实验。实验直接得出的结果是:在井底测得的重力比井顶大出其总量的 1/19286,矿井深度为383米(1,257英尺)。据此,他推算出地球的比密度为6.566。\(^\text{[9]}\)尽管这一数值远高于通过其他方法得出的早期结果,艾里仍坚信自己经过严谨细致的观测与论证所得出的这一结果“至少有资格与其他结果同台竞争”。\(^\text{[8]}\)目前公认的地球平均密度值为5.5153克/立方厘米。

\textbf{参考大地水准面}

1830年,艾里利用在英国境内进行的测量数据,计算出了地球的极半径和赤道半径。尽管他所使用的半径数值后来被更精确的数值所取代(如 GRS 80 和 WGS84 所采用的数值),但他的“艾里大地水准面”(严格来说是一种参考椭球体,称为 OSGB36)至今仍被英国测绘局(Ordnance Survey)用于英格兰、苏格兰和威尔士的地图绘制工作,因为它与当地的海平面拟合得更好(大约比全球平均海平面低80厘米)。\(^\text{[10][11]}\)

\textbf{行星摄动不等式}

艾里在金星与地球运动中的一个新不等式的发现,在某些方面是他最卓越的成就之一。在修正德朗布尔(Delambre)所编太阳表的轨道要素时,他开始怀疑该表的编制者忽略了某项不等项。这一原因他很快就找到了:金星平均运动的13倍几乎等于地球平均运动的8倍,两者之间的差异仅是地球平均运动的一小部分。尽管这一差值本身极小,但在对微分方程进行积分时,这一项会获得一个约为 2,200,000 的乘子,由此艾里推断出一个影响显著、周期长达 240 年的不等式(见《哲学汇刊》第122卷,第67页)。这项研究很可能是当时在行星理论领域中最为繁复的工作,也代表了自万有引力理论建立以来,英国在太阳表方面做出的第一项明确改进。

为表彰这项成就,英国皇家天文学会于1833年授予他金质奖章\(^\text{[8]}\)(他在1846年再次获得该奖)。
