% 费马小定理
% keys 费马小定理
% license Xiao
% type Tutor
\pentry{剩余类环\upref{RRing},整环\upref{Domain},域}
费马小定理是法国律师费马于1636年发现的,其由欧拉在1736年出版的名为“一些与素数有关的定理的证明”的论文集中第一次给出证明。其描述如下
\begin{theorem}{费马小定理}
若 $p$ 是素数,$m$ 是一个不能被 $p$ 整除的整数,则有同余式(\autoref{def_RRing_1}~\upref{RRing})
\begin{equation}
m^{p-1}\equiv 1(p)~.
\end{equation}
\end{theorem}
翻译成自然语言,就是说若整数 $m$ 不能被素数 $p$ 整除,则 $p$ 除 $m^{p-1}$ 余数为1。本文将用群论的方法证明该定理。这依赖于这样一个引理:
\begin{lemma}{}\label{lem_FmLT_1}
剩余类环\upref{RRing} $\mathbb Z_p$ 是个域(\autoref{def_field_4}~\upref{field}),当且仅当 $p$ 是素数。
\end{lemma}
\subsection{\autoref{lem_FmLT_1} 的证明}
若 $p$ 不是素数,则由素数定义,存在整数 $1<r,s<p$, 使得 $rs=p$,于是 $\bar r\bar s=\bar p=\bar 0$,这就是说 $\mathbb Z_p$ 中有零因子(\autoref{def_Domain_1}~\upref{Domain})$\bar r,\bar s$。由于域不可能有零因子,所以$p$ 非素数时 $\mathbb Z_p$ 不是域,由逆否命题的正确性,$\mathbb Z_p$ 是域则 $p$ 是素数。

其次,假设 $p$ 是素数,由于 $\mathbb Z_p$ 是有单位元的交换环(\autoref{the_RRing_1}~\upref{RRing}),证明其是域只需证明其上非零元都有逆元。

任意 $s\not\equiv 0(p)$,当 $k=1,\cdots,p-1$ 时,有 $ks\not\equiv 0(p)$。事实上,设 $s\in\bar r, r\in{1,\cdots,p-1}$, 则$\bar k\bar s=\bar k\bar r$,由于任一整数都可写成同一组两两不等的素数方幂的乘积,并考虑到 $k,r<p$, 那么:
\begin{equation}
k=p_1^{\alpha_1}p_2^{\alpha_2}\cdots p_n^{\alpha_n}p^0, \quad r=p_1^{\beta_1}p_2^{\beta_2}\cdots p_n^{\beta_n}p^0,\quad ks=p_1^{\alpha_1+\beta_1}p_2^{\alpha_2+\beta_2}\cdots p_n^{\alpha_n+\beta_n}p^0~,
\end{equation}
其中 $p_1,\cdots,p_n$ 是小于 $p$ 的素数,而
\begin{equation}
p=p_1^0p_2^0\cdots p_n^{0}p~.
\end{equation}
$p$ 整除 $kr$ 意味着 $p$ 是 $kr$ 的因数,这相当于在 $kr$ 的用素数分解式中函数 $p$ 的分解式(相当于分解式中各素数因子幂次都要大于因数对应素数的幂次)。然而 $kr$ 的分解式中 $p$ 的幂次为0,这意味着 $p$ 不是 $kr$ 的因数,即 $ks\not\equiv 0(p)$ 。

