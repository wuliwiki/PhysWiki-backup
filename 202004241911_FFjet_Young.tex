% 杨氏双缝干涉实验

用两个点波源作光的干涉实验的典型代表, 是杨氏实验.杨氏实验的装置如下图所示,
\begin{figure}[ht]
\centering
\includegraphics[width=8cm]{./figures/Young_1.png}
\caption{杨氏双缝实验} \label{Young_fig1}
\end{figure}
\begin{figure}[ht]
\centering
\includegraphics[width=7cm]{./figures/Young_2.png}
\caption{杨氏双缝实验} \label{Young_fig2}
\end{figure}
在普通单色光源前面放一个开有小孔$S$的屏,作为单色点光源.在$S $的照明范围内再放一个开有两个小孔$S_1,S_2$的屏.按惠更斯原理,$S$将作为两个次波源向前发射次波(球面波),形成交叠的波场.在较远的地方放置一接收屏,屏上可以观测到一组几乎是平行的直线条纹\autoref{Young_fig1}(b)\autoref{Young_fig2}(b) .为了提高干涉条纹的亮度,实际
中S、s, 、82 用三个互相平行的狭缝(杨氏双缝干涉),而且可不用屏幕接
收,而代之以目镜直接观测.在激光出现以后,人们可以用氨氖激光束直接
照明双孔,在屏幕上即可获得一套相当明显的干涉条纹,供许多人同时观
看.现在来分匠少」用普~光源做杨氏实验时,由双孔出射的两束光波之间
的相位差.设ss, =凡, 882 =R2, 用中o 代表点波源S 的初相位,则次波源8, 、
鸟的初相位分别为