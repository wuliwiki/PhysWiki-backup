% 南京航空航天大学 2012 量子真题
% license Usr
% type Note

\textbf{声明}:“该内容来源于网络公开资料,不保证真实性,如有侵权请联系管理员”

\subsection{简答题(本题 30 分,每小题 15 分)}
①光电效应哪些现象是经典理论不能解释的?

②举一个实验说明微观粒子具有波粒二象性。

\subsection{(本题 30 分,每小题 15 分)}
① 写出氢原子的束缚态能级$E_n$,、所有量子数以及这些量子数取值范围,并求能级$E_n$简并度;

②若氢原子处于状态$\psi(r, \theta, \varphi) = \frac{1}{2} R_{2,1}(r) Y_{1,0}(\theta, \varphi) - \frac{\sqrt{3}}{2} R_{2,1}(r) Y_{1,-1}(\theta, \varphi)$,问上述量子数,那些有确定值?那些没有?有确定值的给出其数值;没有确定值的给出其可能值以及出现几率,并求对应物理量平均值。

\subsection{(本题 30 分)}
一维谐振子的哈密密顿量为 $$ H_0 = \frac{\hbar^2}{2m} \frac{d^2}{dx^2} + \frac{1}{2} k x^2~$$,假设它处于基态,若在加上一个弹力作用$H' = \frac{1}{2} b x^2$,使用微扰论计算 $H'$ 对能量的一级修正,并与严格解比较。
