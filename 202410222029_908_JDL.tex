% 角动量(综述)
% license CCBYSA3
% type Wiki

(本文根据 CC-BY-SA 协议转载翻译自维基百科\href{https://en.wikipedia.org/wiki/Angular_momentum}{相关文章})

\begin{figure}[ht]
\centering
\includegraphics[width=6cm]{./figures/0e2f9582f8dddda6.png}
\caption{这个陀螺仪在旋转时由于角动量守恒而保持直立。常用符号:\( L \)   在国际单位制中的基本单位:kg⋅m²⋅s⁻¹   是否守恒:是   由其他量推导:\( L = I\omega = r \times p \)   量纲:\( \mathsf{M L^2 T^{-1}} \)} \label{fig_JDL_1}
\end{figure}
\textbf{角动量}(有时称为动量矩或旋转动量)是线性动量的旋转类比。它是一个重要的物理量,因为它是守恒量——封闭系统的总角动量保持不变。角动量既有方向也有大小,并且两者都守恒。自行车和摩托车、飞盘、膛线子弹以及陀螺仪的有用特性都归因于角动量守恒。角动量守恒也是飓风形成螺旋状以及中子星具有高速旋转率的原因。通常,守恒定律限制了系统可能的运动,但并不能唯一确定其运动方式。

三维角动量在经典力学中表示为伪向量 \( \mathbf{r} \times \mathbf{p} \),即粒子位置向量 \( \mathbf{r} \)(相对于某一原点)与其动量向量的叉积;后者在牛顿力学中为 \( \mathbf{p} = m\mathbf{v} \)。与线性动量不同,角动量取决于原点的选择,因为粒子的位置是从该原点测量的。

角动量是一种广延量;也就是说,任何复合系统的总角动量是其组成部分角动量的总和。对于连续的刚体或流体,系统的总角动量是角动量密度(即单位体积的角动量,当体积趋近于零时)的体积分在整个物体上的积分。

类似于线性动量守恒,如果没有外力作用,线性动量守恒;同样,如果没有外力矩作用,角动量也守恒。力矩可以定义为角动量的变化率,类似于力的作用。任何系统的净外力矩始终等于系统上的总力矩;系统内的所有内力矩之和总是为0(这是牛顿第三运动定律的旋转类比)。因此,对于封闭系统(没有净外力矩),系统的总力矩必须为0,这意味着系统的总角动量是恒定的。

特定相互作用下角动量的变化称为\textbf{角冲量},有时称为“旋转”。角冲量是线性冲量的旋转类比。[3]