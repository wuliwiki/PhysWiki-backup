% 韦达定理
% 代数方程|根与系数|数论|域|伽罗瓦|伽罗华|维达定理|Vieta's formula

\pentry{一元多项式\upref{OnePol}}
\addTODO{加入目录:《初等数学-一元多项式》的后面.}

韦达定理描述了多项式根与系数的关系.在中学数学中常见韦达定理最简单的形式,即实系数一元二次多项式的根与系数的关系.本词条要讨论的是最一般形式下的韦达定理.

\subsection{定理描述}

\begin{theorem}{一元二次多项式的韦达定理}\label{VietaF_the1}

设有域$\mathbb{F}$上的多项式$ax^2+bx+c$,其有两个根$x_1$和$x_2$,则有:
\begin{equation}
x_1+x_2 = -\frac{b}{a}
\end{equation}
\begin{equation}
x_1x_2 = \frac{c}{a}
\end{equation}

\end{theorem}

\autoref{VietaF_the1} 的推导在这里不赘述,用一元二次方程的求根公式就能解决.

事实上,韦达定理还适用于一般的实系数多项式的根,记忆起来也非常方便:

\begin{theorem}{韦达定理}\label{VietaF_the2}
设有实系数多项式$\sum_{i=0}^n a_ix^i$,据\textbf{代数学基本定理}\upref{BscAlg}知其应有$n$个根(重根按重数记),分别记为$x_1, x_2, \cdots, x_n$.则有:
\begin{equation}\label{VietaF_eq1}
\leftgroup{
    x_1+x_2+\cdots+x_n &= -\frac{a_{n-1}}{a_n}\\
    x_1x_2+x_1x_3+\cdots+x_{n-1}x_n &= \frac{a_{n-2}}{a_n}\\
    &\vdots\\
    x_1x_2\cdots x_n &= (-1)^n\frac{a_0}{a_n}
}
\end{equation}
\end{theorem}

显然,\autoref{VietaF_the1} 只是\autoref{VietaF_the2} 的一个特例:实系数一元二次多项式.实系数多项式指“实数域上的多项式”,即“域上的多项式”的特例,二次多项式则是“多项式”的特例.

记忆\autoref{VietaF_eq1} 也不难:第$i$个式子就是每$i$个根为一组相乘、所有组相加,结果是$(-1)^i\frac{a_{n-i}}{a_n}$,分母永远是最高次项的系数.





\subsection{定理证明}

按\autoref{VietaF_the2} 题设,知
\begin{equation}
\sum_{i=0}^n a_ix^i = A(x-x_1)(x-x_2)\cdots(x-x_n)
\end{equation}

展开后,比较各项系数得

\begin{equation}\label{VietaF_eq3}
    a_nx^n = Ax^n
\end{equation}
和
\begin{equation}\label{VietaF_eq2}
\leftgroup{
    a_{n-1}x^{n-1} &= A(-x_1-x_2-\cdots-x_n)x^{n-1}\\
    a_{n-2}x^{n-2} &= A(-x_1x_2-x_1x_3-\cdots-x_{n-1}x_n)x^{n-2}\\
    &\vdots\\
    a_{0} &= A(-x_1)(-x_2)\cdots(-x_n)
}
\end{equation}


将\autoref{VietaF_eq3} 代入\autoref{VietaF_eq2} ,整理后即得\autoref{VietaF_eq1} .








