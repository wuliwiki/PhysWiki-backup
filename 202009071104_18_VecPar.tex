% 平行性 (向量丛)
\pentry{曲率 (向量丛)\upref{VecCur}, 费罗贝尼乌斯定理\upref{FrobTh}}

本节采用爱因斯坦求和约定.

设$M$是$n$维微分流形, $E$是其上秩为$k$的光滑向量丛. 设给定了$E$上的联络$D$.

\subsection{平行截面}

向量丛$E$的截面$\xi\in\Gamma(E)$称为在联络$D$之下\textbf{平行的 (parallel)}, 如果
$$D\xi=0.$$
在局部标架$\{s_\alpha\}_{\alpha=1}^k$之下, 如果$\xi=\xi^\alpha s_\alpha$, $\omega$是此标架下的联络1-形式矩阵, 则$D\xi=0$等价于
$$
d\xi^\alpha+\xi^\beta\omega^\alpha_\beta=0,\,1\leq \alpha\leq k.
$$
进一步给定切丛和余切丛的局部标架$\{e_i\},\{\theta_j\}$ (二者为对偶) 之后, 就有了克氏符$\Gamma_{\beta i}^\alpha$, 因此上式进一步等价于方程组
$$
e_i(\xi^\alpha)+\xi^\beta\Gamma^\alpha_{\beta i}=0,\,1\leq \alpha\leq k,\,1\leq i\leq n.
$$
这是普法夫系. 在局部上, 根据费罗贝尼乌斯定理\upref{FrobTh}, 这方程组可解 (局部上相当于有$k$个函数$\{\xi^\alpha\}_{\alpha=1}^k$ 满足上面的偏微分方程组) 当且仅当
$$
0=d(\omega^\alpha_\beta\xi^\beta)=d\xi^\beta\wedge\omega^\alpha_\beta+\xi^\beta d\omega^\alpha_\beta=\xi^\beta\wedge\Omega_\beta^\alpha.
$$
这也就表示这个方程组可解当且仅当
$$
\xi^\beta\wedge\Omega_\beta^\alpha=0.
$$
因此如果$D$的曲率在某个开集上等于零, 则上述普法夫系是局部上可积的, 于是此开集上存在$E$的平行截面. 这样一来, 曲率是平行截面存在的障碍.

$\mathbb{R}^n$上平凡丛$\mathbb{R}^n\times\mathbb{R}^k$上的平凡联络就是通常的微分运算, 因此当然是可交换的, 曲率算子为零. 设$\partial_\alpha$是$\mathbb{R}^k$上的一个仿射坐标向量, 则它在此联络下就是平行的. 这很符合"平行"的直观意义. 

\subsection{平行移动}
向量丛在区域上的平行截面很可能不存在, 但却可以定义沿着某条道路平行的截面. 

设$\gamma:[0,a]\to M$是 Lipschitz 连续的道路, $\xi_0\in E_{\gamma(0)}$是沿着道路的截面. 它的严格定义是一个映射: $\xi:[0,a]\to E$, 使得$\xi(t)\in E_{\gamma(t)}$. 称此截面是\textbf{沿着道路$\gamma$平行的 (parallel along $\gamma$)}, 如果$D_{\gamma'}\xi(t)=0$. 此时称$\xi(a)$是$\xi(0)$沿着$\gamma$的\textbf{平行移动 (parallel transport)}. 

The \emph{parallel transport of $\xi_0$ along $\gamma$} is a section $s(t)$ of $E$ along $c$ that satisfies $D_{\gamma'}s=0$, or in local coordinates (with $s=\xi^\alpha s_\alpha$),
$$\frac{d\xi^\alpha}{dt}+\Gamma_{\beta i}^\alpha\xi^\beta\frac{d\gamma^i}{dt}=0.$$
This is a linear ODE in $(\xi^\alpha(t))$ with bounded coefficients, so it has a unique solution with any given initial value. This induces a linear operator $P_{\gamma}:E_{\gamma(0)}\to E_{\gamma(1)}$, i.e., sending $\xi_0\in E_{\gamma(0)}$ to $s(1)\in E_{\gamma(1)}$. Note that the differential equations imply that this operator does not depend on the parametrization of the path $\gamma$. Easy calculation gives that $P_\gamma^{-1}$ is in fact the parallel transform along the inverse path $\gamma^{-1}$. Furthermore, let $P_\gamma^{\tau,t}$ be the parallel transport from $E_{\gamma(s)}$ to $E_{\gamma(t)}$. Then an easy calculation gives that for any section $s\in\Gamma(E)$ and almost all $t$ (if $\gamma$ is $C^1$ then every $t$),
$$D_{\gamma'(t)}\xi=\lim_{\tau\to t}\frac{P_\gamma^{\tau,t}s(\gamma(\tau))-s(\gamma(t))}{\tau-t}.$$
Thus up to first order derivative, parallel transports along curves completely determine a connection.