% 一维单原子链晶格
% 声子|一维晶格|振动

为了研究复杂的三维晶格的性质,我们可以先从较为简单的结构入手,例如研究一维单原子链的情形,并且假定\textbf{仅相邻原子之间存在相互作用}.设一排 N 个相同的原子组成的简单晶体,原子质量为 $m$,相邻两个原子的相互作用能是它们之间距离的函数:$V(r)$.当晶体\textbf{处于平衡位置}时,相邻两原子间距离为 $a$,那么原子略微偏离平衡位置可以产生小振动\upref{Oscil},根据理论力学的相关知识,一维单原子链作为 $N$ 个自由度的力学体系,共有 $N$ 个独立简谐振动模式,对这些简谐振动模式的研究可以帮助我们认识一维晶格振动的情形.

为了衡量相邻原子间弹性恢复力的程度,我们把相互作用能 $V(r)$ 在 $a$ 附近进行傅里叶展开:
\begin{equation}
V(a+\delta)=V(a)+\frac{1}{2}\beta \delta^2+\text{高阶项}
\end{equation}
让我们\textbf{暂时地忽略高阶项},