% 向量丛和切丛
% 向量丛|切向量|截面|光滑截面
\begin{issues}
\issueDraft
\end{issues}
\addTODO{创建“环上的模”词条后,应将其引用为本词条的预备知识}

\pentry{纤维丛\upref{Fibre}}

“切丛”是“切向量丛”的简称.顾名思义,它是切向量构成的向量丛,而这里的切向量是在流形上定义的,因此切丛的底空间是流形.

\begin{definition}{切丛、}
给定流形$M$,以$M$为底空间,把各$p\in M$上的$T_pM$视为该点处的一根纤维,得到的纤维丛就称为流形$M$上的\textbf{切丛(tangent bundle)}.
\end{definition}

一个切向量场可以视为切丛的一种特殊的子集,称为“\textbf{截面(section)}”.使用这个术语是为了强调这种子集的特殊性,它在每一根纤维上都取且仅取一个点,看起来就像是纤维丛的一个截面.同样地,一个光滑切向量场有时也被称作切丛上的一个\textbf{光滑截面}.

局部来看,流形$M$上切丛的每根纤维是一个线性空间;整体来看,每个切向量场本身都可以看成一个向量,构成一个线性空间,记为$\mathfrak{X}(M)$.$\mathfrak{X}(M)$作为线性空间的\textbf{基域}和$M$是一致的,具体来说,实流形$M$上的$\mathfrak{X}(M)$,其数乘时使用的“数字”就是实数.但是我们常研究另一种情况,即把对$\mathfrak{X}(M)$进行数乘时的“数字”取为$M$上的一个光滑函数.这个时候,$\mathfrak{X}(M)$不再能被看成一个线性空间,而应该是一个环$C^\infty(M)$上的模.这里$C^\infty(M)$是$M$上全体光滑函数的集合,它只是一个环,不满足乘法逆元存在性.

流形$M$上也可以定义切丛之外的向量丛,同样有截面的概念,只不过此时的截面不再是切向量场了.向量丛$E$上全体光滑截面的集合,记为$\Gamma(E)$.由上所述,光滑向量场的集合$\mathfrak{X}(M)$是$\Gamma(E)$的特例,正如切丛是向量丛的特例.

向量丛之间有两类比较重要的映射:

\begin{definition}{点算子}
设$E$和$F$是同一个底空间$M$上的两个\textbf{向量丛},称丛映射$\varphi:E\rightarrow F$是一个\textbf{点算子(point operator)},如果对于任意的光滑截面$s\in\Gamma(E)$和点$p\in M$,当$s(p)=0$时必有$\varphi(s)(p)=0$.

换句话说,如果称$Z_s\{p\in M|s(p)=0\}$为$s$的\textbf{零化子(null set)},那么$\varphi$是一个点算子当且仅当$Z_s\subseteq Z_{\varphi{s}}$.
\end{definition}


