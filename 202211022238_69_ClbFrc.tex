% 库仑定律
% keys 库仑力|库仑定律|平方反比力|电介质常数

\begin{issues}
\issueDraft
\end{issues}

\pentry{万有引力\upref{Gravty}}

1785 年法国科学家库仑通过实验得到了\textbf{库仑定律}.实验由一个固定不动的带电钢球 $M$,和一个被细线悬挂的小带电钢球 $A$ 组成.库仑发现,悬挂着小钢球 $A$ 的细线倾斜了一个小角度,偏离了垂直位置一段距离,这意味着 $M$ 似乎对 $A$ 产生了一个排斥力.库仑用另一个与 $A$ 具有相同材质、大小和形状的小钢球 $B$ 与 $A$ 接触,使得 $B$ 平分了 $A$ 球上的电荷,于是 $A$ 球携带的电荷变为了原来的 $1/2$ 倍.实验现象是:此时钢球 $A$ 偏离垂直位置的距离大约变为了原来的 $1/2$.库仑由此总结出:两电荷间的静电力是与它们分别所携带的电荷量成正比的.库仑继续进行他的实验,他改变钢球 $A$ 悬挂的位置,使得它与 $M$ 的距离变为原来的 $2$ 倍、$3$ 倍、$4$ 倍……实验结果是,$A$ 球偏离垂直位置的距离依次变为原来的 $1/4$ 倍、$1/9$ 倍、$1/16$ 倍.这个实验结果启发库仑提出了静电相互作用的平方反比律.
\begin{figure}[ht]
\centering
\includegraphics[width=12cm]{./figures/ClbFrc_1.png}
\caption{库仑定律实验装置} \label{ClbFrc_fig1}
\end{figure}

\subsection{库仑定律}
\textbf{真空中}两个\textbf{静止}的点电荷间的相互作用被\textbf{库仑定律}所描述.在阐释这个定律之前我们需要先定义\textbf{点电荷},点电荷就是在质点的基础上(忽略物体的大小与形状), 增加了一个总电荷量的属性.这样以后我们的质点往往携带这些物理量 "$\bvec r=(x,y,z),m,q$" 分别代表它的坐标、质量、电荷量.这些量可能在运动以及相互作用的过程中发生变化,但我们通常假定它的\textbf{静止质量}、它携带的\textbf{电荷}是该质点所不变的属性\footnote{除非多个质点间发生了碰撞,则此时我们需要重新对质点在碰撞前后的变化进行定义,使之符合我们的实验现象.}.库仑定律描述了真空中两个静止点电荷间的相互作用力,它的本质不同于引力,却有与引力几乎相同的数学表达形式:它与两个点电荷的电荷量成正比,且与两点电荷的距离的平方成反比.与万有引力的\autoref{Gravty_eq1}~\upref{Gravty}类似,两个点电荷间的库仑力的矢量表达式为
\begin{equation}\label{ClbFrc_eq1}
\bvec F_{12} = k\frac{q_1 q_2}{r_{12}^2} \uvec r_{12} = \frac{1}{4\pi\epsilon_0} \frac{q_1 q_2}{\abs{\bvec r_2 - \bvec r_1}^3}(\bvec r_2 - \bvec r_1)
\end{equation}
其中 $\bvec F_{1,2}$ 表示点电荷 $2$ 所受到的力的大小,$\hat{\bvec r}_{12}$ 表示从 $1$ 指向 $2$ 的单位矢量.$q_1, q_2$ 分别是两个点电荷的电荷量.注意比起万有引力的\autoref{Gravty_eq1}~\upref{Gravty}, \autoref{ClbFrc_eq1} 没有负号, $q_1, q_2$ 可以是负数(代表负电荷).当它们同号时 $\bvec F_{12}$ 是由 $1$ 指向 $2$ 的矢量,而当它们异号时 $\bvec F_{12}$ 则是从 $2$ 指向 $1$ 的矢量.我们容易看出两电荷同号相吸, 异号相斥.

$k$ 是库仑常数,可以在实验中测量;我们也常常将这个符号替换为
\begin{equation}
k = \frac{1}{4\pi\epsilon_0} \approx 8.9876\e9 \Si{N m^2/C^2}
\end{equation}
$\epsilon_0$ 是\textbf{真空中的电介质常数(vacuum permittivity)},这将在我们之后的学习过程中经常出现;而上式中 $4\pi$ 的出现将使得我们将用到的 “高斯定律\upref{EGauss}” 具有更加简洁的形式.这里我们采取的是\textbf{ SI 单位制}(国际单位制\upref{SIunit}),而描述电磁相互作用力实际上还有许多其他单位制,在不同的场合中经常被使用.