% 分块矩阵

\begin{issues}
\issueDraft
\end{issues}

\subsection{分块矩阵 Partitioned Matrices}
有时,把一个大矩阵$\mat M$视为若干个小矩阵的“组合”可以帮助简化问题
\begin{equation}
\mat M = 
\begin{bmatrix}
\mat A & \mat B\\
\mat C & \mat D\\
\end{bmatrix}
\end{equation}
其中 $\mat A, \mat B, \mat C, \mat D$都是矩阵(可以不是方阵).这种划分是必须有“意义”的,即$\mat A$与$\mat C$的行数相同,与$\mat B$的列数相同,等等.

\subsubsection{分块矩阵的加法、乘法}
若相应的计算都有定义,那么分块矩阵的加法、乘法法则与普通矩阵形式上完全类似.
\begin{equation}
\begin{bmatrix}
\mat A & \mat B\\
\mat C & \mat D\\
\end{bmatrix}
+
\begin{bmatrix}
\mat E & \mat F\\
\mat G & \mat H\\
\end{bmatrix}
=
\begin{bmatrix}
\mat A +\mat E & \mat B +\mat F\\
\mat C +\mat G & \mat D +\mat H\\
\end{bmatrix}
\end{equation}

\begin{equation}
\begin{bmatrix}
\mat A & \mat B\\
\mat C & \mat D\\
\end{bmatrix}
\begin{bmatrix}
\mat E & \mat F\\
\mat G & \mat H\\
\end{bmatrix}
=
\begin{bmatrix}
\mat A \mat E +\mat B \mat G & \mat A \mat F +\mat B \mat H \\
\mat C \mat E +\mat D \mat G  & \mat C \mat F +\mat D \mat H \\
\end{bmatrix}
\end{equation}

%相乘时可以把每块看成一个元素, 元素之间的乘法就是块的矩阵乘法.

\subsection{块对角矩阵}
一般讨论对称的块对角矩阵: $\bvec y_i$ 和 $\bvec x_i$ 的长度和划分相同.

定义对角块

块对角矩阵: 只有对角块为非零的矩阵. 在每个子空间中分别映射, 所以可以对每块分别处理. 例如计算本征问题时.
