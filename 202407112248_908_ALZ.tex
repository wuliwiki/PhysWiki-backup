% Α粒子
% license CCBYSA3
% type Wiki

(本文根据 CC-BY-SA 协议转载自原搜狗科学百科对英文维基百科的翻译)

阿尔法粒子,也称为阿尔法射线或阿尔法辐射,是由两个质子和两个中子结合成的粒子,与氦-4原子核相同。它们通常是在$\alpha$衰变过程中产生的,但也可能以其他方式产生。$\alpha$粒子是以希腊字母表中的第一个字母$\alpha$命名的。$\alpha$粒子的符号是$\alpha$或$\alpha$²⁺。因为它们与氦核相同,所以有时也被写成$He^{2+}$或4$^4$ $_2He^{2+}$表明氦离子带+2电荷(缺少两个电子)。如果离子从其环境中获得电子,阿尔法粒子就变成正常(电中性)氦原子$^4$ $_2He$。

阿尔法粒子像氦原子核一样,净自旋为零。由于它们在标准α放射性衰变中的产生机制,$\alpha$粒子通常具有约5兆电子伏的动能,速度约为光速的5\%。它们是高度电离形式的粒子辐射,并且(当由放射性α衰变产生时)具有低穿透深度。它们可以被几厘米的空气阻挡,也可以被皮肤阻挡。

然而,三元裂变产生的所谓长程$\alpha$粒子具有三倍的能量,穿透距离也是三倍。如前所述,构成宇宙射线10-12\%的氦核通常也比核衰变过程产生的氦核能量高得多,因此能够高度穿透并穿越人体,并根据其能量能够穿透数米的致密固体屏蔽。在较小程度上,粒子加速器产生的高能氦核也是如此。

当辐射$\alpha$粒子的同位素被摄入时,它们比半衰期或衰变率所显示的要危险得多,因为$\alpha$辐射具有较高的生物效应,会造成生物损伤。相比比$\beta$辐射或$\gamma$辐射放射性同位素,$\alpha$粒子平均危险约20倍,在吸入$\alpha$辐射源的实验中,危险高达1000倍。[1]