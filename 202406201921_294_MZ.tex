% μ子
% license CCBYSA3
% type Wiki


(本文根据 CC-BY-SA 协议转载自原搜狗科学百科对英文维基百科的翻译)
\subsection{μ子}

$\mu$子(/ˈmjuːɒn/;来自用于表示它的希腊字母 mu($\mu$))是一种类似于电子的基本粒子,电荷量为−1e,自旋为 1/2,但是$\mu$子拥有比电子更大的质量。$\mu$子被归类为轻子。与其他轻子一样,子不被认为有任何亚结构——也就是说,它不被认为由任何更简单的粒子组成。

$\mu$子是一种不稳定的亚原子粒子,其平均寿命为2.2 $\mu$s,这比许多其他亚原子粒子都要更长。如同非基本粒子的中子的衰变一样(中子寿命约为15分钟),$\mu$子的衰变(按亚原子标准)发生得很慢,这是因为$\mu$子的衰变是完全由弱相互作用引起的(而不是更强大的强相互作用者电磁相互作用);且因为$\mu$子和它的衰变产物间的质量差很小,所以它的衰变只有很少的动力学自由度。$\mu$子的衰变几乎总是产生至少三个粒子,其中必须定括一个与$\mu$子拥有相同电荷的电子以及两个不同类型的中微子。

像所有基本粒子一样,$\mu$子也有相应的反粒子,后者拥有与前者相反的电荷(+1e),但两者的质量和自旋则是相同的,这个反粒子就是反$\mu$子(也称为正$\mu$子)。$\mu$子被表示为$\mu$−,反$\mu$子则是$\mu$+。$\mu$子以前被称为$\mu$介子,但现代粒子物理学家不再将其归类为介子,因此物理界也就不再使用$\mu$介子这个名称了。

$\mu$子的质量为105.66 MeV/c2,这大约是电子的207倍。由于$\mu$子的质量更大,当$\mu$子遇到电磁场时,其加速不会那么快,也不会发射那么多轫致辐射(减速辐射)。这使得拥有给定能量的$\mu$子在物质中可以比电子穿透得更深,因为电子和$\mu$子的减速主要是由于轫致辐射机制所导致的能量损失。例如,由宇宙射线撞击大气层所产生的所谓“次级$\mu$子”可以穿透到地表,甚至深入矿井。

因为$\mu$子的质量和能量比起放射性过程的衰变能量来说非常大,所以$\mu$子从来都不会由放射性衰变产生。然而,它们可以大量地产生于正常物质中的高能相互作用、涉及到强子的某些粒子加速器实验或者宇宙线与物质间自然发生的相互作用中。这些相互作用通常先产生$\pi$介子,$\pi$介子通常会衰变成$\mu$子。

与其他带电轻子一样,$\mu$子有一种相对应的$\mu$子中微子,后者被表示为$v_\mu$。$\mu$子中微子不同于电子中微子,它们不参与相同的核反应介子,$\pi$介子通常会衰变成$\mu$子。

与其他带电轻子一样,$\mu$子有一种相对应的$\mu$子中微子,后者被表示为$v_\mu$。$\mu$子中微子不同于电子中微子,它们不参与相同的核反应