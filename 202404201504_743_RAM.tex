% 随机存储器
% keys 随机存储器|随机存取存储器|内存
% license Xiao
% type Tutor

\begin{issues}
\issueDraft
\end{issues}

\subsection{随机存储器——RAM}

随机存储器(Random Access Memory, RAM)是一种可读可写的存储器,其任何一个存储单元都是可以随机存取的,而且存取时间与单元的物理位置无关。

这种存储器通常用于存储正在运行的程序和当前正在处理的数据。与其他存储介质(如硬盘驱动器、光盘)相比,RAM提供了更快的读写速度,但它是易失性的,意味着当电源关闭时存储的信息会丢失。

根据信息存储的物理原理,又可以分为静态随机存储器(SRAM)和动态存储器(DRAM)。


\subsection{静态随机访问存储器——SRAM}

静态随机访问存储器(Static random-access memory,SRAM)其中的Static意味着:只要保持通电,存储的数据就可以恒常保持。

具体地,一个SRAM基本单元有0 和 1两个电平稳定状态。SRAM使用六个晶体管存储单元的状态存储一个数据位,通常使用六个MOSFETs,这种RAM的生产成本更高,但通常比DRAM更快且动态功耗更低。

SRAM基本单元由两个CMOS反相器组成。两个反相器的输入、输出交叉连接,即第一个反相器的输出连接第二个反相器的输入,第二个反相器的输出连接第一个反相器的输入。这就能实现两个反相器的输出状态的锁定、保存,即储存了1个比特的状态。

\begin{figure}[ht]
\centering
\includegraphics[width=6cm]{./figures/87070ace2adbda7b.png}
\caption{六管SRAM存储单元示意图} \label{fig_RAM_3}
\end{figure}

\subsubsection{特性}


但是SRAM用了太多MOS管,而且总有两个MOS管饱和导通,占空间,功耗大。而且地址线多,导致不能做太大。

SRAM往往集成在芯片内部,一般的使用场景有:
作为x86等微处理器的缓存(如L1、L2、L3)
作为寄存器(寄存器堆)
用于FPGA与CPLD

\subsection{动态随机访问存储器——DRAM}


通常我们俗称的"内存"一般是指计算机中的DRAM。具体的说,内存条通常使用DDR4 DRAM,其容量一般为8G或16G,通常,我们需要插两根来组成双通道内存,即构成16G或32G的内存(memory)。

严格来讲,“内存”应该是所有易失性存储器的统称;相对应的,\textbf{非易失性存储器}统称为“外存”。



% \subsection{NVRAM}

% 我们上述所说的内存介质都需要通电来维持数据,这些存储器都属于易失性存储器。然而NVRAM()


参考文献:
\begin{enumerate}
\item 唐朔飞。 计算机组成原理[M]. 高等教育出版社。 2008
\end{enumerate}
