% 正弦型函数(高中)
% keys 正弦型函数|五点法作图|相位|角频率|信号
% license Usr
% type Tutor

\begin{issues}
\issueDraft
\end{issues}

\pentry{函数视角下的三角函数\nref{nod_HsTFFv},三角恒等变换\nref{nod_HsAnTf}}{nod_373d}

前面所介绍的三角函数的基本图像是理解和应用的重要基础,需要熟记。这些图像源于对各个自变量对应函数值的计算,凝聚了数学家的长期探索成果。

此前,在研究幂函数等函数时,通常通过分析关键点和整体趋势来绘制其图像。同样的方法也适用于三角函数。正弦函数在所有三角函数中具有特殊地位,许多涉及三角函数的函数在化简和推导后,都可以表示为正弦函数的某种变形。其实在前面的介绍中已经接触过这种例子了,根据诱导公式$\cos x$可以表示为:
\begin{equation}\label{eq_HsSinF_1}
\cos x=\sin(x+{\pi\over2})~.
\end{equation}
因此,接下来将以正弦函数为例,探讨如何绘制并理解相关的任意图像。

\subsection{正弦型函数}

目前研究的三角函数均为 $\sin x$ 形式,即未涉及额外参数。为了进一步拓展应用,需要引入正弦型函数,以更一般的形式刻画这些变形。正弦型函数是对基本正弦函数的扩展,它通过调整振幅、频率和相位来适应不同的周期性变化。

\begin{definition}{正弦型函数}
形如
\begin{equation}
f(x) = A\sin(\omega x + \varphi)~.
\end{equation}
的函数称为\textbf{正弦型函数(sinusoidal function)},其中 $A, \omega, \varphi$ 为常数,且满足 $A\omega \neq 0$。其中:
\begin{itemize}
\item $|A|$ 称为\textbf{振幅(amplitude)};
\item $\omega x + \varphi$ 称为\textbf{相位(phase)}。
\end{itemize}
\end{definition}

如 \autoref{eq_HsSinF_1} 所示,$\cos x$ 可以视为正弦型函数的一种特殊形式,其中 $\displaystyle A = \omega = 1, \varphi = \frac{\pi}{2}$。此外,通过适当的变形,可以将参数调整至更规范的范围。利用 $\sin(-x) = -\sin x$ 和 $-\sin x = \sin(x + \pi)$,可以确保 $A$ 和 $\omega$ 取正值,而所有的符号变化都体现在 $\varphi$ 的取值变化上\footnote{需要注意的是,这并不意味着仅仅改变 $\varphi$ 的符号,具体情况将在后续讨论中说明}。因此,在规范化的表达中,参数满足 $A \in (0, +\infty)$,$\omega \in (0, +\infty)$。教科书中通常使用 $|A|$ 和 $|\omega|$ 进行表示。同样,相位  $\phi $ 的取值范围一般是任意实数,但在实际应用中,通常约定它的范围在$[0, 2\pi)$  或  $(-\pi, \pi]$ ,以避免冗余描述。由于三角函数的周期性,如果不在此范围内,可以利用 $\sin x = \sin(x + 2k\pi)$ 进行调整,使其化为符合规范的形式。为了简化讨论,后续内容均默认正弦型函数已转换为上述标准形式。

一下子引入多个参数可能会让人感到眼花缭乱,但它们的核心作用是刻画正弦型函数相较于标准正弦函数 $\sin x$ 的变化。这些参数的设置旨在描述明确函数的变换规律,使其与 $\sin x$ 的对应关系更加清晰。

\subsection{相位}

在上面提及的新概念中,相位 $\omega x + \varphi$ 尤其值得关注。与以往的参数不同,它不是单独作为一个数值出现,而是作为整体引入,从而提供了一种新的视角来理解三角函数的变化。具体而言,由于 $\sin x$ 是一个非线性函数,直接分析其变化规律并不直观。因此,可以借鉴指数函数的处理方式,将 $\sin(\omega x + \varphi)$ 视为一个复合函数,其中 $\omega x + \varphi$ 对 $x$ 进行线性变换,作为 $\sin$ 的输入,而 $\sin$ 仅在最后起到非线性映射的作用,将输入限制在 $[-1,1]$ 之间,而不直接影响 $x$ 的变换过程。

这种分解方式将三角函数的周期性的非线性行为与输入变量的线性变化分离,使得分析更加直观,并有助于理解各参数对函数图像的影响。在数学建模中,许多非线性问题,如神经网络或回归分析,也常采用类似的方法:先处理线性部分,再通过非线性映射得到最终结果。这种思路不仅简化了分析过程,还在广泛的数学和工程领域中发挥了重要作用。

这里要注意的是,尽管在英语中使用相同的单词,但\textbf{相(phase)} 和 \textbf{相位(phase)}却指向两个不同的概念。理解这两个概念的区别,对于准确把握相位的意义至关重要。

先想象这样一个过程:垂直向上抛出一个球,当球达到某个高度时,它可能处于上升阶段,也可能处于下降阶段。仅凭高度本身无法判断球的运动趋势,必须结合其运动方向的信息。“相”指的是周期性变化过程中某个特定的状态,例如某一时刻球的高度和运动方向的组合,就像一张记录了该瞬间所有信息的特殊照片。而“相位”则标识了该状态在整个周期中的位置,类似于给照片附加的时间戳,使其明确对应于周期内的哪个时刻。

同样,在 $\sin x$ 的周期内,虽然同一个 $y$ 值通常对应两个不同的 $x$ 值,但这两个点的导数符号相反,意味着它们的运动趋势不同。因此,$(y, y')$ 这对信息可以唯一地确定周期运动中的某个状态,即“相”,而 $x$ 则是指向该状态的唯一“相位”。在正弦型函数中,$x$ 的位置由 $\omega x + \varphi$ 代替,因此确定 $\omega x + \varphi$ 也就等同于确定 $\sin x$ 在周期内的具体位置,从而确保状态信息的完整性。在相位的表达式中:
\begin{itemize}
\item $\omega$ 称作\textbf{角频率(angular frequency,也称圆频率)},它控制输入值的增长速度,即 $x$ 变化时 $\omega x$ 变化的速率,从而决定了函数的变化频率。
\item $\varphi$ 称为\textbf{初相(initial phase)},它设定了初始相位,即 $x=0$ 时,决定了函数图像相处于标准 $\sin x$ 的哪一个相。
\end{itemize}

\subsection{正弦型函数中参数的含义}

接下来依次分析 $f(x)$ 中的各个参数,并探讨它们对函数行为的具体影响。

\subsubsection{振幅$A$}

$A$ 决定了 $f(x)$ 的值域 $\left[-A, A\right]$。如果将正弦函数视为一种振动,那么 $A$ 代表的是函数的最大偏离值,这也是“振幅”一词的来源。换句话说,$A$ 控制了函数图像在垂直方向上的伸缩,最高点与最低点的差为 $2A$。
\subsubsection{角频率$\omega$}
$\omega$ 决定了函数变化的快慢,即图像在水平方向上的压缩或拉伸程度。$\omega$ 越大,周期越短,函数振荡得越快。\textbf{周期(cycle)} $T$ 与 $\omega$ 之间的关系为:
\begin{equation}
    T = \frac{2\pi}{\omega}~.
\end{equation}
在实际应用中,有时直接用 $T$ 代替 $\omega$ 来描述函数的周期性:
\begin{equation}
    f(x) = A\sin\left(\frac{2\pi}{T} x + \varphi\right)~.
\end{equation}

从圆周运动的视角来理解。角频率 $\omega$ 是圆周运动特有的量,表示单位时间内转过的弧度。在日常生活中,与此类似的另一个常见的概念是\textbf{频率(frequency)},它频率表示单位时间内完成的完整振荡次数,在圆周运动中,它对应于物体转过的圈数。由于一个完整的圆周对应 $2\pi$ 弧度,因此频率 $f$ \footnote{注意,尽管习惯上都用字母$f$,但频率是一个常数,而函数则对应一个映射关系。}与角频率 $\omega$、周期 $T$ 的关系为:
\begin{equation}
    f = \frac{1}{T} = \frac{\omega}{2\pi}~.
\end{equation}
这一关系表明,角频率和频率本质上是同一现象的不同刻画方式。值得注意的是,这里的分析基于圆周运动的概念,这也正体现了“三角函数”也称为“圆函数”这件事。
