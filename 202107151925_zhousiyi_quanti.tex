% 标量场的量子化
% 标量场|量子化
\pentry{经典场论基础\upref{classi}}

这一节的目的是给大家介绍如何对最简单的克莱因-戈登场进行量子化.

量子化的步骤是:把$\phi$场和$\pi$场升级为算符,然后在上面加入合适的对易关系.在场论里,因为$\phi$可以类比于坐标,而$\pi$可以类比于动量,那么场的正则对易关系为
\begin{aligned}
& [\phi(\mathbf x),\pi(\mathbf y)] = i \delta^{(3)}(\mathbf x- \mathbf y) \\
& [\phi(\mathbf x),\phi(\mathbf y)]  = [\pi(\mathbf x),\pi(\mathbf y)] = 0
\end{aligned}
一般来说在动量空间里面研究问题比较方便.那么我们把$\phi$场换到动量空间中.那么克莱因-戈登方程的形式为
\begin{equation}
\bigg[\frac{\partial^2}{\partial t^2}+(|\mathbf p|^2+m^2)\bigg] \phi(\mathbf p, t) = 0
\end{equation}
这也就是一个能量为$\omega_{\mathbf p}$的简谐振子的运动方程.$\omega_{\mathbf p}$的表达式如下
\begin{equation}
\omega_{\mathbf p} = \sqrt{|\mathbf p|^2+m^2}
\end{equation}

现在我们来找克莱因-戈登场的谱.用的是跟量子力学里面学到的方法类似的方法,首先我们要对场$\phi$和场$\phi$进行量子化
\begin{equation}
\begin{aligned}
& \phi(\mathbf x) = \int \frac{d^3p}{(2\pi)^3} \frac{1}{\sqrt{2\omega_{\mathbf p}}}\bigg( a_{\mathbf p} e^{i \mathbf p \cdot \mathbf x} + a_{\mathbf p}^\dagger e^{-\mathbf p \cdot \mathbf x} \bigg) \\
& \pi(\mathbf x) = \int \frac{d^3p}{(2\pi)^3} (-i) \sqrt{\frac{\omega_{\mathbf p}}{2}} \bigg( a_{\mathbf p} e^{i \mathbf p \cdot \mathbf x} - a^{\dagger}_{\mathbf p} e^{-i \mathbf p \cdot \mathbf x} \bigg)
\end{aligned}
\end{equation}
可以证明,正则对易关系可以化简为如下的形式
\begin{equation}
[a_{\mathbf p},a_{\mathbf p'}^\dagger] = (2\pi)^3 \delta^{(3)} (\mathbf p - \mathbf p')
\end{equation}
前面我们已经推导过哈密顿量的表达式,现在把$\phi$场和$\pi$场的biao'da's


