% p5.js 笔记
% license Usr
% type Note

\pentry{JavaScript 入门笔记\upref{JS}}

\begin{itemize}
\item \href{https://wuli.wiki/apps/PtAcc/}{质点加速的例子}
\item \href{https://wuli.wiki/apps/TwoBallSpring/}{百科知识树加速}
\end{itemize}

\subsection{常识}
\begin{itemize}
\item 使用 html5 的一个 \verb`<canvas>`
\item 通过编程的方式创建 GUI
\item 坐标原点默认在左上角,向右为 $x$, 向下为 $y$,单位是像素。
\item \verb`createCanvas(400, 400);` 创建 canvas。
\item \verb`function setup()` 会先执行,然后每一帧执行一次 \verb`function draw()`。
\item \verb`frameRate(30);` 设置每秒钟的帧率
\end{itemize}

\subsection{几何绘图}
\begin{itemize}
\item \verb`stroke(0, 0, 0, 50);` 规定划线的 RGB 和透明度(0-255)。
\item \verb`fill(200);` 规定长方形等的填充颜色。
\item \verb`line(x1, y1, x2, y2);` 画一条线。
\item \verb`background(255);` 设置背景颜色(可以是 RGB 三个输入)
\item \verb`textSize(16);` 设置字体颜色
\item \verb`text("This is a text box!", x, y);` 显示一行字
\end{itemize}

\subsection{鼠标控制}
\begin{itemize}
\item 全局变量 \verb`mouseX, mouseY` 是鼠标的坐标
\item \verb`function mousePressed()` 按下鼠标的回调函数
\item \verb`function mouseReleased()`
\end{itemize}

\subsection{向量运算}
\begin{itemize}
\item \verb`let v = createVector(10, 20, 30);` 生成一个向量(二维或三维)
\item 用 \verb`v.x, v.y, v.z` 来获取
\item \verb`v.mag()` 获取长度, \verb`v.mult(s)` 乘以标量
\end{itemize}

