% 数列的概念与函数特性(高中)
% 高中|数列的概念|数列的函数特性

\begin{issues}
\issueDraft
\end{issues}

\subsection{定义与相关概念}
一般地,按一定次序排列的一列数叫做\textbf{数列(sequence)},数列中的每一个数叫作这个数列的\textbf{项}.数列的一般形式可以写成
\begin{equation}
a_1,a_2,a_3,\cdots,a_n,\cdots
\end{equation}
简记为数列 $\begin{Bmatrix} a_n \end{Bmatrix}$,其中数列的第1项 $a_1$,也称\textbf{首项};$a_n$ 是数列的第 $n$ 项,也叫数列的\textbf{通项}.

项数有限的数列,称为\textbf{有限数列};项数无限的数列,称为\textbf{无穷数列}.

如果数列 $\begin{Bmatrix} a_n \end{Bmatrix}$ 的第 $n$ 项 $a_n$ 与 $n$ 之间的函数关系可以用一个式子表示成 $a_n = f(n)$,那么这个式子就叫作这个数列的\textbf{通项公式},数列的通项公式就是相应函数的解析式.

\textsl{注意:不是所有数列都能写出通项公式.}

\subsection{函数特性}
