% 艾萨克·牛顿(综述)
% license CCBYSA3
% type Wiki

本文根据 CC-BY-SA 协议转载翻译自维基百科\href{https://en.wikipedia.org/wiki/Isaac_Newton}{相关文章})

\begin{figure}[ht]
\centering
\includegraphics[width=6cm]{./figures/0c83e6f3dfbe0a8c.png}
\caption{《46岁的牛顿肖像,1689年》} \label{fig_Newton_1}
\end{figure}
艾萨克·牛顿爵士,皇家学会会员(1642年12月25日-1726/27年3月20日[a]),是一位英国博学家,活跃于数学、物理学、天文学、炼金术、神学和写作领域,在他所在的时代被称为自然哲学家。他是科学革命及其后的启蒙运动中的关键人物。他的开创性著作《自然哲学的数学原理》首次出版于1687年,汇集了许多前人的研究成果,奠定了经典力学的基础。牛顿还在光学方面做出了开创性的贡献,并与德国数学家戈特弗里德·威廉·莱布尼茨共同被认为是微积分的创立者,尽管他在莱布尼茨之前几年就已发展了微积分。[10][11]

在《自然哲学的数学原理》中,牛顿制定了运动定律和万有引力定律,这些理论成为数个世纪以来主导性的科学观点,直到相对论的出现。他利用对重力的数学描述推导了开普勒的行星运动定律,解释了潮汐、彗星轨迹、岁差等现象,消除了关于太阳系日心说的疑虑。他展示了地球上的物体和天体的运动可以由相同的原理解释。牛顿推测地球为扁球体,这一推测后来由莫佩尔蒂、拉康达米娜等人的测地测量所证实,使得大多数欧洲科学家信服于牛顿力学的优越性。

他制造了第一个实用的反射望远镜,并基于棱镜将白光分解为可见光谱的颜色的观察,发展出一套精细的颜色理论。他关于光的研究汇集于其极具影响力的著作《光学》中,1704年出版。他提出了一个经验性的冷却定律,这是第一个热传导的表述,首次对声速进行了理论计算,并引入了牛顿流体的概念。此外,他还对电进行了早期研究,他在《光学》一书中的一个设想可以说是电场理论的开端。作为数学家,除了微积分的研究外,他还对幂级数进行了研究,将二项式定理推广至非整数指数,发展出求解函数根的方法,并分类了大部分的三次平面曲线。

