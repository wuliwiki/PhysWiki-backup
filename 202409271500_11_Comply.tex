% 李代数的复化
% license Usr
% type Tutor

对于李群$G$,我们已经知道,李代数$\mathfrak g\equiv \opn{Lie} G$是李群上的全体左不变切场,该线性空间一定得定义在实数域上,才能导出在该李群的积分曲线。比如$X\in \mathfrak g$的积分曲线若为$c(t)$,则对于任意$s\in \mathbb R$,$sX$的积分曲线是$c(st)$,相当于把原积分曲线重新参数化。
\begin{definition}{}
若$\mathfrak g$是实数域上的李代数,我们可以将其扩展到复数域上。定义一个新的向量空间为
\begin{equation}
\mathfrak g+\I \mathfrak g=\{X+\I Y|X\,,Y\in\mathfrak g\}~.
\end{equation}
\end{definition}
可以验证李括号运算在其上封闭,且李括号依然有双线性和结合性。所以这也是一个李代数,称为$\mathfrak g$的\textbf{复化(complification)}。
\begin{example}{$\mathfrak u(n)$的复化}
类似于$\opn{Lie}GL(n,\mathbb R)\cong \mathfrak{gl}(n,\mathbb R)$的证明,易证$\opn{Lie}GL(n,\mathbb C)\cong \mathfrak{gl}(n,\mathbb C)$。下面主要证明这个复数版本的李代数恰为$\mathfrak u(n)$的复化,简单表示为$\mathfrak{gl}(n,\mathbb C)=\mathfrak u(n)+\I \mathfrak u(n)$。

\textbf{证明:}

$\mathfrak{gl}(n,\mathbb C)$由$n$阶复数矩阵构成,$\mathfrak u(n)$则由$n$阶反厄米矩阵构成。因此显然$\mathfrak{gl}(n,\mathbb C)\supseteq \mathfrak u(n)+\I \mathfrak u(n)$。现设对于任意$A\in \mathfrak gl(n,\mathbb C)$,都有$A=B+\I C$,其中$B,C\in \mathfrak u(n)$。则$A^{\dagger}=-B+\I C$。因此

\begin{equation}
\begin{aligned}
B&=\frac{A-A^{\dagger}}{2}\\
C&= \frac{A+A^{\dagger}}{2\I}~.
\end{aligned}
\end{equation}
由于$B=-B^{\dagger},C=-C^{\dagger}$,所以假设成立,并且这种分解是唯一的。

\end{example}
\begin{example}{$\mathfrak {su}(n)$的复化}
$\mathfrak {su}(n)$的复化版本是$\mathfrak sl(n,\mathbb C)$,简单表示为$\mathfrak sl(n,\mathbb C)=\mathfrak {su}(n)+\I \mathfrak {su}(n)$。

\textbf{证明:}

$\mathfrak sl(n,\mathbb C)$为无迹的$n$阶复数矩阵,$\mathfrak {su}(n)$为无迹的$n$阶反厄米矩阵。由于无迹性不随复化而丧失,因此$\mathfrak sl(n,\mathbb C)\supseteq \mathfrak {su}(n)+\I \mathfrak {su}(n)$。剩余证明过程类似上个例子。
\end{example}
\begin{example}{洛伦兹群}

\end{example}