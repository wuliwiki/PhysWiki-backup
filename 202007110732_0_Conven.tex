% 本书符号与规范
% keys 符号|规范|单位制

这里列出本书中一些可能会产生歧义的符号以及规范, 并给出对应词条的链接
\begin{itemize}
\item 使用粗体与正体表示几何矢量\upref{GVec}, 如 $\bvec v$, 也可以用于表示列矢量或行矢量以及矩阵, 矩阵一般用大写, 如 $\mat A$.
\item 在粗体与正体矢量上方加 hat 表示单位矢量, 如 $\uvec x$
\item 也可以用狄拉克符号表示任意矢量空间的矢量, 如 $\ket{v}$, 对偶矢量如 $\bra{v}$, 内积如 $\braket{u}{v}$
\item $\subseteq$ 和 $\subset$ 表示子集, $\subsetneq$ 表示真子集
\item 如无声明词条默认使用国际单位制\upref{Consts}, 若使用其他单位制, 要在每个词条开头用脚注声明 “本词条使用 xxx 单位制”. 例如原子单位\upref{AU}, 厘米—克—秒\upref{CGS}, 或高斯单位制\upref{GaussU}.
\item 
\end{itemize}
