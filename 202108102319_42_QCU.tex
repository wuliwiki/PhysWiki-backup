% 量类和单位制
% 量|同类量|量类|单位制
\subsection{量类}
现象、物体或物质可定性区别并能定量测量的属性称为\textbf{物理量}(简称\textbf{量}).量的具体意义指大小、轻重、长短等概念,并不是所有的量都可以相互比较,比如表示长短的量和表示大小的量不能相互比较,但表示同一具体意义的量之间可以相互比较.

我们把可以相互比较的量称为\textbf{同类量},比较的结果是一个数.以粗体字母表示量,如 $\boldsymbol{A}$,细体字母表示数,如 $A$.若量 $\boldsymbol{A_1}$ 和量 $\boldsymbol{A_2}$ 可相互比较,则称量 $\boldsymbol{A_1}$ 和 $\boldsymbol{A_2}$ \textbf{同类}.对于所有量构成的集合 $M$,其上的同类关系显然为一等价关系, 由该等价关系确定的等价类称为\textbf{量类}.