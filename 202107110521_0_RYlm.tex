% 实球谐函数

\pentry{球谐函数\upref{SphHar}}

\footnote{参考 Wikipedia \href{https://en.wikipedia.org/wiki/Spherical_harmonics}{相关页面}.}球谐函数中唯一的复数因子是 $\E^{\pm\I m}$(见球谐函数表\upref{YlmTab}), 如果我们需要一套实数的球谐函数作为基底(例如展开实函数), 可以通过欧拉公式(\autoref{CExp_eq2}~\upref{CExp})把该因子变为 $\sin(m\phi)$ 和 $\cos(m\phi)$
\begin{equation}
\E^{\pm\I m \phi} = \cos(m\phi) \pm \I \sin(m\phi)
\end{equation}

定义实球谐函数为
\begin{equation}\label{RYlm_eq1}
\mathcal Y_{l,m} = \leftgroup{
&\frac{1}{\sqrt{2}}[(-1)^{m} Y_{l,m} + Y_{l,-m}] \quad &(m > 0)\\
&Y_{l,0} \qquad &(m = 0)\\
&\frac{1}{\sqrt{2} \I}[(-1)^{m} Y_{l,-m} - Y_{l,m}]  &(m < 0)
}\end{equation}
从球谐函数表\upref{YlmTab} 中可知, 上式两个方括号中第一项 $\sim \E^{\I \phi}$, 第二项 $\sim \E^{-\I \phi}$, 所以 $m > 0$ 时 $\mathcal Y_{l,0} \sim 2\cos(m\phi)$, $m < 0$ 时 $\mathcal Y_{l,0} \sim 2\sin(m\phi)$, $m = 0$ 时与 $\phi$ 无关.

由于不同的 $Y_{l,m}$ 是正交归一的, 所以\autoref{RYlm_eq1} 把两个球谐函数相加后,需要在前面乘以 $1/\sqrt{2}$ 保持正交归一:
\begin{equation}
\int \mathcal Y_{l',m'}\Cj(\uvec r) \mathcal Y_{l,m}(\uvec r) \dd{\Omega} = \delta_{l,l'}\delta_{m,m'}
\end{equation}

在球谐函数表\upref{YlmTab} 中, 要把复球谐函数变为实球谐函数, 只需要把奇数 $m$ 前面的 $\mp$ 去掉, 再把 $\E^{\pm\I \phi}$ 分别替换为 $\sqrt{2}\cos(m\phi)$ 和 $\sqrt{2}\sin(m\phi)$ 即可.

根据\autoref{RYlm_eq1}, 复球谐函数的性质都容易类推到实球谐函数, 这里不再赘述.
