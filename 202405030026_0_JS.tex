% JavaScript 入门笔记
% license Xiao
% type Note

\pentry{HTML 基础}{nod_8086}
JavaScript 常见于网页中, 一般浏览器都可以运行。 我们以 Chrome 浏览器为例演示 \verb|Hello World| 程序。

\begin{lstlisting}[language=js]
<!DOCTYPE html>
<html>
	<body>
		<div id = a></div>
		<script>
            console.log("hello world 1");
            window.alert("hello world 2");
			document.getElementById("a").innerHTML = "hello world 3";
            document.write("hello world 4");
		</script>
    <script src="mycode.js"></script>
	</body>
</html>
\end{lstlisting}
把这段程序复制到一个文本文件, 并命名为 \verb|test.html|, 用 Chrome 打开即可自动运行。 按 F12 可打开调试窗口和命令行。

\verb|<script>...</script>| 中的 4 行程序就是 JavaScript, 它们这里分别演示了用 4 种不同的方法显示 “hello world”:
\begin{enumerate}
\item 输出到命令行(需要 F12 才能看到)
\item 弹出提示窗
\item 插入到 \verb|<div>...</div>| 元素中
\item 插入到 \verb|<script>| 之前
\end{enumerate}

在用 Chrome 调试时, 可以用 F12 打开调试窗口, 可以实现逐行调试, 查看错误和警告信息和命令行输出等。

\subsection{用 VScode 调试 Javascript}
\begin{itemize}
\item 先用 VScode 打开文件夹, 按左边的 debug 图标 (小虫子), 弹出来的窗口中按齿轮按钮, 会自动创建 \verb`.vscode/launch.json` 设置文件, 不用做任何改动,现在就可以 debug 了。
\item 可以在行号左边点击添加 break point, 快捷键 \verb`F5,F10,F11` 是连续执行,下一行,步入(step in)。
\item 菜单 View 里面可以打开 Debug Console, 相当于 python REPL, 可以输入任何命令。
\end{itemize}

\subsection{常识}
\begin{itemize}
\item \textbf{ECMAScript} 是这种语言的规范或标准,而 JavaScript 是该标准的实现。JavaScript 包括所有 ECMAScript 的功能加上对 Web 开发有用的附加功能,这些附加功能不属于 ECMAScript 规范,例如 Web API
\item 其他语言也遵循 ECMAScript 标准,如 JScript(微软的实现)和 ActionScript(用于 Adobe Flash),但 JavaScript 是最突出和广泛使用的
\item 当人们谈论学习 JavaScript 时,他们通常是指 ECMAScript 的核心功能以及 JavaScript 环境(如 Web 浏览器和 Node.js)中可用的扩展功能。
\item JS 最新一次中大更新是版本 \textbf{ES6} 或者 \textbf{ECMAScript 2015},也是考虑兼容性时的标志性版本,后续的更新幅度较小。大部分现代浏览器兼容 ES6 及以上。
\item \verb`Babel` 可以把较新的 JS 版本转译(transpile)成较老的版本
\item \verb|<head>| 和 \verb|<body>| 中任意位置可以包含任意多个 \verb|<script>|
\item 用 \verb|<script src="路径/文件名.js 或者 url.js"></script>| 插入代码文件
\item 在 js 代码第一行插入 \verb|'use strict'| 后使用严格语法。 例如不声明的变量会出错, 又例如不能使用 \verb|with|。
\item 注释和 C 语言一样用 \verb|// 注释| 或 \verb|/*注释*/|
\item 每行末尾的 \verb`;` 可以省略(取决于代码风格)
\item \verb|$| 在变量/函数名中相当于一个字母, 单独一个或几个 \verb|$| 可以作为变量/函数名
\item 命令后的 \verb|;| 不是必须的, 可用于在同一行中分隔两个命令。
\item 弹出输入框如 \verb|person = prompt("提示", "默认");|
\item 按钮如 \verb|<button onclick="函数名()">按钮文字</button>|
\item 输入框如 \verb|<input id="b"><br>|
\item 获取或更改网页上的文字用 \verb|document.getElementById('b').value|
\item 当一个项目有多个文件时(例如 json 文件,多个 js 脚本),往往浏览器不允许在本地运行,需要搭建一个本地 Web Server 或者直接上传到服务器进行调试。 这叫做 \textbf{Same-Origin Policy} 以及  \textbf{CORS (Cross-Origin Resource Sharing)}。 浏览器打开本地文件用的 protocol 是 \verb`file:///`。 \textbf{严重不推荐通过降低浏览器安全选项来解决}。
\end{itemize}

\subsubsection{算符}
\begin{itemize}
\item 大部分和 C 语言一样
\item \verb|%|, \verb|++|, \verb|--|, \verb|+=|, \verb|-=|
\item \verb|&&|, \verb*||*, \verb|&|, \verb*|*, \verb|条件?语句1:语句2|
\item 幂算符是 \verb|**|, 另有 \verb|**=|
\item \verb|==| 判断值, \verb|===| 判断值和类型。 同理 \verb|!=| 和 \verb|!==|
\end{itemize}

\subsection{变量}
\begin{itemize}
\item 声明变量 \verb|let a = 1, b = 2| 或者用 \verb|const| 和 \verb|var|
\item \verb|var| 是全局变量, 且可以重新声明, \verb|let| 和 \verb|const| 是局部的
\item \verb|typeof()| 获取变量类型
\item 自带类型有 \verb|number| (不区分整数和浮点数,浮点是双精度), \verb|string| (单引号双引号通用, 没有 \verb`char` 类型)
\item 所有自定义的类型都叫做 \verb|object|
\end{itemize}

\subsubsection{字符串}
\begin{itemize}
\item \verb|str[ind]| 或 \verb|.charAt(ind)| 可以获取字符串的某个字符
\item \verb`slice(startIndex, endIndex)` 可以或取子串(不包括 \verb`endIndex`)
\item \verb`str.slice(-N);` 可以获取最后 \verb`N` 个字符。
\item 合并字符串用 \verb|字符串 + 字符串|, 也可以 \verb|字符串 + 数字|
\item \verb|num.toString()| 可以数字变字符串, \verb|num.toPrecision(有效数字)| 可以四舍五入到有效数字
\item 可以字符串变数字如 \verb|parseInt("123")|, \verb|parseFloat("12.3e2")|
\item \verb|Number(str)| 既可以转换整数也可以转换小数
\item \verb|.toUpperCase()| 和 \verb|.toLowerCase()| 将所有字母变为大小写
\end{itemize}


\subsubsection{数组}
\begin{itemize}
\item \textbf{数组(array)}如 \verb|v = [1, 'a', [2, 'b']]|, \verb|typeof| 返回 \verb|object|, 可以嵌套
\item 数组索引从 0 开始如 \verb|v[0]|, \verb|v[2][1]|
\item 数组长度为 \verb|v.length|
\item \verb|console.table()| 可以显示二维数组
\item \verb|a.push(元素)| 在最后新增元素, \verb|a.pop()| 删除最后一个元素
\item \verb|a.shift()| 删除第一个元素, \verb|a.unshift(元素)| 新增元素到第一个
\item 子数组如 \verb|v.slice(i, j)|, 复制从 \verb|i| 到 \verb|j-1| 个元素, 注意没有 view
\item \verb|a = [1,2,3]; b = a; c = [1,2,3]| 会有 \verb|a == b|, \verb|a === b|, \verb|c| 永远不等于 \verb|a| 和 \verb|b|。 改变 \verb|a| 的一个元素 \verb|b| 也会变, 
\end{itemize}

\subsection{判断循环}
\begin{itemize}
\item 判断如 \verb|if (条件) {...}|
\item 循环如 \verb|for (var i = 0; i < N; i++) {...}|
\item 
\begin{lstlisting}[language=js]
const array = [10, 20, 30, 40, 50];
for (const item of array)
  console.log(item);
\end{lstlisting}
\end{itemize}

\subsection{函数}
\begin{itemize}
\item 定义函数如 \verb|function myFunction(p1, p2) {return p1 * p2;}|, 可以在定义之前调用函数。 \verb|return| 可选
\item 不存在函数重载,如果定义的函数和之前的名字冲突,那么之后会用新的名字覆盖,不会报错。
\item 一个函数中可以调用在它之后定义的函数,但在函数所在的 scope 中,不能直接在函数定义之前调用该函数。
\item 调用函数必须要有括号, 没有括号返回函数定义。
\item 数组作为函数变量是 pass by reference
\item \verb|let myfun = function(){...}| 是一个(匿名)函数的定义, 可以作为另一个函数的变量, 相当于把某个函数名作为变量。 \verb`myfun` 的类型为 \verb|function|。
\item \verb|(function(){...})()| 叫做 \textbf{Immediately-Invoked Function Expression (IIFE)}, 可以立即执行 \verb|...| 中的代码而避免污染当前范围的变量名。
\item 如果 IIFE 有返回值,可以用 \verb`let myvar = (function(){...})()`。 注意 \verb`myvar` 是返回值不是函数句柄!
\item 普通函数中,\verb`arguments` 是一个类似于数组的对象, \verb`arguments[i]` 是每个函数参数, \verb`arguments.length` 是实际输入参数的个数。
\item \verb`(arg1, arg2) => { ... }` 或者 \verb`arg1 => {...}` 也可以定义匿名函数。只有 \verb`...` 中的 \verb`this` 会有区别。
\end{itemize}

\subsection{常用函数}
\begin{itemize}
\item \verb|var d = new Date()| 用于获取当前日期和时间 object, 成员函数有 \verb|.getDay()| (获取星期几), \verb|.getMonth()|, \verb|.getFullYear()|, \verb|.getHours()|, \verb|.getMinutes|,
\item \verb|.getSeconds()|, \verb|.getMilliseconds|, \verb|.getTime()| (返回从 1970/01/01/0:00 到当前的时长)
\end{itemize}

\subsection{对象}
\begin{itemize}
\item 对象(object)如 \verb|car = {name:"a", model:"500", color:"white"};|
\item 获取属性如 \verb|car.name|
\item \verb|console.table()| 可以列表显示属性
\end{itemize}

定义成员函数, 可以使用关键词 \verb|this|
\begin{lstlisting}[language=js]
const person = {
  firstName: "John",
  lastName : "Doe",
  id       : 5566,
  fullName : function() {
    return this.firstName + " " + this.lastName;
  }
};
\end{lstlisting}

Constructor 的例子(\href{https://www.w3schools.com/js/js_object_constructors.asp}{参考})
\begin{lstlisting}[language=js]
function Person(first, last, age, eyecolor) {
  this.firstName = first;
  this.lastName = last;
  this.age = age;
  this.eyeColor = eyecolor;
  this.name = function() {
    return this.firstName + " " + this.lastName;
  };
}
\end{lstlisting}

\subsection{模块}
\begin{itemize}
\item 注意模块只能用于 \verb|http://| 和 \verb|https://| 协议, 不能用 \verb|file://| 协议。
\item 从电脑上打开的 html 文件就是 file 协议, 如 \verb|file:///C:/路径/diffus.html| 但浏览器上可能不会显示 \verb|file://|。 注意第三个 \verb|/| 表示电脑的根目录。
\item 在 html 中加载模块 \verb|<script type="module"> import message from "./message.js"; </script>|
\item \textbf{named export} 如 \verb|const name = "Jesse"; const age = 40; export {name, age};|
\item 加载如 \verb|import { name, age } from "路径/文件.js";|
\item \textbf{default export} 如 \verb|export default message;|。 一个模块文件中之能有一个 default export。
\end{itemize}

\subsection{容器}
\subsubsection{set}
\begin{itemize}
\item \verb`new Set();`, \verb`new Set([1, 2, 3, 4]);`
\item \verb`mySet.clear();`
\item \verb`mySet.add(3);`
\item \verb`mySet.has(3);`
\end{itemize}


\subsection{math.js}
\href{https://github.com/josdejong/mathjs}{github 仓库}

\verb|math.js| 和内建的 Math 库兼容(可以一起用)。 注意这不是一个模块, 不需要 import, 可以用 \verb|file://| 协议。

在代码前面插入 \verb|<script src="math.js"></script>|, 也可以是 url 如 \verb|https://cdnjs.cloudflare.com/ajax/libs/mathjs/10.0.2/math.js|

\subsubsection{常用}
\begin{itemize}
\item 所有的对象和函数前面要加 \verb|math.|, 也可以在非 \verb|strict mode| 下把代码放到 \verb|with (math) {...}| 中
\item 常数 \verb|pi|, \verb|e|, 
\item 常用函数 \verb|round|,\verb|ceil|, \verb|floor|, \verb|exp|, \verb|sin|, \verb|cos|, \verb|sqrt|, \verb|log|, \verb|atan2|, \verb|pow|
\item \verb|random()| 生成 0 到 1 的随机数, 或者 \verb|random(min, max)|, \verb|random([Nr,Nc])| 生成 0 到 1 的随机矩阵, \verb|random([Nr, Nc], min, max)| 等。
\end{itemize}

\subsubsection{复数}
\begin{itemize}
\item JS 不支持算符重载, 所以复数加减乘除都要用函数实现
\item \verb|let c = math.complex(2, 3)| 生成复数
\item \verb|c.re| 和 \verb|c.im| 获取实部和虚部, 可以赋值 \verb|c.re = 5|。 也可以用 \verb|math.re(c)| 和 \verb|math.im(c)|, 不能赋值。
\item 共轭 \verb|c.conjugate()| 或者 \verb|math.conj(c)|, 绝对值 \verb|math.abs(c)| 或 \verb|c.abs()|, 辐角 \verb|math.arg(c)| 或者 \verb|c.arg()|, 加法 \verb|math.add(a, b)|, 乘法 \verb|math.multiply(a, b)|, 除法 \verb|math.divide(a, b)|, 根号 \verb|math.sqrt(-4)|, 倒数 \verb|a.inverse()|。
\end{itemize}

\subsubsection{线性代数}
\begin{itemize}
\item 常矩阵 \verb|zeros([Nc, Nr])|, \verb|ones([Nc, Nr])|
\item \verb|range(2, 5)|
\item \verb|size(矩阵)| 获取矩阵尺寸, 又如 \verb|size(矩阵)[1]| 获取列数
\item \verb|a.resize([2, 3])| 变形为零矩阵
\item 行列式 \verb|det(矩阵)|
\item 矩阵乘法 \verb|multiply(矩阵1, 矩阵2)|
\item \verb|lusolve(A, b)| 解线性方程组
\end{itemize}

\subsection{minify 和 concatenation}
\begin{itemize}
\item 很多网站的 js 如果直接打开会发现类似于密密麻麻的乱码。 这是因为 minify 和 concatenation。
\item minify 就是把多余的空格,回车,注释等都删掉,保持功能不变。 concatenation 是把多个 js 文件合并为一个。 对于闭源项目,这也可以用于保护源码。
\item 一个流行 minify 工具是 Terser (UglifyJS 的 fork,更好支持 ES6+)
\item Webpack 功能更多,包括 minify,concatenation,等等。
\item 在浏览器中调试时,调试界面会自动把 minify 的代码换行和缩进,但这样行号就改变了
\item 要把 js 代码直接自动格式化(换行和缩进),可以用一个在线工具 \href{https://prettier.io/playground/}{Prettier},也可以用 VScode 右键菜单的 format 功能。 保存文件后,行号就稳定了。但简化的文件名和变量名是不可能找回来的。
\item 一些开源项目也会同时提供一个 source map, 文件名例如 \verb`xxx.js.map`。浏览器的开发者工具可以根据该文件自动把 minify 的 js 显示为原来的源码。 要使用 source map, 一般 js 文件的最后会有一行类似于 \verb`//# sourceMappingURL=app.min.js.map` 的注释,指定 source map 的位置,浏览器会根据这个读取 map 文件。 该文件把 minified 
\item 许多大型的 js 项目会提供用户定制的方法,例如你只需要使用 \href{https://mathjs.org/docs/custom_bundling.html}{math.js 的一小部分功能},或者 \href{https://prismjs.com/download.html#themes=prism&languages=markup+css+clike+javascript}{prism.js 的个别语言},就可以获得定制化的 minified 文件,比完整文件要小很多。
\item 项目文件夹中, \verb`src/` 文件夹一般是源代码, \verb`dist/` (distribution)文件夹一般是 minify 的生产环境代码。
\end{itemize}
