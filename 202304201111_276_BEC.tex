% 玻色爱因斯坦凝聚
% keys 玻色气体|玻色爱因斯坦凝聚|玻色分布


\begin{issues}
\issueDraft
\end{issues}

\pentry{玻尔兹曼分布(统计力学)\upref{MBsta}}

考虑由 $N$ 个\textbf{全同\footnote{需要考虑全同粒子假设,$\Omega=\Omega_{M.B.}/N!$。}、近独立\upref{depsys}\footnote{近独立\upref{depsys}的意思是,系统的总能量近似等于所有单粒子能量的总和,即忽略粒子相互作用势。这是一个极粗糙的近似,但我们可以以此简化计算得到系统可能的一些性质。}}的玻色子组成的系统。根据玻色分布,处在能级 $\epsilon_l$ 上的粒子数为
\begin{equation}
a_l=\frac{\omega_l}{e^{\frac{\epsilon_l-\mu}{kT}}-1}~.
\end{equation}
处于任意能级上的粒子数不能为负的,即 $a_l\ge 0$,这要求对任意能级 $\epsilon_l$ 都 $>\mu$。所以化学势 $\mu<0$。化学势可以由下式确定:
\begin{equation}\label{eq_BEC_1}
\sum_la_l=\sum_l \frac{\omega_l}{e^{\frac{\epsilon_l-\mu}{kT}}-1} =N
\end{equation}
可以看出,化学势随温度的降低而升高。当温度降到某一临界温度 $T_C$ 时,$\mu$ 将趋于 $-0$,大量粒子将聚集在最低的单粒子态(即基态)上,直到绝对零度时,所有粒子都会凝聚到基态上。这种无相互作用系统中,宏观数量的玻色子凝聚到能量最低的单粒子态上的现象就被称为\textbf{玻色爱因斯坦凝聚}。

下面计算临界温度 $T_C$。将\autoref{eq_BEC_1} 中的求和用积分代替,并利用\autoref{eq_IdED1_2}~\upref{IdED1}化简。可以得到
\begin{equation}
\frac{2\pi V}{h^3}(2m)^{3/2}\int_0^\infty \frac{\epsilon^{1/2}\dd \epsilon}{e^{\frac{\epsilon-\mu}{kT}}-1}=N
\end{equation}
当 $T$ 降到 $T_C$ 时,$\mu$ 趋于 $-0$。所以临界温度 $T_C$ 由下式给出
\begin{equation}\label{eq_BEC_2}
\begin{aligned}
&\frac{2\pi V}{h^3}(2m)^{3/2}\int_0^\infty \frac{\epsilon^{1/2}\dd \epsilon}{e^{\frac{\epsilon}{kT_C}}-1}=N\\
&\Rightarrow \frac{2\pi}{h^3}(2mkT_C)^{3/2}\int_0^\infty \frac{x^{1/2}\dd x}{e^x-1}=n\\
&\Rightarrow T_C=\frac{h^2}{2\pi mk}\qty(\frac{n}{2.612})^{2/3}
\end{aligned}
\end{equation}

例如将液 $^4\rm{He}$ 的数据代入,可以得到 $T_C=3.13\rm{K}$。  $^4\rm{He}$ 在 $T_\lambda=2.17\rm{K}$ 时发生一个相变,称为 $\lambda$ 相变。温度高于 $T_\lambda$ 时是正常液体,称为 $\rm{He}$Ⅰ.低于 $T_\lambda$ 时液 $^4\rm{He}$ 具有超流性,称为 $\rm{He}$Ⅱ。由于临界温度 $T_C$ 与 $T_\lambda$ 非常接近。因此发现 $^4\rm{He}$ 的超流性质后,伦敦曾提出 $^4\rm{He}$ 的 $\lambda$ 相变可能是一种玻色凝聚。但后来人们发现并不是这样,液氦的超流性是由于氦原子之间的相互作用引起的。随后,人们一直试图寻找一种真正玻色爱因斯坦·凝聚的例子。要实现真正意义上的玻色-爱因斯坦凝聚,就必须使得气体足够稀薄,以至于其原子或分子间的相互作用可以忽略。这实际上要求极低的温度。因此,只有近年来低温技术(激光冷却和磁约束的技术)得到充分发展后,才可能在实验上实现。当然,与此同时也产生了许多新的诺贝尔奖获得者。

当 $T<T_C$ 时,为了计算在最低能级 $\epsilon=0$ 的粒子数密度,不能再采用 \autoref{eq_BEC_2} 的积分形式,必须将 $\epsilon=0$ 部分单独处理。$\epsilon>0$ 的部分仍可以用积分近似。
\begin{equation}\label{eq_BEC_3}
\begin{aligned}
&n_0(T)+\frac{2\pi}{h^3}(2mkT)^{3/2}\int_0^\infty \frac{x^{1/2}\dd x}{e^x-1}=n\\
&\Rightarrow n_0(T)+n\qty(\frac{T}{T_C})^{3/2}=n\\
&\Rightarrow n_0(T)=n\qty[1-\qty(\frac{T}{T_C})^{3/2}]
\end{aligned}
\end{equation}

\autoref{eq_BEC_3} 表明,在 $T<T_C$ 时会有宏观量级的粒子在能级 $\epsilon=0$ 凝聚。这就是\textbf{玻色·爱因斯坦凝聚},简称玻色凝聚。$T_C$ 称为凝聚温度。凝聚在 $\epsilon_0$ 的粒子集合称为玻色凝聚体。 凝聚体不仅能量、动量为 $0$,由于微观状态完全确定,其熵也为 $0$。凝聚体对压强也没有贡献。

根据玻色分布可以计算玻色气体的内能,从而计算热容。
\begin{equation}
\begin{aligned}
U&=\frac{2\pi V}{h^3}(2m)^{3/2}\int_0^\infty \frac{\epsilon^{3/2}\dd \epsilon}{e^{\epsilon/kT}-1}\\
&=\frac{2\pi V}{h^3}(2mkT)^{3/2}kT\int_0^\infty \frac{x^{3/2}\dd x}{e^{x}-1}\\&=\frac{2\pi V}{h^3}(2mkT)^{3/2}kT\cdot \frac{3\sqrt{\pi}}{4}\cdot 1.341
\\
&=0.770NkT\qty(\frac{T}{T_C})^{3/2}
\end{aligned}
\end{equation}
所以定容热容为
\begin{equation}
C_V=\qty(\frac{\partial U}{\partial T})_V=\frac{5}{2}\frac{U}{T}=1.925Nk\qty(\frac{T}{T_C})^{3/2}~.
\end{equation}

于是 $T<T_C$ 时理想玻色气体的 $C_V$ 与 $T^{3/2}$ 成正比。到 $T=T_C$ 时热容有一处尖峰,$C_V$ 达到极大值;高温时应趋向于经典值 $\frac{3}{2}Nk$。