% 计算机史(综述)
% license CCBYSA3
% type Wiki

本文根据 CC-BY-SA 协议转载翻译自维基百科\href{https://en.wikipedia.org/wiki/History_of_computing}{相关文章}。

计算历史的时间比计算硬件和现代计算技术的历史更为悠久,其中包括了为笔和纸或黑板和粉笔而设计的方法的历史,无论是否借助表格的辅助。
\subsection{具体设备}  
数字计算与数字的表示密切相关。[1] 但在像数字这样的抽象概念出现之前,已经有了为文明服务的数学概念。这些概念隐含于具体的实践中,例如:
\begin{itemize}
\item 一一对应,[2] 一种用于计数物品数量的规则,例如在计数棒上,最终被抽象为数字。
\item 与标准比较,[3] 一种假定测量可重复性的方法,例如,硬币的数量。
\item 3-4-5 直角三角形是一种确保直角的设备,使用带有 12 个均匀间距结的绳子,例如。[4][未验证]
\end{itemize}
\subsection{数字}  
最终,数字的概念变得具体且熟悉,足以用于计数,有时还伴随着歌谣式的记忆法来教别人记住数列。所有已知的人类语言,除了皮拉哈语(Piraha),都有表示至少“一个”和“两个”的词汇,甚至一些动物,如黑鸟,也能够区分出令人惊讶的物品数量。[5]

数字系统和数学符号的进步最终导致了数学运算的发现,例如加法、减法、乘法、除法、平方、平方根等。最终,这些运算被形式化,关于这些运算的概念也得到了足够清晰的理解,可以被正式表述,甚至被证明。例如,可以参考欧几里得算法,用于找出两个数的最大公约数。

到了中世纪晚期,位值的印度-阿拉伯数字系统传入了欧洲,这使得数字的系统化计算成为可能。在这一时期,计算在纸上表示出来,允许数学表达式的计算,并且能够列出数学函数,如平方根、常用对数(用于乘法和除法),以及三角函数。到艾萨克·牛顿的研究时期,纸张或羊皮纸已成为重要的计算资源,甚至在我们今天的时代,像恩里科·费米这样的研究人员会在随机的纸片上进行计算,来满足他们对方程的好奇心。[6] 甚至在可编程计算器的时代,理查德·费曼也毫不犹豫地手动计算任何超出计算器内存限制的步骤,只为了解答案;到1976年,费曼购买了一台HP-25计算器,具有49步程序容量;如果微分方程需要超过49步来解答,他就继续手工计算。[7]
\subsection{早期计算}  
数学陈述不必仅限于抽象;当一个陈述可以用实际数字加以说明时,这些数字就可以被传递,从而形成一个社区。这使得可重复、可验证的陈述成为数学和科学的标志。这类陈述存在了数千年,并且跨越了多个文明,如下所示:

已知最早用于计算的工具是苏美尔的算盘,据信它是在公元前2700年至2300年左右的巴比伦发明的。其最初的使用方式是通过在沙地上划线并使用小石子。[需要引用]

公元前1050至771年左右,古代中国发明了指向南方的战车。这是已知的第一个使用差动齿轮的齿轮机制,后来被用于模拟计算机。中国人还发明了一种更为复杂的算盘,大约在公元前2世纪左右,称为中国算盘。[需要引用]

公元前3世纪,阿基米德使用平衡的机械原理(参见《阿基米德的手稿》 § 机械定理法)来解决数学问题,例如计算宇宙中的沙粒数量(《沙粒计算器》),这也需要一种递归的数字表示法(例如,万万)。  

安提基特拉机制被认为是已知最早的齿轮计算设备。它被设计用来计算天文位置。它于1901年在希腊安提基特拉岛附近的沉船中被发现,估计制作于公元前100年左右。[8]

根据西蒙·辛格的说法,穆斯林数学家在密码学方面也做出了重要的进展,如阿尔金都斯(Alkindus)发展了密码分析和频率分析。[9][10] 穆斯林工程师还发明了可编程机器,如班努·穆萨兄弟的自动笛子演奏机。[11]

在中世纪,几位欧洲哲学家试图制造模拟计算设备。在阿拉伯人和经院哲学的影响下,马略卡哲学家拉蒙·吕尔(Ramon Llull,1232–1315)将他的一生大部分时间都投入到定义和设计几种逻辑机器,这些机器通过结合简单且无可否认的哲学真理,可以产生所有可能的知识。这些机器实际上并未建造出来,因为它们更多的是一种思想实验,旨在以系统化的方式产生新知识;尽管它们可以进行简单的逻辑操作,但仍需要人工来解读结果。此外,它们缺乏多功能的架构,每个机器只能用于非常具体的目的。尽管如此,吕尔的工作对戈特弗里德·莱布尼茨(18世纪初)产生了强烈影响,后者进一步发展了他的思想,并根据这些思想制造了几种计算工具。

这一早期机械计算时代的顶峰可以从查尔斯·巴贝奇的差分机及其后继的解析机中看到。巴贝奇虽然未能完成这两台机器的建造,但在2002年,伦敦科学博物馆的多伦·斯韦德(Doron Swade)和其他工程师们使用1840年代可得的材料,完成了巴贝奇的差分机。[12] 通过遵循巴贝奇的详细设计,他们成功地建造了一个能够运行的机器,使得历史学家可以有信心地说,如果巴贝奇能够完成他的差分机,它本应该是可以工作的。[13] 更加先进的解析机结合了他之前工作的概念以及其他人的研究成果,创造了一种设备,如果按照设计建造,便会具备许多现代电子计算机的特性,例如等同于RAM的内部“暂存记忆”、多种输出形式,包括铃声、图形绘制仪和简单打印机,及可编程的输入输出“硬”记忆(即穿孔卡片),它不仅能够读取信息,还可以修改这些信息。巴贝奇的设备超越前人之处在于,每个组件都是独立于其他部分的,就像现代电子计算机的各个组件一样。这是一种思维上的根本转变;之前的计算设备只能服务于单一的目的,且在解决新问题时,最好的做法是拆解并重新配置机器。巴贝奇的设备则可以通过输入新数据重新编程来解决新问题,并且在同一指令序列中对之前的计算进行操作。阿达·洛夫莱斯将这一概念进一步发展,她为解析机编写了一个程序来计算伯努利数,这一复杂的计算需要递归算法。这被认为是第一个真正的计算机程序的例子,一个指令序列,在程序运行之前并不完全知道操作的数据。

在巴贝奇之后,虽然并不知晓他早期的工作,珀西·拉德盖特(Percy Ludgate)于1909年发布了历史上仅有的两个机械解析机设计中的第二个。[14][15] 另外两位发明家,莱昂纳多·托雷斯·凯维多(Leonardo Torres Quevedo)[17] 和范尼瓦尔·布什(Vannevar Bush)[18],也基于巴贝奇的工作进行了后续研究。在他的《自动化论文集》(Essays on Automatics,1914年)中,托雷斯提出了一种机电计算机的设计,并引入了浮动点算术的概念。[19][20] 1920年,为了庆祝算术仪发明100周年,托雷斯在巴黎展示了机电算术仪,它是一种算术单元,可以连接远程打字机,用户可以输入命令,结果会自动打印出来。[21][22] 布什的论文《仪器分析》(Instrumental Analysis,1936年)讨论了如何利用现有的IBM穿孔卡机器来实现巴贝奇的设计。同年,他开始了快速算术机器项目,研究构建电子数字计算机的问题。