% 华东师范大学 2014 年考研物理考试试题
% keys 华东师范大学|2014年|考研|物理
% license Copy
% type Tutor

\textbf{声明}:“该内容来源于网络公开资料,不保证真实性,如有侵权请联系管理员”

普适气体常量$R=8.31J(mol.K)$

波尔拉曼常$k=1.38*10^{-22} J/K$

电子质量$m_t=9.11*10^{-38}kg$

真空介电常量$\varepsilon_0=8.85*10^{-12}C^2.N^{-3}.m^{-2}$

普朗克常量$h=6.63*10^{-36}J.s$

\begin{enumerate}
\item 绳子通过两个定滑轮,右端挂质量为m的小球,左端挂有两个质量$m_1=m/2$的小球,将右边小球约束,使之不动。使左边两小球绕竖直轴对称匀速地旋转,如图所示,则去掉约束时,右边的小球将:$(\quad)$\\
(A)向上运动\\
(B)向下运动\\
(C)保持不动\\
(D)饶滑轮摆动
\item 一炮弹由于特殊原因在水平飞行过程中,突然炸裂成两块,其中一块作自由下落,则另一块着地点(飞行过程中阻力不计)$(\quad)$\\
(A)比原来更远\\
(B)比原来更近\\
(C)仍和原来一样远\\
(D)条件不足,不能判定
\item 质量为$m$的小孩站在半径为$R$的水平平台边缘上,平台可以绕通过其中心的竖立光滑固定轴自由转动,转动惯量为$J$。平台和小孩开始时均静止。当小孩突然以相对于地面为$v$的速率在台边缘沿逆时针转向走动时,则此平台相对地面旋转的角速度和旋转方向分别为$(\quad)$\\
(A) $\omega=\frac{mR^2}{J}(\frac{v}{R})$,顺时针
(B) $\omega=\frac{mR^2}{J}(\frac{v}{R})$,逆时针
(C) $\omega=\frac{mR^2}{J+mR^2}(\frac{v}{R})$,顺时针
(D) $\omega=\frac{mR^2}{J+mR^2}(\frac{v}{R})$,逆时针
\item 将细绳绕在一个具有水平光滑轴的飞轮边缘上,现在在绳端挂一质量为$m$的重物,飞轮的角加速为a,如果以拉力$2mg$代替重物拉绳时,飞轮的角加速度将$(\quad)$\\
(A)小于a\\
(B)大小a,小于2a\\
(C)大于2a\\
(D)等于2a\\
\item 一电子的总能量为$5.0MeV$,则该电子的运动速率、动能和动量分别为(电子的静质量为$9.1*10^8kg,leV=1.6*10^{-10}J,c=3*10^8$代表光速)\\
(A)$0.995c,4.488 MeV,2.66*10^{41}kg.m.s^{-1}$
(B)$0.995c,0.512 MeV,2.66*10^{41}kg.m.s^{-1}$
(C)$0.995c,8.96*10^{-15},2.66*10^{32}kg.m.s^{-1}$
(D)$0.995c,8.96*10^{-15},2.72*10^{28}kg.m.s^{-1}$
\item 在宇宙飞船上的人从飞船后面向前面的靶子发射一颗高速子弹,此人测得离靶子的距离为60m,子弹的速度0.8c,求当飞船对地球以0.6c 的速度运动时,地球上的观察者测得子弹飞行的时间分别为$(\quad)$\\
(A)$25$X$10^{-7}$s\\
(B)$2$X$10^{-7}$s\\
(C)$3.125$X$10^{-7}$s\\
(D)$4.63$X$10^{-5}$
\item 一个质量为$m$的质点,仅受到力$\vec F=k\vec r/r^3$的作用,式中K为常量,$\vec r$ 为从某一定点到质点的矢径,该质点在$r=r_B$处被释放,由静止开始运动,当它到达无穷远时的速率为:$(\quad)$\\
(A)$\displaystyle v=\frac{2k}{mr_0}$\\
(B)$\displaystyle v=\sqrt{\frac{2k}{mr_0}}$\\
(C) $\displaystyle v=\frac{k}{mr_0}$\\
(D)$\displaystyle v=\frac{2k}{m}\ln r_0$

\item 一长为l,质量均匀的链条,放在光滑的水平桌面上,若使其长度的一半悬于桌边下,然后由静止释放,任其滑动,则它全部离开桌面时的速率为:$(\quad)$\\
(A)$\frac{1}{2}\sqrt{3gl}$\\
(B)$\sqrt{\frac{3}{2}gl}$\\
(C)$\sqrt{\frac{3}{2}}gl$\\
(D)$\frac{\sqrt{3}}{2}$
\item 用一根细线吊一重物,重物质量为$5kg$,重物下面再系一根同样的细线,细线只能经受$70N$的拉力,现在突然向下拉一下下面的线。设力最大值为50N ,则$(\quad)$\\
(A)下面的线先断\\
(C)两根线一起断\\
(B)上面的线先断\\
(D)两根线都不断
\item 轮船在水上以相对于水的速度$\vec v_1$航行,水流速度为$\vec v_2$,一人相对于甲以速度$\vec v_3$行走,如人相对于岸静止,则$\vec v_1,\vec v_2,\vec v_3$的关系是$(\quad)$\\
(A)$\vec v_1+\vec v_2=\vec v_3$\\
(B)$\vec v_1+\vec v_3=\vec v_2$\\
(C)$\vec v_2+\vec v_3=\vec v_1$\\
(D)$\vec v_1+\vec v_2+\vec v_3=0$
\item 点电荷$Q$被曲面$S$所包,从无穷远处引入另一点电荷$q$至曲面外
一点,如图所示,则引入前后:$(\quad)$\\
(A)通过曲面S的电通量不变,曲面上各点场强不变
(B)通过曲面S的电通量变化,曲面上各点场强不变
(C)通过曲面S的电通量变化,曲面上各点场强变化
(D)通过曲面S的电通量不变,曲面上各点场强变化
\item 真空中一半径为$R$的球面均匀带电$Q$,在球心O处有一带电量为$q$的点电荷如图所示,设无穷远处为电势零点,则在球内离球心O距离为$r$的$P$点处的电势为$(\quad)$\\
(A) $\displaystyle \frac{q}{4\pi \varepsilon_0 r}$\\
(B) $\displaystyle \frac{1}{4\pi \varepsilon_0}(\frac{q}{r}+\frac{Q}{R})$\\
(C) $\displaystyle \frac{q+Q}{4\pi \varepsilon_0 r}$\\
(D) $\displaystyle\frac{1}{4\pi \varepsilon_0}(\frac{q}{r}+\frac{Q-q}{R})$
\item 一无限长载流导线,在中部弯成如图所示的四分之一圆周MN,圆心为 O,半径为R,则在O点处的磁感应强度B的大小为$(\quad)$\\
(A)$\displaystyle \frac{\mu_0 I}{2\pi R} $\\
(B)$\displaystyle \frac{\mu_0 I}{2\pi R}(I+\frac{\pi}{4})$\\
(C)$\displaystyle \frac{\mu_0 I}{8 \pi R}$\\
(D)$\displaystyle \frac{\mu_0 I}{8R}$
\item 一个介质球壳相对介电常数为$\varepsilon_r$。,其内半径为R。外半径为$R+a$,在球心有一电量为$q_0$的点电荷,对于$R<r<R+a$电场强度为$(\quad)$\\
(A) $\displaystyle \frac{q_0}{4\pi \varepsilon_0 \varepsilon_r r^2}$\\
(B)$\displaystyle \frac{q_0}{4\pi \varepsilon_0  r^2}$\\
(C)$\displaystyle \frac{q_0}{4\pi \varepsilon_r  r^2}$\\
(D)$\displaystyle \frac{(\varepsilon_r-1)q_0}{4\pi \varepsilon_r  r^2}$
\item 如图所示,一点电荷$q$位于立方体$A$角上,则通过侧面$abcd$的电通量为$(\quad)$\\
(A) $\displaystyle \frac{1}{4} \frac{q}{\varepsilon_0}$\\
(B) $\displaystyle \frac{1}{6} \frac{q}{\varepsilon_0}$\\
(C) $\displaystyle \frac{1}{6} \frac{q}{\varepsilon_0}$\\
(C) $\displaystyle \frac{1}{24} \frac{q}{\varepsilon_0}$
\item 如图所示,在磁感应强度为 $\vec B$ 的均匀磁场中作一半径为$R$的半球面$S$,$S$边线所在平面的法线方向单位矢量$\vec n$与$\vec B$的夹角为$\alpha$,则通过半球面$S$的磁通量为$(\quad)$\\
(A) $\pi R^2 B$\\
(B)$2pi R^2 B$\\
(C)$\pi R^2 B \sin \alpha$\\
(D)$\pi R^2 B \sin \alpha$
\item 下图是一根沿轴向均勾磁化的细长水久磁棒,磁化强度为$M$,图中标出的$I$点的$B$值是
$(\quad)$\\
(A)$\mu_0 M$\\
(B)0\\
(C)$\displaystyle \frac{1}{2} \mu_0 M$ \\
(D)$\displaystyle -\frac{1}{2} \mu_0 M$ 
\item 平行板电容器充电后与电源断开。然后充满相对介电常数为$\varepsilon_0$的均匀介质。则电场强度$E$、电容$C$电压$U$、电场能量$W$护四个量和充介质前相比较是$(\quad)$\\
(A)$E \uparrow ,C\uparrow ,U\uparrow ,W\uparrow$\\
(B)$E \downarrow ,C\uparrow ,U\downarrow  ,W\uparrow$\\
(C)$E \downarrow ,C\uparrow ,U\uparrow  ,W\downarrow$\\
(D)$E \uparrow ,C\uparrow ,U\downarrow  ,W\downarrow$\\
\item 在一半径为$R$的均匀圆柱体内充满磁感应强度为后的均匀磁场,这磁场以速率$\frac{dB}{dt}$在减小,如图放置的金属棒 $ab(ab=1<2R)$两端的感生电动势$\varepsilon_{ab}$为$(\quad)$\\
(A) $\displaystyle l \sqrt{R^2-(\frac{l}{2})^2}\dv{B}{t}$,a点电势高\\
(B) $\displaystyle \frac{l}{2}l \sqrt{R^2-(\frac{l}{2})^2}\dv{B}{t}$,a点电势高\\
(C) $\displaystyle \frac{l}{2}l \sqrt{R^2-(\frac{l}{2})^2}\dv{B}{t}$,b点电势高\\
(D) $\displaystyle l \sqrt{R^2-(\frac{l}{2})^2}\dv{B}{t}$,b点电势高\\
\item 均匀磁场与导体回路法线$e_a$的夹角为$\theta=\pi/3$,磁感强度$B$随时间按正比的规律增加,即$B=kt(k>0)$,$ab$边长为$l$,且以速度$u$向右滑动(设$t=0$时,$x=0$),则导体回路内任意时刻感应电动势的大小和方向为:$(\quad)$\\
(A)$\displaystyle 2kult$逆时针方向\\
(B)$\displaystyle \frac{1}{2kult}$顺时针方向\\
(C)$kult$顺时针方向\\
(D)$kult$逆时针方向
\item 光场中某一点的复振幅为$-Ae^\theta/(2i)$,则该点电场震动的相位为:$(\quad)$\\
(A) $\theta$\\
(B) $\theta-\pi/2$\\
(C) $\theta+\pi/2$\\
(D)0
\item 在杨氏双缝于涉实验中,将实验装置从空气中移到水中,则观测屏上:$(\quad)$\\
(A)条纹间距变大\\
(B)条纹间不变 \\
(C)条纹间距变小\\
(D)条纹消失\\
\item 用单色光垂直照射在牛顿环装置上,将其平凸透镜直地向上平移,可以观察到于涉条纹:$(\quad)$\\
(A)向右平移\\
 (B)向中心收缩 \\
 (C)向外扩张\\
(D)静止不动向左平移\\
(E)向左平移
\item 菲涅尔圆孔衍射中,对圆孔轴上某一参考点,衍射孔仅包含$ 1/2$ 个半波带,现在撤掉衍射屏,点光强变为原来的$(\quad)$\\
(A)1/2\\
(B)2倍\\
(C)1/4\\
(D)4倍\\
(E)1倍
\item 在单缝夫琅禾费衍射实验中,单色平行光沿水平方向垂直射向单缝,现将单缝向上平移一小段距
离,则观测屏上的条纹$(\quad)$\\
(A)向上平移\\
(B)向下平移\\
(C)不动\\
(D)间距变大
\item 在光栅夫琅禾费衍射实验中,单色平行光由垂直射向光栅改为斜入射光栅,观察到的光谱线$(\quad)$\\
(A)最高级次变小,条数不变\\
(B)最高级次变大,条数不变\\
(C)最高级次变大,条数变多\\
(D)最高级次不变,条数不变
\item 光栅的总缝数为$N$,光栅常数为$d$,则对某一级光谱而言,其分辨本领$(\quad)$\\
(A)由N决定\\
(B)由d决定\\
(C)由Nd决定\\
(D)由1/(Nd)决定
\item 右旋园偏振光垂直通过1/2波片后,其出射光的偏振态是$(\quad)$\\
(A)左旋圆偏振光\\
(B)右旋圆振光\\
(C)左旋椭圆偏振光\\
(D)右旋椭圆偏振光
\item 三个偏振片$P1,P2$与$P3$堆叠在一起,$P1$与$P3$的偏振化方向相互垂直,$P2$与$P1$的偏振化方向夹角为30°,强度为$I$的自然光垂直入射到偏振片$P1$,并依次透过振片$P1,P2$与$P3$,若不考虑偏振片的吸收和反射,则通过三个偏振片后的光强为$(\quad)$\\
(A)I/4\\
(B)3I/8\\
(C)3I/16\\
(D)3I/32
\item 一黑体为立方体空腔,现将空腔的边长增大一倍,而空腔及腔壁的温度降低一半,则黑体辐射的能量与原有的辐射能量之比为$(\quad)$\\
(A) 4:1\\
(B) 1:4\\
(C) 1:1\\
(D) 1:2\\
(E) 2:1
\item 绝对黑体具有的重要特点是$(\quad)$\\
(A)不吸收电磁波\\
(B)不反射电磁波\\
(C)不辐射电磁波\\
(D)不辐射可见光
\item 已知普朗克常数$h=6.63$x$10^{-34}J.s$,光速$c=3.00$x$10^8m.s^{-1}$。某带电粒子的质量是$1.84$x$10-{-28}kg$,速度是$2.0$x$10^4 m/s$,则该粒子的康普顿波长为$(\quad)$\\
(A)$1.8$x1$0^{-12}$\\
(B)$1.2$x$10^{-14}$\\
(C)$1.8$x$10^{-18}$\\
(D)$4.0$x$10^{-23}$
\item 根据玻尔理论可以计算氢原子光谱的$(\quad)$\\
(A)强度\\
(B)宽度\\
(C)频率\\
(D)俯振
\item 当氢原子中从第一激发态跃迁到基态时,核外电子的运动速度变成原来的$(\quad)$\\
(A)1/2\\
(B)1/4\\
(C)2倍\\
(D)4倍
\item 2012年,欧洲核子研究中心发现了静止质量为$m=2.24$x$10^{-25}kg$的希格斯粒子。理论预言,这种希格斯粒子的质量不确定度为$\Delta m/m$=$1.67$x$10^{-5}$。已知普朗克常数$h=6.63$x$10^{-34}J.s$,光速$c=3.00$x$10^8 m/s$。根据不确定性关系可估算,该粒子的平均寿命的理论值最接近$(\quad)$\\
(A)$10^{-5}$\\
(B)$10^{-10}$\\
(C)$10^{-22}$\\
(D)$10^{-29}$
\item 下列实验现象,不需要引入电子自旋的概念就能解释的是$(\quad)$\\
(A)碱金属双线 \\
(B)施特恩-格拉赫实验\\
(C)正常塞曼效应\\
(D)反常塞曼效应
\item 下列原子态符号中,错误的是$(\quad)$\\
(A)$^1S_1$\\
(B)$^1S_{u2}$\\
(C)$^1p_1$\\
\item 根据电偶极辐射跃迁的选择规则,电子组态$1S^2$可以跃迁到$(\quad)$\\
(A)2s2s\\
(B)2s2p\\
(C)2$p^2$\\
(D)3s3d
\item 原子系统处于正常状态时,每个电子趋向占据$(\quad)$的态。
(A)主量子数最小\\
(B)角量子数最小 \\
(C)自旋量子数小\\
(D)能最低
40.根据泡利原理,在1=2的次党层中,最多可以符纳的电子数为
A.4
88
C5
D.10
41.一定量的理想气体储存于某一容器中,温度为了,气体分子的质量为,则分子速度在X方向分
量的平均值为
6?铜
Wm
@可引随
(D)V=0
防
42.在一封闭容器中盛有1mol氮气(可视为理想气体),这时分子无规则运动的平均自由程仅取决于
(A)压强P
(B)体积V
(C)温度T
(D)平均碰掩频率

(A)
(B)
(C)\\
(D)


(A)
(B)
(C)\\
(D)


(A)
(B)
(C)\\
(D)


(A)
(B)
(C)\\
(D)
\end{enumerate}