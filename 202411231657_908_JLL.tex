% 伽利略·伽利莱(综述)
% license CCBYSA3
% type Wiki

本文根据 CC-BY-SA 协议转载翻译自维基百科\href{https://en.wikipedia.org/wiki/Galileo_Galilei}{相关文章}。

\begin{figure}[ht]
\centering
\includegraphics[width=6cm]{./figures/f16d6423773a5687.png}
\caption{约1640年的肖像画} \label{fig_JLL_1}
\end{figure}
伽利略·迪·文琴佐·博纳尤蒂·德·伽利略(\textbf{Galileo di Vincenzo Bonaiuti de' Galilei},1564年2月15日-1642年1月8日),通常简称为伽利略·伽利莱(\textbf{Galileo Galilei},/ˌɡælɪˈleɪoʊ ˌɡælɪˈleɪ/,美国英语中也读作 /ˌɡælɪˈliːoʊ -/;意大利语:[ɡaliˈlɛːo ɡaliˈlɛːi]),或单名称为伽利略(\textbf{Galileo}),是一位佛罗伦萨的天文学家、物理学家和工程师,有时被描述为“博学多才之人”。他出生于当时属于佛罗伦萨公国的比萨市。伽利略被誉为观测天文学之父、现代经典物理学之父、科学方法之父以及现代科学之父。

伽利略研究了速度与加速度、重力与自由落体、相对性原理、惯性、抛体运动等,并在应用科学和技术领域开展了工作,描述了摆的特性和“静水天平”。他是文艺复兴早期温度计(即热测量仪)的开发者之一,还发明了多种军用罗盘。通过他改进的望远镜,伽利略观测到了银河的恒星、金星的位相、木星的四大卫星、土星的光环、月球的陨石坑以及太阳黑子。他还制作了一种早期显微镜。

伽利略对哥白尼日心说的支持遭到了天主教会内部和一些天文学家的反对。1615年,罗马宗教裁判所对这一问题进行了调查,得出结论认为他的观点与当时普遍接受的《圣经》解释相矛盾。[9][10][11]

后来,伽利略在《两大世界体系的对话》(1632年)中为自己的观点辩护,但书中似乎对教皇乌尔班八世进行了攻击和嘲讽,这使伽利略疏远了教皇及之前一直大力支持他的耶稣会士。[9] 他因此受到宗教裁判所的审判,被认定为“严重可疑的异端”,并被迫公开认错。伽利略此后被软禁在家中度过余生。[12][13] 在此期间,他撰写了《两种新科学》(1638年),主要探讨了运动学和材料强度问题。[14]
\subsection{早年生活与家庭}  
伽利略于1564年2月15日出生在比萨(当时属于佛罗伦萨公国),是文琴佐·伽利莱和朱莉亚·阿曼纳蒂的长子。他的父亲文琴佐是著名的鲁特琴演奏家、作曲家和音乐理论家,他的母亲朱莉亚是当地一位显赫商人的女儿。两人于1562年结婚,当时文琴佐42岁,而朱莉亚24岁。伽利略自己也成为了一名出色的鲁特琴演奏家,并可能从父亲那里早早学会了对权威的怀疑态度。[15][16]  

伽利略的五个兄弟姐妹中,有三个在婴儿期幸存下来。他最小的弟弟米开朗基罗(或称米开拉尼奥洛)也成为了一名鲁特琴演奏家和作曲家,但这一生都给伽利略带来了经济负担。[17] 米开朗基罗未能履行父亲承诺的嫁妆分担责任,这导致伽利略的妹夫们试图通过法律途径追讨欠款。此外,米开朗基罗有时还需要向伽利略借钱,以支持他的音乐事业和旅行。这些经济压力可能促使伽利略早年就产生了发明一些能够带来额外收入的装置的想法。[18]  

伽利略八岁时,家人搬到了佛罗伦萨,但他被留在比萨,由穆齐奥·泰达尔迪照顾了两年。十岁时,他离开比萨,与家人团聚在佛罗伦萨,并开始接受雅科波·博尔吉尼的指导。[15] 从1575年至1578年,他在佛罗伦萨东南约30公里的瓦隆布罗萨修道院接受教育,特别是在逻辑方面的学习。[19][20]
\subsubsection{名字}  
伽利略通常只用他的名字来称呼自己。在当时的意大利,姓氏并非必需,而他的名字“伽利略”(Galileo)与他有时使用的家族姓氏“伽利莱”(Galilei)源于同一祖先。无论是他的名字还是姓氏,最终都可追溯到他的祖先伽利略·博纳尤蒂(Galileo Bonaiuti),一位15世纪佛罗伦萨的重要医生、教授和政治家。[21] 伽利略·博纳尤蒂被安葬在佛罗伦萨的圣十字大教堂,这也是大约200年后伽利略·伽利莱的安葬地。[22]  

当伽利略使用多于一个名字时,他有时称自己为“伽利略·伽利莱·林切奥”(Galileo Galilei Linceo),以表示他是林切伊学院(Accademia dei Lincei)的一员。该学院是教皇国成立的一所精英科学组织。在16世纪中期的托斯卡纳,长子通常以父母的姓氏命名为名字。[23] 因此,伽利略·伽利莱的名字未必是专门为了纪念他的祖先伽利略·博纳尤蒂。意大利男性名“伽利略”(Galileo)及其派生的姓氏“伽利莱”(Galilei)来源于拉丁语“Galilaeus”,意为“加利利人”。[24][21]  

伽利略名字和姓氏的圣经渊源后来成为一个著名双关语的主题。[25] 1614年,在伽利略事件期间,伽利略的一位反对者、多米尼加会士托马索·卡奇尼(Tommaso Caccini)发表了一篇颇具争议且影响深远的布道词。他在布道中引用《使徒行传》1:11说道:“加利利人哪,你们为什么站着望天呢?”(可能含有对伽利略的讽刺)。[citation needed]  
\subsubsection{子女}  

尽管伽利略是一位虔诚的天主教徒,[26] 他与玛丽娜·甘巴(Marina Gamba)未婚生育了三个孩子。他们育有两个女儿:维尔吉尼亚(Virginia,生于1600年)和利维娅(Livia,生于1601年),以及一个儿子:文琴佐(Vincenzo,生于1606年)。[27]  

由于孩子们是非婚生的,伽利略认为两个女儿难以嫁人,且无法承担高昂的经济支持或嫁妆费用。这些费用可能会使伽利略重蹈帮助两位姐妹嫁人的经济困境。[28] 因此,她们唯一合适的选择就是修道生活。两位女儿都被阿尔切特里的圣马太修道院接纳,并在此度过了余生。[29]  

维尔吉尼亚进入修道院后改名为玛利亚·切莱斯特(Maria Celeste)。她于1634年4月2日去世,并与伽利略合葬在佛罗伦萨的圣十字大教堂。利维娅进入修道院后改名为阿尔坎杰拉修女(Sister Arcangela),大部分时间身体都不好。伽利略的儿子文琴佐后来被确认为合法继承人,并与塞斯蒂莉亚·博基内里(Sestilia Bocchineri)结婚。[30]  