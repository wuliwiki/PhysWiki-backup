% 克莱姆法则
% keys 线性代数|线性方程组|克莱姆法则|行列式

\pentry{矩阵\upref{Mat}}

\textbf{克莱姆法则(也译作克拉默法则(Kramer's rule)}可以直接用行列式表示出以下线性方程组的各个解. 将线性方程组记为
\begin{equation}
\mat A \bvec x = \bvec y
\end{equation}
其中 $\mat A$ 是已知的 $N$ 行 $N$ 列方阵, $\bvec y$ 是已知的 $N$ 维列矢量, $\bvec x$ 是未知的 $N$ 维列矢量. 当系数行列式 $\abs{\mat A}$ 不为零时, 方程有唯一解. $\bvec x$ 的各个元为
\begin{equation}
x_i = \frac{\abs{\mat B_i}}{\abs{\mat A}}
\end{equation}
其中矩阵 $\mat B_i$ 是把矩阵 $\mat A$ 的第 $i$ 列替换为 $\bvec y$ 得到的矩阵.

% 例题: 未完成
