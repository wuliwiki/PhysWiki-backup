% 一维散射态的正交归一化
% keys 散射态|相移|正交|归一化|波函数
% license Xiao
% type Tutor

\begin{issues}
\issueTODO
\end{issues}

\pentry{施密特正交化\nref{nod_SmdtOt}, 幺正变换\nref{nod_Unitar},量子散射(一维)\nref{nod_Sca1D}, 零函数(列)\nref{nod_F0}}{nod_d908}
本文使用原子单位制\upref{AU}。 类似于平面波的归一化\upref{EngNor}, 一维散射态也有不同的归一化方式, 但情况要更为复杂。

\subsection{对称势能}
为简单起见先假设 $V(x)$ 关于原点对称, 且 $V(x)$ 是短程的, 即只在区间 $[-L,L]$ 内不为零。 定态薛定谔方程(\autoref{eq_SchEq_1}~\upref{SchEq})
\begin{equation}
-\frac{1}{2m}\pdv[2]{x}\psi(x) + V(x)\psi(x) = E\psi(x)~
\end{equation}
中,$E > 0$ 的解都是散射态。 由于 $V(x)$ 的对称性, 我们必定能找到\textbf{实值的奇函数和偶函数}两种解。 令 $k = \sqrt{2mE}$, 在区间 $[-L,L]$ 外, 波函数就是正弦函数加上一个相移
\begin{equation}\label{eq_ScaNrm_3}
\psi_k(x) = \leftgroup{
&A\sin(kx + \phi) && (x > L)\\
&\pm \psi_k(-x) && (x < L)~,
}\end{equation}
其中 $A$ 为实数, $\phi$ 是 $k$ 的函数, 称为\textbf{相移(phase shift)}。 为方便书写下文把 $\phi(k),\phi(k')$ 分别记为 $\phi, \phi'$。

记奇函数和偶函数散射态分别为实函数 $\psi_{k,1}(x)$ 和 $\psi_{k,2}(x)$ 我们希望通过指定适当的归一化系数后, 波函数能满足正交归一化条件(\autoref{eq_EngNor_3}~\upref{EngNor})
\begin{equation}\label{eq_ScaNrm_1}
\int_{-\infty}^{+\infty} \psi_{k',i}^*(x) \psi_{k,i}(x) \dd{x} = \delta(k' - k)~\qquad (k > 0,\ i = 1, 2)~,
\end{equation}
\begin{equation}\label{eq_ScaNrm_6}
\int_{-\infty}^{+\infty} \psi_{k',1}^*(x) \psi_{k,2}(x) \dd{x} = z(k' - k) \qquad (k > 0)~.
\end{equation}
其中 $\delta$ 理解为 $\delta$ 函数列\upref{Delta}, $z$ 是零函数(列)\upref{F0}, 以避免积分不收敛的问题。

\begin{theorem}{}\label{the_ScaNrm_1}
\autoref{eq_ScaNrm_1} 和\autoref{eq_ScaNrm_6} 对所有性质良好的偶函数 $V(x)$ 都成立, 且\autoref{eq_ScaNrm_3} 中\textbf{归一化系数和平面波相同}, 即 $A = 1/\sqrt{\pi}$(\autoref{eq_EngNor_5}~\upref{EngNor})。
\end{theorem}

\subsection{不对称势能}
由于势能 $V(x)$ 不对称, 我们仍然可以求出\textbf{两组线性无关的实函数解} $\psi_{k,1},\psi_{k,2}$, 却无法自动保证它们的奇偶性, 所以需要额外进行正交化以满足\autoref{eq_ScaNrm_6}。

任何这样的解在区间 $[-L,L]$ 外的波函数都可以表示为
\begin{equation}\label{eq_ScaNrm_9}
\psi_{k,i}(x) = \leftgroup{
    &A_{+,i} \sin(kx + \phi_{+,i}) &&(x > L)\\
    &A_{-,i} \sin(kx - \phi_{-,i}) &&(x < -L)
} \qquad (i = 1,2)~.
\end{equation}
其中 $A_{\pm,i}$ 都是实数。

\subsubsection{归一化}
\begin{theorem}{}\label{the_ScaNrm_2}
要使\autoref{eq_ScaNrm_9} 的波函数归一化,需要满足
\begin{equation}\label{eq_ScaNrm_4}
\frac{1}{2}\qty(\abs{A_{+,i}}^2 + \abs{A_{-,i}}^2) = \frac{1}{\pi} \qquad (i = 1,2)~,
\end{equation}
即\textbf{振幅的平均模方}为 $1/\pi$。
\end{theorem}
证明见文末。 特殊第, 当 $A_{+,i} = A_{-,i}$ 时就有\autoref{the_ScaNrm_1} 中的归一化系数 $1/\sqrt{\pi}$。

\subsubsection{正交化}
对于非对称势能 $V(x)$, 由于我们缺失了函数的奇偶性, 需要用其他办法保证 $k$ 相同的两个线性无关解正交。 对给定的 $k$, 两个线性无关解张成二维本征矢空间,若在其中定义内积为
\begin{equation}\label{eq_ScaNrm_13}
\braket{\psi_{k,i}}{\psi_{k,i'}} = \lim_{n\to\infty}\frac{\pi}{n}\int_{-n}^{+n} \psi_{k,i}^*(x) \psi_{k,i'}(x) \dd{x} \qquad (i = 1,2;\ i' = 1,2)~,
\end{equation}
那么该二维空间的正交归一条件就是
\begin{equation}\label{eq_ScaNrm_7}
\braket{\psi_{k,i}}{\psi_{k,i'}} = \delta_{i,i'}~.
\end{equation}
显然对称势能的奇函数和偶函解根据该定义是正交的,并且下面会证明该条件等效于\autoref{eq_ScaNrm_1} \autoref{eq_ScaNrm_6}。


作为例子,可以验证该定义下 $\sin(kx+\phi)/\sqrt{\pi}$ 以及 $\exp(\I kx)/\sqrt{2\pi}$ 的模长为 1。 

把\autoref{eq_ScaNrm_9} 的波函数代入\autoref{eq_ScaNrm_13} (令 $\Delta\phi_\pm = \phi_{\pm, 2} - \phi_{\pm, 1}$)得
\begin{equation}
\braket{\psi_{k,1}}{\psi_{k,2}} = \frac{\pi}{2}(A_{-,1}A_{-,2}\cos\Delta\phi_- + A_{+,1}A_{+,2}\cos\Delta\phi_+)~.
\end{equation}


若我们已经解得两个不正交的线性无关解 $\psi'_{k,1}, \psi'_{k,2}$, 可以使用施密特正交化\upref{SmdtOt}: 先把 $\psi'_{k,1}$ 按照\autoref{eq_ScaNrm_4} 即\autoref{eq_ScaNrm_7} 归一化得 $\psi_{k,1}$, 那么
\begin{equation}\label{eq_ScaNrm_14}
\bar\psi_{k,2} = \psi_{k,2}' - \braket{\psi_{k,1}}{\psi'_{k,2}} \psi_{k,1}~
\end{equation}
与 $\psi_{k,1}$ 正交。 这相当于把 $\psi'_{k,2}$ 中与 $\psi_{k,1}$ 不正交的部分减去。 最后再做归一化即可
\begin{equation}\label{eq_ScaNrm_15}
\psi_{k,2} = \bar\psi_{k,2}/\sqrt{\braket*{\bar\psi_{k,2}}{\bar\psi_{k,2}}}~.
\end{equation}
这样就得到了正交归一得 $\psi_{k,1}, \psi_{k,2}$。

\begin{example}{}
令两组不正交不归一的平面波为
\begin{equation}
\psi'_{k,1}(x) = \sin(kx - \varphi)~, \qquad
\psi'_{k,2}(x) = \sin(x)~.
\end{equation}
其中 $\varphi$ 是常数。把 $\psi'_{k,1}(x)$ 归一化易得
\begin{equation}
\psi_{k,1}(x) = \frac{1}{\sqrt{\pi}}\sin(kx - \varphi)~.
\end{equation}
代入\autoref{eq_ScaNrm_14} 后再做归一化, 可得正交归一化的 $\psi'_{k,2}(x)$ 为
\begin{equation}
\psi_{k,2}(x) = \frac{1}{\sqrt{\pi}}\cos(kx - \varphi)~.
\end{equation}
显然 $\psi_{k,1}, \psi_{k,2}$ 是正交的, 因为它们分别是关于 $x = \varphi/k$ 的奇函数和偶函数。
\end{example}

\subsection{行波的正交归一化}
以上只讨论了实数解,它们的物理意义是驻波。 势垒外部的行波解具有以下形式(不要求 $V(x)$ 对称):
\begin{equation}
\psi_{k,i} = \leftgroup{
    &A_i\exp(\I kx) + B_i\exp(-\I kx) && (x < -L)\\
    &C_i\exp(\I kx) + D_i\exp(-\I kx) && (x > L)
} \qquad (i = a,b)~,
\end{equation}
该形式只是对两个实数解用复常数进行了线性组合,就像 $\exp(\pm\I kx)$ 是 $\sin(kx)$ 和 $\cos(kx)$ 的线性组合。

可证明上一节的内积公式\autoref{eq_ScaNrm_13} 对行波解仍然适用:
\begin{equation}
\braket{\psi_{k,i}}{\psi_{k,i'}} = \pi (A_i^* A_{i'} + B_i^* B_{i'} + C_i^* C_{i'} + D_i^* D_{i'})~,
\end{equation}
所以行波解的归一化条件为
\begin{equation}
\abs{A_i}^2 + \abs{B_i}^2 + \abs{C_i}^2 + \abs{D_i}^2= \frac{1}{\pi}~,
\end{equation}
对于平面波, 显然有 $A = C = 1/\sqrt{2\pi}$, $B = D = 0$。 和 “平面波的的正交归一化\upref{EngNor}” 中结论相同。

\begin{figure}[ht]
\centering
\includegraphics[width=9cm]{./figures/b49512be2c9b45ba.pdf}
\caption{对称 $V(x)$ 存在对称且正交的行波解。 左: $\psi_{k,a}$, 右: $\psi_{k,b}$} \label{fig_ScaNrm_1}
\end{figure}

一种 $\psi_{k,1}$ 常用的行波边界条件为 $D_a = 0$, 它的物理意义是粒子从左边入射, 发生向左的反射和向右的透射(\autoref{fig_ScaNrm_1} 左)s。 又根据概率流\upref{PrbJ}守恒得
\begin{equation}
\abs{A_a}^2 = \abs{B_a}^2 + \abs{C_a}^2~,
\end{equation}
所以 $\abs{A_a} = 1/\sqrt{2\pi}$。 所以我们一般直接规定
\begin{equation}\label{eq_ScaNrm_10}
A_a = \frac{1}{\sqrt{2\pi}}~.
\end{equation}
一般情况下我们通过施密特正交化得到与之正交归一的 $\psi_b$。 我们先令不一定正交的 $\psi'_b$ 满足边界条件 $A_b = 0$, 即粒子从右边入射, 投影得
\begin{equation}
\braket{\psi_a}{\psi'_b} = \pi \qty(\frac{A'_b}{\sqrt{2\pi}} + B_a^* B'_b + C_a^*C'_b)~.
\end{equation}
于是正交但未归一化的 $\bar\psi_b$ 为
\begin{equation}
\bar\psi_b = \psi'_b - \braket{\psi_a}{\psi'_b}\psi_a~.
\end{equation}
但我们下面会看到当 $V(x)$ 对称(偶函数)时, 正交的行波解是完全对称的: $\psi_{k,b}(x) = \psi_{k,a}(-x)$。

\subsubsection{变换矩阵}
假设\autoref{eq_ScaNrm_9} 已经正交归一化, 要得到正交归一的两个行波, 进行任意幺正变换\upref{Unitar}即可。 令酉矩阵\upref{UniMat}为 $\pmat{c_1 & c_2\\ c_2^* & -c_1^*}$, 满足 $\abs{c_1}^2 + \abs{c_2}^2 = 1$, 有
\begin{equation}
\begin{aligned}
&\psi_{k,a}(x) = c_1\psi_{k,1}(x) + c_2\psi_{k,2}(x)~,\\
&\psi_{k,b}(x) = c_2^*\psi_{k,1}(x) - c_1^*\psi_{k,2}(x)~.
\end{aligned}
\end{equation}
令 $\psi_{k,a}$ 满足上述边界条件 $D_a = 0, A_a = 1/\sqrt{2\pi}$ 分别得到
\begin{equation}\label{eq_ScaNrm_12}
\begin{aligned}
&c_1 A_{+,1}\E^{-\I\phi_{+,1}} + c_2 A_{+,2}\E^{-\I\phi_{+,2}} = 0~,\\
&c_1 A_{-,1}\frac{\E^{-\I\phi_{-,1}}}{2\I} + c_2 A_{-,2}\frac{\E^{-\I\phi_{-,2}}}{2\I} = \frac{1}{\sqrt{2\pi}}~,
\end{aligned}
\end{equation}
这就可以求出 $c_1, c_2$, 以及 $\psi_{k,a}, \psi_{k,b}$。

可以证明\footnote{证明过程:对 $\psi_{k,b}$ 写出\autoref{eq_ScaNrm_12} 的对应条件, $c_1,c_2$ 替换为 $c_2^*, -c_1^*$, 这时会发现该条件和\autoref{eq_ScaNrm_12} 是相同的。}如果 $V(x)$ 是对称的($A_{\pm,i} = 1/\sqrt{\pi}$, $\phi_{+,i}=\phi_{-,i}$), 那么 $\psi_b$ 就和 $\psi_a$ 镜像对称(\autoref{fig_ScaNrm_1} )
\begin{equation}
\psi_{k,b}(x) = \psi_{k,a}(-x) \qquad (x\in \mathbb{R})~,
\end{equation}
由正交条件\autoref{eq_ScaNrm_7} 得
\begin{equation}\label{eq_ScaNrm_11}
\Re[B^*C] = 0 \Longleftrightarrow \cos(\arg{B} - \arg{C}) = 0~.
\end{equation}
也就是说\textbf{反射波和透射波的相位总是相差 $\pi/2$}。

\addTODO{有没有可能,即使是不对称shi'nen}

\subsection{对称势能的部分证明}
这里给出\autoref{the_ScaNrm_1} 的部分证明。 对于奇偶性不同的两个函数, 他们显然式正交的(\autoref{ex_F0_2}~\upref{F0})。 对奇偶性相同的, 首先(参考\autoref{exe_Delta_3}~\upref{Delta})
\begin{equation}\label{eq_ScaNrm_5}
\int_{0}^{+\infty} \sin(k'x)\sin(kx)\dd{x} = \frac{\pi}{2}\delta(k'-k)
\qquad (k, k' > 0)~.
\end{equation}
现在添加相位 $\phi(k)$ 后, 有不定积分
\begin{equation}
\int \sin(k'x+\phi')\sin(kx+\phi) \dd{x} = \frac{\sin[(k'-k)x + (\phi'-\phi)]}{2(k'-k)}
- \frac{\sin[(k'+k)x+(\phi'+\phi)]}{2(k'+k)}~.
\end{equation}
在 $(0,n)$ 做定积分取极限 $n\to\infty$ 后发现比\autoref{eq_ScaNrm_5} 多了两项:
\begin{equation}
\int_{0}^{+\infty} \sin(k'x+\phi')\sin(kx+\phi) \dd{x} = \frac{\pi}{2}\delta(k'-k)
+ \frac{\sin(\phi'+\phi)}{2(k'+k)} - \frac{\sin(\phi'-\phi)}{2(k'-k)}~,
\end{equation}
所以在区间 $(0, +\infty)$ 上 $\sin(kx+\phi)$ 并不一定正交。 但下面会看到在若积分下限改为 $-\infty$ 则正交。

使用归一化系数 $1/\sqrt{\pi}$, \autoref{eq_ScaNrm_1} 的积分为(利用波函数的奇偶性)
\begin{equation}\label{eq_ScaNrm_2}
\begin{aligned}
\braket{\psi_{k'}}{\psi_k} &= 2\int_{0}^{+\infty} \frac{1}{\sqrt{\pi}}\sin(k'x+\phi')\frac{1}{\sqrt{\pi}}\sin(kx+\phi) \dd{x} + 2I(k,k')\\
&= \delta(k'-k) + \frac{\sin(\phi'+\phi)}{\pi(k'+k)} - \frac{\sin(\phi'-\phi)}{\pi(k'-k)} + 2I(k,k')~,
\end{aligned}
\end{equation}
其中 $2I(k,k')$ 修正了 $[-L,L]$ 区间实际波函数和 $\sin(kx+\phi)$ 的不同, 定义是
\begin{equation}\label{eq_ScaNrm_8}
I(k,k') = \int_0^L \psi_{k'}^*(x) \psi_k(x) \dd{x}
-\int_{0}^{L} \frac{1}{\sqrt{\pi}}\sin(k'x+\phi') \frac{1}{\sqrt{\pi}}\sin(kx+\phi) \dd{x}~.
\end{equation}
如果能证明对于任意偶函数 $V(x)$, \autoref{eq_ScaNrm_2} 的最后的三项之和都为零, 即
\begin{equation}
I(k,k') = \frac{\sin(\phi'-\phi)}{2\pi(k'-k)} - \frac{\sin(\phi'+\phi)}{2\pi(k'+k)}~,
\end{equation}
那么我们就证明了\autoref{eq_ScaNrm_1} 的正交归一关系。 笔者暂时不会证, 但我们可以尝试用一些具体的例子证明, 如方势垒\upref{SqrPot}。

另一种说明 $[-L,L]$ 的积分修正不影响正交归一性的方法是, 把该短程势阱装在 $[-n,n]$ 区间的无限深势阱中, 随着 $n\to\infty$, 在计算不同束缚态之间的内积时 $[-L,L]$ 部分的积分可以忽略不计。

\subsection{非对称势能的部分证明}
\textbf{\autoref{the_ScaNrm_2} 部分证明}: 把正交化积分划分为正负半轴两部分进行($\psi_{k}$ 取 $\psi_{k,1}, \psi_{k,2}$ 中一个)
\begin{equation}
\begin{aligned}
\braket{\psi_{k'}}{\psi_{k}} &=
\abs{A_+}^2\int_0^{\infty} \sin(kx + \phi_{+})\sin(kx + \phi_{+}) \dd{x}   + I_+(k,k')\\
&+ \abs{A_-}^2\int_{-\infty}^0 \sin(kx + \phi_{-})\sin(kx + \phi_{-}) \dd{x} + I_-(k,k')\\
&= \frac{\pi}{2}(\abs{A_+}^2 + \abs{A_-}^2)\delta(k'-k) = \delta(k'-k)~,
\end{aligned}
\end{equation}
其中 $I_{\pm}(k,k')$ 的定义和\autoref{eq_ScaNrm_8} 类似。 第二个等号的具体过程笔者同样不会。

\addTODO{正交化:证明\autoref{eq_ScaNrm_15} 得结果符合零函数正交条件\autoref{eq_ScaNrm_6} }
