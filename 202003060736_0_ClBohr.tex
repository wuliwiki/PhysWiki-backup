% 从天球的音乐到玻尔模型

形或古希腊人所说的 “idea”,有多种含义,比如:形状,这是和视觉有关的;比如风格、分类,这可以是和视觉有关的,也可以无关,比如音乐也可以有风格,这就是和声音有关的了.

和视觉有关的“形”是直观的,我们无须论证,纠结于如何用语言表达,仅凭图形——或者是静态的,或者是想象中动态的——直接给出结果.对形的研究会导向几何学,几何本身是视觉的,而视觉是偏好静的,偏好不动的,但一加“学”,几何“学”或“学”几何就动起来了.

我们如何学呢?或者演示,用圆规和直尺,或者像毕达哥拉斯那样拿根木棍面对沙土,世界是一步一步地被展现出来的,一笔一划本身就是个动态的过程.我们努力说:“首先如何,其次如何,然后,又然后……”

所谓动态就是次序,我们首先只关注首先要解决的,其次,带着对刚刚过去的对首先的记忆,探讨紧接着要解决的问题,我们的思想没法分叉.人在专注的状态下,视觉也需要一个焦点.当我们的视觉遭遇挑战,看不清某物的时候,我们凝眼观瞧,把视线使劲聚焦于某物,凝眼就是凝神,不受诱惑地专注于某物,看清楚一点再继续看下一点.

这个结构很像自然数:“0,1,2,3,……”,一步一步地展示给你看,比如“我是如何用直尺和圆规作图的”,这种线性展开的结构就是时间,“学”的过程,对学的人是学,在Challenge,对展示的人来说是在“证”,在说服,这个过程是世界次第展开的过程,是叙事、是 Chronicle.

\subsection{形与声}

“学” 依赖语言,语言是一种声音现象.

据说人能够发出一个八度再加一个四度的声音.

古代世界,天和地很近,音乐和人也很近.孔子闻韶乐“三月不知肉味”,这种沉浸在声音里的境界和我们今天听流行音乐,把音乐当做一种背景噪音,是完全不同的两种声音技术.今天我们听音乐往往是为了抑制我们心中的背景噪音.

古代的音乐都很简单.简单到好比就是敲击单音音叉发出的声音,单音音叉是校音用的,它在古代世界的对应物是中国的黄钟律管或希腊的单弦琴(Monochord).它们发出很纯的音,基本上就是一个频率.孔子一生关心礼,礼与乐相联,乐就是音及音的混杂与排列.

我们用音高,频率,响度,音色等来描述声音.音高就是频率,是描述“音”诸参数中最重要的一个.人天生就是一个感知音高的灵敏动物,高音激越,使人振奋,低音呜咽,让人伤感.简单的音乐庄重使人入静,而复杂多变的音乐也如一场“视觉的盛宴”,它使我们好奇和沉迷.

听觉和视觉一样,是感觉,同时也是思维,我们的眼睛和耳朵接受信息,同时也处理、歪曲信息以为我们所用.古代的政治传统,古代的教育家都注重音乐教育,这其中最重要的就是对音乐体系的保留和传承.

比如唱歌的时候要先定调,调可以定低点,显得庄重,也可以定高点,显得轻快.定好调后,一系列的声音次第展开,它们的相对音高保持一个固定的结构,比如:

“低,低低,高,高高,低,中中,……”

在给定乐谱的前提下.基准音高的选取,或所谓定调是任意的.我们可以定高点,无非大家唱不上去而已.但因为有人唱不上去,这个定调就也不是完全主观任意的了.

古代政治秩序大多由推崇勇猛进取精神的战士集团建立,对战士共同体而言,最重要的是要保持这种勇猛进取的精神,能够保持这种精神的音乐会与特定音高有关,这是人群的共同经验.比如柏拉图在《理想国》中就说,要摒弃悲伤和软绵绵的吕底亚调和伊奥尼亚调,而推崇多利亚调和佛里吉亚调.

保持这种对声音的共同经验在古代政治传统中是非常重要的,其中之一就是确定音调,或基准音的频率,然后在此基础上给出其他音的定义,其他音是相对于基准音而言的,可以更高,也可以更低,构成一个阶梯状的结构.

原子的“idea”是无所不在的,这里由人的听觉经验,我们再次得到了原子的概念,即存在着“音高”的原子,进一步细分不同音高的原子是没有必要的.

保存音乐制度最简单的方法就是造一套标准的乐器,然后后人反复向这些标准的乐器学习,第一套自然是由城邦的缔造者“铸造”的.

考虑到弦乐器与弦绷紧的程度有关,受湿度、温度影响较大,青铜器制造的发音器会是理想的选择,这是为什么“钟”会成为“国家”符号的原因,塔可夫斯基电影《安德烈\cdot卢布廖夫》再现的是俄罗斯帝国创旦的精神基础,在影片的结尾就出现了工匠之子铸钟的奇迹.

\begin{figure}[ht]
\centering
\includegraphics[width=10cm]{./figures/ClBohr_1.png}
\caption{工匠之子铸钟} \label{ClBohr_fig1}
\end{figure}


钟是要发音的,音高是有标准的,“音高”高一些,低一些,很微妙,但人的耳朵,或某些人的耳朵天生就是辨别音高的灵敏仪器.只有能发出特定音高的钟才是可以被接受,一只发音不准的钟在敲响的时候不嘹亮,不能激发人民激越的精神,这对城邦是不利的.

这里有个似是而非但很有趣的讨论,人有时间感,但人的时间感是非常内在的,几乎不存在什么可以相互交流的基础.这是妨碍人产生运动观念,并研究运动的重要原因.音高即频率,频率就是时间的倒数,人没法准确标记时间的流逝,但人却是辨别频率(时间倒数)的精密仪器.同时我们的发音器官,还能娴熟地对不同音高的声音进行模仿,这是我们具有语言和音乐能力的生物学基础.

类似地,我们还可以讨论视觉,讨论视觉对位置和速度的分辨.人天生就能在相当精确的意义下辨别位置,但我们对速度的判断就要差许多.我们说A比B快,其实是通过位置下的判断,即AB同时出发,但A先撞线,所以A更快.这是亚里士多德无法得到“正确”的落体规律的原因,他受人本身的局限,速度是很难直接被看的.

古代实验技术还没有充分发展起来,而实验技术的充分发展与资本主义的生产方式兴盛有关,近代自然科学与资本主义生产方式同步爆发并非巧合.回顾二者,科学史和资本主义发展史,两者讲的是同一个故事,只是叙事的角度生了变化.

由“造钟”故事,我们得到一个新洞见,即:“音与形有关”.对钟来说这是大大地简单化了,因为材质也很重要,但形状确实决定了钟振动的频率.这意味着:“听音可以定形,定形可以定音”.

形既是形状也是模型,还是形式.在毕达哥拉斯和柏拉图的传统里,形是与数紧密相连的.比如钟的形由何而定呢?长、宽、高、是数字,钟的厚度也是数字,但这一堆数字的集合又有什么意义呢?

当我滔滔不绝地罗列一堆数字的时候,这是没有意义的.我们需要给出数字和数字之间的关系,才有意义.而且最好是只给出一个关系(或最少关系),就能让所有的数字各就各位.找到这样的规律自然是对思维的奖励,是可以向众人夸耀的;同时这也是技术,有了技术我们就能铸钟,小孩的父亲是会铸钟的,但他把技术带到坟墓里去了.

《安德烈\cdot卢布廖夫》中的小孩是幸运的,他必须试试,他也只能试试.在拜占庭衰败之后,东正教来到了俄罗斯与当地的土豪、愚民混合,文明在绝望中重新开始,这就是俄罗斯的宿命.卢布廖夫受不会铸造但却造出钟的小孩的激励,重新拿起画笔开始画注定会塑造俄罗斯民族精神的那些很平、很抽象圣像画.

