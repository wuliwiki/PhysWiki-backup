% 关于小时百科
% license Xiao
% type Tutor

\subsection{我们要做什么}
\addTODO{原来的介绍见\autoref{sub_about_1} 及以后, 有待整合。}

看到《小时百科》这个标题, 可能许多人会以为这是一个类似维基百科的网站。 但我们并不是在做一个国产的维基百科, 虽然我们的网站叫做小时百科, 但我们目前大部分内容都更接近\textbf{教程}、\textbf{讲义}或者\textbf{博客}。 这和维基百科有什么区别呢?

维基百科上每个条目基本可以看做是围绕其标题的一个\textbf{综述}, 追求全面、客观、中立、具有一般性。 例如在维基百科的\href{https://en.wikipedia.org/wiki/Newton's_laws_of_motion}{牛顿运动定律}条目\footnote{我们将只引用维基百科的英文条目, 因为中文条目的质量还有相当大的差距。}中囊括了牛顿力学的发展历史, 牛顿三定律的具体表述, 功和能, 刚体力学, 混沌, 分析力学, 与热力学的关系, 与电磁学、相对论以及量子力学的关系。 全文约 9000 单词, 引用文献约 120 个。 这样的结构作为一个百科条目或者一篇综述是符合标准的, 但对想要自学的初学者却并不那么友好——它包含的信息量太大, 而每个具体的知识点讲得又不够详细。 又例如在\href{https://en.wikipedia.org/wiki/Angular_momentum}{角动量}条目中, 同样涉及了牛顿力学、分析力学和量子力学中的角动量相关内容, 当然还有发展史。

作为对比, 如果我们翻开一些优秀的大学物理系低年级的力学教材, 里面则可能会先用一章介绍一些简单的矢量微积分(多见于英文教材), 然后讲解如何画受力分析图, 力的合成与分解, 然后再讲解牛顿三定律, 刚体力学, 最后才会简单介绍分析力学和相对论等, 也通常不会涉及任何量子力学。 也就是说维基百科上的牛顿运动定律涉及到了一本(甚至多本)几百页的教材!

当然, 很多时候我们并不想把整本教材从头读到尾(即使在大学的课程中由于时间有限也很少把整本书都覆盖到)。 所以姑且假设你的目标是\textbf{较快地学习了解某几个知识点}(例如高中生想了解大学物理中如何描述牛顿三定律), 而不是系统地按部就班地学习整个学科。 在自学维基百科无果后, 你翻开一本教材通过目录找到对应的章节想要学习。 如无意外, 你会发现想要学的章节看不懂, 因为前面的内容没读。 在没有人指引的情况下, 你最后不得不从第一页开始看。 更糟糕的是, 如果这本书开始就假设你已经学过一些其他的课程(例如微积分和线性代数)那你还需要找来这些教材, 仍然从第一页开始看…… 如果你足够有耐心和毅力, 终于把若干本教材都从头看完弄懂 ——诚然你的基础会非常扎实—— 回过头来你会发现这个过程中学到的大部分内容对你最初的目标来说太过深入, 或者根本可以跳过, 但你不可能从一开始就知道哪些才是你真正需要的。

当然如果你一定要通过维基百科自学, 理论上顺着所有链接和引用, 你最终也可以达到目的, 但你通常并不清楚哪些链接和引用是你当前最需要的。 这就会产生不同页面间海量的依赖关系, 形成一个错综复杂的迷宫。 如果假设每个页面链接到 10 个不同的页面, 那么只需 3 层就会达到惊人的 1000 个页面。 而一些至关重要的预备知识可能会隐藏在更深的链接中。

解决这个问题最好的办法是找一个老师说明你想学什么, 学到什么深度, 然后让老师根据这个目标以及你现有的知识背景和愿意投入的精力给你定制一套私人课程。 但显然大部分人并不具有这样优越的条件。 而小时百科的目标正是代替这样一个老师, 让每个人都能根据自己的知识背景和目标给自己量身定制一套课程。

一些比较人性化的教材在前言中会根据你的时间精力给你一些不同的选择(例如说明时间不够可以只看哪些章节, 甚至会画出一个章节关系的树状图)。 但我们想把这种依赖关系做得更细更友好, 对每个小节都给出完整的依赖关系。 甚至对每个使用的公式定理的来源都进行精确的追溯定位。

当然, 相对于百科或综述而言, 教材的内容编排是高度主观且取决于作者和目标读者的。 所以在小时百科的\href{http://wuli.wiki/online}{目录}中, 我们并不是按照话题来分类所有文章, 而是创建许多\textbf{部分}。 有的部分可以是一个非常完整和连续的传统教材, 每个页面是一个小节, 适合从头读到尾(当然你也可以根据依赖关系自己决定读哪些)。 而另一些部分则没那么系统, 只是把某个话题下一些相对独立的小教程、博文、讲义、笔记等放到一起(但仍然需要给出具体的依赖关系)。 至于维基百科那样的综述性页面(目前还几乎没有), 我们同样可以根据话题创建\textbf{专门的部分}来收纳它们(目前还没有), 这些综述的正文中又可以进一步链接到其他部分中的页面(例如详细的证明推导)。 小时百科的每个部分(甚至它的每章)都应该有介绍页面来描述它的内容和结构。

基于这样的想法, 本来属于维基百科中一个条目的内容可能会出现在小时百科的许多不同页面中, 每一个知识点都可能出现针对不同目标读者的不同版本的页面, 具有不一样的预备知识和链接, 分散在不同部分中。 和维基百科一样, 小时百科中每个页面的标题名称必须是全站唯一的, 若同一个内容有不同版本的文章, 我们使用稍微不同的标题或在它后面用括号加以区分。 例如 “角动量(科普)\upref{AngMo}”, “角动量、角动量定理、角动量守恒(单个质点)\upref{AMLaw1}”, “角动量定理、角动量守恒\upref{AMLaw}”, “轨道角动量(量子力学)\upref{QOrbAM}”, “自旋角动量\upref{Spin}” 都分布在不同的部分和章节。

综上, 你可以认为小时百科(的目标)是集百科、教材、 博客/笔记、 \href{http://wuli.wiki/apps}{互动演示}、 \href{http://wuli.wiki/forum}{讨论}于一体的, 支持合作编辑的综合性网站\footnote{而尴尬的是我们的网站已经叫做 “百科” 了, 但主要是一些不那么百科的内容。 或许以后会改名。}。

% === 这段应该放到 “创作指导” 中, 这篇文章是给读者看的 ====
% 每个页面应该有对应的审核员(通常是该页面的第一作者)来负责, 其他人想修改或添加内容需要经过他同意, 如果审核者认为要添加的内容超出了本文的范围, 则应该另外创建一个页面。 不应该像百科(综述)一样, 众多不同的人把众多不同的内容塞进同一个页面。

% ========= 回收的内容 ===========
% 如果我们只有一本公认优秀的传统教材,它包含了丰富的导入,动机,定义,定理,讲解,例题,习题。 而且它有一个很明确的目标人群,这些内容的深度和讲解思路,严谨性,都是为这个人群定制的。

% 那么,如何在一字不变的情况下把它做成灵活的模块化结构呢?

% 这样,我们可以把任意一本教材的内容切割并赋予一个树状结构。 就相当于给这本书生成一个知识地图。 书的内容一字不变,但是读者通过地图很容易看出来全书的结构。 一个极端的例子是,整本书都是严格线性的,任意第 $i+1$ 个节点都需要第 $i$ 个节点作为预备知识。 那么所谓的树状图就只有一个分支,这就要求所有读者把整本书从头看到尾或者看到中间的某个节点。 但几乎没有一本教材是这样严格的结构,它通常有许多不同的话题, 例如教材第一节是理论基础 A, 而第二节和第三节分别是话题 B 和话题 C, B 和 C 之间并没有什么联系, 可能一些读者只需要 B 另一些只需要 C。 在没有树状图时,读者并不知道看完 A 就可以直接看 C, 给出树状图以后读者就知道 B 是可以省略的。


\subsubsection{百科和教材的矛盾}
一个经常遇到的问题是, 有时候一篇文章的范围很难明确界定。 例如 “球谐函数\upref{SphHar}” 页面, 如果它本来是作为某个其他页面的预备知识而创作的一个连贯自洽的小教程, 它不会像维基百科的\href{https://en.wikipedia.org/wiki/Spherical_harmonics}{球谐函数}条目那样包含几乎所有性质以及和其他众多特殊函数的关系, 而只是包含一些在某个场景下比较重要的性质(例如解氢原子的定态波函数\upref{HWF}所需要的那些)以及特定的讲解方式和举例。 但是随着其他需要使用球谐函数作为预备知识的词条的出现, 原作者或其他人可能会倾向于不断往同一个页面补充并引用新的性质, 并调整讲解顺序, 最后导致其越来越接近维基百科——但这样它就变得不那么适合初学/自学了。 可见百科和教材存在不可调和的矛盾。 对于一个话题, 我们不可能写出一个既像教材又像百科的 “部分” 并让所有人都满意。 这就是为什么要创建多个部分。

根据上文提出的思路, 最终理想的情况是, 我们既有维基百科那样关于球谐函数的综述(方便已经学过的人查阅检索, 不包含太多具体的细节和推导,不必照顾初学者), 又有符合不同场景的各种版本分散于其他部分。 我们也\textbf{无需避免相似的内容在不同页面出现重复}, 甚至还可以必要时把已有内容稍作修改创建一个符合不同需求的版本(如果协议兼容)。 这时, 不同的球谐函数页面就会分别被命名为诸如 “球谐函数(综述)”, “球谐函数简介(量子力学)”, “球谐函数表\upref{YlmTab}”, “球谐函数与XXX”, “球谐函数的XX性质” 等等。

\subsection{原小时百科简介}\label{sub_about_1}

小时百科从结构上尝试将教材和百科这两种不同形式的文本融合到一起, 使其既适合\textbf{初学者自学}, 又可以根据需要按照灵活的顺序阅读。 我们计划涵盖理工科专业本科课程中的主要内容, 适用于具有普通高中及以上数学物理基础的读者。 小时百科是一个庞大的工程, 将长期处于更新状态。

在介绍小时百科的特点以前,我们先总结一般数理教材的不足:
\begin{enumerate}
\item 需要按顺序学习,不适合初学者快速了解或查找某个话题或知识点。 例如某高中生需要了解角动量的概念, 直接翻开大学力学教材的相关章节发现看不懂, 却又不知道需要先学什么, 也没时间从头先看完微积分和线性代数的教材再开始学习。
\item 读者不能自己选择所学内容的深度和严谨性。 例如一些常用的高等数学教材在读者对微积分还没有一个大概的了解时就介绍最严谨的定义和证明。 我们认为这样做对初学者并不友好。
\item 不够自洽(self-contained)。 一本教材的自洽性指目标读者在学习前是否还需要学习其他教材。 例如大部分本科物理教材对高中生都是不自洽的, 因为它们往往假设读者具有一定的微积分和线性代数基础。
\end{enumerate}

再来看一般网络百科的不足:
\begin{enumerate}
\item 每个词条都相当于一个综述, 追求大而全。
\item 每个词条都以最广义最严谨的方式呈现, 读者不能选择适合自己的深度和严谨程度。
\item 容易出现公式定理的罗列, 思路不连贯, 缺乏知识点导入和讲述, 缺乏例题, 习题等。
\end{enumerate}

为了克服上述困难,百科采用以下形式:
\begin{enumerate}
\item 将知识点划分为词条, 且在每个词条中列出学习该词条前需要先学习哪些词条。 这样相当于建立了一个知识树(如\autoref{fig_about_1})。 完整的知识树见 \href{https://wuli.wiki/tree}{wuli.wiki/tree}, 可以选中任意词条为目标,生成知识树。
\item 采用词条分级,把同一个话题以不同深度, 严谨度和适用范围等划分成若干个等级的同名词条。 这样读者可以选择螺旋式学习(例如初中,高中,大学物理中所学的话题几乎相同,但程度不同)。 暂定初级词条从科普开始,尽量少使用数学。随着词条级别升高,会使用适用范围更广的定义,更严谨的表述和更抽象的数学等。
\end{enumerate}

\begin{figure}[ht]
\centering
\includegraphics[width=10cm]{./figures/648204cc09583468.pdf}
\caption{由 “预备知识” 画出的知识树(目标词条为“力场、势能\upref{V}”)}\label{fig_about_1}
\end{figure}

\subsection{词条}
小时百科内容繁多,不同词条的重要性相差甚远,如果要系统学习应该以每章给出的导航为主线,再根据兴趣和需要阅读其余词条。

理论上,读者可以直接跳到最感兴的词条,如果 “预备知识” 中列出的词条都已经掌握,就可以开始学习该词条,否则就先掌握 “预备知识” 中的词条。 如果 “预备知识” 出现在词条开始,则必须先掌握,如果出现在正文中,则只有阅读该部分时需要掌握。如果正文中引用了没有出现在“预备知识”中的内容,则读者可自行决定是否阅读。

为了便于书内的跳查,词条之间进行了大量的链接,链接到其他词条如 “导数简介\upref{Der}” 在 pdf 中, 右上角中括号中的数字代表被引用词条的页码。 由于每个词条的公式编号都从 1 开始, 引用其他词条中的公式有时会用类似 “\autoref{eq_Der_2}~\upref{Der}” 的格式, pdf 中右上角的方括号中是公式所在词条的起始页码。 在网页版中, 使用快捷键组合 “Alt + 左箭头” 返回跳查前的位置, 在 app 中也有相应的返回按钮。
