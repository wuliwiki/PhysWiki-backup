% 线性算子
% keys 线性算子
% license Xiao
% type Tutor

\subsection{线性算子}\label{sub_LiOper_4}
\pentry{线性映射\upref{LinMap}}

域 $\mathbb{F}$ 上所有从矢量空间 $V$ 到维矢量空间 $W$ 的线性映射 $f:V\rightarrow W$ 的集合用符号 $\mathcal{L}(V, W)$ (或者 $\mathrm{Hom}(V, W)$) 表示,它仍是一个矢量空间;如果$V, W$是有限维度的,维度分别为$n, m$,那么线性映射 $f \in \mathcal{L}(V, W)$ 和一个 $m\times n$ 的矩阵一一对应\upref{MatLS}。在 $V=W$ 的情形,矢量空间 $\mathcal{L}(V, W)$ 简记为 $\mathcal{L}(V)$ (或 $\mathrm{End}(V)$),它的向量通常称为\textbf{线性算子}。

\textbf{符号约定:}在线性代数部分,线性算子将用拉丁字母 $\mathcal{A,B,C,\cdots}$ 表示,而在矢量空间 $V$ 的基底 $( e_i)$ 之下对应的矩阵用粗体正体字母 $\mat A,\mat B,\mat C,\cdots$ 表示,另一基底 $( e_i')$ 之下对应矩阵则表示为 $\mat A',\mat B',\mat C',\cdots$ 。总是用 $\mathcal{E}=\mathrm{Id}$ 和 $E=(\delta_{ij})$ 表示恒等(单位)映射 $ x\mapsto  x$ 。算子 $\mathcal{A}$ 作用在 $ x$ 上的结果 $\mathcal{A}(x)$ 简写成 $\mathcal{A} x$。

线性算子 $\mathcal{B}$ 称为 $\mathcal{A}$ 的\textbf{逆算子},如果 $\mathcal{AB}=\mathcal{BA}=\mathcal{E}$ 。算子 $\mathcal{A}$ 的逆算子通常记为 $\mathcal{A}^{-1}$。由\autoref{cor_MatLS2_1}~\upref{MatLS2},$\mathcal{A}^{-1}$ 存在等价于 $\mathrm{Ker}\mathcal{A}=0$ 或者 $\mathrm{dim}\;V=\mathrm{dim\;Im}\mathcal{A}$ 。$\mathrm{dim\;Ker}\mathcal{A}$ 称为 $\mathcal{A}$ 的\textbf{亏数}。
\begin{example}{零算子}\label{ex_LiOper_1}
零算子 $\mathcal{O}$ 把每个向量 $ v\in V$ 都变成零:$\mathrm{rank}\; \mathcal{O}=0$。
\end{example}
\begin{example}{相似算子}
$\mathcal{A} x=\lambda x$,其中 $\lambda\in\mathbb{F}$。
\end{example}
\begin{example}{投影算子}
设 $V=U\oplus W$,则 $ x= x_U+ x_W$ 且 $\mathcal{P} x= x_U$,那么称 $\mathcal{P}$ 为\textbf{投影算子}或在子空间 $U$ 平行于 $W$ 的\textbf{投影}。显然 $\mathcal{P}^2=\mathcal{P}$。
\end{example}

\subsection{不同基底下线性算子对应的矩阵}\label{sub_LiOper_1}

\begin{theorem}{}\label{the_LiOper_1}
如果
\begin{equation}\label{eq_LiOper_2}
\mathcal{A}: e_k\mapsto \mathcal{A} e_k=\sum_i^{n}a_{ik} e_i~,\quad \mathcal{B}: e_j\mapsto  \mathcal{B} e_j=\sum_{k=1}^n b_{kj} e_k~.
\end{equation}
是线性空间 $V$ 在基底 $( e_i)$ 之下以 $\mat A=(a_{ij}), \mat B=(b_{kj})$ 为矩阵的线性算子,那么,算子 $\mathcal{AB}$ 在同一基底下的矩阵是 $\mat C=\mat{AB}$
\end{theorem}
\textbf{证明:}
\begin{equation}
\begin{aligned}
\sum_i c_{ij} e_i&=(\mathcal{AB}) e_j=\mathcal{A}(\mathcal{B} e_j)=\mathcal{A}\qty(\sum_k b_{kj} e_k)=\sum_k b_{kj}\mathcal{A} e_k\\
&=\sum_k b_{kj}\sum_i a_{ik} e_i=\sum_{i,k}a_{ik}b_{kj} e_i=\mat{AB} e_i~.
\end{aligned}
\end{equation}
\textbf{证毕!}

\addTODO{移动到线性映射的矩阵表示}

\begin{theorem}{}\label{the_LiOper_2}
若线性算子 $\mathcal A$ 在基底 $( e_1\cdots  e_n)$ 下对应矩阵为 $\mat A$,则在另一基底  $( e'_1\cdots  e'_n)$ 之下对应的矩阵 $\mat A'$ 为
\begin{equation}
\mat A'=\mat B^{-1}\mat{AB}~,
\end{equation}
其中 $\mat B$ 为基底 $( e_i)$ 向基底 $( e_j')$ 的过渡矩阵\upref{TransM}.
\end{theorem}
\textbf{证明:}
由定理条件,若设 $\mat A=(a_{ij}), \mat A'=(a_{kj}'),\mat B=(b_{ij})$ ,则
\begin{equation}
\begin{aligned}
&\mathcal{A} e_i=\sum_{k} a_{ki} e_k
~,\\
&\mathcal{A} e_j'=\sum_{k} a'_{kj} e'_k~,\\
& e_j'=\sum_i b_{ij} e_i~.
\end{aligned}
\end{equation}
引入算子 $\mathcal{B}$ ,它在基底 $( e_1\cdots  e_n)$ 下对应的矩阵为 $\mat B$ ,那么
\begin{equation}
\mathcal{B} e_j=\sum_i b_{ij} e_i= e_j'~.
\end{equation}

由于线性算子与矩阵之间在固定基底之下一一对应,所以可定义一算子 $\mathcal{A'}$ ,它在基底 $( e_1\cdots  e_n)$ 之下对应的矩阵为 $\mat A'$ ,即
\begin{equation}
\mathcal A' e_j=\sum_i a'_{ij} e_i~.
\end{equation}
于是
\begin{equation}
\mathcal{AB} e_j=\mathcal{A} e'_j=\sum_i a'_{ij}  e'_i=\sum_i a'_{ij}\mathcal{B} e_i=\mathcal{B}\qty(\sum_i a'_{ij} e_i)=\mathcal{BA'} e_j~,
\end{equation}
于是 
\begin{equation}
\mathcal{A'}=\mathcal{B}^{-1}\mathcal{AB}~.
\end{equation}
由\autoref{the_LiOper_1} ,上式对应的矩阵的形式就为
\begin{equation}
\mat A'=\mat B^{-1}\mat{AB}~.
\end{equation}
\textbf{证毕}!
\begin{definition}{相似矩阵}
称矩阵 $\mat A'$ 相似于矩阵 $\mat A$ ,如果存在非退化矩阵 $\mat B$ ,使得
\begin{equation}
\mat A'=\mat B^{-1}\mat{AB}~,
\end{equation}
并记作 $\mat A'\sim \mat A$ 。
\end{definition}
容易验证,矩阵的相似关系是一种等价关系\autoref{sub_Relat_1}~\upref{Relat}.

\autoref{the_LiOper_2} 表明,每一个线性算子都对应一个相似矩阵类(所以相似的矩阵构成的集合),而其中每一矩阵都相当于同一线性算子在不同基底下的矩阵。
\subsection{线性算子的行列式与迹}\label{sub_LiOper_2}
设 $\mathcal{A}$ 对应矩阵为 $A$ ,则\autoref{the_trace_1}~\upref{trace} 
\begin{equation}\label{eq_LiOper_3}
\begin{aligned}
&\mathrm{det}(\mat B^{-1}\mat{AB})=\mathrm{det}(\mat A)~,\\
&\mathrm{tr}\;(\mat B^{-1}\mat{AB})=\mathrm{tr}\;(\mat{ABB}^{-1})=\mathrm{tr}\;(\mat A)~.
\end{aligned}
\end{equation}
\begin{definition}{}
称
$
\mathrm{det}\;\mathcal{A}=\mathrm{det}\;A
$
为线性算子$\mathcal{A}$的\textbf{行列式}。

称
$
\mathrm{tr}\;\mathcal{A}=\mathrm{tr}\;A
$
为线性算子 $\mathcal{A}$ 的\textbf{迹}。
\end{definition}
\autoref{eq_LiOper_3} 表明,线性算子 $\mathcal{A}$ 的行列式和迹是 $\mathcal{A}$ 的不变量,即这个定义是适当的。
