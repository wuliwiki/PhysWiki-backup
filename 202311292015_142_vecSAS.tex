% 矢量空间的对称/反对称幂
% keys 对称幂|反对称幂
% license Xiao
% type Wiki

\begin{issues}
\issueTODO
\issueDraft
\issueOther{可以对照张量的对称化和交错化\upref{SIofTe}进行阅读}
\end{issues}

\pentry{空间的张量积\upref{TPofSp},置换的奇偶性\upref{permu},群作用\upref{Group3}}

\subsection{作为子空间的对称/反对称幂}

\subsubsection{对称幂}

向量空间 $V$ 的 $n$ 次张量幂 $V^{\otimes n}$ 上存在一个 $n$ 阶对称群 $S_n$ \upref{Perm}的群作用:
\begin{equation}
\begin{aligned}
\rho(\sigma): V^{\otimes n} &\to V^{\otimes n}~, \\
v_1 \otimes \dots \otimes v_n &\mapsto v_{\sigma(1)} \otimes \dots \otimes v_{\sigma(n)}~.
\end{aligned}
\end{equation}
被称为\textbf{置换作用}。

注意:我们把$\rho(\sigma)$ 被定义为线性的,因此只需要考虑 $V^{\otimes n}$ 的一组基的映射就可以了,下同。

\begin{example}{}
考虑 $n = 2$,$S_2 = \mathbb{Z}/2\mathbb{Z} = \{e, (1 2)\}$,$\rho(e)$是恒等映射,而
\begin{equation}
\begin{aligned}
\rho((1 2)): V \otimes V &\to V \otimes V~, \\
\sum v_1 \otimes v_2 &\mapsto \sum v_2 \otimes v_1~.
\end{aligned}
\end{equation}
\end{example}

特别的,我们把 $n$ 次张量幂 $V^{\otimes n}$ 的不动点集 $(V^{\otimes n})^{S_n}$ (\autoref{def_Group3_2}~\upref{Group3})称为 $V$ 的 $n$ 阶\textbf{对称幂},记做 $\opn{Sym}^n V$ 或者 $S^n(V)$。

\begin{example}{}\label{ex_vecSAS_1}
考虑 $V = \mathbb{R}^2$;$\opn{Sym}^2(V) = \langle e_1 \otimes e_1, e_2 \otimes e_2, e_1 \otimes e_2 + e_2 \otimes e_1 \rangle$ 是一个三维矢量空间。
\end{example}

我们定义(向量的)\textbf{对称积}
\begin{equation}
\begin{aligned}
\cdot: V \times V &\to V^{\otimes 2}~, \\
v \cdot w &:= \frac12 (v \otimes w + w \otimes v)~,
\end{aligned}
\end{equation}
因此\autoref{ex_vecSAS_1} 中 $\opn{Sym}^2(V)$ 的基向量可以写成 $e_1 \cdot e_1, e_2 \cdot e_2$ 和 $2 e_1 \cdot e_2$ (在基向量的意义下,系数不重要)。


注意:我们会说一个向量\textbf{空间} $V$ 的(反)对称\textbf{幂},以及两个\textbf{向量}的对称\textbf{积},有时我们会把二阶对称幂 $\text{Sym}^2 V$ 称为向量空间 $V$ 的对称积,但是这并不严谨。

更一般的,我们可以定义多项对称积
\begin{equation}
\begin{aligned}
\cdot: V^{\times n} &\to V^{\otimes n}~, \\
v_1 \cdots v_n &:= \frac{1}{n!} \sum_{\sigma \in S_n} v_{\sigma(1)} \otimes \dots \otimes v_{\sigma(n)}~,
\end{aligned}
\end{equation}
比如当 $n = 3$ 时,
\begin{equation}
\begin{aligned}
v_1 \cdot v_2 \cdot v_3 = &\frac16 (v_1 \otimes v_2 \otimes v_3 \\
&+ v_1 \otimes v_3 \otimes v_2 \\
&+ v_2 \otimes v_1 \otimes v_3 \\
&+ v_2 \otimes v_3 \otimes v_1 \\
&+ v_3 \otimes v_1 \otimes v_2 \\
&+ v_3 \otimes v_2 \otimes v_1)~,
\end{aligned}
\end{equation}

可以证明,$v_1 \cdot v_2 \cdots = v_1 \cdot v_3 \cdot v_2 \cdots = \cdots$ 在 $S_n$ 的置换作用下固定,这意味着$v \cdots w \in \text{Sym}^n V$;

取 $V$ 的一组基 $\{e_1, \dots, e_k\}$,可以找到 $\opn{Sym}^n V$ 的一组基
\begin{equation}
\left\{ e_{i_1} \cdot \dots \cdot e_{i_n} \mid 1 \leq i_1 \leq \dots \leq i_n \leq k \right\}~,
\end{equation}
特别的,$\dim(\opn{Sym}^n V) = \pmat{k + n - 1\\n}$(隔板法\upref{BarCom})。

\subsubsection{反对称幂(又称交错幂、外幂)}

张量幂 $V^{\otimes n}$ 上存在另一个 $S_n$ \upref{Perm}的群作用:
\begin{equation}
\begin{aligned}
\rho(\sigma): V^{\otimes n} &\to V^{\otimes n}~, \\
v_1 \otimes \dots \otimes v_n &\mapsto \sum \opn{sgn}(\sigma) v_{\sigma(1)} \otimes \dots \otimes v_{\sigma(n)}~.
\end{aligned}
\end{equation}
其中对于偶置换$\sigma = 1$、奇置换$\sigma = -1$;我们把它的不动点集称为 $V$ 的 $n$ 阶\textbf{反对称幂},记做 ${\large \wedge}^n V$。

\begin{example}{}\label{ex_vecSAS_2}
考虑 $V = \mathbb{R}^2$;${\large \wedge}^2 V = \langle e_1 \otimes e_2 - e_2 \otimes e_1 \rangle$ 是一个一维矢量空间。
\end{example}

我们定义(向量的)\textbf{反对称积}(或称\textbf{交错积}、\textbf{外积})
\begin{equation}
\begin{aligned}
\wedge: V \times V &\to V^{\otimes 2}~, \\
v \wedge w &:= v \otimes w - w \otimes v~,
\end{aligned}
\end{equation}
因此\autoref{ex_vecSAS_2} 中 ${\large \wedge}^2 V$ 的基向量可以写成 $e_1 \wedge e_2$。

反对称积 $\wedge: V \times V \to V^{\otimes 2}$ 满足,
\begin{itemize}
\item $v \wedge w = - w \wedge v$,特别的,
\item $v \wedge v = 0$;
\end{itemize}
这意味着$v \wedge w \in {\large \wedge}^2 V$。

更一般的,我们可以定义多项反对称积
\begin{equation}
\begin{aligned}
\cdot: V^{\times n} &\to V^{\otimes n}~, \\
v_1 \wedge \cdots \wedge v_n &:= \sum_{\sigma \in S_n} \text{sgn}(\sigma) v_{\sigma(1)} \otimes \dots \otimes v_{\sigma(n)}~,
\end{aligned}
\end{equation}



取 $V$ 的一组基 $\{e_1, \dots, e_k\}$,可以找到 ${\large \wedge}^n V$ 的一组基
\begin{equation}
\left\{ e_{i_1} \wedge \dots \wedge e_{i_n} \mid 1 \leq i_1 < \dots < i_n \leq k \right\}~,
\end{equation}
特别的,$\dim({\large \wedge}^n V) = \pmat{k \\ n}$。

\subsubsection{张量积空间的对称反对称分解}

\subsubsection{作为商空间的对称/反对称幂}