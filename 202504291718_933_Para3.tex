% 抛物线(高中)
% keys 极坐标系|直角坐标系|圆锥曲线|抛物线
% license Xiao
% type Tutor

\begin{issues}
\issueDraft
\end{issues}

\pentry{解析几何\nref{nod_JXJH},圆\nref{nod_HsCirc}}{nod_7c17}

在没学过之前会下意识地觉得双曲线的一支是抛物线。他们形态上很相像。

抛物线的反射性质特别适合于将所有来自焦点的信号反射成一条平行光束(或将平行光束集中到焦点)。这正是雷达天线、卫星天线等设备需要的特性:发射时集中方向,接收时集中能量。

1. 初步认识
	•	抛物线的直观形象(如水的抛射轨迹、抛物面天线)
	•	抛物线在自然界和工程中的应用举例
\subsection{抛物线的定义}
标准定义:平面上到定点(焦点)和定直线(准线)距离相等的点的轨迹
这就是 “\enref{圆锥曲线的极坐标方程}{Cone}” 中对抛物线的定义。
\begin{figure}[ht]
\centering
\includegraphics[width=4.2cm]{./figures/c89771dd2fef516e.pdf}
\caption{抛物线的定义} \label{fig_Para3_1}
\end{figure}

在 $x$ 轴正半轴作一条与准线平行的直线 $L$, 则抛物线上一点 $P$ 到其焦点的距离 $r$ 与 $P$ 到 $L$ 的距离之和不变。

如\autoref{fig_Para3_1}, 要证明由焦点和准线定义的抛物线满足该性质, 只需过点 $P$ 作从准线到直线 $L$ 的垂直线段 $AB$, 由于 $r$ 等于线段 $PA$ 的长度, 所以 $r$ 加上 $PB$ 的长度等于 $AB$ 的长度, 与 $P$ 的位置无关。 证毕。


\subsection{抛物线的方程}
\begin{theorem}{抛物线的标准方程}

\end{theorem}
	•	顶点在原点,轴为 $y$ 轴的标准式:$x^2=2py$
	•	讨论参数 $p$ 的意义(焦点到顶点的距离)
\begin{theorem}{抛物线的参数方程}
	•	用参数表示抛物线上的点,如 $x=pt^2,,y=2pt$ 等(视教学安排可选讲)
\end{theorem}

\subsection{抛物线的几何性质}
	•	对称性(关于轴对称)
	•	顶点、焦点、准线的定义和关系
	•	开口方向与参数正负有关
	•	通用式推导(顶点在 $(h,k)$ 时的方程)
    反射性质(光线从焦点发出反射后平行于轴)
\subsubsection{切线}
	•	给定抛物线方程和点,求切线方程
	•	切线的几何意义(过点,与焦点、准线的关系)
