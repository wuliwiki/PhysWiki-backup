% 光学系统基本概念
% license Usr
% type Tutor

\pentry{几何光学基本定律\nref{nod_GeOp2}}{nod_0e67}

\begin{definition}{光学系统}
由基本光学元件构成的系统。
\end{definition}

\begin{definition}{基本光学元件}
分为物理光学元件(光栅、偏振片、波片等)和几何光学元件(透镜、反射镜和平面镜等)。
\end{definition}

\begin{definition}{光轴、共轴光学系统、子午面、共轴球面光学系统}
如果一个光学系统有关于一条轴线旋转对称的性质,即一条公共轴线通过各个光学元件的曲率中心,则这条轴线叫做光学系统的光轴,这样的光学系统称为共轴光学系统。过光轴的任一平面称为子午面。如果构成共轴光学系统的各光学元件表面均为球面或平面,这样的光学系统称为共轴球面光学系统。
\end{definition}

\begin{definition}{物点、物、实物点、实物、虚物点、虚物}
进入光学系统的光必然来自于任一物体上一点,这一点我们称为物点,所有物点的集合构成了物。

如果来自于某一物点的光束经过光学系统后,它们的反向延长线能够交于一点,则将其称为实物点,所有实物点的集合构成了实物。

如果来自于某一物点的光束经过光学系统后,它们的延长线能够交于一点,则将其称为虚物点,所有虚物点的集合构成了虚物。
\end{definition}

\begin{definition}{完善像点、完善像、实像点、实像、虚像点、虚像}
物点发出的同心光束经过光学系统后仍为同心光束,我们就说光学系统对该物点形成了一个完善的像点。

如果光学系统对构成物的所有物点均形成了完善像点,我们就说光学系统对物形成了完善像。

如果离开光学系统的光束为会聚的同心光束,则将光束会聚的点称为实像点,所有实像点的集合构成了实像。

如果离开光学系统的光束为发散的同心光束,则将光束反向延长线会聚的点称为虚像点,所有虚像点的集合构成了虚像。
\end{definition}

\begin{definition}{物空间、像空间}
光学系统中物所在的空间称为物空间,像所在的空间称为像空间。
\end{definition}
