% 反常霍尔效应
% 晶体|电子|霍尔效应
\pentry{电子运动的准经典模型\upref{cryele}}
\begin{issues}
\issueDraft
\end{issues}
一般霍尔效应的产生需要磁场,并且满带不出现霍尔效应。但是反常霍尔效应不需要这些条件。
\subsection{介绍}
我们知道一个布洛赫态可以写成:
\begin{equation}
\psi_{n,\boldsymbol{k}}=e^{i\boldsymbol{k}\cdot\boldsymbol{r}}u_{n,\boldsymbol{k}}
\end{equation}
其哈密顿量为:$\widehat{H_0}=-\frac{h^2}{2m}\nabla^2+V(\boldsymbol{r})$,对应的能量是$E_{n,\boldsymbol{k}}$。其中$V(\boldsymbol{r})$是一个周期函数,有$V(\boldsymbol{r}+\boldsymbol{R})=V(\boldsymbol{r})$,$\boldsymbol{R}$是任意一个格矢。

外力作用下哈密顿量变成$\widehat{H}=\widehat{H_0}-\boldsymbol{F}\cdot\boldsymbol{r}$,则dt时间后,布洛赫态变成:
\begin{equation}
%\psi(\boldsymbol{r},dt)=e^{-\frac{i\,\widehat{H}\,dt}{\hbar}}\psi_{n,\boldsymbol{k}}\approx \psi_{{n,\boldsymbol{k}}-%\frac{idt}{\hbar}(\widehat{H_0}-\boldsymbol{F}\cdot\boldsymbol{r})\psi_{{n,\boldsymbol{k}}
11
\end{equation}

