% 双曲线(高中)
% keys 极坐标系|直角坐标系|圆锥曲线|双曲线|渐近线
% license Xiao
% type Tutor

\begin{issues}
\issueDraft
\end{issues}

\pentry{解析几何\nref{nod_JXJH},圆\nref{nod_HsCirc},椭圆\nref{nod_Elips3},复数\nref{nod_CplxNo}}{nod_fd80}

在发电厂中,有一种高大耸立的冷却装置,叫作“冷却塔”(见 \autoref{fig_Hypb3_3} 右侧)。乍一看,它像是两个圆台底朝底叠在一起,但仔细看会发现,它的边缘并非直线,而是略微弯曲的曲线。而与圆台相似,冷却塔的外形也通过一条特定的曲线绕着一条轴旋转得到。它之所以设计成这样,并不单单是为了好看,而是因为这样可以更好地抵抗大风、减少材料使用,还能提高散热效率。而这些效果,其实都和这条曲线在数学上的一些特殊性质有关。

\begin{figure}[ht]
\centering
\includegraphics[width=8cm]{./figures/9d8f8fdb10436630.png}
\caption{冷却塔} \label{fig_Hypb3_3}
\end{figure}

在数学中,这类曲线早就被深入研究过。它由两条对称、互不相交的分支组成,虽然共用同一个坐标原点,但却分别朝着相反的方向无限延伸,永远不会相遇。这种曲线被称为“双曲线”。很多读者第一次听到“双曲线”这个词,可能是在初中学习反比例函数的时候。当时常听到一句话:“反比例函数的图像是双曲线。”那时候的理解可能还很简单,只是觉得图像上有“两根”弯曲的线条,所以叫“双曲线”。至于它更深层的结构和性质,课堂上一般不会讲太细,学生自然也无从深入探究。

不过,相信读者还记得,反比例函数的图像有两条渐近线。那么,既然它是双曲线的图像,就可以合理猜想:双曲线也应该有两条渐近线。这个猜想是对的。虽然在高中阶段,“渐近线”这个概念不常被强调,但它其实在双曲线中非常重要,前面所说冷却塔的特性就是基于ta。双曲线的两个分支会越来越靠近各自的渐近线,但始终不会真正相交。

本文在介绍双曲线的过程中,也会进一步分析它和渐近线之间的关系,帮助大家更全面地理解这种特殊的曲线。



不过,相信读者记得,反比例函数有两条渐近线,而既然反比例函数的图像是双曲线,那么有理由猜想双曲线应该也有两条渐近线。答案是是的,尽管高中阶段对渐近线的讲解较为有限,但这一概念在双曲线中却具有核心意义:双曲线的两个分支会无限接近各自的渐近线,却始终不会与之相交。本文在介绍双曲线的过程中,将进一步探讨其与渐近线之间的关系,以拓展对这一曲线的理解和认识。

\subsection{“反比例函数的图像是双曲线”}

如前面所说,“反比例函数的图像是双曲线”。既然双曲线是比较陌生的概念,下面就从熟悉的反比例函数入手,来仔细研究一下这件事。

\begin{example}{求反比例函数$\displaystyle y={1\over x}$的图像逆时针旋转$45^\circ$后的表达式。}\label{ex_Hypb3_1}
根据\aref{图像旋转的规律}{sub_FunTra_3},将$\displaystyle y={1\over x}$逆时针旋转$45^\circ$,即$\displaystyle\theta=-{\pi\over4}$。
\begin{equation}\label{eq_Hypb3_2}
\begin{cases}
\displaystyle
X_0&=X_1 \cos \left(-{\displaystyle\frac{\pi}{4}}\right) + Y_1 \sin  \left(-{\displaystyle\frac{\pi}{4}}\right)\\
Y_0&=Y_1 \cos  \left(-{\displaystyle\frac{\pi}{4}}\right) - X_1 \sin  \left(-{\displaystyle\frac{\pi}{4}}\right)\\
\end{cases}\implies
\begin{cases}
X_0&={\displaystyle\frac{1}{\sqrt{2}}}(X_1 - Y_1)\\
Y_0&={\displaystyle\frac{1}{\sqrt{2}}}(Y_1+ X_1) \\
\end{cases}~.
\end{equation}
将\autoref{eq_Hypb3_2} 代入$xy=1$有:
\begin{equation}
\displaystyle
{1\over\sqrt{2}}(X_1 - Y_1)\cdot{1\over\sqrt{2}}(Y_1+ X_1)=1\implies {X_1^2\over2}- {Y_1^2\over2}=1~.
\end{equation}
即旋转后的方程是:
\begin{equation}\label{eq_Hypb3_5}
{x^2\over2}- {y^2\over2}=1~.
\end{equation}
\end{example}

\addTODO{旋转图像}

由于旋转变换不会改变图形的形状,因此,原本反比例函数图像的两条渐近线——$x$轴和$y$轴,在经过旋转后,会变为 $y = \pm x$。换句话说,旋转后的图像虽然向两个方向无限延伸,却始终不会越过 $y = \pm x$ 这两条直线所构成的“边界”。

再来看 \autoref{eq_Hypb3_5},这个式子的形式可能会让人感到熟悉。它与半径为 $\sqrt{2}$ 的圆方程极为相似,唯一的差别在于 $y^2$ 前的符号不同。这个细微的差异不禁让人联想到椭圆:椭圆的方程与标准圆的方程之间,同样只是系数上的区别。然而,双曲线的图像与圆或椭圆在外形上差别如此之大,甚至都不是封闭曲线,它们之间会存在什么联系呢?

\subsection{双曲线的几何定义}

在研究椭圆时曾提到,若从圆的定义出发,尝试进行推广,就需要改变原有定义的表达方式。对 $|O_1P| = |O_2P| = r$。其中第二个等号可以被“打开”为垂直平分线的集合,而第一个等号打开时,则将 $|O_1P| = r, |O_2P| = r$ 作为初始条件,而不要求两者始终相等。若将两个距离的和固定,即 $|O_1P| + |O_2P| = m$,便得到了椭圆。那么,如果考虑的是两点间距离的差,即 $|O_1P| - |O_2P| = 0$,这看似回到了 $|O_1P| = |O_2P|$ 的情形,但如果进一步类比椭圆的定义,将这个差固定为一个非零常数 $m$,即$\left||O_1P| - |O_2P|\right| = m$将会得到什么样的图形?

\begin{example}{对两定点 $F_1(c, 0)$ 和 $F_2(-c, 0),(c>0)$,若点$P$满足$|PF_1| - |PF_2| = m,(0<m <2c)$,求$P$方程。}
解:

设椭圆上的任意点为 $P(x, y)$,根据题意有:
\begin{equation}
\sqrt{(x + c)^2 + y^2} - \sqrt{(x - c)^2 + y^2} = m~.
\end{equation}
移项后,两边平方有:
\begin{equation}
(x + c)^2 + y^2 = m^2 + 2m\sqrt{(x - c)^2 + y^2} + (x - c)^2 + y^2~.
\end{equation}
打开整理有:
\begin{equation}
2m\sqrt{(x - c)^2 + y^2}= 4cx - m^2~.
\end{equation}
两边平方,打开有:
\begin{equation}\label{eq_Hypb3_6}
4m^2(x^2 - 2cx+c^2) + 4m^2y^2= m^4-4m^2\cdot2cx+16c^2x^2~.
\end{equation}
整理后得到:
\begin{equation}
4(m^2 -4c^2)x^2 + 4m^2y^2= m^2(m^2-4c^2)~.
\end{equation}
两侧同时除以$(m^2-4c^2)m^2$后得到:
\begin{equation}\label{eq_Hypb3_3}
\frac{x^2}{\left(\displaystyle\frac{m}{2}\right)^2} -\frac{y^2}{\displaystyle c^2-\left(\frac{m}{2}\right)^2}=1~.
\end{equation}
由于$2c>m>0$,也就是$\displaystyle c^2-\left(\frac{m}{2}\right)^2>0$。
\end{example}

关于\autoref{ex_Hypb3_1} 有几点需要特别注意:
\begin{itemize}
\item 虽然题目将不变量由椭圆中的“距离和”变为了“距离差”,但对比 \aref{椭圆的推导过程}{ex_Elips3_1} 可以发现,推导的过程和从\autoref{eq_Hypb3_6} 开始的结果其实完全一致。唯一的区别在于参数大小关系的变化:原本椭圆中为 $m > 2c$,而在这里变为 $m < 2c$。为了使分母保持正值,这一变化最终导致\autoref{eq_Hypb3_3} 与\aref{椭圆的推导结果}{eq_Elips3_4}在形式上有所差异。
\item 题目中隐含了一个默认条件,即 $|PF_1| > |PF_2|$,这使得最终得到的表达式只对应图像的右半部分,即 $x > 0$ 的区域。如果将条件改为 $||PF_1| - |PF_2|| = m$,其中 $0 < m < 2c$,则会对应完整的 \autoref{eq_Hypb3_3} 表达式,图像也将呈现左右对称的完整双曲线。
\end{itemize}

可以看到,推导得到的 \autoref{eq_Hypb3_3} 与 \autoref{eq_Hypb3_5} 在形式上完全一致,这正是“双曲线”的标准代数表达式。而题目所给出的,正是双曲线的几何定义,也被称为双曲线的第一定义。

\begin{definition}{双曲线的几何定义}
在平面上,所有满足到两个定点 $F_1$ 与 $F_2$ 的距离之差的绝对值为常数 $2a$ 的点 $P$ 的轨迹,构成一个几何图形,称为\textbf{双曲线(hyperbola)}。即,对于双曲线上的任意一点 $P$,都有:
\begin{equation}
||PF_1| - |PF_2|| = 2a ,\qquad(2a<|F_1F_2|)~.
\end{equation}
其中,$F_1$ 和 $F_2$ 被称为双曲线的两个\textbf{焦点(focus)},两焦点之间的距离 $|F_1F_2|$ 称为椭圆的\textbf{焦距(focal distance)},记作 $2c$,其中 $c$ 被称为\textbf{半焦距(semi-focal distance)},也称为双曲线的\textbf{线性离心率(linear eccentricity)}。
\end{definition}

此外,从 \autoref{ex_Hypb3_1} 中还可以进一步得到以下结论:

\begin{itemize}
\item 若将两个定点设为 $F_1(0,-c)$ 和 $F_2(0,c)$,其中 $c > 0$,相当于将原来的 $x$ 轴和 $y$ 轴互换,得到的表达式为:
\begin{equation}\label{eq_Hypb3_7}
\frac{y^2}{\left(\displaystyle\frac{m}{2}\right)^2}-\frac{x^2}{\displaystyle c^2-\left(\frac{m}{2}\right)^2} =1~.
\end{equation}
与椭圆不同的是,这里不仅分母位置发生了变化,符号也要随之调整。
\item 对于 \autoref{eq_Hypb3_3},当令 $y = 0$ 时,得到 $x$ 轴上的两个交点$\displaystyle x = \pm \frac{m}{2}$,而当令 $x = 0$ 时,方程在实数范围内无解,但在复数范围内的解为:$y = \displaystyle\pm \I \sqrt{c^2 - \left( \frac{m}{2} \right)^2}$。这一情况与椭圆存在显著区别:标准椭圆图像与坐标轴有四个交点,而该双曲线仅与 $x$ 轴相交于两点,与 $y$ 轴无实数交点。
\end{itemize}

为了便于记录,同时与椭圆的表示方式统一,规定符号为正的参数记作 $a^2$,符号为负的参数分母记作 $b^2$。需要注意的是,这里 $a$ 与 $b$ 的定义与椭圆中的选取略有不同,并未要求二者的大小关系。根据这一约定,有:
\begin{equation}
\left(\displaystyle\frac{m}{2}\right)^2=a^2\qquad c^2-\displaystyle\left(\frac{m}{2}\right)^2=b^2~.
\end{equation}
整理后可得:
\begin{equation}
m=2a\qquad a^2+b^2=c^2~.
\end{equation}

是的,熟悉的平方和形式再次出现了,不过这一次,参数的位置发生了变化,这一点需要特别注意。

结合前面对实解与虚解的讨论,这里也引入“\textbf{实轴(transverse axis)}”与“\textbf{虚轴(conjugate axis)}”的概念\footnote{注意区分这里与复平面的“实轴”和“虚轴”的区别。}。双曲线与坐标轴的交点称为\textbf{顶点(vertex)},连接两个顶点的线段称为\textbf{实轴(transverse axis)},长度为 $2a$,其中 $a$ 被称为\textbf{半实轴(semi-transverse axis)}。就像椭圆的焦点始终位于长轴上一样,双曲线的焦点也始终位于实轴上。

将复数解中虚部所对应的位置标在另一条坐标轴上,得到\textbf{虚顶点(co-vertex)},这两个虚顶点之间的连线称为\textbf{虚轴(conjugate axis)},长度为 $2b$,其中 $b$ 被称为\textbf{半虚轴(semi-conjugate axis)}。

许多学生在学习至此时常感到疑惑:既然是“虚”的,为什么还要在图像中标出?虚轴到底有什么意义?事实上,这些虚轴和虚顶点在复变函数的理论中具有重要作用。当函数的定义域拓展至复数范围时,它们对应的图像结构就会真实展现出来。不过由于高中阶段不涉及复变函数,这里仅作简单介绍,以满足读者的好奇心。

\begin{figure}[ht]
\centering
\includegraphics[width=14.2cm]{./figures/457f263feb62b8e9.png}
\caption{复数域下的双曲线} \label{fig_Hypb3_2}
\end{figure}

\subsection{双曲线的方程}

从\autoref{eq_Hypb3_3} 和\autoref{eq_Hypb3_7} 下手,整理代换后可以得到双曲线的标准方程:

\begin{theorem}{双曲线的标准方程}
\begin{itemize}
\item 实轴在$x$轴上,虚轴在$y$轴上的双曲线方程为:
\begin{equation}\label{eq_Hypb3_4}
\frac{x^2}{a^2} - \frac{y^2}{b^2} = 1~.
\end{equation}
\item 实轴在$y$轴上,虚轴在$x$轴上的双曲线方程为:
\begin{equation}
\frac{y^2}{b^2} -\frac{x^2}{a^2}  = 1~.
\end{equation}
\end{itemize}
\end{theorem}



大多数情况都会认为双曲线是开放的或者说不封闭的,不过转换一下视角,想象一下,当视角来到无穷远处,如果把它们两支相互对称的地方连起来,似乎就形成了一个封闭的环,只不过这个环太大了。不知道你看到双曲线时是否会有这样的想象,他们要是能连起来就太好了。其实,在射影几何的视角下,它们正是在无穷远处相交。而这也带来了非常重要的视角。

\begin{theorem}{双曲线的参数方程}

\end{theorem}

\subsection{渐近线}
\begin{figure}[ht]
\centering
\includegraphics[width=4.8cm]{./figures/eb70b650d9fa932b.pdf}
\caption{双曲线的渐近线} \label{fig_Hypb3_1}
\end{figure}
\textbf{渐近线(asymptotes)}

当 $x,y$ 都无穷大时, \autoref{eq_Hypb3_4} 中的 $1$ 可以忽略不计,有 $y/x = \pm b/a$,\enref{渐近线}{Asmpto}与 $x$ 轴夹角为
\begin{equation}\label{eq_Hypb3_1}
\theta_0 = \arctan(b/a)~.
\end{equation}
两条渐近线到两个焦点的距离都为
\begin{equation}\label{eq_Hypb3_11}
c\sin\theta_0 = c\cdot b/c = b~.
\end{equation}


事实上这么推导渐近线并不严谨, 在学习了高数的相关内容(见“\enref{泰勒展开}{Taylor}”)后,由\autoref{eq_Hypb3_4} 得
\begin{equation}
y = \frac{bx}{a} \sqrt{1-\frac{a^2}{x^2}}~.
\end{equation}
把根号部分关于 $a^2/x^2$ 进行泰勒展开, 有
\begin{equation}\label{eq_Hypb3_13}
y = \frac ba x - \frac{ab}{2x} + \order{\frac{1}{x^3}}~.
\end{equation}
所以当 $x\to\infty$ 时, 就有渐进线 $y = bx/a$。 之所以要这样做, 是为了防止\autoref{eq_Hypb3_13} 右边出现常数项。 如果存在常数项 $\lambda$, 那么双曲线的渐近线就是 $y = bx/a + \lambda$ 了。

\subsection{双曲线的性质}
面积









