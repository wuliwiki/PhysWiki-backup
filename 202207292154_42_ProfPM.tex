% 计数原理
% keys 加法原则|乘法原则

\pentry{集合\upref{Set}}
本节介绍组合学里两个一般性的原则——加法原则和乘法原则.

以下假设 $A$ 和 $B$ 是两类不同、互不关联的事件.
\subsection{加法原则}
\textbf{加法原则:}设事件 $A$ 有 $m$ 种选取方式,事件 $B$ 有 $n$ 种选取方式,则选 $A$ 或 $B$ 共有 $m+n$ 种方式.

用集合\upref{Set}的语言可将加法原则描述成如下定理(\autoref{CardiN_the1}~\upref{CardiN}):
\begin{theorem}{}
设 $A,B$ 为有限集,且 $A\cap B=\emptyset$ ,则
\begin{equation}
\abs{A\cup B}=\abs{A}+\abs{B}
\end{equation}
\end{theorem}
\begin{corollary}{}
设 $n$ 个有限集合 $A_1,\cdots,A_n$ 满足
\begin{equation}
A_i\cap A_j=\emptyset,1\leq i\neq j\leq n
\end{equation}
则
\begin{equation}
\abs{\bigcup_{i=1}^n A_i}=\sum_{i=1}^n\abs{A_i}
\end{equation}

\end{corollary}
\subsection{乘法原则}
\textbf{乘法原则:}设事件 $A$ 有 $m$ 种选取方式,事件 $B$ 有 $n$ 种选取方式,那么选取 $A$ 以后再选取 $B$ 共有 $m\cdot n$ 种方式.

同样,用集合的语言可将乘法原则描述成如下的定理:
\begin{theorem}{}
设 $A,B$ 为有限集,$\abs{A}=m,\abs{B}=n$, 则
\begin{equation}
\abs{A\times B}=\abs{A}\times\abs{B}=m\cdot n
\end{equation}

\end{theorem}
