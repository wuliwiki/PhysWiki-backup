% 功和机械能(高中)
% 功|机械能|势能|动能|能量守恒定律

\pentry{牛顿运动定律\upref{HSPM03}}

\subsection{功}

在直线运动中, 如果一个力 $F$ 作用在物体上, 且物体在这个力的方向上发生了位移 $s$, 就说这个力对物体做了\textbf{机械功}(简称\textbf{功})。功常用字母 $W$ 表示。做功的过程是能量变化的过程,功是能量变化的量度。

在国际单位中,功的单位是\textbf{焦耳},简称\textbf{焦},符号为 $\mathrm{J}$,$1\mathrm{J}=1\mathrm{N \cdot m}$。

\subsubsection{恒力对物体做的功}

当力的方向与物体位移的方向相同时, 功的大小等于力的大小与位移大小的乘积,即:
\begin{equation}\label{HSPM07_eq1}
W=Fs
\end{equation}

若拓展到一般情况, 力和位移都是矢量, 功是力和位移的点乘(内积)\upref{Dot} $W = \bvec F \vdot \bvec s$。 因此功是一个标量。 由于点乘满足分配律, 当物体在多个力的作用下发生了一段位移,它们对物体所做的总功等于各个力对物体所做的功的代数和, 也等于合力在该位移下做的功
\begin{equation}
W = (\bvec F_1 + \bvec F_2 + \dots)\vdot \bvec s = \bvec F_1 \vdot \bvec s + \bvec F_2 \vdot \bvec s + \dots
\end{equation}

当力的方向与物体位移的方向的夹角为 $\theta$ 时,对 $\bvec F$ 进行正交分解可得 $\bvec F$ 沿位移方向的分力 $F_1=F\cos \theta$,垂直于位移方向的分力 $F_2=F\sin \theta$。易知物体在分力 $F_1$ 的方向上发生了位移 $\bvec s$,在 $F_2$ 的方向上没有位移,$\bvec F$ 对物体做功大小为:
\begin{equation}\label{HSPM07_eq2}
W=Fs\cos \theta
\end{equation}

\begin{itemize}
\item 当 $\theta = 0$ 时,\autoref{HSPM07_eq2} 可化简为\autoref{HSPM07_eq1} 的形式。
\item 当 $0\leq \theta < \pi/2$ 时,$W>0$,力 $\bvec F$ 为动力,推动物体的运动,对物体做\textbf{正功}。
\item 当 $\pi/2< \theta \leq \pi$ 时,$W<0$,力 $\bvec F$ 为阻力,阻碍物体的运动,对物体做\textbf{负功}。
\item 当 $\theta = \pi/2$ 时,$W=0$,力 $\bvec F$ 对物体不做功。
\end{itemize}

要注意的是,对于功的计算式,位移必须是在力在作用过程中发生的。

\subsubsection{变力对物体做的功}

对于变力做功,需要根据实际情况,选择不同的方法,此处列举几个常用的方法:

\begin{enumerate}
\item 分解成多个恒力做功的阶段,分别计算再求代数和。

\item 图像法:已知力—位移图像(力和位移在同一直线上)时,曲线与 $x$ 轴上方围成的面积为正功,与 $x$ 轴下方围成的面积为负功,高中阶段适用于规则几何图形及割补法可计算的情况。

\item 求平均力:如果力的方向不变,其大小随位移均匀变化,可求出物体所受的平均力,进而用\autoref{HSPM07_eq2} 求解,这种情况也可从用图像法计算三角形面积的情况推出。

\item 微元法:如一个物体在一个大小不变的拉力 $\bvec F$ 下做半径为 $r$ 的匀速圆周运动,力的方向和运动方向始终一致,总功等于在无数段极小的位移上恒力 $\bvec F$ 做功的和,所有极小位移的大小之和为 $\Delta s_1+\Delta s_2+\dots+\Delta s_n=2\pi r$,则 $W=F\Delta s_1+F\Delta s_2+\dots+F\Delta s_n=2\pi rF$。

\item 已知恒定功率或平均功率以及做功时间,用\autoref{HSPM07_eq3} 求解。

\item 利用动能定理\autoref{HSPM07_eq6} 列式求解,该方法适用范围较广,常为计算变力做功的首选。
\end{enumerate}

\subsection{功率}

功与其对应做功所用时间之比叫做\textbf{功率}。功率是表示做功快慢的物理量,常用字母 $P$ 表示,其定义式为:
\begin{equation}\label{HSPM07_eq3}
P=\frac{W}{t}
\end{equation}

在国际单位中,功率的单位是\textbf{瓦特},简称\textbf{瓦},符号为 $\mathrm{W}$。

\textbf{额定功率}:机械设备长时间正常工作时所能达到的最大输出功率,对某个设备来说其额定功率是一定的。

\textbf{实际功率}:机械设备工作时实际的输出功率。

联立\autoref{HSPM07_eq2}、 \autoref{HSPM07_eq3} 和 $v=s/t$,可得功率与速度的关系:
\begin{equation}\label{HSPM07_eq4}
P=Fv\cos \theta
\end{equation}

研究功率的问题时,通常用\autoref{HSPM07_eq3} 计算平均功率,用\autoref{HSPM07_eq4} 计算瞬时功率。

\subsection{动能}

物体由于运动而具有的能量叫做\textbf{动能},用符号 $E_k$ 表示\footnote{$E_k$ 中的 $k$ 表示 kinetic。}。

质量为 $m$ 的物体,在速度大小为 $v$ 时所具有的动能为:
\begin{equation}\label{HSPM07_eq5}
E_k=\frac12mv^2
\end{equation}

注意:运动具有相对性,因此物体的动能与参考系有关。

因为 $\mathrm{1kg\cdot (m/s)^2=1(kg\cdot m/s^2)\cdot m=1N\cdot m=1J}$,由\autoref{HSPM07_eq5} 可知动能的单位与功的单位相同。

\subsection{动能定理}

物体的运动过程中,合外力对物体所做的功 $W$ 等于物体在这个过程中动能的变化量 $\Delta E_k$。设物体的初动能为 $E_{k1}$,末动能为 $E_{k2}$ 则:
\begin{equation}\label{HSPM07_eq6}
W=\Delta E_k =E_{k2}-E_{k1}
\end{equation}

推导:设质量为 $m$ 的物体,在水平方向受到恒定的合外力 $\bvec F$,以 $\bvec v_1$ 的速度开始做匀变速直线运动,加速到 $\bvec v_2$。根据匀变速直线运动规律(\autoref{HSPM01_eq4}~\upref{HSPM01})、牛顿第二定律(\autoref{HSPM03_eq1}~\upref{HSPM03})和功的计算(\autoref{HSPM07_eq1}) 可得:
\begin{equation}
W=Fs=ma\frac{v_2^2-v_1^2}{2a}=\frac12mv_2^2-\frac12mv_1^2
\end{equation}

动能定律具有普遍性,除了上述推导中恒力做功、直线运动的情况,同样适用于变力做功、曲线运动的情况,当确定了物体运动过程的初末状态后,使用动能定理分析问题会较为方便。

\subsection{势能}\label{HSPM07_sub1}

\textbf{势能}也叫\textbf{位能},是与相互作用的物体的相对位置有关的能量,用符号 $E_p$ 表示\footnote{$E_p$ 中的 $p$ 表示potential。}。

\subsubsection{重力势能}

物体受到重力并处于一定高度时具有的能量叫做\textbf{重力势能}。

质量为 $m$ 的物体,在高度为 $h$ 处所具有的重力势能为:
\begin{equation}\label{HSPM07_eq7}
E_p=mgh
\end{equation}

物体的重力势能与参考平面(重力势能为零的平面)的选取有关,通常选地面或运动最低处为参考平面。

质量为 $m$ 的物体在重力作用下,从高度为 $h_1$ 的位置运动至高度为 $h_2$ 的位置,位移大小为 $s$,位移与重力的夹角为 $\theta$,在此过程中,重力做功为:
\begin{equation}
W_G=mgs\cos \theta=mg(h_1-h_2)=mgh_1-mgh_2
\end{equation}

根据\autoref{HSPM07_eq7} 物体在 $h_1$ 和 $h_2$ 处的重力势能分别为 $E_{p1}=mgh_1$ 和 $E_{p2}=mgh_2$,从 $h_1$ 到 $h_2$ 处,重力势能变化为 $\Delta E_p= E_{p2}-E_{p1}$,由此可得重力做功与重力势能的关系为:
\begin{equation}
W_G=E_{p1}-E_{p2}=-\Delta E_p
\end{equation}
由上式可知,重力做正功时,物体的重力势能减小;重力做负功时,物体的重力势能增大。

\subsubsection{弹性势能}

发生弹性形变的物体的各部分之间,由于有弹力的相互作用而具有的势能,叫做\textbf{弹性势能}。

弹性势能的大小也具有相对性,以弹簧为例,一般选取原长处为零势能点进行分析,对同一弹簧来说,从原长处伸长或压缩相同长度时其弹性势能大小相等。

弹性势能的变化与弹力做功有关,弹力做正功时,弹性势能减小;弹力做负功时,弹性势能增大。

\subsection{机械能守恒定律}

动能和势能统称为\textbf{机械能}。

在只有重力或弹力做功的系统内,系统的动能和势能可以互相转化,而总的机械能保持不变。表达式为:
\begin{equation}
E_{k1}+E_{p1}=E_{k2}+E_{p2}
\end{equation}

