% 中国科学院 2003 年考研普通物理 A 卷
% keys 中国科学院|考研|物理
% license Copy
% type Tutor
\begin{enumerate}
\item 滑雪运动员在一倾斜角为 $\alpha$ 的山破上滑雪,初速度为 $v_0$ ,并以与水平线成 $\theta$ 角跳出,如图1所示,不考虑空气阻力,求运动员落在斜坡上的最大距离。
\item 一质量为$m$的物体从质量为$M$的圆弧形槽顶端由静止滑下,设圆弧形槽的半经为$R$,张角为$\pi /2$。(如图2所示),如所有摩擦都可忽略。求:\\
(1)物体刚高开槽底端时,物体和槽的速度各为多少;\\
(2)在物体从$A$滑到$B$的过程中,物体对槽所的功;\\
(3)物体到达 $B $时对槽的压力。
\item 一根质量为$m$,长为  $l$的均匀细$OA$(如图3所示),可以绕通过其一端的光滑轴$O$在竖直平面内转动。今使棒从水平位置开始自由下摆,求细棒摆到垂直位置时,其中心点$C$和端点 $A$ 的速度。
\item 一均匀带电球壳的面电荷密度为 $\sigma$。利用能量守恒定律,求作用在球壳的单位面积上的静电力。
\item 如图5所示,三根平行直导线1,2,3在真空中相距为$a$,分别通有电流$I_1,I_2,I_3$,导线1和导线2中电流方向相同。试求每根导线单位长度所受的力。
\item 如图6所示,质量为$m$的小珠可沿半径为$r$的圆环形轨道运动,环面为水平面。小珠带有固定的正电荷$q$,设在以环形轨道的外同心圆环为其正截面的圆柱体内有均匀的随时间$t$变化的磁场,磁够应强度$B$垂直于环面。已知$t=0$时,$B=0$,小珠静止于环上;$0<t<T$时,$B$ 随时间线性地增长:$t=T$时,$B=B_0$,设重力和摩擦力可忽略。试求:在$0\ge t \le T$时间内,小珠的运动速度与时间关系及小珠对轨道的作用力。
\item 已知镍原子的基态是 $ ^{3}F_4$。\\
(1)问镍原子在斯特思一盖拉赫实验的不均匀横向磁场中将分裂为几束?简述理由。\\
(2)求基态镍原子的有效磁矩 $\mu_2$,(以破尔磁子$\mu_B$为单位)。
\item 在磁感应强度为 $0.5$ 特斯拉的磁场中,钙的$^{3}D_2 \to ^1 P_1  ,\lambda=732.6nm$的谱线在垂直于磁场方向观察将分裂成几条谱线?若相邻增线间距为$0.7$x$10^{10}Hz$,请计算一下电子的荷质比。
\end{enumerate}
