% 施勒米希-洛希余项公式
% 泰勒公式|余项

\pentry{泰勒公式\upref{Tayl}}
在用带有皮亚诺(Peano)余项的泰勒公式\autoref{Tayl_eq2}~\upref{Tayl}表达函数 $f(x)$ 时,得到的总是属于 “局部” 的性质,即关于点 $x_0$ 的性质的.若取其它数值 $x$ ,必须假定这些数值“充分接近”于 $x_0$,而不能任意选取.这是因为函数 $f(x)$ 与其n阶近似多项式 $p(x)$ 的差为一余项 $o((x-x_0)^n)$ ,而这余项仅仅说在 $x$ 充分接近 $x_0$ 时趋于0.而对于其它的 $x$,并不能保证近似多项式可以表达函数 $f(x)$ 至预先选定的准确度,而这往往是我们希望做到的.因此,我们需要推导余项的其它形式.

假定 $f(x)$ 在区间 $[x_0,x_0+H](H>0)$ 内定义着,并在该区间内有直至 $n$ 阶导数存在并连续,此外,至少在开区间 $(x_0,x_0+H)$ 内 $n+1$ 阶导数 $f^{(n+1)}(x)$ 存在且有限.函数在区间 $[x_0-H,x_0]$ 内定义时的情形,可类似说明.

若用 $n$ 阶多项式
\begin{equation}\label{SchRo_eq2}
p(x)=\sum_{i=0}^n\frac{f^{(i)}(x_0)}{i!}(x-x_0)^i
\end{equation}
作为 $f(x)$ 的近似公式,余项用 $r_n(x)$ 表示,即 
\begin{equation}\label{SchRo_eq3}
r_n(x)=f(x)-p(x)
\end{equation}
施勒米希-洛希(O.Schl$\ddot{o}$milch-Roche)推导出下面的公式
\begin{equation}\label{SchRo_eq1}
r_n(x)=\frac{f^{(n+1)}(x_0+\theta(x-x_0))}{n!p}\cdot(1-\theta)^{n+1-p}(x-x_0)^{n+1}\quad (0<\theta<1)
\end{equation}
其中,$p$ 为大于0的任一数.该式称为\textbf{施勒米希-洛希余项公式}.
\subsection{\autoref{SchRo_eq1} 的证明}
由\autoref{SchRo_eq2} ,\autoref{SchRo_eq3} 
\begin{equation}
r_n(x)=f(x)-\sum_{i=0}^n\frac{f^{(i)}(x_0)}{i!}(x-x_0)^i
\end{equation}
令 $x$ 固定为区间 $[x_0,x_0+H]$ 内任一数值,做辅助函数
\begin{equation}
\varphi(z)=f(x)-\sum_{i=0}^n\frac{f^{(i)}(z)}{i!}(x-z)^i
\end{equation}
