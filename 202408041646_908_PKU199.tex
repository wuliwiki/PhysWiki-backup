% 北京大学 1999 年 考研 普通物理
% license Usr
% type Note

\textbf{声明}:“该内容来源于网络公开资料,不保证真实性,如有侵权请联系管理员”

力学

一、(15分)

质量为$M$,半径为R的均质圆盘静止于粗糙水平面上,在离地面h处突然施以一水平打击,冲量为$F\Delta t$,如图。求:

1)打击后瞬间,圆盘质心的速度,圆盘的角速度以及圆盘接地点的速度。

2)$h$多大时,打击后圆盘作无滑动滚动?
\begin{figure}[ht]
\centering
\includegraphics[width=8cm]{./figures/062d0dd6a8d190aa.png}
\caption{} \label{fig_PKU199_1}
\end{figure}

二、(10分)}

已知某种双原子分子两原子之问的作用力可以表为
$$F = -\frac{a}{r^7} + \frac{b}{r^8}~$$
$$\begin{array}{c}
F \quad \longleftarrow \quad r \quad \longrightarrow \quad F \\
M \quad \quad \quad \quad M 
\end{array}~$$
其中$r$为两原子之问的距离,$a,b$为正常数

(1)求出其势能,两原子问的平衡距离$r_0$,并画出势能曲线示意图:

(2)不考感其他分子的作用,求两原子作小振动时的角频率,设两原子的质量均为$M$。

三、(10分)

半径为$R$的圆盘绕过其中心$0$的重直轴在水平面内以匀角速度$\omega_0$转动。。圆盘上有两条光滑细槽,一条沿半园弧$0A$,一条沿半径$0A$由于转动,两质点将由盘中心分别沿两滑槽滑向盘边(均由静止开始)求两质点到达盘边$A$点时
\begin{figure}[ht]
\centering
\includegraphics[width=8cm]{./figures/4e72883bdf1ee9a6.png}
\caption{} \label{fig_PKU199_2}
\end{figure}

电磁学

一、(15分)}

附图是由理想的导线、电阻器、线图组成的电路,电路中的初始电流为0:
\begin{figure}[ht]
\centering
\includegraphics[width=8cm]{./figures/fc13948c47e20cad.png}
\caption{} \label{fig_PKU199_3}
\end{figure}
(1)定性地描绘与说明,在开关闭合后,导线、电阻和线圈附近的电场
和磁场分布,及能流密度矢量的方向;

(2)定性地描绘与说明,在电路稳定后空间的电场、磁场和能流密度矢量的分布。

二、(10分)

附图是两个同心的导体球壳。内壳的外半径为$R_1$,其上有电量$Q$;外壳的内半径为$R_2$,其上有电量$-Q$.求外壳内部空间任何位置的电场的能量密度。
\begin{figure}[ht]
\centering
\includegraphics[width=8cm]{./figures/9f456c124254b4e9.png}
\caption{} \label{fig_PKU199_4}
\end{figure}

三、(10分)

都镀有银膜的立方形晶体(电解质)通过导线与交流电源联接。试以该系统为例,说明:

(1)何谓全电流,何谓位移电流?

(2)在导线中、晶体中和晶体外部的空间中何处有电流?何处没有电流?各有什么电流?形成这些电流的物理机制有何不同?

(3)试描述旋光晶片及$\frac{\lambda}{2}$片旋光特点的差异。

四、(9 分)

正透镜$L_1$(焦距为$50mm$)后面$10mm$处置一负透镜$L_2$(焦距为 100mm),物在$L_1$前面$50cm$处,问像在何处?描述像的特点,该透镜组的放大率=?

五、(9 分)

用$1mm$内有500条刻痕的平面透射光栅观察钠光谱2=5890A 问(1)光线垂直入射时,最多能看见几级光谱?(2)光线以30斜入射时最多能见几级光谱?说明斜入射观察光谱的利弊。