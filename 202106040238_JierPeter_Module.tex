% 模
% 代数|环|线性空间|向量|module

\pentry{环\upref{Ring},矢量空间\upref{LSpace}}

模是对线性空间的一种推广,相当于要求进行数乘时所用的“数”不再是一个域的元素,而只要求是一个环.也就是说,线性空间是模的一种,但模不一定是线性空间.

\begin{definition}{左模}
给定一个环$(R, \times)$和一个\textbf{阿贝尔}群$(M, +)$,将$R$中的每个元素都定义为$M$上的一个变换,其中对于$r\in R, m\in M$,记$rm\in M$为$r$对$m$进行变换的结果.

如果所定义的变换满足:
\begin{enumerate}
\item 对于任意$r_i\in R, m\in M$,有:$r_2(r_1m)=(r_2\times r_1)m$;
\item 对于任意$r_i\in R, m\in M$,有:$r_2m+r_1m=(r_2+r_1)m$;
\item 对于任意$r\in R, m_i\in M$,有:$r(m_1+m_2)=rm_1+rm_2$;
\item 对于$R$的乘法单位元$1_R$和任意$m\in M$,有:$1_Rm=m$.
\end{enumerate}

那么我们说环$R$和群$M$配合给定的变换定义,构成一个\textbf{左}$R$\textbf{-模(left} $R$\textbf{-modeule)},记为$_RM$.
\end{definition}

类似地,也可以定义$r_i$从右边作用于$m_i$,所得的结构就是\textbf{右}$R$\textbf{-模(right} $R$\textbf{-modeule)},记为$M_R$.

线性空间的数乘并没有左右的区分,而模的“数乘”,即以上定义的变换,是有的.这是因为域的乘法必然是交换的,而环的则不一定,导致$r_1mr_2$的定义不明确.但是如果$R$是交换环,那么我们就可以良好地定义$r_1mr_2=(r_1\times r_2)m=(r_2\times r_1)m=m(r_1\times r_2)=m(r_2\times r_1)$.对于交换环$R$的模,左模和右模是一样的,统称为$R$-模.

\begin{definition}{线性空间}
域上的模,称为\textbf{线性空间(linear space)}.
\end{definition}

我们在线性代数中已经熟悉了线性空间的概念.一般的模有很多和线性空间相似的性质.

\subsection{模的分类}

\begin{definition}{有限生成模}
令$_RM$是$R$上的一个左模,如果存在$M$的有限子集$\{m_1, \cdots, m_n\}$,使得$M=\{r_1m_1+r_2m_2+\cdots+r_nm_n|r_i\in R\}$,则称$_RM$是\textbf{有限生成(finitely generated)}的,子集$\{m_1, \cdots, m_n\}$称为其一个\textbf{生成组}.

特别地,有限生成组只包含一个元素的模,称为一个\textbf{循环模(cyclic module)}.
\end{definition}


\begin{definition}{自由模}
令$R$为一个环,集合$M=\{(r_1, r_2, \cdots, r_n)|r_i\in R\}$.在$M$上定义加法运算为$(r_1, \cdots, r_n)+(s_1, \cdots, s_n)=(r_1+s_1, \cdots, r_n+s_n)$,使$M$构成一个阿贝尔群.如果再定义\textbf{左数乘}为$r(r_1, \cdots, r_n)=(r\times r_1, \cdots, r\times r_n)$,那么称这样的到的模为一个\textbf{自由}$R$\textbf{-模(free} $R$\textbf{-module)}.
\end{definition}





\subsection{模的例子}

\begin{example}{}
线性空间都是模.
\end{example}

\begin{example}{}
给定一个超越数$x$,则$x$的全体\textbf{整系数}多项式构成一个$\mathbb{Z}$-模.
\end{example}

\begin{definition}{流形上的光滑向量场}
一个流形$M$上的全体光滑函数构成一个环$C^{\infty}(M)$.光滑函数乘到光滑向量场上的运算作为左乘,则全体gua
\end{definition}


\subsection{子模}

令$_RM$是$R$上的一个左模,$N$是$M$的一个子集,且对于任意$r\in R$和$n\in N$有$rn\in N$,那么$N$可以继承$_RM$上的作用,构成一个左$R$-模,记为$_RN$.称$_RN$为$_RM$的\textbf{子模(submodule)}.


\subsection{同态}

设$_RM$和$_RN$都是左$R$-模,且有群同态$f:M\to N$,使得对于任何$r, s\in R$和$m\in M, n\in N$,有$f(rm+sn)=rf(m)+sf(n)$,那么说$f$是一个$R$\textbf{-模同态(module)}.











