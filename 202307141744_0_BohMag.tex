% 磁旋比、玻尔磁子
% keys 磁矩|磁旋比|玻尔磁子

%未完成: 将开头内容移动到 “磁矩\upref{MagMom}” 中

\pentry{转动惯量\upref{RigRot}, 磁矩\upref{MagMom}, 自旋角动量\upref{Spin}}

\footnote{参考 Wikipedia \href{https://en.wikipedia.org/wiki/Gyromagnetic_ratio}{相关页面}。}当一个物体做定轴转动时, 磁矩\upref{MagMom}的和角动量\upref{AngMom}的大小之比就叫做\textbf{磁旋比(gyromagnetic ratio)}。 该物理量常在量子力学中出现,但对于宏观的带电物体也可以定义。

\subsection{带电圆环的磁旋比}
令圆环净电荷为 $q$, 旋转周期为 $T$, 电流为 $q/T$, 面积为 $A= \pi r^2$, 因此
\begin{equation}
\mu=\frac{q\pi r^2}{T}~.
\end{equation}
假设圆环的质量为 $m$,那么由于角动量为:
\begin{equation}
\bvec L = I\bvec\omega~.
\end{equation}
这个例子中的转动惯量为:
\begin{equation}
I = mr^2~.
\end{equation}
角速度为:
\begin{equation}
\bvec \omega = \frac{2\pi}{T}~.
\end{equation}
因此
\begin{equation}
L = \frac{2\pi m r^2}{T}~,
\end{equation}
显然对于这样的带电圆环体,磁旋比为:
\begin{equation}
\gamma = \frac{\mu}{S} = \frac{q}{2m}~,
\end{equation}
注意到它和 $r$ 或者 $T$ 是无关的。此外,如果考虑更加复杂的球体,我们也可以将其分割为小的圆环,在将所有圆环累加在一起来得到 $\mu,S$ 的取值。此外,$\bvec \mu$ 和 $\bvec S$ 的方向当电荷为正时是同向的,当电荷为负时是相反的,因此:
\begin{equation}
\bvec \mu = \frac{q}{2m}\bvec L~.
\end{equation}

\subsection{均匀带电体的磁旋比}
一个任意形状的带电刚体, 总质量为 $m$, 总电荷量为 $q$。 如果质量密度 $\rho_m$ 和电荷密度 $\rho_q$ 成正比($\rho_m/\rho_q \equiv m/q$),当它定轴转动时,令角速度矢量为 $\bvec\omega$, 则角动量\upref{CM2}为
\begin{equation}
\bvec L = I\bvec \omega  = \bvec \omega \int r_\bot^2 \rho_m \dd{V}~.
\end{equation}
磁矩定义为
\begin{equation}
\bvec \mu  = Ia \uvec\omega = \int \frac{\dd{Q}}{2\pi/\omega}  \pi r^2 \uvec \omega
= \frac12 \bvec \omega \int r_\bot^2\rho_q \dd{V} ~,
\end{equation}
其中 $r$ 为质量元到转轴的距离。两式比较,得
\begin{equation}
\bvec \mu  = \frac{q}{2m} \bvec L~.
\end{equation}
所以该物体的磁旋比为 $q/(2m)$。

在经典电磁学中,若点电荷绕一点做圆周运动, 那么它的磁旋比也同样是 $q/(2m)$。\footnote{你可以想象这个质点是一个满足上述条件的小刚体,就可以直接套用上面的推导过程了。}

把上面的推导稍加改变可以发现, 若电荷分布在物体的外表面, 那么磁旋比将比该值要大:
\begin{exercise}{}
\begin{itemize}
\item 求一个质量分布均匀,但只有侧面表面带电的圆柱体的磁旋比(转轴为对对称轴)。(答案:$q/m$)
\item 求一个质量分布均匀,但只有面表面带电的球体的磁旋比(转轴过圆心)。(答案:$5q/6m$)
\end{itemize}
\end{exercise}

\subsection{基本粒子的磁旋比}
但对基本粒子(例如电子)的实验中,发现上式还需要一个修正因子, 称为\textbf{朗德 g 因子(Landé g-factor)}或简称 g 因子。
\begin{equation}\label{eq_BohMag_4}
\mu  = g\frac{q}{2m} L~.
\end{equation}
所以磁旋比为
\begin{equation}
\gamma  = g\frac{q}{2m}~.
\end{equation}
\addTODO{\autoref{eq_BohMag_4} 中不可以用矢量, 因为众所周知量子力学中我们不可能直接决定角动量的方向,所以磁矩到底是什么方向测出来的呢? 应该是 $s_z$ 取最大值的时候。}

对于粒子的自旋, $L_s = \sqrt{s(s + 1)}\hbar$(自旋角动量\upref{Spin}), 所以它的磁矩为
\begin{equation}
\mu = \sqrt{s (s + 1)} \hbar g\frac{q}{2m}~.
\end{equation}
对于电子,实验测得
\begin{equation}
g_e \approx 2.00231930436118(27) \approx 2~.
\end{equation}
所以
\begin{equation}
\bvec \mu_e  = g_e \frac{e\hbar}{2 m_e} \sqrt{\frac12 \qty(1 + \frac12)}  = \frac{\sqrt 3}{2} g_e \mu_B \approx \sqrt 3 \mu_B~,
\end{equation}
其中 $\mu_B$ 为\textbf{玻尔磁子(Bohr magneton)},定义为
\begin{equation}
\mu_B = \frac{e\hbar}{2 m_e} \approx 5.788\e{-5} \Si{eV/T}~.
\end{equation}
它可以看作是磁矩的一个单位。


======== 回收 ===========


不过,上述推论都是完全建立在经典的理论上的。实际的实验表明,电子的磁矩是经典结论的两倍,也就是:
\begin{equation}
\bvec \mu_e = -\frac{e}{m}\bvec S~,
\end{equation}
狄拉克的相对论电子理论对于这一现象给出了解释。
%例题未完成 
在之前的例题中,我们就已经认识到直接的将电子看作为一个自旋的球体是危险的,这也就引发了经典推论的错误结论。实际结果于经典结果偏差的被称之为\textbf{$g$ 因子}:
\begin{equation}
\bvec \mu = g\frac{q}{2m}\bvec S~.
\end{equation}
在非相对论量子力学理论下考虑自旋-轨道作用时,当然有 $g=1$。不过在狄拉克所提出来的相对论量子力学理论中,电子的 $g$ 因此恰好是 $2$.在量子电动力学中,因为电子与真空能量的电磁涨落相互作用,朱利安·施温格(Julian Schwinger)对其有微小的修正:
\begin{equation}
g_e = 2+\frac{a}{\pi}+\cdots = 2.002\cdots~
\end{equation}
这些对于电子反常磁矩的实验测量和理论推导是近代物理最为辉煌的成就之一。
