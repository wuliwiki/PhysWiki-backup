% C++ 的宏(笔记)

\begin{issues}
\issueDraft
\end{issues}

\addTODO{解释一下 SLISC 中所用的那些}

\subsection{基础}
\begin{itemize}
\item C++ 的宏功能是从 C 语言直接沿用过来的, 没有做任何拓展。
\item 宏是给预编译器的指令。 预编译器本质上就是对代码中的宏指令进行文本替换操作。
\item 最常见的 \verb|#include 头文件| 指令就是让预编译器把头文件中的代码原封不动地插入到当前位置。
\item 所以从编译器(不是预编译器)看来, 每个 \verb|cpp| 文件其实都是单个文件, 无论它的 include 有多复杂。
\item \verb|#define 宏名 代码| 把一个宏定义为一串别的代码。 \verb|代码| 如果要换行, 可以用 \verb|\|。
\item \verb|#define 宏名| 仅仅定义一个宏, 也可以理解为 \verb|代码| 为空。
\item \verb|#ifdef 宏 ... #else ... #endif| 可以判断宏是否存在, 类似地, 有 \verb|#ifndef 宏|。(注意每一个 \verb|#| 都要换行。
\item 更一般地, 有 \verb|#if 条件 ... #elif ... #else ... #endif|。 例如条件可以是 \verb|#if defined(宏)| 相当于 \verb|#ifdef|, \verb|#if !defined(宏)| 相当于 \verb|#ifndef|。 预编译器会删除不满足条件的 \verb|...| 中的代码, 编译器将看不到它的存在。
\item \verb|defined()| 函数是一个特殊的存在, 没有其他类似的函数了(除了下文的宏函数)。
\item 条件中可以用 \verb|&&|, \verb`||`, 可以用括号, 可以用 \verb|>, <=, ==, !=| 等比较数值的大小, 可以用 \verb|+,-,*,/,%| 等算符。 注意运算和比较仅限于整数。
\item 用 \verb|g++ -E 源文件.cpp| 可以输出预编译的结果(将把所有宏都进行替换)。
\item \verb|#undef 宏| 可以在之后的代码中取消某个宏的定义, 否则该宏将在定义后的所有位置有定义, 而且会覆盖函数名和变量名。 所以使用宏时要特别小心, 一般会严格遵循特定的命名规则以防止和其他变量或宏发生冲突。
\item 一个宏在一个 cpp 文件(包括它使用的头文件)中不能有多次定义。
\item 命名规则: 宏名应该全部使用大写字母和下划线。 不要用下划线开头(下划线开头的宏保留给编译器和基本的库去定义)。 若要避免冲突, 可以你定义的所有宏前面加上一个特殊的前缀。 例如 \verb|前缀_宏名|, 如果你在写库, 通常这个前缀是库的名字。
\end{itemize}

\subsection{C++ 标准中常用的宏}
\begin{itemize}
\item \verb|__cplusplus| 用于判断是否在使用 C++ (而不是 C), 以及判断使用的 C++ 标准, 例如 \verb|#if __cplusplus >= 201103L| (\verb|L| 表示这是一个 \verb|long| 的 literal)
\item \verb|__FILE__| 会替换为当前的文件名(一个字符串 literal)。
\item \verb|__LINE__| 会替换为当前的行号(一个 \verb|int| 的 literal)。
\item \verb|__func__| 会替换为当前的函数(一个字符串 literal)
\end{itemize}

\subsection{宏函数}
\addTODO{...}
