% Hierarchical Data Format(HDF) 笔记
% license Usr
% type Note


\begin{issues}
\issueDraft
\end{issues}

\begin{itemize}
\item 广泛用于科学计算: 高能物理、天文、生物信息
\item 由 HDF Group 开发
\item 支持文件夹结构,多维数组,图片,表格,文字,支持大数据量,支持数据压缩,支持小部分读写(例如数组的某列)
\item 支持 C, C++, Fortran, Java, Python, MATLAB
\item \href{https://www.hdfgroup.org/downloads/hdfview/}{HDFView} 是一个 GUI 程序可以查看 HDF5 文件。
\end{itemize}

\subsection{HDF C++ API}
\begin{itemize}
\item 这是官方提供的 C++ API
\item Ubuntu 安装: \verb|sudo apt install libhdf5-dev|
\end{itemize}

常识:
\begin{itemize}
\item \verb|DataSet|:文件中的一个矩阵
\item \verb|DataSet::write(内存地址, 元素类型, 内存矩阵DataSpace, 文件矩阵DataSpace)|:把内存中一个矩阵的一部分(用 \verb`DataSpace` 表示)写入文件中一个矩阵的一部分
\item \verb|DataSpace|:内存中或文件中矩阵的 layout
\item \verb|DataSpace::selectHyperslab| 选取一个子矩阵
\item \verb`hsize_t` 用于表示矩阵尺寸的整型
\item 文件中的矩阵只支持行主序(row major)储存
\item 对内存中的 row major 密矩阵,可以直接用一个 \verb`write` 命令写整个矩阵
\item 对内存中的 column major 矩阵,需要用 Hyperslab 一列一列的写
\end{itemize}


\begin{lstlisting}[language=cpp,caption=test\_hdf5.cpp]
#include <iostream>
#include <vector>
#include <H5Cpp.h>

// Use the standard HDF5 namespaces
using namespace H5;
using namespace std;

int main() {
	vector<double> data = {1., 2., 3., 4., 5., 6.};

	// H5F_ACC_TRUNC: replace file if exist
	H5File file("testfile.h5", H5F_ACC_TRUNC);

	hsize_t dims[2] = {2, 3}; // array size (ROW major!)
	DataSpace dataspace(2, dims);

	// use Predfined Type (PredType), double
	DataSet dataset = file.createDataSet("myDoubleArray",
		PredType::NATIVE_DOUBLE, dataspace);

	dataset.write(data.data(), PredType::NATIVE_DOUBLE);

	// will be auto called in destructors
	dataset.close(); dataspace.close(); file.close();
}
\end{lstlisting}
\begin{itemize}
\item 可以用 \verb|dpkg -L libhdf5-dev| 来查看所有头文件和 lib 文件的安装路径。
\item 编译:\verb|g++ test_hdf5.cpp -o test_hdf5.x -I/usr/include/hdf5/serial/ -L/usr/lib/x86_64-linux-gnu/hdf5/serial/ -lhdf5_cpp -lhdf5|
\item 运行: \verb|test_hdf5.x|
\item 注意储存的矩阵是行主序的!
\item 默认不开启压缩, 要让上面代码进行压缩,在创建 \verb|dataset| 时添加一个 \verb|DSetCreatPropList|(property list):
\end{itemize}
\begin{lstlisting}[language=cpp]
DSetCreatPropList plist;
plist.setDeflate(6); // compression level from 0 to 9
DataSet dataset = file.createDataSet("myDoubleArray",
	PredType::NATIVE_DOUBLE, dataspace, plist);
\end{lstlisting}
