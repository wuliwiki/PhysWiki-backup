% 李群的指数映射
% license Usr
% type Tutor

%预备知识需要添加.exp全文替换
\pentry{李群的李代数\nref{nod_LieGA},流\nref{nod_flow}}{nod_016b}
(本文默认左不变切场都是从切空间的切向量开始延拓得到)
\begin{definition}{}
设$\mathfrak g$是李群$G$的李代数。定义指数映射$exp:\mathfrak g\to G$,对于任意$X_e\in \mathfrak g$ 有$exp(X_e)=c_X(1)$。其中$c_X$正是左不变切场$X$的积分流。
\end{definition}
在物理上,我们经常要用到矩阵李群。可以证明,对于矩阵李群,指数映射恰为矩阵的指数函数,可说是名副其实了。

\begin{theorem}{指数映射的性质}
\begin{enumerate}
\item 对于任意$X_e\in \mathfrak g$,积分曲线为$\theta_t(e)=c_X(t)=exp(tX_e)$。
\item 
\item 对于任意$x,t\in\mathbb R,X_e\mathfrak g$,指数映射是$\mathbb R\to G$的群同态,满足$exp((s+t)X_e)=(expsX_e)(exptX_e)$。
\item 指数映射是光滑的。
\item 指数映射在$t=0$处的切映射是单位映射。
\item 对于一般线性群$GL(n,\mathbb R)$,有
\begin{equation}
\exp A=\sum_{k=0}^\infty\frac{A^k}{k!},\quad \forall A\in\mathfrak{gl}(n,\mathbb{R})~.
\end{equation}
\end{enumerate}
\end{theorem}

