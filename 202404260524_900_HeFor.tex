% 埃尔米特型
% keys 埃尔米特型|Hermitian|正定埃尔米特型
% license Xiao
% type Tutor
\pentry{半双线性形式\nref{nod_sequil}}{nod_6cf7}
本节半双线性型定义采用物理上习惯的定义,即\autoref{def_sequil_1}~\upref{sequil}。
\begin{definition}{埃尔米特型}
设 $V$ 是定义在复数域 $\mathbb{C}$ 上的矢量空间,其上的半双线性型称为\textbf{埃尔米特型}(Hermitian),若
\begin{equation}\label{eq_HeFor_2}
f(   y,   x)=f^*(   x,   y)~,
\end{equation}
其中*表共轭复数。
\end{definition}
\begin{example}{埃尔米特型对应矩阵元的性质}
试证明埃尔米特型 $f$ 对应的矩阵 $F$ 的系数满足 $f_{ij}=f_{ji}^*$。其中 $f_{ij}=f(   e_i,   e_j)$。这就是说 $\mat F^\dagger=\mat F$,其中 $\mat F^\dagger=\qty(\mat F^T)^*$。

\textbf{证明:}由埃尔米特型定义知
\begin{equation}
\begin{aligned}
\sum_{i,j}x_i^*y_j f_{ij}&=f(   x,   y)=f^*(   y,   x)\\
&=\qty(\sum_{i,j}y_j^* x_i f_{ji})^*=\sum_{i,j}y_j x_i^*f_{ji}^*~,
\end{aligned}
\end{equation}
对比即得 $f_{ij}=f_{ji}^*$。
\end{example}
按照二次型对应的线性型与对应矩阵的命名的惯例(即名为 $name$ 型的线性型对应的矩阵称 $name$ 矩阵),有下面定义
\begin{definition}{埃尔米特矩阵}
称矩阵 $\mat A$ 为\textbf{埃尔米特矩阵},若 $\mat A^\dagger=\mat A$,其中,$\mat A^\dagger=\qty(\mat A^T)^*$
\end{definition}
\begin{example}{埃尔米特矩阵在不同基底下仍是埃尔米特的}
也就是说要证明对基底 $   e_i$ 下的埃尔米特型 $f$ 对应的矩阵 $\mat F=\mat F^\dagger$ ,要证在基底 $   e'_i$ 下对应的矩阵 $\mat F'$ 仍满足 $\mat F'=\mat F'^\dagger$。

\textbf{证明:}设 $\mat A$ 是基底 $  e_i$ 到基底 $   e'_i$ 的转换矩阵。由 $f$ 是半双线性型,知(\autoref{ex_sequil_1}~\upref{sequil})
\begin{equation}
\mat F'=\mat A^\dagger \mat{FA}~,
\end{equation}
 故
\begin{equation}
(\mat F')^\dagger=(\mat A^\dagger \mat{FA})^\dagger=\mat A^\dagger \mat F^\dagger \mat A=\mat A^\dagger \mat{FA}=\mat F'~.
\end{equation}
\end{example}
埃尔米特型 $f(   x,   y)$ 自然对应\textbf{埃尔米特二次型} $f(   x,   x)$ 。因为
\begin{equation}
 f(   x,   x)= f^*(   x,   x) ~,
\end{equation}
所以埃尔米特二次型只取实数值。

\begin{definition}{正定埃尔米特二次型}
埃尔米特二次型 $f(   x,   x)$ 称为\textbf{正定的},若对 $\forall    x$ 有
\begin{equation}
f(   x,   x)\geq0~,
\end{equation}
且
\begin{equation}
f(   x,   x)=0\Rightarrow   x=   0~.
\end{equation}
\end{definition}
\begin{definition}{正定埃尔米特型}\label{def_HeFor_1}
与正定埃尔米特二次型对应的埃尔米特型称为\textbf{正定埃尔米特型}。
\end{definition}
