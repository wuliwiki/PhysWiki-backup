% 状态量和过程量

\begin{issues}
\issueDraft
\end{issues}

系统关于时间的变化可以看作状态空间中一点经过的轨迹. 若定义一个量只和该点在状态空间的位置有关, 那么它就是状态量, 若它和状态空间中的运动轨迹有关, 它就是过程量.

一类常见的情况是, 一个系统的状态用矢量 $\bvec x = (x_1, x_2, \dots, x_N)$ 描述, 在一个特定的过程中, $x_i$ 都是时间的函数. 定义一个量为
\begin{equation}\label{StaPro_eq1}
Q_{12} = \int_{\mathcal L} \sum_i f_i(x_1, \dots, x_N) \dd{x_i} = \int_{t_1}^{t_2} \sum_i f_i(x_1, \dots, x_N) \dv{x_i}{t} \dd{t}
\end{equation}
$\mathcal L$ 表示状态点的 “运动方程” $x_i(t)$ ($i = 1,\dots, N$) 以及起点终点 $\bvec x(t_1), \bvec x(t_2)$. 易得,积分结果只和路径的形状有关而与状态点在轨迹上运动的快慢无关. 所以这里的 $t$ 可以看作轨迹的参数随时间变化而未必是时间本身.

从定义上来说, $Q$ 是一个过程量, 但如果在某个系统中它可以表示为某个状态量的增量, 那么\textbf{对这个系统}区分 $Q$ 是状态量和过程量将没有太大意义. 认为任何状态量在不同时间的差都能看作一个这样的过程量. 此时 \autoref{StaPro_eq1} 积分的结果不取决于路径, 只取决于初末状态. 把该状态量记为 $V(\bvec x)$, 那么总有
\begin{equation}
Q_{12} = V(\bvec x(t_2)) - V(\bvec x(t_1))
\end{equation}
例如在二维或三维状态空间, 若令矢量函数为 $\bvec f = \sum_i f_i \uvec x_i$, 那么当旋度 $\curl \bvec f = \bvec 0$ 时, $\bvec f$ 就是一个保守场, 必存在势函数 $V(\bvec x)$.

但事实上远非所有情况下\autoref{StaPro_eq1} 的积分都可以表示为两个状态量之差. 此时积分的结果必须取决于路径的形状, 那么区分状态量和过程量就至关重要. 例如, 虽然我们往往写出微分关系
\begin{equation}
dQ = \sum_i f_i(x_1, \dots, x_N) \dd{x_i}
\end{equation}
但是却不可能把 $Q$ 表示为 $x_i$ 的函数, $f_i$ 也不能看作偏导 $\pdv*{Q}{x_i}$.

一个具体的例子是力场对质点的做功. 如果是保守场, 那么做功就是势能之差, 如果是非保守场, 做功只能由具体路径决定. 此时 “功” 和 “能” 的区分就很重要了.

另一个例子是热力学第一定律往往记为
\begin{equation}
\dd{Q} = P\dd{V} + \dd{E}
\end{equation}
但状态空间中的环积分并不总是为零, 例如卡诺热机. 所以 $Q$ 不能看作 $V, E$ 的函数, 也不能记
\begin{equation}
\qty(\pdv{Q}{V})_E = P \qquad \qty(\pdv{Q}{E})_V = 1 \qquad \text{(错)}
\end{equation}

