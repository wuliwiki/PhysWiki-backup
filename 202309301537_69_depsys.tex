% 近独立子系、理想玻色气体和费米气体
% keys 热力学系统|近独立子系|无相互作用
% license Xiao
% type Tutor

\pentry{热力学笔记(科普)\upref{HeatIn},热力学与统计力学导航\upref{StatMe}}

\textbf{近独立子系}被定义为:大量粒子组成的热力学系统,且忽略粒子间的力学相互作用,不同粒子间可以看作是近独立的。比如\textbf{理想气体}\upref{Igas}是近独立子系,除了粒子间的碰撞相互作用,其他相互作用可以。而范德瓦尔斯气体\upref{Vand}

近独立子系是热力学的重要研究对象,由于不同粒子间可看作是近独立的,可以通过分析系统的能级来计算系统的配分函数\footnote{例如正则系综法\upref{CEsb}。},可以分析粒子的速率分布函数\footnote{例如麦克斯韦—玻尔兹曼分布\upref{MxwBzm}。},因此系统的各个热力学量就可以方便地用统计力学公式进行计算\footnote{例如玻尔兹曼分布(统计力学)\upref{MBsta}。},可以得到与实验符合得很好的结果。

近独立子系一般有三种分布:玻尔兹曼分布,费米狄拉克分布,玻色爱因斯坦分布,其中玻尔兹曼分布是经典极限情形下的分布。同时,粒子的全同对称性带来等效的交换相互作用,考虑到量子系统中玻色子和费米子的性质,系统会呈现出同经典玻尔兹曼分布不同的结果\footnote{例如玻色爱因斯坦凝聚\upref{BEC}、金属中的自由电子气体\upref{mfcgas}。}。

近独立子系的统计性质可以由最概然分布给出,也被称为玻尔兹曼分布,具体计算见玻尔兹曼分布(统计力学)\upref{MBsta}中第三节的讨论,利用了拉格朗日乘子法来计算最概然分布。这里我们将讨论另一种方法,它本质上是\textbf{单能级巨正则系综法}\upref{QGs1ME},详细的理论介绍可以参考\textbf{巨正则系综法\upref{MCEsb}的近独立子系部分}。

\subsection{单粒子态能级和粒子数表象}
为了研究近独立子系的微观状态和它的统计性质,我们可以在粒子数表象下研究整个系统。

由于近独立子系中粒子可以近似视为独立的,因此研究系统的热力学性质可以通过研究单粒子的微观状态入手。假设系统中单粒子可能处于这样一些能级 $\epsilon_1,\epsilon_2,\cdots$,它们可以通过求解单粒子哈密顿量 $H$ 的本征值得到,因此也可能存在重复的情况: $\epsilon_i=\epsilon_j(i\neq j)$,我们称这种情况为能级\textbf{简并}。在这里我们用下标 $s$,用 $\epsilon_s,s=1,2,\cdots$ 来标记这些单粒子态能级。

一个例子是,对于立方体盒子中的光子气体\upref{PhoGas},如果将它视作近独立的,那么光子的能级为
\begin{equation}
\hbar^2c^2 |\bvec k|^2,\bvec k=\frac{2\pi n}{L}\hat{\bvec x}+\frac{2\pi m}{L}\hat{\bvec y}+\frac{2\pi l}{L}\hat{\bvec z},\quad n,m,l\in \{0,1,2,\cdots\}~.
\end{equation}
又由于电磁波有两个偏振自由度,所以上面每个能级都重复出现两次。由此我们得到了光子气体的所有单粒子能级。

求出了这些能级以后,就可以知道近独立子系的所有可能的微观状态了。


\subsection{近独立子系的玻尔兹曼分布}
\pentry{玻尔兹曼分布(统计力学)\upref{MBsta}}
在玻尔兹曼分布(统计力学)\upref{MBsta} 中,我们求解了经典理想气体的最概然分布,得到了以下结论
\begin{theorem}{}
对于经典理想气体,设系统的 $\epsilon_\lambda$ 能级单粒子态的简并度为 $\omega_\lambda$(这里我们用下标 $\lambda$ 和 $\epsilon_\lambda,\lambda=1,2,\cdots$ 来标记所有互不相同的能级,简并度表示它作为哈密顿量本征值重复出现了多少次),那么该能级的平均粒子数为
\begin{equation}
\bar a_\lambda = \omega_\lambda e^{-\alpha-\beta\epsilon_\lambda}=
\omega_\lambda e^{-(\epsilon_\lambda-\mu)/kT}~.
\end{equation}


或者说一个单粒子微观状态 $s$ 的粒子数的\textbf{期待值}(expectation value)为
\begin{equation}
\bar a_s= e^{-(\epsilon_s-\mu)/kT}~.
\end{equation}
\end{theorem}

事实上这一结论可以推广到更一般的近独立子系,包括费米系统和玻色系统。“近独立子系”告诉我们,系统的不同单粒子态之间可以近似认为是互相独立的,没有相互作用的。对于某个单粒子态,如果是玻色子系统,那么该微观状态的粒子占据数可以是任意的;而如果是费米子系统,那么根据泡利不相容原理,该微观状态上至多只能有一个粒子。考虑粒子数占据数为 $n_s$ 的微观状态,那么子系出于该微观状态的概率可以由 
\begin{equation}
\bar a_s(n_s) =  e^{-\alpha n_s-\beta n_s \epsilon_s}~
\end{equation}
给出,其中 $\epsilon_s$ 为单粒子态的能级大小,$n_s$ 为粒子占据数。
$\beta,\alpha$ 参数可以由系统的温度参数以及子系的化学势确定\footnote{玻尔兹曼分布(统计力学)\upref{MBsta}。}:
\begin{equation}
\alpha=-\frac{\mu}{kT},\quad \beta=\frac{1}{kT}~.
\end{equation}
因此上式可以改写为
\begin{equation}
\bar a_s(n_s)= e^{-n_s(\epsilon_s-\mu)/kT}= e^{-(E_\text{子系}-\mu_\text{子系})/kT}~.
\end{equation}
这实际上也启发了巨正则系综的思想。子系的巨正则系综由 $\sum_N \sum_i \exp((\mu N-E_i)/kT)$ 给出。

\subsection{理想玻色气体和理想费米气体}
\subsubsection{推导玻色分布}
以玻色子系统为例,能级 $\epsilon_l$ 的简并度为 $\omega_l$,且同一个能级上可以有多个粒子占据,那么此处我们将它不仅仅视为一个能级,而是视为能量为 $0,\epsilon_l,2\epsilon_l,3\epsilon_l,\cdots, n\epsilon_l,\cdots$ 的无限多种能级,分别代表在 $\epsilon_l$ 能级上系统可能的状态。在这无穷多个能级上考虑玻尔兹曼分布,$\bar{a}_{n,l} \propto n\cdot \exp(-n\alpha-\beta\cdot n\epsilon_l),n=0,1,2,\cdots $($\alpha$ 正比于单粒子的化学势,所以对于占据数为 $n$ 的状态应当将分布函数改写为 $\exp(-n\alpha-\beta n\epsilon_l)$。)。因此这里我们可以定义玻色系统的子系配分函数。要注意的是,正是因为是我们讨论的是近独立子系,子系与子系间相互作用近似忽略,所以才能够单独地讨论子系配分函数以及它的粒子占据数。
\begin{equation}
Z_1=\sum_{n\ge 0} \exp(-n\alpha-\beta\cdot n\epsilon_l)=\frac{1}{1-\exp(-\alpha-\beta\epsilon_l)}~.
\end{equation}
对所有可能的 $n$ 求和,再乘以简并度 $\omega_l$,除以配分函数,就得到了能级 $\epsilon_l$ 所对应的期望粒子数:
\begin{equation}\label{eq_depsys_1}
\begin{aligned}
\bar a_l&=\omega_l \frac{\sum_n n\cdot \exp(-n\alpha-n\beta\epsilon_l)}{\sum_n \exp(-n\alpha-n\beta\epsilon_l)}=\omega_l (1-\exp(-\alpha-\beta\epsilon_l)) \pdv{(-\beta\epsilon_l)} \frac{1}{1-\exp(-\alpha-\beta\epsilon_l)}\\
&=\omega_l \frac{1}{\exp(\alpha+\beta\epsilon_l)-1}~.
\end{aligned}
\end{equation}
可以发现它恰好与玻色分布的公式\autoref{eq_MBsta_9}~\upref{MBsta} 相符合。
\subsubsection{推导费米分布}
类似地,对于费米系统可以推出费米分布
\begin{equation}
\bar a_l=\omega_l \frac{1}{\exp(\alpha+\beta\epsilon_l)+1}~.
\end{equation}
