% Mathematica 文件操作
% keys Mathematica|IO|文件操作

\begin{issues}
\issueDraft
\issueTODO
\end{issues}

\subsection{常用环境变量}

除了丰富的标准文件操作外,Wolfram 语言的统一符号(\verb`Symbol`)架构,
让我们更容易将算法和高级编程应用于许多文件和系统管理任务.此介绍主要参考官方页面:

\begin{itemize}
\item guide/FileOperations, 比较全的文件系统操作的函数列表
\item tutorial/FilesStreamsAndExternalOperations\#12068
\end{itemize}

其中的关键是:不要直接使用字符串函数硬编码 文件名/文件路径, 
这样生成的路径依赖于操作系统, 应该使用 Mathematica 提供的文件系统接口.

\begin{itemize}
\item \verb`$OperatingSystem` : 给出正在运行的操作系统的名称.
\item \verb`$PathnameSeparator` : 字符串,在构建路径名的时候使用, 例如 \verb|$UserBaseDirectory <> "\abcd\"|. 
Windows 的默认值是 \verb|\|, 其他系统是 \verb|/|.
在 Windows 中, 像 FileNameSplit 这样的函数默认同时允许 \verb|\| 和 \verb|/|.
\end{itemize}

\subsubsection{文件后缀名使用惯例}

\begin{itemize}
\item \verb`.m`  : Wolfram 语言源文件
\item \verb`.nb` : Wolfram 系统笔记本文件
\item \verb`.ma` : Wolfram 系统从第 $3$ 版以前的笔记本文件
\item \verb`.mx` : 输出所有 Wolfram 语言表达式
\item \verb`.exe`: WSTP 可执行程序
\item \verb`.tm` : WSTP 模版文件
\item \verb`.ml` : WSTP 流文件
\end{itemize}

\verb`Get`,\verb`Needs`,\verb`Import`,\verb`Install` 等函数读取本地文件时,默认使用的搜索路径为 \verb`$Path`.
全局变量 \verb`$Path` 被定义为一个字符串的列表, 每个字符串代表一个目录.
每次你要求打开文件时, Wolfram 就依次将这些目录暂时设置为当前工作目录,然后从该目录中尝试寻找你要求的文件.
也就是说,如果两个目录中含有同名的文件,排在前面的目录将遮蔽掉后面的目录,不过你可以通过提供更详细的上层路径来避免歧义.
在 \verb`$Path` 的典型设置中, 当前目录 \verb`.` 和你的主目录 \verb`~` 被列在第一位.

\subsubsection{常用的目录操作}

\begin{itemize}
\item \verb`DirectoryName["name",n]` : 给出路径的父目录, \verb`n` 代表上升 \verb`n` 次. 
默认情形给出父目录, 可以省略 \verb`n`. 可作用于文件和目录, 但不检查目录是否真实存在于硬盘.
\item \verb`DirectoryName[..., OperatingSystem->"os"]` 用来给出某种操作系统风格的路径, 
选项有 \verb`"Windows"`, \verb`"MacOSX"`, 和 \verb`"Unix"`.
\item \verb`ParentDirectory["dir",n]` :给出路径的父目录, \verb`n` 代表上升 \verb`n` 次, 
只能作用于目录, 并且要求目录真实存在.
\end{itemize}

\subsubsection{预定义的环境变量}

\begin{itemize}
\item \verb`$InitialDirectory` : Wolfram 系统启动时的初始目录.
\item \verb`$HomeDirectory` :  你的主目录, 如果被定义过的话
\item \verb`$BaseDirectory` :  Wolfram 系统要加载的全系统文件的基本目录.
\item \verb`$UserBaseDirectory` :  用于 Wolfram 系统加载的, 用户自定义文件的基本目录
\item \verb`$InstallationDirectory` :  你的 Wolfram 系统安装的最高级别目录
\end{itemize}

Wolfram 系统所使用的绝大多数文件都与操作系统无关. 然而, \verb`mx` 和 \verb`.exe` 文件与系统有关.
对于这些文件, 按照惯例, 捆绑上不同计算机系统版本的名称, 形式如 \verb`name/$SystemID/name`.

\subsection{笔记本相关接口}

\begin{itemize}
\item \verb`NotebookFileName[]` : 给出当前笔记本的完整路径.
\item \verb`NotebookDirectory[]`: 笔记本父目录
\end{itemize}

\subsubsection{打开笔记本}

\begin{itemize}
\item \verb`NotebookOpen["name"]`:  打开已经存在的笔记本 \verb`"name"`,返回笔记本对象. \verb`"name"` 可以是绝对路径.
\item \verb`NotebookOpen["name", options]`: 使用指定的选项打开笔记本.
\item \verb`NotebookOpen[File["path"]]` 和 \verb`NotebookOpen[URL["url"]]` 也支持.
\item \verb`NotebookOpen` 通常会导致你的屏幕弹出一个新的笔记本窗口.
如果 \verb`NotebookOpen` 打开指定的文件失败, 则返回 \verb`$Failed`.
\item 若给出相对路径, \verb`NotebookOpen` 则搜索由前端全局选项 \verb`NotebookPath` 指定的目录.
\item 若设置选项 \verb`Visible->False`,\verb`NotebookOpen` 打开的笔记本将带有此选项,它永远不会显示在屏幕上.
\item 调用\verb`NotebookOpen` 时,当前光标(\verb`selection`)的初始化位置,位于笔记本第一行单元之前.
\end{itemize}

\subsubsection{保存和关闭笔记本}

\begin{itemize}
\item \verb`NotebookSave[notebook]`: 保存特定笔记本的当前版本.\verb`notebook` 必须是一个 \verb`NotebookObject`.
\item \verb`NotebookSave[notebook]` 将笔记本保存在文件中, 文件名由笔记本对象 \verb`notebook` 给出.
\item \verb`NotebookSave` : 写入对应的 Wolfram 语言表达式, 以及 Wolfram 语言注释, 以便前端再次读入笔记本.
\item \verb`NotebookSave[notebook, "file"]`, 如果\verb`"file"`存在, 则不加警告地覆盖它.
\item \verb`NotebookSave[notebook, File["file"]]` 也支持.
如果给定选项 \verb`Interactive->True`, 前端将提示用户给笔记本取名.
\item \verb`NotebookClose[notebook]`: 关闭指定的笔记本对象.
\item \verb`NotebookClose[]`:关闭当前在运行的笔记本.
\item \verb`NotebookClose` 将使笔记本从你的屏幕上消失,并使所有引用该笔记本的笔记本对象失效.
如果给定了选项设置 \verb`Interactive->True`, 前端将提示用户是否关闭笔记本而不保存.
\end{itemize}




\subsection{操作文件和目录}

\begin{itemize}
\item tutorial/FilesStreamsAndExternalOperations\#12068
\item Manipulating Files and Directories
\end{itemize}

\begin{itemize}
\item ExpandFileName["name"] : 将"name"文件展开成当前系统规范的绝对路径, 给出相对于你当前目录的名称.
它展开通常的目录指定, 如 \verb`.` 和 \verb`..`;它只是对文件名进行操作:它并不实际搜索指定的文件.
它支持 \verb`ExpandFileName[File["name"]]`, 以及 \verb`ExpandFileName[URL["file:///path"]]`, 
后者将基于文件的 \verb`URL` 转换为绝对文件名.
\item \verb`AbsoluteFileName["name"]`: 给出 \verb`"name"` 文件的绝对路径. 
与 \verb`ExpandFileName` 的区别是, 它会进入文件系统, 检查文件是否真实存在.
同样相对于你当前目录的名称, 可以处理目录指定, 如\verb`.`, \verb`..` 和 \verb`~`.
它也支持 `\verb`AbsoluteFileName[File["name"]]`.
\end{itemize}

\addTODO{...}

\begin{itemize}
\item \verb`FileNameTake["name"]` : 从"name"的完整路径中提取出最后的文件名.
\item \verb`FileBaseName["file"]` : 给出文件的 basename, 也就是不包括拓展名.
\item \verb`FileExtension["file"]`: 给出文件的拓展名.
\item \verb`FileNameDepth["name"]`: 给出文件路径的深度, 文件不必真实存在.
\end{itemize}

\addTODO{...}

\begin{itemize}
\item \verb`FileNameJoin` : 从路径列表中组合出完整的文件名
\item \verb`FileNameSplit` : 将文件的完整路径分割开
\item \verb`FileNameDrop["name",n]` : 去掉文件"name"路径的前n个片段. 如果是 \verb`-n`, 那么去掉从末尾开始的 n 个.
\item \verb`FileExistsQ["name"]`  : 检查文件, 目录等等是否存在.
\item \verb`ContextToFileName["context"]`  : 给出 Mathematica 上下文规范对应的文件名.
\end{itemize}


