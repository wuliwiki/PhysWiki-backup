% 理想气体混合的熵变

\footnote{本文参考了Schroeder的《热物理学导论》}
\pentry{理想气体分压定律\upref{PartiP}, 熵的宏观表达式\upref{MacroS}}

\begin{issues}
\issueDraft
\end{issues}
在本文中,我们将讨论等温(或绝热,因为理想气体混合不导致内能改变,因此等温与绝热在本问题中没有区别)等压情况下,理想气体混合熵变的问题。

\subsection{同种气体的混合}
\begin{figure}[ht]
\centering
\includegraphics[width=5cm]{./figures/ed271e2113ff3534.pdf}
\caption{同种气体的混合} \label{fig_IGME_1}
\end{figure}
首先,我们先思考一种最简单的混合过程:同种气体的混合。当我们拿开挡板之后会发生什么?你觉得可能\textsl{什么都不会发生}:既然两侧的气体已经是完全一样的,那有和没有挡板有什么区别吗?事实正是如此的简单(伴随着些许的诡异):在这种情况下,同种气体的混合并不会导致熵变。
\begin{equation}
\Delta S = 0
\end{equation}

这个问题比想象中的更微妙一些。这个现象暗示了微观粒子是全同的,而这也是统计力学与量子力学的基本假设之一。在微观世界中,你不能说“这个粒子、那个粒子”,因为他们是完全一样而不可区分的。

\subsection{异种气体的混合}
\begin{figure}[ht]
\centering
\includegraphics[width=5cm]{./figures/f6a63cf8de808fbc.pdf}
\caption{异种气体的混合} \label{fig_IGME_fig2}
\end{figure}

现在我们来看异种气体的混合。先定义摩尔分数$x_A = \frac{n_A}{\sum n_i} = \frac{n_A}{n_A+n_B+...}$

尽管移除挡板后系统的总压强还是$p$,但$A, B$各自的\textsl{分压}\upref{PartiP} 却降低了。例如,$A$气体的分压从混合前的$p$降为$p'=p \cdot x_A$。

先计算A气体的熵变。根据熵的计算公式\upref{MacroS},此时
$$
\Delta S_A=-nR\ln p |^{p \cdot x_A}_p=n_A R \ln x_A
$$
同理,
$$
\Delta S_B=n_B R \ln x_B
$$
因此系统的总熵变
$$
\Delta S = \Delta S_A+\Delta S_B= n_A R \ln x_A + n_B R \ln x_B
$$
写为更为一般的形式
\begin{equation}
\Delta S = R \sum n_i \ln x_i
\end{equation}

还可以从另一个角度理解:在移除挡板前,$A$气体只能在左侧容器中运动;而在移除挡板后,$A$气体可以同时在左、右容器中运动。由于$A$气体的运动范围扩大了,因此$A$气体的混乱程度上升了(或者按统计力学的说法,$A$气体可能的微观态个数增加了),也就是熵上升了。

\subsection{其他混合过程}
尽管上述两个例子非常简单,但他们是理解理想气体混合熵变问题所必不可少的。
%TODO: 举更多例子
