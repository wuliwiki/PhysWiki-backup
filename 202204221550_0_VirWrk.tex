% 虚位移、虚功原理

\begin{issues}
\issueTODO
\end{issues}

\pentry{广义力\upref{LagEqQ}}

在拉格朗日方程(\autoref{LagEqQ_eq1}~\upref{LagEqQ})($i=1,\dots,N$, 下同)
\begin{equation}\label{VirWrk_eq1}
\dv{t} \pdv{L}{\dot q_i} = \pdv{L}{q_i} + Q_i^{(e)}
\end{equation}
中, 假设系统中没有含时约束, 即每个质点的位置 $\bvec r_i$ 不显含时, 且 $V$ 不显含 $\dot q$. 若系统不随时间变化(即每个广义坐标 $q_i$ 不随时间变化), 那么每个广义力都恒为零
\begin{equation}\label{VirWrk_eq2}
Q_i = -\pdv{V}{q_i} + Q_i^{(e)} = 0
\end{equation}
而根据广义力的定义(\autoref{LagEqQ_eq3}~\upref{LagEqQ}), 有
\begin{equation}
Q_i = \sum_j \bvec F_j^{(a)} \vdot \pdv{\bvec r_j}{q_i} = 0
\end{equation}
其中 $\bvec F_j^{(a)}$ 是作用于点 $\bvec r_j$ 的所有非约束力的合力. 也可以写成类似于微分形式
\begin{equation}\label{VirWrk_eq3}
\sum_j \bvec F_j^{(a)} \vdot \delta \bvec r_j = 0
\end{equation}
这里之所以把 $\bvec r_j$ 的微小增量记为 $\delta \bvec r_j$ 而不是微分 $\dd{\bvec r_j}$, 是因为他们是一些假想的位移, 而现实中系统中所有的点 $\bvec r_j$ 都不随时间变化, 有 $\dd{\bvec r_i} = \bvec 0$, 所以要加以区分. 这种假想的位移就叫做\textbf{虚位移(virtual displacement)}, 而式子左边的求和就叫做\textbf{虚功(virtual work)}. 所以\autoref{VirWrk_eq3} (或者等价的\autoref{VirWrk_eq2})称为\textbf{虚功原理(principle of virtual work)}. 虚功原理给出了使一个系统ba

\subsubsection{证明虚功原理}
在静止的系统中, 显然\autoref{VirWrk_eq1} 左边第一项为零, 而右边第一项中
\begin{equation}
\eval{\pdv{T}{q_i}}_{\dot q = 0} = 0
\end{equation}
这是因为 $\dot q = 0$ 时, 动能 $T$ 恒为零(每个质点不显含时间). 所以\autoref{VirWrk_eq2} 成立. 证毕.

\addTODO{高低不同的两点间悬挂一条质量不计的细线, 线比两点之间的距离要长. 一个质点在细线上无摩擦滑动, 求平衡点.}

\subsection{虚功和实功的关系}
某个力 $\bvec F_j$ 作用点为 $\bvec r_i$, 对系统的功率为
\begin{equation}
\dot W = \bvec F_j \vdot \dot {\bvec r}_j
= \bvec F_j \vdot \sum_i \qty(\pdv{\bvec r_j}{q_i} \dot q_i + \pdv{\bvec r_j}{t})
= \sum_i \qty(\bvec F_j \vdot \pdv{\bvec r_j}{q_i}) \dot q_i + \sum_i 
\end{equation}
