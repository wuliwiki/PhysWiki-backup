% 极小多项式(线性代数)
% keys 零化多项式|null polynomial|最小多项式|minimal polynomial|环|线性代数|域|Hamilton-Cayley 定理|Cayley-Hamilton 定理|哈密尔顿-凯莱定理
% license Xiao
% type Tutor


\pentry{线性映射\upref{LinMap}}


极小多项式又称最小多项式,描述了一个元素关于给定环的代数性质。本节是线性代数的一部分,因此我们只讨论线性变换的极小多项式。


\subsection{零化多项式}





\begin{definition}{线性变换的多项式}
给定域$\mathbb{F}$上的线性空间$V$,令$A$是$V$上的\textbf{线性变换}。

定义$A^k$是$A$与自身复合$k$次的结果,即$A^k=\overbrace{A\circ A\circ\cdots\circ A}^{k\text{个}A}$,则可以定义线性变换的多项式:若$f$是$\mathbb{F}$上的多项式,表达为
\begin{equation}
f(x) = \sum_{i=0}^m a_ix^i~, 
\end{equation}
那么$f(A)$就是$V$上的线性变换,表达为
\begin{equation}
f(A) = \sum_{i=0}^m a_ixA^i~. 
\end{equation}
\end{definition}



\begin{definition}{零化多项式}\label{def_MinPol_1}
给定域$\mathbb{F}$上的线性空间$V$,令$A$是$V$上的\textbf{线性变换}。

对于$V$的子空间$W$,若$\mathbb{F}$上的多项式$f$满足$f(A)\bvec{w}=\bvec{0}$对任意$\bvec{w}\in W$成立,则称$f$是$A$在$W$上的\textbf{零化多项式(null polynomial)}。
\end{definition}


由\autoref{def_MinPol_1} 显然可知,$f$是$A$在$V$上的零化多项式,当且仅当$f(A)=0$。


对于任意线性变换,其零化多项式一定存在,如以下定理所说:

\begin{theorem}{Hamilton-Cayley定理}\label{the_MinPol_1}
给定线性空间$V$上的线性变换$A$,$f(\lambda)$为其特征多项式,则$f(A)=0$。
\end{theorem}

\textbf{证明}:

任取$V$的一组基,将线性变换$A$表示为矩阵$\bvec{A}$,恒等变换$I$的矩阵则必为$\bvec{I}$。

令
\begin{equation}\label{eq_MinPol_1}
f(\lambda) = \sum_{i=0}^n a_i\lambda^i~, 
\end{equation}
再定义$\lambda\bvec{I}-\bvec{A}$的伴随矩阵为
\begin{equation}\label{eq_MinPol_3}
\bvec{B}(\lambda)=\sum_{i=0}^n \bvec{B}_i\lambda^i~, 
\end{equation}
其中各$\bvec{B}_i$是常数矩阵。

由伴随矩阵和特征多项式的定义,
\begin{equation}\label{eq_MinPol_2}
\bvec{B}(\lambda)\qty(\lambda\bvec{I}-\bvec{A}) = \bvec{I}\opn{det}(\lambda\bvec{I}-\bvec{A}) = \bvec{I}f(\lambda)~. 
\end{equation}

将\autoref{eq_MinPol_1} 和\autoref{eq_MinPol_3} 代入\autoref{eq_MinPol_2} ,比较各$\lambda^i$的系数可得
\begin{equation}
\left\{
\begin{aligned}
\bvec{B}_{n-1} ={}& a_n\bvec{I}; \\
\bvec{B}_{n-2}-\bvec{B}_{n-1}\bvec{A} ={}& a_{n-1}\bvec{I}; \\
\cdots{}&\\
\bvec{B}_{0}-\bvec{B}_{1}\bvec{A} ={}& a_1\bvec{I};\\
-\bvec{B}_{0}\bvec{A} ={}& a_0\bvec{I}. 
\end{aligned}
\right. ~
\end{equation}

用$\bvec{A}^{n+1-k}$\textbf{右乘}上式的第$k$个等式,即得
\begin{equation}\label{eq_MinPol_4}
\left\{
\begin{aligned}
\bvec{B}_{n-1} \bvec{A}^n={}& a_n\bvec{A}^n; \\
\bvec{B}_{n-2}\bvec{A}^{n-1}-\bvec{B}_{n-1}\bvec{A}^n ={}& a_{n-1}\bvec{A}^{n-1}; \\
\cdots{}&\\
\bvec{B}_{0}\bvec{A}-\bvec{B}_{1}\bvec{A}^2 ={}& a_1\bvec{A};\\
-\bvec{B}_{0}\bvec{A} ={}& a_0\bvec{I}. 
\end{aligned}
\right. ~
\end{equation}
将\autoref{eq_MinPol_4} 中所有式子加起来,比较左右两端即可得
\begin{equation}
\bvec{0} = \sum_{i=0}^n a_i\bvec{A}^i~. 
\end{equation}

\textbf{证毕}。



\subsection{极小多项式}

另见\autoref{sub_RPlynm_1}~\upref{RPlynm}.

\begin{definition}{极小多项式}

设$V$是域$\mathbb{F}$上的线性空间,$A$是$V$上的线性变换。若$\mathbb{F}$上的多项式$f$是$A$的零化多项式,且对于$A$的任意零化多项式$g$,总存在多项式$h$使得$g=fh$,那么称$f$是$A$的\textbf{极小多项式(minimal polynomial)},或\textbf{最小多项式}。

\end{definition}


由\autoref{the_MinPol_1} ,任何一个线性变换都有零化多项式;那么极小多项式是否总是存在呢?


\begin{theorem}{极小多项式存在性}
设$V$是域$\mathbb{F}$上的线性空间,$A$是$V$上的线性变换,则$A$有极小多项式。
\end{theorem}

\textbf{证明}:

任取$A$的零化多项式$f$,如果不存在次数比$f$更低的零化多项式,那么$f$就是极小多项式。这是因为,此时对如果$f$不是极小多项式,则可以取零化多项式$g$,使得
\begin{equation}\label{eq_MinPol_5}
g=fh+r~, 
\end{equation}
其中$h$和$r$都是多项式,且$\opn{deg}r<\opn{deg}f$。显然,$f$和$gh$都是$A$的零化多项式,因此$r$也是,而$r$的次数比$f$的次数低,与假设矛盾。

如果存在比$f$更低的零化多项式$g$,那么存在多项式$$


\textbf{证毕}。





取线性变换$A$的\textbf{零化多项式}$f$和$g$,那么存在$\mathbb{F}$上的多项式$h$和$r$使得











