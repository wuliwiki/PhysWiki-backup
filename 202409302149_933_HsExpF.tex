% 指数函数(高中)
% keys 指数|指数函数|自然常数
% license Usr
% type Tutor

\pentry{函数\nref{nod_functi},函数的性质\nref{nod_HsFunC},幂运算与幂函数\nref{nod_power}}{nod_d767}

\begin{issues}
\issueDraft
\end{issues}

在日常生活中,我们常常遇到指数函数的应用。想象一下,细菌每过一小时就分裂一次,数量翻倍。刚开始可能只是几百个细菌,但很快,数量会成千上万地增长,这正是指数增长的典型例子。通过指数函数,你能理解这种现象,并学会如何计算这些快速变化的量。再比如,银行里的利息也是按照一定的比例增长,时间越长,利息越多,当时间长到一定程度,大部分的帐目都将是利息,它将远超过本金,这也是典型的指数增长现象。就像雪球越滚越大,指数函数也能迅速变得非常大。

这种增长速度与自身数量直接相关的性质就指数增长,它的速度比幂函数快得多。在一个环境比较理想的情况下,大部分事物的增长都会遵循这个规律,直到环境变化或达到环境的承载能力,才会改变。指数函数的这个性质使它在很多实际问题中应用广泛。比如,地球人口在几十年内翻倍之类的现象背后都有指数函数在起作用。通过学习指数函数,你不仅能解决这些问题,还能更好地理解世界中的很多快速变化现象。

\subsection{指数函数}

回看幂运算的\aref{定义}{def_power_1},如果将底数作为参数,指数作为自变量的函数就称为指数函数,指数函数的名称指的就是自变量的在指数位置上,注意不要与幂函数相混淆。

\begin{definition}{指数函数}
形如
\begin{equation}
f(x) = a^x~.
\end{equation}
的函数称作\textbf{指数函数(exponential function)},其中 $a\in\mathbb R^+$。
\end{definition}

这里之所以要求参数$a\in\mathbb R^+$,是因为负数、$0$的实数幂次在非常多点上是未定义的,造成函数的定义域不连续,难以研究\footnote{类比狄利克雷函数可知,这个函数无法画出图像}。

\subsection{指数函数的性质}

有了幂函数的经验,同样,下面根据$a(a>0)$的性质讨论指数函数的性质。

\subsubsection{定义域和值域}

根据指数函数参数的限定,$x$可以取任意实数。此时根据幂运算的法则,$f(x)>0$恒成立,即函数的值域是$(0,+\infty)$。

这里有一个特例,$a=1$的情况是平凡的,它是直线$y=1$,它在后面的研究中会作为一个参考基准,讨论时不会参与$a=1$的情况。

根据$0^a=1$可知,函数恒过定点$(0,1)$。

\subsubsection{单调性}

对于$f(x)=a^x$,任取定义域上的两点$x_1>x_2$,则平均变化率为

\begin{equation}\label{eq_HsExpF_1}
\frac{f(x_1)-f(x_2)}{x_1-x_2}=\frac{a^{x_2}(a^{x_1-x_2}-1)}{(x_1-x_2)}~.
\end{equation}

由于\autoref{eq_HsExpF_1} 的分母和$a^{x_2}$为正,因此讨论$a^{x_1-x_2}$和$1$的关系。由于$x_1>x_2$,$x_1-x_2>0$。为了讨论方便,取$1^{x_1-x_2}$。由于幂函数在参数为正时在第一象限内是增函数,因此$a>1$时,$a^{x_1-x_2}>1^{x_1-x_2}$,$0<a<1$时,$a^{x_1-x_2}<1^{x_1-x_2}$。

综上,$a>0$时,\autoref{eq_HsExpF_1} 的值大于$0$,函数在定义域上是递增的;反之$0<a<1$时,函数则是递减的。

\subsubsection{不同的$a$的函数图象的关系}

当$a_1>a_2$时,有$a_1-a_2>0$,讨论两个函数间的关系\footnote{这种写法表示参数是$a_1,a_2$}

\begin{equation}
f(x;a_1)-f(x;a_2)=(a_1)^x-(a_2)^x=(a_2)^x\left(\left(\frac{a_1}{a_2}\right)^x-1\right)~.
\end{equation}

由于$\displaystyle (a_2)^x>0$,取$p=\frac{a_1}{a_2}>1$,下面讨论$p^x$和$1$的关系。对$x>0$时,$p^x>1$,$x<0$时,$\displaystyle p^x=\left(\frac{1}{p}\right)^{|x|}<1$。

形象地来说,在$x>0$时,如果从下到上画一条垂线,则会先穿过$a$值较小的函数图象,后穿过$a$值较大的函数图象;在$x<0$时从下到上画一条垂线,则会先穿过$a$值较大的函数图象,后穿过$a$值较小的函数图象。

由于$1^x=1$,根据上面的分析,当$a>1$时,$x<0$时$0<y<1$,$x>0$时$y>1$;当$0<a<1$时,$x<0$时$y>1$,$x>0$时$0<y<1$。

仔细对比,可以发现这里“单调性”和“不同的$a$的函数图象的关系”的讨论与幂函数讨论的表达式形式正好相反。

\subsubsection{其他性质}

由于$\displaystyle \left(\frac{1}{a}\right)^{-x}=a^{x}$,所以如果代入$(-x,y)$到$\displaystyle\left(\frac{1}{a}\right)^{x}$的表达式中,得到的它关于$y$轴对称的函数是$a^{x}$。

顺便一提,若$a>1$,则$a^x$在$x$趋于$-\infty$时,趋于$0$;在$x$趋于$+\infty$时,趋于$+\infty$。若$1>a>0$,则$a^x$在$x$趋于$-\infty$时,趋于$+\infty$;在$x$趋于$+\infty$时,趋于$0$。这是根据幂运算的性质得到的。同样关于“趋于”、“无穷”这两个词,现在只需要有一个感性的理解就可以了,它是符合几何直观的。关于他们的具体内涵,会在大学阶段学习。

\subsubsection{函数图像}

根据上面的分析可以得到两类函数的图像,分别是$a>1$和$0<a<1$情况的。
\addTODO{图像}
函数是光滑的,并且$|\ln(a)|$\footnote{这里的符号是\enref{对数}{Ln},此处如果看不懂可以先跳过。}越大,越会靠近$y$轴;$|\ln(a)|$越小,越会靠近直线$y=1$。

\subsection{指数爆炸}

\textbf{指数爆炸(exponential growth )}指的是函数值随自变量呈指数级别的快速增长,它的显著特征是初期增速缓慢,但随后会急剧加速。指数函数的增长速度非常快,对于初等函数而言,当参数 x 足够大时,指数函数的增长速度是最快的。具体来说,若参数 $a > 1$,在第一象限内($x > 0$)的典型函数增长速度从慢到快通常满足以下顺序:

\begin{equation}
 a < \log_a{x} <x^a < a^x~.
\end{equation}

式子中,常数 $a$ 是一个固定值,不随 $x$ 改变,或者说不增加。\enref{对数函数}{Ln} $\log_a{x}$ 的值在 $x$越大时,仍在增加,但增速会越来越慢,仅略大于不增。\enref{幂函数}{power} $x^a$ 和指数函数 $a^x$的增长速度都会随着 $x$ 增加,$x^a$ 增速逐渐加快,但$x^a$比指数函数 $a^x$ 慢,一般认为相较于指数函数,幂函数是线性或近似线性的。指数增长会呈现“爆炸式”的加速,远超其他初等函数。

\subsection{柯西函数方程}

\textbf{柯西函数方程 (Cauchy functional equation)}是柯西提出的是数学分析中具有加性和乘性特征的几个方程。它们的形式如下:
\begin{enumerate}
\item $f(x+y)=f(x)+f(y)$
\item $f(xy) = f(x) f(y)$
\item $f(x+y)=f(x)f(y)$
\item $f(xy) = f(x)+f(y)$
\end{enumerate}

刚看到可能觉得有点吓人,下面以第一个作为例子表示一下它的含义:函数$f(x)$满足,任取两个自变量的值时,这两个自变量$x,y$的和$x+y$对应的函数值$f(x+y)$与他们对应的函数值$f(x),f(y)$的和$f(x)+f(y)$相等。这里的$y$只表示某一个可以任意赋值的,与$x$无关的自变量。

由于他们无关,可以任意给他们赋值来研究它的性质:
\begin{itemize}
\item 令$y=x$,有$f(2x)=2f(x)$,进而更多地得到对自然数$n$有$f(nx)=nf(x)$,代入$x=1$有$f(n)=nf(1)$。
\item 令$x=y=0$,有$f(0)=2f(0)$,因此$f(0)=0$,即函数恒过定点$(0,0)$。
\item 令$y=-x$,有$f(0)=f(x)+f(-x)$,又因为$f(0)=0$,从而$-f(x)=f(-x)$,这说明函数应该是奇函数。
\item 令$\displaystyle x=\frac{q}{p}$,即$q=px$。因为$p,q$为互质的整数,有$f(p)=pf(1)$,$f(q)=qf(1)$。由于$p$是整数,$f(q)=f(px)=pf(x)$。综上$\displaystyle f(x)=\frac{f(q)}{p}=\frac{q}{p}f(1)=xf(x)$
\end{itemize}

上面的方法是研究此类抽象函数的一种普遍方法,通过上面的推理已经可以证明$f(x)=xf(1)$对任意有理数$x$都成立,设$f(1)=c$的话,可知函数表达式为$f(x)=cx$。事实上,这个表达式对$x\in\mathbb{R}$都成立,不过证明就超过高中范畴了。

其他的方程也可以通过类似的方法进行研究,不过好在,这些方程已经被很多人研究过,下面直接给出各个函数的某种解的形式\footnote{可能有其他的解,但不在此处的讨论范畴内。},可以自行带入验证:

\begin{enumerate}
\item $f(x+y)=f(x)+f(y)\implies f(x)=cx$(正比例函数、线性函数)
\item $f(xy)=f(x)f(y)\implies f(x)=x^a$(\enref{幂函数}{power})
\item $f(x+y)=f(x)f(y)\implies f(x)=a^x$(指数函数)
\item $f(xy)=f(x)+f(y)\implies f(x)=\log_ax$(\enref{对数函数}{Ln})
\end{enumerate}

因此,也可以说,在某个视角上,这几种函数根本的性质就是柯西函数方程描述的样子。

\subsection{自然常数$\E$}

自然常数一般会通过下面这个例子来引入。假设在银行存入1元,银行承诺年利率为$100\%$,利息的计算公式是“利息=本金\times年利率\times存款年数(时间)”。下面的计算不要关注每次计算的得到的数值,而是要关注计算过程的变化整合。

最简单的情况是银行一年只结算一次利息,这时年末得到的收入就是$1\times100\%\times1=1$。这样,一年后1元会变成$1+1\times100\%=1\times(1+100\%)^1=2$元。

如果要求银行“每半年结算一次利息” ,这样计算的好处是,第一次结算之后的利息会作为本金参与到第二次的计算中。于是第一次结算时,利息为$1\times100\%\times\frac{1}{2}=0.5$。第二次计息时的本金变成了$1+1\times100\%\times\frac{1}{2}=1.5$。第二次的利息就是

银行答应了,但规定每次给你50%的利息。六个月后,你的1元变成了 1.5元(因为1元加上50%的利息),接下来,再过六个月,银行会给你 1.5元 的50%的利息(不是1元了),这时你有 2.25元。每次提前结算,你都会比上一阶段赚得更多一点。

3. 每季度计算一次利息

现在你要求银行每三个月(季度)计算一次利息,这样每次的利息就是25%。第一次三个月后,你有 1.25元,接着再过三个月,你的 1.25元 得到25%的利息,变成 1.5625元,再接着,你的利息会继续增长,最终一年结束时,你会有 2.4414元 左右。看起来,每次频繁计算利息,你的钱都会稍微多一些。

4. 每月计算一次利息

你再进一步,要求银行每个月计算一次利息(12次/年)。现在每个月你拿到的利息是1/12的100%,也就是8.33%。第一个月结束时,你有 1.0833元,第二个月有 1.1694元,如此继续,年底你会有大约 2.613元。相比之前,你赚得又多了一些。



想象你有一个储蓄账户,银行给你存的每一元钱每年利息1元。那么假如你每年结一次息,存一年后你的账户里就会有2元。如果我们想让银行更频繁地给你计算利息,比如每半年算一次,每次给你一半的利息,这样你的钱就会稍微多一点:你半年后会有1.5元,再半年后会有2.25元。再继续细化,按季度、按月、按天计算,利息虽然每次越来越小,但账户里的钱却在不断增长。

那么问题来了:如果我们让银行每天甚至每秒都给你计算利息,那你的钱能涨到多少?这个增长速度有一个极限值,这个值就是大约2.71828…… 这就是自然常数 e!它描述了增长速度的极限,是数学中研究“不断变化”的核心工具,比如在微积分中讨论的瞬时变化率,e就是这种变化的“终极助手”。

简单来说,e 代表着在不受限制的情况下,某种东西增长到最快时能达到的程度。

这就是自然常数$\E \approx 2.71828$,不仅与利息有关,它还出现在很多自然现象中,比如人口增长和放射性衰变,它描述的是指数增长和衰减的规律。它和早已在小学时就接触过的$\pi$有许多相似点。

他们都是无理数,这意味着它们不能表示为两个整数的比值。它们的小数部分是无限且不循环的,也就是说,在任何整数进制中它们都永远不会终止或重复。

他们也都是超越数,意思是它们不能作为任何有理系数多项式方程的解。换句话说,这比无理数的要求更加严格。它们不仅不能表示为整数之比,也不能通过各阶的根式表示。$\E$的超越性由查尔斯·埃尔米特(Charles Hermite)在1873年证明,$\pi$的超越性由费迪南德·冯·林德曼(Ferdinand von Lindemann)在1882年证明。

二者都可以用无穷展开的方式来表示,下面给出两个常见的展开方式\footnote{关于求和符号可以参考\enref{求和符号(高中)}{SumSym},关于阶乘可以参考\enref{阶乘(高中)。}{factor}}:
\begin{equation}
\pi=4\sum_{n=0}^\infty\frac{(-1)^i}{2i+1}~.
\end{equation}
\begin{equation}
\E=\sum_{n=0}^\infty\frac{1}{i!}~.
\end{equation}

$e$的定义有很多种方式,极限定义是被广为了解的,但它有些抽象。本文将另给出一个定义,这个定义比较简单,但在了解微分运算之后,才能体会这个定义的简洁。
假设某个初始值进行增长,随着增长次数变得无限多且增长的频率越来越频繁,最终增长的极限值是 $\E$。

\begin{equation}
\E = \lim_{n \to \infty} \left( 1 + \frac{1}{n} \right)^n~.
\end{equation}


微积分中的自然对数函数:$e$ 是使得
\begin{equation}
\frac{d}{dx} f(x) = f(x)~.
\end{equation}
成立的指数函数的底数,意味着以 $e$ 为底的指数函数是唯一的保持自身斜率不变的增长函数。

