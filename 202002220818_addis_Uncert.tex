% 不确定原理

\pentry{平均值\upref{QMavg}, 柯西不等式\upref{CSNeq}}

\subsection{位置—动量不确定原理}
单个粒子一维运动的波函数 $\psi(x)$ 的位置和动量的标准差为 $\sigma_x$ 和 $\sigma_p$
\begin{equation}\label{Uncert_eq2}
\sigma_x \sigma_p \leqslant \frac{\hbar}{2}
\end{equation}

\begin{example}{高斯波包}\label{Uncert_ex1}
(未完成)
\end{example}

\subsection{不确定原理的拓展}
任意两个物理量 $A$ 和 $B$ 都满足
\begin{equation}
\sigma_a \sigma_b \leqslant \frac{1}{2\I}\mel{v}{[A, B]}{v}
\end{equation}

\subsection{证明}
令 $f = (A-a)v$, $g = (B-b)v$, 使用柯西不等式\upref{CSNeq} 有
\begin{equation}\label{Uncert_eq1}
\sigma_a^2 \sigma_b^2 = \braket{f}{f} \braket{g}{g} \leqslant \abs{\braket{f}{g}}^2 = (\Re\braket{f}{g})^2 + (\abs{\Im\braket{f}{g}})^2
\end{equation}
其中
\begin{equation}
\begin{aligned}
\Im\braket{f}{g} &= \frac{1}{2\I}(\braket{f}{g} - \braket{g}{f})
= \frac{1}{2\I}\mel{v}{[(A-a), (B-b)]}{v}\\
&= \frac{1}{2\I}\mel{v}{[A, B]}{v}
\end{aligned}
\end{equation}
忽略\autoref{Uncert_eq1} 中的 $\Re$ 项(大于等于 0), 则有\footnote{容易证明, $\mel{v}{[A, B]}{v}$ 必然是一个纯虚数, 所以\autoref{Uncert_eq4} 右边是实数.}
\begin{equation}\label{Uncert_eq4}
\sigma_a \sigma_b \leqslant \frac{1}{2\I}\mel{v}{[A, B]}{v}
\end{equation}
令 $A = x, B = p = -\I\hbar \dv*{x}$, 有正则对易关系
\begin{equation}
[A, B] = [x, p] = \I\hbar
\end{equation}
带入\autoref{Uncert_eq4} 得\autoref{Uncert_eq2}. 

最后我们需要证明存在波函数 $\ket{v}$ 使得不等式可以取等号, 而高斯波包可以满足这个要求(\autoref{Uncert_ex1}).% 未完成:链接
证毕.
