% 系统的角动量
% 动量|角动量|角动量分析|坐标系变换|质心系

\pentry{角动量、角动量定理、角动量守恒(单个质点)\upref{AMLaw1}, 质点系\upref{PSys}}

\subsection{系统的角动量}
角动量是矢量,若把系统看做质点系,则系统的角动量等于所有质点的角动量矢量相加\upref{GVecOp}。
\begin{equation}\label{AngMom_eq1}
\bvec L = \sum_i \bvec L_i = \sum_i \bvec r_i \cross\bvec p_i = \sum_i m_i \bvec r_i \cross\bvec v_i~.
\end{equation}
\begin{example}{旋转圆环}\label{AngMom_ex1}
在\autoref{AMLaw1_ex2}~\upref{AMLaw1}中, 我们知道单个质点做圆周运动的角动量为 $\bvec L = m r^2 \omega \bvec z$。 现在考虑一个半径为 $r$ 质量为 $M$ 的细圆环绕处于原点的圆心做逆时针圆周运动, 每个质点的轨迹就是圆环本身。 如果我把圆环看成由许多质点 $m_i$ 组成, 每个质点都做上述圆周运动, 则总角动量为
\begin{equation}
\bvec L = \sum_i m_i \omega r^2 \uvec z = M \omega r^2 \uvec z~.
\end{equation}
\end{example}

\subsection{角动量的坐标系变换}
可类比力矩的坐标系变换(\autoref{Torque_eq5}~\upref{Torque}),坐标系 $A$ 中总角动量为
\begin{equation}
\bvec L_A = \sum_i \bvec r_{Ai} \cross \bvec p_i ~.
\end{equation}
变换到与 $A$ \textbf{相对静止的}坐标系 $B$ 中, 从 $B$ 原点指向 $A$ 原点的矢量为 $\bvec r_{BA}$, 那么总角动量为
\begin{equation}\label{AngMom_eq4}
\bvec L_B = \sum_i (\bvec r_{BA} + \bvec r_{Ai})\cross \bvec p_i = \bvec r_{BA}\cross \sum_i \bvec p_i + \bvec L_A~.
\end{equation}
可见当系统总动量为零时, 两参考系中计算系统角动量的结果相同。 注意由于两参考系相对静止, 以上的 $\bvec p_i$ 在两参考系中都是相同的。

\begin{example}{旋转圆环 2}\label{AngMom_ex2}
如果我们把\autoref{AngMom_ex1} 中的旋转圆环平移一下再计算(关于原点)的角动量, 根据上述讨论, 结果仍然是相同的。这是因为圆环中的每个质点都存在一个反方向运动的质点, 圆环的总动量为零。
\end{example}

\subsection{角动量的质心系分解}
若当前参考系不是质心系, 令质心的位置矢量为 $\bvec r_c$, 速度为 $\bvec v_c$。 每个质点在质心系中的位置为 $\bvec r_{ci}$, 速度为 $\bvec v_{ci}$, 令系统总质量为 $M = \sum_i m_i$, 则系统角动量可以表示为
\begin{equation}\label{AngMom_eq2}
\begin{aligned}
\bvec L &= \sum_i m_i \bvec r_i \bvec v_i = \sum_i m_i (\bvec r_c + \bvec r_{ci})\cross( \bvec v_c + \bvec v_{ci})\\
&= \bvec r_c \cross (M\bvec v_c) + \sum_i m_i \bvec r_{ci}\cross \bvec v_{ci} + \bvec r_c \cross \sum_i m_i \bvec v_{ci} + \qty(\sum_i m_i \bvec r_{ci}) \cross \bvec v_c~.
\end{aligned}
\end{equation}
根据质心系的性质, 最后两项中的求和为零。 另外根据\autoref{SysMom_eq2}~\upref{SysMom}, $M\bvec v_c$ 就是系统总动量, 也相当于所有质量集中于质心获得的动量。 所以我们可以把第一项定义所谓的 “质心的角动量”
\begin{equation}
\bvec L_0 = \bvec r_c \cross (M \bvec v_c)~.
\end{equation}
而第二项就是质心系中系统的角动量
\begin{equation}
\bvec L_c = \sum_i m_i \bvec r_{ci}\cross \bvec v_{ci}
所以\autoref{AngMom_eq2} 可以记为
\begin{equation}\label{AngMom_eq3}
\bvec L = \bvec L_0 + \bvec L_c
\end{equation}
所以\textbf{任何坐标系中,系统的总角动量等于其质心的角动量加上相其相对质心的角动量}。 注意与上一小节不同的是, 我们不要求两个参考系相对静止。

\begin{exercise}{旋转圆环 3}
在\autoref{AngMom_ex2} 的基础上, 如果圆环一边旋转, 圆心一边以速度 $\bvec v$ 做匀速直线运动, 直线与当前原点距离为 $d$, 求系统的总角动量。 角动量是否守恒?
\end{exercise}
