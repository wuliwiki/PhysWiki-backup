% 高阶微分(多元函数)
% keys 多元函数|高阶微分
\subsection{二阶微分}
\pentry{全微分(简明微积分)\upref{TDiff}}
设在区域 $\mathcal{D}$ 中给定一函数 $u=f(x_1,\cdots,x_n)$ ,它有着一阶连续偏导数\upref{ParDif}.那时,称为全微分 $\dd{u}$ 的,就是如下表达式\autoref{TDiff_eq1}~\upref{TDiff}
\begin{equation}
\dd{u}=\sum_{i=1}^n \pdv{u}{x_i}\dd{x_i}
\end{equation}
式中的 $\dd{x_i}\quad(i=1,\cdots,n)$ 是自变量 $x_i$ 的任意增量.全微分也称为\textbf{一阶微分}.

可以看出,$\dd{u}$ 也是一个 $x_1,\cdots,x_n$ 的函数.若假定 $u$ 有二阶连续偏导数,则 $\dd{u}$ 就有一阶连续偏导数,于是就能说微分 $\dd{u}$ 的全微分 $\dd{(\dd u)}$ ,它称为 $u$ 的\textbf{二阶微分},用记号 $\mathrm{d}^2u$ 表示.

\textbf{需重点指出},在这时增量 $\dd{x_1},\cdots,\dd{x_n}$ 被看着常数,且当由一个微分转移到下一微分时,仍保持同一数值.

利用微分法则(链接),就有
\begin{equation}
\begin{aligned}
\dd{^2u}&=\dd{(\dd{u})}=\dd{\qty(\sum_{i=1}^n \pdv{u}{x_i}\dd{x_i})}\\
&=\sum_{i=1}^n \dd{\qty(\pdv{u}{x_i})}\dd{x_i}\\
&=\sum_{i=1}^n\qty(\sum_{j=1}^n\pdv{^2u}{x_i\partial x_j} \dd{x_j})\dd{x_i}\\
&=\sum_{i,j=1}^n\pdv{^2u}{x_i\partial x_j} \dd{x_i}\dd{x_j}
\end{aligned}
\end{equation}
\subsubsection{记号约定}
在一阶微分的情形,约定“将字母 $u$ 移到求和号外”;于是可记为
\begin{equation}
\dd{u}=\qty(\sum_{i=1}^n\pdv{}{x_i}\dd{x_i})u
\end{equation}
若在二阶微分情形,也约定 “将字母 $u$ 移到求和号外”,就有
\begin{equation}
\dd{^2u}=\qty(\sum_{i,j=1}^n\pdv{^2}{x_i\partial x_j} \dd{x_i}\dd{x_j})u
\end{equation}
因此,二阶微分可以记号化地写成
\begin{equation}
\dd{^2u}=\qty(\sum_{i=1}^n\pdv{}{x_i} \dd{x_i})^2u
\end{equation}
\subsection{高阶微分}
仿照二阶微分的定义,若 $(k-1)$ 阶微分 $\dd{^{k-1}}u$ 已确定, 则\textbf{$k$阶微分} $\dd{^ku}$ 就定义为 $(k-1)$ 阶微分的全微分:
\begin{equation}
\dd{^k u}=\dd{(\dd{^{k-1}}u)}
\end{equation}
若函数 $u$ 存在着直至 $k$ 阶为止的所有各阶的连续偏导数,则 $k$ 阶微分的存在就有了保证.

 同样的,对任意的 $k$ ,有记号化的等式
 \begin{equation}\label{MHDiff_eq1}
 \dd{^ku}=\qty(\sum_{i=1}^n\pdv{}{x_i}\dd{x_i})^k u
 \end{equation}
 \textbf{这个公式必须这样理解:}首先把括号内的多项式按照代数学的乘幂法则形式地展开,以后所有各项“乘” 以 $u$(补写在分子 $\partial^k$ 后面),仅在这一步后,一切记号方才回到导数及微分的意义.
 \subsubsection{\autoref{MHDiff_eq1} 的证明}
 当 $k=1,2$ 时,我们已看到为真;故只需证明,若它对于 $\dd{^k u}$ 为真,则对于 $\dd{^{k+1}u}$ 为真即可.
 
 假设这法则对于 $\dd{^k u}$ 能成立,就有展开式:
 \begin{equation}\label{MHDiff_eq2}
 \begin{aligned}
 \dd{^k u}&=\qty(\sum_{i=1}\pdv{}{x_i}\dd{x_i})^n u\\
 &=\sum C_{\alpha_1,\cdots,\alpha_n}\cdot\pdv{^k u}{x_1^{\alpha_1}\cdots\partial{x_n}^{\alpha_n}}\dd{x_1^{\alpha_1}}\cdots\dd{x_n^{\alpha_n}}
 \end{aligned}
 \end{equation}
 其中的总和是关于条件 $\sum_{i=1}^n\alpha_i=k$ 的非负整数 $\alpha_i$ 的一切可能组合而取的,且
 \begin{equation}
 C_{\alpha_1,\cdots,\alpha_n}=\frac{k!}{\alpha_1!\cdots\alpha_n!}
 \end{equation}
 由假定,存在 $(k+1)$ 阶连续偏导数,微分