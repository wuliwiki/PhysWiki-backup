% C++ 原子变量原子操作笔记
% keys 原子变量|原子操作|C++|多线程|并发
% license Usr
% type Tutor

\begin{issues}
\issueDraft
\issueTODO
\end{issues}


C++ 11 中引入的原子变量(atomic)是在多线程编程中的常用同步机制。它可以避免竟态条件(race condition)和死锁(dead lock)问题。它可以保证对共享变量的操作在执行时不会被其他线程的操作干扰。

举个例子,以下 C++ 代码:
\begin{lstlisting}[language=cpp]
// C++ 11
#include <iostream>
#include <thread>
using namespace std;
int n(0);

void cnt1e4() {
	for(int i(1); i <= 1e4; ++i) ++ n;
}

int main() {
	thread ths[100];
	for(thread &thr: ths) thr = thread(cnt1e4);
	for(thread &thr: ths) thr.join();
	cout << n << endl;
	return 0;
}
\end{lstlisting}
理论输出应为 $10^4 \times 100 = 1000000$,而实际输出总小一些,例如在某计算机上运行三次结果分别为:
\begin{figure}[ht]
\centering
\includegraphics[width=10cm]{./figures/a6a955b5d96c1353.png}
\caption{第一次运行结果} \label{fig_atomVO_1}
\end{figure}
\begin{figure}[ht]
\centering
\includegraphics[width=10cm]{./figures/5000b1ff19d8f7a3.png}
\caption{第二次运行结果} \label{fig_atomVO_2}
\end{figure}
\begin{figure}[ht]
\centering
\includegraphics[width=10cm]{./figures/bf4fe703627db114.png}
\caption{第三次运行结果} \label{fig_atomVO_3}
\end{figure}
这类问题尤其在大量不同线程对于同一变量进行操作时出现,更多的线程数量可能导致更大的误差。以两个线程为例,从计算机对变量的操作的角度来考虑这个问题:
对于某一个线程,电脑此时要对 $n$ 执行 $+1$ 操作,顺序为:
\begin{enumerate}
\item 从主内存中加载 $n$ 的值到线程工作内存;
\item 执行 $+1$ 操作;
\item 把第二步的执行结果从工作内存写入到主内存。
\end{enumerate}
而当有两个线程或是更多个线程的时候,可能会出现以下情况:
\begin{enumerate}
\item 线程 1 从主内存中加载 $n$ 的值 $n_{old}$ 到线程 1 到工作内存;
\item 线程 2 从主内存中加载 $n$ 的值 $n_{old}$ 到线程 2 到工作内存;
\item 线程 1 执行 $+1$ 运算得到结果 $n_{old}+1$;
\item 线程 2 执行 $+1$ 运算得到结果 $n_{old}+1$;
\item 线程 1 把 $n_{old}+1$ 写入主内存中的 $n$ 变量;
\item 线程 2 把 $n_{old}+1$ 写入主内存中的 $n$ 变量;
\end{enumerate}
此时 $n$ 的值本应为 $n_{old} + 2$,实际却是 $n_{old}+1$,当大量不同线程同时对 $n$ 进行类似操作时更容易出现这类问题,所以引入了原子变量与原子操作的概念。

\subsection{原子变量}
\verb`atomic` 是一个模板类,可以通过 \verb`atomic<_typName> varName;` 来声明类型为 \verb`_typName` 的、变量名为 \verb`varName` 的原子变量,前提是 \verb`_typName` 类型合法。

对于所有的原子变量都有以下四个常用的公开成员函数:

下面通过介绍一些使用 \verb`atomic` 类型的例子来介绍其用法,并给出一些例程。

\subsubsection{对整数类型的特化}
原子变量对整数类型进行了特化,具体的有:
\begin{enumerate}
\item 字符类型 \verb`char`、\verb`char8_t` (C++20 起)、\verb`char16_t`、\verb`char32_t` 和 \verb`wchar_t`;
\item 标准有符号整数类型:\verb`signed char`、\verb`short`、\verb`int`、\verb`long` 和 \verb`long long`;
\item 标准无符号整数类型:\verb`unsigned char`、\verb`unsigned short`、\verb`unsigned int`、\verb`unsigned long` 和 \verb`unsigned long long`。
\end{enumerate}
这些类型可以直接使用 \verb`atomic_类型` 的方式直接声明变量。
其中,\verb`unsigned` 的类型加前缀 \verb`u`,\verb`long long` 缩写为 \verb`llong`。
例如:\verb`atomic_ullong a;`。
对于这些整数变量,提供了专门的 \verb`fetch_add`,\verb`fetch_sub`、\verb`fetch_or` 等函数可以直接进行原子操作。例如:\verb`a.fetch_add(3);` 可以对原子变量 \verb`a` 原子地加 $3$ 并赋值给它。

\subsubsection{C++ 标准}
关于原子变量,有下列相关的 C++ 标准:
C++ 标准在头文件(标头)中提供了以下关于 \verb`atomic` 的定义:
\begin{enumerate}
\item (C++11 起) \verb`template< class T > struct atomic;`
\item (C++11 起) 在标头 <memory> 定义:\verb`template< class U > struct atomic<U*>;`
\item (C++20 起) \verb`template<class U> struct atomic<std::shared_ptr<U>>;`
\item (C++20 起) 在标头 <stdatomic.h> 定义:\verb`template<class U> struct atomic<std::weak_ptr<U>>;`
\end{enumerate}

原子类型要求是满足可复制构造 (CopyConstructible) 、可复制赋值 (CopyAssignable) 的可平凡复制 (TriviallyCopyable) 类型。
其中,可复制构造是指该类型的实例可以从左值表达式进行复制构造,可复制赋值是指该类型的实例可以从左值表达式复制赋值,可平凡复制类型有下表的要求。
\begin{enumerate}
\item 至少有一个合格的复制构造函数,移动构造函数,复制赋值运算符或移动赋值运算符;
\item 每个合格的复制构造函数都是平凡的;
\item 每个合格的移动构造函数都是平凡的;
\item 每个合格的复制赋值运算符都是平凡的;
\item 每个合格的移动赋值运算符都是平凡的;
\item 有一个未被弃置的平凡析构函数。
\end{enumerate}