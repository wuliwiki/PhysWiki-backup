% 过渡矩阵
% 基底|坐标变换|过度矩阵|线性变换|矩阵变换|相似变换|过渡矩阵

\pentry{矢量空间的表示\upref{VecRep}}

%相似变换的本质是同一个线性变换的矩阵表示,在不同基下的变换.因此线性变换和相似变换的相关词条需要重新整理.

\subsection{过渡矩阵}
给定域 $\mathbb{F}$ 上的 $n$ 维矢量空间 $V$.如果 $V$ 有两个基 $\{\bvec{e}_i\}_{i=1}^{n}$ 和 $\{\bvec{e}'_i\}_{i=1}^{n}$,那么由于各 $\bvec{e}'_j$ 也是 $V$ 中的元素,故可以表示成 $\bvec{e}'_j=\sum\limits_{i=1}^na_{ij}\bvec{e}_i$,因此我们可以把矢量也当作矩阵元素,写出以下矩阵等式:

\begin{equation}\label{TransM_eq1}
(\bvec{e}'_1, \cdots\bvec{e}'_n)=(\bvec{e}_1, \cdots\bvec{e}_n)\pmat{a_{11},a_{12},\cdots,a_{1n}\\ a_{21},a_{22},\cdots,a_{2n}\\ \vdots\ddots\vdots\\ a_{n1},a_{n2},\cdots,a_{nn}}
\end{equation}

矩阵 $\bvec{Q}=\pmat{a_{11},a_{12},\cdots,a_{1n}\\ a_{21},a_{22},\cdots,a_{2n}\\ \vdots\ddots\vdots\\ a_{n1},a_{n2},\cdots,a_{nn}}$ 就被称为基 $\{\bvec{e}_i\}_{i=1}^{n}$ 到基 $\{\bvec{e}'_i\}_{i=1}^{n}$ 的一个\textbf{过渡矩阵(transition matrix)}.

\begin{example}{}
证明:过渡矩阵必须是可逆矩阵.
\end{example}

\subsection{过渡矩阵和基的变换}

过渡矩阵本身是表示基之间的变换的,但它也可以用来表示两个不同的基下各种表示的变换,比如向量的坐标、线性变换的矩阵等.

\subsubsection{不同基下坐标的变换}


如果在基 $\{\bvec{e}_i\}_{i=1}^{n}$ 中,向量 $\bvec{v}$ 的坐标是列向量 $\bvec{c}=\pmat{c_1,\cdots, c_n}^T$,那么就有

\begin{equation}\label{TransM_eq2}
\bvec{v}=(\bvec{e}_1, \cdots\bvec{e}_n)\pmat{c_1\\ \vdots \\ c_n}
\end{equation}

同样地,如果在基 $\{\bvec{e}'_i\}_{i=1}^{n}$ 中,$\bvec{v}$ 的坐标是列向量 $\bvec{c}'=\pmat{c'_1,\cdots, c'_n}^T$,则也有

\begin{equation}\label{TransM_eq3}
\bvec{v}=(\bvec{e}'_1, \cdots\bvec{e}'_n)\pmat{c'_1\\ \vdots \\ c_n'}
\end{equation}

考虑到 $\bvec{v}=\bvec{v}$,联立\autoref{TransM_eq1},\autoref{TransM_eq2} 和\autoref{TransM_eq3} 即可得到:

\begin{equation}\label{TransM_eq4}
\pmat{a_{11},a_{12},\cdots,a_{1n}\\ a_{21},a_{22},\cdots,a_{2n}\\ \vdots\ddots\vdots\\  a_{n1},a_{n2},\cdots,a_{nn}}\pmat{c'_1\\ \vdots\\ c_n'}=\pmat{c_1\\ \vdots\\ c_n}
\end{equation}

\autoref{TransM_eq4} 也可以简单记为 $\bvec{Q}\bvec{c}'=\bvec{c}$,或者 $\bvec{c}'=\bvec{Q}^{-1}\bvec{c}$.

\subsubsection{不同基下线性变换的矩阵的变换}


设 $T$ 是 $V$ 上的一个线性变换,基 $\{\bvec{e}_i\}_{i=1}^{n}$ 到 $\{\bvec{e}'_i\}_{i=1}^{n}$ 的过渡矩阵是 $\bvec{Q}$.向量 $\bvec{v}$ 在基 $\{\bvec{e}_i\}_{i=1}^{n}$ 和 $\{\bvec{e}'_i\}_{i=1}^{n}$ 下的坐标分别是 $\bvec{c}$ 和 $\bvec{c}'$,而 $T$ 的矩阵分别是 $\bvec{M}$ 和 $\bvec{M}'$.

$T\bvec{v}$ 在基 $\{\bvec{e}_i\}_{i=1}^{n}$ 和 $\{\bvec{e}'_i\}_{i=1}^{n}$ 下的坐标就分别是 $\bvec{M}\bvec{c}$ 和 $\bvec{M}'\bvec{c}'$.由坐标的变换可知,$\bvec{Q}\bvec{M}'\bvec{c}'=\bvec{M}\bvec{c}$,即 $\bvec{M}'\bvec{c}'=\bvec{Q}^{-1}\bvec{M}\bvec{c}$.

又因为 $\bvec{Q}\bvec{c}'=\bvec{c}$,故有 $\bvec{M}'\bvec{c}'=\bvec{Q}^{-1}\bvec{M}\bvec{Q}\bvec{c}$.

也就是说,如果在基 $\{\bvec{e}_i\}_{i=1}^{n}$ 下,线性变换 $T$ 的矩阵是 $\bvec{M}$,那么在基 $\{\bvec{e}'_i\}_{i=1}^{n}$ 下,$T$ 的矩阵应该是 $\bvec{Q}^{-1}\bvec{M}\bvec{Q}\bvec{c}$.

由此可见,矩阵的\textbf{相似变换}就是同一个线性变换在不同基下的矩阵之间的变换.
