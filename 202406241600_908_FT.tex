% 沸腾(综述)
% license CCBYSA3
% type Wiki

(本文根据 CC-BY-SA 协议转载自原搜狗科学百科对英文维基百科的翻译)

沸腾是一种液体被加热到沸点后快速蒸发的现象,而沸点则是液体蒸气压等于周围大气对液体施加的压力时的温度。沸腾主要有两种类型:成核沸腾和临界热流沸腾。成核沸腾时,小气泡离散地形成。临界热流沸腾时,液体表面被加热到某一临界温度以上,表面会形成一层蒸汽膜。过渡沸腾是一种介于两种沸腾状态下的不稳定的中间态。水的沸点是100摄氏度或212华氏度,但随着海拔升高时,大气压力的降低会造成沸点的降低。

把水煮沸是一种杀死水中微生物的方法,然后水就可以被饮用。不同微生物对热量的敏感性各不相同,一般来说,把水煮到70℃(158℉)并保持十分钟,许多微生物就会死亡,但有些微生物更耐热,所以需要在沸点温度下保持一分钟。

沸腾也可以用于烹饪,可以用来煮蔬菜、淀粉类食物 (如米饭、面条)、土豆、鸡蛋、肉、酱、原汁和汤,是一种简单且适合大规模烹饪的方法。坚韧的肉或家禽也可以被长时间缓慢的烹饪,并产生营养丰富的原汁,但这个过程会造成水溶性维生素和矿物质的流失。商业制备的食品有时用聚乙烯小袋包装起来,作为“袋中可煮”的产品出售。

\subsection{类型}
\subsubsection{1.1 成核}
\begin{figure}[ht]
\centering
\includegraphics[width=6cm]{./figures/aaf350a76195eb3e.png}
\caption{厨房炉灶燃烧器上的水成核沸腾} \label{fig_FT_1}
\end{figure}

成核沸腾的特征是受热表面上气泡离散地出现并且上升,此时表面温度仅略高于液体。通常,成核点的数量随着表面温度的升高而增加。

沸腾容器的不规则表面(即增加了粗糙度的表面)及一些流体添加剂(表面活性剂或纳米颗粒)[1][2]可以产生额外的成核点,[3] 而异常光滑的表面(例如塑料)则会使液体过热。在这种条件下,加热的液体可能会显示沸腾延迟,温度可能会稍微高于沸点而不沸腾。

\subsubsection{1.2 临界热流}
临界热流(CHF)描述了一种加热过程中出现的极限状态。在加热过程中发生相变时(例如在加热水的金属表面上形成气泡),会导致热导率突然降低,从而导致加热表面局部过热。当沸腾表面继续被加热到临界温度以上时,连续的气泡会在加热表面形成一层蒸汽膜。由于这种蒸汽膜散热效率慢,导致温度上升得非常快并且很快进入过渡沸腾状态。临界热流点取决于沸腾流体的特性和加热的表面。[2]

\subsubsection{1.3 过渡}
过渡沸腾可以定义为不稳定沸腾,它的温度介于成核沸腾的最高温度和薄膜沸腾的最低温度之间。

受热液体中气泡的形成是一个复杂的物理过程,通常涉及气穴现象和声学效应。例如人们在热水壶中听到的嘶嘶声时,表示该水壶尚未加热到气泡沸腾到表面的程度。

\subsubsection{1.4 膜}
如果加热液体的表面比液体要热得多,会形成水蒸汽膜,那么就会发生薄膜沸腾。在这种情况下,低热导率的水蒸汽膜会将热表面与液体绝缘。

\subsection{用途}
\subsubsection{2.1 用于饮用水}
将水加热到100℃ (212℉)下达到沸点100℃ (212℉) 是最古老和最有效的对水进行消毒的方式,且不影响味道。尽管水中存在污染物或颗粒,但它是一个有效的消除大多数导致肠道相关疾病的微生物的单步过程。[4]水的沸点在在海拔为0时是100摄氏度。[5] 在有适当水净化系统的地方,煮沸水仅作为紧急处理或在野外和农村地区获得饮用水的方法,因为它不能去除化学毒素或杂质。[6][7]

通过煮沸去除微生物的过程遵循一级动力学——高温则完成快,低温则完成慢。微生物的热敏性各不相同,在70℃(158℉)时,贾第鞭毛虫物种(导致贾第鞭毛虫病)可能需要十分钟才能完全失活,大多数影响肠道的微生物和大肠杆菌(导致胃肠炎)则需要不到一分钟的时间;在沸点,霍乱弧菌(霍乱)需要10秒钟,甲型肝炎病毒(引起黄疸症状)需要1分钟。沸腾并不能确保消灭所有微生物;细菌孢子梭菌能在100℃(212℉)下存活,但不会通过水传播或影响肠道。因此,为了人类健康,并不需要对水进行完全消毒。[4]

将水煮沸十分钟这一传统建议主要是为了增加安全性,因为微生物在高于60摄氏度(140华氏度)的温度下就开始被清除。当水沸腾时,水被消毒,且沸腾现象不用温度计就能看到。虽然沸点随着海拔的升高而降低,但这不足以影响消毒过程。[4][8]

\subsubsection{2.2 烹饪方面}
\begin{figure}[ht]
\centering
\includegraphics[width=6cm]{./figures/0ed51dbc322dc8aa.png}
\caption{煮意大利面} \label{fig_FT_2}
\end{figure}
