% 线性规划
% keys 线性规划|单纯形法
% license Usr
% type Tutor
\pentry{线性方程组(高中)\nref{nod_LinEqu}}{nod_9aef}

1939年,苏联数学家Kantorovich出版了《生产组织与计划中的线性规划模型》一书,为列宁格勒胶合板厂的计划任务建立了一个线性规划的数学模型,为用数学方法解决管理并使二者结合做出了开创性的工作。后来,由于战争的需要,美国经济学家Koopmans重新独立地研究运输问题,并很快看到了线性规划在经济学中应用的意义。此后线性规划也被广泛应用于军事、经济等各方面。鉴于他们在线性规划方面的突出贡献,1975年的诺贝尔经济学奖授予了他们。1947年美国数学家Dantzig提出了求解一般线性规划问题的方法——单纯性法,之后线性规划问题在理论上日益成熟,并在实际中日益广泛应用。

线性规划研究的是在线性不等式或等式限制下,使得某一线性目标取得最大(或最小)的问题。

\subsection{线性规划}

\begin{definition}{线性规划}
若研究的问题可以归结到求解一个关于某些变量的线性函数,使得该函数在变量的某些线性限制条件下,取最大或最小的问题,则称该问题为\textbf{线性规划}模型。若记变量为 $x_1,\cdots,x_n$,所求最大或最小的线性函数为 $z=z(x_1,\cdots,x_n)$,
\end{definition}











