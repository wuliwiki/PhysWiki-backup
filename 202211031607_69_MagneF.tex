% 磁场
% keys 磁感应强度|磁场|磁感线

\begin{issues}
\issueDraft
\end{issues}

\pentry{电场\upref{Efield},矢量叉乘\upref{Cross}}
\subsection{磁场的定义}
在学习静电学的过程中,我们已经知道,空间中某静止的点电荷 $q$ 所受到的电场力可以写为 $\bvec F=q\bvec E$,而 $E$ 是当前这一点的电场强度\footnote{在经典电磁学中,这个电场不包含来自试探电荷 $q$ 本身的部分,否则会出现发散困难.}.然而对于一个运动的电荷,人们发现除了 $q\bvec E$ 以外,还存在一个力的分量,它总是垂直于运动方向而且与速度和电荷量成正比.这意味着空间中还存在某个\textbf{矢量场}\upref{Vfield},我们把它记为 $\bvec B(\bvec r)$,那么完整的电磁力公式中应当还有 $\bvec F'=q\bvec v\times \bvec B$ 这一项.这个矢量场被称为\textbf{磁场}.在本书中\textbf{磁场(magnetic field)}指的是\textbf{磁感应强度(magnetic inductance)}, 一般记为 $\bvec B$.一般地,我们可以用广义洛伦兹力\autoref{Lorenz_eq2}~\upref{Lorenz}来定义: 空间中某点的磁场使得运动时经过该点的点电荷所受的电磁力为
\begin{equation}\label{MagneF_eq1}
\bvec F = q(\bvec E + \bvec v \cross \bvec B)
\end{equation}
其中 $q$ 是电荷量, $\bvec E$ 是该点的电场, $\bvec v$ 是速度.

由于历史原因, “磁场强度” 这个而名字已经被占用% 链接未完成
, 所以 $\bvec B$ 只好叫做磁感应强度. 在比较新的教材中, 磁场一般指磁感应强度. 磁场也可以使用安培力\upref{FAmp}来定义, 但安培力在微观本质上也是洛伦兹力. 


磁感应强度单位为\textbf{特斯拉(Tesla)},也就是 $\Si{kg\cdot C^{-1}s^{-1}}$.注意它乘上速度的量纲和电荷的量纲以后,$\Si{kg\cdot ms^{-2}}$ 刚好是力的量纲,这与洛伦兹力公式\autoref{MagneF_eq1} 是相符的.

与电场线一样, 我们可以在空间中画出许多有方向的\textbf{磁感线}(有时候也称为磁力线), 使得磁感线上任意一点的方向等于该点处磁场的方向, 磁感线在单位横截面的条数与截面处的磁场大小近似成正比. 磁感线通常用于磁场的可视化和帮助理解, 并不是实际存在的,也并不精确.在静电学中,电场能够由电荷的密度分布来确定,那么磁场又应当如何确定呢?

\subsection{静磁学}
\pentry{电流密度\upref{Idens}}
对于随时间变化的电荷与电流分布,其激发的电场与磁场往往是复杂而难以计算的.所以我们希望从最简单的情形入手,研究“静态”的问题.例如,静态的电荷