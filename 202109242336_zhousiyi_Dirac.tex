% 狄拉克场
% 狄拉克|场论|费米子

\begin{definition}{相对论性不变}
如果$\phi$是一个场或者是多个场,$\mathcal D$是微分算符.那么,我们说\textbf{$\mathcal D \phi = 0$是相对论不变的},就是说如果$\phi(x)$满足这个方程,我们再\textbf{对参考系}进行转动或者boost这样的操作,换到别的参考系,则变换后的场,在新的参考系下,满足同样的方程.我们也可以考虑\textbf{物理上}对所有的粒子或者场进行转动或者boost这样的操作,这时候方程$\mathcal\phi = 0$仍然保持不变.这种物理上对场进行操作的办法叫做\textbf{主动}的办法.
\end{definition}
用拉式量写出的场论让洛仑兹不变的讨论变得非常容易.如果一个理论的运动方程是从洛仑兹标量的拉式量推导出来的,那么这个理论的运动方程一定是自动洛仑兹不变的.

考虑如下的洛仑兹变换
\begin{equation}
x^\mu \rightarrow x'^\mu = \Lambda^\mu{}_\nu x^\nu ~. 
\end{equation}
在这边变换下,$\phi$的变换为
\begin{equation}\label{Dirac_eq1}
\phi(x)\rightarrow \phi'(x)=\phi(\Lambda^{-1}x)~.
\end{equation}
这个变换让克莱因-戈登场的拉式量保持不变.
\begin{equation}
\mathcal L(x)\rightarrow \mathcal L(\Lambda^{-1}x)
\end{equation}
运动方程同样是保持不变的
\begin{equation}
(\partial^2+m^2)\phi'(x) = 0 ~.
\end{equation}
\autoref{Dirac_eq1} 这样的变换规则是对于$\phi$这样的场最简单的变换规则.这是对于只有一个分量的场的唯一的一种可能.但是对于有多个分量的场来说,变换规则会更为复杂一些.我们以矢量场的变换来做个例子.

在三维旋转下,矢量场的变换规则为
\begin{equation}
V^i(x) \rightarrow R^{ij} V^j (R^{-1}x)~.
\end{equation}
洛仑兹变换下,矢量场的变换规则为
\begin{equation}
V^\mu(x) \rightarrow \Lambda^\mu{}_{\nu}V^\nu(\Lambda^{-1}x)~.
\end{equation}
任意阶的张量可以通过从矢量增加更多的指标来得到.每增加一个指标,我们就在变换规则前面多加一个$\Lambda$.用这些矢量场和张量场,我们可以写出一些洛仑兹不变的方程,比如说麦克斯韦方程
\begin{equation}
\partial^\mu F_{\mu\nu} = 0~,\quad \partial^2 A_\nu -\partial_\nu \partial^\mu A_\mu = 0~.  
\end{equation}
这些麦克斯韦方程可以从拉式量直接推出
\begin{equation}
\mathcal L_{\rm Maxwell} = - \frac{1}{4} (F_{\mu\nu})^2 = - \frac{1}{4} (\partial_\mu A_\nu - \partial_\nu A_\mu)^2~. 
\end{equation}
我们的目标是写出洛仑兹不变的方程.那么如何找到这些洛仑兹不变的方程呢?我们可以首先来研究一下场的变换规则.然后写出洛仑兹不变的拉式量就不难了.

首先我们可以考虑线性的变换.如果$\Phi_a$是一个$n$个组分的场.那么洛伦兹变换的规则可以由一个$n\times n$的矩阵$M(\Lambda)$来给定
\begin{equation}
\Phi_a (x) \rightarrow M_{ab} (\Lambda) \Phi_b(\Lambda^{-1} x)~.
\end{equation}
我们把指标去掉,写成下面这个更简单的形式
\begin{equation}
\Phi\rightarrow M(\Lambda)\Phi~.
\end{equation}
现在我们来考虑两个连续的$M(\Lambda)$作用在场$\Phi$上面.
\begin{equation}
\Phi\rightarrow M(\Lambda')M(\Lambda)\Phi = M(\Lambda'')\Phi~.
\end{equation}

