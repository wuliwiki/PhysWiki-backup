% 极端曲线
% 高维|极端曲线|极值|必要条件

\pentry{绝对极值与相对极值(变分学)\upref{AbPol},欧拉方程(变分学)\upref{ElueEV}}

从欧拉方程\upref{ElueEV} 一节中,可以知道极值曲线是使二维空间中泛函 
\begin{equation}
J(y)=\int_a^b F(x,y,y')\dd x
\end{equation}
取极值的曲线,它满足欧拉方程.当推广到高维空间时,我们称使 $n$ 维空间中泛函 
\begin{equation}
J(y_1,\cdots, y_n)=\int_{a_0}^{a_1}F(x;y_1,\cdots,y_n;y_1',\cdots,y_n') 
\end{equation}
取极值的曲线 
\begin{equation}
y_1=y_1(x);\cdots;y_n=y_n(x)
\end{equation}
为\textbf{极端曲线}.式中,$a_0,a_1$ 分别是 $n$ 维空间中可取曲线的起止点的参数值. 