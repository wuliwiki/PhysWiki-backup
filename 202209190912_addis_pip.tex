% Python 包管理

\pentry{Python 简介\upref{Python}}

\begin{itemize}
\item 另见 Conda 笔记\upref{CondaN}.
\item \verb|python| 在运行时从哪里搜索包(module)? 如果 \verb|import| 命令搜不到, 会出现错误 \verb|ModuleNotFoundError|. 答案是在(\verb|import sys|) \verb|sys.path| 中, 而其中列出的路径按优先级有三个来源: 1. current directory. 2. 环境变量 \verb|PYTHONPATH|. 3. \verb|sys.path| 中的其他默认路径. 这可以直接在代码中修改, 如 \verb|sys.path.append('some/folder')|.
\item import 以后, 可以用 \verb|module.__file__| 查看 module 的文件路径.
\item \verb|sudo apt install python3.9|
\item \verb|sudo apt install python3-pip|
\item \verb|pip3 ...| 和 \verb|python3 -m pip ...| 应该是一样的, 见\href{https://stackoverflow.com/questions/41307101/difference-between-pip3-and-python3-m-pip}{这里}.
\item \verb|pip3 install numpy|
\item \verb|pip3 list| 列出所有包
\item \verb|pip3 uninstall 包名|
\item 老系统上面(例如 16.04), 如果 apt 不能直接安装新版本, 要自己编译安装. 从官网下载源码: \verb|wget https://www.python.org/ftp/python/3.6.3/Python-3.6.3.tgz| 解压 \verb|tar -xvf Python-3.6.3.tgz|, \verb|cd Python-3.6.3|, \verb|apt-get install build-essential zlib1g-dev|, \verb|./configure|, \verb|make|(可以加 \verb|-j4| 选项), \verb|make install| 就可以了. 检查版本: \verb|python3.6 -V|, 但安装 \verb|3.10| 的时候会出现编译错误.
\item \verb|pip| 是 python 的默认包管理器, 最广泛使用. 另外也可以用 conda, 但不能免费商用.
\item \verb|Python Package Index (PyPI)| 是 \verb|pip| 中安装包的主要来源.
\item 注意 \verb|pip3 install 包名称| 中的名称未必是 \verb|import 包名称| 的名称! 如果找不到前者, 会提示 \verb|Could not find a version that satisfies the requirement|
\item \verb|python| 命令的路径一般是 \verb|/usr/bin/python|, 它是一个 soft link, 链接到具体版本. 注意如果直接修改这个连接可能会发生一些错误(导致 \verb|lsb_release| 不能用等)
\item 在 \verb|module| 的代码中, 如果 module 是被 import 的, 那么 \verb|__name__| (被 import 以后就是 \verb|包名.__name__|)是包的名字, 但是如果 module 是作为 script 来执行的(\verb|python3 -m 包名|), 那么 \verb|__name__| 就是 \verb|__main__|, 在 module 中, \verb|if __name__ == '__main__'| 可以用于判断到底是哪种情况.
\item 在命令行运行 \verb|python3 -m 包名称 参数...| 相当于把 module import 以后再 run as script.
\end{itemize}
