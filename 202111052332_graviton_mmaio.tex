% Mathematica文件操作
% Mathematica|IO|文件操作

\subsection{常用变量}

guide/FileOperations, 比较全的文件系统操作的函数列表
tutorial/FilesStreamsAndExternalOperations#12068

不要直接使用字符串函数操作文件名/文件路径, 这样的到的路径依赖于操作系统, 应该使用 Mathematica 提供的文件系统接口.

`$OperatingSystem`; 给出正在运行的操作系统的名称.
`$PathnameSeparator` ; 字符串,在构建路径名的时候使用.
`Windows`的默认值时`"\\"`, 其他系统是`"/"`. 在`Windows`中, 像`FileNameSplit`这样的函数默认同时允许`\` 和 `/`.

文件名使用惯例.
    `name.m`  ; Wolfram 语言源文件
    `name.nb` ;  Wolfram 系统笔记本文件
    `name.ma` ;  Wolfram 系统从第3版以前的笔记本文件
    `name.mx` ;  输出所有 Wolfram 语言表达式
    `name.exe`;   WSTP 可执行程序
    `name.tm` ;  WSTP 模版文件
    `name.ml` ;  WSTP 流文件

`$Path`; 默认的目录列表, 用于搜索输入文件的相关目录. 一般来说, 全局变量 `$Path` 被定义为一个字符串的列表, 每个字符串代表一个目录.
每次你要求打开文件时, `Wolfram` 就暂时将这些目录中的依次变成你的当前工作目录, 然后从该目录中尝试找到你要求的文件.

在`$Path`的典型设置中, 当前目录`.`和你的主目录`~`被列在第一位.

`DirectoryName["name",n]` ;  给出路径的父目录, `n`代表上升`n`次. 默认情形给出父目录, 不用写`n`. 作用于文件和目录, 不检查目录是否真实存在.
可以用`DirectoryName[...,OperatingSystem->"os"]`给出某种操作系统风格的路径, 选项有 `"Windows"`, `"MacOSX"`, 和 `"Unix"`.

`ParentDirectory["dir",n]` ;  给出路径的父目录, `n`代表上升`n`此, 只能作用于目录, 且要求目录真实存在.

`$InitialDirectory` ;  是 `Wolfram` 系统启动时的初始目录.
`$HomeDirectory` ;  你的主目录, 如果被定义过的话
`$BaseDirectory` ;  是 `Wolfram` 系统要加载的全系统文件的基本目录.
`$UserBaseDirectory` ;  用于 `Wolfram` 系统加载的用户特定文件的基本目录
`$InstallationDirectory` ;  你的 `Wolfram` 系统安装的最高级别目录

`Wolfram` 系统所使用的绝大多数文件都与`操作系统`无关. 然而, `.mx` 和 `.exe` 文件与系统有关.
对于这些文件, 按照惯例, 对不同计算机系统版本的名称进行捆绑, 形式如 `name/$SystemID/name`.

## 笔记本

`NotebookFileName[]` ; 给出当前笔记本的完整路径.
`NotebookDirectory[]`; 笔记本父目录

`NotebookOpen["name"]`;  打开已经存在的笔记本`"name"`, 返回笔记本对象. `"name"`可以是绝对路径.
`NotebookOpen["name",options]`; 使用指定的选项打开笔记本.
    `NotebookOpen[File["path"]]`和`NotebookOpen[URL["url"]]`也被支持.
    `NotebookOpen`通常会导致一个新的笔记本窗口在你的屏幕上被打开.
    如果`NotebookOpen`打开指定的文件失败, 则返回`$Failed`.
    若给出相对路径, `NotebookOpen`搜索由前端的全局选项`NotebookPath`指定的目录
    若使用选项 `Visible->False` 设置, `NotebookOpen` 将打开带有此选项的笔记本,它永远不会显示在屏幕上.
    `NotebookOpen` 将当前`selection`初始化设置在笔记本的第一行单元之前.

`NotebookSave[notebook]`; 保存特定笔记本的当前版本.
    `notebook`必须是一个`NotebookObject`.
    `NotebookSave[notebook]` 将笔记本保存在一个文件中, 文件名由笔记本对象 `notebook` 给出.
    `NotebookSave` 写入对应的 `Wolfram` 语言表达式, 以及 Wolfram 语言注释, 以便于前端再次读入笔记本.
    `NotebookSave[notebook, "file"]`, 如果`"file"`存在, 则不加警告地覆盖它.
    `NotebookSave[notebook,File["file"]]`也被支持.
    如果给定选项 `Interactive->True`, 前端将提示用户为笔记本选择一个文件名.

`NotebookClose[notebook]`; 关闭指定的笔记本对象.
`NotebookClose[] `; 关闭当前在运行的笔记本.
    `NotebookClose`将使笔记本从你的屏幕上消失, 并将使所有引用该笔记本的笔记本对象失效.
    如果给定了选项设置`Interactive->True`, 前端将提示用户是否关闭笔记本而不保存.

## 操作文件和目录

tutorial/FilesStreamsAndExternalOperations#12068
Manipulating Files and Directories

`ExpandFileName["name"] `; 将`"name"`文件展开成当前系统规范的绝对路径, 给出相对于你当前目录的名称.
它展开通常的目录指定, 如`.`和 `..`.
它只是对文件名进行操作;它并不实际搜索指定的文件.
它支持 `ExpandFileName[File["name"]]`, 以及`ExpandFileName[URL["file:///path"]]`, 后者将基于文件的`URL`转换为绝对文件名.

`AbsoluteFileName["name"]`; 给出`"name"`文件的绝对路径. 与`ExpandFileName`的区别是, 它会进入文件系统, 检查文件是否真实存在.
同样相对于你当前目录的名称, 可以处理目录指定, 如`.`, `..`和 `~`.
它也支持 `AbsoluteFileName[File["name"]]`.

`FileNameTake["name"]` ; 从`"name"`的完整路径中提取出最后的文件名.
`FileBaseName["file"]`; 给出文件的 basename, 也就是不包括拓展名.
`FileExtension["file"] ` ; 给出文件的拓展名.
`FileNameDepth["name"] `; 给出文件路径的深度, 文件不必真实存在.

`FileNameJoin` ; 从路径列表中组合出完整的文件名
`FileNameSplit` ; 将文件的完整路径分割开
`FileNameDrop["name",n] `; 去掉文件`"name"`路径的前`n`个片段. 如果是`-n`, 那么去掉从末尾开始的`n`个.
`FileExistsQ["name"] ` ; 检查文件, 目录等等是否存在.
`ContextToFileName["context"] ` ; 给出 Mathematica 上下文规范对应的文件名.
