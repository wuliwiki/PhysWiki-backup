% 欧拉运动定律(综述)
% license CCBYSA3
% type Wiki

本文根据 CC-BY-SA 协议转载翻译自维基百科\href{https://en.wikipedia.org/wiki/Euler\%27s_laws_of_motion}{相关文章}。

在经典力学中,欧拉运动定律是将牛顿针对质点的运动定律扩展到刚体运动的运动方程。这些定律由莱昂哈德·欧拉在艾萨克·牛顿提出其运动定律约50年后提出。

\subsection{概述}  
\subsubsection{欧拉第一定律}  
欧拉第一定律指出,刚体的线动量 \( p \) 的变化率等于作用在刚体上的所有外力 \( F_{\text{ext}} \) 的合力:[2]  
\[
F_{\text{ext}} = \frac{d\mathbf{p}}{dt}~
\]  
构成刚体的粒子之间的内力不会改变刚体的动量,因为它们互相作用并产生相等且相反的力,最终没有净效应。[3]  

刚体的线动量是刚体质量与其质心速度 \( v_{\text{cm}} \) 的乘积。[1][4][5]
\subsubsection{欧拉第二定律}  
欧拉第二定律指出,关于惯性参考系中固定点(通常是刚体的质心)的角动量 \( L \) 的变化率,等于该点处作用于刚体的外力矩(扭矩)之和 \( M \):[1][4][5]  
\[
\mathbf{M} = \frac{d\mathbf{L}}{dt}~
\]  
需要注意的是,上述公式仅在 \( M \) 和 \( L \) 都是相对于固定惯性参考系计算,或相对于与惯性系平行但固定在质心的参考系计算时成立。对于仅在二维空间中进行平移和旋转的刚体,可以表示为:[6]  
\[
\mathbf{M} = \mathbf{r}_{\text{cm}} \times \mathbf{a}_{\text{cm}} m + I \boldsymbol{\alpha}~
\]  
其中:  
\begin{itemize}
\item \( \mathbf{r}_{\text{cm}} \) 是刚体质心相对于计算力矩的点的位置矢量,  
\item \( \mathbf{a}_{\text{cm}} \) 是刚体质心的线加速度,  
\item \( m \) 是刚体的质量,  
\item \( \boldsymbol{\alpha} \) 是刚体的角加速度,  
\item \( I \) 是刚体关于其质心的转动惯量。  
\end{itemize}
另见:欧拉方程(刚体动力学)。
\subsection{解释与推导} 
在一个可变形物体中,内部力的分布不一定是均匀的,即应力在不同点之间存在差异。物体内部力的这种变化由牛顿的第二定律,即线动量和角动量守恒定律支配,在最简单的形式下,这些定律应用于质点,但在连续介质力学中被扩展到具有连续分布质量的物体。对于连续物体,这些定律被称为欧拉运动定律。[7]

对于一个具有质量 \( m \)、质量密度 \( \rho \) 和体积 \( V \) 的连续物体,施加的总体力为通过体积积分计算得到:

\[
\mathbf{F}_B = \int_V \mathbf{b} \, dm = \int_V \mathbf{b} \rho \, dV~
\]

其中 \( \mathbf{b} \) 是单位质量上作用的力(加速度的维度,通常称为“体力”),\( dm = \rho dV \) 是物体的一个微小质量元素。

作用于物体的体力和接触力会导致相对于给定点的相应力矩(扭矩)。因此,关于原点的总施加扭矩 \( \mathbf{M} \) 由以下公式给出:

\[
\mathbf{M} = \mathbf{M}_B + \mathbf{M}_C
\]

其中 \( \mathbf{M}_B \) 和 \( \mathbf{M}_C \) 分别表示由体力和接触力引起的力矩。

因此,施加在物体上的所有力和扭矩(相对于坐标系的原点)之和可以表示为体积和表面积积分的和:

\[
\mathbf{F} = \int_V \mathbf{a} \, dm = \int_V \mathbf{a} \rho \, dV = \int_S \mathbf{t} \, dS + \int_V \mathbf{b} \rho \, dV
\]

\[
\mathbf{M} = \mathbf{M}_B + \mathbf{M}_C = \int_S \mathbf{r} \times \mathbf{t} \, dS + \int_V \mathbf{r} \times \mathbf{b} \rho \, dV
\]

其中 \( \mathbf{t} = t(n) \) 称为表面牵引力,是在物体表面上的积分,\( n \) 是单位法向量,指向表面 \( S \) 的外部。

假设坐标系 \( (x_1, x_2, x_3) \) 是惯性参考系,\( \mathbf{r} \) 是物体中某个质点相对于坐标系原点的位置矢量,\( \mathbf{v} = \frac{d\mathbf{r}}{dt} \) 是该点的速度矢量。

**欧拉第一公设或定律**(线动量平衡定律或力的平衡定律)指出,在惯性参考系中,连续物体任意部分的线动量 \( \mathbf{p} \) 的时间变化率等于作用在该部分上的总外力 \( \mathbf{F} \),其表达式为:

\[
\frac{d\mathbf{p}}{dt} = \mathbf{F}
\]

\[
\frac{d}{dt} \int_V \rho \mathbf{v} \, dV = \int_S \mathbf{t} \, dS + \int_V \mathbf{b} \rho \, dV
\]

**欧拉第二公设或定律**(角动量平衡定律或扭矩平衡定律)指出,在惯性参考系中,连续物体任意部分的角动量 \( \mathbf{L} \) 的时间变化率等于作用在该部分上的总外扭矩 \( \mathbf{M} \),其表达式为:

\[
\frac{d\mathbf{L}}{dt} = \mathbf{M}
\]

\[
\frac{d}{dt} \int_V \mathbf{r} \times \rho \mathbf{v} \, dV = \int_S \mathbf{r} \times \mathbf{t} \, dS + \int_V \mathbf{r} \times \mathbf{b} \rho \, dV
\]

其中,  
\( \mathbf{v} \) 是速度,  
\( V \) 是体积,  
\( p \) 和 \( L \) 的导数是物质导数。