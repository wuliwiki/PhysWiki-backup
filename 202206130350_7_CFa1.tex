% 从分析力学到场论
% 经典场|狭义相对论|作用量|拉格朗日函数

%在相对论中创建目录:经典场论

\addTODO{预备知识待确定.}

\pentry{向量丛和切丛\upref{TanBun},欧拉—拉格朗日方程\upref{Lagrng}}



所谓场论,就是研究场的运动,或者说变化的理论.经典场论所研究的是四维时空中的场,与其说是一种理论,不如说是建立理论的一种范式.我们可以把电磁场解释为场,用场论的方法研究它的变化;也可以把单个的粒子解释为场;诸如此类.场的类型也有很多,标量场,向量场,一般的张量场,都有可能.从经典场论延伸,我们还可以在千奇百怪的时空流形上建立各式各样的张量场来解释物质,但那就是量子场论以及其它内容要讨论的了.本章中,我们只讨论经典场.

牛顿力学认为物质是由粒子构成的,而粒子是一种无大小的数学对象,并且被赋予了一个描述其特征的标量,称为“质量”.牛顿动力学所研究的,就是“粒子”这种对象运动的规律.最古典的处理方式,是用\textbf{牛顿三定律}来描述这种规律,相当于描述每时每刻粒子的运动状态和其改变;拉格朗日力学则提出了另一种研究范式,即从整体着手,研究哪些粒子运动轨迹是允许的,此时描述规律的方式变成了\textbf{最小作用量原理}.角度虽有不同,但两种处理方式都是描述“粒子”的运动.

现在,我们要从拉格朗日力学的方法出发,讨论如何用场论来处理牛顿力学,也即经典场论思想的由来.


\subsection{一个粒子的运动:从粒子观点到场论观点}

\subsubsection{粒子观点}

考虑一个粒子在三维空间中的运动.粒子的位置由一个三维向量函数$\bvec{r}(t)$描述,它是时间$t$的函数.定义拉格朗日函数
\begin{equation}
\mathcal{L}=\frac{1}{2}\qty(\dot{\bvec{r}}(t))^2-V(\bvec{r})
\end{equation}
其中$V$是空间的(标量)函数.

那么就可以得到其所关联的作用量,从给定的时间$a$到$b$:
\begin{equation}
\mathcal{S}=\int ^b_a \mathcal{L} \dd t
\end{equation}
于是,粒子的合法运动就是满足“在任意时间段中的作用量取最小值”的运动.用一点变分的技巧,我们就能得出粒子运动的\textbf{欧拉-拉格朗日方程}:
\begin{equation}\label{CFa1_eq1}
\frac{\dd}{\dd t}\frac{\partial\mathcal{L}}{\partial \dot{\bvec{r}}} = \frac{\partial\mathcal{L}}{\partial \bvec{r}}
\end{equation}

符合给定初值条件\autoref{CFa1_eq1} 的向量值函数$\bvec{r}(t)$,就是粒子在该初值条件下的运动.


\subsubsection{场论观点}

换一种视角来理解这个粒子的运动:粒子在某一时刻的位置,可以理解为时间轴上的某一点处粘着的三维向量丛上的一点;粒子所有时刻的位置构成的轨迹,也可以理解为时间轴上三维向量丛的截面.

再准确些说,四维时空中粒子运动的轨迹,可以理解为一维时空上$\mathbb{R}^3$丛的一个截面.所谓一维时空,就是时间轴,没有空间部分.也就是说,三维的空间部分被理解为一个向量丛了,这样就自然把轨迹的概念化为场的概念了.

现在,我们用场论观点重新叙述一遍粒子的运动规律:

一维时空上有一个向量丛$\mathbb{R}^3$,对于该丛上的每个截面$\phi(t)$,能诱导另一个截面$\dot{\phi}(t)=\frac{\dd}{\dd t}\phi(t)$.

定义拉格朗日函数为
\begin{equation}
\mathcal{L} = \frac{1}{2}\qty(\partial_t \phi)^2 - V(\phi)
\end{equation}
它可以理解为给定一个截面$\phi$后,一维时空上的函数.

于是作用量为
\begin{equation}
\mathcal{S}=\int_a^b \mathcal{L}\dd t
\end{equation}
它是给定截面和一维时空上的一个区域后,由拉格朗日函数确定的一个数字.

最小作用量原理用\textbf{欧拉-拉格朗日方程}描述:
\begin{equation}
\frac{\dd}{\dd t}\frac{\partial\mathcal{L}}{\partial\dot{\phi}} = \frac{\partial \mathcal{L}}{\partial\phi}
\end{equation}

场论观点基本上和粒子观点没有区别,只不过用不同的概念解释了相同的东西.四维时空的轨迹变成了一维时空的向量丛截面,相空间上的拉格朗日函数变成了给定截面时一维时空的函数,诸如此类.最重要的是,求作用量时的积分区间,被解释为一维时空上的一个区域.

这是用场论处理单个粒子运动的示例,在这个例子里,场是粒子的运动轨迹.很简单,很直白,但是很有启发意义.我们可以由此出发,推广出场论应该具有的形式,得到新的建立理论的范式.


\subsection{经典场论}
















