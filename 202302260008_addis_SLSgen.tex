% SLISC 库代码生成

\begin{issues}
\issueDraft
\end{issues}

\pentry{SLISC 库概述\upref{SLISC}}

SLISC 中许多头文件(\verb|xxx.h|)都是根据模板文件(\verb|xxx.h.in|) 生成的。 如果你需要修改代码, 那么一种方法是直接修改头文件。 另一种是修改头文件的模板。 模板文件的语法是 SLISC 库原创的。

下面来看一个模板文件的例子:
\begin{lstlisting}[language=cpp, caption=demo.h.in]
#include <complex>
#include <vector>
using namespace std;

//% types = {
//%  'vector<double>', 'Int', 'vector<float>';
//%  'vector<complex<double>>', 'vector<complex<double>>', 'vector<Int>';
//% };
//%----------------------------------
//% [Tz, Tx, Ty] = varargin{:};
void add(@Tz@ &z, const @Tx@ &x, const @Ty@ &y)
{
//% if is_vec(Tz) && is_vec(Tx) && is_scalar(Ty)
	for (size_t i = 0; i < z.size(); ++i)
		z[i] = x[i] + y;
//% elseif is_vec(Tz) && is_scalar(Tx) && is_vec(Ty)
	for (size_t i = 0; i < z.size(); ++i)
		z[i] = x + y[i];
//% elseif is_vec(Tz) && is_vec(Tx) && is_vec(Ty)
	for (size_t i = 0; i < z.size(); ++i)
		z[i] = x[i] + y[i];
//% else
//%     error('not implemented!');
//% end
}
//%-----------------------------------
\end{lstlisting}

安装 \verb|octave| 后, 在命令行中用 \verb`$octave auto_gen.m`, 就会自动生成文件:
\begin{lstlisting}[language=cpp,caption=demo.h]
#include <complex>
#include <vector>
using namespace std;

void add(vector<double> &z, const Int &x, const vector<float> &y)
{
	for (size_t i = 0; i < z.size(); ++i)
		z[i] = x + y[i];
}

void add(vector<complex<double>> &z, const vector<complex<double>> &x,
        const vector<Int> &y)
{
	for (size_t i = 0; i < z.size(); ++i)
		z[i] = x[i] + y[i];
}
\end{lstlisting}
