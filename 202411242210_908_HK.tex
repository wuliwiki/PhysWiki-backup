% 罗伯特·胡克(综述)
% license CCBYSA3
% type Wiki

本文根据 CC-BY-SA 协议转载翻译自维基百科\href{https://en.wikipedia.org/wiki/Robert_Hooke}{相关文章}。

罗伯特·胡克 (Robert Hooke) FRS (/hʊk/;1635年7月18日-1703年3月3日)[4][a] 是一位英国博学者,活跃于物理学(“自然哲学”)、天文学、地质学、气象学和建筑学领域。[5] 他被认为是最早在1665年利用其设计的复合显微镜研究微观生物的科学家之一。[6][7] 胡克年轻时是一位贫困的科学研究者,后来成为他那个时代最重要的科学家之一。[8] 在1666年的伦敦大火之后,胡克以测量员和建筑师的身份,通过完成超过一半的地产界线测绘工作以及协助城市的快速重建,获得了财富和声誉。[9][8] 在他去世后的几个世纪中,胡克经常受到作家的贬低,但在20世纪末,他的名誉得以恢复,并被誉为“英格兰的达·芬奇”。[10]

胡克是皇家学会的院士,从1662年起担任其首任实验策展人。[9] 从1665年至1703年,他还担任格雷沙姆学院的几何学教授。[11] 胡克的科学生涯始于担任物理科学家罗伯特·波义耳(Robert Boyle)的助手。胡克制作了用于波义耳气体定律实验的真空泵,并亲自进行了实验。[12] 1664年,胡克观测到火星和木星的自转。[11] 胡克在1665年出版的著作《显微图谱》(*Micrographia*)中首次提出了“细胞”(cell)一词,这本书激发了显微研究的热潮。[13][14] 在光学领域的研究中——特别是对光折射的研究——胡克提出了光的波动理论。[15] 他是第一个提出以下假说的人:物质因热膨胀的原因,[16] 空气由不断运动的小颗粒组成,并由此产生压力,[17] 以及热是一种能量的概念。[18]

在物理学中,胡克推测重力遵循反平方定律,并且可以说是第一个提出行星运动中这种关系假设的人。[19][20] 这一原理后来被艾萨克·牛顿(Isaac Newton)进一步发展并形式化为牛顿的万有引力定律。[21] 对这一见解的优先权争议促成了胡克与牛顿之间的竞争。在地质学和古生物学中,胡克创立了“水陆球”理论,[22] 因而质疑了《圣经》中关于地球年龄的观点;他还提出了物种灭绝的假说,并认为山丘和山脉是由地质过程抬升而成的。[23] 通过识别已灭绝物种的化石,胡克预示了生物进化论的诞生。[22][24]
\subsection{生平与作品}
\subsubsection{早年生活}  
有关胡克早年生活的大部分信息来自他于1696年开始撰写但未完成的自传;理查德·沃勒 (Richard Waller) 在1705年出版的《罗伯特·胡克博士遗著》序言中提到了这部自传。[25][b] 沃勒的作品,以及约翰·沃德 (John Ward) 的《格雷沙姆教授的生平》[27] 和约翰·奥布里的《简短生平记》[28],构成了胡克生平最重要的同时期传记资料。

胡克于1635年出生在怀特岛的弗雷什沃特 (Freshwater),父母是塞西莉·贾尔斯 (Cecily Gyles) 和英国国教牧师约翰·胡克 (John Hooke),后者是弗雷什沃特全圣教堂的助理牧师。[29] 罗伯特是四个兄弟姐妹中最小的,比其他人小七岁(两个男孩和两个女孩);他身体虚弱,起初并不被认为能活下来。[30][31] 虽然他的父亲教授了他一些英语、(拉丁)语法和神学,但罗伯特的教育大体上被忽视了。[32] 自己摸索着成长的胡克制作了许多机械玩具;看到一个拆开的黄铜时钟后,他用木头制作了一个复制品,并“能正常运转”。[32]

胡克的父亲于1648年10月去世,遗嘱中留给罗伯特40英镑(外加祖母留给他的10英镑)。[33][c] 13岁时,他带着这笔钱去了伦敦,成为著名画家彼得·莱利 (Peter Lely) 的学徒。[35] 胡克还接受了肖像画家塞缪尔·库珀 (Samuel Cowper) 的“绘画指导”,[34] 但“油画颜料的气味不适合他的体质,导致他经常头痛”,于是他成为威斯敏斯特学校校长理查德·布斯比 (Richard Busby) 的学生。[37] 胡克很快掌握了拉丁语、希腊语和欧几里得《几何原本》;[11] 他还学会了弹奏管风琴[38],并开始了对力学的终生研究。[11] 胡克后来在为罗伯特·波义耳的作品和自己《显微图谱》配图时,展示了他精湛的绘画技艺。[39]
\subsubsection{牛津}
\begin{figure}[ht]
\centering
\includegraphics[width=6cm]{./figures/00ca14e549d75fcd.png}
\caption{罗伯特·波义耳的画像,由约翰·克尔斯布姆创作,收藏于兰开夏郡的高瑟普庄园。} \label{fig_HK_1}
\end{figure}
1653年,胡克进入牛津大学基督教会学院,他以风琴师和唱诗班成员的身份获得了免费学费和住宿,并通过担任勤务生获得了基本收入,[40][d] 尽管他直到1658年才正式注册。[40] 1662年,胡克获得了文学硕士学位。[38]

在牛津学习期间,胡克还受雇为托马斯·威利斯博士的助理——威利斯是一位医生、化学家和牛津哲学俱乐部的成员。[42][e] 哲学俱乐部由沃达姆学院院长约翰·威尔金斯创立,他领导了这个重要的科学家团体,这个团体后来成为皇家学会的核心。[44] 1659年,胡克向俱乐部描述了一些比空气重的飞行方法的要素,但他得出结论认为人类的肌肉力量不足以实现这一目标。[45] 通过俱乐部,胡克认识了赛思·沃德(大学的萨维利安天文学教授),并为沃德开发了一种机制,以改进用于天文时间测量的摆钟的规律性。[46] 胡克将他在牛津的日子描述为他终生科学热情的基础。[47] 他在这里结识的朋友,尤其是克里斯托弗·雷恩,在他整个职业生涯中都对他至关重要。威利斯还将胡克介绍给罗伯特·波义耳,俱乐部试图吸引波义耳来牛津工作。[48]

1655年,波义耳搬到牛津,胡克名义上成为他的助理,但实际上是他的共同实验者。[48] 波义耳一直在研究气体压力;尽管亚里士多德的格言“自然厌恶真空”广为流传,但真空可能存在的想法刚刚开始被讨论。胡克为波义耳的实验开发了一种空气泵,而不是使用拉尔夫·格雷特雷克斯的泵,胡克认为后者“过于粗糙,无法完成任何重要任务”。[49] 胡克的设备促成了随后归因于波义耳的著名定律的发展;[50][f] 胡克具有特别敏锐的观察力,并且是一个熟练的数学家,这些特质波义耳并不具备。胡克教授波义耳学习了欧几里得的《几何原本》和笛卡尔的《哲学原理》;[9] 他们还通过实验认识到火是一种化学反应,而不是亚里士多德所说的自然界的基本元素。[52]
\subsubsection{皇家学会}
霍克在皇家学会任职期间的科学工作概述见下文“科学”部分。

根据1935年皇家学会图书管理员亨利·罗宾逊的说法:

如果没有他每周的实验和丰富的工作,皇家学会几乎不可能存续,或者至少会以完全不同的方式发展。毫不夸张地说,他在历史上是皇家学会的创造者。[53]

皇家学会(全称“通过实验改进自然知识的皇家学会”[g])成立于1660年,并于1662年7月获得皇家特许状。[54] 1661年11月5日,罗伯特·莫雷提议任命一位实验策展人为学会提供实验支持,这一提议获得一致通过,并在波义耳的推荐下任命霍克为策展人。[9] 学会没有稳定的收入来完全支付“实验策展人”职位的费用,但在1664年,约翰·卡特勒设立了一笔每年50英镑的津贴,用于在格雷沙姆学院开设“机械”讲座,[55] 条件是学会必须任命霍克担任此任务。[56] 1664年6月27日,霍克被正式任命为此职务,并于1665年1月11日被授予“终身策展人”头衔,年薪为80英镑,[h] 其中包括学会支付的30英镑和卡特勒的50英镑年金。[56][i]

1663年6月,霍克被选为皇家学会会士(FRS)。[57] 1665年3月20日,他还被任命为格雷沙姆学院几何学教授。[58][59] 1667年9月13日,霍克成为学会的代理秘书,[60] 并于1677年12月19日被任命为联合秘书。[61]
\subsubsection{性格、人际关系、健康与去世}
\begin{figure}[ht]
\centering
\includegraphics[width=8cm]{./figures/c967f4e5b4d04f35.png}
\caption{插图选自《罗伯特·胡克的遗作……》,发表于《学者学报》(Acta Eruditorum),1707年。} \label{fig_HK_3}
\end{figure}
尽管约翰·奥布里形容霍克是一个“具有伟大美德和善良”的人,[62] 但关于霍克性格中令人不快的一面也有许多记录。据他的第一位传记作者理查德·沃勒描述,霍克“其人卑微”,“性情忧郁、多疑且嫉妒”。[63] 沃勒的评论在随后的200多年里影响了其他作者,导致许多书籍和文章——特别是艾萨克·牛顿的传记——将霍克描绘成一个不满、自私、反社会的牢骚满腹者。例如,亚瑟·贝里称霍克“声称几乎所有当时的科学发现都是他的功劳”。[64] 萨利文写道,他是“彻底不择手段的”,在与牛顿的交往中表现出“不安的自负”。[65] 曼纽尔形容霍克“爱争吵、嫉妒且报复心强”。[66] 根据莫尔的描述,霍克有着“愤世嫉俗的性情”和“刻薄的言辞”。[67] 安德拉德对霍克更具同情心,但仍将其描述为“难以相处”、“多疑”和“易怒”。[68] 1675年10月,皇家学会理事会曾考虑过一项将霍克开除的动议,原因是他就钟表设计的科学优先权问题攻击克里斯蒂安·惠更斯,但该动议未获通过。[69] 霍克的传记作者艾伦·德雷克指出:

> 如果研究当时的知识氛围,他所涉及的争议和竞争几乎是普遍现象,而非例外。相比于他的一些同时代人,霍克在面对涉及自己发现和发明的争议时的反应显得相对温和。[70]

1935年霍克日记的出版[71] 揭示了关于他社会和家庭关系的许多此前未知的细节。他的传记作者玛格丽特·埃斯皮纳斯指出:“通常将霍克描绘成一个忧郁...隐士的形象完全是错误的”。[72] 他与一些著名的工匠有交往,比如钟表匠托马斯·汤皮恩[73] 和仪器制造师克里斯托弗·考克斯。[74] 霍克经常与克里斯托弗·雷恩会面,两人有许多共同兴趣,并与约翰·奥布里保持了持久的友谊。他的日记还经常提到在咖啡馆和酒馆的会面,以及与罗伯特·波义耳共进晚餐的记录。在许多场合,他会与实验室助理哈里·亨特一起喝茶。尽管他大部分时间独自生活,除了负责打理他家务的仆人外,他的侄女格雷斯·霍克和他的堂弟汤姆·贾尔斯曾在他们年幼时与他同住过几年。[75]

胡克从未结婚。根据他的日记,胡克在侄女格蕾丝年满16岁后与她发生了性关系。自10岁起,格蕾丝一直由胡克监护。[76][77]他还与几名女仆和管家有过性关系。胡克的传记作者斯蒂芬·英伍德认为格蕾丝是他生命中的挚爱,当她在1687年去世时,他深受打击。英伍德还提到“他和格蕾丝之间的年龄差距很常见,不会像我们这样让同时代的人感到不安”。尽管如此,这种乱伦关系如果被发现,仍然会受到教会法庭的谴责和审判,但在1660年后,这并不是一项死罪。[78][]

自幼,霍克患有偏头痛、耳鸣、眩晕和失眠发作。[80] 他还患有脊柱畸形,被诊断为舍尔曼氏驼背,这使得他在中年和晚年时期身体“瘦弱且弯曲,头部过大,眼睛凸出”。[81] 霍克以科学的态度对待这些问题,通过自我治疗进行实验,在日记中认真记录症状、药物和效果。他经常使用氨水、催吐剂、泻药和鸦片,这些药物似乎随着时间的推移对他的身心健康产生了越来越大的影响。[82]

霍克于1703年3月3日逝世于伦敦,在生命的最后一年中,他失明且卧病在床。在格雷沙姆学院的房间里发现了一个装有8000英镑现金和黄金的箱子。[83][k] 他的藏书包括超过3000本拉丁文、法文、意大利文和英文书籍。[83] 尽管他曾提到希望为皇家学会留下丰厚的遗赠,以他的名字命名图书馆、实验室和讲座,但没有遗嘱被找到,资金最终传给了一位名为伊丽莎白·斯蒂芬斯的堂亲。[84] 霍克被安葬在伦敦城毕晓普门的圣海伦教堂,[85] 但他的墓地确切位置至今未知。