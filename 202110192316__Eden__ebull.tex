% 沸腾
% 沸腾|暴沸|水

\textbf{沸腾}是在液体表面及液体内部同时发生的剧烈的汽化现象.

我们用水壶烧水时将看到几个不同的阶段.\textbf{烧到一定程度},可以在水壶底看到一些小气泡积聚在水壶底部.一些小气泡可能会脱离底部上升,但在上升过程中会越来越小直至消失.\textbf{再过一段时间},一些气泡能够到达液面变成很小的空气气泡而破裂,此时能听到“吱吱”的声音.\textbf{再后来},气泡在上升的过程中不断增大而冒出液面,整个液体呈现上下翻滚的剧烈汽化状态,这就是\textbf{沸腾现象}.
\begin{figure}[ht]
\centering
\includegraphics[width=6cm]{./figures/ebull_1.png}
\caption{附在容器底部的气泡}} \label{ebull_fig1}
\end{figure}
要解释沸腾现象,我们需要借助一定热学知识.\textbf{小气泡的产生原因}是:空气在水中的溶解度随水温升高而降低,温度较高的下层水的部分空气分子首先脱溶,容易在器壁的微孔处形成气泡,所以一般气泡会先积聚在底部.由于气泡依附于微孔(有时也依附于液体中的杂质颗粒),这些气泡一般较大(尺度远大于分子尺度),这是液体不会发生\textbf{过热暴沸}的关键.

气泡要经历一个过程才能脱离容器底部向上浮:\autoref{ebull_fig1} 的 (a) 中气泡逐渐增大到 (b),此时气泡虽受浮力作用,但气泡颈处有\textbf{表面张力}与浮力抗衡;当气泡大到一定程度后则上浮,如图 (c).

现在设液体内部一个半径为 $r$ 的气泡距表面距离为 $h$,附在容器底部.设大气压强为 $p_0$,液体的密度为 $\rho$,则在这个深度上液体压强为
\begin{equation}
p_0+\rho gh 
\end{equation}
再由液体表面张力的\textbf{拉普拉斯公式}(\upref{sftens}\autoref{sftens_eq1}),气泡内气压为
\begin{equation}\label{ebull_eq1}
p=p_0+\frac{2\sigma}{r}+\rho gh
\end{equation}

\textbf{气泡内气体可分为两部分}:从液体中脱溶的空气、液体蒸汽.设空气分子摩尔数为 $\nu$,根据理想气体状态方程(\upref{PVnRT}\autoref{PVnRT_eq1}),空气的分压\upref{PartiP}为 $\nu RT/V$.根据饱和蒸气压方程(\upref{Clapey}\autoref{Clapey_eq2}),气泡内液体蒸汽的分压为 $p_r=A-B/T$(其中 $A,B$ 是依赖于系统的常数.\textbf{实际上 $p_r$ 还与气泡半径 $r$ 有关\upref{sftens},这里由于气泡较大而忽略不计}).我们可以将前面的\autoref{ebull_eq1} 改写为(通常 $\rho gh\ll p_0$,所以略去该项)
\begin{equation}\label{ebull_eq2}
\frac{\nu RT}{V}+p_r=p_0+\frac{2\sigma}{r}
\end{equation}

随着温度的增大,饱和蒸气压 $p_r$ 也增大,上式左侧(泡内)压强将大于右侧压强,所以气泡会胀大.但是在胀大的同时,左侧的 $\nu RT/V$ 又会减小(比 $2\sigma/r$ 的减小幅度大),达到一个平衡.在水加热的过程中,当气泡脱离容器壁上升时,由于\textbf{液体温度随深度的减小而降低},$p_r=A-B/T$ 会减小,所以 $V$ 会减小,直到消失.加热到一定程度后,\textbf{气泡能够在消失之前到达液面},这时就能听到吱吱的声音.当液体逐渐加热到\textbf{饱和蒸气压等于大气压强}时(即 $p_r=p_0$).此时,液体\textbf{上下温度差一般不大},气泡上升时 $p_r$ 的变化可以忽略,从\autoref{ebull_eq2} 可以看出,气泡内压强将始终大于泡外压强.因此气泡将不断扩大(当然,由于这个过程中 $\nu RT/V$的减小,气泡不会无限制地扩大),从而导致了\textbf{沸腾现象}.

通常人们简单地认为,沸点就是\textbf{液体饱和蒸气压等于液体上方气体压强时的液体温度}.饱和蒸气压随液体温度的升高而升高,所以沸点也随温度的升高而升高.需要说明的是,在上面的物理解释中,气泡的产生依赖于容器器壁上的微孔、水中的颗粒物或溶于水中的空气.所以液体发生\textbf{正常沸腾的条件是},这些微孔或颗粒提供了\textbf{足够的小气泡},从而起到\textbf{汽化核}的作用.外界的干扰可以使液体分子间互相推动产生一些局部的极小气泡,但其尺度只是数倍于分子半径,根本无法继续增大.如果缺乏汽化核,液体加热到沸点仍无法沸腾,从而继续升温,这种液体称为\textbf{过热液体}.由于过热液体温度已高于沸点,饱和蒸气压也随之增大;当极小气泡内的饱和蒸气压 $p_r$ 大于液体上方的气体压强时,气泡将扩大,而随着半径的增大 $p_r$ 又继续增大\footnote{见\upref{sftens}弯曲液面的饱和蒸气压张力公式}.这样就使气泡迅速膨胀,甚至发生爆炸.这就是\textbf{暴沸现象}.为了防止暴沸现象,实验中常加入\textbf{沸石}或加入一些溶有空气的新水等.
