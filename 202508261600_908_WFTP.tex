% 微分拓扑(综述)
% license CCBYSA3
% type Wiki

本文根据 CC-BY-SA 协议转载翻译自维基百科\href{https://en.wikipedia.org/wiki/Differential_topology}{相关文章}

在数学中,微分拓扑是研究光滑流形的拓扑性质和光滑性质\(^\text{[a]}\)的领域。从这个意义上说,微分拓扑不同于密切相关的微分几何。微分几何关注的是光滑流形的几何性质,包括大小、距离和刚性形状等概念;而微分拓扑关注的是更为粗略的性质,例如流形中有多少个“洞”、它的同伦型或其微分同胚群的结构。由于这些粗略性质中的许多可以通过代数手段来刻画,微分拓扑与代数拓扑有着紧密的联系\(^\text{[1]}\)。
\begin{figure}[ht]
\centering
\includegraphics[width=6cm]{./figures/6d7860d55520d137.png}
\caption{环面上高度函数的莫尔斯理论可以用来描述它的同伦类型。} \label{fig_WFTP_1}
\end{figure}
微分拓扑领域的核心目标是对所有光滑流形按微分同胚(diffeomorphism)进行分类**。由于维数在微分同胚意义下是光滑流形的不变量,这一分类工作通常通过分别研究各个维度中的(连通)流形来进行:
\begin{itemize}
\item 一维(dimension 1)在一维中,按微分同胚分类后,唯一的光滑流形包括:圆、实数轴,以及带边界的半闭区间 $[0,1)$ 和闭区间 $[0,1]$\(^\text{[2]}\)。
\item 二维(dimension 2)在二维中,每一个闭曲面都可以通过其亏格(genus,洞的数量,或者等价地说其欧拉示性数)以及是否可定向来按微分同胚进行分类。这就是著名的闭曲面分类定理\(^\text{[3][4]}\)[。然而,即便在二维情况下,非紧曲面的分类也变得困难,例如由于存在“Jacob's ladder”这类特殊空间,使问题更加复杂。
\item 三维(dimension 3)在三维中,威廉·瑟斯顿提出的几何化猜想(,由格里高利·佩雷尔曼证明,给出了紧三维流形的部分分类。其中包括著名的庞加莱猜想,它指出任何闭的、单连通的三维流形都与三维球面(3-sphere)同胚(实际上也是微分同胚的)。
\end{itemize}
\begin{figure}[ht]
\centering
\includegraphics[width=6cm]{./figures/fb73b0fbf14774d0.png}
\caption{一个余流形$(W; M, N)$,它推广了微分同胚的概念。} \label{fig_WFTP_2}
\end{figure}
从四维开始,分类问题变得更加复杂,原因有两点\(^\text{[5][6]}\):

1. 任意有限表示群都可以作为某个四维流形的基本群。由于基本群是微分同胚不变量,这意味着四维流形的分类至少和有限表示群的分类一样困难。而群的分类涉及群的词问题,该问题等价于停机问题,因此无法得到完整的拓扑分类。

2. 在四维及更高维中,存在同胚但非微分同胚的光滑流形。即使是欧几里得空间 $\mathbb{R}^4$,也存在许多异构的 $\mathbb{R}^4$(exotic $\mathbb{R}^4$)结构。这意味着,四维及更高维流形的微分拓扑研究必须使用超出普通连续拓扑范畴的工具。

微分拓扑中的一个核心未解难题是四维光滑庞加莱猜想:是否每个与四维球面同胚的光滑四维流形都与四维球面微分同胚?换句话说,四维球面是否只存在一种光滑结构?在1维、2维和3维中,这一猜想根据前述分类结果是成立的;但在7维中,这一猜想已知是错误的,因为存在米尔诺球这种非标准的光滑结构。

研究光滑流形微分拓扑的重要工具包括构造光滑拓扑不变量,例如德拉姆上同调或交叉形式,以及可光滑化的拓扑构造,如光滑手术理论或余流形的构造。莫尔斯理论是一种重要方法,它通过研究流形上可微函数的临界点来分析光滑流形,展现了流形的光滑结构如何融入这些分析工具\(^\text{[2、7]}\)。在很多情况下,还可以采用更加几何化或分析化的技术,例如给光滑流形赋予黎曼度量,或者研究定义在流形上的微分方程。但需要谨慎的是,必须确保所得信息不依赖于所选附加结构,以便其真正反映流形底层的拓扑性质。例如,霍奇定理为德拉姆上同调提供了几何和分析的解释;而规范理论则被西蒙·唐纳森用于证明关于单连通四维流形交叉形式的重要结果\(^\text{[8]}\)。在某些情况下,当代物理学中的方法也会出现,例如拓扑量子场论,它可以用于计算光滑空间的拓扑不变量。

微分拓扑中的著名定理包括:惠特尼嵌入定理,毛球定理,霍普夫定理庞加莱–霍普夫定理,唐纳森定理,庞加莱猜想。
\subsection{描述}
微分拓扑研究的是那些仅依赖于流形的光滑结构才能定义的性质和结构。与带有额外几何结构的流形相比,光滑流形更“柔性”,因为额外的几何结构可能成为某些微分拓扑中等价或变形的障碍。例如,体积和黎曼曲率是可以区分同一光滑流形上不同几何结构的不变量。换句话说,某些流形可以通过光滑变形被“展平”,但这可能需要扭曲空间,从而改变其曲率或体积。

另一方面,光滑流形又比单纯的拓扑流形更具“刚性”。约翰·米尔诺发现,有些球面存在不止一种光滑结构(见异构球和唐纳森定理);米歇尔·凯尔韦尔则展示了一些完全不存在光滑结构的拓扑流形\(^\text{[9]}\)。某些光滑流形理论中的构造(如切丛(的存在\(^\text{[10]}\))虽然也可以在拓扑环境下实现,但过程复杂得多;而有些构造则在拓扑环境下根本无法实现。

微分拓扑的主要研究方向之一是考察流形之间的特殊光滑映射,例如浸入、满射,以及通过横截性研究子流形之间的交互关系。更广泛地,人们关注的是那些在微分同胚下保持不变的光滑流形的性质和不变量。莫尔斯理论是微分拓扑的另一个重要分支,它通过研究函数雅可比矩阵秩的变化来推导流形的拓扑信息。

有关微分拓扑主题的完整列表,请参考微分几何主题列表。
\subsection{微分拓扑与微分几何的区别}
微分拓扑与微分几何首先体现出它们的相似性:两者都主要研究可微流形的性质,有时还会在这些流形上施加各种附加结构进行研究。两者的主要区别在于所研究问题的性质。[4] 从一个角度看,微分拓扑与微分几何的区别在于:微分拓扑主要研究那些本质上是全局性的问题。举一个例子:咖啡杯和甜甜圈。从微分拓扑的角度来看,咖啡杯和甜甜圈是“相同的”(某种意义下)。这种观点是全局性的,因为单凭观察它们的某一个局部部分,无法判断它们是否相同;只有看到整个对象,才能做出这种判断。

而从微分几何的角度来看,咖啡杯和甜甜圈是不同的,因为无论怎么旋转咖啡杯,都无法让它的形状完全匹配甜甜圈。这虽然也是一个全局的思路,但关键区别在于:几何学家无需依赖整个物体。只要观察一个小小的局部,比如杯柄的一部分,就能判断咖啡杯和甜甜圈是不同的,因为杯柄比甜甜圈的任何部分都更细(或者曲率更大)。

简而言之,微分拓扑研究的是流形上那些在局部上没有有趣结构的性质;而微分几何则研究那些在局部(甚至无穷小尺度)具有丰富结构的性质。

更严格地用数学语言来说,例如:在两个相同维数的流形之间构造微分同胚的问题本质上是全局性的,因为局部来看,这样的两个流形总是微分同胚的。类似地,计算流形上一些在可微映射下保持不变的量也是全局性问题,因为任何局部不变量都是平凡的——它们在 $\mathbb{R}^n$ 的拓扑结构中已经自然存在。此外,微分拓扑并不仅限于研究微分同胚。例如,辛拓扑——微分拓扑的一个分支——研究的是辛流形的全局性质。相比之下,微分几何关注的是总会涉及某些非平凡局部性质的问题,无论它们是局部的还是全局的。因此,微分几何会研究那些配备了联络、度量(metric,可能是黎曼度量、伪黎曼度量或芬斯勒度量)、或特殊分布(如 CR 结构)的可微流形,等等。

不过,在涉及局部微分同胚不变量的问题上,这两者的界限会变得模糊。例如,对于某一点的切空间的讨论,微分拓扑也会涉及类似的问题,还包括研究 $\mathbb{R}^n$ 上可微映射的性质,例如切丛、射流丛、惠特尼延拓定理等内容。

抽象地总结,这一区别可以简明地表述为:
\begin{itemize}
\item 微分拓扑:研究的是流形上那些局部参数平凡的结构的(无穷小、局部和全局)性质。
\item 微分几何:研究的是流形上那些至少有一个非平凡局部参数的结构的(无穷小、局部和全局)性质。
\end{itemize}
\subsection{参见}、
\begin{itemize}
\item 微分几何主题列表
\item 微分几何与拓扑术语表
\item 微分几何的重要出版物
\item 微分拓扑的重要出版物
\item 弯曲时空数学的基础介绍
\end{itemize}
\subsection{注释}
\begin{enumerate}
\item Bott, R. 和 Tu, L.W., 1982. Differential forms in algebraic topology (Vol. 82, pp. xiv+-331). 纽约: Springer.
\item Milnor, J. 和 Weaver, D.W., 1997. Topology from the differentiable viewpoint. 普林斯顿大学出版社.
\item Lee, J., 2010. Introduction to topological manifolds (Vol. 202). Springer Science & Business Media.
\item Hirsch, Morris, 1997. Differential Topology. Springer-Verlag. ISBN 978-0-387-90148-0.
\item Scorpan, A., 2005. The wild world of 4-manifolds. American Mathematical Society.
\item Freed, D.S. 和 Uhlenbeck, K.K., 2012. Instantons and four-manifolds (Vol. 1). Springer Science & Business Media.
\item Milnor, J., 2016. Morse Theory (AM-51), Volume 51. 普林斯顿大学出版社.
\item  Donaldson, S.K., Donaldson, S.K. 和 Kronheimer, P.B., 1997. The geometry of four-manifolds. 牛津大学出版社.
\item Kervaire, 1960.
\item Lashof, 1972.\\
a,流形的光滑性质是指在微分同胚下保持不变的性质。这不包括某些几何性质,例如点之间的距离或体积,因为这些性质依赖于黎曼度量的进一步选择,并且只在同构(isometry)意义下才是不变的。
\end{enumerate}
\subsection{参考文献}
\begin{itemize}
\item Bloch, Ethan D. (1996). A First Course in Geometric Topology and Differential Geometry. 波士顿: Birkhäuser. ISBN 978-0-8176-3840-5.
\item Hirsch, Morris (1997). Differential Topology. Springer-Verlag. ISBN 978-0-387-90148-0.
\item Lashof, Richard (1972年12月). "The Tangent Bundle of a Topological Manifold". American Mathematical Monthly. 79 (10): 1090–1096. doi:10.2307/2317423. JSTOR 2317423.
\item Kervaire, Michel A. (1960年12月). "A manifold which does not admit any differentiable structure". Commentarii Mathematici Helvetici. 34 (1): 257–270. doi:10.1007/BF02565940.
\end{itemize}
\subsection{外部链接}
\begin{itemize}
\item "Differential topology", Encyclopedia of Mathematics, EMS Press, 2001 [1994].
\end{itemize}
