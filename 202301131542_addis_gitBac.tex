% 用 git  备份文件夹

\begin{issues}
\issueDraft
\end{issues}

\pentry{Git 笔记\upref{Git}}

以下 bash 程序可以把若干个文件夹中的每个都用 git 备份到 \verb|备份目录|。 运行完后, 每个 \verb|文件夹*| 会保存为 \verb|备份目录/文件夹*.git|。 这个其实就是 git 仓库中的 \verb|.git| 文件夹(注意并不是 bare repo, 不可以用于上游)。

要注意的是, Git (尤其是 Windows 上)处理大的二进制文件的速度较慢。 笔者在 Windows 上在两个 HDD 硬盘之间用该脚本备份, 写入速度平均 12MiB/s。 但笔者认为 Git 的丰富功能和灵活性、以及广泛的普及可以弥补这一不足。 另见 bup\upref{bupBac}(待研究)、git-annex\upref{gitanx}(最新版本依赖于 symlink) 以及 git-lfs(没有本地repo)。 更简单直接地, 可以用 rsync 甚至 cp 备份\upref{rsyncB}。

备份脚本: 用法 \verb|git-backup.sh 备份目录|。 所有子目录中, 如果包含 \verb|.gitattributes|, 就会备份到 \verb|备份目录/子目录名.git|。

\addTODO{如何批量检查变化? 批量 commit? push? 多重备份?}
