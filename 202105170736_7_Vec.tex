% 切向量场
% 光滑函数|流形|切向量|切空间|tangent space|tangent vector|方向导数|李括号|Lie bracket
\pentry{流形上的切空间\upref{tgSpa}}

\subsection{光滑切向量场的定义}

我们知道,集合上的一个实函数可以理解为给集合中的每个点都分配了一个实数的结果.流形上的一个实函数,有时也被称为“标量场”.类似地,如果给流形上每个点都分配一个该点的切向量,这种分配就被称为流形上的一个\textbf{切向量场(tangent vector field)}.更进一步,流形上每个点分配一个该点上的张量,就得到流形上的一个\textbf{张量场(tensor field)};切向量场和函数都是张量场的特例.


在流形$M$上某一点$p$附近取一个图$(U, \varphi)$,那么切空间$T_pM$中的每一个向量$\bvec{v}$都唯一对应一个坐标,即$\varphi(\bvec{v})$在$\mathbb{R}^n$中的坐标,其中$n=\opn{dim}M$是$M$的维度.

仅仅依靠“图”这一概念,足够我们定义什么是光滑切向量场.这一点很神奇,因为我们甚至还无法讨论怎么在流形上给切向量求导\footnote{流形上给切向量求(方向)导数的概念,见后续的\textbf{仿射联络(流形)}\upref{affcon}词条.},就已经能够定义流形上的光滑切向量场了,也就是“可以任意求导的切向量场”.

\begin{definition}{光滑切向量场}\label{Vec_def1}
给定实流形$M$,令$X:M\to TM$为$M$上的一个映射,其中对于任意$p\in M$,有$X(p)\in T_pM$.称$X$为$M$上的一个\textbf{切向量场(tangent vector field)},有时也直接将切向量简称为向量.如果对于任意$p\in M$,存在一个包含$p$的图$(U_p, \varphi_p)$,使得$\dd\varphi_p\circ X\circ\varphi_p^{-1}$在欧几里得空间中的每一个坐标分量都是一个光滑函数,那么称$X$是$M$上的一个\textbf{光滑切向量场(smooth tangent vector field)}.
\end{definition}

简单来说,流形上的光滑切向量场$X$,就是对于每一个图$(U, \varphi)$,其映射到图上的结果$\dd\varphi\circ X$是一个光滑切向量场.$\dd\varphi$将流形上的切向量$\bvec{v}$映射为$\varphi(U)$上的切向量$\dd\varphi(\bvec{v})$,因此$\dd\varphi\circ X\circ\varphi^{-1}$就是将$\varphi(U)$上的个点映射到该点处的切向量,即$\varphi(U)$上的切向量场,也就是微积分中所讨论的“欧氏空间里的向量值函数”.

上述阐释和\autoref{Vec_def1} 有一处区别:阐释要求$X$被任何图映射后都还是一个光滑向量值函数,而定义只要求在任何点附近都存在一个图,使之映射后还是一个光滑向量值函数就行了.这两个表述其实是等价的,因为我们要求图与图之间相容,也就是彼此之间的变换是光滑的,因此只要有一个图中$\dd\varphi\circ X$是一个光滑向量值函数,那么任何与之相交的图中它依然是光滑的.

如前所述,尽管“光滑”的含义是“任意阶导数存在”,我们却尚无法对流形上的光滑切向量场进行求导运算.一方面,流形上的“光滑”继承自“欧几里得空间中的光滑”,后者虽然可以计算导数,但不同的图计算结果往往不同,无法规定谁才是正统的导数;另一方面,欧几里得空间中,不同点处的切向量是天然有一一对应关系的,而任意流形却不行.流形上的切向量只能用导子、道路等方式来定义,然而不同起点的道路,哪些相等、哪些相等,我们都还尚未讨论.这些都留待\textbf{仿射联络(流形)}\upref{affcon}一节详细阐释.在此你只需要意识到切向量的\textbf{导子}和\textbf{道路}两种定义,与\textbf{几何}、\textbf{坐标}的定义有本质区别:前者无法确定不同切点间切向量的关系,后者已经暗自包含了关系\footnote{这个关系,就是所谓的“联络”.}.


\subsection{切向量对于场的作用}
%李括号
我们使用切向量的\textbf{道路}和\textbf{导子}定义.

流形上的一个“张量场”,就是一个映射.如果是把流形上的点映射到实数(0阶张量),那么我们一般就称之为\textbf{函数};更高阶的张量映射,就被称为\textbf{场}.现代理论物理中所讨论的“张量”,绝大多数情况下特指“流形上的张量场”.

考虑流形$M$.把切向量$\bvec{v}_p$看成点$p\in M$出发的一条道路$v:[0, 1]\to M$\footnote{即$\bvec{v}_p=\frac{\dd}{\dd t}v(t)$,$v(0)=p$.},再定义$M$上的一个光滑函数$f$,那么切向量对函数的作用就定义为:
\begin{equation}
\bvec{v}_pf=\frac{\dd f(v(t))}{\dd t}
\end{equation}
也就是说,$\bvec{v}_pf$可以看作$f$沿着$\bvec{v}_p$方向求\textbf{方向导数}.

对于一般的流形,只有函数(0阶张量)可以像上面那样,定义切向量对函数的作用.切向量对切向量场、乃至更高阶的张量场,没法直接定义作用.这一方面是因为不同切点上的高阶张量无法xiang


