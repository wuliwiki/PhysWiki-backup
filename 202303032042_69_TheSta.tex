% 热力学量的统计表达式(玻尔兹曼分布)
% 热力学量|统计力学|配分函数|玻尔兹曼分布

\pentry{玻尔兹曼分布(统计力学)\upref{MBsta},理想气体的内能\upref{IdgEng},熵\upref{Entrop}}

\subsection{配分函数,热力学第一定律}

满足经典极限\footnote{在玻尔兹曼分布\upref{MBsta} 词条中谈到了玻色分布和费米分布的表达式,式中如果 $e^\alpha\gg 1$,那么将过度到经典情况的玻尔兹曼分布。我们称这个条件为经典极限。}的大量粒子组成的系统中,粒子遵从玻尔兹曼分布。我们可以试图用统计力学中配分函数来推出一切热力学量。

配分函数表达式为:
\begin{equation}
Z_1=\sum_l \omega_l e^{-\beta \epsilon_l}
\end{equation}

式中 $\omega_l$ 为能级的简并度。根据玻尔兹曼分布,每个能级上的粒子数为 $e^{-\alpha-\beta\epsilon_l}$。于是有
\begin{equation}\label{TheSta_eq4}
\begin{aligned}
&N=\sum_l \omega_l e^{-\alpha-\beta\epsilon_l}=e^{-\alpha} Z_1\\
&E=\sum_l \epsilon_l \omega_l e^{-\alpha-\beta\epsilon_l}=-e^{-\alpha}\frac{\partial Z_1}{\partial \beta}=-\frac{N}{Z_1}\frac{\partial Z_1}{\partial \beta}=-N\frac{\partial \ln Z_1}{\partial \beta}
\end{aligned}
\end{equation}

这里 $\alpha,\beta$ 仍然是待定的参量。但对于一个粒子数 $N$ 和内能 $E$ 确定的系统,当我们知道了它的能级信息 $\epsilon_l,\omega_l$,我们就可以写出它的配分函数 $Z_1(\beta)$,并代入上面的公式联立两个方程,最终确定 $\alpha,\beta$。这种方法理论上可行,但实际上较为复杂。下面我们将继续从配分函数出发分析系统的其他热力学量。
\subsubsection{做功与热量传递}
由热力学第一定律,系统可以通过功和热量两种方式与外界交换能量。例如在可逆过程中做功可以写为 $Y\dd y$,$Y$ 为广义力,$y$ 为广义位移\footnote{例如,$Y$ 取 $-P$,$y$ 取 $V$ 对应体积压缩做功;$Y$ 取电场强度 $E$,$y$ 取电极化强度 $P$ 可以表示外界使介质极化需要做的功}。当系统发生广义位移,能级也会发生变化。外界对能级 $\epsilon_l$ 上一个粒子的力为 $\frac{\partial \epsilon_l}{\partial y}$。因此可以表示出 $Y$:
\begin{equation}
\begin{aligned}
Y&=\sum_l a_l\frac{\partial \epsilon_l}{\partial y}=\sum_l \frac{\partial \epsilon_l}{\partial y}\omega_le^{-\alpha-\beta\epsilon_l}\\
&=e^{-\alpha}\qty(-\frac{1}{\beta}\frac{\partial }{\partial y})Z_1\\
&=-\frac{N}{\beta}\frac{\partial }{\partial y}\ln Z_1
\end{aligned}
\end{equation}
例如将上面的 $y$ 替换成 $V$,可以得到压强 $P$ 的表达式。

假设系统发生一个 $\dd y$ 的广义位移的变化。则外界对系统做的功为
\begin{equation}
Y\dd y=\sum_l a_l\dd \epsilon_l=-\frac{N}{\beta}\frac{\partial }{\partial y}\ln Z_1
\end{equation}
注意到上面我们将做功定义为了广义位移对粒子所对应能级 $\epsilon_l$ 的移动。这部分当然是对系统内能有贡献的,但另一部分对系统内能的贡献来自于粒子在各个能级上的重新分布,即 $a_l$ 的变化量。也就是说,对于无穷小的广义位移 $\dd y$,内能的改变量(全微分)为:
\begin{equation}
\dd E=\sum_l a_l\dd \epsilon_l+\sum_l\epsilon_l\dd a_l=\delta W+\delta Q=Y\dd y+\delta Q
\end{equation}
由此可以推出热量的传递:
\begin{equation}\label{TheSta_eq2}
\delta Q=\dd E-Y\dd y=
-N\dd{\qty(\frac{\partial \ln Z_1}{\partial \beta})}+\frac{N}{\beta}\frac{\partial \ln Z_1}{\partial y}\dd y
\end{equation}
\subsection{熵的统计表达式,玻尔兹曼关系}
\subsubsection{从配分函数得到熵}
用 $\beta$ 乘\autoref{TheSta_eq2} 得到
\begin{equation}\label{TheSta_eq1}
\begin{aligned}
\beta \delta Q&=-N\beta\dd{\qty(\frac{\partial \ln Z_1}{\partial \beta})}+N\frac{\partial \ln Z_1}{\partial y}\dd y\\
&=-N\dd{\qty(\beta\frac{\partial \ln Z_1}{\partial \beta})}+N\frac{\partial \ln Z_1}{\partial \beta}\dd \beta+N\frac{\partial \ln Z_1}{\partial y}\dd y\\
&=N\dd{\qty(\ln Z_1-\beta\frac{\partial }{\partial \beta}\ln Z_1)}
\end{aligned}
\end{equation}
根据热力学量熵\upref{Entrop} 的相关知识,我们知道 $\delta Q$ 乘上积分因子 $1/T$ 就可以变成全微分 $\dd S$。于是\autoref{TheSta_eq1} 表明 $\beta$ 式 $1/T$ 乘上一个常数,熵就是等式右侧括号内的表达式乘以一个常数。我们令
\begin{equation}
\beta=\frac{1}{kT}
\end{equation}
$k$ 称为玻尔兹曼常量,其数值为
\begin{equation}
k=1.381\times 10^{-23} \rm{J\cdot K^{-1}}
\end{equation}
并且可以验证,如此设定 $\beta$ 后,对单原子理想气体系统的配分函数进行计算,得到的各热力学量结果与已知结论相符,例如内能是 $E=\frac{3}{2}NkT$\upref{IdgEng},有状态方程 $PV=NkT$…… 

因为 $\dd S=\delta Q/T$,再对 \autoref{TheSta_eq1} 积分得:
\begin{equation}\label{TheSta_eq3}
S=Nk(\ln Z_1-\beta\frac{\partial }{\partial \beta}\ln Z_1)
\end{equation}

至此,我们用配分函数表达出了熵 $S$。还可以从另一个角度来看式\autoref{TheSta_eq3} 的含义。由于 $N=e^{-\alpha}Z_1$,对它取对数可以得到 $\ln Z_1=\ln N+\alpha$,
将\autoref{TheSta_eq4} 代入熵的表达式,有
\begin{equation}
\begin{aligned}
S&=k(N\ln N+\alpha N+\beta E)
\\&=k[N\ln N+\sum_l(\alpha+\beta\epsilon_l)a_l]
\end{aligned}
\end{equation}
其中 $a_l=\omega_le^{-\alpha-\beta\epsilon_l}$,所以$\alpha+\beta\epsilon_l=\ln(\omega/a_l)$,所以 $S$ 可以表示为
\begin{equation}
S=k(N\ln N+\sum_l a_l\ln\omega_l-\sum_l a_l\ln a_l)
\end{equation}
与\autoref{MBsta_eq6}~\upref{MBsta} 相比较,可以得到著名的\textbf{玻尔兹曼关系}\footnote{这个公式被刻在了玻尔兹曼的墓碑上。}:
\begin{equation}
S=k\ln \Omega
\end{equation}
\subsubsection{全同粒子假设,吉布斯佯谬的解决}
为了使 $S$ 满足广延量要求(即固定温度、粒子数密度的平衡态热力学系统的熵与它的粒子数成正比),还要对上式减去 $k\ln (N!)\approx k N(\ln N-1)$,相当于将微观状态数除去 $N!$。在量子力学出现以前吉布斯(Gibbs)为了解决吉布斯佯谬,就曾提倡减去这一项。直到\textbf{全同粒子假设} 给了这个理论一个充分解释。正确的表达式应为
\begin{equation}\label{TheSta_eq5}
\begin{aligned}
&S=k\ln \frac{\Omega_{M.B.}}{N!}\\
&S=Nk(\ln Z_1-\beta\frac{\partial }{\partial \beta}\ln Z_1)-k\ln N!
\end{aligned}
\end{equation}
\subsubsection{确定 $\alpha$ 与 $\beta$}
之前我们求得 $\beta=1/kT$,现在考虑求解 $\alpha$ 与热力学量的联系。我们假定体系可以和外界交换粒子,即体系的粒子数 $N$ 能发生改变。那么内能的全微分式应该写成 $\dd E=\delta W+\delta Q+\mu\dd N$,可以写成 $T\dd S=\dd E-Y\dd y-\mu\dd N$。将 $S,E,Y\dd y$ 用配分函数的表达式代入,令 $\dd T=\dd \beta=0$,即温度不变,进行化简。可以得到
\begin{equation}
\begin{aligned}
&T\dd(Nk(\ln Z_1-\beta\frac{\partial }{\partial \beta}\ln Z_1)-k\ln N!)=-\dd(N \frac{\partial \ln Z_1}{\partial \beta})+\frac{N}{\beta}\frac{\partial \ln Z_1}{\partial y}\dd y-\mu\dd N\\
\Rightarrow &\dd(NkT\ln Z_1-kT\ln N!)=NkT \dd \ln Z_1 -\mu \dd N\\
\Rightarrow & kT\ln Z_1 \dd N-kT\ln N\dd N=-\mu\dd N\\
\Rightarrow &\mu=-\frac{\alpha}{\beta}
\end{aligned}
\end{equation}

因此
\begin{equation}
\alpha=-\frac{\mu}{kT},\beta=\frac{1}{kT}
\end{equation}
玻尔兹曼分布可以写为:
\begin{equation}
n_l=\omega_l e^{-\frac{\epsilon_l-\mu}{kT}}
\end{equation}

或者我们直接利用推导玻尔兹曼分布时用的拉格朗日乘子公式\autoref{MBsta_eq7}~\upref{MBsta},可以得到
\begin{equation}
\delta(S/k)-\alpha \delta N-\beta \delta E=0
\end{equation}
由此可得
\begin{equation}
\qty(\frac{\partial S}{\partial N})_{V,E}=k\alpha,\qty(\frac{\partial S}{\partial E})_{V,N}=k\beta
\end{equation}
再根据 $E$ 的全微分公式 $T\dd S=\dd E-Y\dd y-\mu\dd N$ 容易求得
\begin{equation}
\alpha=-\frac{\mu}{kT}
\end{equation}

\subsection{自由能}
亥姆霍兹自由能\upref{HelmF} $F=E-TS$,将\autoref{TheSta_eq4} 和\autoref{TheSta_eq5} 代入,自由能的表达式为
\begin{equation}
F=-NkT\ln Z_1+kT\ln(N!)
\end{equation}

根据自由能的全微分式,我们可以用它的偏微商表示化学势 $\mu$:
\begin{equation}
\begin{aligned}
\mu&=\left(\frac{\partial F}{\partial N}\right)_{T,V}=-kT\ln Z_1+kT\ln N\\
&=kT\ln N/Z_1=-kT\alpha=\frac{-\alpha}{\beta}
\end{aligned}
\end{equation}
由这个关系式我们又重新得到了 $\alpha$ 和化学势的关系。

代入系统的配分函数进行计算,可以发现理想气体的化学势是负的。对于量子效应显著(不能作经典近似)的气体,例如金属的自由电子气体,则配分函数要根据费米分布修改,可以得到,金属中自由电子的化学势是正的。类似的,对于玻色气体,配分函数要根据玻色分布进行修改。