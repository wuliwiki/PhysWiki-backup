% 旋转参考系的 “机械能守恒”
% 机械能守恒|旋转参考系|势能|雅可比常量

\pentry{科里奥利力\upref{Corio}, 机械能守恒(单个质点)\upref{ECnst}}

在匀速旋转参考系中, 如果加入惯性力的修正, 则牛顿定律同样适用。 那么我们是否同样有某种 “修正版” 的单质点机械能守恒呢? 答案是肯定的, 在旋转参考系中, 我们仍然定义每个质点的动能为 $E_k = mv^2/2$ ($v$ 是相对旋转参考系的速度), 假设质点所受的所有非惯性力在旋转参考系中都是保守力\upref{V}, 对应的势能函数 $V(\bvec r)$。 那么机械能修正后的守恒量为
\begin{equation}\label{Rconst_eq1}
\frac{1}{2}m v^2 + V(\bvec r) - \frac{1}{2}m\omega^2 r^2 = \text{const}~,
\end{equation}
其中 $\omega$ 是旋转参考系相对于惯性系的角速度。

对于质点系, 如果质点两两之间的力也是保守力, 例如库仑力或者万有引力, 那么我们只需要对每个质点计算上式并求和即可得到系统的守恒量。

例子: 雅可比常量\upref{JacCon}。

\subsection{推导}
证明的思路很简单, 在 “机械能守恒(单个质点)\upref{ECnst}” 中证明的基础上, 我们只需要额外考虑惯性力的做功即可。

由于科里奥利力始终与每个质点的运动方向垂直, 所以对质点不做功, 而离心力却有可能会做功, 所以\autoref{Rconst_eq1} 中的前两项不是守恒量。 每个质点所受的离心力只与位置有关, 于是我们可以得到一个离心力场
\begin{equation}
F_c(\bvec r) = m\omega^2 \bvec r
\end{equation}
 这是一个中心力场, 所以必定是一个保守场\upref{CenFrc}, 沿径向积分容易得到对应的等效势能函数为
\begin{equation}
V_c(\bvec r) = -\frac{1}{2}m\omega^2 r^2
\end{equation}
现在, 离心力做功等于 $V_c(\bvec r)$ 的减小, 所以把这个等效势能加入原来的机械能, 就又得到了一个守恒量。
