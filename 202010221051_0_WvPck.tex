% 波包
% keys 包络线|波动|高斯波包|振幅

\pentry{平面波\upref{PWave}}

\footnote{参考 Wikipedia \href{https://en.wikipedia.org/wiki/Wave_packet}{相关页面}.}\textbf{波包(wave packet)}是一种局域的波动. 也就是说在某个时刻, 波包的振幅关于位置变化, 超出一定范围后, 振幅以某种方式衰减(但不一定严格为零, 例如高斯波包\upref{GausPk}).

\begin{figure}[ht]
\centering
\includegraphics[width=8cm]{./figures/WvPck_1.pdf}
\caption{波包, 红色虚线是包络线(来自维基百科)} \label{WvPck_fig1}
\end{figure}

\subsection{波函数的复数表示}
简单的波包通常可以表示为以下形式
\begin{equation}
f(\bvec r, t) = A(\bvec r, t) \exp(\bvec k \vdot \bvec r - \omega t)
\end{equation}
其中 $A$ 是复振幅. 经典物理中使用复数表示振动或者波动时.

这种形式的常见的波包是高斯波包\upref{GausPk}.
