% 弧度制与任意角(高中)
% keys 高中|角|弧度|
% license Xiao
% type Tutor

\begin{issues}
\issueDraft
\end{issues}
\pentry{几何与解析几何初步\nref{nod_HsGeBa}}{nod_2571}

任意角和弧度制在高中数学中属于基础概念。类似于“实数”,尽管在后续课程中起着重要作用,但许多教材在讲解时往往只是定义角的正负方向,并给出它们与坐标系的关系,而缺乏直观的动机和背景,使得学生在初学时难以理解其意义。而关于弧度制,传统教科书在介绍时,通常直接抛出弧度与角度之间的转化公式,而没有详细解释引入弧度制的必要性,这让许多人在初次接触弧度制时感到困惑:“为什么需要弧度制?”。另外,日常生活中以“度”作为单位描述角度早已深入人心,且这些角度通常是简单的整数,而弧度值却常以分数乘以$\pi$的形式出现,其表达的复杂性在视觉上与角度形成了鲜明对比,这进一步加剧了初学者对弧度制的抗拒心理。

本文针对这些问题进行补充,不仅介绍弧度制和任意角的定义,还会探讨它们的必要性和直观理解,使读者不会有“空降知识”的感觉,而是能够循序渐进地理解这些概念为何被引入,以及它们如何在数学中发挥作用。

\subsection{为什么采用弧度制?}

在日常生活中,角的使用主要集中在两种情境:一是描述两条直线之间的倾斜关系,二是通过圆心角来分割圆周。在表达倾斜的角度时,“度分秒”系统十分方便,它精细且规整,足以满足日常使用的需求。对于圆的分割,由于这一操作在人们的生活中极为常见,通常并不需要精确量取角度值,因而人们往往忽略了这种分割实际上依赖于角度的概念。

早在约公元前2000年,古巴比伦人选择用$360^\circ$来表示圆周的完整度量,这一选择可能与360的独特性质有关——360是最小的、能够被除7以外所有一位数整除的整数。这一性质使得$360^\circ$在处理常见的“将圆分成几份”的问题时极为便利,既能满足分割的精细需求,又避免了度量数值过大而难以操作的问题。但这种基于便利性的选择,并不具有数学上的普遍性。例如,有人可能主张,为了使圆能够被分成7等份且每一份的度数为整数,可以采用 $2520^\circ$ 作为圆周的度量。这样的选择在理论上完全可行。此外,刚接触度分秒的初中生可能会感到困惑:既然现代社会普及了十进制,为何仍然沿用六十进制的度分秒?他们或许会想,若直接将圆周设定为 $100^\circ$,不是更简单直观吗?这种随意性与英制单位、市制单位,甚至曾经的某些公制单位类似,它们都是根据人们的习惯和实际需求设定的。

然而,在数学领域,这种依赖习惯的单位定义可能会带来问题。数学需要一种普适且逻辑严密的方式来定义角的度量,以避免随意性,同时在各种情境下保持简洁。此外,数学计算通常涉及无单位的纯数字,而非带有单位的量,由此问题在公元6世纪时被印度数学家\textbf{阿耶波多(Aryabhata)}发现。他在研究正弦函数时遇到了一个不知如何处理的情况\footnote{\href{https://www.hanspub.org/journal/PaperInformation?paperID=71063&utm_source=chatgpt.com}{马瑞芳, 库在强. 数学史融入弧度制的教学设计研究[J]. 教育进展, 2023, 13(8): 5911-5919.}}:
\begin{equation}
60^\circ+\sin30^\circ~.
\end{equation}
其中,$\sin 30^\circ$是一个无单位的实数,而$60^\circ$则是一个带有角度单位的量。这种单位的不一致使得它们无法直接进行代数运算(例如加法或乘法),类似于“$1\text{元} + 2$”,使人无法理解结果的意义。此外,随着数学研究的深入,人们还需要计算角$\alpha$的平方、立方等幂次以及它们的和。如果继续使用“度”为单位,这将引入诸如$\circ^2$和$\circ^3$等奇怪的单位。这些实际需求促使数学家思考如何定义一种新的角度度量方式,以满足以下要求:
\begin{enumerate}
\item 能够将角度转化为无单位的量,或者单位是“1”的量,从而便于代数运算。
\item 与传统的“度分秒”系统具有明确的换算关系。
\item 定义应独立于历史、习惯,具有普适性和数学上的严密性,并能在复杂分析、物理学等领域中表现出自然的适用性。
\end{enumerate}

在这个问题被发现之前,人们已经有了分割圆的经验。由于生活中容易量取长度,人们常将角度的分割转化为长度的分割,从而降低测量的难度。尽管,现在人们知道圆周的长度通常是一个无理数,这种方法事实上也是一种近似。然而,这一思路为角度测量问题的解决提供了重要启发。

阿耶波多通过选择用同一单位度量圆的半径和圆周,实际上提出了一个与现代弧度制几乎一致的概念。1748年,瑞士数学家\textbf{欧拉(Euler)}在其著作《无穷小分析概论(Introduction to the Analysis of the Infinite)》中,正式定义了以圆的半径作为弧长的度量单位。下面看看这样的定义可以推知些什么:

已知圆周长度 $C$ 与半径 $r$ 的比值是一个固定值,由于以圆的半径作为弧长的度量单位,这就得到了算数式:
\begin{equation}
C=2\pi r~.
\end{equation}
而一段圆弧$l$如果是周长的$\displaystyle\frac{n}{m}$,那么,它自然就可以表示为:
\begin{equation}
l=\frac{n}{m}C=\frac{2n}{m}\pi r~.
\end{equation}
因此,弧长 $l$ 与半径 $r$ 之间建立了一一对应的关系。在\aref{圆的相关概念}{sub_HsGeBa_1}中提到,每段弧长都对应着一个圆心角。

由于所有的圆都是相似的,对应同一圆心角的不同弧长之间的差异仅来源于圆的半径 $r$ 的不同。如果希望使新的圆心角度量方法不受半径 $r$ 的影响,需要消除半径 $r$ 的作用。将 $r$ 移到等式的左侧,可以得到:
\begin{equation}\label{eq_HsAngl_1}
\frac{l}{r}=\frac{2n\pi}{m}~.
\end{equation}
这样,对于占圆周相同比例的圆弧,不论圆的半径 $r$ 多大,右侧的值始终保持不变。

\subsection{弧度制}

如果将\autoref{eq_HsAngl_1} 右侧的值作为弧对应的圆心角的度量方式,那么通过定义“以圆的半径作为弧长的度量单位”,人们得到了一个新的角度度量方法,这种方法满足之前提出的所有要求:
\begin{enumerate}
\item 它是一个无单位的纯数值。
\item 由于“度”在定义时采用了类似的按比例等分的原则,若规定整圆的圆心角为一个固定值,这种新的方式可以实现与传统的度分秒系统的快速转换。
\item 它仅依赖于常数 $\pi$。
\end{enumerate}
这种新的角度度量方法也就是\textbf{弧度制(radian measure)}。

\begin{definition}{弧度}
对于任意圆,圆的弧长 $l$ 与半径 $r$ 存在固定比例关系,即:
\begin{equation}
\theta  = \frac{l}{r}~.
\end{equation}
称$\theta$为弧所对圆心角的\textbf{弧度(radian)}。
\end{definition}

尽管之前说过$\theta$是个数,但通常在物理上根据习惯,为了表示他是与角相关的量,会在后面加上单位$\Si{rad}$,读作弧度。注意,这只是为了在物理计算中与其他单位复合,表示它是角,本质上还是\enref{无量纲的}{USD}。在数学研究中,一般也不会写$\Si{rad}$。

根据圆的周长公式,圆周对应的圆心角为:
\begin{equation}
\frac{C}{r}=2\pi~.
\end{equation}



也就是说,弧度中与$360^\circ$对应的是$2\pi$。根据上面的原理可以得到$2\pi\Si{rad}$与$360^\circ$对应,而根据定义$0\Si{rad}$与$0^\circ$对应。若同一个角为$a^\circ$与$\theta\Si{rad}$通过对应关系化简可以得到二者的换算方法:
\begin{equation}
\theta=\frac{a}{180}\pi\qquad a=\frac{\theta}{\pi}\times180~.
\end{equation}

由于弧度制消除了不同半径带来的影响,因此一般只研究\textbf{单位圆(unit circle)},即半径为$1$的圆。可以认为,弧度就是圆心角在单位圆上所对的弧长。在单位圆中长度与半径相等的弧所对的圆心角就是$1 \Si{rad}$。

弧度制的引入使得圆的许多性质表达得更加简洁。例如,设半径为$r$的圆弧对应的圆心角为$\theta$(以弧度计),则有:
\begin{itemize}
\item 由弧度的定义可得弧长 $l = r\theta$;
\item 扇形面积为 $\displaystyle S = \frac{1}{2} r^2 \theta$,或用弧长表示为 $\displaystyle S = \frac{1}{2} l r$。
\end{itemize}

值得注意的是,扇形面积公式 $\displaystyle S = \frac{1}{2} l r$ 与三角形面积公式 $\displaystyle S = \frac{1}{2} bh$ 形式相同。这可以通过几何解释来直观理解。设想一个角度 $\theta$ 极小的扇形,其形状近似于一个三角形,其中弧长 $l$ 近似为底边,高为半径\footnote{因为半径与弧的切线垂直,而角度很小时弧近似于切线} $r$。更一般地,可以将扇形视为由无数个极窄的三角形组成,每个三角形的高均为 $r$,而底边依弧线分布。由于这一极限过程,扇形面积公式自然呈现出三角形面积的形式。需要注意的是,此处的分析是定性的,旨在帮助理解公式的直观来源。


\subsection{任意角}

在初中阶段,角通常被分类为锐角、直角、钝角、平角和周角。然而,在描述超过 $180^\circ$ 的角时,人们往往采用等价的较小角度来表示,即用 $360^\circ$ 减去该角度。例如,一个 $210^\circ$ 的角常被描述为 $360^\circ - 210^\circ = 150^\circ$,这样可以将角限制在 $180^\circ$ 以内,以符合熟悉的锐角、直角和钝角的分类。这种方法在仅关注角度大小时非常方便。

传统上,角的定义是由一个公共端点引出的两条射线之间的夹角。在实际生活中,如跳水运动员的空中翻转、机械部件的旋转等,角度不仅仅用于测量静态夹角,还用于描述动态的旋转过程。然而,角也可以通过旋转的视角来理解:设定一条\textbf{初始边(initial side)},然后围绕其端点旋转,最终到达\textbf{终边(terminal side)}。旋转的角度就是从初始边到终边的变化量。这种定义方式使角度能够直接描述旋转现象,但如果仅使用 $0^\circ$ 到 $360^\circ$ 之间的角度来表示旋转,可能会导致信息丢失。

首先,当旋转角度超过 $180^\circ$ 时,虽然逆时针旋转 $210^\circ$ 和顺时针旋转 $150^\circ$ 使终边位置相同,但旋转路径不同。逆时针旋转 $210^\circ$ 经过的路径更长,而顺时针旋转 $150^\circ$ 则是较短路径到达终边。因此,仅用终边的位置来描述角度,无法准确体现旋转的过程,需要加以区分。

其次,设想一个车轮上的某个辐条在不断旋转。虽然辐条会多次回到相同的位置,但它实际经历了不同的旋转圈数。如果只用 $0^\circ$ 到 $360^\circ$ 之间的角度来描述,就无法区分它转动的圈数,也无法体现旋转的方向。

由此可见,初学时所理解的角是一种静态的概念,主要用于描述两个射线之间的夹角。而在涉及旋转运动时,需要扩展角的概念,使其不仅能表示一个固定的几何量,还能描述物体的连续运动。这种扩展类似于数系的拓展。例如,有理数只能表示分数,无法表示 $\sqrt{2}$ 这样的无理数,因此需要引入实数,使数学系统更加完整。同样,传统角度仅限于 $0^\circ$ 到 $360^\circ$ 之间,无法区分旋转的方向和圈数,因此在研究旋转运动时,需要引入超过一周的角,甚至包括负角,使角的定义更加广泛,以适用于更复杂的运动分析。



在实际生活中,如跳水运动员的翻转、机械部件的旋转等活动都有旋转的身影,而旋转往往与角度密切相关。传统上,角的定义是由一个点引出两条射线。然而,也可以换一种视角来理解角:先固定一条射线作为\textbf{初始边(initial side)},然后让这条射线绕着端点旋转,最终到达\textbf{终边(terminal side)},此时旋转的角度就是两条射线构成的角的大小。这种理解方式就将旋转与角度的定义联系了起来。然而,直接用当前限制在周角以内的角度来描述这种相对旋转的关系,会带来一些问题:

首先,当旋转角度超过 $180^\circ$ 时,尽管逆时针旋转 $210^\circ$ 和顺时针旋转 $150^\circ$ 后的终边位置相同,但旋转的过程不同。逆时针旋转 $210^\circ$ 表示射线经过了更大范围的运动,而顺时针旋转 $150^\circ$ 则是较短路径到达终边。因此,仅仅用终边的位置来表示角,无法准确体现旋转的过程,有必要加以区分。

其次,想象一个车轮上的某个辐条在不断旋转。随着车轮的持续转动,辐条会一次次回到相同的位置,表面上看起来与最初的状态相同,但实际上它已经转动了许多圈。如果只用 $0^\circ$ 到 $360^\circ$ 之间的角度来描述,就无法区分它经历了多少次完整的旋转。

也就是说,原本学习的角是一种静态的角,主要用于描述两个射线之间的夹角。而现在,希望通过角来表示动态的变化,例如物体的旋转、机械的转动等。因此,需要对角的概念进行扩展。这种扩展与数的扩展类似。例如,有理数只能表示分数,但无法表示 $\sqrt{2}$ 这样的无理数,因此需要将其扩展到实数,使得数学系统更加完整。同样,原本的角只能表示 $0^\circ$ 到 $360^\circ$ 之间的静态角度,无法区分多次旋转的情况,也无法描述旋转的方向。因此,在研究旋转运动时,需要引入超过一周的角,甚至包括负角,使得角不仅仅是一个静态的几何对象,还能用于描述连续变化的运动。


也就是说原本学习的角是一种静态的角,而现在希望通过角能够表示一个动态的变化。因此需要对角的概念进行扩张。这里就和有理数和实数一样,有理数无法表示根号等运算,于是需要将其扩张到实数,才能够。因此,在研究旋转运动时,需要拓展角的概念,使其能够表示超过一周的旋转,甚至包括负方向的旋转。这种扩展使得角不仅仅是一个静态的几何对象,还能用于描述连续变化的运动。

在初中学习脚的时候,有一个锐角直角和钝角的概念,这个在之前已经介绍过了,但是在学习角度时接触过大于80度的角,比如周角就是360度,而通常在描述一个角的大小时只会关注这个角小于百80度的部分,所以有的时候当一个脚大于180度的时候会用360度来减去他描述他从另一个方向来看的角度,这样就把360度以内的角都可以用直角锐角和钝角来描述了,但是这是一种简化的表示方法,事实上如果一个角大于180度,我们也可以称这个角为多少多少度,这时这个角,尽管看上去和他对应于360度的那个角的样式是相同的,但是在实际生活中这样的角还有其他的含义是那个角不能替代的。

最开始接触的脚都是说一个点引出两条射线,但是我们也可以把视角转换成如下的样子,先有一个点先有一条射线,然后我们保持这个射线在基准的位置住在基准的位置,然后去旋转这条射线,于是就得到了一条射线,而这个旋转过的角度就是这个两条射线所构成的角的角度那么这里就有一个有趣的点就是可以,如果这个旋转的度数超过了180度,他仍然是逆时针旋转的,但如果这时用360度减去它去描述的话就成为了顺时针旋转的这两种旋转,尽管最终的效果看起来像是一样的,但是在这个旋转的过程中这个射线扫过的地方确实不同的,因此有必要区分这两种情况。

另外就像车轮,他的某一个点在持续不断地旋转,这样的旋转已超过了一周的范围,那么如何要来形容这样的形式呢?

比较直觉的方式就是把逆时针和顺时针对应成正数和负数,然后把这种限定为最大的约束给取消掉,这样这个度数可以不断地向外扩张,得到比如720度或者-540等等。

几个中心:
旋转角的视角将与弧度制联系成为自然。
任意角解决了顺逆时针的描述问题。
一般地,正角的弧度数是一个正数,负角的弧度数是一个负数,零角的弧度数是 $0$。这样弧度制确立了用十进制实数表示角的大小的方法,即实数在和线段上的点一一对应之外,也和角一一对应。角的概念推广后,在弧度制下,角的集合与实数集R之间建立起一一对应关系。正角对应正实数,负角对应负实数,零角对应0。





在初中阶段,角被分为锐角、直角、钝角、平角以及周角等。然而,在更深入地研究角的测量时,会遇到超过 $180^\circ$ 的角。在只是描述它们的大小时,人们往往关注小于 $180^\circ$ 的部分。当角大于 $180^\circ$ 时,会用 $360^\circ$ 减去该角度,从另一方向来描述它。例如,一个 $210^\circ$ 的角可以被描述为 $360^\circ - 210^\circ = 150^\circ$。这种方法使得 $360^\circ$ 以内的角仍然可以用不大于$180^\circ$的锐角、直角和钝角的概念来描述。这种简化的表达方式在只关注某个具体值时很好用,但角度的意义却不止于此。

传统上,角的定义是由一个点引出两条射线。然而,可以换一种视角来理解角:先固定一条射线作为\textbf{初始边(initial side)},然后让这条射线绕着端点旋转,最终到达\textbf{终边(terminal side)},此时旋转的角度就是两条射线构成的角的大小。(补充:这样的方法在生活中有很多应用的场景,跳水啊,机械旋转啊等等)但直接用当前的角的范围来描述这种相对旋转的关系会存在一些问题。

首先,当旋转角度超过 $180^\circ$ 时,尽管逆时针旋转$210^\circ$和顺时针旋转$150^\circ$后,最终的角的形状看起来一样,但旋转过程涉及不同的运动轨迹,因此有必要加以区分。

其次,想象一个车轮上的辐条在不断旋转。当车轮持续转动时,尽管它会一次次来到相同的位置,或者说它与最开始的角度看上去一样,但它已经转动了很多圈,用同样的数值来描述它无法区分这些运动的不同点。




这种方法使得角的测量更加直观,并且能够自然地扩展到大于 $180^\circ$ 的情况。而在这时,$210^\circ$ 与 $150^\circ$ 就意味着完全不同的事情。

它的角度已经远远超过 $360^\circ$,甚至可能达到 $720^\circ$、$1080^\circ$,甚至更多。这时如果都统一使用

在实际应用中,一个大于 $180^\circ$ 的角依然可以按原本的角度表示,并且它所代表的意义不同于用 $360^\circ$ 进行转换后的角。


类似地,为了表达这种旋转,最自然的方式是\textbf{允许角度突破 $360^\circ$ 的限制},使其可以任意增长。此外,可以用\textbf{正负号}区分旋转方向:逆时针旋转对应\textbf{正角},顺时针旋转对应\textbf{负角}。这样,角度就不再局限于 $0^\circ$ 到 $360^\circ$ 之间,而是可以扩展为 $720^\circ, -540^\circ$ 等,从而更准确地描述连续旋转的情况。这种扩展不仅让角的表示更加灵活,也为之后的数学分析(如三角函数的周期性)奠定了基础。


按逆时针方向旋转形成的角叫做\textbf{正角};按顺时针方向旋转形成的角叫做\textbf{负角};如果一条射线从起始位置没有作任何旋转,或终止位置与起始位置重合,我们称这样的角为\textbf{零度角},又称\textbf{零角},记作 $\alpha = 0$

角的终边(除端点外)在第几象限,我们就说这个角是第几象限角。

一般地,所有与角 $\alpha$ 终边相同的角,连同角 $\alpha$ 在内,可构成一个集合
\begin{equation}
S = \begin{Bmatrix} \beta|\beta=\alpha+2k\pi,k \in Z \end{Bmatrix}~.
\end{equation}


