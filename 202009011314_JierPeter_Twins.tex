% 双生子佯谬

\pentry{斜坐标系表示洛伦兹变换\upref{SROb}}

双生子佯谬,又称孪生子佯谬,是一个著名的相对论问题.“佯谬”一词的意思是“看起来像是错误但实际上不是”,其中“谬”指“错误”,而“佯”指“假的”.它曾经被认为是一个悖论,但在今天已经被完美解决了,所以成了一个佯谬.和\textbf{时间的变换与钟慢效应}\upref{SRtime}词条的结尾所指出的一样,我们将从双生子佯谬入手,尝试讨论非惯性系眼中的时空.

\subsection{问题描述}

假设地球在某个惯性系$K_1$中静止,忽略一切引力等作用,把地球考虑成一个质点.为方便理解,也可以说$K_1$是“地球系”.在地球上有一对完全同龄的双胞胎,其中弟弟始终留在地球上,而哥哥则乘坐飞船离开地球.称飞船的参考系$K_2$是“飞船系”,同样看成一个质点.一段时间以哥哥返回并降落在地球上.称飞船的参考系$K_2$是“飞船系”.当飞船降落后,兄弟俩的年龄是否有差异?差异是什么?

\subsection{双生子佯谬的解答}







