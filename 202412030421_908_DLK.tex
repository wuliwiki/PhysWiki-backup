% 狄拉克方程(综述)
% license CCBYSA3
% type Wiki

本文根据 CC-BY-SA 协议转载翻译自维基百科\href{https://en.wikipedia.org/wiki/Dirac_equation}{相关文章}。

在粒子物理学中,狄拉克方程是由英国物理学家保罗·狄拉克于1928年推导出的相对论波动方程。在其自由形式或包括电磁相互作用的情况下,它描述了所有自旋为1/2的有质量粒子,被称为“狄拉克粒子”,例如电子和夸克,这些粒子具有宇称对称性。它与量子力学原理和相对论的特殊理论一致,[1]并且是第一个在量子力学中完全考虑特殊相对论的理论。通过严格地解释氢谱的精细结构,它得到了验证。它在标准模型的构建中变得至关重要。[2]

该方程还暗示了一种新的物质形式——反物质,之前未曾被怀疑或观察到,几年的实验验证了这一点。它还为保罗的自旋现象学理论中引入多个分量波函数提供了理论依据。狄拉克理论中的波函数是四个复数值的向量(称为双自旋数),其中两个在非相对论极限下类似于保罗波函数,而与描述单一复数值波函数的薛定谔方程不同。此外,在零质量极限下,狄拉克方程简化为魏尔方程。

在量子场论的背景下,狄拉克方程被重新解释为描述与自旋1/2粒子相对应的量子场。

狄拉克没有完全意识到自己结果的重要性;然而,他关于自旋的解释——作为量子力学和相对论结合的结果——以及最终发现正电子,代表了理论物理学的伟大胜利之一。这一成就被认为与牛顿、麦克斯韦和爱因斯坦的工作相提并论。[3] 有些物理学家认为这方程是“现代物理学的真正种子”。[4] 该方程还被描述为“相对论量子力学的核心”,并且有人说“这方程可能是所有量子力学中最重要的方程”。[5]

狄拉克方程被刻在威斯敏斯特大教堂的地板上的一块纪念牌上。该纪念牌于1995年11月13日揭幕,纪念狄拉克的一生。[6]

\subsection{历史}
狄拉克方程在狄拉克最初提出的形式是:[7]:291 [8]
\[
\left( \beta mc^{2} + c \sum_{n=1}^{3} \alpha_{n} p_{n} \right) \psi(x,t) = i\hbar \frac{\partial \psi(x,t)}{\partial t}~
\]
其中,\(\psi(x,t)\) 是具有静止质量 \(m\) 的电子的波函数,\(x\) 和 \(t\) 为时空坐标,\(p_1, p_2, p_3\) 是动量的分量,被理解为薛定谔方程中的动量算符。\(c\) 是光速,\(\hbar\) 是约化普朗克常数;这些基本物理常数分别反映了特殊相对论和量子力学。

狄拉克提出此方程的目的是解释相对论性运动电子的行为,从而使得原子可以以与相对论一致的方式进行处理。他希望通过这种方式引入的修正可能对原子光谱的问题有所帮助。

在此之前,试图使旧量子理论与相对论理论兼容的努力——这些努力基于将电子在原子核周围可能非圆形轨道中存储的角动量离散化——都失败了,而海森堡、保利、约旦、薛定谔和狄拉克自己提出的新量子力学还未充分发展以处理这一问题。尽管狄拉克最初的目的已得到满足,但他的方程对物质结构有着更深远的影响,并引入了新的数学对象类,这些对象现在是基础物理学的重要元素。

方程中的新元素是四个 4 × 4 矩阵 \(\alpha_1, \alpha_2, \alpha_3\) 和 \(\beta\),以及四分量波函数 \(\psi\)。\(\psi\) 有四个分量,因为在配置空间中的任意一点处对其求值是一个双自旋数。它被解释为自旋向上的电子、自旋向下的电子、自旋向上的正电子和自旋向下的正电子的叠加。

这四个 4 × 4 矩阵 \(\alpha_k\) 和 \(\beta\) 都是厄米矩阵,并且是自反的:
\[
\alpha_i^2 = \beta^2 = I_4~
\]
它们互相反对易:
\[
\alpha_i \alpha_j + \alpha_j \alpha_i = 0 \quad (i \neq j)~
\]
\[
\alpha_i \beta + \beta \alpha_i = 0~
\]
这些矩阵及波函数的形式具有深刻的数学意义。伽马矩阵所表示的代数结构早在50年前由英国数学家 W. K. Clifford 创造。反过来,Clifford 的思想源于19世纪中期德国数学家赫尔曼·格拉斯曼在其《线性展开理论》(\textbf{Lineare Ausdehnungslehre})中的工作。
\subsubsection{使薛定谔方程相对论化}
狄拉克方程在表面上与描述自由粒子的薛定谔方程相似:
\[
- \frac{\hbar^2}{2m} \nabla^2 \phi = i\hbar \frac{\partial}{\partial t} \phi~
\]
左侧表示动量算符的平方除以两倍的质量,这是非相对论性动能。因为相对论将空间和时间视为一个整体,因此这一方程的相对论性推广要求空间和时间的导数必须对称地出现,就像麦克斯韦方程中描述光的行为一样——方程在空间和时间上的微分阶数必须相同。在相对论中,动量和能量是时空四矢量的空间和时间部分,称为四动量,它们通过相对论不变关系相互关联:
\[
E^2 = m^2 c^4 + p^2 c^2~
\]
这表示该四矢量的长度与静质量 \(m\) 成正比。将薛定谔理论中能量和动量的算符替代进入后,得到描述波动传播的克莱因–戈尔登方程,构建于相对论不变的物体基础之上:
\[
\left( -\frac{1}{c^2} \frac{\partial^2}{\partial t^2} + \nabla^2 \right) \phi = \frac{m^2 c^2}{\hbar^2} \phi~
\]
其中波函数 \(\phi\) 是相对论标量:它是一个复数,在所有参考系中具有相同的数值。空间和时间的导数都以二阶形式出现。这对方程的解释有重要影响。由于方程在时间导数上是二阶的,必须指定波函数本身和其一阶时间导数的初始值,才能求解具体问题。由于这两个初始值可以或多或少任意选择,波函数无法再像在薛定谔理论中那样,保持其决定电子在给定运动状态下的概率密度的作用。

在薛定谔理论中,概率密度由正定的表达式给出:
\[
\rho = \phi^* \phi~
\]
该密度根据概率流向量进行对流:
\[
J = - \frac{i \hbar}{2m} (\phi^* \nabla \phi - \phi \nabla \phi^*)~
\]
并且概率流和密度的守恒来自于连续方程:
\[
\nabla \cdot J + \frac{\partial \rho}{\partial t} = 0~
\]
密度为正定并且根据这个连续方程进行对流,意味着可以在某个区域内对密度积分并将总和设为 1,这个条件将由守恒定律保持。具有概率密度流的正确相对论性理论也必须具备这一特性。为了保持对流密度的概念,必须将薛定谔的密度和电流表达式推广,使得空间和时间的导数再次以对称的方式进入标量波函数的关系。薛定谔的表达式可以保留在电流中,但概率密度必须替换为对称形成的表达式(需要进一步解释):
\[
\rho = \frac{i \hbar}{2m c^2} \left( \psi^* \partial_t \psi - \psi \partial_t \psi^* \right)~
\]
这个表达式现在成为了时空四矢量的第四分量,整个概率四流密度具有相对论协变的表达式:
\[
J^\mu = \frac{i \hbar}{2m} \left( \psi^* \partial^\mu \psi - \psi \partial^\mu \psi^* \right)~
\]
连续方程依旧成立。现在,一切都与相对论兼容,但密度的表达式不再是正定的;\(\psi\) 和 \(\partial_t \psi\) 的初始值可以自由选择,因此密度可能变为负值,这对于合法的概率密度来说是不可能的。因此,在简单地假设波函数是相对论标量,并且它满足一个时间二阶方程的情况下,无法得到薛定谔方程的简单相对论性推广。

尽管它不是薛定谔方程的成功相对论性推广,这个方程在量子场论的背景下复兴,并被称为克莱因–戈尔登方程,用来描述无自旋粒子场(例如 \(\pi\) 介子或希格斯玻色子)。历史上,薛定谔本人在他的名字所命名的方程之前就得到了这个方程,但很快就放弃了它。在量子场论的背景下,这种不确定的密度被理解为对应于电荷密度,可以为正或负,而不是概率密度。










































































































































































































































































































































































































































































































































































































































































