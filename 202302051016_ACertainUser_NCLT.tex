% 经典形核理论
\footnote{本文参考了刘智恩的《材料科学基础》与Callister的 Material Science and Engineering An Introduction}

\begin{issues}
\issueDraft
\issueOther{未完成:非均匀形核}
\issueOther{未完成:不知道要加入什么前置}
\end{issues}

\pentry{吉布斯自由能\upref{GibbsG}, 表面张力\upref{sftens}}
\begin{figure}[ht]
\centering
\includegraphics[width=14cm]{./figures/NCLT_6.pdf}
\caption{液体凝固过程示意图} \label{NCLT_fig6}
\end{figure}
实际中,如果温度只略微低于理想的热力学凝固点(熔点),液体往往不凝固;同时,液体凝固时,往往是一部分液体先凝固,随后凝固的部分逐渐长大,直至液体完全凝固(又称形核-长大过程)。

此类现象需要更细致的热力学-动力学理论来解释。以下简要介绍液体结晶的经典形核理论。

\subsection{液体动力学模型}
%原书也没有给这个取名字...
\begin{figure}[ht]
\centering
\includegraphics[width=10cm]{./figures/NCLT_5.pdf}
\caption{液体中的分子运动示意图} \label{NCLT_fig5}
\end{figure}

相较于热力学平衡态的“处处相同、均匀”,液体动力学模型提出了一个更“动态”的液体微观模型。从结构上看,液体中存在时聚时散的小原子团(晶胚);从能量上看,液体中的能量分布不是完全均匀的,液体中每个小区域的能量总是在微小范围内不断浮动、“涨落”;从成分看,某区域内各组分的浓度也是不断浮动的。在一些教材中,这些动力学现象被归纳为“三大起伏”,即结构起伏、能量起伏与成分起伏。

\subsection{均匀形核}
均匀形核理论假定固体晶胚直接从均匀液体中产生。与理想情况不同的是,实际凝固过程中需要考虑固-液相界面的阻碍作用:液体凝固时,虽然液-固相变降低了体积自由能,但新产生的相界面又提高了表面自由能,二者间存在竞争。
\begin{figure}[ht]
\centering
\includegraphics[width=8cm]{./figures/NCLT_1.pdf}
\caption{形核时,虽然体积自由能降低,但表面自由能升高} \label{NCLT_fig1}
\end{figure}

假设由于结构、能量起伏,液体中已形成一半径为r的球形固体晶胚。前后总自由能变化: 
\begin{equation}\label{NCLT_eq1}
\Delta G  = \Delta G_V +\Delta G_S
\end{equation}
其中,
\begin{itemize}
\item $\Delta G_V = \frac{4}{3}\pi r^3 \Delta G_B$是体积自由能变
\item $\Delta G_B = \Delta H \frac{\Delta T}{T_M}$ 是单位体积自由能变
\item $\Delta H$是相变焓变(记为负值)
\item $\Delta T=T_M-T$是\textbf{过冷度}(当前实际温度与理想凝固点的差值,记为正值)
\item $T_M$是液体的理想热力学凝固点
\item $\Delta G_S = 4\pi r^2 \gamma$是表面自由能变
\end{itemize}

\begin{figure}[ht]
\centering
\includegraphics[width=10cm]{./figures/NCLT_2.pdf}
\caption{总自由能变与晶核半径关系的示意图。本人制作的一个更详细的\href{https://www.geogebra.org/m/prktxhhk}{可交互模型}(站外链接)} \label{NCLT_fig2}
\end{figure}

\subsubsection{临界形核半径}
随后,系统将自发发生减少总自由能的过程。如\autoref{NCLT_fig2} 所示,只有当晶胚的初始半径r大于某个半径时,自由能减少的方向才是晶胚自发长大的方向,否则晶胚将自发衰亡。此半径称为\textbf{临界形核半径}。
\begin{figure}[ht]
\centering
\includegraphics[width=8cm]{./figures/NCLT_3.pdf}
\caption{“生存还是毁灭,这是一个问题”} \label{NCLT_fig3}
\end{figure}

为求解临界形核半径,对\autoref{NCLT_eq1} 求导,并令 $\dv{\Delta G}{r} = 0$。解得
\begin{equation}
r_k=-\frac{2\gamma}{\Delta G_B}=-\frac{2\gamma T_M}{\Delta H \Delta T}
\end{equation}
可见,临界形核半径与过冷度和物质自身性质(表面张力、体积焓变等)等均有关。过冷度越高,临界形核半径越小。

\subsubsection{临界过冷度}

\begin{figure}[ht]
\centering
\includegraphics[width=6cm]{./figures/NCLT_4.pdf}
\caption{只有过冷度高于临界过冷度时,液体才会开始凝固} \label{NCLT_fig4}
\end{figure}
根据液体理论,当温度降低、过冷度升高时,液体中能自发形成的晶胚也就越大。因此,存在一个\textbf{临界过冷度$\Delta T^*$},只有过冷度高于临界过冷度时,液体中自发形成的晶胚半径才会大于临界形核半径、液体才会开始凝固。

\subsubsection{形核功}
根据\autoref{NCLT_eq1} 与\autoref{NCLT_fig2} ,晶胚即使达到了临界形核半径,也仍有$\Delta G>0$。这仍需额外的功 $W=\Delta G$,称为\textbf{形核功}。在均匀形核中,这部分能量来自于成分起伏。

当晶核半径等于临界形核半径$r=r_k$时,所需的形核功最大,称为\textbf{临界形核功 A}。可证明,此时$A=\frac{1}{3}\Delta G_S=\frac{4}{3}\pi r_k^2 \gamma$。由于$r_k$随过冷度增高而减小,因此形核功也随过冷度增高而减小。%;换句话说,体积自由能变只能弥补表面自由能升高的2/3,剩余1/3的表面自由能变必须由周围液体对其额外做功。

\subsubsection{形核率}
那么,过冷液体中晶核的形成速率与过冷度有关吗?以下以形核率表示晶核的形成速率,形核率即为单位体积、时间内液体中产生的晶核数量。%\textsl{太长不看:随温度降低、过冷度升高,形核速率先快后慢。}

一方面,随过冷度升高(温度降低),形核功降低,这有利于形核;但另一方面,随过冷度升高,原子从液相到固相的短程扩散被抑制,这不利于形核。如\autoref{NCLT_fig7} 所示,这对矛盾使形核率随过冷度升高而先快后慢。用Arrhenius公式可以表示为
$$N \propto \E^{-\frac{\Delta G}{kT}} \E^{-\frac{Q}{kT}}$$
其中N是形核率,$\Delta G$是形核功,Q是原子扩散的激活能。
\begin{figure}[ht]
\centering
\includegraphics[width=8cm]{./figures/NCLT_7.pdf}
\caption{随过冷度升高,形核率呈现先快后慢} \label{NCLT_fig7}
\end{figure}

对于纯金属的凝固等简单的液-固相变过程,在过冷度足够高前,凝固速度已足够快、液体已完全凝固,因此形核率不体现下降。
\begin{figure}[ht]
\centering
\includegraphics[width=6cm]{./figures/NCLT_8.pdf}
\caption{金属形核率与过冷度的关系} \label{NCLT_fig8}
\end{figure}

\subsection{非均匀形核}
非均匀形核理论假定晶胚从容器表面、液体中的杂质处产生。

可以发现,非均匀形核所需的临界过冷度(一般小10倍)、临界形核功更小,因而往往是实际中\textbf{主要}的形核过程。

\begin{example}{过冷水}
在不少科普作品中,我们往往见到“过冷水”,即放置在低温($<0 ^\circ C$)下的纯净水仍然保持液态。这便与水的形核机制有关。当水足够干净、容器壁上可供非均匀形核的位点很少时,冰的非均匀形核(均匀形核远远难于非均匀形核)就难以开始,因此水就保持液态。此时,若放入少量杂质或冰晶,就触发了冰的形核(或长大)机制,因此过冷水马上凝固为冰。
\end{example}

尽管本文的介绍主要基于液体凝固的例子,但不少结论、思想可以被推广至其他基于形核-长大机制的相变过程,例如固态相变过程。
