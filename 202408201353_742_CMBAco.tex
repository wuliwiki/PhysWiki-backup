% CMB 声学峰
% license Usr
% type Tutor



在这一部分,我们讨论宇宙微波背景(CMB)如何携带暗物质存在的印记。主要的可观测性是CMB功率谱(图1.5中间左侧板):通过执行光子温度场(同一图的顶部左侧的各向异性)的球面傅里叶变换,然后通过平均每个角动量的2ℓ+1个方向来计算方差。这个方差被测量并与膨胀宇宙大爆炸宇宙学的预测进行比较:这种比较允许我们推断宇宙的内容,包括关键的,关于暗物质的存在。

更精确地说,CMB峰值是由于重子/光子流体的声学振荡。这些声学峰的位置取决于暗物质密度(较少的暗物质意味着辐射-物质平等较晚),它们的幅度取决于相对于普通物质的暗物质的相对数量(只有后者经历声学振荡)。全局拟合发现,这些和其他宇宙学可观测性可以通过包括暗物质的标准宇宙学模型很好地再现(具有具有高斯高斯初始扰动的ΛCDM模型),并允许确定其宇宙学参数的值。这提供了目前可用的暗物质密度的最准确确定。

下面,我们扩展讨论并半定量地展示暗物质如何影响CMB功率谱的形状。我们过度简化物理并仅关注主要与暗物质相关的特征; 

CMB功率谱与1.3.1节中的物质功率谱P(k)具有相同的直观含义,但现在是针对光子各向异性的。大致上,Cℓ是大约为θ ∼ π/ℓ的各向异性量。角尺度θ大致对应于波数k ∼ ℓH0,其中H0是当前的哈勃常数。CMB功率谱随着ℓ的增加而下降,这是由于“Silk阻尼”,即小尺度结构由于光子扩散而平滑[25]。然而,第二和第三CMB峰值大致相等。这表明,如果能够去除Silk阻尼,第三个峰值将特别突出,高于第二个峰值。这就是暗物质的足迹,正如我们接下来所说明的。

在宇宙中特定位置r和时间t的光子流体中的不均匀性由函数Θ(r, t, ˆp) ≡ δT/T表示,或者等价地,由其傅里叶变换Θ(k, t, ˆp)表示。这里T是光子温度,ˆp是光子方向。这个函数服从以下玻尔兹曼演化方程:

%\[ \dot{Θ} - ik \mu a \Theta + \dot{\Psi} - ik \mu a \Phi = - \dot{[Θ_0 - Θ + \mu v_b k]}~. \]

其中μ = 