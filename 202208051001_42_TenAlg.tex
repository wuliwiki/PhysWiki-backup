% 张量代数
% 张量代数|对称代数

\begin{issues}
\issueDraft
\end{issues}

\pentry{直和(线性空间)\upref{DirSum},张量的对称化和交错化\upref{SIofTe}}
这一节将用 $\mathbb T_1^0(V)$ 来构造一个代数 $\mathbb T(V^*)$,其上的乘法为张量积.之所以用 $\mathbb T(V^*)$ 而不用 $T(V)$ 是因为 $\mathbb T_1^0(V)=V^*$. 
\subsection{共变张量代数}
由矢量空间的张量积\upref{TPofSp}知道, $\mathbb T_p^0(V)$ 上矢量和 $\mathbb T_q^0(V)$ 上矢量的张量积所在的空间为  $\mathbb T_{p+q}^0(V)$.所以要 $\mathbb T_1^0(V)$ 构造的代数其乘法为张量积,那么该代数需包含矢量空间 $\mathbb T_2^0(V)$ .于是该代数需包含子空间 $\mathbb T_1^0(V)\oplus\mathbb T_2^0(V) $

 考虑到张量积的单位元为 $1\in\mathbb F$,所以 1在该代数上,而代数的矢量空间性质要求 $\mathbb F$ 也在该代数上.所以该代数需包含子空间
\begin{equation}
\mathbb F\oplus\mathbb T_1^0(V)\oplus\mathbb T_2^0(V) 
\end{equation} 
同理,继续将该矢量空间上进行张量积,可得该代数包含子空间
\begin{equation}
\mathbb F\oplus\mathbb T_1^0(V)\oplus\mathbb T_2^0(V)\oplus\mathbb T_3^0(V)\oplus\mathbb T_4^0(V)  
\end{equation}
重复这一过程,便得所需的代数为
\begin{equation}
\mathbb F\oplus\mathbb T_1^0(V)\oplus\mathbb T_2^0(V)\oplus\cdots
\end{equation}
由直和\upref{DirSum}的性质知道,该代数上的矢量 $f$ 可记作
\begin{equation}
f=\sum_{i=0}^\infty f_i=(f_0,f_1,f_2,\cdots),\quad f_i\in\mathbb T_i^0(V)
\end{equation}
其中 $\mathbb T_0^0(V)=\mathbb F$.

设 $f=\sum_\limits i f_i,g=\sum_\limits i g_i$.显然,该代数上的\textbf{加法}为
\begin{equation}\label{TenAlg_eq1}
f+g=\sum_{i}f_i+g_i=(f_0+g_0,f_1+g_1,\cdots)
\end{equation}
\textbf{乘法}便是
\begin{equation}\label{TenAlg_eq2}
f\otimes g=\sum_{i,j}f_i\otimes g_j=\sum_k h_k
\end{equation}
其中 $h_k=\sum_\limits{i=0}^k f_i\otimes g_{k-i}$.

乘法\autoref{TenAlg_eq2} 的结合性和纯量与张量积的乘法定律直接由张量积的运算性质(\autoref{TsrPrd_the1}~\upref{TsrPrd})得到.

\begin{definition}{共变张量代数}
称代数
\begin{equation}
\mathbb T(V^*)=\mathbb F\oplus\mathbb T_1^0(V)\oplus\mathbb T_2^0(V)\oplus\cdots
\end{equation}
为矢量空间 $V$ 上的\textbf{共变张量代数},其上的加法和乘法分别由\autoref{TenAlg_eq1} ,\autoref{TenAlg_eq2} 定义.
\end{definition}

容易知道,代数 $\mathbb T(V^*)$ 是个无穷维结合代数.