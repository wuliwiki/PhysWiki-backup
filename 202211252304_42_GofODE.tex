% 常微分方程的几何图像
% keys 向量场|单参数微分同胚群|微分方程|解

\pentry{相空间和相流\upref{PSaPF},}
这里,将从几何上来理解常微分方程.这需要先引入一些概念.
\subsection{微分同胚}
以下新名词的理解只需掌握函数可微性的知识.
\begin{definition}{可微函数}
设 $U$ 是矢量空间 $\mathbb R^n$ 上的区域,其上坐标为 $x_1,\cdots,x_n$,称函数
\begin{equation}
f:U\rightarrow\mathbb R
\end{equation}
是 $U$ 上的\textbf{可微函数},若 $f(x_1,\cdots,x_n)$ 是 $r$ 次连续可微的,此处,$1\leq r\leq\infty$.通常人们都不关心 $r$ 的具体值,因此并不指明.若有需要将指出“$r$次可微”或函数类 $C^r$.
\end{definition}
\begin{definition}{可微映射}
设 $U$ 是 $\mathbb R^n$ 中区域,$V$ 是 $\mathbb R^m$ 中区域,其中 $x_1,\cdots,x_n$ 是 $U$ 中的坐标,$y_1,\cdots,y_m$ 是 $V$ 中的坐标,称映射
\begin{equation}
f:U\rightarrow V,\quad f(x_1,\cdots,x_n)=(y_1,\cdots,y_m)
\end{equation}
为\textbf{可微映射},若 $y_i=f_i(x_1,\cdots,x_n)$ 是可微函数.其中 $1\leq i\leq m$.
\end{definition}
\begin{definition}{微分同胚}
若映射 $f:U\rightarrow V$ 是个双射,且 $f$ 和其逆 $f^{-1}$ 都是可微映射,则称 $f$ 为\textbf{微分同胚}.
\end{definition}

\begin{definition}{单参数微分同胚群}
设 $M$ 是一流形(可认为是欧氏空间中一区域),若映射
\begin{equation}
g:\mathbb R\times M\rightarrow M,\quad g(t,x) =g^t x
\end{equation}
满足:1.$g$ 是可微映射;2.族 $\{g^t|t\in\mathbb R\}$ 是 $M$ 的单参数变换群(\autoref{PSaPF_def1}~\upref{PSaPF}).则称 $\{g^t|t\in\mathbb R\}$ 是 $M$ 的\textbf{单参数微分同胚群}.
\end{definition}
\begin{theorem}{}
若 $\{g^t|t\in\mathbb R\}$ 是单参数微分同胚群,则对每一 $t$,映射 $g^t$ 是微分同胚.
\end{theorem}
\textbf{证明:}因为 $g$ 是可微映射,且
\begin{equation}
\dv{g^tx}{x}=\pdv{g(t,x)}{x}
\end{equation}
所以 $g^t$ 对 $x$ 可微.又 $\{g^t\}$ 是单参数变换群,所以对每一 $t$,$g^t$ 是双射\upref{PSaPF},这意味着 $\dv{g^t x}{x}\neq 0$(因为否则 $g^t(x+\dd x)=g^t(x)$,这违反了 $g^t$ 的单射性).所以
\begin{equation}
\dv{{g^{t}}^{-1}}{x}=\frac{1}{\dv{g^t x}{x}}
\end{equation}
可微.

\textbf{证毕!}
\subsection{向量场}
掌握了以上概念,下面的概念是容易理解的.常微分方程研究的过程是可微过程,即要求相空间 $M$ 是微分流形,状态随时间的变化由可微函数描述.由于状态x经过t时间后变为 $g^tx$,要这个可微函数为 $g(t,x)=g^tx$,






