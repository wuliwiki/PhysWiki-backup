% Tychonoff定理
% 紧致性|紧性|compact|积拓扑|乘积拓扑

\pentry{紧致性\upref{Topo2},积拓扑\upref{Topo6}}

\subsection{定理的描述}

紧空间具有仅次于有限空间的优良性质.然而,一个有限空间的无限次积空间通常是无限集(只要这个有限空间的元素数大于$1$),我们可能会猜想,一个紧空间的无限次积空间可能会失去紧性.然而Tychonoff定理表明,对于积拓扑,紧空间的任意次笛卡尔积仍然是紧空间,紧性被保留了.

我们先列出该定理,然后讨论并给出其证明.

\begin{theorem}{Tychonoff定理}
给定一族拓扑空间$\{(X_\alpha, \mathcal{T}_\alpha)\}_{\alpha\in \Lambda}$,如果对每个$\alpha\in\Lambda$,$(X_\alpha, \mathcal{T}_\alpha)$都是紧空间,那么乘积空间$(\prod_{\alpha\in\Lambda}X_\alpha, \mathcal{T})$也是紧空间.这里$\mathcal{T}$是各$\mathcal{T}_\alpha$的积拓扑.
\end{theorem}



\subsection{有限积空间继承紧致性}

考虑两个紧空间$(X_1, \mathcal{T}_1)$和$(X_2, \mathcal{T}_2)$的积拓扑.为了方便视觉化理解,我们可以把$X_1$和$X_2$分别画成一根轴,于是它们的积空间就是一个平面.把各$X_i$的开集都想象成一条线段,于是$X_i$的紧致性可以描述为“任意覆盖了整条轴的线段集合,总存在有限子覆盖”.那么$X_i$的紧致性能继承到$X_1\times X_2$上吗?

乍一看似乎没关联,因为$X_1\times X_2$的开集和$X_i$开集的联系似乎只有\textbf{投影}.如果记$\{U_\alpha\}$是$X_1\times X_2$的一族开覆盖,那么投影后的$\{\pi_i(U_\alpha)\}$就是$X_i$的一族开覆盖.由$X_i$的紧致性虽然能得出$\{\pi_i(U_\alpha)\}$存在有限子覆盖$\{\pi_i(U_i)\}_{i\in I}$($I$是某个有限指标集),但没法保证$\{U_i\}_{i\in I}$是$X_1\times X_2$的覆盖.

不过,稍加优化我们会发现,由于开集都是基本开集的并、基本开集都是开集,可以得到“任意开覆盖都有有限子覆盖”等价于“任意拓扑基覆盖都有有限子覆盖”,这里拓扑基覆盖是指取拓扑基中的基本开集来覆盖.那是不是用拓扑基覆盖可以证明呢?不行,没有解决上述问题.

尽管如此,我们还是把这个结论记下来:

\begin{exercise}{}
证明:对于拓扑空间$X$的任意子空间$A$,选定一个拓扑基后,下列命题等价:
\begin{enumerate}
\item $A$的任意开覆盖都有有限子覆盖;
\item $A$的任意\textbf{拓扑基}覆盖都有有限子覆盖.
\end{enumerate}
\end{exercise}

再优化思路:“任意拓扑基覆盖都有有限子覆盖”可以等价于“任意子基覆盖都有有限子覆盖”吗?这里的子基覆盖就是指用子基中的元素紧性覆盖.

答案是肯定的.

\begin{theorem}{}
对于拓扑空间$X$和它的子空间$A$,选定一个子基和它生成的拓扑基后,下列命题等价:
\begin{enumerate}
\item $A$的任意\textbf{拓扑基}覆盖都有有限子覆盖;
\item $A$的任意\textbf{子基}覆盖都有有限子覆盖.
\end{enumerate}
\end{theorem}

\textbf{证明}:

1. $\implies$ 2. 是显然的,因为子基的元素都是拓扑基的元素,从而子基覆盖都是拓扑基覆盖.下证 2. $\implies$1. .

反设存在$A$的拓扑基覆盖,它没有有限子覆盖.由\textbf{Zorn 引理}\upref{ZornLe},在全体“没有有限子覆盖”的拓扑基覆盖构成的集合中,存在“极大”的拓扑基覆盖\footnote{这个集合中的偏序关系$\leq$由包含关系定义,即$\{B_\alpha\}_{\alpha\in \Lambda}\leq\{B_\beta\}_{\beta\in \Gamma}$当且仅当各$B_\alpha\in\{B_\beta\}$.}.即,如果$\{B_\alpha\}$是一个上述的极大拓扑基覆盖,那么对于任意基本开集$B_0\not\in\{B_\alpha\}$,覆盖$\{B_0\}\cup\{B_\alpha\}$是有有限子覆盖的.

给定子基$\mathcal{S}$,以及$A$的一个\textbf{极大的}没有有限子覆盖的\textbf{拓扑基}覆盖$\{B_\alpha\}_{\alpha\in \Lambda}$.

取指标集$I\subseteq \Lambda$,定义为:拓扑基的元素$B_i$也是子基的元素,当且仅当$i\in I$.

由于$A$的\textbf{子基}覆盖都有有限子覆盖,而按定义,\textbf{上述拓扑基}覆盖$\{B_\alpha\}_{\alpha\in \Lambda}$没有有限子覆盖,因此$\bigcup_{i\in I} B_i$无法覆盖$A$.

于是存在$x\in A-\bigcup_{i\in I} B_i$.又因为$\bigcup_{\alpha\in \Lambda}B_\alpha$覆盖$A$,故存在$\alpha_0\in\Lambda$使得$x\in B_{\alpha_0}$.

由于该拓扑基由给定子基$\mathcal{S}$导出,因此$B_{\alpha_0}$是有限多个子基的元素的交集.因此存在$S_1, S_2, \cdots, S_n\in\mathcal{S}$使得
\begin{equation}
x\in S_1\cap S_2\cap\cdots\cap S_n \subseteq B_{\alpha_0}
\end{equation}

按照$x$的取法,可知$S_1, S_2, \cdots, S_n$不等于任何$B_\alpha$,因此由于$\{B_\alpha\}_{\alpha\in\Lambda}$的极大性,







% 设$\{B_\alpha\}_{\alpha\in\Lambda}$是$A$的拓扑基覆盖,则各$B_\alpha$是由子基中\textbf{有限多个}开集的\textbf{并}:
% \begin{equation}
% B_\alpha = \bigcup_{i_\alpha\in I_\alpha}S_{\alpha, i_\alpha}
% \end{equation}
% 即$\Gamma$的基数任意,而各$I_\alpha$是有限集.



\textbf{证毕}.























