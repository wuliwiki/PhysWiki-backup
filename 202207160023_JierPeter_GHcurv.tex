% 高斯曲率和平均曲率
% 曲率|微分几何|脐点|umbilical|法曲率|形状算子|高斯映射

\pentry{高斯映射\upref{GMap}}




\begin{definition}{}
给定一个正则曲面 $S$ 和其上一条正则曲线 $C$.任取 $p\in C$,记 $k$ 为 $C$ 在 $p$ 处的曲率,$n$ 为 $C$ 在 $p$ 处的\textbf{单位}法向量,$N$ 为 $S$ 在 $p$ 处的单位法向量,则定义 $k_n=kn\cdot N$ 为曲线 $C$ 在曲面 $S$ 上的\textbf{法曲率(normal curvature)}.
\end{definition}

法曲率和曲线本身的曲率不一定相同.比如说,考虑在纸面上画一个圆,然后把纸卷成圆柱,那么圆上总有两个点的法曲率为零,但是圆本身的曲率处处不为零.至于是哪两个点,你可以先发挥一下想象力,而接下来介绍的定理可以帮助你验证想象是否准确.

事实上,法曲率的值和曲线本身关系不是特别大,只需要知道曲线在一点处的切线,就可以唯一确定其法曲率了.这一点被表述为以下定理.

\begin{theorem}{Meusnier定理}
给定正则曲面 $S$ 和其上一点 $p$,如果 $C_1$ 和 $C_2$ 是两条 $S$ 上过 $p$ 的曲线,且在 $p$ 点处二者有相同的切线,那么二者的法曲率相同.
\end{theorem}

\begin{definition}{主曲率}
给定正则曲面 $S$ 和其上一点 $p$,由Meusnier定理,该点处每一个切线方向都唯一对应一个法曲率值.这些法曲率中的最大和最小值被称为 $S$ 在点 $p$ 处的\textbf{主曲率(principal curvature)},对应的方向则是\textbf{主方向(principal direction)}.
\end{definition}

\begin{definition}{曲率曲线}
如果曲面上一条曲线处处都是主方向,那么称其为曲面上的一条\textbf{曲率曲线(line of curvature)}.\footnote{Olinde Rodrigues定理给出了曲率曲线的充要条件.}
\end{definition}

\begin{definition}{高斯曲率与平均曲率}
设曲面 $S$ 在 $p$ 处的主曲率为 $k_1$ 和 $k_2$,那么称 $K_p=k_1k_2$ 为曲面在该点处的\textbf{高斯曲率(Gaussian Curvature)},$H_p=\frac{k_1+k_2}{2}$ 为曲面在该点处的\textbf{平均曲率(mean curvature)}.
\end{definition}

高斯曲率实际上就是形状算子的\textbf{行列式},而平均曲率是形状算子\textbf{迹的一半}.

高斯曲率和形状算子还引出了对曲面上的点的一种分类:

\begin{definition}{}
对于曲面 $S$ 上的一点 $p$,我们有:
\begin{itemize}
\item 如果 $K_p>0$,则称 $p$ 是一个\textbf{椭圆点(elliptic)};
\item 如果 $K_p<0$,则称 $p$ 是一个\textbf{双曲点(hyperbolic)};
\item 如果 $K_p=0$ 但 $L_p\not=0$,则称 $p$ 是一个\textbf{抛物点(parabolic)};
\item 如果 $L_p=0$,则称 $p$ 是一个\textbf{平面点(planar)};
\end{itemize}
\end{definition}

其中对于椭圆点和双曲点,还特别地有以下性质:
\begin{theorem}{}
对于曲面 $S$ 上的一点 $p$,如果 $p$ 是椭圆点,则存在 $S$ 上 $p$ 的邻域 $V$,使得 $V-\{p\}$ 的所有点都在 $T_pS$ 的同一侧;如果 $p$ 是双曲点,则对于 $S$ 上 $p$ 的任何邻域 $U$,$T_pS$ 的两侧总有 $U$ 的点.
\end{theorem}


\begin{definition}{脐点}

对于曲面 $S$ 上的一点 $p$,如果 $k_1=k_2$,即所有方向都是主方向,那么称 $p$ 为 $S$ 上的一个\textbf{脐点(umbilical point)}.

\end{definition}

平面、球面上的每一个点都是脐点;抛物线的顶点也是脐点.

\begin{theorem}{}
如果一个正则曲面 $S$ 上处处是脐点,那么 $S$ 必然是球面的一部分.此处将平面视为球面的特例.
\end{theorem}

\begin{definition}{渐进曲线}
对于曲面 $S$ 上的一点 $p$,如果某个方向对应的法曲率为零,那么称这个方向是 $T_pS$ 上的\textbf{渐进方向(asymptotic direction)}.如果 $S$ 上一条正则曲线处处在渐进方向上,则称它为 $S$ 上的一条\textbf{渐进曲线(asymptotic curve)}.


\end{definition}


