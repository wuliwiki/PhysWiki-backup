% FFmpeg 笔记

\begin{issues}
\issueDraft
\end{issues}

\begin{itemize}
\item 查看视频全部流的编码信息 \verb|ffprobe -show_streams 视频文件|, 或者 \verb|ffprobe -i 视频文件 -show_streams| 如果只查看视频流或音频流, 用 \verb|v| 或 \verb|a|, 可以用 \verb|a:0| 查看第一个音频流。 要查看字幕流,用 \verb|s|, 数据流用 \verb|d|,附件用 \verb|t|
\item 3d 视频通常有两个视频流,音频流可以用于不同语言的配音,注意每个音频流都可以支持双声道而不是用不同流区分。
\item 数据流可以用于储存多媒体信息, 视频分节信息, 等
\item 在 \verb|-show_streams| 的输出中, \verb|codec| 是编码格式, MP4 最常用的是 H264, 高清视频也常用 H265。 音频最常用 ACC
\item \verb|bit_rate| 是每个流每秒钟的平均比特数。 一个 4GB 的两小时视频大概是 \verb|4.4Mbit/s|, 也就是 \verb|550kB/s|, iPhone Xs max 录像大约是 \verb|1MB/s|, 录屏(886x1920px)大约 \verb|2MB/s|。
\item 微信传视频如果不选原视频,会压缩到 \verb|440kbit/s|。
\end{itemize}
