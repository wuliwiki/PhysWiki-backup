% 库仑势垒
% license CCBYSA3
% type Wiki

(本文根据 CC-BY-SA 协议转载自原搜狗科学百科对英文维基百科的翻译)

\textbf{库伦势垒}是由于静电相互作用而产生的能量势垒,两个原子核需要克服静电相互作用才能靠得足够近以进行核反应。\textbf{库伦势垒}以库仑定律命名,库仑定律又以物理学家查尔斯-奥古斯丁·德·库仑(Charles-Augustin de Coulomb)命名。

\subsection{能量势垒}
这个能垒由静电势能给出:
$$U_{\text{coul}} = k \frac{q_1 q_2}{r} = \frac{1}{4\pi\epsilon_0} \frac{q_1 q_2}{r}$$

其中
\begin{equation}
k &\text{ 是库仑常数 } = 8.9876 \times 10^9 \\, \text{N m}^2 \text{C}^{-2}; \\\\
\epsilon_0 &\text{ 是自由空间的介电常数}; \\\\
q_1, q_2 &\text{ 是相互作用粒子的电荷}; \\\\
r &\text{ 是相互作用半径}.~
\end{equation}
正值表示排斥力,所以相互作用的粒子越靠近能量水平越高。负势能表示束缚态(由于吸引力)。

库仑势垒随着碰撞原子核的原子序数(即质子数)而增加:

U_{\\text{coul}} = \\frac{k Z_1 Z_2 e^2}{r}

其中
\\begin{align*}
e &\\text{ 是基本电荷 }, 1.602 \\, 176 \\, 53 \\times 10^{-19} \\, \\text{C}; \\\\
Z_i &\\text{ 是相应的原子序数}.
\\end{align*}
