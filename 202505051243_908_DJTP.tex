% 点集拓扑学(综述)
% license CCBYSA3
% type Wiki

本文根据 CC-BY-SA 协议转载翻译自维基百科\href{https://en.wikipedia.org/wiki/General_topology}{相关文章}。

\begin{figure}[ht]
\centering
\includegraphics[width=10cm]{./figures/16b10920f52e4ba5.png}
\caption{} \label{fig_DJTP_1}
\end{figure}
在数学中,一般拓扑(或称点集拓扑)是拓扑学的一个分支,主要研究拓扑学中使用的基本集合论定义和构造。它是大多数其他拓扑学分支的基础,包括微分拓扑、几何拓扑和代数拓扑。

点集拓扑中的基本概念是连续性、紧致性和连通性:

\begin{itemize}
\item 连续函数直观上是将相邻的点映射到相邻的点。
\item 紧致集是指可以被有限多个任意小的集合覆盖的集合。
\item 连通集是指不能被分成两个彼此远离的部分的集合。
\end{itemize}
