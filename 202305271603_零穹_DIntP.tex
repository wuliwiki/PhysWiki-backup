% 定积分的性质
% 定积分|性质

\pentry{定积分\upref{DInt}}
在定积分的定义(\autoref{def_DInt_3}~\upref{DInt}),说“$a$ 到 $b$ 的区间上的定积分” 时,总是理解成 $a<b$。现在我们先去除这一限制,即对定积分的定义中 $a,b$ 的大小关系不施加任何限制,进而讨论定积分的性质。为此,先介绍定向区间的概念。

\begin{definition}{定向区间}
满足不等式
\begin{equation}
a\leq x\leq b\quad or\quad a\geq x\geq b
\end{equation}
的顺序从 $a$ 到 $b$ 的 $x$ 的集合称为从 $a$ 到 $b$ 的\textbf{定向区间},记作 $[a,b]$。即 $a<b$ 则是递增顺序,$a>b$ 就是递减顺序,$a=b$ 就是不增不减。
\end{definition}
对 $a>b$ 时的定向区间 $[a,b]$ 上的积分的定义,可以用完全类似的方法定义,只需从 $a$ 到 $b$ 的方向插入分点:
\begin{equation}
x_0=a>x_1>\cdots>x_n=n
\end{equation}
于是,若积分和
\begin{equation}
\sigma=\sum_{i=0}^{n-1}f(\xi_i)\Delta x_i,\quad\Delta x_i=x_{i+1}-x_i<0
\end{equation}
在 $\lambda=\max\qty{\Delta x_i|i=0,\cdots n-1}$ 趋于 0 时的极限(\autoref{def_DInt_1}~\upref{DInt})存在,则该极限就是 $f(x0$ 在定向区间 $[a,b]$ 上的定积分,并记作
\begin{equation}
\int_a^b f(x)\dd x=\lim_{\lambda\rightarrow0}\sigma
\end{equation}

\subsection{定积分的性质}
\begin{theorem}{}\label{the_DIntP_2}
若 $f(x)$ 在区间 $[b,a]$ 上可积,则它们在 $[a,b]$ 也可积,并且
\begin{equation}
\int_a^b f(x)\dd x=-\int_b^a f(x)\dd x
\end{equation}
\end{theorem} 

\textbf{证明:}
只需对 $[a,b],[b,a]$ 取同样的分点和 $\xi_i$ ,则它们的积分和仅仅差一负号,因此其极限也只差一负号。

\textbf{证毕!}
\begin{theorem}{}\label{the_DIntP_1}
$\int_a^a f(x)\dd x=0$
\end{theorem}
\textbf{证明:}
此时 $[a,a]$ 上所有的分点和 $\xi_i$ 都是 $a$,而 $\Delta x_i=0$ 且 $f(a)$ 有限,所以积分和每一项都是0。

\textbf{证毕!} 

当然,如果定向区间 $[a,b]$ 的定义不允许 $a=b$ ,那么可以把 $\int_a^a f(x)\dd x$ 定义为 $\lim\limits_{b\rightarrow a}\int_a^b f(x)\dd x$,此时也有\autoref{the_DIntP_1} 的性质。所以两种定义带来的结果都是等价的,如何理解都可。

为方便下面的证明,使用的符号 $\sum\limits_{c}^d f(\xi)\Delta x$ 代表 $\sum\limits_{i=0}^{n-1} f(\xi_i)\Delta x_i$ 其中 $x_0=c,x_{n-1}=d$,可以理解成矢量 $f(\xi)=(f(\xi_0),\cdots,f(\xi_{n-1}))$ 和 $\Delta x=(\Delta x_0,\cdots,\Delta x_{n-1})$ 的内积 ,其中 $x_0=c,x_{n-1}=d.$
\begin{theorem}{}
设 $f(x)$ 在区间 $[a,b]$ 上可积,那么对任意的 $c,d,e\in[a,b]$,成立
\begin{equation}
\int_c^d f(x)\dd x=\int_c^e f(x)\dd x+\int_e^d f(x)\dd x
\end{equation}
\end{theorem}
\textbf{证明:}
由\autoref{the_DIntP_2} ,只需对 $c<d$ 的情形证明即可。由\autoref{the_InFun_1}~\upref{InFun},$f(x)$ 在 $[a,b]$ 上的每一部分区间都可积。

设 $e\in [c,d]$,并令 $e$ 是分点之一,于是
\begin{equation}
\sum_{c}^d f(\xi)\Delta x=\sum_c^e f(\xi)\Delta x+\sum_e^d f(\xi)\Delta x
\end{equation}
两边取极限就得所需证的等式。

设 $e\notin[c,d]$,不失一般性,令 $e\leq c$,于是由上一情形
\begin{equation}
\begin{aligned}
&\int_e^d f(x)\dd x=\int_e^c f(x)\dd x+\int_c^d f(x)\dd x\\
&\Downarrow\\
&\int_c^d f(x)\dd x=-\int_e^c f(x)\dd x+\int_e^d f(x)\dd x=\int_c^e f(x)\dd x+\int_e^d f(x)\dd x+
\end{aligned}
\end{equation}

\textbf{证毕!}

\begin{theorem}{}
设 $f(x),g(x)$ 在区间 $[a,b]$ 上可积,则
\begin{enumerate}
\item $\int_a^b kf(x)\dd x=k\int_a^b f(x)\dd x$,($k$是常数);
\item $\int_a^b[f(x)\pm g(x)]\dd x=\int_a^b f(x)\dd x\pm\int_a^b g(x)\dd x$.
\end{enumerate}
\end{theorem}
\textbf{证明:}
利用积分和出发证明即可:
\begin{equation}
\begin{aligned}
&\sum_{a}^bkf(\xi)\Delta x=k\sum_{a}^bf(\xi)\Delta x;\\
&\sum_a^b \qty[f(\xi)\pm g(\xi)]\Delta x=\sum_a^bf(\xi)\Delta x\pm \sum_a^bg(\xi)\Delta x
\end{aligned}
\end{equation}
上面两边取极限便得所需证的等式。

\textbf{证毕!} 

\begin{theorem}{}
如果在 $[a,b]$ 上可积函数 $f(x)$ 是非负的,并且 $a<b$ ,则
\begin{equation}
\int_a^b f(x)\dd x\geq0
\end{equation}
此时,若可积函数 $f(x)$ 在 $[a,b]$ 上是正的,则
\begin{equation}\label{eq_DIntP_1}
\int_a^b f(x)\dd x>0
\end{equation}
\end{theorem}
\textbf{证明:}
写出达布下和
\begin{equation}
s=\sum_{a}^bm\Delta x
\end{equation}
由于 $f(\xi),\Delta x\geq0$,所以和的每一项都非负,由定积分存在条件和\autoref{eq_Rieman_2}~\upref{Rieman},
\begin{equation}
0\leq s\leq\int_a^b f(x)\dd x.
\end{equation}
对于\autoref{eq_DIntP_1} 的证明,完全一样,由于在某一区间分划下,达布下和大于0,可以该分划的分点为基础获得新的 $\lambda\rightarrow0$ 的分划,由达布下和的性质\autoref{the_Rieman_1}~\upref{Rieman} 和\autoref{eq_Rieman_2}~\upref{Rieman} 
$0<s\leq\int_a^b f(x)\dd x.$

\textbf{证毕!}