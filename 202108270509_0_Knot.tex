% 绳结问题

\pentry{力的分解与合成\upref{Fdecom}}

作为力的分解与合成的一个应用, 考虑 $N$ 根质量和粗细不计的绳子的末端连接到一个质量不计的绳结, 假设每根绳子拉绳结的力为 $\bvec F_i$, 那么绳结受力平衡的充分必要条件是合力为零
\begin{equation}
\sum_{i=1}^N \bvec F_i = \bvec 0
\end{equation}
显然, 当 $N = 2$ 时, 两个拉力必须等大反向.

\begin{example}{三点拉绳}
从 $P_1,P_2,P_3$ 三点以固定大小的力 $F_1, F_2, F_3$ 拉一个绳结, 求绳结平衡的条件以及平衡位置.
\begin{figure}[ht]
\centering
\includegraphics[width=6.5cm]{./figures/Knot_1.pdf}
\caption{三点拉绳示意图} \label{Knot_fig1}
\end{figure}

解: 首先平衡位置必定在三角形内. 绳结到三点的单位矢量分别为 $\uvec r_1, \uvec r_2, \uvec r_3$, 那么
\begin{equation}
F_1 \uvec r_1 + F_2 \uvec r_2 + F_3 \uvec r_3 = \bvec 0
\end{equation}
根据该关系以及矢量相加的平行四边形法则\upref{GVecOp}, 容易求出 $\uvec r_1, \uvec r_2, \uvec r_3$ 两两之间的夹角, 记为 $\theta_{12}, \theta_{23}, \theta_{13}$. 但 $F_1, F_2, F_3$ 必须满足任意两个之和大于第三个才可能有解.

然后根据根据 $\theta_{12}$ 过 $P_1, P_2$ 作一条弧线, 半径为(\autoref{SphTri_eq1}~\upref{SphTri})
\begin{equation}
R = \frac{l_{12}}{\sin\theta_{12}}
\end{equation}
那么绳结必定落在弧线上. 接下来在三角形内的弧线上移动绳结, 若满足另外一个角 $\theta_{23}$ 或 $\theta_{13}$ 的值, 就可以找到平衡点, 若不存在, 则无平衡点.
\end{example}

另一个例题见\autoref{Spring_ex1}~\upref{Spring}.
