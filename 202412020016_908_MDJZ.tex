% 密度矩阵(综述)
% license CCBYSA3
% type Wiki

本文根据 CC-BY-SA 协议转载翻译自维基百科\href{https://en.wikipedia.org/wiki/Density_matrix}{相关文章}。

在量子力学中,密度矩阵(或密度算符)是描述物理系统集体(即使该集体仅包含一个系统)作为量子态的矩阵。它允许通过博恩规则计算对集体中各系统进行测量后的结果概率。密度矩阵是比常见的状态向量或波函数更一般的概念:状态向量或波函数只能表示纯态,而密度矩阵也能表示混合集体(有时模糊地称为混合态)。混合集体在量子力学中有两种情况:
\begin{enumerate}
\item 当系统的准备导致集体中有多个纯态时,需要处理可能准备的统计分布;
\item 当想要描述一个与另一个系统纠缠的物理系统时,而不描述它们的组合态;这种情况通常出现在系统与某些环境交互时(例如去相干)。在这种情况下,纠缠系统的密度矩阵不同于由多个纯态组成的集体的密度矩阵,后者在测量时会给出相同的统计结果。
\end{enumerate}
因此,密度矩阵在处理混合集体的量子力学领域中是至关重要的工具,如量子统计力学、开放量子系统和量子信息等。
\subsection{定义与动机}
密度矩阵是一个线性算符的表示,称为密度算符。密度矩阵是通过在基础空间中选择一个正交规范基来从密度算符中得到的。[2]实际上,密度矩阵和密度算符这两个术语常常可以互换使用。

假设选择一个基,其中有两个状态 \(|0\rangle\) 和 \(|1\rangle\) 在二维希尔伯特空间中,那么密度算符可以表示为矩阵:
\[
(\rho_{ij}) = \left( \begin{matrix} \rho_{00} & \rho_{01} \\ \rho_{10} & \rho_{11} \end{matrix} \right) = \left( \begin{matrix} p_0 & \rho_{01} \\ \rho_{01}^* & p_1 \end{matrix} \right)~
\]
其中对角元素是实数,并且它们的和为1(也称为两个状态 \(|0\rangle\) 和 \(|1\rangle\) 的占据数)。非对角元素是彼此的复共轭(也称为相干性);它们的幅度受到密度算符是正半定算符这一要求的限制,见下文。

密度算符是一个正半定的自伴算符,其迹为1,并作用于系统的希尔伯特空间。[3][4][5]这一定义可以通过考虑一个情况来动机化,其中每个纯态 \(|\psi_j\rangle\) 以概率 \(p_j\) 被准备,描述的是一个纯态的集体。在使用投影算符 \(\Pi_m\) 时,获得投影测量结果 \(m\) 的概率由以下公式给出[6]:
\[
p(m) = \sum_j p_j \langle \psi_j | \Pi_m | \psi_j \rangle = \operatorname{tr} \left[\Pi_m \left( \sum_j p_j |\psi_j \rangle \langle \psi_j | \right) \right]~
\]
这使得密度算符定义为:
\[
\rho = \sum_j p_j |\psi_j \rangle \langle \psi_j |~
\]
成为这个集体状态的一个方便表示。可以很容易地验证,这个算符是正半定的,自伴的,并且具有迹为1。反过来,根据谱定理,任何具有这些性质的算符都可以写成:\(\sum_j p_j |\psi_j \rangle \langle \psi_j |\)其中 \( |\psi_j \rangle \) 是某些状态,且系数 \(p_j\) 非负且其和为1。[7][6]然而,如施罗德–HJW 定理所示,这个表示并不唯一。

密度算符定义的另一个动机来自于考虑对纠缠态的局部测量。设 \(|\Psi\rangle\) 是复合希尔伯特空间 \(\mathcal{H}_1 \otimes \mathcal{H}_2\) 中的一个纯纠缠态。当在希尔伯特空间 \(\mathcal{H}_1\) 上对投影算符 \(\Pi_m\) 进行测量时,获得测量结果 \(m\) 的概率由下式给出[6]:
\[
p(m) = \langle \Psi | (\Pi_m \otimes I) | \Psi \rangle = \operatorname{tr} \left[ \Pi_m \left( \operatorname{tr}_2 |\Psi \rangle \langle \Psi | \right) \right]~
\]
其中,\(\operatorname{tr}_2\) 表示对希尔伯特空间 \(\mathcal{H}_2\) 的部分迹运算。这使得算符:
\[
\rho = \operatorname{tr}_2 |\Psi \rangle \langle \Psi |~
\]
成为计算这些局部测量概率的一个方便工具。它被称为纯态 \(|\Psi\rangle\) 在子系统 1 上的约化密度矩阵。可以很容易地验证,这个算符具有所有密度算符的性质。反过来,施罗德–HJW定理意味着,所有密度算符都可以写成:\(\operatorname{tr}_2 |\Psi \rangle \langle \Psi |\)对于某个状态 \(|\Psi \rangle\)。