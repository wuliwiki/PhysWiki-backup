% 毕达哥拉斯(综述)
% license CCBYSA3
% type Wiki

本文根据 CC-BY-SA 协议转载翻译自维基百科\href{https://en.wikipedia.org/wiki/Pythagoras}{相关文章}。

\begin{figure}[ht]
\centering
\includegraphics[width=6cm]{./figures/76de0efc99539521.png}
\caption{萨摩斯的毕达哥拉斯雕像,位于罗马的卡比托利尼博物馆[1]} \label{fig_Pythag_1}
\end{figure}
萨摩斯的毕达哥拉斯(古希腊语:Πυθαγόρας;约公元前570年-约公元前495年),通常以单名“毕达哥拉斯”而为人熟知,是一位古代伊奥尼亚希腊哲学家、博学家,也是毕达哥拉斯学派的创始人。他的政治和宗教教义在大希腊地区广为人知,并且影响了柏拉图、亚里士多德的哲学思想,进而影响了整个西方。关于他生平的知识被传说所笼罩;现代学者对毕达哥拉斯的教育背景和影响存在分歧,但他们一致认为,在公元前530年左右,他前往意大利南部的克罗顿,在那里创立了一所学校,要求学员发誓保守机密,并过着共同体的禁欲主义生活。

在古代,毕达哥拉斯被誉为许多数学和科学发现的创立者,包括毕达哥拉斯定理、毕达哥拉斯调律、五种规则立体、比例理论、地球的球形学说,以及早晨星和晚星是金星的天体认知。据说他是第一个自称为“哲学家”(即“智慧之爱者”)的人,也是第一个将地球划分为五个气候带的人。古代历史学家对毕达哥拉斯是否做出了这些发现存在争议,许多归功于他的成就可能在他之前就已被发现,或者是由他的同事或继承者完成的。一些记载提到,与毕达哥拉斯相关的哲学与数学有关,数字在其中占有重要地位,但关于他在数学或自然哲学方面的实际贡献程度仍存在争议。

与毕达哥拉斯最为密切相关的教义是“灵魂的轮回”或“转世说”,即认为每个灵魂是永恒的,死后会进入另一个身体。他可能还创立了“宇宙音乐”(musica universalis)的学说,认为行星的运动遵循数学方程,因此产生一种不可听见的音乐交响曲。学者们争论毕达哥拉斯是否发展了与数字学和音乐相关的教义,或者这些教义是否由他的后继者,特别是克罗顿的菲洛劳斯(Philolaus)所创立。在公元前510年左右,克罗顿在与希巴里斯的决定性战役中取得胜利后,毕达哥拉斯的追随者与民主派支持者发生冲突,毕达哥拉斯学派的集会场所被焚毁。毕达哥拉斯可能在这次迫害中被杀害,或者他逃亡到梅塔庞图姆并在那里去世。

毕达哥拉斯对柏拉图产生了影响,柏拉图的对话录,特别是《蒂迈欧篇》,展示了毕达哥拉斯的教义。毕达哥拉斯关于数学完美的思想也影响了古希腊的艺术。他的教义在公元前1世纪经历了复兴,尤其是在中期柏拉图主义者中,并与新毕达哥拉主义的兴起相吻合。毕达哥拉斯在中世纪一直被视为伟大的哲学家,他的哲学对尼古拉斯·哥白尼、约翰内斯·开普勒和艾萨克·牛顿等科学家产生了重大影响。毕达哥拉斯的象征主义也在早期现代欧洲的神秘主义中广泛使用,而他在奥维德《变形记》中所描绘的教义后来还影响了现代素食主义运动。
\subsection{传记资料}  
没有毕达哥拉斯的真实著作流传下来,[2][3][4] 关于他的生活几乎没有确凿的资料。[5][6][7] 最早关于毕达哥拉斯生平的资料简短、模糊,并且常带有讽刺性。[4][8][9] 最早的关于毕达哥拉斯教义的资料是一首讽刺诗,可能是毕达哥拉斯去世后由他的一位同时代人——科洛丰的希腊哲学家克塞诺芬尼(约公元前570年-公元前478年)所写。[10][11] 诗中,克塞诺芬尼描述了毕达哥拉斯为一只被打的狗求情,声称在狗的哀嚎中听出了已故朋友的声音。[d][9][10][12] 克罗顿的医生阿尔克迈翁(约公元前450年,约与毕达哥拉斯生活在同一时期)将许多毕达哥拉斯的教义融入自己的著作中[13],并暗示自己可能与毕达哥拉斯有过私人接触。[13] 以弗所的诗人赫拉克利特(约公元前500年,出生地距离萨摩斯岛不远,可能与毕达哥拉斯同时代)讽刺毕达哥拉斯为一个狡猾的骗子,[8][14] 并评论道:“毕达哥拉斯,孟内萨尔库斯之子,比任何人都更擅长探求,他从这些著作中选择并制造了一种属于自己的智慧——博学多才,巧妙的伪善。”[8][14]
\begin{figure}[ht]
\centering
\includegraphics[width=6cm]{./figures/920ad764d7f392b0.png}
\caption{17世纪版画中的毕达哥拉斯虚构肖像} \label{fig_Pythag_2}
\end{figure}
古希腊诗人基俄斯的伊翁(约公元前480年—公元前421年)和阿克拉加斯的恩培多克勒斯(约公元前493年—公元前432年)都在他们的诗篇中表达了对毕达哥拉斯的钦佩。[15] 关于毕达哥拉斯的第一次简洁描述出自历史学家哈利卡纳苏斯的希罗多德(约公元前484年—公元前420年),他将毕达哥拉斯描述为最伟大的希腊教师之一,并且表示毕达哥拉斯教导他的追随者如何获得不朽。[16] 然而,希罗多德的著作准确性存在争议。[17][18][19][20][21] 被归于毕达哥拉斯哲学家克罗顿的菲洛劳斯(约公元前470年—公元前385年)的著作,是最早描述数字学和音乐理论的文本,这些理论后来归于毕达哥拉斯。[22] 雅典修辞学家伊索克拉底(约公元前436年—公元前338年)是第一个描述毕达哥拉斯曾访问埃及的人。[16] 亚里士多德(约公元前384年—公元前322年)曾写过一篇关于毕达哥拉斯的论文,现已失传。[23] 其中的一些内容可能被保存在《劝学篇》里。亚里士多德的弟子迪凯阿科斯、阿里斯托克修斯和赫拉克利德(他们都生活在公元前3世纪)也写过相关的著作。[24]

关于毕达哥拉斯生平的主要来源大多来自罗马时期,[25] 根据德国古典学者沃尔特·伯克特的说法,“毕达哥拉斯主义的历史已经是……一项艰难的重建,恢复已失去的东西。”[24] 三部来自晚期古代的古代传记仍然存世,[7][25] 它们主要充满了神话和传说。[7][25][26] 其中最早且最值得尊敬的传记出自狄俄尼修斯·拉尔修斯的《名哲学家列传》。[25][26] 另外两部传记是新柏拉图主义哲学家波尔菲里(Porphyry)和扬布利库斯(Iamblichus)所写,[25][26] 部分内容意在反对基督教的兴起。[26] 后来的来源比早期的传记更长,[25] 对毕达哥拉斯成就的描述也更加神奇。[25][26] 波尔菲里和扬布利库斯使用了亚里士多德弟子(迪凯阿科斯、阿里斯托克修斯和赫拉克利德)的失传著作中的材料,[24] 这些材料通常被认为是最可靠的。[24]