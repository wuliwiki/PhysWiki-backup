% 斯坦尼斯瓦夫·乌拉姆(综述)
% license CCBYSA3
% type Wiki

本文根据 CC-BY-SA 协议转载翻译自维基百科\href{https://en.wikipedia.org/wiki/Stanis\%C5\%82aw_Ulam}{相关文章}。
\begin{figure}[ht]
\centering
\includegraphics[width=6cm]{./figures/15377c1b183c5f57.png}
\caption{乌拉姆在洛斯阿拉莫斯} \label{fig_Ulam_1}
\end{figure}
斯坦尼斯瓦夫·马尔钦·乌拉姆(波兰语:[sta'ɲiswaf 'mart͡ɕin 'ulam];1909年4月13日 – 1984年5月13日)是波兰数学家、核物理学家和计算机科学家。他参与了曼哈顿计划,提出了特勒–乌拉姆热核武器设计,发现了细胞自动机的概念,发明了蒙特卡罗计算方法,并提出了核脉冲推进技术。在纯数学和应用数学领域,他证明了多个定理并提出了若干猜想。

乌拉姆出生于奥匈帝国利沃夫的一个富裕的波兰犹太家庭;他在利沃夫工艺大学学习数学,并于1933年在卡齐米日·库拉特科夫斯基(Kazimierz Kuratowski)和弗沃季米日·斯托热克(Włodzimierz Stożek)的指导下获得博士学位。[1] 1935年,乌拉姆在华沙遇到了约翰·冯·诺依曼,后者邀请他到新泽西州普林斯顿的高等研究院待几个月。从1936年到1939年,他每年夏天都回波兰,学年则在马萨诸塞州剑桥的哈佛大学度过,在那里他致力于建立关于遍历理论的重要成果。1939年8月20日,他和17岁的弟弟亚当·乌拉姆一起最后一次乘船前往美国。1940年,他成为威斯康星大学麦迪逊分校的助理教授,并于1941年成为美国公民。

1943年10月,乌拉姆收到了汉斯·贝特的邀请,加入位于新墨西哥州洛斯阿拉莫斯的曼哈顿计划秘密实验室。在那里,他负责进行流体动力学计算,以预测爆炸透镜在内爆型武器中的行为。他被分配到爱德华·泰勒的团队,在泰勒和恩里科·费米的指导下,参与了泰勒的“超级”炸弹项目。战后,他离开洛斯阿拉莫斯,成为南加州大学的副教授,但在1946年回到洛斯阿拉莫斯,继续从事热核武器的研究。在一群女性“计算员”的帮助下,他发现泰勒的“超级”设计不可行。1951年1月,乌拉姆和泰勒共同提出了泰勒–乌拉姆设计,这一设计成为所有热核武器的基础。

乌拉姆考虑了火箭核推进的问题,这一问题由“罗孚计划”(Project Rover)进行研究。他提出了一种替代“罗孚计划”核热火箭的方法,即利用小规模的核爆炸进行推进,这一方案后来成为了“猎鹰计划”(Project Orion)。与费米、约翰·帕斯塔(John Pasta)和玛丽·青果(Mary Tsingou)一起,乌拉姆研究了著名的费米–帕斯塔–乌拉姆–青果问题(Fermi–Pasta–Ulam–Tsingou problem),这一问题成为了非线性科学领域的启发来源。他可能最为人知的是意识到,电子计算机使得将统计方法应用于没有已知解的函数变得可行。随着计算机的发展,蒙特卡罗方法(Monte Carlo method)已成为解决许多问题的常见且标准的方法。
\subsection{波兰}  
乌拉姆于1909年4月13日出生在加利西亚的莱姆堡(Lemberg)。当时,加利西亚属于奥匈帝国的加利西亚和洛多梅里亚王国,波兰人称其为奥地利分治区。1918年,莱姆堡成为新恢复的波兰第二共和国的一部分,并重新取回了其波兰名字——利沃夫(Lwów)。

乌拉姆家族是一个富裕的波兰犹太家庭,从事银行业、工业和其他专业工作。乌拉姆的直系家庭“生活富足,但并不算非常富有”。他的父亲,约瑟夫·乌拉姆(Józef Ulam),出生在利沃夫,是一名律师;母亲安娜(Anna, née Auerbach)出生于斯特里(Stryj)。他的叔叔米哈乌·乌拉姆(Michał Ulam)是一名建筑师、建筑承包商和木材工业家。从1916年到1918年,约瑟夫的家庭曾暂时居住在维也纳。返回后,利沃夫成为波兰–乌克兰战争的中心,期间该市遭遇了乌克兰的围困。
\begin{figure}[ht]
\centering
\includegraphics[width=6cm]{./figures/d11840c23803e51e.png}
\caption{位于乌克兰利沃夫的苏格兰咖啡馆大楼现在是Szkocka餐厅和酒吧的所在地(该餐厅以原苏格兰咖啡馆命名)。} \label{fig_Ulam_2}
\end{figure}
1919年,乌拉姆进入了利沃夫第七中学,并于1927年毕业。[10] 随后,他在利沃夫理工学院学习数学。在卡兹米日·库拉托夫斯基的指导下,他于1932年获得文学硕士学位,并于1933年获得科学博士学位。[9][11] 在1929年,年仅20岁的乌拉姆在《数学基础》杂志上发表了他的第一篇论文《关于集合的函数》。[11] 从1931年到1935年,他前往并在维尔纽斯(今立陶宛首都)、维也纳、苏黎世、巴黎和英国剑桥学习,在那里他结识了G·H·哈迪和苏布拉马尼扬·钱德拉塞卡。[12]

乌拉姆与斯坦尼斯瓦夫·马祖尔、马克·卡茨、弗沃季米日·斯托热克、卡兹米日·库拉托夫斯基等人一起,是利沃夫数学学派的成员。该学派的创始人是胡戈·施泰因豪斯和斯特凡·巴纳赫,他们是雅努什·卡兹米日大学的教授。这些数学家常常在苏格兰咖啡馆聚会,讨论他们的问题,这些问题被收录在《苏格兰书》中,这是由巴纳赫的妻子提供的一本厚重的笔记本。乌拉姆是这本书的主要贡献者之一。在1935年至1941年间记录的193个问题中,他作为唯一作者贡献了40个问题,又与巴纳赫和马祖尔一起合作贡献了11个问题,并与其他人共同贡献了15个问题。1957年,他从施泰因豪斯那里收到了这本幸存下来的书,并将其翻译成了英语。[13] 1981年,乌拉姆的朋友R·丹尼尔·莫尔丁发布了扩展版和注释版。[14]
\subsection{移居美国}  
1935年,乌拉姆在华沙遇见的约翰·冯·诺依曼邀请他前往新泽西州普林斯顿的高等研究院待几个月。那年12月,乌拉姆启程前往美国。在普林斯顿,他参加了讲座和研讨会,听到了奥斯瓦尔德·维布伦、詹姆斯·亚历山大和阿尔伯特·爱因斯坦的演讲。在冯·诺依曼家的一次茶话会上,他遇到了G·D·伯克霍夫,伯克霍夫建议他申请哈佛大学学者协会的职位。[9] 根据伯克霍夫的建议,乌拉姆从1936年到1939年夏季在波兰度过,学年则在马萨诸塞州剑桥的哈佛大学度过,并与约翰·C·奥克斯托比合作,研究了遍历理论的相关成果。这些成果于1941年发表在《数学年刊》上。[10][15] 1938年,乌拉姆的母亲安娜·汉娜·乌拉姆(原名奥尔巴赫)因癌症去世。

1939年8月20日,在格丁尼亚,约瑟夫·乌拉姆与他的兄弟希蒙一起,将他的两个儿子,斯坦尼斯瓦夫和17岁的亚当,送上了一艘前往美国的船。[9] 十一天后,德国人入侵了波兰。两个月内,德国完成了对西波兰的占领,苏联则入侵并占领了东波兰。两年内,约瑟夫·乌拉姆和他的家人,包括斯坦尼斯瓦夫的妹妹斯特凡尼亚·乌拉姆,成为了大屠杀的受害者,胡戈·斯坦豪斯则藏匿起来,卡齐米日·库拉特科夫斯基在华沙的地下大学讲课,弗沃季米日·斯托日克和他的两个儿子在利沃夫教授大屠杀中丧生,最后一个问题被记录在《苏格兰书》里。斯特凡·巴纳赫通过在鲁道夫·韦格尔的伤寒研究所喂食虱子,幸存于纳粹占领期间。1963年,亚当·乌拉姆(他在哈佛大学成为了著名的克里姆林学家)收到了乔治·沃尔斯基的信件,[17] 沃尔斯基曾在波兰军队叛逃后藏匿在约瑟夫·乌拉姆的家中。这段回忆记录了1939年末利沃夫混乱的景象。[18] 晚年,乌拉姆自称“是一个不可知论者。有时我深思那些对我而言不可见的力量。当我几乎接近上帝的概念时,我立刻因这个世界的恐怖而感到疏离,他似乎容忍这些恐怖”。[19]

1940年,在比尔科夫的推荐下,乌拉姆成为威斯康星大学麦迪逊分校的助理教授。在这里,他于1941年成为美国公民。[9] 同年,他与弗朗索瓦丝·阿隆结婚。[10] 她曾是霍利奥克学院的法国交换生,他们在剑桥相识。他们有一个女儿,克莱尔。在麦迪逊,乌拉姆结识了他的朋友和同事C·J·埃弗雷特,他们合作发表了多篇论文。[20]
\subsection{曼哈顿计划}
\begin{figure}[ht]
\centering
\includegraphics[width=6cm]{./figures/d0b3ffc64d56e78d.png}
\caption{乌拉姆在洛斯阿拉莫斯国家实验室的身份证照片} \label{fig_Ulam_3}
\end{figure}
1943年初,乌拉姆请求冯·诺依曼为他找到一份战争工作。10月,他收到了一个邀请函,邀请他加入一个位于新墨西哥州圣菲附近的未透露名称的项目。[9] 这封信是由汉斯·贝特签署的,他被罗伯特·奥本海默任命为洛斯阿拉莫斯国家实验室的理论部门负责人。[21] 乌拉姆对这个地方一无所知,于是他借了一本新墨西哥州的指南。在借书卡上,他发现了他在威斯康星大学的同事们——琼·欣顿、大卫·弗里施和约瑟夫·麦基本的名字,他们都神秘地消失了。[9] 这就是乌拉姆与曼哈顿计划的第一次接触,曼哈顿计划是美国在二战期间为了制造原子弹而开展的工作。[22]
\subsubsection{湍流塌缩的水动力学计算}  
乌拉姆于1944年2月抵达洛斯阿拉莫斯后几周,项目遭遇了一场危机。1944年4月,埃米里奥·塞格雷发现,反应堆中制造的钚无法用于像“瘦子”那样的枪式钚武器,该武器与同时开发的铀武器“原子男孩”并行,而“原子男孩”最后被投放到广岛。这一问题威胁到对汉福德现场新建反应堆的大量投资,并使得缓慢的铀同位素分离成为准备适用于武器的裂变材料的唯一途径。为应对这一问题,奥本海默在8月实施了一次大规模的实验室重组,将重点放在开发内爆型武器上,并任命乔治·基斯季亚科夫斯基为内爆部门负责人。他是哈佛大学的教授,也是精确使用爆炸物的专家。[23]

内爆的基本概念是使用化学炸药将一块裂变材料压缩到临界质量,在那里中子倍增导致核链式反应,释放大量能量。赛斯·内德迈尔曾研究过圆柱形内爆配置,但约翰·冯·诺依曼具有使用装甲穿透弹药中形状炸药的经验,他是球形内爆驱动爆炸透镜的强烈倡导者。他意识到内爆压缩钚的对称性和速度是关键问题,[23]并邀请乌拉姆协助设计能够提供几乎球形内爆的透镜配置。在内爆中,由于巨大的压力和高温,固体材料的行为更像流体。这意味着需要进行水动力学计算来预测并最小化可能破坏核爆炸的非对称性。对于这些计算,乌拉姆说:

水动力学问题表述简单,但计算起来非常困难——不仅是细节上的,甚至是数量级上的。在这次讨论中,我强调了纯粹的实用主义,并认为必须通过简单的蛮力方法获得问题的启发式概览,而不是依赖大量的数值计算。[9]

尽管当时的设施非常原始,乌拉姆和冯·诺依曼还是进行了数值计算,最终得出了一个令人满意的设计。这促使他们在洛斯阿拉莫斯推动了强大的计算能力的建设,这一工作始于战争期间,[24]并延续到了冷战时期,至今仍然存在。[25] 奥托·弗里施回忆道,乌拉姆是“一位才华横溢的波兰拓扑学家,拥有迷人的法国妻子。他立刻告诉我,他是一个纯粹的数学家,已经堕落到自己的最新论文中居然出现了带小数点的数字!”[26]
\subsubsection{分支和乘法过程的统计学 } 
即使是在链式反应中,中子倍增的固有统计波动,也会对聚爆速度和对称性产生影响。1944年11月,David Hawkins 和 Ulam 在一篇题为《乘法过程理论》的报告中讨论了这个问题[27]。这篇报告利用了概率生成函数,也是关于分支和乘法过程统计学的早期文献之一[28]。1948年,Ulam 和 Everett 扩展了这一研究的范围[29]。

在曼哈顿计划初期,恩里科·费米将注意力集中在利用反应堆生产钚上。1944年9月,他到达洛斯阿拉莫斯,不久后他成功恢复了第一座汉福德反应堆的正常运行,该反应堆曾因氙同位素污染而停运。[30] 在费米到达后不久,泰勒的‘超级’原子弹小组(乌拉姆是其中一员)被调到由费米领导的新部门。[31] 费米和乌拉姆建立了密切的合作关系,战后这一关系取得了丰硕的成果。[32]
\subsection{战后洛斯阿拉莫斯}  
1945年9月,乌拉姆离开洛斯阿拉莫斯,成为洛杉矶南加州大学的副教授。1946年1月,他患上了急性脑炎,生命垂危,但通过紧急脑部手术得以缓解。在恢复期间,许多朋友前来探望,其中包括来自洛斯阿拉莫斯的尼古拉斯·梅特罗波利斯和著名数学家保罗·厄尔德什[33],厄尔德什说道:“斯坦,你和以前一样。”[9] 这让乌拉姆感到宽慰,因为他当时担心自己的智力状态,因为在危机期间他失去了说话的能力。另一位朋友,吉安-卡洛·罗塔,在1987年的一篇文章中指出,这次脑炎发作改变了乌拉姆的性格:此后,他从严格的纯数学转向更具推测性的数学猜想,关注数学在物理学和生物学中的应用;罗塔还引用了乌拉姆的前合作者保罗·斯坦的说法,称乌拉姆此后在穿着上变得不那么讲究,约翰·奥克斯托比则指出,乌拉姆在患脑炎之前可以连续几个小时进行计算,而在与罗塔合作时,他甚至不愿解一个二次方程。[34] 这一说法未被乌拉姆的妻子弗朗索瓦丝·阿龙·乌拉姆接受。[35]

到1946年4月底,乌拉姆已康复到足以参加在洛斯阿拉莫斯举行的一个秘密会议,讨论热核武器。与会者包括乌拉姆、冯·诺依曼、梅特罗波利斯、泰勒、斯坦·弗兰克尔以及其他人。在曼哈顿计划期间,泰勒的努力一直集中在基于核聚变的‘超级’武器的开发上,而不是实用裂变炸弹的开发。在广泛讨论之后,与会者达成共识,认为他的想法值得进一步探索。几周后,乌拉姆收到了梅特罗波利斯和新任理论部主任罗伯特·D·里希特迈尔的邀请,提供了一个薪资更高的职位,乌拉姆一家人便回到了洛斯阿拉莫斯。[36]
\subsubsection{蒙特卡洛方法}
\begin{figure}[ht]
\centering
\includegraphics[width=6cm]{./figures/feeb318326d46f16.png}
\caption{斯坦·乌拉姆拿着FERMIAC} \label{fig_Ulam_4}
\end{figure}
在战争后期,在冯·诺依曼的支持下,弗兰克尔和梅特罗波利斯开始对第一台通用电子计算机——位于马里兰州阿伯丁试验场的ENIAC进行计算。乌拉姆在返回洛斯阿拉莫斯后不久,参与了对这些计算结果的审查。[37] 早些时候,在从手术中恢复期间,乌拉姆在玩接龙时曾想过通过玩几百局游戏来统计估算成功结果的概率。[38] 有了ENIAC,他意识到计算机的可用性使得这种统计方法变得非常实际。约翰·冯·诺依曼立即看到了这一洞察的重要性。1947年3月,他提出了一种统计方法来解决中子在可裂变物质中的扩散问题。[39] 由于乌拉姆经常提到他的叔叔米哈乌·乌拉姆,“他总是得去蒙特卡洛”去赌博,梅特罗波利斯便将这种统计方法称为“蒙特卡洛方法”。[37] 梅特罗波利斯和乌拉姆于1949年发布了关于蒙特卡洛方法的第一篇非机密论文。[40]

费米得知乌拉姆的突破后,设计了一种模拟计算机,称为蒙特卡洛电车,后来被称为FERMIAC。这种装置对中子的随机扩散进行了机械模拟。随着计算机在速度和可编程性方面的改进,这些方法变得更加实用。特别是,许多在现代大规模并行超级计算机上进行的蒙特卡洛计算是非常适合并行处理的应用,其结果可以非常精确。[25]
\subsubsection{泰勒–乌拉姆设计}  
1949年8月29日,苏联进行了首次裂变炸弹测试,即RDS-1。在拉夫连季·贝利亚的监督下,该炸弹旨在复制美国的努力,这款武器几乎与‘胖子’炸弹相同,因为它的设计基于间谍克劳斯·弗克斯、西奥多·霍尔和大卫·格林格拉斯提供的信息。作为回应,1950年1月31日,哈里·S·杜鲁门总统宣布启动一个紧急计划,开发聚变炸弹。[41]

为了倡导积极的发展计划,厄内斯特·劳伦斯和路易斯·阿尔瓦雷斯来到洛斯阿拉莫斯,与实验室主任诺里斯·布拉德伯里、乔治·伽莫夫、爱德华·泰勒和乌拉姆进行了磋商。很快,泰勒、伽莫夫和乌拉姆成为布拉德伯里任命的一个短命委员会的成员,研究该问题,泰勒担任委员会主席。[9] 当时,关于使用裂变武器引发聚变反应的研究自1942年以来一直在进行,但设计仍然基本沿用了泰勒最初提出的方案。他的概念是将氚和/或氘与裂变炸弹放置在接近的位置,希望炸弹爆炸时释放的热量和强烈的中子通量能够点燃自持的聚变反应。氢的这些同位素的反应之所以受到关注,是因为它们在聚变时释放的单位质量燃料的能量要远大于重核裂变释放的能量。[42]
\begin{figure}[ht]
\centering
\includegraphics[width=6cm]{./figures/5cdc6db6f9970615.png}
\caption{艾薇·迈克(Ivy Mike),泰勒–乌拉姆设计的首次完整测试(分级聚变炸弹),于1952年11月1日爆炸,威力为1040万吨当量。} \label{fig_Ulam_5}
\end{figure}
由于基于泰勒概念的计算结果令人沮丧,许多科学家认为这无法导致成功的武器,而另一些人则出于道德和经济原因反对继续进行。因此,曼哈顿计划的几位高级人物反对发展这一方案,包括贝特和奥本海默。[43] 为了澄清情况,乌拉姆和冯·诺依曼决定进行新的计算,确定泰勒的方法是否可行。为了进行这些研究,冯·诺依曼决定使用电子计算机:位于阿伯丁的ENIAC,普林斯顿的新计算机MANIAC,以及在洛斯阿拉莫斯建设中的它的双胞胎。乌拉姆邀请埃弗雷特采用一种完全不同的方法,这种方法以物理直觉为指导。弗朗索瓦丝·乌拉姆是[44] 一群“计算员”之一,她们在机械计算器上进行艰巨而广泛的热核场景计算,并通过埃弗雷特的计算尺进行补充和确认。乌拉姆和费米进一步合作分析这些场景。结果表明,在可行的配置中,热核反应不会点燃,如果点燃,也不会是自持的。乌拉姆利用他在组合数学方面的专长,分析了氘的链式反应,这比铀和钚的反应复杂得多,他得出结论,考虑到泰勒所考虑的(低)密度,不会发生自持链式反应。[45] 1950年底,这些结论得到了冯·诺依曼的结果确认。[35][46]

1951年1月,乌拉姆提出了另一个想法:利用核爆炸的机械冲击来压缩聚变燃料。在妻子的建议下,[35] 乌拉姆在告诉泰勒之前,先与布拉德伯里和马克讨论了这一想法。[47] 几乎立刻,泰勒看到了这个想法的优点,但指出裂变炸弹发出的软X射线会比机械冲击更强烈地压缩热核燃料,并建议增强这种效果的方法。1951年3月9日,泰勒和乌拉姆提交了一份联合报告,描述了这些创新。[48] 几周后,泰勒建议将一个裂变杆或圆柱体放置在聚变燃料的中心。这个“火花塞”的引爆[49] 将有助于启动并增强聚变反应。基于这些想法的设计,被称为分级辐射内爆,已成为制造热核武器的标准方法。它通常被称为“泰勒–乌拉姆设计”。[50]
\begin{figure}[ht]
\centering
\includegraphics[width=6cm]{./figures/5737eeb6e26d43c3.png}
\caption{‘香肠’装置是迈克核试验中的一部分(威力为1040万吨),该试验在恩纽埃塔环礁进行。该试验是‘常春藤行动’的一部分。‘香肠’是第一次真正测试的氢弹,意味着它是基于泰勒–乌拉姆的分级辐射内爆原理建造的第一种热核装置。} \label{fig_Ulam_6}
\end{figure}
1951年9月,在与布拉德伯里和其他科学家发生一系列分歧后,泰勒辞去了洛斯阿拉莫斯的职务,返回芝加哥大学。[51] 大约在同一时间,乌拉姆请假作为哈佛大学的访问教授,讲授一个学期。[52] 尽管泰勒和乌拉姆共同提交了他们的设计报告[48],并共同申请了专利[22],他们很快就卷入了一场关于谁应获得功劳的争论。[47] 战后,贝特返回康奈尔大学,但他作为顾问深度参与了热核武器的发展。1954年,他写了一篇关于氢弹历史的文章[53],在其中他表达了自己的观点,认为两人都为这一突破做出了非常重要的贡献。其他参与其中的人,包括马克和费米,也持有这种平衡的观点,但泰勒坚持试图淡化乌拉姆的作用。[54] “氢弹制造出来后,”贝特回忆道,“记者开始称泰勒为氢弹之父。为了历史的准确性,我认为更精确的说法是乌拉姆是父亲,因为他提供了种子,而泰勒是母亲,因为他与孩子始终在一起。至于我,我想我是接生婆。”[55]

在基本的聚变反应得到确认,并且有了可行的设计后,洛斯阿拉莫斯测试热核装置再没有任何障碍。1952年11月1日,首次热核爆炸发生,当时‘艾薇·迈克’在恩纽埃塔环礁的美国太平洋试验场引爆。这个装置使用液态氘作为聚变燃料,体积庞大,完全无法作为武器使用。然而,它的成功验证了泰勒–乌拉姆设计,并刺激了实用武器的密集开发。[52]

