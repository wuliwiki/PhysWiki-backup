% 线性方程组的仿射解释
% 仿射线性函数|线性方程组|几何解释

\pentry{仿射空间\upref{AfSp},线性方程组\upref{LinEqu}}
在解释线性方程组的仿射几何意义前,先总结一下矩阵和矢量观点下的意义.

一般的线性方程组具有如下形式:
\begin{equation}\label{AS2LF_eq1}
\leftgroup{
&a_{11}x_1 + a_{12}x_2 + \dots + a_{1n}x_n\;\;\;=y_1\\
&a_{21}x_1 + a_{22}x_2 + \dots + a_{2n}x_n\;\;\;=y_2\\
&\qquad \qquad \dots  \qquad \dots \qquad  \dots\\
&a_{m1}x_1 + a_{m2}x_2 + \dots + a_{mn}x_n=y_m}\\
\end{equation}
写成\textbf{矩阵形式}则为
\begin{equation}\label{AS2LF_eq2}
\mat A \bvec x=\bvec y
\end{equation}
其中
\begin{equation}
\mat A=\begin{pmatrix}
a_{11}&a_{12}&\cdots&a_{1n}\\
a_{21}&a_{22}&\cdots&a_{2n}\\
\vdots&\vdots&\vdots&\vdots\\
a_{m1}&a_{m2}&\cdots&a_{mn}
\end{pmatrix}
,\quad \bvec x=\begin{pmatrix}
x_1\\x_2\\\vdots\\x_n
\end{pmatrix}
,\quad \bvec y=\begin{pmatrix}
y_1\\y_2\\\vdots\\y_n
\end{pmatrix}
\end{equation}
在\textbf{矢量观点}下,\autoref{AS2LF_eq1} 又可写为
\begin{equation}\label{AS2LF_eq3}
\sum_{i=1}^n x_ia_i=y
\end{equation}
其中
\begin{equation}
a_i=(a_{1i},a_{2i},\cdots,a_{mi})^T,\quad y=(y_1,y_2,\cdots,y_m)^T
\end{equation}

在\textbf{矩阵观点}下,线性方程组\autoref{AS2LF_eq2} 有解,当且仅当 $\mathrm{rank}\;\mat A=\mathrm{rank}\;(\mat A,\bvec y)$ .这里, $(\mat A,\bvec y)$ 是在矩阵 $\mat A$ 后再添上一列 $\bvec y$ 构成的矩阵,即 $\mat A$ 的\textbf{增广矩阵}.$\mathrm{rank}\; \mat A$ 表 $\mat A$ 的秩.在有解的情况下,线性方程组的解构成的集合为 $X_s= X_0+x_1 $ ,其中 $X_0$ 为对应齐次方程组的通解,$x_1$ 是任一特解(\autoref{LinEq2_the1}~\upref{LinEq2}).

在\textbf{矢量观点}下,线性方程组\autoref{AS2LF_eq3} 有解,当且仅当矢量 $y$ 可被矢量组 $\{a_1,\cdots,a_n\}$ 线性表示;也可理解为 $y$ 在 $a_1,\cdots,a_n$ 生成的张成空间\upref{VecSpn}当中,即$y\in\langle a_1,\cdots,a_n\rangle$ .

在以上解释中,并没有把线性方程组与通常观点下的几何对象(点线面)联系起来.当然,在矩阵理论和矢量空间里,还没有引入点的概念,要将线性方程组与几何对象联系起来,需要在仿射空间当中去完成.在此之前先说明几个必要的概念.
\subsection{仿射线性函数}
在仿射线性映射定义(\autoref{AfSp_def2}~\upref{AfSp})中,若 $\mathbb A'=\mathbb F$, $Df$ 是 $V$ 上的线性函数,则仿射线性映射称之为\textbf{仿射线性函数}.
\begin{definition}{仿射线性函数}
设 $(\mathbb A,V)$ 是域 $\mathbb {F}$ 上的仿射空间,称 $f:\mathbb A\rightarrow \mathbb F$ 是个\textbf{仿射线性函数},如果
\begin{equation}
f(\dot p+v)=f(\dot p)+Df\cdot v\quad\forall \dot p\in\mathbb A,v\in V
\end{equation}
其中 $Df$ 是 $V$ 上的一个线性函数,称为 $f$ 的\textbf{线性部分}(或\textbf{微商}).
\end{definition}
如此一来,线性部分为零的仿射线性函数对应常量.
\begin{theorem}{}
 $f:\mathbb A\rightarrow \mathbb F$ 是仿射线性函数,当且仅当在坐标系 $\{\dot o;e_1,\cdots,e_n\}$ 下,对 $\forall \dot p\in \mathbb A$,有
 \begin{equation}\label{AS2LF_eq4}
 f(\dot p)=\sum_{i=1}^n\alpha_ix_i+\alpha_0
 \end{equation}
 其中 $\alpha_0=f(\dot o),\alpha_i=Df\cdot e_i$ ,$x_1,\cdots,x_n$ 是点 $\dot p$ 坐标.
\end{theorem}
\textbf{证明:}1. $\Rightarrow$

\begin{equation}
f(\dot p)=f(\dot o+\vec{op})=f(\dot o)+Df\cdot \vec{op}=\sum_{i=1}^n \alpha_i x_i+\alpha_0
\end{equation}

2. $\Leftarrow$

设 $v=\sum\limits_{i=1}^n v_ie_i$ ,于是由\autoref{AfSp_the2}~\upref{AfSp} 第2条,$\dot p+v$ 的坐标为 $x_i+v_i,i=1,\cdots ,n$.由\autoref{AS2LF_eq4} ,有
\begin{equation}
\begin{aligned}
f(\dot p+v)&=\sum_{i=1}^n\alpha_i(x_i+v_i)+\alpha_0=\qty(\sum_{i=1}^n\alpha_i x_i+\alpha_0)+\sum_{i=1}^n\alpha_i v_i\\
&=f(\dot p)+Df\cdot v
\end{aligned}
\end{equation}

\textbf{证毕!}
