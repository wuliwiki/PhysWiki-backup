% 分式域
% fraction field|环

\addTODO{等待加入目录}

\pentry{环和域\upref{field}}

\footnote{本小节节选自《小时百科教材系列》的《代数学》.}


整环$R$可以添加元素从而成为一个域,这个域就叫做$R$的\textbf{分式域(field of fractions)}.我们不妨用一个熟悉的例子来展示怎么添加新的元素:

\begin{example}{}
考虑整数环$\mathbb{Z}$,绝大多数元素都是没有乘法逆元的,除了$\pm 1$.对于$2\in\mathbb{Z}$,它在环上没有乘法逆元,那么我们就引入一个新的元素$1/2$,使得$2\cdot 1/2=1$.

如果用$1/2$去乘以别的元素,考虑到$\mathbb{Z}$是整环,可以进行乘法消去律,那么我们可以从别的元素中抽取出$2$的乘积,来抵消掉$1/2$,比如说,$4\cdot 1/2=(2\cdot 2)\cdot 1/2=2\cdot(2\cdot 1/2)=2\cdot 1=2$.但是$3\cdot 1/2$并不存在于$\mathbb{Z}\cup\{1/2\}$中,所以也需要把它作为一个新元素加入进去,记为$3/2$,并且满足$2\cdot 3/2=3$.同时,$1/2$之间也可以彼此互相乘积,得到新的元素,如$1/4$、$1/8$等.也就是说,用$1/2$不停地去乘以已有的任何元素,如果得到新元素就添加进我们的集合中,再进行这样的运算.

除了乘法可以引出新的元素,加法也可以.$1/2+1$如何计算呢?记住我们是用$2\cdot 1/2$来定义$1/2$的,所以此处可以巧用该定义来推导:$2\cdot(1/2+1)=1+2=3$,因此$1/2+1=3/2$.

以上累赘的概括新元素的方法,也可以更紧凑地表达为:对新元素$1/2$,其关于加法和乘法扩充后的集合为$\{r\cdot (1/2)^n|r\in\mathbb{Z}, n\in \mathbb{Z}\}$.

对每一个元素都这样操作,添加它们的逆元,以及每一次添加的逆元和所有已有元素运算的结果,得到的集合就是一个域.对于$\mathbb{Z}$环的情况,这个域就是有理数域$\mathbb{Q}$.
\end{example}

整环添加新元素来构成域的核心思想,就是不停地用新元素和已有元素进行乘法、加法运算,如果得到新的元素就囊括进去,这是为了保证域的“乘法封闭性”.

我们把上述累赘的例子抽象成以下定义,可能比例子要逻辑更清晰一些:

\begin{definition}{}
给定一个带有运算的集合$S$,它有一个子集$T$,那么记$<T>$是$T$中所有元素彼此任意进行运算的结果的集合,称为由子集$T$通过给定运算生成的子集.
\end{definition}

\begin{definition}{}
给定整环$R$,记$R$中没有乘法逆元的元素构成的集合为$S$,对任意$s\in S$,定义新元素$s^{-1}$,其运算规则由$s\cdot s^{-1}=e$决定\footnote{为了理解为什么这条式子可以决定这个新元素的运算规则,参考例子,思考一下如何从这条式子得出$s^{-1}$与已有元素的\textbf{加法}.}.把所有这样的元素$s^{-1}$构成的集合记为$S'$,那么$R\cup S'$通过加法和乘法生成的集合$<R\cup S'>$,就是$R$的\textbf{分式域(fraction field)}\footnote{这里通过加法和乘法生成的集合一定包含$R$,通常不是$R$的子集.这和上一条定义略有不同,但是我想并不会造成理解困难.}.
\end{definition}
















