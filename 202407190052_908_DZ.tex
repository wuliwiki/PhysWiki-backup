% 电子
% license CCBYSA3
% type Wiki

(本文根据 CC-BY-SA 协议转载自原搜狗科学百科对英文维基百科的翻译)

\textbf{电子}是一种亚原子粒子,符号是$e^-$或者$\beta^-$,其电荷量是负一个单位的基本电荷。[1]电子属于第一代轻子族,且通常被认为是基本粒子,因为它们没有已知的成分或亚结构。电子的质量大约是质子质量的1/1836。 电子的量子力学性质包括半整数值的内禀角动量(自旋),单位为约化普朗克常数ħ。根据泡利不相容原理,作为费米子,没有两个电子可以占据相同的量子态。像所有基本粒子一样,电子表现出波粒二象性:它们可以与其他粒子发生碰撞,也可以像光一样发生衍射。比其他粒子(例如中子和质子)更容易通过实验观察到,因为电子的质量更低,对于给定的能量电子拥有更大的德布罗意波长。

电子在许多物理现象中起着重要作用,如电学、磁学、化学和热导率,电子还参与重力、电磁相互作用和弱相互作用。[2]因为电子带有电荷,所以它周围有一个电场,如果电子相对于某个观察者发生运动,这个观察者将观察到它产生一个磁场。其他来源所产生的电磁场将影响电子的运动,这种影响由洛伦兹力定律所描述。当电子被加速时,它们以光子的形式辐射或吸收能量。实验室仪器能够通过使用电磁场囚禁单个电子和电子等离子体。特殊的望远镜可以探测外层空间的电子等离子体。电子涉及许多应用,例如电子学、焊接、阴极射线管、电子显微镜、放射疗法、激光器、气体电离探测器和粒子加速器。

电子与其他亚原子粒子的相互作用在化学和核物理等领域都很有意义。原子核内带正电荷的质子和没有原子核的负电荷电子之间的库仑力相互作用允许它们共同组成原子。电离或负电荷电子与正电荷原子核之间的电荷量差异改变了原子系统的结合能。两个或多个原子之间的电子交换或共享是化学键形成的主要原因。1838年,英国自然哲学家理查德·拉明首先假设了一个不可分割的电荷量的概念来解释原子的化学性质。爱尔兰物理学家乔治·约翰斯顿·斯通尼在1891年将这种电荷命名为“电子”,而约瑟夫·汤姆孙和他的英国物理学家团队在1897年将它确定为粒子。电子也可以参与核反应,例如恒星中的核合成,在其中电子被称为贝塔粒子。电子可以通过放射性同位素的β衰变以及高能碰撞产生,后者的一个例子发生在宇宙射线进入大气层的时候。电子的反粒子被称为正电子;它与电子拥有很多相同的性质,除了它携带有与电子相反的电荷以及其他的荷。当一个电子与正电子碰撞时,两个粒子可以发生湮灭,并产生$\gamma$射线光子。

\subsection{历史}
\subsubsection{1.1 电力效应的发现}
古希腊人注意到琥珀在摩擦皮毛后会吸引小物体。与闪电一样,这种现象是人类最早记录的电学经验之一。[3]在1600年的论文《De Magnete》中,英国科学家威廉·吉尔伯特创造了新拉丁语术语electrica来指代那些性质类似琥珀、摩擦后会吸引小物体的物质。[4]electric和electricity这两个词都源自拉丁语ēlectrum(这也是同名合金的名称由来),来源于希腊语中的“琥珀”ἤλεκτρον(ēlektron)。
\subsubsection{1.2 两种电荷的发现}
17世纪初,法国化学家Charles Franç ois du Fay发现如果带电的金箔被用丝绸摩擦过的玻璃所排斥,那么同样的带电金箔就会被用毛皮摩擦过的琥珀所吸引。从这个和其他类似实验的结果中,他得出结论:电由两种电流体所组成,分别是来自被丝绸摩擦过的玻璃的玻璃液体以及来自被毛皮摩擦过的琥珀的树脂液体。这两种流体混合时可以互相中和。[4][5]美国科学家埃比尼泽·金纳斯利后来也独立得出了同样的结论。[6]十年后本杰明·富兰克林提出电不是来自不同类型的电流体,而是表现出过量(+)或不足(-)的单一电流体。他将这两种情况分别命名为正和负,这也是现代的电荷命名规则。[7]富兰克林认为电荷载体是正的,但他没有正确地识别哪种情况是电荷载体的盈余、哪种情况是不足。[8]

1838年至1851年间,英国自然哲学家理查德·拉明发展了原子由被亚原子粒子包围的物质核心所构成的观点,这些包围核心的亚原子粒子具有单位数量的电荷。[9]从1846年开始,德国物理学家威廉·韦伯建立了电流由带正电荷和负电荷的流体组成的理论,它们的相互作用受平方反比定律支配。在1874年研究了电解现象后,爱尔兰物理学家乔治·约翰斯顿·斯通尼提出存在“单一确定量的电”,即单价离子的电荷。借助于法拉第电解定律,他能够估算出这个基本电荷的数值e。[10]然而,斯通尼认为这些电荷永久地附着在原子上并且无法移除。1881年,德国物理学家赫尔曼·赫尔姆霍茨认为正电荷和负电荷都被分成基本部分,每个都“表现得像电的原子”。[11]

斯通尼在1881年创造了术语electrolion。十年后,他转而用electron来描述这些基本电荷,他在1894年写道:“...有人估计了这个最引人注目的基本电学单位的实际量,后来我大胆地提出了这个名字电子(electron)”。1906年有人提议改用electrion一词,但这个提议没有成功,这是因为亨德里克·洛伦兹更愿意继续使用electron这个词。[11][12]electron这个词是单词electric和ion的组合。[13]来自electron的后缀-on现在被用来为其他亚原子粒子命名,如质子或中子。[14][15]
\subsubsection{1.3 物质外自由电子的发现}
约瑟夫·汤姆孙对电子的发现与许多物理学家几十年来对阴极射线的实验和理论研究密切相关。[11]在1859年研究稀薄气体的电导率时,德国物理学家朱利叶斯·普吕克尔观察到由阴极发射的辐射引起的磷光出现在阴极附近的管壁上,并且磷光区域可以通过施加磁场来移动。1869年,普吕克的学生希托夫发现置于阴极和磷光体之间的固体会在管的磷光区域上投下阴影。希托夫推断阴极发出笔直的射线,而磷光是由撞击管壁的射线所引起的。1876年,德国物理学家尤金·戈德斯坦证明这些射线被垂直于阴极表面发射出来,这区分了从阴极发射的射线和白炽光。戈德斯坦称这种射线为阴极射线。[16] [17]

19世纪70年代,英国化学家和物理学家威廉·克鲁克斯爵士开发了第一个内部具有高度真空的阴极射线管。[18]接着,他在1874年证明阴极射线可以转动放在它们的路径上的一个小桨轮。因此,他得出结论:这种射线带有动量。此外,通过施加磁场,他能够偏转这种射线,从而证明这种射线表现得好像带了负电荷。[17]1879年,他提出这些性质可以通过将阴极射线视为由处于第四种物质状态的带负电的气态分子组成来解释,在这里粒子的平均自由程太长以致于它们之间的碰撞可以被忽略。[16]

德国出生的英国物理学家阿瑟·舒斯特扩展了克鲁克斯的实验,他将金属板相对阴极射线平行放置,并在金属板之间施加电势。电场将射线偏转向带正电荷的板上,这进一步证明了这种射线带有负电荷。通过测量给定水平的电流的偏转量,舒斯特在1890年就能够估计出射线成分质荷比。然而,这产生了一个比预期大一千多倍的数值,所以当时他的计算很少得到认可。[17]

1892年,亨德里克·洛伦兹提出这些粒子(电子)的质量可能是它们电荷的结果。[19]

法国物理学家昂利·贝可勒尔在1896年研究自然发荧光矿物时发现,它们在没有暴露于外部能源的情况下就会发出辐射。这些放射性材料成为科学家们非常感兴趣的主题,这其中就包括后来发现它们所发射粒子的本质的新西兰物理学家欧内斯特·卢瑟福。他根据这些粒子穿透物质的能力将它们命名为α粒子和β粒子。[20]1900年,贝克勒尔证明了镭发出的β射线可以被电场偏转,并且它们的质荷比与阴极射线相同。[21]这一证据支持了电子作为原子成分存在的观点。[22][23]
\begin{figure}[ht]
\centering
\includegraphics[width=6cm]{./figures/1dbcc8852c33c51a.png}
\caption{约瑟夫·汤姆孙} \label{fig_DZ_1}
\end{figure}
1897年,英国物理学家约瑟夫·汤姆孙和他的同事约翰·汤森和H. A .威尔逊进行的实验表明阴极射线确实是一种独特的粒子,而不是以前认为的波、原子或分子。[24]汤姆森对这种粒子的电荷e和质量m都做了很好的估计,他发现阴极射线粒子(他称之为“微粒”)的质量可能是已知质量最小的离子(氢离子)的千分之一。[24]他证明了这种粒子的荷质比e/m不取决于阴极材料。他进一步证明,放射性材料、被加热材料和照明材料所产生的负电荷粒子是相同的。[24][24]科学界将这些粒子命名为电子,这主要是因为G. F. 菲茨杰拉德、J. 拉莫和H. A. 洛伦兹的倡导。[25]
\begin{figure}[ht]
\centering
\includegraphics[width=6cm]{./figures/9a283a42b0af6b49.png}
\caption{罗伯特·密立根} \label{fig_DZ_2}
\end{figure}
美国物理学家罗伯特·密立根和哈维·弗莱彻在他们1909年的油滴实验中更仔细地测量了电子的电荷,其结果发表于1911年。在这个实验中,他们使用电场来阻止带电的油滴因重力而下落。该装置可以测量少至1-150个离子的电荷,误差小于0.3\%。汤姆森的团队早些时候也做过类似的实验,[24]他们利用了电解产生的带电水滴云。1911年亚伯兰·约菲使用带电金属微粒独立地获得了与密立根相同的结果,并在1913年发表了他的结果。[26]然而,油滴比水滴更稳定,因为它们的蒸发速度更慢,所以更适合长时间的精确实验。[27]

大约在二十世纪初,人们发现,在某些条件下,快速移动的带电粒子会导致过饱和水蒸气沿其路径发生凝结。1911年,查理·威尔逊利用这一原理设计了他的威尔逊云雾室,这样他就可以拍摄带电粒子的轨迹,例如快速移动的电子。[28]
\subsubsection{1.4 原子理论}
\begin{figure}[ht]
\centering
\includegraphics[width=8cm]{./figures/d736d3f77e6f2f50.png}
\caption{玻尔原子模型,显示了能量由n所量子化的电子状态。下降到较低轨道的电子发射的光子的能量等于轨道之间的能量差。} \label{fig_DZ_3}
\end{figure}
到1914年,物理学家欧内斯特·卢瑟福、亨利·莫塞莱、詹姆斯·弗兰克和古斯塔夫·赫兹的实验已经在很大程度上确立了原子的结构,即被低质量电子包围的带正电荷的致密核。[29]1913年,丹麦物理学家尼尔斯·玻尔假设电子处于量子化的能量状态,其能量由电子绕核轨道的角动量所决定。通过发射或吸收特定频率的光子,电子可以在这些状态或轨道之间移动。通过这些量子化的轨道,他准确地解释了氢原子的谱线。[30]然而,玻尔模型未能解释谱线的相对强度,也未能解释更复杂的原子的光谱。[29]

吉尔伯特·牛顿·路易斯解释了原子之间的化学键,他在1916年提出两个原子之间的共价键是由它们之间共享的一对电子所维持的。[31]后来,在1927年,沃尔特·海特勒和弗里茨·伦敦为电子对形成和化学键给出了完整的量子力学解释。[32]1919年,美国化学家欧文·朗缪尔阐述了路易斯的原子静态模型,并提出所有电子都分布在连续的“(接近)同心的厚度相等的球形壳层中”。[33]接着,他将壳分成许多单元,每个单元包含一对电子。利用这个模型,朗缪尔能够定性地解释元素周期表中所有元素的化学性质,[32]根据周期定律,它们在很大程度上重复自身。[34]

1924年,奥地利物理学家沃尔夫冈·泡利观察到,原子的壳状结构可以用一组定义每个量子能量状态的四个参数来解释,只要每个状态被不超过一个电子占据。这种禁止一个以上的电子占据相同量子能量态的禁令被称为泡利不相容原理。[35]荷兰物理学家塞缪尔·古德施密特和乔治·乌伦贝克提供了解释第四个参数的物理机制,该参数有两个不同的可能值。1925年,他们提出电子除了轨道角动量之外,还拥有一个内禀角动量和磁偶极矩。[29][36]这类似于地球绕着太阳转动的同时也绕着自身的轴旋转。内禀角动量被称为自旋,它解释了以前用高分辨率光谱仪观察到的神秘的谱线分裂,这种现象被称为精细结构分裂。[37]
\subsubsection{1.5 量子力学}
\begin{figure}[ht]
\centering
\includegraphics[width=6cm]{./figures/7b9fbb20551d0820.png}
\caption{在量子力学中,原子中电子的行为由轨道所描述,轨道是概率分布而不是轨迹。在图中,阴影表示在该点“找到”具有对应于给定量子数的能量的电子的相对概率。} \label{fig_DZ_4}
\end{figure}
在1924年的论文《Recherches sur la théorie des quanta》(量子理论研究)中,法国物理学家路易·德布罗意假设所有物质都可以用类似光的方式被表示为德布罗意波。也就是说,在适当的条件下,电子和其他物质将显示出粒子或波的性质。粒子的微粒性质表现在任何给定时刻沿着其轨道它都在空间中具有一个局域的位置。[38]光的波动性质则表现在穿过平行狭缝进而产生干涉图案的时候。1927年,乔治·佩杰特·汤姆孙发现当电子束穿过薄金属箔时存在干涉效应。美国物理学家克林顿·戴维孙和雷斯特·革末在电子从镍晶体反射的过程中也观察到了干涉效应。[39]

德布罗意对电子的波动性质的预测导致埃尔温·薛定谔假设了电子在原子核影响下运动的波动方程。1926年,这个被称为薛定谔方程的方程成功地描述了电子波是如何传播的。[40]这个波动方程不是产生一个确定电子随时间变化的位置的解,而是可以用来预测在某个位置附近找到电子的概率,特别是在电子被束缚在空间中的位置附近,在这种情况下电子的波动方程不随时间变化。这种方法导致了量子力学的第二种表述(海森堡在1925年提出了第一种),薛定谔方程的解像海森堡方程一样提供了对氢原子中的电子能量状态的推导,这些状态跟玻尔在1913年首次推导得到的结果是一样的,并且已知可以再现氢光谱。[41]一旦自旋和多个电子间相互作用被描述出来,量子力学就可以预测原子序数大于氢的原子中的电子构型。[42]

1928年,在沃尔夫冈·泡利的工作的基础上,保罗·狄拉克通过对电磁场的量子力学的哈密顿表述应用相对论和对称性考虑而创建了一个电子模型,这就是遵循相对论的狄拉克方程。[43]为了解决他的相对论方程中的一些问题,狄拉克在1930年建立了一个真空模型,它是一个负能量粒子的无限海,后来被称为狄拉克海。这使他预测了正电子的存在,后者是与电子相对应的反物质。[44]这种粒子是由卡尔·安德森于1932年发现的,安德森提议将普通电子命名为negatons,而使用electron作为描述带正电荷和负电荷的变体的通用术语。

1947年威利斯·兰姆与研究生罗伯特·雷瑟福德合作发现氢原子的某些本应具有相同能量的量子态相对于彼此发生了位移,这种差异被称为兰姆位移。大约在同一时间,波利卡普·库施与亨利·福利一起发现电子的磁矩略大于狄拉克理论的预测值。这个微小的差异后来被称为电子的反常磁矩。这种差异后来被朝永振一郎、朱利安·施温格和理查德·费曼于20世纪40年代末所发展的量子电动力学所解释。[45]
\subsubsection{1.6 粒子加速器}
随着20世纪上半叶粒子加速器的发展,物理学家开始更深入地研究亚原子粒子的性质。[46]1942年,唐纳德·克尔斯特首次成功地尝试使用电磁感应加速电子。他最初的β加速器的能量达到2.3MeV,而随后的β加速器则达到300MeV。在1947年,同步辐射是用通用电气的一台70MeV的电子同步加速器被发现的。这种辐射是由电子以接近光速移动时穿过磁场的加速度引起的。[47]

第一台高能粒子对撞机是于1968年开始运行的ADONE,粒子束能量为1.5GeV。[48]这种设备在相反的方向上加速了电子和正电子,与用电子撞击静态目标相比,它有效地将粒子碰撞的能量增加了一倍。[49]欧洲核子研究中心(CERN)于1989年至2000年运行的大型电子正子对撞机(LEP)的碰撞能量达到209GeV,物理学家利用它对粒子物理标准模型进行了重要测量。[50][51]
\subsubsection{1.7 囚禁单个电子}
单个电子现在可以很容易地被限制在超小($L = 20 nm,W = 20 nm$)的CMOS晶体管中,后者运行于-269摄氏度(4K)到大约258摄氏度(15K)的低温下。[52]电子波函数分散于半导体晶格中,与价带电子的相互作用很小,因此可以通过将质量替换成有效质量张量而作为单粒子来处理。

\subsection{特征}
\subsubsection{2.1 分类}
\begin{figure}[ht]
\centering
\includegraphics[width=10cm]{./figures/b5c187a00339db95.png}
\caption{基本粒子的标准模型。电子(符号e)在左边。} \label{fig_DZ_5}
\end{figure}
在粒子物理学的标准模型中,电子属于被称为轻子的亚原子粒子,它们被认为是基本粒子。在所有带电轻子(或任何类型的带电粒子)中,电子的质量是最低的,属于第一代基本粒子。[53]第二代和第三代包含带电轻子($\mu$子和$\tau$子),它们与电子拥有相同的电荷、自旋和相互作用,但质量更大。轻子不同于物质的其他基本成分(即夸克),因为它们不参与强相互作用。轻子群的所有成员都是费米子,因为它们都有半奇整数的自旋。电子的自旋为12。[54]
\subsubsection{2.2 基本性质}
电子的不变质量近似为$9.109\times10^{-31}$ 千克,[55]或者$5.489\times10^-4$原子质量单位。根据爱因斯坦的质能公式,这个质量对应于0.511 MeV的静止能量。质子和电子的质量之比约为1836。[56][56]天文测量表明,正如标准模型预测的那样,质子电子质量比至少在宇宙年龄一半的时间内保持相同的值。[57]

电子的电荷量为$-1.602\times10^{-19}$ 库仑,[55]这被用作亚原子粒子的标准电荷单位,也称为基本电荷。这个基本电荷的相对标准不确定度为$2.2\times10^{-8}$。[55]在实验精度的限制内,电子的电荷等于质子的电荷,但符号相反。[58]由于e被用于表示基本电荷,电子通常被写成$e^-$,其中的负号代表负电荷。正电子的符号是$e^+$,因为它具有与电子相同的性质,但带有正电荷而不是负电荷。[54][55]

电子具有内禀角动量或自旋,大小为12。[55]由于这一性质,我们通常将电子称为一种自旋-12粒子。[54]这样的粒子的自旋大小是√32$\hbar$。而对任何轴上的自旋投影的测量结果只能是$\pm\hbar2$。除了自旋,电子还拥有沿着其旋转轴的内禀磁矩。[55]它大约等于一个玻尔磁子,[59]后者是一个等于$9.27400915(23)\times10^{-24}$ joules/ tesla的物理学常数。[55]自旋相对于电子动量的方向定义了一个称为螺旋度的基本粒子性质。[60]

电子没有已知的亚结构[61][61],并且它被认为是没有空间范围的带有点电荷的点粒子。[62]

电子半径问题是现代理论物理中一个具有挑战性的问题。承认电子半径有限的假设不符合相对论的前提。另一方面,点状电子(半径为零)使得电子的自能趋于无穷大,这会导致严重的数学困难。[62]对彭宁离子阱中一个单独电子的观察表明电子的半径的上限为10-22米。[63]电子半径的上限10-18米[64]可以由能量的不确定关系导出。还存在一个叫“经典电子半径”的物理学常数,它的值为2.8179×10−15 m,大于质子的半径。这个术语来源于一个忽略了量子力学效应的简单计算;在现实中,所谓的经典电子半径与电子真正的基本结构没有什么关系。[65]

有一些基本粒子可以自发地衰变成质量较小的粒子。其中一个例子是μ子,它们的平均寿命为2.2×10−6秒,可以衰变为电子、μ子中微子和电子反中微子。另一方面,从理论上来说,电子被认为是稳定的:电子是带有非零电荷的质量最小的粒子,因此它的衰变将违反电荷守恒定律。[66]电子平均寿命的实验下限是6.6×1028年,置信度为90%。[67][68][69]