% 完备空间

\pentry{度量空间\upref{Metric},极限\upref{Lim}}

\subsection{柯西列}

度量空间是定义了“距离”的空间,因此可以讨论点与点之间的远近.微积分的基础是\textbf{极限}的概念;直观来说,如果一个数列$\{a_n\}$在$n$趋于正无穷时有极限$a_0$,意思就是说$a_n$随着$n$的增大,越来越靠近$a_0$.因此,表达式$\lim\limits_{n\rightarrow\infty}a_n=a_0$可以表述为“不管我们要求的接近程度$\epsilon$有多小(同时大于零),总能找到足够大的正整数$N$,使得只要$n>N$,那么$a_n$到$a_0$的距离就小于$\epsilon$”;或者用数学符号,更紧凑一点,写为:$\forall\epsilon>0$,$\exists N\in\mathbb{Z}^+$,使得$\forall n>N$,有$|a_n-a_0|<\epsilon$.

什么情况下,一个数列$\{a_n\}$存在这样的极限$a_0$呢?这样的数列被称作柯西数列,定义如下.

\begin{definition}{柯西数列}

在$\mathbb{R}$中,一个数列$\{a_n\}$被称为\textbf{柯西数列},当且仅当对于任意的\textbf{正}距离$\epsilon$,存在$N\in\mathbb{Z}$,使得只要$n, m>N$,就有:$|a_n-a_m|<\epsilon$.

\end{definition}

这样的数列就是可以有极限的数列.

\begin{exercise}{柯西数列的收敛性}

证明:柯西数列必有极限;有极限的数列必是柯西数列.

\end{exercise}

以上定义中,我们只讨论了“距离”这一性质,因此这些定义都可以推广到任意有距离概念的集合中,也就是度量空间中.

\begin{definition}{柯西列}

给定度量空间$\mathcal{M}$,记两点$x$,$y\in\mathcal{M}$之间的度量为$d(x, y)$称点列$\{a_n\}_{n=1}^{\infty}\subseteq\mathcal{M}$为一个\textbf{柯西点列},当且仅当$\forall\epsilon>0$,$\exists N\in\mathbb{Z}^+$,使得$\forall n, m>N$,有$d(a_n, a_m)<\epsilon$.

\end{definition}


柯西点列、柯西数列也可以简称/统称为柯西列.