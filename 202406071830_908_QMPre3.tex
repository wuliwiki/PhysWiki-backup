% 投影和表示 (转载)
% license Usr
% type Art


\subsection{太阳比喻}

在晴朗的日子里出去走一走,我们看不见自己,但却能看见自己的影子。这是很有意思的事情。

如果太阳不是很晒的话,我们可以站在一个空旷平坦的地上观察我们的影子,它和阳光射来的方向相对,在地上留下一个阴影,如果时间早的话,太阳升的不是很“高”,光线会斜斜地在地上投下一个较长的阴影,随着时间的流逝,太阳会沿着自己的轨道在天空中划出一个圆弧,随着太阳的升“高”,阴影会越来越短,当太阳升到最“高”的时候,阴影也最短。

但说高并不精确,我们可以把眼睛眯起,朝太阳的方向看,所谓“高”就是我们要仰起脖子才能“追踪”到太阳,我们仰起脖子的角度越大、太阳越高,我们可以把这个仰角定义为“太阳-观察者”连接线与地面的夹角$\theta$。当这个角度为$90^o$的时候,太阳在天顶,光线垂直地射下来,此时我们在地上的影子会“消失”\footnote{阴影之内没有光线是暗的,而阴影之外会被阳光照亮,光在这里更多地体现出“粒子性”,它以直线传播,绝对不会绕过障碍物。光从$\theta$方向照射到物体上,在地面上留下一个影子,假设物体的高度是$H$,影子的长度将是$H \cdot \frac{\cos \theta}{\sin \theta } = H \cdot \cot \theta$。}。

\begin{figure}[ht]
\centering
\includegraphics[width=6cm]{./figures/2f63172810481ba4.png}
\caption{太阳光⼊射,与竖直⽅向成 α 角。} \label{fig_QMPre3_4}
\end{figure}

有时我们也以竖直的方向为基准,定义太阳光与竖直方向的夹角为$\alpha$($\alpha = \frac{\pi}{2} - \theta $),当$\alpha = 0$时,阳光笔直地照射在地面上,这时照射到单位面积上太阳光的能量最大,当角度$\alpha$逐渐增大时,照射到单位面积上太阳光的能量会变小,变小的比例正比于$\cos \alpha$。
人类走出非洲后,一路向北,先来到中近东,然后扩散到欧洲、亚洲等其它地方。中近东、欧洲、亚洲比非洲的纬度高,太阳会以一个更大的角度$\alpha$照射下来,随着$\alpha$的增大,单位表面积上地球吸收到的能量会减少,气温会随之降低,尤其是夜晚温度会更低。

我们现在都是住在屋子里的,但在远古人类甚至连制造房屋的技术都没有发明,冷了只能去山洞。但山洞里已经有其他动物占领了,比如曾广泛分布于欧洲和中近东各地的洞熊(cave bear)。洞熊的体型庞大,雄性洞熊的体重可高达1吨,可以想象与洞熊争夺山洞的战役是人类走出非洲后碰到的一大挑战。在这个过程中,火的使用是决定性的,因为在各种动物中只有人类不怕火,甚至还学会了使用火,发明了保存火种的方法,甚至制造火种的技术\footnote{维特鲁威在《建筑十书》中说:“远古时候,人类生来就像出没森林、洞穴和丛林中的野兽一样,茹毛饮血,辛苦度日。那时有一个地方,生长着密集繁茂的森林,狂风袭来,树木剧烈摇晃,树枝相互摩擦而起火。住在附近的人们被火焰吓坏了,逃之夭夭。但后来他们凑近时发现,火的热量对人体有极大的好处,他们将原木投入火中,将火种保存下来。”}。

可以想象人类曾长期生活在生有篝火的洞穴里,而这样的一个原始记忆也被用于比喻说理中,比如柏拉图在《理想国》中借用“洞穴”比喻了城邦和知识。

那么我们的洞穴经验是什么样的呢?

首先需要一个封闭的空间,比如在伸手不见五指的夜晚,任何一个房屋都可以是个洞穴,山洞无非也是个封闭的空间。

漆黑的夜晚,我们呆在山洞或封闭的房子里。我们什么都看不见,我们看不见自己,也看不见他人和物体。我们点燃一个火把或蜡烛。人是喜欢光亮的,于是都凑过去,此时我们在墙壁上看到影子,因为火把的光比较弱,反射一次后就基本没亮光了,洞穴中的影子会比阳光下的更显著和夸张,光和影在一起给我们的视觉极大的刺激。

阳光下我们不能清晰地看到物体的轮廓,但在洞穴经验中,阴影和光亮是截然分开的,我们甚至可以想象一个人去描摹阴影的轮廓。

用简单的线条去对象化一个物体是认识活动的开始,比如自我是不可见的,俗话说我们是在别人的眼睛(其实就是镜子)里认识自己的,但在洞穴经验里,我们在墙壁上能直接看见自己的阴影,比如我们可以面对着墙壁,背对着火把,伸出一只手,举过头顶……然后,我看见我面对的那个阴影会同步地作出这种种动作。

这就从视觉经验上把自我对象化了,同时我还能看见别人的阴影和其他物体的阴影……

火把的好处是可以随意移动,要想看清楚什么东西我们只需要把火把拿过来照一照就可以了。这意味着我们可以控制光线行进的方向,我可以让光向上方射,只需要我们把火把放在物体的下方,我们也可以让光向左射,只需要把火把放在物体的右边……

在洞穴中,我们举着火把从各个方向照物体,为的是要看清某物,光从某个方向射过来,我们看到的是光照亮的那个“面”,物体其他面的形象对我们是隐藏的,我们必须移动火把,使光从另外的方向射向物体,这个动作其实就是选择,我们选择从另一个角度“照亮”物体,刚才对我们显现的将隐藏在黑暗里,但新的面,新的形象会对我们显现。

\begin{figure}[ht]
\centering
\includegraphics[width=6cm]{./figures/d272e38bdd758e77.png}
\caption{三视图就是往三个⽅向做投影。} \label{fig_QMPre3_5}
\end{figure}

同时照亮所有的面则需要很多火把,比如我们可以从两个、三个,甚至六个方向上照亮物体。假设物体是三维的,并且假设物体是“透明”的,我们需要至少从三个互相垂直的方向上照亮物体,才能获得对物体的整体认识。这个其实就是工程里的三视图,上视、侧视和前视\footnote{假如物体不是透明的那就很复杂,因为还涉及物体内部构造的问题,即便不考虑内部构造,我们也得假设物体必须是“凸起”的,才能通过六视图获得物体的整体概念。}。

光源(太阳)、物体、阴影也构成一个常见的“认识论比喻”,这就是柏拉图的“太阳喻”。我们能“看”,是因为有光,而光是源自太阳的;光照射在物体上,我们像洞穴中背对着光源的原始人一样只能看到物体的阴影,即物体本身是不对我们显现的,对我们显现的只是物体的阴影。

这里我们的兴趣并不是介绍哲学上的“太阳喻”,我们只是借助这一图像建立量子力学中的“表示概念”。

在量子力学中没有物体,只有量子态,使用狄拉克记号,记作$\left| \alpha \right\rangle$,量子态本身是无法直接被“看”到的。我们需要对量子态建立一个表示,所谓表示就是选择一个观看的方式。

以观看物体为例,就是我们拿着火把以什么样的方式把物体仔细打量一番?比如我们可以选择从$x$,$y$,$z$三个方向上照亮物体,从三个方向照亮物体其实就是把物体对这三个方向做投影。

我们把矢量$V$看做是最简单的物体,往三个方向做投影就是:

\begin{equation}
\text{x方向,方向是e_x,投影是V_x = e_x \cdot V}~
\end{equation}