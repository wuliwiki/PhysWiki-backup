% 华东师范大学 2011 年 考研 量子力学
% license Usr
% type Note

\textbf{声明}:“该内容来源于网络公开资料,不保证真实性,如有侵权请联系管理员”


\begin{enumerate}
\item 什么是光电效应?其实验现象与经典理论存在什么矛盾?爱因斯坦是如何解释这一效应的?
\item 简述波函数的统计解释。
\item 自由粒子的波函数是否一定是平面波?为什么?
\item 量子力学中的物理力学量用什么算符表示?力学量算符在自身表象中的矩
阵有何特点?
\item 频率为$\omega$的一维线性谐振子的零点能是多少?零点能的存在意味着什么?
第三激发态能量是多少?
\item 什么是能级简并?简并度又是什么?举个例子说明。
\item 考虑一个一维运动问题,哈密顿算量是 $H_0 = \frac{p^2}{2m} + V(x)$ 时,已知 $H_0$ 的本征值为 $E_n^{(0)}$,$n = 1, 2, \dots$。现考虑哈密顿算量为 $H = H_0 + \frac{\lambda}{m} p$,其中 $\lambda$ 为给定参数,求 $H$ 的本征值。
\item 中心力场中,力学量完全集可否取为($p_xp_yp_z$,)?为什么通常不这样取?
\item 质量为 $m$ 的粒子在一维均匀力场 $f(x) = -F$ ($F > 0$) 中运动。若 $\rho(p, t)$ 为其在动量空间的几率密度,试证明:
$$\frac{\partial \rho(p, t)}{\partial t} = F \frac{\partial \rho(p, t)}{\partial p}~$$
\item 有两个力学量算符 $\dot A$, $\dot B$ 在某个表象下的表示为
$$A = \begin{pmatrix}3 & \sqrt{2} \\\sqrt{2} & 2\end{pmatrix} a, \quad B = \begin{pmatrix}4 & -\sqrt{2} \\-\sqrt{2} & 5\end{pmatrix} b,~$$
其中 $a, b$ 为两个实参量。\\
1.证明这两个力学量互相对易;\\
2.求出它们的共同本征态;\\
3.求能够利用表象变换将该两个力学量算符对角化的变换矩阵。\\
\item 由两个自旋为$1/2$的非全同粒子组成的体系,置于$z$轴方向的均匀外磁场中,体系的哈密顿量可写成$\hat{H} = a\sigma_1 + b\sigma_2 + c\vec{\sigma_1} \cdot \vec{\sigma_2}$, 设 $e \ll |a|, |b|, |a \pm b|$视$\hat{H}_0 = a\sigma_1 + b\sigma_2$,和$\hat{H}' = c\vec{\sigma_1} \cdot \vec{\sigma_2}$\\
1)试用定态微扰论求$\hat{H}$的能级近似值(准确到二级微扰);\\
2)解出$\hat{H}'$的能级精确值,并与微扰论的结果进行比较\\
\end{enumerate}
