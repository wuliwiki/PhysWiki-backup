% 刘维尔定理
% keys 统计力学|刘维尔定理|相空间|流体密度
% license Xiao
% type Tutor

\begin{issues}
\issueDraft
\end{issues}

$(q, p)$ 空间(广义坐标、广义d称为相空间, 复杂系统的所有系综看成多维相空间中的流体, 每个具体系统的状态是相空间中的一点, 随时间变化。 跟随一点时, 周围密度不随时间变化。

$t$ 等于零时在相空间中取一块小区域, 具有边界 $\mathcal B$。 可以证明随着时间变化, 虽然边界开始变形, 但边界两边的点不会跨越边界。 也可以证明, 这个区域的体积始终保持不变。

刘维尔定理的一个最直接推论是, 如果开始时相空间中这种流体的密度处处相同, 那么接下来在任意时刻 $t$, 流体密度仍然处处相同。

在这样的流体里面随机抽取一个点, 那么这个点几乎肯定处于平衡态。 热力学第二定律就是在这个 “几乎肯定” 上成立的。
