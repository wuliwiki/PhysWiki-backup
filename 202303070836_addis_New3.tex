% 牛顿运动定律、惯性系
% 牛顿定律|惯性系|运动定律|第二定律|第三定律

\pentry{加速度\upref{VnA}, 力的合成与分解\upref{Fdecom}, 速度的参考系变换\upref{Vtrans}}

牛顿的三定律可表述如下, 为了避免讨论物体的质心及转动, 这里我们只讨论质点。
\begin{itemize}
\item \textbf{第一定律}:不受力或受合力为零的质点做匀速直线运动或静止。
\item \textbf{第二定律}:质点所受合外力等于质点的质量乘以加速度。
\item \textbf{第三定律}:两质点的相互作用力等大反向。
\end{itemize}

\subsection{第一定律}

牛顿第一定律的作用是定义\textbf{惯性坐标系(inertial frame of reference, 简称惯性系)}:定义满足牛顿第一定律的参考系就是惯性系, 且惯性系存在。

\begin{corollary}{推论}
相对某惯性系静止或匀速运动且没有相对转动的参考系也是惯性系,否则不是惯性系。
\end{corollary}
\textbf{证明}:若已知 $A$ 系为惯性系,$B$ 系相对 $A$ 系的平移速度为 $\bvec v_{AB}$, 质点在两系中的瞬时速度分别记为 $\bvec v_A, \bvec v_B$, 则由高中的 “绝对速度 = 相对速度 + 牵连速度”(\autoref{Vtrans_eq1}~\upref{Vtrans})得
\begin{equation}
\bvec v_{B} = \bvec v_{A} + \bvec v_{AB}~.
\end{equation}
由于 $A$ 是惯性系, 所以任何不受力的质点 $\bvec v_{A}$ 都不随时间变化。 与 $\bvec v_{AB}$ 都不随时间变化,那么显然 $\bvec v_{B}$ 也不随时间变化,即 $B$ 系为惯性系。 若两系之间有任何相对的加速度(包括加速平移和转动),那么 $\bvec v_{AB}$ 将随时间或位置变化,也就不能保证 $\bvec v_B$ 一定是常矢量,所以 $B$ 系就不是惯性系。

%要证明这个推论,只需使用参考系间的速度变换,即。只有满足推论中的条件,才能保证“牵连速度”等于常矢量,进而使任意不受力或合力为零的质点在两个参考系中都做匀速运动或静止。进一步分析可以得出,任何两个惯性系的相对速度都是常矢量。

\subsection{第二定律}
牛顿第二定律只能在惯性系中使用,在非惯性系中需要用惯性力\upref{Iner} 进行修正。用矢量 $\bvec F$ 表示合力,牛顿第二定律记为
\begin{equation}\label{New3_eq1}
\bvec F = m\bvec a
\end{equation}
高中物理只强调圆周运动和直线运动, 所以一般 $F$ 和 $a$ 都记为标量, 请读者不要局限在高中思维。 这是一条矢量表达式, 加速度矢量 $\bvec a$ 是位置矢量 $\bvec r$ 关于时间的二阶导数, 或者速度矢量 $\bvec v$ 关于时间的导数。 加速度和速度不必共线也不必垂直。

\subsubsection{动量定理}
牛顿本人对第二定律的表述使用了动量定理\upref{PLaw1},记质点的动量为 $\bvec p$,则
\begin{equation}\label{New3_eq2}
\bvec F = \dv{\bvec p}{t}
\end{equation}
在经典力学中,由于质量不发生变化,\autoref{New3_eq1} 和\autoref{New3_eq2} 是等效的,但令人惊讶的是,牛顿所用的形式在狭义相对论中仍然成立\footnote{在狭义相对论中,动量的定义有所不同。},而\autoref{New3_eq1} 却不成立。

\subsection{第三定律}
广义来说,牛顿第三定律就是动量守恒定律\upref{PLaw}。牛顿第三定律在任何参考系中都适用,但是要注意两点。第一,在非惯性系中,由于惯性力作为一个数学上的修正,并不是真正的力,所以不存在反作用力。第二,在考虑电磁力时,由于电磁场可能具有动量,所以动量守恒定律要求所有物体与电磁场的动量之和守恒,而不仅仅是质点的总动量守恒。在考虑两带电粒子的相互作用力时,若假设粒子的运动速度较慢,则磁场可以忽略,电磁场动量始终为零,此时两粒子的总动量守恒,相互作用力等大反向。
