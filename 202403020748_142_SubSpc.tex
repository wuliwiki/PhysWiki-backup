% 向量子空间
% keys 向量空间|线性空间|子空间|基底|向量
% license Xiao
% type Tutor

\begin{issues}
\issueDraft
\end{issues}


\pentry{向量空间\nref{nod_LSpace}}{nod_4dd7}

\addTODO{向量子空间的定义}

如果一个向量空间 $S$ 中的所有向量都属于另一个向量空间 $V$, 且两个向量空间中加法和数乘运算的定义相同。 那么前者就是后者的\textbf{子空间}。注意每个向量空间都是它本身的子空间。

\begin{definition}{向量子空间}
向量空间 $V$ (记其域为 $\mathbb{F}$)的子集 $S$ 被称为 $V$ 的向量子空间意味着
\begin{enumerate}
\item $0_V \in S$,
\item 对任意的 $a, b \in \mathbb{F}$,$x, y \in S$,$a x + b y \in S$。
\end{enumerate}
\end{definition}

% Giacomo:TODO:重写

注意若我们在 $V$ 中任意找一组基底, 不一定能恰好从中选出 $N_S$ 个使其作为 $S$ 空间的基底。 若想让 $V$ 空间的一组基底包含 $S$ 空间的基底, 我们可以先在 $S$ 空间中选 $N_S$ 个基底, 再在 $S$ 空间外选取 $N - N_S$ 个基底即可\footnote{这样的基底一定是存在的, 因为 $N$ 维空间中任意给出 $N_S$ 个线性无关的向量, 就必定能再找到另外 $N - N_S$ 个线性无关的向量}。

\begin{example}{三维空间中的平面}
在三维的几何向量空间 $V$ 中, 若建立直角坐标系, 可以选 $\uvec x, \uvec y, \uvec z$ 作为一组正交归一的基底。 现在来看过原点且以向量 $\uvec x + \uvec y + \uvec z$ 为法向量的平面, 平面方程为\footnote{参考高中数学教材中的立体几何章节}
\begin{equation}\label{eq_SubSpc_1}
x + y + z = 0~,
\end{equation}
所有与该平面重合的向量可以构成这个三维空间中的一个二维子空间 $S$。 证明: 平面上的两个向量相加仍然落在平面上, 数乘也同样落在平面上, 详细过程略。 

但是, 由于 $\uvec x, \uvec y, \uvec z$ 中任意一个都落在该子空间外面, 所以不可能选出两个作为子空间的基底。 如果需要是选一组 $V$ 的基底且包含 $S$ 空间的基底, 可以现在 $S$ 空间中选两个基底(坐标满足\autoref{eq_SubSpc_1}), 例如坐标为 $(1, -2, 1)/\sqrt{6}$ 和 $(1, 0, -1)/\sqrt{2}$ 的两个向量。 再在空间外取一个基底, 如 $(1, 1, 1)/\sqrt{3}$。 注意这里给出的三个向量是一组正交归一基底, 但原则上只需要线性无关即可。 在该情况下, 线性无关意味着三个几何向量两两不共线且不共面(\autoref{ex_linDpe_1}~\upref{linDpe})。
\end{example}
\begin{theorem}{}\label{the_SubSpc_1}
若 $V_1, V_2$ 是向量空间 $V$ 的子空间,则 $V_1 \cap V_2$ 仍是 $V$ 的子空间。
\end{theorem}
\textbf{证明:}

1. \textbf{封闭性:} $\forall v_1,v_2\in V_1 \cap V_2$ ,由于 $V_1$ 是子空间 ,且 $v_1, v_2 \in V_1$,所以有 $v_1 + v_2 \in V_1$,同理 $v_1 + v_2 \in V_2$,故 $v_1 + v_2 \in V_1 \cap V_2$。

2.\textbf{结合性,交换律和数乘运算}直接从 $V$ 中继承过来。

3.\textbf{零向量存在性}显然。

4.\textbf{逆元存在性:}$\forall v \in V_1 \cap V_2$ ,作为 $V_1$ 的元,$-v \in V_1$,同理 $-v \in V_2$,故 $-v \in V_1 \cap V_2$。

\textbf{证毕!}


\subsection{向量张成的空间}

\addTODO{移动到基(线性代数)\upref{VecSpn}}

如果在 $N$ 维向量空间 $V$ 中给定 $M$ 个向量(不一定要求 $M \leqslant N$, 也不一定都线性无关) ${v_1}, \dots, {v_M} \in V$, 那么容易证明
\begin{equation}
{v} = \sum_{i=1}^M c_i {v_i}~
\end{equation}
的集合就是 $V$ 的一个子空间(可以是 $V$ 本身)。 我们把这个空间叫做 ${v_1}, \dots, {v_M}$ \textbf{张成}的空间。 当 ${v_1}, \dots, {v_M}$ 线性无关时, 该空间为 $M$ 维, 否则就小于 $M$ 维。

要获得张成空间的一组基底, 我们可以在这 $M$ 个向量中, 将所有可以表示为其他向量线性组合的向量剔除, 直到所有剩下的向量线性无关, 就得到了一组基底。
