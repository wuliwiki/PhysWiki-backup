% 朴素集合论(综述)
% license CCBYSA3
% type Wiki

本文根据 CC-BY-SA 协议转载翻译自维基百科\href{https://en.wikipedia.org/wiki/Naive_set_theory}{相关文章}。

朴素集合论是用于讨论数学基础的几种集合论之一。\(^\text{[3]}\)与公理化集合论不同,公理化集合论是使用形式逻辑定义的,而朴素集合论则是以自然语言非正式地定义的。它描述了离散数学中常见的数学集合的各个方面(例如维恩图和关于其布尔代数的符号推理),并且足以满足当代数学中集合论概念的日常使用。\(^\text{[4]}\)

集合在数学中具有重要意义;在现代形式化处理方式中,大多数数学对象(如数字、关系、函数等)都是通过集合来定义的。朴素集合论足以应对许多目的,同时也为更为正式的处理方法提供了一个基础。
\subsection{方法}  
在“朴素集合论”的意义上,朴素理论是一种非形式化的理论,即使用自然语言描述集合及其运算的理论。这种理论将集合视为柏拉图式的绝对对象。诸如“和”、“或”、“如果...那么”、“非”、“存在某个”、“对每个”之类的词汇,按照普通数学中的用法处理。出于方便,即使在更高级的数学中,朴素集合论及其形式化方法仍然占主导地位——包括在集合论本身更为正式的设置中。

集合论的最初发展是朴素集合论。它是在19世纪末由乔治·康托尔创立的,作为他研究无限集合的一部分\(^\text{[5]}\),并由戈特洛布·弗雷格在他的《算术基础》一书中进一步发展。