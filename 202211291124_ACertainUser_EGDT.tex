% 地球测地线

众所周知,球面上两点间的最短路径不再是一条直线段,而是一条圆弧曲线段.这也是为什么在平面地图上两点间的最短路径不再是直线段,而是一条有弧度的曲线段.

以下是绘制地球上两点间最短路径的octave/matlab代码.该程序同时将返回该路径的长度,单位为$km$. 本代码的运行需要函数\verb|Sph2Cart| \upref{SphCar}.

这个简略的模型只适用于演示,并且可能存在bug.%代码还写的很丑

\begin{lstlisting}[language=matlab]
clc
clear

%L %-180~180 W:-, E:+
%A -90~90, N:+, S:-

%在这里分别设定两点的角度制经纬度,L为经度(记E为正,W为负,取值范围-180~180),A为纬度(记N为正,S为负,取值范围-90~90).
L(1)=rand*360-180;
A(1)=rand*180-90;

L(2)=rand*360-180;
A(2)=rand*180-90;

%将经纬度换算为球坐标系坐标
if L>=0
  phi = L;
else
  phi = 360 + L;
end

theta = -A +90;

phi = phi/360*2*pi;
theta = theta/360*2*pi;

hold on
axis equal
axis([-1 1 -1 1 -1 1])

%绘制球面
[gx gy gz] = sphere(18);
g = surf(gx,gy,gz,'FaceColor',[0.8 0.8 1],'EdgeColor',[0.6 0.6 0.6]);

%绘制两点
[x y z] = Sph2Cart(1, theta, phi);
scatter3(x,y,z,'r');
text(x(1),y(1),z(1)+0.2,'Point 1');
text(x(2),y(2),z(2)+0.2,'Point 2');
scatter3(0,0,0,'k');
line([0 x(1)],[0, y(1)],[0 z(1)]);
line([0 x(2)],[0, y(2)],[0 z(2)]);

pointA=[x(1),y(1),z(1)];
pointB=[x(2),y(2),z(2)];

%计算路径长度
angle = abs(acos(dot(pointA,pointB) /(norm(pointA) * norm(pointB))));
length = 6370 * angle

%绘制路径
n = cross(pointA, pointB);
n = n / norm(n);

Bd = cross(n, pointA);
Bd = Bd/norm(Bd);

P(:,1) = pointA/norm(pointA);
P(:,2) = Bd;
P(:,3) = n;

t = linspace(0,angle,10);

cPoint(1,:)= 1*cos(t);
cPoint(2,:)= 1*sin(t);
cPoint(3,:)= 0;

for i =1:size(t,2)
  ccPoint(:,i) = P*cPoint(:,i);
end

plot3(ccPoint(1,:),ccPoint(2,:),ccPoint(3,:));

\end{lstlisting}
