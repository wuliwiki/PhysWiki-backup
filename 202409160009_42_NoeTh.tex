% Noether 定理
% keys 对称性|守恒量
% license Usr
% type Tutor

\pentry{最小作用量、哈密顿原理\nref{nod_HamPrn}}{nod_a5cc}
Nother定理是关于对称性和守恒量的,它于1915年由Noether证明,其表明每一个使拉氏量不变的变换都对应一个守恒量。正如理论物理中最深刻的定理一样,Noether定理的证明极其的简单。

\subsection{Noether定理}
\begin{theorem}{Noether定理}
对每一个使得拉氏量 $L(q_1(t),\ldots,q_n(t),\dot q_1(t),\ldots,\dot q_n(t))$ 不变的变换,都有一个守恒量存在。即若无穷小变换
\begin{equation}
q_a(t)\rightarrow q_a(t)+\delta q_a(t)
\end{equation}
使得 $L$

\end{theorem}
