% 导数(简明微积分)
% 微积分|导数|导函数|切线|极限

% 未完成: 举几个例子,说明常数导数为零, 直线导数为定值等!
\pentry{切线与割线\upref{TanL}, 函数\upref{functi}}

\subsection{函数一点处的导数}
\subsubsection{几何理解}
一个一元函数 $y = f(x)$ 在直角坐标系中表示为一条曲线. 在切线与割线\upref{TanL}中,我们已经了解什么是切线与割线.现在,让我们进行更细致的探讨.

\begin{figure}[ht]
\centering
\includegraphics[width=14cm]{./figures/Der_2.png}
\caption{割线} \label{Der_fig2}
\end{figure}

先让我们写出割线的直线方程.由于割线就是一条经过A,B两点的直线,根据高中数学知识,很容易得到
\begin{equation}
y-y_A=\frac{y_B-y_A}{x_B-x_A}(x-x_A)=\frac{f(x_B)-f(x_A)}{x_B-x_A}(x-x_A)
\end{equation}

现在,如\autoref{Der_fig2} 所示,我们固定A点不动,让B点趋近于A点,这使割线逐步趋向于切线.让我们观察割线的方程,其中除了斜率项$k=\frac{f(x_B)-f(x_A)}{x_B-x_A}$,其余位置均与$x_B$无关.当B点趋近A点时($x_B\rightarrow x_A$)时,割线(此时被称为切线)的斜率即可用极限来定义$k=\lim_{x_B\to x_A}\frac{f(x_B)-f(x_A)}{x_B-x_A}$.

事实上,我们把这个斜率定义为$x=x_A$处$f(x)$的导数.因此,我们说函数一点处的导数值就是这点处切线的斜率.更一般的,有
\begin{definition}{函数一点处的导数}
\begin{equation}
f'(x_0)=\lim_{x\to x_0}\frac{f(x)-f(x_0)}{x-x_0}
\end{equation}
\end{definition}

% \begin{figure}[ht]
% \centering
% \includegraphics[width=5cm]{./figures/Der_1.pdf}
% \caption{点 $A$ 的切线}
% \end{figure}

% 若切线存在,该切线与 $x$ 轴的夹角 $\theta$ 的正切值,就叫点 $A$ 的导数.当函数在 $A$ 点递增时,可能的取值为 $\theta \in (0,\pi/2)$, 即 $\tan \theta  \in (0, + \infty)$. 递减时,取 $\theta  \in (-\pi/2,0)$, 即 $\tan \theta \in (-\infty ,0)$. 当切线水平时,$\theta  = \tan \theta  = 0$. 
\subsubsection{“物理”理解}
仔细观察$x=x_0$附近切线的形状与$f(x)$的形状,你很容易看出(\textsl{再重复一次,在$x=x_0$附近的小范围内})这两者是几乎一样的.这启发我们用一条直线来在一个小区域内近似原函数.此时,若想计算$x$轻微增加时,函数值$y$的增量,就没有必要复杂地计算
$$\Delta y = f(x_0+\Delta x) - f(x_0)$$
,而只需要计算
\begin{equation}
\Delta y = f'(x_0) \Delta x
\end{equation}
事实上,这就引入了微分的概念\upref{Diff}.同时,这也给告诉你我们导数的另一层含义:因变量$y$对自变量$x$的“敏感度”,即$x$轻微变化时,$y$会做出多大的响应.例如,一个绝对值大的导数值意味着$y$会因为$x$的轻微变化而剧烈变化.

理解这一含义的最直白例子或许是速度\upref{VnA1},速度被定义为物体位置关于时间的导数$v=\dx{r}{t}$.即使只靠直觉,我们也能理解\textsl{“速度快”就是指“一瞬间他就从我眼前飞过去了”}.这就是说当时间轻微增加时,物体的位置大幅变化.

若函数曲线在 $x$ 的某一开区间内的每一点都可导, 则这个区间上每一个 $x$ 对应一个导数.将其写成关于 $x$ 的函数 $g(x)$,  $g(x)$  就是该区间上的 \textbf{导函数}. 通常将导函数记为以下的一种(后3种记号的来源见下文)

\begin{equation}\label{Der_eq1}
f'(x),\quad [f(x)]',\quad \dv{y}{x},\quad \dv{f}{x},\quad \dv{x}f(x)
\end{equation}
在物理中, 常常在物理量上方加一点表示对时间求导(注意仅限于对时间求导), 例如 $\dot f(t) = \dv*{f(t)}{t}$.

若切线不存在(例如折线的棱角处,但也有其他更复杂的情况), 我们说点 $A$ 不可导. 如果某区间内是 “光滑” 的, 那么的该区间内处处可导.

\begin{figure}[ht]
\centering
\includegraphics[width=5cm]{./figures/Der_3.pdf}
\caption{棱角处不可导}
\end{figure}

若函数曲线在某一点附近是光滑的,那么在这点附近取一小段,当这一段取得足够小,可以近似认为它是线段且与切线重合(如下图). 以这条线段为斜边,作一直角三角形,令其底边长为 $\dd{x}$ (在微积分中,通常把非常小的一段 $\Delta x$ 记为 $\dd{x}$,  $\dd{x}$ 是一不能分割的整体符号,而不是两个量相乘),竖直边的边长为 $\dd{y}$ (当函数递增时, $\dd{y}$ 取正值,反之取负值).根据上面导数的定义,$\dv*{y}{x} = \tan \theta $ 就是函数的导数.所以导数通常表示为 $\dv*{y}{x}$, 导数的倒数则为 $\dv*{x}{y}$. 

\begin{figure}\label{Der_fig1}[ht]
\centering
\includegraphics[width=14cm]{./figures/Der_2.pdf}
\caption{将切点放大,会发现切线和曲线在切点附近 “重合”}
\end{figure}

由上面的讨论可得,当 $x$ 增加一小段 $\Delta x$ 时,$y$ 轴的增量约为 $\Delta y \approx f'(x)\Delta x$,且当 $\Delta x$ 越小,这条式子就越精确成立, 记为 $\dd{y} = f'(x) \dd{x}$.这个关系就叫函数的微分.

\subsection{导数的严谨理解}
导数的代数理解就是: 一个量关于另一个量的变化率. 例如质点直线运动时,速度的大小就是其路程对时间的导数.把这种描述用极限\upref{Lim}表达出来就是
\begin{equation}\label{Der_eq2}
f'(x) = \lim_{\Delta x \to 0} \frac{f(x + \Delta x) - f(x)}{\Delta x}
\end{equation}
在图3的右图中,$\Delta x$ 的始末位置并不非常重要,既可以从 $x$ 取到 $x + \Delta x$, 也可以从 $x - \Delta x$  取到 $x$ 等等( 因为当 $\Delta x$ 非常小的时候,$x$ 附近的曲线基本处处跟切线重合,它们的斜率都是一样的). 所以导数的定义也有其他类似的形式

\begin{equation}
f'(x) = \lim_{\Delta x \to 0} \frac{f(x) - f(x - \Delta x)}{\Delta x} = \lim_{\Delta x \to 0} \frac{f(x + \Delta x) - f(x - \Delta x)}{2\Delta x}
\end{equation}
虽然上面用到了诸如“近似”等词,但根据定义,极限都是精确的.

例子: 速度 加速度(一维)\upref{VnA1}.







