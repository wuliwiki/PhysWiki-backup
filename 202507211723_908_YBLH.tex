% 亚伯拉罕·棣莫弗(综述)
% license CCBYSA3
% type Wiki

本文根据 CC-BY-SA 协议转载翻译自维基百科\href{https://en.wikipedia.org/wiki/Abraham_de_Moivre}{相关文章}。

\begin{figure}[ht]
\centering
\includegraphics[width=6cm]{./figures/a68a19e0be08cf7f.png}
\caption{约瑟夫·海默绘《德·莫弗像》,1736年} \label{fig_YBLH_1}
\end{figure}
亚伯拉罕·德·莫弗尔爵士(Abraham de Moivre FRS,法语发音:[abʁaam də mwavʁ],1667年5月26日-1754年11月27日)是一位法国数学家,以“德·莫弗公式”闻名——这是一个将复数与三角函数联系起来的公式。他还因在正态分布和概率论方面的工作而享有盛誉。

由于法国对胡格诺派的新教徒进行宗教迫害,特别是在1685年《枫丹白露敕令》颁布后达到高潮,他年少时迁居英格兰。他是艾萨克·牛顿、埃德蒙·哈雷和詹姆斯·斯特林的朋友。在英格兰的胡格诺派流亡者中,他也是编辑和翻译家皮埃尔·德·梅佐的同僚。

德·莫弗尔撰写了一本关于概率论的著作《机会论》,据说广受赌徒欢迎。他最早发现了比奈公式,这是一个将黄金分割数 φ 的 n 次幂与第 n 个斐波那契数联系起来的闭式表达式。他同样是最早提出中心极限定理的人之一,该定理是现代概率论的基石。
\subsection{生平}
\begin{figure}[ht]
\centering
\includegraphics[width=6cm]{./figures/832799eab7c34141.png}
\caption{《机遇论》,1756年} \label{fig_YBLH_2}
\end{figure}
\subsubsection{早年经历}
亚伯拉罕·德·莫弗于1667年5月26日出生在香槟地区的维特里勒弗朗索瓦。他的父亲丹尼尔·德·莫弗是一名外科医生,非常重视教育。尽管德·莫弗一家是新教徒,他最初却在当地的天主教基督兄弟会学校就读,这在当时宗教紧张的法国是相当少见的宽容表现。在他11岁时,父母将他送往色当新教学院就读。在那里他学习了四年希腊文,老师是雅克·迪朗代尔。这所新教学院建于1579年,由亨利-罗贝尔·德·拉·马克的遗孀弗朗索瓦丝·德·波旁倡议创办。

1682年,色当新教学院被当局关闭,德·莫弗转而前往索米尔学习逻辑学,为期两年。尽管数学并不在他的正式课程中,德·莫弗却自学了多部数学著作,包括法国奥拉托利会神父及数学家让·普雷斯特的《数学基础》,以及荷兰物理学家、数学家、天文学家和发明家克里斯蒂安·惠更撰写的赌博概率小册子《掷骰游戏中的推理》。1684年,德·莫弗搬到巴黎学习物理学,并首次接受系统的数学训练,跟随雅克·奥扎南进行私人学习。

1685年,法国国王路易十四颁布《枫丹白露敕令》,废除了给予法国新教徒广泛权利的《南特敕令》。这一法令严重压制新教信仰,禁止新教礼拜,强制要求所有儿童由天主教神父施洗。德·莫弗被送往圣马丹修道院附属学校,这是当局用来对新教儿童进行天主教灌输的场所。

目前尚不清楚德·莫弗何时离开圣马丹修道院并迁往英格兰。修道院的记录显示他在1688年离开,而德·莫弗和他的兄弟则于1687年8月28日以胡格诺派身份加入伦敦的萨伏伊教会。
\subsubsection{中年时期}
抵达伦敦时,德·莫弗已经是一位具有扎实数学功底的人,熟悉当时许多标准教材。\(^\text{[1]}\)为了谋生,他成为一名数学私人教师,在伦敦各地拜访学生授课,或在咖啡馆里讲授数学。他在一次拜访德文郡伯爵时看到牛顿的新书《自然哲学的数学原理》,便继续深入研究数学。翻阅该书后,他意识到这本书远比他以前读过的任何书都要深奥,于是下定决心要通读并理解它。然而,由于他每天要在伦敦各地奔波授课,没有太多时间静心学习,他便将书页撕下放入口袋,利用课间穿梭的空隙阅读。

据一个可能是传说的故事称,牛顿晚年时常把前来请教数学问题的人转交给德·莫弗,并说:“这些问题他比我更懂。”\(^\text{[2]}\)

到1692年,德·莫弗已与埃德蒙·哈雷成为朋友,不久后也与牛顿结识。1695年,哈雷将德·莫弗根据研究《原理》中“流数”而写的第一篇数学论文提交给皇家学会,并于同年发表在《哲学汇刊》上。不久之后,他又将牛顿著名的二项式定理推广为多项式定理。该方法于1697年被皇家学会知悉,并于当年11月30日将德·莫弗选为会员。

成为会员后,在哈雷的鼓励下,德·莫弗开始将注意力转向天文学。1705年,他凭直觉发现:“任一行星的向心力与其到中心的距离成正比,与回转体直径与垂直于切线的立方乘积成反比。”换言之,若某行星 M 绕焦点 F 作椭圆轨道运动,在某点 P,PM 是椭圆切线,且 FPM 为直角(即 FP 为切线的垂线),则 P 点处的向心力与 FM / (R·(FP)³) 成正比,其中 R 是 M 点处的曲率半径。瑞士数学家约翰·伯努利于1710年证明了这一公式。

尽管学术上取得诸多成果,德·莫弗始终未能获得大学数学教授职位,这使他不得不长期依赖费时的家教工作维生,而这对他造成的负担远大于同时代许多数学家。这种境况部分源于当时英国人对法国出身者的偏见。\(^\text{[3][4][5]}\)

1697年11月,他被选为皇家学会会员\(^\text{[1]}\),1712年被任命为一个由该学会组建的委员会成员,负责评议牛顿和莱布尼茨在微积分发明上的优先权争议。该委员会成员还包括阿布斯诺特、希尔、哈雷、琼斯、马金、伯内特、罗巴茨、博内、阿斯顿和泰勒等人。关于这场争议的详细内容,可参见“莱布尼茨与牛顿的微积分争议”词条。

德·莫弗终生贫困。据说他是圣马丁巷与克兰本街交汇处“老斯劳特咖啡馆”的常客,在那里靠下棋赚取微薄收入。
\subsubsection{晚年时期}
德·莫弗直到1754年去世前一直在研究概率与数学领域。去世后,还有几篇他的论文陆续发表。随着年龄增长,他变得愈发迟钝,睡眠时间也越来越长。人们普遍流传这样一个说法:德·莫弗注意到自己每晚比前一晚多睡15分钟,并据此精确计算出自己死亡的日期——当他的睡眠时间累计达到24小时的那天,即1754年11月27日。恰巧,他的确在那一天于伦敦去世,遗体最初安葬于圣马丁教堂,但后来被迁葬。

不过,这一“预言自己死亡日期”的说法并没有在当时留下任何文献记录,因此真实性受到质疑。\(^\text{[7]}\)
\subsection{概率论}
德·莫弗在解析几何和概率论的发展中起到了开创性作用,他在前人的基础上,尤其是克里斯蒂安·惠更斯和伯努利家族成员的工作之上,进一步扩展了这些理论。他撰写了概率论的第二本教科书《机遇论:一种计算游戏中事件概率的方法》。(关于博弈的第一本著作是吉罗拉莫·卡尔达诺在16世纪60年代撰写的《掷骰之书》,但直到1663年才出版。)

这本书共有四个版本:1711年拉丁文版,以及1718年、1738年和1756年的英文版。在后续版本中,德·莫弗加入了他1733年未发表的一项成果,即用我们今天称为正态分布或高斯函数的形式,对二项分布进行近似的首次陈述\(^\text{[8]}\)。这是第一次提出用分布的变异性(即标准差)为单位来计算某个误差大小的概率的方法,也是首次识别“可能误差”的计算。这些理论他还应用于赌博问题与年金计算表。

在概率论中,常见表达式之一是阶乘 $n!$,但在没有计算器的年代,计算较大 $n$ 的阶乘非常耗时。德·莫弗于1733年提出了一个估算阶乘的公式:$n! \approx cn^{(n + 1/2)} e^{-n}$他得出了常数 $c$ 的近似表达式,而后由詹姆斯·斯特林(James Stirling)指出该常数为 $\sqrt{2\pi}$\(^\text{[9]}\)。

德·莫弗还发表了题为《生命年金》的文章,首次揭示了随年龄变化的死亡率呈正态分布。基于此,他提出了一个简便公式,可用于估算基于个人年龄的年金支付收入。这类似于现代保险公司使用的公式类型。
\subsubsection{关于泊松分布的优先权问题}
一些关于泊松分布的结果最早由德·莫弗在《运气衡量,或关于由偶然事件决定的游戏中事件概率的研究》)中首次提出,并发表于《皇家学会哲学汇刊》第219页\(^\text{[10]}\)。因此,有些学者主张应将泊松分布命名为“德·莫弗分布”\(^\text{[11][12]}\)。
\subsection{德·莫弗公式}
1707 年,德·莫弗推导出一个方程式,从中可以得到如下公式:
$$
\cos x = \tfrac{1}{2}(\cos(nx) + i\sin(nx))^{1/n} + \tfrac{1}{2}(\cos(nx) - i\sin(nx))^{1/n}~
$$
他证明了该公式对所有正整数 $n$ 都成立\(^\text{[13][14]}\)。1722 年,他提出了另一组方程式,从中可以推导出更为人熟知的德·莫弗公式形式:
$$
(\cos x + i \sin x)^n = \cos(nx) + i \sin(nx)
^\text{[15][16]}~
$$
1749 年,欧拉使用他自己的欧拉公式对该公式进行了推广,证明了它对于任意实数 \( n \) 成立,这使得证明过程相当简洁\(^\text{[17]}\)。这条公式非常重要,因为它将复数与三角函数联系了起来。此外,该公式还可用于从 \(\cos x\) 和 \(\sin x\) 推导出 \(\cos(nx)\) 和 \(\sin(nx)\) 的有用表达式。
\subsection{斯特林近似}
德·莫弗长期研究概率论,他的研究要求他计算二项式系数,而这又需要计算阶乘\(^\text{[18][19]}\)。1730 年,德·莫弗出版了著作《分析杂集:级数与求积》(拉丁文原名 *Miscellanea Analytica de Seriebus et Quadraturis*),其中包含了 $\log(n!)$ 的对数表\(^\text{[20]}\)。对于较大的 $n$ 值,德·莫弗提出了对二项展开中项系数的近似计算方法。具体来说,对于一个偶数且足够大的正整数 $n$,德·莫弗给出如下对 $(1 + 1)^n$ 中间项系数的近似公式\(^\text{[21][22]}\):
$$
\binom{n}{n/2} = \frac{n!}{\left(\left(\frac{n}{2}\right)!\right)^2} \approx 2^n \cdot \frac{2 \cdot \frac{21}{125}(n-1)^{n-\frac{1}{2}}}{n^n}~
$$
1729 年 6 月 19 日,詹姆斯·斯特林(James Stirling)给德·莫弗写了一封信,展示了他是如何对大的 $n$ 值下的二项式展开 $(a + b)^n$ 中的中项系数进行计算的\(^\text{[23][24]}\)。1730 年,斯特林出版了他的著作《微分方法》(拉丁文原名 Methodus Differentialis),其中给出了 $\log(n!)$ 的级数表达式\(^\text{[25]}\):
$$
\log_{10} \left(n + \tfrac{1}{2}\right)! \approx \log_{10} \sqrt{2\pi} + n \log_{10} n - \frac{n}{\ln 10}~
$$
因此,对于大的 $n$,可以得到:$n! \approx \sqrt{2\pi} \left(\frac{n}{e}\right)^n$

1733 年 11 月 12 日,德·莫弗私下出版并分发了一本小册子,名为《对二项式 \((a + b)^n\)展开式各项之和的近似》(拉丁文原名 Approximatio ad Summam Terminorum Binomii \((a + b)^n\)in Seriem expansi),在其中他承认了斯特林的信件内容,并提出了一个对二项式展开中项的新近似公式\(^\text{[26]}\)。
\subsection{参见}
\begin{itemize}
\item 德·莫弗数
\item 德·莫弗五次方程
\item 经济模型
\item 高斯积分
\item 泊松分布
\end{itemize}
\subsection{注释}
\begin{enumerate}
\item O'Connor, John J.; Robertson, Edmund F., “Abraham de Moivre”,圣安德鲁斯大学 MacTutor 数学史档案
\item Bellhouse, David R.(2011)。《Abraham De Moivre: Setting the Stage for Classical Probability and Its Applications》。伦敦:Taylor & Francis,第99页。ISBN 978-1-56881-349-3。
\item Coughlin, Raymond F.; Zitarelli, David E.(1984)。《The Ascent of Mathematics》。McGraw-Hill,第437页。ISBN 0-07-013215-1。不幸的是,因为德·莫弗不是英国人,他从未能获得大学的教职。
\item Jungnickel, Christa;McCormmach, Russell(1996)。《Cavendish》。美国哲学学会回忆录,第220卷。美国哲学学会,第52页。ISBN 9780871692207。尽管他在数学界人脉广泛,工作也备受推崇,但他仍然无法找到一份好工作。甚至他在1705年皈依英格兰教会,也未能改变他是“外来者”的事实。
\item Tanton, James Stuart(2005)。《数学百科全书》。Infobase Publishing,第122页。ISBN 9780816051243。他原希望获得数学教师职位,但作为外国人,从未获得此类任命。
\item Cajori, Florian(1991)。《数学史》(第5版)。美国数学学会,第229页。ISBN 9780821821022。
\item “传记细节——亚伯拉罕·德·莫弗真的预测了自己的死亡吗?”
\item 参见:
  \begin{itemize}
  \item Abraham De Moivre(1733年11月12日),“Approximatio ad summam terminorum binomii (a+b)^n in seriem expansi”(自印小册子),7页。
  \item 英文译本:A. De Moivre,《The Doctrine of Chances…》,第2版(伦敦:H. Woodfall,1738年),第235–243页
  \end{itemize}
\item Pearson, Karl(1924)。“关于正态误差曲线起源的历史说明”。《Biometrika》,16(3–4):402–404。doi:10.1093/biomet/16.3-4.402。
\item Johnson, N.L., Kotz, S., Kemp, A.W.(1993)。《单变量离散分布》(第2版)。Wiley。ISBN 0-471-54897-9,第157页。
\item Stigler, Stephen M.(1982)。“泊松论泊松分布”。《统计与概率通讯》,1:33–35。doi:10.1016/0167-7152(82)90010-4。
\item Hald, Anders;de Moivre, Abraham;McClintock, Bruce(1984)。《A. de Moivre:《De Mensura Sortis》或《机会测量论》》。《国际统计评论》,1984年第3期:229–262。JSTOR 1403045。

\item 莫弗,Ab. de(1707年)。“Aequationum quarundam potestatis tertiae, quintae, septimae, nonae, & superiorum, ad infinitum usque pergendo, in termimis finitis, ad instar regularum pro cubicis quae vocantur Cardani, resolutio analytica”(关于三次、五次、七次、九次以及更高次方的某些方程,通过有限项的展开,类似于所谓卡尔达诺三次方程法则的解析解法)《伦敦皇家学会哲学汇刊》(拉丁文),第25卷,第309期:2368–2371。doi:10.1098/rstl.1706.0037。S2CID 186209627。
\begin{itemize}
\item 理查德·J·普尔斯坎普(Richard J. Pulskamp)于2009年提供英文翻译。
在第2370页,德·莫弗指出,如果一个级数具有以下形式:
$$
ny + \frac{1 - nn}{2 \times 3}ny^3 + \frac{1 - nn}{2 \times 3} \cdot \frac{9 - nn}{4 \times 5}ny^5 + \frac{1 - nn}{2 \times 3} \cdot \frac{9 - nn}{4 \times 5} \cdot \frac{25 - nn}{6 \times 7}ny^7 + \cdots = a~
$$
其中 $n$ 是任意给定的奇整数(正或负),而 $y$ 和 $a$ 可以是函数,那么在求解 $y$ 后,得到的结果是同页的公式(2):
$$
y = \frac{1}{2} \sqrt[n]{a + \sqrt{aa - 1}} + \frac{1}{2} \sqrt[n]{a - \sqrt{aa - 1}}~
$$
如果设 $y = \cos x$,且 $a = \cos(nx)$,则结果为:
$$
\cos x = \frac{1}{2}(\cos(nx) + i \sin(nx))^{1/n} + \frac{1}{2}(\cos(nx) - i \sin(nx))^{1/n}~
$$
这正是后来的“莫弗公式”的一种形式。
\item 1676年,艾萨克·牛顿发现了两个弦长之比为 $n:1$ 时的关系,并用上述级数表达了这种关系。该级数出现在牛顿于1676年6月13日写给皇家学会秘书亨利·奥登堡的信中——《Epistola prior D. Issaci Newton, Mathescos Professoris in Celeberrima Academia Cantabrigiensi; …》。该信件的副本也被转交给了戈特弗里德·威廉·莱布尼茨。参见以下资料第106页:Biot, J.-B.; Lefort, F. 编 (1856). 《Commercium epistolicum J. Collins et aliorum de analysi promota, etc: ou …》(拉丁文),法国巴黎:Mallet-Bachelier出版社,第102–112页。
\item 1698年,德·莫弗独立推导出了相同的级数。参见:
de Moivre, A. (1698). “A method of extracting roots of an infinite equation”(从无限方程中提取根的方法),发表于《伦敦皇家学会哲学汇刊》, 第20卷,第240期,第190–193页,doi:10.1098/rstl.1698.0034,S2CID 186214144;见第192页。
\item 1730年,德·莫弗明确考虑了函数为 cos θ 和 cos nθ 的情形。参见:Moivre, A. de(1730年)《杂项分析:级数与求积》(拉丁文),英国伦敦:J. Tonson & J. Watts,第1页。

原文第1页写道:

“引理1:若 l 与 x 为两个圆弧 A 与 B 的余弦值,这两个圆弧由同一半径1所描出,且前者是后者的 n 倍,即两者之比为 n : 1,那么有:
$$
x = \tfrac{1}{2} \sqrt[n]{l + \sqrt{ll - 1}} + \tfrac{1}{2} \cdot \frac{1}{\sqrt[n]{l + \sqrt{ll - 1}}}~
$$
(若 l 与 x 分别为两个圆弧 A 与 B 的余弦值,这两个圆弧由相同的单位半径1所描出,且弧 A 是弧 B 的 n 倍,那么就有上述关系成立。)”
因此,如果弧 A = n × 弧 B,那么 l = cos A = cos nB,x = cos B。
于是可以得到如下表达式:
$$
\cos B = \tfrac{1}{2} \left( \cos(nB) + \sqrt{-1} \sin(nB) \right)^{1/n} + \tfrac{1}{2} \left( \cos(nB) + \sqrt{-1} \sin(nB) \right)^{-1/n}~
$$
参见:
\item 康托尔,莫里茨(1898):《数学史讲义》,Teuber数学图书馆系列第8–9卷,第3卷,德国莱比锡:B.G. Teubner出版社,第624页。
\item 冯·布劳恩米尔,A.(1901):《关于所谓莫弗公式起源的历史》,《数学图书馆》第三辑第2卷,第97–102页;参见第98页。
\end{itemize}
\item 史密斯,戴维·尤金(1959):《数学原典选读》第三卷,Courier Dover 出版社,第444页,ISBN 9780486646909。
\item 德·莫弗,A.(1722):《关于角的分割》,载于《伦敦皇家学会哲学汇刊》第32卷第374期,第228–230页。doi:10.1098/rstl.1722.0039,S2CID 186210081。检索日期:2020年6月6日。
\begin{itemize}
\item 英文翻译:理查德·J·普尔斯坎普,2009年完成,2020年11月28日存档于 Wayback Machine。

第229页摘录如下:

“设 x 为任意弧的正矢(即 x = 1 – cos θ);
设 t 为另一弧的正矢;
设圆的半径为 1;
若第一弧与第二弧之比为 1 : n(即 t = 1 – cos nθ),那么,设以下两个可称为“相关”的方程:
 1 – 2zⁿ + z²ⁿ = –2zⁿt
 1 – 2z + z² = –2zx
消去 z 后,即得出确定 x 与 t 之间关系的方程。”

换句话说,给定以下两个方程:1 – 2zⁿ + z²ⁿ = –2zⁿ (1 – cos nθ)1 – 2z + z² = –2z (1 – cos θ)

使用求解一元二次方程的方法,分别对第一个方程中的 zⁿ 和第二个方程中的 z 解出,结果将是:zⁿ = cos nθ ± i sin nθ z = cos θ ± i sin θ
从而立即得出:
  (cos θ ± i sin θ)ⁿ = cos nθ ± i sin nθ\\

参见:
\item 史密斯,戴维·尤金,《数学原典选读》,第二卷,纽约市:多佛出版公司,1959年,第444–446页,见第445页脚注1。
\end{itemize}

\end{enumerate}