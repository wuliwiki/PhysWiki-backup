% 利萨茹曲线

\begin{issues}
\issueDraft
\end{issues}

\pentry{简谐振子\upref{SHO}}

\footnote{参考 Wikipedia \href{https://en.wikipedia.org/wiki/Lissajous_curve}{相关页面}.}\textbf{利萨茹曲线(Lissajous curve)}是平面上一点在两个垂直的方向分别做相简谐运动形成的轨迹. 把这两个方向作为 $x, y$ 坐标轴, 可以用以下参数方程表示
\begin{equation}\label{Lissaj_eq1}
\leftgroup{
x &= A\sin(a t + \phi)\\
y &= B\sin(b t)
}\end{equation}
通常来说, 我们讨论的是闭合的利萨茹曲线, 此时要求频率 $a/b$ 是一个有理数, 通常是两个较小的整数之比. 周期为两个方向周期的最小公倍数. 式中 $\phi$ 被称为相位差, 顾名思义是两个简谐振动的相位之差. 当曲线闭合时, 把\autoref{Lissaj_eq1} 中的 $\sin$ 都换成 $\cos$ 曲线形状也不变, 因为这相当于给 $x, y$ 同时加上 $\pi/2$ 相位. 给 $x, y$ 同时加上任意相同的相位都不会改变曲线形状.

下文中,我们假设 $\phi \in (-\pi, \pi]$. 不难证明, 给 $\phi$ 取相反数同样不改变曲线的形状: $\phi = \pm\abs{\phi}$ 和 $\phi = $

\subsection{频率相同}
当 $\phi = 0$ 时曲线是延 $y=x$ 的线段, $\phi = \pi$ 时是延 $y=-x$ 的线段. 当 $\phi = \pm\pi/2$ 时轨迹是一个椭圆, 椭圆的长短两个轴与 $x,y$ 坐标轴平行. 特殊地, 如果 $\phi = \pm\pi/2$ 且 $A = B$ 则轨迹是一个圆.



当 $\phi$ 是其他值时, 曲线是一个斜置的椭圆. 当 $A = B$ 时, 椭圆长轴与 $x$ 轴的夹角为
\addTODO{推导未完成}
\addTODO{画图}

一般而言, 我们可以令 $t = 0$, 那么曲线与 $x$ 轴的截线长度为 $A\sin(\phi)$, 它与曲线的 $x$ 坐标最大值 $A$ 之比为 $\sin\phi$, 由此可判断相位差.

\subsection{频率不同}
\addTODO{如何从闭合曲线判断频率之比是多少?通过数一个周期内两个方向的峰值.}
\addTODO{画图}
