% 安德烈-马里·安培(综述)
% license CCBYSA3
% type Wiki

本文根据 CC-BY-SA 协议转载翻译自维基百科\href{https://en.wikipedia.org/wiki/Andr\%C3\%A9-Marie_Amp\%C3\%A8re}{相关文章}。

安德烈-玛丽·安培(André-Marie Ampère ForMemRS,英国发音:/ˈɒ̃pɛər, ˈæmpɛər/,美国发音:/ˈæmpɪər/,[1] 法语:[ɑ̃dʁe maʁi ɑ̃pɛʁ];1775年1月20日-1836年6月10日)[2] 是法国物理学家和数学家,是经典电磁学学科的奠基人之一,他将其称为‘电动力学’。他还发明了众多应用,例如由他命名的螺线管和电报机。作为一位自学成才的科学家,安培是法国科学院的院士,并在巴黎综合理工学院和法兰西学院担任教授。

国际单位制中的电流单位安培(A)以他的名字命名。他的名字还被刻在埃菲尔铁塔上的72个名字之一。‘运动学’(kinematic)一词是他创造的法语‘cinématique’的英语版本,[3] 它来源于希腊语 κίνημα kinema(意为‘运动’),其本身衍生自 κινεῖν kinein(意为‘移动’)。[4][5]
\subsection{传记}
\subsubsection{早年生活} 
安德烈-玛丽·安培于1775年1月20日出生在里昂,他的父亲是成功的商人让-雅克·安培,母亲是让娜·安托瓦内特·德苏蒂耶尔-萨尔塞·安培。他出生于法国启蒙运动的鼎盛时期,童年和少年时期大部分时间都在靠近里昂的家族庄园——波莱米约-蒙多尔(Poleymieux-au-Mont-d'Or)度过。[6] 让-雅克·安培是一位成功的商人,同时也是让-雅克·卢梭哲学的仰慕者。卢梭在其著作《爱弥儿》中阐述的教育理论成为安培教育的基础。卢梭认为,男孩应该避免接受正式学校教育,而是从“自然中直接获得教育”。安培的父亲将这一理念付诸实践,允许儿子在家族藏书丰富的图书馆中自学。因此,像乔治-路易·勒克莱尔(乔治·路易·布封)的《自然史》(自1749年开始撰写)和丹尼斯·狄德罗与让·勒隆·达朗贝尔的《百科全书》(1751年至1772年间新增的卷册)等法国启蒙时期的杰作,成为了安培的“老师”。[需要引用]  

然而,年轻的安培很快重新开始了拉丁语课程,这使他能够深入研究莱昂哈德·欧拉和丹尼尔·伯努利的著作。[7]
\subsubsection{法国大革命}
此外,安培利用接触最新书籍的机会,从12岁起自学高等数学。晚年时,安培声称自己在18岁时对数学和科学的了解已经达到了人生的巅峰。然而,作为一位博学多才的人,他的阅读范围还包括历史、旅行、诗歌、哲学和自然科学。[7] 由于他的母亲是一位虔诚的天主教徒,安培在接触启蒙科学的同时,也接受了天主教信仰的熏陶。法国大革命(1789-1799)在他的少年时代爆发,这对他产生了深远的影响。安培的父亲因革命新政府的召唤而参与公共事务,[8] 并在里昂附近的一个小镇担任地方法官(*juge de paix*)。1792年,雅各宾派掌控革命政府后,他的父亲让-雅克·安培因抵制新的政治浪潮,于1793年11月24日被送上断头台,这是雅各宾派清洗的一部分。

1796年,安培结识了朱莉·卡隆(Julie Carron),并于1799年结婚。同年,安培开始了他的第一份正式工作,担任数学教师,这使他有了经济保障,与卡隆结婚,并在次年迎来了他们的第一个孩子让-雅克(以其父亲命名)。后来,让-雅克·安培因语言学研究而声名鹊起。安培的成长与法国向拿破仑政权的过渡同步,他在拿破仑支持的技术官僚体系中找到了新的成功机会。

1802年,安培被任命为布雷斯堡中央学校(École Centrale)的物理和化学教授,并将生病的妻子和婴儿让-雅克·安托万·安培留在里昂。他利用在布雷斯堡的时间进行数学研究,并于1802年完成了《关于博弈数学理论的思考》(*Considérations sur la théorie mathématique du jeu*)的论文,这是一部关于数学概率的著作。他于1803年将其送交巴黎科学院审议。
\subsubsection{教学生涯}
\begin{figure}[ht]
\centering
\includegraphics[width=6cm]{./figures/23e944b2e0dca4a0.png}
\caption{《科学哲学试论》} \label{fig_AP_1}
\end{figure}
1803年7月,安培的妻子去世后,[9][10] 他搬到了巴黎,并于1804年在新成立的综合理工学院(École Polytechnique)担任导师。尽管他没有正式的学术资格,安培于1809年被任命为该校的数学教授。在1828年之前,他一直在该校担任各种职务。此外,1819年和1820年,安培分别在巴黎大学开设了哲学和天文学课程;1824年,他被选为法国学院(Collège de France)实验物理学著名讲席的教授。1814年,安培被邀请加入新成立的帝国学院数学班,这是改革后的国家科学院的上级机构。

在被选入科学院之前的几年,安培参与了广泛的科学研究,撰写论文并探讨从数学、哲学到化学和天文学的各种主题,这在当时顶尖的科学知识分子中是很常见的。安培曾声称:“在他18岁时,他的人生中有三个至高点:第一次领圣体、阅读安托万·莱昂纳·托马的《笛卡尔的颂辞》,以及攻占巴士底狱的消息。”在他妻子去世的那一天,他写下了《诗篇》中的两句经文,并祈祷:“主啊,慈悲的上帝,请在天上让我与那些您允许我在地上所爱的人团聚。”在困境中,他会通过阅读《圣经》和教会教父的作品寻求慰藉。[11]

作为一名虔诚的平信徒,他曾一度接纳年轻学生弗雷德里克·奥扎南(Frédéric Ozanam,1813–1853)住在家中。奥扎南后来成为慈善会议(Conference of Charity,后称圣文森·德保禄会)的创始人之一。[citation needed] 奥扎南于1998年由教宗若望·保禄二世册封为真福。通过安培的引荐,奥扎南接触到了新天主教运动的领袖,例如弗朗索瓦-勒内·德·夏多布里昂(François-René de Chateaubriand)、让-巴蒂斯特·亨利·拉科代尔(Jean-Baptiste Henri Lacordaire)和夏尔·福布斯·勒内·德·蒙塔朗贝尔(Charles Forbes René de Montalembert)。[citation needed]
\subsubsection{电磁学领域的研究}
1820年9月,安培的朋友、后来的悼词作者弗朗索瓦·阿拉戈向法国科学院的成员展示了丹麦物理学家汉斯·克里斯蒂安·奥斯特德的惊人发现:电流能够使邻近的磁针偏转。安培随即开始发展一种数学和物理理论,以理解电与磁之间的关系。在奥斯特德实验工作的基础上,安培进一步研究发现,平行导线中的电流会相互吸引或排斥,具体取决于电流方向是否相同或相反——这一发现奠定了电动力学的基础。他还通过数学的方法,将这些实验结果推广为普遍的物理定律,其中最重要的就是后来被称为**安培定律**的原理。该定律指出,两段载流导线之间的相互作用力与导线长度及其电流强度成正比。安培还将这一原理应用于磁学,证明了其定律与法国物理学家夏尔·奥古斯丁·库仑的电作用定律之间的和谐一致性。安培对实验技术的专注和精湛技能,使他的科学研究牢牢根植于新兴的实验物理领域。

安培还对电磁关系提供了物理学上的理解,他推测了“电动力分子”(即电子概念的雏形)的存在,这种分子是电与磁的基本组成单元。基于这一物理解释,安培发展出了一套关于电磁现象的物理学理论,这一理论既可通过实验验证,又具有数学上的预测能力。近百年后,在1915年,阿尔伯特·爱因斯坦和万德·约翰内斯·德哈斯通过**爱因斯坦-德哈斯效应**证明了安培假设的正确性。

1827年,安培发表了他的代表作《仅通过实验得出的电动力学现象数学理论的回忆录》(**Mémoire sur la théorie mathématique des phénomènes électrodynamiques uniquement déduite de l'experience**)。这部著作为其新科学——电动力学命名,并成为该领域的奠基之作。

同年,安培被选为英国皇家学会的外籍会员,次年被选为瑞典皇家科学院的外籍会员。[12] 或许,他最崇高的荣誉来自詹姆斯·克拉克·麦克斯韦,后者在《电与磁理论》一书中称安培为“电学领域的牛顿”。
\subsection{荣誉}
\begin{itemize}
\item 1825年10月8日:安培被选为比利时皇家科学、文学与美术学院院士。[13]
\end{itemize}
\subsubsection{遗产}
1881年国际电力博览会签署的一项国际公约,将**安培**(ampere)确立为电流的标准单位之一,以表彰安培在现代电学科学创建中的贡献。同一公约中,还有库仑(coulomb)、伏特(volt)、欧姆(ohm)、瓦特(watt)和法拉(farad)等单位,它们分别以安培同时代的科学家命名:法国的夏尔-奥古斯丁·库仑、意大利的亚历山德罗·伏打、德国的乔治·欧姆、苏格兰的詹姆斯·瓦特和英格兰的迈克尔·法拉第。安培的名字还被刻在埃菲尔铁塔的72个名字之一中。

许多街道和广场以安培的名字命名。此外,还有以他命名的学校、一座里昂地铁站、一种图形处理器微架构、月球上的一座山,以及挪威的一艘电动渡轮。[14]
\subsection{著作}

\begin{itemize}
\item 《关于数学游戏理论的思考》(Considérations sur la théorie mathématique du jeu),Perisse,里昂,巴黎,1802年,可在线阅读于互联网档案馆。
\item 安德烈-玛丽·安培**(1822),《电动力学观察集:包含各种论文、通知、书信或科学期刊摘录,涉及两个电流之间的相互作用、磁针或地球与两个磁铁之间的相互作用》(**Recueil d'observations électro-dynamiques**,法文),出版商:Chez Crochard,2010年9月26日获取。
\item 安德烈-玛丽·安培;雅克·M. 巴比内**(1822),《关于电和磁的新发现的展示》(**Exposé des nouvelles découvertes sur l'électricité et le magnétisme**,德文),出版商:Chez Méquignon-Marvis,2010年9月26日获取。
\item 安德烈-玛丽·安培**(1824),《电动力学装置的描述》(**Description d'un appareil électro-dynamique**,法文),出版商:Chez Crochard ... et Bachelier,2010年9月26日获取。
\item 安德烈-玛丽·安培**(1826),《电动力学现象理论,仅从实验中推导》(**Théorie des phénomènes électro-dynamiques, uniquement déduite de l'expérience**,法文),出版商:Méquignon-Marvis,2010年9月26日获取。
   \item 安德烈-玛丽·安培**(1883),《电动力学现象数学理论,仅从实验中推导》(*Théorie mathématique des phénomènes électro-dynamiques: uniquement déduite de l'expérience**,法文,第2版),出版商:A. Hermann,2010年9月26日获取。

\item 安德烈-玛丽·安培(1834),《科学哲学随笔,或对所有人类知识的自然分类的分析性展示》(Essai sur la philosophie des sciences, ou, Exposition analytique d'une classification naturelle de toutes les connaissances humaines,德文),出版商:Chez Bachelier,2010年9月26日获取。
  \item 安德烈-玛丽·安培(1834),《科学哲学随笔》(Essai sur la philosophie des sciences,德文,第1卷),出版商:Chez Bachelier,2010年9月26日获取。
   \item 安德烈-玛丽·安培(1843),《科学哲学随笔》(Essai sur la philosophie des sciences,德文,第2卷),出版商:Bachelier,2010年9月26日获取。
\end{itemize}