% 矩阵李群
% 矩阵李群|李群

\begin{issues}
\issueDraft
\end{issues}

\pentry{一般线性群\upref{GL},域上的代数\upref{AlgFie}}

% 我希望把这个词条写成不需要微分几何前置的样子,参见GTM222
\subsection{矩阵李群}

\footnote{本文参考\cite{GTM222}} $M_n(\mathbb{C})$ 是全体 $n \times n$ 复矩阵的集合。全体 $n \times n$ 可逆矩阵的集合 $\opn{GL}(n, \mathbb{C})$ 构成一个群,同时也是拓扑空间 $M_n(\mathbb{C})$ 的一个开集合(因此是个子流形)。

\begin{definition}{矩阵李群}
对于群 $\opn{GL}(n, \mathbb{C})$ 的子群 $G$,$G$ 被称为一个\textbf{矩阵李群}如果它是 $\opn{GL}(n, \mathbb{C})$ 的一个闭子集。 Definition 1.4 \cite{GTM222} 
\end{definition}

对于一个矩阵李群 $G$,我们有
$$
G \subseteq \opn{GL}(n, \mathbb{C}) \subseteq M_n(\mathbb{C})
$$
$G$ 在 $\opn{GL}(n, \mathbb{C})$ 中是闭的,但在 $M_n(\mathbb{C})$ 中不一定。

\subsection{例子}

\begin{example}{}
$$\opn{SL}(n, \mathbb{C}) \subseteq \opn{GL}(n, \mathbb{C})$$
$$\opn{GL}(n, \mathbb{R}) \subseteq \opn{GL}(n, \mathbb{C})$$
$$\opn{SL}(n, \mathbb{R}) \subseteq \opn{GL}(n, \mathbb{C})$$
\end{example}

\subsubsection{幺正群和正交群}
\addTODO{定义}
\subsubsection{广义正交群}
\addTODO{定义}

\addTODO{定义:洛伦兹群}

\addTODO{定理:在复数下只有一种正交群}

\subsubsection{辛群和紧辛群}
\addTODO{定义}
% 可以专门开个篇章讲

\subsubsection{(广义)欧几里得群}
\addTODO{定义}
\addTODO{定义:伽利略群和庞加莱群}
% 可以专门开个篇章讲

\subsubsection{海森堡群}
\addTODO{定义}
% 如果物理人认为有必要的话

\subsubsection{性质}
\pentry{李群\upref{LieGrp},紧致性\upref{Topo2},道路连通性\upref{Topo4},}

\begin{theorem}{}
矩阵李群是李群 $\opn{GL}(n, \mathbb{C})$ 的子李群
\end{theorem}

\begin{exercise}{}
证明它。
\end{exercise}

\begin{exercise}{}
判断例子中的那些矩阵李群是
\begin{itemize}
\item 紧致的;
\item (道路)连通的;
\item 单连通的。
\end{itemize}
\end{exercise}
\addTODO{单连通的词条}

