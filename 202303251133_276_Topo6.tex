% 积拓扑
% keys 乘积拓扑|拓扑空间|笛卡儿积|拓扑基|子基|乘积映射

\pentry{拓扑空间\upref{Topol}}

\subsection{有限维积拓扑}

给定两个拓扑空间 $(X_1, \mathcal{T}_1)$ 和 $(X_2, \mathcal{T}_2)$,那么我们可以在笛卡尔积 $X=X_1\times X_2$ 中定义一个拓扑 $\mathcal{T}$,其拓扑基为 $\{O_1\times O_2|O_1\in\mathcal{T_1}, O_2\in\mathcal{T_2}\}$。就是说,两个拓扑空间的笛卡尔积的拓扑基,是两个空间中开集的笛卡尔积的集合。

\begin{figure}[ht]
\centering
\includegraphics[width=6cm]{./figures/Topo6_2.pdf}
\caption{二维乘积拓扑的拓扑基中每个元素都是如图所示的一个“矩形”。} \label{Topo6_fig2}~.
\end{figure}

由拓扑基生成拓扑的方式(取任意并),容易发现,如果上述 $O_1$ 和 $O_2$ 取的只是 $X_1$ 和 $X_2$ 空间的某两个拓扑基,也能得到一样的定义。

一般地,$N$ 个拓扑空间的集合做笛卡尔积,这个笛卡尔积集合上的\textbf{积拓扑}定义为:

\begin{definition}{有限维积拓扑}
设 $(X_n, \mathcal{T}_n)$ 是若干拓扑空间,$n$ 取值范围为 $[1, N]\cap\mathbb{Z}$。那么\textbf{积空间} $X_1\times X_2\times X_3\times\cdots\times X_N=\prod\limits_{n=1,2,\cdots,N}X_n=X$ 中的拓扑由\textbf{拓扑基} $\mathcal{B}$ 生成,其中 $\mathcal{B}=\{\prod\limits_{n=1,2,\cdots,N}O_n|\forall n, O_n\in\mathcal{T}_n\}$。
称这个拓扑为各$\mathcal{T}_n$的\textbf{乘积拓扑(product topology)},或译作\textbf{积拓扑}。
\end{definition}

\subsection{任意维积拓扑}

如果用于进行笛卡尔积的拓扑空间数量大于等于 $\aleph_0$,那么我们常用的乘积拓扑定义会和有限维情况的说法略有不同。在这里,我们使用\textbf{子基(sub-basis)}来定义积拓扑:

\begin{definition}{任意维积拓扑}

设 $(X_\alpha, \mathcal{T}_\alpha)$ 是若干拓扑空间,$\alpha$ 不再是整数指标,而是用一个集合 $\Lambda$ 中的元素来表达的指标:$\alpha\in\Lambda$。这样的集合 $\Lambda$ 称为一个\textbf{指标集(set of indexes)}。

空间 $X=\prod\limits_{\alpha\in\Lambda}X_\alpha$ 的拓扑 $\mathcal{T}$ 由\textbf{子基}$\mathcal{S}$ 生成,其中 $\mathcal{S}=\{O_{\alpha_0}\times\prod\limits_{\alpha\in\Lambda-\{\alpha_0\}}X_\alpha|\alpha_0\in\Lambda, O_{\alpha_0}\in\mathcal{T}_{\alpha_0}\}$。就是说,$\mathcal{S}$ 中的每一个元素,都是某个 $X_{\alpha_0}$ 的开集 $O_{\alpha_0}$ 和其它所有 $X_\alpha$ 乘积的结果。

\begin{figure}[ht]
\centering
\includegraphics[width=7.5cm]{./figures/Topo6_1.pdf}
\caption{子基 $\mathcal{S}$ 中的一个元素,类似一个“柱形”(类比 $\mathbb{R}$ 空间之间的乘积)。} \label{Topo6_fig1}
\end{figure}

其中,各 $X_\alpha$ 空间称作 $X$ 的一个\textbf{分量(component)},类比向量空间的称呼。

\end{definition}

任意维积拓扑也可以用拓扑基定义,只不过拓扑基中的元素不再是各分量空间的开集之乘积,而是有限个分量空间的开集和其它全体分量空间本身的乘积。

任意维积拓扑的定义是包含了有限维的定义的,因为当进行乘积的空间是有限个(即 $|\Lambda|<\aleph_0$)的时候,任意维积拓扑的定义就和有限维积拓扑的定义一致了。

\subsection{积映射}

给定集合间的映射 $f:A\rightarrow X$ 和 $g:B\rightarrow Y$,可以定义映射 $f\times g:A\times B\rightarrow X\times Y$。其中 $\forall a\in A, b\in B$,有 $f\times g(a, b)=(f(a), g(b))$。

\begin{exercise}{}\label{Topo6_exe1}
证明:如果上述集合是拓扑空间,并且 $f$ 和 $g$ 都是连续映射,那么 $f\times g$ 也是乘积空间之间的连续映射。
\end{exercise}



