% 数学分析笔记
% rudin|数学分析|实数|复数|集合论|集合

\begin{issues}
\issueMissDepend
\issueOther{这是一个总结, 应该放到所有相应内容之后, 以及给出详细词条的链接}
\end{issues}

本文参考 \cite{Rudin}.

\subsection{Chap 1. 实数系和复数系}

\begin{itemize}
\item \textbf{有理数(rational number)}记为 $Q$, 实数记为 $R$

\item 虽然任意两个不同的有理数间还有一个有理数, 但是有理数集中还是会有 “间隙”, 而实数集填补了这些间隙.

\item \textbf{集合(set)}:\textbf{属于(in)} $x \in A$, \textbf{不属于(not in)} $x \notin A$

\item \textbf{空集(empty set)}, \textbf{非空(none empty)},\textbf{子集(subset)} $A \subseteq B$,\textbf{超集(superset)} $B \supseteq A$, \textbf{真子集(proper subset)}

\item \textbf{有序集(ordered set)}, 任意不相等的两个元可以比较大小

\item \textbf{有上界(bounded above)}: 任意元小于等于超集中的某个元. \textbf{上界(upper bound)}

\item \textbf{最小上界(least upper bound, supremem)} : $\alpha = \sup E$; \textbf{最大下界(greatest lower bound, infimum)}: $\alpha = \inf E$

\item 如果对于任意非空有上界的 $E \subset S$, 都有 $\sup E \in S$, 那 $S$ 就具有 \textbf{upper bound property}

\item \textbf{域(field)} 集合 $F$ 定义了\textbf{加法}和\textbf{乘法}. 加法满足: 闭合性, 交换律, 结合律, 存在 0 元, 存在逆元. 乘法满足: 闭合性, 交换律, 结合律, 存在单位元, 存在倒数. 加法和乘法满足分配律.

\item 有理数集是一个域

\item \textbf{有序域(ordered field)}

\item 存在一个有序域 $R$ 具有 upper bound property, 且有理数集 $Q$ 是其子集. $R$ 就是实数.

\item 实数的\textbf{阿基米德性质}: 存在整数 $n$ 使 $nx > y$ ($x > 0$)

\item $x \in R$, $x > 0$, $n$ 为整数, 存在实数 $y$ 使 $y^n = x$

\item \textbf{稠密(dense)}: 两个不同的实数间必有一个有理数

\item \textbf{extended real number system} 是在实数集基础上加入 $\pm\infty$ 两个符号. 对任何实数有 $-\infty < x < +\infty$. 所有非空子集都有最小上界和最大下界. 相比于无穷, 实数集中的元被称为 \textbf{finite}.

\item 复数是一对有序实数 $(a, b)$, 定义了加法和乘法后, 就变成了一个域. 定义 $\I = (0, 1)$.

\item 对正整数 $k$, $R^k$ 定义为所有 $k$ 个有序实数的集合 $\bvec x = (x_1, \dots, x_k)$, 其中 $x_i$ 叫做坐标.

\item 定义 $R^k$ 中的内积为 $\bvec x \vdot \bvec y = \sum_{i = 1}^k x_i y_i$

\item 定义模长为 $\abs{x} = (\bvec x \vdot \bvec x)^{1/2}$

\item 定义了内积和模长的 $R^k$ 被称为\textbf{欧几里得 $k$ 空间(euclidean k-space)}. 这也是一个\textbf{度量空间(metric space)}(见下文). 通常 $R^1$ 叫做线, $R^2$ 叫做面
\end{itemize}

\subsection{Chap 2. 基本拓扑}

\begin{itemize}
\item \textbf{函数}是两个集合 $A$, $B$ 之间的\textbf{映射}; 定义域, 值域, 值. $f(A)$ 就是集合 $A$ 的 \textbf{image}; $f(A) \subset B$. 如果 $f(A) = B$, 那么 $f$ 把 $A$ 映射到(onto) $B$. $f^{-1}(E)$ \textbf{inverse image}. \textbf{1-1 映射}

\item 如果 $A$ 到 $B$ 存在 1-1 映射, 记为 $A \sim B$: \textbf{reflexive} $A \sim A$, \textbf{symmetric} $A \sim B \to B \sim A$, \textbf{transitive} $A \sim B, B \sim C \to A \sim C$. 此时称 $A$ 和 $B$ 等效, 他们具有相同的\textbf{基数(cardinal number)}即元素个数.

\item 定义 $J_n$ 为集合 $1,2,\dots, n$, 定义 $J$ 为 $1, 2, \dots$

\item $A \sim J_n$ 为\textbf{有限(finite)}, 不是有限就是\textbf{无限(infinite)}, $A \sim J$ 为\textbf{可数(countable)}\footnote{也叫 enumerable 或者 denumerable}. \textbf{至多可数(at most countable)} 就是有限或者可数

\item 对无限集来说, “含有同样多个元素” 变得很模糊, 但是 1-1 映射的定义仍然有效(只要写出一个表达式)

\item 有限集不可能与其真子集等效, 而无限集可以

\item 数列是 $J$ 的映射 $f(n) = x_n$, 记为 $\{x_n\}$. $x_n$ 叫做一项. 如果 $x_n \in A$, 那该序列就叫 $A$ 中(元素)的序列.

\item 可数集的无穷子集仍然是可数的

\item \textbf{Sequence of set} $\{E_\alpha\}$, 每个 $\alpha \in A$ 都对应一个 $E_\alpha$(其实就是 set of sets, 但不这么叫)

\item \textbf{并集(Union)}: $\bigcup\limits_{m = 1}^n E_m$, \textbf{交集(intersection)}: $\bigcap\limits_{m = 1}^n E_m$

\item 并集和交集的混合运算法则与加法和乘法差不多

\item 如果 $E_n\ (n = 1, 2\dots)$ (无穷多个)是可数的, 那么它们的并集仍然是可数的

\item 如果 $A$ 是至多可数, 且对 $\alpha \in A$, $B_\alpha$ 也是至多可数, 那么 $T = \bigcup_{\alpha \in A} B_\alpha$ 也是至多可数

\item 如果 $A$ 可数, $a_i \in A$,  而 $B_n$ ($n$ 为固定的正整数)是所有 $(a_1, \dots, a_n)$ 的集合, 那么 $B_n$ 也是可数的

\item 自然数和有理数(可以看作两个有序整数)都可数, 无理数不可数

\item 如果集合 $X$ 中的元素可以叫做\textbf{点(point)}, 如果一个值为实数的函数 $d(p, q), \ p \in X,\ q \in X$ 满足: 当 $p = q$, $d(p, q) = 0$, 当 $p \ne q$, $d(p, q) > 0$, $d(p, q) = d(q, p)$, $d(p, q) \leqslant d(p, r) + d(r, q)$, $r \in X$, 我们就说这是一个\textbf{度量空间(metrix space)}, 函数 $d$ 叫做\textbf{距离函数(distance function)}, 或者\textbf{度规(metric)}. 度读 du 第四声.

\item \textbf{segment} $(a, b)$ 是所有 $a < x < b$ 的实数, \textbf{interval} $[a, b]$ 是所有 $a \leqslant x \leqslant b$ 的实数.

\item interval 也叫 1-方格, 类似地, $R^k$ 中可以定义 \textbf{k-方格(k-cell)}, 2-方格是长方形

\item 类似地, $R^k$ 空间也可以定义 \textbf{开/闭球(open/closed ball)}

\item \textbf{凸(convex)}: $E \subset R^k$ 对任意 $0 < \lambda < 1$ 和 $\bvec x, \bvec y \in E$ 满足 $\lambda \bvec x + (1 - \lambda) \bvec y \in E$. 例如, ball 和 k-方格都是 convex 的.

\item 度量空间中, \textbf{邻域(neighborhood)} $N_r$: 到某点距离小于 $r$ 的集合($r > 0$)

\item 度量空间中, \textbf{极限点(limit point)} $p$: 所有邻域存在一个与 $p$ 不同的点(无论半径有多小)

\item 如果不是极限点, 那就是 \textbf{孤立点(isolated point)}

\item 如果所有极限点都属于集合, 这个集合就是\textbf{闭(closed)}的.

\item 如果关于点 $p$ 的某个邻域是集合 $E$ 的子集, $p$ 就是 $E$ 的\textbf{内点(interior point)}

\item 如果 $E$ 中的任意一点都是内点, $E$ 就是 \textbf{开(open)} 的

\item \textbf{补集(complement)}

\item 如果一个闭集合中每一点都是它的极限点, 那么该集合就是\textbf{完全(perfect)} 的

\item 如果集合中任意一点都在某个 $r$ 为实数的邻域内, 这个集合就是 \textbf{有界的(bounded)}

\item 集合 $E$ 在集合 $X$ 上\textbf{稠密(dense)}: $X$ 中任意一点都是 $E$ 的一个极限点或者 $E$ 中的一点. (例如有理数在实数上稠密)

\item 任何邻域都是开的

\item 如果 $p$ 是一个极限点, 那么它的任何邻域都有无限多个点

\item 有限个点的集合没有极限点

\item $(\bigcup_\alpha E_\alpha)^c = \bigcap_\alpha (E_\alpha^c)$ 其中 $c$ 代表补集

\item 集合 $E$ 是开的当且仅当它的补集是闭的. $E$ 是闭的当且仅当它的补集是开的.

\item 任意多开集合的并集仍然是开的, 任意多闭集合的交集仍然是闭的

\item 有限个开集合的交集仍然是开的, 有限个闭集合的并集仍然是闭的

\item 设 $X$ 是度量空间, 如果 $E \subset X$, $E'$ 表示 $E$ 在 $X$ 中所有极限点组成的集. 那么, 把 $\bar E = E \cup E'$ 叫做 $E$ 的\textbf{闭包(closure)}

\item 设 $X$ 是度量空间,而 $E \subset X$, 那么 (a) $\bar E$ 是闭的, (b) $E = \bar E$ 当且仅当 $E$ 闭, (c) 如果闭集 $F \subset X$ 且 $E \subset F$, 那么 $\bar E \subset F$. 由 (a) 和 (c), $E$ 是 $X$ 中包含 $E$ 的最小闭子集

\item 设 $E$ 是一个不空实数集, 上有界, 令 $y = \sup E$ ,那么 $y \in \bar E$.
因此, 如果 $E$ 闭, 那么 $y \in E$.

\item 令 $Y \subset X$, $E \subset Y$, $E$ 是开的当且仅当 $E = Y \bigcap G$, 对某个 $G \subset X$

\item 若 $X$ 的一组开子集 $\{G_\alpha\}$ 使 $E \subset \bigcup_\alpha G_\alpha$, 那么 $\{G_\alpha\}$ 就是 $E$ 的\textbf{开覆盖(open cover)}

\item \textbf{紧集(compact set)}: 如果 $\{G_\alpha\}$ 是 $K$ 的开覆盖, 那么存在有限个 $\alpha_1,\dots, \alpha_n$ 使得 $K \subset G_{\alpha_1} \bigcup \dots \bigcup G_{\alpha_n}$

\item 有限集都是紧集

\item 如果 $E \subset Y \subset X$, 那么 $E$ 可能是 $Y$ 中的开区间而不是 $X$ 的开区间.

\item 假设 $K \subset Y \subset X$. 那么 $K$ 在 $X$ 中是紧的当且仅当它在 $Y$ 中也是紧的.

\item 度量空间的紧子集是闭的

\item 紧集的闭子集也是紧的

\item 如果 $F$ 是闭的且 $K$ 是紧的, 那么 $F \bigcap K$ 是紧的

\item 如果 $\{K_\alpha\}$ 是度量空间 $X$ 的一组紧子集且任意有限个 $\{K_\alpha\}$ 的交集为非空, 那么 $\bigcap K_\alpha$ 也是非空的
\item 如果 $E$ 是紧集 $K$ 的无穷子集, 那么 $E$ 在 $K$ 中存在极限点

\item 如果 $\{I_n\}$ 是 $R^1$ 中的一组区间, 使得 $I_n \supset I_{n+1} (n = 1, 2, 3,\dots)$, 那么 $\bigcap_1^\infty I_n$ 非空

\item k-方格是紧的

\item 对 $R^k$ 中的集合 $E$, 这三个条件等价: (a) $E$ 闭且有界. (b) $E$ 是紧的. (c) $E$ 中的任意无限集在 $E$ 中存在极限点

\item $R^k$ 中任何有界的无限集在 $R^k$ 中有(至少)一个极限点

\item 令 $P$ 为 $R^k$ 内的非空完全集. 那么 $P$ 是不可数的

\item The \textbf{Cantor set} shows that there exist perfect sets in $R^1$ which contain no segment.

\item Two subsets $A$ and $B$ of a metric space $X$ are said to be \textbf{separated} if both $A \cap \bar B$ and $\bar A \cap B$ are empty, i.e., if no point of $A$ lies in the closure of $B$ and no point of $B$ lies in the closure of $A$.

\item A subset $E$ of the real line $R^1$ is connected if and only if it has the following property:  If $x \in E$, $y \in E$, and $x < z < y$, then $z \in E$.
\end{itemize}

\subsection{Chap 3. 数列与级数}
\begin{itemize}
\item A sequence $\{p_n\}$ in a metric space X is said to converge if there is a point $p \in X$ with the following property: For every $\epsilon > 0$ there is an integer $N$ such that $n \geqslant N$ implies that $d(p_n, p) < \epsilon$. (Here $d$ denotes the distance in X.) In this case we also say that $\{p_n\}$ converges to $p$, or that $p$ is the limit of $\{p_n\}$ and we write $p_n \to p$, or $\lim_{n\to \infty} p_n = p$.
\item the set of all points $p_n$ is the range of $\{p_n\}$. The range of a sequence may be a finite set, or it may be infinite. The sequence $\{p_n\}$ is said to be bounded if its range is bounded.
\end{itemize}

\addTODO{以下所有内容}

\subsection{Chap 4. 连续性}

\subsection{Chap 5. 微分法}

\subsection{Chap 6. Riemann-Stieltjes 积分}

\subsection{Chap 7. 函数序列与函数项级数}

\subsection{Chap 8. 一些特殊函数}

\subsection{Chap 9. 多元函数}

\subsection{Chap 10. 微分形式的积分}

\subsection{Chap 11. Lebesgue 理论}
