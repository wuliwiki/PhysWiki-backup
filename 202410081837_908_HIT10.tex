% 哈尔滨工业大学 2010 年 考研 量子力学
% license Usr
% type Note

\textbf{声明}:“该内容来源于网络公开资料,不保证真实性,如有侵权请联系管理员”

\subsection{共50分,每小题10分}
\begin{enumerate}
\item 证明:若$\hat{A}$和$\hat{B}$均为厄米算符,则$i[\hat{A},\hat{B}]$也为厄米算符
\item 设氢原子在$t=0$时出于状态\\\\
求其能量、角动最平方及角动量$Z$分量的的可能取值
\item 若一个算符与角动量算符$\hat{j}$的两个分量对易,则其必与$\hat{j}$的另一个分量对易。
\item 
\item 
\end{enumerate}
\subsection{(15分)}
对于一个系统,力学量算符$\hat{A}$与哈密顿算符$\hat{H}$(不易含时间$t$)不对易,已知$\hat{A}$的两个本征值为$a_1$和$a_2$,相应的本征函数分别为:
$$\psi_1 = \frac{1}{\sqrt{2}} (\varphi_1 + \varphi_2), \quad \psi_1 = \frac{1}{\sqrt{2}} (\varphi_1 - \varphi_2)~$$
其中 $\varphi_1$ 和 $\varphi_2$ 为本征函数, 相应的本征值分别为 $E_1$ 和 $E_2$. 若 $t = 0$ 时, 系统处于 $\psi_1$ 态, 求 $t$ 时刻力学量 $\hat{A}$ 的平均值.
\subsection{(15分)}
设 $E_m$ 为系统哈密顿量的本征值, 相应的本征矢为 $\ket{m}$, $\hat{F}(\vec{r}, \hat{\vec{p}})$ 为一厄米算符. 证明:
\[
\sum^\infty_{n=0}(E_n - E_0) |F_n|^2 =\frac{1}{2}\langle j |[\hat{F}, [\hat{H}, \hat{F}]] |j \rangle ~
\]
\subsection{(15分)}

设
\[
\varphi_1(x) = ae^{-\frac{1}{2}x^2} \quad \text{和} \quad \varphi_2(x) = b(x^2 + Ax + B)e^{-\frac{1}{2}x^2}~
\]
是某粒子一推束缚态的两个定态本征函数,其中 $a, b, \alpha$ 为已知实常数$-\infty < x < +\infty$。求这两个束缚态的能级差,并确定实常数 $A$ 和 $B$。
\subsection{(15分)}
线性谐振子的哈密顿量为 $\hat{H} = (\hat{N} + \frac{1}{2})\hbar\omega$ ,其中 $\hat{N} = \hat{a}^\dagger \hat{a}$,而 $[\hat{a}, \hat{a}^\dagger] = 1$,满足 $[\hat{a}, \hat{a}] = [\hat{a}^\dagger, \hat{a}^\dagger] = 0$。

(1) 设 $m$ 为正整数,证明:
$$[\hat{N}, (\hat{a}^\dagger)^m] = -m(\hat{a}^\dagger)^m~$$ 
$$[\hat{N}, (\hat{a})^m] = m(\hat{a})^m~$$

(2) 若 $\ket{n}$ 是 $\hat{H}$ 的归一化本征矢,$\hat{a}\ket{0} = 0$,$\ket{n} = N_n ( \hat{a}^\dagger )^n\ket{n}$,求因子 $N_n$。

(3) 若外加一微扰,$\hat{H'} = g\hat{a}^\dagger \hat{a}^\dagger \hat{a} \hat{a}$,$g$ 为系数,求能量的一阶近似值。
\subsection{(20分)}

\subsection{(20分)}