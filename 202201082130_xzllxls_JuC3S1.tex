% 变量的定义
% 变量 定义

本文授权转载自郝林的 《Julia 编程基础》. 原文链接:\href{https://github.com/hyper0x/JuliaBasics/blob/master/book/ch03.md}{第3章:变量与常量}.

\subsubsection{3.1 变量的定义}

变量不只能代表所谓的中间结果,而是可以代表任何值.当我们在 REPL 环境中定义一个变量的时候,它就会回显该变量名所代表的值.例如:

\begin{lstlisting}[language=julia]
julia> x = 2020
2020

julia> 
\end{lstlisting}

在这之后,我们也可以输入这个变量名,以此让 REPL 环境回显它代表的那个值:

\begin{lstlisting}[language=julia]
julia> x
2020

julia> 
\end{lstlisting}

然而,当我们输入\verb|y|这个标识符的时候,会发现它无法回显某个值:

\begin{lstlisting}[language=julia]
julia> y
ERROR: UndefVarError: y not defined

julia> 
\end{lstlisting}

这是因为\verb|y|还没有被定义,Julia 并不知道它代表了什么.那什么是定义呢?更确切地说,什么是变量的定义呢?我们在前面说过,变量相当于一个标识符与值的绑定.那么,对这种绑定的描述就是变量的定义.在 Julia 中,变量的定义一般由标识符、赋值符号\verb|=|和值字面量构成.就像我们在前面展示的那样.