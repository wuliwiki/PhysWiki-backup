% 纳什嵌入定理

\pentry{子流形\upref{SubMnf}, 黎曼度量与伪黎曼度量\upref{RiMetr}}

我们知道, 在 $N$ 维欧氏空间中, $n$ 维的子流形自然有一个黎曼度量. 例如, 三维空间中的二维光滑曲面 $M$ 就自然带有诱导出来的黎曼度量, 它在古典微分几何中被称为"第一微分型". 不过我们自然也可以关心反过来的问题: 

\textbf{给定了一个抽象的 $n$ 维微分流形 $M$ 和上面的一个黎曼度量 $g$, 是否存在足够大的 $N$, 使得 $(M,g)$ 能光滑等距嵌入为 $\mathbb{R}^N$ 中的子流形? 也就是说, 是否有某个 $n$ 维光滑子流形恰好就是具有度量 $g$ 的流形 $M$?}
