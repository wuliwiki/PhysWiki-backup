% 狄拉克矩阵
% keys 狄拉克|矩阵|费米子

之前的词条中我们已经介绍过,$\bar \psi\psi$是一个洛仑兹标量.$\bar\psi\gamma^\mu\psi$是一个4-矢量.现在我们来考虑一个更广义的情形$\bar\psi\Gamma\psi$,其中$\Gamma$是任意的$4\times 4$常数矩阵.现在我们的问题是,我们能不能把这个表达式分解成一些项,这些项在洛仑兹群的变换下,具有特定的变换的特性?答案是肯定的.我们可以把$\Gamma$用下面这组基矢进行展开
\begin{table}[ht]
\centering
\caption{$\Gamma$矩阵的基矢}\label{diracm_tab1}
\begin{tabular}{|c|c|}
\hline
$1$ & 1个 \\
\hline
$\gamma^\mu$ & 4个 \\
\hline
$\gamma^{\mu\nu}=\frac{1}{2}[\gamma^\mu,\gamma^\nu]\equiv \gamma^{[\mu}\gamma^{\nu]}\equiv -i\sigma^{\mu\nu}$ & 6个 \\
\hline
$\gamma^{\mu\nu\rho}=\gamma^{[\mu}\gamma^\nu\gamma^{\rho]}$ & 4个 \\
\hline
$\gamma^{\mu\nu\rho\sigma}=\gamma^{[\mu}\gamma^\nu\gamma^\rho\gamma^{\rho]}$ & 1个 \\
\hline
总共 & 16个 \\
\hline
\end{tabular}
\end{table}
\begin{exercise}{写出这样的项在洛仑兹变换下是如何变换的}
\begin{align}\nonumber
\bar\psi\gamma^{\mu\nu}\psi & \rightarrow(\bar\psi\Lambda_{\frac{1}{2}}^{-1})(\frac{1}{2}[\gamma^\mu,\gamma^\nu])(\Lambda_{\frac{1}{2}}\psi)\\
\nonumber
& = \frac{1}{2} \bar\psi (\Lambda_{\frac{1}{2}}^{-1}\gamma^\mu\Lambda_{\frac{1}{2}}\Lambda_{\frac{1}{2}}^{-1}\gamma^\nu\Lambda_{\frac{1}{2}} - \Lambda_{\frac{1}{2}}^{-1}\gamma^\nu\Lambda_{\frac{1}{2}}\Lambda_{\frac{1}{2}}^{-1}\gamma^\mu\Lambda_{\frac{1}{2}})\psi \\
& = \Lambda^\mu{}_\alpha \Lambda^\nu{}_\beta\bar\psi\gamma^{\alpha\beta}\psi~.
\end{align}
\end{exercise}
我们再定义一个$\gamma$矩阵
\begin{equation}
\gamma^5\equiv i \gamma^0 \gamma^1\gamma^2\gamma^3 = -\frac{i}{4!}\epsilon^{\mu\nu\rho\sigma}\gamma_\mu\gamma_\nu\gamma_\rho\gamma_\sigma~.
\end{equation}
所以$\gamma^{\mu\nu\rho\sigma}=-i\epsilon^{\mu\nu\rho\sigma}\gamma^5$, $\gamma^{\mu\nu\rho} = -i\epsilon^{\mu\nu\rho\sigma}\gamma_\sigma\gamma^5$. $\gamma^5$满足的性质总结如下
\begin{align}
(\gamma^5)^\dagger = \gamma^5~, \\
(\gamma^5)^2 = 1~, \\
\{\gamma^5,\gamma^\mu\}=0~. 
\end{align}
从最后一个式子我们可以推出
\begin{equation}
[\gamma^5,S^{\mu\nu}] = 0~.
\end{equation}
因此狄拉克表示一定是可约的.
\begin{equation}
\gamma^5 = \begin{pmatrix}
-1 & 0 \\ 0 & 1 
\end{pmatrix}~.
\end{equation}
只有左手的分量的狄拉克旋量是$\gamma^5$的本征值为$-1$的态.而只有右手的分量的狄拉克旋量是$\gamma^5$的本征值为$1$的态.

现在我们来重写我们的$4\times 4$矩阵
\begin{table}[ht]
\centering
\caption{$\Gamma$矩阵的基矢}\label{diracm_tab2}
\begin{tabular}{|c|c|}
\hline
$1$ &标量& 1个 \\
\hline
$\gamma^\mu$ &矢量& 4个 \\
\hline
$\sigma^{\mu\nu}=\frac{i}{2}[\gamma^\mu,\gamma^\nu]$ &二阶张量& 6个 \\
\hline
$\gamma^{\mu}\gamma^5 $ &赝矢量& 4个 \\
\hline
$\gamma^{5} $ &赝标量& 1个 \\
\hline
总共 && 16个 \\
\hline
\end{tabular}
\end{table}
赝矢量和赝标量中的赝子字的意思是,这些量在洛伦兹变换下像矢量和标量一样变换,只不过变换规则前面多了一个负号.

我们可以写下狄拉克场的两个流
\begin{equation}
j^\mu(x) = \bar\psi(x)\gamma^\mu\psi(x)~, \quad j^{\mu 5}(x) = \bar\psi(x)\gamma^\mu\gamma^5\psi(x)~.
\end{equation}
现在我们来计算这些流的散度
\begin{align}\nonumber
\partial_\mu j^\mu & = (\partial_\mu \bar\psi)\gamma^\mu\psi + \bar\psi\gamma^\mu\partial_\mu\psi \\\nonumber
& = (im \bar\psi) \psi + \bar\psi(-i m \psi)\\
& = 0 ~.
\end{align}
因此,如果$\psi$满足狄拉克方程的话,$j^\mu$总是守恒的.当我们把狄拉克场跟电磁场耦合起来,$j^\mu$就是电流密度.同样地,我们可以计算
\begin{equation}
\partial_\mu j^{\mu 5} = 2 i m \bar\psi \gamma^5 \psi ~.
\end{equation}
如果$m=0$,这个流也是守恒的.因此我们可以定义下面的线性组合
\begin{equation}
j^\mu_L = \bar \psi \gamma^\mu \bigg( \frac{1-\gamma^5}{2} \bigg)\psi ~, \quad j^\mu_R = \bar \psi \gamma^\mu \bigg( \frac{1+\gamma^5}{2} \bigg)\psi ~.
\end{equation}
两个流$j^\mu(x)$和$j^{\mu 5}(x)$是对应于下面两个变换的诺特流
\begin{equation}
\psi(x) \rightarrow e^{i\alpha} \psi(x) ~, \quad {\rm and} \quad \psi(x)\rightarrow e^{i\alpha\gamma^5}\psi(x)~.
\end{equation}
第一个是狄拉克拉式量的对称性.第二个被称为手性变换.诺特定理证明了这种axial的矢量流当且仅当$m=0$的时候,是守恒的.

下面一个恒等式被称为Fierz恒等式,非常有用
\begin{equation}
(\sigma^\mu)_{\alpha\beta}(\sigma_\mu)_{\gamma\delta} = 2 \epsilon_{\alpha\gamma} \epsilon_{\beta\delta}~.
\end{equation}
其中$\alpha$和$\gamma$在$\psi_L$的洛仑兹表示下变换.$\beta$和$\delta$在$\psi_R$的表示下变换.

另一个有用的等式如下
\begin{align}\nonumber
(\bar u_{1R}\sigma^\mu u_{2R})(\bar u_{3R}\sigma_{\mu}u_{4R}) & = 2 \epsilon_{\alpha\gamma} \bar u_{1R\alpha} \bar u_{3R\gamma} \epsilon_{\beta\delta } u_{2 R \beta} u_{4 R \delta }\\
& = - (\bar u_{1R} \sigma^\mu u_{4R})(\bar u_{3R}\sigma_\mu u_{2R})~.
\end{align}
同样的步骤也可以用于$\bar\sigma^\mu$,我们有
\begin{equation}
(\bar u_{1L}\bar \sigma^\mu u_{2L})(\bar u_{3L}\bar \sigma_\mu u_{4L}) = -(\bar u_{1L}\bar \sigma^\mu u_{4L} )(\bar u_{3L}\bar \sigma_\mu u_{2L})~.
\end{equation}
我们可以把Fierz恒等式跟上面的式子结合起来,推出
\begin{equation}
\epsilon_{\alpha\beta}(\sigma^\mu)_{\beta\gamma} = (\bar \sigma^{\mu T})_{\alpha \beta} \epsilon_{\beta\gamma}~.
\end{equation}




