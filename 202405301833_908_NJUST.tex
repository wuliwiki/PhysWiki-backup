% 南京理工大学  2015 量子真题
% license Usr
% type Note

\textbf{声明}:“该内容来源于网络公开资料,不保证真实性,如有侵权请联系管理员”

\section*{简答题}

1. 量子理论实验表明了微观粒子的波粒二象性(至少写出两条)。

2. 库仑定律和把电场定义为力分别为别。

3. 写出量子力学工大基本假设中的任意两个。

4. 量子力学中能量算符和动量算符分别为。

5. 证明关系式 $[\alpha, \beta] = i \hbar$。

\section*{计算题}

1. 有一波函数 $\psi(r, \theta, \varphi) = \frac{1}{\sqrt{4 \pi}} \left( \frac{1}{a_0^3} \right)^{1/2} e^{-r/a_0}$,求 (1) 在点 $r$ 近似体积元 $dr$ 内找到粒子的概率;(2) 在 $r \to r + dr$ 球壳内找到粒子的概率;(3) 在什么位置发现粒子的概率最大。

2. 设在 $H^0$ 表象中,$\hat{H}$ 的矩阵为
$\hat{H} = \begin{pmatrix}E_1^0 & 0 & a \\\\0 & E_2^0 & b \\\\a & b & E_3^0\end{pmatrix}$
其中 $E_0^1 < E_0^2 < E_0^3$。试用微扰论求能量的一级修正。

3. 利用波尔—索末菲量子化条件计算匀场磁场中的回圈运动的电子的可能能量和轨道半径。

(波尔—索末菲量子化条件为 $\oint p dq = (n + \frac{1}{2}) h$,$p, q$ 为广义动量和广义坐标,$n$ 为对的量子数)