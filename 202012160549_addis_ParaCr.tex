% 抛物线坐标系

\begin{issues}
\issueTODO
\end{issues}

\pentry{抛物线的三种定义\upref{Para3}}

\footnote{本文参考 Wikipedia \href{https://en.wikipedia.org/wiki/Parabolic_coordinates}{相关页面}.}抛物线的极坐标方程为(\autoref{Para3_eq1}~\upref{Para3})
\begin{equation}\label{ParaCr_eq1}
r = \frac{\xi}{1 - \cos \theta }
\end{equation}
若选用不同的半通径 $\xi$ ($\xi > 0$), 将得到一系列缩放的抛物线(\autoref{ParaCr_fig1} 中的绿色). 我们也可以把这些抛物线旋转 180 度, 得到
\begin{equation}
r = \frac{\eta}{1 + \cos \theta }
\end{equation}
这时我们把半通径记为 $\eta$ ($\eta > 0$), 当它取不同的值也得到一系列抛物线(\autoref{ParaCr_fig1} 中的红色). 这样, 通过 $\xi, \eta$ 两个坐标, 我们就能确定平面上的唯一一点.

\begin{figure}[ht]
\centering
\includegraphics[width=10cm]{./figures/ParaCr_1.pdf}
\caption{抛物线坐标系, 极轴向上(来自 Wikipedia)} \label{ParaCr_fig1}
\end{figure}

若把这些曲线绕极轴旋转一周变为一系列曲面, 那么我们只需要再指定一个方位角 $\phi$ 就可以用坐标 $\xi, \eta, \phi$ 确定空间中的任意一点.

\subsection{与直角坐标和球坐标的转换}
一般令极轴与直角坐标的 $z$ 轴重合, 则根据定义有
\begin{equation}
\xi = r(1 + \cos\theta) = \sqrt{x^2 + y^2 + z^2} + z
\end{equation}
\begin{equation}
\eta = r(1 - \cos\theta) = \sqrt{x^2 + y^2 + z^2} - z
\end{equation}
\begin{equation}
\phi = \Arctan(y, x)
\end{equation}
其中 $\Arctan$ 见 “四象限 Arctan 函数\upref{Arctan}”. 可以解得
\begin{equation}\label{ParaCr_eq2}
z = \frac{\xi - \eta}{2}
\end{equation}
\begin{equation}\label{ParaCr_eq3}
x = \sqrt{\xi\eta}\cos\phi \qquad
y = \sqrt{\xi\eta}\sin\phi
\end{equation}

\subsection{正交曲线坐标系}
可以证明\autoref{ParaCr_eq1} 中过任意一点的两条坐标曲线都垂直, 即抛物线坐标系是一个正交曲线坐标系:

在某点 $\bvec r$, 延 $\eta$ 曲线的切线方向为 $\pdv*{\bvec r}{\xi}$, 延 $\xi$ 方向曲线的切线方向为 $\pdv*{\bvec r}{\eta}$, 所以只需证明
\begin{equation}
\pdv{\bvec r}{\xi} \vdot \pdv{\bvec r}{\eta} = \bvec 0
\end{equation}
只需分别证明三个分量即可, 使用\autoref{ParaCr_eq2} 和\autoref{ParaCr_eq3} 易证. 证毕.

\subsection{矢量算符}
\begin{equation}
\laplacian = \frac{4}{\xi + \eta} \qty[\pdv{\xi}\qty(\xi\pdv{\xi}) + \pdv{\eta}\qty(\eta\pdv{\eta})] + \frac{1}{\xi\eta}\pdv[2]{\phi}
\end{equation}
\addTODO{未完成, 新开词条}
