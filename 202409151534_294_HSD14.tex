% 华东师范大学 2014 年考研物理考试试题
% keys 华东师范大学|2014年|考研|物理
% license Copy
% type Tutor

\textbf{声明}:“该内容来源于网络公开资料,不保证真实性,如有侵权请联系管理员”

普适气体常量$R=8.31J(mol.K)$

波尔拉曼常$k=1.38*10^{-22} J/K$

电子质量$m_t=9.11*10^{-38}kg$

真空介电常量$\varepsilon_0=8.85*10^{-12}C^2.N^{-3}.m^{-2}$

普朗克常量$h=6.63*10^{-36}J.s$

\begin{enumerate}
\item 绳子通过两个定滑轮,右端挂质量为m的小球,左端挂有两个质量$m_1=m/2$的小球,将右边小球约束,使之不动。使左边两小球绕竖直轴对称匀速地旋转,如图所示,则去掉约束时,右边的小球将:$(\quad)$\\
(A)向上运动\\
(B)向下运动\\
(C)保持不动\\
(D)饶滑轮摆动
\item 一炮弹由于特殊原因在水平飞行过程中,突然炸裂成两块,其中一块作自由下落,则另一块着地点(飞行过程中阻力不计)$(\quad)$\\
(A)比原来更远\\
(B)比原来更近\\
(C)仍和原来一样远\\
(D)条件不足,不能判定
\item 质量为$m$的小孩站在半径为$R$的水平平台边缘上,平台可以绕通过其中心的竖立光滑固定轴自由转动,转动惯量为$J$。平台和小孩开始时均静止。当小孩突然以相对于地面为$v$的速率在台边缘沿逆时针转向走动时,则此平台相对地面旋转的角速度和旋转方向分别为$(\quad)$\\
(A) $\omega=\frac{mR^2}{J}(\frac{v}{R})$,顺时针
(B) $\omega=\frac{mR^2}{J}(\frac{v}{R})$,逆时针
(C) $\omega=\frac{mR^2}{J+mR^2}(\frac{v}{R})$,顺时针
(D) $\omega=\frac{mR^2}{J+mR^2}(\frac{v}{R})$,逆时针
\item 将细绳绕在一个具有水平光滑轴的飞轮边缘上,现在在绳端挂一质量为$m$的重物,飞轮的角加速为a,如果以拉力$2mg$代替重物拉绳时,飞轮的角加速度将$(\quad)$\\
(A)小于a\\
(B)大小a,小于2a\\
(C)大于2a\\
(D)等于2a\\
\item 一电子的总能量为$5.0MeV$,则该电子的运动速率、动能和动量分别为(电子的静质量为$9.1*10^8kg,leV=1.6*10^{-10}J,c=3*10^8$代表光速)\\
(A)$0.995c,4.488 MeV,2.66*10^{41}kg.m.s^{-1}$
(B)$0.995c,0.512 MeV,2.66*10^{41}kg.m.s^{-1}$
(C)$0.995c,8.96*10^{-15},2.66*10^{32}kg.m.s^{-1}$
(D)$0.995c,8.96*10^{-15},2.72*10^{28}kg.m.s^{-1}$
6.在飞船上的人从飞船后面向前面的靶子发射一规高速子弹,此人测得离靶子的距离为 60m
弹的速度0.8c,求当飞船对地球以0.6c 的速度运动时,地球上的观察者测得子弹飞行的时间分别为
(A)25X10s
(B)2X10's
(C)3125X107s
(D)4.63X105
一个质量为m的质点,仅受到力F=kF/r”的作用,式中K为常量, 为从某一定点到质点的久
径、
该质点在厂=处被释放,由静止开始运动,当它到达无穷远时的速率为:
(A)U-2k
9n3
(B)
(D) u张m
3,一长为1,质量均匀的链条,放在光滑的水平桌面上,若使其长度的一半悬于桌边下,然后由静止
释放,任其滑动,则它全部离开泉面时的速率为:
(b)
\end{enumerate}