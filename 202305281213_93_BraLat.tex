% 布拉伐格子、基元与原胞
% 布拉伐格子、基元与原胞

晶体最主要的特征是具有周期性重复的规则结果,可以看成是由不同原子组成的最小单元以某种方式在空间周期性重复平移的结果。这里有两个主要方面:
一是重复单元,二是重复排列的方式。只要确定了这两点,我们就可以得到整个晶体。

早在1866年,布拉伐首先提出用\textbf{晶格点阵}来表示晶体中原子周期排列的方式,用\textbf{基元}表示最小重复单元。关于晶体、基元与晶格点阵的关系,我们可以这样理解:晶体 = 基元 + 晶格点阵。晶格点阵又称为\textbf{布拉伐格子}。
% 插入图片

通过布拉伐格子与基元的简化,我们抓住了晶体结构中最主要的特征————即晶体中原子周期性的排列,体现了其所具有的平移对称性。布拉伐格子是一个无限延展的理想点阵,忽略实际晶体中的缺陷,同时,它\textbf{不是真实的原子},是抽象的\textbf{点},是基元所在的位置。

打个比方,见过超市里的盒装鸡蛋吗?一个个鸡蛋整齐规律地垒在小坑里形成一整盒,相当于一个个“基元”处在规律的“布拉伐格子”上。