% 量子霍尔效应(综述)
% license CCBYSA3
% type Wiki

(本文根据 CC-BY-SA 协议转载自原搜狗科学百科对英文维基百科的翻译)

\begin{figure}[ht]
\centering
\includegraphics[width=6cm]{./figures/9144033c12386920.png}
\caption{概念图} \label{fig_LZHRXY_3}
\end{figure}

\textbf{量子霍尔效应 (或整数量子霍尔效应)}是霍尔效应的量子力学版本。一般看作是整数量子霍尔效应和分数量子霍尔效应的统称。在低温强磁场的二维电子系统中可以观察到这个效应,霍尔电导$\sigma$在发生量子霍尔跃相变后,被量子化。

\begin{equation}
\sigma = \frac{I_{\text{channel}}}{V_{\text{Hall}}} = \nu \frac{e^2}{h}~,
\end{equation}

其中Ichannel 是通道电流,$V_{Hall}$ 是霍尔电压, e 是电荷和h是普朗克常数。前因子$\nu$ 称为填充因子,可以取整数值($\nu =\text{1, 2, 3,…}$),也可以取分数值( $\nu=\text{13, 25, 37, 23, 35, 15, 29, 313, 52, 125,…}$)。量子霍尔效应根据$\nu$ 的取值不同相应地称为整数量子霍尔效应或者分数量子霍尔效应。

整数量子霍尔效应的显著特征是,随着电子密度的变化,量子霍尔效应的量子化(即霍尔平台)会持续存在。由于电子密度在费米能级处于一个光洁的谱隙时保持不变,这种情况对应于费米能级是有限态密度能量,并且这些态是局域的。

分数量子霍尔效应更复杂,因为它从根本上依赖于电子-电子的相互作用。分数量子霍尔效应也可以被理解为整数量子霍尔效应,但是并不依赖电子之间的相互作用,而是电荷-磁通复合物的复合费米子相互作用。1988年,有人提出没有朗道能级的量子霍尔效应。[1] 这种量子霍尔效应被称为量子反常霍尔(QAH)效应。还有一个量子自旋霍尔效应的新概念,它与量子霍尔效应类似,区别在于由自旋电流流动代替了电荷电流。[2]

\subsection{应用}
霍尔电导的量子化是非常精确的。霍尔电导的实际测量值是$e^2h$ 的整数倍或分数倍,接近十亿分之一。这种现象,被称为 精确量子化,已被证明是规范不变性的巧妙表现形式。 它允许定义一个新的实用标准的电阻,基于电阻量子给出的冯克利青常数$R_K = he^2 = 25812.807557(18)\Omega$。[3] 这是以精确量子化的发现者克劳斯·冯·克里辛的名字命名的。自1990年以来,世界范围内的电阻校准都采用固定的常规值$R_{k-90}$ 。[4] 2018年11月16日,国际度量衡大会第26次会议决定确定$h$(普朗克常数)和$e$(基本电荷)的值,弃用了传统值。[5] 量子霍尔效应还给出了量子电动力学中非常重要的精细结构常数的确定值。

\subsection{历史}
霍尔电导的整数量子化最初是由安藤忠雄、松本和Uemura在1975年根据他们自己都不相信的近似计算预测的。[6] 随后,几名研究人员在场效应管的倒置层上进行的实验中观察到了这种效应 。[7] 在1980年在格勒诺布尔的高磁场实验室工作,负责迈克尔·佩珀(Michael Pepper)和格哈德·多达(Gerhard Dorda)开发的硅基样品的克劳斯·冯·克利青(Klaus von Klitzing)意外地发现霍尔电导率被精确地量子化了。[8] 由于这一发现,冯·克利青获得了1985年的 诺贝尔物理学奖。罗伯特·劳克林随后发现了精确量子化和规范不变性之间的联系,他把量子化电导率与索利斯电荷泵的量子化电荷输运联系起来。[9][9] 虽然许多其他半导体材料也可以使用,但是多数整数量子霍尔实验都是在砷化镓异质结构材料上开展的。2007年,在室温[10] 条件下, 石墨烯中氧化锌镁ZnO–MgxZn1−xO展现了整数量子霍尔效应。[11]

\subsection{整数量子霍尔效应——朗道能级}
\begin{figure}[ht]
\centering
\includegraphics[width=10cm]{./figures/7646db978854eb6b.png}
\caption\label{fig_LZHRXY_1}
\end{figure}
在二维空间中,当经典电子受到磁场作用时,它们按照圆形回旋轨道运动。当系统被量子化时,这些轨道被量子化。量子轨道的能级呈现离散值:

\subsection{数学}
\begin{figure}[ht]
\centering
\includegraphics[width=8cm]{./figures/d0957ea6719aa82a.png}
\caption{霍夫斯塔德蝴蝶} \label{fig_LZHRXY_2}
\end{figure}
霍尔效应中出现的整数是 拓扑量子数。他们在数学上被认为是第一个 陈数 与贝里相位密切相关。在这种背景下,一个引人注目的模型是阿兹贝尔-哈珀-霍夫斯塔德模型,其量子相图是 霍夫斯塔德蝴蝶 如图所示。纵轴是 磁场强度 横轴是 化学势,也就是确定了电子密度。颜色代表整数霍尔电导。暖色代表正整数,冷色代表负整数。然而,请注意,量子化霍尔电导的这些区域中的态密度为零;因此,它们不能产生实验中观察到的平台结构。相图是分形的,各种尺度的结构都包含其中。在图中有一个明显的 自相似性。图中无序的情况,来源于实验中的平台结构部分,使得这张图看起来非常不同,因为其中的分形结构没有得到展现。

考虑到物理机制,杂质和/或特定的状态(例如边缘电流)对于“整数”和“分数”效应都很重要。此外,库仑相互作用在 分数量子霍尔效应中也起到了重要作用。整数量子霍尔效应和分数量子霍尔效应之间的强相似性可以用具有偶数磁通量量子数的电子趋向于形成束缚态来解释,也就是 复合费米子。

\subsection{冯·克利青常数的玻尔原子解释}
用单原子的玻尔模型考虑成单电子霍尔效应时,可以得到冯·克利青常数的值。此时电子在圆形轨道上做回旋运动 ,在离心力和由横向感应电压引起的 洛伦兹力的作用下保持运动的平衡。 可以把玻尔原子中的库仑电势差看作感应原子的霍尔电压,电子在圆周上的周期性运动看作是霍尔电流。将单原子霍尔电流定义为电子$e$以角频率$\omega$进行开普勒旋转

感应霍尔电压作为电子轨道上一点和无穷远处的氢核库仑势之间的差:

$U = V_C(\infty) - V_C(r) = 0 - V_C(r) = \frac{e}{4 \pi \epsilon_0 r}$

我们重新定义量子化的玻尔轨道霍尔电阻,以冯·克利青常数的形式表达:
\begin{equation}
\mathbf{R}_{\text{Bohr}}(n) = \frac{U}{I} = n \frac{h}{e^2}~
\end{equation}
其中玻尔原子是线性的,但不是以 $n$ 倒数的形式。

\subsection{相对论变形}
整数量子霍尔效应的相对论性例子有量子自旋霍尔效应,参见格点规范理论。[12][13]

\subsection{参考文献}
[1]
^F. D. M. Haldane (1988). "Model for a Quantum Hall Effect without Landau Levels: Condensed-Matter Realization of the 'Parity Anomaly'". Physical Review Letters. 61 (18): 2015–2018. Bibcode:1988PhRvL..61.2015H. doi:10.1103/PhysRevLett.61.2015. PMID 10038961..

[2]
^Ezawa, Zyun F. (2013). Quantum Hall Effects: Recent Theoretical and Experimental Developments (3rd ed.). World Scientific. ISBN 978-981-4360-75-3..

[3]
^A. Tzalenchuk; S. Lara-Avila; A. Kalaboukhov; S. Paolillo; M. Syväjärvi; R. Yakimova; O. Kazakova; T. J. B. M. Janssen; V. Fal'ko; S. Kubatkin (2010). "Towards a quantum resistance standard based on epitaxial graphene". Nature Nanotechnology. 5 (3): 186–189. arXiv:0909.1220. Bibcode:2010NatNa...5..186T. doi:10.1038/nnano.2009.474. PMID 20081845..

[4]
^"conventional value of von Klitzing constant". NIST..

[5]
^"26th CGPM Resolutions" (PDF). BIPM..

[6]
^T. Ando; Y. Matsumoto; Y. Uemura (1975). "Theory of Hall effect in a two-dimensional electron system". J. Phys. Soc. Jpn. 39 (2): 279–288. Bibcode:1975JPSJ...39..279A. doi:10.1143/JPSJ.39.279..

[7]
^J. Wakabayashi; S. Kawaji (1978). "Hall effect in silicon MOS inversion layers under strong magnetic fields". J. Phys. Soc. Jpn. 44 (6): 1839. Bibcode:1978JPSJ...44.1839W. doi:10.1143/JPSJ.44.1839..

[8]
^K. v. Klitzing; G. Dorda; M. Pepper (1980). "New method for high-accuracy determination of the fine-structure constant based on quantized Hall resistance". Phys. Rev. Lett. 45 (6): 494–497. Bibcode:1980PhRvL..45..494K. doi:10.1103/PhysRevLett.45.494..

[9]
^R. B. Laughlin (1981). "Quantized Hall conductivity in two dimensions". Phys. Rev. B. 23 (10): 5632–5633. Bibcode:1981PhRvB..23.5632L. doi:10.1103/PhysRevB.23.5632..

[10]
^K. S. Novoselov; Z. Jiang; Y. Zhang; S. V. Morozov; H. L. Stormer; U. Zeitler; J. C. Maan; G. S. Boebinger; P. Kim; A. K. Geim (2007). "Room-temperature quantum Hall effect in graphene". Science. 315 (5817): 1379. arXiv:cond-mat/0702408. Bibcode:2007Sci...315.1379N. doi:10.1126/science.1137201. PMID 17303717..

[11]
^Tsukazaki, A.; Ohtomo, A.; Kita, T.; Ohno, Y.; Ohno, H.; Kawasaki, M. (2007). "Quantum Hall effect in polar oxide heterostructures". Science. 315 (5817): 1388–91. Bibcode:2007Sci...315.1388T. doi:10.1126/science.1137430. PMID 17255474..

[12]
^D. B. Kaplan (1992). "A Method for simulating chiral fermions on the lattice". Physics Letters. B288: 342–347. arXiv:hep-lat/9206013. doi:10.1016/0370-2693(92)91112-M..

[13]
^M. F. L. Golterman; K. Jansen; D. B. Kaplan (1993). "Chern-Simons currents and chiral fermions on the lattice". Physics Letters. B301: 219–223. arXiv:hep-lat/9209003. doi:10.1016/0370-2693(93)90692-B..