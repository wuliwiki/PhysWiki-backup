% Bertrand 定理
% keys Bertrand 定理|Bertrand 猜想|Bertrand 假设
% license Usr
% type Tutor

\pentry{数论函数 theta 与 psi 的阶\nref{nod_tpont}}{nod_61f5}

\begin{theorem}{Bertrand 定理}
对于足够大的 $n$,那么至少存在一个素数 $p$ 使得
\begin{equation}\label{eq_bertTh_1}
n < p \le 2n ~,
\end{equation}
或者等价地说,若 $p_m$ 为第 $m$ 个素数,则对于每个足够大的 $m$ 都有
\begin{equation}
p_{m+1} < 2 p_m ~.
\end{equation}

\end{theorem}

\textbf{证明}:这两个命题是等价的。我们考虑\autoref{eq_bertTh_1} 的证明。我们取足够大的 $n$,不妨令 $n > 2^9 = 512$,假设没有满足 $n < p \le 2n$ 的素数存在,沿用数论函数 theta 与 psi 的阶\upref{tpont}中\autoref{eq_tpont_4}~\upref{tpont}的记号,设 $p$ 是 $N$ 的一个素因子,就有 $\alpha_p \ge 1$,而由于假设,$p \ge n$,若 $\frac 23 n < p \le n$ 则
\begin{equation}
2p \le 2n < 3p, ~ p^2  > \frac 49 n^2 > 2n~,
\end{equation}
从而 $\alpha_p$ 可以表示为
\begin{equation}
\alpha_p = \left\lfloor \frac{2n}p \right\rfloor - 2 \left \lfloor \frac np \right \rfloor = 2 - 2 = 0 ~,
\end{equation}
从而对于 $N$ 的每个素因子 $p$ 都有 $p \le \frac23 n$。利用\autoref{cor_tpont_1}~\upref{tpont}可以得到
\begin{equation}\label{eq_bertTh_2}
\sum_{p | N} \ln p \le \sum_{p\le \frac 23 n} \ln p = \vartheta(\frac 23 n) \le \frac 43 n \ln 2 ~.
\end{equation}
若 $\alpha_p \ge 2$ 则利用\autoref{eq_tpont_5}~\upref{tpont}可以得到
\begin{equation}
2 \ln p \le \alpha_p \ln p \le \ln (2n), ~ p \le \sqrt{2n } ~,
\end{equation}
因此最多有 $\sqrt{2n}$ 个这样的 $p$,从而
\begin{equation}
\sum_{\alpha_p \ge 2} \alpha_p \ln p \le \sqrt{2n} \ln(2n) ~,
\end{equation}
而利用\autoref{eq_bertTh_2} 可以得到
\begin{equation}
\begin{aligned}
\ln N &\le \sum_{\alpha_p = 1} \ln p + \sum_{\alpha_p \ge 2} \alpha_p \ln p ~\\
& \le \sum_{p | N} \ln p + \sqrt{2n} \ln(2n) ~\\
& \le \frac43 n \ln 2 + \sqrt{2n} \ln(2n) ~.
\end{aligned}
\end{equation}



