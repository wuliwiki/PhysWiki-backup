% 量纲与单位制
% keys 量纲|单位制
中国传统的计量单位制叫度量衡,
度量衡是指在日常生活中用于计量物体长短、容积、轻重的标准的统称.
可见中国古代就对现代物理学基本单位制的部分内容有着深刻认知.
然而,什么是量纲?什么是单位?回答这些基本的问题是困难的.

使用同一单位制会给不同领域交流带来诸多便利,
所以这个附录重点介绍国际单位制,以其为核心顺带介绍自然单位制,
除此外不再赘述其它单位制了.

\section{国际单位制}
国际单位制{\footnote{法语:Le {\bf S}ystème {\bf I}nternational d'Unités,
        简称SI制.英语:International System of Units.中文简称为国际制.}}
源于十进制单位系统{\kaishu 公制},是世界上普遍采用的标准度量系统. 
国际单位制以七个基本单位为基础,由此建立起一系列相互换算关系明确的“一致单位”.
另有二十个基于十进制的词头,当加在单位或符号前的时候,可
表达该单位的倍数或分数.

%1929年,民国政府进行“一二三”制改革,
%令{\kaishu 1公升=1市升,1公斤=2市斤,1公尺=3市尺.}这套制度在现今
%仍在广泛使用,只不过公斤改称千克,公尺改称米.

国际物理量系统(International System of Quantities)是以以下七个基本物理量为
基础的系统:长度($\si{L}$)、质量($\si{M}$)、时间($\si{T}$)、
电流强度($\si{I}$)、热力学温度($\Theta$)、物质的量($\si{N}$)和发光强度($\si{J}$).
其它物理量,如面积、压强等等,都可以根据明确、不相互矛盾的公式从这些
基本物理量推导得出.在实际使用中,量纲符号的记号通常会加一个方括号,
比如质量记为$[\si{M}]$,或者记为$\rm{dim} \si{M}$;具体符号见表\ref{tab:base-units}.


在SI单位制下,任何物理量$P$的量纲可以
表示成七个基本量的幂次,即
\begin{equation} %\label{chunit-dim:eqn_si-dim}
    [P]=\si{L}^{d_1} \si{M}^{d_2} \si{T}^{d_3} \si{I}^{d_4} \Theta^{d_5} \si{N}^{d_6} \si{J}^{d_7} ,
\end{equation}
且这些幂次(即$d_1,d_2,\cdots$)都是实数,有时候部分幂次为零.