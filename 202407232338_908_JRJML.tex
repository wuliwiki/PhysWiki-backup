% 加尔加梅勒
% license CCBYSA3
% type Wiki

(本文根据 CC-BY-SA 协议转载自原搜狗科学百科对英文维基百科的翻译)

\textbf{Gargamelle}是1970年到1979年在欧洲核子研究中心(CERN)运行的一种重液体气泡室探测器。它被设计用来探测中微子和反中微子,这些中微子和反中微子是在1970年至1976年由质子同步加速器(\textbf{著名图象处理软件})发出的光束产生的,因此探测器被移到超级质子同步加速器(\textbf{SPS})上。[1] 1979年,由于在气泡室发现了一个不可修复的裂缝,因此探测器停止使用了。它目前是在欧洲核子研究中心微观展览的那一部分,对公众开放。

加尔加梅勒以发现中性电流的实验闻名于世。1973年7月提出的中性线电流是Z0 玻色子存在的第一个实验表明,因此这意味着向验证弱电理论迈出的重要一步。

加尔加梅勒既可以指气泡室探测器本身,也可以指同名的高能物理学实验。这个名字来源于16世纪弗朗索瓦·拉伯雷的一部小说《巨人和潘塔格鲁的生活》,其中女巨人加尔加梅勒是巨人的母亲。[1]

\subsection{背景}
在20世纪60年代的一系列独立作品中,谢尔登·格拉秀,史蒂芬·温伯格和阿卜杜勒·萨拉姆提出了一种理论,即统一了基本粒子之间的电磁和弱相互作用——弱电理论,因此一起获得了分享了1979年的诺贝尔物理学奖。[2] 他们的理论预言了$W^\pm$和$Z^0$玻色子作为弱相互作用的传播者。$W^\pm$ 玻色子带有正电荷$^{+}$)或负数($W^−$,而$Z^0$没有。当交换一个$Z^0$玻色子的转移动力,旋转,和活力,但是留下粒子的量子数的性能未受影响—如充电,风味,重子数, 轻子数等等。因为没有电荷转移,所$Z^0$被称为“中性线电流”。中性电流是弱电理论的预测。

1960年,梅尔文·施瓦茨提出了一种产生高能中微子束的方法。[3] 1962年,施瓦茨等人在布鲁克海文进行了了一项实验,这证明了μ介子和电子中微子的存在。施瓦茨因为这个发现获得了1988年的诺贝尔物理学奖。[4] 在施瓦茨的想法之前,弱相互作用只在基本粒子的衰变中被研究过,特别是奇怪的粒子。使用这些新中微子束大大增加了研究弱相互作用的能量。加尔加梅勒是第一批利用中微子束进行实验的人之一,中微子束是由粒子系统的质子束产生的。

气泡室只是一个装满过热液体的容器。带电粒子穿过腔室会留下电离轨道,液体在电离轨道周围蒸发,形成微小的气泡。整个电离室受到恒定磁场的作用,导致带电粒子的轨迹弯曲。曲率半径与粒子的动量成正比。这些轨迹被拍摄下来,通过研究这些轨迹,人们可以了解到所探测到的粒子的性质。因为中微子没有电荷,所以穿过伽格米尔气泡室的中微子束没有在探测器中留下任何轨迹。因此,通过观察中微子与物质成分相互作用产生的粒子,可以检测到与中微子的相互作用。中微子的横截面非常小,这表示中微子交互的可能性非常小。虽然气泡室通常充满液体氢,但加尔加梅勒却被灌满了一种重液体——CBrF3(氟利昂)——这增加了看到中微子相互作用的可能性。[1]