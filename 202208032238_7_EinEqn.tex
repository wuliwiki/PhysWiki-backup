% 爱因斯坦场方程
% keys 引力场|引力|gravity|场方程|Ricci张量|测地线|geodesic|广义相对论|相对论|relativity|时空|spacetime|弯曲|曲率

\pentry{引力的弱场近似\upref{WeakG},尘埃云的能动张量\upref{SRFld}}

本节规定度量张量的号差是$-2$.号差不同会导致$g^{\mu\nu}T_{\mu\nu}=\pm T_{00}$(\autoref{EinEqn_eq10} )中正负号的选择不同,进而导致\autoref{EinEqn_eq6} 和\autoref{EinEqn_eq9} 的正负号选择不同.

\subsection{爱因斯坦张量}

我们直接介绍一个有用的性质,在稍后猜测爱因斯坦场方程的时候我们自然会讨论到它的用处.

\begin{definition}{Ricci标量曲率}
给定流形上的Ricci曲率场 $R_{ij}$,则有光滑函数 $R=g^{ij}R_{ij}$,称之为流形上的\textbf{标量曲率(scalar curvature)}.
\end{definition}

考虑黎曼曲率张量的第二Bianchi恒等式\autoref{RicciC_eq8}~\upref{RicciC}:
\begin{equation}
\nabla_\lambda R_{\mu\nu\rho\sigma}+\nabla_\rho R_{\mu\nu\sigma\lambda}+\nabla_\sigma R_{\mu\nu\lambda\rho}=0
\end{equation}

等式两端同时乘以 $g^{\nu\sigma}g^{\mu\lambda}$ 后,再考虑到联络对度量的相容性\footnote{即 $\nabla_ag_{ij}=0$.},得
\begin{equation}
\begin{aligned}
0&=g^{\nu\sigma}g^{\mu\lambda}\qty(\nabla_\lambda R_{\mu\nu\rho\sigma}+\nabla_\rho R_{\mu\nu\sigma\lambda}+\nabla_\sigma R_{\mu\nu\lambda\rho})\\
&=g^{\nu\sigma}\qty(\nabla^\mu R_{\mu\nu\rho\sigma} - \nabla_\rho R_{\mu\nu\lambda\sigma}g^{\mu\lambda} + \nabla_\sigma R_{\mu\nu\lambda\rho}g^{\mu\lambda})\\
&=g^{\nu\sigma}\qty(\nabla^\mu R_{\nu\mu\sigma\rho} - \nabla_\rho R_{\nu\sigma} + \nabla_\sigma R_{\nu\rho})\\
&=\nabla^\mu R_{\nu\mu\sigma\rho}g^{\nu\sigma} - \nabla_\rho R_{\nu\sigma}g^{\nu\sigma} + \nabla_\sigma R_{\nu\rho}g^{\nu\sigma}\\
&=\nabla^\mu R_{\mu\rho} - \nabla_\rho R + \nabla^\nu R_{\nu\rho}\\
&=2\nabla^\mu R_{\mu\rho} - \nabla_\rho R
\end{aligned}
\end{equation}

因此
\begin{equation}\label{EinEqn_eq1}
\nabla^\mu R_{\mu\rho}=\frac{1}{2}\nabla_\rho R
\end{equation}

% ,得到\footnote{注意负号的来源:$g^{\nu\sigma}g^{\mu\lambda}R_{\mu\nu\sigma\lambda}=-g^{\nu\sigma}g^{\mu\lambda}R_{\nu\mu\sigma\lambda}=-g^{\mu\lambda}R_{\mu\lambda}=R$}:
% \begin{equation}
% \begin{aligned}
% 0&=g^{\nu\sigma}g^{\mu\lambda}(\nabla_\lambda R_{\mu\nu\rho\sigma}+\nabla_\rho R_{\mu\nu\sigma\lambda}+\nabla_\sigma R_{\mu\nu\lambda\rho})\\
% &=\nabla^\mu R_{\mu\rho}-\nabla_\rho R+\nabla^{\nu}R_{\nu\rho}
% \end{aligned}
% \end{equation}

% 也就是说
% \begin{equation}
% \nabla^\mu R_{\rho\mu}=\frac{1}{2}\nabla_\rho R
% \end{equation}

这么一来,如果我们定义一个张量
\begin{equation}
G_{\mu\nu}=R_{\mu\nu}-\frac{1}{2}Rg_{\mu\nu}
\end{equation}
那么\autoref{EinEqn_eq1} 就可以写为
\begin{equation}\label{EinEqn_eq2}
\nabla^\mu G_{\mu\nu}=0
\end{equation}

这个张量 $G_{\mu\nu}$ 就被称为\textbf{爱因斯坦张量(Einstein tensor)},由于度量和Ricci张量的对称性,它也是对称的.\autoref{EinEqn_eq2} 就是我们稍后要用到的有用性质.



% \subsection{能动张量}

% 爱因斯坦场方程的引出过程中,我们考虑的是最简单的宏观模型,即\textbf{尘埃云的能动张量}\upref{SRFld}中所介绍的\textbf{理想流体}.对于任何物质,其四动量的各分量,都会随着参考系的不同而有不同取值,因此这些量只能是流形上某种量在具体坐标系中的坐标分量而已.这样一来,在流形上描述能量质量分布的量,就不能是简单的光滑函数,或者说标量场,而只能是更高阶的张量场.能描述理想流体四动量分布的张量,可以是四动量本身,也可以是能动张量,而我们会选择能动张量,这样才能和Ricci曲率张量的阶数吻合.

%能动张量的讨论尚未确定最终流程




\subsection{爱因斯坦场方程}

\textbf{引力的弱场近似}\upref{WeakG}一节中,我们看到了作为平直时空的微小扰动,带曲率时空中的测地线确实能描述稳定、低速、弱场近似下的引力效应.我们现在希望利用这个原则,将它推广到任意情况下的时空中去.

曲率是引力的体现,而引力是由物质产生的.牛顿引力论中描述物质的引力效应使用的是物质的质量,在牛顿理论中这是时空中的一个光滑函数.从狭义相对论中我们就知道,描述物质分布时统一的、无视坐标系选择的量,应该是四动量和能动张量.那么我们在推测场方程的时候,应该用四动量还是能动张量呢?四动量是 $(1, 0)$ 型张量,能动张量是 $(0, 2)$ 型的,前者的类型是无法和任何曲率张量匹配的.因此我们选择用能动张量来描述物质分布.

初步的猜测是,将Ricci张量和能动张量匹配,得到如下场方程:
\begin{equation}\label{EinEqn_eq3}
R_{\mu\nu}=KT_{\mu\nu}
\end{equation}
其中 $R_{\mu\nu}$ 是Ricci张量场,$T_{\mu\nu}$ 是物质分布的能动张量场,$K$ 是常数.

但这个式子有个问题,它不满足四动量守恒假设.四动量守恒假设体现为 $\nabla^\mu T_{\mu\nu}=0$,而如果\autoref{EinEqn_eq3} 要满足四动量守恒,就只有 $K=0$ 一种情况,整个式子就毫无意义了.因此,我们根据\autoref{EinEqn_eq2} 做一点小小的修正,得到:
\begin{equation}
G_{\mu\nu}=KT_{\mu\nu}
\end{equation}
这就是我们所\textbf{猜想}的,曲率依赖于物质分布的方式.



\subsubsection{计算常数 $K$}


最后一步就是要得到这个常数 $K$.思路很简单:计算 $K$ 的目的是确定引力如何受物质影响,那么我们就从牛顿引力论中引力产生的规律入手.

取稳定、低速、弱场近似,有牛顿引力方程:
\begin{equation}\label{EinEqn_eq4}
\nabla^2\Phi=4\pi G\rho
\end{equation}
其中 $G$ 是牛顿万有引力常数,$\rho$ 是物质质量密度在牛顿时空中的分布,是个光滑函数.

考虑到 $h_{00}=-2\Phi$,$T_{00}=\rho$,\autoref{EinEqn_eq4} 意味着
\begin{equation}\label{EinEqn_eq7}
\nabla^2h_{00}=-8\pi G T_{00}
\end{equation}



如果我们能用 $h_{00}$ 计算出 $R_{00}$,那么再结合\autoref{EinEqn_eq7} 就可以确定 $K$ 了.由于曲率张量可以用Christoffel符号计算出来,而Christoffel符号可以由度量张量计算出来,因此我们按以下\autoref{EinEqn_eq8} 到\autoref{EinEqn_eq5} 的思路进行.

在稳定、低速、弱场近似下,$T_{00}=\rho$,而 $h_{00}=-2\Phi$\footnote{见\textbf{引力的弱场近似}\upref{WeakG}最后的讨论.}.度量张量为 $g_{\mu\nu}=\eta_{\mu\nu}+h_{\mu\nu}$,其中 $h_{\mu\nu}\ll 1$.考虑到低速近似下,流体压强很小,因此 
\begin{equation}\label{EinEqn_eq10}
g^{\mu\nu}T_{\mu\nu}:=T=g^{00}T_{00}=T_{00}
\end{equation}

特别要注意,\textbf{稳定}假设:$\partial_0h_{\mu\nu}=0$.


根据\autoref{CrstfS_eq3}~\upref{CrstfS},誊抄如下:
\begin{equation}\label{EinEqn_eq8}
\Gamma^{r}_{ij}=\frac{1}{2}g^{kr}(\partial_ig_{jk}+\partial_jg_{ki}-\partial_kg_{ij})
\end{equation}
稳定、低速、弱场近似下的Christoffel符号为:
\begin{equation}
\Gamma^{r}_{ij}=\frac{1}{2}\eta^{kr}(\partial_ih_{jk}+\partial_jh_{ki}-\partial_kh_{ij})
\end{equation}
再代入\textbf{曲率张量场}\upref{RicciC}中的\autoref{RicciC_eq9}~\upref{RicciC},誊抄如下:
\begin{equation}
R_{jk}=R_{kj}=\partial_i\Gamma^i_{jk}-\partial_j\Gamma^{i}_{ik}+\Gamma^s_{jk}\Gamma^i_{is}-\Gamma^s_{ik}\Gamma^i_{js}
\end{equation}
由于 $h_{\mu\nu}$ 很小,我们忽略掉高阶项 $\Gamma^s_{jk}\Gamma^i_{is}-\Gamma^s_{ik}\Gamma^i_{js}$,得:
\begin{equation}
\begin{aligned}
R_{\mu\nu}=&\partial_\lambda\Gamma^\lambda_{\mu\nu}-\partial_{\mu}\Gamma^\lambda_{\lambda\nu}\\
=&\frac{1}{2}\eta^{\lambda\alpha}[\partial_\lambda(\partial_\mu h_{\nu \alpha}+\partial_\nu h_{\alpha\mu}-\partial_\alpha h_{\mu\nu})\\
&-\partial_\mu(\partial_\lambda h_{\nu\alpha}+\partial_\nu h_{\alpha\lambda}-\partial_\alpha h_{\lambda\nu})]\\
\end{aligned}
\end{equation}

因此
\begin{equation}\label{EinEqn_eq5}
\begin{aligned}
R_{00}=&\frac{1}{2}\eta^{\lambda\alpha}[\partial_\lambda(\partial_0 h_{0 \alpha}+\partial_0 h_{\alpha0}-\partial_\alpha h_{00})\\
&-\partial_0(\partial_\lambda h_{0\alpha}+\partial_0 h_{\alpha\lambda}-\partial_\alpha h_{\lambda0})]\\
=&-\frac{1}{2}\eta^{\lambda\alpha}\partial_\lambda\partial_\alpha h_{00}\\
=&-\frac{1}{2}\nabla^2 h_{00}\\
=&4\pi G\rho\\
=&4\pi GT_{00}
\end{aligned}
\end{equation}

注意,这里的 $\nabla^2$ 是三维空间中的Labpace算子,没有时间项,因为\textbf{稳定}假设.


到这里\textbf{还差一步},我们得出的是 $R_{00}$ 和 $T_{00}$ 的关系,但是方程中使用的是 $G_{\mu\nu}=R_{\mu\nu}-\frac{1}{2}Rg_{\mu\nu}$.由于 $R=g^{\mu\nu}R_{\mu\nu}$,所以我们可以通过两边都乘以 $g^{\mu\nu}$ 来处理,这时候就要用到 $g^{\mu\nu}T_{\mu\nu}=T_{00}$ 的假设了.

$R_{\mu\nu}-\frac{1}{2}Rg_{\mu\nu}=KT_{\mu\nu}$ 两边同乘以 $g^{\mu\nu}$\footnote{其中 $g^{\mu\nu}g_{\mu\nu}=1^2+(-1)^2+(-1)^2+(-1)^2=4$.},得到 $R=-KT$.因此,我们有:$R_{00}+\frac{1}{2}KTg_{00}=KT_{00}$.



整理得:
\begin{equation}\label{EinEqn_eq6}
R_{00}=\frac{1}{2}KT_{00}
\end{equation}

联立\autoref{EinEqn_eq5} 和\autoref{EinEqn_eq6} ,我们得到 $K=8\pi G$.

\subsubsection{最终的场方程}

由以上讨论,计算出了 $K$ 以后,我们终于可以把爱因斯坦场方程写下来了:

\begin{equation}\label{EinEqn_eq9}
R_{\mu\nu}-\frac{1}{2}Rg_{\mu\nu}=8\pi GT_{\mu\nu}
\end{equation}



如果选用$+2$的号差,那么\autoref{EinEqn_eq10} 应为
\begin{equation}
g^{\mu\nu}T_{\mu\nu}:=T=g^{00}T_{00}=-T_{00}
\end{equation}

从而 应为











