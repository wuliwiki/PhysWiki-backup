% 浙江大学 2006 年 考研 量子力学
% license Usr
% type Note

\textbf{声明}:“该内容来源于网络公开资料,不保证真实性,如有侵权请联系管理员”

\subsection{第一题(50 分)简答题:}
(1) 从坐标与动量算符的对易关系($[\hat{x}, \hat{p}] = i\hbar$ 等)推出角动量算符与动量算符的对易关系。

(2) 请用泡利矩阵 $\sigma^x = \begin{pmatrix} 0 & 1 \\\\ 1 & 0 \end{pmatrix}$,$\sigma^y = \begin{pmatrix} 0 & -i \\\\ i & 0 \end{pmatrix}$,$\sigma^z = \begin{pmatrix} 1 & 0 \\\\ 0 & -1 \end{pmatrix}$ 定义电子的自旋算符,并验证它们满足角动量对易关系。

(3) 量子力学中的可观测量算符为什么应为厄米算符?

(4) 你知道量子力学中的哪些数定律在经典物理中没有对应?

(5) 设$\Psi$ $\hat{H}_0$ 为$\hat{H}$的简并本征函数,相应的能量本征值为 $E_n$,如果 $\hat{H} = \hat{H}_0 + \hat{H}'$,其中 $\hat{H}'$ 可看作微扰。试写出能级的微扰修正公式(写到二级修正)。

(6) 什么叫受激辐射?什么叫自发辐射?

(7) 写出由 $\frac{1}{2}$ 自旋态构成的总自旋为0的态矢和自旋为1的态矢。