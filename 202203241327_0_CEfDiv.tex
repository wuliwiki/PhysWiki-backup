% 点电荷的散度
% keys 点电荷|散度|无穷大|狄拉克 delta 函数

\begin{issues}
\issueDraft
\end{issues}

\pentry{电场的高斯定律证明\upref{EGausP}}

在电场的高斯定律证明\upref{EGausP}, 我们看到通常意义下, 如果取一个包含点电荷的闭合高斯面, 那么散度定理(\autoref{Divgnc_eq1}~\upref{Divgnc})并不能直接使用, 因为在点电荷处电场的散度没有定义, 而在曲面内的其他位置电场的散度处处为零.

但是我们可以先把每个点电荷替换成一个半径为 $R$ 的带电小球, 然后令 $R\to 0$ 即可.




未完成: 参考 \cite{GriffE}

========= 回收内容 ==============

相同. 同样, 这个结论也包含.

可以推出原点处有无穷大的散度, 使得积分后恰好等于 $q/\epsilon_0$. 这让我们马上想到(三维)狄拉克 $\delta$ 函数\upref{Delta}, 在原点处为无穷大, 但体积分为有限值
\begin{equation}
\int \delta(\bvec r)\dd{V} = 1
\end{equation}

所以如果我们假设电场在原点处的散度为
\begin{equation}
\div \bvec E = \frac{q}{\epsilon_0}\delta(\bvec r)
\end{equation}
那么恰好可使电场满足散度定理
\begin{equation}
\int \div \bvec E \dd{V} = \frac{q}{\epsilon_0}
\end{equation}
