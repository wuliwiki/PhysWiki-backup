% 热力学第一定律
% 热力学第一定律|能量守恒|做功|传热|内能

\begin{issues}
\issueDraft
\end{issues}

\pentry{压力体积图\upref{PVgraf}, 理想气体内能\upref{IdgEng}}

\textbf{热力学第一定律}是\textbf{能量守恒}在热力学中的形式. \textbf{热量}是指当热学系统出现温度差时引起的能量转移. 热力学第一定律表明, 外部对系统传递的热量 $Q$ 等于系统对外做功 $W$ 加上系统的\textbf{内能} $E$ 增加:
\begin{equation}\label{Th1Law_eq1}
\Delta Q = W + \Delta E
\end{equation}
系统的内能有时也用 $U$ 表示. 热力学第一定律的另一种表述是:\textbf{第一类永动机}是不可能造成的.
\addTODO{什么是第一类永动机?链接到永动机词条}

虽然功和热都是能量的转移, 但热是通过微观粒子的无规则相互作用传递的, 而功\upref{Fwork}在热力学中特指宏观的作用力和宏观位移产生的. 对于容器中的气体, 把系统的压强记为 $P$, 体积记为 $V$, 对外做功可以写成积分形式:
\begin{equation}
W = \int_{V_1}^{V_2} P \dd{V}
\end{equation}
热力学第一定律写成微分形式是
\begin{equation}\label{Th1Law_eq2}
\delta Q = \delta W + \dd E = P\dd V + \dd E
\end{equation}
我们在 $Q$ 和 $W$ 用 $\delta$ 符号而不是全微分符号 $\dd{}$, 是因为 $Q$ 和 $W$ 和系统变化的过程本身有关: 也就是说即使系统的初末状态确定, 中间的过程不一样也会导致它们的不同. 这样的量被称为\textbf{过程量}\upref{StaPro}. $E$ 前面用的是全微分符号,是因为内能 $E$ 之和系统的状态有关, 被称为\textbf{状态量}\upref{StaPro}. 所以 $\Delta E$ 也只与系统的初始和最终的状态有关, 与中间的过程无关.

\addTODO{写一些理想气体的例题, 例如 PV 图种, 计算两点间延着不同轨迹的热量}
\addTODO{这个词条不要提熵, 下文移动到其他地方}

虽然 $\delta W$ 是过程量,但 $\delta W/P=\dd V$ 是全微分($V$ 是态函数).$\delta Q/T=\dd S$ 也是全微分,其中 $S$ 为热力学熵\upref{Entrop}.

\subsection{内能和态函数}
如果某个函数只和系统的热力学参量有关,也就是只和系统状态有关,我们称它为\textbf{态函数}.热力学研究的就是热力学系统的态函数之间的关系.

我们可以用几个宏观的热力学参量来完整地刻画一个热力学平衡系统.例如,对于一个无外场的孤立气体系统,压强 $P$ 和温度 $T$ 足以刻画这个气体系统的一切宏观特征.对于理想气体,有状态方程\upref{PVnRT} $PV=nRT$,压强 $P$ 和温度 $T$ 足以描绘整个理想气体系统(对任意均匀的单元系统也有类似的结论).因此可以写出 $E$ 的全微分形式:
\begin{equation}
\dd E=\left(\frac{\partial E}{\partial T}\right)_p \dd T + \left(\frac{\partial E}{\partial P}\right)_T \dd P
\end{equation}

如果将 $E$ 看成是熵 $S$ \upref{Entrop} 和体积 $V$ 的函数,则可以写成
\begin{equation}
\dd E=T\dd S-P\dd V
\end{equation}

这里 $S=\left(\frac{\partial E}{\partial T}\right)_V$,除此以外熵还有统计物理的定义.如果这个全微分刚好对应系统的一个可逆过程,那么我们可以看出 $\delta W$ 就是 $P\dd V$,由热力学第一定律,就有 $\delta Q=T\dd S$,这给出了熵的另一个定义——对于可逆过程 $\delta Q/T$ 的积分.

由热力学第二定律\upref{Td2Law},对于不可逆的热力学过程,有 $\delta Q<T\dd S$.所以代入第一定律可以得到 $T\dd S\ge \dd U+\delta W=\dd U+P\dd V$,等号在可逆过程中成立.

对理想气体\upref{Igas}, 令分子自由度为 $i$, 有
\begin{equation}
E = \frac{i}{2}n RT
\end{equation}

\addTODO{需要加一个范德瓦尔斯气体的词条,作为经典例子}
