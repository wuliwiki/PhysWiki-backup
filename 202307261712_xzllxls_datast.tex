% 数据
% 数据

\textbf{数据}(Data)是对客观事物各种信息、各种特征的记录,能够反应事物的各种性质和状态。机器学习是基于经验的学习。对于机器学习来说,数据就是经验,是机器学习的学习素材。机器学习算法所做的事情,就是从大量已知数据中寻找出背后的规律——\textbf{数据分布}(Data distribution),从而能够预测未来可能发生的事情。

\textbf{数据集}(Data set)是多条数据记录所构成的集合。在实际应用中,通常可以以表格的形式来表示数据集。

\begin{table}[ht]
\centering
\caption{请输入表格标题}\label{tab_datast2}
\begin{tabular}{|c|c|c|c|c|c|c|c|c|c|c|}
\hline
编号 & 性别 & 年龄 & 职业 & 睡眠时间 & BMI指数 & 心率 & 舒张压 & 收缩压 & 每日走路步数 & 睡眠障碍 \\\hline
1 & * & * & * & * & * & * & * & * & * & * \\
\hline
2 & * & * & * & * & * & * & * & * & * & * \\
\hline
3 & * & * & * & * & * & * & * & * & * & * \\
\hline
4 & * & * & * & * & * & * & * & * & * & * \\
\hline
\end{tabular}
\end{table}

上述表格就是关于  的一个简单数据集。表格中的每一行(除表头外)都是对于记录对象的描述,称为一个\textbf{实例}(Instance)或者\textbf{样本}(Sample)。