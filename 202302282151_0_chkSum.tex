% 校验和
% keys 哈希|指纹|安全性|sha1|md5

\begin{issues}
\issueOther{术语可能不严谨}
\issueOther{需要预备词条:计算机文件: 文件名,内容, metadata, 文件系统}
\end{issues}

\footnote{本文参考 Wikipedia \href{https://en.wikipedia.org/wiki/Checksum}{相关页面}。}\textbf{校验和(checksum)}是指一类算法, 可以将任意字节的数据(以下统称为文件)转换为一个简短的值。 我们通常也将其称为文件的\textbf{指纹(fingerprint)}或者\textbf{哈希值(hash)}\footnote{严格来说这些不是同义词, 但人们经常混用}。 这可以用于检查两个文件的内容是否相同。 % 未完成: 引用 ”计算机文件”

通常来说, 文件在储存或传输的过程中有一定可能会被损坏, 例如网络故障, 黑客攻击, 或者储存介质的个别字节出现故障。 如果我们知道文件原来的校验和, 就可以随时重新做一次校验和, 如果结果和以前不一样, 就说明文件被改变了。 这也是为什么一些网站的提供文件下载时同时也会提供各种校验和, 以便用户下载以后对照校验和确保文件无误。

\subsection{碰撞}
然而要达到上述目的, 理想的情况就是确保任何文件和它的指纹都是一一对应\upref{map}的。 但实际上, 我们只能确保内容相同的文件得到同样的指纹, 或者说不同的指纹必定对应内容不同的文件。 不同的文件有可能对应相同的指纹, 我们把这种情况叫做\textbf{碰撞(collision/clash)}。

哈希碰撞的概率通常很小, 但却无法避免。 例如某文件有 100 个比特, 那么就有可能有 $2^{100}$ 种不同的内容, 但如果哈希值只有 10 个比特, 那么只可能有 $2^{10}$ 个不同的哈希值, 所以必定会出现碰撞。 为了减小碰撞的可能性, 我们可以使用更好的算法, 或者用更长的指纹。

为什么不同的算法会影响碰撞的可能性呢? 我们可以考虑一个最简单的算法: 把文件中每个字节对应的整数相加得到指纹。 这显然是一个不好的算法, 因为实际操作中很可能会出现两个文件仅仅存在字节顺序上的不同, 例如 \verb|abcd| 和 \verb|dcba| 按照这个算法得到的结果都是一样的。 所以好的算法在于能尽量减小实际使用时发生碰撞的概率。

\subsection{安全性}
除了碰撞, 一个校验和算法的安全性也十分重要。 安全性是指能否创造或修改某个文件, 使得它具有指定的指纹。 这种操作越难实现, 算法就越安全。

例如一个黑客在某文件被传输时将其篡改, 为了防止被发现, 将其篡改的文件略作不重要的修改使其具有和源文件相同的指纹, 那么即使文件的接收者对比指纹, 也不会有所察觉, 这就使得这个算法不安全。

\subsubsection{用于服务器保存密码}
网站的服务器中经常使用某种安全的指纹算法保存密码。 这是因为如果一个网站的服务器中直接保存用户的密码明文, 那么一旦这些明文密码被泄露, 那么得到密码的人将可以任意访问用户的数据。

更好的做法是服务器仅仅保存每个用户密码的哈希值, 用户每次输入密码时, 服务器先将密码转换为哈希值再验证是否正确。 这样即使这些哈希值泄露, 也不能把他们作为密码直接使用, 或者逆向得到密码原文。

\subsection{常用算法}
\textbf{SHA1(Secure Hash Algorithm 1)}著名文件版本控制软件 Git % 链接未完成
就是使用 SHA1 散列值来检查文件是否发生变化。 这是一个广泛使用的算法, 但 SHA1 已被破解, 所以安全性无保障。

\textbf{MD5} 是一个更安全的算法。

\subsection{常用软件}
大部分时候我们在命令行中操作, 如 linux 的 \verb|sha1sum 文件名|, \verb|md5sum 文件名| 等命令, 但也有一些软件提 GUI 界面, 可以通过鼠标右键菜单的方式获得校验和, 如压缩软件 7zip。

Linux 用 \verb`find 文件夹 -type f -exec sha1sum {} \; | sort | sha1sum` 可以对比两个文件夹。 对比包含文件名, 文件夹名, 和每个文件的 hash, 不包括 metadata (例如权限, 最后修改时间等)。
