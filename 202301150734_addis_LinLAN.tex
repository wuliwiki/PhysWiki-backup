% 搭建 Linux 局域网

\begin{issues}
\issueDraft
\end{issues}

\pentry{Linux 网络笔记\upref{LinWeb}}

VirtualBox 虚拟机搭建局域网见: \autoref{VirBox_sub1}~\upref{VirBox}

\subsection{硬件}
\begin{itemize}
\item 交换机(managed, unmanaged)
\item 网线(cat6, 类型需要交换机支持)
\item 笔记本电脑
\end{itemize}

\subsection{wifi 路由器连接}
\begin{itemize}
\item wifi 路由器同样也支持有线连接,可以充当 switch, 还有路由器, 和 DHCH 服务器。
\item 路由器是否会改变机器的 ip? (如果机器太多应该会的)
\end{itemize}

\subsection{switch 连接}
\subsubsection{用 Windows 共享网络}
\begin{itemize}
\item 可以用 Windows 电脑来提供网络, 例如通过 wifi 接互联网, 就再控制面板中找到 view network status and tasks, 点 Wi-Fi 2 (Wifi 名称), 点最下面的 properties, 点 sharing 面板, 勾选 Allow other network users to connect through this computer's Internet connection, 然后在下面选中要分享的 Ethernet 网卡。 下面那个勾也可以打上(... control or disable the shared Internet connection)。 这样就成功了, 链接 switch 的所有电脑都可以上网了。
\item 如果重启 Windows 以后, Linux 连不上, 就把上面的 share 关闭, 连 Linux, 再开一次。
\item Switch 的 DHCP 功能可以关掉, 因为这个设置下 DHCP 和 NAT 都由 Windows 提供。 NAT 根据 DHCP 分配的 ip 工作, 所以如果 Linux 机器自行设置 ip 地址, 那么局域网中该 ip 可以用, 但是这个机器就上不了外网了(因为 Windows 的 NAT 转发失败)。 所以最好还是不碰, Linux 被分配的 ip 貌似也不会改变。
\end{itemize}

\subsubsection{基本操作}
\begin{itemize}
\item Windows 上可以用 cmd 里面的 \verb|ipconfig| 命令来查看 ip。 Linux 上用 \verb|ifconfig|。 如果 Win 上装了 WSL, 那么 WSL 里面看到的 ip 和 cmd 里面是一样的。
\item 每个机器可以自己设置自己的 ip 地址, 也可以从 DH 什么服务器获取设置, 一般好像不用管它。
\item 每台机器上, 要搜索局域网中所有设备, 用 \verb|sudo arp-scan -l --interface=网卡名| 其中 \verb|网卡名| 就是 \verb|ifconfig| 里面显示的。
\end{itemize}

\subsubsection{笔记本合上盖子不作为}
\begin{itemize}
\item \verb|sudo vim /etc/systemd/logind.conf|, 找到所有包含 \verb|LidSwitch| 的选项, 取消注释, 值改成 \verb|ignore|
\item 然后运行 \verb|systemctl restart systemd-logind.service| 生效。 注意系统会重新 login
\end{itemize}


\subsection{ssh 互通}
\begin{itemize}
\item 安装 \verb|openssh-server|, 设置见\upref{SSH}。 注意每个用户还是单独生成自己的密钥比较好, 否则可能连接会被拒绝。
\end{itemize}

\subsection{共享文件夹}
Linux 详见 “Linux 创建网络文件夹(sshfs 和 NFS)\upref{NFS}”。 要把 windows 上的目录挂载到 linux, 建议用 sshfs, 好处是 windows 上无需安装任何第三方的 server, 只要 linux 能 ssh 到 windows 的 WSL1 就行。 但是 windows 重启以后可能就会需要重新连一次。

Windows Server 自带 NFS, 普通 Windows 可以用第三方的 \href{https://sourceforge.net/projects/freenfs/files/latest/download}{freeNFS}

\subsection{远程桌面}
\begin{itemize}
\item 安装向日葵, 或者 teamviewer 之类的远程桌面即可。
\item 远程桌面对 wayland 支持一般欠佳。 较新的 ubuntu 如果用的显示服务器是 wayland, 可以退出登录, 在登录的时候选择 xorg。
\end{itemize}
