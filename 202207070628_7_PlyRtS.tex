% 多项式的根式解
% Galois理论|求根公式|五次方程|5次方程|Galois群


\pentry{Galois扩张\upref{GExt}}

\subsection{古典数学难题}

代数方程的根式解表达,是古典代数学中的经典难题.根式解即用任意多项式的系数进行有限次加、减、乘、除以及开根号运算,得到其根.比如,二次多项式$ax^2+bx+c$的根总可以表达为$\frac{-b\pm\sqrt{b^2-4ac}}{2a}$.

得到二次多项式求根公式的方法很简单,配方即可.

三次多项式的求根则复杂许多.1494年,意大利数学家卢卡·帕乔利(Lusa Pacioli)发表了他近三十年心血的结晶,《算术、几何、比及比例概要》(\textsl{Summa de arithmetica, geometria, Proportioni et proportionalita})\footnote{关于这本书的介绍,可参考https://en.wikipedia.org/wiki/Summa_de_arithmetica.},有时也被翻译为《数学大全》或者《算术大全》.在这本书里,他列出了两种无法解出的三次代数方程:
\begin{equation}\label{PlyRtS_eq1}
n=ax+bx^3
\end{equation}
\begin{equation}
n=ax^2+bx^3
\end{equation}

但是就在约一个世纪后,一个名叫希皮奥内·德尔·费罗(Scipione del Ferro)的意大利数学家就发现了\autoref{PlyRtS_eq1} 的解法.这个人很“自闭”,他不喜欢公开交流思想,只喜欢和自己的朋友或学生交流——这大概就是为什么没多少人记得他.所幸,费罗在三次方程求根公式上的成果被记录在他的笔记本上,在他去世后由女婿哈尼瓦·纳威(Hannival Nave)继承了,这位女婿也是个数学家.

戏剧的是,在费罗去世之前,还秘密将





















