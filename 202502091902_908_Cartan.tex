% 埃利·嘉当(综述)
% license CCBYSA3
% type Wiki

本文根据 CC-BY-SA 协议转载翻译自维基百科\href{https://en.wikipedia.org/wiki/\%C3\%89lie_Cartan}{相关文章}。

埃利·约瑟夫·卡尔坦(Élie Joseph Cartan,法语发音:[kaʁtɑ̃];1869年4月9日 – 1951年5月6日)是法国一位具有深远影响的数学家,他在李群理论、微分系统(偏微分方程的无坐标几何形式化)和微分几何领域做出了基础性工作。他还对广义相对论和间接地对量子力学做出了重要贡献。普遍认为,他是20世纪最伟大的数学家之一。

他的儿子亨利·卡尔坦(Henri Cartan)是位于代数拓扑学领域的具有深远影响的数学家。
\subsection{生平}  
埃利·卡尔坦(Élie Cartan)于1869年4月9日出生在法国伊泽尔省的多洛米厄村(Dolomieu),父亲是约瑟夫·卡尔坦(Joseph Cartan,1837–1917),母亲是安妮·科塔兹(Anne Cottaz,1841–1927)。约瑟夫·卡尔坦是村里的铁匠;埃利·卡尔坦回忆说,他的童年是在“铁砧的敲打声中度过的,每天早晨天刚亮就开始了”,而且“他的母亲在那少数几分钟的时间里,解放了自己从照顾孩子和家务的繁忙中,便开始使用纺车”。埃利有一个姐姐让娜-玛丽(Jeanne-Marie,1867–1931),她成为了一名裁缝;一个弟弟莱昂(Léon,1872–1956),他成为了父亲的铁匠铺的工匠;以及一个妹妹安娜·卡尔坦(Anna Cartan,1878–1923),在埃利的影响下,她进入了高等师范学校(École Normale Supérieure),并选择了成为一名数学教师的职业,担任中学(lycée)的教师。

埃利·卡尔坦在多洛米厄的一个小学上学,并且是班级中最好的学生。曾任他老师的杜普依(M. Dupuis)回忆道:“埃利·卡尔坦是一个害羞的学生,但他眼中闪烁着一种不寻常的智慧光芒,并且具备了极好的记忆力。”当时伊泽尔的代表安托南·杜博(Antonin Dubost)参观了学校,并对卡尔坦的非凡才能印象深刻。他建议卡尔坦参加一项中学奖学金的竞赛。卡尔坦在杜普依老师的指导下准备了这次竞赛,并在十岁时顺利通过。他在维耶讷(Vienne)学院度过了五年(1880–1885),然后在格勒诺布尔(Grenoble)中学度过了两年(1885–1887)。1887年,他搬到巴黎的尚松·德·塞伊(Lycée Janson de Sailly)中学,学习科学,并在两年内结识了同班同学让-巴蒂斯特·佩朗(Jean-Baptiste Perrin,1870–1942),后者后来成为法国著名的物理学家。

卡尔坦于1888年进入巴黎高等师范学校(École Normale Supérieure),在那里他听取了查尔斯·埃尔米特(Charles Hermite,1822–1901)、朱尔·塔内里(Jules Tannery,1848–1910)、加斯顿·达尔布(Gaston Darboux,1842–1917)、保罗·阿佩尔(Paul Appell,1855–1930)、埃米尔·皮卡尔(Émile Picard,1856–1941)、爱德华·古尔萨(Édouard Goursat,1858–1936)和亨利·庞加莱(Henri Poincaré,1854–1912)等人的讲座,他特别推崇庞加莱的讲座。

卡尔坦于1891年从高等师范学校毕业后,被征召入法国军队,服役一年并晋升为中士。在接下来的两年里(1892–1894),卡尔坦返回高等师范学校,并在同班同学阿瑟·特雷斯(Arthur Tresse,1868–1958)的建议下,开始研究由威廉·基林(Wilhelm Killing)提出的简单李群分类问题。特雷斯曾在1888–1889年间师从索弗斯·李(Sophus Lie)。1892年,李受达尔布和塔内里的邀请来到巴黎,并首次与卡尔坦见面。

卡尔坦于1894年在索邦大学(Sorbonne)的理学院捍卫了他的博士论文《有限连续变换群的结构》。在1894至1896年间,卡尔坦在蒙彼利埃大学担任讲师;1896年至1903年,他在里昂大学理学院担任讲师。

1903年,卡尔坦在里昂结婚,妻子是玛丽-路易丝·比安科尼(Marie-Louise Bianconi,1880–1950);同年,卡尔坦成为南锡大学理学院的教授。1904年,卡尔坦的长子亨利·卡尔坦(Henri Cartan)出生,后来成为一位有影响力的数学家;1906年,他的次子让·卡尔坦(Jean Cartan)出生,后来成为一位作曲家。1909年,卡尔坦将家人迁往巴黎,并在索邦大学理学院担任讲师。1912年,卡尔坦根据庞加莱的推荐成为索邦大学的教授,并一直在此工作,直到1940年退休。在晚年,卡尔坦在巴黎高等师范学校女子部教授数学。

作为卡尔坦的学生,几何学家陈省身写道:[4]

通常,在[与卡尔坦会面]的第二天,我会收到他的信。他会说:“你走后,我又想了想你的问题……”——他有了一些结果,还有一些新问题,等等。他记得所有关于简单李群、李代数的论文,都是熟记于心。当你在街上遇到他,某个问题冒出来时,他会从口袋里拿出一个旧信封,写点东西,然后给你答案。有时我花了几个小时甚至几天才能得到同样的答案……我必须非常努力地工作。

1921年,他成为波兰科学院的外籍院士,1937年成为荷兰皇家艺术与科学学院的外籍院士。[5] 1938年,他参与了组织国际科学统一大会的国际委员会。[6]

他于1951年在巴黎因长期病患去世。

1976年,一座月球陨石坑以他的名字命名,原本它被命名为阿波罗纽斯D。