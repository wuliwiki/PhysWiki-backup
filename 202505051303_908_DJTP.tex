% 点集拓扑学(综述)
% license CCBYSA3
% type Wiki

本文根据 CC-BY-SA 协议转载翻译自维基百科\href{https://en.wikipedia.org/wiki/General_topology}{相关文章}。

\begin{figure}[ht]
\centering
\includegraphics[width=10cm]{./figures/16b10920f52e4ba5.png}
\caption{} \label{fig_DJTP_1}
\end{figure}
在数学中,一般拓扑(或称点集拓扑)是拓扑学的一个分支,主要研究拓扑学中使用的基本集合论定义和构造。它是大多数其他拓扑学分支的基础,包括微分拓扑、几何拓扑和代数拓扑。

点集拓扑中的基本概念是连续性、紧致性和连通性:
\begin{itemize}
\item 连续函数直观上是将相邻的点映射到相邻的点。
\item 紧致集是指可以被有限多个任意小的集合覆盖的集合。
\item 连通集是指不能被分成两个彼此远离的部分的集合。
\end{itemize}
“附近”、“任意小”和“远离”这些术语都可以通过使用开集的概念来精确定义。如果我们改变“开集”的定义,就会改变连续函数、紧致集和连通集的定义。对于“开集”的每一种定义选择,都称为一种拓扑。具有拓扑的集合称为拓扑空间。

度量空间是拓扑空间中的一个重要类别,在这些空间中,可以为集合中的点对定义一个实数的非负距离,也称为度量。具有度量简化了许多证明,且许多最常见的拓扑空间都是度量空间。
\subsection{历史}
一般拓扑学起源于多个领域,其中最重要的包括:
\begin{itemize}
\item 对实数线子集的详细研究(曾被称为点集的拓扑;这种用法现已过时)
\item 流形概念的引入
\item 在功能分析的早期阶段对度量空间,特别是赋范线性空间的研究。
\end{itemize}
一般拓扑学在大约1940年形成了现在的形式。可以说,它几乎囊括了连续性直觉的所有内容,以一种技术上足够的形式,能够应用于数学的任何领域。
\subsection{集合上的拓扑}
设 $X$ 是一个集合,$\tau$ 是 $X$ 的子集族。如果 $\tau$ 满足以下条件,则称 $\tau$ 是 $X$ 上的拓扑:\(^\text{[1][2]}\)
\begin{enumerate}
\item 空集和 $X$ 本身是 $\tau$ 的元素
\item $\tau$ 的元素的任意并集是 $\tau$ 的元素
\item $\tau$ 的有限多个元素的任意交集是 $\tau$ 的元素
\end{enumerate}
如果 $\tau$ 是 $X$ 上的拓扑,则对偶 $(X, \tau)$ 称为拓扑空间。可以使用符号 $X_\tau$ 来表示 $X$ 上带有特定拓扑 $\tau$ 的集合。

$\tau$ 的成员称为 $X$ 中的开集。如果 $X$ 的一个子集的补集在 $\tau$ 中(即其补集是开集),则称该子集是闭集。$X$ 的一个子集可以是开集、闭集、两者(闭开集)或都不是。空集和 $X$ 本身总是既是闭集又是开集。
\subsubsection{拓扑的基}
一个拓扑空间 $X$ 及其拓扑 $T$ 的基 $B$ 是 $T$ 中开集的一个集合,使得 $T$ 中的每个开集都可以写成 $B$ 中元素的并集。\(^\text{[3][4]}\)我们说基 $B$ 生成了拓扑 $T$。基是有用的,因为拓扑的许多性质可以简化为关于生成该拓扑的基的陈述——并且许多拓扑可以通过定义生成它们的基来最容易地定义。
\subsubsection{子空间与商空间}
每个拓扑空间的子集都可以赋予子空间拓扑,其中开集是大空间的开集与子集的交集。对于任何索引族的拓扑空间,可以赋予积空间积拓扑,该拓扑由在投影映射下,各个因子的开集的逆像生成。例如,在有限积中,积拓扑的基由所有开集的积组成。在无限积中,还需要额外的要求:在一个基本开集中的所有投影中,除了有限个投影之外,其他的投影都是整个空间。

商空间定义如下:如果 $X$ 是一个拓扑空间,$Y$ 是一个集合,并且 $f : X \to Y$ 是一个满射函数,则 $Y$ 上的商拓扑是那些在 $f$ 下具有开逆像的 $Y$ 的子集。换句话说,商拓扑是使得 $f$ 连续的最细拓扑。商拓扑的一个常见例子是当在拓扑空间 $X$ 上定义一个等价关系时。此时,映射 $f$ 是到等价类集合的自然投影。
\subsubsection{拓扑空间的例子}
给定的集合可以具有多种不同的拓扑。如果给一个集合赋予不同的拓扑,它就被视为一个不同的拓扑空间。

\textbf{离散拓扑与平凡拓扑}

任何集合都可以赋予离散拓扑,在该拓扑中,所有子集都是开集。在这种拓扑中,唯一收敛的序列或网是那些最终恒定的序列或网。此外,任何集合也可以赋予平凡拓扑(也叫不可分拓扑),在这种拓扑中,只有空集和整个空间是开集。在这种拓扑中的每个序列和网都收敛到空间的每一点。这个例子表明,在一般的拓扑空间中,序列的极限不一定是唯一的。然而,通常拓扑空间必须是豪斯多夫空间,在这种空间中极限点是唯一的。

\textbf{余有限拓扑与余可数拓扑}

任何集合都可以赋予余有限拓扑,在这种拓扑中,开集是空集和补集是有限集的集合。这是任何无限集合上的最小的 $T_1$ 拓扑。

任何集合也可以赋予余可数拓扑,在这种拓扑中,如果一个集合是空集或者它的补集是可数的,则该集合被定义为开集。当集合是不可数时,这种拓扑在许多情况下作为反例。

\textbf{实数和复数的拓扑}

在 $\mathbf{R}$(实数集)上有许多方法可以定义拓扑。标准的实数拓扑是由开区间生成的。所有开区间的集合构成了该拓扑的基(或基底),这意味着每个开集都是从基中某些集合的并集。特别地,这意味着一个集合是开集,如果在该集合中的每个点周围都存在一个非零半径的开区间。更一般地,欧几里得空间 $\mathbf{R}^n$ 也可以赋予拓扑。在 $\mathbf{R}^n$ 的通常拓扑中,基本开集是开球。类似地,复数集 $\mathbf{C}$ 和 $\mathbf{C}^n$ 也有一个标准拓扑,其中基本开集是开球。

实数线也可以赋予下限拓扑。在这里,基本开集是半开区间 $[a, b)$。这种拓扑在实数集 $\mathbf{R}$ 上比上面定义的欧几里得拓扑要细致;如果一个序列在这种拓扑中收敛到一个点,当且仅当它在欧几里得拓扑中从上方收敛到该点。这个例子表明,一个集合可以有许多不同的拓扑定义。

\textbf{度量拓扑}

每个度量空间都可以赋予度量拓扑,其中基本开集是由度量定义的开球。这是任何赋范向量空间上的标准拓扑。在有限维向量空间中,对于所有范数,这个拓扑是相同的。

\textbf{更多例子}

\begin{itemize}
\item 在任何给定的有限集合上都存在众多拓扑。这类空间被称为有限拓扑空间。有限空间有时用于提供关于拓扑空间的一般猜想的例子或反例。
\item 每个流形都有一个自然的拓扑,因为它是局部欧几里得的。类似地,每个单纯形和每个单纯复形从 $\mathbf{R}^n$ 中继承一个自然的拓扑。
\item 扎里斯基拓扑在一个环或代数簇的谱上按代数方式定义。在 $\mathbf{R}^n$ 或 $\mathbf{C}^n$ 上,扎里斯基拓扑的闭集是多项式方程组的解集。
\item 线性图具有一种自然的拓扑,它概括了图形中顶点和边的许多几何方面。
在泛函分析中,许多线性算子集合被赋予了通过指定某个特定函数序列何时收敛到零函数来定义的拓扑。
任何局部域都有一个原生拓扑,并且可以扩展到该域上的向量空间。
谢尔宾斯基空间是最简单的非离散拓扑空间。它与计算理论和语义学有重要的关系。
如果 $\Gamma$ 是一个序数,则集合 $\Gamma = [0, \Gamma)$ 可以赋予由区间 $(a, b)$、\[0, b) ) 和 $(a, \Gamma)$ 生成的顺序拓扑,其中 $a$ 和 $b$ 是 $\Gamma$ 的元素。
\end{itemize}
