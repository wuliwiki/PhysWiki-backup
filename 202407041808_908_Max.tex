% 马克斯·普朗克
% license CCBYSA3
% type Wiki

(本文根据 CC-BY-SA 协议转载自原搜狗科学百科对英文维基百科的翻译)

马克斯·卡尔·恩斯特·路德维希·普朗克(德语:[ˈplaŋk];[1] English: /ˈplæŋk/;[2] 1858年4月23日 – 1947年10月4日),德国理论物理学家,1918年因发现能量量子获得诺贝尔物理学奖。[3]

普朗克对理论物理做出了许多贡献,但他作为物理学家的名声主要取决于他作为量子理论创始人的角色,[4]这彻底改变了人类对原子和亚原子过程的理解。1948年,德国科学机构凯泽·威廉学会(普朗克曾两次担任主席)更名为马克斯·普朗克学会(MPS)。议员现在包括代表广泛科学方向的83个机构。

\subsection{生活和事业}
\begin{figure}[ht]
\centering
\includegraphics[width=10cm]{./figures/25347aee0b1e0c17.png}
\caption{马克思普朗克十岁时的签名} \label{fig_Max_1}
\end{figure}
