<script type="text/javascript" src="http://cdn.mathjax.org/mathjax/latest/MathJax.js?config=default"></script>

$$
\mathrm{若五阶行列式的展开式中项}a_{13}a_{2k}a_{34}a_{42}a_{5l}\mathrm{带负号},\mathrm 则k,l\mathrm{的值分别为}(\;).
$$
$$
A.
k=1,l=5 \quad B.k=5,l=1 \quad C.k=2,l=5 \quad D.k=5,l=2 \quad E. \quad F. \quad G. \quad H.
$$
$$
\begin{array}{l}\begin{array}{l}\mathrm{根据行列式的定义},k,l\mathrm{只能取}1或5,\\若k=5,l=1,则\tau(35421)=8,\end{array}\\\begin{array}{l}若k=1,l=5,则τ(31425)=3,\\\mathrm{所以}k=1,l=5.\\\end{array}\end{array}
$$



$$
\mathrm{在四阶行列式的展开式中},\mathrm{下列各项中带正号的是}(\;).
$$
$$
A.
a_{13}a_{24}a_{32}a_{41} \quad B.a_{14}a_{23}a_{32}a_{41} \quad C.a_{14}a_{22}a_{33}a_{41} \quad D.a_{11}a_{23}a_{32}a_{44} \quad E. \quad F. \quad G. \quad H.
$$
$$
a_{14}a_{23}a_{32}a_{41}\mathrm{的列标的逆序数为}1+2+3=6,\mathrm{故此项前边带有正号}.
$$



$$
\mathrm{下列哪一个不是五阶行列式的展开式中的项}(\;).
$$
$$
A.
a_{21}a_{13}a_{34}a_{55}a_{42} \quad B.a_{21}a_{13}a_{24}a_{55}a_{42} \quad C.a_{11}a_{23}a_{34}a_{55}a_{42} \quad D.a_{13}a_{34}a_{51}a_{42}a_{25} \quad E. \quad F. \quad G. \quad H.
$$
$$
a_{21}a_{13}a_{24}a_{55}a_{42}\mathrm{在第二行取了两个元素},\mathrm{与行列式定义不符}.
$$



$$
在n\;\mathrm{阶行列式中},\mathrm{关于主对角线与元素}a_{ij}\mathrm{对称的元素为}(\;).
$$
$$
A.
a_{ij} \quad B.a_{ji} \quad C.a_{(i-1)(j-1)} \quad D.a_{(j-1)(i-1)} \quad E. \quad F. \quad G. \quad H.
$$
$$
\mathrm{由行列式定义知},\mathrm{与元素}a_{ij}\mathrm{关于主对角线对称的元素行列刚好互换},\mathrm{即为}a_{ji}.
$$



$$
若a_{1i}a_{23}a_{35}a_{4j}a_{54}\mathrm{为五阶行列式的展开式中带正号的一项},则\;i,j\;\mathrm{分别为}(\;).
$$
$$
A.
i=1,j=2 \quad B.i=3,j=2 \quad C.i=2,j=1 \quad D.i=2,j=3 \quad E. \quad F. \quad G. \quad H.
$$
$$
\begin{array}{l}\begin{array}{l}\mathrm{由题意可知},i,j\mathrm{只能分别取值}1,2\\\mathrm{不妨设}i=1,j=2,\mathrm{则排列}13524\mathrm{的逆序数为}3,\mathrm{故此排列为奇排列},\mathrm{对应项带负号},\mathrm{不合题意},\end{array}\\\mathrm{因此}i=2\;,j=1.\end{array}
$$



$$
若-a_{32}a_{1r}a_{25}a_{4s}a_{53}\mathrm{是五阶行列式展开式中的一项},则\;r,s\;\mathrm{的值分别为}(\;).
$$
$$
A.
r=1,s=1 \quad B.r=1,s=4 \quad C.r=4,s=1 \quad D.r=4,s=4 \quad E. \quad F. \quad G. \quad H.
$$
$$
\begin{array}{l}r和s\mathrm{只能取值为}1,4\\当r=4,s=1时,\mathrm{判断}a_{32}a_{14}a_{25}a_{41}a_{53}即a_{14}a_{25}a_{32}a_{41}a_{53}\mathrm{的列标的逆序数为}7(\mathrm{奇数}),\mathrm{故其项的符号为负}.\end{array}
$$



$$
若a_{13}a_{2k}a_{34}a_{42}a_{5l}\mathrm{是五阶行列式的展开式中带有正号的项},则\;k,l\mathrm{分别为}(\;).
$$
$$
A.
k=1,l=5 \quad B.k=5,l=1 \quad C.k=1,l=4 \quad D.k=4,l=1 \quad E. \quad F. \quad G. \quad H.
$$
$$
\mathrm{根据行列式的定义},k\;,l\mathrm{只能取值}1或5,若k=5,l=1,则τ(\;35421)=8,若k=1,l=5,则τ(\;31425)=3,\mathrm{所以}k=5,l=1.
$$



$$
\mathrm{六阶行列式的展开式中共有}(\;)项.
$$
$$
A.
36 \quad B.720 \quad C.240 \quad D.160 \quad E. \quad F. \quad G. \quad H.
$$
$$
\mathrm{由行列式的定义可知}n\mathrm{阶行列式表示所有取自不同行不同列}n\mathrm{个元素乘积的代数和},即n!\mathrm{项的代数和}.
$$



$$
\mathrm{在四阶行列式的展开式中},\mathrm{下列带负号的是}(\;).
$$
$$
A.
a_{13}a_{24}a_{32}a_{41} \quad B.a_{14}a_{23}a_{32}a_{41} \quad C.a_{12}a_{24}a_{33}a_{41} \quad D.a_{13}a_{21}a_{32}a_{44} \quad E. \quad F. \quad G. \quad H.
$$
$$
a_{13}a_{24}a_{32}a_{41}\mathrm{的列标的逆序数为}2+3=5,\mathrm{所以此项前面带有负号}.
$$



$$
\mathrm{四阶行列式展开式中带负号且包含元素}a_{12}和a_{21}\mathrm{的项为}(\;).
$$
$$
A.
a_{12}a_{21}a_{33}a_{44} \quad B.a_{12}a_{21}a_{34}a_{43} \quad C.a_{12}a_{21}a_{32}a_{44} \quad D.a_{12}a_{21}a_{33}a_{41} \quad E. \quad F. \quad G. \quad H.
$$
$$
\mathrm{根据行列式的定义},\mathrm{四阶行列式中包含元素}a_{12}和a_{21}\mathrm{的项为}a_{12}a_{21}a_{33}a_{44}或a_{12}a_{21}a_{34}a_{43},而a_{12}a_{21}a_{33}a_{44}\mathrm{带有负号}.
$$



$$
n\mathrm{阶行列式}\begin{vmatrix}a_1&0&⋯&0\\0&a_2&⋯&0\\⋯&⋯&⋯&0\\0&0&⋯&a_n\end{vmatrix}=(\;).
$$
$$
A.
0 \quad B.1 \quad C.a_1+a_2+⋯+a_n \quad D.a_1a_2⋯ a_n \quad E. \quad F. \quad G. \quad H.
$$
$$
\mathrm{根据行列式的定义},\begin{vmatrix}a_1&0&⋯&0\\0&a_2&⋯&0\\⋯&⋯&⋯&0\\0&0&⋯&a_n\end{vmatrix}=a_1a_2⋯ a_n.
$$



$$
\mathrm{四阶行列式}\begin{vmatrix}0&0&2&0\\-1&0&3&0\\6&0&0&-3\\2&2&5&4\end{vmatrix}=().
$$
$$
A.
0 \quad B.4 \quad C.12 \quad D.-12 \quad E. \quad F. \quad G. \quad H.
$$
$$
\mathrm{由行列式的定义可知}\begin{vmatrix}0&0&2&0\\-1&0&3&0\\6&0&0&-3\\2&2&5&4\end{vmatrix}=(-1)^{τ(3142)}\;\;2×(-1)×(-3)×2=-12.
$$



$$
\mathrm{设多项式}f(x)=\begin{vmatrix}x&x&0&2\\3&x&1&1\\2&1&x&0\\5&4&2&x\end{vmatrix},则f(x)中x^4\mathrm{的系数为}(\;).
$$
$$
A.
1 \quad B.3 \quad C.2 \quad D.4 \quad E. \quad F. \quad G. \quad H.
$$
$$
\mathrm{根据行列式定义可以得出},含x^4\mathrm{的项只有主对角线元素的乘积},\mathrm{因此系数为}1.
$$



$$
n\mathrm{阶行列式}\begin{vmatrix}1&0&0&...&0\\0&2&0&...&0\\0&0&3&...&0\\...&...&...&...&...\\0&0&0&...&n\end{vmatrix}=(\;).
$$
$$
A.
n! \quad B.0 \quad C.n \quad D.-n! \quad E. \quad F. \quad G. \quad H.
$$
$$
\mathrm{原式}=1×2×3....× n=n!.
$$



$$
\mathrm{四阶行列式}D=\begin{vmatrix}a_{11}&a_{22}&a_{32}&a_{13}\\a_{21}&a_{12}&a_{31}&a_{14}\\a_{33}&a_{41}&a_{24}&a_{34}\\a_{44}&a_{43}&a_{42}&a_{23}\end{vmatrix},\mathrm{下列哪一个不是}D\mathrm{的展开式中的项}().
$$
$$
A.
a_{11}a_{22}a_{33}a_{44} \quad B.a_{32}a_{12}a_{44}a_{34} \quad C.-a_{21}a_{22}a_{23}a_{24} \quad D.a_{11}a_{12}a_{23}a_{24} \quad E. \quad F. \quad G. \quad H.
$$
$$
a_{11}a_{22}a_{33}a_{44}\mathrm{不是}D\mathrm{中的项},\mathrm{因为}a_{11},a_{22}\mathrm{在同一行}.
$$



$$
设\begin{vmatrix}-1&1&1\\1&-1&x\\1&1&-1\end{vmatrix}\mathrm{是关于}x\mathrm{的一次多项式},\mathrm{该多项式一次项系数为}(\;).
$$
$$
A.
-1 \quad B.1 \quad C.-2 \quad D.2 \quad E. \quad F. \quad G. \quad H.
$$
$$
含x\mathrm{的项为}x(-1)^{2+3}\begin{vmatrix}-1&1\\1&1\end{vmatrix}=2x,\mathrm{故一次项系数为}2.
$$



$$
\mathrm{五阶行列式的展开式中},\mathrm{带负号的项是}(\;).
$$
$$
A.
a_{12}a_{25}a_{31}a_{43}a_{54} \quad B.a_{12}a_{31}a_{25}a_{54}a_{43} \quad C.a_{11}a_{22}a_{33}a_{44}a_{55} \quad D.a_{12}a_{24}a_{31}a_{43}a_{55} \quad E. \quad F. \quad G. \quad H.
$$
$$
\mathrm{对于项}a_{12}a_{24}a_{31}a_{43}a_{55},τ(\;24135)=3.
$$



$$
\mathrm{四阶行列式}\begin{vmatrix}0&0&1&0\\0&1&0&0\\0&0&0&1\\1&0&0&0\end{vmatrix}=().
$$
$$
A.
0 \quad B.1 \quad C.-1 \quad D.-4 \quad E. \quad F. \quad G. \quad H.
$$
$$
\begin{vmatrix}0&0&1&0\\0&1&0&0\\0&0&0&1\\1&0&0&0\end{vmatrix}=(-1)^{τ(3241)}a_{13}a_{22}a_{34}a_{41}=1.
$$



$$
\mathrm{四阶行列式}\begin{vmatrix}0&0&0&a\\0&0&b&0\\0&c&0&0\\d&0&0&0\end{vmatrix}=(\;).
$$
$$
A.
0 \quad B.a \quad C.abcd \quad D.-abcd \quad E. \quad F. \quad G. \quad H.
$$
$$
\begin{vmatrix}0&0&0&a\\0&0&b&0\\0&c&0&0\\d&0&0&0\end{vmatrix}=(-1)^{τ(4321)}abcd=(-1)^6abcd=abcd.
$$



$$
\mathrm{设多项式}f(x)=\begin{vmatrix}2x&3&1&2\\x&x&0&1\\2&1&x&4\\x&2&1&4x\end{vmatrix},则f(x)中x^4\mathrm{系数为}(\;).
$$
$$
A.
2 \quad B.4 \quad C.8 \quad D.6 \quad E. \quad F. \quad G. \quad H.
$$
$$
\mathrm{根据行列式定义可以得出},含x^4\mathrm{的项只能是主对角线上元素的乘积},\mathrm{因此系数为}2×4=8.
$$



$$
\mathrm{四阶行列式}\begin{vmatrix}a&0&0&0\\0&0&b&0\\0&c&0&0\\0&0&0&d\end{vmatrix}=(\;).
$$
$$
A.
0 \quad B.1 \quad C.abcd \quad D.-abcd \quad E. \quad F. \quad G. \quad H.
$$
$$
\begin{vmatrix}a&0&0&0\\0&0&b&0\\0&c&0&0\\0&0&0&d\end{vmatrix}=(-1)^{τ(1324)}abcd=-abcd.
$$



$$
\mathrm{四阶行列式的展开式中含有因子}a_{11}a_{23}\mathrm{的项为}(\;).
$$
$$
A.
-a_{11}a_{23}a_{32}a_{44}和a_{11}a_{23}a_{34}a_{42} \quad B.a_{11}a_{23}a_{32}a_{44}和a_{11}a_{23}a_{34}a_{42} \quad C.-a_{11}a_{23}a_{32}a_{44}和-a_{11}a_{23}a_{34}a_{42} \quad D.a_{11}a_{23}a_{32}a_{44}和-a_{11}a_{23}a_{34}a_{42} \quad E. \quad F. \quad G. \quad H.
$$
$$
\begin{array}{l}\mathrm{由定义知},\mathrm{四阶行列式的一般项为}:(-1)^τ a_{1p_1}a_{2p_2}a_{3p_3}a_{4p_4},\mathrm{其中}τ 为p_1p_2p_3p_4\mathrm{的逆序数},\mathrm{由于}p_1=1,p_2=3\mathrm{已固定},p_1p_2p_3p_4\mathrm{只能形如}13xx,即1324或1342,\\其τ\mathrm{分别为}0+0+1+0或0+0+0+2=2,\mathrm{所以}-a_{11}a_{23}a_{32}a_{44}和a_{11}a_{23}a_{34}a_{42}\mathrm{为所求}.\\\end{array}
$$



$$
\mathrm{多项式}f(x)=\begin{vmatrix}x&3&1&2\\x&2x&0&1\\2&1&x&4\\0&2&1&x\end{vmatrix},则f(x)中x^4\mathrm{系数为}(\;).
$$
$$
A.
2 \quad B.1 \quad C.4 \quad D.8 \quad E. \quad F. \quad G. \quad H.
$$
$$
\mathrm{由于行列式是不同行不同列元素乘积的和},\mathrm{故含有}x^4\mathrm{项只有行列式主对角线上元素的乘积},\mathrm{即系数为}2.
$$



$$
若n\mathrm{阶行列式中有}n^2-n\mathrm{个以上的元素为零},\mathrm{则该行列式为}(\;).
$$
$$
A.
0 \quad B.n^2-n \quad C.n \quad D.\mathrm{无法确定} \quad E. \quad F. \quad G. \quad H.
$$
$$
\begin{array}{l}\mathrm{如果}n\mathrm{阶行列式中有}n^2-n\mathrm{个以上元素为零},\mathrm{则至多有}n-1\mathrm{个不为零元素}\\\mathrm{由于}n\;\mathrm{阶行列式的每一项为}n\mathrm{个不同元素的乘积},\mathrm{从而每一项均为零},\mathrm{故该行列式为为零}.\end{array}
$$



$$
\mathrm{四阶行列式}\begin{vmatrix}0&0&1&0\\0&1&0&0\\0&0&0&1\\1&0&0&0\end{vmatrix}和\begin{vmatrix}1&1&1&0\\0&1&0&1\\0&1&1&1\\0&0&1&0\end{vmatrix}\mathrm{的值分别为}(\;).
$$
$$
A.
1,0 \quad B.0,0 \quad C.1,1 \quad D.0,1 \quad E. \quad F. \quad G. \quad H.
$$
$$
\begin{array}{l}(1)\begin{vmatrix}0&0&1&0\\0&1&0&0\\0&0&0&1\\1&0&0&0\end{vmatrix}=(-1)^{τ(3241)}=1,\\(2)\begin{vmatrix}1&1&1&0\\0&1&0&1\\0&1&1&1\\0&0&1&0\end{vmatrix}=(-1)^{τ(1243)}+(-1)^{τ(1423)}=0.\end{array}
$$



$$
设\begin{vmatrix}x&x&1\\2x&x&1\\3&2&x\end{vmatrix},\mathrm{则展开式中}x^2\;\mathrm{的系数为}(\;).
$$
$$
A.
1 \quad B.-1 \quad C.0 \quad D.2 \quad E. \quad F. \quad G. \quad H.
$$
$$
\begin{vmatrix}x&x&1\\2x&x&1\\3&2&x\end{vmatrix}=\begin{vmatrix}x&x&1\\0&-x&-1\\3&2&x\end{vmatrix}=\begin{vmatrix}x&0&1\\0&-x&-1\\3&-1&x\end{vmatrix}\mathrm{展开式不会出现}x^2.
$$



$$
\mathrm{设多项式}f(x)=\begin{vmatrix}3&-1&x\\x&2&5\\1&4&x\end{vmatrix},则f(x)是()\mathrm{次多项式}.
$$
$$
A.
0 \quad B.1 \quad C.2 \quad D.3 \quad E. \quad F. \quad G. \quad H.
$$
$$
f(x)=\begin{vmatrix}3&-1&x\\x&2&5\\1&4&x\end{vmatrix}=\begin{vmatrix}3&-1&x\\x&2&5\\-2&5&0\end{vmatrix}取a_{13}a_{21}a_{32}\mathrm{就会出现}x\mathrm{的最高次数}2.
$$



$$
\mathrm{用行列式的定义计算五阶行列式}\begin{vmatrix}a_{11}&a_{12}&a_{13}&a_{14}&a_{15}\\a_{21}&a_{22}&a_{23}&a_{24}&a_{25}\\a_{31}&a_{32}&0&0&0\\a_{41}&a_{42}&0&0&0\\a_{51}&a_{52}&0&0&0\end{vmatrix}=(\;).
$$
$$
A.
1 \quad B.a_{11}a_{12} \quad C.0 \quad D.a_{11}a_{22}a_{33}a_{44}a_{55} \quad E. \quad F. \quad G. \quad H.
$$
$$
\begin{array}{l}\mathrm{由于在五阶行列式的展开式中每一项的五个元素均分布在不同的行和不同的列},\mathrm{显然任意的这样五个元素中都至少有一个为零}\\\mathrm{从而行列式的每一项为零},\mathrm{故所求行列式为零}.\end{array}
$$



$$
\mathrm{用行列式的定义计算五阶行列式}\begin{vmatrix}0&a_{12}&a_{13}&0&0\\a_{21}&a_{22}&a_{23}&a_{24}&a_{25}\\a_{31}&a_{32}&a_{33}&a_{34}&a_{35}\\0&a_{42}&a_{43}&0&0\\0&a_{52}&a_{53}&0&0\end{vmatrix}=(\;\;\;).
$$
$$
A.
0 \quad B.a_{11}a_{22}a_{33}a_{44}a_{55} \quad C.-a_{11}a_{22}a_{33}a_{44}a_{55} \quad D.a_{22}a_{33} \quad E. \quad F. \quad G. \quad H.
$$
$$
\begin{array}{l}设D\mathrm{中第}1,2,3,4,5\mathrm{行的元素分别为}a_{1p_1},a_{2p_2},a_{3p_3},a_{4p4},a_{5p_5},\mathrm{则由}D\mathrm{中第}1,2,3,4,5\mathrm{行可能的非零元素分别得到}p_1=2,3;p_2=1,2,3,4,5;p_3=1,2,3,4,5\\p_4=2,3;p_5=2,3;\;p_1,p_2,p_3,p_4,p_5\mathrm{在上述可能的代码中},\mathrm{一个}5\mathrm{元排列也不能组成},故D=0.\\\\\end{array}
$$



$$
\mathrm{设多项式}f(x)=\begin{vmatrix}3&-1&x\\x&2&5\\1&4&x\end{vmatrix},则f(x)\mathrm{一次项的系数为}(\;).
$$
$$
A.
-2 \quad B.-4 \quad C.2 \quad D.4 \quad E. \quad F. \quad G. \quad H.
$$
$$
\mathrm{一次项为}(-1)^{τ(321)}x×2×1+3×2× x=4x,\mathrm{故系数为}4.
$$



$$
\mathrm{按行列式的定义计算}n\mathrm{阶行列式}\begin{vmatrix}0&0&...&0&a_{1n}\\0&0&...&a_{2(n-1)}&a_{2n}\\...&...&...&...&...\\0&a_{(n-1)2}&...&a_{(n-1)(n-1)}&a_{(n-1)n}\\a_{n1}&a_{n2}&...&a_{n(n-1)}&a_{nn}\end{vmatrix}=(\;\;\;).
$$
$$
A.
0 \quad B.-a_{1n}a_{2(n-1)}a_{3(n-2)}....a_{(n-1)2}a_{n1} \quad C.a_{1n}a_{2(n-1)}a_{3(n-2)}....a_{(n-1)2}a_{n1} \quad D.(-1)^{\textstyle\frac{n(n-1)}2}{\textstyle{}^{}}a_{1n}a_{2(n-1)}...a_{(n-1)2}a_{n1} \quad E. \quad F. \quad G. \quad H.
$$
$$
\mathrm{原式}=(-1)^{τ((n-1)....123)}a_{1n}a_{2(n-1)}...a_{(n-1)2}a_{n1}=(-1)^\frac{n(n-1)}2a_{1n}a_{2(n-1)}...a_{(n-1)2}a_{n1}.
$$



$$
n\mathrm{阶行列式}\begin{vmatrix}0&0&...&0&1\\0&0&...&2&0\\...&...&...&...&...\\0&n-1&...&0&0\\n&0&...&0&0\end{vmatrix}=(\;).
$$
$$
A.
n! \quad B.-n! \quad C.(-1)^{{\textstyle\frac12}n(n-1)}n! \quad D.(-1)^{\textstyle(n-1)}n! \quad E. \quad F. \quad G. \quad H.
$$
$$
\begin{array}{l}\mathrm{行列式的展开式中除了}(-1)^{\textstyleτ(n(n-1)....321)}n!外,\mathrm{其余各项都为零},\mathrm{由于}(-1)^{\textstyleτ(n(n-1)....321)}n!=(-1)^{{\textstyle\frac12}n(n-1)}n!,\mathrm{故原行列式}\\\mathrm{等于}(-1)^{\textstyle\frac12n(n-1)}n!.\end{array}
$$



$$
\mathrm{设多项式}f(x)=\begin{vmatrix}2x&x&1&2\\1&2x&1&-1\\3&2&x&1\\1&1&1&x\end{vmatrix},则f(x)中x^4与x^3\mathrm{的系数分别为}(\;).
$$
$$
A.
4,1 \quad B.4,-1 \quad C.1,2 \quad D.1,-2 \quad E. \quad F. \quad G. \quad H.
$$
$$
\begin{array}{l}含x^4\mathrm{的项只能由}a_{11}a_{22}a_{33}a_{44}\mathrm{组成},\mathrm{其系数为}4,\\含x^3\mathrm{的项只能由}a_{12}a_{21}a_{33}a_{44}\mathrm{组成},\mathrm{其系数为}-1.\end{array}
$$



$$
n\mathrm{阶行列式}\begin{vmatrix}a_{11}&...&...&a_{1(n-1)}&a_{1n}\\a_{21}&...&...&a_{2(n-1)}&0\\a_{31}&...&...&0&0\\...&...&...&...&...\\a_{n1}&...&...&0&0\end{vmatrix}=(\;\;).
$$
$$
A.
(-1)^{{\textstyle\frac12}n(n-1)}a_{1n}a_{2(n-1)}....a_{n1} \quad B.a_{1n}a_{2(n-1)}....a_{n1} \quad C.-a_{1n}a_{2(n-1)}....a_{n1} \quad D.(-1)^na_{1n}a_{2(n-1)}....a_{n1} \quad E. \quad F. \quad G. \quad H.
$$
$$
\begin{array}{l}\mathrm{根据行列式的定义},\mathrm{行列式中唯一可能不为零的项为}(-1)^{τ(n(n-1)....321)}a_{1n}a_{2(n-1)}....a_{n1}\\τ(n(n-1)....321)=\frac{n(n-1)}2,\mathrm{故原行列式为}(-1)^\frac{n(n-1)}2a_{1n}a_{2(n-1)}....a_{n1}.\end{array}
$$



$$
\mathrm{行列式}D=\begin{vmatrix}0&0&...&0&1&0\\0&0&...&2&0&0\\...&...&...&...&...&...\\0&2015&...&0&0&0\\2016&0&...&0&0&0\\0&0&...&0&0&2017\end{vmatrix}=(\;\;\;).
$$
$$
A.
-2017! \quad B.2017! \quad C.0 \quad D.2017 \quad E. \quad F. \quad G. \quad H.
$$
$$
\begin{array}{l}\mathrm{第一行的非零元素只有}a_{1,2016},故p_1\mathrm{只能取}2016,\mathrm{同理第},3,4,...2016\mathrm{可知}p_2=2001,p_3=2000...j_{2016}=1,p_{2017}=2017,\mathrm{于是在可能取的数码中},\\\mathrm{只能组成一个}2017\mathrm{级排列},\mathrm{故中非零项只有一项},即\;\;\;D=(-1)^{τ(2016\;2015....2\;1\;2017)}\;a_{1,2016}a_{2,2015}\;.....a_{2016,1}a_{2017,2017}=2017!\;.\;\;\;\;\;\end{array}
$$



$$
\mathrm{四阶行列式}\begin{vmatrix}-1&0&x&1\\1&1&-1&-1\\1&-1&1&-1\\1&-1&-1&1\end{vmatrix}\;,\mathrm{则展开式中}x\mathrm{的系数为}(\;)\;.
$$
$$
A.
-4 \quad B.4 \quad C.2 \quad D.-2 \quad E. \quad F. \quad G. \quad H.
$$
$$
\begin{array}{l}x\mathrm{的系数应为}(-1)^{1+3}\begin{vmatrix}1&1&-1\\1&-1&-1\\1&-1&1\end{vmatrix}=-4.\\\end{array}
$$



$$
n\mathrm{阶行列式}D_n=\begin{vmatrix}0&0&...&0&1&0\\0&0&...&2&0&0\\...&...&...&...&...&...\\n-1&0&...&0&0&0\\0&0&...&0&0&n\end{vmatrix}=(\;\;\;).
$$
$$
A.
(-1)^\frac{(n-1)(n-2)}2\;\;\;n! \quad B.n! \quad C.(-1)^nn! \quad D.(-1)^\frac{(n+1)(n+2)}2\;\;n! \quad E. \quad F. \quad G. \quad H.
$$
$$
\begin{array}{l}D_n=(-1)^τ a_{1(n-1)}a_{2(n-2)}...a_{nn}=(-1)^τ n!,\mathrm{其中}τ=τ((n-1)(n-2)...21n)=0+1+2+....+(n-2)+0=\frac{(n-1)(n-2)}2\\故D_n=(-1)^\frac{(n-1)(n-2)}2n!.\end{array}
$$



$$
n\mathrm{阶行列式}\begin{vmatrix}a_{11}&&&&a_{1n}\\&a_{22}&&&\\&&...&&\\&&&...&\\0&&&&a_{nn}\end{vmatrix}=\begin{vmatrix}a_{11}&&&&0\\&a_{22}&&&\\&&...&&\\&&&...&\\a_{n1}&&&&a_{nn}\end{vmatrix}=(\;).
$$
$$
A.
0 \quad B.a_{11}a_{22}...a_{nn} \quad C.-a_{11}a_{22}...a_{nn} \quad D.1 \quad E. \quad F. \quad G. \quad H.
$$
$$
\begin{vmatrix}a_{11}&&&&a_{1n}\\&a_{22}&&&\\&&...&&\\&&&...&\\0&&&&a_{nn}\end{vmatrix}=\begin{vmatrix}a_{11}&&&&0\\&a_{22}&&&\\&&...&&\\&&&...&\\a_{n1}&&&&a_{nn}\end{vmatrix}=a_{11}a_{22}...a_{nn}.
$$



$$
n\mathrm{阶行列式}\begin{vmatrix}0&&&&a_{1n}\\&&&a_{2(n-1)}&\\&&...&&\\&...&&&\\a_{n1}&&&&0\end{vmatrix}=(\;).
$$
$$
A.
a_{1n}a_{2(n-1)}.....a_{n1} \quad B.-a_{1n}a_{2(n-1)}.....a_{n1} \quad C.(-1)^{\textstyle\frac{n(n-1)}2}a_{1n}a_{2(n-1)}.....a_{n1} \quad D.(-1)^{\textstyle\frac n2}a_{1n}a_{2(n-1)}.....a_{n1} \quad E. \quad F. \quad G. \quad H.
$$
$$
\begin{vmatrix}0&&&&a_{1n}\\&&&a_{2(n-1)}&\\&&...&&\\&...&&&\\a_{n1}&&&&0\end{vmatrix}=(-1)^{τ(n(n-1)....1)}a_{1n}a_{2(n-1)}.....a_{n1}=(-1)^{\textstyle\frac{n(n-1)}2}a_{1n}a_{2(n-1)}.....a_{n1}.
$$



$$
\mathrm{双曲面}x^2-y^2/4-z^2/9=1\mathrm{与平面}y=4\mathrm{交线为}(\;).
$$
$$
A.
\mathrm{双曲线} \quad B.\mathrm{椭圆} \quad C.\mathrm{抛物线} \quad D.\mathrm{一对相交直线} \quad E. \quad F. \quad G. \quad H.
$$
$$
将y=4\mathrm{代入到双曲面方程得}x^2-z^2/9=5,\mathrm{所以相交的曲线为双曲线}.
$$



$$
\mathrm{若五阶行列式的展开式中项}a_{13}a_{2k}a_{34}a_{42}a_{5l}\mathrm{带负号},\mathrm 则k,l\mathrm{的值分别为}(\;).
$$
$$
A.
k=1,l=5 \quad B.k=5,l=1 \quad C.k=2,l=5 \quad D.k=5,l=2 \quad E. \quad F. \quad G. \quad H.
$$
$$
\begin{array}{l}\begin{array}{l}\mathrm{根据行列式的定义},k,l\mathrm{只能取}1或5,\\若k=5,l=1,则τ(35421)=8,\end{array}\\\begin{array}{l}若k=1,l=5,则τ(31425)=3,\\\mathrm{所以}k=1,l=5.\\\end{array}\end{array}
$$



$$
\begin{array}{l}\mathrm{二阶行列式}\begin{vmatrix}\cos\;α&-\sin\;α\\\sinα&\cos\;α\end{vmatrix}=(\;).\;\\\end{array}
$$
$$
A.
-1 \quad B.1 \quad C.2\sin^2α \quad D.2\cos^2α \quad E. \quad F. \quad G. \quad H.
$$
$$
\begin{vmatrix}\cos\;α&-\sin\;α\\\sinα&\cos\;α\end{vmatrix}=\cos^2α-(-\sin^2α)=\cos^2\alpha+\sin^2α=1.
$$



$$
\mathrm{若三阶行列式}\begin{vmatrix}1&2&5\\1&3&-2\\2&5&x\end{vmatrix}=0,则x\mathrm{的值为}().
$$
$$
A.
-3 \quad B.-2 \quad C.2 \quad D.3 \quad E. \quad F. \quad G. \quad H.
$$
$$
\begin{vmatrix}1&2&5\\1&3&-2\\2&5&x\end{vmatrix}=3x-8+25-30+10-2x=0,x=3.
$$



$$
\mathrm{三阶行列式}\begin{vmatrix}1&2&3\\4&0&5\\-1&0&6\end{vmatrix}\mathrm{的值为}().
$$
$$
A.
0 \quad B.-58 \quad C.1 \quad D.-48 \quad E. \quad F. \quad G. \quad H.
$$
$$
\begin{vmatrix}1&2&3\\4&0&5\\-1&0&6\end{vmatrix}=1×0×6+2×5×(-1)+3×0×4-3×0×(-1)-2×4×6-1×0×5=-10-48=-58.
$$



$$
\mathrm{三阶行列式}\begin{vmatrix}a&b&c\\b&c&a\\c&a&b\end{vmatrix}\mathrm{的值为}().
$$
$$
A.
3abc-a^3-b^3-c^3 \quad B.3abc+a^3+b^3+c^3 \quad C.abc-a^3-b^3-c^3 \quad D.abc+a^3+b^3+c^3 \quad E. \quad F. \quad G. \quad H.
$$
$$
\begin{vmatrix}a&b&c\\b&c&a\\c&a&b\end{vmatrix}=acb+bac+cba-bbb-aaa-ccc=3abc-a^3-b^3-c^3.
$$



$$
\mathrm{三阶行列式}\begin{vmatrix}a&b&b\\b&a&b\\b&b&a\end{vmatrix}\mathrm{的值为}(\;).
$$
$$
A.
(a-b)^3 \quad B.a^3-2b^3-3ab^2 \quad C.a^3+2b^3-3ab^2 \quad D.a^2+2b^3+3ab^3 \quad E. \quad F. \quad G. \quad H.
$$
$$
\begin{vmatrix}a&b&b\\b&a&b\\b&b&a\end{vmatrix}=a^3+b^3+b^3-ab^2-b^2a-b^2a=a^3+2b^3-3ab^2.
$$



$$
若a,b∈ R,\begin{vmatrix}a&b&0\\-b&a&0\\100&0&-1\end{vmatrix}=0,则a,b\mathrm{的值分别为}().
$$
$$
A.
a=0,b=1 \quad B.a=0,b=0 \quad C.a=1,b=0 \quad D.a=1,b=1 \quad E. \quad F. \quad G. \quad H.
$$
$$
\begin{vmatrix}a&b&0\\-b&a&0\\100&0&-1\end{vmatrix}=-a^2-b^2=0,a^2+b^2=0,a=b=0.
$$



$$
\mathrm{三阶行列式}\begin{vmatrix}a&1&1\\0&-1&0\\4&a&a\end{vmatrix}<0\mathrm{的充要条件是}(\;\;\;).
$$
$$
A.
a<2 \quad B.a>-2 \quad C.\left|a\right|>2 \quad D.\left|a\right|<2 \quad E. \quad F. \quad G. \quad H.
$$
$$
\begin{vmatrix}a&1&1\\0&-1&0\\4&a&a\end{vmatrix}=-a^2+4<0,\left|a\right|>2.
$$



$$
\mathrm{三阶行列式}\begin{vmatrix}1&2&3\\4&0&5\\7&0&6\end{vmatrix}\mathrm{的值为}().
$$
$$
A.
23 \quad B.-24 \quad C.22 \quad D.-25 \quad E. \quad F. \quad G. \quad H.
$$
$$
\begin{vmatrix}1&2&3\\4&0&5\\7&0&6\end{vmatrix}=1×0×6+2×5×7+3×0×4-3×0×7-2×4×6-1×0×5=22.
$$



$$
\mathrm{若三阶行列式}\begin{vmatrix}1&2&5\\3&7&x\\6&9&0\end{vmatrix}=0,则x=(\;).
$$
$$
A.
-25 \quad B.-30 \quad C.25 \quad D.30 \quad E. \quad F. \quad G. \quad H.
$$
$$
\begin{vmatrix}1&2&5\\3&7&x\\6&9&0\end{vmatrix}=12x+135-210-9x=3x-75=0,x=25.
$$



$$
\mathrm{三阶行列式}\begin{vmatrix}a&-1&1\\0&1&0\\9&a&a\end{vmatrix}>0\mathrm{的充要条件是}(\;).
$$
$$
A.
a<3 \quad B.a>-3 \quad C.\left|a\right|>3 \quad D.\left|a\right|<3 \quad E. \quad F. \quad G. \quad H.
$$
$$
\begin{vmatrix}a&-1&1\\0&1&0\\9&a&a\end{vmatrix}=a^2-9>0,则\left|a\right|>3.
$$



$$
\mathrm{三阶行列式}\begin{vmatrix}1&2&0\\0&3&1\\3&0&2\end{vmatrix}\mathrm{的值为}(\;).
$$
$$
A.
-18 \quad B.18 \quad C.12 \quad D.-12 \quad E. \quad F. \quad G. \quad H.
$$
$$
\begin{vmatrix}1&2&0\\0&3&1\\3&0&2\end{vmatrix}=1×2×3+1×2×3-0=12.
$$



$$
\mathrm{三阶行列式}\begin{vmatrix}2&1&0\\1&0&2\\0&2&1\end{vmatrix}\mathrm{的值为}(\;).
$$
$$
A.
45 \quad B.-45 \quad C.9 \quad D.-9 \quad E. \quad F. \quad G. \quad H.
$$
$$
\begin{vmatrix}2&1&0\\1&0&2\\0&2&1\end{vmatrix}=-1^3-2^3=-9.
$$



$$
\mathrm{三阶行列式}\begin{vmatrix}1&0&-1\\3&5&0\\0&4&1\end{vmatrix}\mathrm{的值为}(\;).
$$
$$
A.
7 \quad B.-7 \quad C.5 \quad D.-5 \quad E. \quad F. \quad G. \quad H.
$$
$$
\begin{vmatrix}1&0&-1\\3&5&0\\0&4&1\end{vmatrix}=1×5×1+0×0×0+(-1)×3×4-(-1)×5×0-3×0×1-1×4×0=-7.
$$



$$
\mathrm{三阶行列式}\begin{vmatrix}2&0&0\\1&-4&-1\\-1&8&3\end{vmatrix}\mathrm{的值为}(\;).
$$
$$
A.
-5 \quad B.-8 \quad C.-4 \quad D.3 \quad E. \quad F. \quad G. \quad H.
$$
$$
\begin{vmatrix}2&0&0\\1&-4&-1\\-1&8&3\end{vmatrix}=2×(-4)×3-2×8×(-1)=-24+16=-8.
$$



$$
\mathrm{三阶行列式}\begin{vmatrix}0&a&0\\b&0&c\\0&d&0\end{vmatrix}\mathrm{的值为}(\;\;\;).
$$
$$
A.
0 \quad B.abcd \quad C.-abcd \quad D.acd \quad E. \quad F. \quad G. \quad H.
$$
$$
\begin{vmatrix}0&a&0\\b&0&c\\0&d&0\end{vmatrix}=0×0×0+b× d×0+0× c× a-0×0×0-a× b×0-0× d× c=0.
$$



$$
\mathrm{三阶行列式}\begin{vmatrix}1&1&1\\a&b&c\\a^2&b^2&c^2\end{vmatrix}\mathrm{的值为}(\;).
$$
$$
A.
(a-b)(b-c)(c-a) \quad B.(a-b)(b-c)(a-c) \quad C.(a-1)(b-1)(c-1) \quad D.(1-b)(1-c)(1-a) \quad E. \quad F. \quad G. \quad H.
$$
$$
\begin{vmatrix}1&1&1\\a&b&c\\a^2&b^2&c^2\end{vmatrix}=bc^2+ca^2+ab^2-ac^2-ba^2-cb^2=(a-b)(b-c)(c-a)\mathrm{或者利用范德蒙行列式可得}.
$$



$$
\mathrm{三阶行列式}\begin{vmatrix}1&1&1\\3&1&4\\0&0&5\end{vmatrix}\mathrm{的值为}(\;\;\;).
$$
$$
A.
8 \quad B.5 \quad C.15 \quad D.-10 \quad E. \quad F. \quad G. \quad H.
$$
$$
\begin{vmatrix}1&1&1\\3&1&4\\0&0&5\end{vmatrix}=1×1×5-3×1×5=-10.
$$



$$
\mathrm{三阶行列式}\begin{vmatrix}2&0&1\\1&-4&-1\\-1&0&3\end{vmatrix}\mathrm{的值为}(\;).
$$
$$
A.
8 \quad B.-24 \quad C.16 \quad D.-28 \quad E. \quad F. \quad G. \quad H.
$$
$$
\begin{vmatrix}2&0&1\\1&-4&-1\\-1&0&3\end{vmatrix}=2×(-4)×3-1×(-4)×(-1)=-28.
$$



$$
\mathrm{二阶行列式}\begin{vmatrix}x-1&1\\x^3&x^2+x+1\end{vmatrix}\mathrm{的值为}(\;\;\;).
$$
$$
A.
0 \quad B.1 \quad C.x^2+x \quad D.-1 \quad E. \quad F. \quad G. \quad H.
$$
$$
\begin{vmatrix}x-1&1\\x^3&x^2+x+1\end{vmatrix}=(x-1)(x^2+x+1)-x^3=-1.
$$



$$
\mathrm{二阶行列式}\begin{vmatrix}a+b&b+d\\a+c&c+d\end{vmatrix}\mathrm{的值为}(\;\;\;).
$$
$$
A.
(a-d)(c-b) \quad B.(a-b)(c-d) \quad C.(a-d)(c+b) \quad D.(a-b)(c+d) \quad E. \quad F. \quad G. \quad H.
$$
$$
\begin{vmatrix}a+b&b+d\\a+c&c+d\end{vmatrix}=(a+b)(c+d)-(a+c)(b+d)=ac+bc+ad+bd-ab-bc-ad-cd=ac-ab+bd-cd=a(c-b)-d(c-b)=(a-d)(c-b).
$$



$$
\mathrm{二阶行列式}\begin{vmatrix}a^2&ab\\b&b\end{vmatrix}\mathrm{的值为}(\;\;\;).
$$
$$
A.
a^3b-b^2 \quad B.b^2-a^3b \quad C.a^2b-ab^2 \quad D.a^2b+ab^2 \quad E. \quad F. \quad G. \quad H.
$$
$$
\begin{vmatrix}a^2&ab\\b&b\end{vmatrix}=a^2b-ab^2.
$$



$$
\mathrm{二阶行列式}\begin{vmatrix}a+b&b\\a+c&c\end{vmatrix}\mathrm{的值为}(\;\;\;).
$$
$$
A.
ac-ab \quad B.ac+b \quad C.ac-b \quad D.ac+bc \quad E. \quad F. \quad G. \quad H.
$$
$$
\begin{vmatrix}a+b&b\\a+c&c\end{vmatrix}=(a+b)c-(a+c)b=ac+bc-ab-bc=ac-ab.
$$



$$
\mathrm{二阶行列式}\begin{vmatrix}1+\sqrt2&2-\sqrt3\\2+\sqrt3&1-\sqrt2\end{vmatrix}\mathrm{的值为}(\;\;\;).
$$
$$
A.
\sqrt2 \quad B.\sqrt3 \quad C.2 \quad D.-2 \quad E. \quad F. \quad G. \quad H.
$$
$$
\begin{vmatrix}1+\sqrt2&2-\sqrt3\\2+\sqrt3&1-\sqrt2\end{vmatrix}=(1+\sqrt2)(1-\sqrt2)-(2+\sqrt3)(2-\sqrt3)=1-2-4+3=-2.
$$



$$
\mathrm{二阶行列式}\begin{vmatrix}a_{11}&a_{12}\\a_{21}&a_{22}\end{vmatrix}\mathrm{的值为}(\;\;\;).
$$
$$
A.
a_{11}a_{12}-a_{22}a_{21} \quad B.a_{11}a_{12}+a_{22}a_{21} \quad C.a_{11}a_{22}-a_{12}a_{21} \quad D.a_{11}a_{22}+a_{12}a_{21} \quad E. \quad F. \quad G. \quad H.
$$
$$
\begin{vmatrix}a_{11}&a_{12}\\a_{21}&a_{22}\end{vmatrix}=a_{11}a_{22}-a_{12}a_{21}.
$$



$$
\mathrm{二阶行列式}\begin{vmatrix}a&b\\a^2&b\end{vmatrix}\mathrm{的值为}(\;\;\;).
$$
$$
A.
ab^2 \quad B.a^2b \quad C.ab(1-a) \quad D.ab(a-b) \quad E. \quad F. \quad G. \quad H.
$$
$$
\begin{vmatrix}a&b\\a^2&b\end{vmatrix}=ab-a^2b=ab(1-a).
$$



$$
\mathrm{二阶行列式}\begin{vmatrix}x-1&1\\x^2&x^2+x+1\end{vmatrix}\mathrm{的值为}(\;).
$$
$$
A.
1 \quad B.-1 \quad C.x^3-x^2-1 \quad D.x^3-x^2+1 \quad E. \quad F. \quad G. \quad H.
$$
$$
\begin{vmatrix}x-1&1\\x^2&x^2+x+1\end{vmatrix}=(x-1)(x^2+x+1)-1× x^2=x^3-x^2-1.
$$



$$
\mathrm{二阶行列式}\begin{vmatrix}1-t^2&2t\\-2t&1-t^2\end{vmatrix}\mathrm{的值为}(\;\;\;).
$$
$$
A.
(1+t^2)^2 \quad B.1+t^2 \quad C.(1-t^2)^2 \quad D.1 \quad E. \quad F. \quad G. \quad H.
$$
$$
\begin{vmatrix}1-t^2&2t\\-2t&1-t^2\end{vmatrix}=(1-t^2)(1-t^2)-(-2t)×2t=1+t^4+2t^2=(1+t^2)^2.
$$



$$
\mathrm{三阶行列式}\begin{vmatrix}0&2&3\\3&0&2\\2&3&0\end{vmatrix}\mathrm{的值为}(\;).
$$
$$
A.
12 \quad B.15 \quad C.35 \quad D.21 \quad E. \quad F. \quad G. \quad H.
$$
$$
\begin{vmatrix}0&2&3\\3&0&2\\2&3&0\end{vmatrix}=2×2×2+3×3×3=35.
$$



$$
\mathrm{三阶行列式}\begin{vmatrix}3&1&x\\4&x&0\\1&0&x\end{vmatrix}\neq0,\mathrm{则有}(\;).
$$
$$
A.
x\neq0 \quad B.x\neq2 \quad C.x\neq0且x\neq2 \quad D.x\neq0或x\neq2 \quad E. \quad F. \quad G. \quad H.
$$
$$
\begin{vmatrix}3&1&x\\4&x&0\\1&0&x\end{vmatrix}=3× x× x+0+0-x× x×1-0-1×4× x=3x^2-x^2-4x=2x(x-2)\neq0.
$$



$$
\mathrm{已知}ω^3=1,则\begin{vmatrix}ω&ω^2&1\\ω^2&1&ω\\1&ω&ω^2\end{vmatrix}=().
$$
$$
A.
1 \quad B.-1 \quad C.0 \quad D.ω \quad E. \quad F. \quad G. \quad H.
$$
$$
\begin{vmatrix}ω&ω^2&1\\ω^2&1&ω\\1&ω&ω^2\end{vmatrix}=ω^3+ω^3+ω^3-1-ω^3-ω^6=0.
$$



$$
\mathrm{三阶行列式}\begin{vmatrix}b^2&ab&ac\\ab&c^2&bc\\ca&cb&a^2\end{vmatrix}\mathrm{的值为}(\;\;\;).
$$
$$
A.
a^2b^2c^2 \quad B.2a^2b^2c^2 \quad C.3a^2b^2c^2-a^2c^4-b^2a^4-c^2b^4 \quad D.0 \quad E. \quad F. \quad G. \quad H.
$$
$$
\begin{vmatrix}b^2&ab&ac\\ab&c^2&bc\\ca&cb&a^2\end{vmatrix}=3a^2b^2c^2-a^2c^4-b^2a^4-c^2b^4.
$$



$$
\mathrm{三阶行列式}\begin{vmatrix}1&2&3\\4&5&6\\3&2&1\end{vmatrix}\mathrm{的值为}(\;\;\;).
$$
$$
A.
-1 \quad B.-2 \quad C.0 \quad D.-4 \quad E. \quad F. \quad G. \quad H.
$$
$$
\begin{vmatrix}1&2&3\\4&5&6\\3&2&1\end{vmatrix}=5+24+36-45-8-12=0.
$$



$$
\mathrm{三阶行列式}\begin{vmatrix}1&2&3\\2&3&4\\3&4&5\end{vmatrix}\mathrm{的值为}(\;).
$$
$$
A.
0 \quad B.1 \quad C.2 \quad D.3 \quad E. \quad F. \quad G. \quad H.
$$
$$
\begin{vmatrix}1&2&3\\2&3&4\\3&4&5\end{vmatrix}=15+24+24-27-20-16=0.
$$



$$
\begin{vmatrix}a&b\\c&d\end{vmatrix}+\begin{vmatrix}0&b&a\\1&a&b\\0&d&c\end{vmatrix}=(\;).
$$
$$
A.
ad-cd \quad B.2(ad-bc) \quad C.0 \quad D.ad+bc \quad E. \quad F. \quad G. \quad H.
$$
$$
\begin{vmatrix}a&b\\c&d\end{vmatrix}+\begin{vmatrix}0&b&a\\1&a&b\\0&d&c\end{vmatrix}=(ad-bc)+(ad-bc)=2(ad-bc).
$$



$$
\begin{vmatrix}0&b&a\\1&a&b\\0&d&c\end{vmatrix}-\begin{vmatrix}a&b\\c&d\end{vmatrix}=(\;).
$$
$$
A.
ad-dc \quad B.2(ad-dc) \quad C.0 \quad D.1 \quad E. \quad F. \quad G. \quad H.
$$
$$
\begin{vmatrix}0&b&a\\1&a&b\\0&d&c\end{vmatrix}-\begin{vmatrix}a&b\\c&d\end{vmatrix}=(ad-bc)-(ad-bc)=0.
$$



$$
\mathrm{二阶行列式}\begin{vmatrix}1&\log_ba\\\log_ab&1\end{vmatrix}\mathrm{的值为}(\;).
$$
$$
A.
1 \quad B.0 \quad C.-1 \quad D.与a,b\mathrm{有关} \quad E. \quad F. \quad G. \quad H.
$$
$$
\begin{vmatrix}1&\log_ba\\\log_ab&1\end{vmatrix}=1×1-\log_ba×\log_ab=0.
$$



$$
\mathrm{方程}\begin{vmatrix}1&1&1\\2&3&x\\4&9&x^2\end{vmatrix}=0\mathrm{的解为}(\;).
$$
$$
A.
x=2或x=3 \quad B.x=2 \quad C.x=3 \quad D.\mathrm{无解} \quad E. \quad F. \quad G. \quad H.
$$
$$
\begin{vmatrix}1&1&1\\2&3&x\\4&9&x^2\end{vmatrix}=3x^2+4x+18-12-9x-2x^2=x^2-5x+6,则x=2或x=3.
$$



$$
\begin{array}{l}\mathrm{下列计算正确的个数有}(\;)个.\\(1)\begin{vmatrix}1&0&-1\\3&5&0\\0&4&1\end{vmatrix}=-7\;\;\;(2)\begin{vmatrix}2&0&1\\1&-4&-1\\-1&8&3\end{vmatrix}=4\;\;\;(3)\begin{vmatrix}0&a&0\\b&0&c\\0&d&0\end{vmatrix}=0\end{array}
$$
$$
A.
0 \quad B.2个 \quad C.1个 \quad D.3个 \quad E. \quad F. \quad G. \quad H.
$$
$$
\begin{array}{l}(1)\begin{vmatrix}1&0&-1\\3&5&0\\0&4&1\end{vmatrix}=1×5×1+0×0×0+(-1)×3×4-(-1)×5×0-3×0×1-1×4×0=-7,\\\;(2)\begin{vmatrix}2&0&1\\1&-4&-1\\-1&8&3\end{vmatrix}=2×(-4)×3+0×(-1)×(-1)+1×1×8-0×1×3-2×(-1)×8-1×(-4)×(-1)=-24+8+16-4=-4,\\\;(3)\begin{vmatrix}0&a&0\\b&0&c\\0&d&0\end{vmatrix}=0×0×0+0× b× d+0× a× c-0×0×0-b× a×0-0× d× c=0.\end{array}
$$



$$
\mathrm{三阶行列式}D_1=\begin{vmatrix}1&3&1\\2&2&3\\3&1&5\end{vmatrix},D_2=\begin{vmatrix}λ&0&1\\0&λ-1&0\\1&0&λ\end{vmatrix},若D_1=D_2,则λ\mathrm{取值为}(\;).
$$
$$
A.
0,1 \quad B.0,2 \quad C.1,-1 \quad D.2,-1 \quad E. \quad F. \quad G. \quad H.
$$
$$
\begin{array}{l}D_1=10+2+27-6-30-3=0,\\D_2=(λ+1)(λ-1)^2,\\若D_1=D_2,则(λ+1)(λ-1)^2=0,\mathrm{解得λ}=1\mathrm{或者}λ=-1.\end{array}
$$



$$
\mathrm{三阶行列式}\begin{vmatrix}x&y&x+y\\y&x+y&x\\x+y&x&y\end{vmatrix}\mathrm{的值为}(\;).
$$
$$
A.
-2(x^3+y^3) \quad B.2(x^3+y^3) \quad C.-(x^3+y^3) \quad D.(x^3+y^3) \quad E. \quad F. \quad G. \quad H.
$$
$$
\begin{array}{l}\begin{vmatrix}x&y&x+y\\y&x+y&x\\x+y&x&y\end{vmatrix}=x(x+y)y+yx(x+y)+(x+y)yx-y^3-(x+y)^3-x^3\\=3xy(x+y)-y^3-3x^2y-3y^2x-x^3-y^3-x^3=-2(x^3+y^3).\end{array}
$$



$$
设τ(\;)\mathrm{表示排列的逆序数},则3+(-1)^{τ(23541)}=(\;).
$$
$$
A.
2 \quad B.4 \quad C.8 \quad D.-2 \quad E. \quad F. \quad G. \quad H.
$$
$$
3+(-1)^{τ(23541)}=3+(-1)^5=2.
$$



$$
\mathrm{下列五级排列中},\mathrm{第一},\mathrm{第二位置是}4,1,\mathrm{则是偶排列的为}(\;).
$$
$$
A.
41235 \quad B.41352 \quad C.41532 \quad D.41523 \quad E. \quad F. \quad G. \quad H.
$$
$$
\mathrm{排列}41532\mathrm{的逆序数为}1+2+3=6,\mathrm{为偶排列}.
$$



$$
\mathrm{排列}1432\mathrm{的逆序数为}(\;).
$$
$$
A.
1 \quad B.2 \quad C.3 \quad D.4 \quad E. \quad F. \quad G. \quad H.
$$
$$
\mathrm{排列}1432\mathrm{的逆序数为}1+2=3.
$$



$$
\mathrm{排列}2431\mathrm{的逆序数为}(\;).
$$
$$
A.
1 \quad B.2 \quad C.3 \quad D.4 \quad E. \quad F. \quad G. \quad H.
$$
$$
\mathrm{排列}2431\mathrm{的逆序数为}1+3=4.
$$



$$
\mathrm{排列}431256\mathrm{的逆序数为}(\;).
$$
$$
A.
6 \quad B.7 \quad C.4 \quad D.5 \quad E. \quad F. \quad G. \quad H.
$$
$$
\mathrm{排列}431256\mathrm{的逆序数为}1+2+2=5.
$$



$$
\mathrm{排列}53124\mathrm{的逆序数为}(\;).
$$
$$
A.
7 \quad B.4 \quad C.5 \quad D.6 \quad E. \quad F. \quad G. \quad H.
$$
$$
\mathrm{排列}53124\mathrm{的逆序数为}1+2+2+1=6.
$$



$$
\mathrm{排列}123654\mathrm{的逆序数为}(\;).
$$
$$
A.
5 \quad B.2 \quad C.3 \quad D.4 \quad E. \quad F. \quad G. \quad H.
$$
$$
\mathrm{排列}123654\mathrm{的逆序数为}1+2=3.
$$



$$
在n\mathrm{阶排列中}(n⩾2),\mathrm{奇排列与偶排列的个数相同},\mathrm{各为}(\;)个.
$$
$$
A.
n \quad B.n! \quad C.\frac12n! \quad D.n^2 \quad E. \quad F. \quad G. \quad H.
$$
$$
\begin{array}{l}设n!\mathrm{个排列中},\mathrm{奇偶排列分别为}p,q个,对p\mathrm{个奇排列进行同样的一次对换},\mathrm{得到}p\;\mathrm{个偶排列},则p⩽ q,\mathrm{同理可证}q⩽ p,\\\mathrm{所以}p=q,\mathrm{因全部排列为}n!个,故p=q=\frac12n!.\end{array}
$$



$$
设τ(\;)\mathrm{表排列的逆序数},则τ(23541\;)\lbrackτ(\;7564132)-τ(\;631254)\rbrack=(\;).
$$
$$
A.
10 \quad B.0 \quad C.50 \quad D.25 \quad E. \quad F. \quad G. \quad H.
$$
$$
τ(23541)\lbrackτ(7564132)-τ(631254)\rbrack=5(18-8)=50.
$$



$$
\begin{array}{l}\mathrm{下列排列中奇排列的个数为}(\;).\\(1)4132\;\;\;(2)2413\;\;\;(3)36715284\;\;\;(4)3712456\end{array}
$$
$$
A.
1 \quad B.2 \quad C.3 \quad D.4 \quad E. \quad F. \quad G. \quad H.
$$
$$
\begin{array}{l}\mathrm{排列}4132\mathrm{的逆序数}τ=1+1+2=4,\\\mathrm{排列}2413\mathrm{逆序数为}τ=1+2=3,\\\mathrm{排列}36715284\mathrm{逆序数为}τ=3+2+4+4=13,\\\mathrm{排列}3712456\mathrm{逆序数为}τ=2+2+1+1+1=7.\\\end{array}
$$



$$
\begin{array}{l}\mathrm{下列排列逆序数正确的有}(\;)个.\\(1\;)τ(4132\;)=4\\(2)τ(2413)=3\\(3)τ(36715284\;)=10\\(4)τ(3712456\;)=7\end{array}
$$
$$
A.
1 \quad B.2 \quad C.3 \quad D.4 \quad E. \quad F. \quad G. \quad H.
$$
$$
\begin{array}{l}(1)4132\mathrm{逆序数}τ=4,\\(2)2413\mathrm{逆序数}τ=3,\\(3)36715284\mathrm{逆序数}τ=13,\\(4)3712456\mathrm{逆序数}τ=7.\end{array}
$$



$$
设τ(\;)\mathrm{表示排列的逆序数},则(-1)^{τ(312645)}+(-1{)^{τ(234156)}=(\;)}.
$$
$$
A.
0 \quad B.1 \quad C.-1 \quad D.2 \quad E. \quad F. \quad G. \quad H.
$$
$$
τ(312645)=4,τ(234156)=3,则(-1)^{τ(312645)}+(-1)^{τ(234156)}=(-1)^4+(-1)^3=0.
$$



$$
\;设τ(\;\;)\mathrm{表示排列的逆序数},则(-1)^{τ(3421)}+(-1)^{τ(132456)}=(\;\;\;).
$$
$$
A.
0 \quad B.2 \quad C.-2 \quad D.1 \quad E. \quad F. \quad G. \quad H.
$$
$$
\begin{array}{l}τ(3421)=2+3=5\\τ(132456)=1\\\mathrm{故结果为}(-1)^{τ(3421)}+(-1)^{τ(132456)}=(-1)^5+(-1)^1=-2.\end{array}
$$



$$
\mathrm{下列排列中}(\;)\mathrm{是偶排列}.
$$
$$
A.
4312 \quad B.51432 \quad C.45312 \quad D.654321 \quad E. \quad F. \quad G. \quad H.
$$
$$
\begin{array}{l}\begin{array}{l}\mathrm{排列}4312\mathrm{的逆序数为}5,\\\mathrm{排列}51432\mathrm{的逆序数为}7,\\\mathrm{排列}45312\mathrm{的逆序数为}8,\\\mathrm{排列}654321\mathrm{的逆序数为}15,\\\mathrm{根据奇偶排列的定义可知排列}45312\mathrm{为偶排列}.\end{array}\\\end{array}
$$



$$
\begin{vmatrix}a&b\\c&d\end{vmatrix}+(-1)^{τ(234156)}\begin{vmatrix}0&b&a\\1&a&b\\0&d&c\end{vmatrix}=(\;).
$$
$$
A.
ad-cd \quad B.2(ad-cd) \quad C.0 \quad D.ad+bc \quad E. \quad F. \quad G. \quad H.
$$
$$
\begin{vmatrix}a&b\\c&d\end{vmatrix}+(-1)^{τ(234156)}\begin{vmatrix}0&b&a\\1&a&b\\0&d&c\end{vmatrix}=ad-bc+(-1)^3(-bc+ad)=0.
$$



$$
\mathrm{排列}2n(2n-1)(2n-2)…21\mathrm{的逆序数为}(\;).
$$
$$
A.
2n \quad B.2n-1 \quad C.n(2n-1) \quad D.2n(n-1) \quad E. \quad F. \quad G. \quad H.
$$
$$
2n(2n-1)(2n-2)…21\mathrm{的逆序数为}0+1+2+3+...+(2n-1)=n(2n-1).
$$



$$
\mathrm{排列}n(n-1)...21,当n=(\;)时,\mathrm{排列为奇排列}.
$$
$$
A.
n=4k+3,k∈ Z \quad B.n=4k+1,k∈ Z \quad C.n=4k+2,k∈ Z \quad D.n=4k+2,n=4k+3,k∈ Z \quad E. \quad F. \quad G. \quad H.
$$
$$
\begin{array}{l}\mathrm{排列}n(n-1).....21\mathrm{的逆序数为}0+1+2+⋯+(n-1)=\frac{n(n-1)}2,\\当n=4k,4k+1时,\mathrm{排列为偶排列},\\当n=4k+2,4k+3时,\mathrm{排列为奇排列}.\end{array}
$$



$$
\mathrm{排列}n(n-1)(n-2)……321\mathrm{的逆序数为}(\;).
$$
$$
A.
\frac{n(n-1)}2 \quad B.\frac{n(n+1)}2 \quad C.\frac{n(n-2)}2 \quad D.\frac{n(n+2)}2 \quad E. \quad F. \quad G. \quad H.
$$
$$
\begin{array}{l}\mathrm{排列}n(n-1)(n-2)⋯⋯321\\\mathrm{其逆序数}τ=1+2+...+(n-3)+(n-2)+(n-1)=\frac{n(n-1)}2.\\\end{array}
$$



$$
\mathrm{排列}13⋯(2n-1)24⋯(2n)\mathrm{的逆序数为}(\;).
$$
$$
A.
\frac{n(n-1)}2 \quad B.\frac{n(n+1)}2 \quad C.\frac{n(n-2)}2 \quad D.\frac{n(n+2)}2 \quad E. \quad F. \quad G. \quad H.
$$
$$
\begin{array}{l}\mathrm{排列}\;\;\;\;\;\;\;\;\;\;\;\;\;1\;\;\;3\;\;⋯\;\;(2n-1)\;\;\;\;\;\;\;2\;\;\;\;\;\;\;\;\;\;\;\;\;4\;\;\;\;\;\;⋯\;\;\;\;\;(2n-2)\;\;\;\;2n\\\mathrm{其逆序数为}\;\;\;\;0\;\;\;0\;⋯\;\;\;\;\;\;\;\;\;0\;\;\;\;\;\;\;(n-1)\;\;\;\;\;(n-2)\;\;\;\;⋯\;\;\;\;\;\;1\;\;\;\;\;\;\;\;\;0\\\mathrm{故排列逆序数}τ=(n-1)+(n-2)+...+2+1+0=\frac{n(n-1)}2.\end{array}
$$



$$
\mathrm{排列}246...(2n)135...(2n-1)\mathrm{的逆序数为}(\;).
$$
$$
A.
n \quad B.0 \quad C.\frac{n(n-1)}2 \quad D.\frac{n(n+1)}2 \quad E. \quad F. \quad G. \quad H.
$$
$$
\mathrm{由定义可知},\mathrm{排列的逆序数为}n+(n-1)+...+1=\frac{n(n+1)}2.
$$



$$
\mathrm{四阶行列式}\begin{vmatrix}0&0&0&2\\0&0&-1&1\\0&3&0&-1\\2&0&2&1\end{vmatrix}=().
$$
$$
A.
12 \quad B.-11 \quad C.-12 \quad D.11 \quad E. \quad F. \quad G. \quad H.
$$
$$
\begin{vmatrix}0&0&0&2\\0&0&-1&1\\0&3&0&-1\\2&0&2&1\end{vmatrix}\overset{r_1↔ r_4}{\overset{r_{2\;}↔ r_3}=}\begin{vmatrix}2&0&2&1\\0&3&0&-1\\0&0&-1&1\\0&0&0&2\end{vmatrix}=2×3×\left(-1\right)×2=-12.
$$



$$
\mathrm{四阶行列式}\begin{vmatrix}0&0&0&2\\0&2&-1&0\\0&3&1&0\\2&0&0&0\end{vmatrix}=().
$$
$$
A.
10 \quad B.-20 \quad C.30 \quad D.-40 \quad E. \quad F. \quad G. \quad H.
$$
$$
\begin{vmatrix}0&0&0&2\\0&2&-1&0\\0&3&1&0\\2&0&0&0\end{vmatrix}=-\begin{vmatrix}2&0&0&0\\0&2&-1&0\\0&3&1&0\\0&0&0&2\end{vmatrix}=-\begin{vmatrix}2&0&0&0\\0&2&-1&0\\0&0&\frac52&0\\0&0&0&2\end{vmatrix}=-2×2×\frac52×2=-20.
$$



$$
\mathrm{四阶行列式}\begin{vmatrix}a_{11}&a_{12}&a_{13}&a_{14}\\a_{21}&a_{22}&a_{23}&0\\a_{31}&a_{32}&0&0\\a_{41}&0&0&0\end{vmatrix}=().
$$
$$
A.
-a_{14}a_{23}a_{32}a_{41} \quad B.a_{14}a_{23}a_{32}a_{41} \quad C.a_{14}a_{13}a_{12}a_{11} \quad D.0 \quad E. \quad F. \quad G. \quad H.
$$
$$
\begin{vmatrix}a_{11}&a_{12}&a_{13}&a_{14}\\a_{21}&a_{22}&a_{23}&0\\a_{31}&a_{32}&0&0\\a_{41}&0&0&0\end{vmatrix}=\begin{vmatrix}a_{14}&a_{13}&a_{12}&a_{11}\\0&a_{23}&a_{22}&a_{21}\\0&0&a_{32}&a_{31}\\0&0&0&a_{41}\end{vmatrix}=a_{14}a_{23}a_{32}a_{41}.
$$



$$
设\;\begin{vmatrix}x&1&1\\1&x&1\\1&1&x\end{vmatrix}=0,则x=().
$$
$$
A.
0或1 \quad B.0\;或\;-2 \quad C.1或\;-2 \quad D.0或-1 \quad E. \quad F. \quad G. \quad H.
$$
$$
\begin{vmatrix}x&1&1\\1&x&1\\1&1&x\end{vmatrix}=\begin{vmatrix}x+2&1&1\\x+2&x&1\\x+2&1&x\end{vmatrix}=\left(x+2\right)\begin{vmatrix}1&1&1\\1&x&1\\1&1&x\end{vmatrix}=\left(x+2\right)\begin{vmatrix}1&1&1\\0&x-1&0\\0&0&x-1\end{vmatrix}=\left(x+2\right)\left(x-1\right)\;^2=0.
$$



$$
设\begin{vmatrix}x&2&2\\2&x&2\\2&2&x\end{vmatrix}=0,则\;x=().
$$
$$
A.
0或2 \quad B.0\;或\;-4 \quad C.-4或\;2 \quad D.0或4 \quad E. \quad F. \quad G. \quad H.
$$
$$
\begin{vmatrix}x&2&2\\2&x&2\\2&2&x\end{vmatrix}=\begin{vmatrix}x+4&2&2\\x+4&x&2\\x+4&2&x\end{vmatrix}=\left(x+4\right)\begin{vmatrix}1&2&2\\1&x&2\\1&2&x\end{vmatrix}=\left(x+4\right)\begin{vmatrix}1&2&2\\0&x-2&0\\0&0&x-2\end{vmatrix}=\left(x+4\right)\left(x-2\right)\;^2=0.
$$



$$
设\begin{vmatrix}x&3&3\\3&x&3\\3&3&x\end{vmatrix}=0,则x=().
$$
$$
A.
0或3 \quad B.0\;或\;-6 \quad C.-6或\;3 \quad D.0或6 \quad E. \quad F. \quad G. \quad H.
$$
$$
\begin{vmatrix}x&3&3\\3&x&3\\3&3&x\end{vmatrix}=\begin{vmatrix}x+6&3&3\\x+6&x&3\\x+6&3&x\end{vmatrix}=\left(x+6\right)\begin{vmatrix}1&3&3\\1&x&3\\1&3&x\end{vmatrix}=\left(x+6\right)\begin{vmatrix}1&3&3\\0&x-3&0\\0&0&x-3\end{vmatrix}=\left(x+6\right)\left(x-3\right)^2=0.
$$



$$
设\begin{vmatrix}2x&2&2\\2&2x&2\\2&2&2x\end{vmatrix}=0,则\;x=().
$$
$$
A.
0或1 \quad B.0\;或\;-2 \quad C.1或\;-2 \quad D.0或-1 \quad E. \quad F. \quad G. \quad H.
$$
$$
\begin{vmatrix}2x&2&2\\2&2x&2\\2&2&2x\end{vmatrix}=8\begin{vmatrix}x&1&1\\1&x&1\\1&1&x\end{vmatrix}=8\begin{vmatrix}x+2&1&1\\x+2&x&1\\x+2&1&x\end{vmatrix}=8(x+2)\begin{vmatrix}1&1&1\\1&x&1\\1&1&x\end{vmatrix}=8(x+2)\begin{vmatrix}1&1&1\\0&x-1&0\\0&0&x-1\end{vmatrix}=8(x+2)(x-1)^2=0.
$$



$$
设\begin{vmatrix}x&1&1&1\\1&x&1&1\\1&1&x&1\\1&1&1&x\end{vmatrix}=0,则x=().
$$
$$
A.
-3或0 \quad B.0或1 \quad C.-3或1 \quad D.1或3 \quad E. \quad F. \quad G. \quad H.
$$
$$
\begin{vmatrix}x&1&1&1\\1&x&1&1\\1&1&x&1\\1&1&1&x\end{vmatrix}=\begin{vmatrix}x+3&1&1&1\\x+3&x&1&1\\x+3&1&x&1\\x+3&1&1&x\end{vmatrix}=\left(x+3\right)\begin{vmatrix}1&1&1&1\\1&x&1&1\\1&1&x&1\\1&1&1&x\end{vmatrix}=\left(x+3\right)\begin{vmatrix}1&1&1&1\\0&x-1&0&0\\0&0&x-1&0\\0&0&0&x-1\end{vmatrix}=\left(x+3\right)\left(x-1\right)^3=0.
$$



$$
设\begin{vmatrix}x&2&2&2\\2&x&2&2\\2&2&x&2\\2&2&2&x\end{vmatrix}=0,则x=().
$$
$$
A.
-6或0 \quad B.0或2 \quad C.-6或2 \quad D.1或6 \quad E. \quad F. \quad G. \quad H.
$$
$$
\begin{vmatrix}x&2&2&2\\2&x&2&2\\2&2&x&2\\2&2&2&x\end{vmatrix}=\begin{vmatrix}x+6&2&2&2\\x+6&x&2&2\\x+6&2&x&2\\x+6&2&2&x\end{vmatrix}=(x+6)\begin{vmatrix}1&2&2&2\\1&x&2&2\\1&2&x&2\\1&2&2&x\end{vmatrix}=(x+6)\begin{vmatrix}1&2&2&2\\0&x-2&0&0\\0&0&x-2&0\\0&0&0&x-2\end{vmatrix}=\left(x+6\right)\left(x-2\right)^3=0.
$$



$$
设\begin{vmatrix}x&3&3&3\\3&x&3&3\\3&3&x&3\\3&3&3&x\end{vmatrix}=0,则x=().
$$
$$
A.
-9或0 \quad B.0或3 \quad C.-9或3 \quad D.3或9 \quad E. \quad F. \quad G. \quad H.
$$
$$
\begin{vmatrix}x&3&3&3\\3&x&3&3\\3&3&x&3\\3&3&3&x\end{vmatrix}=\begin{vmatrix}x+9&3&3&3\\x+9&x&3&3\\x+9&3&x&3\\x+9&3&3&x\end{vmatrix}=(x+9)\begin{vmatrix}1&3&3&3\\1&x&3&3\\1&3&x&3\\1&3&3&x\end{vmatrix}=(x+9)\begin{vmatrix}1&3&3&3\\0&x-3&0&0\\0&0&x-3&0\\0&0&0&x-3\end{vmatrix}=\left(x+9\right)\left(x-3\right)^3=0.
$$



$$
\mathrm{三阶行列式}\begin{vmatrix}x+1&2&3\\1&x+2&3\\1&2&x+3\end{vmatrix}=(\;\;\;\;).\;
$$
$$
A.
\left(x+6\right)x^2 \quad B.\left(x+6\right) \quad C.\left(x+6\right)x \quad D.x^2 \quad E. \quad F. \quad G. \quad H.
$$
$$
\begin{vmatrix}x+1&2&3\\1&x+2&3\\1&2&x+3\end{vmatrix}=\begin{vmatrix}x+6&2&3\\x+6&x+2&3\\x+6&2&x+3\end{vmatrix}=\left(x+6\right)\begin{vmatrix}1&2&3\\1&x+2&3\\1&2&x+3\end{vmatrix}=\left(x+6\right)\begin{vmatrix}1&2&3\\0&x&0\\0&0&x\end{vmatrix}=\left(x+6\right)x^2.
$$



$$
\mathrm{四阶行列式}\begin{vmatrix}1&1&1&1\\-1&2&1&1\\-1&-1&3&1\\-1&-1&-1&4\end{vmatrix}=(\;\;\;\;).\;
$$
$$
A.
40 \quad B.50 \quad C.60 \quad D.30 \quad E. \quad F. \quad G. \quad H.
$$
$$
\begin{vmatrix}1&1&1&1\\-1&2&1&1\\-1&-1&3&1\\-1&-1&-1&4\end{vmatrix}=\begin{vmatrix}1&1&1&1\\0&3&2&2\\0&0&4&2\\0&0&0&5\end{vmatrix}=1×3×4×5=60.
$$



$$
\mathrm{下列哪个不是方程}D\left(x\right)=\begin{vmatrix}1&1&1&1\\1&x&2&2\\2&2&x&3\\3&3&3&x\end{vmatrix}=0\mathrm{的解}(\;\;\;).
$$
$$
A.
1 \quad B.2 \quad C.3 \quad D.0 \quad E. \quad F. \quad G. \quad H.
$$
$$
D\left(x\right)=\begin{vmatrix}1&1&1&1\\1&x&2&2\\2&2&x&3\\3&3&3&x\end{vmatrix}=\begin{vmatrix}1&1&1&1\\0&x-1&1&1\\0&0&x-2&1\\0&0&0&x-3\end{vmatrix}=\left(x-1\right)\left(x-2\right)\left(x-3\right)=0,即x_1=1,x_2=2,x_3=3.
$$



$$
\mathrm{四阶行列式}\begin{vmatrix}1&1&1&1\\-1&1&1&1\\-1&-1&1&1\\-1&-1&-1&1\end{vmatrix}=(\;\;\;\;).\;
$$
$$
A.
1 \quad B.4 \quad C.8 \quad D.0 \quad E. \quad F. \quad G. \quad H.
$$
$$
\begin{vmatrix}1&1&1&1\\-1&1&1&1\\-1&-1&1&1\\-1&-1&-1&1\end{vmatrix}=\begin{vmatrix}1&1&1&1\\0&2&2&2\\0&0&2&2\\0&0&0&2\end{vmatrix}=8.
$$



$$
\mathrm{四阶行列式}\begin{vmatrix}4&1&2&4\\1&2&0&2\\10&5&2&0\\0&1&1&7\end{vmatrix}=(\;\;\;).
$$
$$
A.
0 \quad B.112 \quad C.4 \quad D.1 \quad E. \quad F. \quad G. \quad H.
$$
$$
\begin{vmatrix}4&1&2&4\\1&2&0&2\\10&5&2&0\\0&1&1&7\end{vmatrix}=-\begin{vmatrix}1&2&0&2\\4&1&2&4\\10&5&2&0\\0&1&1&7\end{vmatrix}=-\begin{vmatrix}1&2&0&2\\0&-7&2&-4\\0&-15&2&-20\\0&1&1&7\end{vmatrix}=\begin{vmatrix}1&2&0&2\\0&1&1&7\\0&-15&2&-20\\0&-7&2&-4\end{vmatrix}=\begin{vmatrix}1&2&0&2\\0&1&1&7\\0&0&17&85\\0&0&9&45\end{vmatrix}=\begin{vmatrix}1&2&0&2\\0&1&1&7\\0&0&17&85\\0&0&0&0\end{vmatrix}=0.
$$



$$
\mathrm{四阶行列式}\begin{vmatrix}1&1&1&1\\-1&1&1&1\\0&-1&1&1\\0&0&-1&1\end{vmatrix}=().
$$
$$
A.
8 \quad B.-8 \quad C.4 \quad D.-4 \quad E. \quad F. \quad G. \quad H.
$$
$$
\begin{vmatrix}1&1&1&1\\-1&1&1&1\\0&-1&1&1\\0&0&-1&1\end{vmatrix}=\begin{vmatrix}1&1&1&1\\0&2&2&2\\0&-1&1&1\\0&0&-1&1\end{vmatrix}=2\begin{vmatrix}1&1&1&1\\0&1&1&1\\0&-1&1&1\\0&0&-1&1\end{vmatrix}=2\begin{vmatrix}1&1&1&1\\0&1&1&1\\0&0&2&2\\0&0&-1&1\end{vmatrix}=4\begin{vmatrix}1&1&1&1\\0&1&1&1\\0&0&1&1\\0&0&-1&1\end{vmatrix}=4\begin{vmatrix}1&1&1&1\\0&1&1&1\\0&0&1&1\\0&0&0&2\end{vmatrix}=8.
$$



$$
\mathrm{四阶行列式}\begin{vmatrix}1&1&1&0\\1&2&3&4\\1&3&6&10\\0&4&10&20\end{vmatrix}=().
$$
$$
A.
1 \quad B.0 \quad C.2 \quad D.3 \quad E. \quad F. \quad G. \quad H.
$$
$$
\mathrm{原式}=\begin{vmatrix}1&1&1&0\\0&1&2&4\\0&2&5&10\\0&4&10&20\end{vmatrix}=0.
$$



$$
\mathrm{三阶行列式}\begin{vmatrix}1&2&3\\2&3&1\\0&1&2018\end{vmatrix}=(\;\;\;).
$$
$$
A.
-2018 \quad B.2013 \quad C.2018 \quad D.-2013 \quad E. \quad F. \quad G. \quad H.
$$
$$
\begin{vmatrix}1&2&3\\2&3&1\\0&1&2018\end{vmatrix}=\begin{vmatrix}1&2&3\\0&-1&-5\\0&1&2018\end{vmatrix}=\begin{vmatrix}1&2&3\\0&-1&-5\\0&0&2013\end{vmatrix}=-2013.
$$



$$
\mathrm{四阶行列式}\begin{vmatrix}1&1&1&1\\1&-1&1&1\\1&1&-1&1\\1&1&1&-1\end{vmatrix}=(\;\;\;).\;
$$
$$
A.
-1 \quad B.-4 \quad C.-8 \quad D.1 \quad E. \quad F. \quad G. \quad H.
$$
$$
D\overset{}=\begin{vmatrix}1&1&1&1\\0&-2&0&0\\0&0&-2&0\\0&0&0&-2\end{vmatrix}=-8.
$$



$$
\mathrm{方程}\begin{vmatrix}1&1&1&1\\1&1+x&1&1\\1&1&2+x&1\\1&1&1&3+x\end{vmatrix}=0,\mathrm{下列哪一个不是方程的解}(\;\;\;).
$$
$$
A.
0 \quad B.-1 \quad C.-2 \quad D.-3 \quad E. \quad F. \quad G. \quad H.
$$
$$
\begin{vmatrix}1&1&1&1\\1&1+x&1&1\\1&1&2+x&1\\1&1&1&3+x\end{vmatrix}=\begin{vmatrix}1&1&1&1\\0&x&0&0\\0&0&1+x&0\\0&0&0&2+x\end{vmatrix}=x(1+x)(2+x)=0.
$$



$$
\mathrm{四阶行列式}\begin{vmatrix}0&0&0&1\\0&0&-2&1\\0&3&0&-1\\4&0&2&1\end{vmatrix}=(\;\;\;).
$$
$$
A.
-12 \quad B.-24 \quad C.-36 \quad D.-48 \quad E. \quad F. \quad G. \quad H.
$$
$$
\begin{vmatrix}0&0&0&1\\0&0&-2&1\\0&3&0&-1\\4&0&2&1\end{vmatrix}\overset{\overset{r_1↔ r4}{r_2↔ r_3}}=\begin{vmatrix}4&0&2&1\\0&3&0&-1\\0&0&-2&1\\0&0&0&1\end{vmatrix}=4×3×(-2)×1=-24.
$$



$$
\mathrm{三阶行列式}\begin{vmatrix}1&2&3\\0&2&1\\2&0&18\end{vmatrix}=(\;\;\;).\;
$$
$$
A.
26 \quad B.28 \quad C.27 \quad D.24 \quad E. \quad F. \quad G. \quad H.
$$
$$
\begin{vmatrix}1&2&3\\0&2&1\\2&0&18\end{vmatrix}=\begin{vmatrix}1&2&3\\0&2&1\\0&-4&12\end{vmatrix}=\begin{vmatrix}1&2&3\\0&2&1\\0&0&14\end{vmatrix}=28.
$$



$$
\mathrm{三阶行列式}\begin{vmatrix}1&0&1\\1&1&0\\1&2&2018\end{vmatrix}=(\;\;\;\;).
$$
$$
A.
2019 \quad B.2018 \quad C.2017 \quad D.2015 \quad E. \quad F. \quad G. \quad H.
$$
$$
\begin{vmatrix}1&0&1\\1&1&0\\1&2&2018\end{vmatrix}=\begin{vmatrix}1&0&1\\0&1&-1\\0&2&2017\end{vmatrix}=\begin{vmatrix}1&0&1\\0&1&-1\\0&0&2019\end{vmatrix}=2019.
$$



$$
\mathrm{三阶行列式}\begin{vmatrix}1&1&1\\2&2&4\\2&3&2018\end{vmatrix}=(\;\;\;\;).
$$
$$
A.
2018 \quad B.-1 \quad C.-2 \quad D.-2018 \quad E. \quad F. \quad G. \quad H.
$$
$$
\begin{vmatrix}1&1&1\\2&2&4\\2&3&2018\end{vmatrix}=\begin{vmatrix}1&1&1\\0&0&2\\0&1&2016\end{vmatrix}=-\begin{vmatrix}1&1&1\\0&1&2016\\0&0&2\end{vmatrix}=-2.
$$



$$
\mathrm{四阶行列式}\begin{vmatrix}1&1&1&1\\1&3&1&1\\1&1&3&1\\1&1&1&3\end{vmatrix}=(\;\;\;).
$$
$$
A.
5 \quad B.6 \quad C.7 \quad D.8 \quad E. \quad F. \quad G. \quad H.
$$
$$
\begin{vmatrix}1&1&1&1\\1&3&1&1\\1&1&3&1\\1&1&1&3\end{vmatrix}=\begin{vmatrix}1&1&1&1\\0&2&0&0\\0&0&2&0\\0&0&0&2\end{vmatrix}=8.
$$



$$
\mathrm{四阶行列式}\begin{vmatrix}1&1&1&1\\-2&1&1&1\\-2&-2&1&1\\-2&-2&-2&1\end{vmatrix}=(\;\;\;).
$$
$$
A.
9 \quad B.27 \quad C.18 \quad D.36 \quad E. \quad F. \quad G. \quad H.
$$
$$
\begin{vmatrix}1&1&1&1\\-2&1&1&1\\-2&-2&1&1\\-2&-2&-2&1\end{vmatrix}=\begin{vmatrix}1&1&1&1\\0&3&3&3\\0&0&3&3\\0&0&0&3\end{vmatrix}=27.
$$



$$
\mathrm{三阶行列式}\begin{vmatrix}1&2&3\\2&3&1\\3&1&2\end{vmatrix}=(\;\;\;).\;
$$
$$
A.
-18 \quad B.-19\;\; \quad C.-20 \quad D.-21 \quad E. \quad F. \quad G. \quad H.
$$
$$
\begin{vmatrix}1&2&3\\2&3&1\\3&1&2\end{vmatrix}=\begin{vmatrix}6&2&3\\6&3&1\\6&1&2\end{vmatrix}=6\begin{vmatrix}1&2&3\\1&3&1\\1&1&2\end{vmatrix}=6\begin{vmatrix}1&2&3\\0&1&-2\\0&-1&-1\end{vmatrix}=6\begin{vmatrix}1&2&3\\0&1&-2\\0&0&-3\end{vmatrix}=-18.
$$



$$
\mathrm{四阶行列式}\begin{vmatrix}1&2&3&4\\0&1&2&3\\0&0&1&2\\1&1&1&1\end{vmatrix}=(\;\;\;\;).\;
$$
$$
A.
0 \quad B.1 \quad C.2 \quad D.3 \quad E. \quad F. \quad G. \quad H.
$$
$$
\begin{vmatrix}1&2&3&4\\0&1&2&3\\0&0&1&2\\1&1&1&1\end{vmatrix}=\begin{vmatrix}1&2&3&4\\0&1&2&3\\0&0&1&2\\0&-1&-2&-3\end{vmatrix}=\begin{vmatrix}1&2&3&4\\0&1&2&3\\0&0&1&2\\0&0&0&0\end{vmatrix}=0.
$$



$$
\mathrm{三阶行列式}\begin{vmatrix}1&1&3\\0&2&3\\2&2&18\end{vmatrix}=(\;\;\;\;).\;
$$
$$
A.
20 \quad B.-20 \quad C.24 \quad D.0 \quad E. \quad F. \quad G. \quad H.
$$
$$
\begin{vmatrix}1&1&3\\0&2&3\\2&2&18\end{vmatrix}=\begin{vmatrix}1&1&3\\0&2&3\\0&0&12\end{vmatrix}=24.
$$



$$
\mathrm{四阶行列式}\begin{vmatrix}4&3&2&1\\0&1&2&3\\4&3&3&3\\8\;&6&6&3\end{vmatrix}=(\;\;\;\;).\;
$$
$$
A.
-6 \quad B.7 \quad C.-12 \quad D.9 \quad E. \quad F. \quad G. \quad H.
$$
$$
\begin{vmatrix}4&3&2&1\\0&1&2&3\\4&3&3&3\\8\;&6&6&3\end{vmatrix}=\begin{vmatrix}4&3&2&1\\0&1&2&3\\0&0&1&2\\0&0&2&1\end{vmatrix}=\begin{vmatrix}4&3&2&1\\0&1&2&3\\0&0&1&2\\0&0&0&-3\end{vmatrix}=-12.
$$



$$
\mathrm{四阶行列式}\begin{vmatrix}2&3&2&2\\0&3&2&2\\2&3&4&4\\0&0&4&3\end{vmatrix}=(\;\;\;\;).\;
$$
$$
A.
-1 \quad B.-11 \quad C.-2 \quad D.-12 \quad E. \quad F. \quad G. \quad H.
$$
$$
\begin{vmatrix}2&3&2&2\\0&3&2&2\\2&3&4&4\\0&0&4&3\end{vmatrix}=\begin{vmatrix}2&3&2&2\\0&3&2&2\\0&0&2&2\\0&0&4&3\end{vmatrix}=\begin{vmatrix}2&3&2&2\\0&3&2&2\\0&0&2&2\\0&0&0&-1\end{vmatrix}=-12.
$$



$$
\mathrm{四阶行列式}\begin{vmatrix}1&1&2&2\\2&3&3&3\\3&3&4&4\\4&4&4&5\end{vmatrix}=(\;\;\;\;).\;
$$
$$
A.
0 \quad B.-1 \quad C.-2 \quad D.-3 \quad E. \quad F. \quad G. \quad H.
$$
$$
\begin{vmatrix}1&1&2&2\\2&3&3&3\\3&3&4&4\\4&4&4&5\end{vmatrix}=\;\begin{vmatrix}1&1&2&2\\0&1&-1&-1\\0&0&-2&-2\\0&0&-4&-3\end{vmatrix}=\begin{vmatrix}1&1&2&2\\0&1&-1&-1\\0&0&-2&-2\\0&0&0&1\end{vmatrix}=-2.
$$



$$
\mathrm{四阶行列式}\begin{vmatrix}1&1&1&0\\-1&0&1&1\\-1&-1&1&1\\-1&-1&-1&-1\end{vmatrix}=(\;\;\;).
$$
$$
A.
-2 \quad B.-8 \quad C.4 \quad D.-4 \quad E. \quad F. \quad G. \quad H.
$$
$$
\begin{vmatrix}1&1&1&0\\-1&0&1&1\\-1&-1&1&1\\-1&-1&-1&-1\end{vmatrix}=\begin{vmatrix}1&1&1&0\\0&1&2&1\\0&0&2&1\\0&0&0&-1\end{vmatrix}=-2.
$$



$$
\mathrm{四阶行列式}\begin{vmatrix}1&1&1&1\\1&2&3&4\\1&3&6&10\\1&4&10&20\end{vmatrix}=(\;\;\;).\;
$$
$$
A.
0 \quad B.1 \quad C.2 \quad D.3 \quad E. \quad F. \quad G. \quad H.
$$
$$
\mathrm{原式}=\begin{vmatrix}1&1&1&1\\0&1&2&3\\0&2&5&9\\0&3&9&19\end{vmatrix}=\begin{vmatrix}1&1&1&1\\0&1&2&3\\0&0&1&3\\0&0&0&1\end{vmatrix}=1.
$$



$$
\mathrm{四阶行列式}\begin{vmatrix}1&1&1&0\\0&1&1&1\\0&0&1&1\\1&0&0&1\end{vmatrix}=(\;\;\;\;).\;
$$
$$
A.
1 \quad B.2 \quad C.3 \quad D.4 \quad E. \quad F. \quad G. \quad H.
$$
$$
\begin{vmatrix}1&1&1&0\\0&1&1&1\\0&0&1&1\\1&0&0&1\end{vmatrix}=\begin{vmatrix}1&1&1&0\\0&1&1&1\\0&0&1&1\\0&-1&-1&1\end{vmatrix}=\begin{vmatrix}1&1&1&0\\0&1&1&1\\0&0&1&1\\0&0&0&2\end{vmatrix}=2.
$$



$$
\mathrm{四阶行列式}\begin{vmatrix}-1&2&-4&0\\0&-1&3&0\\3&1&-2&-3\\1&0&5&1\end{vmatrix}=(\;\;\;).
$$
$$
A.
27 \quad B.28 \quad C.29 \quad D.30 \quad E. \quad F. \quad G. \quad H.
$$
$$
\begin{vmatrix}-1&2&-4&0\\0&-1&3&0\\3&1&-2&-3\\1&0&5&1\end{vmatrix}=\begin{vmatrix}-1&2&-4&0\\0&-1&3&0\\0&7&-14&-3\\0&2&1&1\end{vmatrix}=\begin{vmatrix}-1&2&-4&0\\0&-1&3&0\\0&0&7&-3\\0&0&7&1\end{vmatrix}=\begin{vmatrix}-1&2&-4&0\\0&-1&3&0\\0&0&7&-3\\0&0&0&4\end{vmatrix}=28.
$$



$$
\mathrm{四阶行列式}\begin{vmatrix}2&1&0&0\\1&2&1&0\\0&1&2&1\\0&0&1&2\end{vmatrix}=(\;\;\;\;\;).\;
$$
$$
A.
4 \quad B.5 \quad C.6 \quad D.7 \quad E. \quad F. \quad G. \quad H.
$$
$$
\begin{array}{l}\begin{vmatrix}2&1&0&0\\1&2&1&0\\0&1&2&1\\0&0&1&2\end{vmatrix}=-\begin{vmatrix}1&2&1&0\\2&1&0&0\\0&1&2&1\\0&0&1&2\end{vmatrix}=-\begin{vmatrix}1&2&1&0\\0&-3&-2&0\\0&1&2&1\\0&0&1&2\end{vmatrix}\\=\begin{vmatrix}1&2&1&0\\0&1&2&1\\0&-3&-2&0\\0&0&1&2\end{vmatrix}=\begin{vmatrix}1&2&1&0\\0&1&2&1\\0&0&4&3\\0&0&1&2\end{vmatrix}=\begin{vmatrix}1&2&1&0\\0&1&2&1\\0&0&4&3\\0&0&0&\frac54\end{vmatrix}=5.\end{array}
$$



$$
\mathrm{四阶行列式}\begin{vmatrix}1&19&23&1\\0&1&33&1\\-1&-19&-21&1\\2&38&44&1\end{vmatrix}=(\;\;).\;\;\;\;
$$
$$
A.
1 \quad B.2 \quad C.3 \quad D.4 \quad E. \quad F. \quad G. \quad H.
$$
$$
\begin{vmatrix}1&19&23&1\\0&1&33&1\\-1&-19&-21&1\\2&38&44&1\end{vmatrix}=\begin{vmatrix}1&19&23&1\\0&1&33&1\\0&0&2&2\\0&0&-2&-1\end{vmatrix}=\begin{vmatrix}1&19&23&1\\0&1&33&1\\0&0&2&2\\0&0&0&1\end{vmatrix}=2.
$$



$$
\mathrm{四阶行列式}\begin{vmatrix}6&2&2&2\\1&7&2&2\\1&1&8&2\\1&1&1&2\end{vmatrix}=(\;\;\;\;).\;
$$
$$
A.
410 \quad B.420 \quad C.430 \quad D.440 \quad E. \quad F. \quad G. \quad H.
$$
$$
\begin{array}{l}\begin{vmatrix}6&2&2&2\\1&7&2&2\\1&1&8&2\\1&1&1&2\end{vmatrix}=\begin{vmatrix}1&1&1&2\\1&1&8&2\\1&7&2&2\\6&2&2&2\end{vmatrix}=\begin{vmatrix}1&1&1&2\\0&0&7&0\;\\0&6&1&0\\0&-4&-4&-10\end{vmatrix}\\=\begin{vmatrix}1&1&1&2\\0&4&4&10\;\\0&6&1&0\\0&0&7&0\end{vmatrix}=\begin{vmatrix}1&1&1&2\\0&4&4&10\;\\0&0&-5&-15\\0&0&7&0\end{vmatrix}=\begin{vmatrix}1&1&1&2\\0&4&4&10\\0&0&-5&-15\\0&0&0&-21\end{vmatrix}=420.\end{array}
$$



$$
\mathrm{四阶行列式}\begin{vmatrix}6&2&3&4\\1&7&3&4\\1&2&8&4\\1&2&3&9\end{vmatrix}=(\;\;\;\;\;\;).\;
$$
$$
A.
15×5^2 \quad B.15×5^3 \quad C.15×5 \quad D.5^3 \quad E. \quad F. \quad G. \quad H.
$$
$$
\begin{array}{l}\begin{vmatrix}6&2&3&4\\1&7&3&4\\1&2&8&4\\1&2&3&9\end{vmatrix}=\begin{vmatrix}15&2&3&4\\15&7&3&4\\15&2&8&4\\15&2&3&9\end{vmatrix}\\=15\begin{vmatrix}1&2&3&4\\1&7&3&4\\1&2&8&4\\1&2&3&9\end{vmatrix}=15\begin{vmatrix}1&2&3&4\\0&5&0&0\\0&0&5&0\\0&0&0&5\end{vmatrix}=15×5^3.\end{array}
$$



$$
\mathrm{四阶行列式}\begin{vmatrix}0&1&1&1\\-1&0&1&1\\-1&-1&0&1\\-1&-1&-1&0\end{vmatrix}=(\;\;).\;
$$
$$
A.
1 \quad B.2 \quad C.3 \quad D.4 \quad E. \quad F. \quad G. \quad H.
$$
$$
\begin{array}{l}\begin{vmatrix}0&1&1&1\\-1&0&1&1\\-1&-1&0&1\\-1&-1&-1&0\end{vmatrix}=\begin{vmatrix}0&1&1&1\\-1&0&1&1\\0&-1&-1&0\\0&-1&-2&-1\end{vmatrix}=-\begin{vmatrix}-1&0&1&1\\0&1&1&1\\0&-1&-1&0\\0&-1&-2&-1\end{vmatrix}\\=-\begin{vmatrix}-1&0&1&1\\0&1&1&1\\0&0&0&1\\0&0&-1&0\end{vmatrix}=\begin{vmatrix}-1&0&1&1\\0&1&1&1\\0&0&-1&0\\0&0&0&1\end{vmatrix}=1.\end{array}
$$



$$
\mathrm{四阶行列式}\begin{vmatrix}0&1&-1&2\\1&0&0&2\\1&2&-1&0\\2&1&1&0\end{vmatrix}=(\;\;).
$$
$$
A.
-4 \quad B.-5 \quad C.-6 \quad D.-7 \quad E. \quad F. \quad G. \quad H.
$$
$$
\begin{array}{l}\begin{vmatrix}0&1&-1&2\\1&0&0&2\\1&2&-1&0\\2&1&1&0\end{vmatrix}=-\begin{vmatrix}1&0&0&2\\0&1&-1&2\\1&2&-1&0\\2&1&1&0\end{vmatrix}=-\begin{vmatrix}1&0&0&2\\0&1&-1&2\\0&2&-1&-2\\0&1&1&-4\end{vmatrix}\\=-\begin{vmatrix}1&0&0&2\\0&1&-1&2\\0&0&1&-6\\0&0&2&-6\end{vmatrix}=-\begin{vmatrix}1&0&0&2\\0&1&-1&2\\0&0&1&-6\\0&0&0&6\end{vmatrix}=-6.\end{array}
$$



$$
\mathrm{四阶行列式}\begin{vmatrix}-1&0&1&1\\0&1&1&1\\1&1&-1&-1\\-1&-1&1&0\end{vmatrix}=(\;\;\;).
$$
$$
A.
-1 \quad B.-2 \quad C.-3 \quad D.-4 \quad E. \quad F. \quad G. \quad H.
$$
$$
\begin{vmatrix}-1&0&1&1\\0&1&1&1\\1&1&-1&-1\\-1&-1&1&0\end{vmatrix}=\begin{vmatrix}-1&0&1&1\\0&1&1&1\\0&1&0&0\\0&-1&0&-1\end{vmatrix}=\begin{vmatrix}-1&0&1&1\\0&1&1&1\\0&0&-1&-1\\0&0&1&0\end{vmatrix}=\begin{vmatrix}-1&0&1&1\\0&1&1&1\\0&0&-1&-1\\0&0&0&-1\end{vmatrix}=-1.
$$



$$
\mathrm{四阶行列式}\begin{vmatrix}1&0&1&0\\0&-1&1&1\\1&1&-1&1\\0&1&1&-1\end{vmatrix}=(\;\;\;).
$$
$$
A.
1 \quad B.2 \quad C.3 \quad D.4 \quad E. \quad F. \quad G. \quad H.
$$
$$
\begin{vmatrix}1&0&1&0\\0&-1&1&1\\1&1&-1&1\\0&1&1&-1\end{vmatrix}=\begin{vmatrix}1&0&1&0\\0&-1&1&1\\0&1&-2&1\\0&1&1&-1\end{vmatrix}=\begin{vmatrix}1&0&1&0\\0&-1&1&1\\0&0&-1&2\\0&0&2&0\end{vmatrix}=\begin{vmatrix}1&0&1&0\\0&-1&1&1\\0&0&-1&2\\0&0&0&4\end{vmatrix}=4.
$$



$$
\mathrm{四阶行列式}\begin{vmatrix}1&3&0&0\\3&1&3&0\\0&3&1&3\\0&0&3&1\end{vmatrix}=(\;\;\;).\;
$$
$$
A.
54 \quad B.55 \quad C.56 \quad D.57 \quad E. \quad F. \quad G. \quad H.
$$
$$
\begin{vmatrix}1&3&0&0\\3&1&3&0\\0&3&1&3\\0&0&3&1\end{vmatrix}=\begin{vmatrix}1&3&0&0\\0&-8&3&0\\0&3&1&3\\0&0&3&1\end{vmatrix}=\begin{vmatrix}1&3&0&0\\0&-8&3&0\\0&0&\frac{17}8&3\\0&0&3&1\end{vmatrix}=\begin{vmatrix}1&3&0&0\\0&-8&3&0\\0&0&\frac{17}8&3\\0&0&0&-\frac{55}{17}\end{vmatrix}=55.
$$



$$
\mathrm{四阶行列式}\begin{vmatrix}1&1&1&1\\1&0&2&3\\4&1&9&16\\8&1&27&64\end{vmatrix}=(\;\;).\;\;\;
$$
$$
A.
-11 \quad B.-12 \quad C.-13 \quad D.-14 \quad E. \quad F. \quad G. \quad H.
$$
$$
\begin{vmatrix}1&1&1&1\\1&0&2&3\\4&1&9&16\\8&1&27&64\end{vmatrix}=\begin{vmatrix}1&1&1&1\\0&-1&1&2\\0&-3&5&12\\0&-7&19&56\end{vmatrix}=\begin{vmatrix}1&1&1&1\\0&-1&1&2\\0&0&2&6\\0&0&12&42\end{vmatrix}=\begin{vmatrix}1&1&1&1\\0&-1&1&2\\0&0&2&-4\\0&0&0&6\end{vmatrix}=-12.
$$



$$
\mathrm{四阶行列式}\begin{vmatrix}1&0&0&1\\0&2&3&4\\0&4&6&10\\1&4&10&20\end{vmatrix}=(\;\;\;).
$$
$$
A.
-12 \quad B.-16 \quad C.-36 \quad D.-48 \quad E. \quad F. \quad G. \quad H.
$$
$$
\begin{vmatrix}1&0&0&1\\0&2&3&4\\0&4&6&10\\1&4&10&20\end{vmatrix}=\begin{vmatrix}1&0&0&1\\0&2&3&4\\0&4&6&10\\0&4&10&19\end{vmatrix}=\begin{vmatrix}1&0&0&1\\0&2&3&4\\0&0&0&2\\0&0&4&11\end{vmatrix}=-\begin{vmatrix}1&0&0&1\\0&2&3&4\\0&0&4&11\\0&0&0&2\end{vmatrix}=\;-16.
$$



$$
\mathrm{四阶行列式}\begin{vmatrix}2&2&2&2\\2&2&3&4\\2&3&6&10\\2&4&10&20\end{vmatrix}=(\;\;\;\;).
$$
$$
A.
1 \quad B.-2 \quad C.3 \quad D.-4 \quad E. \quad F. \quad G. \quad H.
$$
$$
\begin{vmatrix}2&2&2&2\\2&2&3&4\\2&3&6&10\\2&4&10&20\end{vmatrix}=\begin{vmatrix}2&2&2&2\\0&0&1&2\\0&1&4&8\\0&2&8&18\end{vmatrix}=\begin{vmatrix}2&2&2&2\\0&0&1&2\\0&1&4&8\\0&0&0&2\end{vmatrix}=-\begin{vmatrix}2&2&2&2\\0&1&4&8\\0&0&1&2\\0&0&0&2\end{vmatrix}=-4.
$$



$$
\mathrm{四阶行列式}\begin{vmatrix}1&2&3&4\\2&3&4&4\\3&4&4&4\\4&4&4&4\end{vmatrix}=(\;\;\;).\;
$$
$$
A.
1 \quad B.-1 \quad C.4 \quad D.-4 \quad E. \quad F. \quad G. \quad H.
$$
$$
\begin{vmatrix}1&2&3&4\\2&3&4&4\\3&4&4&4\\4&4&4&4\end{vmatrix}=\begin{vmatrix}1&2&3&4\\0&-1&-2&-4\\0&-2&-5&-8\\0&-4&-8&-12\end{vmatrix}=\begin{vmatrix}1&2&3&4\\0&-1&-2&-4\\0&0&-1&0\\0&0&0&4\end{vmatrix}=4.
$$



$$
\mathrm{四阶行列式}\begin{vmatrix}1&19&23&27\\0&1&33&37\\-1&-19&-21&14\\2&38&44&18\end{vmatrix}=(\;\;\;\;\;).\;
$$
$$
A.
1 \quad B.5 \quad C.10 \quad D.20 \quad E. \quad F. \quad G. \quad H.
$$
$$
\begin{vmatrix}1&19&23&27\\0&1&33&37\\-1&-19&-21&14\\2&38&44&18\end{vmatrix}=\begin{vmatrix}1&19&23&27\\0&1&33&37\\0&0&2&41\\0&0&\;-2&-36\end{vmatrix}=\begin{vmatrix}1&19&23&27\\0&1&33&37\\0&0&2&41\\0&0&\;0&5\end{vmatrix}=\;10.
$$



$$
\mathrm{四阶行列式}\begin{vmatrix}1&3&3&3\\3&2&3&3\\3&3&3&3\\3&3&3&4\end{vmatrix}=(\;\;\;\;\;).\;
$$
$$
A.
3 \quad B.-3 \quad C.6 \quad D.-6 \quad E. \quad F. \quad G. \quad H.
$$
$$
\begin{vmatrix}1&3&3&3\\3&2&3&3\\3&3&3&3\\3&3&3&4\end{vmatrix}=\begin{vmatrix}-2&0&0&0\\0&-1&0&0\\3&3&3&3\\0&0&0&1\end{vmatrix}=\begin{vmatrix}-2&0&0&0\\0&-1&0&0\\0&3&3&3\\0&0&0&1\end{vmatrix}=\begin{vmatrix}-2&0&0&0\\0&-1&0&0\\0&0&3&3\\0&0&0&1\end{vmatrix}=6.
$$



$$
\mathrm{四阶行列式}\begin{vmatrix}3&2&2&2\\2&3&2&2\\2&2&3&2\\2&2&2&3\end{vmatrix}=(\;\;\;\;).
$$
$$
A.
9 \quad B.3 \quad C.6 \quad D.12 \quad E. \quad F. \quad G. \quad H.
$$
$$
\begin{vmatrix}3&2&2&2\\2&3&2&2\\2&2&3&2\\2&2&2&3\end{vmatrix}=\begin{vmatrix}9&9&9&9\\2&3&2&2\\2&2&3&2\\2&2&2&3\end{vmatrix}=9\begin{vmatrix}1&1&1&1\\2&3&2&2\\2&2&3&2\\2&2&2&3\end{vmatrix}=9\begin{vmatrix}1&1&1&1\\0&1&0&0\\0&0&1&0\\0&0&0&1\end{vmatrix}=9.
$$



$$
\mathrm{四阶行列式}\begin{vmatrix}2&3&0&0\\4&2&3&0\\0&4&2&3\\0&0&4&2\end{vmatrix}=(\;\;\;).
$$
$$
A.
16 \quad B.-16 \quad C.8 \quad D.-8 \quad E. \quad F. \quad G. \quad H.
$$
$$
\begin{vmatrix}2&3&0&0\\4&2&3&0\\0&4&2&3\\0&0&4&2\end{vmatrix}=\begin{vmatrix}2&3&0&0\\0&-4&3&0\\0&4&2&3\\0&0&4&2\end{vmatrix}=\begin{vmatrix}2&3&0&0\\0&-4&3&0\\0&0&5&3\\0&0&4&2\end{vmatrix}=\begin{vmatrix}2&3&0&0\\0&-4&3&0\\0&0&5&3\\0&0&0&-\frac25\end{vmatrix}=\;2×\left(-4\right)×5×\left(-\frac25\right)=16.
$$



$$
\mathrm{四阶行列式}\begin{vmatrix}2&1&1&1\\1&2&1&1\\1&1&2&1\\1&1&1&2\end{vmatrix}=(\;\;\;).
$$
$$
A.
5 \quad B.8 \quad C.16 \quad D.1 \quad E. \quad F. \quad G. \quad H.
$$
$$
\mathrm{原式}\overset{r_1+r_2+r_3+r_4}=\begin{vmatrix}5&5&5&5\\1&2&1&1\\1&1&2&1\\1&1&1&2\end{vmatrix}=5\begin{vmatrix}1&1&1&1\\1&2&1&1\\1&1&2&1\\1&1&1&2\end{vmatrix}=5\begin{vmatrix}1&1&1&1\\0&1&0&0\\0&0&1&0\\0&0&0&1\end{vmatrix}=5.
$$



$$
\mathrm{三阶行列式}\begin{vmatrix}x_1-m&x_2&x_3\\x_1&x_2-m&x_3\\x_1&x_2&x_3-m\end{vmatrix}=(\;\;\;).
$$
$$
A.
\left(x_1+x_2+x_3-m\right)m^2 \quad B.\left(x_1+x_2+x_3-m\right)m^3 \quad C.\left(x_1+x_2+x_3-m\right)m \quad D.\left(x_1+x_2+x_3-m\right) \quad E. \quad F. \quad G. \quad H.
$$
$$
\begin{array}{l}\begin{vmatrix}x_1-m&x_2&x_3\\x_1&x_2-m&x_3\\x_1&x_2&x_3-m\end{vmatrix}=\begin{vmatrix}x_1+x_2+x_3-m&x_2&x_3\\x_1+x_2+x_3-m&\;\;x_2-m&x_3\\x_1+x_2+x_3-m&x_2&\;\;x_3-m\end{vmatrix}=(x_1+x_2+x_3-m)\begin{vmatrix}1&x_2&x_3\\1&x_2-m&x_3\\1&x_2&x_3-m\end{vmatrix}\\=(x_1+x_2+x_3-m)\begin{vmatrix}1&x_2&x_3\\0&-m&0\\0&0&-m\end{vmatrix}=(x_1+x_2+x_3-m)m^2.\end{array}
$$



$$
\mathrm{四阶行列式}\begin{vmatrix}2&-1&1&0\\2&0&0&3\\1&0&0&-1\\0&2&-1&1\end{vmatrix}=(\;\;\;).
$$
$$
A.
-5 \quad B.5 \quad C.-6 \quad D.6 \quad E. \quad F. \quad G. \quad H.
$$
$$
\begin{vmatrix}2&-1&1&0\\2&0&0&3\\1&0&0&-1\\0&2&-1&1\end{vmatrix}=\begin{vmatrix}1&0&0&-1\\0&2&-1&1\\2&-1&1&0\\2&0&0&3\end{vmatrix}=\begin{vmatrix}1&0&0&-1\\0&2&-1&1\\0&-1&1&2\\0&0&0&5\end{vmatrix}=-\begin{vmatrix}1&0&0&-1\\0&-1&1&2\\0&2&-1&1\\0&0&0&5\end{vmatrix}=-\begin{vmatrix}1&0&0&-1\\0&-1&1&2\\0&0&1&5\\0&0&0&5\end{vmatrix}=5.
$$



$$
\mathrm{四阶行列式}\begin{vmatrix}1&1&2&2\\0&4&0&2\\-1&2&-3&5\\-2&0&1&2\end{vmatrix}=(\;\;\;).\;
$$
$$
A.
-45 \quad B.-130 \quad C.-65 \quad D.65 \quad E. \quad F. \quad G. \quad H.
$$
$$
\begin{vmatrix}1&1&2&2\\0&4&0&2\\-1&2&-3&5\\-2&0&1&2\end{vmatrix}=\begin{vmatrix}1&1&2&2\\0&4&0&2\\0&3&-1&7\\0&2&5&6\end{vmatrix}=\begin{vmatrix}1&1&2&2\\0&4&0&2\\0&0&-1&\frac{11}2\\0&0&5&5\end{vmatrix}=\begin{vmatrix}1&1&2&2\\0&4&0&2\\0&0&-1&\frac{11}2\\0&0&0&\frac{65}2\end{vmatrix}=1×4×\left(-1\right)×\frac{65}2=-130.
$$



$$
\mathrm{四阶行列式}\begin{vmatrix}1&0&-1&-1\\0&1&1&1\\-1&-1&1&1\\-1&-1&1&0\end{vmatrix}=(\;\;\;\;).\;
$$
$$
A.
-1 \quad B.0 \quad C.1 \quad D.2 \quad E. \quad F. \quad G. \quad H.
$$
$$
\begin{array}{l}\begin{vmatrix}1&0&-1&-1\\0&1&1&1\\-1&-1&1&1\\-1&-1&1&0\end{vmatrix}=\begin{vmatrix}1&0&-1&-1\\0&1&1&1\\0&-1&0&0\\0&-1&0&-1\end{vmatrix}=\begin{vmatrix}1&0&-1&-1\\0&1&1&1\\0&0&1&1\\0&0&1&0\end{vmatrix}\\=\begin{vmatrix}1&0&-1&-1\\0&1&1&1\\0&0&1&1\\0&0&0&-1\end{vmatrix}=-1.\end{array}
$$



$$
\mathrm{四阶行列式}\begin{vmatrix}a&b&b&b\\b&a&b&b\\b&b&a&b\\b&b&b&a\end{vmatrix}=(\;\;\;).
$$
$$
A.
\left(a+3b\right)\left(a-b\right)^3 \quad B.a^4 \quad C.b^4 \quad D.\left(a+3b\right)\left(a+b\right)^3 \quad E. \quad F. \quad G. \quad H.
$$
$$
\mathrm{原式}=(a+3b)\begin{vmatrix}1&b&b&b\\1&a&b&b\\1&b&a&b\\1&b&b&a\end{vmatrix}=(a+3b)\begin{vmatrix}1&b&b&b\\0&a-b&0&0\\0&0&a-b&0\\0&0&0&a-b\end{vmatrix}=\left(a+3b\right)\left(a-b\right)^3.
$$



$$
\mathrm{四阶行列式}\begin{vmatrix}-2&2&-4&0\\4&-1&3&5\\3&1&-2&-3\\2&0&5&1\end{vmatrix}=(\;\;\;).\;
$$
$$
A.
270 \quad B.-270 \quad C.27 \quad D.-27 \quad E. \quad F. \quad G. \quad H.
$$
$$
\begin{vmatrix}-2&2&-4&0\\4&-1&3&5\\3&1&-2&-3\\2&0&5&1\end{vmatrix}=-2\begin{vmatrix}1&-1&2&0\\0&3&-5&5\\0&4&-8&-3\\0&2&1&1\end{vmatrix}=2\begin{vmatrix}1&-1&2&0\\0&2&1&1\\0&4&-8&-3\\0&3&-5&5\end{vmatrix}=2\begin{vmatrix}1&-1&2&0\\0&2&1&1\\0&0&-10&-5\\0&0&-\frac{13}2&\frac72\end{vmatrix}=-270.
$$



$$
\mathrm{四阶行列式}\begin{vmatrix}0&1&1&1\\1&0&1&1\\1&1&0&1\\1&1&1&0\end{vmatrix}=(\;\;\;\;).\;
$$
$$
A.
3 \quad B.-3 \quad C.1 \quad D.-1 \quad E. \quad F. \quad G. \quad H.
$$
$$
\mathrm{原式}\overset{c_1+c_2+c_3+c_4}=3\begin{vmatrix}1&1&1&1\\1&0&1&1\\1&1&0&1\\1&1&1&0\end{vmatrix}=3\begin{vmatrix}1&1&1&1\\0&-1&0&0\\0&0&-1&0\\0&0&0&-1\end{vmatrix}=-3.
$$



$$
\mathrm{四阶行列式}\begin{vmatrix}0&-1&-1&2\\1&-1&0&2\\-1&2&-1&0\\2&1&1&0\end{vmatrix}=(\;\;\;\;).\;
$$
$$
A.
4 \quad B.-4 \quad C.2 \quad D.-2 \quad E. \quad F. \quad G. \quad H.
$$
$$
\begin{vmatrix}0&-1&-1&2\\1&-1&0&2\\-1&2&-1&0\\2&1&1&0\end{vmatrix}=-\begin{vmatrix}1&-1&0&2\\0&-1&-1&2\\-1&2&-1&0\\2&1&1&0\end{vmatrix}=-\begin{vmatrix}1&-1&0&2\\0&-1&-1&2\\0&1&-1&2\\0&3&1&-4\end{vmatrix}=-\begin{vmatrix}1&-1&0&2\\0&-1&-1&2\\0&0&-2&4\\0&0&-2&2\end{vmatrix}=-\begin{vmatrix}1&-1&0&2\\0&-1&-1&2\\0&0&-2&4\\0&0&0&-2\end{vmatrix}=4.
$$



$$
\mathrm{四阶行列式}\begin{vmatrix}1&0&-1&-1\\0&1&1&1\\-1&-1&1&1\\-1&1&1&0\end{vmatrix}=(\;\;\;).\;
$$
$$
A.
0 \quad B.-1 \quad C.1 \quad D.2 \quad E. \quad F. \quad G. \quad H.
$$
$$
\begin{vmatrix}1&0&-1&-1\\0&1&1&1\\-1&-1&1&1\\-1&1&1&0\end{vmatrix}=\begin{vmatrix}1&0&-1&-1\\0&1&1&1\\0&-1&0&0\\0&1&0&-1\end{vmatrix}=\begin{vmatrix}1&0&-1&-1\\0&1&1&1\\0&0&1&1\\0&0&-1&-2\end{vmatrix}=\begin{vmatrix}1&0&-1&-1\\0&1&1&1\\0&0&1&1\\0&0&0&-1\end{vmatrix}=-1.
$$



$$
\mathrm{四阶行列式}\begin{vmatrix}3&1&-1&2\\-5&1&3&-4\\2&0&1&-1\\1&-5&3&-3\end{vmatrix}=(\;\;\;).\;
$$
$$
A.
40 \quad B.-40 \quad C.20 \quad D.-20 \quad E. \quad F. \quad G. \quad H.
$$
$$
\mathrm{原式}\overset{c_1↔ c_2}=-\begin{vmatrix}1&3&-1&2\\1&-5&3&-4\\0&2&1&-1\\-5&1&3&-3\end{vmatrix}=-\begin{vmatrix}1&3&-1&2\\0&-8&4&-6\\0&2&1&-1\\0&16&-2&7\end{vmatrix}=\begin{vmatrix}1&3&-1&2\\0&2&1&-1\\0&-8&4&-6\\0&16&-2&7\end{vmatrix}=\begin{vmatrix}1&3&-1&2\\0&2&1&-1\\0&0&8&-10\\0&0&-10&15\end{vmatrix}=\begin{vmatrix}1&3&-1&2\\0&2&1&-1\\0&0&8&-10\\0&0&0&5/2\end{vmatrix}=40.
$$



$$
\mathrm{四阶行列式}\begin{vmatrix}3&1&1&1\\1&3&1&1\\1&1&3&1\\1&1&1&3\end{vmatrix}=(\;\;\;).\;
$$
$$
A.
48 \quad B.-48 \quad C.81 \quad D.-81 \quad E. \quad F. \quad G. \quad H.
$$
$$
\begin{array}{l}\mathrm{原式}\overset{r_1+r_2+r_3+r_4}=\begin{vmatrix}6&6&6&6\\1&3&1&1\\1&1&3&1\\1&1&1&3\end{vmatrix}=6\begin{vmatrix}1&1&1&1\\1&3&1&1\\1&1&3&1\\1&1&1&3\end{vmatrix}=6\begin{vmatrix}1&1&1&1\\0&2&0&0\\0&0&2&0\\0&0&0&2\end{vmatrix}=48.\end{array}
$$



$$
\mathrm{四阶行列式}\begin{vmatrix}a_1&-a_1&0&0\\0&a_2&-a_2&0\\0&0&a_3&-a_3\\1&1&1&1\end{vmatrix}=(\;\;\;).\;
$$
$$
A.
4a_1a_2a_3 \quad B.a_1a_2a_3 \quad C.-a_1a_2a_3 \quad D.-4a_1a_2a_3 \quad E. \quad F. \quad G. \quad H.
$$
$$
\begin{array}{l}\\\mathrm{原式}\;\overset{c_2+c_1}=\;\begin{vmatrix}a_1&0&0&0\\0&a_2&-a_2&0\\0&0&a_3&-a_3\\1&2&1&1\end{vmatrix}=\begin{vmatrix}a_1&0&0&0\\0&a_2&0&0\\0&0&a_3&-a_3\\1&2&3&1\end{vmatrix}=\begin{vmatrix}a_1&0&0&0\\0&a_2&0&0\\0&0&a_3&0\\1&2&3&4\end{vmatrix}=4a_1a_2a_3.\end{array}
$$



$$
\mathrm{三阶行列式}\begin{vmatrix}a&b&b\\b&a&b\\b&b&a\end{vmatrix}=(\;\;\;).\;
$$
$$
A.
\left(a-b\right)^3 \quad B.\left(a+2b\right)^2\left(a-b\right) \quad C.\left(a+2b\right)\left(a-b\right)^2 \quad D.\left(a-2b\right)\left(a+b\right)^2 \quad E. \quad F. \quad G. \quad H.
$$
$$
\begin{vmatrix}a&b&b\\b&a&b\\b&b&a\end{vmatrix}=\begin{vmatrix}a+2b&a+2b&a+2b\\b&a&b\\b&b&a\end{vmatrix}=\left(a+2b\right)\begin{vmatrix}1&1&1\\0&a-b&0\\0&0&a-b\end{vmatrix}=\left(a+2b\right)\left(a-b\right)^2.
$$



$$
\mathrm{四阶行列式}\begin{vmatrix}1&2&3&4\\2&3&4&1\\3&4&1&2\\4&1&2&3\end{vmatrix}=(\;\;\;).\;
$$
$$
A.
160 \quad B.-160\; \quad C.80 \quad D.-80 \quad E. \quad F. \quad G. \quad H.
$$
$$
\begin{vmatrix}1&2&3&4\\2&3&4&1\\3&4&1&2\\4&1&2&3\end{vmatrix}=\begin{vmatrix}10&2&3&4\\10&3&4&1\\10&4&1&2\\10&1&2&3\end{vmatrix}=10\begin{vmatrix}1&2&3&4\\1&3&4&1\\1&4&1&2\\1&1&2&3\end{vmatrix}=10\begin{vmatrix}1&2&3&4\\0&1&1&-3\\0&2&-2&-2\\0&-1&-1&-1\end{vmatrix}=10\begin{vmatrix}1&2&3&4\\0&1&1&-3\\0&0&-4&4\\0&0&0&-4\end{vmatrix}=160.
$$



$$
\mathrm{四阶行列式}\begin{vmatrix}a&b&c&d\\a&a+b&a+b+c&a+b+c+d\\a&2a+b&3a+2b+c&4a+3b+2c+d\\a&3a+b&6a+3b+c&10a+6b+3c+d\end{vmatrix}=(\;\;\;).\;
$$
$$
A.
a^4 \quad B.a^3 \quad C.abcd \quad D.abc \quad E. \quad F. \quad G. \quad H.
$$
$$
\begin{array}{l}\mathrm{从第}4\mathrm{行开始},\mathrm{后一行减前一行}:\\\mathrm{原式}\overset{\overset{r_4-r_3}{r_3-r_2}}{\underset{r_2-r_1}=}\begin{vmatrix}a&b&c&d\\0&a&a+b&a+b+c\\0&a&2a+b&3a+2b+c\\0&a&3a+b&6a+3b+c\end{vmatrix}\overset{r_4-r_3}{\underset{r_3-r_2}=}\begin{vmatrix}a&b&c&d\\0&a&a+b&a+b+c\\0&0&a&2a+b\\0&0&a&3a+b\end{vmatrix}\overset{r_4-r_3}=\begin{vmatrix}a&b&c&d\\0&a&a+b&a+b+c\\0&0&a&2a+b\\0&0&0&a\end{vmatrix}=a^4.\end{array}
$$



$$
\mathrm{四阶行列式}\begin{vmatrix}2&1&1&x\\1&2&1&y\\1&1&2&z\\1&1&1&t\end{vmatrix}=(\;\;\;\;).\;
$$
$$
A.
x-y-z+4t \quad B.x+y-z+4t \quad C.-x-y-z+4t \quad D.-x-y-z-4t \quad E. \quad F. \quad G. \quad H.
$$
$$
\begin{vmatrix}2&1&1&x\\1&2&1&y\\1&1&2&z\\1&1&1&t\end{vmatrix}=-\begin{vmatrix}1&1&1&t\\1&2&1&y\\1&1&2&z\\2&1&1&x\end{vmatrix}=-\begin{vmatrix}1&1&1&t\\0&1&0&y-t\\0&0&1&z-t\\0&-1&-1&x-2t\end{vmatrix}=-\begin{vmatrix}1&1&1&t\\0&1&0&y-t\\0&0&1&z-t\\0&0&0&x+y+z-4t\end{vmatrix}=-x-y-z+4t.
$$



$$
\mathrm{方程}\begin{vmatrix}1&1&2&3\\1&2-x^2&2&3\\2&3&1&5\\2&3&1&9-x^2\end{vmatrix}=0,则\;x=(\;\;\;\;\;).\;
$$
$$
A.
x=±1或x=±2 \quad B.x=1或x=2 \quad C.x=-1或x=2 \quad D.x=1或x=-2 \quad E. \quad F. \quad G. \quad H.
$$
$$
\begin{array}{l}\begin{vmatrix}1&1&2&3\\1&2-x^2&2&3\\2&3&1&5\\2&3&1&9-x^2\end{vmatrix}=\begin{vmatrix}1&1&2&3\\0&1-x^2&0&0\\2&3&1&5\\0&0&0&4-x^2\end{vmatrix}\overset{\begin{array}{c}c_1-2c_3\\c_2-3c_3\end{array}}=\begin{vmatrix}-3&-5&2&3\\0&1-x^2&0&0\\0&0&1&5\\0&0&0&4-x^2\end{vmatrix}\\=-3\left(1-x^2\right)\left(4-x^2\right)=0⇒ x=±1,x=±2.\end{array}
$$



$$
\mathrm{方程}\begin{vmatrix}2&2&2&2\\2&x&3&3\\3&3&x&4\\4&4&4&x\end{vmatrix}=0,\mathrm{方程的解是}\left(\;\;\;\;\;\right).
$$
$$
A.
x_1=x_2=x_3=4 \quad B.x_1=1,x_2=2,x_3=4 \quad C.x_1=x_2=x_3=2 \quad D.x_1=2,x_2=3,x_3=4 \quad E. \quad F. \quad G. \quad H.
$$
$$
\begin{array}{l}\begin{vmatrix}2&2&2&2\\2&x&3&3\\3&3&x&4\\4&4&4&x\end{vmatrix}=2\begin{vmatrix}1&1&1&1\\2&x&3&3\\3&3&x&4\\4&4&4&x\end{vmatrix}=2\begin{vmatrix}1&1&1&1\\0&x-2&1&1\\0&0&x-3&1\\0&0&0&x-4\end{vmatrix}=2\left(x-2\right)\left(x-3\right)\left(x-4\right)=0,\\故x_1=2,x_2=3,x_3=4.\end{array}
$$



$$
\mathrm{设四阶行列式}\begin{vmatrix}1&x&x&x\\x&1&0&0\\x&0&1&0\\x&0&0&1\end{vmatrix}=-3,则x=(\;\;\;).\;
$$
$$
A.
±\frac{\sqrt[{}]2}3 \quad B.\frac{\sqrt[{}]2}3 \quad C.\frac{\sqrt[{}]3}3 \quad D.±\frac{2\sqrt[{}]3}3 \quad E. \quad F. \quad G. \quad H.
$$
$$
\begin{array}{l}\begin{vmatrix}1&x&x&x\\x&1&0&0\\x&0&1&0\\x&0&0&1\end{vmatrix}\overset{\begin{array}{c}c_1-xc_2\\c_1-xc_3\\c_1-xc_4\end{array}}=\begin{vmatrix}1-3x^2&x&x&x\\0&1&0&0\\0&0&1&0\\0&0&0&1\end{vmatrix}=1-3x^2=-3,\\则x=±\frac{2\sqrt[{}]3}3.\end{array}
$$



$$
设\begin{vmatrix}1&1&0&0\\1&k&1&0\\0&0&k&2\\0&0&2&k\end{vmatrix}\neq0,则k\mathrm{应满足}\left(\;\;\;\right).
$$
$$
A.
k\neq1,k\neq2 \quad B.k\neq±2 \quad C.k\neq1,k\neq-2 \quad D.k\neq1,k\neq-2,k\neq2 \quad E. \quad F. \quad G. \quad H.
$$
$$
\begin{array}{l}\begin{vmatrix}1&1&0&0\\1&k&1&0\\0&0&k&2\\0&0&2&k\end{vmatrix}=\begin{vmatrix}1&1&0&0\\0&k-1&1&0\\0&0&k&2\\0&0&2&k\end{vmatrix}=-\begin{vmatrix}1&1&0&0\\0&k-1&1&0\\0&0&2&k\\0&0&k&2\end{vmatrix}=-\begin{vmatrix}1&1&0&0\\0&k-1&1&0\\0&0&2&k\\0&0&0&2-\frac{k^2}2\end{vmatrix}\\=-2\left(k-1\right)\left(2-\frac{k^2}2\right)=-\left(k-1\right)\left(4-k^2\right)=\left(k-1\right)\left(k+2\right)\left(k-2\right)\neq0,\\故k\neq1,k\neq-2,k\neq2.\\\end{array}
$$



$$
\mathrm{四阶行列式}\begin{vmatrix}1&-1&1&x-1\\1&-1&x+1&-1\\1&x-1&1&-1\\1+x&-1&1&-1\end{vmatrix}=(\;\;\;).\;
$$
$$
A.
x^4 \quad B.1 \quad C.-1 \quad D.x^3 \quad E. \quad F. \quad G. \quad H.
$$
$$
\begin{array}{l}将2,3,4\mathrm{列都加到第}1\mathrm{列后提取公因子}x,有\\\mathrm{原式}=x\begin{vmatrix}1&-1&1&x-1\\1&-1&x+1&-1\\1&x-1&1&-1\\1&-1&1&-1\end{vmatrix}=x\begin{vmatrix}1&-1&1&x-1\\0&0&x&-x\\0&x&0&-x\\0&0&0&-x\end{vmatrix}=x^4.\end{array}
$$



$$
\mathrm{四阶行列式}\begin{vmatrix}1&x&y&z\\x&1&0&0\\y&0&1&0\\z&0&0&1\end{vmatrix}=(\;\;\;).\;
$$
$$
A.
1-x^2-y^2-z^2 \quad B.1 \quad C.1+x^2+y^2+z^2 \quad D.1-x-y-z \quad E. \quad F. \quad G. \quad H.
$$
$$
\begin{vmatrix}1&x&y&z\\x&1&0&0\\y&0&1&0\\z&0&0&1\end{vmatrix}\overset{c_1-c_2· x-c_3· y-c_4· z}=\begin{vmatrix}1-x^2-y^2-z^2&x&y&z\\0&1&0&0\\0&0&1&0\\0&0&0&1\end{vmatrix}=1-x^2-y^2-z^2.
$$



$$
\mathrm{四阶行列式}\begin{vmatrix}a_0&1&2&3\\1&a_1&0&0\\2&0&a_2&0\\3&0&0&a_3\end{vmatrix}=(\;\;\;),\left(a_1a_2a_3\neq0\right).\;
$$
$$
A.
\left(a_0-\frac1{a_1}-\frac{2^2}{a_2}-\frac{3^2}{a_3}\right)a_1a_2a_3 \quad B.\left(a_0-\frac1{a_1}-\frac2{a_2}-\frac3{a_3}\right)a_1a_2a_3 \quad C.a_0a_1a_2a_3 \quad D.a_0-\frac1{a_1}-\frac{2^2}{a_2}-\frac{3^2}{a_3} \quad E. \quad F. \quad G. \quad H.
$$
$$
\begin{array}{l}\mathrm{行列式的第}i+1\mathrm{列乘}-\frac i{a_i}\left(i=1,2,3\right)\mathrm{加到第一列上去},得\\\begin{vmatrix}a_0-\frac1{a_1}-\frac{2^2}{a_2}-\frac{3^2}{a_3}&1&2&3\\0&a_1&0&0\\0&0&a_2&0\\0&0&0&a_3\end{vmatrix}=\left(a_0-\frac1{a_1}-\frac{2^2}{a_2}-\frac{3^2}{a_3}\right)a_1a_2a_3.\end{array}
$$



$$
\mathrm{四阶行列式}\begin{vmatrix}a_0&b_1&b_2&b_3\\c_1&a_1&0&0\\c_2&0&a_2&0\\c_3&0&0&a_3\end{vmatrix}=(\;\;\;),\mathrm{其中}a_i\neq0.\;
$$
$$
A.
a_1a_2a_3(a_0-∑_{i=1}^3\frac{b_ic_i}{a_i}) \quad B.a_1a_2a_3a_0 \quad C.0 \quad D.b_1b_2b_3 \quad E. \quad F. \quad G. \quad H.
$$
$$
\begin{array}{l}\mathrm{化为三角形行列式},\mathrm{把第}i+1\left(i=1,2,3\right)\mathrm{列的}-\frac{c_i}{a_i}\mathrm{倍加到第一列},得\\\begin{vmatrix}a_0-∑_{i=1}^3\frac{b_ic_i}{a_i}&b_1&b_2&b_3\\0&a_1&0&0\\0&0&a_2&0\\0&0&0&a_3\end{vmatrix}=a_1a_2a_3(a_0-∑_{i=1}^3\frac{b_ic_i}{a_i}).\end{array}
$$



$$
\left(n+1\right)\mathrm{阶行列式}D=\begin{vmatrix}a_0&1&1&⋯&1\\1&a_0&1&⋯&1\\1&1&a_0&⋯&1\\\vdots&\vdots&\vdots&&\vdots\\1&1&1&⋯&a_0\end{vmatrix}=(\;\;\;).\;
$$
$$
A.
\left(a_0+n\right)\left(a_0-1\right)^n \quad B.\left(a_0+n\right)\left(a_0-1\right) \quad C.\left(a_0+n\right)\left(a_0+1\right)^n \quad D.\left(a_0-n\right)\left(a_0-1\right)^n \quad E. \quad F. \quad G. \quad H.
$$
$$
\begin{array}{l}D=\begin{vmatrix}a_0&1&1&⋯&1\\1&a_0&1&⋯&1\\1&1&a_0&⋯&1\\\vdots&\vdots&\vdots&&\vdots\\1&1&1&⋯&a_0\end{vmatrix}=\begin{vmatrix}a_0+n&1&1&⋯&1\\a_0+n&a_0&1&⋯&1\\a_0+n&1&a_0&⋯&1\\\vdots&\vdots&\vdots&&\vdots\\a_0+n&1&1&⋯&a_0\end{vmatrix}\\=\left(a_0+n\right)\begin{vmatrix}1&1&1&⋯&1\\1&a_0&1&⋯&1\\1&1&a_0&⋯&1\\\vdots&\vdots&\vdots&&\vdots\\1&1&1&⋯&a_0\end{vmatrix}=\left(a_0+n\right)\begin{vmatrix}1&1&1&⋯&1\\0&a_0-1&0&⋯&0\\0&0&a_0-1&⋯&0\\\vdots&\vdots&\vdots&&\vdots\\0&0&0&⋯&a_0-1\end{vmatrix}\\=\left(a_0+n\right)\left(a_0-1\right)^n.\end{array}
$$



$$
\mathrm{方程}\begin{vmatrix}1&1&1&⋯&1&1\\1&1-x&1&⋯&1&1\\1&1&2-x&⋯&1&1\\⋯&⋯&⋯&⋯&⋯&⋯\\1&1&1&⋯&\left(n-2\right)-x&1\\1&1&1&⋯&1&\left(n-1\right)-x\end{vmatrix}=0,则x=(\;\;\;).\;
$$
$$
A.
x_1=0,x_2=1,⋯,\;x_{n-2}=n-3,\;x_{n-1}=n-2 \quad B.x_1=1,x_2=2,⋯,\;x_{n-2}=n-2,\;x_{n-1}=n-1 \quad C.x_1=2,x_2=3,⋯,\;x_{n-2}=n-1,\;x_{n-1}=n \quad D.x_1=x_2=⋯ x_{n-1}=0 \quad E. \quad F. \quad G. \quad H.
$$
$$
\begin{array}{l}\mathrm{方程左边}=\begin{vmatrix}1&1&1&⋯&1&1\\0&-x&0&⋯&0&0\\0&0&1-x&⋯&0&0\\⋯&⋯&⋯&⋯&⋯&⋯\\0&0&0&⋯&\left(n-3\right)-x&0\\0&0&0&⋯&0&\left(n-2\right)-x\end{vmatrix}\\\;\;\;\;\;\;\;\;\;\;\;\;\;\;\;\;\;\;\;\;\;=-x\left(1-x\right)⋯\left[\left(n-3\right)-x\right]\left[\left(n-2\right)-x\right]\\\;\;\;\;\;\;\;\;\;\;\;\;\;\;\;\;\;\;\;\;\;⇒ x_1=0,\;x_2=1,⋯,\;x_{n-2}=n-3,\;x_{n-1}=n-2.\end{array}
$$



$$
\left(n+1\right)\mathrm{阶行列式}D_{n+1}\begin{vmatrix}x&a_1&a_2&a_3&⋯&a_n\\a_1&x&a_2&a_3&⋯&a_n\\a_1&a_2&x&a_3&⋯&a_n\\⋯&⋯&⋯&⋯&⋯&⋯\\a_1&a_2&a_3&a_4&⋯&x\end{vmatrix}=(\;\;\;).\;
$$
$$
A.
\left(x+∑_{i=1}^na_i\right)\prod_{i=1}^n\left(x-a_i\right) \quad B.\left(x+∑_{i=1}^na_i\right)\prod_{i=1}^n\left(x+a_i\right) \quad C.\left(x-∑_{i=1}^na_i\right)\prod_{i=1}^n\left(x-a_i\right) \quad D.\left(x-∑_{i=1}^na_i\right)\prod_{i=1}^n\left(x+a_i\right) \quad E. \quad F. \quad G. \quad H.
$$
$$
\begin{array}{l}\mathrm{将第}2,3,⋯,n+1\mathrm{列都加到第一列},得\;\\D_{n+1}=\left(x+∑_{i=1}^na_i\right)\begin{vmatrix}1&a_1&a_2&a_3&⋯&a_n\\1&x&a_2&a_3&⋯&a_n\\1&a_2&x&a_3&⋯&a_n\\⋯&⋯&⋯&⋯&⋯&⋯\\1&a_2&a_3&a_4&⋯&x\end{vmatrix},\\\mathrm{将第}j\mathrm{列减去第}1\mathrm{列乘以}a_j,有\\D_{n+1}=\left(x+∑_{i=1}^na_i\right)\begin{vmatrix}1&0&0&⋯&0\\1&x-a_1&0&⋯&0\\1&a_2-a_1&x-a_2&⋯&0\\⋯&⋯&⋯&⋯&⋯\\1&a_2-a_1&a_3-a_2&⋯&x-a_n\end{vmatrix}=\left(x+∑_{i=1}^na_i\right)\prod_{i=1}^n\left(x-a_i\right).\end{array}
$$



$$
n\mathrm{阶行列式}D_n=\begin{vmatrix}1&2&3&⋯&n\\-1&0&3&⋯&n\\-1&-2&0&⋯&n\\\vdots&\vdots&\vdots&&\vdots\\-1&-2&-3&⋯&0\end{vmatrix}=(\;\;\;).\;
$$
$$
A.
n! \quad B.\left(n-1\right)! \quad C.\left(-1\right)^nn! \quad D.\left(-1\right)^{n-1}\left(n-1\right)! \quad E. \quad F. \quad G. \quad H.
$$
$$
D_n\overset{\begin{array}{c}r_2+r_1\\r_3+r_1\\\vdots\\r_n+r_1\end{array}}=\begin{vmatrix}1&2&3&⋯&n\\0&2&6&\cdots&2n\\0&0&3&⋯&2n\\\vdots&\vdots&\vdots&&\vdots\\0&0&0&⋯&n\end{vmatrix}=n!.
$$



$$
\left(n+1\right)\mathrm{阶行列式}\begin{vmatrix}1&a_1&a_2&\cdots&a_n\\1&a_1+b_1&a_2&⋯&a_n\\1&a_1&a_2+b_2&⋯&a_n\\⋯&⋯&⋯&⋯&⋯\\1&a_1&a_2&⋯&a_n+b_n\end{vmatrix}=(\;\;\;).\;
$$
$$
A.
a_1a_2⋯ a_n \quad B.\left(a_1+b_1\right)\left(a_2+b_2\right)⋯\left(a_n+b_n\right) \quad C.b_1b_2⋯ b_n \quad D.\left(a_1-b_1\right)\left(a_2-b_2\right)⋯\left(a_n-b_n\right) \quad E. \quad F. \quad G. \quad H.
$$
$$
\mathrm{原式}=\begin{vmatrix}1&a_1&a_2&⋯&a_n\\0&b_1&0&⋯&0\\0&0&b_2&⋯&0\\⋯&⋯&⋯&⋯&⋯\\0&0&0&⋯&b_n\end{vmatrix}=b_1b_2⋯ b_n.
$$



$$
n\mathrm{阶行列式}D_n=\begin{vmatrix}2&1&1&⋯&1\\1&2&1&⋯&1\\1&1&2&⋯&1\\⋯&⋯&⋯&⋯&⋯\\1&1&1&⋯&2\end{vmatrix}=(\;\;\;).\;
$$
$$
A.
n \quad B.n+1 \quad C.2^n \quad D.2^{n-1} \quad E. \quad F. \quad G. \quad H.
$$
$$
\begin{array}{l}\mathrm{将第}2,3,⋯,n\mathrm{行元素都加到第}1\mathrm{行上},得\\D_n=\begin{vmatrix}n+1&n+1&n+1&⋯&n+1\\1&2&1&⋯&1\\1&1&2&⋯&1\\⋯&⋯&⋯&⋯&⋯\\1&1&1&⋯&2\end{vmatrix}=\left(n+1\right)\begin{vmatrix}1&1&1&⋯&1\\1&2&1&\cdots&1\\1&1&2&⋯&1\\⋯&⋯&⋯&⋯&\cdots\\1&1&1&⋯&2\end{vmatrix}\\=\left(n+1\right)\begin{vmatrix}1&1&1&⋯&1\\0&1&0&⋯&0\\0&0&1&⋯&0\\⋯&⋯&⋯&⋯&⋯\\0&0&0&⋯&1\end{vmatrix}=n+1.\end{array}
$$



$$
\left(n+1\right)\mathrm{阶行列式}\begin{vmatrix}1&a_1&a_2&⋯&a_n\\1&a_1+a_2&a_2&⋯&a_n\\1&a_1&a_2+a_3&⋯&a_n\\\vdots&\vdots&\vdots&&\vdots\\1&a_1&a_2&⋯&a_n+a_1\end{vmatrix}=(\;\;\;).\;
$$
$$
A.
a_1a_2⋯ a_n \quad B.a_1+a_2+⋯+a_n \quad C.1 \quad D.0 \quad E. \quad F. \quad G. \quad H.
$$
$$
\mathrm{原式}=\begin{vmatrix}1&a_1&a_2&⋯&a_n\\0&a_2&0&⋯&0\\0&0&a_3&⋯&0\\\vdots&\vdots&\vdots&\vdots&\vdots\\0&0&0&⋯&a_1\end{vmatrix}=a_1a_2⋯ a_n.
$$



$$
\mathrm{四阶行列式}\begin{vmatrix}1&1&1&1\\2x&1&0&0\\3x&0&1&0\\4x&0&0&1\end{vmatrix}=(\;\;\;).\;\mathrm{其中}x\neq0
$$
$$
A.
1-9x \quad B.1-2x \quad C.1-3x \quad D.1-4x \quad E. \quad F. \quad G. \quad H.
$$
$$
\begin{vmatrix}1&1&1&1\\2x&1&0&0\\3x&0&1&0\\4x&0&0&1\end{vmatrix}=\begin{vmatrix}1-2x&1&1&1\\0&1&0&0\\3x&0&1&0\\4x&0&0&1\end{vmatrix}=\begin{vmatrix}1-2x-3x-4x&1&1&1\\0&1&0&0\\0&0&1&0\\0&0&0&1\end{vmatrix}=1-9x.
$$



$$
n\mathrm{阶行列式}\begin{vmatrix}x&a&⋯&a\\a&x&⋯&a\\⋯&⋯&⋯&⋯\\a&a&⋯&x\end{vmatrix}=(\;\;\;).\;
$$
$$
A.
\left(x-a\right)^n \quad B.\left(x-a\right)^{n-1}\left[x+\left(n-1\right)a\right] \quad C.\left(x-a\right)^{n-1}\left[x-\left(n-1\right)a\right] \quad D.\left(x+a\right)^n \quad E. \quad F. \quad G. \quad H.
$$
$$
\begin{array}{l}\mathrm{将原行列式的第}2\mathrm{列至第}n\mathrm{列加到第}1列,\mathrm{并提出第}1\mathrm{列公因子}x+\left(n-1\right)a,得\;\\\mathrm{原式}=\left[x+\left(n-1\right)a\right]\begin{vmatrix}1&a&\cdots&a\\1&x&⋯&a\\\cdots&⋯&⋯&⋯\\1&a&⋯&x\end{vmatrix}\\=\left[x+\left(n-1\right)a\right]\begin{vmatrix}1&a&a&⋯&a\\0&x-a&0&⋯&0\\0&0&x-a&\cdots&0\\⋯&⋯&⋯&⋯&⋯\\0&0&0&⋯&x-a\end{vmatrix}\\=\left[x+\left(n-1\right)a\right]\left(x-a\right)^{n-1}.\end{array}
$$



$$
n\mathrm{阶行列式}D=\begin{vmatrix}1&1&1&⋯&1\\1&0&1&⋯&1\\1&1&0&⋯&1\\⋯&⋯&⋯&⋯&⋯\\1&1&1&⋯&0\end{vmatrix}=(\;\;\;).\;
$$
$$
A.
1 \quad B.\left(-1\right)^{n-1} \quad C.0 \quad D.-1 \quad E. \quad F. \quad G. \quad H.
$$
$$
D=\begin{vmatrix}1&1&1&⋯&1\\1&0&1&⋯&1\\1&1&0&\cdots&1\\⋯&\cdots&⋯&⋯&⋯\\1&1&1&⋯&0\end{vmatrix}=\begin{vmatrix}1&1&1&⋯&1\\0&-1&0&⋯&0\\0&0&-1&⋯&0\\⋯&⋯&⋯&⋯&⋯\\0&0&0&⋯&-1\end{vmatrix}=\left(-1\right)^{n-1}.
$$



$$
\left(n+1\right)\mathrm{阶行列式}\begin{vmatrix}1&a_1&0&0&⋯&0&0\\-1&1-a_1&a_2&0&⋯&0&0\\0&-1&1-a_2&a_3&⋯&0&0\\\vdots&\vdots&\vdots&\vdots&&\vdots&\vdots\\0&0&0&0&⋯&1-a_{n-1}&a_n\\0&0&0&0&⋯&-1&1-a_n\end{vmatrix}=(\;\;\;).\;
$$
$$
A.
0 \quad B.1 \quad C.-1 \quad D.a_1a_2⋯ a_n \quad E. \quad F. \quad G. \quad H.
$$
$$
\begin{array}{l}\mathrm{从第一行开始},\mathrm{每行都逐次往下一行上加},\mathrm{即变为对角线都是}1\mathrm{的三角行列式},即\;\\\mathrm{原式}=\begin{vmatrix}\begin{array}{ccccccc}1&a_1&0&0&⋯&0&0\\0&1&a_2&0&⋯&0&0\\0&-1&1-a_2&a_3&⋯&0&0\\\vdots&\vdots&\vdots&\vdots&&\vdots&\vdots\\0&0&0&0&⋯&1-a_{n-1}&a_n\\0&0&0&0&⋯&-1&1-a_n\end{array}\end{vmatrix}=\begin{vmatrix}\begin{array}{ccccccc}1&a_1&0&0&⋯&0&0\\0&1&a_2&0&⋯&0&0\\0&0&1&a_3&⋯&0&0\\\vdots&\vdots&\vdots&\vdots&&\vdots&\vdots\\0&0&0&0&⋯&1-a_{n-1}&a_n\\0&0&0&0&⋯&-1&1-a_n\end{array}\end{vmatrix}=...=\\\begin{vmatrix}\begin{array}{ccccccc}1&a_1&0&0&⋯&0&0\\0&1&a_2&0&⋯&0&0\\0&0&1&a_3&⋯&0&0\\\vdots&\vdots&\vdots&\vdots&&\vdots&\vdots\\0&0&0&0&⋯&1&a_n\\0&0&0&0&⋯&0&1\end{array}\end{vmatrix}=1.\end{array}
$$



$$
n\mathrm{阶行列式}D_n=\begin{vmatrix}4&5&5&⋯&5\\5&4&5&⋯&5\\5&5&4&⋯&5\\\vdots&\vdots&\vdots&&\vdots\\5&5&5&⋯&4\end{vmatrix}=(\;\;\;).\;
$$
$$
A.
\left(5n-1\right)\left(-1\right)^{n-1} \quad B.\left(5n-1\right)\left(-1\right)^n \quad C.\left(5n+1\right)\left(-1\right)^{n-1} \quad D.\left(5n+1\right)\left(-1\right)^n \quad E. \quad F. \quad G. \quad H.
$$
$$
\begin{array}{l}D_n=\begin{vmatrix}4&5&5&⋯&5\\5&4&5&⋯&5\\5&5&4&⋯&5\\\vdots&\vdots&\vdots&&\vdots\\5&5&5&⋯&4\end{vmatrix}=\left(4+5\left(n-1\right)\right)\begin{vmatrix}1&5&5&⋯&5\\1&4&5&⋯&5\\1&5&4&⋯&5\\\vdots&\vdots&\vdots&&\vdots\\1&5&5&⋯&4\end{vmatrix}=\left(4+5\left(n-1\right)\right)\begin{vmatrix}1&5&5&⋯&5\\0&-1&0&⋯&0\\0&0&-1&⋯&0\\\vdots&\vdots&\vdots&&\vdots\\0&0&0&⋯&-1\end{vmatrix}\\=\left(5n-1\right)\left(-1\right)^{n-1}.\end{array}
$$



$$
\left(n+1\right)\mathrm{阶行列式}D=\begin{vmatrix}-1&1&0&⋯&0&0\\0&-2&2&⋯&0&0\\0&0&-3&⋯&0&0\\\vdots&\vdots&\vdots&&\vdots&\vdots\\0&0&0&⋯&-n&n\\1&1&1&⋯&1&1\end{vmatrix}=(\;\;\;).\;
$$
$$
A.
\left(-1\right)^n\left(n+1\right)! \quad B.\left(n+1\right)! \quad C.\left(-1\right)^nn! \quad D.n! \quad E. \quad F. \quad G. \quad H.
$$
$$
D=\begin{vmatrix}-1&1&0&⋯&0&0\\0&-2&2&⋯&0&0\\0&0&-3&⋯&0&0\\\vdots&\vdots&\vdots&&\vdots&\vdots\\0&0&0&⋯&-n&n\\0&2&1&⋯&1&1\end{vmatrix}=...=\begin{vmatrix}-1&1&0&⋯&0&0\\0&-2&0&⋯&0&0\\0&0&-3&⋯&0&0\\\vdots&\vdots&\vdots&&\vdots&\vdots\\0&0&0&⋯&-n&0\\0&0&0&⋯&0&n+1\end{vmatrix}=\left(-1\right)^n\left(n+1\right)!.
$$



$$
n\mathrm{阶行列式}D=\begin{vmatrix}1&2&3&⋯&n\\2&2&0&⋯&0\\3&0&3&⋯&0\\\vdots&\vdots&\vdots&&\vdots\\n&0&0&⋯&n\end{vmatrix}=(\;\;\;).\;
$$
$$
A.
n!\begin{pmatrix}1-\frac{n\left(n+1\right)}2\end{pmatrix} \quad B.n! \quad C.n!\frac{n\left(n+1\right)}2 \quad D.n!\begin{pmatrix}2-\frac{n\left(n+1\right)}2\end{pmatrix} \quad E. \quad F. \quad G. \quad H.
$$
$$
D\overset{c_1-c_2-⋯-c_n}=\begin{vmatrix}1-2-⋯-n&2&3&⋯&n\\0&2&0&⋯&0\\0&0&3&⋯&0\\\vdots&\vdots&\vdots&&\vdots\\0&0&0&⋯&n\end{vmatrix}=n!\begin{pmatrix}2-\frac{n\left(n+1\right)}2\end{pmatrix}.
$$



$$
\mathrm{四阶行列式}\begin{vmatrix}1&1&1&1\\-1&0&1&1\\0&-1&1&1\\0&0&-1&1\end{vmatrix}=().
$$
$$
A.
8 \quad B.-8 \quad C.6 \quad D.-4 \quad E. \quad F. \quad G. \quad H.
$$
$$
\begin{vmatrix}1&1&1&1\\-1&0&1&1\\0&-1&1&1\\0&0&-1&1\end{vmatrix}=\begin{vmatrix}1&1&1&1\\0&1&2&2\\0&-1&1&1\\0&0&-1&1\end{vmatrix}=\begin{vmatrix}1&1&1&1\\0&1&2&2\\0&0&3&3\\0&0&-1&1\end{vmatrix}=3\begin{vmatrix}1&1&1&1\\0&1&2&2\\0&0&1&1\\0&0&-1&1\end{vmatrix}=3\begin{vmatrix}1&1&1&1\\0&1&2&2\\0&0&1&1\\0&0&0&2\end{vmatrix}=6.
$$



$$
\mathrm{四阶行列式}\begin{vmatrix}1&1&1&0\\1&2&3&4\\1&3&6&10\\0&4&10&18\end{vmatrix}=().
$$
$$
A.
0 \quad B.-2 \quad C.2 \quad D.3 \quad E. \quad F. \quad G. \quad H.
$$
$$
\mathrm{原式}=\begin{vmatrix}1&1&1&0\\0&1&2&4\\0&2&5&10\\0&4&10&18\end{vmatrix}=\begin{vmatrix}1&1&1&0\\0&1&2&4\\0&0&1&2\\0&0&2&2\end{vmatrix}=\begin{vmatrix}1&1&1&0\\0&1&2&4\\0&0&1&2\\0&0&0&-2\end{vmatrix}=-2.
$$



$$
\mathrm{三阶行列式}\begin{vmatrix}1&2&3\\2&3&1\\2&1&3\end{vmatrix}=(\;\;\;).
$$
$$
A.
-12 \quad B.12 \quad C.-16 \quad D.16 \quad E. \quad F. \quad G. \quad H.
$$
$$
\begin{vmatrix}1&2&3\\2&3&1\\2&1&3\end{vmatrix}=\begin{vmatrix}1&2&3\\0&-1&-5\\0&-3&-3\end{vmatrix}=\begin{vmatrix}1&2&3\\0&-1&-5\\0&0&12\end{vmatrix}=-12.
$$



$$
\mathrm{三阶行列式}\begin{vmatrix}1&2&3\\2&0&1\\0&1&2\end{vmatrix}=(\;\;\;).
$$
$$
A.
-2 \quad B.-1 \quad C.1 \quad D.-3 \quad E. \quad F. \quad G. \quad H.
$$
$$
\begin{vmatrix}1&2&3\\2&0&1\\0&1&2\end{vmatrix}=\begin{vmatrix}1&2&3\\0&-4&-5\\0&1&2\end{vmatrix}=\begin{vmatrix}1&2&3\\0&1&2\\0&4&5\end{vmatrix}=-3.
$$



$$
\mathrm{三阶行列式}\begin{vmatrix}x_1-m&x_2&1\\x_1&x_2-m&1\\x_1&x_2&1-m\end{vmatrix}=(\;\;\;).
$$
$$
A.
\left(x_1+x_2+1-m\right)m^2 \quad B.\left(x_1+x_2+1-m\right)m^3 \quad C.\left(x_1+x_2+1-m\right)m \quad D.x_1+x_2+1-m \quad E. \quad F. \quad G. \quad H.
$$
$$
\begin{array}{l}\begin{vmatrix}x_1-m&x_2&1\\x_1&x_2-m&1\\x_1&x_2&1-m\end{vmatrix}=\begin{vmatrix}x_1+x_2+1-m&x_2&1\\x_1+x_2+1-m&\;\;x_2-m&1\\x_1+x_2+1-m&x_2&\;\;1-m\end{vmatrix}=(x_1+x_2+1-m)\begin{vmatrix}1&x_2&1\\1&x_2-m&1\\1&x_2&1-m\end{vmatrix}\\=(x_1+x_2+1-m)\begin{vmatrix}1&x_2&1\\0&-m&0\\0&0&-m\end{vmatrix}=(x_1+x_2+1-m)m^2.\end{array}
$$



$$
\mathrm{四阶行列式}\begin{vmatrix}1&b&b&b\\b&1&b&b\\b&b&1&b\\b&b&b&1\end{vmatrix}=(\;\;\;).
$$
$$
A.
\left(1+3b\right)\left(1-b\right)^3 \quad B.1 \quad C.b^4 \quad D.\left(1+3b\right)\left(1+b\right)^3 \quad E. \quad F. \quad G. \quad H.
$$
$$
\mathrm{原式}=(1+3b)\begin{vmatrix}1&b&b&b\\1&1&b&b\\1&b&1&b\\1&b&b&1\end{vmatrix}=(1+3b)\begin{vmatrix}1&b&b&b\\0&1-b&0&0\\0&0&1-b&0\\0&0&0&1-b\end{vmatrix}=\left(1+3b\right)\left(1-b\right)^3.
$$



$$
\mathrm{四阶行列式}\begin{vmatrix}1&1&1&0\\1&2&3&4\\-1&-3&-6&10\\0&0&6&14\end{vmatrix}=().
$$
$$
A.
122 \quad B.-122 \quad C.92 \quad D.-92 \quad E. \quad F. \quad G. \quad H.
$$
$$
\begin{vmatrix}1&1&1&0\\1&2&3&4\\-1&-3&-6&10\\0&0&6&14\end{vmatrix}=\begin{vmatrix}1&1&1&0\\0&1&2&4\\0&-2&-5&10\\0&0&6&14\end{vmatrix}=\begin{vmatrix}1&1&1&0\\0&1&2&4\\0&0&-1&18\\0&0&6&14\end{vmatrix}=\begin{vmatrix}1&1&1&0\\0&1&2&4\\0&0&-1&18\\0&0&0&122\end{vmatrix}=-122.
$$



$$
\mathrm{四阶行列式}\begin{vmatrix}1&1&1&1\\-1&1&1&1\\0&-1&1&1\\0&0&0&1\end{vmatrix}=().
$$
$$
A.
8 \quad B.-8 \quad C.4 \quad D.-4 \quad E. \quad F. \quad G. \quad H.
$$
$$
\begin{vmatrix}1&1&1&1\\-1&1&1&1\\0&-1&1&1\\0&0&0&1\end{vmatrix}=\begin{vmatrix}1&1&1&1\\0&2&2&2\\0&-1&1&1\\0&0&0&1\end{vmatrix}=2\begin{vmatrix}1&1&1&1\\0&1&1&1\\0&-1&1&1\\0&0&0&1\end{vmatrix}=2\begin{vmatrix}1&1&1&1\\0&1&1&1\\0&0&2&2\\0&0&0&1\end{vmatrix}=4.
$$



$$
\mathrm{四阶行列式}\begin{vmatrix}1&1&1&0\\1&2&3&4\\0&3&6&10\\0&4&10&20\end{vmatrix}=().
$$
$$
A.
1 \quad B.4 \quad C.2 \quad D.3 \quad E. \quad F. \quad G. \quad H.
$$
$$
\mathrm{原式}=\begin{vmatrix}1&1&1&0\\0&1&2&4\\0&3&6&10\\0&4&10&20\end{vmatrix}=\begin{vmatrix}1&1&1&0\\0&1&2&4\\0&0&0&-2\\0&0&2&4\end{vmatrix}=-\begin{vmatrix}1&1&1&0\\0&1&2&4\\0&0&2&4\\0&0&0&-2\end{vmatrix}=4.
$$



$$
\mathrm{四阶行列式}\begin{vmatrix}1&1&1&1\\2&-1&1&1\\3&0&-1&1\\4&1&3&-1\end{vmatrix}=(\;\;\;).\;
$$
$$
A.
36 \quad B.-34 \quad C.-36 \quad D.16 \quad E. \quad F. \quad G. \quad H.
$$
$$
\begin{vmatrix}1&1&1&1\\2&-1&1&1\\3&0&-1&1\\4&1&3&-1\end{vmatrix}=\begin{vmatrix}1&1&1&1\\0&-3&-1&-1\\0&-3&-4&-2\\0&-3&-1&-5\end{vmatrix}=\begin{vmatrix}1&1&1&1\\0&-3&-1&-1\\0&0&-3&-1\\0&0&0&-4\end{vmatrix}=-36.
$$



$$
\mathrm{方程}\begin{vmatrix}1&1&1&1\\2&2+x&1&1\\3&3&3+x&1\\4&4&4&4+x\end{vmatrix}=0\mathrm{的解是}(\;\;\;).
$$
$$
A.
0 \quad B.-1 \quad C.-2 \quad D.-3 \quad E. \quad F. \quad G. \quad H.
$$
$$
\begin{vmatrix}1&1&1&1\\2&2+x&1&1\\3&3&3+x&1\\4&4&4&4+x\end{vmatrix}=\begin{vmatrix}1&1&1&1\\0&x&-1&-1\\0&0&x&-2\\0&0&0&x\end{vmatrix}=x^3=0.
$$



$$
\mathrm{四阶行列式}\begin{vmatrix}0&0&0&-1\\0&0&2&2\\0&-3&0&-3\\4&0&4&4\end{vmatrix}=(\;\;\;).
$$
$$
A.
-24 \quad B.24 \quad C.-36 \quad D.-48 \quad E. \quad F. \quad G. \quad H.
$$
$$
\begin{vmatrix}0&0&0&-1\\0&0&2&2\\0&-3&0&-3\\4&0&4&4\end{vmatrix}\overset{\overset{r_1↔ r4}{r_2↔ r_3}}=\begin{vmatrix}4&0&4&4\\0&-3&0&-3\\0&0&2&2\\0&0&0&-1\end{vmatrix}=24.
$$



$$
\mathrm{四阶行列式}\begin{vmatrix}2&2&2&2\\2&3&2&2\\2&2&3&2\\2&2&2&3\end{vmatrix}=(\;\;\;).
$$
$$
A.
5 \quad B.6 \quad C.4 \quad D.2 \quad E. \quad F. \quad G. \quad H.
$$
$$
\begin{vmatrix}2&2&2&2\\2&3&2&2\\2&2&3&2\\2&2&2&3\end{vmatrix}\overset{\overset{r_2-r_1}{r_3-r_1}}{\underset{r_4-r_1}=}\begin{vmatrix}2&2&2&2\\0&1&0&0\\0&0&1&0\\0&0&0&1\end{vmatrix}=2.
$$



$$
\mathrm{四阶行列式}\begin{vmatrix}2&2&2&2\\-2&1&1&1\\-2&-2&1&1\\-2&-2&-2&1\end{vmatrix}=(\;\;\;).
$$
$$
A.
49 \quad B.54 \quad C.18 \quad D.36 \quad E. \quad F. \quad G. \quad H.
$$
$$
\begin{vmatrix}2&2&2&2\\-2&1&1&1\\-2&-2&1&1\\-2&-2&-2&1\end{vmatrix}=\begin{vmatrix}2&2&2&2\\0&3&3&3\\0&0&3&3\\0&0&0&3\end{vmatrix}=54.
$$



$$
\mathrm{四阶行列式}\begin{vmatrix}1&2&3&4\\0&1&2&3\\0&0&1&2\\2&2&2&2\end{vmatrix}=(\;\;\;\;).\;
$$
$$
A.
0 \quad B.1 \quad C.2 \quad D.3 \quad E. \quad F. \quad G. \quad H.
$$
$$
\begin{vmatrix}1&2&3&4\\0&1&2&3\\0&0&1&2\\2&2&2&2\end{vmatrix}=\begin{vmatrix}1&2&3&4\\0&1&2&3\\0&0&1&2\\0&-2&-4&-6\end{vmatrix}=\begin{vmatrix}1&2&3&4\\0&1&2&3\\0&0&1&2\\0&0&0&0\end{vmatrix}=0.
$$



$$
\mathrm{四阶行列式}\begin{vmatrix}1&3&0&0\\3&2&3&0\\0&3&2&3\\0&0&3&2\end{vmatrix}=(\;\;\;).\;
$$
$$
A.
54 \quad B.17 \quad C.18 \quad D.57 \quad E. \quad F. \quad G. \quad H.
$$
$$
\begin{vmatrix}1&3&0&0\\3&2&3&0\\0&3&2&3\\0&0&3&2\end{vmatrix}=\begin{vmatrix}1&3&0&0\\0&-7&3&0\\0&3&2&3\\0&0&3&2\end{vmatrix}=\begin{vmatrix}1&3&0&0\\0&-7&3&0\\0&0&\frac{23}7&3\\0&0&3&2\end{vmatrix}=\begin{vmatrix}1&3&0&0\\0&-7&3&0\\0&0&\frac{23}7&3\\0&0&0&-\frac{17}{23}\end{vmatrix}=17.
$$



$$
\mathrm{四阶行列式}\begin{vmatrix}2&2&1&1\\1&2&2&1\\1&1&2&2\\2&1&1&2\end{vmatrix}=(\;\;\;).
$$
$$
A.
6 \quad B.8 \quad C.16 \quad D.1 \quad E. \quad F. \quad G. \quad H.
$$
$$
\mathrm{原式}\overset{r_1+r_2+r_3+r_4}=\begin{vmatrix}6&6&6&6\\1&2&1&1\\1&1&2&2\\1&1&1&2\end{vmatrix}=6\begin{vmatrix}1&1&1&1\\1&2&1&1\\1&1&2&2\\1&1&1&2\end{vmatrix}=6\begin{vmatrix}1&1&1&1\\0&1&0&0\\0&0&1&1\\0&0&0&1\end{vmatrix}=6.
$$



$$
\mathrm{四阶行列式}\begin{vmatrix}1&-1&1&0\\2&0&0&6\\1&0&0&-1\\0&2&-1&1\end{vmatrix}=(\;\;\;).
$$
$$
A.
-5 \quad B.8 \quad C.-6 \quad D.6 \quad E. \quad F. \quad G. \quad H.
$$
$$
\begin{vmatrix}1&-1&1&0\\2&0&0&6\\1&0&0&-1\\0&2&-1&1\end{vmatrix}=\begin{vmatrix}1&-1&1&0\\0&2&-2&6\\0&1&-1&-1\\0&2&-1&1\end{vmatrix}=2\begin{vmatrix}1&-1&1&0\\0&1&-1&3\\0&1&-1&-1\\0&2&-1&1\end{vmatrix}=2\begin{vmatrix}1&-1&1&0\\0&1&-1&3\\0&0&0&-4\\0&0&1&-5\end{vmatrix}=8.
$$



$$
\mathrm{四阶行列式}\begin{vmatrix}a&b&c&4\\a&a+b&a+b+c&a+b+c+4\\a&2a+b&3a+2b+c&4a+3b+2c+4\\a&3a+b&6a+3b+c&10a+6b+3c+4\end{vmatrix}=(\;\;\;).\;
$$
$$
A.
a^4 \quad B.a^3 \quad C.4abc \quad D.abc \quad E. \quad F. \quad G. \quad H.
$$
$$
\begin{array}{l}\mathrm{从第}4\mathrm{行开始},\mathrm{后一行减前一行}:\\\mathrm{原式}\overset{\overset{r_4-r_3}{r_3-r_2}}{\underset{r_2-r_1}=}\begin{vmatrix}a&b&c&4\\0&a&a+b&a+b+c\\0&a&2a+b&3a+2b+c\\0&a&3a+b&6a+3b+c\end{vmatrix}\overset{r_4-r_3}{\underset{r_3-r_2}=}\begin{vmatrix}a&b&c&4\\0&a&a+b&a+b+c\\0&0&a&2a+b\\0&0&a&3a+b\end{vmatrix}\overset{r_4-r_3}=\begin{vmatrix}a&b&c&4\\0&a&a+b&a+b+c\\0&0&a&2a+b\\0&0&0&a\end{vmatrix}=a^4.\end{array}
$$



$$
n\mathrm{阶行列式}D_n=\begin{vmatrix}3&1&1&⋯&1\\1&3&1&⋯&1\\1&1&3&⋯&1\\⋯&⋯&⋯&⋯&⋯\\1&1&1&⋯&3\end{vmatrix}=(\;\;\;).\;
$$
$$
A.
(n+2)2^{n-1} \quad B.(n+3)2^{n-1} \quad C.(n+2)2^n \quad D.2^{n-1} \quad E. \quad F. \quad G. \quad H.
$$
$$
\begin{array}{l}\mathrm{将第}2,3,⋯,n\mathrm{行元素都加到第}1\mathrm{行上},得\\D_n=\begin{vmatrix}n+2&n+2&n+2&⋯&n+2\\1&3&1&⋯&1\\1&1&3&⋯&1\\⋯&⋯&⋯&⋯&⋯\\1&1&1&⋯&3\end{vmatrix}=\left(n+2\right)\begin{vmatrix}1&1&1&⋯&1\\1&3&1&⋯&1\\1&1&3&⋯&1\\⋯&⋯&⋯&⋯&⋯\\1&1&1&⋯&3\end{vmatrix}\\=\left(n+2\right)\begin{vmatrix}1&1&1&⋯&1\\0&2&0&⋯&0\\0&0&2&⋯&0\\⋯&⋯&⋯&⋯&⋯\\0&0&0&⋯&2\end{vmatrix}=(n+2)2^{n-1}.\end{array}
$$



$$
n\mathrm{阶行列式}D_n=\begin{vmatrix}0&5&5&⋯&5\\5&0&5&⋯&5\\5&5&0&⋯&5\\\vdots&\vdots&\vdots&&\vdots\\5&5&5&⋯&0\end{vmatrix}=(\;\;\;).\;
$$
$$
A.
\left(-1\right)^{n-1}\left(n-1\right)5^n \quad B.\left(n-1\right)5^n \quad C.\left(-1\right)^{n-1}n5^n \quad D.\left(5n+1\right)\left(-1\right)^n \quad E. \quad F. \quad G. \quad H.
$$
$$
\begin{array}{l}D_n=\begin{vmatrix}0&5&5&⋯&5\\5&0&5&⋯&5\\5&5&0&⋯&5\\\vdots&\vdots&\vdots&&\vdots\\5&5&5&⋯&0\end{vmatrix}=5\left(n-1\right)\begin{vmatrix}1&5&5&⋯&5\\1&0&5&⋯&5\\1&5&0&⋯&5\\\vdots&\vdots&\vdots&&\vdots\\1&5&5&⋯&0\end{vmatrix}=5\left(n-1\right)\begin{vmatrix}1&5&5&⋯&5\\0&-5&0&⋯&0\\0&0&-5&⋯&0\\\vdots&\vdots&\vdots&&\vdots\\0&0&0&⋯&-5\end{vmatrix}\\=5\left(n-1\right)\left(-1\right)^{n-1}5^{n-1}=\left(-1\right)^{n-1}\left(n-1\right)5^n.\end{array}
$$



$$
\mathrm{四阶行列式}\begin{vmatrix}1&b&0&b\\b&1&b&0\\0&b&1&b\\b&0&b&1\end{vmatrix}=(\;\;\;).
$$
$$
A.
1-4b^2 \quad B.1+4b^2 \quad C.b^4 \quad D.\left(1+3b\right)\left(1+b\right)^3 \quad E. \quad F. \quad G. \quad H.
$$
$$
\begin{array}{l}\mathrm{原式}=(1+2b)\begin{vmatrix}1&b&0&b\\1&1&b&0\\1&b&1&b\\1&0&b&1\end{vmatrix}=(1+2b)\begin{vmatrix}1&b&0&b\\0&1-b&b&-b\\0&0&1&0\\0&-b&b&1-b\end{vmatrix}\overset{c_2+c_3}=(1+2b)\begin{vmatrix}1&b&0&b\\0&1&b&-b\\0&1&1&0\\0&0&b&1-b\end{vmatrix}=(1+2b)\begin{vmatrix}1&b&0&b\\0&1&b&-b\\0&0&1-b&b\\0&0&b&1-b\end{vmatrix}=(1+2b)(1-2b)=1-4b^2.\\\end{array}
$$



$$
n\mathrm{阶行列式}D_n=\begin{vmatrix}0&9&9&⋯&9\\9&0&9&⋯&9\\9&9&0&⋯&9\\\vdots&\vdots&\vdots&&\vdots\\9&9&9&⋯&0\end{vmatrix}=(\;\;\;).\;
$$
$$
A.
\left(-1\right)^{n-1}\left(n-1\right)9^n \quad B.\left(n-1\right)9^n \quad C.\left(-1\right)^{n-1}n9^n \quad D.\left(9n+1\right)\left(-1\right)^n \quad E. \quad F. \quad G. \quad H.
$$
$$
\begin{array}{l}D_n=\begin{vmatrix}0&9&9&⋯&9\\9&0&9&⋯&9\\9&9&0&⋯&9\\\vdots&\vdots&\vdots&&\vdots\\9&9&9&⋯&0\end{vmatrix}=9\left(n-1\right)\begin{vmatrix}1&9&9&⋯&9\\1&0&9&⋯&9\\1&9&0&⋯&9\\\vdots&\vdots&\vdots&&\vdots\\1&9&9&⋯&0\end{vmatrix}=9\left(n-1\right)\begin{vmatrix}1&9&9&⋯&9\\0&-9&0&⋯&0\\0&0&-9&⋯&0\\\vdots&\vdots&\vdots&&\vdots\\0&0&0&⋯&-9\end{vmatrix}\\=9\left(n-1\right)\left(-1\right)^{n-1}9^{n-1}=\left(-1\right)^{n-1}\left(n-1\right)9^n.\end{array}
$$



$$
\mathrm{五阶行列式}\begin{vmatrix}1&1&1&0&0\\2&3&-1&0&0\\0&9&1&0&0\\0&0&0&2&1\\0&0&0&4&3\end{vmatrix}=\left(\;\;\;\;\right).
$$
$$
A.
56 \quad B.12 \quad C.58 \quad D.-58 \quad E. \quad F. \quad G. \quad H.
$$
$$
\begin{vmatrix}1&1&1&0&0\\2&3&-1&0&0\\0&9&1&0&0\\0&0&0&2&1\\0&0&0&4&3\end{vmatrix}=\begin{vmatrix}1&1&1&0&0\\0&1&-3&0&0\\0&9&1&0&0\\0&0&0&2&1\\0&0&0&4&3\end{vmatrix}=\begin{vmatrix}1&1&1&0&0\\0&1&-3&0&0\\0&0&28&0&0\\0&0&0&2&1\\0&0&0&4&3\end{vmatrix}=\begin{vmatrix}1&1&1&0&0\\0&1&-3&0&0\\0&0&28&0&0\\0&0&0&2&1\\0&0&0&0&1\end{vmatrix}=56.
$$



$$
\mathrm{四阶行列式}\begin{vmatrix}1&1&1&7\\1&1&7&1\\1&7&1&1\\7&1&1&1\end{vmatrix}=(\;\;\;\;).
$$
$$
A.
-2170 \quad B.-2160 \quad C.2170 \quad D.2160 \quad E. \quad F. \quad G. \quad H.
$$
$$
\mathrm{原式}=10\begin{vmatrix}1&1&1&7\\1&1&7&1\\1&7&1&1\\1&1&1&1\end{vmatrix}=10\begin{vmatrix}1&1&1&7\\0&0&6&-6\\0&6&0&-6\\0&0&0&-6\end{vmatrix}=-10\begin{vmatrix}1&1&1&7\\0&6&0&-6\\0&0&6&-6\\0&0&0&-6\end{vmatrix}=2160.
$$



$$
\mathrm{行列式}D=\begin{vmatrix}1&4&3&7&-6&-1\\1&3&5&8&4&0\\0&0&2&-1&9&8\\0&0&1&1&7&-2\\0&0&0&0&-1&-3\\0&0&0&0&3&5\end{vmatrix}=\left(\;\;\;\;\;\;\right).
$$
$$
A.
-12 \quad B.-24 \quad C.60 \quad D.12 \quad E. \quad F. \quad G. \quad H.
$$
$$
D=\begin{vmatrix}1&4&3&7&-6&-1\\1&3&5&8&4&0\\0&0&2&-1&9&8\\0&0&1&1&7&-2\\0&0&0&0&-1&-3\\0&0&0&0&3&5\end{vmatrix}=\begin{vmatrix}1&4&3&7&-6&-1\\0&-1&2&1&10&1\\0&0&2&-1&9&8\\0&0&1&1&7&-2\\0&0&0&0&-1&-3\\0&0&0&0&3&5\end{vmatrix}=-\begin{vmatrix}1&4&3&7&-6&-1\\0&-1&2&1&10&1\\0&0&1&1&7&-2\\0&0&2&-1&9&8\\0&0&0&0&-1&-3\\0&0&0&0&3&5\end{vmatrix}=-\begin{vmatrix}1&4&3&7&-6&-1\\0&-1&2&1&10&1\\0&0&1&1&7&-2\\0&0&0&-3&-5&12\\0&0&0&0&-1&-3\\0&0&0&0&0&-4\end{vmatrix}=-12.
$$



$$
\mathrm{四阶行列式}\begin{vmatrix}1&0&-1&-1\\0&-1&-1&1\\a&0&0&0\\-1&-1&1&0\end{vmatrix}=(\;\;\;).
$$
$$
A.
3a \quad B.-3a \quad C.2a \quad D.a \quad E. \quad F. \quad G. \quad H.
$$
$$
\begin{vmatrix}1&0&-1&-1\\0&-1&-1&1\\a&0&0&0\\-1&-1&1&0\end{vmatrix}=\begin{vmatrix}1&0&-1&-1\\0&-1&-1&1\\0&0&a&a\\0&-1&0&-1\end{vmatrix}=\begin{vmatrix}1&0&-1&-1\\0&-1&-1&1\\0&0&a&a\\0&0&1&-2\end{vmatrix}=-\begin{vmatrix}1&0&-1&-1\\0&-1&-1&1\\0&0&1&-2\\0&0&a&a\end{vmatrix}=-\begin{vmatrix}1&0&-1&-1\\0&-1&-1&1\\0&0&1&-2\\0&0&0&3a\end{vmatrix}=3a.
$$



$$
\mathrm{四阶行列式}\begin{vmatrix}1&1&0&2\\b&0&3&1\\0&2&1&1\\0&1&1&1\end{vmatrix}=(\;\;\;).
$$
$$
A.
-2-2b \quad B.2-2b \quad C.1-2b \quad D.-2+2b \quad E. \quad F. \quad G. \quad H.
$$
$$
\mathrm{原式}=\begin{vmatrix}1&1&0&2\\0&-b&3&1-2b\\0&2&1&1\\0&1&1&1\end{vmatrix}=-\begin{vmatrix}1&1&0&2\\0&1&1&1\\0&2&1&1\\0&-b&3&1-2b\end{vmatrix}=-\begin{vmatrix}1&1&0&2\\0&1&1&1\\0&0&-1&-1\\0&0&3+b&1-b\end{vmatrix}=-\begin{vmatrix}1&1&0&2\\0&1&1&1\\0&0&-1&-1\\0&0&0&-2-2b\end{vmatrix}=-2-2b.
$$



$$
\mathrm{四阶行列式}\begin{vmatrix}a&1&0&0\\-1&b&1&0\\0&-1&1&1\\0&0&-1&1\end{vmatrix}=(\;\;\;).
$$
$$
A.
2ab+a+2 \quad B.2ab+a+1 \quad C.ab+a+2 \quad D.ab+1 \quad E. \quad F. \quad G. \quad H.
$$
$$
\begin{vmatrix}a&1&0&0\\-1&b&1&0\\0&-1&1&1\\0&0&-1&1\end{vmatrix}=-\begin{vmatrix}-1&b&1&0\\a&1&0&0\\0&-1&1&1\\0&0&-1&1\end{vmatrix}=-\begin{vmatrix}-1&b&1&0\\0&1+ab&a&0\\0&-1&1&1\\0&0&-1&1\end{vmatrix}=\begin{vmatrix}-1&b&1&0\\0&-1&1&1\\0&1+ab&a&0\\0&0&-1&1\end{vmatrix}=\begin{vmatrix}-1&b&1&0\\0&-1&1&1\\0&0&a+1+ab&1+ab\\0&0&-1&1\end{vmatrix}=2ab+a+2.
$$



$$
\mathrm{四阶行列式}\begin{vmatrix}1&0&a&1\\0&-1&b&-1\\-1&-1&c&-1\\-1&1&0&0\end{vmatrix}=(\;\;\;\;).
$$
$$
A.
a+b \quad B.a+b+c \quad C.a-b \quad D.a \quad E. \quad F. \quad G. \quad H.
$$
$$
\begin{vmatrix}1&0&a&1\\0&-1&b&-1\\-1&-1&c&-1\\-1&1&0&0\end{vmatrix}=\begin{vmatrix}1&0&a&1\\0&-1&b&-1\\0&-1&a+c&0\\0&1&a&1\end{vmatrix}=\begin{vmatrix}1&0&a&1\\0&-1&b&-1\\0&0&a+c-b&1\\0&0&a+b&0\end{vmatrix}=-\begin{vmatrix}1&0&1&a\\0&-1&-1&b\\0&0&1&a+c-b\\0&0&0&a+b\end{vmatrix}=a+b.
$$



$$
\mathrm{四阶行列式}\begin{vmatrix}x&1&0&0\\0&x&1&0\\0&0&x&1\\1&0&0&x\end{vmatrix}=(\;\;\;\;\;).
$$
$$
A.
x^4-1 \quad B.x^4+1 \quad C.x^3-1 \quad D.x^3+1 \quad E. \quad F. \quad G. \quad H.
$$
$$
\begin{array}{l}\mathrm{原式}=\begin{vmatrix}x+1&1&0&0\\x+1&x&1&0\\x+1&0&x&1\\x+1&0&0&x\end{vmatrix}=(x+1)\begin{vmatrix}1&1&0&0\\1&x&1&0\\1&0&x&1\\1&0&0&x\end{vmatrix}=(x+1)\begin{vmatrix}1&1&0&0\\0&x-1&1&0\\0&-1&x&1\\0&-1&0&x\end{vmatrix}=-(x+1)\begin{vmatrix}1&1&0&0\\0&-1&x&1\\0&x-1&1&0\\0&-1&0&x\end{vmatrix}=-(x+1)\begin{vmatrix}1&1&0&0\\0&-1&x&1\\0&0&x^2-x+1&x-1\\0&0&-x&x-1\end{vmatrix}\\\;\;\;\;\;\;\;=-(x+1)\begin{vmatrix}1&1&0&0\\0&-1&x+1&1\\0&0&x^2&x-1\\0&0&-1&x-1\end{vmatrix}=(x+1)\begin{vmatrix}1&1&0&0\\0&-1&x+1&1\\0&0&-1&x-1\\0&0&x^2&x-1\end{vmatrix}=(x+1)\begin{vmatrix}1&1&0&0\\0&-1&x+1&1\\0&0&-1&x-1\\0&0&0&x^3-x^2+x-1\end{vmatrix}=x^4-1.\end{array}
$$



$$
\mathrm{五阶行列式}\begin{vmatrix}1&0&0&0&c\\b&1&0&0&0\\0&b&1&0&0\\0&0&b&1&0\\0&0&0&b&1\end{vmatrix}=(\;\;\;\;).
$$
$$
A.
1+b^4c \quad B.1+b^4 \quad C.1-b^4c \quad D.1-b^4 \quad E. \quad F. \quad G. \quad H.
$$
$$
\begin{vmatrix}1&0&0&0&c\\b&1&0&0&0\\0&b&1&0&0\\0&0&b&1&0\\0&0&0&b&1\end{vmatrix}=\begin{vmatrix}1&0&0&0&c\\0&1&0&0&-bc\\0&b&1&0&0\\0&0&b&1&0\\0&0&0&b&1\end{vmatrix}=\begin{vmatrix}1&0&0&0&c\\0&1&0&0&-bc\\0&0&1&0&b^2c\\0&0&b&1&0\\0&0&0&b&1\end{vmatrix}=\begin{vmatrix}1&0&0&0&c\\0&1&0&0&-bc\\0&0&1&0&b^2c\\0&0&0&1&-b^3c\\0&0&0&b&1\end{vmatrix}=\begin{vmatrix}1&0&0&0&c\\0&1&0&0&-bc\\0&0&1&0&b^2c\\0&0&0&1&-b^3c\\0&0&0&0&1+b^4c\end{vmatrix}=1+b^4c.
$$



$$
\mathrm{四阶行列式}\begin{vmatrix}1&1&0&0\\1&2&b&c\\1&b&b^2+1&bc\\1&c&bc&c^2+1\end{vmatrix}=(\;\;\;\;).
$$
$$
A.
1+b+c \quad B.2+b+c \quad C.2-b+c \quad D.2+b-c \quad E. \quad F. \quad G. \quad H.
$$
$$
\begin{array}{l}\begin{vmatrix}1&1&0&0\\1&2&b&c\\1&b&b^2+1&bc\\1&c&bc&c^2+1\end{vmatrix}=\begin{vmatrix}1&1&0&0\\0&1&b&c\\0&b-1&b^2+1&bc\\0&c-1&bc&c^2+1\end{vmatrix}=\begin{vmatrix}1&1&0&0\\0&1&b&c\\0&0&b+1&c\\0&0&b&c+1\end{vmatrix}=\begin{vmatrix}1&1&0&0\\0&1&b&c\\0&0&1&-1\\0&0&b&c+1\end{vmatrix}=\begin{vmatrix}1&1&0&0\\0&1&b&c\\0&0&1&-1\\0&0&0&b+c+1\end{vmatrix}=b+c+1.\\\\\end{array}
$$



$$
\mathrm{四阶行列式}\begin{vmatrix}2&8&-5&1\\1&9&0&-6\\0&-5&-1&2\\1&0&-7&6\end{vmatrix}=().
$$
$$
A.
120 \quad B.-110 \quad C.-108 \quad D.110 \quad E. \quad F. \quad G. \quad H.
$$
$$
\begin{vmatrix}2&8&-5&1\\1&9&0&-6\\0&-5&-1&2\\1&0&-7&6\end{vmatrix}=-\begin{vmatrix}1&9&0&-6\\2&8&-5&1\\0&-5&-1&2\\1&0&-7&6\end{vmatrix}=-\begin{vmatrix}1&9&0&-6\\0&-10&-5&13\\0&-5&-1&2\\0&-9&-7&12\end{vmatrix}\;\overset{r_2-r_4}=\;-\begin{vmatrix}1&9&0&-6\\0&-1&2&1\\0&-5&-1&2\\0&-9&-7&12\end{vmatrix}\;=-\begin{vmatrix}1&9&0&-6\\0&-1&2&1\\0&0&-11&-3\\0&0&-25&3\end{vmatrix}=-\begin{vmatrix}1&9&0&-6\\0&-1&2&1\\0&0&-11&-3\\0&0&0&\frac{108}{11}\end{vmatrix}=-108.
$$



$$
\mathrm{五阶行列式}\begin{vmatrix}1&1&1&1&0\\0&1&1&1&1\\1&2&3&0&0\\0&1&2&3&0\\0&0&1&2&3\end{vmatrix}=().
$$
$$
A.
16 \quad B.-16 \quad C.2 \quad D.-2 \quad E. \quad F. \quad G. \quad H.
$$
$$
\begin{vmatrix}1&1&1&1&0\\0&1&1&1&1\\1&2&3&0&0\\0&1&2&3&0\\0&0&1&2&3\end{vmatrix}=\begin{vmatrix}1&1&1&1&0\\0&1&1&1&1\\0&1&2&-1&0\\0&1&2&3&0\\0&0&1&2&3\end{vmatrix}=\begin{vmatrix}1&1&1&1&0\\0&1&1&1&1\\0&0&1&-2&-1\\0&0&1&2&-1\\0&0&1&2&3\end{vmatrix}=\begin{vmatrix}1&1&1&1&0\\0&1&1&1&1\\0&0&1&-2&-1\\0&0&0&4&0\\0&0&0&4&4\end{vmatrix}=16.
$$



$$
\mathrm{三阶行列式}\begin{vmatrix}3&1&301\\1&2&102\\2&4&200\end{vmatrix}=().
$$
$$
A.
25 \quad B.-20 \quad C.0 \quad D.1 \quad E. \quad F. \quad G. \quad H.
$$
$$
\mathrm{原式}=\begin{vmatrix}3&1&300\\1&2&100\\2&4&200\end{vmatrix}+\begin{vmatrix}3&1&1\\1&2&2\\2&4&0\end{vmatrix}=\begin{vmatrix}3&1&1\\1&2&2\\2&4&0\end{vmatrix}=-20.
$$



$$
\mathrm{三阶行列式}\begin{vmatrix}1&2&3\\0&1&2\\1&1&1\end{vmatrix}=().
$$
$$
A.
0 \quad B.1 \quad C.-1 \quad D.2 \quad E. \quad F. \quad G. \quad H.
$$
$$
\begin{vmatrix}1&2&3\\0&1&2\\1&1&1\end{vmatrix}=\begin{vmatrix}1&2&3\\0&1&2\\0&-1&-2\end{vmatrix}=0.
$$



$$
\mathrm{三阶行列式}\begin{vmatrix}10&20&15\\25&75&125\\18&36&27\end{vmatrix}=().
$$
$$
A.
5^3×9 \quad B.5^3×9×35 \quad C.0 \quad D.\mathrm{以上都不对}. \quad E. \quad F. \quad G. \quad H.
$$
$$
\begin{vmatrix}10&20&15\\25&75&125\\18&36&27\end{vmatrix}=5×25×9×\begin{vmatrix}2&4&3\\1&3&5\\2&4&3\end{vmatrix}=0.
$$



$$
\mathrm{如果}n\mathrm{阶行列式}D(D\neq0)\mathrm{所有元素变号},\mathrm{得到新的行列式}D_1,则D_1\mathrm{与的}D\mathrm{关系是}(\;\;\;\;).
$$
$$
A.
D_1=D \quad B.D_1=-D \quad C.D_1=\left(-1\right)^nD \quad D.D_1=0 \quad E. \quad F. \quad G. \quad H.
$$
$$
\mathrm{将行列式的每行都提取公因数}\left(-1\right),\mathrm{则行列式乘以}\left(-1\right)^n.
$$



$$
\mathrm{三阶行列式}D=\begin{vmatrix}101&100&201\\201&200&400\\301&300&600\end{vmatrix}=().
$$
$$
A.
100 \quad B.1000 \quad C.-1000 \quad D.-2000 \quad E. \quad F. \quad G. \quad H.
$$
$$
D=\begin{vmatrix}1&100&1\\1&200&0\\1&300&0\end{vmatrix}=100\begin{vmatrix}1&1&1\\1&2&0\\1&3&0\end{vmatrix}=100.
$$



$$
在n\mathrm{阶行列式}D(D\neq0)中,\mathrm{如果把第一列移到最后},\mathrm{而其余各列保持原来次序向左移动},\mathrm{得行列式}D_1,\mathrm{行列式}D_1与D\mathrm{关系为}(\;\;\;).
$$
$$
A.
D_1=\left(-1\right)^{n-1}D \quad B.D_1=\left(-1\right)^nD \quad C.D_1=-D \quad D.D_1=D \quad E. \quad F. \quad G. \quad H.
$$
$$
\mathrm{把行列式}D\mathrm{的第一列移到最后},\mathrm{则第一列分别与第}2,3,…,n\mathrm{列都进行了交换},\mathrm{即交换了}n-1次,故D\;_1\;=\;\left(-1\right)\;^{n-1}\;D.
$$



$$
\mathrm{三阶行列式}\begin{vmatrix}ax&ay&az\\ay&az&ax\\az&ax&ay\end{vmatrix}=().(a\neq0)
$$
$$
A.
a^3\begin{vmatrix}x&y&z\\y&z&x\\z&x&y\end{vmatrix} \quad B.a\begin{vmatrix}x&y&z\\y&z&x\\z&x&y\end{vmatrix} \quad C.\begin{vmatrix}x&y&z\\y&z&x\\z&x&y\end{vmatrix} \quad D.a^2\begin{vmatrix}x&y&z\\y&z&x\\z&x&y\end{vmatrix} \quad E. \quad F. \quad G. \quad H.
$$
$$
\begin{vmatrix}ax&ay&az\\ay&az&ax\\az&ax&ay\end{vmatrix}=a^3\begin{vmatrix}x&y&z\\y&z&x\\z&x&y\end{vmatrix}.
$$



$$
\mathrm{三阶行列式}\begin{vmatrix}1&2&3\\10&20&30\\11&21&31\end{vmatrix}=().
$$
$$
A.
6 \quad B.60 \quad C.160 \quad D.0 \quad E. \quad F. \quad G. \quad H.
$$
$$
\mathrm{第一行},\mathrm{第二行成比例},\begin{vmatrix}1&2&3\\10&20&30\\11&21&31\end{vmatrix}=0.
$$



$$
\mathrm{四阶行列式}\begin{vmatrix}1&2&3&4\\5&6&7&8\\0&0&0&0\\2&0&1&8\end{vmatrix}=().
$$
$$
A.
1 \quad B.2 \quad C.0 \quad D.-1 \quad E. \quad F. \quad G. \quad H.
$$
$$
\mathrm{第三行为}0,\begin{vmatrix}1&2&3&4\\5&6&7&8\\0&0&0&0\\2&0&1&8\end{vmatrix}=0.
$$



$$
\mathrm{四阶行列式}\begin{vmatrix}1&2&2&1\\2&3&4&1\\3&4&6&1\\4&5&8&1\end{vmatrix}=().
$$
$$
A.
0 \quad B.1 \quad C.2 \quad D.3 \quad E. \quad F. \quad G. \quad H.
$$
$$
\mathrm{第一列与第三列对应成比例},\mathrm{所以}\begin{vmatrix}1&2&2&1\\2&3&4&1\\3&4&6&1\\4&5&8&1\end{vmatrix}=0.
$$



$$
\mathrm{设行列式}\begin{vmatrix}x&y&z\\4&0&3\\1&1&1\end{vmatrix}=6,\mathrm{则行列式}\;\begin{vmatrix}2x&2y&2z\\\frac43&0&1\\1&1&1\end{vmatrix}=().
$$
$$
A.
1 \quad B.2 \quad C.3 \quad D.4 \quad E. \quad F. \quad G. \quad H.
$$
$$
\;\begin{vmatrix}2x&2y&2z\\\frac43&0&1\\1&1&1\end{vmatrix}=\frac23\begin{vmatrix}x&y&z\\4&0&3\\1&1&1\end{vmatrix}=\frac23×6=4.
$$



$$
\mathrm{设行列式}\begin{vmatrix}x&y&z\\3&0&2\\1&1&1\end{vmatrix}=2,则\begin{vmatrix}x&y&z\\3&0&2\\4&1&3\end{vmatrix}=().
$$
$$
A.
1 \quad B.2 \quad C.3 \quad D.4 \quad E. \quad F. \quad G. \quad H.
$$
$$
\begin{vmatrix}x&y&z\\3&0&2\\4&1&3\end{vmatrix}\overset{r_3-r_2}=\begin{vmatrix}x&y&z\\3&0&2\\1&1&1\end{vmatrix}=2.
$$



$$
若\begin{vmatrix}a_{11}&a_{12}\\a_{21}&a_{22}\end{vmatrix}=m\neq0,则\begin{vmatrix}2a_{11}&2a_{12}\\2a_{21}&2a_{22}\end{vmatrix}=().
$$
$$
A.
m \quad B.2m \quad C.3m \quad D.4m \quad E. \quad F. \quad G. \quad H.
$$
$$
\begin{vmatrix}2a_{11}&2a_{12}\\2a_{21}&2a_{22}\end{vmatrix}=4\begin{vmatrix}a_{11}&a_{12}\\a_{21}&a_{22}\end{vmatrix}=4m.
$$



$$
若\begin{vmatrix}a_{11}&a_{12}\\a_{21}&a_{22}\end{vmatrix}=m\neq0,则\begin{vmatrix}4a_{11}&4a_{12}\\3a_{21}&3a_{22}\end{vmatrix}=().
$$
$$
A.
m \quad B.3m \quad C.4m \quad D.12m \quad E. \quad F. \quad G. \quad H.
$$
$$
\begin{vmatrix}4a_{11}&4a_{12}\\3a_{21}&3a_{22}\end{vmatrix}=12\begin{vmatrix}a_{11}&a_{12}\\a_{21}&a_{22}\end{vmatrix}=12m.
$$



$$
\mathrm{三阶行列式}\begin{vmatrix}1&0&100\\3&2&200\\0&2&300\end{vmatrix}=().
$$
$$
A.
800 \quad B.200 \quad C.300 \quad D.400 \quad E. \quad F. \quad G. \quad H.
$$
$$
\begin{vmatrix}1&0&100\\3&2&200\\0&2&300\end{vmatrix}=100\begin{vmatrix}1&0&1\\3&2&2\\0&2&3\end{vmatrix}=800.
$$



$$
若\begin{vmatrix}a_{11}&a_{12}\\a_{21}&a_{22}\end{vmatrix}=m\neq0,则\begin{vmatrix}a_{11}&a_{12}\\a_{21}+a_{11}&a_{22}+a_{12}\end{vmatrix}=().
$$
$$
A.
m \quad B.2m \quad C.0 \quad D.4m \quad E. \quad F. \quad G. \quad H.
$$
$$
\begin{vmatrix}a_{11}&a_{12}\\a_{21}&a_{22}\end{vmatrix}\overset{r_{2+}r_1}=\begin{vmatrix}a_{11}&a_{12}\\a_{21}+a_{11}&a_{22}+a_{12}\end{vmatrix}=m.
$$



$$
若\begin{vmatrix}a_{11}&a_{12}\\a_{21}&a_{22}\end{vmatrix}=n\neq0,则\begin{vmatrix}a_{11}&a_{12}-2a_{11}\\a_{21}&a_{22}-2a_{21}\end{vmatrix}=().
$$
$$
A.
0 \quad B.n \quad C.-n \quad D.-2n \quad E. \quad F. \quad G. \quad H.
$$
$$
\begin{vmatrix}a_{11}&a_{12}\\a_{21}&a_{22}\end{vmatrix}\overset{c_2-2c_1}=\begin{vmatrix}a_{11}&a_{12}-2a_{11}\\a_{21}&a_{22}-2a_{21}\end{vmatrix}=n.
$$



$$
若\begin{vmatrix}1&2&3\\4&5&6\\7&8&9\end{vmatrix}=a,则\begin{vmatrix}1&4&7\\2&5&8\\3&6&9\end{vmatrix}=().
$$
$$
A.
0 \quad B.a \quad C.-a \quad D.1 \quad E. \quad F. \quad G. \quad H.
$$
$$
\mathrm{转置}:\begin{vmatrix}1&2&3\\4&5&6\\7&8&9\end{vmatrix}=\begin{vmatrix}1&4&7\\2&5&8\\3&6&9\end{vmatrix}=a.
$$



$$
若\begin{vmatrix}1&2&3\\3&2&1\\1&1&2\end{vmatrix}=b,则\begin{vmatrix}1&1&2\\3&2&1\\1&2&3\end{vmatrix}=().
$$
$$
A.
0 \quad B.1 \quad C.b \quad D.-b \quad E. \quad F. \quad G. \quad H.
$$
$$
\mathrm{行列式互换两行},\mathrm{变号}.
$$



$$
设D_1=\begin{vmatrix}3a_1&0&…&0\\0&3a_2&…&0\\…&…&…&…\\0&0&0&3a_n\end{vmatrix},D_2=\begin{vmatrix}a_1&0&…&0\\0&a_2&…&0\\…&…&…&…\\0&0&0&a_n\end{vmatrix},\mathrm{其中}a_1… a_n\neq0则().
$$
$$
A.
D_1=D_2 \quad B.D_1=-3^nD_2 \quad C.D_1=3^nD_2 \quad D.D_1={\textstyle\frac1{3^n}}D_2 \quad E. \quad F. \quad G. \quad H.
$$
$$
D_1=\begin{vmatrix}3a_1&0&…&0\\0&3a_2&…&0\\…&…&…&…\\0&0&0&3a_n\end{vmatrix}=3^n\begin{vmatrix}a_1&0&…&0\\0&a_2&…&0\\…&…&…&…\\0&0&0&a_n\end{vmatrix}=3^nD_2.
$$



$$
\mathrm{若行列式}D=\begin{vmatrix}a_{11}&a_{12}&a_{13}\\a_{21}&a_{22}&a_{23}\\a_{31}&a_{32}&a_{33}\end{vmatrix}=M(M\neq0),则D_1=\begin{vmatrix}3a_{11}&4a_{11}-a_{12}&a_{13}\\3a_{21}&4a_{21}-a_{22}&a_{23}\\3a_{31}&4a_{31}-a_{32}&a_{33}\end{vmatrix}=().
$$
$$
A.
-3M \quad B.3M \quad C.12M \quad D.-12M \quad E. \quad F. \quad G. \quad H.
$$
$$
D_1=\begin{vmatrix}3a_{11}&4a_{11}-a_{12}&a_{13}\\3a_{21}&4a_{21}-a_{22}&a_{23}\\3a_{31}&4a_{31}-a_{32}&a_{33}\end{vmatrix}=\begin{vmatrix}3a_{11}&4a_{11}&a_{13}\\3a_{21}&4a_{21}&a_{23}\\3a_{31}&4a_{31}&a_{33}\end{vmatrix}-\begin{vmatrix}3a_{11}&a_{12}&a_{13}\\3a_{21}&a_{22}&a_{23}\\3a_{31}&a_{32}&a_{33}\end{vmatrix}=0-3M=-3M.
$$



$$
\mathrm{若行列式}\begin{vmatrix}a_{11}&a_{12}&a_{13}\\a_{21}&a_{22}&a_{23}\\a_{31}&a_{32}&a_{33}\end{vmatrix}=10,则\begin{vmatrix}3a_{11}&-a_{12}&-5a_{13}\\-3a_{21}&a_{22}&5a_{23}\\-3a_{31}&a_{32}&5a_{33}\end{vmatrix}=().
$$
$$
A.
140 \quad B.150 \quad C.160 \quad D.170 \quad E. \quad F. \quad G. \quad H.
$$
$$
\begin{array}{l}\mathrm{利用行列式性质},有\\\begin{vmatrix}3a_{11}&-a_{12}&-5a_{13}\\-3a_{21}&a_{22}&5a_{23}\\-3a_{31}&a_{32}&5a_{33}\end{vmatrix}=-\;\begin{vmatrix}-3a_{11}&a_{12}&5a_{13}\\-3a_{21}&a_{22}&5a_{23}\\-3a_{31}&a_{32}&5a_{33}\end{vmatrix}=3×5\begin{vmatrix}a_{11}&a_{12}&a_{13}\\a_{21}&a_{22}&a_{23}\\a_{31}&a_{32}&a_{33}\end{vmatrix}=15×10=150.\end{array}
$$



$$
\mathrm{三阶行列式}D_1=\begin{vmatrix}a_{11}&a_{12}&a_{13}\\a_{21}&a_{22}&a_{23}\\a_{31}&a_{32}&a_{33}\end{vmatrix},D_2=\begin{vmatrix}a_{12}&a_{32}&a_{22}\\a_{11}&a_{31}&a_{21}\\a_{13}&a_{33}&a_{23}\end{vmatrix},则(\;\;\;\;).
$$
$$
A.
D_1=D_2 \quad B.D_1=-D_2 \quad C.D_1=2D_2 \quad D.D_1=\frac12D_2 \quad E. \quad F. \quad G. \quad H.
$$
$$
D_1\mathrm{互换第二行与第三行},\mathrm{转置},\mathrm{最后互换第二行与第一行},\mathrm{得到}D_2,∴ D_1=D_2.
$$



$$
\mathrm{下列关于}x\mathrm{的方程}D\left(x\right)=\begin{vmatrix}1&1&2&3\\1&2-x^2&2&3\\2&3&1&5\\2&3&1&9-x^2\end{vmatrix}=0\mathrm{的解},\mathrm{不正确的是}(\;\;\;).
$$
$$
A.
±1 \quad B.2 \quad C.-2 \quad D.±4 \quad E. \quad F. \quad G. \quad H.
$$
$$
∵ D\left(±1\right)=0,D\left(±2\right)=0,\;而D\left(x\right)\mathrm{的的最高次方为}4.故D\left(x\right)=0\mathrm{解为}x_1=1,x_2=-1,x_3=2,x_4=-2.
$$



$$
D\left(λ\right)=\begin{vmatrix}1&1&1&1\\2&λ&4&5\\4&λ^2&16&25\\8&λ^3&64&125\end{vmatrix},\mathrm{不能使}D\left(λ\right)=0\mathrm{的是}(\;\;\;\;).\;
$$
$$
A.
1 \quad B.2 \quad C.4 \quad D.5 \quad E. \quad F. \quad G. \quad H.
$$
$$
D\left(2\right)=0,D\left(4\right)=0,\;\;D\left(5\right)=0,\;\;D\left(1\right)\neq0.\;\;\mathrm{或者利用行列式为范德蒙德行列式}.
$$



$$
\mathrm{若三阶行列式}\begin{vmatrix}a_1&a_2&a_3\\2b_1-a_1&2b_2-a_2&2b_3-a_3\\c_1&c_2&c_3\end{vmatrix}=6,则\begin{vmatrix}a_1&a_2&a_3\\b_1&b_2&b_3\\c_1&c_2&c_3\end{vmatrix}=().
$$
$$
A.
3 \quad B.2 \quad C.-1 \quad D.8 \quad E. \quad F. \quad G. \quad H.
$$
$$
\begin{vmatrix}a_1&a_2&a_3\\2b_1-a_1&2b_2-a_2&2b_3-a_3\\c_1&c_2&c_3\end{vmatrix}=\begin{vmatrix}a_1&a_2&a_3\\2b_1&2b_2&2b_3\\c_1&c_2&c_3\end{vmatrix}+\begin{vmatrix}a_1&a_2&a_3\\-a_1&-a_2&-a_3\\c_1&c_2&c_3\end{vmatrix}\;=2\begin{vmatrix}a_1&a_2&a_3\\b_1&b_2&b_3\\c_1&c_2&c_3\end{vmatrix}\;=6.\mathrm{所以}\begin{vmatrix}a_1&a_2&a_3\\b_1&b_2&b_3\\c_1&c_2&c_3\end{vmatrix}\;=3.
$$



$$
\mathrm{若行列式}D=\begin{vmatrix}a_{11}&a_{12}&a_{13}\\a_{21}&a_{22}&a_{23}\\a_{31}&a_{32}&a_{33}\end{vmatrix}=M\neq0,则D_1=\begin{vmatrix}3a_{11}&4a_{11}-a_{12}&-a_{13}\\3a_{21}&4a_{21}-a_{22}&-a_{23}\\3a_{31}&4a_{31}-a_{32}&-a_{33}\end{vmatrix}=().
$$
$$
A.
-3M \quad B.3M \quad C.12M \quad D.-12M \quad E. \quad F. \quad G. \quad H.
$$
$$
\;D_1=\begin{vmatrix}3a_{11}&4a_{11}-a_{12}&\;\;-a_{13}\\3a_{21}&4a_{21}-a_{22}&\;\;-a_{23}\\3a_{31}&4a_{31}-a_{32}&\;\;-a_{33}\end{vmatrix}=\begin{vmatrix}3a_{11}&4a_{11}&-a_{13}\\3a_{21}&4a_{21}&-a_{23}\\3a_{31}&4a_{31}&\;\;-a_{33}\end{vmatrix}-\begin{vmatrix}3a_{11}&a_{12}&-a_{13}\\3a_{21}&a_{22}&-a_{23}\\3a_{31}&a_{32}&\;\;-a_{33}\end{vmatrix}=0-\left(-3M\right)=3M.
$$



$$
设D=\begin{vmatrix}a_{11}&a_{12}&a_{13}\\a_{21}&a_{22}&a_{23}\\a_{31}&a_{32}&a_{33}\end{vmatrix}\neq0,D_1=\begin{vmatrix}2a_{11}&3a_{11}-a_{12}&-a_{13}\\2a_{21}&3a_{21}-a_{22}&-a_{23}\\2a_{31}&3a_{31}-a_{32}&-a_{33}\end{vmatrix},则D_1=().
$$
$$
A.
-2D \quad B.0 \quad C.2D \quad D.6D \quad E. \quad F. \quad G. \quad H.
$$
$$
D_1=\begin{vmatrix}2a_{11}&3a_{11}-a_{12}&-a_{13}\\2a_{21}&3a_{21}-a_{22}&-a_{23}\\2a_{31}&3a_{31}-a_{32}&-a_{33}\end{vmatrix}=\begin{vmatrix}2a_{11}&3a_{11}&-a_{13}\\2a_{21}&3a_{21}&-a_{23}\\2a_{31}&3a_{31}&-a_{33}\end{vmatrix}+\begin{vmatrix}2a_{11}&-a_{12}&-a_{13}\\2a_{21}&-a_{22}&-a_{23}\\2a_{31}&-a_{32}&-a_{33}\end{vmatrix}=2\begin{vmatrix}a_{11}&a_{12}&a_{13}\\a_{21}&a_{22}&a_{23}\\a_{31}&a_{32}&a_{33}\end{vmatrix}=2D.
$$



$$
设D=\begin{vmatrix}a_{11}&a_{12}&a_{13}\\a_{21}&a_{22}&a_{23}\\a_{31}&a_{32}&a_{33}\end{vmatrix}=2,D_1=\begin{vmatrix}3a_{11}&4a_{11}+a_{12}&a_{13}\\3a_{21}&4a_{21}+a_{22}&a_{23}\\3a_{31}&4a_{31}+a_{32}&a_{33}\end{vmatrix},则D_1=\left(\;\;\;\right).
$$
$$
A.
-4 \quad B.0 \quad C.4 \quad D.6 \quad E. \quad F. \quad G. \quad H.
$$
$$
\begin{array}{l}D_1=\begin{vmatrix}2a_{11}&4a_{11}+a_{12}&a_{13}\\2a_{21}&4a_{21}+a_{22}&a_{23}\\2a_{31}&4a_{31}+a_{32}&a_{33}\end{vmatrix}\;=\;\begin{vmatrix}3a_{11}&4a_{11}&a_{13}\\3a_{21}&4a_{21}&a_{23}\\3a_{31}&4a_{31}&a_{33}\end{vmatrix}+\begin{vmatrix}3a_{11}&a_{12}&a_{13}\\3a_{21}&a_{22}&a_{23}\\3a_{31}&a_{32}&a_{33}\end{vmatrix}\\=3\begin{vmatrix}a_{11}&a_{12}&a_{13}\\a_{21}&a_{22}&a_{23}\\a_{31}&a_{32}&a_{33}\end{vmatrix}=3D=6.\end{array}
$$



$$
设D=\begin{vmatrix}a_{11}&a_{12}&a_{13}\\a_{21}&a_{22}&a_{23}\\a_{31}&a_{32}&a_{33}\end{vmatrix}=2,D_1=\begin{vmatrix}2a_{11}&2a_{11}-3a_{12}&3a_{13}\\2a_{21}&2a_{21}-3a_{22}&3a_{23}\\2a_{31}&2a_{31}-3a_{32}&3a_{33}\end{vmatrix},\;则D_1=\left(\;\;\;\right).
$$
$$
A.
-12 \quad B.24 \quad C.-36 \quad D.48 \quad E. \quad F. \quad G. \quad H.
$$
$$
\begin{array}{l}D_1=\begin{vmatrix}2a_{11}&2a_{11}-3a_{12}&3a_{13}\\2a_{21}&2a_{21}-3a_{22}&3a_{23}\\2a_{31}&2a_{31}-3a_{32}&3a_{33}\end{vmatrix}=\begin{vmatrix}2a_{11}&2a_{11}&3a_{13}\\2a_{21}&2a_{21}&3a_{23}\\2a_{31}&2a_{31}&3a_{33}\end{vmatrix}-\begin{vmatrix}2a_{11}&3a_{12}&3a_{13}\\2a_{21}&3a_{22}&3a_{23}\\2a_{31}&3a_{32}&3a_{33}\end{vmatrix}\\=-2×3×3\begin{vmatrix}a_{11}&a_{12}&a_{13}\\a_{21}&a_{22}&a_{23}\\a_{31}&a_{32}&a_{33}\end{vmatrix}=-18×2=-36.\end{array}
$$



$$
\mathrm{若行列式}\begin{vmatrix}a_{11}&a_{12}&a_{13}\\a_{21}&a_{22}&a_{23}\\a_{31}&a_{32}&a_{33}\end{vmatrix}=1,则\begin{vmatrix}5a_{11}&4a_{11}-a_{12}&a_{13}\\5a_{21}&4a_{21}-a_{22}&a_{23}\\5a_{31}&4a_{31}-a_{32}&a_{33}\end{vmatrix}=\left(\;\;\;\;\;\;\right).
$$
$$
A.
5 \quad B.20 \quad C.-5 \quad D.-20 \quad E. \quad F. \quad G. \quad H.
$$
$$
\begin{vmatrix}5a_{11}&4a_{11}-a_{12}&a_{13}\\5a_{21}&4a_{21}-a_{22}&a_{23}\\5a_{31}&4a_{31}-a_{32}&a_{33}\end{vmatrix}=\begin{vmatrix}5a_{11}&4a_{11}&a_{13}\\5a_{21}&4a_{21}&a_{23}\\5a_{31}&4a_{31}&a_{33}\end{vmatrix}-\begin{vmatrix}5a_{11}&a_{12}&a_{13}\\5a_{21}&a_{22}&a_{23}\\5a_{31}&a_{32}&a_{33}\end{vmatrix}=0-5=-5.
$$



$$
\mathrm{若行列式}\begin{vmatrix}a_{11}&a_{12}&a_{13}\\a_{21}&a_{22}&a_{23}\\a_{31}&a_{32}&a_{33}\end{vmatrix}=1,则\begin{vmatrix}6a_{11}&-2a_{12}&-10a_{13}\\-3a_{21}&a_{22}&5a_{23}\\-3a_{31}&a_{32}&5a_{33}\end{vmatrix}=\left(\;\;\;\;\;\right).
$$
$$
A.
30 \quad B.-30 \quad C.-10 \quad D.10 \quad E. \quad F. \quad G. \quad H.
$$
$$
\begin{array}{l}\mathrm{利用行列式性质},有\\\begin{vmatrix}6a_{11}&-2a_{12}&-10a_{13}\\-3a_{21}&a_{22}&5a_{23}\\-3a_{31}&a_{32}&5a_{33}\end{vmatrix}=-2\begin{vmatrix}-3a_{11}&a_{12}&5a_{13}\\-3a_{21}&a_{22}&5a_{23}\\-3a_{31}&a_{32}&5a_{33}\end{vmatrix}\\\;\;\;\;\;\;\;\;\;\;\;\;\;\;\;\;\;\;\;\;\;\;\;\;\;\;\;\;\;\;\;\;\;\;\;\;\;\;\;\;=-2·\left(-3\right)·5\begin{vmatrix}a_{11}&a_{12}&a_{13}\\a_{21}&a_{22}&a_{23}\\a_{31}&a_{32}&a_{33}\end{vmatrix}=-2·\left(-3\right)·5·1=30.\end{array}
$$



$$
\mathrm{三阶行列式}\begin{vmatrix}-ab&ac&ae\\bd&-cd&de\\bf&cf&-ef\end{vmatrix}=\left(\;\;\;\;\;\;\right).
$$
$$
A.
abcdef \quad B.2abcdef \quad C.3abcdef \quad D.4abcdef \quad E. \quad F. \quad G. \quad H.
$$
$$
\begin{array}{l}\begin{vmatrix}-ab&ac&ae\\bd&-cd&de\\bf&cf&-ef\end{vmatrix}=adf\begin{vmatrix}-b&c&e\\b&-c&e\\b&c&-e\end{vmatrix}=adfbce\begin{vmatrix}-1&1&1\\1&-1&1\\1&1&-1\end{vmatrix}=4adfbce.\\\;\;\;\;\;\;\;\;\;\;\;\;\;\;\;\;\;\;\;\;\;\;\;\;\;\;\;\;\;\;\;\;\;\;\;\;\;\;\;\;\;\end{array}
$$



$$
\begin{array}{l}\mathrm{设行列式}\begin{vmatrix}a_{ij}\end{vmatrix}=m\left(i,j=1,2,3⋯5,\right),\mathrm{依下列次序对}\begin{vmatrix}a_{ij}\end{vmatrix}\mathrm{进行变换}:\;\mathrm{交换第一行与第五行},\mathrm{再转置},用2\mathrm{乘所有元素},\\\mathrm{在用}\left(-3\right)\mathrm{乘以第二列加到第四列},\mathrm{最后用}4\mathrm{除以第二行各元素},\mathrm{则变换后的结果为}(\;\;\;).\end{array}
$$
$$
A.
-8m \quad B.8m \quad C.-32m \quad D.32m \quad E. \quad F. \quad G. \quad H.
$$
$$
\begin{array}{l}由\begin{vmatrix}a_{ij}\end{vmatrix}=m,\mathrm{交换第一行与第五行得}-m,\;\\\mathrm{再转置得}-m,用2\mathrm{乘所有元素得}-32m,\;\\\mathrm{再用}(-3)\mathrm{乘第二列加到第四列得}-32m,\;\\\mathrm{最后用}4\mathrm{除以第二行各元素得}\;-8m.\end{array}
$$



$$
\mathrm{行列式}D_n为0\mathrm{的充分条件是}\left(\;\;\;\;\;\right).
$$
$$
A.
\mathrm{零元素的个数大于}n \quad B.D_n\mathrm{中各行元素之和为零} \quad C.\mathrm{主对角线上元素为零} \quad D.\mathrm{次对角线上元素全为零} \quad E. \quad F. \quad G. \quad H.
$$
$$
\begin{array}{l}\mathrm{由于行列式的某一行}(列)\mathrm{都加到另一行对应位置的元素上},\mathrm{行列式的性质不变},\\\mathrm{可知若}D_n\mathrm{中各行元素之和为零},\mathrm{可将各列都加到第一列},\mathrm{得到第一列全为零},\\\mathrm{由行列式的定义可知},\mathrm{行列式为零}.\end{array}
$$



$$
\mathrm{四阶行列式}\begin{vmatrix}2&1&4&-1\\3&-1&2&-1\\1&2&3&-2\\5&0&6&-2\end{vmatrix}=\left(\;\;\;\;\;\right).
$$
$$
A.
0 \quad B.1 \quad C.-1 \quad D.2 \quad E. \quad F. \quad G. \quad H.
$$
$$
\mathrm{原式}\overset{r_2+r_1}=\begin{vmatrix}2&1&4&-1\\5&0&6&-2\\1&2&3&-2\\5&0&6&-2\end{vmatrix}=0.
$$



$$
\mathrm{已知行列式}D=\begin{vmatrix}x&y&z\\x_1&y_1&z_1\\x_2&y_2&z_2\end{vmatrix}=a,\mathrm 则\begin{vmatrix}y+z&z+x&x+y\\y_1+z_1&z_1+x_1&x_1+y_1\\y_2+z_2&z_2+x_2&x_2+y_2\end{vmatrix}=(\;\;\;).
$$
$$
A.
0 \quad B.a \quad C.2a \quad D.-2a \quad E. \quad F. \quad G. \quad H.
$$
$$
\begin{array}{l}\begin{vmatrix}y+z&z+x&x+y\\y_1+z_1&z_1+x_1&x_1+y_1\\y_2+z_2&z_2+x_2&x_2+y_2\end{vmatrix}=\begin{vmatrix}y&z+x&x+y\\y_1&z_1+x_1&x_1+y_1\\y_2&z_2+x_2&x_2+y_2\end{vmatrix}+\begin{vmatrix}z&z+x&x+y\\z_1&z_1+x_1&x_1+y_1\\z_2&z_2+x_2&x_2+y_2\end{vmatrix}\\=\;(\;\begin{vmatrix}y&z&x+y\\y_1&z_1&x_1+y_1\\y_2&z_2&x_2+y_2\end{vmatrix}+\begin{vmatrix}y&x&x+y\\y_1&x_1&x_1+y_1\\y_2&x_2&x_2+y_2\end{vmatrix}\;)\;\;+(\;\begin{vmatrix}z&z&x+y\\z_1&z_1&x_1+y_1\\z_2&z_2&x_2+y_2\end{vmatrix}+\begin{vmatrix}z&x&x+y\\z_1&x_1&x_1+y_1\\z_2&x_2&x_2+y_2\end{vmatrix}\;)\\=\begin{vmatrix}y&z&x\\y_1&z_1&x_1\\y_2&z_2&x_2\end{vmatrix}+\begin{vmatrix}z&x&y\\z_1&x_1&y_1\\z_2&x_2&y_2\end{vmatrix}\\=\begin{vmatrix}x&y&z\\x_1&y_1&z_1\\x_2&y_2&z_2\end{vmatrix}+\begin{vmatrix}x&y&z\\x_1&y_1&z_1\\x_2&y_2&z_2\end{vmatrix}=2a.\end{array}
$$



$$
\mathrm{三阶行列式}\begin{vmatrix}a_1+kb_1&b_1+c_1&c_1\\a_2+kb_2&b_2+c_2&c_2\\a_3+kb_3&b_3+c_3&c_3\end{vmatrix}=(\;\;\;).
$$
$$
A.
0 \quad B.\begin{vmatrix}a_1&b_1&c_1\\a_2&b_2&c_2\\a_3&b_3&c_3\end{vmatrix} \quad C.k\begin{vmatrix}a_1&b_1&c_1\\a_2&b_2&c_2\\a_3&b_3&c_3\end{vmatrix} \quad D.k \quad E. \quad F. \quad G. \quad H.
$$
$$
\begin{array}{l}\begin{vmatrix}a_1+kb_1&b_1+c_1&c_1\\a_2+kb_2&b_2+c_2&c_2\\a_3+kb_3&b_3+c_3&c_3\end{vmatrix}=\begin{vmatrix}a_1+kb_1&b_1&c_1\\a_2+kb_2&b_2&c_2\\a_3+kb_3&b_3&c_3\end{vmatrix}+\begin{vmatrix}a_1+kb_1&c_1&c_1\\a_2+kb_2&c_2&c_2\\a_3+kb_3&c_3&c_3\end{vmatrix}\\\;\;\;\;\;\;\;\;\;\;\;\;\;\;\;\;\;\;\;\;\;\;\;\;\;\;\;\;\;\;\;\;\;\;\;=\begin{vmatrix}a_1+kb_1&b_1&c_1\\a_2+kb_2&b_2&c_2\\a_3+kb_3&b_3&c_3\end{vmatrix}=\begin{vmatrix}a_1&b_1&c_1\\a_2&b_2&c_2\\a_3&b_3&c_3\end{vmatrix}+k\begin{vmatrix}b_1&b_1&c_1\\b_2&b_2&c_2\\b_3&b_3&c_3\end{vmatrix}=\begin{vmatrix}a_1&b_1&c_1\\a_2&b_2&c_2\\a_3&b_3&c_3\end{vmatrix}.\end{array}
$$



$$
\mathrm{四阶行列式}\begin{vmatrix}c&a&d&b\\a&c&d&b\\a&c&b&d\\c&a&b&d\end{vmatrix}=(\;\;\;).
$$
$$
A.
acbd \quad B.0 \quad C.1 \quad D.4abcd \quad E. \quad F. \quad G. \quad H.
$$
$$
\mathrm{原式}=\begin{vmatrix}c-a&a-c&0&0\\a&c&d&b\\a&c&b&d\\c-a&a-c&0&0\end{vmatrix}=0.
$$



$$
\mathrm{四阶行列式}\begin{vmatrix}1&1&1&1\\a&b&c&d\\b&c&d&a\\c+d&a+d&a+b&b+c\end{vmatrix}=(\;\;).
$$
$$
A.
1 \quad B.0 \quad C.abcd \quad D.a+b+c+d \quad E. \quad F. \quad G. \quad H.
$$
$$
\mathrm{把第二行},\mathrm{第三行都加到第四行上去},\mathrm{有左边}=(a+b+c+d)\begin{vmatrix}1&1&1&1\\a&b&c&d\\b&c&d&a\\1&1&1&1\end{vmatrix}=0.
$$



$$
\mathrm{三阶行列式}\begin{vmatrix}a_1+ka_2+la_3&a_2+ma_3&a_3\\b_1+kb_2+lb_3&b_2+mb_3&b_3\\c_1+kc_2+lc_3&c_2+mc_3&c_3\end{vmatrix}=(\;\;\;).
$$
$$
A.
\begin{vmatrix}a_1&a_2&a_3\\b_1&b_2&b_3\\c_1&c_2&c_3\end{vmatrix} \quad B.0 \quad C.klm\begin{vmatrix}a_1&a_2&a_3\\b_1&b_2&b_3\\c_1&c_2&c_3\end{vmatrix} \quad D.k^3l^3m^3\begin{vmatrix}a_1&a_2&a_3\\b_1&b_2&b_3\\c_1&c_2&c_3\end{vmatrix} \quad E. \quad F. \quad G. \quad H.
$$
$$
\begin{array}{l}-m×\mathrm{第三列加到第二列},-l×\mathrm{第三列加到第一列};\mathrm{然后}-k×\mathrm{第二列加到第一列}.\\\begin{vmatrix}a_1+ka_2+1a_3&a_2+ma_3&a_3\\b_1+kb_2+1b_3&b_2+mb_3&b_3\\c_1+kc_2+1c_3&c_2+mc_3&c_3\end{vmatrix}\;=\begin{vmatrix}a_1&a_2&a_3\\b_1&b_2&b_3\\c_1&c_2&c_3\end{vmatrix}.\end{array}
$$



$$
\mathrm{已知}231,105,182\mathrm{能被}7\mathrm{整除},\mathrm{不用计算行列式的值},\mathrm{下列行列式不能被}7\mathrm{整除的是}(\;\;\;\;).
$$
$$
A.
\begin{vmatrix}2&3&1\\1&0&5\\1&8&2\end{vmatrix} \quad B.\begin{vmatrix}2&1&1\\3&0&8\\1&5&2\end{vmatrix} \quad C.\begin{vmatrix}2&1&3\\1&5&0\\1&2&8\end{vmatrix} \quad D.\begin{vmatrix}2&3&1\\1&1&0\\1&8&2\end{vmatrix} \quad E. \quad F. \quad G. \quad H.
$$
$$
\begin{array}{l}\begin{vmatrix}2&3&1\\1&0&5\\1&8&2\end{vmatrix}=\begin{vmatrix}2&3&231\\1&0&105\\1&8&182\end{vmatrix}∴\mathrm{能被整除}\\\begin{vmatrix}2&1&1\\3&0&8\\1&5&2\end{vmatrix}=\begin{vmatrix}2&1&1\\3&0&8\\231&105&182\end{vmatrix}∴\mathrm{能被整除}\\\begin{vmatrix}2&1&3\\1&5&0\\1&2&8\end{vmatrix}=\begin{vmatrix}2&231&3\\1&105&0\\1&182&8\end{vmatrix}∴\mathrm{能被整除}.\end{array}
$$



$$
设f(x)=\begin{vmatrix}a_{11}+x&a_{12}+x&a_{13}+x\\a_{21}+x&a_{22}+x&a_{23}+x\\a_{31}+x&a_{32}+x&a_{33}+x\end{vmatrix},则f(x)\mathrm{展开式中}x\mathrm{的最高次数是}(\;\;).
$$
$$
A.
0 \quad B.1 \quad C.2 \quad D.3 \quad E. \quad F. \quad G. \quad H.
$$
$$
\begin{array}{l}\begin{vmatrix}a_{11}+x&a_{12}+x&a_{13}+x\\a_{21}+x&a_{22}+x&a_{23}+x\\a_{31}+x&a_{32}+x&a_{33}+x\end{vmatrix}\overset{c_2-c_1}{\underset{c_3-c_1}=}\begin{vmatrix}a_{11}+x&a_{12}-a_{11}&a_{13}-a_{11}\\a_{21}+x&a_{22}-a_{21}&a_{23}-a_{21}\\a_{31}+x&a_{32}-a_{31}&a_{33}-a_{31}\end{vmatrix}\\=\begin{vmatrix}a_{11}&a_{12}-a_{11}&a_{13}-a_{11}\\a_{21}&a_{22}-a_{21}&a_{23}-a_{21}\\a_{31}&a_{32}-a_{31}&a_{33}-a_{31}\end{vmatrix}+\begin{vmatrix}x&a_{12}-a_{11}&a_{13}-a_{11}\\x&a_{22}-a_{21}&a_{23}-a_{21}\\x&a_{32}-a_{31}&a_{33}-a_{31}\end{vmatrix}\\=\begin{vmatrix}a_{11}&a_{12}-a_{11}&a_{13}-a_{11}\\a_{21}&a_{22}-a_{21}&a_{23}-a_{21}\\a_{31}&a_{32}-a_{31}&a_{33}-a_{31}\end{vmatrix}+x\begin{vmatrix}1&a_{12}-a_{11}&a_{13}-a_{11}\\1&a_{22}-a_{21}&a_{23}-a_{21}\\1&a_{32}-a_{31}&a_{33}-a_{31}\end{vmatrix}.\end{array}
$$



$$
\mathrm{四阶行列式}\begin{vmatrix}a&b&c&1\\b&c&a&1\\c&a&b&1\\a&c&b&1\end{vmatrix}=(\;\;\;).
$$
$$
A.
0 \quad B.1 \quad C.-1 \quad D.2 \quad E. \quad F. \quad G. \quad H.
$$
$$
\begin{vmatrix}a+b+c+1&b&c&1\\a+b+c+1&c&a&1\\a+b+c+1&a&b&1\\a+b+c+1&c&b&1\end{vmatrix}=\left(a+b+c+1\right)\begin{vmatrix}1&b&c&1\\1&c&a&1\\1&a&b&1\\1&c&b&1\end{vmatrix}=0.
$$



$$
\mathrm{四阶行列式}\begin{vmatrix}4&4&5&5\\2&2&4&4\\2&2&0&3\\0&0&0&2\end{vmatrix}=(\;\;\;\;\;).\;
$$
$$
A.
1 \quad B.2 \quad C.3 \quad D.0 \quad E. \quad F. \quad G. \quad H.
$$
$$
\mathrm{前两列对应相等},\mathrm{所以行列式等于}0.
$$



$$
\mathrm{四阶行列式}\begin{vmatrix}a&b&0&1\\b&0&a&1\\0&a&b&1\\a&0&b&1\end{vmatrix}=(\;\;\;).
$$
$$
A.
0 \quad B.1 \quad C.-1 \quad D.2 \quad E. \quad F. \quad G. \quad H.
$$
$$
\begin{vmatrix}a+b+1&b&0&1\\a+b+1&0&a&1\\a+b+1&a&b&1\\a+b+1&0&b&1\end{vmatrix}=\left(a+b+1\right)\begin{vmatrix}1&b&0&1\\1&0&a&1\\1&a&b&1\\1&0&b&1\end{vmatrix}=0
$$



$$
若n\mathrm{阶行列式}D=4,D^T为D\mathrm{的转置行列式},则D+D^T=(\;\;)
$$
$$
A.
4 \quad B.-4 \quad C.8 \quad D.-8 \quad E. \quad F. \quad G. \quad H.
$$
$$
\mathrm{根据行列式的性质}D^T=D,\mathrm{所以}D+D^T=8
$$



$$
\mathrm{四阶行列式}\begin{vmatrix}1&1&1&1\\-1&2&-3&4\\1&4&9&16\\-1&8&-27&64\end{vmatrix}=(\;\;).
$$
$$
A.
2100 \quad B.1200 \quad C.24 \quad D.-24 \quad E. \quad F. \quad G. \quad H.
$$
$$
\mathrm{是范德蒙德行列式},则\begin{vmatrix}1&1&1&1\\-1&2&-3&4\\1&4&9&16\\-1&8&-27&64\end{vmatrix}=\left(2+1\right)\left(-3+1\right)\left(4+1\right)\left(-3-2\right)\left(4-2\right)\left(4+3\right)=2100.
$$



$$
\mathrm{三阶行列式}\begin{vmatrix}1&1&1\\14&12&15\\196&144&225\end{vmatrix}=\left(\;\;\;\;\;\right).
$$
$$
A.
6 \quad B.-6 \quad C.4 \quad D.-4 \quad E. \quad F. \quad G. \quad H.
$$
$$
\begin{vmatrix}1&1&1\\14&12&15\\196&144&225\end{vmatrix}=\left(12-14\right)\left(15-14\right)\left(15-12\right)=-6.
$$



$$
\mathrm{三阶行列式}\begin{vmatrix}1&1&1\\a&b&e\\a^2&b^2&e^2\end{vmatrix}=\left(\;\;\;\;\right).
$$
$$
A.
\left(a-b\right)\left(b-e\right) \quad B.\left(a-b\right)\left(b-e\right)\left(a-e\right) \quad C.\left(b-a\right)\left(b-e\right) \quad D.\left(b-a\right)\left(e-a\right)\left(e-b\right) \quad E. \quad F. \quad G. \quad H.
$$
$$
\begin{vmatrix}1&1&1\\a&b&e\\a^2&b^2&e^2\end{vmatrix}=\left(b-a\right)\left(e-a\right)\left(e-b\right).
$$



$$
\mathrm{四阶行列式}\begin{vmatrix}1&2&4&8\\1&3&9&27\\1&4&16&64\\1&5&25&125\end{vmatrix}=(\;\;\;\;).
$$
$$
A.
10 \quad B.11 \quad C.12 \quad D.13 \quad E. \quad F. \quad G. \quad H.
$$
$$
\begin{vmatrix}1&2&4&8\\1&3&9&27\\1&4&16&64\\1&5&25&125\end{vmatrix}=\begin{vmatrix}1&2&2^2&2^3\\1&3&3^2&3^3\\1&4&4^2&4^3\\1&5&5^2&5^3\end{vmatrix}=\left(3-2\right)\left(4-2\right)\left(5-2\right)\left(4-3\right)\left(5-3\right)\left(5-4\right)=12.
$$



$$
\mathrm{四阶行列式}\begin{vmatrix}1&1&1&1\\1&2&3&4\\1&4&9&16\\1&8&27&64\end{vmatrix}=(\;\;\;\;\;).
$$
$$
A.
12 \quad B.13 \quad C.14 \quad D.15 \quad E. \quad F. \quad G. \quad H.
$$
$$
\begin{vmatrix}1&1&1&1\\1&2&3&4\\1&4&9&16\\1&8&27&64\end{vmatrix}=\begin{vmatrix}1&1&1&1\\1&2&3&4\\1^2&2^2&3^2&4^2\\1^3&2^3&3^3&4^3\end{vmatrix}=\left(2-1\right)\left(3-1\right)\left(4-1\right)\left(3-2\right)\left(4-2\right)\left(4-3\right)=2×3×2=12.
$$



$$
设D\left(x\right)=\begin{vmatrix}1&x&x^2&x^3\\1&2&4&8\\1&3&9&27\\1&4&16&64\end{vmatrix},则(\;\;\;\;)\mathrm{不是}D\left(x\right)=0\mathrm{的根}.
$$
$$
A.
1 \quad B.2 \quad C.3 \quad D.4 \quad E. \quad F. \quad G. \quad H.
$$
$$
D\left(x\right)=\begin{vmatrix}1&x&x^2&x^3\\1&2&2^2&2^3\\1&3&3^2&3^3\\1&4&4^2&4^3\end{vmatrix}=2\left(2-x\right)\left(3-x\right)\left(4-x\right),D\left(x\right)=0⇒ x_1=2,x_2=3,x_3=4.
$$



$$
\mathrm{下列}(\;\;\;\;)\mathrm{不是方程}\begin{vmatrix}1&1&1&1\\1&2&3&x\\1&4&9&x^2\\1&8&27&x^3\end{vmatrix}=0\mathrm{的根}.
$$
$$
A.
1 \quad B.2 \quad C.3 \quad D.4 \quad E. \quad F. \quad G. \quad H.
$$
$$
\begin{array}{l}\mathrm{由范德蒙行列式的计算公式可得}\;\;,\;\\\;\;\;\;\;\;\;\;\;\;\;\;\;\;\;\;\;\;\;\;\;\;\;\;\;\;\;\;\begin{vmatrix}1&1&1&1\\1&2&3&x\\1&4&9&x^2\\1&8&27&x^3\end{vmatrix}=2\left(x-1\right)\left(x-2\right)\left(x-3\right)=0,\\\;\mathrm{方程的全部根为}1,2,3.\end{array}
$$



$$
\mathrm{方程}\begin{vmatrix}1&2&4\\1&3&9\\1&x&x^2\end{vmatrix}=0,\mathrm{方程的根有}2和(\;\;\;\;).
$$
$$
A.
1 \quad B.2 \quad C.3 \quad D.4 \quad E. \quad F. \quad G. \quad H.
$$
$$
\begin{vmatrix}1&2&4\\1&3&9\\1&x&x^2\end{vmatrix}=\left(3-2\right)\left(x-2\right)\left(x-3\right)=0,\;\;\mathrm{所以}x=2,\;x=3.
$$



$$
\mathrm{四阶行列式}\begin{vmatrix}1&1&1&1\\2&3&4&5\\4&9&16&25\\8&27&64&125\end{vmatrix}=\left(\;\;\;\;\;\right).
$$
$$
A.
12 \quad B.-12 \quad C.24 \quad D.-24 \quad E. \quad F. \quad G. \quad H.
$$
$$
\begin{vmatrix}1&1&1&1\\2&3&4&5\\4&9&16&25\\8&27&64&125\end{vmatrix}=(3-2)(4-2)(5-2)(4-3)(5-3)(5-4)=12.
$$



$$
\mathrm{多项式}f\left(x\right)=\begin{vmatrix}1&1&1&1\\x&a&b&c\\x^2&a^2&b^2&c^2\\x^3&a^3&b^3&c^3\end{vmatrix}=(\;\;\;\;\;).
$$
$$
A.
\left(x-a\right)\left(x-b\right)\left(x-c\right)\left(b-a\right)\left(c-a\right)\left(c-b\right) \quad B.\left(b-a\right)\left(c-a\right)\left(c-b\right) \quad C.-\left(x-a\right)\left(x-b\right)\left(x-c\right)\left(b-a\right)\left(c-a\right)\left(c-b\right) \quad D.-\left(b-a\right)\left(c-a\right)\left(c-b\right) \quad E. \quad F. \quad G. \quad H.
$$
$$
f\left(x\right)=\begin{vmatrix}1&1&1&1\\x&a&b&c\\x^2&a^2&b^2&c^2\\x^3&a^3&b^3&c^3\end{vmatrix}=\left(a-x\right)\left(b-x\right)\left(c-x\right)\left(b-a\right)\left(c-a\right)\left(c-b\right)=-\left(x-a\right)\left(x-b\right)\left(x-c\right)\left(b-a\right)\left(c-a\right)\left(c-b\right).
$$



$$
\mathrm{三阶行列式}\begin{vmatrix}a&a^2&a^3\\b&b^2&b^3\\c&c^2&c^3\end{vmatrix}=\left(\;\;\;\;\;\right).
$$
$$
A.
\left(b-a\right)\left(c-a\right)\left(c-b\right) \quad B.abc\left(b-a\right)\left(c-a\right)\left(c-b\right) \quad C.\left(b+a\right)\left(c+a\right)\left(c+b\right) \quad D.abc\left(b+a\right)\left(c+a\right)\left(c+b\right) \quad E. \quad F. \quad G. \quad H.
$$
$$
\mathrm{原式}=abc\begin{vmatrix}1&a&a^2\\1&b&b^2\\1&c&c^2\end{vmatrix}=abc\left(b-a\right)\left(c-a\right)\left(c-b\right).
$$



$$
\mathrm{四阶行列式}\begin{vmatrix}1&1&1&1\\2&-2&3&-1\\4&4&9&1\\8&-8&27&-1\end{vmatrix}=(\;\;\;\;).
$$
$$
A.
240 \quad B.-240 \quad C.250 \quad D.-250 \quad E. \quad F. \quad G. \quad H.
$$
$$
D=\begin{vmatrix}1&1&1&1\\2&-2&3&-1\\2^2&\left(-2\right)^2&3^2&\left(-1\right)^2\\2^3&\left(-2\right)^3&3^3&\left(-1\right)^3\end{vmatrix}=\left(-2-2\right)\left(3-2\right)\left(-1-2\right)\left(3+2\right)\left(-1+2\right)\left(-1-3\right)=-240.
$$



$$
\mathrm{三阶行列式}\begin{vmatrix}1&1&1\\b&b^2&b^3\\c&c^2&c^3\end{vmatrix}=\left(\;\;\;\;\;\right).
$$
$$
A.
\left(b-1\right)\left(c-1\right)\left(c-b\right) \quad B.bc\left(b-1\right)\left(c-1\right)\left(c-b\right) \quad C.\left(b+1\right)\left(c+1\right)\left(c+b\right) \quad D.bc\left(b+1\right)\left(c+1\right)\left(c+b\right) \quad E. \quad F. \quad G. \quad H.
$$
$$
\mathrm{原式}=bc\begin{vmatrix}1&1&1\\1&b&b^2\\1&c&c^2\end{vmatrix}=bc\left(b-1\right)\left(c-1\right)\left(c-b\right).
$$



$$
\mathrm{四阶行列式}\begin{vmatrix}a&b&c&d\\a^2&b^2&c^2&d^2\\a^3&b^3&c^3&d^3\\1&1&1&1\end{vmatrix}=(\;\;\;\;\;\;).
$$
$$
A.
0 \quad B.\left(b-a\right)\left(c-a\right)\left(d-a\right)\left(c-b\right)\left(d-b\right)\left(d-c\right) \quad C.-\left(b-a\right)\left(c-a\right)\left(d-a\right)\left(c-b\right)\left(d-b\right)\left(d-c\right) \quad D.-\left(c-a\right)\left(d-a\right)\left(c-b\right)\left(d-b\right)\left(d-c\right) \quad E. \quad F. \quad G. \quad H.
$$
$$
\begin{vmatrix}a&b&c&d\\a^2&b^2&c^2&d^2\\a^3&b^3&c^3&d^3\\1&1&1&1\end{vmatrix}=\left(-1\right)^3\begin{vmatrix}1&1&1&1\\a&b&c&d\\a^2&b^2&c^2&d^2\\a^3&b^3&c^3&d^3\end{vmatrix}=-\left(b-a\right)\left(c-a\right)\left(d-a\right)\left(c-b\right)\left(d-b\right)\left(d-c\right).
$$



$$
\mathrm{四阶行列式}\begin{vmatrix}1&1&1&1\\1&-2&2&3\\4&1&9&16\\8&-1&27&64\end{vmatrix}=(\;\;).
$$
$$
A.
-120 \quad B.120 \quad C.24 \quad D.-24 \quad E. \quad F. \quad G. \quad H.
$$
$$
\begin{vmatrix}1&1&1&1\\1&-2&2&3\\4&1&9&16\\8&-1&27&64\end{vmatrix}\overset{\;r_2+r_1\;}=\begin{vmatrix}1&1&1&1\\2&-1&3&4\\4&1&9&16\\8&-1&27&64\end{vmatrix}=(-1-2)(3-2)(4-2)(3+1)(4+1)(4-3)=-120.
$$



$$
\mathrm{行列式}\begin{vmatrix}1&1&1&0&0\\2&3&-1&0&0\\4&9&1&0&0\\0&0&0&2&1\\0&0&0&4&3\end{vmatrix}=\left(\;\;\;\;\right).
$$
$$
A.
24 \quad B.12 \quad C.-12 \quad D.-24 \quad E. \quad F. \quad G. \quad H.
$$
$$
\mathrm{原式}=\begin{vmatrix}1&1&1\\2&3&-1\\4&9&1\end{vmatrix}×\begin{vmatrix}2&1\\4&3\end{vmatrix}=\begin{vmatrix}1&1&1\\0&1&-3\\0&5&-3\end{vmatrix}×2=12×2=24.
$$



$$
\mathrm{四阶行列式}D=\begin{vmatrix}0&a&b&0\\e&0&0&f\\g&0&0&h\\0&c&d&0\end{vmatrix}=\left(\;\;\;\;\;\;\right).
$$
$$
A.
aehd-bfgc \quad B.aehd+bfgc \quad C.\left(ae-bf\right)\left(hd-gc\right) \quad D.\left(eh-fg\right)\left(ad-bc\right) \quad E. \quad F. \quad G. \quad H.
$$
$$
\begin{array}{l}\mathrm{按行列式的第一行和第四行展开得}\\D=\begin{vmatrix}0&a&b&0\\e&0&0&f\\g&0&0&h\\0&c&d&0\end{vmatrix}=\begin{vmatrix}a&b\\c&d\end{vmatrix}×\left(-1\right)^{1+4+2+3}\begin{vmatrix}e&f\\g&h\end{vmatrix}=\left(ad-bc\right)\left(eh-gf\right).\end{array}
$$



$$
\mathrm{行列式}\begin{vmatrix}2&3&0&0\\1&2&3&0\\0&1&2&3\\0&0&1&2\end{vmatrix}\mathrm{的值为}\left(\;\;\;\;\;\;\;\right).
$$
$$
A.
-11 \quad B.11 \quad C.-13 \quad D.13 \quad E. \quad F. \quad G. \quad H.
$$
$$
\begin{array}{l}\mathrm{按第一行和第二行展开}\\\begin{vmatrix}2&3&0&0\\1&2&3&0\\0&1&2&3\\0&0&1&2\end{vmatrix}=\begin{vmatrix}2&3\\1&2\end{vmatrix}×\left(-1\right)^{1+2+1+2}\begin{vmatrix}2&3\\1&2\end{vmatrix}+\begin{vmatrix}2&0\\1&3\end{vmatrix}×\left(-1\right)^{1+2+1+3}\begin{vmatrix}1&3\\0&2\end{vmatrix}\\+\begin{vmatrix}3&0\\2&3\end{vmatrix}×\left(-1\right)^{1+2+2+3}\begin{vmatrix}0&3\\0&2\end{vmatrix}=1-12+0=-11.\end{array}
$$



$$
\mathrm{行列式}D=\begin{vmatrix}2&4&3&7&-6&-1\\1&3&5&8&4&0\\0&0&2&-1&9&8\\0&0&1&1&7&-2\\0&0&0&0&-1&-3\\0&0&0&0&3&5\end{vmatrix}=\left(\;\;\;\;\;\;\right).
$$
$$
A.
24 \quad B.-24 \quad C.60 \quad D.-60 \quad E. \quad F. \quad G. \quad H.
$$
$$
D=\begin{vmatrix}2&4\\1&3\end{vmatrix}·\begin{vmatrix}2&-1\\1&1\end{vmatrix}·\begin{vmatrix}-1&-3\\3&5\end{vmatrix}=2·3·4=24.
$$



$$
\mathrm{行列式}D=\begin{vmatrix}7&6&5&4&3&2\\9&7&8&9&4&3\\7&4&9&7&0&0\\5&3&6&1&0&0\\0&0&5&6&0&0\\0&0&6&8&0&0\end{vmatrix}\mathrm{的值为}\left(\;\;\;\;\;\right).
$$
$$
A.
0 \quad B.4 \quad C.-4 \quad D.-1 \quad E. \quad F. \quad G. \quad H.
$$
$$
D=\begin{vmatrix}3&2\\4&3\end{vmatrix}·\left(-1\right)^{5+6+1+2}·\begin{vmatrix}7&4&9&7\\5&3&6&1\\0&0&5&6\\0&0&6&8\end{vmatrix}=\begin{vmatrix}3&2\\4&3\end{vmatrix}·\begin{vmatrix}7&4\\5&3\end{vmatrix}·\begin{vmatrix}5&6\\6&8\end{vmatrix}=\left(9-8\right)\left(21-20\right)\left(40-36\right)=4.
$$



$$
\mathrm{行列式}\begin{vmatrix}1&1&1&0&0&0\\2&3&4&0&0&0\\3&10&16&1&1&1\\-1&1&0&1&1&1\\-2&4&1&1&2&3\\-3&16&1&1&4&9\end{vmatrix}\mathrm{的值为}\left(\;\;\;\;\right).
$$
$$
A.
0 \quad B.1 \quad C.4 \quad D.12 \quad E. \quad F. \quad G. \quad H.
$$
$$
\begin{array}{l}\mathrm{原式}\overset{r_3-r_4}=\begin{vmatrix}1&1&1&0&0&0\\2&3&4&0&0&0\\4&9&16&0&0&0\\-1&1&0&1&1&1\\-2&4&1&1&2&3\\-3&16&1&1&4&9\end{vmatrix}=\begin{vmatrix}1&1&1\\2&3&4\\4&9&16\end{vmatrix}\begin{vmatrix}1&1&1\\1&2&3\\1&4&9\end{vmatrix},\\\mathrm{上式两个范德蒙行列式},\mathrm{所以原式}=\left(3-2\right)\left(4-2\right)\left(4-3\right)\left(2-1\right)\left(3-1\right)\left(3-2\right)=4.\end{array}
$$



$$
\mathrm{四阶行列式}\begin{vmatrix}0&a&0&0\\b&c&0&0\\0&0&c&e\\0&0&0&d\end{vmatrix}\mathrm{的值为}\left(\;\;\right).
$$
$$
A.
abcd \quad B.-abce \quad C.abce \quad D.-abcd \quad E. \quad F. \quad G. \quad H.
$$
$$
\mathrm{把第一行展开},有\begin{vmatrix}0&a&0&0\\b&c&0&0\\0&0&c&e\\0&0&0&d\end{vmatrix}=-a\begin{vmatrix}b&0&0\\0&c&e\\0&0&d\end{vmatrix}=-abcd.
$$



$$
若\begin{vmatrix}a_{11}&a_{12}\\a_{21}&a_{22}\end{vmatrix}=6,则\begin{vmatrix}a_{11}&2a_{12}&0\\a_{21}&2a_{22}&0\\0&-2&-1\end{vmatrix}\mathrm{的值为}\left(\;\;\;\right).
$$
$$
A.
-12 \quad B.12 \quad C.18 \quad D.0 \quad E. \quad F. \quad G. \quad H.
$$
$$
\begin{vmatrix}a_{11}&2a_{12}&0\\a_{21}&2a_{22}&0\\0&-2&-1\end{vmatrix}=\left(-1\right)×(-1)^{3+3}\begin{vmatrix}a_{11}&2a_{12}\\a_{21}&2a_{22}\end{vmatrix}=-2\begin{vmatrix}a_{11}&a_{12}\\a_{21}&a_{22}\end{vmatrix}=-2×6=-12.
$$



$$
\mathrm{三阶行列式}\begin{vmatrix}a_1&b_1&c_1\\a_2&b_2&c_2\\a_3&b_3&c_3\end{vmatrix}=\left(\;\;\;\;\right).
$$
$$
A.
a_1\begin{vmatrix}b_2&c_2\\b_3&c_3\end{vmatrix}+b_1\begin{vmatrix}a_2&c_2\\a_3&c_3\end{vmatrix}+c_1\begin{vmatrix}a_2&b_2\\a_3&b_3\end{vmatrix} \quad B.a_1\begin{vmatrix}b_2&c_2\\b_3&c_3\end{vmatrix}-b_1\begin{vmatrix}a_2&c_2\\a_3&c_3\end{vmatrix}-c_1\begin{vmatrix}a_2&b_2\\a_3&b_3\end{vmatrix} \quad C.a_1\begin{vmatrix}b_2&c_2\\b_3&c_3\end{vmatrix}+b_1\begin{vmatrix}a_2&c_2\\a_3&c_3\end{vmatrix}-c_1\begin{vmatrix}a_2&b_2\\a_3&b_3\end{vmatrix} \quad D.a_1\begin{vmatrix}b_2&c_2\\b_3&c_3\end{vmatrix}-b_1\begin{vmatrix}a_2&c_2\\a_3&c_3\end{vmatrix}+c_1\begin{vmatrix}a_2&b_2\\a_3&b_3\end{vmatrix} \quad E. \quad F. \quad G. \quad H.
$$
$$
\begin{vmatrix}a_1&b_1&c_1\\a_2&b_2&c_2\\a_3&b_3&c_3\end{vmatrix}=a_1\begin{vmatrix}b_2&c_2\\b_3&c_3\end{vmatrix}-b_1\begin{vmatrix}a_2&c_2\\a_3&c_3\end{vmatrix}+c_1\begin{vmatrix}a_2&b_2\\a_3&b_3\end{vmatrix}.
$$



$$
\mathrm{三阶行列式}\begin{vmatrix}-3&0&4\\5&4&3\\2&-2&1\end{vmatrix}\mathrm{中的元素}2\mathrm{的代数余子式为}(\;\;\;).\;
$$
$$
A.
-16 \quad B.16 \quad C.2 \quad D.4 \quad E. \quad F. \quad G. \quad H.
$$
$$
2\mathrm{的代数余子式为}\left(-1\right)^{3+1}\begin{vmatrix}0&4\\4&3\end{vmatrix}=-16.
$$



$$
\mathrm{四阶行列式}\begin{vmatrix}1&0&2&1\\4&-1&x&0\\2&2&-1&0\\1&5&-2&1\end{vmatrix}\mathrm{中元素}x\mathrm{的代数余子式是}(\;\;\;\;\;\;).
$$
$$
A.
10 \quad B.-10 \quad C.20 \quad D.-20 \quad E. \quad F. \quad G. \quad H.
$$
$$
x\mathrm{的代数余子式为}\left(-1\right)^{2+3}\begin{vmatrix}1&0&1\\2&2&0\\1&5&1\end{vmatrix}=-\begin{vmatrix}1&0&0\\0&2&-2\\0&5&0\end{vmatrix}=-\begin{vmatrix}2&-2\\5&0\end{vmatrix}=-10.
$$



$$
\mathrm{三阶行列式}\begin{vmatrix}-3&0&4\\5&0&3\\2&-2&1\end{vmatrix}\mathrm{中元素}-2\mathrm{的代数余子式为}(\;\;\;).
$$
$$
A.
29 \quad B.-29 \quad C.27 \quad D.-27 \quad E. \quad F. \quad G. \quad H.
$$
$$
\mathrm{元素}-2\mathrm{的代数余子式为}\left(-1\right)^{3+2}\begin{vmatrix}-3&4\\5&3\end{vmatrix}=29.
$$



$$
\mathrm{五阶行列式按第三列展开计算}\begin{vmatrix}a_{11}&a_{12}&a_{13}&a_{14}&a_{15}\\a_{21}&a_{22}&a_{23}&a_{24}&a_{25}\\a_{31}&a_{32}&0&0&0\\a_{41}&a_{42}&0&0&0\\a_{51}&a_{52}&0&0&0\end{vmatrix}=(\;\;\;).\;
$$
$$
A.
1 \quad B.0 \quad C.a_{11}a_{22} \quad D.a_{11}+a_{22} \quad E. \quad F. \quad G. \quad H.
$$
$$
\begin{vmatrix}a_{11}&a_{12}&a_{13}&a_{14}&a_{15}\\a_{21}&a_{22}&a_{23}&a_{24}&a_{25}\\a_{31}&a_{32}&0&0&0\\a_{41}&a_{42}&0&0&0\\a_{51}&a_{52}&0&0&0\end{vmatrix}=a_{13}\begin{vmatrix}a_{21}&a_{22}&a_{24}&a_{25}\\a_{31}&a_{32}&0&0\\a_{41}&a_{42}&0&0\\a_{51}&a_{52}&0&0\end{vmatrix}-a_{23}\begin{vmatrix}a_{11}&a_{12}&a_{14}&a_{15}\\a_{31}&a_{32}&0&0\\a_{41}&a_{42}&0&0\\a_{51}&a_{52}&0&0\end{vmatrix}=0.
$$



$$
\mathrm{设五阶行列式}\begin{vmatrix}a_{11}&a_{12}&a_{13}\;\;\;a_{14}&a_{15}&\\a_{21}&a_{22}&a_{23\;}\;\;a_{24}&a_{25}&\\a_{31}&a_{32}&a_{33\;}\;\;a_{34}&a_{35}&\\a_{41}&a_{42}&a_{43}\;\;\;a_{44}&\;a_{45}&\\a_{51}&a_{52}&a_{53}\;\;a_{54}&a_{55}&\end{vmatrix},a_{34}\mathrm{的代数余子式}A_{34}是(\;\;\;\;).
$$
$$
A.
A_{34}=\left(-1\right)^{3+4}\begin{vmatrix}a_{11}&a_{12}&a_{13}&a_{15}\\a_{21}&a_{22}&a_{23}&a_{25}\\a_{41}&a_{42}&a_{43}&a_{45}\\a_{51}&a_{52}&a_{53}&a_{55}\end{vmatrix} \quad B.A_{34}=\left(-1\right)^4\begin{vmatrix}a_{11}&a_{12}&a_{13}&a_{15}\\a_{21}&a_{22}&a_{23}&a_{25}\\a_{41}&a_{42}&a_{43}&a_{45}\\a_{51}&a_{52}&a_{53}&a_{55}\end{vmatrix} \quad C.A_{34}=\begin{vmatrix}a_{11}&a_{12}&a_{13}&a_{15}\\a_{21}&a_{22}&a_{23}&a_{25}\\a_{41}&a_{42}&a_{43}&a_{45}\\a_{51}&a_{52}&a_{53}&a_{55}\end{vmatrix} \quad D.A_{34}=\left(-1\right)^{3×4}\begin{vmatrix}a_{11}&a_{12}&a_{13}&a_{15}\\a_{21}&a_{22}&a_{23}&a_{25}\\a_{41}&a_{42}&a_{43}&a_{45}\\a_{51}&a_{52}&a_{53}&a_{55}\end{vmatrix} \quad E. \quad F. \quad G. \quad H.
$$
$$
A_{34}=\left(-1\right)^{3+4}\begin{vmatrix}a_{11}&a_{12}&a_{13}&a_{15}\\a_{21}&a_{22}&a_{23}&a_{25}\\a_{41}&a_{42}&a_{43}&a_{45}\\a_{51}&a_{52}&a_{53}&a_{55}\end{vmatrix}.
$$



$$
\mathrm{按第一行展开},\mathrm{计算三阶行列式}\begin{vmatrix}i&j&k\\1&2&3\\2&1&3\end{vmatrix}=(\;\;\;\;).
$$
$$
A.
3i+3j-3k \quad B.3i+3j+3k \quad C.3i-3j-3k \quad D.2i+3j-3k \quad E. \quad F. \quad G. \quad H.
$$
$$
\mathrm{原式}=\begin{vmatrix}2&3\\1&3\end{vmatrix}i-\begin{vmatrix}1&3\\2&3\end{vmatrix}j+\begin{vmatrix}1&2\\2&1\end{vmatrix}k=3i+3j-3k.
$$



$$
\mathrm{设三阶行列式}\begin{vmatrix}2&1&3\\4&-1&2\\1&2&-1\end{vmatrix},\mathrm{则代数余子式}A_{23}为(\;\;\;\;).
$$
$$
A.
-3 \quad B.3 \quad C.2 \quad D.-2 \quad E. \quad F. \quad G. \quad H.
$$
$$
A_{23}=\left(-1\right)^{2+3}\begin{vmatrix}2&1\\1&2\end{vmatrix}=-3.
$$



$$
\mathrm{设四阶行列式}\begin{vmatrix}a&b&c&d\\1&-1&0&1\\2&0&2&-1\\3&1&3&-1\end{vmatrix},a_{12}\mathrm{的代数余子式为}(\;\;\;\;).
$$
$$
A.
0 \quad B.-1 \quad C.-2 \quad D.-3 \quad E. \quad F. \quad G. \quad H.
$$
$$
A_{12}=\left(-1\right)^{1+2}\begin{vmatrix}1&0&1\\2&2&-1\\3&3&-1\end{vmatrix}=-1.
$$



$$
\mathrm{四阶行列式}\begin{vmatrix}a_{11}&a_{12}&a_{13}&a_{14}\\a_{21}&a_{22}&a_{23}&a_{24}\\a_{31}&a_{32}&a_{33}&a_{34}\\a_{41}&a_{42}&a_{43}&a_{44}\end{vmatrix}中a_{32}\mathrm{的代数余子式为}(\;\;\;\;\;).
$$
$$
A.
\begin{vmatrix}a_{11}&a_{13}&a_{14}\\a_{21}&a_{23}&a_{24}\\a_{41}&a_{43}&a_{44}\end{vmatrix} \quad B.\left(-1\right)^{3+2}\begin{vmatrix}a_{11}&a_{13}&a_{14}\\a_{21}&a_{23}&a_{24}\\a_{41}&a_{43}&a_{44}\end{vmatrix} \quad C.\left(-1\right)^{3×2}\begin{vmatrix}a_{11}&a_{13}&a_{14}\\a_{21}&a_{23}&a_{24}\\a_{41}&a_{43}&a_{44}\end{vmatrix} \quad D.\left(-1\right)^2\begin{vmatrix}a_{11}&a_{13}&a_{14}\\a_{21}&a_{23}&a_{24}\\a_{41}&a_{43}&a_{44}\end{vmatrix} \quad E. \quad F. \quad G. \quad H.
$$
$$
a_{32}\mathrm{的代数余子式}A_{32}=\left(-1\right)^{3+2}M_{32}=-\begin{vmatrix}a_{11}&a_{13}&a_{14}\\a_{21}&a_{23}&a_{24}\\a_{41}&a_{43}&a_{44}\end{vmatrix}.
$$



$$
\mathrm{四阶行列式}\begin{vmatrix}1&2&1&4\\0&-1&2&-2\\4&7&2&1\\0&2&1&3\end{vmatrix}\mathrm{的代数余子式}A_{24}=(\;\;\;\;\;).
$$
$$
A.
3 \quad B.4 \quad C.5 \quad D.6 \quad E. \quad F. \quad G. \quad H.
$$
$$
A_{24}=\left(-1\right)^{2+4}\begin{vmatrix}1&2&1\\4&7&2\\0&2&1\end{vmatrix}=3.
$$



$$
\mathrm{四阶行列式}\begin{vmatrix}1&2&1&4\\0&-1&2&-2\\4&7&2&1\\0&2&1&3\end{vmatrix}\mathrm{的代数余子式}A_{32}=(\;\;\;\;\;).
$$
$$
A.
-8 \quad B.8 \quad C.-9 \quad D.9 \quad E. \quad F. \quad G. \quad H.
$$
$$
A_{32}=\left(-1\right)^{3+2}\begin{vmatrix}1&1&4\\0&2&-2\\0&1&3\end{vmatrix}=-8.
$$



$$
n\mathrm{阶行列式的元素}a_{ij}\mathrm{的代数余子式}A_{ij}\mathrm{与余子式}M_{ij}\mathrm{之间的关系是}(\;\;\;\;).\;
$$
$$
A.
A_{ij}=M_{ij} \quad B.A_{ij}=-M_{ij} \quad C.A_{ij}=a_{ij}M_{ij} \quad D.A_{ij}=\left(-1\right)^{i+j}M_{ij} \quad E. \quad F. \quad G. \quad H.
$$
$$
A_{ij}=\left(-1\right)^{i+j}M_{ij}.
$$



$$
若\begin{vmatrix}a_{11}&a_{12}\\a_{21}&a_{22}\end{vmatrix}=6,则\begin{vmatrix}a_{11}&a_{12}&0\\a_{21}&a_{22}&0\\0&-2&2\end{vmatrix}\mathrm{的值为}(\;\;\;\;\;).\;
$$
$$
A.
-12 \quad B.12 \quad C.18 \quad D.0 \quad E. \quad F. \quad G. \quad H.
$$
$$
\begin{vmatrix}a_{11}&a_{12}&0\\a_{21}&a_{22}&0\\0&-2&2\end{vmatrix}=2\begin{vmatrix}a_{11}&a_{12}\\a_{21}&a_{22}\end{vmatrix}=2×6=12.
$$



$$
\mathrm{三阶行列式}\begin{vmatrix}-3&0&4\\5&0&3\\2&-2&1\end{vmatrix}\mathrm{中的元素}2\mathrm{的代数余子式为}(\;\;\;).\;
$$
$$
A.
0 \quad B.1 \quad C.2 \quad D.4 \quad E. \quad F. \quad G. \quad H.
$$
$$
2\mathrm{的代数余子式为}\left(-1\right)^{3+1}\begin{vmatrix}0&4\\0&3\end{vmatrix}=0.
$$



$$
\mathrm{四阶行列式}\begin{vmatrix}1&0&2&0\\4&-1&x&0\\2&2&-1&0\\1&5&-2&1\end{vmatrix}\mathrm{中元素}x\mathrm{的代数余子式是}(\;\;\;\;\;\;).
$$
$$
A.
2 \quad B.-2 \quad C.20 \quad D.-20 \quad E. \quad F. \quad G. \quad H.
$$
$$
x\mathrm{的代数余子式为}\left(-1\right)^{2+3}\begin{vmatrix}1&0&0\\2&2&0\\1&5&1\end{vmatrix}=-2.
$$



$$
\mathrm{设三阶行列式}\begin{vmatrix}2&1&3\\4&-1&2\\1&2&-1\end{vmatrix},\mathrm{则代数余子式}A_{13}为(\;\;\;\;).
$$
$$
A.
9 \quad B.10 \quad C.11 \quad D.12 \quad E. \quad F. \quad G. \quad H.
$$
$$
A_{13}=\left(-1\right)^{1+3}\begin{vmatrix}4&-1\\1&2\end{vmatrix}=9.
$$



$$
\mathrm{四阶行列式}\begin{vmatrix}3&6&9&12\\2&4&6&8\\1&2&0&3\\5&6&4&3\end{vmatrix},则3A_{41}+6A_{42}+9A_{43}+12A_{44}=(\;\;\;\;\;).
$$
$$
A.
0 \quad B.1 \quad C.-1 \quad D.-2 \quad E. \quad F. \quad G. \quad H.
$$
$$
\begin{array}{l}\mathrm{根据行列式某一行}(列)\mathrm{的元素与另一行}(列)\mathrm{的对应元素的代数余子式乘积之和等于零可得}:\\3A_{41}+6A_{42}+9A_{43}+12A_{44}=0.\end{array}
$$



$$
\mathrm{设四阶行列式}\begin{vmatrix}a&b&c&d\\1&-1&0&1\\2&0&2&-1\\3&1&3&-1\end{vmatrix},则A_{11}+2A_{12}+3A_{13}+4A_{14}=(\;\;\;\;).
$$
$$
A.
15 \quad B.16 \quad C.17 \quad D.18 \quad E. \quad F. \quad G. \quad H.
$$
$$
\begin{array}{l}\begin{array}{l}A_{11}+2A_{12}+3A_{13}+4A_{14}=\begin{vmatrix}1&2&3&4\\1&-1&0&1\\2&0&2&-1\\3&1&3&-1\end{vmatrix}=\begin{vmatrix}1&2&3&4\\0&-3&-3&-3\\0&-4&-4&-9\\0&-5&-6&-13\end{vmatrix}\\\end{array}\\=-3\begin{vmatrix}1&2&3&4\\0&1&1&1\\0&-4&-4&-9\\0&-5&-6&-13\end{vmatrix}=-3\begin{vmatrix}1&2&3&4\\0&1&1&1\\0&0&0&-5\\0&0&-1&-8\end{vmatrix}=3\begin{vmatrix}1&2&3&4\\0&1&1&1\\0&0&-1&-8\\0&0&0&-5\end{vmatrix}=15.\end{array}
$$



$$
\mathrm{已知四阶行列式}D\mathrm{中第}3\mathrm{列元素依次为}-1,2,3,1,\mathrm{它们的余子式依次为}5,3,-1,4,则D=(\;\;).
$$
$$
A.
-18 \quad B.18 \quad C.8 \quad D.-8 \quad E. \quad F. \quad G. \quad H.
$$
$$
\begin{array}{l}\mathrm{根据条件可知},第3\mathrm{列元素的代数余子式依次为}5,-3,-1,-4,\\则D=\left(-1\right)×5+2×\left(-3\right)+3×\left(-1\right)-1×4=-18.\end{array}
$$



$$
\mathrm{如果对于}n\mathrm{阶行列式}D\mathrm{的元素}a_{ij}\mathrm{及代数余子式}A_{ij},\mathrm{总有}∑_{k=1}^na_{ki}A_{kj}=0\left(i,j=1,2…,n\right),则D=(\;\;\;\;).
$$
$$
A.
0 \quad B.1 \quad C.-1 \quad D.\mathrm{无法确定} \quad E. \quad F. \quad G. \quad H.
$$
$$
∑_{k=1}^na_{ki}A_{kj}=\left\{\begin{array}{lc}D,&i=j\\0,&i\neq j\end{array}\right.,\mathrm{由于恒有}∑_{k=1}^na_{ki}A_{kj}=0\left(i,j=1,2…,n\right),故D=0.
$$



$$
\mathrm{设四阶行列式}\begin{vmatrix}1&2&3&4\\4&3&2&1\\1&0&-1&2\\5&1&-4&6\end{vmatrix},则4A_{41}+3A_{42}+2A_{43}+A_{44}=(\;\;\;\;\;\;\;\;).
$$
$$
A.
1 \quad B.0 \quad C.12 \quad D.-12 \quad E. \quad F. \quad G. \quad H.
$$
$$
\begin{array}{l}4A_{41}+3A_{42}+2A_{43}+A_{44}\mathrm{表示的是第}2\mathrm{行元素与第}4\mathrm{行元素的代数余子式的乘积},\mathrm{由行列式的展开定理可知}\\4A_{41}+3A_{42}+2A_{43}+A_{44}=0.\end{array}
$$



$$
\mathrm{已知四阶行列式}D\mathrm{中第}3\mathrm{列元素依次为}-1,2,3,1,\mathrm{他们的余子式依次为}6,1,-5,2,则D\mathrm{的值为}(\;\;).
$$
$$
A.
-25 \quad B.24 \quad C.25 \quad D.-24 \quad E. \quad F. \quad G. \quad H.
$$
$$
\begin{array}{l}\mathrm{由条件可知第}3\mathrm{列元素的代数余子式依次为}6,-1,-5,-2,\mathrm{则由行列式的展开定理可得}\\D=-1×6+2×\left(-1\right)+3×\left(-5\right)+1×\left(-2\right)=-25.\end{array}
$$



$$
\mathrm{已知四阶行列式中第}3\mathrm{行元素依次为}-1,2,0,1,\mathrm{它们的余子式分别为}5,3,-7,4,则D=(\;\;\;).
$$
$$
A.
15 \quad B.-15 \quad C.5 \quad D.-5 \quad E. \quad F. \quad G. \quad H.
$$
$$
\begin{array}{l}\mathrm{由条件可知},第3\mathrm{列元素的代数余子式为}5,-3,-7,-4,则\\D=-1×5+2×\left(-3\right)+0×\left(-7\right)+1×\left(-4\right)=-15.\end{array}
$$



$$
\mathrm{设四阶行列式}D=\begin{vmatrix}3&-5&2&1\\1&1&0&-5\\-1&3&1&3\\2&-4&-1&-3\end{vmatrix},D\mathrm{中元素}a_{ij}\mathrm{的代数余子式记作}A_{ij},\;则A_{11}+A_{12}+A_{13}+A_{14}为\left(\;\;\;\;\right).
$$
$$
A.
4 \quad B.-4 \quad C.-1 \quad D.1 \quad E. \quad F. \quad G. \quad H.
$$
$$
\begin{array}{l}\mathrm{注意到}A_{11}+A_{12}+A_{13}+A_{14}\mathrm{等于}1,1,1,1\mathrm{代替}D\mathrm{的第}1\mathrm{行所得的行列式},即\\A_{11}+A_{12}+A_{13}+A_{14}=\begin{vmatrix}1&1&1&1\\1&1&0&-5\\-1&3&1&3\\2&-4&-1&-3\end{vmatrix}\overset{r_4+r_3}{\underset{r_3-r_1}=}\begin{vmatrix}1&1&1&1\\1&1&0&-5\\-2&2&0&2\\1&-1&0&0\end{vmatrix}\\\;\;\;\;\;\;\;\;\;\;\;\;\;\;\;\;\;\;\;\;\;\;\;\;\;\;\;\;\;\;\;\;\;=\begin{vmatrix}1&1&-5\\-2&2&2\\1&-1&0\end{vmatrix}=\begin{vmatrix}2&-5\\0&2\end{vmatrix}=4.\end{array}
$$



$$
\mathrm{设四阶行列式}\begin{vmatrix}3&6&9&12\\2&4&6&8\\1&2&0&3\\5&6&4&3\end{vmatrix},\mathrm{元素}a_{ij}\mathrm{的代数余子式记为}A_{ij},则A_{41}+2A_{42}+3A_{44}=\left(\;\;\;\;\;\;\;\right).\;
$$
$$
A.
0 \quad B.1 \quad C.2 \quad D.4 \quad E. \quad F. \quad G. \quad H.
$$
$$
A_{41}+2A_{42}+3A_{44}=1· A_{41}+2· A_{42}+0· A_{43}+3· A_{44}=\begin{vmatrix}3&6&9&12\\2&4&6&8\\1&2&0&3\\1&2&0&3\end{vmatrix}=0.
$$



$$
\mathrm{四阶行列式}\begin{vmatrix}a&b&c&d\\b&a&c&d\\d&a&c&b\\d&b&c&a\end{vmatrix},\mathrm{元素}a_{ij}\mathrm{的代数余子式记作}A_{ij},\;\;则\;A_{11}+A_{21}+A_{31}+A_{41}=\left(\;\;\;\;\;\;\;\right).
$$
$$
A.
1 \quad B.0 \quad C.-1 \quad D.abcd \quad E. \quad F. \quad G. \quad H.
$$
$$
\begin{array}{l}\mathrm{由行列式按行}(列)\mathrm{展开定理可知}:\\A_{11}+A_{21}+A_{31}+A_{41}=\begin{vmatrix}1&b&c&d\\1&a&c&d\\1&a&c&b\\1&b&c&a\end{vmatrix},\\\mathrm{上述行列式的第}1,3\mathrm{列对应成比例},\mathrm{由行列式的性质可知此行列式的值为零}.\end{array}
$$



$$
设A_{ij}\left(i,j=1,2\right)\mathrm{为行列式}\begin{vmatrix}2&1\\3&1\end{vmatrix}\mathrm{中元素}a_{ij}\mathrm{的代数余子式},则\;\begin{vmatrix}A_{11}&A_{21}\\A_{12}&A_{22}\end{vmatrix}=\left(\;\;\;\;\;\right).
$$
$$
A.
1 \quad B.0 \quad C.-1 \quad D.2 \quad E. \quad F. \quad G. \quad H.
$$
$$
\begin{array}{l}\begin{array}{l}解1:\mathrm{根据行列式可分别计算出代数余子式}A_{11},A_{21},A_{12},A_{22}\mathrm{分别为}1,-1,-3,2,则\\\begin{vmatrix}A_{11}&A_{21}\\A_{12}&A_{22}\end{vmatrix}\;=\begin{vmatrix}1&-1\\-3&2\end{vmatrix}=2-3=-1.\end{array}\\解2:\begin{vmatrix}A_{11}&A_{21}\\A_{12}&A_{22}\end{vmatrix}=\vert A^*\vert=\vert A\vert^{2-1}=\vert A\vert=-1.\end{array}
$$



$$
\mathrm{设四阶行列式}\begin{vmatrix}3&0&4&0\\2&2&2&2\\0&-7&0&0\\5&3&-2&2\end{vmatrix},\mathrm{则第四行各元素代数余子式之和的值为}\left(\;\;\;\;\;\;\right).
$$
$$
A.
28 \quad B.-28 \quad C.-7 \quad D.0 \quad E. \quad F. \quad G. \quad H.
$$
$$
\begin{array}{l}\begin{array}{l}\mathrm{根据余子式和代数余子式的定义及行列式的展开定理有}:\\A_{41}+A_{42}+A_{43}+A_{44}\end{array}\\=\begin{vmatrix}3&0&4&0\\2&2&2&2\\0&-7&0&0\\1&1&1&1\end{vmatrix}=0.\\\\\end{array}
$$



$$
\mathrm{四阶行列式}\begin{vmatrix}3&0&4&0\\2&2&2&2\\0&-7&0&0\\5&3&-2&2\end{vmatrix},\mathrm{则第}4\mathrm{行各元素余子式之和为}\left(\;\;\;\;\;\;\;\;\;\right).
$$
$$
A.
28 \quad B.-28 \quad C.-21 \quad D.21 \quad E. \quad F. \quad G. \quad H.
$$
$$
\begin{array}{l}\mathrm{行列式的第}4\mathrm{行各元素余子式之和}\\M_{41}+M_{42}+M_{43}+M_{44}=-A_{41}+A_{42}-A_{43}+A_{44}=\begin{vmatrix}3&0&4&0\\2&2&2&2\\0&-7&0&0\\-1&1&-1&1\end{vmatrix}=-28.\end{array}
$$



$$
\begin{array}{l}\mathrm{已知四阶行列式}D\mathrm{中第}1\mathrm{行的元素分别为}1,2,0,-4,第3\mathrm{行的元素的余子式依次为}6,x,19,2,\\则x\mathrm{的值为}\left(\;\;\;\;\;\right).\;\end{array}
$$
$$
A.
7 \quad B.8 \quad C.0 \quad D.1 \quad E. \quad F. \quad G. \quad H.
$$
$$
\begin{array}{l}\mathrm{由题设知},a_{11},a_{12},a_{13},a_{14}\mathrm{分别为}1,2,0,-4;\\\;\;\;\;\;\;\;\;\;\;\;\;\;\;\;\;\;\;\;\;\;\;M_{31},M_{32},M_{33},M_{34},\;\mathrm{分别为}6,x,19,2,\;\\\mathrm{从而得}A_{31},A_{32},A_{33},A_{34}\;,\mathrm{分别为}6,-x,19,-2,\;\\\mathrm{由行列式按行}(列)\mathrm{展开定理},得\;\;\;\\\;\;\;\;\;\;\;\;\;\;\;\;\;\;\;\;\;\;a_{11}A_{31}+\;a_{12}A_{32}+\;a_{13}A_{33}+\;a_{14}A_{34}=0\\即\;\;\;\;\;1×6+2×\left(-x\right)\;\;+0×19+\left(-4\right)×\left(-2\right)=0\\\mathrm{所以}\;x=7.\end{array}
$$



$$
\mathrm{设四阶行列式}\begin{vmatrix}a&b&c&d\\1&-1&0&1\\2&0&2&-1\\3&1&3&-1\end{vmatrix},\mathrm{元素}a_{ij}\mathrm{的余子式记作}M_{ij}\;,\;\;则M_{11}+2M_{12}+3M_{13}+4M_{14}=(\;\;\;\;).
$$
$$
A.
1 \quad B.2 \quad C.3 \quad D.4 \quad E. \quad F. \quad G. \quad H.
$$
$$
\begin{array}{l}M_{11}+2M_{12}+3M_{13}+4M_{14}=A_{11}-2A_{12}+3A_{13}-4A_{14}=\begin{vmatrix}1&-2&3&-4\\1&-1&0&1\\2&0&2&-1\\3&1&3&-1\end{vmatrix}\\=\begin{vmatrix}1&-2&3&-4\\0&1&-3&5\\0&4&-4&7\\0&7&-6&11\end{vmatrix}=\begin{vmatrix}1&-2&3&-4\\0&1&-3&5\\0&0&8&-13\\0&0&15&-24\end{vmatrix}=3.\end{array}
$$



$$
\mathrm{设四阶行列式}\begin{vmatrix}3&-5&2&1\\1&1&0&-5\\-1&3&1&3\\2&-4&-1&-3\end{vmatrix},\mathrm{元素}a_{ij}\mathrm{的余子式记作}M_{ij},\;则M_{11}+M_{21}+M_{31}+M_{41}为\left(\;\;\;\;\;\right).
$$
$$
A.
4 \quad B.0 \quad C.-4 \quad D.-1 \quad E. \quad F. \quad G. \quad H.
$$
$$
\begin{array}{l}M_{11}+M_{21}+M_{31}+M_{41}=A_{11}-A_{21}+A_{31}-A_{41}=\begin{vmatrix}1&-5&2&1\\-1&1&0&-5\\1&3&1&3\\-1&-4&-1&-3\end{vmatrix}\\\;\;\;\;\;\;\;\;\;\;\;\;\;\;\overset{r_4+r_3}=\;\begin{vmatrix}1&-5&2&1\\-1&1&0&-5\\1&3&1&3\\0&-1&0&0\end{vmatrix}=\left(-1\right)\begin{vmatrix}1&2&1\\-1&0&-5\\1&1&3\end{vmatrix}\overset{r_1-2r_3}=-\begin{vmatrix}-1&0&-5\\-1&0&-5\\1&1&3\end{vmatrix}=0.\end{array}
$$



$$
n\mathrm{阶行列式}\;\;\begin{vmatrix}x&y&0&⋯&0&0\\0&x&y&⋯&0&0\\⋯&⋯&⋯&⋯&⋯&⋯\\0&0&0&⋯&x&y\\y&0&0&⋯&0&x\end{vmatrix}=\left(\;\;\;\;\;\;\;\;\right).
$$
$$
A.
x^n+y^n \quad B.x^n+\left(-1\right)^{n+1}y^n \quad C.x^n-y^n \quad D.x^n-\left(-1\right)^{n+1}y^n \quad E. \quad F. \quad G. \quad H.
$$
$$
\begin{array}{l}\mathrm{按第一列展开得},\mathrm{原式}=x\begin{vmatrix}x&y&⋯&0&0\\⋯&⋯&⋯&⋯&⋯\\0&0&⋯&x&y\\0&0&⋯&0&x\end{vmatrix}+\left(-1\right)^{n+1}y\begin{vmatrix}y&0&⋯&0&0\\x&y&⋯&0&0\\⋯&⋯&⋯&⋯&⋯\\0&0&⋯&x&y\end{vmatrix}\\\;\;\;\;\;\;\;\;=x^n+\left(-1\right)^{n+1}y^n.\end{array}
$$



$$
\mathrm{设四阶行列式}D=\begin{vmatrix}1&1&1&0\\4&3&-5&1\\-2&5&2&1\\3&-2&1&1\end{vmatrix},A_{i2}为D\mathrm{中元素}a_{i2}\left(i=1,2,3,4\right)\mathrm{的代数余子式},则∑_{i=1}^4A_{i2}=(\;\;\;\;\;).
$$
$$
A.
29 \quad B.28 \quad C.27 \quad D.26 \quad E. \quad F. \quad G. \quad H.
$$
$$
\begin{array}{l}∑_{i=1}^4A_{i2}=\begin{vmatrix}1&1&1&0\\4&1&-5&1\\-2&1&2&1\\3&1&1&1\end{vmatrix}=\begin{vmatrix}1&1&1&0\\4&0&-5&1\\-2&0&2&1\\3&0&1&1\end{vmatrix}\\\;\;=-\begin{vmatrix}4&-5&1\\-2&2&1\\3&1&1\end{vmatrix}=-\begin{vmatrix}1&-6&0\\-5&1&0\\3&1&1\end{vmatrix}=-\begin{vmatrix}1&-6\\-5&1\end{vmatrix}=29.\end{array}
$$



$$
\mathrm{四阶行列式}\begin{vmatrix}0&a&0&0\\b&c&0&0\\0&0&c&a\\0&0&0&a\end{vmatrix}\mathrm{的值为}\left(\;\;\;\right).
$$
$$
A.
abc \quad B.-abc \quad C.a^2bc \quad D.-a^2bc \quad E. \quad F. \quad G. \quad H.
$$
$$
\mathrm{把第一行展开},\mathrm{有原式}\begin{vmatrix}0&a&0&0\\b&c&0&0\\0&0&c&a\\0&0&0&a\end{vmatrix}=-a\begin{vmatrix}b&0&0\\0&c&a\\0&0&a\end{vmatrix}=-a^2bc.
$$



$$
若\begin{vmatrix}a_{11}&a_{12}\\a_{21}&a_{22}\end{vmatrix}=6,则\begin{vmatrix}a_{11}&a_{12}&0\\0&2&1\\a_{21}&a_{22}&0\end{vmatrix}\mathrm{的值为}\left(\;\;\;\right).
$$
$$
A.
-6 \quad B.6 \quad C.18 \quad D.0 \quad E. \quad F. \quad G. \quad H.
$$
$$
\begin{vmatrix}a_{11}&a_{12}&0\\0&2&1\\a_{21}&a_{22}&0\end{vmatrix}=1×\left(-1\right)^{2+3}\begin{vmatrix}a_{11}&a_{12}\\a_{21}&a_{22}\end{vmatrix}=-6.
$$



$$
\mathrm{四阶行列式}\begin{vmatrix}1&2&1&4\\0&-1&2&-2\\4&7&2&1\\0&2&1&3\end{vmatrix}\mathrm{的代数余子式}A_{14}=(\;\;\;\;\;).
$$
$$
A.
-20 \quad B.21 \quad C.-5 \quad D.6 \quad E. \quad F. \quad G. \quad H.
$$
$$
A_{14}=\left(-1\right)^{1+4}\begin{vmatrix}0&-1&2\\4&7&2\\0&2&1\end{vmatrix}=-20.
$$



$$
(n+1)\mathrm{阶行列式}\;\;\begin{vmatrix}-a_1&a_1&0&…&0&0\\0&-a_2&a_2&…&0&0\\…&…&…&…&…&…\\0&0&0&…&-a_n&a_n\\1&1&1&…&1&1\end{vmatrix}=\left(\;\;\;\;\;\;\right).
$$
$$
A.
\left(-1\right)^nn· a_1a_2⋯ a_n \quad B.\left(-1\right)^{n+1}\left(n+1\right)a_1a_2⋯ a_n \quad C.\left(-1\right)^{n+1}n· a_1a_2⋯ a_n \quad D.\left(-1\right)^n\left(n+1\right)a_1a_2⋯ a_n \quad E. \quad F. \quad G. \quad H.
$$
$$
\begin{array}{l}\mathrm{从第二列开始},\mathrm{都加到第一列上},\mathrm{原式}=\begin{vmatrix}0&a_1&0&…&0&0\\0&-a_2&a_2&…&0&0\\…&…&…&…&…&…\\0&0&0&…&-a_n&a_n\\n+1&1&1&…&1&1\end{vmatrix}\\\;\;\;\;\;\;\;\;\;\;=\left(-1\right)^{n+1+1}\;\left(n+1\right)\;\begin{vmatrix}a_1&0&…&0&0\\-a_2&a_2&…&0&0\\…&…&…&…&…\\0&0&…&-a_n&a_n\end{vmatrix}\\\;\;\;\;\;\;\;\;\;\;=\left(-1\right)^n\left(n+1\right)a_1a_2⋯ a_n.\end{array}
$$



$$
\mathrm{三阶行列式}\begin{vmatrix}2&-2&0\\2&-2&1\\1&3&-3\end{vmatrix}=\left(\;\;\;\;\right).\;
$$
$$
A.
8 \quad B.-8 \quad C.6 \quad D.-6 \quad E. \quad F. \quad G. \quad H.
$$
$$
\begin{vmatrix}2&-2&0\\2&-2&1\\1&3&-3\end{vmatrix}=\begin{vmatrix}2&-2&0\\0&0&1\\1&3&-3\end{vmatrix}=\left(-1\right)^{2+3}\begin{vmatrix}2&-2\\1&3\end{vmatrix}=-\left(6+2\right)=-8.
$$



$$
\mathrm{三阶行列式}\begin{vmatrix}1&3&2\\-1&0&3\\2&1&5\end{vmatrix}=\left(\;\;\;\;\;\;\right).
$$
$$
A.
28 \quad B.-28 \quad C.16 \quad D.-16 \quad E. \quad F. \quad G. \quad H.
$$
$$
\begin{vmatrix}1&3&2\\-1&0&3\\2&1&5\end{vmatrix}=\begin{vmatrix}1&3&5\\-1&0&0\\2&1&11\end{vmatrix}=\begin{vmatrix}3&5\\1&11\end{vmatrix}=28.
$$



$$
\mathrm{三阶行列式}\begin{vmatrix}2&1&4\\3&-1&2\\1&2&3\end{vmatrix}=\left(\;\;\;\;\;\;\right).
$$
$$
A.
6 \quad B.7 \quad C.-6 \quad D.-7 \quad E. \quad F. \quad G. \quad H.
$$
$$
\begin{vmatrix}2&1&4\\3&-1&2\\1&2&3\end{vmatrix}=\begin{vmatrix}0&-3&-2\\0&-7&-7\\1&2&3\end{vmatrix}=\begin{vmatrix}-3&-2\\-7&-7\end{vmatrix}=7.
$$



$$
\mathrm{四阶行列式}\begin{vmatrix}-1&2&1&6\\0&0&-1&0\\-1&5&2&15\\1&1&1&6\end{vmatrix}=(\;).
$$
$$
A.
9 \quad B.-9 \quad C.12 \quad D.-12 \quad E. \quad F. \quad G. \quad H.
$$
$$
\begin{vmatrix}-1&2&1&6\\0&0&-1&0\\-1&5&2&15\\1&1&1&6\end{vmatrix}=(-1)×(-1)^{3+2}\begin{vmatrix}-1&2&6\\-1&5&15\\1&1&6\end{vmatrix}=3\begin{vmatrix}-1&2&2\\-1&5&5\\1&1&2\end{vmatrix}\overset{\begin{array}{c}c_2+2c_1\\c_3+2c_1\end{array}}=3\begin{vmatrix}-1&0&0\\-1&3&3\\1&3&4\end{vmatrix}=-3\begin{vmatrix}3&3\\3&4\end{vmatrix}=-3×3=-9.
$$



$$
\mathrm{四阶行列式}\begin{vmatrix}a&3&0&5\\0&b&0&2\\1&2&c&3\\0&0&0&d\end{vmatrix}=\left(\;\;\;\;\right).\;
$$
$$
A.
abcd \quad B.5cd \quad C.0 \quad D.9abcd \quad E. \quad F. \quad G. \quad H.
$$
$$
\mathrm{原式}=d×(-1)^{4+4}\begin{vmatrix}a&3&0\\0&b&0\\1&2&c\end{vmatrix}=cd\begin{vmatrix}a&3\\0&b\end{vmatrix}=abcd.
$$



$$
\mathrm{三阶行列式}\begin{vmatrix}1&0&0\\0&2&3\\0&3&4\end{vmatrix}=\left(\;\;\;\right).
$$
$$
A.
0 \quad B.-1 \quad C.1 \quad D.-2 \quad E. \quad F. \quad G. \quad H.
$$
$$
\begin{vmatrix}1&0&0\\0&2&3\\0&3&4\end{vmatrix}=1×\begin{vmatrix}2&3\\3&4\end{vmatrix}=1×\left(-1\right)=-1.
$$



$$
\mathrm{三阶行列式}\begin{vmatrix}0&b&a\\1&a&b\\0&d&c\end{vmatrix}=\left(\;\;\;\;\right).
$$
$$
A.
0 \quad B.1 \quad C.ad-bc \quad D.bc-ad \quad E. \quad F. \quad G. \quad H.
$$
$$
\begin{vmatrix}0&b&a\\1&a&b\\0&d&c\end{vmatrix}=-\begin{vmatrix}b&a\\d&c\end{vmatrix}=ad-bc.
$$



$$
\mathrm{四阶行列式}\begin{vmatrix}1&0&2&a\\2&0&b&0\\3&c&4&5\\d&0&0&0\end{vmatrix}=(\;\;\;).\;
$$
$$
A.
abcd \quad B.1 \quad C.0 \quad D.-abcd \quad E. \quad F. \quad G. \quad H.
$$
$$
\mathrm{原式}=d×(-1)^{4+1}\begin{vmatrix}0&2&a\\0&b&0\\c&4&5\end{vmatrix}=-cd\begin{vmatrix}2&a\\b&0\end{vmatrix}=abcd.
$$



$$
\mathrm{已知三阶行列式}\begin{vmatrix}1&x&y\\x&1&0\\y&0&1\end{vmatrix}=1,x=\left(\;\;\;\;\right).
$$
$$
A.
0 \quad B.1 \quad C.-1 \quad D.2 \quad E. \quad F. \quad G. \quad H.
$$
$$
\mathrm{左边}=y\begin{vmatrix}x&y\\1&0\end{vmatrix}+\begin{vmatrix}1&x\\x&1\end{vmatrix}=-y^2-x^2+1,则x^2+y^2=0.\mathrm{从而}x=y=0.
$$



$$
\mathrm{四阶行列式}\begin{vmatrix}1&2&1&1\\0&2&1&11\\0&3&6&6\\0&0&-3&6\end{vmatrix}=(\;\;\;\;).
$$
$$
A.
-9 \quad B.9 \quad C.-19 \quad D.19 \quad E. \quad F. \quad G. \quad H.
$$
$$
\begin{vmatrix}1&2&1&1\\0&2&1&11\\0&3&6&6\\0&0&-3&6\end{vmatrix}=\begin{vmatrix}2&1&11\\3&6&6\\0&-3&6\end{vmatrix}=\begin{vmatrix}2&1&13\\3&6&18\\0&-3&0\end{vmatrix}=3\begin{vmatrix}2&13\\3&18\end{vmatrix}=3\left(36-39\right)=-9.
$$



$$
\mathrm{四阶行列式}\begin{vmatrix}3&1&-1&1\\-2&2&2&0\\2&0&1&0\\1&-5&3&0\end{vmatrix}=(\;\;\;\;).
$$
$$
A.
10 \quad B.20 \quad C.30 \quad D.40 \quad E. \quad F. \quad G. \quad H.
$$
$$
\begin{vmatrix}3&1&-1&1\\-2&2&2&0\\2&0&1&0\\1&-5&3&0\end{vmatrix}=-\begin{vmatrix}-2&2&2\\2&0&1\\1&-5&3\end{vmatrix}=-\begin{vmatrix}-2&0&0\\2&2&3\\1&-4&4\end{vmatrix}=40.
$$



$$
\mathrm{四阶行列式}\begin{vmatrix}2&-1&1&-2\\4&0&-1&4\\-1&2&1&0\\0&0&1&0\end{vmatrix}=(\;\;\;).
$$
$$
A.
26 \quad B.27 \quad C.28 \quad D.29 \quad E. \quad F. \quad G. \quad H.
$$
$$
\;\begin{vmatrix}2&-1&1&-2\\4&0&-1&4\\-1&2&1&0\\0&0&1&0\end{vmatrix}=-\begin{vmatrix}2&-1&-2\\4&0&4\\-1&2&0\end{vmatrix}=-\begin{vmatrix}4&-1&-2\\0&0&4\\-1&2&0\end{vmatrix}=4\begin{vmatrix}4&-1\\-1&2\end{vmatrix}=28.
$$



$$
\mathrm{四阶行列式}\;\begin{vmatrix}2&2&2&0\\1&0&3&1\\-1&-1&-2&0\\-1&2&-12&0\end{vmatrix}=(\;\;\;).
$$
$$
A.
6 \quad B.7 \quad C.8 \quad D.9 \quad E. \quad F. \quad G. \quad H.
$$
$$
\;\begin{vmatrix}2&2&2&0\\1&0&3&1\\-1&-1&-2&0\\-1&2&-12&0\end{vmatrix}=\begin{vmatrix}2&2&2\\-1&-1&-2\\-1&2&-12\end{vmatrix}=\begin{vmatrix}0&0&-2\\-1&-1&-2\\-1&2&-12\end{vmatrix}=-2\begin{vmatrix}-1&-1\\-1&2\end{vmatrix}=6.
$$



$$
\mathrm{四阶行列式}\;\begin{vmatrix}1&2&1&1\\0&0&0&3\\2&5&-5&24\\0&2&10&-7\end{vmatrix}=(\;\;\;).
$$
$$
A.
72 \quad B.73 \quad C.74 \quad D.75 \quad E. \quad F. \quad G. \quad H.
$$
$$
\;\begin{vmatrix}1&2&1&1\\0&0&0&3\\2&5&-5&24\\0&2&10&-7\end{vmatrix}=3\begin{vmatrix}1&2&1\\2&5&-5\\0&2&10\end{vmatrix}=3\begin{vmatrix}1&2&1\\0&1&-7\\0&2&10\end{vmatrix}=3\begin{vmatrix}1&-7\\2&10\end{vmatrix}=72.
$$



$$
\mathrm{四阶行列式}\;\begin{vmatrix}0&-1&0&0\\-3&2&2&-2\\2&-5&-2&-5\\-4&3&2&0\end{vmatrix}=\left(\;\;\;\right).
$$
$$
A.
15 \quad B.16 \quad C.17 \quad D.18 \quad E. \quad F. \quad G. \quad H.
$$
$$
\;\begin{vmatrix}0&-1&0&0\\-3&2&2&-2\\2&-5&-2&-5\\-4&3&2&0\end{vmatrix}=\begin{vmatrix}-3&2&-2\\2&-2&-5\\-4&2&0\end{vmatrix}=\begin{vmatrix}1&2&-2\\-2&-2&-5\\0&2&0\end{vmatrix}=-2\begin{vmatrix}1&-2\\-2&-5\end{vmatrix}=-2\left(-5-4\right)=18.
$$



$$
\mathrm{四阶行列式}\begin{vmatrix}-1&0&0&0\\-5&-7&-3&-3\\4&7&1&6\\7&13&3&9\end{vmatrix}=(\;\;\;\;).
$$
$$
A.
4 \quad B.5 \quad C.6 \quad D.7 \quad E. \quad F. \quad G. \quad H.
$$
$$
\;\begin{vmatrix}-1&0&0&0\\-5&-7&-3&-3\\4&7&1&6\\7&13&3&9\end{vmatrix}=-\begin{vmatrix}-7&-3&-3\\7&1&6\\13&3&9\end{vmatrix}=3\begin{vmatrix}-7&-3&1\\7&1&-2\\13&3&-3\end{vmatrix}=3\begin{vmatrix}0&0&1\\-7&-5&-2\\-8&-6&-3\end{vmatrix}=3\begin{vmatrix}7&5\\8&6\end{vmatrix}=3\left(42-40\right)=6.
$$



$$
\mathrm{四阶行列式}\begin{vmatrix}7&0&6&7\\2&1&4&6\\5&0&8&7\\4&0&5&6\end{vmatrix}=(\;\;\;\;).\;
$$
$$
A.
10 \quad B.20 \quad C.30 \quad D.40 \quad E. \quad F. \quad G. \quad H.
$$
$$
\begin{vmatrix}7&0&6&7\\2&1&4&6\\5&0&8&7\\4&0&5&6\end{vmatrix}=\begin{vmatrix}7&6&7\\5&8&7\\4&5&6\end{vmatrix}\overset{c_3-c_2}=\begin{vmatrix}7&6&1\\5&8&-1\\4&5&1\end{vmatrix}=\begin{vmatrix}12&14&0\\5&8&-1\\9&13&0\end{vmatrix}=\begin{vmatrix}12&14\\9&13\end{vmatrix}=\left(156-126\right)=30.
$$



$$
\mathrm{四阶行列式}\begin{vmatrix}2&-2&4&3\\3&1&7&5\\0&0&0&-1\\0&6&2&8\end{vmatrix}=(\;\;\;\;).
$$
$$
A.
1 \quad B.3 \quad C.2 \quad D.4 \quad E. \quad F. \quad G. \quad H.
$$
$$
\begin{vmatrix}2&-2&4&3\\3&1&7&5\\0&0&0&-1\\0&6&2&8\end{vmatrix}=\begin{vmatrix}2&-2&4\\3&1&7\\0&6&2\end{vmatrix}=\begin{vmatrix}2&-14&4\\3&-20&7\\0&0&2\end{vmatrix}=2\begin{vmatrix}2&-14\\3&-20\end{vmatrix}=3\left(-40+42\right)=4.
$$



$$
\mathrm{四阶行列式}\begin{vmatrix}0&6&3&7\\0&-1&2&-2\\0&4&3&5\\1&6&5&4\end{vmatrix}=(\;\;\;\;).
$$
$$
A.
-7 \quad B.-8 \quad C.-9 \quad D.-10 \quad E. \quad F. \quad G. \quad H.
$$
$$
\begin{vmatrix}0&6&3&7\\0&-1&2&-2\\0&4&3&5\\1&6&5&4\end{vmatrix}=-\begin{vmatrix}6&3&7\\-1&2&-2\\4&3&5\end{vmatrix}\overset{c_2+2c_1}{\underset{c_3-2c_1}=}\begin{vmatrix}6&15&-5\\-1&0&0\\4&11&-3\end{vmatrix}=-\begin{vmatrix}15&-5\\11&-3\end{vmatrix}=45-55=-10.
$$



$$
\mathrm{四阶行列式}\begin{vmatrix}-2&-1&8&4\\5&1&5&2\\-4&-1&3&7\\0&0&0&1\end{vmatrix}=(\;\;\;\;).
$$
$$
A.
11 \quad B.12 \quad C.13 \quad D.14 \quad E. \quad F. \quad G. \quad H.
$$
$$
\begin{vmatrix}-2&-1&8&4\\5&1&5&2\\-4&-1&3&7\\0&0&0&1\end{vmatrix}=\begin{vmatrix}-2&-1&8\\5&1&5\\-4&-1&3\end{vmatrix}=\begin{vmatrix}-2&-1&8\\3&0&13\\-2&0&-5\end{vmatrix}=\begin{vmatrix}3&13\\-2&-5\end{vmatrix}=11.
$$



$$
\mathrm{四阶行列式}\begin{vmatrix}4&1&12&-3\\3&1&8&-4\\-2&0&0&0\\2&5&3&-2\end{vmatrix}=(\;\;\;\;)
$$
$$
A.
216 \quad B.217 \quad C.218 \quad D.219 \quad E. \quad F. \quad G. \quad H.
$$
$$
\begin{vmatrix}4&1&12&-3\\3&1&8&-4\\-2&0&0&0\\2&5&3&-2\end{vmatrix}=-2\begin{vmatrix}1&12&-3\\1&8&-4\\5&3&-2\end{vmatrix}=218.
$$



$$
\mathrm{四阶行列式}\begin{vmatrix}6&8&1&0\\0&2&5&2\\7&8&1&0\\29&-1&-6&0\end{vmatrix}=(\;\;\;\;).
$$
$$
A.
97 \quad B.95 \quad C.94 \quad D.96 \quad E. \quad F. \quad G. \quad H.
$$
$$
\begin{vmatrix}6&8&1&0\\0&2&5&2\\7&8&1&0\\29&-1&-6&0\end{vmatrix}=2\begin{vmatrix}6&8&1\\7&8&1\\29&-1&-6\end{vmatrix}=2\begin{vmatrix}6&8&1\\1&0&0\\29&-1&-6\end{vmatrix}=-2\begin{vmatrix}8&1\\-1&-6\end{vmatrix}=94.
$$



$$
\mathrm{四阶行列式}\begin{vmatrix}27&44&40&1\\20&64&21&0\\67&60&67&0\\-6&10&-6&0\end{vmatrix}=(\;\;\;\;).
$$
$$
A.
0 \quad B.-1030 \quad C.1040 \quad D.-1050 \quad E. \quad F. \quad G. \quad H.
$$
$$
\begin{vmatrix}27&44&40&1\\20&64&21&0\\67&60&67&0\\-6&10&-6&0\end{vmatrix}=-\begin{vmatrix}20&64&21\\67&60&67\\-6&10&-6\end{vmatrix}\overset{c_3-c_1}=-\begin{vmatrix}20&64&1\\67&60&0\\-6&10&0\end{vmatrix}=-\begin{vmatrix}67&60\\-6&10\end{vmatrix}=-1030.
$$



$$
\mathrm{四阶行列式}\begin{vmatrix}1&2&3&4\\0&2&3&4\\0&0&2&3\\0&0&3&4\end{vmatrix}=(\;\;\;).\;
$$
$$
A.
0 \quad B.-1 \quad C.-2 \quad D.-3 \quad E. \quad F. \quad G. \quad H.
$$
$$
\begin{vmatrix}1&2&3&4\\0&2&3&4\\0&0&2&3\\0&0&3&4\end{vmatrix}=\begin{vmatrix}2&3&4\\0&2&3\\0&3&4\end{vmatrix}=2\begin{vmatrix}2&3\\3&4\end{vmatrix}=-2.
$$



$$
\mathrm{三阶行列式}\begin{vmatrix}3&5&8\\9&1&4\\2&0&0\end{vmatrix}=(\;\;\;).\;
$$
$$
A.
24 \quad B.25 \quad C.27 \quad D.20 \quad E. \quad F. \quad G. \quad H.
$$
$$
\begin{vmatrix}3&5&8\\9&1&4\\2&0&0\end{vmatrix}=2\begin{vmatrix}5&8\\1&4\end{vmatrix}=2×12=24.
$$



$$
\mathrm{四阶行列式}\begin{vmatrix}1&5&1&1\\0&2&0&0\\5&6&0&0\\6&7&8&9\end{vmatrix}=(\;\;\;\;\;\;).\;
$$
$$
A.
-9 \quad B.-10 \quad C.-12 \quad D.-16 \quad E. \quad F. \quad G. \quad H.
$$
$$
\begin{vmatrix}1&5&1&1\\0&2&0&0\\5&6&0&0\\6&7&8&9\end{vmatrix}=2\begin{vmatrix}1&1&1\\5&0&0\\6&8&9\end{vmatrix}=-2×5\begin{vmatrix}1&1\\8&9\end{vmatrix}=-10.
$$



$$
\mathrm{四阶行列式}\begin{vmatrix}4&4&5&5\\3&3&4&4\\2&0&0&3\\0&0&0&2\end{vmatrix}=(\;\;\;\;\;).\;
$$
$$
A.
1 \quad B.2 \quad C.3 \quad D.4 \quad E. \quad F. \quad G. \quad H.
$$
$$
\begin{vmatrix}4&4&5&5\\3&3&4&4\\2&0&0&3\\0&0&0&2\end{vmatrix}=2\begin{vmatrix}4&4&5\\3&3&4\\2&0&0\end{vmatrix}=4\begin{vmatrix}4&5\\3&4\end{vmatrix}=4.
$$



$$
\mathrm{四阶行列式}\begin{vmatrix}3&3&3&3\\3&2&2&3\\3&0&0&2\\2&0&0&0\end{vmatrix}=(\;\;\;).\;
$$
$$
A.
0 \quad B.1 \quad C.2 \quad D.3 \quad E. \quad F. \quad G. \quad H.
$$
$$
\begin{vmatrix}3&3&3&3\\3&2&2&3\\3&0&0&2\\2&0&0&0\end{vmatrix}=-2\begin{vmatrix}3&3&3\\2&2&3\\0&0&2\end{vmatrix}=-4\begin{vmatrix}3&3\\2&2\end{vmatrix}=0.
$$



$$
\mathrm{四阶行列式}\begin{vmatrix}a&1&0&0\\-1&b&1&0\\0&-1&0&1\\0&0&-1&0\end{vmatrix}=(\;\;\;\;).
$$
$$
A.
ab+1 \quad B.ab-1 \quad C.a+b+1 \quad D.a-b+1 \quad E. \quad F. \quad G. \quad H.
$$
$$
\begin{vmatrix}a&1&0&0\\-1&b&1&0\\0&-1&0&1\\0&0&-1&0\end{vmatrix}=-\begin{vmatrix}a&1&0\\-1&b&1\\0&0&-1\end{vmatrix}=\begin{vmatrix}a&1\\-1&b\end{vmatrix}=ab+1.
$$



$$
\mathrm{四阶行列式}\begin{vmatrix}1&2&3&4\\2&3&0&4\\3&0&0&4\\0&0&0&4\end{vmatrix}=(\;\;\;\;\;\;).\;
$$
$$
A.
100 \quad B.110 \quad C.-108 \quad D.109 \quad E. \quad F. \quad G. \quad H.
$$
$$
\mathrm{按第四行展开}\begin{vmatrix}1&2&3&4\\2&3&0&4\\3&0&0&4\\0&0&0&4\end{vmatrix}=4\begin{vmatrix}1&2&3\\2&3&0\\3&0&0\end{vmatrix}=12\begin{vmatrix}2&3\\3&0\end{vmatrix}=-108.
$$



$$
\mathrm{四阶行列式}\begin{vmatrix}2&2&0&0\\1&0&2&1\\1&0&3&2\\1&0&4&3\end{vmatrix}=(\;\;\;\;).\;
$$
$$
A.
0 \quad B.1 \quad C.2 \quad D.3 \quad E. \quad F. \quad G. \quad H.
$$
$$
\begin{vmatrix}2&2&0&0\\1&0&2&1\\1&0&3&2\\1&0&4&3\end{vmatrix}=-2\begin{vmatrix}1&2&1\\1&3&2\\1&4&3\end{vmatrix}=-2\begin{vmatrix}1&2&1\\0&1&1\\0&2&2\end{vmatrix}=0.
$$



$$
\mathrm{四阶行列式}\begin{vmatrix}3&2&2&2\\9&-8&5&8\\5&-8&5&8\\6&-5&4&7\end{vmatrix}=(\;\;\;).\;
$$
$$
A.
-96 \quad B.97 \quad C.-98 \quad D.99 \quad E. \quad F. \quad G. \quad H.
$$
$$
\mathrm{原式}\overset{r_2-r_3}=\begin{vmatrix}3&2&2&2\\4&0&0&0\\5&-8&5&8\\6&-5&4&7\end{vmatrix}=-4\begin{vmatrix}2&2&2\\-8&5&8\\-5&4&7\end{vmatrix}=-8\begin{vmatrix}1&1&1\\-8&5&8\\-5&4&7\end{vmatrix}=-8\begin{vmatrix}1&1&1\\0&13&16\\0&9&12\end{vmatrix}=-96.
$$



$$
\mathrm{四阶行列式}\begin{vmatrix}2&1&4&-1\\3&-1&2&-1\\1&2&3&-2\\5&0&6&-2\end{vmatrix}=\left(\;\;\;\;\;\;\right).\;
$$
$$
A.
1 \quad B.2 \quad C.0 \quad D.-1 \quad E. \quad F. \quad G. \quad H.
$$
$$
\begin{vmatrix}2&1&4&-1\\3&-1&2&-1\\1&2&3&-2\\5&0&6&-2\end{vmatrix}=\begin{vmatrix}2&1&4&-1\\5&0&6&-2\\-3&0&-5&0\\5&0&6&-2\end{vmatrix}=-1×\begin{vmatrix}5&6&-2\\-3&-5&0\\5&6&-2\end{vmatrix}=0.
$$



$$
\mathrm{四阶行列式}\begin{vmatrix}0&2&1&0\\3&3&2&4\\1&1&3&1\\3&1&4&4\end{vmatrix}=(\;\;\;\;\;).
$$
$$
A.
-4 \quad B.-5 \quad C.-6 \quad D.-7 \quad E. \quad F. \quad G. \quad H.
$$
$$
\begin{vmatrix}0&2&1&0\\3&3&2&4\\1&1&3&1\\3&1&4&4\end{vmatrix}\;\overset{c_2-2c_3}=\;\begin{vmatrix}0&0&1&0\\3&-1&2&4\\1&-5&3&1\\3&-7&4&4\end{vmatrix}=\begin{vmatrix}3&-1&4\\1&-5&1\\3&-7&4\end{vmatrix}\overset{r_3-r_1}=\begin{vmatrix}3&-1&4\\1&-5&1\\0&-6&0\end{vmatrix}=6\begin{vmatrix}3&4\\1&1\end{vmatrix}=-6.
$$



$$
\mathrm{四阶行列式}\begin{vmatrix}2&3&2&2\\3&6&8&3\\4&6&6&5\\1&1&1&1\end{vmatrix}=(\;\;\;).
$$
$$
A.
-5 \quad B.-6 \quad C.-7 \quad D.-8 \quad E. \quad F. \quad G. \quad H.
$$
$$
\begin{vmatrix}2&3&2&2\\3&6&8&3\\4&6&6&5\\1&1&1&1\end{vmatrix}\overset{\;r_1-2r_{4\;}}{\;=}\begin{vmatrix}0&1&0&0\\3&6&8&3\\4&6&6&5\\1&1&1&1\end{vmatrix}=-\begin{vmatrix}3&8&3\\4&6&5\\1&1&1\end{vmatrix}\overset{c_3-c_1}=-\begin{vmatrix}3&8&0\\4&6&1\\1&1&0\end{vmatrix}=\begin{vmatrix}3&8\\1&1\end{vmatrix}=-5.
$$



$$
\mathrm{四阶行列式}\begin{vmatrix}1&1&1&9\\1&1&9&1\\1&9&1&1\\9&1&1&1\end{vmatrix}=(\;\;\;\;).
$$
$$
A.
4554 \quad B.4664 \quad C.4774 \quad D.6144 \quad E. \quad F. \quad G. \quad H.
$$
$$
D=12\begin{vmatrix}1&1&1&9\\1&1&9&1\\1&9&1&1\\1&1&1&1\end{vmatrix}=12\begin{vmatrix}1&1&1&9\\0&0&8&-8\\0&8&0&-8\\0&0&0&-8\end{vmatrix}=12\begin{vmatrix}0&8&-8\\8&0&-8\\0&0&-8\end{vmatrix}=6144
$$



$$
\mathrm{四阶行列式}\begin{vmatrix}-1&1&2&3\\-1&2&4&6\\-11&1&-1&-8\\2&2&1&5\end{vmatrix}=(\;\;\;\;\;).
$$
$$
A.
28 \quad B.29 \quad C.30 \quad D.31 \quad E. \quad F. \quad G. \quad H.
$$
$$
\begin{vmatrix}-1&1&2&3\\-1&2&4&6\\-11&1&-1&-8\\2&2&1&5\end{vmatrix}\overset{r_2-2r_1}=\begin{vmatrix}-1&1&2&3\\1&0&0&0\\-11&1&-1&-8\\2&2&1&5\end{vmatrix}=-\begin{vmatrix}1&2&3\\1&-1&-8\\2&1&5\end{vmatrix}\overset{r_2-r_1}{\underset{r_3-2r_1}=}-\begin{vmatrix}1&2&3\\0&-3&-11\\0&-3&-1\end{vmatrix}=-\begin{vmatrix}3&11\\3&1\end{vmatrix}=30.
$$



$$
\mathrm{四阶行列式}\begin{vmatrix}1&1&1&1\\1&1&-1&-1\\1&-1&1&-1\\1&-1&-1&1\end{vmatrix}=(\;\;\;).
$$
$$
A.
-15 \quad B.-16 \quad C.-17 \quad D.-18 \quad E. \quad F. \quad G. \quad H.
$$
$$
\mathrm{原式}=\begin{vmatrix}1&1&1&1\\0&0&-2&-2\\0&-2&0&-2\\0&-2&-2&0\end{vmatrix}=\begin{vmatrix}0&-2&-2\\-2&0&-2\\-2&-2&0\end{vmatrix}=\begin{vmatrix}0&-2&-2\\-2&0&-2\\0&-2&2\end{vmatrix}=2\begin{vmatrix}-2&-2\\-2&2\end{vmatrix}=-16.
$$



$$
\mathrm{四阶行列式}\begin{vmatrix}1&2&3&4\\2&3&4&1\\3&4&1&2\\4&1&2&3\end{vmatrix}=(\;\;\;\;).\;
$$
$$
A.
150 \quad B.160 \quad C.170 \quad D.180 \quad E. \quad F. \quad G. \quad H.
$$
$$
\mathrm{原式}=\begin{vmatrix}10&2&3&4\\10&3&4&1\\10&4&1&2\\10&1&2&3\end{vmatrix}=10\begin{vmatrix}1&2&3&4\\0&1&1&-3\\0&2&-2&-2\\0&-1&-1&-1\end{vmatrix}=20\begin{vmatrix}1&1&-3\\1&-1&-1\\-1&-1&-1\end{vmatrix}=20\begin{vmatrix}1&1&-3\\0&-2&2\\0&0&-4\end{vmatrix}=20\begin{vmatrix}-2&2\\0&-4\end{vmatrix}=20×8=160.
$$



$$
\mathrm{四阶行列式}\begin{vmatrix}2&0&1&8\\1&2&1&0\\0&1&2&1\\0&0&1&2\end{vmatrix}=\left(\;\;\;\;\;\;\;\right).\;
$$
$$
A.
2 \quad B.6 \quad C.7 \quad D.8 \quad E. \quad F. \quad G. \quad H.
$$
$$
\begin{vmatrix}2&0&1&8\\1&2&1&0\\0&1&2&1\\0&0&1&2\end{vmatrix}=2\begin{vmatrix}2&1&0\\1&2&1\\0&1&2\end{vmatrix}-\begin{vmatrix}0&1&8\\1&2&1\\0&1&2\end{vmatrix}=2.
$$



$$
\mathrm{四阶行列式}\begin{vmatrix}1&0&-1&-1\\0&-1&-1&1\\a&b&0&0\\-1&-1&1&0\end{vmatrix}=(\;\;\;).
$$
$$
A.
3a-b \quad B.3a+b \quad C.2a-b \quad D.a-b \quad E. \quad F. \quad G. \quad H.
$$
$$
\begin{vmatrix}1&0&-1&-1\\0&-1&-1&1\\a&b&0&0\\-1&-1&1&0\end{vmatrix}=\begin{vmatrix}1&0&0&0\\0&-1&-1&1\\a&b&a&a\\-1&-1&0&-1\end{vmatrix}=\begin{vmatrix}-1&-1&1\\b&a&a\\-1&0&-1\end{vmatrix}=\begin{vmatrix}-2&-1&1\\b-a&a&a\\0&0&-1\end{vmatrix}=-\begin{vmatrix}-2&-1\\b-a&a\end{vmatrix}=3a-b.
$$



$$
\mathrm{四阶行列式}\begin{vmatrix}a&1&0&2\\b&0&3&1\\0&2&1&1\\0&1&1&1\end{vmatrix}=(\;\;\;).
$$
$$
A.
-2a-2b \quad B.2a-2b \quad C.a-2b \quad D.-2a+2b \quad E. \quad F. \quad G. \quad H.
$$
$$
\mathrm{原式}=\begin{vmatrix}0&1&0&0\\b&0&3&1\\-2a&2&1&-3\\-a&1&1&-1\end{vmatrix}=-\begin{vmatrix}b&3&1\\-2a&1&-3\\-a&1&-1\end{vmatrix}=-\begin{vmatrix}b+6a&0&10\\-2a&1&-3\\a&0&2\end{vmatrix}=-\begin{vmatrix}b+6a&10\\a&2\end{vmatrix}=-2a-2b.
$$



$$
\mathrm{四阶行列式}\begin{vmatrix}3&4&1&2\\0&3&0&1\\1&2&5&4\\4&7&-1&-2\end{vmatrix}=(\;\;\;\;).
$$
$$
A.
-154 \quad B.154 \quad C.155 \quad D.-155 \quad E. \quad F. \quad G. \quad H.
$$
$$
\begin{vmatrix}3&4&1&2\\0&3&0&1\\1&2&5&4\\4&7&-1&-2\end{vmatrix}=\begin{vmatrix}3&-2&1&2\\0&0&0&1\\1&-10&5&4\\4&13&-1&-2\end{vmatrix}=\begin{vmatrix}3&-2&1\\1&-10&5\\4&13&-1\end{vmatrix}=\begin{vmatrix}3&-2&1\\-14&0&0\\4&13&-1\end{vmatrix}=14\begin{vmatrix}-2&1\\13&-1\end{vmatrix}=-154.
$$



$$
\mathrm{四阶行列式}\begin{vmatrix}3&0&-2&1\\1&0&-1&0\\2&3&6&3\\4&5&-7&23\end{vmatrix}=(\;\;\;).
$$
$$
A.
103 \quad B.100 \quad C.110 \quad D.106 \quad E. \quad F. \quad G. \quad H.
$$
$$
\mathrm{原式}=\begin{vmatrix}3&0&1&1\\1&0&0&0\\2&3&8&3\\4&5&-3&23\end{vmatrix}=-\begin{vmatrix}0&1&1\\3&8&3\\5&-3&23\end{vmatrix}=-\begin{vmatrix}0&1&0\\3&8&-5\\5&-3&26\end{vmatrix}=103.
$$



$$
\mathrm{四阶行列式}\begin{vmatrix}1&0&0&2\\3&6&4&6\\2&2&3&3\\-1&1&4&7\end{vmatrix}=(\;\;\;).
$$
$$
A.
100 \quad B.110 \quad C.120 \quad D.130 \quad E. \quad F. \quad G. \quad H.
$$
$$
\mathrm{原式}=\begin{vmatrix}1&0&0&0\\3&6&4&0\\2&2&3&-1\\-1&1&4&9\end{vmatrix}=\begin{vmatrix}6&4&0\\2&3&-1\\1&4&9\end{vmatrix}=110.
$$



$$
\mathrm{五阶行列式}\begin{vmatrix}x&a&b&0&c\\0&y&0&0&d\\0&e&z&0&f\\g&h&k&u&l\\0&0&0&0&v\end{vmatrix}=(\;\;\;\;).
$$
$$
A.
xyzuv \quad B.1 \quad C.0 \quad D.xyz \quad E. \quad F. \quad G. \quad H.
$$
$$
\mathrm{原式}=v\begin{vmatrix}x&a&b&0\\0&y&0&0\\0&e&z&0\\g&h&k&u\end{vmatrix}=uv\begin{vmatrix}x&a&b\\0&y&0\\0&e&z\end{vmatrix}=xuv\begin{vmatrix}y&0\\e&z\end{vmatrix}=xyzuv.
$$



$$
\mathrm{四阶行列式}\begin{vmatrix}a&1&0&0\\-1&b&1&0\\0&-1&c&1\\0&0&-1&d\end{vmatrix}=(\;\;\;).
$$
$$
A.
abcd+cd+ab+ad+1 \quad B.abcd+cd+ab+ad \quad C.cd+ab+ad+1 \quad D.abcd+1 \quad E. \quad F. \quad G. \quad H.
$$
$$
\mathrm{按第一列展开},\mathrm{原式}=a\begin{vmatrix}b&1&0\\-1&c&1\\0&-1&d\end{vmatrix}+\begin{vmatrix}1&0&0\\-1&c&1\\0&-1&d\end{vmatrix}=abcd+cd+ab+ad+1.
$$



$$
\mathrm{四阶行列式}\begin{vmatrix}1&0&a&1\\0&-1&b&-1\\-1&-1&c&-1\\-1&1&d&0\end{vmatrix}=(\;\;\;\;).
$$
$$
A.
a+b+d \quad B.a+b+c+d \quad C.a-b+d \quad D.a+b-d \quad E. \quad F. \quad G. \quad H.
$$
$$
\begin{vmatrix}1&0&a&1\\0&-1&b&-1\\-1&-1&c&-1\\-1&1&d&0\end{vmatrix}=\begin{vmatrix}1&0&a&1\\0&-1&b&-1\\0&-1&a+c&0\\0&1&a+d&1\end{vmatrix}=\begin{vmatrix}-1&b&-1\\-1&a+c&0\\1&a+d&1\end{vmatrix}=\begin{vmatrix}-1&b&-1\\0&a-b+c&1\\0&a+b+d&0\end{vmatrix}=a+b+d.
$$



$$
\mathrm{四阶行列式}\begin{vmatrix}a_1&0&0&b_1\\0&a_2&b_2&0\\0&b_3&a_3&0\\b_4&0&0&a_4\end{vmatrix}=(\;\;\;\;).
$$
$$
A.
\left(a_2a_3-b_2b_3\right)\left(a_1a_4-b_1b_4\right) \quad B.\left(a_2a_3+b_2b_3\right)\left(a_1a_4-b_1b_4\right) \quad C.\left(a_2a_3-b_2b_3\right)\left(a_1a_4+b_1b_4\right) \quad D.\left(a_2a_3+b_2b_3\right)\left(a_1a_4+b_1b_4\right) \quad E. \quad F. \quad G. \quad H.
$$
$$
\begin{array}{l}\mathrm{按第一行展开},得\\\mathrm{原式}=a_1\begin{vmatrix}a_2&b_2&0\\b_3&a_3&0\\0&0&a_4\end{vmatrix}-b_1\begin{vmatrix}0&a_2&b_2\\0&b_3&a_3\\b_4&0&0\end{vmatrix}=a_1a_4\left(a_2a_3-b_2b_3\right)-b_1b_4\left(a_2a_3-b_2b_3\right)=\left(a_2a_3-b_2b_3\right)\left(a_1a_4-b_1b_4\right).\\\end{array}
$$



$$
\mathrm{四阶行列式}\begin{vmatrix}x&y&0&0\\0&x&y&0\\0&0&x&y\\y&0&0&x\end{vmatrix}=(\;\;\;\;\;).
$$
$$
A.
x^4-y^4 \quad B.x^4+y^4 \quad C.x^3-y^3 \quad D.x^3+y^3 \quad E. \quad F. \quad G. \quad H.
$$
$$
\mathrm{原式}=x\begin{vmatrix}x&y&0\\0&x&y\\0&0&x\end{vmatrix}-y\begin{vmatrix}y&0&0\\x&y&0\\0&x&y\end{vmatrix}=x^4-y^4.
$$



$$
\mathrm{四阶行列式}\begin{vmatrix}1&2&3&4\\1&0&1&2\\3&-1&-1&0\\1&2&0&-5\end{vmatrix}=(\;\;\;\;).
$$
$$
A.
-24 \quad B.24 \quad C.2 \quad D.4 \quad E. \quad F. \quad G. \quad H.
$$
$$
\begin{vmatrix}1&2&3&4\\1&0&1&2\\3&-1&-1&0\\1&2&0&-5\end{vmatrix}=\begin{vmatrix}1&2&3&4\\0&-2&-2&-2\\0&-7&-10&-12\\0&0&-3&-9\end{vmatrix}=-\begin{vmatrix}1&2&3&4\\0&2&2&2\\0&7&10&12\\0&0&3&9\end{vmatrix}=-\begin{vmatrix}2&2&2\\7&10&12\\0&3&9\end{vmatrix}=-\begin{vmatrix}2&2&-4\\7&10&-18\\0&3&0\end{vmatrix}=3\begin{vmatrix}2&-4\\7&-18\end{vmatrix}=3×\left(-8\right)=-24.
$$



$$
\mathrm{三阶行列式}\begin{vmatrix}a&b&a+b\\b&a+b&a\\a+b&a&b\end{vmatrix}=(\;\;\;\;).
$$
$$
A.
-2\left(a^3+b^3\right) \quad B.2\left(a^3+b^3\right) \quad C.-2\left(a^3-b^3\right) \quad D.2\left(a^3-b^3\right) \quad E. \quad F. \quad G. \quad H.
$$
$$
\begin{array}{l}\mathrm{原式}\overset{C_2,C_3\mathrm{加到}C_1}=2\left(a+b\right)\begin{vmatrix}1&b&a+b\\1&a+b&a\\1&a&b\end{vmatrix}\\=2\left(a+b\right)\begin{vmatrix}1&b&a+b\\0&a&-b\\0&a-b&-a\end{vmatrix}=2\left(a+b\right)\begin{vmatrix}a&-b\\a-b&-a\end{vmatrix}\\=2\left(a+b\right)\left(-a^2+ab-b^2\right)=-2\left(a^3+b^3\right).\end{array}
$$



$$
\mathrm{四阶行列式}\begin{vmatrix}1&2&-3&4\\2&3&-4&7\\-1&-2&5&-8\\1&3&-5&10\end{vmatrix}=(\;\;\;).
$$
$$
A.
10 \quad B.-10 \quad C.1 \quad D.-1 \quad E. \quad F. \quad G. \quad H.
$$
$$
\mathrm{原式}=\begin{vmatrix}1&2&-3&4\\0&-1&2&-1\\0&0&2&-4\\0&1&-2&6\end{vmatrix}=\begin{vmatrix}-1&2&-1\\0&2&-4\\1&-2&6\end{vmatrix}=\begin{vmatrix}-1&2&-1\\0&2&-4\\0&0&5\end{vmatrix}=-10.
$$



$$
\mathrm{三阶行列式}\begin{vmatrix}ab&2ac&ae\\bd&cd&-de\\3bf&cf&ef\end{vmatrix}=(\;\;\;\;\;).
$$
$$
A.
-8abcdef \quad B.-6abcdef \quad C.-abcdef \quad D.abcdef \quad E. \quad F. \quad G. \quad H.
$$
$$
\begin{vmatrix}ab&2ac&ae\\bd&cd&-de\\3bf&cf&ef\end{vmatrix}=adfbce\begin{vmatrix}1&2&1\\1&1&-1\\3&1&1\end{vmatrix}=adfbce\begin{vmatrix}1&2&1\\0&-1&-2\\0&-5&-2\end{vmatrix}=abcdef\begin{vmatrix}-1&-2\\-5&-2\end{vmatrix}=-8abcdef.
$$



$$
\mathrm{三阶行列式}\begin{vmatrix}a^2&ab&b^2\\2a&a+b&2b\\1&1&1\end{vmatrix}=(\;\;\;\;\;\;).
$$
$$
A.
\left(a-b\right)^3 \quad B.\left(a+b\right)^3 \quad C.\left(a-b\right)^2 \quad D.\left(a+b\right)^2 \quad E. \quad F. \quad G. \quad H.
$$
$$
\begin{vmatrix}a^2&ab&b^2\\2a&a+b&2b\\1&1&1\end{vmatrix}=\begin{vmatrix}a^2&ab-a^2&b^2-a^2\\2a&b-a&2b-2a\\1&0&0\end{vmatrix}=\left(b-a\right)\left(b-a\right)\begin{vmatrix}a&b+a\\1&2\end{vmatrix}=\left(a-b\right)^3.
$$



$$
\mathrm{四阶行列式}\begin{vmatrix}1&2&1&-2\\2&4&-1&4\\3&1&1&0\\-1&-4&-1&2\end{vmatrix}=\left(\;\;\;\;\right).
$$
$$
A.
4 \quad B.-4 \quad C.2 \quad D.-2 \quad E. \quad F. \quad G. \quad H.
$$
$$
\begin{vmatrix}1&2&1&-2\\2&4&-1&4\\3&1&1&0\\-1&-4&-1&2\end{vmatrix}=\begin{vmatrix}1&2&1&-2\\0&0&-3&8\\0&-5&-2&6\\0&-2&0&0\end{vmatrix}=\begin{vmatrix}0&-3&8\\-5&-2&6\\-2&0&0\end{vmatrix}=-2\begin{vmatrix}-3&8\\-2&6\end{vmatrix}=4.
$$



$$
\mathrm{四阶行列式}\begin{vmatrix}1&-1&1&-2\\2&0&-1&4\\3&2&1&0\\-1&2&-1&2\end{vmatrix}=\left(\;\;\;\;\;\right).
$$
$$
A.
2 \quad B.-2 \quad C.1 \quad D.-1 \quad E. \quad F. \quad G. \quad H.
$$
$$
\mathrm{把第一行加到第四行},\begin{vmatrix}1&-1&1&-2\\2&0&-1&4\\3&2&1&0\\-1&2&-1&2\end{vmatrix}=\begin{vmatrix}1&-1&1&-2\\2&0&-1&4\\3&2&1&0\\0&1&0&0\end{vmatrix}=\begin{vmatrix}1&1&-2\\2&-1&4\\3&1&0\end{vmatrix}=\begin{vmatrix}1&1&-2\\4&1&0\\3&1&0\end{vmatrix}=-2.
$$



$$
\mathrm{四阶行列式}\begin{vmatrix}2&1&-5&1\\1&-3&0&-6\\0&2&-1&2\\1&4&-7&6\end{vmatrix}=\left(\;\;\;\;\right).
$$
$$
A.
27 \quad B.-27 \quad C.99 \quad D.-99 \quad E. \quad F. \quad G. \quad H.
$$
$$
\begin{array}{l}\begin{vmatrix}2&1&-5&1\\1&-3&0&-6\\0&2&-1&2\\1&4&-7&6\end{vmatrix}=\begin{vmatrix}0&7&-5&13\\1&-3&0&-6\\0&2&-1&2\\0&7&-7&12\end{vmatrix}=-\begin{vmatrix}7&-5&13\\2&-1&2\\7&-7&12\end{vmatrix}=-\begin{vmatrix}2&-5&13\\1&-1&2\\0&7&12\end{vmatrix}=-\begin{vmatrix}0&-3&9\\1&-1&2\\0&-7&12\end{vmatrix}=\begin{vmatrix}-3&9\\-7&12\end{vmatrix}=27.\\\end{array}
$$



$$
\mathrm{四阶行列式}\begin{vmatrix}2&1&8&1\\1&-3&9&-6\\0&2&5&2\\1&4&0&6\end{vmatrix}=\left(\;\;\;\;\;\;\right).
$$
$$
A.
27 \quad B.-27 \quad C.43 \quad D.-43 \quad E. \quad F. \quad G. \quad H.
$$
$$
\begin{vmatrix}2&1&8&1\\1&-3&9&-6\\0&2&5&2\\1&4&0&6\end{vmatrix}=\begin{vmatrix}0&7&-10&13\\1&-3&9&-6\\0&2&5&2\\0&7&-9&12\end{vmatrix}=-\begin{vmatrix}7&-10&13\\2&5&2\\7&-9&12\end{vmatrix}=-\begin{vmatrix}7&-10&13\\2&5&2\\0&1&-1\end{vmatrix}=-\begin{vmatrix}7&3&13\\2&7&2\\0&0&-1\end{vmatrix}=43.
$$



$$
\mathrm{四阶行列式}\begin{vmatrix}2&3&5&8\\-1&0&2&3\\0&1&7&4\\4&1&-2&1\end{vmatrix}=\left(\;\;\;\right).
$$
$$
A.
106 \quad B.-106 \quad C.-110 \quad D.110 \quad E. \quad F. \quad G. \quad H.
$$
$$
\begin{array}{l}\begin{vmatrix}2&3&5&8\\-1&0&2&3\\0&1&7&4\\4&1&-2&1\end{vmatrix}=\begin{vmatrix}2&0&-16&-4\\-1&0&2&3\\0&1&7&4\\4&0&-9&-3\end{vmatrix}=-\begin{vmatrix}2&-16&-4\\-1&2&3\\4&-9&-3\end{vmatrix}=-\begin{vmatrix}0&-12&2\\-1&2&3\\0&-1&9\end{vmatrix}=-\begin{vmatrix}-12&2\\-1&9\end{vmatrix}=106.\\\end{array}
$$



$$
\mathrm{四阶行列式}\begin{vmatrix}1&2&1&4\\0&-1&2&-1\\4&7&2&1\\0&2&1&3\end{vmatrix}=\left(\;\;\;\;\;\right).
$$
$$
A.
66 \quad B.74 \quad C.-66 \quad D.-74 \quad E. \quad F. \quad G. \quad H.
$$
$$
\begin{array}{l}\mathrm{原式}=\begin{vmatrix}1&2&1&4\\0&-1&2&-1\\0&-1&-2&-15\\0&2&1&3\end{vmatrix}=\begin{vmatrix}-1&2&-1\\-1&-2&-15\\2&1&3\end{vmatrix}=\begin{vmatrix}-1&2&-1\\0&-4&-14\\0&5&1\end{vmatrix}=-\begin{vmatrix}-4&-14\\5&1\end{vmatrix}=-66.\\\end{array}
$$



$$
\mathrm{五阶行列式}\begin{vmatrix}0&0&0&1&0\\0&0&2&0&0\\0&3&10&0&0\\4&11&0&12&0\\9&8&7&6&5\end{vmatrix}=\left(\;\;\;\right).
$$
$$
A.
120 \quad B.-120 \quad C.216 \quad D.-216 \quad E. \quad F. \quad G. \quad H.
$$
$$
\mathrm{按最后一列展开降阶},\mathrm{原式}=5×\begin{vmatrix}0&0&0&1\\0&0&2&0\\0&3&10&0\\4&11&0&12\end{vmatrix}=120.
$$



$$
\mathrm{四阶行列式}\begin{vmatrix}1&1&1&1\\2&1&3&4\\3&1&6&10\\4&1&10&20\end{vmatrix}=\left(\;\;\;\;\;\right).
$$
$$
A.
1 \quad B.-1 \quad C.2 \quad D.0 \quad E. \quad F. \quad G. \quad H.
$$
$$
\mathrm{原式}=\begin{vmatrix}1&1&1&1\\1&0&2&3\\2&0&5&9\\3&0&9&19\end{vmatrix}=-\begin{vmatrix}1&2&3\\2&5&9\\3&9&19\end{vmatrix}=-\begin{vmatrix}1&2&3\\0&1&3\\0&3&10\end{vmatrix}=-\begin{vmatrix}1&3\\3&10\end{vmatrix}=-1.
$$



$$
\mathrm{四阶行列式}\begin{vmatrix}1&-2&3&4\\1&0&1&2\\3&-1&-1&0\\1&2&0&-5\end{vmatrix}=\left(\;\;\;\;\;\right).
$$
$$
A.
64 \quad B.-44 \quad C.20 \quad D.-20 \quad E. \quad F. \quad G. \quad H.
$$
$$
\begin{array}{l}\begin{vmatrix}1&-2&3&4\\1&0&1&2\\3&-1&-1&0\\1&2&0&-5\end{vmatrix}=\begin{vmatrix}-5&0&5&4\\1&0&1&2\\3&-1&-1&0\\7&0&-2&-5\end{vmatrix}\\=\left(-1\right)×\left(-1\right)^{3+2}\begin{vmatrix}-5&5&4\\1&1&2\\7&-2&-5\end{vmatrix}=\begin{vmatrix}-10&0&-6\\1&1&2\\9&0&-1\end{vmatrix}\\=1×\left(-1\right)^{2+2}\begin{vmatrix}-10&-6\\9&-1\end{vmatrix}=64.\\\end{array}
$$



$$
\mathrm{五阶行列式}\begin{vmatrix}5&3&-1&2&0\\1&7&2&5&2\\0&-2&3&1&0\\0&-4&-1&4&0\\0&2&3&5&0\end{vmatrix}=\left(\;\;\;\;\right).
$$
$$
A.
-1080 \quad B.1080 \quad C.-108 \quad D.108 \quad E. \quad F. \quad G. \quad H.
$$
$$
\begin{array}{l}\begin{vmatrix}5&3&-1&2&0\\1&7&2&5&2\\0&-2&3&1&0\\0&-4&-1&4&0\\0&2&3&5&0\end{vmatrix}=\left(-1\right)^{2+5}·2\begin{vmatrix}5&3&-1&2\\0&-2&3&1\\0&-4&-1&4\\0&2&3&5\end{vmatrix}=-2·5\begin{vmatrix}-2&3&1\\-4&-1&4\\2&3&5\end{vmatrix}\\\;\;\;\;\;\;\;\;\;\;\;\;\;\;\;\;\;\;\;\;\;\;\;\;\;\;\;\;\;\;\;\;\;\;\;\overset{r_2-2r_1}{\underset{r_3+r_1}=}-10\begin{vmatrix}-2&3&1\\0&-7&2\\0&6&6\end{vmatrix}=-10·\left(-2\right)\begin{vmatrix}-7&2\\6&6\end{vmatrix}\\\;\;\;\;\;\;\;\;\;\;\;\;\;\;\;\;\;\;\;\;\;\;\;\;\;\;\;\;\;\;\;\;\;\;\;\;\;\;=20\left(-42-12\right)=-1080.\end{array}
$$



$$
\mathrm{四阶行列式}\begin{vmatrix}0&a&b&a\\a&0&a&b\\b&a&0&a\\a&b&a&0\end{vmatrix}=\left(\;\;\;\;\right).
$$
$$
A.
b^2\left(b^2-4a^2\right) \quad B.b^2\left(b^2+4a^2\right) \quad C.a^2\left(b^2-4a^2\right) \quad D.a^2\left(b^2+4a^2\right) \quad E. \quad F. \quad G. \quad H.
$$
$$
\begin{array}{l}将c_2,c_3,c_4\mathrm{都加到}c_1,得\;\\\;\mathrm{原式}=\begin{vmatrix}2a+b&a&b&a\\2a+b&0&a&b\\2a+b&a&0&a\\2a+b&b&a&0\end{vmatrix}=\left(2a+b\right)\begin{vmatrix}1&a&b&a\\1&0&a&b\\1&a&0&a\\1&b&a&0\end{vmatrix}\\=\left(2a+b\right)\begin{vmatrix}1&a&b&a\\0&-a&a-b&b-a\\0&0&-b&0\\0&b\left(-b\right)-a&a-b&-a\end{vmatrix}=\left(2a+b\right)\begin{vmatrix}-a&a-b&b-a\\0&-b&0\\b-a&a-b&-a\end{vmatrix}\\=\left(2a+b\right)\left(-b\right)\begin{vmatrix}-a&b-a\\b-a&-a\end{vmatrix}=b^2\left(b^2-4a^2\right).\end{array}
$$



$$
\mathrm{方程}\begin{vmatrix}x&-2&2&2\\x&x&1&1\\4&2&x&2\\0&0&1&1\end{vmatrix}=0\mathrm{的根是}(\;\;\;\;).
$$
$$
A.
0 \quad B.-2,0 \quad C.-2,0,2 \quad D.0,2 \quad E. \quad F. \quad G. \quad H.
$$
$$
\begin{array}{l}\mathrm{第三列减去第四列},得\\\begin{vmatrix}x&-2&2&2\\x&x&1&1\\4&2&x&2\\0&0&1&1\end{vmatrix}=\begin{vmatrix}x&-2&0&2\\x&x&0&1\\4&2&x-2&2\\0&0&0&1\end{vmatrix}=\begin{vmatrix}x&-2&0\\x&x&0\\4&2&x-2\end{vmatrix}=\left(x-2\right)\begin{vmatrix}x&-2\\x&x\end{vmatrix}=x\left(x+2\right)\left(x-2\right)=0.\mathrm{所以}x_1=0,\;x_2=2,\;x_1=-2.\end{array}
$$



$$
\mathrm{五阶行列式}\begin{vmatrix}a&0&0&0&c\\b&a&0&0&0\\0&b&a&0&0\\0&0&b&a&0\\0&0&0&b&a\end{vmatrix}=(\;\;\;\;).
$$
$$
A.
a^5+b^4c \quad B.a^4+b^4c \quad C.a^5-b^4c \quad D.a^4-b^4c \quad E. \quad F. \quad G. \quad H.
$$
$$
\begin{array}{l}\mathrm{将行列式按第一行展开得}\\\begin{vmatrix}a&0&0&0&c\\b&a&0&0&0\\0&b&a&0&0\\0&0&b&a&0\\0&0&0&b&a\end{vmatrix}=a\begin{vmatrix}a&&&\\b&a&&\\&b&a&\\&&b&a\end{vmatrix}+\left(-1\right)^{1+5}c\begin{vmatrix}b&a&&\\&b&a&\\&&b&a\\&&&b\end{vmatrix}=a^5+b^4c.\end{array}
$$



$$
\mathrm{四阶行列式}\begin{vmatrix}3&2&1&4\\15&29&2&14\\16&19&3&17\\33&39&8&38\end{vmatrix}=\left(\;\;\;\;\right).
$$
$$
A.
6 \quad B.-6 \quad C.-2 \quad D.2 \quad E. \quad F. \quad G. \quad H.
$$
$$
\begin{array}{l}\mathrm{原式}=\begin{vmatrix}3&2&1&4\\9&25&0&6\\7&13&0&5\\9&23&0&6\end{vmatrix}=\begin{vmatrix}9&25&6\\7&13&5\\9&23&6\end{vmatrix}=\begin{vmatrix}0&2&0\\7&13&5\\9&23&6\end{vmatrix}=-2\begin{vmatrix}7&5\\9&6\end{vmatrix}=6.\\\end{array}
$$



$$
\mathrm{四阶行列式}\begin{vmatrix}4&1&2&4\\1&2&0&2\\10&5&2&0\\0&1&1&8\end{vmatrix}=(\;\;\;\;).
$$
$$
A.
-16 \quad B.16 \quad C.2 \quad D.12 \quad E. \quad F. \quad G. \quad H.
$$
$$
\mathrm{原式}=\begin{vmatrix}4&-7&2&-4\\1&0&0&0\\10&-15&2&-20\\0&1&1&8\end{vmatrix}=-\begin{vmatrix}-7&2&-4\\-15&2&-20\\1&1&8\end{vmatrix}=-\begin{vmatrix}-9&0&-20\\-17&0&-36\\1&1&7\end{vmatrix}=\begin{vmatrix}-9&-20\\-17&-36\end{vmatrix}=-16.
$$



$$
\mathrm{四阶行列式}\begin{vmatrix}3&-5&2&-1\\-3&4&-5&0\\-5&7&-7&0\\8&-8&5&2\end{vmatrix}=(\;\;\;).
$$
$$
A.
16 \quad B.17 \quad C.18 \quad D.19 \quad E. \quad F. \quad G. \quad H.
$$
$$
\begin{array}{l}\begin{vmatrix}3&-5&2&-1\\-3&4&-5&0\\-5&7&-7&0\\8&-8&5&2\end{vmatrix}=\begin{vmatrix}3&-5&2&-1\\-3&4&-5&0\\-5&7&-7&0\\14&-18&9&0\end{vmatrix}=\begin{vmatrix}-3&4&-5\\-5&7&-7\\14&-18&9\end{vmatrix}\\=\begin{vmatrix}-3&-1&-5\\-5&0&-7\\14&-9&9\end{vmatrix}=\begin{vmatrix}-3&-1&-5\\-5&0&-7\\41&0&54\end{vmatrix}=\begin{vmatrix}-5&-7\\41&54\end{vmatrix}=-270+287=17.\end{array}
$$



$$
\mathrm{四阶行列式}\begin{vmatrix}a&b&0&0\\-b&a&0&0\\0&0&a&-b\\0&0&b&a\end{vmatrix}=\left(\;\;\;\;\;\right).
$$
$$
A.
\left(a^2+b^2\right)^2 \quad B.\left(a^2+b^2\right)^3 \quad C.\left(a^2-b^2\right)^2 \quad D.a^2+b^2 \quad E. \quad F. \quad G. \quad H.
$$
$$
\mathrm{按第一列展开},\mathrm{原式}=a\begin{vmatrix}a&0&0\\0&a&-b\\0&b&a\end{vmatrix}+(-b)×(-1)^{2+1}\begin{vmatrix}b&0&0\\0&a&-b\\0&b&a\end{vmatrix}=\left(a^2+b^2\right)^2.
$$



$$
\mathrm{多项式}f\left(λ\right)=\begin{vmatrix}λ&-1&-1&1\\-1&λ&1&-1\\-1&1&λ&-1\\1&-1&-1&λ\end{vmatrix}=(\;\;\;\;).
$$
$$
A.
\left(λ-1\right)^3\left(λ+3\right) \quad B.\left(λ-1\right)^2\left(λ+3\right) \quad C.\left(λ-1\right)\left(λ+3\right) \quad D.\left(λ+1\right)^3\left(λ+3\right) \quad E. \quad F. \quad G. \quad H.
$$
$$
\begin{array}{l}\mathrm{将行列式的}2,3,4\mathrm{列都加到第}1列,\mathrm{然后提出公因式},得\\f\left(λ\right)=\left(λ-1\right)\begin{vmatrix}1&-1&-1&1\\1&λ&1&-1\\1&1&λ&-1\\1&-1&-1&λ\end{vmatrix}=\left(λ-1\right)\begin{vmatrix}1&-1&-1&1\\0&λ+1&2&-2\\0&2&λ+1&-2\\0&0&0&λ-1\end{vmatrix}=\left(λ-1\right)\begin{vmatrix}λ+1&2&-2\\2&λ+1&-2\\0&0&λ-1\end{vmatrix}\\=\left(λ-1\right)^2\begin{vmatrix}λ+1&2\\2&λ+1\end{vmatrix}=\left(λ-1\right)^3\left(λ+3\right).\end{array}
$$



$$
(n+1)\mathrm{阶行列式}D_{n+1}=\begin{vmatrix}a_0&a_1&a_2&⋯&a_n\\-x&x&0&⋯&0\\0&-x&x&⋯&0\\\vdots&\vdots&\vdots&&\vdots\\0&0&0&⋯&x\end{vmatrix}=(\;\;\;\;\;).
$$
$$
A.
\left(a_0+a_1+⋯+a_n\right)x^n\;\; \quad B.\left(a_0+a_1+⋯+a_n\right)x \quad C.x^n\;\; \quad D.\left(a_0+a_1+⋯+a_n\right)x^{n+1}\;\; \quad E. \quad F. \quad G. \quad H.
$$
$$
\begin{array}{l}D_{n+1}\overset{c_2,⋯,c_n}{\underset{\mathrm{都加到}c_1}=}\begin{vmatrix}a_0+a_1+⋯ a_n&a_1&a_2&⋯&a_n\\0&x&0&⋯&0\\0&-x&x&⋯&0\\\vdots&\vdots&\vdots&&\vdots\\0&0&0&⋯&x\end{vmatrix}=\left(a_0+a_1+⋯+a_n\right)\begin{vmatrix}x&0&0&⋯&0\\-x&x&0&⋯&0\\0&-x&x&⋯&0\\\vdots&\vdots&\vdots&&\vdots\\0&0&0&⋯&x\end{vmatrix}\\\;\;\;\;\;\;\;\;\;\;\;\;\;\;\;\;\;\;\;\;\;\;\;\;\;\;\;\;\;\;\;\;\;\;\;\;\;\;\;\;\;\;\;\;\;\;\;\;\;\;\;\;\;\;=\left(a_0+a_1+⋯+a_n\right)x^n\;.\;\;\;\;\;\;\;\;\;\;\;\;\;\;\;\;\;\;\;\;\;\;\;\;\;\;\;\;\;\;\;\;\;\;\;\;\;\;\end{array}
$$



$$
n\mathrm{阶行列式}\begin{vmatrix}1&y&0&⋯&0&0\\0&1&y&⋯&0&0\\⋯&⋯&⋯&⋯&⋯&⋯\\0&0&0&⋯&1&y\\y&0&0&⋯&0&1\end{vmatrix}=(\;\;\;\;).\;
$$
$$
A.
1+\left(-1\right)^{n+1}y^n \quad B.1-\left(-1\right)^{n+1}y^n \quad C.1+y^n \quad D.1-y^n \quad E. \quad F. \quad G. \quad H.
$$
$$
\begin{array}{l}\mathrm{原式}=\begin{vmatrix}1&y&⋯&0&0\\0&1&⋯&0&0\\⋯&⋯&⋯&⋯&⋯\\0&0&⋯&1&y\\0&0&⋯&0&1\end{vmatrix}+\left(-1\right)^{n+1}y\begin{vmatrix}y&0&⋯&0&0\\1&y&⋯&0&0\\⋯&⋯&⋯&⋯&⋯\\0&0&⋯&1&y\end{vmatrix}\\\;\;\;\;\;\;\;\;\;\;\;\;\;\;\;\;\;\;\;\;\;\;\;\;=1+\left(-1\right)^{n+1}y^n.\end{array}
$$



$$
n\mathrm{阶行列式}\begin{vmatrix}1&2&0&⋯&0&0\\0&1&2&⋯&0&0\\\vdots&\vdots&\vdots&&\vdots&\vdots\\0&0&0&⋯&1&2\\2&0&0&⋯&0&1\end{vmatrix}=(\;\;\;\;\;).
$$
$$
A.
1+\left(-1\right)^{n+1}2^n \quad B.1-\left(-1\right)^{n+1}2^n \quad C.1+2^n \quad D.\left(-1\right)^{n+1}2^n \quad E. \quad F. \quad G. \quad H.
$$
$$
\mathrm{原式}=\begin{vmatrix}1&2&⋯&0&0\\\vdots&\vdots&&\vdots&\vdots\\0&0&⋯&1&2\\0&0&⋯&0&1\end{vmatrix}+\left(-1\right)^{n+1}2\begin{vmatrix}2&0&⋯&0&0\\1&2&⋯&0&0\\\vdots&\vdots&&\vdots&\vdots\\0&0&⋯&1&2\end{vmatrix}=1+\left(-1\right)^{n+1}2^n.
$$



$$
\mathrm{四阶行列式}\begin{vmatrix}1&1&0&0\\1&a^2+1&ab&ac\\1&ab&b^2+1&bc\\1&ac&bc&c^2+1\end{vmatrix}=(\;\;\;\;).
$$
$$
A.
a\left(a+b+c\right) \quad B.a+b+c \quad C.a\left(a-b+c\right) \quad D.a\left(a+b-c\right) \quad E. \quad F. \quad G. \quad H.
$$
$$
\begin{array}{l}\begin{vmatrix}1&1&0&0\\1&a^2+1&ab&ac\\1&ab&b^2+1&bc\\1&ac&bc&c^2+1\end{vmatrix}=\begin{vmatrix}1&0&0&0\\1&a^2&ab&ac\\1&ab-1&b^2+1&bc\\1&ac-1&bc&c^2+1\end{vmatrix}=\begin{vmatrix}a^2&ab&ac\\ab-1&b^2+1&bc\\ac-1&bc&c^2+1\end{vmatrix}\\=a\begin{vmatrix}a&b&c\\ab-1&b^2+1&bc\\ac-1&bc&c^2+1\end{vmatrix}\\=a\begin{vmatrix}a&b&c\\-1&1&0\\-1&0&1\end{vmatrix}=a\left(a+b+c\right).\end{array}
$$



$$
\mathrm{四阶行列式}\begin{vmatrix}1&1&0&0\\1&a^2+1&a&a\\1&a&2&1\\1&a&1&2\end{vmatrix}=(\;\;\;\;).
$$
$$
A.
a\left(a+2\right) \quad B.\left(a+2\right) \quad C.a \quad D.a\left(a+1\right) \quad E. \quad F. \quad G. \quad H.
$$
$$
\begin{array}{l}\begin{vmatrix}1&1&0&0\\1&a^2+1&a&a\\1&a&2&1\\1&a&1&2\end{vmatrix}=\begin{vmatrix}1&0&0&0\\1&a^2&a&a\\1&a-1&2&1\\1&a-1&1&2\end{vmatrix}=\begin{vmatrix}a^2&a&a\\a-1&2&1\\a-1&1&2\end{vmatrix}\\=a\begin{vmatrix}a&1&1\\a-1&2&1\\a-1&1&2\end{vmatrix}\\=a\begin{vmatrix}a&1&1\\-1&1&0\\-1&0&1\end{vmatrix}=a\left(a+2\right).\end{array}
$$



$$
\mathrm{四阶行列式}\begin{vmatrix}0&1&b&1\\1&0&1&b\\b&1&0&1\\1&b&1&0\end{vmatrix}=\left(\;\;\;\;\right).
$$
$$
A.
b^2\left(b^2-4\right) \quad B.b^2\left(b^2+4\right) \quad C.b^2-4 \quad D.b^2+4 \quad E. \quad F. \quad G. \quad H.
$$
$$
\begin{array}{l}将c_2,c_3,c_4\mathrm{都加到}c_1,得\;\\\;\mathrm{原式}=\begin{vmatrix}2+b&1&b&1\\2+b&0&1&b\\2+b&1&0&1\\2+b&b&1&0\end{vmatrix}=\left(2+b\right)\begin{vmatrix}1&1&b&1\\1&0&1&b\\1&1&0&1\\1&b&1&0\end{vmatrix}\\=\left(2+b\right)\begin{vmatrix}1&1&b&1\\0&-1&1-b&b-1\\0&0&-b&0\\0&b-1&1-b&-1\end{vmatrix}=\left(2+b\right)\begin{vmatrix}-1&1-b&b-1\\0&-b&0\\b-1&1-b&-1\end{vmatrix}\\=\left(2+b\right)\left(-b\right)\begin{vmatrix}-1&b-1\\b-1&-1\end{vmatrix}=b^2\left(b^2-4\right).\end{array}
$$



$$
\mathrm{五阶行列式}\begin{vmatrix}2&0&0&0&c\\b&2&0&0&0\\0&b&2&0&0\\0&0&b&2&0\\0&0&0&b&2\end{vmatrix}=(\;\;\;\;).
$$
$$
A.
2^5+b^4c \quad B.2^5+b^3c \quad C.2^5-b^3c \quad D.2^5-b^4 \quad E. \quad F. \quad G. \quad H.
$$
$$
\begin{array}{l}\mathrm{将行列式按第一行展开得}\\\begin{vmatrix}2&0&0&0&c\\b&2&0&0&0\\0&b&2&0&0\\0&0&b&2&0\\0&0&0&b&2\end{vmatrix}=2×\begin{vmatrix}2&&&\\b&2&&\\&b&2&\\&&b&2\end{vmatrix}+\left(-1\right)^{1+5}c\begin{vmatrix}b&2&&\\&b&2&\\&&b&2\\&&&b\end{vmatrix}=2^5+b^4c.\end{array}
$$



$$
\mathrm{三阶行列式}\begin{vmatrix}a^2&a&1\\2a&a+1&2\\1&1&1\end{vmatrix}=(\;\;\;\;\;\;).
$$
$$
A.
\left(a-1\right)^3 \quad B.\left(a+1\right)^3 \quad C.\left(a-1\right)^2 \quad D.\left(a+1\right)^2 \quad E. \quad F. \quad G. \quad H.
$$
$$
\begin{vmatrix}a^2&a&1\\2a&a+1&2\\1&1&1\end{vmatrix}=\begin{vmatrix}a^2&a-a^2&1-a^2\\2a&1-a&2-2a\\1&0&0\end{vmatrix}=\left(1-a\right)\left(1-a\right)\begin{vmatrix}a&1+a\\1&2\end{vmatrix}=\left(a-1\right)^3.
$$



$$
\mathrm{四阶行列式}\begin{vmatrix}1&2&1&1\\0&2&1&11\\0&3&6&6\\1&0&0&0\end{vmatrix}=(\;\;\;\;).
$$
$$
A.
90 \quad B.9 \quad C.-19 \quad D.19 \quad E. \quad F. \quad G. \quad H.
$$
$$
\begin{vmatrix}1&2&1&1\\0&2&1&11\\0&3&6&6\\1&0&0&0\end{vmatrix}=-\begin{vmatrix}2&1&1\\2&1&11\\3&6&6\end{vmatrix}=-\begin{vmatrix}2&1&1\\0&0&10\\3&6&6\end{vmatrix}=10\begin{vmatrix}2&1\\3&6\end{vmatrix}=90.
$$



$$
\mathrm{四阶行列式}\begin{vmatrix}-2&-1&8&9\\5&1&5&9\\-4&-1&3&9\\0&0&0&3\end{vmatrix}=(\;\;\;\;).
$$
$$
A.
33 \quad B.36 \quad C.39 \quad D.30 \quad E. \quad F. \quad G. \quad H.
$$
$$
\begin{vmatrix}-2&-1&8&9\\5&1&5&9\\-4&-1&3&9\\0&0&0&3\end{vmatrix}=3\begin{vmatrix}-2&-1&8\\5&1&5\\-4&-1&3\end{vmatrix}=3\begin{vmatrix}-2&-1&8\\3&0&13\\-2&0&-5\end{vmatrix}=3\begin{vmatrix}3&13\\-2&-5\end{vmatrix}=33.
$$



$$
\mathrm{四阶行列式}\;\begin{vmatrix}9&1&12&-3\\3&1&8&-4\\-3&0&0&0\\6&5&3&-2\end{vmatrix}=(\;\;\;\;).
$$
$$
A.
216 \quad B.217 \quad C.327 \quad D.330 \quad E. \quad F. \quad G. \quad H.
$$
$$
\begin{vmatrix}9&1&12&-3\\3&1&8&-4\\-3&0&0&0\\6&5&3&-2\end{vmatrix}=-3×\begin{vmatrix}1&12&-3\\1&8&-4\\5&3&-2\end{vmatrix}\overset{c_2-8c_1}{\underset{c_3+4c_1}=}-3×\begin{vmatrix}1&4&1\\1&0&0\\5&-37&18\end{vmatrix}=-3×\left(-1\right)×\begin{vmatrix}4&1\\-37&18\end{vmatrix}=327.
$$



$$
\mathrm{四阶行列式}\begin{vmatrix}0&1&4&-1\\3&-1&2&-1\\1&2&3&-2\\5&0&6&-2\end{vmatrix}=\left(\;\;\;\;\;\;\right).\;
$$
$$
A.
1 \quad B.2 \quad C.20 \quad D.-12 \quad E. \quad F. \quad G. \quad H.
$$
$$
\begin{vmatrix}0&1&4&-1\\3&-1&2&-1\\1&2&3&-2\\5&0&6&-2\end{vmatrix}=\begin{vmatrix}0&1&4&-1\\0&-7&-7&5\\1&2&3&-2\\0&-10&-9&8\end{vmatrix}=\begin{vmatrix}1&4&-1\\-7&-7&5\\-10&-9&8\end{vmatrix}=\begin{vmatrix}1&4&-1\\0&21&-2\\0&31&-2\end{vmatrix}=\begin{vmatrix}21&-2\\31&-2\end{vmatrix}=20.
$$



$$
\mathrm{四阶行列式}\begin{vmatrix}1&y&0&0\\0&2&y&0\\0&0&3&y\\y&0&0&4\end{vmatrix}=(\;\;\;\;\;).
$$
$$
A.
24-y^4 \quad B.24+y^4 \quad C.24-y^3 \quad D.24+y^3 \quad E. \quad F. \quad G. \quad H.
$$
$$
\mathrm{原式}=\begin{vmatrix}2&y&0\\0&3&y\\0&0&4\end{vmatrix}-y\begin{vmatrix}y&0&0\\2&y&0\\0&3&y\end{vmatrix}=24-y^4.
$$



$$
\mathrm{四阶行列式}\begin{vmatrix}1&3&5&1\\-1&0&2&3\\0&1&7&4\\4&1&-2&1\end{vmatrix}=\left(\;\;\;\right).
$$
$$
A.
134 \quad B.-134 \quad C.-110 \quad D.110 \quad E. \quad F. \quad G. \quad H.
$$
$$
\begin{array}{l}\begin{vmatrix}1&3&5&1\\-1&0&2&3\\0&1&7&4\\4&1&-2&1\end{vmatrix}=\begin{vmatrix}1&3&7&4\\-1&0&0&0\\0&1&7&4\\4&1&6&13\end{vmatrix}=\begin{vmatrix}3&7&4\\1&7&4\\1&6&13\end{vmatrix}=\begin{vmatrix}2&0&0\\1&7&4\\1&6&13\end{vmatrix}=134.\\\end{array}
$$



$$
\mathrm{五阶行列式}\begin{vmatrix}0&0&0&0&c\\b&a&0&0&0\\0&b&a&0&0\\0&0&b&a&0\\0&0&0&b&a\end{vmatrix}=(\;\;\;\;).
$$
$$
A.
b^4c \quad B.a^4c \quad C.-b^4c \quad D.ab^4c \quad E. \quad F. \quad G. \quad H.
$$
$$
\begin{array}{l}\mathrm{将行列式按第一行展开得}\\\begin{vmatrix}0&0&0&0&c\\b&a&0&0&0\\0&b&a&0&0\\0&0&b&a&0\\0&0&0&b&a\end{vmatrix}=\left(-1\right)^{1+5}c\begin{vmatrix}b&a&&\\&b&a&\\&&b&a\\&&&b\end{vmatrix}=b^4c.\end{array}
$$



$$
\mathrm{四阶行列式}\begin{vmatrix}2&b&0&0\\-b&2&0&0\\0&0&2&-b\\0&0&b&2\end{vmatrix}=\left(\;\;\;\;\;\right).
$$
$$
A.
\left(4+b^2\right)^2 \quad B.4+b^2 \quad C.\left(4-b^2\right)^2 \quad D.4-b^2 \quad E. \quad F. \quad G. \quad H.
$$
$$
\mathrm{原式}=\begin{vmatrix}2&b&0&0\\0&2+\frac{b^2}2&0&0\\0&0&2&-b\\0&0&b&2\end{vmatrix}=2\begin{vmatrix}2+\frac{b^2}2&0&0\\0&2&-b\\0&b&2\end{vmatrix}=2\left(2+\frac{b^2}2\right)\begin{vmatrix}2&-b\\b&2\end{vmatrix}=\left(4+b^2\right)^2.
$$



$$
\mathrm{四阶行列式}\begin{vmatrix}-1&2&1&1\\0&2&1&11\\0&3&6&6\\-1&0&0&0\end{vmatrix}=(\;\;\;\;).
$$
$$
A.
-90 \quad B.9 \quad C.-19 \quad D.19 \quad E. \quad F. \quad G. \quad H.
$$
$$
\begin{vmatrix}-1&2&1&1\\0&2&1&11\\0&3&6&6\\-1&0&0&0\end{vmatrix}=\begin{vmatrix}2&1&1\\2&1&11\\3&6&6\end{vmatrix}=\begin{vmatrix}2&1&1\\0&0&10\\3&6&6\end{vmatrix}=-10\begin{vmatrix}2&1\\3&6\end{vmatrix}=-90.
$$



$$
\mathrm{四阶行列式}\begin{vmatrix}2&1&11&5\\1&1&5&2\\2&1&3&4\\1&1&1&4\end{vmatrix}=(\;).
$$
$$
A.
9 \quad B.-20 \quad C.12 \quad D.-12 \quad E. \quad F. \quad G. \quad H.
$$
$$
\begin{vmatrix}2&1&11&5\\1&1&5&2\\2&1&3&4\\1&1&1&4\end{vmatrix}=\begin{vmatrix}2&1&11&5\\-1&0&-6&-3\\0&0&-8&-1\\-1&0&-10&-1\end{vmatrix}=1×(-1)^{1+2}\begin{vmatrix}-1&-6&-3\\0&-8&-1\\-1&-10&-1\end{vmatrix}=-\begin{vmatrix}-1&-6&-3\\0&-8&-1\\0&-4&2\end{vmatrix}=\begin{vmatrix}-8&-1\\-4&2\end{vmatrix}=-20.
$$



$$
\mathrm{四阶行列式}\begin{vmatrix}1&1&2&4\\2&-1&3&-2\\1&6&1&-2\\1&-5&2&4\end{vmatrix}=(\;\;\;).
$$
$$
A.
-12 \quad B.12 \quad C.24 \quad D.-24 \quad E. \quad F. \quad G. \quad H.
$$
$$
\begin{vmatrix}1&1&2&4\\2&-1&3&-2\\1&6&1&-2\\1&-5&2&4\end{vmatrix}=\begin{vmatrix}1&1&2&4\\0&-3&-1&-10\\0&5&-1&-6\\0&-6&0&0\end{vmatrix}=\begin{vmatrix}-3&-1&-10\\5&-1&-6\\-6&0&0\end{vmatrix}=-6×\begin{vmatrix}-1&-10\\-1&-6\end{vmatrix}=-6×(-4)=24.
$$



$$
n\mathrm{阶行列式}\;\;\begin{vmatrix}x&-1&0&…&0&0\\0&x&-1&…&0&0\\…&…&…&…&…&…\\0&0&0&…&x&-1\\a_n&a_{n-1}&a_{n-2}&…&a_2&x+a_1\end{vmatrix}=\left(\;\;\;\;\;\;\right).
$$
$$
A.
x^n-a_1x^{n-1}-...-a_{n-1}x-a_n \quad B.a_1x^{n-1}+...+a_{n-1}x+a_n \quad C.x^n+a_1x^{n-1}+...+a_{n-1}x \quad D.x^n+a_1x^{n-1}+...+a_{n-1}x+a_n \quad E. \quad F. \quad G. \quad H.
$$
$$
\begin{array}{l}\begin{array}{l}\;\;\begin{vmatrix}x&-1&0&…&0&0\\0&x&-1&…&0&0\\…&…&…&…&…&…\\0&0&0&…&x&-1\\a_n&a_{n-1}&a_{n-2}&…&a_2&x+a_1\end{vmatrix}\end{array}=x\begin{vmatrix}x&-1&…&0&0\\…&…&…&…&…\\0&0&…&x&-1\\a_{n-1}&a_{n-2}&…&a_2&x+a_1\end{vmatrix}+a_n×(-1)^{n+1}\begin{vmatrix}-1&0&…&0&0\\x&-1&…&0&0\\…&…&…&…&…\\0&0&…&x&-1\end{vmatrix}\\\;\;=x\begin{vmatrix}x&-1&…&0&0\\…&…&…&…&…\\0&0&…&x&-1\\a_{n-1}&a_{n-2}&…&a_2&x+a_1\end{vmatrix}+a_n=x(x\begin{vmatrix}x&…&0&0\\…&…&…&…\\0&…&x&-1\\a_{n-2}&…&a_2&x+a_1\end{vmatrix}+a_{n-1})+a_n=x^2\begin{vmatrix}x&…&0&0\\…&…&…&…\\0&…&x&-1\\a_{n-2}&…&a_2&x+a_1\end{vmatrix}+xa_{n-1}+a_n=...=x^n+a_1x^{n-1}+...+a_{n-1}x+a_n.\end{array}
$$



$$
\mathrm{五阶行列式}\begin{vmatrix}1&1&1&0&0\\2&3&-1&0&0\\4&9&1&0&0\\0&0&0&2&1\\0&0&0&0&3\end{vmatrix}=\left(\;\;\;\;\right).
$$
$$
A.
72 \quad B.64 \quad C.-12 \quad D.-24 \quad E. \quad F. \quad G. \quad H.
$$
$$
\begin{vmatrix}1&1&1&0&0\\2&3&-1&0&0\\4&9&1&0&0\\0&0&0&2&1\\0&0&0&0&3\end{vmatrix}=3×\begin{vmatrix}1&1&1&0\\2&3&-1&0\\4&9&1&0\\0&0&0&2\end{vmatrix}=3×2×\begin{vmatrix}1&1&1\\2&3&-1\\4&9&1\end{vmatrix}=6×\begin{vmatrix}1&1&1\\3&4&0\\3&8&0\end{vmatrix}=6×12=72.
$$



$$
\mathrm{若线性方程组}\left\{\begin{array}{l}λ x-y=a\\-x+λ y=b\end{array}\right.\mathrm{有唯一的解},则λ().
$$
$$
A.
\mathrm{可为任意实数} \quad B.\mathrm{等于}\;±1\;\; \quad C.\mathrm{不等于}±1\; \quad D.\mathrm{不等于零} \quad E. \quad F. \quad G. \quad H.
$$
$$
\begin{array}{l}\mathrm{由克莱姆法则可知},\mathrm{若题设中的线性方程组有唯一解},\mathrm{则其系数行列式不等于零},即\\\begin{vmatrix}λ&-1\\-1&λ\end{vmatrix}=λ^2-1\neq0⇒λ\neq±1.\end{array}
$$



$$
\mathrm{若齐次线性方程组有非零解},\mathrm{则它的系数行列式}D().
$$
$$
A.
\mathrm{必为}0 \quad B.\mathrm{必不为}0 \quad C.\mathrm{必为}1 \quad D.\mathrm{可取任何值} \quad E. \quad F. \quad G. \quad H.
$$
$$
\mathrm{由克莱姆法则可知},\mathrm{齐次线性方程组有非零解},\mathrm{则系数行列式必为零}.
$$



$$
\;\mathrm{若齐次线性方程组}\left\{\begin{array}{c}λ x_1+x_2+x_3=0\\x_1+μ x_2+x_3=0\\x_1+2μ x_2+x_3=0\end{array}\right.\mathrm{有非零解},则().
$$
$$
A.
μ=0或λ=1 \quad B.μ=0 \quad C.λ=1 \quad D.μ=0且λ=1 \quad E. \quad F. \quad G. \quad H.
$$
$$
\begin{array}{l}\;\;\;\;\;\;\;\;\;\;\;\;\;\;\;\;\;\;\;\;\;\;\;\;\;\;\;\;\;\;\;\;\;\;\;\;\;\;\;\;D=\begin{vmatrix}λ&1&1\\1&μ&1\\1&2μ&1\end{vmatrix}=μ-μλ\\\mathrm{齐次线性方程组有非零解},则D=0\;,\;即\;μ-μλ=0⇒μ=0或\;λ=1.\\\end{array}
$$



$$
当k\neq(\;\;)时,\mathrm{方程组}\left\{\begin{array}{l}\begin{array}{c}kx+z=0\\2x+ky+z=0\\kx-2y+z=0\end{array}\\\end{array}\right.\mathrm{只有零解}.
$$
$$
A.
0 \quad B.-1 \quad C.2 \quad D.-2 \quad E. \quad F. \quad G. \quad H.
$$
$$
\begin{array}{l}\mathrm{已知方程组的系数行列式为}D=\begin{vmatrix}k&0&1\\2&k&1\\k&-2&1\end{vmatrix},\mathrm{由题意及克莱姆法则得}\\\mathrm{当且仅当}D\neq0即k(k+2)+(-4-k^2)=2(k-2)\neq0时,\mathrm{已知方程组有零解},\mathrm{解得}k\neq2.\end{array}
$$



$$
\mathrm{若线性方程组}\left\{\begin{array}{l}\begin{array}{c}ax_1+x_2=0\\2x_1+ax_2+2x_3=0\\x_2+ax_3=0\end{array}\\\end{array}\right.\mathrm{有非零解},则a\mathrm{的值为}().
$$
$$
A.
a=0或a=2\; \quad B.a=0 \quad C.a=±2 \quad D.a=0或a=±2 \quad E. \quad F. \quad G. \quad H.
$$
$$
\begin{array}{l}\mathrm{齐次方程组有非零解},\mathrm{则其系数行列式等于零},即\begin{vmatrix}a&1&0\\2&a&2\\0&1&a\end{vmatrix}=a^3-4a=0,\\\mathrm{所以}a=0或a=±2.\end{array}
$$



$$
\mathrm{齐次线性方程组}\left\{\begin{array}{l}\begin{array}{c}λ x_1+x_2+2x_3=0\\x_1+λ x_2-x_3=0\\λ x_3=0\end{array}\\\end{array}\right.\;\mathrm{有非零解},则().
$$
$$
A.
λ=0或λ=±1 \quad B.λ\neq0且λ\neq±1 \quad C.λ\neq±1 \quad D.λ=0 \quad E. \quad F. \quad G. \quad H.
$$
$$
\mathrm{因为}D=\begin{vmatrix}λ&1&2\\1&λ&-1\\0&0&λ\end{vmatrix}=λ(λ^2-1),\mathrm{所以当}\;λ=0或λ=±1\mathrm{时方程组有非零解}.
$$



$$
\mathrm{齐次线性方程组}\left\{\begin{array}{l}\begin{array}{c}x_1+kx_2+x_3=0\\kx_1+x_2-x_3=0\\2x_1+x_2+x_3=0\end{array}\\\end{array}\right.\mathrm{只有零解},则k().
$$
$$
A.
k\neq0且k\neq-1 \quad B.k=0或-1 \quad C.k=0 \quad D.k\neq0 \quad E. \quad F. \quad G. \quad H.
$$
$$
\begin{array}{l}\mathrm{因为方程组的系数行列式}D=\begin{vmatrix}1&k&1\\k&1&-1\\2&1&1\end{vmatrix}=-k-k^2.\\\mathrm{所以当}k\neq0,-1\mathrm{时方程组有唯一的零解}.\end{array}
$$



$$
\mathrm{齐次方程组}\left\{\begin{array}{l}\begin{array}{c}kx_{}+y+z=0\\x+ky-z=0\\2x-y+z=0\end{array}\\\end{array}\right.\mathrm{有非零解},\text{则}k\mathrm{的值为}().
$$
$$
A.
k\neq-1\text{且}k\neq4 \quad B.k=4 \quad C.k=-1 \quad D.k=-1\text{或4} \quad E. \quad F. \quad G. \quad H.
$$
$$
\begin{array}{l}\mathrm{齐次方程组的系数行列式}D=\begin{vmatrix}k&1&1\\1&k&-1\\2&-1&1\end{vmatrix}=(k+1)(k-4).\\故k=-1或4时,\mathrm{原方程组有非零解}.\end{array}
$$



$$
\mathrm{方程组}\left\{\begin{array}{l}\begin{array}{c}λ x_1+x_2+x_3=1\\x_1+λ x_2+x_3=1\\x_1+x_2+λ x_3=1\end{array}\\\end{array}\right.\mathrm{有唯一解},则λ\mathrm{的值为}().
$$
$$
A.
λ\neq1\text{且}λ\neq-2 \quad B.λ=1\text{或}-2 \quad C.λ\neq1 \quad D.λ=-2 \quad E. \quad F. \quad G. \quad H.
$$
$$
\begin{array}{l}D=\begin{vmatrix}λ&1&1\\1&λ&1\\1&1&λ\end{vmatrix}=(λ-1)^2(λ+2),\\\mathrm{故当}λ\neq1,-2\mathrm{时方程组有唯一解}.\end{array}
$$



$$
\mathrm{方程组}\left\{\begin{array}{l}\begin{array}{c}λ x_1+x_2+x_3=1\\x_1+λ x_2+x_3=λ\\x_1+x_2+λ x_3=λ^2\end{array}\\\end{array}\right.\mathrm{有唯一解时},\text{对}λ\mathrm{的要求是}\;()
$$
$$
A.
λ\neq1,λ\neq2 \quad B.λ\neq-1,λ\neq2 \quad C.λ\neq1,λ\neq-2 \quad D.λ\neq-1,λ\neq-2 \quad E. \quad F. \quad G. \quad H.
$$
$$
\begin{array}{l}\begin{array}{l}\mathrm{由克莱姆法则可知},\mathrm{当所给非齐次线性方程组的系数行列式不等于零时},\mathrm{该方程组有唯一解},\mathrm{于是令行列式}\\\end{array}\\D=\begin{vmatrix}λ&1&1\\1&λ&1\\1&1&λ\end{vmatrix}=λ^3-3λ+2=(λ-1)^2(λ+2)\neq0,\\\text{即}λ\neq1,λ\neq-2.\\\end{array}
$$



$$
\text{设}n\mathrm{阶行列式}D_n,\text{则}D_n=0\mathrm{的必要条件是}().
$$
$$
A.
D_n\mathrm{中两行}(\mathrm{或列})\mathrm{元素对应成比例} \quad B.D_n\mathrm{中有一行}(\mathrm{或列})\mathrm{元素全为零} \quad C.D_n\mathrm{各列元素之和为零} \quad D.\text{以}D_n\mathrm{为系数行列式的齐次线性方程组有非零解} \quad E. \quad F. \quad G. \quad H.
$$
$$
\begin{array}{l}由D_n=0\mathrm{可推导出以}D_n\mathrm{为系数行列式的齐次线性方程组有非零解},\\\;\mathrm{因此以}D_n\mathrm{为系数行列式的齐次线性方程组有非零解是}D_n=0\mathrm{的必要条件}.\;\\\mathrm{其余选项都是}D_n=0\mathrm{的充分条件}.\end{array}
$$



$$
当a,b\mathrm{满足}(\;)时,\mathrm{方程组}\left\{\begin{array}{l}ax_1+ax_2+ax_3+ax_4+bx_5=0\\ax_1+ax_2+ax_3+bx_4+ax_5=0\\ax_1+ax_2+bx_3+ax_4+ax_5=0\\ax_1+bx_2+ax_3+ax_4+ax_5=0\\bx_1+ax_2+ax_3+ax_4+ax_5=0\end{array}\right.\mathrm{只有零解}.
$$
$$
A.
a\neq b\text{且}a\neq-\frac b4 \quad B.a=b或a\neq-\frac b4 \quad C.a=b\text{或}a=-\frac b4 \quad D.a\neq b\text{或}a=-\frac b4 \quad E. \quad F. \quad G. \quad H.
$$
$$
\begin{array}{l}\mathrm{由克莱姆法则可知},\mathrm{当系数行列式}\\D=\begin{vmatrix}a&a&a&a&b\\a&a&a&b&a\\a&a&b&a&a\\a&b&a&a&a\\b&a&a&a&a\end{vmatrix}=(4a+b)(b-a^{})^4\neq0\\\text{时},\mathrm{齐次线性方程组只有零解},\text{即}a\neq b,\text{且}a\neq-\frac b4\text{时},\mathrm{原方程组有唯一零解}.\end{array}
$$



$$
\mathrm{平面上三条不同的直线}\;ax+by+c=0,bx+cy+a=0,cx+ay+b=0\mathrm{相交于一点的必要条件是}().\;
$$
$$
A.
a+b+c=0 \quad B.a-b+c=0 \quad C.a+b-c=0 \quad D.a-b-c=0 \quad E. \quad F. \quad G. \quad H.
$$
$$
\begin{array}{l}\;\mathrm{设三条直线交于一点}M(x_0,y_0)\;则x=x_0,y=y_0,z=1\mathrm{可视为齐次线性方程组}\left\{\begin{array}{c}ax+by+cz=0\\\begin{array}{l}bx+cy+az=0\\cx+ay+bz=0\end{array}\end{array}\right.\text{的}\\\mathrm{非零解}.\;\mathrm{从而有系数行列式}\begin{vmatrix}a&b&c\\b&c&a\\c&a&b\end{vmatrix}=(-\frac12)(a+b+c)\lbrack(a-b)^2+(b-c)^2+(c-a)^2\rbrack=0.\\\mathrm{因为三条直线互不相同},\mathrm{所以}a,b,c\mathrm{也不完全相同},\text{故}a+b+c=0.\\\end{array}
$$



$$
设f(x)\;=c_0+c_1x+c_2x^2+...+c_nx^n,\mathrm{如果}f(x)=0有n+1\mathrm{个互不相同的根},\text{则( )}.\;
$$
$$
A.
f(x)\mathrm{是零多项式} \quad B.f(x)\mathrm{是非零常数} \quad C.f(x)\text{是}n\mathrm{次多项式} \quad D.f(x)\text{是}n-1\mathrm{次多项式} \quad E. \quad F. \quad G. \quad H.
$$
$$
\begin{array}{l}设a_1,a_2,...,a_{n+1}\text{为}f(x)\text{的}n+1\mathrm{个互不相同的根},\mathrm{其中}\\a_i\neq a_j(i\neq j,且i,j=1,2,...,n+1),\\\mathrm{则由}\;\;f(a_i)=0(i=1,2,...,n+1)\\\mathrm{得线性方程组}\left\{\begin{array}{c}c_0+a_1c_1+a_1^2c_2+...+a_1^nc_n=0\\c_0+a_2c_1+a_2^2c_2+...+a_2^nc_n=0\\\begin{array}{l}......\\c_0+a_{n+1}c_1+a_{n+1}^2c_2+...+a_{n+1}^nc_n=0\end{array}\end{array}\right.\\\mathrm{它是关于}c_0,c_1,c_2,...,c_n\mathrm{的齐次线性方程组},\mathrm{其系数行列式}\\D_{n+1}=\begin{vmatrix}1&a_1&a_1^2&...&a_1^n\\1&a_2&a_2^2&...&a_2^n\\...&...&...&...&...\\1&a_{n+1}&a_{n+1}^2&...&a_{n+1}^n\end{vmatrix}=\prod_{n+1\geq i>j\geq1}(a_i-a_j)\neq0\\\mathrm{故由克莱姆法则知}:\mathrm{齐次线性方程组只有唯一的零解},即\;\\c_0=c_1=c_2=...=c_n=0\\\text{故}f(x)\;=0\;\text{即}f(x)\;\mathrm{是零多项式}.\end{array}
$$



$$
\mathrm{齐次方程组}\left\{\begin{array}{l}\begin{array}{c}(1-λ)x_1-2x_2+4x_3=0\\2x_1+(3-λ)x_2+x_3=0\\x_1+x_2+(1-\lambda)x_3=0\end{array}\\\end{array}\right.\mathrm{有非零解},则λ\mathrm{的值为}().
$$
$$
A.
λ=0,或\lambda=2\text{或}λ=3 \quad B.λ=0 \quad C.λ=3 \quad D.λ=2 \quad E. \quad F. \quad G. \quad H.
$$
$$
\begin{array}{l}\\D=\begin{vmatrix}1-λ&-2&4\\2&3-λ&1\\1&1&1-λ\end{vmatrix}=λ(λ-2)(3-\lambda)\\\\\mathrm{齐次线性方程组有非零解},则D=0,\mathrm{所以}λ=0,λ=2\text{或}λ=3\text{时},\mathrm{齐次线性方程组有非零解}.\end{array}
$$



$$
n\mathrm{阶行列式}\begin{vmatrix}0&0&...&0&λ_1\\0&0&...&λ_2&0\\...&...&...&...&...\\λ_n&0&...&0&0\end{vmatrix}=(\;\;).
$$
$$
A.
0 \quad B.λ_1λ_2…λ_n \quad C.(-1)^\frac{n(n-1)}2λ_1λ_2…λ_n \quad D.-λ_1λ_2…λ_n \quad E. \quad F. \quad G. \quad H.
$$
$$
\begin{vmatrix}0&0&...&0&λ_1\\0&0&...&λ_2&0\\...&...&...&...&...\\λ_n&0&...&0&0\end{vmatrix}=(-1)^{τ(n(n-1)(n-2)...321)}λ_1λ_2…λ_n=(-1)^\frac{n(n-1)}2λ_1λ_2…λ_n.
$$



$$
当a\mathrm{满足}(\;)时,\mathrm{方程组}\left\{\begin{array}{l}ax_1+ax_2+ax_3+ax_4+x_5=0\\ax_1+ax_2+ax_3+x_4+ax_5=0\\ax_1+ax_2+x_3+ax_4+ax_5=0\\ax_1+x_2+ax_3+ax_4+ax_5=0\\x_1+ax_2+ax_3+ax_4+ax_5=0\end{array}\right.\mathrm{只有零解}.
$$
$$
A.
a\neq1\text{且}a\neq-\frac14 \quad B.a=1或a\neq-\frac14 \quad C.a=1\text{或}a=-\frac14 \quad D.a\neq1\text{或}a=-\frac14 \quad E. \quad F. \quad G. \quad H.
$$
$$
\begin{array}{l}\mathrm{由克莱姆法则可知},\mathrm{当系数行列式}\\D=\begin{vmatrix}a&a&a&a&1\\a&a&a&1&a\\a&a&1&a&a\\a&1&a&a&a\\1&a&a&a&a\end{vmatrix}=(4a+1)(1-a^{})^4\neq0\\\text{时},\mathrm{齐次线性方程组只有零解},\text{即}a\neq1,\text{且}a\neq-\frac14\text{时},\mathrm{原方程组有唯一零解}.\end{array}
$$



$$
\mathrm{方程组}\left\{\begin{array}{l}\begin{array}{c}λ x_1+2x_2+2x_3=1\\2x_1+λ x_2+2x_3=1\\2x_1+2x_2+λ x_3=1\end{array}\\\end{array}\right.\mathrm{有唯一解},则λ\mathrm{的值为}().
$$
$$
A.
λ\neq2\text{且}λ\neq-4 \quad B.λ=2\text{或}-4 \quad C.λ\neq1 \quad D.λ=-4 \quad E. \quad F. \quad G. \quad H.
$$
$$
\begin{array}{l}D=\begin{vmatrix}λ&2&2\\2&λ&2\\2&2&λ\end{vmatrix}=(λ-2)^2(λ+4),\\\mathrm{故当}\lambda\neq2\text{,}-4\mathrm{时方程组有唯一解}.\end{array}
$$



$$
\mathrm{若线性方程组}\left\{\begin{array}{l}λ x_1+x_2=0\\x_1+λ x_2=0\end{array}\right.\mathrm{有非零解},则λ().
$$
$$
A.
\mathrm{可为任意实数} \quad B.\mathrm{等于}\;±1\;\; \quad C.\mathrm{不等于}±1\; \quad D.\mathrm{等于零} \quad E. \quad F. \quad G. \quad H.
$$
$$
\begin{array}{l}\mathrm{由克莱姆法则可知},\\\begin{vmatrix}λ&1\\1&λ\end{vmatrix}=λ^2-1=0⇒λ=±1.\end{array}
$$



$$
\;\mathrm{若齐次线性方程组}\left\{\begin{array}{c}λ x_1+x_2+x_3=0\\x_1+μ x_2+x_3=0\\x_1+2μ x_2+x_3=0\end{array}\right.\mathrm{只有零解},则().
$$
$$
A.
μ=0或λ=1 \quad B.μ=0 \quad C.λ\neq1 \quad D.μ\neq0且λ\neq1 \quad E. \quad F. \quad G. \quad H.
$$
$$
\begin{array}{l}\;\;\;\;\;\;\;\;\;\;\;\;\;\;\;\;\;\;\;\;\;\;\;\;\;\;\;\;\;\;\;\;\;\;\;\;\;\;\;\;D=\begin{vmatrix}λ&1&1\\1&μ&1\\1&2μ&1\end{vmatrix}=μ-μλ\\\mathrm{齐次线性方程组只有零解},则D\neq0\;,\;即\;μ-μλ\neq0⇒μ\neq0且\;λ\neq1.\\\end{array}
$$



$$
当k=(\;\;)时,\mathrm{方程组}\left\{\begin{array}{l}\begin{array}{c}kx+z=0\\2x+ky+z=0\\kx-2y+z=0\end{array}\\\end{array}\right.\mathrm{有非零解}.
$$
$$
A.
0 \quad B.-1 \quad C.2 \quad D.-2 \quad E. \quad F. \quad G. \quad H.
$$
$$
\begin{array}{l}\mathrm{已知方程组的系数行列式为}D=\begin{vmatrix}k&0&1\\2&k&1\\k&-2&1\end{vmatrix},\\\mathrm{当且仅当}D=0即k(k+2)+(-4-k^2)=2(k-2)=0时,\mathrm{已知方程组有非零解},\mathrm{解得}k=2.\end{array}
$$



$$
\mathrm{方程组}\left\{\begin{array}{lc}\begin{array}{c}\;x+y-2z=-3\\5x-2y+7z=22\\2x-5y+4z=4\end{array}中,&z=\left(\right).\end{array}\right.
$$
$$
A.
3 \quad B.-3 \quad C.2 \quad D.-2 \quad E. \quad F. \quad G. \quad H.
$$
$$
D=\begin{vmatrix}1&1&-2\\5&-2&7\\2&-5&4\end{vmatrix}=63,D_3=\begin{vmatrix}1&1&-3\\5&-2&22\\2&-5&4\end{vmatrix}=189,则z=\frac{D_3}D=\frac{189}{63}=3.
$$



$$
\mathrm{方程组}\left\{\begin{array}{lc}\begin{array}{c}bx-ay+2ab=0\\-2cy+3bz-bc=0,\\cx+az=0\end{array}&(abc\neq0)中,\begin{array}{c}x=().\end{array}\end{array}\right.
$$
$$
A.
-a \quad B.a \quad C.bc \quad D.-bc \quad E. \quad F. \quad G. \quad H.
$$
$$
\begin{array}{l}D=\begin{vmatrix}b&-a&0\\0&-2c&3b\\c&0&a\end{vmatrix}=-5abc,D_1=\begin{vmatrix}-2ab&-a&0\\bc&-2c&3b\\0&0&a\end{vmatrix}=5a^2bc\\\mathrm{于是得}x=\frac{D_1}D=\frac{5a^2bc}{-5abc}=-a.\end{array}
$$



$$
\mathrm{如果}\begin{vmatrix}a_{11}&a_{12}\\a_{21}&a_{22}\end{vmatrix}=1,\mathrm{则方程组}\left\{\begin{array}{l}a_{11}x_1-a_{12}x_2+b_1=0\\a_{21}x_1-a_{22}x_2+b_2=0\end{array}\right.\mathrm{的解是}().
$$
$$
A.
x_1=\begin{vmatrix}b_1&a_{12}\\b_2&a_{22}\end{vmatrix},x_2=-\begin{vmatrix}a_{11}&b_1\\a_{21}&b_2\end{vmatrix} \quad B.x_1=-\begin{vmatrix}b_1&a_{12}\\b_2&a_{22}\end{vmatrix},x_2=-\begin{vmatrix}a_{11}&b_1\\a_{21}&b_2\end{vmatrix} \quad C.x_1=-\begin{vmatrix}-b_1&a_{12}\\-b_2&a_{22}\end{vmatrix},x_2=-\begin{vmatrix}a_{11}&-b_1\\a_{21}&-b_2\end{vmatrix} \quad D.x_1=-\begin{vmatrix}-b_1&-a_{12}\\-b_2&-a_{22}\end{vmatrix},x_2=-\begin{vmatrix}a_{11}&-b_1\\a_{21}&-b_2\end{vmatrix} \quad E. \quad F. \quad G. \quad H.
$$
$$
x_1=-\begin{vmatrix}-b_1&-a_{12}\\-b_2&-a_{22}\end{vmatrix},x_2=-\begin{vmatrix}a_{11}&-b_1\\a_{21}&-b_2\end{vmatrix}
$$



$$
\mathrm{方程组}\left\{\begin{array}{lc}\begin{array}{c}2x_1+x_2-5x_3+x_4=8\\x_1-3x_2-6x_4=9\\2x_2-x_3+2x_4=-5\\x_1+4x_2-7x_3+6x_4=0\end{array}&中,x_2=().\end{array}\right.
$$
$$
A.
-4 \quad B.4 \quad C.-1 \quad D.1 \quad E. \quad F. \quad G. \quad H.
$$
$$
\begin{array}{l}D=\begin{vmatrix}2&1&-5&1\\1&-3&0&-6\\0&2&-1&2\\1&4&-7&6\end{vmatrix}\begin{array}{c}r_1-2r_2\\=\\r_4-r_2\end{array}\begin{vmatrix}0&7&-5&13\\1&-3&0&-6\\0&2&-1&2\\0&7&-7&12\end{vmatrix}=-\begin{vmatrix}7&-5&13\\2&-1&2\\7&-7&12\end{vmatrix}\begin{array}{c}c_1+2c_2\\=\\c_3+2c_2\end{array}-\begin{vmatrix}-3&-5&3\\0&-1&0\\-7&-7&-2\end{vmatrix}=\begin{vmatrix}-3&3\\-7&-2\end{vmatrix}=27.\\D_2=\begin{vmatrix}2&8&-5&1\\1&9&0&-6\\0&-5&-1&2\\1&0&-7&6\end{vmatrix}=-108,x_2=\frac{D_2}D=\frac{-108}{27}=-4\end{array}
$$



$$
\mathrm{方程组}\left\{\begin{array}{l}\begin{array}{c}2x_1+\;x_2-5x_3+x_4=8\;\\x_1-3x_2-6x_4=9\\2x_2-x_3+2x_4=-5\end{array}中,x_3=\left(\right).\\x_1+4x_2-7x_3+6x_4=0\end{array}\right.
$$
$$
A.
-1 \quad B.1 \quad C.3 \quad D.-3 \quad E. \quad F. \quad G. \quad H.
$$
$$
\begin{array}{l}\mathrm{系数行列式}\\D=\begin{vmatrix}2&1&-5&1\\1&-3&0&-6\\0&2&-1&2\\1&4&-7&6\end{vmatrix}=27\neq0,D_3=\begin{vmatrix}2&1&8&1\\1&-3&9&-6\\0&2&-5&2\\1&4&0&6\end{vmatrix}=-27,则x_3=\frac{D_3}D=\frac{-27}{27}=-1\end{array}
$$



$$
\mathrm{方程组}\left\{\begin{array}{l}\begin{array}{c}2x_1+3x_2+11x_3+5x_4=6\\x_1+x_2+5x_3+2x_4=2\\2x_1+x_2+3x_3+4x_4=2\end{array}中,x_4=\left(\right).\\x_1+x_2+3x_3+4x_4=2\end{array}\right.
$$
$$
A.
0 \quad B.20 \quad C.1 \quad D.-1 \quad E. \quad F. \quad G. \quad H.
$$
$$
\mathrm{系数行列式}D=\begin{vmatrix}2&3&11&5\\1&1&5&2\\2&1&3&4\\1&1&3&4\end{vmatrix}=10\neq0,D_4=\begin{vmatrix}2&3&11&6\\1&1&5&2\\2&1&3&2\\1&1&3&2\end{vmatrix}=0,\mathrm{于是得}x_4=\frac{D_4}D=\frac0{10}=0
$$



$$
\mathrm{方程组}\left\{\begin{array}{l}\begin{array}{c}\begin{array}{c}x_1+x_2+x_3+x_4=0\\x_2+x_3+x_4+x_5=0\\x_1+2x_2+3x_3=2\end{array}\\x_2+2x_3+3x_4=-2\\x_3+2x_4+3x_5=2\end{array}中,x_5=\left(\right).\\\end{array}\right.
$$
$$
A.
1 \quad B.-1 \quad C.2 \quad D.-2 \quad E. \quad F. \quad G. \quad H.
$$
$$
D=\begin{vmatrix}1&1&1&1&0\\0&1&1&1&1\\1&2&3&0&0\\0&1&2&3&0\\0&0&1&2&3\end{vmatrix}=16\neq0,D_5=\begin{vmatrix}1&1&1&1&0\\0&1&1&1&0\\1&2&3&0&2\\0&1&2&3&-2\\0&0&1&2&2\end{vmatrix}=16,则x_5=\frac{16}{16}=1
$$



$$
\mathrm{方程组}\left\{\begin{array}{l}\begin{array}{c}x_1+x_2+x_3+x_4=1\\2x_1+3x_2+4x_3+5x_4=5\\4x_1+9x_2+16x_3+25x_4=25\end{array}\\8x_1+27x_2+64x_3+125x_4=125\end{array}\right.\mathrm{有解}\left(\right).
$$
$$
A.
x_1=0,x_2=-1,x_3=2,x_4=0 \quad B.x_1=1,x_2=1,x_3=0,x_4=-1 \quad C.x_1=0,\;x_2=0,\;x_3=0,\;x_4=1 \quad D.x_1=0,x_2=0,x_3=1,x_4=1 \quad E. \quad F. \quad G. \quad H.
$$
$$
\begin{array}{l}\mathrm{方程组的系列行列式为}D=\begin{vmatrix}1&1&1&1\\2&3&4&5\\4&9&16&25\\8&27&64&125\end{vmatrix},\mathrm{由克莱姆法则可知}x_1=\frac{D_1}D,x_2=\frac{D_2}D,x_3=\frac{D_3}D,x_4=\frac{D_4}D,\mathrm{由于}\\D_1,D_2,D_3\mathrm{都有相同的两列},\mathrm{因此行列式都为零},且D_4=D,故\\x_1=0,\;\;x_2=0,\;\;x_3=0,x_4=1.\end{array}
$$



$$
设a,b,c\mathrm{是互不相等的数},\mathrm{则方程组}\left\{\begin{array}{l}\begin{array}{c}x+y+z=a+b+c\\ax+by+cz=a^2+b^2+c^2\;\;\\bcx+cay+abz=3abc\end{array}中,y=\left(\right).\\\end{array}\right.
$$
$$
A.
0 \quad B.a \quad C.b \quad D.c \quad E. \quad F. \quad G. \quad H.
$$
$$
\begin{array}{l}D=\begin{vmatrix}1&1&1\\a&b&c\\bc&ca&ab\end{vmatrix}=\begin{vmatrix}1&0&0\\a&b-a&c-a\\bc&c(a-b)&b(a-c)\end{vmatrix}=(b-a)(c-a)(c-b).\\D_2=\begin{vmatrix}1&a+b+c&1\\a&a^2+b^2+c^2&c\\bc&3abc&ab\end{vmatrix}=\begin{vmatrix}1&b&1\\a&b^2&c\\bc&abc&ab\end{vmatrix}=bD.\\y=\frac{D^2}D=b\end{array}
$$



$$
\mathrm{用克莱姆法则解线性方程组}\left\{\begin{array}{l}\begin{array}{c}x+3y+2z+4t=1\\2x-y+3z-2t=-1\\x+2y+z-2t=6\end{array},则x=\left(\right).\\x+2z+4t=-5\end{array}\right.
$$
$$
A.
-1 \quad B.1 \quad C.2 \quad D.-2 \quad E. \quad F. \quad G. \quad H.
$$
$$
\begin{array}{l}D=\begin{vmatrix}1&3&2&4\\2&-1&3&-2\\1&2&1&-2\\1&0&2&4\end{vmatrix}=12\\D_1=\begin{vmatrix}1&3&2&4\\-1&-1&3&-2\\6&2&1&-2\\-5&0&2&4\end{vmatrix}=12,\\\\\mathrm{解之得}x=1.\end{array}
$$



$$
\mathrm{用克莱姆法则解线性方程组}\left\{\begin{array}{l}\begin{array}{c}x+3y+2z+4t=1\\2x-y+3z-2t=-1\\x+2y+z-2t=6\end{array},则y=\left(\right).\\x+2z+4t=-5\end{array}\right.
$$
$$
A.
-1 \quad B.1 \quad C.2 \quad D.-2 \quad E. \quad F. \quad G. \quad H.
$$
$$
\begin{array}{l}D=\begin{vmatrix}1&3&2&4\\2&-1&3&-2\\1&2&1&-2\\1&0&2&4\end{vmatrix}=12\\D_2=\begin{vmatrix}1&1&2&4\\2&-1&3&-2\\1&6&1&-2\\1&-5&2&4\end{vmatrix}=24\\\\\mathrm{解之得}y=2.\end{array}
$$



$$
\mathrm{用克莱姆法则解线性方程组}\left\{\begin{array}{l}\begin{array}{c}x+3y+2z+4t=1\\2x-y+3z-2t=-1\\x+2y+z-2t=6\end{array},则z=\left(\right).\\x+2z+4t=-5\end{array}\right.
$$
$$
A.
-1 \quad B.1 \quad C.2 \quad D.-2 \quad E. \quad F. \quad G. \quad H.
$$
$$
\begin{array}{l}D=\begin{vmatrix}1&3&2&4\\2&-1&3&-2\\1&2&1&-2\\1&0&2&4\end{vmatrix}=12\\\\D_3=\begin{vmatrix}1&3&1&4\\2&-1&-1&-2\\1&2&6&-2\\1&0&-5&4\end{vmatrix}=-12\\\mathrm{解之得}z=-1.\end{array}
$$



$$
\mathrm{用克莱姆法则解线性方程组}\left\{\begin{array}{l}\begin{array}{c}x+3y+2z+4t=1\\2x-y+3z-2t=-1\\x+2y+z-2t=6\end{array},则t=\left(\right).\\x+2z+4t=-5\end{array}\right.
$$
$$
A.
-1 \quad B.1 \quad C.2 \quad D.-2 \quad E. \quad F. \quad G. \quad H.
$$
$$
\begin{array}{l}D=\begin{vmatrix}1&3&2&4\\2&-1&3&-2\\1&2&1&-2\\1&0&2&4\end{vmatrix}=12\\\\D_4=\begin{vmatrix}1&3&2&1\\2&-1&3&-1\\1&2&1&6\\1&0&2&-5\end{vmatrix}=-12,\\\mathrm{解之得}t=-1.\end{array}
$$



$$
\mathrm{用克莱姆法则求解方程组}\left\{\begin{array}{l}\begin{array}{c}2x_1+3x_2+11x_3+5x_4=6\\x_1+x_2+5x_3+2x_4=2\\2x_1+x_2+3x_3+4x_4=2\end{array},则x_1=\left(\right).\\x_1+x_2+3x_3+4x_4=2\end{array}\right.
$$
$$
A.
0 \quad B.1 \quad C.-1 \quad D.2 \quad E. \quad F. \quad G. \quad H.
$$
$$
\begin{array}{l}D=\begin{vmatrix}2&3&11&5\\1&1&5&2\\2&1&3&4\\1&1&3&4\end{vmatrix}=10,D_1=\begin{vmatrix}6&3&11&5\\2&1&5&2\\2&1&3&4\\2&1&3&4\end{vmatrix}=0,\\\\则x_1=0.\end{array}
$$



$$
\mathrm{用克莱姆法则求解方程组}\left\{\begin{array}{l}\begin{array}{c}2x_1+3x_2+11x_3+5x_4=6\\x_1+x_2+5x_3+2x_4=2\\2x_1+x_2+3x_3+4x_4=2\end{array},则x_2=\left(\right).\\x_1+x_2+3x_3+4x_4=2\end{array}\right.
$$
$$
A.
0 \quad B.1 \quad C.-1 \quad D.2 \quad E. \quad F. \quad G. \quad H.
$$
$$
\begin{array}{l}D=\begin{vmatrix}2&3&11&5\\1&1&5&2\\2&1&3&4\\1&1&3&4\end{vmatrix}=10,D_2=\begin{vmatrix}2&6&11&5\\1&2&5&2\\2&2&3&4\\1&2&3&4\end{vmatrix}=20,\\\\则x_2=2.\end{array}
$$



$$
\mathrm{用克莱姆法则求解方程组}\left\{\begin{array}{l}\begin{array}{c}2x_1+x_2-5x_3+x_4=8\\x_1-3x_2-6x_4=9\\2x_2-x_3+2x_4=-5\end{array},\mathrm{其中}x_1=\left(\right).\\x_1+4x_2-7x_3+6x_4=0\end{array}\right.
$$
$$
A.
3 \quad B.-4 \quad C.-1 \quad D.1 \quad E. \quad F. \quad G. \quad H.
$$
$$
\begin{array}{l}D=\begin{vmatrix}2&1&-5&1\\1&-3&0&-6\\0&2&-1&2\\1&4&-7&6\end{vmatrix}=27,D_1=\begin{vmatrix}8&1&-5&1\\9&-3&0&-6\\-5&2&-1&2\\0&4&-7&6\end{vmatrix}=81,\\\\\mathrm{解之得}x_1=3.\end{array}
$$



$$
\mathrm{用克莱姆法则求解方程组}\left\{\begin{array}{l}\begin{array}{c}2x_1+x_2-5x_3+x_4=8\\x_1-3x_2-6x_4=9\\2x_2-x_3+2x_4=-5\end{array},\mathrm{其中}x_4=\left(\right).\\x_1+4x_2-7x_3+6x_4=0\end{array}\right.
$$
$$
A.
3 \quad B.-4 \quad C.-1 \quad D.1 \quad E. \quad F. \quad G. \quad H.
$$
$$
\begin{array}{l}D=\begin{vmatrix}2&1&-5&1\\1&-3&0&-6\\0&2&-1&2\\1&4&-7&6\end{vmatrix}=27,\\D_4=\begin{vmatrix}2&1&-5&8\\1&-3&0&9\\0&2&-1&-5\\1&4&-7&0\end{vmatrix}=27,\\\mathrm{解之得}x_4=1.\end{array}
$$



$$
\mathrm{设曲线}y=a_0+a_1x+a_2x^2+a_3x^3\mathrm{通过四点}\left(1,3\right)、\left(2,4\right)、\left(3,3\right)、\left(4,-3\right),\mathrm{则系数}a_0,a_1,a_2,a_3\mathrm{分别为}\left(\right).
$$
$$
A.
a_0=3,a_1=-3/2,a_2=2,a_3=-1/2 \quad B.a_0=3,a_1=-3/2,a_2=-2,a_3=1/2 \quad C.a_0=3,a_1=3/2,a_2=-2,a_3=-1/2 \quad D.a_0=3,a_1=3/2,a_2=2,a_3=1/2 \quad E. \quad F. \quad G. \quad H.
$$
$$
\begin{array}{l}\begin{array}{l}\mathrm{把四个点的坐标代入曲线方程},\mathrm{得线性方程组},\mathrm{得线性方程组}\\\left\{\begin{array}{l}\begin{array}{c}a_0+a_1+a_2+a_3=3\\a_0+2a_1+4a_2+8a_3=4\\a_0+3a_1+9a_2+27a_3=3\end{array}\\a_0+4a_1+16a_2+64a_3=-3\end{array}\right.\\\mathrm{其系数行列式}D=\begin{vmatrix}1&1&1&1\\1&2&4&8\\1&3&9&27\\1&4&16&64\end{vmatrix}=1·2·3·1·2·1=12,而\\D_1=\begin{vmatrix}3&1&1&1\\4&2&4&8\\3&3&9&27\\-3&4&16&64\end{vmatrix}\begin{array}{c}\begin{array}{c}c_4-c_3\\c_3-c_2\\=\end{array}\\c_1-3c_2\\\end{array}\begin{vmatrix}0&1&0&0\\-2&2&2&4\\-6&3&6&18\\-15&4&12&48\end{vmatrix}=\left(-1\right)^3\begin{vmatrix}-2&2&4\\-6&6&18\\-15&12&48\end{vmatrix}\end{array}\\\overset{c_1+c_2}=-\begin{vmatrix}0&2&4\\0&6&18\\-3&12&48\end{vmatrix}=-\left(-3\right)\begin{vmatrix}2&4\\6&18\end{vmatrix}=36;\\\mathrm{类似的},\mathrm{计算得}:\\\;\;\;\;\;\;\;\;\;\;\;\;\;\;\;\;\;\;\;\;D_2=\begin{vmatrix}1&3&1&1\\1&4&4&8\\1&3&9&27\\1&-3&16&64\end{vmatrix}=-18;\;\;\;\;\;D_3=\begin{vmatrix}1&1&3&1\\1&2&4&8\\1&3&3&27\\1&4&-3&64\end{vmatrix}=24;\;\\\;\;\;\;\;\;\;\;\;\;\;\;\;\;\;\;\;\;\;\;\;D_4=\begin{vmatrix}1&1&1&3\\1&2&4&4\\1&3&9&3\\1&4&16&-3\end{vmatrix}=-6;\;\\\mathrm{故由克莱姆法则},\mathrm{得唯一解}\\\;\;\;\;\;\;\;\;\;\;\;\;\;\;\;\;\;\;\;\;\;\;a_0=3,\;a_1=-3/2,\;a_2=2,\;a_3=-1/2,\\\mathrm{即曲线方程为}\\y=3-\frac32x+2x^2-\frac12x^3.\end{array}
$$



$$
\mathrm{方程组}\left\{\begin{array}{lc}\begin{array}{c}\;x+y-2z=-3\\5x-2y+7z=22中.\\2x-5y+4z=4\end{array}&x=\left(\right).\end{array}\right.
$$
$$
A.
1 \quad B.-1 \quad C.2 \quad D.-2 \quad E. \quad F. \quad G. \quad H.
$$
$$
D=\begin{vmatrix}1&1&-2\\5&-2&7\\2&-5&4\end{vmatrix}=63,\;\;D_1=\begin{vmatrix}-3&1&-2\\22&-2&7\\4&-5&4\end{vmatrix}=63\;,则\;x=\frac{D_1}D=\frac{63}{63}=1.
$$



$$
\mathrm{方程组}\left\{\begin{array}{lc}\begin{array}{c}\;x+y-2z=-3\\5x-2y+7z=22\\2x-5y+4z=4\end{array}中,&y=\left(\right).\end{array}\right.
$$
$$
A.
2 \quad B.-2 \quad C.3 \quad D.-3 \quad E. \quad F. \quad G. \quad H.
$$
$$
D=\begin{vmatrix}1&1&-2\\5&-2&7\\2&-5&4\end{vmatrix}=63,\;\;D_2=\begin{vmatrix}1&-3&-2\\5&22&7\\2&4&4\end{vmatrix}=126\;,则\;y=\frac{D_2}D=\frac{126}{63}=2.
$$



$$
设A=\begin{pmatrix}2&5\\1&-2\end{pmatrix},B=\begin{pmatrix}1&2\\3&-1\end{pmatrix},则A^2+9B^2=\left(\;\;\;\;\right)
$$
$$
A.
72E \quad B.O \quad C.63E \quad D.3E \quad E. \quad F. \quad G. \quad H.
$$
$$
A^2+9B^2=9E+9·7E=72E
$$



$$
\begin{array}{l}设A,B,C,O\mathrm{为同阶方阵},\mathrm{则下列命题成立的是}(\;).\\\end{array}
$$
$$
A.
若AB=AC,则B=C \quad B.若AB=O,则A=O或B=O \quad C.若A\neq O,\;则\begin{vmatrix}A\end{vmatrix}\neq0 \quad D.若\begin{vmatrix}A\end{vmatrix}\neq0,\;则A\neq O \quad E. \quad F. \quad G. \quad H.
$$
$$
\begin{array}{l}\mathrm{由于矩阵的乘法不满足消去律},\mathrm{因此由}AB=AC,\mathrm{不能推出}B=C;\\\;\mathrm{由若}AB=O,\mathrm{不能推出}A=O或B=O,\mathrm{例如非零矩阵}A=\begin{pmatrix}0&1\\0&0\end{pmatrix},B=\begin{pmatrix}1&0\\0&0\end{pmatrix},\;AB=O且\begin{vmatrix}A\end{vmatrix}=0.\;\\\mathrm{因此只有选项若}\begin{vmatrix}A\end{vmatrix}\neq0,则A\neq O\mathrm{正确}.\end{array}
$$



$$
设A,B\mathrm{均为}n\mathrm{阶方阵},\mathrm{满足}AB=O,则(\;).
$$
$$
A.
A=B=O \quad B.A+B=O \quad C.\begin{vmatrix}A\end{vmatrix}=0或\begin{vmatrix}B\end{vmatrix}=0 \quad D.\begin{vmatrix}A\end{vmatrix}+\begin{vmatrix}B\end{vmatrix}=0 \quad E. \quad F. \quad G. \quad H.
$$
$$
AB=O⇒\begin{vmatrix}AB\end{vmatrix}=0,即\begin{vmatrix}A\end{vmatrix}=0或\begin{vmatrix}B\end{vmatrix}=0.
$$



$$
设A,B\mathrm{为同阶方阵},则(AB)^n为(\;).
$$
$$
A.
A^nB^n \quad B.AB^nA^{n-1} \quad C.B^nA^n \quad D.\underbrace{ABAB⋯ AB}_n \quad E. \quad F. \quad G. \quad H.
$$
$$
\mathrm{矩阵的乘法一般不满足交换律},\mathrm{且由方阵的幂运算可知}(AB)^n\;\;=\;\;\underbrace{ABAB⋯ AB}_n
$$



$$
设A,B\mathrm{均为}n\mathrm{阶方阵},且A(B-E)=O,则().
$$
$$
A.
A=O或B=E \quad B.\begin{vmatrix}A\end{vmatrix}=0或\begin{vmatrix}B-E\end{vmatrix}=0 \quad C.\begin{vmatrix}A\end{vmatrix}=0或\begin{vmatrix}B\end{vmatrix}=1 \quad D.A=BA \quad E. \quad F. \quad G. \quad H.
$$
$$
\begin{array}{l}∵ A(B-E)=O,∴\begin{vmatrix}A(B-E)\end{vmatrix}=\begin{vmatrix}A\end{vmatrix}\begin{vmatrix}B-E\end{vmatrix}=0\\即\begin{vmatrix}A\end{vmatrix}=0或\begin{vmatrix}B-E\end{vmatrix}=0.\end{array}
$$



$$
设A,B\mathrm{均为}4\mathrm{阶方阵},且\begin{vmatrix}A\end{vmatrix}=2,\begin{vmatrix}B\end{vmatrix}=-5,则\begin{vmatrix}-AB\end{vmatrix}\mathrm{等于}(\;).
$$
$$
A.
30 \quad B.-30 \quad C.10 \quad D.-10 \quad E. \quad F. \quad G. \quad H.
$$
$$
\begin{vmatrix}-AB\end{vmatrix}=(-1)^4\begin{vmatrix}A\end{vmatrix}\begin{vmatrix}B\end{vmatrix}=2×(-5)=-10
$$



$$
\mathrm{设矩阵}A\mathrm{为三阶矩阵},\mathrm{且已知}\begin{vmatrix}A\end{vmatrix}=m,则\begin{vmatrix}-mA\end{vmatrix}=(\;).
$$
$$
A.
-m^4 \quad B.m^4 \quad C.-m^2 \quad D.m^2 \quad E. \quad F. \quad G. \quad H.
$$
$$
\begin{array}{l}\\\left|-mA\right|=(-m)^3·\left|A\right|=-m^4\end{array}
$$



$$
设A为2\mathrm{阶方阵},且\left|A\right|=4,则\left|\begin{pmatrix}\frac12A\end{pmatrix}^2\right|=(\;).
$$
$$
A.
2 \quad B.1 \quad C.4 \quad D.-1 \quad E. \quad F. \quad G. \quad H.
$$
$$
\left|\begin{pmatrix}\frac12A\end{pmatrix}^2\right|=\left|\frac14A^2\right|=\begin{pmatrix}\frac14\end{pmatrix}^2\left|A^2\right|=\frac1{16}\left|A\right|\left|A\right|=1
$$



$$
设A是4\mathrm{阶方阵},\mathrm{且行列式}\left|A\right|=8,B=-\frac12A,则\left|B\right|=(\;).
$$
$$
A.
-4 \quad B.4 \quad C.-\frac12 \quad D.\frac12 \quad E. \quad F. \quad G. \quad H.
$$
$$
B=-\frac12A⇒\left|B\right|=\begin{pmatrix}-\frac12\end{pmatrix}^4\left|A\right|=\frac12.
$$



$$
\begin{array}{l}设A\mathrm{是三阶方阵},且A^2=O,\mathrm{下列各式中},\mathrm{正确的有}(\;)个.\;\\(1)A=O\;\;\;\;\;\;\;\;\;\;\;(2)A^3=O\;\;\;\;\;\;\;\;\;\;\;\;(3)\left|A\right|=0\end{array}
$$
$$
A.
0个 \quad B.1个 \quad C.2个 \quad D.3个 \quad E. \quad F. \quad G. \quad H.
$$
$$
\begin{array}{l}A^2=O,A^3=A^2A=OA=O,且\left|A^2\right|=\left|A\right|^2=0⇒\left|A\right|=0.\\但A\;\mathrm{不一定为零},\mathrm{例如}A=\begin{pmatrix}0&1\\0&0\end{pmatrix}\mathrm{不等于零},但A^2=O.\end{array}
$$



$$
设A\mathrm{为五阶方阵},若\left|A\right|=3,则\left|-3A\right|=(\;).\;
$$
$$
A.
3^6 \quad B.3^5 \quad C.-3^6 \quad D.-3^5 \quad E. \quad F. \quad G. \quad H.
$$
$$
\left|-3A\right|=(-3)^5\left|A\right|=-3^6
$$



$$
若A,B为n\mathrm{阶方阵},\mathrm{满足}AB=O;\mathrm{则必有}(\;).
$$
$$
A.
A=O或B=O \quad B.\left|A\right|=0或\left|B\right|=0 \quad C.A+B=O \quad D.\mathrm{以上都不对} \quad E. \quad F. \quad G. \quad H.
$$
$$
\left|A\right|=0或\left|B\right|=0
$$



$$
若A,B为n\mathrm{阶方阵},\mathrm{则下列恒成立的是}(\;).\;
$$
$$
A.
\left|A+B\right|=\left|A\right|+\left|B\right| \quad B.AB=BA \quad C.\left|AB\right|=\left|BA\right| \quad D.(AB)^T=A^TB^T \quad E. \quad F. \quad G. \quad H.
$$
$$
\left|AB\right|=\left|BA\right|
$$



$$
设A,B\mathrm{为同阶方阵},则(AB)^2为(\;).
$$
$$
A.
A^2B^2 \quad B.AB^2A \quad C.B^2A^2 \quad D.ABAB \quad E. \quad F. \quad G. \quad H.
$$
$$
\mathrm{矩阵的乘法一般不满足交换律},\mathrm{且由方阵的幂运算可知}(AB)^2=ABAB
$$



$$
设A,B\mathrm{为同阶方阵},则(AB)^3为(\;).
$$
$$
A.
A^3B^3 \quad B.AB^3A^2 \quad C.B^3A^3 \quad D.ABABAB \quad E. \quad F. \quad G. \quad H.
$$
$$
\begin{array}{l}\mathrm{矩阵的乘法一般不满足交换律},\mathrm{且由方阵的幂运算可知}(AB)^3\end{array}=ABABAB
$$



$$
设A,B\mathrm{为同阶方阵},则(AB)^4为(\;).
$$
$$
A.
A^4B^4 \quad B.AB^4A^3 \quad C.ABABABAB \quad D.B^4A^4 \quad E. \quad F. \quad G. \quad H.
$$
$$
\mathrm{矩阵的乘法一般不满足交换律},\mathrm{且由方阵的幂运算可知}(AB)^4=\underbrace{ABAB… AB}_4\;\;\;
$$



$$
设A,B\mathrm{都是}2\mathrm{阶方阵},且\left|A\right|=2,\left|B\right|=-3,则\left|-3AB\right|\mathrm{等于}(\;).
$$
$$
A.
48 \quad B.54 \quad C.-48 \quad D.-54 \quad E. \quad F. \quad G. \quad H.
$$
$$
\left|-3AB\right|=(-3)^2\left|A\right|\left|B\right|=9×2×(-3)=-54
$$



$$
设A,B\mathrm{都是}3\mathrm{阶方阵},且\left|A\right|=2,\left|B\right|=-1,则\left|3AB\right|\mathrm{等于}(\;).
$$
$$
A.
48 \quad B.-48 \quad C.-54 \quad D.54 \quad E. \quad F. \quad G. \quad H.
$$
$$
\left|3AB\right|=3^3\left|A\right|\left|B\right|=27×2×(-1)=-54
$$



$$
设A,B\mathrm{都是}2\mathrm{阶方阵},且\left|A\right|=2,\left|B\right|=-3,则\left|-2AB\right|\mathrm{等于}(\;).
$$
$$
A.
48 \quad B.-48 \quad C.24 \quad D.-24 \quad E. \quad F. \quad G. \quad H.
$$
$$
\left|-2AB\right|=(-2)^2\left|A\right|\left|B\right|=4×2×(-3)=-24
$$



$$
设A,B\mathrm{均为}n\mathrm{阶方阵},\mathrm{则下列恒成立的是}\;\;(\;\;\;)
$$
$$
A.
\left|A+B\right|=\left|A\right|+\left|B\right| \quad B.AB=BA \quad C.\left|AB\right|=\left|A\right|\left|B\right| \quad D.(2A)^T=\frac12A^T \quad E. \quad F. \quad G. \quad H.
$$
$$
\left|AB\right|=\left|A\right|\left|B\right|
$$



$$
设A,B\mathrm{均为}n\mathrm{阶方阵},\mathrm{则下列恒成立的是}\;\;(\;\;\;).
$$
$$
A.
(AB)^k=A^kB^k \quad B.B^2-A^2=(B-A)(B+A) \quad C.\left|AB\right|=\left|BA\right| \quad D.\left|-A\right|=-\left|A\right| \quad E. \quad F. \quad G. \quad H.
$$
$$
\left|AB\right|=\left|BA\right|=\left|A\right|\left|B\right|
$$



$$
\mathrm{设矩阵}A\mathrm{为三阶矩阵},\mathrm{且已知}\left|A\right|=m,\;则\left|-2A\right|=(\;).\;
$$
$$
A.
-6m \quad B.-8m \quad C.-m^2 \quad D.-2m \quad E. \quad F. \quad G. \quad H.
$$
$$
\left|-2A\right|=(-2)^3× m=-8m
$$



$$
\mathrm{设矩阵}A\mathrm{为三阶矩阵},\mathrm{且已知}\left|A\right|=n\;,则\left|-3A\right|=(\;).
$$
$$
A.
-9n \quad B.-27n \quad C.-n^3 \quad D.-3n \quad E. \quad F. \quad G. \quad H.
$$
$$
\left|-3A\right|=(-3)^3\left|A\right|=-27n
$$



$$
设A,B\mathrm{均为}n\mathrm{阶方阵},\mathrm{则下列恒成立的是}\;\;(\;\;\;)
$$
$$
A.
若\left|A\right|=0,\;则A=O \quad B.若A^2=O,则A=O \quad C.若A\mathrm{是对称矩阵},则A^2\mathrm{也是对称矩阵} \quad D.(A+B)(A-B)=A^2-B^2 \quad E. \quad F. \quad G. \quad H.
$$
$$
∵ A^T=A,(A^2)^T=A^TA^T=(A^T)^2=A^2,∴ A^2\mathrm{是对称矩阵}
$$



$$
设A,B\mathrm{均为}n\mathrm{阶方阵},且A(B-2E)=O,则(\;).
$$
$$
A.
A=O或B=2E \quad B.\left|A\right|=0或\left|B-2E\right|=0 \quad C.\left|A\right|=0或\left|2B\right|=1 \quad D.A=BA \quad E. \quad F. \quad G. \quad H.
$$
$$
A(B-2E)=O,\mathrm{两边取行列式得}\left|A(B-2E)\right|=0⇒\left|A\right|=0或\left|B-2E\right|=0
$$



$$
设A是n\mathrm{阶方阵},\mathrm{且行列式}\left|A\right|=25,\mathrm{则行列式}\;\left|-4A\right|=(\;).
$$
$$
A.
(-4)^n·25 \quad B.4^n·25 \quad C.(-4)^{n-1}·25 \quad D.4^{n-1}·25 \quad E. \quad F. \quad G. \quad H.
$$
$$
\left|-4A\right|=(-4)^n\left|A\right|=(-4)^n·25
$$



$$
若A,B为n\mathrm{阶方阵},且AB=O;则\;(\;).\;
$$
$$
A.
A=O或B=O \quad B.(A-B)^2=A^2-B^2 \quad C.\left|A\right|=0或\left|B\right|=0 \quad D.\left|A\right|+\left|B\right|=0 \quad E. \quad F. \quad G. \quad H.
$$
$$
\mathrm{因为}A,B\mathrm{都是方阵},\mathrm{两边取行列式},\mathrm{即得答案}\left|A\right|=0或\left|B\right|=0
$$



$$
\mathrm{已知矩阵}A=\begin{pmatrix}10&20&15\\25&75&125\\18&36&27\end{pmatrix},\;则\;\left|A\right|=(\;).
$$
$$
A.
5^3×9 \quad B.5^3×9×35 \quad C.0 \quad D.\mathrm{以上都不对} \quad E. \quad F. \quad G. \quad H.
$$
$$
\left|A\right|=\begin{vmatrix}10&20&15\\25&75&125\\18&36&27\end{vmatrix}=5×25×9\begin{vmatrix}2&4&3\\1&3&5\\2&4&3\end{vmatrix}=0
$$



$$
设A是4\mathrm{阶方阵},\mathrm{且行列式}\left|A\right|=8,B=-\frac12A,则\left|B\right|=(\;).
$$
$$
A.
-4 \quad B.4 \quad C.-\frac12 \quad D.\frac12 \quad E. \quad F. \quad G. \quad H.
$$
$$
B=-\frac12A⇒\left|B\right|=(-\frac12)^4\left|A\right|=\frac12
$$



$$
\mathrm{设矩阵}A=\begin{pmatrix}1&1&0\\2&1&2\end{pmatrix},B=\begin{pmatrix}1&2\\-1&1\\1&-1\end{pmatrix},则\;\left|BA\right|为(\;).
$$
$$
A.
12 \quad B.-12 \quad C.0 \quad D.1 \quad E. \quad F. \quad G. \quad H.
$$
$$
BA=\begin{pmatrix}1&2\\-1&1\\1&-1\end{pmatrix}\begin{pmatrix}1&1&0\\2&1&2\end{pmatrix}=\begin{pmatrix}5&3&4\\1&0&2\\-1&0&-2\end{pmatrix}∴\left|BA\right|=0
$$



$$
\begin{array}{l}\mathrm{设矩阵}A=\begin{pmatrix}1&1\\0&1\\1&-3\end{pmatrix},B=\begin{pmatrix}1&2&1\\1&1&0\end{pmatrix},则\left|-3BA\right|=(\;).\\\end{array}
$$
$$
A.
-16 \quad B.16 \quad C.36 \quad D.-36 \quad E. \quad F. \quad G. \quad H.
$$
$$
∵\begin{pmatrix}1&2&1\\1&1&0\end{pmatrix}\begin{pmatrix}1&1\\0&1\\1&-3\end{pmatrix}=\begin{pmatrix}2&0\\1&2\end{pmatrix},∴\left|-3BA\right|=(-3)^2×4=36
$$



$$
\begin{array}{l}\mathrm{设矩阵}A=\begin{pmatrix}1&1\\1&0\\1&-1\end{pmatrix},B=\begin{pmatrix}1&2&1\\1&1&0\end{pmatrix},则\left|-2BA\right|=(\;).\\\end{array}
$$
$$
A.
-16 \quad B.16 \quad C.32 \quad D.-32 \quad E. \quad F. \quad G. \quad H.
$$
$$
\begin{array}{l}∵\begin{pmatrix}1&2&1\\1&1&0\end{pmatrix}\begin{pmatrix}1&1\\1&0\\1&-1\end{pmatrix}=\begin{pmatrix}4&0\\2&1\end{pmatrix},∴\left|-2BA\right|=(-2)^2×4=16\\\end{array}
$$



$$
\begin{array}{l}设3\mathrm{阶矩阵}A\;有\left|A\right|=2,\left|A+E\right|=3,则\left|-2A^2-2A\right|=(\;).\\\end{array}
$$
$$
A.
-48 \quad B.48 \quad C.12 \quad D.-12 \quad E. \quad F. \quad G. \quad H.
$$
$$
\left|-2A^2-2A\right|=\left|-2A(A+E)\right|=(-2)^3\left|A\right|\left|(A+E)\right|=(-2)^3×2×3=-48
$$



$$
\begin{array}{l}\mathrm{设矩阵}A=\begin{pmatrix}3&1\\4&a\end{pmatrix},B=\begin{pmatrix}1&0\\5&2\end{pmatrix},若\left|AB\right|=10,则a=(\;).\\\end{array}
$$
$$
A.
5 \quad B.2 \quad C.3 \quad D.1 \quad E. \quad F. \quad G. \quad H.
$$
$$
\left|AB\right|=(3a-4)×2=10⇒ a=3
$$



$$
\begin{array}{l}\mathrm{设矩阵}A=\begin{pmatrix}2&1\\3&a\end{pmatrix},B=\begin{pmatrix}1&0\\5&3\end{pmatrix},若\left|AB\right|=9,则a=(\;).\\\end{array}
$$
$$
A.
5 \quad B.2 \quad C.1 \quad D.3 \quad E. \quad F. \quad G. \quad H.
$$
$$
\left|AB\right|=(2a-3)×3=9⇒ a=3
$$



$$
\begin{array}{l}\mathrm{设矩阵}A=\begin{pmatrix}3&1\\1&a\end{pmatrix},B=\begin{pmatrix}1&5\\0&2\end{pmatrix},若\left|AB\right|=10,则a=(\;).\\\end{array}
$$
$$
A.
5 \quad B.2 \quad C.3 \quad D.1 \quad E. \quad F. \quad G. \quad H.
$$
$$
\left|AB\right|=(3a-1)×2=10⇒ a=2
$$



$$
\begin{array}{l}\mathrm{设矩阵}A=\begin{pmatrix}3&3&3\\1&2&3\\1&3&6\end{pmatrix},则\left|A^2\right|=(\;).\\\end{array}
$$
$$
A.
5 \quad B.6 \quad C.9 \quad D.1 \quad E. \quad F. \quad G. \quad H.
$$
$$
\left|A\right|=\begin{vmatrix}3&3&3\\1&2&3\\1&3&6\end{vmatrix}=3,∴\left|A^2\right|=\left|A\right|^2=9
$$



$$
\begin{array}{l}\mathrm{设矩阵}A=\begin{pmatrix}4&4&4\\1&2&3\\1&3&6\end{pmatrix},则\left|A^2\right|=(\;).\\\end{array}
$$
$$
A.
1 \quad B.6 \quad C.9 \quad D.16 \quad E. \quad F. \quad G. \quad H.
$$
$$
\left|A\right|=\begin{vmatrix}4&4&4\\1&2&3\\1&3&6\end{vmatrix}=4\;\;∴\left|A^2\right|=\left|A\right|^2=16
$$



$$
\begin{array}{l}\mathrm{设矩阵}A=\begin{pmatrix}4&4&4\\1&2&3\\1&3&6\end{pmatrix},则\left|A^3\right|=(\;).\\\end{array}
$$
$$
A.
1 \quad B.6 \quad C.16 \quad D.64 \quad E. \quad F. \quad G. \quad H.
$$
$$
\left|A\right|=\begin{vmatrix}4&4&4\\1&2&3\\1&3&6\end{vmatrix}=4\;\;∴\left|A^3\right|=\left|A\right|^3=64
$$



$$
\begin{array}{l}\mathrm{设矩阵}A=\begin{pmatrix}2&2&2\\1&2&3\\1&3&6\end{pmatrix},则\left|A^3\right|=(\;).\\\end{array}
$$
$$
A.
8 \quad B.\frac18 \quad C.-8 \quad D.-\frac18 \quad E. \quad F. \quad G. \quad H.
$$
$$
\left|A\right|=\begin{vmatrix}2&2&2\\1&2&3\\1&3&6\end{vmatrix}=2\;\;∴\left|A^3\right|=\left|A\right|^3=8
$$



$$
\begin{array}{l}设A=\begin{pmatrix}1&0&1\\-1&1&0\\0&0&2\end{pmatrix},\;B=\begin{pmatrix}1&2&2\\2&2&4\\3&0&0\end{pmatrix},则\left|AB\right|=(\;).\\\end{array}
$$
$$
A.
2 \quad B.24 \quad C.10 \quad D.12 \quad E. \quad F. \quad G. \quad H.
$$
$$
\left|AB\right|=\left|A\right|\left|B\right|=2×12=24.
$$



$$
设A是n\mathrm{阶矩阵},且\left|A\right|=2,\left|A-E\right|=3,则\left|A^2-A\right|=(\;).
$$
$$
A.
5 \quad B.6 \quad C.1 \quad D.0 \quad E. \quad F. \quad G. \quad H.
$$
$$
\left|A^2-A\right|=\left|A(\;A-E)\right|=\left|A\right|\left|A-E\right|=6.
$$



$$
\begin{array}{l}设4\mathrm{阶矩阵}A有\left|A\right|=2,\left|2A+E\right|=3,则\left|4A^2+2A\right|=(\;).\\\end{array}
$$
$$
A.
6 \quad B.48 \quad C.96 \quad D.24 \quad E. \quad F. \quad G. \quad H.
$$
$$
\left|4A^2+2A\right|=\left|2A\left(2A+E\right)\right|=\left|2A\right|·\left|2A+E\right|=2^4×2·3=96
$$



$$
\begin{array}{l}设A=\begin{pmatrix}1&1\\0&1\\1&-3\end{pmatrix},\;B=\begin{pmatrix}1&2&1\\1&1&0\end{pmatrix},则\left|-2BA\right|=(\;).\\\end{array}
$$
$$
A.
-16 \quad B.16 \quad C.4 \quad D.-4 \quad E. \quad F. \quad G. \quad H.
$$
$$
\begin{pmatrix}1&2&1\\1&1&0\end{pmatrix}\begin{pmatrix}1&1\\0&1\\1&-3\end{pmatrix}=\begin{pmatrix}2&0\\1&2\end{pmatrix},\;则\left|-2BA\right|=\left(-2\right)^2×4=16
$$



$$
\mathrm{设矩阵}A=\begin{pmatrix}1&0&0\\-1&3&0\\5&4&2\end{pmatrix},则\left|\left(4E-A\right)^T\left(4E-A\right)\right|=(\;).
$$
$$
A.
36 \quad B.6 \quad C.\frac1{36} \quad D.1 \quad E. \quad F. \quad G. \quad H.
$$
$$
\begin{array}{l}\left|\left(4E-A\right)^T\left(4E-A\right)\right|=\left|4E-A\right|^2.\\\left|4E-A\right|=\begin{vmatrix}3&0&0\\1&1&0\\-5&-4&2\end{vmatrix}=6,\;\left|\left(4E-A\right)^T\left(4E-A\right)\right|=36.\end{array}
$$



$$
\begin{array}{l}设A=\begin{pmatrix}1&1&0\\2&1&2\end{pmatrix},\;B=\begin{pmatrix}1&2\\-1&2\\1&-3\end{pmatrix},则\left|AB\right|为(\;\;).\\\end{array}
$$
$$
A.
-12 \quad B.12 \quad C.10 \quad D.-10 \quad E. \quad F. \quad G. \quad H.
$$
$$
\left|AB\right|=\begin{vmatrix}0&4\\3&0\end{vmatrix}=-12.
$$



$$
\mathrm{已知矩阵}A=\begin{pmatrix}2&0&0\\0&3&1\end{pmatrix},B=\begin{pmatrix}1&0&1\\2&1&3\\1&1&0\end{pmatrix},\mathrm{则矩阵}(AB)^T=(\;\;).
$$
$$
A.
\begin{pmatrix}2&7\\0&4\\2&9\end{pmatrix} \quad B.\begin{pmatrix}2&0&2\\7&4&9\end{pmatrix} \quad C.\begin{pmatrix}2&1&2\\7&4&9\end{pmatrix} \quad D.\begin{pmatrix}2&7\\1&4\\2&9\end{pmatrix} \quad E. \quad F. \quad G. \quad H.
$$
$$
(AB)^T=B^TA^T=\begin{pmatrix}1&2&1\\0&1&1\\1&3&0\end{pmatrix}\begin{pmatrix}2&0\\0&3\\0&1\end{pmatrix}=\begin{pmatrix}2&7\\0&4\\2&9\end{pmatrix}.
$$



$$
\mathrm{设矩阵}A=\begin{pmatrix}3&-1\\4&a\end{pmatrix},B=\begin{pmatrix}1&0\\5&2\end{pmatrix},若\left|AB\right|=8,则a=(\;).
$$
$$
A.
\frac43 \quad B.\frac{16}3 \quad C.\frac83 \quad D.0 \quad E. \quad F. \quad G. \quad H.
$$
$$
\left|AB\right|=\left|A\right|\left|B\right|=8,又\left|B\right|=2,则\left|A\right|=4,即3a+4=4⇒ a=0.
$$



$$
设n\mathrm{阶矩阵}A\mathrm{满足}A^2=O,E是n\mathrm{阶单位矩阵},则(\;\;).\;\;
$$
$$
A.
\left|E-A\right|\neq0,\;但\left|E+A\right|=0 \quad B.\left|E-A\right|=0,\;但\left|E+A\right|\neq0 \quad C.\left|E-A\right|=0,\;且\left|E+A\right|=0 \quad D.\left|E-A\right|\neq0,\;且\left|E+A\right|\neq0 \quad E. \quad F. \quad G. \quad H.
$$
$$
\begin{array}{l}由A^2=O⇒ E^2-A^2=E^2,即\left(E+A\right)\left(E-A\right)=E^2,\\则\left|E+A\right|\left|E-A\right|=\left|E\right|^2=1⇒\left|E+A\right|\neq0且\left|E-A\right|\neq0.\end{array}
$$



$$
\mathrm{设矩阵}A=\begin{pmatrix}t&-1\\2&3\end{pmatrix},B=\begin{pmatrix}2&1\\3&4\end{pmatrix},若\left|AB\right|=25,则t=(\;\;\;).
$$
$$
A.
5 \quad B.2 \quad C.1 \quad D.-1 \quad E. \quad F. \quad G. \quad H.
$$
$$
\left|AB\right|=\left|A\right|\left|B\right|=25,\;又\left|B\right|=8-3=5,\;\;则\left|A\right|=3t+2=5⇒ t=1.
$$



$$
\mathrm{设矩阵}A=\begin{pmatrix}1&0\\2&1\end{pmatrix},则A^k为(\;\;\;).
$$
$$
A.
\begin{pmatrix}1&0\\2^k&1\end{pmatrix} \quad B.\begin{pmatrix}1&0\\2^{k-1}&1\end{pmatrix} \quad C.\begin{pmatrix}1&0\\2k&1\end{pmatrix} \quad D.\begin{pmatrix}1&0\\2(k-1)&1\end{pmatrix} \quad E. \quad F. \quad G. \quad H.
$$
$$
\begin{array}{l}\mathrm{此题可根据归纳法得出结论},\mathrm{然后加以验证}\\A=\begin{pmatrix}1&0\\2&1\end{pmatrix},A^2=\begin{pmatrix}1&0\\4&1\end{pmatrix}=\begin{pmatrix}1&0\\2×2&1\end{pmatrix},A^3=\begin{pmatrix}1&0\\6&1\end{pmatrix}=\begin{pmatrix}1&0\\2×3&1\end{pmatrix},…,A^k=\begin{pmatrix}1&0\\2k&1\end{pmatrix}\end{array}
$$



$$
设A=\begin{pmatrix}2&5\\1&-2\end{pmatrix},B=\begin{pmatrix}1&2\\3&-1\end{pmatrix},则\;A^2+9B^2-3AB-3BA=(\;).
$$
$$
A.
E \quad B.9E \quad C.-9E \quad D.-E \quad E. \quad F. \quad G. \quad H.
$$
$$
\begin{array}{l}\mathrm{原式}=(A-3B)\end{array}^2=\begin{pmatrix}-1&-1\\-8&1\end{pmatrix}^2=\begin{pmatrix}9&0\\0&9\end{pmatrix}=9E.
$$



$$
\mathrm{已知则}A=\begin{pmatrix}1\\2\\3\end{pmatrix}\begin{pmatrix}1&\frac12&\frac13\end{pmatrix},则A^5=(\;).
$$
$$
A.
3^4\begin{pmatrix}1&\frac12&\frac13\\2&1&\frac23\\3&\frac32&1\end{pmatrix} \quad B.3^5\begin{pmatrix}1&\frac12&\frac13\\2&1&\frac23\\3&\frac32&1\end{pmatrix} \quad C.\begin{pmatrix}1&\frac12&\frac13\\2&1&\frac23\\3&\frac32&1\end{pmatrix} \quad D.3^6\begin{pmatrix}1&\frac12&\frac13\\2&1&\frac23\\3&\frac32&1\end{pmatrix} \quad E. \quad F. \quad G. \quad H.
$$
$$
\begin{array}{l}∵\begin{pmatrix}1\\2\\3\end{pmatrix}\begin{pmatrix}1&\frac12&\frac13\end{pmatrix}=\begin{pmatrix}1&\frac12&\frac13\\2&1&\frac23\\3&\frac32&1\end{pmatrix},\begin{pmatrix}1&\frac12&\frac13\end{pmatrix}\begin{pmatrix}1\\2\\3\end{pmatrix}=3\\\;\;\;\;\;\;故A^5=3^4\begin{pmatrix}1&\frac12&\frac13\\2&1&\frac23\\3&\frac32&1\end{pmatrix}.\end{array}
$$



$$
\mathrm{已知矩阵}A=\begin{pmatrix}-1&1\\1&-1\end{pmatrix},则A^6=(\;).\;
$$
$$
A.
(-2)^5A \quad B.2^5A \quad C.-2^6A \quad D.2^6A \quad E. \quad F. \quad G. \quad H.
$$
$$
\begin{array}{l}A^2=\begin{pmatrix}-1&1\\1&-1\end{pmatrix}\begin{pmatrix}-1&1\\1&-1\end{pmatrix}=-2\begin{pmatrix}-1&1\\1&-1\end{pmatrix}=-2A\\\;\;\;\;\;\;\;\;A^3=A^2A=-2A^2=(-2)^2A,⋯,A^6=(-2)^5A.\end{array}
$$



$$
设A=\begin{pmatrix}1&-1&1\\1&1&1\\1&1&2\end{pmatrix},则\;\left|A^3\right|=(\;).
$$
$$
A.
-8 \quad B.8 \quad C.4 \quad D.-4 \quad E. \quad F. \quad G. \quad H.
$$
$$
\left|A\right|=2,\left|A^3\right|=\left|A\right|^3=8
$$



$$
\mathrm{设矩阵}A=\begin{pmatrix}1&-1&1&1\\1&1&-2&1\\1&1&0&2\\1&0&1&1\end{pmatrix},则\;\left|A^4\right|=(\;).
$$
$$
A.
81 \quad B.0 \quad C.1 \quad D.16 \quad E. \quad F. \quad G. \quad H.
$$
$$
\left|A\right|=\begin{vmatrix}1&-1&1&1\\1&1&-2&1\\1&1&0&2\\1&0&1&1\end{vmatrix}=-3,\left|A^4\right|=\left|A\right|^4=81
$$



$$
\mathrm{矩阵}A=\begin{pmatrix}3&5\\-4&3\end{pmatrix},B=\begin{pmatrix}1&1\\-2&1\end{pmatrix},则\;A^2+4B^2-2AB-2BA=(\;).
$$
$$
A.
\begin{pmatrix}-31&70\\-80&-31\end{pmatrix} \quad B.\begin{pmatrix}1&6\\0&1\end{pmatrix} \quad C.\begin{pmatrix}-19&22\\-8&5\end{pmatrix} \quad D.\begin{pmatrix}1&9\\0&1\end{pmatrix} \quad E. \quad F. \quad G. \quad H.
$$
$$
A^2+4B^2-2AB-2BA=(A-2B)^2=\begin{pmatrix}1&3\\0&1\end{pmatrix}\begin{pmatrix}1&3\\0&1\end{pmatrix}=\begin{pmatrix}1&6\\0&1\end{pmatrix}
$$



$$
\mathrm{设矩阵}A=\begin{pmatrix}1&2\\-2&0\\3&1\end{pmatrix},B=\begin{pmatrix}1&0&-1\\-2&1&3\end{pmatrix},则\;\left|AB\right|=(\;).
$$
$$
A.
2 \quad B.0 \quad C.1 \quad D.-1 \quad E. \quad F. \quad G. \quad H.
$$
$$
\begin{array}{l}AB=\begin{pmatrix}-3&2&5\\-2&0&2\\1&1&0\end{pmatrix},\left|AB\right|=\begin{vmatrix}-5&2&5\\-2&0&2\\0&1&0\end{vmatrix}=0\\\end{array}
$$



$$
\mathrm{设矩阵}A=\begin{pmatrix}1&1&1&0\\0&1&1&2\end{pmatrix},B=\begin{pmatrix}1&0\\-1&1\\0&1\\1&-1\end{pmatrix},则\;\left|AB\right|=(\;).
$$
$$
A.
-2 \quad B.-5 \quad C.6 \quad D.1 \quad E. \quad F. \quad G. \quad H.
$$
$$
AB=\begin{pmatrix}1&1&1&0\\0&1&1&2\end{pmatrix}\begin{pmatrix}1&0\\-1&1\\0&1\\1&-1\end{pmatrix}=\begin{pmatrix}0&2\\1&0\end{pmatrix},∴\left|AB\right|=-2
$$



$$
\begin{array}{l}\mathrm{矩阵}A=\frac12\begin{pmatrix}2&3&5\\3&-1&2\end{pmatrix}\begin{pmatrix}1&1\\1&-3\\3&3\end{pmatrix},则\left|A\right|\mathrm{的值为}(\;).\\\end{array}
$$
$$
A.
20 \quad B.44 \quad C.-12 \quad D.-44 \quad E. \quad F. \quad G. \quad H.
$$
$$
\begin{array}{l}A=\frac12\begin{pmatrix}20&8\\8&12\end{pmatrix}=\begin{pmatrix}10&4\\4&6\end{pmatrix}.\\\left|A\right|=44.\end{array}
$$



$$
\mathrm{设矩阵}A\mathrm{满足}A-\begin{pmatrix}1&0&-2\\0&1&-3\\3&0&1\end{pmatrix}=\begin{pmatrix}2&0\\2&1\\1&1\end{pmatrix}\begin{pmatrix}1&0&1\\2&2&1\end{pmatrix},则\left|A\right|=(\;).
$$
$$
A.
1 \quad B.0 \quad C.27 \quad D.-3 \quad E. \quad F. \quad G. \quad H.
$$
$$
A=\begin{pmatrix}2&0&2\\4&2&3\\3&2&2\end{pmatrix}+\begin{pmatrix}1&0&-2\\0&1&-3\\3&0&1\end{pmatrix}=\begin{pmatrix}3&0&0\\4&3&0\\6&2&3\end{pmatrix},∴\left|A\right|=27
$$



$$
\begin{array}{l}\mathrm{设矩阵}A=\begin{pmatrix}-4&5\\-3&4\end{pmatrix},则A^n=(\;).\\\end{array}
$$
$$
A.
A^n=\left\{\begin{array}{l}A,n=2k-1\\E,n=2k\end{array}\right. \quad B.A或E \quad C.-A或E \quad D.A^n=\left\{\begin{array}{l}E,n=2k-1\\A,n=2k\end{array}\right. \quad E. \quad F. \quad G. \quad H.
$$
$$
\begin{array}{l}\mathrm{由于}A^2=E=\begin{pmatrix}1&0\\0&1\end{pmatrix},\\\mathrm{所以},A^n=\left\{\begin{array}{l}A,n=2k-1\\E,n=2k\end{array}\right.,\mathrm{其中}\;k\mathrm{是自然数}.\end{array}
$$



$$
\mathrm{设行矩阵}C=\begin{pmatrix}\frac12,0,⋯,0,\frac12\end{pmatrix},\mathrm{矩阵}A=E-C^TC\;,\;B=E+2C^TC,\mathrm{其中}A,B为n\mathrm{阶矩阵},E为n\mathrm{阶单位矩阵},则AB\mathrm{等于}(\;).
$$
$$
A.
O \quad B.-E \quad C.E \quad D.E+C^TC \quad E. \quad F. \quad G. \quad H.
$$
$$
\begin{array}{l}\mathrm{由题意可知}CC^T=\frac12,则\\AB=(E-C^TC)(E+2C^TC)\\=E+2C^TC-C^TC-2C^TCC^TC=E+C^TC-2C^T\frac12C=E\end{array}
$$



$$
设A=\begin{pmatrix}1&0&0\\-1&2&0\\5&4&2\end{pmatrix},则\left|(4E-A)^T(4E-A)\right|=(\;).
$$
$$
A.
-144 \quad B.144 \quad C.12 \quad D.-12 \quad E. \quad F. \quad G. \quad H.
$$
$$
\begin{array}{l}\left|(4E-A)^T(4E-A)\right|=\left|(4E-A)^T\right|\left|(4E-A)\right|=\left|4E-A\right|^2.\\\left|4E-A\right|=\begin{vmatrix}3&0&0\\1&2&0\\-5&-4&2\end{vmatrix}=12,\left|(4E-A)^T\right|\left|(4E-A)\right|=144.\end{array}
$$



$$
\mathrm{设行矩阵}A=\begin{pmatrix}a_1&a_2&a_3\end{pmatrix},B=\begin{pmatrix}b_1&b_2&b_3\end{pmatrix},则\left|A^TB\right|=(\;).
$$
$$
A.
1 \quad B.0 \quad C.\mathrm{常数}c \quad D.\mathrm{无法确定} \quad E. \quad F. \quad G. \quad H.
$$
$$
\left|A^TB\right|=\left|\begin{pmatrix}a_1\\a_2\\a_3\end{pmatrix}\begin{pmatrix}b_1&b_2&b_3\end{pmatrix}\right|=b_1b_2b_3\begin{vmatrix}a_1&a_1&a_1\\a_2&a_2&a_2\\a_3&a_3&a_3\end{vmatrix}=0
$$



$$
设A为n\mathrm{阶反对称矩阵},B为n\mathrm{阶对称矩阵},\;\mathrm{则下列为反对称矩阵的是}\;\;(\;\;\;)
$$
$$
A.
BA-AB \quad B.BA+AB \quad C.(BA)^2 \quad D.ABA \quad E. \quad F. \quad G. \quad H.
$$
$$
\begin{array}{l}(BA-AB)^T=(BA)^T-(AB)^T=A^TB^T-B^TA^T=-AB+BA\\(BA+AB)^T=(BA)^T+(AB)^T=A^TB^T+B^TA^T=-AB-BA\\=-(BA+AB)\end{array}
$$



$$
\mathrm{设矩阵}A=\begin{pmatrix}1&-1&1&1\\1&1&-2&1\\1&1&0&2\\1&0&1&1\end{pmatrix},则\left|A^3\right|=(\;).
$$
$$
A.
9 \quad B.-9 \quad C.-27 \quad D.27 \quad E. \quad F. \quad G. \quad H.
$$
$$
\left|A\right|=\begin{vmatrix}1&-1&1&1\\1&1&-2&1\\1&1&0&2\\1&0&1&1\end{vmatrix}=-3,\left|A^3\right|=\left|A\right|^3=-27
$$



$$
\mathrm{设矩阵}A=\begin{pmatrix}1&-1&1&1\\1&1&-2&1\\1&1&0&2\\1&0&1&1\end{pmatrix},则\left|A^2\right|=(\;).
$$
$$
A.
9 \quad B.-9 \quad C.27 \quad D.-27 \quad E. \quad F. \quad G. \quad H.
$$
$$
\left|A\right|=\begin{vmatrix}1&-1&1&1\\1&1&-2&1\\1&1&0&2\\1&0&1&1\end{vmatrix}=-3,\left|A^2\right|=\left|A\right|^2=9.
$$



$$
设A=\begin{pmatrix}x_1&b_1&c_1\\x_2&b_2&c_2\\x_3&b_3&c_3\end{pmatrix},B=\begin{pmatrix}y_1&b_1&c_1\\y_2&b_2&c_2\\y_3&b_3&c_3\end{pmatrix},且\left|A\right|=2,\left|B\right|=-3,则\left|A+B\right|\mathrm{等于}(\;).
$$
$$
A.
4 \quad B.-4 \quad C.16 \quad D.-16 \quad E. \quad F. \quad G. \quad H.
$$
$$
\begin{array}{l}A+B=\begin{bmatrix}x_1&b_1&c_1\\x_2&b_2&c_2\\x_3&b_3&c_3\end{bmatrix}+\begin{bmatrix}y_1&b_1&c_1\\y_2&b_2&c_2\\y_3&b_3&c_3\end{bmatrix}=\begin{bmatrix}x_1+y_1&2b_1&2c_1\\x_2+y_2&2b_2&2c_2\\x_3+y_3&2b_3&2c_3\end{bmatrix}\\\left|A+B\right|=\begin{vmatrix}x_1+y_1&2b_1&2c_1\\x_2+y_2&2b_2&2c_2\\x_3+y_3&2b_3&2c_3\end{vmatrix}=4(\left|A\right|+\left|B\right|)=-4.\end{array}
$$



$$
\mathrm{已知矩阵}A=\begin{pmatrix}1\\2\\3\end{pmatrix}\begin{pmatrix}1&\frac12&\frac13\end{pmatrix},则A^4=(\;).
$$
$$
A.
3^4\begin{pmatrix}1&\frac12&\frac13\\2&1&\frac23\\3&\frac32&1\end{pmatrix} \quad B.3^3\begin{pmatrix}1&\frac12&\frac13\\2&1&\frac23\\3&\frac32&1\end{pmatrix} \quad C.\begin{pmatrix}1&\frac12&\frac13\\2&1&\frac23\\3&\frac32&1\end{pmatrix} \quad D.3^6\begin{pmatrix}1&\frac12&\frac13\\2&1&\frac23\\3&\frac32&1\end{pmatrix} \quad E. \quad F. \quad G. \quad H.
$$
$$
∵\begin{pmatrix}1&\frac12&\frac13\end{pmatrix}\begin{pmatrix}1\\2\\3\end{pmatrix}=3,∴ A^4=3^3\begin{pmatrix}1&\frac12&\frac13\\2&1&\frac23\\3&\frac32&1\end{pmatrix}
$$



$$
\mathrm{已知矩阵}A=\begin{pmatrix}1\\2\\1\end{pmatrix}\begin{pmatrix}\frac12&1&\frac12\end{pmatrix},则A^4=(\;).
$$
$$
A.
3^3\begin{pmatrix}\frac12&1&\frac12\\1&2&1\\\frac12&1&\frac12\end{pmatrix} \quad B.3^3\begin{pmatrix}\frac12&1&\frac12\\1&2&1\\\frac12&1&1\end{pmatrix} \quad C.3^3\begin{pmatrix}1&1&\frac12\\1&2&1\\\frac12&1&\frac12\end{pmatrix} \quad D.3^3\begin{pmatrix}\frac12&\frac12&\frac12\\1&2&1\\\frac12&1&\frac12\end{pmatrix} \quad E. \quad F. \quad G. \quad H.
$$
$$
∵\begin{pmatrix}\frac12&1&\frac12\end{pmatrix}\begin{pmatrix}1\\2\\1\end{pmatrix}=3\;\;∴ A^4=3^3\begin{pmatrix}\frac12&1&\frac12\\1&2&1\\\frac12&1&\frac12\end{pmatrix}
$$



$$
\mathrm{已知矩阵}A=\begin{pmatrix}-1&1&1&-1\\1&-1&-1&1\\1&-1&-1&1\\-1&1&1&-1\end{pmatrix},则A^4=(\;).
$$
$$
A.
(-4)^3A \quad B.4{}^3A \quad C.-4^2A \quad D.-4A \quad E. \quad F. \quad G. \quad H.
$$
$$
\begin{array}{l}A^2=\begin{pmatrix}-1&1&1&-1\\1&-1&-1&1\\1&-1&-1&1\\-1&1&1&-1\end{pmatrix}\begin{pmatrix}-1&1&1&-1\\1&-1&-1&1\\1&-1&-1&1\\-1&1&1&-1\end{pmatrix}=\begin{pmatrix}4&-4&-4&4\\-4&4&4&-4\\-4&4&4&-4\\4&-4&-4&4\end{pmatrix}\\=-4\begin{pmatrix}-1&1&1&-1\\1&-1&-1&1\\1&-1&-1&1\\-1&1&1&-1\end{pmatrix}=-4A,\\A^4=A^2A^2=-4A(-4A)=(-4)^2A^2=(-4)^3A\;.\end{array}
$$



$$
设A=\begin{pmatrix}x_1&b_1&c_1\\x_2&b_2&c_2\\x_3&b_3&c_3\end{pmatrix},B=\begin{pmatrix}y_1&b_1&c_1\\y_2&b_2&c_2\\y_3&b_3&c_3\end{pmatrix},且\left|A\right|=2,\left|B\right|=-4,则\left|A+B\right|\mathrm{等于}(\;).
$$
$$
A.
8 \quad B.16 \quad C.-8 \quad D.-16 \quad E. \quad F. \quad G. \quad H.
$$
$$
\left|A+B\right|=\begin{vmatrix}x_1+y_1&2b_1&2c_1\\x_2+y_2&2b_2&2c_2\\x_3+y_3&2b_3&2c_3\end{vmatrix}=4(\left|A\right|+\left|B\right|)=-8.
$$



$$
\mathrm{设矩阵}A=\begin{pmatrix}1&0&0\\-1&3&0\\5&4&2\end{pmatrix},则\left|(3E-A)^T(3E-A)\right|=(\;).
$$
$$
A.
36 \quad B.6 \quad C.1 \quad D.0 \quad E. \quad F. \quad G. \quad H.
$$
$$
\left|3E-A\right|=\begin{vmatrix}2&0&0\\1&0&0\\-5&-4&1\end{vmatrix}=0,\left|(3E-A)^T(3E-A)\right|=\left|3E-A\right|^2=0
$$



$$
设A=\begin{pmatrix}1&-1&1&2\\1&1&1&1\\1&1&2&1\\1&0&1&-1\end{pmatrix},则\left|A^2\right|=(\;).
$$
$$
A.
25 \quad B.-25 \quad C.-5 \quad D.5 \quad E. \quad F. \quad G. \quad H.
$$
$$
\left|A\right|=-5,\left|A^2\right|=\left|A\right|^2=25
$$



$$
设A,B,C\mathrm{都是}n\mathrm{阶方阵},且AB=BA,AC=CA,则ABC=(\;).
$$
$$
A.
ACB \quad B.CBA \quad C.BCA \quad D.CAB \quad E. \quad F. \quad G. \quad H.
$$
$$
\mathrm{运用条件中的交换律可知}ABC=BAC=BCA,\mathrm{同理可证其它选项中的计算不正确}.
$$



$$
设A=\begin{pmatrix}1&0&0\\-1&2&0\\5&4&2\end{pmatrix},则\left|(3E-A)^T(3E-A)\right|=(\;).
$$
$$
A.
-4 \quad B.4 \quad C.16 \quad D.1 \quad E. \quad F. \quad G. \quad H.
$$
$$
\left|3E-A\right|=\begin{vmatrix}2&0&0\\1&1&0\\-5&-4&1\end{vmatrix}=2\;,\left|(3E-A)^T(3E-A)\right|=4
$$



$$
\mathrm{设矩阵}A=\begin{pmatrix}1&0&1\\0&1&0\\0&0&1\end{pmatrix},求A^4
$$
$$
A.
\begin{pmatrix}1&0&4\\0&1&0\\0&0&1\end{pmatrix} \quad B.\begin{pmatrix}1&0&1\\0&1&0\\0&0&1\end{pmatrix} \quad C.\begin{pmatrix}4&0&4\\0&4&0\\0&0&4\end{pmatrix} \quad D.\begin{pmatrix}4&0&1\\0&4&0\\0&0&4\end{pmatrix} \quad E. \quad F. \quad G. \quad H.
$$
$$
\begin{array}{l}∵ A^2=\begin{pmatrix}1&0&2\\0&1&0\\0&0&1\end{pmatrix},A^3=\begin{pmatrix}1&0&3\\0&1&0\\0&0&1\end{pmatrix},\mathrm{一般的设}A^{n-1}=\begin{pmatrix}1&0&n-1\\0&1&0\\0&0&1\end{pmatrix}\\\\A^n=A^{n-1}A=\begin{pmatrix}1&0&n\\0&1&0\\0&0&1\end{pmatrix},\;∴ A^4=\begin{pmatrix}1&0&4\\0&1&0\\0&0&1\end{pmatrix}\end{array}
$$



$$
\mathrm{设矩阵}A=\begin{pmatrix}1&0&1\\0&1&0\\0&0&1\end{pmatrix},求A^n
$$
$$
A.
\begin{pmatrix}1&0&n\\0&1&0\\0&0&1\end{pmatrix} \quad B.\begin{pmatrix}1&0&1\\0&1&0\\0&0&1\end{pmatrix} \quad C.\begin{pmatrix}n&0&n\\0&n&0\\0&0&n\end{pmatrix} \quad D.\begin{pmatrix}n&0&1\\0&n&0\\0&0&n\end{pmatrix} \quad E. \quad F. \quad G. \quad H.
$$
$$
∵ A^2=\begin{pmatrix}1&0&2\\0&1&0\\0&0&1\end{pmatrix},A^3=\begin{pmatrix}1&0&3\\0&1&0\\0&0&1\end{pmatrix},\mathrm{一般的设}A^{n-1}=\begin{pmatrix}1&0&n-1\\0&1&0\\0&0&1\end{pmatrix}\;,\;A^n=A^{n-1}A=\begin{pmatrix}1&0&n\\0&1&0\\0&0&1\end{pmatrix}
$$



$$
\begin{array}{l}设\;A,B,C\mathrm{都是}n\mathrm{阶方阵},且AB=BC=CA=E,则A^2+B^2+C^2=(\;\;).\end{array}
$$
$$
A.
3E \quad B.2E \quad C.E \quad D.O \quad E. \quad F. \quad G. \quad H.
$$
$$
\begin{array}{l}A^2=B^{-1}C^{-1}=(CB)^{-1}=E^{-1}=E,\mathrm{同理}B^2=E,C^2=E\\\mathrm{所以}A^2+B^2+C^2=3E\end{array}
$$



$$
\mathrm{已知矩阵}A=\begin{pmatrix}1&1&1\\2&2&2\\3&3&3\end{pmatrix}=\begin{pmatrix}1\\2\\3\end{pmatrix}\begin{array}{ccc}(1&1&1\end{array}),则A^{100}=(\;)
$$
$$
A.
6^{99}A \quad B.6^{100}A \quad C.6^{98}A \quad D.6^{101}A \quad E. \quad F. \quad G. \quad H.
$$
$$
\begin{array}{l}\mathrm{直接计算}A^2,A^3,A^{100}\mathrm{是复杂的},\mathrm{这里我们将}A\mathrm{转化为列矩阵乘行矩阵},\mathrm{这可利用结合律简化计算},\\\mathrm{因为}A=\begin{pmatrix}1&1&1\\2&2&2\\3&3&3\end{pmatrix}=\begin{pmatrix}1\\2\\3\end{pmatrix}\begin{array}{ccc}(1&1&1\end{array}),\\令P=\begin{pmatrix}1\\2\\3\end{pmatrix},Q=\begin{array}{ccc}(1&1&1\end{array}),则A=PQ,且\\QP=\begin{array}{ccc}(1&1&1\end{array})\begin{pmatrix}1\\2\\3\end{pmatrix}=6,\\\mathrm{于是}\\A^2=PQ· PQ=P(QP)Q=6PQ=6A\\\;\;\;\;\;A^4=A^2· A^2=6A·6A=6^2A^2=6^3A,\\\mathrm{一般地}\\A^{100}=PQ· PQ··· PQ=P(QP)(QP)···(QP)Q\\\;\;\;\;\;\;\;\;\;\;\;\;\;\;\;\;\;\;\;\;\;\;\;\;=(QP)^{99}· PQ=6^{99}A.\\\end{array}
$$



$$
设A=\begin{pmatrix}1&1&1&1\\1&1&-1&-1\\1&-1&1&-1\\1&-1&-1&1\end{pmatrix},则A^6=(\;\;).
$$
$$
A.
4^3E \quad B.4^2E \quad C.4E \quad D.E \quad E. \quad F. \quad G. \quad H.
$$
$$
\begin{array}{l}\mathrm{由于}A^2=AA=4E,A^3=A^2A=4A,A^4=(A^2)^2=16E\\则A^6=A^4A^2=4^3E.\end{array}
$$



$$
\begin{array}{l}\mathrm{若有等式}\\\;\;\;\;\;\;\;\;\;\;\;\;\;\;\;\;\;\;\;\;\;\;\begin{pmatrix}3&-1\\4&-1\end{pmatrix}=\begin{pmatrix}1&3\\2&5\end{pmatrix}\begin{pmatrix}1&1\\0&1\end{pmatrix}\begin{pmatrix}-5&3\\2&-1\end{pmatrix}及\begin{pmatrix}-5&3\\2&-1\end{pmatrix}\begin{pmatrix}1&3\\2&5\end{pmatrix}=E,\\则\begin{pmatrix}3&-1\\4&-1\end{pmatrix}^7=(\;\;).\\\end{array}
$$
$$
A.
\begin{pmatrix}15&-7\\28&-13\end{pmatrix} \quad B.\begin{pmatrix}-15&7\\28&-13\end{pmatrix} \quad C.\begin{pmatrix}15&7\\-28&-13\end{pmatrix} \quad D.\begin{pmatrix}15&-7\\-28&13\end{pmatrix} \quad E. \quad F. \quad G. \quad H.
$$
$$
\begin{array}{l}\begin{pmatrix}3&-1\\4&-1\end{pmatrix}^7\\=\begin{pmatrix}1&3\\2&5\end{pmatrix}\begin{pmatrix}1&1\\0&1\end{pmatrix}^7\begin{pmatrix}-5&3\\2&-1\end{pmatrix}\\=\begin{pmatrix}1&3\\2&5\end{pmatrix}\begin{pmatrix}1&7\\0&1\end{pmatrix}\begin{pmatrix}-5&3\\2&-1\end{pmatrix}\\=\begin{pmatrix}15&-7\\28&-13\end{pmatrix}\end{array}
$$



$$
设A=\begin{pmatrix}1&1\\0&1\end{pmatrix},B=\begin{pmatrix}2&7\\1&4\end{pmatrix},C=\begin{pmatrix}4&-7\\-1&2\end{pmatrix},则\;(A+A^2+A^3+···+A^{10})(BC-A)=(\;\;).
$$
$$
A.
\begin{pmatrix}0&-10\\0&0\end{pmatrix} \quad B.\begin{pmatrix}0&10\\0&0\end{pmatrix} \quad C.\begin{pmatrix}10&0\\0&10\end{pmatrix} \quad D.\begin{pmatrix}20&0\\0&-20\end{pmatrix} \quad E. \quad F. \quad G. \quad H.
$$
$$
\begin{array}{l}BC=E\\\mathrm{原式}=A(E+A+···+A^9)(E-A)=A-A^{11}=A-\begin{pmatrix}1&11\\0&1\end{pmatrix}=\begin{pmatrix}0&-10\\0&0\end{pmatrix}\end{array}
$$



$$
设A=\begin{pmatrix}1&0\\1&1\end{pmatrix},B\begin{pmatrix}2&1\\5&3\end{pmatrix},C=\begin{pmatrix}-3&1\\5&-2\end{pmatrix},\;则\;(A+A^2+···+A^9)(BC+A)=(\;\;).
$$
$$
A.
\begin{pmatrix}0&0\\11&0\end{pmatrix} \quad B.\begin{pmatrix}0&0\\-9&0\end{pmatrix} \quad C.\begin{pmatrix}0&0\\9&0\end{pmatrix} \quad D.\begin{pmatrix}0&-9\\0&0\end{pmatrix} \quad E. \quad F. \quad G. \quad H.
$$
$$
\begin{array}{l}BC=-E\\\mathrm{原式}=A(E+A+···+A^8)(A-E)=A^{10}-A=\begin{pmatrix}1&0\\10&1\end{pmatrix}-\begin{pmatrix}1&0\\1&1\end{pmatrix}=\begin{pmatrix}0&0\\9&0\end{pmatrix}\end{array}
$$



$$
\begin{array}{l}\mathrm{若有等式}\\\;\;\;\;\;\;\;\;\;\;\;\;\;\;\;\;\;\;\;\;\;\;\begin{pmatrix}-3&2\\-2&1\end{pmatrix}=\begin{pmatrix}1&3\\1&2\end{pmatrix}\begin{pmatrix}1&1\\0&1\end{pmatrix}\begin{pmatrix}2&-3\\-1&1\end{pmatrix}及\begin{pmatrix}2&-3\\-1&1\end{pmatrix}\begin{pmatrix}1&3\\1&2\end{pmatrix}=-E,\\则\begin{pmatrix}-3&2\\-2&1\end{pmatrix}^7=(\;\;).\\\\\;\;\;\;\;\;\;\;\;\;\end{array}
$$
$$
A.
\begin{pmatrix}-8&7\\7&-6\end{pmatrix} \quad B.\begin{pmatrix}-8&7\\-7&-6\end{pmatrix} \quad C.\begin{pmatrix}8&7\\-7&6\end{pmatrix} \quad D.\begin{pmatrix}-8&7\\-7&6\end{pmatrix} \quad E. \quad F. \quad G. \quad H.
$$
$$
\begin{array}{l}\begin{pmatrix}-3&2\\-2&1\end{pmatrix}^7\\=(-1)^6\begin{pmatrix}1&3\\1&2\end{pmatrix}\begin{pmatrix}1&1\\0&1\end{pmatrix}^7\begin{pmatrix}2&-3\\-1&1\end{pmatrix}\\\;\;\;\;\;\;=\begin{pmatrix}1&3\\1&2\end{pmatrix}\begin{pmatrix}1&7\\0&1\end{pmatrix}\begin{pmatrix}2&-3\\-1&1\end{pmatrix}\\=\begin{pmatrix}-8&7\\-7&6\end{pmatrix}.\end{array}
$$



$$
\begin{array}{l}设A=\begin{pmatrix}1&1\\0&1\end{pmatrix},B=\begin{pmatrix}0&1\\0&0\end{pmatrix},C=E+A+A^2+···+A^{10},则BC=(\;\;).\\\end{array}
$$
$$
A.
\begin{pmatrix}0&11\\0&1\end{pmatrix} \quad B.\begin{pmatrix}1&11\\0&1\end{pmatrix} \quad C.\begin{pmatrix}0&11\\0&0\end{pmatrix} \quad D.\begin{pmatrix}0&10\\0&0\end{pmatrix} \quad E. \quad F. \quad G. \quad H.
$$
$$
\begin{array}{l}BC=(A-E)(A^{10}+A^9+···+A+E)=A^{11}-E.\\\\\mathrm{所以}BC=A^{11}-E=\begin{pmatrix}1&11\\0&1\end{pmatrix}-\begin{pmatrix}1&0\\0&1\end{pmatrix}=\begin{pmatrix}0&11\\0&0\end{pmatrix}\end{array}
$$



$$
设A为n\mathrm{阶矩阵},n\mathrm{为奇数},且AA^T=E_n,\left|A\right|=1,则\left|A-E_n\right|=(\;\;).
$$
$$
A.
0 \quad B.1 \quad C.-1 \quad D.2 \quad E. \quad F. \quad G. \quad H.
$$
$$
\begin{array}{l}\begin{array}{l}AA^T=E_n,\left|A\right|=1,得\\\;\;\;\;\;\;\;\;\;\;\;\;\;\;\;\;\;\;\;\;\;\;\;\;\;\;\;\;\;\;\;\;\;\;\;\;\left|A-E_n\right|=\left|A-AA^T\right|=\left|A(E_n-A^T)\right|\\\;\;\;\;\;\;\;\;\;\;\;\;\;\;\;\;\;\;\;\;\;\;\;\;\;\;\;\;\;\;\;\;\;\;\;\;\;\;\;\;\;\;\;\;\;\;\;\;\;\;\;\;\;=\left|A\right|\left|E_n-A^T\right|=\left|E_n-A^T\right|\;\\\;\;\;\;\;\;\;\;\;\;\;\;\;\;\;\;\;\;\;\;\;\;\;\;\;\;\;\;\;\;\;\;\;\;\;\;\;\;\;\;\;\;\;\;\;\;\;\;\;\;\;\;\;=\left|E_n-A\right|=(-1)^n\left|A-E_n\right|\;\;\;\;\;\;\;\;\;\;\;\;\;\;\;\;\;\;\;\;\;\;\;\;\;\;\;\;\;\;\;\;\;\;\\\;\;\;\;\;\;\;\;\;\;\;\;\;\;\;\;\;\;\;\;\;\;\;\;\;\;\;\;\;\;\;\;\;\;\\由n\mathrm{为奇数},知\end{array}\\\;\;\;\;\;\;\;\;\;\;\;\;\;\;\;\;\;\;\;\;\;\;\;\;\;\;\;\;\;\;\;\;\left|A-E_n\right|=-\left|A-E_n\right|⇒\left|A-E_n\right|=0.\\\end{array}
$$



$$
设A=\begin{pmatrix}-11&4\\-30&11\end{pmatrix},则\;(A+E)(E-A+A^2)=(\;\;).
$$
$$
A.
\begin{pmatrix}-10&4\\-30&12\end{pmatrix} \quad B.\begin{pmatrix}10&4\\30&12\end{pmatrix} \quad C.\begin{pmatrix}12&4\\30&10\end{pmatrix} \quad D.\begin{pmatrix}-12&4\\-30&10\end{pmatrix} \quad E. \quad F. \quad G. \quad H.
$$
$$
\begin{array}{l}\mathrm{由于}A^2=\begin{pmatrix}-11&4\\-30&11\end{pmatrix}\begin{pmatrix}-11&4\\-30&11\end{pmatrix}=E,\\\mathrm{故原式}=A^3+E=A+E=\begin{pmatrix}-10&4\\-30&12\end{pmatrix}\end{array}
$$



$$
\begin{array}{l}\mathrm{设矩阵}A=\begin{pmatrix}1&0&1\\0&2&0\\1&0&1\end{pmatrix},\mathrm{正整数}\;n⩾2,则A^n-2A^{n-1}=(\;\;).\\\end{array}
$$
$$
A.
O \quad B.E \quad C.A \quad D.\mathrm{无法计算} \quad E. \quad F. \quad G. \quad H.
$$
$$
\begin{array}{l}A^2=\begin{pmatrix}1&0&1\\0&2&0\\1&0&1\end{pmatrix}\begin{pmatrix}1&0&1\\0&2&0\\1&0&1\end{pmatrix}=\begin{pmatrix}2&0&2\\0&4&0\\2&0&2\end{pmatrix}=2A,\\\;\;\;\;\;\;\;\;\;A^3=A^2· A=2A^2,···\\\mathrm{一般地},\mathrm{可得到如下递推公式}:\\\;\;\;\;\;\;\;\;\;\;\;\;\;\;\;\;\;\;A^n=A^2· A^{n-2}=2A^{n-1},\\故\;\;\;\;\;\;\;\;\;\;\;\;\;\;A^n-2A^{n-1}=O\end{array}
$$



$$
\begin{array}{l}\mathrm{下列矩阵的幂的计算中正确的有}\;(\;\;\;).\\(1)\begin{pmatrix}1&1\\0&0\end{pmatrix}^n=\begin{pmatrix}1&1\\0&0\end{pmatrix};\;\;\;\;\;\;\;\;\;\;\;\;\;\;\;\;\;\;\;\;\;\;\;\;\;\;\;\;\;\;(2)\begin{pmatrix}1&0\\λ&1\end{pmatrix}^n=\begin{pmatrix}1&0\\nλ&1\end{pmatrix};\\(3)\begin{pmatrix}a&0&0\\0&b&0\\0&0&c\end{pmatrix}^n=\begin{pmatrix}a^n&0&0\\0&b^n&0\\0&0&c^n\end{pmatrix};\;\;\;\;\;\;\;\;\;\;\;\;\;\;\;\;(4)\begin{pmatrix}λ&1&0\\0&λ&1\\0&0&λ\end{pmatrix}^n=\begin{pmatrix}λ^n&1&0\\0&λ^n&1\\0&0&λ^n\end{pmatrix}\end{array}
$$
$$
A.
1个 \quad B.2个 \quad C.3个 \quad D.4个 \quad E. \quad F. \quad G. \quad H.
$$
$$
\begin{array}{l}(1)\begin{pmatrix}1&1\\0&0\end{pmatrix}^2=\begin{pmatrix}1&1\\0&0\end{pmatrix}\begin{pmatrix}1&1\\0&0\end{pmatrix}=\begin{pmatrix}1&1\\0&0\end{pmatrix},\\\;\;\;\;\;\;\begin{pmatrix}1&1\\0&0\end{pmatrix}^3=\begin{pmatrix}1&1\\0&0\end{pmatrix}\begin{pmatrix}1&1\\0&0\end{pmatrix}^2=\begin{pmatrix}1&1\\0&0\end{pmatrix}\begin{pmatrix}1&1\\0&0\end{pmatrix}=\begin{pmatrix}1&1\\0&0\end{pmatrix},\\\mathrm{由此推测}\begin{pmatrix}1&1\\0&0\end{pmatrix}^n=\begin{pmatrix}1&1\\0&0\end{pmatrix}.\\(2)A^2=\begin{pmatrix}1&0\\λ&1\end{pmatrix}\begin{pmatrix}1&0\\λ&1\end{pmatrix}=\begin{pmatrix}1&0\\2λ&1\end{pmatrix},\\\;\;\;\;\;\;\;\;\;A^3=A^2A=\begin{pmatrix}1&0\\2λ&1\end{pmatrix}\begin{pmatrix}1&0\\λ&1\end{pmatrix}=\begin{pmatrix}1&0\\3λ&1\end{pmatrix},\\\mathrm{由数学归纳法原理知}:A^n=\begin{pmatrix}1&0\\nλ&1\end{pmatrix}.\\(3)\begin{pmatrix}a&0&0\\0&b&0\\0&0&c\end{pmatrix}^2=\begin{pmatrix}a&0&0\\0&b&0\\0&0&c\end{pmatrix}\begin{pmatrix}a&0&0\\0&b&0\\0&0&c\end{pmatrix}=\begin{pmatrix}a^2&0&0\\0&b^2&0\\0&0&c^2\end{pmatrix},······,\\\mathrm{由此易推测}\begin{pmatrix}a&0&0\\0&b&0\\0&0&c\end{pmatrix}^n=\begin{pmatrix}a^n&0&0\\0&b^n&0\\0&0&c^n\end{pmatrix}.\\(4)\mathrm{首先观察}\\\;\;\;\;\;\;\;\;\;\;\;\;\;\;A^2=\begin{pmatrix}λ&1&0\\0&λ&1\\0&0&λ\end{pmatrix}\begin{pmatrix}λ&1&0\\0&λ&1\\0&0&λ\end{pmatrix}=\begin{pmatrix}λ^2&2λ&1\\0&λ^2&2λ\\0&0&λ^2\end{pmatrix},\\\;\;\;\;\;\;\;\;\;\;\;\;\;\;A^3=A^2· A=\begin{pmatrix}λ^3&3λ^2&3λ\\0&λ^3&3λ^2\\0&0&λ^3\end{pmatrix},\\\mathrm{由此推断}\;\;\;\;A^k=\begin{pmatrix}λ^k&kλ^{k-1}&\frac{k(k-1)}2λ^{k-2}\\0&λ^k&kλ^{k-1}\\0&0&λ^k\end{pmatrix}(k\geq2).\end{array}
$$



$$
\mathrm{设矩阵}A=\begin{pmatrix}1&-2\\1&0\end{pmatrix},B=\begin{pmatrix}1&0\\2&1\end{pmatrix},则4A^2-B^2-2AB+2BA=(\;\;).
$$
$$
A.
\begin{pmatrix}3&-8\\4&-16\end{pmatrix} \quad B.\begin{pmatrix}3&-8\\5&-17\end{pmatrix} \quad C.\begin{pmatrix}3&-9\\4&-17\end{pmatrix} \quad D.\begin{pmatrix}3&-8\\4&-17\end{pmatrix} \quad E. \quad F. \quad G. \quad H.
$$
$$
(2A+B)(2A-B)=\left(2\begin{pmatrix}1&-2\\1&0\end{pmatrix}+\begin{pmatrix}1&0\\2&1\end{pmatrix}\right)\left(2\begin{pmatrix}1&-2\\1&0\end{pmatrix}-\begin{pmatrix}1&0\\2&1\end{pmatrix}\right)=\begin{pmatrix}3&-8\\4&-17\end{pmatrix}
$$



$$
\mathrm{设矩阵}A=\begin{pmatrix}1&-2\\1&0\end{pmatrix},B=\begin{pmatrix}1&0\\2&1\end{pmatrix},则4A^2+B^2+2AB+2BA=(\;\;).
$$
$$
A.
\begin{pmatrix}-7&-16\\16&-15\end{pmatrix} \quad B.\begin{pmatrix}-7&-16\\16&15\end{pmatrix} \quad C.\begin{pmatrix}-7&-16\\-16&-15\end{pmatrix} \quad D.\begin{pmatrix}7&-16\\16&-15\end{pmatrix} \quad E. \quad F. \quad G. \quad H.
$$
$$
\begin{array}{l}\mathrm{原式}=(2A+B)^2\\=\begin{pmatrix}3&-4\\4&1\end{pmatrix}\begin{pmatrix}3&-4\\4&1\end{pmatrix}=\begin{pmatrix}-7&-16\\16&-15\end{pmatrix}.\end{array}
$$



$$
设A=\begin{pmatrix}λ&1&0\\0&λ&1\\0&0&λ\end{pmatrix},则A^3=(\;\;).
$$
$$
A.
\begin{pmatrix}λ^3&3λ^2&3λ\\0&λ^3&3λ^2\\0&0&λ^3\end{pmatrix} \quad B.\begin{pmatrix}λ^3&1&0\\0&λ^3&1\\0&0&λ^3\end{pmatrix} \quad C.\begin{pmatrix}λ^3&3&0\\0&λ^3&3\\0&0&λ^3\end{pmatrix} \quad D.\begin{pmatrix}λ^3&3λ^2&3\\0&λ^3&3λ^2\\0&0&λ^3\end{pmatrix} \quad E. \quad F. \quad G. \quad H.
$$
$$
\begin{array}{l}A^2=\begin{pmatrix}λ&1&0\\0&λ&1\\0&0&λ\end{pmatrix}\begin{pmatrix}λ&1&0\\0&λ&1\\0&0&λ\end{pmatrix}=\begin{pmatrix}λ^2&2λ&1\\0&λ^2&2λ\\0&0&λ^2\end{pmatrix},\\A^3=A^2A=\begin{pmatrix}λ^2&2λ&1\\0&λ^2&2λ\\0&0&λ^2\end{pmatrix}\begin{pmatrix}λ&1&0\\0&λ&1\\0&0&λ\end{pmatrix}=\begin{pmatrix}λ^3&3λ^2&3λ\\0&λ^3&3λ^2\\0&0&λ^3\end{pmatrix}.\end{array}
$$



$$
\begin{array}{l}\mathrm{设列矩阵}X=(x_1,x_2,··· x_n)^T\mathrm{满足}X^TX=1,E为n\mathrm{阶单位矩阵},H=E-2XX^T,\mathrm{则下列说法正确的是}(\;\;).\\(1)H\mathrm{是对称矩阵};\;\;\;\;\;\;\;\;\;\;\;\;\;\;(2)HH^T=E;\;\;\;\;\;\;\;\;\;\;\;(3)HH^T=0.\end{array}
$$
$$
A.
(1)(2) \quad B.(1)(3) \quad C.(1)(2)(3) \quad D.(3) \quad E. \quad F. \quad G. \quad H.
$$
$$
\begin{array}{l}∵ H^T=(E-2XX^T)^T=E^T-2(XX^T)=E-2XX^T=H,\\∵ H\mathrm{是对称矩阵}.\\\;\;\;\;\;\;\;\;\;\;\;\;\;\;\;\;\;\;\;\;\;\;\;\;\;\;HH^T=H^2=(E-2XX^T)^2\\\;\;\;\;\;\;\;\;\;\;\;\;\;\;\;\;\;\;\;\;\;\;\;\;\;\;\;\;\;\;\;\;\;\;\;=E-4XX^T+4(XX^T)(XX^T)=E-4XX^T+4X(X^TX)X^T\\\;\;\;\;\;\;\;\;\;\;\;\;\;\;\;\;\;\;\;\;\;\;\;\;\;\;\;\;\;\;\;\;\;\;=E-4XX^T+4XX{}^{}T=E.\end{array}
$$



$$
\begin{array}{l}\mathrm{下列命题错误的是}(\;\;).\\\end{array}
$$
$$
A.
\mathrm{若矩阵}A\mathrm{与矩阵}B\mathrm{可交换},\mathrm{则矩阵}AB^2\mathrm{与矩阵}BA^2\mathrm{也可交换} \quad B.\mathrm{若矩阵}A-B\mathrm{与矩阵}A+B\mathrm{可交换},\mathrm{则矩阵}A\mathrm{与矩阵}B\mathrm{也可交换} \quad C.\mathrm{若矩阵}A\mathrm{与矩阵}B\mathrm{可交换},\mathrm{则矩阵}A^T\mathrm{与矩阵}B^T\mathrm{也可交换} \quad D.\mathrm{若矩阵}AB\mathrm{与矩阵}BA\mathrm{可交换},\mathrm{则矩阵}A\mathrm{与矩阵}B\mathrm{也可交换} \quad E. \quad F. \quad G. \quad H.
$$
$$
\begin{array}{l}(1)A,B\mathrm{可交换},则AB=BA,有(AB)^T=(BA)^T⇒ B^TA^T=A^TB^T即A^T,B^T\mathrm{可交换};\\且AB^2· BA^2=ABBBAA=BABBAA=BABABA=BAABAB=BAAABB=BA^2· AB^2\mathrm{即矩阵}AB^2\mathrm{与矩阵}BA^2\mathrm{也可交换};\\(2)\mathrm{若矩阵}A-B\mathrm{与矩阵}A+B\mathrm{可交换},则(A-B)(A+B)=(A+B)(A-B),即\\A^2-BA+AB+B^2=A^2+BA-AB+B^2,得AB=BA,\mathrm{则矩阵}A\mathrm{与矩阵}B\mathrm{也可交换};\\(3)\mathrm{若矩阵}AB\mathrm{与矩阵}BA\mathrm{可交换},则AB· BA=BA· AB,\mathrm{不能推出}AB=BA,\mathrm{即矩阵}A\mathrm{与矩阵}B\mathrm{不一定可交换}.\end{array}
$$



$$
设A=\begin{pmatrix}1&0&0\\0&2&0\\0&0&-1\end{pmatrix},B=\begin{pmatrix}1&2&0\\2&2&1\\1&1&1\end{pmatrix},则ABA=(\;\;).
$$
$$
A.
\begin{pmatrix}1&4&0\\4&8&-2\\-1&-2&1\end{pmatrix} \quad B.\begin{pmatrix}1&4&0\\4&8&-1\\-1&-2&1\end{pmatrix} \quad C.\begin{pmatrix}1&4&0\\2&8&-2\\-1&-2&1\end{pmatrix} \quad D.\begin{pmatrix}1&2&0\\4&8&-2\\-1&-2&1\end{pmatrix} \quad E. \quad F. \quad G. \quad H.
$$
$$
\begin{array}{l}ABA=\begin{pmatrix}1&0&0\\0&2&0\\0&0&-1\end{pmatrix}\begin{pmatrix}1&2&0\\2&2&1\\1&1&1\end{pmatrix}\begin{pmatrix}1&0&0\\0&2&0\\0&0&-1\end{pmatrix}\\=\begin{pmatrix}1&4&0\\4&8&-2\\-1&-2&1\end{pmatrix}\end{array}
$$



$$
设A,B\mathrm{均为}n\mathrm{阶非零矩阵},且AB=O,\mathrm{则下列说法正确的是}().
$$
$$
A.
\left|A\right|=0且\left|B\right|=0 \quad B.\left|A\right|=0或\left|B\right|=0 \quad C.A,B\mathrm{不均为奇异阵} \quad D.A,B\mathrm{均为非奇异阵} \quad E. \quad F. \quad G. \quad H.
$$
$$
由AB=0⇒\left|AB\right|=\left|A\right|\left|B\right|=0,则\left|A\right|=0或\left|B\right|=0.
$$



$$
设A,B\mathrm{均为}3\mathrm{阶方阵},且\vert A\vert=3,\vert B\vert=-2,则\vert-AB\vert\mathrm{等于}().
$$
$$
A.
6 \quad B.-6 \quad C.12 \quad D.-12 \quad E. \quad F. \quad G. \quad H.
$$
$$
\vert-AB\vert=(-1)^3\vert A\vert·\vert B\vert=6
$$



$$
\begin{array}{l}设2\mathrm{阶矩阵}A\;有\left|A\right|=2,\left|A+E\right|=3,则\left|-2A^2-2A\right|=(\;).\\\end{array}
$$
$$
A.
24 \quad B.-24 \quad C.12 \quad D.-12 \quad E. \quad F. \quad G. \quad H.
$$
$$
\left|-2A^2-2A\right|=\left|-2A(A+E)\right|=(-2)^2\left|A\right|\left|(A+E)\right|=4×2×3=24
$$



$$
\begin{array}{l}设4\mathrm{阶矩阵}A\;有\left|A\right|=2,\left|A+E\right|=3,则\left|-2A^2-2A\right|=(\;).\\\end{array}
$$
$$
A.
-96 \quad B.-48 \quad C.96 \quad D.48 \quad E. \quad F. \quad G. \quad H.
$$
$$
\left|-2A^2-2A\right|=\left|-2A(A+E)\right|=(-2)^4\left|A\right|\left|(A+E)\right|=16×2×3=96
$$



$$
设A=\begin{pmatrix}1&0&0\\-1&2&0\\5&4&2\end{pmatrix},则\left|(3E-A)^T(3E-A)\right|=(\;).
$$
$$
A.
-16 \quad B.16 \quad C.-4 \quad D.4 \quad E. \quad F. \quad G. \quad H.
$$
$$
\begin{array}{l}\left|(3E-A)^T(3E-A)\right|=\left|(3E-A)^T\right|\left|(3E-A)\right|=\left|3E-A\right|^2.\\\left|3E-A\right|=\begin{vmatrix}2&0&0\\1&1&0\\-5&-4&1\end{vmatrix}=2,\left|(3E-A)^T\right|\left|(3E-A)\right|=4.\end{array}
$$



$$
设A=\begin{pmatrix}1&0&0\\-1&2&0\\5&4&4\end{pmatrix},则\left|(3E-A)^T(3E-A)\right|=(\;).
$$
$$
A.
-4 \quad B.4 \quad C.-8 \quad D.16 \quad E. \quad F. \quad G. \quad H.
$$
$$
\begin{array}{l}\left|(3E-A)^T(3E-A)\right|=\left|(3E-A)^T\right|\left|(3E-A)\right|=\left|3E-A\right|^2.\\\left|3E-A\right|=\begin{vmatrix}2&0&0\\1&1&0\\-5&-4&-1\end{vmatrix}=-2,\left|(3E-A)^T\right|\left|(3E-A)\right|=4.\end{array}
$$



$$
设A=\begin{pmatrix}x_1&b_1&c_1\\x_2&b_2&c_2\\x_3&b_3&c_3\end{pmatrix},B=\begin{pmatrix}y_1&b_1&c_1\\y_2&b_2&c_2\\y_3&b_3&c_3\end{pmatrix},且\left|A\right|=2,\left|B\right|=-1,则\left|A+B\right|\mathrm{等于}(\;).
$$
$$
A.
-8 \quad B.16 \quad C.4 \quad D.-4 \quad E. \quad F. \quad G. \quad H.
$$
$$
\left|A+B\right|=\begin{vmatrix}x_1+y_1&2b_1&2c_1\\x_2+y_2&2b_2&2c_2\\x_3+y_3&2b_3&2c_3\end{vmatrix}=4(\left|A\right|+\left|B\right|)=4.
$$



$$
设A=\begin{pmatrix}x_1&b_1&c_1\\x_2&b_2&c_2\\x_3&b_3&c_3\end{pmatrix},B=\begin{pmatrix}y_1&b_1&c_1\\y_2&b_2&c_2\\y_3&b_3&c_3\end{pmatrix},且\left|A\right|=1,\left|B\right|=-1,则\left|A+B\right|\mathrm{等于}(\;).
$$
$$
A.
-8 \quad B.2 \quad C.4 \quad D.0 \quad E. \quad F. \quad G. \quad H.
$$
$$
\left|A+B\right|=\begin{vmatrix}x_1+y_1&2b_1&2c_1\\x_2+y_2&2b_2&2c_2\\x_3+y_3&2b_3&2c_3\end{vmatrix}=4(\left|A\right|+\left|B\right|)=0.
$$



$$
设A=\begin{pmatrix}x_1&b_1&c_1\\x_2&b_2&c_2\\x_3&b_3&c_3\end{pmatrix},B=\begin{pmatrix}y_1&b_1&c_1\\y_2&b_2&c_2\\y_3&b_3&c_3\end{pmatrix},且\left|A\right|=4,\left|B\right|=-1,则\left|A+B\right|\mathrm{等于}(\;).
$$
$$
A.
-8 \quad B.4 \quad C.-12 \quad D.12 \quad E. \quad F. \quad G. \quad H.
$$
$$
\left|A+B\right|=\begin{vmatrix}x_1+y_1&2b_1&2c_1\\x_2+y_2&2b_2&2c_2\\x_3+y_3&2b_3&2c_3\end{vmatrix}=4(\left|A\right|+\left|B\right|)=12.
$$



$$
设A=\begin{pmatrix}x_1&b_1&c_1\\x_2&b_2&c_2\\x_3&b_3&c_3\end{pmatrix},B=\begin{pmatrix}y_1&b_1&c_1\\y_2&b_2&c_2\\y_3&b_3&c_3\end{pmatrix},且\left|A\right|=5,\left|B\right|=-1,则\left|A+B\right|\mathrm{等于}(\;).
$$
$$
A.
-25 \quad B.25 \quad C.16 \quad D.-16 \quad E. \quad F. \quad G. \quad H.
$$
$$
\left|A+B\right|=\begin{vmatrix}x_1+y_1&2b_1&2c_1\\x_2+y_2&2b_2&2c_2\\x_3+y_3&2b_3&2c_3\end{vmatrix}=4(\left|A\right|+\left|B\right|)=16.
$$



$$
\mathrm{设矩阵}A=\begin{pmatrix}1&0\\2&1\end{pmatrix},则A^4为(\;\;).
$$
$$
A.
\begin{pmatrix}1&0\\2^4&1\end{pmatrix} \quad B.\begin{pmatrix}1&0\\2^{}&1\end{pmatrix} \quad C.\begin{pmatrix}1&0\\8&1\end{pmatrix} \quad D.\begin{pmatrix}1&0\\6&1\end{pmatrix} \quad E. \quad F. \quad G. \quad H.
$$
$$
\begin{array}{l}\mathrm{此题可根据归纳法得出结论},\mathrm{然后加以验证}\\A=\begin{pmatrix}1&0\\2&1\end{pmatrix},A^2=\begin{pmatrix}1&0\\4&1\end{pmatrix}=\begin{pmatrix}1&0\\2×2&1\end{pmatrix},A^3=\begin{pmatrix}1&0\\6&1\end{pmatrix}=\begin{pmatrix}1&0\\2×3&1\end{pmatrix},…,A^4=\begin{pmatrix}1&0\\2×4&1\end{pmatrix}\end{array}
$$



$$
\mathrm{设矩阵}A=\begin{pmatrix}1&0\\2&1\end{pmatrix},则A^7为\left(\;\;\;\right).
$$
$$
A.
\begin{pmatrix}1&0\\2^7&1\end{pmatrix} \quad B.\begin{pmatrix}1&0\\2^6&1\end{pmatrix} \quad C.\begin{pmatrix}1&0\\12&1\end{pmatrix} \quad D.\begin{pmatrix}1&0\\14&1\end{pmatrix} \quad E. \quad F. \quad G. \quad H.
$$
$$
\begin{array}{l}\mathrm{此题可根据归纳法得出结论},\mathrm{然后加以验证}\\A=\begin{pmatrix}1&0\\2&1\end{pmatrix},A^2=\begin{pmatrix}1&0\\4&1\end{pmatrix}=\begin{pmatrix}1&0\\2×2&1\end{pmatrix},A^3=\begin{pmatrix}1&0\\6&1\end{pmatrix}=\begin{pmatrix}1&0\\2×3&1\end{pmatrix},…,A^k=\begin{pmatrix}1&0\\2k&1\end{pmatrix}\end{array}
$$



$$
\mathrm{设矩阵}A=\begin{pmatrix}3&1\\1&-3\end{pmatrix},则A^7为(\;\;).
$$
$$
A.
10^3E \quad B.10^6A \quad C.10^3A \quad D.10^5E \quad E. \quad F. \quad G. \quad H.
$$
$$
A^2=\begin{pmatrix}3&1\\1&-3\end{pmatrix}\begin{pmatrix}3&1\\1&-3\end{pmatrix}=10E,\;\;A^6=10^3E,\;A^7=10^3A
$$



$$
\mathrm{设矩阵}A=\begin{pmatrix}3&1\\1&-3\end{pmatrix},则A^7为(\;\;).
$$
$$
A.
10^3E \quad B.10^6A \quad C.10^3A \quad D.10^5E \quad E. \quad F. \quad G. \quad H.
$$
$$
A^2=\begin{pmatrix}3&1\\1&-3\end{pmatrix}\begin{pmatrix}3&1\\1&-3\end{pmatrix}=10E,\;\;A^6=10^3E,\;A^7=10^3A
$$



$$
\mathrm{设矩阵}A=\begin{pmatrix}3&1\\1&-3\end{pmatrix},则A^{100}为(\;\;).
$$
$$
A.
10^{100}E \quad B.10^{50}A \quad C.10^{50}E \quad D.10^{25}E \quad E. \quad F. \quad G. \quad H.
$$
$$
A^2=\begin{pmatrix}3&1\\1&-3\end{pmatrix}\begin{pmatrix}3&1\\1&-3\end{pmatrix}=10E,\;\;A^6=10^3E,\;A^{100}=10^{50}E
$$



$$
\mathrm{设矩阵}A=\begin{pmatrix}3&1\\1&-3\end{pmatrix},则A^{50}为(\;\;).
$$
$$
A.
10^{50}E \quad B.10^{25}A \quad C.10^5A \quad D.10^{25}E \quad E. \quad F. \quad G. \quad H.
$$
$$
A^2=\begin{pmatrix}3&1\\1&-3\end{pmatrix}\begin{pmatrix}3&1\\1&-3\end{pmatrix}=10E,\;\;A^6=10^3E,\;A^{50}=10^{25}E
$$



$$
\mathrm{已知则}A=\begin{pmatrix}1\\2\\3\end{pmatrix}\begin{pmatrix}1&\frac12&\frac13\end{pmatrix},则A^6=(\;).
$$
$$
A.
3^5\begin{pmatrix}1&\frac12&\frac13\\2&1&\frac23\\3&\frac32&1\end{pmatrix} \quad B.3^6\begin{pmatrix}1&\frac12&\frac13\\2&1&\frac23\\3&\frac32&1\end{pmatrix} \quad C.\begin{pmatrix}1&\frac12&\frac13\\2&1&\frac23\\3&\frac32&1\end{pmatrix} \quad D.3^4\begin{pmatrix}1&\frac12&\frac13\\2&1&\frac23\\3&\frac32&1\end{pmatrix} \quad E. \quad F. \quad G. \quad H.
$$
$$
\begin{array}{l}∵\begin{pmatrix}1\\2\\3\end{pmatrix}\begin{pmatrix}1&\frac12&\frac13\end{pmatrix}=\begin{pmatrix}1&\frac12&\frac13\\2&1&\frac23\\3&\frac32&1\end{pmatrix},\begin{pmatrix}1&\frac12&\frac13\end{pmatrix}\begin{pmatrix}1\\2\\3\end{pmatrix}=3\\\;\;\;\;\;\;故A^6=3^5\begin{pmatrix}1&\frac12&\frac13\\2&1&\frac23\\3&\frac32&1\end{pmatrix}.\end{array}
$$



$$
\mathrm{已知则}A=\begin{pmatrix}1\\2\\3\end{pmatrix}\begin{pmatrix}1&\frac12&\frac13\end{pmatrix},则A^{10}=(\;).
$$
$$
A.
3^{11}\begin{pmatrix}1&\frac12&\frac13\\2&1&\frac23\\3&\frac32&1\end{pmatrix} \quad B.3^{10}\begin{pmatrix}1&\frac12&\frac13\\2&1&\frac23\\3&\frac32&1\end{pmatrix} \quad C.\begin{pmatrix}1&\frac12&\frac13\\2&1&\frac23\\3&\frac32&1\end{pmatrix} \quad D.3^9\begin{pmatrix}1&\frac12&\frac13\\2&1&\frac23\\3&\frac32&1\end{pmatrix} \quad E. \quad F. \quad G. \quad H.
$$
$$
\begin{array}{l}∵\begin{pmatrix}1\\2\\3\end{pmatrix}\begin{pmatrix}1&\frac12&\frac13\end{pmatrix}=\begin{pmatrix}1&\frac12&\frac13\\2&1&\frac23\\3&\frac32&1\end{pmatrix},\begin{pmatrix}1&\frac12&\frac13\end{pmatrix}\begin{pmatrix}1\\2\\3\end{pmatrix}=3\\\;\;\;\;\;\;故A^{10}=3^9\begin{pmatrix}1&\frac12&\frac13\\2&1&\frac23\\3&\frac32&1\end{pmatrix}.\end{array}
$$



$$
\mathrm{已知则}A=\begin{pmatrix}1\\2\\3\end{pmatrix}\begin{pmatrix}1&\frac12&\frac13\end{pmatrix},则A^{100}=(\;).
$$
$$
A.
3^{101}\begin{pmatrix}1&\frac12&\frac13\\2&1&\frac23\\3&\frac32&1\end{pmatrix} \quad B.3^{98}\begin{pmatrix}1&\frac12&\frac13\\2&1&\frac23\\3&\frac32&1\end{pmatrix} \quad C.\begin{pmatrix}1&\frac12&\frac13\\2&1&\frac23\\3&\frac32&1\end{pmatrix} \quad D.3^{99}\begin{pmatrix}1&\frac12&\frac13\\2&1&\frac23\\3&\frac32&1\end{pmatrix} \quad E. \quad F. \quad G. \quad H.
$$
$$
\begin{array}{l}∵\begin{pmatrix}1\\2\\3\end{pmatrix}\begin{pmatrix}1&\frac12&\frac13\end{pmatrix}=\begin{pmatrix}1&\frac12&\frac13\\2&1&\frac23\\3&\frac32&1\end{pmatrix},\begin{pmatrix}1&\frac12&\frac13\end{pmatrix}\begin{pmatrix}1\\2\\3\end{pmatrix}=3\\\;\;\;\;\;\;故A^{100}=3^{99}\begin{pmatrix}1&\frac12&\frac13\\2&1&\frac23\\3&\frac32&1\end{pmatrix}.\end{array}
$$



$$
\mathrm{已知则}A=\begin{pmatrix}2\\1\\-3\end{pmatrix}\begin{pmatrix}1&2&4\end{pmatrix},则A^{100}=(\;).
$$
$$
A.
8^{99}\begin{pmatrix}2&4&8\\1&2&4\\-3&-6&-12\end{pmatrix} \quad B.-8^{99}\begin{pmatrix}1&0&0\\0&1&0\\0&0&1\end{pmatrix} \quad C.8^{99}\begin{pmatrix}1&0&0\\0&1&0\\0&0&1\end{pmatrix} \quad D.-8^{99}\begin{pmatrix}2&4&8\\1&2&4\\-3&-6&-12\end{pmatrix} \quad E. \quad F. \quad G. \quad H.
$$
$$
\begin{array}{l}∵\begin{pmatrix}2\\1\\-3\end{pmatrix}\begin{pmatrix}1&2&4\end{pmatrix}=\begin{pmatrix}2&4&8\\1&2&4\\-3&-6&-12\end{pmatrix},\begin{pmatrix}1&2&4\end{pmatrix}\begin{pmatrix}2\\1\\-3\end{pmatrix}=-8\\\;\;\;\;\;\;故A^{100}=-8^{99}\begin{pmatrix}2&4&8\\1&2&4\\-3&-6&-12\end{pmatrix}\end{array}
$$



$$
\begin{array}{l}\mathrm{有矩阵}A_{3×2}\;,B_{2×3},C_{3×3},\mathrm{下列}(\;)\mathrm{运算可行}.\\\end{array}
$$
$$
A.
AC \quad B.BC \quad C.ACB \quad D.AB-BC \quad E. \quad F. \quad G. \quad H.
$$
$$
\mathrm{同阶矩阵即行列都相同的矩阵才能进行加减运算},\mathrm{且只有当左边矩阵的列数等于右边矩阵的行数时},\mathrm{两个矩阵才能进行乘法运算},故BC\mathrm{可行};
$$



$$
\mathrm{已知矩阵}A,B,C\;\mathrm{满足}AC=CB\;,\mathrm{其中}C=(c_{ij})_{s× n}\;,则A与B\mathrm{分别是}().
$$
$$
A.
A_{s× s}\;,\;B_{n× n} \quad B.A_{s× n}\;,\;B_{n× s} \quad C.A_{n× s}\;,\;B_{n× n} \quad D.A_{s× s}\;,\;B_{s× n} \quad E. \quad F. \quad G. \quad H.
$$
$$
\begin{array}{l}\mathrm{由矩阵乘法的条件可知},A\mathrm{的列数与}C\mathrm{的行数相等},B\mathrm{的行数与}C\mathrm{的列数相等},\;\;\mathrm{又由}AC=CB\mathrm{可得},A\mathrm{的行数与}C\mathrm{的行数相等},\\B\mathrm{的列数与}C\mathrm{的列数相等},则A_{s× s}\;,\;B_{n× n}.\end{array}
$$



$$
\mathrm{设矩阵}A_{m× n}\;,\;B_{k× l}且AB\mathrm{有意义},则\;().
$$
$$
A.
n=k \quad B.l=m \quad C.m=k \quad D.n=l \quad E. \quad F. \quad G. \quad H.
$$
$$
AB\mathrm{可以相乘的充要条件是矩阵}A\mathrm{的列数等于矩阵}B\mathrm{的行数},即n=k.
$$



$$
\begin{array}{l}A=\begin{pmatrix}1&-2&1\\2&-1&2\\0&2&4\end{pmatrix},\;B=\begin{pmatrix}2&1&2\\3&-1&4\\2&0&5\end{pmatrix},C=(c_{ij})=AB,则c_{13}=().\\\end{array}
$$
$$
A.
2 \quad B.1 \quad C.-1 \quad D.12 \quad E. \quad F. \quad G. \quad H.
$$
$$
c_{13}=1×2-2×4+1×5=-1.
$$



$$
\begin{array}{l}设A=\begin{pmatrix}1&-2&1\\2&-1&2\\-1&-2&4\end{pmatrix},\;B=\begin{pmatrix}2&-1&1\\3&1&-4\\1&-2&-1\end{pmatrix}\;,C=(c_{ij})=AB,则c_{32}=().\\\end{array}
$$
$$
A.
-9 \quad B.-11 \quad C.4 \quad D.2 \quad E. \quad F. \quad G. \quad H.
$$
$$
c_{32}=-1×(-1)-2×1+4×(-2)=-9
$$



$$
\begin{array}{l}设A=\begin{pmatrix}1&-2&1\\2&-1&2\\-1&-1&4\end{pmatrix},\;B=\begin{pmatrix}2&-1&1\\-3&1&4\\1&2&-1\end{pmatrix}\;,C=(c_{ij})=BA,则c_{23}=().\\\end{array}
$$
$$
A.
-4 \quad B.8 \quad C.-3 \quad D.15 \quad E. \quad F. \quad G. \quad H.
$$
$$
c_{23}=-3×1+1×2+4×4=15
$$



$$
\begin{array}{l}设A=\begin{pmatrix}1&2&1\\2&-1&2\\3&4&0\end{pmatrix},\;B=\begin{pmatrix}2&1&2\\3&-1&4\\2&0&5\end{pmatrix}\;,C=(c_{ij})=AB,则c_{23}=().\\\end{array}
$$
$$
A.
22 \quad B.10 \quad C.3 \quad D.-1 \quad E. \quad F. \quad G. \quad H.
$$
$$
c_{23}=2×2-1×4+2×5=10
$$



$$
\mathrm{若矩阵}A=\begin{pmatrix}1&2-x&3\\2&6&5z\end{pmatrix},B=\begin{pmatrix}1&x&3\\y&6&z-8\end{pmatrix},且A=B,则x,y,z\mathrm{分别为}().
$$
$$
A.
x=1\;,\;y=2\;,\;z=-2 \quad B.x=1\;,\;y=2\;,\;z=2 \quad C.x=2\;,\;y=2\;,\;z=-1 \quad D.x=2\;,\;y=2\;,\;z=1 \quad E. \quad F. \quad G. \quad H.
$$
$$
\begin{array}{l}∵\;2-x=x\;,\;2=y\;,\;5z=z-8\;,\\∴\;x=1\;,\;y=2\;,\;z=\;-2\;.\end{array}
$$



$$
\mathrm{已知}A=\begin{pmatrix}-1&2&3&1\\0&3&-2&1\\4&0&3&2\end{pmatrix},B=\begin{pmatrix}4&3&2&-1\\5&-3&0&1\\1&2&-5&0\end{pmatrix},\;\mathrm{则矩阵}3A-2B\mathrm{中第}3\mathrm{行第}3\mathrm{列元素为}().
$$
$$
A.
19 \quad B.6 \quad C.-6 \quad D.-4 \quad E. \quad F. \quad G. \quad H.
$$
$$
\begin{array}{l}3A-2B=3\begin{pmatrix}-1&2&3&1\\0&3&-2&1\\4&0&3&2\end{pmatrix}-2\begin{pmatrix}4&3&2&-1\\5&-3&0&1\\1&2&-5&0\end{pmatrix}\\=\begin{pmatrix}-3-8&6-6&9-4&3+2\\0-10&9+6&-6-0&3-2\\12-2&0-4&9+10&6-0\end{pmatrix}=\begin{pmatrix}-11&0&5&5\\-10&15&-6&1\\10&-4&19&6\end{pmatrix}.\end{array}
$$



$$
\mathrm{设矩阵}A=\begin{pmatrix}1&2&1&2\\2&1&2&1\\1&2&3&4\end{pmatrix},B=\begin{pmatrix}4&3&2&1\\-2&1&-2&1\\0&-1&0&-1\end{pmatrix},且C=2A+3B=(c_{ij}),则c_{24}=(\;).
$$
$$
A.
4 \quad B.-4 \quad C.-5 \quad D.5 \quad E. \quad F. \quad G. \quad H.
$$
$$
2A+3B=2\begin{pmatrix}2&4&2&4\\4&2&4&2\\2&4&6&8\end{pmatrix}+3\begin{pmatrix}4&3&2&1\\-2&1&-2&1\\0&-1&0&-1\end{pmatrix}=\begin{pmatrix}14&13&8&7\\-2&5&-2&5\\2&1&6&5\end{pmatrix};∴ c_{24}=5
$$



$$
设A,B\mathrm{都是}n\mathrm{阶方阵},则(A+B)^2=A^2+2AB+B^2\mathrm{的充要条件是}(\;).
$$
$$
A.
A=E \quad B.B=O \quad C.AB=BA \quad D.A=B \quad E. \quad F. \quad G. \quad H.
$$
$$
(A+B)^2=A^2+AB+BA+B^2,若(A+B)^2=A^2+2AB+B^2,则AB+BA=2AB,即AB=BA.
$$



$$
若A=\begin{pmatrix}2&3\\1&-2\\3&1\end{pmatrix},B=\begin{pmatrix}1&-2&-3\\2&-1&0\end{pmatrix},则AB\mathrm{的第}2\mathrm{行第}3\mathrm{列的元素为}(\;).
$$
$$
A.
0 \quad B.-3 \quad C.-6 \quad D.-9 \quad E. \quad F. \quad G. \quad H.
$$
$$
\begin{array}{l}AB=\begin{pmatrix}2&3\\1&-2\\3&1\end{pmatrix}\begin{pmatrix}1&-2&-3\\2&-1&0\end{pmatrix}\\=\begin{pmatrix}2×1+3×2&2×(-2)+3×(-1)&2×(-3)+3×0\\1×1+(-2)×2&1×(-2)+(-2)×(-1)&1×(-3)+(-2)×0\\3×1+1×2&3×(-2)+1×(-1)&3×(-3)+1×0\end{pmatrix}\\=\begin{pmatrix}8&-7&-6\\-3&0&-3\\5&-7&-9\end{pmatrix}\\\end{array}
$$



$$
\mathrm{设矩阵}A=\begin{pmatrix}1&2\\x&-1\end{pmatrix},B=\begin{pmatrix}2&y\\1&0\end{pmatrix},且AB=BA,则x\;,\;y\mathrm{分别为}().
$$
$$
A.
x=1,y=2 \quad B.x=-1,y=-2 \quad C.x=-1,y=2 \quad D.x=1,y=-2 \quad E. \quad F. \quad G. \quad H.
$$
$$
\begin{array}{l}\;由AB=BA,得\begin{pmatrix}1&2\\x&-1\end{pmatrix}\begin{pmatrix}2&y\\1&0\end{pmatrix}=\begin{pmatrix}2&y\\1&0\end{pmatrix}\begin{pmatrix}1&2\\x&-1\end{pmatrix}.\\即\begin{pmatrix}4&y\\2x-1&xy\end{pmatrix}=\begin{pmatrix}2+xy&4-y\\1&2\end{pmatrix}.\\则y=4-y,2x-1=1,\;\mathrm{解方程组},得x=1,y=2.\end{array}
$$



$$
\mathrm{设矩阵}A=\begin{pmatrix}1&2&1&2\\2&1&2&1\\1&2&3&4\end{pmatrix},B=\begin{pmatrix}4&3&2&1\\-2&1&-2&1\\0&-1&0&-1\end{pmatrix},若X=(x_{ij})\mathrm{满足}A+X=B,则x_{23}=().
$$
$$
A.
2 \quad B.-4 \quad C.4 \quad D.-2 \quad E. \quad F. \quad G. \quad H.
$$
$$
\begin{array}{l}X=B-A\\=\begin{pmatrix}4&3&2&1\\-2&1&-2&1\\0&-1&0&-1\end{pmatrix}-\begin{pmatrix}1&2&1&2\\2&1&2&1\\1&2&3&4\end{pmatrix}\\=\begin{pmatrix}3&1&1&-1\\-4&0&-4&0\\-1&-3&-3&-5\end{pmatrix}.∴ x_{23}=-4\end{array}
$$



$$
2\begin{pmatrix}1&0\\0&0\end{pmatrix}+4\begin{pmatrix}0&1\\0&0\end{pmatrix}+6\begin{pmatrix}0&0\\1&0\end{pmatrix}=().
$$
$$
A.
\begin{pmatrix}2&4\\6&8\end{pmatrix} \quad B.\begin{pmatrix}20&0\\0&1\end{pmatrix} \quad C.\begin{pmatrix}2&4\\6&0\end{pmatrix} \quad D.\begin{pmatrix}1&1\\1&20\end{pmatrix} \quad E. \quad F. \quad G. \quad H.
$$
$$
\mathrm{原式}=\begin{pmatrix}2&0\\0&0\end{pmatrix}+\begin{pmatrix}0&4\\0&0\end{pmatrix}+\begin{pmatrix}0&0\\6&0\end{pmatrix}=\begin{pmatrix}2&4\\6&0\end{pmatrix}
$$



$$
\mathrm{计算}3\begin{pmatrix}1&0&2\\3&-3&1\end{pmatrix}\;-2\begin{pmatrix}1&1&3\\4&-5&0\end{pmatrix}=(\;).
$$
$$
A.
\begin{pmatrix}1&-2&0\\1&1&3\end{pmatrix} \quad B.\begin{pmatrix}1&-2&0\\1&-1&3\end{pmatrix} \quad C.\begin{pmatrix}1&-2&0\\-1&1&3\end{pmatrix} \quad D.\begin{pmatrix}1&-2&0\\-1&1&-3\end{pmatrix} \quad E. \quad F. \quad G. \quad H.
$$
$$
\mathrm{原式}=\begin{pmatrix}1&-2&0\\1&1&3\end{pmatrix}
$$



$$
设A=\begin{pmatrix}1&-2&1\\2&-1&2\\0&2&4\end{pmatrix},B=\begin{pmatrix}2&1&2\\3&-1&4\\2&0&5\end{pmatrix},C=(c_{ij})=AB,则c_{22}=(\;).
$$
$$
A.
2 \quad B.0 \quad C.3 \quad D.-2 \quad E. \quad F. \quad G. \quad H.
$$
$$
c_{22}=2×1+(-1)×(-1)+2×0=3
$$



$$
\mathrm{设矩阵}A=(a_{ij})=\begin{pmatrix}2&3&2\\1&4&5\\2&0&1\end{pmatrix}-2\begin{pmatrix}1&0&1\\2&1&1\\1&0&0\end{pmatrix},则a_{32}=(\:).
$$
$$
A.
0 \quad B.1 \quad C.2 \quad D.3 \quad E. \quad F. \quad G. \quad H.
$$
$$
\begin{pmatrix}2&3&2\\1&4&5\\2&0&1\end{pmatrix}-\begin{pmatrix}2&0&12\\4&2&2\\2&0&0\end{pmatrix}=\begin{pmatrix}0&3&0\\-3&2&3\\0&0&1\end{pmatrix}
$$



$$
\mathrm{若矩阵}C=2\begin{pmatrix}2&3&5\\1&4&6\end{pmatrix}\;-3\begin{pmatrix}1&0&2\\1&1&3\end{pmatrix}=(c_{ij}),则c_{13}=(\;)
$$
$$
A.
3 \quad B.4 \quad C.5 \quad D.6 \quad E. \quad F. \quad G. \quad H.
$$
$$
2\begin{pmatrix}2&3&5\\1&4&6\end{pmatrix}\;-3\begin{pmatrix}1&0&2\\1&1&3\end{pmatrix}=\begin{pmatrix}4&6&10\\2&8&12\end{pmatrix}-\begin{pmatrix}3&0&6\\3&3&9\end{pmatrix}=\begin{pmatrix}1&6&4\\-1&5&3\end{pmatrix}
$$



$$
\mathrm{若矩阵}C=2\begin{pmatrix}3&4&5\\1&4&6\end{pmatrix}\;-3\begin{pmatrix}1&0&2\\1&1&3\end{pmatrix}=(c_{ij}),则c_{21}=(\;)
$$
$$
A.
-1 \quad B.3 \quad C.4 \quad D.5 \quad E. \quad F. \quad G. \quad H.
$$
$$
C=\begin{bmatrix}6&6&10\\2&8&12\end{bmatrix}\;-\begin{bmatrix}3&0&6\\3&3&9\end{bmatrix}=\begin{bmatrix}3&6&4\\-1&5&3\end{bmatrix}
$$



$$
\mathrm{设矩阵}A_{m× n},B_{n× m}(m\neq n),\mathrm{则下列运算结果不为}n\mathrm{阶方阵的是}(\;\;)
$$
$$
A.
BA \quad B.(BA)^T \quad C.A^TB^T \quad D.AB \quad E. \quad F. \quad G. \quad H.
$$
$$
\mathrm{根据矩阵相乘即可得出}
$$



$$
\mathrm{已知矩阵}A^{-1}=\begin{pmatrix}1&1&1\\1&1&-1\\1&-1&-1\end{pmatrix},则(A^T)^{-1}=(\;).
$$
$$
A.
\begin{pmatrix}1&1&1\\1&1&-1\\1&-1&-1\end{pmatrix} \quad B.\begin{pmatrix}1&1&1\\1&1&1\\1&-1&-1\end{pmatrix} \quad C.\begin{pmatrix}1&1&1\\1&1&-1\\1&1&-1\end{pmatrix} \quad D.\begin{pmatrix}1&1&1\\-1&1&1\\-1&-1&1\end{pmatrix} \quad E. \quad F. \quad G. \quad H.
$$
$$
(A^T)^{-1}\;=(A^{-1})^{T=}\begin{pmatrix}1&1&1\\1&1&-1\\1&-1&-1\end{pmatrix}
$$



$$
\mathrm{已知}A=\begin{pmatrix}1&1&1&1\\a&b&c&d\\a^2&b^2&c^2&d^2\end{pmatrix},\mathrm{其中}a<\;b<\;c<\;d,则\left(\right).\;
$$
$$
A.
\;\left|A^TA\right|<0 \quad B.\;\left|A^TA\right|>0 \quad C.\;\left|A^TA\right|\neq0 \quad D.\;\left|A^TA\right|=0 \quad E. \quad F. \quad G. \quad H.
$$
$$
\begin{array}{l}\;\;\mathrm{矩阵}A\mathrm{存在三阶子式}\begin{vmatrix}1&1&1\\a&b&c\\a^2&b^2&c^2\end{vmatrix}=(b-a)(c-a)(c-b)\neq0,\mathrm{且不存在四阶子式},故r(A_{3×4})=3<4,又\\\;\;r(A^TA)\leq min\{r(A),r(A^T)\}=3,而A^TA\mathrm{为四阶矩阵},故\vert A^TA\vert=0.\end{array}
$$



$$
\mathrm{下列矩阵中能与矩阵}A=\begin{pmatrix}1&0\\1&1\end{pmatrix}\mathrm{交换的为}(\;).
$$
$$
A.
\begin{pmatrix}1&2\\3&1\end{pmatrix} \quad B.\begin{pmatrix}1&4\\0&1\end{pmatrix} \quad C.\begin{pmatrix}3&1\\0&3\end{pmatrix} \quad D.\begin{pmatrix}2&0\\3&2\end{pmatrix} \quad E. \quad F. \quad G. \quad H.
$$
$$
\begin{array}{l}\mathrm{设与}A\mathrm{可交换的矩阵为}X=\begin{pmatrix}x_{11}&x_{12}\\x_{21}&x_{22}\end{pmatrix},则\\\begin{pmatrix}x_{11}&x_{12}\\x_{21}&x_{22}\end{pmatrix}\begin{pmatrix}1&0\\1&1\end{pmatrix}=\begin{pmatrix}1&0\\1&1\end{pmatrix}\begin{pmatrix}x_{11}&x_{12}\\x_{21}&x_{22}\end{pmatrix},即\begin{pmatrix}x_{11}+x_{12}&x_{12}\\x_{21}+x_{22}&x_{22}\end{pmatrix}=\begin{pmatrix}x_{11}&x_{12}\\x_{11}+x_{21}&x_{12}+x_{22}\end{pmatrix}\\即\;\;\;\;\;\;\;⇒ x_{11}=x_{22}=a,x_{12}=0,x_{21}=b.\\\mathrm{所以与}A\mathrm{可交换的矩阵为}\begin{pmatrix}a&0\\b&a\end{pmatrix}.\\\end{array}
$$



$$
\mathrm{下列矩阵中能与矩阵}A=\begin{pmatrix}2&2\\0&2\end{pmatrix}\mathrm{交换的为}(\;).
$$
$$
A.
\begin{pmatrix}1&0\\1&2\end{pmatrix} \quad B.\begin{pmatrix}3&1\\0&3\end{pmatrix} \quad C.\begin{pmatrix}5&0\\1&5\end{pmatrix} \quad D.\begin{pmatrix}1&0\\1&1\end{pmatrix} \quad E. \quad F. \quad G. \quad H.
$$
$$
\begin{array}{l}\begin{array}{l}\mathrm{设与}A\mathrm{可交换的矩阵为}X=\begin{pmatrix}x_{11}&x_{12}\\x_{21}&x_{22}\end{pmatrix},则\\\begin{pmatrix}x_{11}&x_{12}\\x_{21}&x_{22}\end{pmatrix}\begin{pmatrix}1&1\\0&1\end{pmatrix}=\begin{pmatrix}1&1\\0&1\end{pmatrix}\begin{pmatrix}x_{11}&x_{12}\\x_{21}&x_{22}\end{pmatrix},即\begin{pmatrix}x_{11}&x_{11}+x_{12}\\x_{21}&x_{21}+x_{22}\end{pmatrix}=\begin{pmatrix}x_{11}+x_{21}&x_{12}+x_{22}\\x_{21}&x_{22}\end{pmatrix}\end{array}\\即\;\;\;\;\;\;\;⇒ x_{11}=x_{22}=a,x_{12}=b,x_{21}=0.\\\mathrm{所以与}A\mathrm{可交换的矩阵为}\begin{pmatrix}a&b\\0&a\end{pmatrix}.\end{array}
$$



$$
\mathrm{下列矩阵中不能与矩阵}A=\begin{pmatrix}2&0\\2&2\end{pmatrix}\mathrm{交换的为}(\;).
$$
$$
A.
\begin{pmatrix}1&0\\3&1\end{pmatrix} \quad B.\begin{pmatrix}1&0\\4&1\end{pmatrix} \quad C.\begin{pmatrix}1&0\\-1&1\end{pmatrix} \quad D.\begin{pmatrix}3&1\\0&3\end{pmatrix} \quad E. \quad F. \quad G. \quad H.
$$
$$
\begin{array}{l}\begin{array}{l}\mathrm{设与}A\mathrm{可交换的矩阵为}X=\begin{pmatrix}x_{11}&x_{12}\\x_{21}&x_{22}\end{pmatrix},则\\\begin{pmatrix}x_{11}&x_{12}\\x_{21}&x_{22}\end{pmatrix}\begin{pmatrix}1&1\\0&1\end{pmatrix}=\begin{pmatrix}1&1\\0&1\end{pmatrix}\begin{pmatrix}x_{11}&x_{12}\\x_{21}&x_{22}\end{pmatrix},即\begin{pmatrix}x_{11}&x_{11}+x_{12}\\x_{21}&x_{21}+x_{22}\end{pmatrix}=\begin{pmatrix}x_{11}+x_{21}&x_{12}+x_{22}\\x_{21}&x_{22}\end{pmatrix}\end{array}\\即\;\;\;\;\;\;\;⇒ x_{11}=x_{22}=a,x_{12}=b,x_{21}=0.\\\mathrm{所以与}A\mathrm{可交换的矩阵为}\begin{pmatrix}a&b\\0&a\end{pmatrix}.\end{array}
$$



$$
设A=\begin{pmatrix}2&1&-2\\-2&1&1\\3&-2&1\end{pmatrix},B=\begin{pmatrix}1&2\\1&1\\1&2\end{pmatrix},则AB-2B=(\;).
$$
$$
A.
\begin{pmatrix}-1&-3\\-2&-3\\0&2\end{pmatrix} \quad B.\begin{pmatrix}-1&-3\\-2&-1\\0&2\end{pmatrix} \quad C.\begin{pmatrix}-1&-3\\-2&-3\\0&1\end{pmatrix} \quad D.\begin{pmatrix}-1&-3\\2&-3\\0&2\end{pmatrix} \quad E. \quad F. \quad G. \quad H.
$$
$$
AB-2B=\begin{pmatrix}1&1\\0&-1\\2&6\end{pmatrix}-\begin{pmatrix}2&4\\2&2\\2&4\end{pmatrix}=\begin{pmatrix}-1&-3\\-2&-3\\0&2\end{pmatrix}.
$$



$$
设A=\begin{pmatrix}2&1&-2\\-2&1&1\\3&-2&1\end{pmatrix},B=\begin{pmatrix}1&2\\1&1\\1&2\end{pmatrix},则AB-B=(\;).
$$
$$
A.
\begin{pmatrix}0&-1\\-1&-2\\1&3\end{pmatrix} \quad B.\begin{pmatrix}0&-1\\-1&-2\\1&4\end{pmatrix} \quad C.\begin{pmatrix}0&1\\-1&-2\\1&4\end{pmatrix} \quad D.\begin{pmatrix}0&-1\\-1&-2\\-1&4\end{pmatrix} \quad E. \quad F. \quad G. \quad H.
$$
$$
AB-B=\begin{pmatrix}1&1\\0&-1\\2&6\end{pmatrix}-\begin{pmatrix}1&2\\1&1\\1&2\end{pmatrix}=\begin{pmatrix}0&-1\\-1&-2\\1&4\end{pmatrix}.
$$



$$
\mathrm{计算}\begin{pmatrix}1&-1&2\\-1&1&0\end{pmatrix}\begin{pmatrix}1&2\\0&1\\-1&-2\end{pmatrix}+\begin{pmatrix}3&2\\1&1\end{pmatrix}=(\;).
$$
$$
A.
\begin{pmatrix}2&-1\\1&0\end{pmatrix} \quad B.\begin{pmatrix}2&1\\0&0\end{pmatrix} \quad C.\begin{pmatrix}2&-1\\0&1\end{pmatrix} \quad D.\begin{pmatrix}2&-1\\0&0\end{pmatrix} \quad E. \quad F. \quad G. \quad H.
$$
$$
\begin{array}{l}\begin{pmatrix}1&-1&2\\-1&1&0\end{pmatrix}\begin{pmatrix}1&2\\0&1\\-1&-2\end{pmatrix}+\begin{pmatrix}3&2\\1&1\end{pmatrix}\\=\begin{pmatrix}-1&-3\\-1&-1\end{pmatrix}+\begin{pmatrix}3&2\\1&1\end{pmatrix}=\begin{pmatrix}2&-1\\0&0\end{pmatrix}\end{array}
$$



$$
\mathrm{计算}\begin{pmatrix}1&-1&2\\-1&1&0\end{pmatrix}\begin{pmatrix}1&2\\0&1\\-1&-2\end{pmatrix}-\begin{pmatrix}-1&-2\\1&1\end{pmatrix}=(\;).
$$
$$
A.
\begin{pmatrix}1&-1\\-2&-2\end{pmatrix} \quad B.\begin{pmatrix}0&-1\\-1&-2\end{pmatrix} \quad C.\begin{pmatrix}0&-1\\-2&-2\end{pmatrix} \quad D.\begin{pmatrix}0&-2\\-2&-2\end{pmatrix} \quad E. \quad F. \quad G. \quad H.
$$
$$
\begin{array}{l}\begin{pmatrix}1&-1&2\\-1&1&0\end{pmatrix}\begin{pmatrix}1&2\\0&1\\-1&-2\end{pmatrix}-\begin{pmatrix}-1&-2\\1&1\end{pmatrix}\\=\begin{pmatrix}-1&-3\\-1&-1\end{pmatrix}-\begin{pmatrix}-1&-2\\1&1\end{pmatrix}=\begin{pmatrix}0&-1\\-2&-2\end{pmatrix}\end{array}
$$



$$
\mathrm{计算}\begin{pmatrix}1&-3&2\\-1&3&0\end{pmatrix}\begin{pmatrix}3&2\\0&1\\-1&-2\end{pmatrix}-\begin{pmatrix}3&2\\1&1\end{pmatrix}=(\;).
$$
$$
A.
\begin{pmatrix}-2&-7\\-4&0\end{pmatrix} \quad B.\begin{pmatrix}2&7\\4&0\end{pmatrix} \quad C.\begin{pmatrix}2&-7\\-4&0\end{pmatrix} \quad D.\begin{pmatrix}-2&-7\\4&0\end{pmatrix} \quad E. \quad F. \quad G. \quad H.
$$
$$
\mathrm{原式}=\begin{pmatrix}1&-5\\-3&1\end{pmatrix}-\begin{pmatrix}3&2\\1&1\end{pmatrix}=\begin{pmatrix}-2&-7\\-4&0\end{pmatrix}.
$$



$$
设A=\begin{pmatrix}2&1&-2\\-2&1&1\\3&-2&1\end{pmatrix},B=\begin{pmatrix}3&2\\1&1\\3&2\end{pmatrix},AB-2B=C=(c_{ij}),则c_{11}=(\;).
$$
$$
A.
-5 \quad B.5 \quad C.-3 \quad D.2 \quad E. \quad F. \quad G. \quad H.
$$
$$
AB-2Bc_{11}=∑_{k=1}^3a_{1k}b_{k1}-2b_{11}=-5.
$$



$$
设A=\begin{pmatrix}1&2&-3\\1&-2&1\end{pmatrix},B=\begin{pmatrix}1&2&1\\2&1&1\\3&2&1\end{pmatrix},则AB-3A=(\:).
$$
$$
A.
\begin{pmatrix}-7&-8&9\\-3&8&-3\end{pmatrix} \quad B.\begin{pmatrix}-7&-8&9\\-3&6&-3\end{pmatrix} \quad C.\begin{pmatrix}7&-8&9\\-3&8&-3\end{pmatrix} \quad D.\begin{pmatrix}-7&-8&9\\-3&6&3\end{pmatrix} \quad E. \quad F. \quad G. \quad H.
$$
$$
\begin{array}{l}AB-3A=\begin{pmatrix}-4&-2&0\\0&2&0\end{pmatrix}-\begin{pmatrix}3&6&9\\3&-6&3\end{pmatrix}\\=\begin{pmatrix}-7&-8&9\\-3&8&-3\end{pmatrix}.\end{array}
$$



$$
设A=\begin{pmatrix}3&3\\1&2\end{pmatrix},B=\begin{pmatrix}1&-1\\-2&1\end{pmatrix},则ABA=(\;).
$$
$$
A.
\begin{pmatrix}-6&9\\5&-8\end{pmatrix} \quad B.\begin{pmatrix}-9&-9\\-8&-7\end{pmatrix} \quad C.\begin{pmatrix}6&-9\\-5&8\end{pmatrix} \quad D.\begin{pmatrix}6&9\\5&8\end{pmatrix} \quad E. \quad F. \quad G. \quad H.
$$
$$
ABA=\begin{pmatrix}-3&0\\-3&1\end{pmatrix}\begin{pmatrix}3&3\\1&2\end{pmatrix}=\begin{pmatrix}-9&-9\\-8&-7\end{pmatrix}.
$$



$$
\mathrm{下列矩阵中能与矩阵}A=\begin{pmatrix}1&1\\0&1\end{pmatrix}\mathrm{交换的为}(\;).
$$
$$
A.
\begin{pmatrix}1&0\\0&2\end{pmatrix} \quad B.\begin{pmatrix}3&1\\0&3\end{pmatrix} \quad C.\begin{pmatrix}5&0\\1&5\end{pmatrix} \quad D.\begin{pmatrix}1&0\\1&1\end{pmatrix} \quad E. \quad F. \quad G. \quad H.
$$
$$
\begin{array}{l}\mathrm{设与}A\mathrm{可交换的矩阵为}X=\begin{pmatrix}x_{11}&x_{12}\\x_{21}&x_{22}\end{pmatrix},则\\\begin{pmatrix}x_{11}&x_{12}\\x_{21}&x_{22}\end{pmatrix}\begin{pmatrix}1&1\\0&1\end{pmatrix}=\begin{pmatrix}1&1\\0&1\end{pmatrix}\begin{pmatrix}x_{11}&x_{12}\\x_{21}&x_{22}\end{pmatrix},\\即\begin{pmatrix}x_{11}&x_{11}+x_{12}\\x_{21}&x_{21}+x_{22}\end{pmatrix}=\begin{pmatrix}x_{11}+x_{21}&x_{12}+x_{22}\\x_{21}&x_{22}\end{pmatrix},\\⇒ x_{11}=x_{22}=a,x_{12}=b,x_{21}=0.\\\mathrm{所以与}A\mathrm{可交换的矩阵为}\begin{pmatrix}a&b\\0&a\end{pmatrix}.\\\end{array}
$$



$$
\begin{array}{l}\mathrm{设矩阵}A=\begin{pmatrix}1&2\\4&3\end{pmatrix},B=\begin{pmatrix}x&1\\2&y\end{pmatrix},则A与B\mathrm{可交换的充分必要条件是}(\;).\\\end{array}
$$
$$
A.
x-y=1 \quad B.x-y=-1 \quad C.x=y \quad D.x=2y \quad E. \quad F. \quad G. \quad H.
$$
$$
\begin{array}{l}A与B\mathrm{可交换的充分必要条件是}AB=BA,即\\\begin{pmatrix}x+4&1+2y\\4x+6&4+3y\end{pmatrix}=\begin{pmatrix}x+4&2x+3\\2+4y&4+3y\end{pmatrix}\\\Leftrightarrow\left\{\begin{array}{l}1+2y=2x+3\\4x+6=2+4y\end{array}\right.,\mathrm{解之得}x-y=-1.\end{array}
$$



$$
\begin{array}{l}\mathrm{设矩阵}A=\begin{pmatrix}1&2\\0&3\end{pmatrix},B=\begin{pmatrix}a&b\\c&d\end{pmatrix},\mathrm{则当}a,b\mathrm{为任意常数},c,d\mathrm{分别为}(\;)\mathrm{时恒有}AB=BA.\\\end{array}
$$
$$
A.
c=d=0 \quad B.c=0,d=a+b \quad C.c=a+b,d=0 \quad D.c=d=a+b \quad E. \quad F. \quad G. \quad H.
$$
$$
\begin{array}{l}由AB=BA\mathrm{可得}\\\begin{pmatrix}1&2\\0&3\end{pmatrix}\begin{pmatrix}a&b\\c&d\end{pmatrix}=\begin{pmatrix}a&b\\c&d\end{pmatrix}\begin{pmatrix}1&2\\0&3\end{pmatrix}⇒\begin{pmatrix}a+2c&b+2d\\3c&3d\end{pmatrix}=\begin{pmatrix}a&2a+3b\\c&2c+3d\end{pmatrix},\\即\left\{\begin{array}{c}a+2c=a\\b+2d=2a+3b\\3c=c\\3d=2c+3d\end{array}\right.⇒\left\{\begin{array}{l}c=0\\d=a+b\end{array}\right.\end{array}
$$



$$
设A与B\mathrm{都是}n\mathrm{阶方阵},且AB=BA,AC=CA,\mathrm{则下列式子正确的是}(\;\;).
$$
$$
A.
ABC=BCA \quad B.ACB=ABC \quad C.BCA=ACB \quad D.CBA=ABC \quad E. \quad F. \quad G. \quad H.
$$
$$
\mathrm{运用条件中的交换律可知}ABC=BAC=BCA,\mathrm{同理可证其它选项中的计算不正确}.
$$



$$
设A_{m× n}\;,\;B_{n× m}(m\neq n),\mathrm{则下列运算结果不为}n\mathrm{阶方阵的是}(\;).
$$
$$
A.
BA \quad B.AB \quad C.(BA)^T \quad D.A^TB^T \quad E. \quad F. \quad G. \quad H.
$$
$$
\begin{array}{l}A_{m× n}B_{n× m}=(AB)_{m× m},故AB是m\mathrm{阶方阵}.\\\mathrm{其他选项都是}n\mathrm{阶方阵}.\end{array}
$$



$$
设A=\begin{pmatrix}x&0\\7&y\end{pmatrix},B=\begin{pmatrix}u&v\\y&2\end{pmatrix},C=\begin{pmatrix}3&-4\\x&v\end{pmatrix},且A+2B-C=O,\mathrm{则下列不正确的}(\;\;).
$$
$$
A.
x=-5 \quad B.y=-6 \quad C.u=4 \quad D.v=2 \quad E. \quad F. \quad G. \quad H.
$$
$$
\begin{array}{l}A+2B-C=\begin{pmatrix}x&0\\7&y\end{pmatrix}+2\begin{pmatrix}u&v\\y&2\end{pmatrix}-\begin{pmatrix}3&-4\\x&v\end{pmatrix}\\\;\;\;\;\;\;\;\;\;\;\;\;\;\;\;\;\;\;=\begin{pmatrix}x+2u-3&2v+4\\7+2y-x&y+4-v\end{pmatrix}=\begin{pmatrix}0&0\\0&0\end{pmatrix}\\\;\;\;\;\;\;\;\;\;\;\;\;\;\;\;\;\;\;\;\;⇒\left\{\begin{array}{l}x+2u-3=0\\7+2y-x=0\\2v+4=0\\y+4-v=0\end{array}\right.\\\;\;\;\;\;\;\;\;\;\;\;\;\;\;\;\;\;\;\;\;\;\;⇒ x=-5,y=-6,u=4,v=-2.\end{array}
$$



$$
\mathrm{计算}\begin{pmatrix}1&-2&3\\3&2&-1\\2&3&1\end{pmatrix}\begin{pmatrix}1&0&0\\0&3&0\\0&0&2\end{pmatrix}+\begin{pmatrix}1&0&0\\0&3&0\\0&0&2\end{pmatrix}\begin{pmatrix}3&2&1\\1&-2&1\\2&3&3\end{pmatrix}=(\;)
$$
$$
A.
\begin{pmatrix}4&-4&7\\6&0&1\\6&3&8\end{pmatrix} \quad B.\begin{pmatrix}4&4&7\\6&0&1\\6&3&8\end{pmatrix} \quad C.\begin{pmatrix}4&-4&7\\6&0&1\\6&15&8\end{pmatrix} \quad D.\begin{pmatrix}4&4&7\\6&0&1\\6&-3&8\end{pmatrix} \quad E. \quad F. \quad G. \quad H.
$$
$$
\mathrm{原式}=\begin{pmatrix}1&-6&6\\3&6&-2\\2&9&2\end{pmatrix}+\begin{pmatrix}3&2&1\\3&-6&3\\4&6&6\end{pmatrix}=\begin{pmatrix}4&-4&7\\6&0&1\\6&15&8\end{pmatrix}.
$$



$$
\mathrm{设矩阵}A=\begin{pmatrix}2&-2&3\\5&1&4\end{pmatrix},B=\begin{pmatrix}2&1&5\\-3&2&4\\1&3&1\end{pmatrix},C=\begin{pmatrix}1&1&5\\-3&1&4\\1&3&0\end{pmatrix},则AB-AC=(\;\;).
$$
$$
A.
\begin{pmatrix}2&-2&3\\5&1&4\end{pmatrix} \quad B.\begin{pmatrix}2&2&3\\5&1&4\end{pmatrix} \quad C.\begin{pmatrix}2&-2&3\\5&1&-4\end{pmatrix} \quad D.\begin{pmatrix}2&2&3\\5&1&-4\end{pmatrix} \quad E. \quad F. \quad G. \quad H.
$$
$$
\begin{array}{l}\mathrm{由于}B-C=E=\begin{pmatrix}1&0&0\\0&1&0\\0&0&1\end{pmatrix},\\AB-AC=A(B-C)=AE=A=\begin{pmatrix}2&-2&3\\5&1&4\end{pmatrix}.\end{array}
$$



$$
\mathrm{有矩阵}A_{3×2},B_{2×3},C_{3×3},\mathrm{下列}(\;\;)\mathrm{运算不可行}.
$$
$$
A.
CA \quad B.BC \quad C.ACB \quad D.AB-C \quad E. \quad F. \quad G. \quad H.
$$
$$
\mathrm{根据两矩阵相乘的条件},\mathrm{易知}ACB错
$$



$$
\begin{array}{l}设A,B,C\mathrm{是同阶矩阵},且A\mathrm{可逆},\mathrm{下列命题正确的是}\left(\;\;\;\right).\\(1)若AB=AC,则B=C;\;\\\;(2)若AB=CB,则A=C.\end{array}
$$
$$
A.
\left(1\right) \quad B.\left(2\right) \quad C.\left(1\right)\left(2\right) \quad D.\mathrm{都不正确} \quad E. \quad F. \quad G. \quad H.
$$
$$
\begin{array}{l}(1)\mathrm{正确}.\;\mathrm{因为若}AB=AC,\mathrm{等式两边左乘以}A^{-1},有\\\;\;\;\;\;\;\;\;\;\;\;\;\;\;\;\;\;\;\;\;\;\;\;\;\;\;\;\;\;\;\;\;\;\;\;\;\;\;\;\;\;\;A^{-1}AB=A^{-1}AC⇒ EB=EC⇒ B=C\;.\\(2)\mathrm{不正确}.\;\mathrm{例如},设\\\;\;\;\;\;\;\;\;\;\;\;\;\;\;\;\;\;\;\;\;\;\;\;\;\;\;\;\;\;\;\;\;\;\;\;\;\;\;\;\;\;A=\begin{pmatrix}1&2\\0&1\end{pmatrix},\;B=\begin{pmatrix}1&1\\1&1\end{pmatrix},C=\begin{pmatrix}3&0\\0&1\end{pmatrix}\\则\;\;\;\;\;\;\;\;\;\;\;\;\;\;\;\;\;\;\;\;\;\;\;\;\;\;\;\;\;\;\;\;\;\;\;\;AB=\begin{pmatrix}1&2\\0&1\end{pmatrix}\begin{pmatrix}1&1\\1&1\end{pmatrix}=\begin{pmatrix}3&3\\1&1\end{pmatrix}\\\;\;\;\;\;\;\;\;\;\;\;\;\;\;\;\;\;\;\;\;\;\;\;\;\;\;\;\;\;\;\;\;\;\;\;\;\;\;\;\;CB=\begin{pmatrix}3&0\\0&1\end{pmatrix}\begin{pmatrix}1&1\\1&1\end{pmatrix}=\begin{pmatrix}3&3\\1&1\end{pmatrix}\\\mathrm{显然有}AB=CB,但A\neq C\;.\end{array}
$$



$$
\mathrm{设方阵}A\mathrm{满足方程}aA^2+bA+cE=O,若A\mathrm{可逆},则A^{-1}=\left(\;\;\;\right)(a,b,c\mathrm{为常数},c\neq0).\;
$$
$$
A.
-\frac acA-\frac bcE \quad B.\frac acA+\frac bcE \quad C.-\frac bcA-\frac acE \quad D.\frac bcA+\frac acE \quad E. \quad F. \quad G. \quad H.
$$
$$
\begin{array}{l}由aA^2+bA+cE=O⇒ aA^2+bA=-cE,\\∵ c\neq0,\\∴-\frac acA^2-\frac bcA=E⇒\left(-\frac acA-\frac bcE\right)A=E,\\\mathrm{由定理}1\mathrm{的推论知},A\mathrm{可逆},且\\\;\;\;\;\;\;\;\;\;\;\;\;\;\;\;\;\;\;\;\;\;\;\;\;\;\;\;\;\;\;\;\;\;\;\;A^{-1}=-\frac acA-\frac bcE\end{array}
$$



$$
\begin{array}{l}\mathrm{已知矩阵方程}A^2+A+E=O,则\;\left(A-E\right)^{-1}=\left(\;\;\;\right).\\\end{array}
$$
$$
A.
\frac{A-2E}3 \quad B.\frac{A+2E}3 \quad C.-\frac{A-2E}3 \quad D.-\frac{A+2E}3 \quad E. \quad F. \quad G. \quad H.
$$
$$
\begin{array}{l}A^2+A-2E+3E=O⇒\left(A-E\right)\left(A+2E\right)=-3E,即\\\;\;\;\;\;\;\;\;\;\;\;\;\;\;\;\;\;\;\;\;\;\;\;\;\;\;\;\;\;\left(A-E\right)\left[-\frac13\left(A+2E\right)\right]=E,故A^{-1}=-\frac{A+2E}3\end{array}
$$



$$
\mathrm{已知}A为n\mathrm{阶矩阵},且A-E\mathrm{可逆},\mathrm{则下列式子正确的是}().\;
$$
$$
A.
(A-E)^{-1}(A-E)=(A-E)(A-E)^{-1} \quad B.(A-E)^{-1}=A^{-1}-E \quad C.\left|A\right|-1\neq0 \quad D.\left|A\right|+1\neq0 \quad E. \quad F. \quad G. \quad H.
$$
$$
\mathrm{矩阵}A-E\mathrm{可逆},则(A-E)^{-1}(A-E)=(A-E)(A-E)^{-1}=E.
$$



$$
\mathrm{已知}n\mathrm{阶矩阵}A\mathrm{满足}:2A^2+3A-5E=O.则A^{-1}=\left(\;\;\;\right).
$$
$$
A.
\frac{2A+3E}5 \quad B.\frac{2A-3E}5 \quad C.\frac{5A+3E}3 \quad D.\frac{-2A-3E}5 \quad E. \quad F. \quad G. \quad H.
$$
$$
\begin{array}{l}由2A^2+3A-5E=O⇒2A^2+3A=5E,2A^2+3A=5E⇒\frac{2A+3E}5A=E\\故A^{-1}=\frac{2A+3E}5\end{array}
$$



$$
设A,B,C\mathrm{是同阶矩阵},\mathrm{下列命题正确的是}(\;\;\;)
$$
$$
A.
若AB=AC,则B=C \quad B.若AB=O,则A=O,B=O \quad C.若AB=E,则A,B\mathrm{是互逆矩阵} \quad D.\mathrm{以上结论均不正确} \quad E. \quad F. \quad G. \quad H.
$$
$$
\mathrm{矩阵的乘法不满足消去律},\mathrm{所以}A,B\mathrm{答案不对},\mathrm{根据逆矩阵的定义},\mathrm{答案}C\mathrm{是正确的}
$$



$$
设A,B,C\mathrm{是同阶矩阵},若AB=AC\mathrm{必能推出}B=C,则A\mathrm{应满足条件}\left(\;\;\;\;\right).
$$
$$
A.
A\neq O \quad B.A=O \quad C.\left|A\right|=0 \quad D.\left|A\right|\neq0 \quad E. \quad F. \quad G. \quad H.
$$
$$
\mathrm{由矩阵乘法的性质以及逆矩阵的性质不难得到}.
$$



$$
\mathrm{矩阵}A=\begin{pmatrix}1&0&1\\2&1&0\\-3&2&-5\end{pmatrix},则\left(E-A\right)^{-1}=(\;\;).
$$
$$
A.
\begin{pmatrix}0&-1/2&0\\-3&-3/4&-1/2\\-1&0&0\end{pmatrix} \quad B.\begin{pmatrix}0&-1/2&1\\3&-3/4&-1/2\\1&0&0\end{pmatrix} \quad C.\begin{pmatrix}1&-1/2&1\\-3&-3/4&-1/2\\0&1&0\end{pmatrix} \quad D.\begin{pmatrix}1&-1/2&0\\3&3/4&1/2\\-1&0&0\end{pmatrix} \quad E. \quad F. \quad G. \quad H.
$$
$$
\begin{array}{l}A=\begin{pmatrix}1&0&1\\2&1&0\\-3&2&-5\end{pmatrix},E-A=\begin{pmatrix}0&0&-1\\-2&0&0\\3&-2&6\end{pmatrix}\\\left(E-A\vdots E\right)=\begin{pmatrix}0&0&-1&1&0&0\\-2&0&0&0&1&0\\3&-2&6&0&0&1\end{pmatrix}\rightarrow\begin{pmatrix}0&0&-1&1&0&0\\-2&0&0&0&1&0\\3&-2&6&0&0&1\end{pmatrix}\\\;\;\;\;\;\;\;\;\;\;\;\;\;\;\;\;\;\;\;\;\;\;\;\;\rightarrow\begin{pmatrix}-2&0&0&0&1&0\\3&-2&6&0&0&1\\0&0&-1&1&0&0\end{pmatrix}\rightarrow\begin{pmatrix}1&0&0&0&-1/2&0\\3&-2&6&0&0&1\\0&0&-1&1&0&0\end{pmatrix}\\\;\;\;\;\;\;\;\;\;\;\;\;\;\;\;\;\;\;\;\;\;\;\;\;\;\;\rightarrow\begin{pmatrix}1&0&0&0&-1/2&0\\0&-2&6&0&3/2&1\\0&0&-1&1&0&0\end{pmatrix}\rightarrow\begin{pmatrix}1&0&0&0&-1/2&0\\0&1&-3&0&-3/4&-1/2\\0&0&1&-1&0&0\end{pmatrix}\;\;\;\\\;\;\;\;\;\;\;\;\;\;\;\;\;\;\;\;\;\;\;\;\;\;\;\;\;\;\;\rightarrow\begin{pmatrix}1&0&0&0&-1/2&0\\0&1&0&-3&-3/4&-1/2\\0&0&1&-1&0&0\end{pmatrix}\;\\∴\left(E-A\right)^{-1}=\begin{pmatrix}0&-1/2&0\\-3&-3/4&-1/2\\-1&0&0\end{pmatrix}.\end{array}
$$



$$
设A=\begin{pmatrix}0&2&-1\\1&1&2\\-1&-1&-1\end{pmatrix},\mathrm{用初等变换求矩阵}A^{-1}为(\;\;)
$$
$$
A.
\begin{pmatrix}-1/2&-3/2&-5/2\\1/2&-1/2&1/2\\0&1&1\end{pmatrix} \quad B.\begin{pmatrix}-1/2&3/2&-5/2\\1/2&1/2&1/2\\0&1&1\end{pmatrix} \quad C.\begin{pmatrix}-1/2&-3/2&5/2\\1/2&1/2&1/2\\0&1&1\end{pmatrix} \quad D.\begin{pmatrix}-1/2&-3/2&-5/2\\1/2&1/2&1/2\\0&1&1\end{pmatrix} \quad E. \quad F. \quad G. \quad H.
$$
$$
\begin{array}{l}\mathrm{作矩阵}(A\vert E),\mathrm{施行初等变换}.\\\begin{pmatrix}0&2&-1&1&0&0\\1&1&2&0&1&0\\-1&-1&-1&0&0&1\end{pmatrix}\rightarrow\begin{pmatrix}1&1&2&0&1&0\\0&2&-1&1&0&0\\-1&-1&-1&0&0&1\end{pmatrix}\\\rightarrow\begin{pmatrix}1&1&2&0&1&0\\0&2&-1&1&0&0\\0&0&1&0&1&1\end{pmatrix}\rightarrow\begin{pmatrix}1&1&0&0&-1&-2\\0&2&0&1&1&1\\0&0&1&0&1&1\end{pmatrix}\\\rightarrow\begin{pmatrix}1&1&0&0&-1&-2\\0&1&0&1/2&1/2&1/2\\0&0&1&0&1&1\end{pmatrix}\rightarrow\begin{pmatrix}1&0&0&-1/2&-3/2&-5/2\\0&1&0&1/2&1/2&1/2\\0&0&1&0&1&1\end{pmatrix}\\故A^{-1}=\begin{pmatrix}-1/2&-3/2&-5/2\\1/2&1/2&1/2\\0&1&1\end{pmatrix}\end{array}
$$



$$
\mathrm{已知矩阵}A=\begin{pmatrix}1&1&-1\\2&1&0\\1&-1&0\end{pmatrix},\mathrm{用初等变换求}A\mathrm{的逆矩阵}A^{-1}=\left(\;\;\;\right)
$$
$$
A.
\begin{pmatrix}0&\frac13&\frac13\\0&\frac13&-\frac23\\-1&\frac23&-\frac13\end{pmatrix} \quad B.\begin{pmatrix}0&\frac13&\frac13\\0&\frac13&\frac23\\-1&\frac23&\frac13\end{pmatrix} \quad C.\begin{pmatrix}0&\frac13&\frac13\\0&-\frac13&\frac23\\-1&\frac23&\frac13\end{pmatrix} \quad D.\begin{pmatrix}0&\frac13&\frac13\\0&\frac13&-\frac23\\1&\frac23&\frac13\end{pmatrix} \quad E. \quad F. \quad G. \quad H.
$$
$$
\begin{array}{l}\left(A\vdots E\right)=\begin{pmatrix}1&1&-1&\vdots&1&0&0\\2&1&0&\vdots&0&1&0\\1&-1&0&\vdots&0&0&1\end{pmatrix}∼\begin{pmatrix}1&1&-1&\vdots&1&0&0\\0&-1&2&\vdots&-2&1&0\\0&-2&1&\vdots&-1&0&1\end{pmatrix}\\\;\;\;\;\;\;\;\;\;\;∼\begin{pmatrix}1&1&-1&\vdots&1&0&0\\0&-1&2&\vdots&-2&1&0\\0&0&-3&\vdots&3&-2&1\end{pmatrix}∼\begin{pmatrix}1&0&0&\vdots&0&\frac13&\frac13\\0&1&0&\vdots&0&\frac13&-\frac23\\0&0&1&\vdots&-1&\frac23&-\frac13\end{pmatrix},故A^{-1}=\begin{pmatrix}0&\frac13&\frac13\\0&\frac13&-\frac23\\-1&\frac23&-\frac13\end{pmatrix}\end{array}
$$



$$
\mathrm{设方阵}A\mathrm{满足}A^2-A-5E=O,\mathrm{则下列说法错误的是}().\;
$$
$$
A.
A\mathrm{不可逆} \quad B.A-E\mathrm{可逆}, \quad C.A-3E\mathrm{可逆} \quad D.A+2E\mathrm{可逆} \quad E. \quad F. \quad G. \quad H.
$$
$$
\begin{array}{l}设A为n\mathrm{阶方阵},由A^2-A-5E=0⇒ A^2-A=5E⇒ A(A-E)=5E,\\故A和A-E\mathrm{可逆};\\A^2-A-5E=0⇒ A^2-A-6E=-E⇒(A-3E)(A+2E)=-E\\故A+2E,A-3E\mathrm{也可逆}.\end{array}
$$



$$
若n\mathrm{阶矩阵满足}A^3+A^2-A-E=O,则A^{-1}=().
$$
$$
A.
A^2+A \quad B.A-2E \quad C.A+2E \quad D.A^2+A-E \quad E. \quad F. \quad G. \quad H.
$$
$$
\begin{array}{l}A^3+A^2-A-E=O⇒ A(A^2+A-E)=E\\∴ A^{-1}=A^2+A-E\end{array}
$$



$$
\mathrm{若方阵}A\mathrm{满足}A^2-5A+3E=O,则(A-3E)^{-1}=().\;
$$
$$
A.
\frac13(A-2E) \quad B.\frac13(A+3E) \quad C.-\frac13(A-3E) \quad D.\frac13(A-E) \quad E. \quad F. \quad G. \quad H.
$$
$$
\begin{array}{l}A^2-5E+6E=3E⇒(A-2E)(A-3E)=3E\\(A-3E)^{-1}=\frac13(A-2E)\end{array}
$$



$$
\mathrm{设矩阵}A=\begin{pmatrix}2&0&1\\0&2&0\\0&0&1\end{pmatrix},则(A+E)^{-1}(A^2-E)=().\;
$$
$$
A.
\begin{pmatrix}1&0&-1\\0&1&0\\0&0&0\end{pmatrix} \quad B.\begin{pmatrix}2&0&1\\0&2&0\\0&0&0\end{pmatrix} \quad C.\begin{pmatrix}1&0&1\\0&1&0\\0&0&0\end{pmatrix} \quad D.\begin{pmatrix}1&0&1\\0&1&0\\0&0&1\end{pmatrix} \quad E. \quad F. \quad G. \quad H.
$$
$$
(A+E)^{-1}(A^2-E)=A-E=\begin{pmatrix}1&0&1\\0&1&0\\0&0&0\end{pmatrix}
$$



$$
\mathrm{设矩阵}A=\begin{pmatrix}4&0&1\\0&1&0\\0&0&3\end{pmatrix},则\;(A-2E)^{-1}(A^2-4E)=().
$$
$$
A.
\begin{pmatrix}6&0&1\\0&3&0\\0&0&3\end{pmatrix} \quad B.\begin{pmatrix}4&0&1\\0&3&0\\0&0&5\end{pmatrix} \quad C.\begin{pmatrix}6&0&3\\0&3&0\\0&0&5\end{pmatrix} \quad D.\begin{pmatrix}6&0&1\\0&3&0\\0&0&5\end{pmatrix} \quad E. \quad F. \quad G. \quad H.
$$
$$
(A-2E)^{-1}(A^2-4E)=A+2E=\begin{pmatrix}6&0&1\\0&3&0\\0&0&5\end{pmatrix}
$$



$$
\mathrm{若方阵}A\mathrm{满足}A^2-5A+3E=O,则(A-2E)^{-1}=().\;
$$
$$
A.
\frac13(A-3E) \quad B.\frac13(A+3E) \quad C.-\frac13(A-3E) \quad D.\frac13(A-E) \quad E. \quad F. \quad G. \quad H.
$$
$$
\begin{array}{l}A^2-5E+6E=3E⇒(A-2E)(A-3E)=3E\\(A-2E)^{-1}=\frac13(A-3E)\end{array}
$$



$$
若n\mathrm{阶矩阵满足}A^2-A-4E=0,则(A+E)^{-1}=().
$$
$$
A.
\frac12(A+2E) \quad B.A-2E \quad C.A+2E \quad D.\frac12(A-2E) \quad E. \quad F. \quad G. \quad H.
$$
$$
A^2-A-4E⇒ A^2-A-2E=2E,则(A-2E)(A+E)=2E,即\frac12(A-2E)(A+E)=E,故(A+E)^{-1}=\frac12(A-2E)
$$



$$
设A是n\mathrm{阶方阵},A^2=E,\mathrm{下列结论正确的是}().
$$
$$
A.
A\neq E时,A+E\mathrm{可逆} \quad B.A+E\mathrm{可逆} \quad C.A=E时,A+E\mathrm{可逆} \quad D.A-E\mathrm{可逆} \quad E. \quad F. \quad G. \quad H.
$$
$$
\begin{array}{l}A^2=E⇒ A^2-E^2=O,即(A+E)(A-E)=O,有\left|A+E\right|=0或\left|A-E\right|=0,\\若A=E⇒,则A+E\mathrm{可逆}.\end{array}
$$



$$
\mathrm{设方阵}A\mathrm{满足}A^2-A-2E=O,\;\mathrm{则下列说法不正确的是}().\;
$$
$$
A.
A+2E\mathrm{可逆} \quad B.A\mathrm{可逆} \quad C.A-E\mathrm{可逆} \quad D.A+E\mathrm{可逆} \quad E. \quad F. \quad G. \quad H.
$$
$$
\begin{array}{l}设A为n\mathrm{阶方阵},由A^2-A-2E=0⇒ A^2-A=2E,\\\mathrm{两端同时取行列式}\left|A^2-A\right|=2^n,即\left|A\right|\left|A-E\right|=2^n,\\故\left|A\right|\neq0,\left|A-E\right|\neq0,故A和A-E\mathrm{可逆};\\而\;A+2E=A^2,\left|A+2E\right|=\left|A^2\right|=\left|A\right|^2\neq0,\\故A+2E\mathrm{也可逆}.\end{array}
$$



$$
设A,B,C为n\mathrm{阶方阵},且ABC=E,\mathrm{则必成立的等式为}().
$$
$$
A.
ACB=E \quad B.CBA=E \quad C.BAC=E \quad D.BCA=E \quad E. \quad F. \quad G. \quad H.
$$
$$
\begin{array}{l}\mathrm{通过左乘}A\mathrm{的逆或右乘的逆},\;\mathrm{注意到}E与A\mathrm{的可交换性},由ABC=E知\\A(BC)=(BC)A=E或(AB)C=C(AB)=E.\end{array}
$$



$$
\begin{array}{l}设n\mathrm{阶方阵}A\mathrm{满足}A^2-A-2E=0\;,\mathrm{则下列说法一定正确的是}().\\(1)A+2E\mathrm{可逆};(2)A\mathrm{可逆};(3)A-E\mathrm{可逆};(4)A+E\mathrm{可逆}.\end{array}
$$
$$
A.
(1),(4) \quad B.(2),(4) \quad C.(3),(4) \quad D.(1),(2),(3) \quad E. \quad F. \quad G. \quad H.
$$
$$
\begin{array}{l}由\;\;\;A^2-A-2E=0⇒ A^2-A=2E,\\\mathrm{两端同时取行列式}\left|A^2-A\right|=2^n,即\left|A\right|\left|A-E\right|=2^n,\\故\left|A\right|\neq0,\left|A-E\right|\neq0,故A和A-E\mathrm{可逆};\\而\;A+2E=A^2,\left|A+2E\right|=\left|A^2\right|=\left|A\right|^2\neq0,\\故A+2E\mathrm{也可逆}.\\\mathrm{由于}(A-2E)(A+E)=0,\mathrm{所以}A+E\mathrm{是否可逆不能确定}.\end{array}
$$



$$
若(E-A)^{-1}=E+A+A^2+⋯+A^{k-1}(\mathrm{是正整数}),\mathrm{则下列说法正确的是}().
$$
$$
A.
A=kE \quad B.A^k=E \quad C.A^k=O(k\geq1) \quad D.A^3=E \quad E. \quad F. \quad G. \quad H.
$$
$$
\begin{array}{l}∵(E-A)(E+A+A^2+⋯+A^{k-1})=E-A^k=E,\\∴ A^k=O.\end{array}
$$



$$
设A,B为n\mathrm{阶矩阵},\mathrm{且满足}2B^{-1}A=A-4E\;\mathrm{其中}E为n\mathrm{阶单位矩阵},则(B-2E)^{-1}=().
$$
$$
A.
\frac18(A-4E) \quad B.\frac14(A-2E) \quad C.\frac18(A+4E) \quad D.\frac14(A+2E) \quad E. \quad F. \quad G. \quad H.
$$
$$
\begin{array}{l}\mathrm{由于}2B^{-1}A=A-4E,\mathrm{方程两边同时左乘}B,有\\\;\;2A=BA-4B,BA-4B-2A+8E=8E,(B-2E)(A-4E)=8E,\\故\;B-2E\mathrm{可逆},\\\;(B-2E)^{-1}=\frac18(A-4E).\end{array}
$$



$$
设A=\begin{pmatrix}1&0&1\\0&2&0\\0&0&1\end{pmatrix},则\;(A+3E)^{-1}(A^2-9E)=().
$$
$$
A.
\begin{pmatrix}-2&0&1\\0&-1&0\\0&0&-2\end{pmatrix} \quad B.\begin{pmatrix}-2&0&-1\\0&-1&0\\0&0&-2\end{pmatrix} \quad C.\begin{pmatrix}-2&0&1\\0&1&0\\0&0&-2\end{pmatrix} \quad D.\begin{pmatrix}-1&0&1\\0&-1&0\\0&0&-2\end{pmatrix} \quad E. \quad F. \quad G. \quad H.
$$
$$
(A+3E)^{-1}(A^2-9E)=(A-3E)=\begin{pmatrix}-2&0&1\\0&-1&0\\0&0&-2\end{pmatrix}
$$



$$
\mathrm{已知矩阵}A=\begin{pmatrix}1&0&3\\0&2&0\\0&0&1\end{pmatrix},则(A+3E)^{-1}(A^2-9E)=().
$$
$$
A.
\begin{pmatrix}-2&0&-3\\0&-1&0\\0&0&-2\end{pmatrix} \quad B.\begin{pmatrix}-2&0&3\\0&-1&0\\0&0&1\end{pmatrix} \quad C.\begin{pmatrix}-1&0&3\\0&-1&0\\0&0&-1\end{pmatrix} \quad D.\begin{pmatrix}-2&0&3\\0&-1&0\\0&0&-2\end{pmatrix} \quad E. \quad F. \quad G. \quad H.
$$
$$
(A+3E)^{-1}(A^2-9E)=(A+3E)^{-1}(A+3E)(A-3E)=A-3E=\begin{pmatrix}-2&0&3\\0&-1&0\\0&0&-2\end{pmatrix}
$$



$$
设A=\begin{pmatrix}-1&0&0\\1&-1&0\\1&1&-1\end{pmatrix},则(A+2E)^{-1}(A^2-4E)=().
$$
$$
A.
\begin{pmatrix}-3&0&0\\1&-3&0\\1&1&-3\end{pmatrix} \quad B.\begin{pmatrix}-2&0&0\\1&-2&0\\1&1&-2\end{pmatrix} \quad C.\begin{pmatrix}-1&0&0\\1&-1&0\\1&1&-1\end{pmatrix} \quad D.\begin{pmatrix}-4&0&0\\1&-4&0\\1&1&-4\end{pmatrix} \quad E. \quad F. \quad G. \quad H.
$$
$$
\begin{array}{l}(A+2E)^{-1}(A^2-4E)\\=(A+2E)^{-1}(A+2E)(A-2E)\\=A-2E\\=\begin{pmatrix}-3&0&0\\1&-3&0\\1&1&-3\end{pmatrix}\end{array}
$$



$$
\begin{array}{l}\mathrm{设实矩阵}A=(a_{ij})_{3×3}\mathrm{满足条件}:\\(1)a_{ij}=A_{ij}(i,j=1,2,3);\\(2)a_{11}\neq0,\mathrm{其中}A_{ij}为a_{ij}\mathrm{的代数余子式}.\\则\left|A\right|=().\\\end{array}
$$
$$
A.
0 \quad B.1 \quad C.-1 \quad D.\mathrm{无法确定} \quad E. \quad F. \quad G. \quad H.
$$
$$
\begin{array}{l}AA^*=\left|A\right|E,又A^*=A^T,即AA^T=\left|A\right|E,\mathrm{两边取行列式},有\\\left|A\right|^2=\left|A\right|^3,即\left|A\right|^2(1-\left|A\right|)=0.\\又\left|A\right|\mathrm{按第一行展开},有\\\left|A\right|=a_{11}A_{11}+a_{12}A_{12}+a_{13}A_{13}=a_{11}^2+a_{12}^2+a_{13}^2\neq0,\\故\left|A\right|=1.\end{array}
$$



$$
\mathrm{已知}n\mathrm{阶矩阵}A,B\mathrm{满足}A+B=AB,则(A-E)^{-1}是\;
$$
$$
A.
B-E \quad B.B+E \quad C.B+2E \quad D.B-2E \quad E. \quad F. \quad G. \quad H.
$$
$$
\begin{array}{l}A+B=AB⇒ AB-A-B+E=E\\∴(A-E)(B-E)=E⇒(A-E)^{-1}=B-E\end{array}
$$



$$
设A为n\mathrm{阶方阵},且\left|A\right|=2,则(A^*)^*=().\;
$$
$$
A.
2^nA \quad B.2^{n-1}A \quad C.2^{n-2}A^* \quad D.2^{n-2}A \quad E. \quad F. \quad G. \quad H.
$$
$$
\begin{array}{l}∵ A^*(A^*)^*=\left|A^*\right|E,又∵\left|A\right|=2,A,A^*\mathrm{都可逆},\\∴(A^*)^*=\left|A\right|^{n-1}(A^*)^{-1}E,∴(A^*)^{-1}=\frac A{\left|A\right|},∴(A^*)^*=\left|A\right|^{n-2}A=2^{n-2}A\end{array}
$$



$$
设A,B,A^{-1}+B^{-1}为n\mathrm{阶可逆矩阵},则(A+B)^{-1}=().
$$
$$
A.
B^{-1}(A^{-1}+B^{-1})^{-1}A^{-1} \quad B.A+B \quad C.B(A+B)^{-1}A \quad D.(A-B)^{-1} \quad E. \quad F. \quad G. \quad H.
$$
$$
\begin{array}{l}令(A+B)^{-1}=C,\;则(A+B)C=E⇒ A^{-1}(A+B)C=A^{-1}\\(E+A^{-1}B)C=A^{-1}⇒(B^{-1}B+A^{-1}B)C=A^{-1},(A^{-1}+B^{-1})BC=A^{-1}⇒ C=B^{-1}(A^{-1}+B^{-1})^{-1}A^{-1}\;\end{array}
$$



$$
设A,B,C\mathrm{均为}n\mathrm{阶矩阵},E为n\mathrm{阶单位矩阵},若B=E+AB,C=A+CA,则B-C=().\;
$$
$$
A.
-E \quad B.A \quad C.E \quad D.-A \quad E. \quad F. \quad G. \quad H.
$$
$$
\begin{array}{l}B=E+AB,C=A+CA,⇒(E-A)B=E,C(E-A)=A\\则B,E-A\mathrm{互为逆矩阵},(B-C)(E-A)=E-A,∴ B-C=E\end{array}
$$



$$
设A,B为3\mathrm{阶矩阵},且\left|A\right|=3,\left|B\right|=2,\left|A^{-1}+B\right|=2,则\left|A+B^{-1}\right|=().
$$
$$
A.
1 \quad B.2 \quad C.3 \quad D.4 \quad E. \quad F. \quad G. \quad H.
$$
$$
\begin{array}{l}\mathrm{应填}3.\\\mathrm{由于}A(A^{-1}+B)B^{-1}=(E+AB)B^{-1}=B^{-1}+A,\mathbf{所以}\\\left|\mathbf A\boldsymbol+\mathbf B^{\boldsymbol-\mathbf1}\right|\boldsymbol=\left|\mathbf A\boldsymbol(\mathbf A^{\boldsymbol-\mathbf1}\boldsymbol+\mathbf B\boldsymbol)\mathbf B^{\boldsymbol-\mathbf1}\right|\boldsymbol=\left|\mathbf A\right|\left|\mathbf A^{\boldsymbol-\mathbf1}\boldsymbol+\mathbf B\right|\left|\mathbf B^{\boldsymbol-\mathbf1}\right|\boldsymbol.\\\mathbf{因为}\left|\mathbf B\right|\boldsymbol=\mathbf2\boldsymbol.\mathbf{所以}\left|\mathbf B^{\boldsymbol-\mathbf1}\right|\boldsymbol=\left|\mathbf B\right|^{\boldsymbol-\mathbf1}\boldsymbol=\frac{\mathbf1}{\mathbf2}\boldsymbol,\mathbf{因此}\\\left|\mathbf A\boldsymbol+\mathbf B^{\boldsymbol-\mathbf1}\right|\boldsymbol=\left|\mathbf A\right|\left|\mathbf A^{\boldsymbol-\mathbf1}\boldsymbol+\mathbf B\right|\left|\mathbf B^{\boldsymbol-\mathbf1}\right|\boldsymbol=\mathbf3\boldsymbol×\mathbf2\boldsymbol×\frac{\mathbf1}{\mathbf2}\boldsymbol=\mathbf3\\\end{array}
$$



$$
设A,B为n\mathrm{阶矩阵},2A-B-AB=E,A^2=A\;,\mathrm{其中}E为n\mathrm{阶单位矩阵},\;则(A-B)^{-1}=().
$$
$$
A.
E+A \quad B.E-A \quad C.E+B \quad D.E-B \quad E. \quad F. \quad G. \quad H.
$$
$$
\begin{array}{l}\mathrm{由于}A^2=A,\mathrm{于是}\\2A-B-AB=A-B+A-AB=A-B+A^2-AB\\=(A-B)+A(A-B)=(E+A)(A_{}-B)=E,\\故A-B\mathrm{为可逆矩阵},且(A-B)^{-1}=E+A\end{array}
$$



$$
设A,B\mathrm{均是}n\mathrm{阶矩阵},且AB=E,BC=2E,则(A-C)^2B=().
$$
$$
A.
\frac C2 \quad B.\frac A2 \quad C.2A \quad D.2C \quad E. \quad F. \quad G. \quad H.
$$
$$
\begin{array}{l}AB=E⇒ ABC=C,BC=2E⇒ ABC=2A,B=2C^{-1},故C=2A,B=2C^{-1},则\\(A-C)^2· B=(\frac C2-C)^2B=\frac{C^2}4×2C^{-1}=\frac C2.\end{array}
$$



$$
设A,B,A+B,A^{-1}+B^{-1}为n\mathrm{阶可逆矩阵},则(A^{-1}+B^{-1})^{-1}=().
$$
$$
A.
A^{-1}+B^{-1} \quad B.A+B \quad C.B(A+B)^{-1}A \quad D.(A+B)^{-1} \quad E. \quad F. \quad G. \quad H.
$$
$$
\begin{array}{l}令(A^{-1}+B^{-1})^{-1}=C\mathrm{由可逆矩阵的性质},有\;\\(A^{-1}+B^{-1})C=E⇒ A(A^{-1}+B^{-1})C=A,(E+AB^{-1})C=A⇒(BB^{-1}+AB^{-1})C=A\\(B+A)B^{-1}C=A⇒ C=B(A+B)^{-1}A\\\mathrm{本题还可以根据逆矩阵的定义直接将选项中的答案代入进行计算}.\end{array}
$$



$$
\mathrm{设矩阵}A=(a_{ij})_{3×3}\mathrm{满足}A^*=A^T,\mathrm{其中}A^*,A^T\mathrm{分别为}A\mathrm{的伴随矩阵与转置矩阵}.若a_{11},a_{12},a_{13}\mathrm{为三个相等的正数},则a_{11}=()
$$
$$
A.
3 \quad B.\frac13 \quad C.\frac{\sqrt3}3 \quad D.\sqrt3 \quad E. \quad F. \quad G. \quad H.
$$
$$
\begin{array}{l}A^*=A^T,AA^*=\left|A\right|E⇒ a_{ij}=A_{ij},AA^T=\left|A\right|E⇒\left|A\right|^2=\left|A\right|^3,\mathrm{所以}\left|A\right|=1,\left|A\right|=0\\\left|A\right|=a_{11}A_{11}+a_{12}A_{12}+a_{13}A_{13}=3{a^2}_{11}\neq0,\mathrm{所以}\left|A\right|=1,a_{11}=\frac{\sqrt3}3\end{array}
$$



$$
\begin{array}{l}设A,B\mathrm{均为}n\mathrm{阶可逆矩阵},则().\\\end{array}
$$
$$
A.
A+B\mathrm{可逆} \quad B.kA\mathrm{可逆}(k\mathrm{为常数}) \quad C.AB\mathrm{可逆} \quad D.(AB)^{-1}=A^{-1}B^{-1} \quad E. \quad F. \quad G. \quad H.
$$
$$
\begin{array}{l}\mathrm{矩阵可逆的充要条件是其行列式不等于零},\mathrm{由题设可知}\left|A\right|\neq0,\left|B\right|\neq0,即AB\mathrm{可逆},\\\mathrm{其它选项不正确},(AB)^{-1}=B^{-1}A^{-1};\\\left|kA\right|=k^n\left|A\right|,若k\neq0,则kA\mathrm{可逆};\mathrm{无法判断}\left|A+B\right|\mathrm{是否非零}.\\\end{array}
$$



$$
设A,B\mathrm{均为}n\mathrm{阶矩阵},\mathrm{则必有}().
$$
$$
A.
\left|A+B\right|=\left|A\right|+\left|B\right| \quad B.AB=BA \quad C.\left|AB\right|=\left|BA\right| \quad D.(A+B)^{-1}=A^{-1}+B^{-1} \quad E. \quad F. \quad G. \quad H.
$$
$$
\mathrm{选项中只有}\left|AB\right|=\left|B\right|\left|A\right|=\left|BA\right|\mathrm{正确},\mathrm{其他都不符合运算规律}.
$$



$$
设A,B\mathrm{均为}n\mathrm{阶方阵},\mathrm{下列结论中},\mathrm{正确的是}().
$$
$$
A.
若A,B\mathrm{均可逆},则A+B\mathrm{可逆} \quad B.若A,B\mathrm{均可逆},则AB\mathrm{可逆} \quad C.若A+B\mathrm{可逆},则A-B\mathrm{可逆} \quad D.若A+B\mathrm{可逆},则A,B\mathrm{均可逆} \quad E. \quad F. \quad G. \quad H.
$$
$$
\begin{array}{l}\mathrm{矩阵可逆的充要条件是其行列式不等于零},若A,B\mathrm{均可逆},则\left|A\right|\neq0,\left|B\right|\neq0,\left|AB\right|=\left|A\right|\left|B\right|\neq0,\\故AB\mathrm{可逆}.\end{array}
$$



$$
设A,B\mathrm{均为}n\mathrm{阶可逆矩阵},\mathrm{则下列结论成立的是}().
$$
$$
A.
(A^2)^{-1}=(A^{-1})^2 \quad B.(kA)^{-1}=kA^{-1}(k\neq0) \quad C.(A+B)^{-1}=A^{-1}+B^{-1} \quad D.(A+B)(A-B)=A^2-B^2 \quad E. \quad F. \quad G. \quad H.
$$
$$
\begin{array}{l}\mathrm{由逆矩阵的运算性质可知}:\\(A^2)^{-1}=(AA)^{-1}=A^{-1}A^{-1}=(A^{-1})^2;\;\;(kA)^{-1}=\frac1kA^{-1}((k\neq0);\\\mathrm{由于}(A+B)(A^{-1}+B^{-1})=2E+BA^{-1}+AB^{-1}\neq E,\mathrm{所以}(A+B)^{-1}\neq A^{-1}+B^{-1};\\而(A+B)(A-B)=A^2-AB+BA-B^2,若A,B\mathrm{不可交换},则AB\neq BA\end{array}
$$



$$
\mathrm{设矩阵}A=\begin{pmatrix}a_{11}&a_{12}&⋯&a_{1n}\\a_{21}&a_{22}&⋯&a_{2n}\\\vdots&\vdots&&\vdots\\a_{n1}&a_{n2}&⋯&a_{nn}\end{pmatrix},B=\begin{pmatrix}A_{11}&A_{12}&⋯&A_{1n}\\A_{21}&A_{22}&⋯&A_{2n}\\\vdots&\vdots&&\vdots\\A_{n1}&A_{n2}&⋯&A_{nm}\end{pmatrix}\mathrm{其中}A_{ij}是a_{ij}\mathrm{的代数余子式}(i,j=1,2,⋯,n),则().\;
$$
$$
A.
A是B\mathrm{的伴随矩阵} \quad B.B是A\mathrm{的伴随矩阵} \quad C.B是A^T\mathrm{的伴随矩阵} \quad D.B\mathrm{不是}A^T\mathrm{的伴随矩阵} \quad E. \quad F. \quad G. \quad H.
$$
$$
\mathrm{根据伴随矩阵的定义可知}B是A^T\mathrm{的伴随矩阵}
$$



$$
\mathrm{设矩阵}A=\begin{pmatrix}1&2&1\\-1&0&1\\0&1&0\end{pmatrix},则\left|A^*\right|=().
$$
$$
A.
-{\textstyle\frac12} \quad B.\textstyle\frac14 \quad C.2 \quad D.4 \quad E. \quad F. \quad G. \quad H.
$$
$$
\left|A^*\right|=\left|A\right|^2=(-2)^2=4
$$



$$
设A=\begin{pmatrix}1&1&0\\0&-1&3\\2&3&-1\end{pmatrix},则\left|A^*\right|=().\;
$$
$$
A.
-4 \quad B.4 \quad C.-8 \quad D.8 \quad E. \quad F. \quad G. \quad H.
$$
$$
\left|A^*\right|=\left|A\right|^{3-1}=\left|A\right|^2,而\left|A\right|=-2,故\left|A^*\right|=4
$$



$$
A,B\mathrm{均为}n\mathrm{阶矩阵},\mathrm{则下列说法正确的是}().
$$
$$
A.
(A-B)^2=A^2-2AB+B^2 \quad B.(A-B)(A+B)=A^2-B^2 \quad C.(AB)^{-1}=A^{-1}B^{-1} \quad D.当\left|AB\right|\neq0时,A,B\mathrm{均可逆} \quad E. \quad F. \quad G. \quad H.
$$
$$
\begin{array}{l}\mathrm{矩阵的乘法一般不满足交换律},且A,B\mathrm{均为}n\mathrm{阶矩阵},\mathrm{但不能确定行列式是否非零},\mathrm{因此不一定可逆};\\当\left|A\right|\left|B\right|\neq0时,A,B\mathrm{都是可逆的}.\end{array}
$$



$$
\mathrm{设矩阵}A=\begin{pmatrix}1&2\\3&4\end{pmatrix},则A^*\mathrm{等于}().\;
$$
$$
A.
\begin{pmatrix}-2&1\\\frac32&-\frac12\end{pmatrix} \quad B.\begin{pmatrix}1&-3\\-2&4\end{pmatrix} \quad C.\begin{pmatrix}4&2\\3&1\end{pmatrix} \quad D.\begin{pmatrix}4&-2\\-3&1\end{pmatrix} \quad E. \quad F. \quad G. \quad H.
$$
$$
\mathrm{根据伴随矩阵的定义可知}A^*=\begin{pmatrix}4&-2\\-3&1\end{pmatrix}.
$$



$$
设A为n\mathrm{阶方阵},且\left|A\right|=a\neq0,则\left|A^*\right|=().
$$
$$
A.
a \quad B.\frac1a \quad C.a^{n-1} \quad D.a^n \quad E. \quad F. \quad G. \quad H.
$$
$$
\mathrm{伴随矩阵的性质}:AA^*=\left|A\right|E,\mathrm{两边取行列式得}\left|AA^*\right|=\left|A\right|\left|A^*\right|=\left|A\right|^n,则\left|A\right|=\left|A\right|^{n-1}=a^{n-1}.
$$



$$
设A是5\mathrm{阶的可逆方阵},且\left|A\right|\neq1,A^* 是A\mathrm{的伴随矩阵},\mathrm{则有}().
$$
$$
A.
\left|A^*\right|=A \quad B.\left|A^*\right|=\frac1{\left|A\right|} \quad C.\left|A^*\right|=\left|A\right|^4 \quad D.\left|A^*\right|=\left|A\right|^5 \quad E. \quad F. \quad G. \quad H.
$$
$$
\left|A^*\right|=\left|A\right|^{n-1}=\left|A\right|^4
$$



$$
设A,B为n\mathrm{阶矩阵},\mathrm{下列运算正确的是}().\;
$$
$$
A.
(AB)^k=A^kB^k \quad B.\left|-A\right|=-\left|A\right| \quad C.A^2-B^2=(A-B)(A+B) \quad D.若A\mathrm{可逆},k\neq0,则(kA)^{-1}=k^{-1}A^{-1} \quad E. \quad F. \quad G. \quad H.
$$
$$
\begin{array}{l}\mathrm{由于矩阵乘法一般不满足交换律},\mathrm{因此}(AB)^k\mathrm{不一定等于}A^kB^k,且A^2-B^2\mathrm{不一定等于}(A-B)(A+B);\\\left|-A\right|=(-1)^n\left|A\right|;若A\mathrm{可逆},k\neq0,则(kA)^{-1}=k^{-1}A^{-1}.\end{array}
$$



$$
\begin{array}{l}设A,B\mathrm{为同阶可逆矩阵},λ\mathrm{为非零实数},\mathrm{则下列命题中不正确的是}().\\\end{array}
$$
$$
A.
(A^{-1})^{-1}=A \quad B.(λ A)^{-1}=λ A^{-1} \quad C.(AB)^{-1}=B^{-1}A^{-1} \quad D.(A^T)^{-1}=(A^{-1})^T \quad E. \quad F. \quad G. \quad H.
$$
$$
\mathrm{由逆矩阵的性质可知}(λ A)^{-1}=λ^{-1}A^{-1}=\frac1λ A^{-1},\mathrm{其余选项都正确}.
$$



$$
设A,B\mathrm{均为}n\mathrm{阶可逆矩阵},\mathrm{则下列结论成立的是}().
$$
$$
A.
(A^2)^{-1}=(A^{-1})^2 \quad B.(kA)^{-1}=kA^{-1}(k\neq0) \quad C.(A+B)^{-1}=A^{-1}+B^{-1} \quad D.(A+B)(A-B)=A^2-B^2 \quad E. \quad F. \quad G. \quad H.
$$
$$
\begin{array}{l}\mathrm{由逆矩阵的运算性质可知}:\\(A^2)=(A· A)^{-1}=A^{-1}A^{-1}=(A^{-1})^2;(kA)^{-1}=\frac1kA^{-1};\\\mathrm{由于}(A+B)(A^{-1}+B^{-1})=2E+BA^{-1}+AB^{-1}\neq E,\mathrm{所以}(A+B)^{-1}\neq A^{-1}+B^{-1};\\而(A+B)(A-B)=A^2-AB+BA-B^2,若A,B\mathrm{不可交换},则AB\neq BA.\\\end{array}
$$



$$
设A\mathrm{是上}(下)\mathrm{三角矩阵},\mathrm{那么}A\mathrm{可逆的充分必要条件是}A\mathrm{的主对角线元素}().\;
$$
$$
A.
\mathrm{全都为负} \quad B.\mathrm{全为零} \quad C.\mathrm{全不为零} \quad D.\mathrm{没有限制} \quad E. \quad F. \quad G. \quad H.
$$
$$
\begin{array}{l}\mathrm{矩阵}A\mathrm{可逆的充要条件是}\left|A\right|\neq0,\mathrm{三角形矩阵的行列式为其主对角线上元素的乘积},\\\mathrm{即主对角线上元素全不为零}.\end{array}
$$



$$
设A,B为n\mathrm{阶矩阵},\mathrm{下列运算正确的是}().
$$
$$
A.
(AB)^k=A^kB^k \quad B.\left|-A\right|=-\left|A\right| \quad C.若A,B\mathrm{可交换},则A^2-B^2=(A-B)(A+B) \quad D.若A\mathrm{可逆},则(kA)^{-1}=k^{-1}A^{-1}. \quad E. \quad F. \quad G. \quad H.
$$
$$
\begin{array}{l}\begin{array}{l}A.\;当A,B\mathrm{可交换时才成立}.B.\left|-A\right|\;=(-1)^n\left|A\right|,\;D.若A\mathrm{可逆},当k\neq0,kA\mathrm{才可逆},\end{array}\\\end{array}
$$



$$
设A为n\mathrm{阶矩阵},且\left|A\right|=3,则\left|\left|A\right|A^{-1}\right|=().
$$
$$
A.
3^{n-1} \quad B.3^n \quad C.3^{n+1} \quad D.3^{2n} \quad E. \quad F. \quad G. \quad H.
$$
$$
\left|\left|A\right|A^{-1}\right|=3^n×\frac13=3^{n-1}
$$



$$
\mathrm{设矩阵}A=\begin{pmatrix}1&2&2\\-1&1&0\\-1&-2&2\end{pmatrix},A^* 为A\mathrm{的伴随矩阵},\;则AA^*=().
$$
$$
A.
4E \quad B.4 \quad C.12E \quad D.12 \quad E. \quad F. \quad G. \quad H.
$$
$$
\left|A\right|=\begin{vmatrix}1&2&2\\-1&1&0\\-1&-2&2\end{vmatrix}=\begin{vmatrix}1&2&2\\0&3&2\\0&0&4\end{vmatrix}=12,AA^*=12E.
$$



$$
\mathrm{设矩阵}A=\begin{pmatrix}1&-2&-2\\-2&1&0\\2&-1&1\end{pmatrix},A^* 是A\mathrm{的伴随矩阵则}\left|A^*\right|=().\;
$$
$$
A.
-3 \quad B.3 \quad C.-9 \quad D.9 \quad E. \quad F. \quad G. \quad H.
$$
$$
\mathrm{因为}\left|A\right|=-3,\left|A^*\right|=\left|A\right|^2=9.
$$



$$
\mathrm{设矩阵}A=\begin{pmatrix}1&-2&3\\-2&0&0\\7&4&9\end{pmatrix},A^{-1}为A\mathrm{的逆矩阵},则\left|A^{-1}\right|=().\;
$$
$$
A.
-60 \quad B.\frac1{60} \quad C.-\frac1{60} \quad D.60 \quad E. \quad F. \quad G. \quad H.
$$
$$
\begin{array}{l}\left|A\right|=(-1)^{2+1}(-2)\begin{vmatrix}-2&3\\4&9\end{vmatrix}=-60.\\则\left|A^{-1}\right|=\frac1{\left|A\right|}=-\frac1{60}\end{array}
$$



$$
设A=\begin{pmatrix}1&2\\5&4\end{pmatrix},则A^*\mathrm{等于}().
$$
$$
A.
\begin{pmatrix}1&-2\\-5&4\end{pmatrix} \quad B.\begin{pmatrix}1&-5\\-2&4\end{pmatrix} \quad C.\begin{pmatrix}4&-2\\-5&1\end{pmatrix} \quad D.\begin{pmatrix}4&2\\-5&1\end{pmatrix} \quad E. \quad F. \quad G. \quad H.
$$
$$
A^*=\begin{pmatrix}4&-2\\-5&1\end{pmatrix}
$$



$$
设A为5\mathrm{阶矩阵},且\left|A\right|=3,则\left|\left|A\right|A^{-1}\right|=().
$$
$$
A.
3^4 \quad B.3^5 \quad C.3^6 \quad D.3^{10} \quad E. \quad F. \quad G. \quad H.
$$
$$
\left|\left|A\right|A^{-1}\right|=\left|A\right|^{5-1}=\left|A\right|^4
$$



$$
\mathrm{设矩阵}A=\begin{pmatrix}1&-2&3\\-2&0&0\\3&2&5\end{pmatrix},A^{-1}为A\mathrm{的逆矩阵},则\left|A^{-1}\right|=().
$$
$$
A.
-12 \quad B.\frac1{32} \quad C.-\frac1{32} \quad D.-32 \quad E. \quad F. \quad G. \quad H.
$$
$$
\left|A\right|=(-1)^{2+1}(-2)\begin{vmatrix}-2&3\\2&5\end{vmatrix}=-32,\left|A^{-1}\right|=-\frac1{32}
$$



$$
设A是n\mathrm{阶可逆方阵},且\left|A\right|=2,则\left|(A^TA^{-1})^{-1}\right|\mathrm{等于}().
$$
$$
A.
1 \quad B.2^n \quad C.2^{n+1} \quad D.\frac1{2^n} \quad E. \quad F. \quad G. \quad H.
$$
$$
\left|(A^TA^{-1})^{-1}\right|=\left|A(A^{-1})^T\right|=\left|A\right|\left|A^{-1}\right|=1
$$



$$
设A,B\mathrm{都是}3\mathrm{阶可逆矩阵},且\left|A\right|=2,\left|B\right|=\frac32,则\left|(AB)^T\right|=().
$$
$$
A.
3 \quad B.9 \quad C.4 \quad D.\frac19 \quad E. \quad F. \quad G. \quad H.
$$
$$
\left|(AB)^T\right|=\left|A\right|\left|B\right|=3
$$



$$
设A,B\mathrm{都是}3\mathrm{阶可逆矩阵},且\left|A\right|=2,\left|B\right|=\frac32,则\left|(AB)^{-1}\right|=().
$$
$$
A.
3 \quad B.\frac13 \quad C.2 \quad D.\frac12 \quad E. \quad F. \quad G. \quad H.
$$
$$
\left|(AB)^{-1}\right|=\frac1{\left|A\right|\left|B\right|}=\frac13
$$



$$
\mathrm{设矩阵}A=\begin{pmatrix}0&0&1\\1&2&0\\3&4&0\end{pmatrix},B=A^{-1},则\left|B^4\right|=().
$$
$$
A.
\frac1{32} \quad B.32 \quad C.16 \quad D.\frac1{16} \quad E. \quad F. \quad G. \quad H.
$$
$$
\begin{array}{l}\left|A\right|=\begin{vmatrix}0&0&1\\1&2&0\\3&4&0\end{vmatrix}=-2,B=A^{-1}\\\left|B^4\right|=\left|B\right|^4=\frac1{16}.\end{array}
$$



$$
设A是n\mathrm{阶可逆方阵},且\left|A\right|=3,则\left|(A^TA^{-1})^{-1}\right|\mathrm{等于}().
$$
$$
A.
1 \quad B.3^n \quad C.3^{n+1} \quad D.\frac1{3^n} \quad E. \quad F. \quad G. \quad H.
$$
$$
\left|(A^TA^{-1})^{-1}\right|=\left|A(A^{-1})^T\right|=\left|A\right|\left|A^{-1}\right|=1
$$



$$
设A,B\mathrm{都是}3\mathrm{阶可逆矩阵},且\left|A\right|=3,\left|B\right|=\frac53,则\left|(AB)^{-1}\right|=().\;
$$
$$
A.
\frac13 \quad B.\frac15 \quad C.-\frac13 \quad D.-\frac15 \quad E. \quad F. \quad G. \quad H.
$$
$$
\left|(AB)^{-1}\right|=\frac1{\left|A\right|\left|B\right|}=\frac15
$$



$$
设A,B\mathrm{都是}3\mathrm{阶可逆矩阵},且\left|A\right|=2,\left|B\right|=\frac32,则\left|(-2AB)^T\right|=().
$$
$$
A.
3 \quad B.9 \quad C.16 \quad D.-24 \quad E. \quad F. \quad G. \quad H.
$$
$$
\left|(-2AB)^T\right|=(-2)^3\left|A\right|\left|B\right|=-24
$$



$$
\mathrm{已知}A,B为3\mathrm{阶方阵},且\left|A\right|=-\frac14,\left|B\right|=2\;,则\left|((AB)^{-1})^T\right|=().
$$
$$
A.
\frac12 \quad B.-\frac12 \quad C.2 \quad D.-2 \quad E. \quad F. \quad G. \quad H.
$$
$$
\left|((AB)^{-1})^T\right|=\frac1{\left|A\right|\left|B\right|}=-2
$$



$$
\mathrm{设矩阵}A=\begin{pmatrix}0&1&0&0&0\\0&0&2&0&0\\0&0&0&3&0\\0&0&0&0&4\\5&0&0&0&0\end{pmatrix},B=A^{-1},则\left|B\right|=().
$$
$$
A.
120 \quad B.\frac1{120} \quad C.125 \quad D.1 \quad E. \quad F. \quad G. \quad H.
$$
$$
\begin{array}{l}\left|A\right|=\begin{vmatrix}0&1&0&0&0\\0&0&2&0&0\\0&0&0&3&0\\0&0&0&0&4\\5&0&0&0&0\end{vmatrix}=(-1)^65!=120,\\\left|B\right|=\frac1{120}\end{array}
$$



$$
设A,B\mathrm{均为}n\mathrm{阶方阵},\left|A\right|=2,\left|B\right|=-3,则\left|2AB^{-1}\right|=().
$$
$$
A.
-\frac{2^n}3 \quad B.-\frac{2^{n+1}}3 \quad C.\frac{2^{2n}}3 \quad D.-\frac{2^{2n}}3 \quad E. \quad F. \quad G. \quad H.
$$
$$
\left|2AB^{-1}\right|=2^n×2×(-\frac13)=-\frac{2^{n+1}}3
$$



$$
\mathrm{已知}n\mathrm{阶矩阵}A\mathrm{的行列式}\left|A\right|=0时,A^*\;是A\mathrm{的伴随矩阵}\;,则\left|A^*\right|=().\;\;
$$
$$
A.
A \quad B.\frac1{\left|A\right|} \quad C.\left|A\right|^{n-1} \quad D.\left|A\right|^n \quad E. \quad F. \quad G. \quad H.
$$
$$
\begin{array}{l}由AA^*=\left|A\right|E,\;\mathrm{两边取行列式得到}:\\\;\;\;\;\;\left|A\right|\left|A^*\right|=\left|A\right|^n,\\若\left|A\right|\neq0,则\\\;\;\;\;\;\left|A^*\right|=\left|A\right|^{n-1},\\若\left|A\right|=0,由\left|A^*\right|=0\mathrm{可知},\;\mathrm{此时命题也成立},\mathrm{故有}\\\;\;\;\;\;\left|A^*\right|=\left|A\right|^{n-1}.\end{array}
$$



$$
\mathrm{已知}A,B为3\mathrm{阶方阵},且\left|A\right|=-\frac14\;,\left|B\right|=2,则\left|((AB)^T)^{-1}\right|=().
$$
$$
A.
2 \quad B.4 \quad C.-2 \quad D.-4 \quad E. \quad F. \quad G. \quad H.
$$
$$
\left|((AB)^T)^{-1}\right|=\left|((AB)^{-1})^T\right|=\left|(AB)^{-1}\right|=-2
$$



$$
n\mathrm{阶方阵}A,若AA^TA^{-1}=2E,则\left|AA^TA^{-1}\right|=().\;
$$
$$
A.
\sqrt2 \quad B.2 \quad C.2^n \quad D.2^\frac n2 \quad E. \quad F. \quad G. \quad H.
$$
$$
因AA^TA^{-1}=2E,则\left|AA^TA^{-1}\right|=\left|2E\right|=2^n.
$$



$$
\mathrm{设矩阵}A=\begin{pmatrix}1&0&1\\2&1&0\\-3&2&-5\end{pmatrix},\mathrm{则矩阵}A\mathrm{的伴随矩阵}A^*=().
$$
$$
A.
\begin{pmatrix}-5&2&-1\\10&-2&2\\7&-2&1\end{pmatrix} \quad B.\begin{pmatrix}10&1&-1\\5&-2&2\\7&-2&1\end{pmatrix} \quad C.\begin{pmatrix}1&0&-1\\7&-2&2\\5&2&1\end{pmatrix} \quad D.\begin{pmatrix}2&1&0\\8&-2&2\\9&-2&1\end{pmatrix} \quad E. \quad F. \quad G. \quad H.
$$
$$
\begin{array}{l}\mathrm{按定义},\mathrm{因为}\\A_{11}=\begin{vmatrix}1&0\\2&-5\end{vmatrix}=-5,A_{12}=\begin{vmatrix}2&0\\-3&-5\end{vmatrix}=10,A_{13}=\begin{vmatrix}2&1\\-3&2\end{vmatrix}=7,\\A_{21}=-\begin{vmatrix}0&1\\2&-5\end{vmatrix}=2,A_{22}=\begin{vmatrix}1&1\\-3&-5\end{vmatrix}=-2,A_{23}=-\begin{vmatrix}1&0\\-3&2\end{vmatrix}=-2,\\A_{31}=\begin{vmatrix}0&1\\1&0\end{vmatrix}=-1,A_{32}=-\begin{vmatrix}1&1\\2&0\end{vmatrix}=2,A_{33}=\begin{vmatrix}1&0\\2&1\end{vmatrix}=1.\\∴ A^*=\begin{pmatrix}A_{11}&A_{21}&A_{31}\\A_{12}&A_{22}&A_{32}\\A_{13}&A_{23}&A_{33}\end{pmatrix}=\begin{pmatrix}-5&2&-1\\10&-2&2\\7&-2&1\end{pmatrix}\\\;\;\;\;\;\end{array}
$$



$$
设A,B\mathrm{是两个}n\mathrm{阶可逆矩阵},则\left[(AB)^T\right]^{-1}\mathrm{等于}().
$$
$$
A.
(A^T)^{-1}(B^T)^{-1} \quad B.(B^T)^{-1}(A^T)^{-1} \quad C.(B^{-1})^T(A^T)^{-1} \quad D.(B^{-1})(A^T)^{-1} \quad E. \quad F. \quad G. \quad H.
$$
$$
\left[(AB)^T\right]^{-1}=(B^TA^T)^{-1}=(A^T)^{-1}(B^T)^{-1}
$$



$$
设A=\begin{pmatrix}1&1&-1\\2&1&0\\1&-1&0\end{pmatrix},则A^*=().
$$
$$
A.
\begin{pmatrix}0&1&1\\0&1&-2\\-3&2&-1\end{pmatrix} \quad B.\begin{pmatrix}0&1&1\\0&1&-2\\-3&2&1\end{pmatrix} \quad C.\begin{pmatrix}1&0&0\\1&1&-2\\-3&2&-1\end{pmatrix} \quad D.\begin{pmatrix}1&0&0\\1&1&-2\\-3&2&-1\end{pmatrix} \quad E. \quad F. \quad G. \quad H.
$$
$$
\begin{array}{l}A_{11}=(-1)^2\begin{vmatrix}1&0\\-1&0\end{vmatrix}=0,A_{21}=(-1)^3\begin{vmatrix}1&-1\\-1&0\end{vmatrix}=1,\\A_{31}=(-1)^4\begin{vmatrix}1&-1\\1&0\end{vmatrix}=1,A_{12}=(-1)^3\begin{vmatrix}2&0\\1&0\end{vmatrix}=0,\\A_{22}=(-1)^4\begin{vmatrix}1&-1\\1&0\end{vmatrix}=1,A_{32}=(-1)^5\begin{vmatrix}1&-1\\2&0\end{vmatrix}=-2,\\A_{13}=(-1)^4\begin{vmatrix}2&1\\1&-1\end{vmatrix}=-3,A_{23}=(-1)^5\begin{vmatrix}1&1\\1&-1\end{vmatrix}=2,\\A_{33}=(-1)^6\begin{vmatrix}1&1\\2&1\end{vmatrix}=-1,\\故\;A^*=\begin{pmatrix}0&1&1\\0&1&-2\\-3&2&-1\end{pmatrix}\end{array}
$$



$$
设A,B\mathrm{是三阶方阵},且\left|A\right|=-1,\left|B\right|=4,则\left|8(A^TB^{-1})^2\right|=().\;
$$
$$
A.
32 \quad B.128 \quad C.64 \quad D.256 \quad E. \quad F. \quad G. \quad H.
$$
$$
\left|8(A^TB^{-1})^2\right|=8^3\left|A^T\right|^2\left|B^{-1}\right|^2=8^3\left|A\right|^2\left|B\right|^{-2}=8^3×1×\frac1{4^2}=32
$$



$$
\mathrm{设三阶方阵}A\mathrm{的行列式}\left|A\right|=\frac12,又B=(2A^2)^{-1}-2(A^{-1})^2,则\left|B\right|=().\;
$$
$$
A.
\frac{27}2 \quad B.-\frac{27}2 \quad C.\frac{27}8 \quad D.0 \quad E. \quad F. \quad G. \quad H.
$$
$$
\begin{array}{l}\mathrm{根据逆矩阵的性质可知},B=(2A^2)^{-1}-2(A^{-1})^2=\frac12(A^{-1})^2-2(A^{-1})^2=-\frac32(A^{-1})^2,则\\\left|B\right|=\left|-\frac32(A^{-1})^2\right|=(-\frac32)^3\left|A^{-1}\right|^2=-\frac{27}8×4=-\frac{27}2.\end{array}
$$



$$
设A,B\mathrm{为三阶方阵},且\left|A\right|=-1,\left|B\right|=2,则\left|2(A^TB^{-1})^2\right|=().\;
$$
$$
A.
2 \quad B.-2 \quad C.1 \quad D.-1 \quad E. \quad F. \quad G. \quad H.
$$
$$
\left|2(A^TB^{-1})^2\right|=2^3\left|A^TB^{-1}\right|^2=8\left|A\right|^2\left|B^{-1}\right|^2=8×(-1)^2×\frac14=2
$$



$$
设3\mathrm{阶矩阵}A有\left|A\right|=4,\left|A^2+E\right|=8则\left|A+A^{-1}\right|=(\;\;).
$$
$$
A.
2 \quad B.1 \quad C.4 \quad D.\frac12 \quad E. \quad F. \quad G. \quad H.
$$
$$
\left|A^2+E\right|=\left|A(A+A^{-1})\right|=\left|A\right|\left|A+A^{-1}\right|,即4\left|A+A^{-1}\right|=8,故\left|A+A^{-1}\right|=2.
$$



$$
设A\mathrm{是任一}n(n\geq3)\mathrm{阶方阵},A^*\mathrm{是其伴随矩阵},又k\mathrm{为常数},且k\neq0,±1,\mathrm{则必有}(kA)^*=().
$$
$$
A.
kA^* \quad B.k^{n-1}A^* \quad C.k^nA^* \quad D.k^{n+1}A^* \quad E. \quad F. \quad G. \quad H.
$$
$$
\begin{pmatrix}ka_{11}&⋯&ka_{1n}\\⋯&⋯&⋯\\ka_{na}&⋯&ka_{nm}\end{pmatrix}^*=\begin{pmatrix}k^{n-1}A_{11}&⋯&k^{n-1}A_{n1}\\⋯&⋯&⋯\\k^{n-1}A_{1n}&⋯&k^{n-1}A_{nm}\end{pmatrix}=k^{n-1}\begin{pmatrix}A_{11}&⋯&A_{n1}\\⋯&⋯&⋯\\A_{1n}&⋯&A_{nm}\end{pmatrix}=k^{n-1}A^*
$$



$$
\mathrm{已知矩阵}A=\begin{pmatrix}1&0&0\\0&\frac12&\frac32\\0&1&\frac52\end{pmatrix},则(A^*)^{-1}=().
$$
$$
A.
\begin{pmatrix}-4&0&0\\0&-2&-3\\0&-4&-10\end{pmatrix} \quad B.\begin{pmatrix}-4&0&0\\0&2&-6\\0&-4&-10\end{pmatrix} \quad C.\begin{pmatrix}-4&0&0\\0&-2&-6\\0&-4&-10\end{pmatrix} \quad D.\begin{pmatrix}-4&0&0\\0&-2&-6\\0&-4&10\end{pmatrix} \quad E. \quad F. \quad G. \quad H.
$$
$$
∵\left|A\right|=-\frac14,(A^*)^{-1}=\frac A{\left|A\right|}=-4A=\begin{pmatrix}-4&0&0\\0&-2&-6\\0&-4&-10\end{pmatrix}
$$



$$
设A=\begin{pmatrix}2&1&1&1\\1&2&1&1\\1&1&2&1\\1&1&1&2\end{pmatrix},B=A^{-1},则\;\left|B^2\right|=().
$$
$$
A.
\frac1{25} \quad B.-\frac1{25} \quad C.-35 \quad D.25 \quad E. \quad F. \quad G. \quad H.
$$
$$
\begin{array}{l}\left|A\right|=\begin{vmatrix}2&1&1&1\\1&2&1&1\\1&1&2&1\\1&1&1&2\end{vmatrix}=5\begin{vmatrix}1&1&1&1\\1&2&1&1\\1&1&2&1\\1&1&1&2\end{vmatrix}=5\begin{vmatrix}1&1&1&1\\0&1&0&0\\0&0&1&0\\0&0&0&1\end{vmatrix}=5,\\\left|B\right|=\left|A^{-1}\right|=\frac1{\left|A\right|}=\frac15,\\\left|B^2\right|=\left|B\right|^2=\frac1{25}.\end{array}
$$



$$
设A=\begin{pmatrix}2&2&2\\1&2&3\\1&3&6\end{pmatrix},B=A^{-1},则\left|B^3\right|=().
$$
$$
A.
\frac18 \quad B.\frac12 \quad C.\frac14 \quad D.1 \quad E. \quad F. \quad G. \quad H.
$$
$$
\begin{array}{l}\left|A\right|=\begin{vmatrix}2&2&2\\1&2&3\\1&3&6\end{vmatrix}=2,\left|B\right|=\left|A^{-1}\right|=\frac1{\left|A\right|}=\frac12,\\\left|B^3\right|=\left|B\right|^3=\frac18.\end{array}
$$



$$
设A,B\mathrm{都是}n\mathrm{阶方阵},A\mathrm{可逆},则\left|AA^{-1}\right|(\left|AB\right|-\frac{\left|B^T\right|}{\left|A^{-1}\right|})=().
$$
$$
A.
0 \quad B.1 \quad C.\left|A\right| \quad D.\left|A\right|·\left|B\right| \quad E. \quad F. \quad G. \quad H.
$$
$$
\mathrm{原式}=\left|E\right|(\left|A\right|\left|B\right|-\left|A\right|\left|B\right|)=1×0=0.
$$



$$
设A=\begin{pmatrix}1&2&0\\0&-1&1\\3&-2&-1\end{pmatrix},\mathrm{则行列式}\left|A^{-1}(A^T)^{-1}\right|\mathrm{的值为}().
$$
$$
A.
81 \quad B.\frac1{81} \quad C.\frac19 \quad D.9 \quad E. \quad F. \quad G. \quad H.
$$
$$
\left|A\right|=\begin{vmatrix}1&2&0\\0&-1&1\\3&-2&-1\end{vmatrix}=9,\left|A^{-1}(A^T)^{-1}\right|=\left|A^{-1}\right|^2=\frac1{\left|A\right|^2}=\frac1{81}.
$$



$$
\mathrm{当矩阵}A=\begin{pmatrix}\frac12&\frac{-\sqrt3}2\\\frac{\sqrt3}2&\frac12\end{pmatrix}时,A^6=E,则A^{11}=().\;
$$
$$
A.
\begin{pmatrix}\frac12&\frac{\sqrt3}2\\\frac{\sqrt3}2&\frac12\end{pmatrix} \quad B.\begin{pmatrix}-\frac12&\frac{\sqrt3}2\\-\frac{\sqrt3}2&-\frac12\end{pmatrix} \quad C.\begin{pmatrix}\frac12&-\frac{\sqrt3}2\\\frac{\sqrt3}2&\frac12\end{pmatrix} \quad D.\begin{pmatrix}\frac12&\frac{\sqrt3}2\\-\frac{\sqrt3}2&\frac12\end{pmatrix} \quad E. \quad F. \quad G. \quad H.
$$
$$
A^{11}=A^{12}A^{-1}=A^{-1}=\begin{pmatrix}\frac12&\frac{\sqrt3}2\\-\frac{\sqrt3}2&\frac12\end{pmatrix}
$$



$$
设A\mathrm{为三阶矩阵},\left|A\right|=\frac12,求\left|2A\right|-4\left|A^*\right|=().
$$
$$
A.
2 \quad B.8 \quad C.4 \quad D.3 \quad E. \quad F. \quad G. \quad H.
$$
$$
\left|2A\right|-4\left|A^*\right|=8\left|A\right|-4\left|A\right|^2=3
$$



$$
设AB=\begin{pmatrix}1&0&0\\1&1&0\\0&0&1\end{pmatrix},且A=\begin{pmatrix}1&0&3\\2&-1&1\\1&-2&1\end{pmatrix},则B^{-1}=().\;
$$
$$
A.
\begin{pmatrix}1&0&3\\1&-1&-2\\1&-2&1\end{pmatrix} \quad B.\begin{pmatrix}1&-1&3\\2&-3&1\\1&-3&1\end{pmatrix} \quad C.\begin{pmatrix}1&0&3\\3&-1&1\\3&-2&1\end{pmatrix} \quad D.\begin{pmatrix}-1&1&2\\2&-1&1\\1&-2&1\end{pmatrix} \quad E. \quad F. \quad G. \quad H.
$$
$$
\begin{array}{l}AB=\begin{pmatrix}1&0&0\\1&1&0\\0&0&1\end{pmatrix}⇒(AB)^{-1}=B^{-1}A^{-1}=\begin{pmatrix}1&0&0\\1&1&0\\0&0&1\end{pmatrix}^{-1},故\\B^{-1}=\begin{pmatrix}1&0&0\\1&1&0\\0&0&1\end{pmatrix}^{-1}· A=\begin{pmatrix}1&0&0\\1&1&0\\0&0&1\end{pmatrix}^{-1}\begin{pmatrix}1&0&3\\2&-1&1\\1&-2&1\end{pmatrix}=\begin{pmatrix}1&0&3\\1&-1&-2\\1&-2&1\end{pmatrix}\end{array}
$$



$$
设A为3×3\mathrm{矩阵},A^* 是A\mathrm{的伴随矩阵},若\left|A\right|=2,则\left|A^*\right|=().\;
$$
$$
A.
1 \quad B.0 \quad C.2 \quad D.4 \quad E. \quad F. \quad G. \quad H.
$$
$$
\begin{array}{l}\mathrm{因为}\;\left|A\right|=2,\mathrm{所以}A\mathrm{可逆}.\;\mathrm{由求逆公式得}\\\left|A^*\right|=\left|\left|A\right|A^{-1}\right|=\left|A\right|^3\left|A^{-1}\right|.\\\mathrm{又由}AA^{-1}=E得\\\left|A\right|\left|A^{-1}\right|=\left|E\right|,即\;\left|A^{-1}\right|=\frac1{\left|A\right|},\\\mathrm{代入}\left|A^*\right|得\\\left|A^*\right|=\left|A\right|^3·\frac1{\left|A\right|}=\left|A\right|^2=4.\end{array}
$$



$$
\begin{array}{l}设A是4\mathrm{阶方阵},A^*\mathrm{是它的伴随矩阵},\mathrm{则下列式子正确的是}().\\\end{array}
$$
$$
A.
AA^*=AE \quad B.AA^*=\left|A\right|^4E \quad C.AA^*=\left|A\right|E \quad D.AA^*=\left|A\right|^3 \quad E. \quad F. \quad G. \quad H.
$$
$$
\mathrm{伴随矩阵的基本性质为}:AA^*=\left|A\right|E,\mathrm{两边取行列式可得}\left|A\right|\left|A^*\right|=\left|A\right|^n=\left|A\right|^4.
$$



$$
设A为n\mathrm{阶可逆矩阵},E为n\mathrm{阶单位矩阵},\mathrm{则下列结论不正确的是}().\;
$$
$$
A.
A+E\mathrm{为可逆矩阵} \quad B.A\mathrm{的伴随矩阵}A^*\mathrm{可逆} \quad C.(A^{-1})^{-1}=A \quad D.\left|A\right|\neq0 \quad E. \quad F. \quad G. \quad H.
$$
$$
A+E\mathrm{不一定是可逆矩阵}
$$



$$
设A为n\mathrm{阶方阵},且\left|A\right|=2,则\left|A^*\right|=().
$$
$$
A.
2 \quad B.\frac12 \quad C.2^{n-1} \quad D.2^n \quad E. \quad F. \quad G. \quad H.
$$
$$
\left|A^* A\right|=\left|A\right|^n,\left|A^*\right|=\left|A\right|^{n-1}=2^{n-1}
$$



$$
\mathrm{设矩阵}A=\begin{pmatrix}1&4&5\\-1&0&0\\-1&-4&1\end{pmatrix},则AA^*=().
$$
$$
A.
24E \quad B.24 \quad C.12E \quad D.12 \quad E. \quad F. \quad G. \quad H.
$$
$$
\left|A\right|=\begin{vmatrix}1&4&5\\-1&0&0\\-1&-4&1\end{vmatrix}=\begin{array}{c}24,AA^*=24E\end{array}
$$



$$
设A为n\mathrm{阶方阵},且\left|A\right|=3,则\left|A^*\right|=().
$$
$$
A.
3 \quad B.\frac13 \quad C.3^{n-1} \quad D.3^n \quad E. \quad F. \quad G. \quad H.
$$
$$
\left|A^*\right|=\left|A\right|^{n-1}=3^{n-1}
$$



$$
设A为3\mathrm{阶方阵},A^* 是A\mathrm{的伴随矩阵},若\left|A\right|=2,则\left|-A^*\right|=().
$$
$$
A.
1 \quad B.4 \quad C.2 \quad D.-4 \quad E. \quad F. \quad G. \quad H.
$$
$$
\left|-A^*\right|=(-1)^3\left|A\right|^2=-4
$$



$$
设A为3\mathrm{阶方阵},A^* 是A\mathrm{的伴随矩阵且}\left|A\right|=3,则\left|2A^*\right|=().
$$
$$
A.
18 \quad B.9 \quad C.72 \quad D.24 \quad E. \quad F. \quad G. \quad H.
$$
$$
\left|2A^*\right|=2^3\left|A\right|^2=72
$$



$$
\mathrm{设矩阵}A=\begin{pmatrix}1&4&2\\2&0&0\\-1&4&1\end{pmatrix},A^* 为A\mathrm{的伴随矩阵},则AA^*=().\;
$$
$$
A.
8E \quad B.8 \quad C.12E \quad D.12 \quad E. \quad F. \quad G. \quad H.
$$
$$
\left|A\right|=8,AA^*=\left|A\right|E=8E
$$



$$
\mathrm{设矩阵}A=\begin{pmatrix}1&4&5\\-1&0&0\\-1&4&4\end{pmatrix},则\left|AA^*\right|=().
$$
$$
A.
64 \quad B.-64 \quad C.16 \quad D.-16 \quad E. \quad F. \quad G. \quad H.
$$
$$
\left|AA^*\right|=\left|A\right|^3=-64
$$



$$
\mathrm{设矩阵}A=\begin{pmatrix}1&0&0\\2&\frac12&0\\0&3&1\end{pmatrix},则\;(A^*)^{-1}=().
$$
$$
A.
\begin{pmatrix}1&0&0\\4&1&0\\0&6&2\end{pmatrix} \quad B.\begin{pmatrix}2&0&0\\4&1&0\\0&6&1\end{pmatrix} \quad C.\begin{pmatrix}2&0&0\\2&1&0\\0&6&2\end{pmatrix} \quad D.\begin{pmatrix}2&0&0\\4&1&0\\0&6&2\end{pmatrix} \quad E. \quad F. \quad G. \quad H.
$$
$$
(A^*)^{-1}=\frac1{\left|A\right|}A=2A=\begin{pmatrix}2&0&0\\4&1&0\\0&6&2\end{pmatrix}
$$



$$
\mathrm{设矩阵}A=\begin{pmatrix}1&0&0\\2&\frac14&0\\0&3&1\end{pmatrix},则(A^*)^{-1}=().
$$
$$
A.
\begin{pmatrix}4&0&0\\4&1&0\\0&6&4\end{pmatrix} \quad B.\begin{pmatrix}4&0&0\\8&1&0\\0&12&4\end{pmatrix} \quad C.\begin{pmatrix}2&0&0\\2&1&0\\0&6&2\end{pmatrix} \quad D.\begin{pmatrix}2&0&0\\4&1&0\\0&6&2\end{pmatrix} \quad E. \quad F. \quad G. \quad H.
$$
$$
(A^*)^{-1}=\frac A{\left|A\right|}=4A
$$



$$
设A=\begin{pmatrix}1&6&2\\-1&0&0\\-1&4&1\end{pmatrix},则\left|AA^*\right|=().\;
$$
$$
A.
8 \quad B.-8 \quad C.16 \quad D.-16 \quad E. \quad F. \quad G. \quad H.
$$
$$
\left|AA^*\right|=\left|A\right|^3=(-2)^3=-8
$$



$$
设A=\begin{pmatrix}2&1&0\\-1&1&0\\-1&4&1\end{pmatrix},则\;\left|AA^*\right|=().
$$
$$
A.
27 \quad B.-27 \quad C.9 \quad D.-9 \quad E. \quad F. \quad G. \quad H.
$$
$$
\left|AA^*\right|=\left|A\right|^3=27
$$



$$
\mathrm{设矩阵}A=\begin{pmatrix}1&0&0\\2&\frac12&0\\0&3&1\end{pmatrix},则(-A^*)^{-1}=().\;
$$
$$
A.
\begin{pmatrix}-1&0&0\\-4&-1&0\\0&-6&-2\end{pmatrix} \quad B.\begin{pmatrix}2&0&0\\-4&-1&0\\0&6&1\end{pmatrix} \quad C.\begin{pmatrix}-2&0&0\\-2&-1&0\\0&-6&-2\end{pmatrix} \quad D.\begin{pmatrix}-2&0&0\\-4&-1&0\\0&-6&-2\end{pmatrix} \quad E. \quad F. \quad G. \quad H.
$$
$$
(-A^*)^{-1}=-\frac A{\left|A\right|}=-2A
$$



$$
\mathrm{设矩阵}A=\begin{pmatrix}1&0&0\\2&\frac14&0\\0&3&1\end{pmatrix},则(-2A^*)^{-1}=().
$$
$$
A.
\begin{pmatrix}2&0&0\\4&1&0\\0&6&4\end{pmatrix} \quad B.\begin{pmatrix}-2&0&0\\-4&-\frac12&0\\0&-6&-2\end{pmatrix} \quad C.\begin{pmatrix}-2&0&0\\-4&-\frac12&0\\0&-6&-1\end{pmatrix} \quad D.\begin{pmatrix}-2&0&0\\-4&-1&0\\0&-6&-2\end{pmatrix} \quad E. \quad F. \quad G. \quad H.
$$
$$
(-2A^*)^{-1}=-\frac12(A^*)^{-1}=-\frac12×\frac A{\left|A\right|}=-2A
$$



$$
设A为3\mathrm{阶方阵},A^* 是A\mathrm{的伴随矩阵},若\left|A\right|=3,则\left|-2A^*\right|\;=()
$$
$$
A.
9 \quad B.-9 \quad C.72 \quad D.-72 \quad E. \quad F. \quad G. \quad H.
$$
$$
\left|-2A^*\right|=(-2)^3\left|A\right|^2=-72^{}
$$



$$
设A是n\mathrm{阶可逆方阵},且\left|A\right|=2,则\left|(A^TA^*)^T\right|\mathrm{等于}().
$$
$$
A.
\frac1{2^{n+1}} \quad B.2^n \quad C.2^{n+1} \quad D.\frac1{2^n} \quad E. \quad F. \quad G. \quad H.
$$
$$
\left|(A^TA^*)^T\right|=\left|A^TA^*\right|=\left|A\right|\left|A\right|^{n-1}=2^n
$$



$$
\mathrm{设矩阵}A\;是n\mathrm{阶可逆方阵},\;且\left|A\right|=2,则\left|(A^{-1}A^*)^{-1}\right|=().\;
$$
$$
A.
\frac1{2^{n-2}} \quad B.2^{n-2} \quad C.2^{n+1} \quad D.\frac1{2^n} \quad E. \quad F. \quad G. \quad H.
$$
$$
\left|(A^{-1}A^*)^{-1}\right|=\left|A\right|\left|(A^*)^{-1}\right|=\left|A\right|(\left|A\right|^{n-1})^{-1}=\frac1{2^{n-2}}
$$



$$
设A=\begin{pmatrix}1&0&0\\2&-1&0\\3&4&\frac12\end{pmatrix},则\;(A^*)^{-1}=().
$$
$$
A.
\begin{pmatrix}-2&0&0\\-4&2&0\\-6&-8&-2\end{pmatrix} \quad B.\begin{pmatrix}-2&0&0\\-4&-2&0\\-6&-8&-1\end{pmatrix} \quad C.\begin{pmatrix}-2&0&0\\-4&2&0\\-6&-8&-1\end{pmatrix} \quad D.\begin{pmatrix}-2&0&0\\-4&2&0\\-6&8&-1\end{pmatrix} \quad E. \quad F. \quad G. \quad H.
$$
$$
因A^*=\left|A\right|A^{-1},故(A^*)^{-1}=\frac1{\left|A\right|}A,\mathrm{又因}\left|A\right|=-\frac12,故(A^*)^{-1}=\begin{pmatrix}-2&0&0\\-4&2&0\\-6&-8&-1\end{pmatrix}.
$$



$$
设A=\begin{pmatrix}1&0&0\\2&2&0\\3&1&\frac14\end{pmatrix},A^* 是A\mathrm{的伴随矩阵},则(A^*)^{-1}=().
$$
$$
A.
\begin{pmatrix}2&0&0\\4&4&0\\6&2&1\end{pmatrix} \quad B.\begin{pmatrix}2&0&0\\4&4&0\\3&2&\frac12\end{pmatrix} \quad C.\begin{pmatrix}2&0&0\\4&4&0\\6&2&\frac12\end{pmatrix} \quad D.\begin{pmatrix}2&0&0\\4&4&0\\6&2&\frac14\end{pmatrix} \quad E. \quad F. \quad G. \quad H.
$$
$$
\begin{array}{l}\mathrm{因为}\left|A\right|=\frac12,\mathrm{所以}A\mathrm{可逆}.\;\mathrm{由求逆公式得}\\(A^*)^{-1}=\frac A{\left|A\right|}=2A=\begin{pmatrix}2&0&0\\4&4&0\\6&2&\frac12\end{pmatrix}\end{array}
$$



$$
设A是n\mathrm{阶可逆方阵},且\left|A\right|=2,则\left|(A^TA^*)^{-1}\right|\mathrm{等于}().
$$
$$
A.
\frac1{2^{n+1}} \quad B.2^n \quad C.2^{n+1} \quad D.\frac1{2^n} \quad E. \quad F. \quad G. \quad H.
$$
$$
\left|(A^TA^*)^{{}^{-1}}\right|=\frac1{\left|A\right|}\frac1{\left|A^*\right|}=\frac12\frac1{2^{n-1}}=\frac1{2^n}.
$$



$$
设A是n\mathrm{阶可逆方阵},且\left|A\right|=2,则\left|(A^*)^{-1}\right|\mathrm{等于}().
$$
$$
A.
\frac1{2^{n-1}} \quad B.2^n \quad C.2^{n-1} \quad D.\frac1{2^n} \quad E. \quad F. \quad G. \quad H.
$$
$$
\left|(A^*)^{-1}\right|=\frac1{\left|A^*\right|}=\frac1{\left|A\right|^{n-1}}=\frac1{2^{n-1}}
$$



$$
设A为3\mathrm{阶方阵},A^* 是A\mathrm{的伴随矩阵},若\left|A\right|=2,则\;\left|-3A^*\right|=().
$$
$$
A.
72 \quad B.-108 \quad C.108 \quad D.-72 \quad E. \quad F. \quad G. \quad H.
$$
$$
\left|-3A^*\right|=(-3)^3×\left|A\right|^2=-108
$$



$$
\mathrm{设矩阵}\;A=\begin{pmatrix}1&0&0\\2&2&0\\3&1&\frac14\end{pmatrix},A^* 是A\mathrm{的伴随矩阵},则(2A^*)^{-1}=().
$$
$$
A.
\begin{pmatrix}2&0&0\\2&2&0\\6&2&\frac14\end{pmatrix} \quad B.\begin{pmatrix}1&0&0\\2&2&0\\3&1&\frac12\end{pmatrix} \quad C.\begin{pmatrix}2&0&0\\4&4&0\\6&1&\frac12\end{pmatrix} \quad D.\begin{pmatrix}1&0&0\\2&2&0\\3&1&\frac14\end{pmatrix} \quad E. \quad F. \quad G. \quad H.
$$
$$
\begin{array}{l}\mathrm{因为}\;\left|A\right|=\frac12\mathrm{所以}A\mathrm{可逆}.\;\mathrm{由求逆公式得}\\(2A^*)^{-1}=\frac12\frac A{\left|A\right|}=A=\begin{pmatrix}1&0&0\\2&2&0\\3&1&\frac14\end{pmatrix}\;\end{array}
$$



$$
\mathrm{设矩阵}A=\begin{pmatrix}2&0&0\\4&4&6\\4&2&2\end{pmatrix},A^* 是A\mathrm{的伴随矩阵},则\;(\frac18A^*)^{-1}=().
$$
$$
A.
\begin{pmatrix}2&0&0\\4&4&6\\4&2&2\end{pmatrix} \quad B.\begin{pmatrix}1&0&0\\2&2&3\\2&1&1\end{pmatrix} \quad C.\begin{pmatrix}-2&0&0\\-4&-4&-6\\-4&-2&-2\end{pmatrix} \quad D.\begin{pmatrix}-1&0&0\\-2&-2&-3\\-2&-1&-1\end{pmatrix} \quad E. \quad F. \quad G. \quad H.
$$
$$
\begin{array}{l}\mathrm{因为}\;\left|A\right|=-8,\mathrm{所以}A\mathrm{可逆}.\;\mathrm{由求逆公式得}\\(\frac18A^*)^{-1}=8\frac A{\left|A\right|}=8×\frac1{-8}A=-A=\begin{pmatrix}-2&0&0\\-4&-4&-6\\-4&-2&-2\end{pmatrix}\end{array}
$$



$$
\mathrm{设三阶方阵}A\mathrm{的行列式为}\left|A\right|=2,A^* 为A\mathrm{的伴随矩阵},\mathrm{则行列式}\left|(\frac13A)^{-1}-A^*\right|=(\;).\;
$$
$$
A.
-\frac12 \quad B.\frac12 \quad C.-\frac13 \quad D.\frac13 \quad E. \quad F. \quad G. \quad H.
$$
$$
\left|(\frac13A)^{-1}-A^*\right|=\left|3A^{-1}-\left|A\right|A^{-1}\right|=\left|A^{-1}\right|=\frac1{\left|A\right|}=\frac12
$$



$$
设A是n\mathrm{阶可逆方阵},且\left|A\right|=3,\left|(A^{-1}A^\ast)^{-1}\right|\mathrm{则等于}().
$$
$$
A.
\frac1{3^{n-2}} \quad B.3^{n-2} \quad C.3^{n+1} \quad D.\frac1{3^n} \quad E. \quad F. \quad G. \quad H.
$$
$$
\left|(A^{-1}A^*)^{-1}\right|=\left|(A^*)^{-1}\right|\left|A\right|=3×\frac1{\left|A^*\right|}=\frac1{3^{n-2}}
$$



$$
设A是n\mathrm{阶可逆方阵},且\left|A\right|=3,则\left|(A^TA^*)^T\right|\mathrm{等于}().\;
$$
$$
A.
\frac1{3^{n+1}} \quad B.3^n \quad C.3^{n+1} \quad D.\frac1{3^n} \quad E. \quad F. \quad G. \quad H.
$$
$$
\left|(A^TA^*)^T\right|=\left|A^TA^*\right|=\left|A^\ast\right|\left|A\right|=\left|A\right|^n=3^n
$$



$$
设A是n\mathrm{阶可逆方阵},且\left|A\right|=3,则\left|A^TA^*\right|\mathrm{等于}().
$$
$$
A.
\frac1{3^{n+1}} \quad B.3^n \quad C.3^{n+1} \quad D.\frac1{3^n} \quad E. \quad F. \quad G. \quad H.
$$
$$
\left|A^TA^*\right|=\left|A\right|\left|A\right|^{n-1}=3^n
$$



$$
设A是n\mathrm{阶可逆方阵},且\left|A\right|=2,则\left|(A^{-1}A^*)^T\right|\mathrm{等于}().
$$
$$
A.
\frac1{2^n} \quad B.2^n \quad C.2^{n-2} \quad D.2^{n-1} \quad E. \quad F. \quad G. \quad H.
$$
$$
\left|(A^{-1}A^*)^T\right|=\left|A\right|^{-1}\left|A\right|^{n-1}=\left|A\right|^{n-2}=2^{n-2}
$$



$$
设A\mathrm{均为}n\mathrm{阶可逆矩阵},A^* 是A\mathrm{的伴随矩阵},0是n\mathrm{阶零矩阵},\mathrm{则下列各式正确的是}().
$$
$$
A.
(2A^{-1})=2A^{-1} \quad B.AA^*\neq0 \quad C.\left|A^*\right|^{-1}=\frac{A^{-1}}{\left|A\right|} \quad D.\left[(A^{-1})^T\right]^{-1}=\left[(A^T)^{-1}\right]^T \quad E. \quad F. \quad G. \quad H.
$$
$$
AA^*\neq0
$$



$$
\mathrm{设矩阵}A=\begin{pmatrix}1&0&0\\2&\frac12&0\\3&4&1\end{pmatrix},A^* 是A\mathrm{的伴随矩阵},则(A^*)^{-1}=(\;)
$$
$$
A.
\begin{pmatrix}2&0&0\\4&1&0\\6&8&1\end{pmatrix} \quad B.\begin{pmatrix}2&0&0\\2&1&0\\6&8&2\end{pmatrix} \quad C.\begin{pmatrix}2&0&0\\4&1&0\\6&8&2\end{pmatrix} \quad D.\begin{pmatrix}2&0&0\\4&1&0\\6&4&2\end{pmatrix} \quad E. \quad F. \quad G. \quad H.
$$
$$
(A^*)^{-1}=(\left|A\right|A^{-1})^{-1}=\frac1{\left|A\right|}A,\mathrm{因为}\left|A\right|=\frac12,(A^*)^{-1}=2A=\begin{pmatrix}2&0&0\\4&1&0\\6&8&2\end{pmatrix}
$$



$$
设A是3\mathrm{阶可逆方阵},且\left|A\right|=2,则\left|-A^TA^*\right|=().
$$
$$
A.
8 \quad B.-8 \quad C.4 \quad D.-4 \quad E. \quad F. \quad G. \quad H.
$$
$$
\left|-A^TA^*\right|=(-1)^3\left|A\right|\left|A\right|^2=-8
$$



$$
\mathrm{设矩阵}A=\begin{pmatrix}1&0&0\\0&\frac12&\frac32\\0&1&\frac52\end{pmatrix}\;,A^* 是A\mathrm{的伴随矩阵},则\left[(A^*)^{-1}\right]^T=().
$$
$$
A.
\begin{pmatrix}-4&0&0\\0&-2&-4\\0&-6&-10\end{pmatrix} \quad B.\begin{pmatrix}4&0&0\\0&-2&4\\0&-6&10\end{pmatrix} \quad C.\begin{pmatrix}-4&0&0\\0&2&-4\\0&6&-10\end{pmatrix} \quad D.\begin{pmatrix}4&0&0\\0&2&4\\0&6&10\end{pmatrix} \quad E. \quad F. \quad G. \quad H.
$$
$$
\left[(A^*)^{-1}\right]^T=\lbrack\frac A{\left|A\right|}\rbrack^T\begin{array}{l}=(-4A)^T=\begin{pmatrix}-4&0&0\\0&-2&-4\\0&-6&-10\end{pmatrix}.\end{array}
$$



$$
\mathrm{设矩阵}\;A=\begin{pmatrix}1&0&0\\0&1&2\\0&1&\frac52\end{pmatrix},A^* 是A\mathrm{的伴随矩阵},则\left[(A^*)^T\right]^{-1}=().
$$
$$
A.
\begin{pmatrix}-4&0&0\\0&-2&-4\\0&-6&-10\end{pmatrix} \quad B.\begin{pmatrix}4&0&0\\0&-2&4\\0&-6&10\end{pmatrix} \quad C.\begin{pmatrix}2&0&0\\0&2&2\\0&4&5\end{pmatrix} \quad D.\begin{pmatrix}2&0&0\\0&2&4\\0&6&10\end{pmatrix} \quad E. \quad F. \quad G. \quad H.
$$
$$
\left[(A^*)^T\right]^{-1}=\left[(A^*)^{-1}\right]^T=\left[\frac A{\left|A\right|}\right]^T=(2A)^T=\begin{pmatrix}2&0&0\\0&2&4\\0&2&5\end{pmatrix}^T=\begin{pmatrix}2&0&0\\0&2&2\\0&4&5\end{pmatrix}
$$



$$
设A是n\mathrm{阶可逆方阵},且\left|A\right|=3,则\left|(3A^TA^*)^T\right|\mathrm{等于}().
$$
$$
A.
\frac1{3^{2n+1}} \quad B.3^{2n} \quad C.3^{n+1} \quad D.\frac1{3^n} \quad E. \quad F. \quad G. \quad H.
$$
$$
\left|(3A^TA^*)^T\right|=3^n×\left|A\right|\left|A\right|^{n-1}=3^{2n}
$$



$$
\mathrm{若三阶矩阵}A\mathrm{的伴随矩阵为}A^*,\;\mathrm{已知}\left|A\right|=3\;,则\left|(\frac12A)^{-1}-A^*\right|=().\;
$$
$$
A.
\frac13 \quad B.-\frac13 \quad C.\frac19 \quad D.-\frac19 \quad E. \quad F. \quad G. \quad H.
$$
$$
\left|(\frac12A)^{-1}-A^*\right|=\left|2A^{-1}-\left|A\right|A^{-1}\right|=-\frac13
$$



$$
设4\mathrm{阶方阵}A=\begin{pmatrix}0&0&1&2\\0&0&0&1\\3&3&0&0\\2&1&0&0\end{pmatrix},则\;(A^*)^{-1}=().
$$
$$
A.
\begin{pmatrix}0&0&-\frac13&\frac23\\0&0&0&-\frac13\\-1&-1&0&0\\-\frac23&-\frac13&0&0\end{pmatrix} \quad B.\begin{pmatrix}0&0&-\frac13&-\frac23\\0&0&0&-\frac13\\-1&1&0&0\\-\frac23&-\frac13&0&0\end{pmatrix} \quad C.\begin{pmatrix}0&0&-\frac13&-\frac23\\0&0&0&-\frac13\\1&1&0&0\\-\frac23&-\frac13&0&0\end{pmatrix} \quad D.\begin{array}{l}\begin{pmatrix}0&0&-\frac13&-\frac23\\0&0&0&-\frac13\\-1&-1&0&0\\-\frac23&-\frac13&0&0\end{pmatrix}\\\end{array} \quad E. \quad F. \quad G. \quad H.
$$
$$
∵ AA^*=\left|A\right|E,∵\left|A\right|=-3,∴(A^*)^{-1}=\frac A{\left|A\right|}=-\frac13A=\begin{pmatrix}0&0&-\frac13&-\frac23\\0&0&0&-\frac13\\-1&-1&0&0\\-\frac23&-\frac13&0&0\end{pmatrix}
$$



$$
A,B\mathrm{都是}3\mathrm{阶可逆矩阵},且\left|A\right|=2,\left|B\right|=\frac32,则\left|(AB)^*\right|=().
$$
$$
A.
3 \quad B.9 \quad C.4 \quad D.\frac19 \quad E. \quad F. \quad G. \quad H.
$$
$$
由\left|A^*\right|=\left|A\right|^{n-1}\mathrm{可知}\left|(AB)^*\right|=\left|AB\right|^{n-1}=(\left|A\right|\left|B\right|)^2=(2×\frac32)^2=9.
$$



$$
设AB=\begin{pmatrix}1&1&0\\0&1&0\\0&0&1\end{pmatrix},且B=\begin{pmatrix}1&0&3\\2&1&-1\\1&-2&1\end{pmatrix},则A^{-1}=().\;
$$
$$
A.
\begin{pmatrix}1&1&3\\2&3&-1\\1&-1&1\end{pmatrix} \quad B.\begin{pmatrix}1&0&3\\1&1&-1\\3&-2&1\end{pmatrix} \quad C.\begin{pmatrix}-2&0&3\\3&1&-1\\0&-2&1\end{pmatrix} \quad D.\begin{pmatrix}1&-1&3\\2&-1&-1\\1&-3&1\end{pmatrix} \quad E. \quad F. \quad G. \quad H.
$$
$$
\begin{array}{l}AB=\begin{pmatrix}0&1&0\\0&1&0\\0&0&1\end{pmatrix},且B=\begin{pmatrix}1&0&3\\2&1&-1\\1&-2&1\end{pmatrix},则\\(AB)^{-1}=B^{-1}A^{-1}=\begin{pmatrix}1&1&0\\0&1&0\\0&0&1\end{pmatrix}^{-1},\\A^{-1}=B\begin{pmatrix}1&1&0\\0&1&0\\0&0&1\end{pmatrix}^{-1}=\begin{pmatrix}1&-1&3\\2&-1&-1\\1&-3&1\end{pmatrix}.\end{array}
$$



$$
\mathrm{设矩阵}A=\begin{pmatrix}1&0&0\\0&\frac12&\frac32\\0&1&\frac52\end{pmatrix},A^* 是A\mathrm{的伴随矩阵},则\left[(A^*)^T\right]^{-1}=().
$$
$$
A.
\begin{pmatrix}-4&0&0\\0&-2&-4\\0&-6&-10\end{pmatrix} \quad B.\begin{pmatrix}4&0&0\\0&-2&4\\0&-6&10\end{pmatrix} \quad C.\begin{pmatrix}-4&0&0\\0&2&-4\\0&6&-10\end{pmatrix} \quad D.\begin{pmatrix}4&0&0\\0&2&4\\0&6&10\end{pmatrix} \quad E. \quad F. \quad G. \quad H.
$$
$$
\begin{array}{l}因(A^*)^{-1}=(\left|A\right|A^{-1})^{-1}=\frac1{\left|A\right|}A,\\故\left[(A^*)^T\right]^{-1}=\left[(A^*)^{-1}\right]^T=\begin{pmatrix}\frac1{\left|A\right|}A\end{pmatrix}^T=\begin{array}{c}\frac1{\left|A\right|}A^T\end{array}=\begin{pmatrix}-4&0&0\\0&-2&-4\\0&-6&-10\end{pmatrix}\end{array}
$$



$$
\mathrm{若三阶矩阵}A\mathrm{的伴随矩阵为}A^*,\;\mathrm{已知}\left|A\right|=\frac12,\;则\left|(3A)^{-1}-2A^*\right|=().\;
$$
$$
A.
-\frac{16}{27} \quad B.\frac{16}{27} \quad C.-\frac8{27} \quad D.\frac8{27} \quad E. \quad F. \quad G. \quad H.
$$
$$
\begin{array}{l}∵ A^*=\left|A\right|A^{-1}=\frac12A^{-1},\\∴\left|(3A)^{-1}-2A^*\right|=\left|\frac13A^{-1}-A^{-1}\right|=\left|-\frac23A^{-1}\right|\\=(-\frac23)^3\left|A^{-1}\right|=-\frac{16}{27}\end{array}
$$



$$
设A\mathrm{为三阶矩阵},A^*\mathrm{为其伴随矩阵},\left|A\right|=\frac12,则\left|(\frac13A)^{-1}-10A^*\right|=().\;
$$
$$
A.
16 \quad B.-16 \quad C.-4 \quad D.4 \quad E. \quad F. \quad G. \quad H.
$$
$$
\begin{array}{l}\left|(\frac13A)^{-1}-10A^*\right|=\left|3A^{-1}-10\left|A\right|A^{-1}\right|\\=\left|3A^{-1}-5A^{-1}\right|=\left|-2A^{-1}\right|=-8\left|A\right|^{-1}\\=-16\end{array}
$$



$$
设A\mathrm{为三阶矩阵},且\left|A\right|=2,则\left|3A-(A^*)^*\right|=().
$$
$$
A.
2 \quad B.1 \quad C.\frac12 \quad D.\frac14 \quad E. \quad F. \quad G. \quad H.
$$
$$
\begin{array}{l}\left|3A-(A^*)^*\right|=\left|3A-\left|A^*\right|(A^*)^{-1}\right|=\left|3A-\left|A\right|^{3-1}\frac1{\left|A\right|}A\right|\\=\left|3A-\left|A\right|A\right|=\left|A\right|=2.\end{array}
$$



$$
设A,B\mathrm{均为}n\mathrm{阶可逆矩阵},A^*,B^*,(AB)^*\mathrm{分别为}A,B,AB\mathrm{的伴随矩阵},\mathrm{则下列正确的是}().
$$
$$
A.
A^* B^*\mathrm{不可逆} \quad B.(AB)^*=B^{-1}A^{-1} \quad C.(AB)^*=\left|AB\right|B^{-1}A^{-1} \quad D.(AB)^*=\left|AB\right|A^{-1}B^{-1} \quad E. \quad F. \quad G. \quad H.
$$
$$
\begin{array}{l}(AB)^*(AB)=\left|AB\right|E⇒(AB)^*=\left|AB\right|B^{-1}A^{-1}\\B^* A^*=\left|B\right|\left|A\right|B^{-1}A^{-1}\end{array}
$$



$$
设A,B\mathrm{均为}n\mathrm{阶方阵},\left|A\right|=2,\left|B\right|=-3,则\left|2A^* B^{-1}\right|=().\;
$$
$$
A.
-\frac{2^n}3 \quad B.-\frac{2^{2n-1}}3 \quad C.\frac{2^{2n}}3 \quad D.\frac{2^n}3 \quad E. \quad F. \quad G. \quad H.
$$
$$
\left|2A^* B^{-1}\right|=2^n\left|A^*\right|\left|B^{-1}\right|=2^n\left|A\right|^{n-1}\left|B\right|^{-1}=2^n×2^{n-1}×(-\frac13)=-\frac{2^{2n-1}}3
$$



$$
设A\mathrm{是三阶方阵},且\left|A\right|=3,则\left|3A^{-1}-2A^*\right|=().
$$
$$
A.
9 \quad B.-9 \quad C.27 \quad D.-27 \quad E. \quad F. \quad G. \quad H.
$$
$$
\left|3A^{-1}-2A^*\right|=\left|3A^{-1}-2\left|A\right|A^{-1}\right|=\left|-3A^{-1}\right|=(-3)^3\left|A\right|^{-1}=-9.
$$



$$
设A=\begin{pmatrix}-1&3&5\\0&1&4\\0&0&2\end{pmatrix},则\left|A+(A^*)^{-1}\right|=().
$$
$$
A.
\frac14 \quad B.-\frac14 \quad C.\frac12 \quad D.-\frac12 \quad E. \quad F. \quad G. \quad H.
$$
$$
\begin{array}{l}\mathrm{由于}AA^*=\left|A\right|E,则(A^*)^{-1}=\frac1{\left|A\right|}A,又\left|A\right|=-2,故\left(A^*\right)^{-1}=-\frac12A,则\\\left|A+(A^*)^{-1}\right|=\left|A-\frac12A\right|=(\frac12)^3\left|A\right|=-\frac14.\end{array}
$$



$$
\mathrm{若三阶矩阵}A\mathrm{的伴随矩阵为}A^*,\;\mathrm{已知}\left|A\right|=\frac12,\;则\left|(\frac13A)^{-1}-2A^*\right|=().\;
$$
$$
A.
-16 \quad B.16 \quad C.-8 \quad D.8 \quad E. \quad F. \quad G. \quad H.
$$
$$
\begin{array}{l}∵ A^*=\left|A\right|A^{-1}=\frac12A^{-1},\\∴\left|(\frac13A)^{-1}-2A^*\right|=\left|3A^{-1}-A^{-1}\right|=\left|2A^{-1}\right|=16\end{array}
$$



$$
设A,B为4\mathrm{阶方阵},且\left|A\right|=\left|B\right|=2,\mathrm{则行列式}\left|A^* B^{-1}A^T\right|=(\;)
$$
$$
A.
64 \quad B.32 \quad C.8 \quad D.16 \quad E. \quad F. \quad G. \quad H.
$$
$$
\left|A^* B^{-1}A^T\right|=\frac{\left|A\right|^3}{\left|B\right|}\left|A\right|=8
$$



$$
设A,B为4\mathrm{阶方阵},且\left|A\right|=\left|B\right|=2,\mathrm{则行列式}\left|(A^* B^{-1})^2A^T\right|=(\;)
$$
$$
A.
64 \quad B.32 \quad C.8 \quad D.16 \quad E. \quad F. \quad G. \quad H.
$$
$$
\left|(A^* B^{-1})^2A^T\right|=(\left|A^*\right|\left|B\right|^{-1})^2\left|A^T\right|=(\left|A\right|^3\left|B\right|^{-1})^2\left|A\right|=\left|A\right|^6\frac1{\left|B\right|^2}\left|A\right|=\left|A\right|^7\frac1{\left|B\right|^2}=2^5=32
$$



$$
设n\mathrm{阶矩阵}A\mathrm{非奇异}(n\geq2),A^*\mathrm{是矩阵}A\mathrm{的伴随矩阵},\mathrm{则下列等式正确的}()
$$
$$
A.
(A^*)^*=\left|A\right|^{n-1}A \quad B.(A^*)^*=\left|A\right|^{n+1}A \quad C.(A^*)^*=\left|A\right|^{n-2}A \quad D.(A^*)^*=\left|A\right|^{n+2}A \quad E. \quad F. \quad G. \quad H.
$$
$$
\begin{array}{l}\mathrm{根据条件}A\mathrm{可逆},即\left|A\right|\neq0,而A^{-1}=\frac1{\left|A\right|}A^*⇒\left|A^*\right|=\left|A\right|^{n-1}\neq0\\\mathrm{因而}A^*\mathrm{可逆},由A^*=\left|A\right|A^{-1}知\\(A^*)^*=\left|A^*\right|(A^*)^{-1}=\left|A^*\right|(\left|A\right|A^{-1})^{-1}=\left|A\right|^{n-1}\frac A{\left|A\right|}=\left|A\right|^{n-2}A\\\end{array}
$$



$$
设A\mathrm{为三阶矩阵},\left|A\right|=3,A^* 为A\mathrm{的伴随矩阵},\mathrm{若交换}A\mathrm{的第一行与第二行得到矩阵}B,则\left|BA^*\right|=(\;)
$$
$$
A.
-9 \quad B.-27 \quad C.27 \quad D.9 \quad E. \quad F. \quad G. \quad H.
$$
$$
\left|BA^*\right|=\left|B\right|\left|A^*\right|=-\left|A\right|\left|A\right|^{3-1}=-27
$$



$$
设A\mathrm{为三阶矩阵},\left|A\right|=2,A^* 为A\mathrm{的伴随矩阵},\mathrm{若把}A\mathrm{的第一行元素的}3\mathrm{倍加到第二行得到矩阵}B,则\left|BA^*\right|=(\;)
$$
$$
A.
-8 \quad B.8 \quad C.6 \quad D.-6 \quad E. \quad F. \quad G. \quad H.
$$
$$
\left|BA^*\right|=\left|B\right|\left|A\right|^{3-1}=\left|A\right|^3=8
$$



$$
设A,B为n\mathrm{阶方阵},且\left|A\right|=2,\left|B\right|=3,\mathrm{则行列式}\left|A^* B^{-1}A^T\right|=(\;)
$$
$$
A.
\frac{2^n}3 \quad B.\frac{2^{n-1}}3 \quad C.\frac{2^{n-2}}3 \quad D.\frac{2^{n+1}}3 \quad E. \quad F. \quad G. \quad H.
$$
$$
\left|A^* B^{-1}A^T\right|=\left|A\right|^n\left|B\right|^{-1}=\frac{2^n}3
$$



$$
设A,B为4\mathrm{阶方阵},且\left|A\right|=\left|B\right|=2,\mathrm{则行列式}\left|(A^* B^{-1})^2A^{-1}\right|=(\;)
$$
$$
A.
8 \quad B.16 \quad C.12 \quad D.32 \quad E. \quad F. \quad G. \quad H.
$$
$$
\left|(A^* B^{-1})^2A^{-1}\right|=(\left|A\right|^{n-1}\left|B\right|^{-1})^2\left|A\right|^{-1}=8
$$



$$
A,B\mathrm{都是}3\mathrm{阶可逆矩阵},且\vert A\vert=2,\vert B\vert=\frac34,则\vert(AB)^*\vert=().
$$
$$
A.
4 \quad B.9 \quad C.\frac94 \quad D.\frac49 \quad E. \quad F. \quad G. \quad H.
$$
$$
\vert(AB)^*\vert=\left|AB\right|^2=\frac94
$$



$$
A,B\mathrm{都是}3\mathrm{阶可逆矩阵},且\vert A\vert=3,\vert B\vert=\frac53,则\vert(AB)^*\vert=().
$$
$$
A.
5 \quad B.25 \quad C.\frac{25}9 \quad D.\frac9{25} \quad E. \quad F. \quad G. \quad H.
$$
$$
\vert(AB)^*\vert=\left|AB\right|^2=25
$$



$$
A,B\mathrm{都是}4\mathrm{阶可逆矩阵},且\vert A\vert=3,\vert B\vert=\frac13,则\vert(AB)^*\vert=().
$$
$$
A.
1 \quad B.\frac13 \quad C.\frac19 \quad D.\frac1{27} \quad E. \quad F. \quad G. \quad H.
$$
$$
\vert(AB)^*\vert=\left|AB\right|^3=1
$$



$$
A,B\mathrm{都是}4\mathrm{阶可逆矩阵},且\vert A\vert=3,\vert B\vert=\frac43,则\vert(AB)^*\vert=().
$$
$$
A.
4 \quad B.8 \quad C.16 \quad D.64 \quad E. \quad F. \quad G. \quad H.
$$
$$
则\vert(AB)^*\vert=\left|AB\right|^3=4^3=64
$$



$$
A,B\mathrm{都是}4\mathrm{阶可逆矩阵},且\vert A\vert=4,\vert B\vert=\frac34,则\vert(AB)^*\vert=().
$$
$$
A.
27 \quad B.3 \quad C.9 \quad D.81 \quad E. \quad F. \quad G. \quad H.
$$
$$
\vert(AB)^*\vert=\left|AB\right|^3=27
$$



$$
\;\mathrm{若三阶矩阵}A\mathrm{的伴随矩阵为}A^*,\mathrm{已知}\vert A\vert=2,则\left|\right|(\frac13A)^{-1}-2A^*\vert=().
$$
$$
A.
\frac12 \quad B.\frac14 \quad C.-\frac12 \quad D.-\frac14 \quad E. \quad F. \quad G. \quad H.
$$
$$
\left|(\frac13A)^{-1}-2A^*\right|=\left|3A^{-1}-2\left|A\right|A^{-1}\right|=\left|-A^{-1}\right|=-\frac12
$$



$$
\mathrm{若三阶矩阵}A\mathrm{的伴随矩阵为}A^*,\mathrm{已知}\vert A\vert=2,则\vert(3A)^{-1}-\frac12A^*\vert=().
$$
$$
A.
\frac4{27} \quad B.-\frac4{27} \quad C.\frac23 \quad D.-\frac49 \quad E. \quad F. \quad G. \quad H.
$$
$$
\vert(3A)^{-1}-\frac12A^*\vert=\left|\frac13A^{-1}-\frac12\left|A\right|A^{-1}\right|=\left|-\frac23A^{-1}\right|-(-\frac23)^3×\frac1{\left|A\right|}=-\frac4{27}
$$



$$
\mathrm{设矩阵}A=\begin{pmatrix}2&3\\4&5\end{pmatrix},A^* 为A\mathrm{的伴随矩阵},则A^*=().\;
$$
$$
A.
\begin{pmatrix}5&-3\\-4&2\end{pmatrix} \quad B.\begin{pmatrix}-\frac52&\frac32\\2&-1\end{pmatrix} \quad C.\begin{pmatrix}-5&-3\\-4&-2\end{pmatrix} \quad D.\begin{pmatrix}5&-4\\-3&2\end{pmatrix} \quad E. \quad F. \quad G. \quad H.
$$
$$
A=\begin{bmatrix}2&3\\4&5\end{bmatrix},则A_{11}=5,\;\;A_{12}=-4,\;\;A_{21}=-3,\;\;A_{22}=2,A^*=\begin{bmatrix}5&-3\\-4&2\end{bmatrix}
$$



$$
设n\mathrm{阶矩阵}A及s\mathrm{阶矩阵}B\mathrm{都可逆},则\begin{pmatrix}O\;\;A\\B\;\;O\end{pmatrix}^{-1}\;=(\;)·
$$
$$
A.
\begin{pmatrix}O\;\;B^{-1}\\A^{-1}\;\;O\end{pmatrix} \quad B.\begin{pmatrix}O\;\;A^{-1}\\B^{-1}\;\;O\end{pmatrix} \quad C.\begin{pmatrix}B^{-1}\;\;O\\O\;\;A^{-1}\end{pmatrix} \quad D.\begin{pmatrix}A^{-1}\;\;O\\O\;\;B^{-1}\end{pmatrix} \quad E. \quad F. \quad G. \quad H.
$$
$$
\begin{array}{l}将\begin{pmatrix}O\;\;A\\B\;\;O\end{pmatrix}^{-1}\;\mathrm{分块为}\begin{pmatrix}C_1\;\;C_2\\C_3\;\;C_4\end{pmatrix},\mathrm{其中}C_1为s× n\mathrm{矩阵},C_2为s× s\mathrm{矩阵},C_3为n× n\mathrm{矩阵},C4为n× s\mathrm{矩阵}\\则\;\;\;\;\;\;\;\;\;\;\;\;\;\;\;\;\;\;\;\;\;\;\;\;\;\;\;\;\;\;\;\;\\\;\;\;\;\;\;\;\;\;\;\;\;\;\;\;\;\;\;\;\;\;\;\;\;\;\;\;\;\;\;\;\;\;\;\;\;\;\;\;\;\;\;\;\;\;\;\;\begin{pmatrix}O\;\;A_{n× n}\\Bs× s\;\;O\end{pmatrix}\;\begin{pmatrix}C_1\;\;C_2\\C_3\;\;C_4\end{pmatrix}=E=\begin{pmatrix}E_n\;\;O\\O\;\;E\end{pmatrix}·\\\mathrm{由此得}\;\;\;\;\;\;\;\;\;\;\;\;\;\;\;\;\;\;AC_3=E_n⇒ C_3=A^{-1}\\\;\;\;\;\;\;\;\;\;\;\;\;\;\;\;\;\;\;\;\;\;\;\;\;\;\;\;\;\;\;AC_4=O⇒ C_4=O(A^{-1}\;\mathrm{存在}\;)\\\;\;\;\;\;\;\;\;\;\;\;\;\;\;\;\;\;\;\;\;\;\;\;\;\;\;\;\;\;\;BC_1=O⇒ C_1=O(B^{-1}\;\mathrm{存在}\;)\\\;\;\;\;\;\;\;\;\;\;\;\;\;\;\;\;\;\;\;\;\;\;\;\;\;\;\;\;\;\;BC_2=E_s⇒ C_2=B^{-1}\;\;\;\;\;\;\\故\;\;\;\;\;\;\;\;\;\;\;\;\;\;\;\;\;\;\;\;\;\;\;\;\;\;\;\;\;\;\;\;\;\;\;\;\begin{pmatrix}O\;\;A\\B\;\;O\end{pmatrix}^{-1}=\;\begin{pmatrix}O\;\;B^{-1}\\A^{-1}\;\;O\end{pmatrix}·\\\end{array}
$$



$$
设A=\begin{pmatrix}A_1\;O\\O\;\;A_2\end{pmatrix},\mathrm{其中}A_1,A_2\mathrm{都是方阵},若A\mathrm{可逆},\mathrm{则下列结论中成立的有}().
$$
$$
A.
仅A_1\mathrm{可逆} \quad B.仅A_2\mathrm{可逆} \quad C.A_1与A_2\mathrm{的可逆性不定} \quad D.A_1与A_2\mathrm{均可逆} \quad E. \quad F. \quad G. \quad H.
$$
$$
\begin{array}{l}A\mathrm{可逆}⇒\vert A\vert=\vert A_1\vert\vert A_2\vert\neq0,即\vert A_1\vert\neq0,且\vert A_2\vert\neq0\\则A_1,A_2\;\mathrm{均可逆}\end{array}
$$



$$
\mathrm{设矩阵}A,B\mathrm{都是}n\mathrm{阶可逆矩阵},则\begin{pmatrix}O\;\;2A\\B\;\;O\end{pmatrix}^{-1}\;=(\;).
$$
$$
A.
\begin{pmatrix}O\;\;2A^{-1}\\B^{-1}\;\;O\end{pmatrix} \quad B.\begin{pmatrix}O\;\;{\textstyle\frac12}A^{-1}\\B^{-1}\;\;O\end{pmatrix} \quad C.\begin{pmatrix}O\;\;B^{-1}\\2A^{-1}\;\;O\end{pmatrix} \quad D.\left(\begin{array}{c}O\;\;B^{-1}\\\frac12A^{-1}\;O\end{array}\;\right) \quad E. \quad F. \quad G. \quad H.
$$
$$
\begin{pmatrix}O\;\;2A\\B\;\;O\end{pmatrix}^{-1}\;=\begin{pmatrix}O\;\;\;\;B^{-1}\\(2A)^{-1}\;\;O\end{pmatrix}=\begin{pmatrix}O\;\;\;\;B^{-1}\\{\textstyle\frac12}A^{-1}\;\;O\end{pmatrix}·
$$



$$
令A\;=\begin{pmatrix}0&0&0&1\\0&0&1/2&0\\0&1/3&0&0\\1/4&0&0&0\end{pmatrix},则A^{-1}\;=()
$$
$$
A.
\begin{pmatrix}0&0&0&4\\0&0&3&0\\0&2&0&0\\1&0&0&0\end{pmatrix} \quad B.\begin{pmatrix}0&0&0&1\\0&0&2&0\\0&3&0&0\\4&0&0&0\end{pmatrix} \quad C.\begin{pmatrix}1&0&0&0\\0&1/2&0&0\\0&0&1/3&0\\0&0&0&1/4\end{pmatrix} \quad D.\begin{pmatrix}0&0&0&1\\0&0&1/2&0\\0&1/3&0&0\\1/4&0&0&0\end{pmatrix} \quad E. \quad F. \quad G. \quad H.
$$
$$
A^{-1}\;=\begin{pmatrix}0&0&0&4\\0&0&3&0\\0&2&0&0\\1&0&0&0\end{pmatrix}
$$



$$
\mathrm{设矩阵}A,B\mathrm{都是}n\mathrm{阶可逆矩阵},则\begin{pmatrix}O&A\\{\textstyle\frac12}B&O\end{pmatrix}^{-1}\;=(\;).
$$
$$
A.
\begin{pmatrix}O&A^{-1}\\2B^{-1}&O\end{pmatrix} \quad B.\begin{pmatrix}O&-A^{-1}\\-2B^{-1}&O\end{pmatrix} \quad C.\begin{pmatrix}O&2B^{-1}\\A^{-1}&O\end{pmatrix} \quad D.\begin{pmatrix}O&-2B^{-1}\\-A^{-1}&O\end{pmatrix} \quad E. \quad F. \quad G. \quad H.
$$
$$
\mathrm{根据分块矩阵的性质}:\begin{pmatrix}O&A\\{\textstyle\frac12}B&O\end{pmatrix}^{-1}\;=\begin{pmatrix}O&2B^{-1}\\\textstyle A^{-1}&O\end{pmatrix}
$$



$$
\mathrm{设矩阵}A,B\mathrm{都是}n\mathrm{阶可逆矩阵},则\begin{pmatrix}A&O\\O&{\textstyle\frac12}B\end{pmatrix}^{-1}\;=(\;).
$$
$$
A.
\begin{pmatrix}2B^{-1}&O\\O&\textstyle A^{-1}\end{pmatrix} \quad B.\begin{pmatrix}A^{-1}&O\\O&\textstyle2B^{-1}\end{pmatrix} \quad C.\begin{pmatrix}O&2B^{-1}\\\textstyle A^{-1}&\textstyle O\end{pmatrix} \quad D.\begin{pmatrix}-2B^{-1}&O\\O&-A^{-1}\end{pmatrix} \quad E. \quad F. \quad G. \quad H.
$$
$$
\mathrm{根据分块矩阵的性质}:\begin{pmatrix}A&O\\O&{\textstyle\frac12}B\end{pmatrix}^{-1}\;=\begin{pmatrix}A^{-1}&O\\O&2B^{-1}\end{pmatrix}
$$



$$
令A=\begin{pmatrix}1&0&0&0\\0&1/2&0&0\\0&0&1/3&0\\0&0&0&1/4\end{pmatrix},则A^{-1}\;=().
$$
$$
A.
\begin{pmatrix}4&0&0&0\\0&3&0&0\\0&0&2&0\\0&0&0&1\end{pmatrix} \quad B.\begin{pmatrix}0&0&0&1\\0&0&2&0\\0&3&0&0\\4&0&0&0\end{pmatrix} \quad C.\begin{pmatrix}1&0&0&0\\0&1/2&0&0\\0&0&1/3&0\\0&0&0&1/4\end{pmatrix} \quad D.\begin{pmatrix}1&0&0&0\\0&2&0&0\\0&0&3&0\\0&0&0&4\end{pmatrix} \quad E. \quad F. \quad G. \quad H.
$$
$$
\mathrm{根据分块矩阵的性质}:\begin{pmatrix}1&0&0&0\\0&1/2&0&0\\0&0&1/3&0\\0&0&0&1/4\end{pmatrix}^{-1}=\begin{pmatrix}1&0&0&0\\0&2&0&0\\0&0&3&0\\0&0&0&4\end{pmatrix}
$$



$$
\mathrm{已知矩阵}A=\begin{pmatrix}1&0&0\\0&\textstyle\frac12&\textstyle\frac32\\0&1&\textstyle\frac52\end{pmatrix},则A^{-1}\;=\;().
$$
$$
A.
\begin{pmatrix}1&0&0\\0&\textstyle-10&\textstyle-6\\0&-4&\textstyle2\end{pmatrix} \quad B.\begin{pmatrix}1&0&0\\0&\textstyle10&\textstyle6\\0&4&\textstyle2\end{pmatrix} \quad C.\begin{pmatrix}1&0&0\\0&\textstyle-10&\textstyle6\\0&4&\textstyle-2\end{pmatrix} \quad D.\begin{pmatrix}1&0&0\\0&\textstyle-10&\textstyle6\\0&4&\textstyle2\end{pmatrix} \quad E. \quad F. \quad G. \quad H.
$$
$$
\mathrm{利用分块矩阵求逆得}:A^{-1}\;=\begin{pmatrix}1&0&0\\0&\textstyle-10&\textstyle6\\0&4&\textstyle-2\end{pmatrix}
$$



$$
\mathrm{设矩阵}A=\begin{pmatrix}1&2&0&0\\0&1&0&0\\0&0&3&2\\0&0&2&1\end{pmatrix},则A^{-1}\;=()
$$
$$
A.
\begin{pmatrix}1&-2&0&0\\0&1&0&0\\0&0&-1&2\\0&0&2&-3\end{pmatrix} \quad B.\begin{pmatrix}1&2&0&0\\0&1&0&0\\0&0&-1&2\\0&0&2&-3\end{pmatrix} \quad C.\begin{pmatrix}1&-2&0&0\\0&1&0&0\\0&0&1&2\\0&0&2&3\end{pmatrix} \quad D.\begin{pmatrix}1&-2&0&0\\0&1&0&0\\0&0&1&-2\\0&0&-2&3\end{pmatrix} \quad E. \quad F. \quad G. \quad H.
$$
$$
A^{-1}\;=\begin{pmatrix}1&-2&0&0\\0&1&0&0\\0&0&-1&2\\0&0&2&-3\end{pmatrix}
$$



$$
\mathrm{设矩阵}A=\begin{pmatrix}3&-4&0&0\\2&-3&0&0\\0&0&-1&0\\0&0&0&-1\end{pmatrix},则A^5\;=().
$$
$$
A.
\begin{pmatrix}3&-4&0&0\\2&-3&0&0\\0&0&-1&0\\0&0&0&-1\end{pmatrix} \quad B.\begin{pmatrix}1&0&0&0\\0&1&0&0\\0&0&-1&0\\0&0&0&-1\end{pmatrix} \quad C.\begin{pmatrix}3&-4&0&0\\2&-3&0&0\\0&0&1&0\\0&0&0&1\end{pmatrix} \quad D.\begin{pmatrix}1&0&0&0\\0&1&0&0\\0&0&1&0\\0&0&0&1\end{pmatrix} \quad E. \quad F. \quad G. \quad H.
$$
$$
\begin{array}{l}B=\begin{pmatrix}3&-4\\2&-3\end{pmatrix},C=\begin{pmatrix}-1&0\\0&-1\end{pmatrix},B^2\;=\begin{pmatrix}1&0\\0&1\end{pmatrix},C^5\;=\begin{pmatrix}-1&0\\0&-1\end{pmatrix},\\A^5\;=\begin{pmatrix}3&-4&0&0\\2&-3&0&0\\0&0&-1&0\\0&0&0&-1\end{pmatrix}\end{array}
$$



$$
\mathrm{设矩阵}A=\begin{pmatrix}1&2&0&0\\0&1&0&0\\0&0&3&3\\0&0&2&3\end{pmatrix},则(\;A^*)^{-1}\;=()
$$
$$
A.
-{\textstyle\frac13}\begin{pmatrix}1&2&0&0\\0&1&0&0\\0&0&3&3\\0&0&2&3\end{pmatrix} \quad B.{\textstyle\frac13}\begin{pmatrix}1&2&0&0\\0&1&0&0\\0&0&3&3\\0&0&2&3\end{pmatrix} \quad C.-{\textstyle\frac16}\begin{pmatrix}1&2&0&0\\0&1&0&0\\0&0&3&3\\0&0&2&3\end{pmatrix} \quad D.\frac16\begin{pmatrix}1&2&0&0\\0&1&0&0\\0&0&3&3\\0&0&2&3\end{pmatrix} \quad E. \quad F. \quad G. \quad H.
$$
$$
(\;A^*)^{-1}=\frac A{\vert A\vert}={\textstyle\frac13}\begin{pmatrix}1&2&0&0\\0&1&0&0\\0&0&3&3\\0&0&2&3\end{pmatrix}
$$



$$
\mathrm{设矩阵}A=\begin{pmatrix}1/2&0&0&0\\0&1&-1&0\\0&-2&3&0\\0&0&0&1/3\end{pmatrix},则A^{-1}\;=().
$$
$$
A.
\begin{pmatrix}2&0&0&0\\0&3&1&0\\0&2&1&0\\0&0&0&3\end{pmatrix} \quad B.\begin{pmatrix}2&0&0&0\\0&-3&1&0\\0&2&-1&0\\0&0&0&3\end{pmatrix} \quad C.\begin{pmatrix}2&0&0&0\\0&3&-1&0\\0&-2&1&0\\0&0&0&3\end{pmatrix} \quad D.\begin{pmatrix}2&0&0&0\\0&3&-1&0\\0&2&-1&0\\0&0&0&3\end{pmatrix} \quad E. \quad F. \quad G. \quad H.
$$
$$
A^{-1}\;=\begin{pmatrix}2&0&0&0\\0&3&1&0\\0&2&1&0\\0&0&0&3\end{pmatrix}
$$



$$
设\;A=\begin{pmatrix}0&0&1&2\\0&0&0&1\\3&3&0&0\\2&1&0&0\end{pmatrix},则\;(A^*)^{-1}=(\;).
$$
$$
A.
-\frac13\begin{pmatrix}0&0&1&2\\0&0&0&1\\3&3&0&0\\2&1&0&0\end{pmatrix} \quad B.\frac13\begin{pmatrix}0&0&1&2\\0&0&0&1\\3&3&0&0\\2&1&0&0\end{pmatrix} \quad C.-\frac16\begin{pmatrix}0&0&1&2\\0&0&0&1\\3&3&0&0\\2&1&0&0\end{pmatrix} \quad D.\frac16\begin{pmatrix}0&0&1&2\\0&0&0&1\\3&3&0&0\\2&1&0&0\end{pmatrix} \quad E. \quad F. \quad G. \quad H.
$$
$$
(A^*)^{-1}\;=\frac A{\vert A\vert}\;=-\frac13\begin{pmatrix}0&0&1&2\\0&0&0&1\\3&3&0&0\\2&1&0&0\end{pmatrix}
$$



$$
\mathrm{设矩阵}A=\begin{pmatrix}1&2&0\\1&3&0\\0&0&2\end{pmatrix},则\;A^{-1}\;=\;(\;).
$$
$$
A.
\begin{pmatrix}1&2&0\\1&3&0\\0&0&-2\end{pmatrix} \quad B.\begin{pmatrix}3&-2&0\\-1&1&0\\0&0&-\frac12\end{pmatrix} \quad C.\begin{pmatrix}-3&2&0\\1&-1&0\\0&0&\frac12\end{pmatrix} \quad D.\begin{pmatrix}3&-2&0\\-1&1&0\\0&0&\frac12\end{pmatrix} \quad E. \quad F. \quad G. \quad H.
$$
$$
A^{-1}=\begin{pmatrix}3&-2&0\\-1&1&0\\0&0&\frac12\end{pmatrix}
$$



$$
\mathrm{设矩阵}A\;=\begin{pmatrix}5&2&0&0\\2&1&0&0\\0&0&2&1\\0&0&1&1\end{pmatrix},则A^{-1}\;=(\;).
$$
$$
A.
\begin{pmatrix}1&2&0&0\\2&5&0&0\\0&0&1&-1\\0&0&-1&2\end{pmatrix} \quad B.\begin{pmatrix}5&-2&0&0\\-2&1&0&0\\0&0&1&-1\\0&0&-1&2\end{pmatrix} \quad C.\begin{pmatrix}1&-2&0&0\\-2&5&0&0\\0&0&1&1\\0&0&1&2\end{pmatrix} \quad D.\begin{pmatrix}1&-2&0&0\\-2&5&0&0\\0&0&1&-1\\0&0&-1&2\end{pmatrix} \quad E. \quad F. \quad G. \quad H.
$$
$$
B=\begin{pmatrix}5&2\\1&1\end{pmatrix},C=\begin{pmatrix}2&1\\1&1\end{pmatrix},A^{-1}=\begin{pmatrix}B^{-1}&O\\O&C^{-1}\end{pmatrix}=\begin{pmatrix}1&-2&0&0\\-2&5&0&0\\0&0&1&-1\\0&0&-1&2\end{pmatrix}
$$



$$
设A=\begin{pmatrix}1&0&0&0\\0&1&0&0\\0&0&3&-4\\0&0&2&-3\end{pmatrix},则\;\vert A^{16}\vert\;=(\;).
$$
$$
A.
-1 \quad B.1 \quad C.4 \quad D.-4 \quad E. \quad F. \quad G. \quad H.
$$
$$
\begin{array}{l}令B=\begin{pmatrix}1&0\\0&1\end{pmatrix},C=\begin{pmatrix}3&-4\\2&-3\end{pmatrix}\\\vert B\vert=1,\vert C\vert=-1.\\\vert A^{16}\vert=\vert B\vert^{16}\left|C\right|^{16}\;=1.\end{array}
$$



$$
设A=\begin{pmatrix}3&-4&0&0\\2&-3&0&0\\0&0&-1&1\\0&0&0&-1\end{pmatrix},则A^{13}\;=(\;).
$$
$$
A.
\begin{pmatrix}3&-4&0&0\\2&-3&0&0\\0&0&-1&13\\0&0&0&-1\end{pmatrix} \quad B.\begin{pmatrix}3&-4&0&0\\2&-3&0&0\\0&0&-1&-13\\0&0&0&-1\end{pmatrix} \quad C.\begin{pmatrix}3&-4&0&0\\2&-3&0&0\\0&0&-1&1\\0&0&0&-1\end{pmatrix} \quad D.\begin{pmatrix}3&-4&0&0\\2&-3&0&0\\0&0&-1&-1\\0&0&0&-1\end{pmatrix} \quad E. \quad F. \quad G. \quad H.
$$
$$
\begin{array}{l}令B=\begin{pmatrix}3&-4\\2&-3\end{pmatrix},C=\begin{pmatrix}-1&1\\0&-1\end{pmatrix},\\A^{13}\;=\;\begin{pmatrix}B&O\\O&C\end{pmatrix}^{13}\;=\begin{pmatrix}B^{13}&O\\O&C^{13}\end{pmatrix},\\B^{13}\;=B,C^{13}\;=\begin{pmatrix}-1&\;13\\0&-1\end{pmatrix},A^{13}\;=\begin{pmatrix}3&-4&0&0\\2&-3&0&0\\0&0&-1&13\\0&0&0&-1\end{pmatrix}.\end{array}
$$



$$
\mathrm{设矩阵}A=\begin{pmatrix}3&-2&0&0\\4&-3&0&0\\0&0&2&0\\0&0&2&1\end{pmatrix},则\vert A^4\vert\;=(\;).
$$
$$
A.
-16 \quad B.16 \quad C.-8 \quad D.8 \quad E. \quad F. \quad G. \quad H.
$$
$$
\begin{array}{l}\mathrm{按如图方法将原矩阵分块}A=\begin{pmatrix}A_1&O\\O&A_2\end{pmatrix},则\\\;\;\;\;\;\;\;\;\;\;\;\;\;\;\;\;\;\;\;\;\;\;\;\;\;\;\;\;\;\;\;\;\;\;\;\;\;\;\;\;\;\;\;\;\;\;\;\;\;\;\;\;\;\;\;\;\;\;\;\;\;\;\;\;\;\;\;\;\;\;\;\;\;\;\;\;\vert A^4\vert=\vert A_1\vert^4\vert A_2\vert^4\;=16.\end{array}
$$



$$
\mathrm{设矩阵}A=\begin{pmatrix}1&0&0&0\\1&1&0&0\\0&0&1&2\\0&0&1&3\end{pmatrix},则A^{-1}\;=(\;).
$$
$$
A.
\begin{pmatrix}1&0&0&0\\-1&1&0&0\\0&0&1&-2\\0&0&-1&3\end{pmatrix} \quad B.\begin{pmatrix}1&0&0&0\\-1&1&0&0\\0&0&3&2\\0&0&1&1\end{pmatrix} \quad C.\begin{pmatrix}1&0&0&0\\1&1&0&0\\0&0&3&-2\\0&0&-1&1\end{pmatrix} \quad D.\begin{pmatrix}1&0&0&0\\-1&1&0&0\\0&0&3&-2\\0&0&-1&1\end{pmatrix} \quad E. \quad F. \quad G. \quad H.
$$
$$
\begin{array}{l}A=\begin{pmatrix}1&0&0&0\\1&1&0&0\\0&0&1&2\\0&0&1&3\end{pmatrix},\\\;\;\;A\begin{pmatrix}A_1&O\\O&A_2\end{pmatrix},\\\;\;\;\;\;\;A_1\;=\begin{pmatrix}1&0\\1&1\end{pmatrix},A_2=\begin{pmatrix}1&2\\1&3\end{pmatrix},;\\\;\;\;\;\;\;\;\;\;A^{-1}\;=\begin{pmatrix}A_1^{-1}&O\\O&A_2^{-1}\end{pmatrix}=\begin{pmatrix}1&0&0&0\\-1&1&0&0\\0&0&3&-2\\0&0&-1&1\end{pmatrix}.\\\end{array}
$$



$$
\mathrm{设矩阵}A=\frac12\begin{pmatrix}1&0&0\\0&1&3\\0&2&5\end{pmatrix},则A^{-1}\;=(\;).
$$
$$
A.
\begin{pmatrix}2&0&0\\0&-5&6\\0&4&-2\end{pmatrix} \quad B.\begin{pmatrix}2&0&0\\0&10&-6\\0&-4&2\end{pmatrix} \quad C.\begin{pmatrix}2&0&0\\0&-10&6\\0&4&-2\end{pmatrix} \quad D.\begin{pmatrix}1&0&0\\0&-10&6\\0&4&-2\end{pmatrix} \quad E. \quad F. \quad G. \quad H.
$$
$$
A=\frac12\begin{pmatrix}1&0&0\\0&1&3\\0&2&5\end{pmatrix},A^{-1}\;=2\begin{pmatrix}1&0&0\\0&-5&3\\0&2&-1\end{pmatrix}=\begin{pmatrix}2&0&0\\0&-10&6\\0&4&-2\end{pmatrix}
$$



$$
\mathrm{设矩阵}A=\frac12\begin{pmatrix}1&0&0\\0&1&3\\0&2&5\end{pmatrix},则A^{-1}\;=(\;).
$$
$$
A.
\begin{pmatrix}2&0&0\\0&-5&6\\0&4&-2\end{pmatrix} \quad B.\begin{pmatrix}2&0&0\\0&10&-6\\0&-4&2\end{pmatrix} \quad C.\begin{pmatrix}2&0&0\\0&-10&6\\0&4&-2\end{pmatrix} \quad D.\begin{pmatrix}1&0&0\\0&-10&6\\0&4&-2\end{pmatrix} \quad E. \quad F. \quad G. \quad H.
$$
$$
A=\frac12\begin{pmatrix}1&0&0\\0&1&3\\0&2&5\end{pmatrix},A^{-1\;}=2\begin{pmatrix}1&0&0\\0&-5&3\\0&2&-1\end{pmatrix}=\begin{pmatrix}2&0&0\\0&-10&6\\0&4&-2\end{pmatrix}
$$



$$
\mathrm{设矩阵}A=\frac12\begin{pmatrix}1&0&0\\0&2&1\\0&3&2\end{pmatrix},则A^{-1\;}=().
$$
$$
A.
\begin{pmatrix}2&0&0\\0&4&-2\\0&-6&4\end{pmatrix} \quad B.\begin{pmatrix}2&0&0\\0&-4&-2\\0&-6&4\end{pmatrix} \quad C.\begin{pmatrix}2&0&0\\0&4&2\\0&6&4\end{pmatrix} \quad D.\begin{pmatrix}2&0&0\\0&-4&2\\0&6&-4\end{pmatrix} \quad E. \quad F. \quad G. \quad H.
$$
$$
A^{-1}\;=\begin{pmatrix}2&0&0\\0&4&-2\\0&-6&4\end{pmatrix}
$$



$$
\mathrm{设矩阵}A=\begin{pmatrix}3&-4&0&0\\2&-3&0&0\\0&0&-1&0\\0&0&0&-1\end{pmatrix},则A^4\;=(\;).
$$
$$
A.
\begin{pmatrix}1&0&0&0\\0&1&0&0\\0&0&-1&0\\0&0&0&-1\end{pmatrix} \quad B.\begin{pmatrix}1&0&0&0\\0&1&0&0\\0&0&1&0\\0&0&0&1\end{pmatrix} \quad C.\begin{pmatrix}3&-4&0&0\\2&-3&0&0\\0&0&1&0\\0&0&0&1\end{pmatrix} \quad D.\begin{pmatrix}3&-4&0&0\\2&-3&0&0\\0&0&-1&0\\0&0&0&-1\end{pmatrix} \quad E. \quad F. \quad G. \quad H.
$$
$$
\begin{array}{l}令B=\begin{pmatrix}3&-4\\2&-3\end{pmatrix},C=\begin{pmatrix}-1&0\\0&-1\end{pmatrix},\\A^4=\begin{pmatrix}B&O\\O&C\end{pmatrix}^4\;=\begin{pmatrix}B^4&O\\O&C^4\end{pmatrix},\\B^4=E,C^4=E,A^4\;=\begin{pmatrix}1&0&0&0\\0&1&0&0\\0&0&1&0\\0&0&0&1\end{pmatrix}\end{array}
$$



$$
设3\mathrm{阶方阵}A=d\begin{pmatrix}a&0&0\\0&0&b\\0&c&0\end{pmatrix},\mathrm{其中}abcd\neq0,则A^{-1}=\;().
$$
$$
A.
\frac1d\begin{pmatrix}\frac1a&0&0\\0&0&\frac1c\\0&\frac1b&0\end{pmatrix} \quad B.\frac1d\begin{pmatrix}\frac1a&0&0\\0&0&\frac1b\\0&\frac1c&0\end{pmatrix} \quad C.d\begin{pmatrix}\frac1a&0&0\\0&0&\frac1c\\0&\frac1b&0\end{pmatrix} \quad D.d\begin{pmatrix}\frac1a&0&0\\0&0&\frac1b\\0&\frac1c&0\end{pmatrix} \quad E. \quad F. \quad G. \quad H.
$$
$$
\begin{array}{l}\mathrm{将矩阵}A\mathrm{分块为}A=d\begin{pmatrix}A_1&\\&A_2\end{pmatrix},\mathrm{其中}A_1=(a),A_2=\begin{pmatrix}0&b\\c&0\end{pmatrix},\mathrm{运用分块矩阵的性质可得}\\A^{-1}\;=\;d^{-1}\;\begin{pmatrix}A_1^{-1}&\\&A_2^{-1}\end{pmatrix}=\frac1d\begin{pmatrix}\frac1a&0&0\\0&0&\frac1c\\0&\frac1b&0\end{pmatrix}\end{array}
$$



$$
设A,B\mathrm{均为}n\mathrm{阶可逆阵},且\vert A\vert=2,\vert B\vert=2,则\left|-2\begin{pmatrix}A^T&O\\O&B^{-1}\end{pmatrix}\right|=(\;).
$$
$$
A.
(-2)^n \quad B.-2 \quad C.2 \quad D.(-2)^{2n} \quad E. \quad F. \quad G. \quad H.
$$
$$
\begin{array}{l}\mathrm{因为}A,B\mathrm{都是}n\mathrm{阶可逆阵},\mathrm{所以}\\\left|-2\begin{pmatrix}A^T\;&O\\O&B^{-1}\end{pmatrix}\right|=(-2)^{2n}\begin{vmatrix}A^T\;&O\\O&B^{-1}\end{vmatrix}\;=(-2)^{2n}\vert A\vert\vert B\vert^{-1}\;=(-2)^{2n}.\end{array}
$$



$$
设A=\begin{pmatrix}0&0&0&1\\0&0&1&0\\1&2&0&0\\3&4&0&0\end{pmatrix},则\vert A^5\vert=(\;).
$$
$$
A.
32 \quad B.-32 \quad C.\frac1{32} \quad D.-\frac1{32} \quad E. \quad F. \quad G. \quad H.
$$
$$
\begin{array}{l}\vert A\vert=\begin{vmatrix}0&0&0&1\\0&0&1&0\\1&2&0&0\\3&4&0&0\end{vmatrix}=2,\\\vert A^5\vert=\vert A\vert^5=32.\end{array}
$$



$$
\begin{array}{l}设A=\begin{pmatrix}5&0&0\\0&3&1\\0&2&1\end{pmatrix},则A^{-1}\;=().\\\end{array}
$$
$$
A.
\begin{pmatrix}1/5&0&0\\0&1&-1\\0&-2&3\end{pmatrix} \quad B.\begin{pmatrix}1/5&0&0\\0&3&-1\\0&-2&1\end{pmatrix} \quad C.\begin{pmatrix}1/5&0&0\\0&1&-2\\0&-1&3\end{pmatrix} \quad D.\begin{pmatrix}1/5&0&0\\0&1&-1\\0&2&3\end{pmatrix} \quad E. \quad F. \quad G. \quad H.
$$
$$
\begin{array}{l}A=\begin{pmatrix}5&0&0\\0&3&1\\0&2&1\end{pmatrix}=\begin{pmatrix}A_1&O\\O&A_2\end{pmatrix},\\\;\;\;\;\;A_1=(5),A_2=\begin{pmatrix}3&1\\2&1\end{pmatrix},A_1^{-1}=\left(\frac15\right);A_2^{-1}=\begin{pmatrix}1&-1\\-2&3\end{pmatrix};\\\;\;\;\;\;\;A^{-1}=\begin{pmatrix}A_1^{-1}&O\\O&A_2^{-1}\end{pmatrix}=\begin{pmatrix}1/5&0&0\\0&1&-1\\0&-2&3\end{pmatrix}.\\\;\;\;\\\end{array}
$$



$$
\mathrm{矩阵}\begin{pmatrix}0&0&2\\1&2&0\\3&4&0\end{pmatrix}\mathrm{的逆矩阵}A^{-1}\;=(\;).
$$
$$
A.
\begin{pmatrix}0&-2&1\\0&3/2&-1/2\\1/2&0&0\end{pmatrix} \quad B.\begin{pmatrix}0&2&-1\\0&-3/2&1/2\\1/2&0&0\end{pmatrix} \quad C.\begin{pmatrix}0&0&1/2\\-2&3/2&0\\1&-1/2&0\end{pmatrix} \quad D.\begin{pmatrix}0&0&1/2\\2&-3/2&0\\-1&1/2&0\end{pmatrix} \quad E. \quad F. \quad G. \quad H.
$$
$$
\begin{array}{l}\mathrm{对原矩阵作分块}\;A=\begin{pmatrix}O&A_1\\A_2&O\end{pmatrix},\mathrm{其中}\\\;\;\;\;\;\;\;\;\;\;\;\;\;\;\;\;\;\;\;\;\;\;\;\;\;\;\;\;\;A_1=(2),\;\;\;\;\;\;\;\;\;\;\;\;\;\;\;\;A_1^{-1}\;=\left(\frac12\right)\;;\\\;\;\;\;\;\;\;\;\;\;\;\;\;\;\;\;\;\;\;\;\;\;\;\;\;\;\;\;\;A_2=\begin{pmatrix}1&2\\3&4\end{pmatrix},\;\;\;\;\;\;\;\;\;\;\;\;\;\;\;\;A_2^{-1}\;=\begin{pmatrix}-2&1\\\frac32&-\frac12\end{pmatrix}\;,\\故\;\;\;\;\;\;\;\;\;\;\;\;\;\;\;\;\;\;\;\;\;\;\;\;\;A^{-1}=\begin{pmatrix}O&A_2^{-1}\\A_1^{-1}&O\end{pmatrix}=\begin{pmatrix}0&-2&1\\0&3/2&-1/2\\1/2&0&0\end{pmatrix}.\end{array}
$$



$$
\begin{array}{l}设A=\begin{pmatrix}3&4&0&0\\2&3&0&0\\0&0&2&0\\0&0&8&1\end{pmatrix},则\vert A^5\vert=().\\\end{array}
$$
$$
A.
32 \quad B.-32 \quad C.16 \quad D.-16 \quad E. \quad F. \quad G. \quad H.
$$
$$
\begin{array}{l}\mathrm{按如图方法将原矩阵分块}\;A=\begin{pmatrix}A_1&O\\O&A_2\end{pmatrix},则\\\;\;\;\;\;\;\;\;\;\;\;\;\;\;\;\;\;\;\;\;\;\;\;\;\;\;\;\;\;\;\;\;\;\;\;\;\;\;\;\;\;\;\;\;\;A^5=\begin{pmatrix}A_1&O\\O&A_2\end{pmatrix}^5\;=\begin{pmatrix}A_1^5&O\\O&A_2^5\end{pmatrix},\\\;\;\;\;\;\;\;\;\;\;\;\;\;\;\;\;\;\;\;\;\;\;\;\;\;\;\;\;\;\;\;\;\;\;\;\;\;\;\;\;\;\;\;\;\;\;\vert A^5\vert=\vert A_1^5\vert\vert A_2^5\vert=\vert A_1\vert^5\vert A_2\vert^5=32.\end{array}
$$



$$
设A=\frac12\begin{pmatrix}0&0&2\\1&3&0\\2&5&0\end{pmatrix},则A^{-1}\;=().
$$
$$
A.
\begin{pmatrix}0&-10&-6\\0&4&-2\\1&0&0\end{pmatrix} \quad B.\begin{pmatrix}0&10&6\\0&4&2\\2&0&0\end{pmatrix} \quad C.\begin{pmatrix}0&10&-6\\0&-4&2\\1&0&0\end{pmatrix} \quad D.\begin{pmatrix}0&-10&6\\0&4&-2\\1&0&0\end{pmatrix} \quad E. \quad F. \quad G. \quad H.
$$
$$
\begin{array}{l}\mathrm{利用矩阵分块法},\\\;\;\;\;\;\;\;\;\;\;\;\;\;\;\;\;\;\;\;\;\;\;\;\;\;\;\;\;\;\;\;\begin{pmatrix}0&0&2\\1&3&0\\2&5&0\end{pmatrix}=\begin{pmatrix}O&B\\C&O\end{pmatrix},\\\mathrm{而当}B,C\mathrm{均可逆时},有\\\;\;\;\;\;\;\;\;\;\;\;\;\;\;\;\;\;\;\;\;\;\;\;\;\;\;\;\;\;\;\begin{pmatrix}O&B\\C&O\end{pmatrix}^{-1}=\begin{pmatrix}O&C^{-1}\\B^{-1}&O\end{pmatrix},\\又\;\;\;B^{-1}=\begin{pmatrix}\frac12\end{pmatrix},C^{-1}=\begin{pmatrix}-5&3\\2&-1\end{pmatrix},\\\mathrm{再利用公式}(kA)^{-1}=\frac1kA^{-1},\mathrm{易见选}\begin{pmatrix}0&-10&6\\0&4&-2\\1&0&0\end{pmatrix}\end{array}
$$



$$
设A,B\mathrm{都是}n\mathrm{阶可逆阵},\vert A\vert=2,\vert B\vert=2,则\begin{vmatrix}(-3)\begin{pmatrix}A^T&O\\O&B^{-1}\end{pmatrix}\end{vmatrix}=(\;).
$$
$$
A.
(-3)^n \quad B.-3 \quad C.3 \quad D.(-3)^{2n} \quad E. \quad F. \quad G. \quad H.
$$
$$
\begin{vmatrix}(-3)\begin{pmatrix}A^T&O\\O&B^{-1}\end{pmatrix}\end{vmatrix}=(-3)^{2n}\begin{vmatrix}A^T&O\\O&B^{-1}\end{vmatrix}=(-3)^{2n}\vert A\vert\vert B\vert^{-1}=(-3)^{2n}
$$



$$
\mathrm{矩阵}\begin{pmatrix}0&a_1&0&⋯&0\\0&0&a_2&⋯&0\\⋯&⋯&⋯&&⋯\\0&0&0&⋯&a_{n-1}\\a_n&0&0&⋯&0\end{pmatrix}\left(a_i\neq0,i=1,2,⋯,n\right)\mathrm{的逆矩阵}\left(\;\;\;\;\;\right)
$$
$$
A.
\begin{pmatrix}0&0&0&⋯&0&0&\frac1{a_n}\\\frac1{a_1}&0&0&⋯&0&0&0\\0&\frac1{a_2}&0&⋯&0&0&0\\⋯⋯&⋯&⋯&⋯&⋯&⋯&⋯\\0&0&0&⋯&\frac1{a_{n-2}}&0&0\\0&0&0&⋯&0&\frac1{a_{n-1}}&0\end{pmatrix} \quad B.\begin{pmatrix}0&0&0&⋯&0&0&\frac1{a_1}\\\frac1{a_2}&0&0&⋯&0&0&0\\0&\frac1{a_3}&0&⋯&0&0&0\\⋯⋯&⋯&⋯&⋯&⋯&⋯&⋯\\0&0&0&⋯&\frac1{a_{n-1}}&0&0\\0&0&0&⋯&0&\frac1{a_n}&0\end{pmatrix} \quad C.\begin{pmatrix}0&0&0&⋯&0&0&\frac1{a_2}\\\frac1{a_1}&0&0&⋯&0&0&0\\0&\frac1{a_3}&0&⋯&0&0&0\\⋯⋯&⋯&⋯&⋯&⋯&⋯&⋯\\0&0&0&⋯&\frac1{a_{n-1}}&0&0\\0&0&0&⋯&0&\frac1{a_n}&0\end{pmatrix} \quad D.\begin{pmatrix}0&0&0&⋯&0&0&-\frac1{a_1}\\-\frac1{a_2}&0&0&⋯&0&0&0\\0&-\frac1{a_3}&0&⋯&0&0&0\\⋯⋯&⋯&⋯&⋯&⋯&⋯&⋯\\0&0&0&⋯&-\frac1{a_{n-1}}&0&0\\0&0&0&⋯&0&-\frac1{a_n}&0\end{pmatrix} \quad E. \quad F. \quad G. \quad H.
$$
$$
\begin{array}{l}\rightarrow\begin{pmatrix}a_n&0&0&⋯&0&0&0&0&0&0&⋯&0&0&1\\0&a_1&0&⋯&0&0&0&1&0&0&⋯&0&0&0\\0&0&a_2&⋯&0&0&0&0&1&0&⋯&0&0&0\\⋯&⋯&⋯&⋯&⋯&⋯&⋯&⋯&⋯&⋯&⋯&⋯&⋯&⋯\\0&0&0&⋯&0&a_{n-2}&0&0&0&0&⋯&1&0&0\\0&0&0&⋯&0&0&a_{n-1}&0&0&0&⋯&0&1&0\end{pmatrix}\\\rightarrow\begin{pmatrix}1&0&0&⋯&0&0&0&0&0&0&⋯&0&0&1/a_n\\0&1&0&⋯&0&0&0&1/a_1&0&0&⋯&0&0&0\\0&0&1&⋯&0&0&0&0&1/a_2&0&⋯&0&0&0\\⋯&⋯&⋯&⋯&⋯&⋯&⋯&⋯&⋯&⋯&⋯&⋯&⋯&⋯\\0&0&0&⋯&0&1&0&0&0&0&⋯&1/a_{n-2}&0&0\\0&0&0&⋯&0&0&1&0&0&0&⋯&0&1/a_{n-1}&0\end{pmatrix}\\\mathrm{故所求逆矩阵为}\\\begin{pmatrix}0&0&0&⋯&0&0&\frac1{a_n}\\\frac1{a_1}&0&0&⋯&0&0&0\\0&\frac1{a_2}&0&⋯&0&0&0\\⋯⋯&⋯&⋯&⋯&⋯&⋯&⋯\\0&0&0&⋯&\frac1{a_{n-2}}&0&0\\0&0&0&⋯&0&\frac1{a_{n-1}}&0\end{pmatrix}\end{array}
$$



$$
\mathrm{设三阶方阵}A\mathrm{的列分块矩阵为}A=\begin{pmatrix}A_1&A_2&A_3\end{pmatrix},若A_3=2A_1+3A_2,则\vert A\vert=().
$$
$$
A.
1 \quad B.0 \quad C.6 \quad D.5 \quad E. \quad F. \quad G. \quad H.
$$
$$
\begin{array}{l}\mathrm{根据行列式的运算性质},得\\\vert A\vert=\vert A_1,\;\;\;\;A_2,\;\;\;\;A_3\vert=\vert A_1,\;\;\;\;A_2,\;\;\;\;2A_1+3A_2\vert=\vert A_1,\;\;\;\;A_2,\;\;\;2A_1\vert+\vert A_1,\;\;\;\;A_2,\;\;\;\;3A_2\vert=0\end{array}
$$



$$
若3\mathrm{阶矩阵}A\mathrm{按列分块为}A=(A_1,\;\;\;\;A_2,\;\;\;\;A_3),\mathrm{其中}\vert A\vert\;=\;-2,则\vert A_3-2A_1,3A_2,A_1\vert=().
$$
$$
A.
6 \quad B.-6 \quad C.18 \quad D.-18 \quad E. \quad F. \quad G. \quad H.
$$
$$
\begin{array}{l}\vert A_3-2A_1,3A_2,A_1\vert=3\vert A_3,A_2,\;A_1\vert-6\vert A_1,A_2,A_1\vert\\\;\;\;\;\;\;\;\;\;\;\;\;\;\;\;\;\;\;\;\;\;\;\;\;\;\;\;\;\;=-3\vert A_1,A_2,A_3\vert-6×0\\\;\;\;\;\;\;\;\;\;\;\;\;\;\;\;\;\;\;\;\;\;\;\;\;\;\;\;\;\;=-3×(-2)\;=6.\end{array}
$$



$$
设3\mathrm{阶方阵}A\mathrm{的列分块矩阵为}A=(A_1,\;\;\;\;A_2,\;\;\;\;A_3),\mathrm{是数},若A_3=2A_1-3A_2,则\vert A\vert=().
$$
$$
A.
-3 \quad B.0 \quad C.1 \quad D.-1 \quad E. \quad F. \quad G. \quad H.
$$
$$
\vert A\vert\;=\vert A_1\;\;A_{2\;\;}2A_1-3A_2\vert=\vert A_{1\;\;}A_2\;\;2A_1\vert+\vert A_1\;\;A_2\;-3A_2\vert\;=0
$$



$$
设A=\begin{pmatrix}A_{11}\;\;\;A_{12}\\\;A_{21\;\;}\;A_{22}\end{pmatrix}\;\mathrm{为分块矩阵},则A^T=().
$$
$$
A.
\begin{pmatrix}A_{11}^T\;A_{12}^T\\A_{21}^T\;A_{22}^T\end{pmatrix} \quad B.\begin{pmatrix}A_{11}^T\;A_{21}^T\\A_{12}^T\;A_{22}^T\end{pmatrix} \quad C.\begin{pmatrix}A_{12}^T\;A_{11}^T\\A_{22}^T\;A_{21}^T\end{pmatrix} \quad D.\begin{pmatrix}A_{21}^T\;A_{22}^T\\A_{11}^T\;A_{12}^T\end{pmatrix} \quad E. \quad F. \quad G. \quad H.
$$
$$
\mathrm{由分块矩阵转置的公式可知}A^T=\begin{pmatrix}A_{11}^T\;A_{21}^T\\A_{12}^T\;A_{22}^T\end{pmatrix}
$$



$$
若3\mathrm{阶矩阵}A\mathrm{按列分块为}A=(A_1,A_2,A_3),\mathrm{其中}\left|A\right|=-2,则\left|A_1,2A_2,A_3\right|=().
$$
$$
A.
4 \quad B.-4 \quad C.2 \quad D.-2 \quad E. \quad F. \quad G. \quad H.
$$
$$
\vert A_1,2A_2,A_3\vert=2\left|A_1,A_2,A_3\right|=2×(-2)\;=-4;
$$



$$
\mathrm{设三阶方阵}A\mathrm{的列分块矩阵为}A=(A_1,A_2,A_3),a,b,\mathrm{是数},若A_3=aA_1+bA_2,则\vert A\vert=().
$$
$$
A.
0 \quad B.1 \quad C.a^3b^3 \quad D.ab \quad E. \quad F. \quad G. \quad H.
$$
$$
\begin{array}{l}\mathrm{根据行列式的运算性质},得\\\;\;\;\;\;\;\;\;\;\;\;\;\;\;\;\;\;\;\;\;\;\;\;\;\;\;\;\;\;\;\;\;\;\;\;A=\vert A_1\;A_2\;\;A_3\vert\;=\vert A_1\;\;\;\;A_2\;\;\;aA_1+bA_2\vert=\vert A_1\;\;\;A_2\;\;aA_1\vert+\vert A_1\;\;\;A_2\;\;bA_2\vert=0\end{array}
$$



$$
\mathrm{矩阵}A=\begin{pmatrix}2&0&\;0\\0&5&5\\0&1&2\end{pmatrix},则\left(E-A\right)^{-1}=(\;\;).
$$
$$
A.
\begin{pmatrix}-1&0&\;0\\0&1&-5\\0&-1&-4\end{pmatrix} \quad B.\begin{pmatrix}-1&0&\;0\\0&1&-5\\0&1&4\end{pmatrix} \quad C.\begin{pmatrix}-1&0&\;0\\0&1&-5\\0&-1&4\end{pmatrix} \quad D.\begin{pmatrix}1&0&\;0\\0&1&-5\\0&-1&4\end{pmatrix} \quad E. \quad F. \quad G. \quad H.
$$
$$
\begin{array}{l}E-A=\begin{pmatrix}-1&0&\;0\\0&-4&-5\\0&-1&-1\end{pmatrix},令A_1=\left[-1\right],\;\;A_2=\begin{bmatrix}-4&-5\\-1&-1\end{bmatrix},则\begin{bmatrix}-4&-5\\-1&-1\end{bmatrix}^{-1}=\begin{bmatrix}1&-5\\-1&4\end{bmatrix}^{}\\\mathrm{所以}:\begin{pmatrix}-1&0&\;0\\0&-4&-5\\0&-1&-1\end{pmatrix}^{-1}=\begin{pmatrix}-1&0&\;0\\0&1&-5\\0&-1&4\end{pmatrix}\end{array}
$$



$$
\mathrm{设矩阵}A=\begin{bmatrix}2&1&0&0&0&0\\0&0&1&2&1&2\\5&3&0&0&0&0\\0&0&2&3&2&3\\0&0&4&1&4&1\\0&0&1&1&1&1\end{bmatrix},\;\;B=\begin{bmatrix}3&0&2&0&0&0\\-5&0&-3&0&0&0\\0&2&0&3&1&-2\\0&-1&0&1&1&-3\\0&-2&0&-3&-1&2\\0&1&0&-1&-1&3\end{bmatrix},C=E_{23}\mathrm{是阶初等矩阵},则(CABC)^n=(\;).
$$
$$
A.
\begin{bmatrix}1&1&0&0&0&0\\0&1&0&0&0&0\\0&0&0&0&0&0\\0&0&0&0&0&0\\0&0&0&0&0&0\\0&0&0&0&0&0\end{bmatrix} \quad B.\begin{bmatrix}1&0&0&0&0&0\\0&1&0&0&0&0\\0&0&0&0&0&0\\0&0&0&0&0&0\\0&0&0&0&0&0\\0&0&0&0&0&0\end{bmatrix} \quad C.\begin{bmatrix}1&n&0&0&0&0\\0&1&0&0&0&0\\0&0&0&0&0&0\\0&0&0&0&0&0\\0&0&0&0&0&0\\0&0&0&0&0&0\end{bmatrix} \quad D.\begin{bmatrix}1&n-1&0&0&0&0\\0&1&0&0&0&0\\0&0&0&0&0&0\\0&0&0&0&0&0\\0&0&0&0&0&0\\0&0&0&0&0&0\end{bmatrix} \quad E. \quad F. \quad G. \quad H.
$$
$$
\begin{array}{l}\begin{array}{l}CA=\begin{bmatrix}2&1&0&0&0&0\\5&3&0&0&0&0\\0&0&1&2&1&2\\0&0&2&3&2&3\\0&0&4&1&4&1\\0&0&1&1&1&1\end{bmatrix}=\begin{pmatrix}A_1&O\\O&A_2\end{pmatrix},\;BC=\begin{bmatrix}3&2&0&0&0&0\\-5&-3&0&0&0&0\\0&0&2&3&1&-2\\0&0&-1&1&1&-3\\0&0&-2&-3&-1&2\\0&0&1&-1&-1&3\end{bmatrix}=\begin{pmatrix}B_1&O\\O&B_2\end{pmatrix}\\(CABC)^n=\begin{bmatrix}A_1B_1&O\\O&A_2B_2\end{bmatrix}^n=\begin{bmatrix}1&n&0&0&0&0\\0&1&0&0&0&0\\0&0&0&0&0&0\\0&0&0&0&0&0\\0&0&0&0&0&0\\0&0&0&0&0&0\end{bmatrix}\\\\\;\;\;\;\;\;\;\;\;\\\end{array}\\\\\end{array}
$$



$$
设X\begin{pmatrix}1&0&0\\0&0&1\\0&1&0\end{pmatrix}=\begin{pmatrix}-2&1&1\\3&1&1\\2&8&5\end{pmatrix},则X=\left(\;\;\;\right)
$$
$$
A.
\begin{pmatrix}-2&1&1\\3&1&1\\2&5&8\end{pmatrix} \quad B.\begin{pmatrix}-2&1&1\\2&8&5\\3&1&1\end{pmatrix} \quad C.\begin{pmatrix}2&5&8\\3&1&1\\-2&1&1\end{pmatrix} \quad D.\begin{pmatrix}1&1&-2\\1&1&3\\8&5&2\end{pmatrix} \quad E. \quad F. \quad G. \quad H.
$$
$$
\mathrm{根据初等变换和初等矩阵的有关内容},\mathrm{用初等矩阵右乘矩阵},\mathrm{相当于对该矩阵进行初等列变换}.
$$



$$
\mathrm{已知}\begin{pmatrix}0&1&0\\1&0&0\\0&0&1\end{pmatrix}X\begin{pmatrix}1&0&0\\0&0&1\\0&1&0\end{pmatrix}=\begin{pmatrix}1&-4&3\\2&0&-1\\1&-2&0\end{pmatrix},则\;X=\left(\;\;\;\right).
$$
$$
A.
\begin{pmatrix}2&-1&0\\1&3&4\\1&0&2\end{pmatrix} \quad B.\begin{pmatrix}2&1&0\\1&3&-4\\1&0&-2\end{pmatrix} \quad C.\begin{pmatrix}2&1&0\\-1&3&-4\\-2&0&1\end{pmatrix} \quad D.\begin{pmatrix}2&-1&0\\1&3&-4\\1&0&-2\end{pmatrix} \quad E. \quad F. \quad G. \quad H.
$$
$$
\begin{array}{l}令\;P_1=\begin{pmatrix}0&1&0\\1&0&0\\0&0&1\end{pmatrix},P_2=\begin{pmatrix}1&0&0\\0&0&1\\0&1&0\end{pmatrix},B=\begin{pmatrix}1&-4&3\\2&0&-1\\1&-2&0\end{pmatrix}\\则P_1,P_2\mathrm{均为初等矩阵},且P_1^{-1}=P1,P_2^{-1}=P_2,故X=P_1BP_2=\begin{pmatrix}2&-1&0\\1&3&-4\\1&0&-2\end{pmatrix}\end{array}
$$



$$
\mathrm{矩阵}\begin{pmatrix}0&2&-3&1\\0&3&-4&3\\0&4&-7&-1\end{pmatrix}\mathrm{的标准形矩阵为}(\;\;\;).
$$
$$
A.
\begin{pmatrix}1&0&0&0\\0&1&0&0\\0&0&0&0\end{pmatrix} \quad B.\begin{pmatrix}1&0&0&0\\0&1&0&0\\0&0&1&0\end{pmatrix} \quad C.\begin{pmatrix}1&0&0&0\\0&1&0&0\\0&0&0&1\end{pmatrix} \quad D.\begin{pmatrix}1&0&0&0\\0&0&0&0\\0&0&0&0\end{pmatrix} \quad E. \quad F. \quad G. \quad H.
$$
$$
\begin{pmatrix}0&2&-3&1\\0&3&-4&3\\0&4&-7&-1\end{pmatrix}\rightarrow\begin{pmatrix}0&2&-3&1\\0&0&1&3\\0&0&-1&-3\end{pmatrix}\rightarrow\begin{pmatrix}0&1&0&5\\0&0&1&3\\0&0&0&0\end{pmatrix}\rightarrow\begin{pmatrix}1&0&0&0\\0&1&0&0\\0&0&0&0\end{pmatrix}.
$$



$$
设A\mathrm{是三阶方阵},\mathrm{交换}A\mathrm{的第一列},\mathrm{第二列得}B,将B\mathrm{的第二列加到第三列得}C,\mathrm{若有可逆阵}Q\mathrm{使得}AQ=C,则Q=(\;\;\;).
$$
$$
A.
\begin{pmatrix}1&0&0\\0&1&1\\0&0&1\end{pmatrix} \quad B.\begin{pmatrix}0&1&1\\0&0&1\\1&0&0\end{pmatrix} \quad C.\begin{pmatrix}0&0&1\\1&0&0\\0&1&1\end{pmatrix} \quad D.\begin{pmatrix}0&1&1\\1&0&0\\0&0&1\end{pmatrix} \quad E. \quad F. \quad G. \quad H.
$$
$$
\begin{array}{l}记E(1,2)=\begin{pmatrix}0&1&0\\1&0&0\\0&0&1\end{pmatrix},E(2,3(1))=\begin{pmatrix}1&0&0\\0&1&1\\0&0&1\end{pmatrix},则AE(1,2)=B,B(E(2,3(1)))=C,故\;\\AE(1,2)E(2,3(1))=C,取Q=E(1,2)E(2,3(1))=\begin{pmatrix}0&1&1\\1&0&0\\0&0&1\end{pmatrix}\mathrm{即可}.\end{array}
$$



$$
\begin{pmatrix}0&0&1\\0&1&0\\1&0&0\end{pmatrix}^{2000}\begin{pmatrix}1&2&3\\4&5&6\\7&8&9\end{pmatrix}\begin{pmatrix}1&0&0\\0&0&1\\0&1&0\end{pmatrix}^{2001}=(\;\;\;).
$$
$$
A.
\begin{pmatrix}7&9&8\\4&6&5\\1&3&2\end{pmatrix} \quad B.\begin{pmatrix}1&3&2\\4&6&5\\7&9&8\end{pmatrix} \quad C.\begin{pmatrix}3&1&2\\5&6&4\\7&9&8\end{pmatrix} \quad D.\begin{pmatrix}7&8&9\\4&6&5\\1&2&3\end{pmatrix} \quad E. \quad F. \quad G. \quad H.
$$
$$
\begin{array}{l}\begin{array}{l}\mathrm{注意到}P=\begin{pmatrix}0&0&1\\0&1&0\\1&0&0\end{pmatrix}与Q=\begin{pmatrix}1&0&0\\0&0&1\\0&1&0\end{pmatrix}\mathrm{均为初等矩阵},PA是A\mathrm{作一次行变换}(一,\mathrm{三两行互换}),\\\;P^{2000}A为A\mathrm{的一},\mathrm{三两行作了偶数次对换},故\;P^{2000}A=A.\\\;\mathrm{类似地},AQ^{2001}\mathrm{相当于}A\mathrm{的二},\mathrm{三列作了一次对换}.\\\;\mathrm{故应填}\begin{pmatrix}1&3&2\\4&6&5\\7&9&8\end{pmatrix}.\end{array}\\\end{array}
$$



$$
\mathrm{下列各矩阵中},\mathrm{初等矩阵是}\left(\right).
$$
$$
A.
\begin{pmatrix}0&1&0\\0&0&1\\1&0&0\end{pmatrix} \quad B.\begin{pmatrix}0&0&1\\0&1&0\\2&0&0\end{pmatrix} \quad C.\begin{pmatrix}1&0&2\\0&1&0\\0&0&1\end{pmatrix} \quad D.\begin{pmatrix}0&0&1\\0&1&0\\1&0&2\end{pmatrix} \quad E. \quad F. \quad G. \quad H.
$$
$$
\mathrm{初等矩阵是单位矩阵经过一次初等变换后得到的矩阵},则\begin{pmatrix}1&0&2\\0&1&0\\0&0&1\end{pmatrix}\mathrm{是初等矩阵},\mathrm{其它选项中的矩阵不符合条件}.
$$



$$
\mathrm{初等矩阵}\left(\right).\;
$$
$$
A.
\mathrm{都可以经过初等变换化为单位阵} \quad B.\mathrm{所对应的行列式的值为}1 \quad C.\mathrm{相乘仍为初等方阵} \quad D.\mathrm{相加仍为初等方阵} \quad E. \quad F. \quad G. \quad H.
$$
$$
\begin{array}{l}\mathrm{初等矩阵是由单位矩阵经过一次初等变换后得到的矩阵},\mathrm{因此可经过初等变换化为单位阵};\\\mathrm{初等矩阵的行列式可为}±1,\;k\left(k∈ R\right)\mathrm{三个值},\mathrm{且初等矩阵相加和相乘不一定为初等矩阵}.\end{array}
$$



$$
\mathrm{两个}n\mathrm{阶初等矩阵的乘积为}\left(\right).
$$
$$
A.
\mathrm{初等矩阵} \quad B.\mathrm{单位矩阵} \quad C.\mathrm{可逆矩阵} \quad D.\mathrm{不可逆矩阵} \quad E. \quad F. \quad G. \quad H.
$$
$$
\mathrm{由初等矩阵的性质可知初等矩阵的行列式等于}±1\mathrm{或非零常数}k,\mathrm{因此两个初等矩阵乘积的行列式也非零},\mathrm{故可逆}.\mathrm{其它选项不能确定}.
$$



$$
\mathrm{下列矩阵中不是初等矩阵的是}\left(\right).
$$
$$
A.
\begin{pmatrix}1&1\\0&1\end{pmatrix} \quad B.\begin{pmatrix}0&0&1\\0&-1&0\\1&0&0\end{pmatrix} \quad C.\begin{pmatrix}1&0&0\\0&3&0\\0&0&1\end{pmatrix} \quad D.\begin{pmatrix}1&0&0\\0&1&0\\5&0&1\end{pmatrix} \quad E. \quad F. \quad G. \quad H.
$$
$$
\begin{array}{l}\mathrm{初等矩阵的定义是对单位矩阵施以一次初等变换得到的矩阵},\mathrm{而矩阵}\begin{pmatrix}0&0&1\\0&-1&0\\1&0&0\end{pmatrix}\mathrm{是先将单位矩阵的}1,\;3\mathrm{列互}\\换,\mathrm{再将第}2\mathrm{行乘以}\left(-1\right)\mathrm{后的结果},\mathrm{进行了两次变换},\mathrm{所以不是初等矩阵},\mathrm{其余选项中的矩阵都是初等矩阵}.\end{array}
$$



$$
F=\begin{pmatrix}1&2&3\\3&-1&2\end{pmatrix},\;E_{12}是2\mathrm{阶初等矩阵},则E_{12}F\mathrm{等于}\left(\right).\;
$$
$$
A.
\begin{pmatrix}3&-1&2\\1&2&3\end{pmatrix} \quad B.\begin{pmatrix}2&1&3\\-1&3&2\end{pmatrix} \quad C.\begin{pmatrix}2&4&6\\3&-1&2\end{pmatrix} \quad D.\begin{pmatrix}1&3&2\\3&2&-1\end{pmatrix} \quad E. \quad F. \quad G. \quad H.
$$
$$
\mathrm{用初等矩阵左乘矩阵相当于对矩阵作相应的初等行变换},E_{12}F\mathrm{等价于将矩阵}F\mathrm{的一}、\mathrm{二行互换}.
$$



$$
F=\begin{pmatrix}1&2&3\\0&1&4\\2&3&0\end{pmatrix},\;E_3\left(2\right)是3\mathrm{阶初等方阵},则\;E_3\left(2\right)F\mathrm{等于}\left(\right).\;
$$
$$
A.
\begin{pmatrix}1&3&2\\0&4&1\\2&0&3\end{pmatrix} \quad B.\begin{pmatrix}1&2&3\\2&3&0\\0&1&4\end{pmatrix} \quad C.\begin{pmatrix}1&2&3\\0&1&4\\4&6&0\end{pmatrix} \quad D.\begin{pmatrix}1&2&6\\0&1&8\\2&3&0\end{pmatrix} \quad E. \quad F. \quad G. \quad H.
$$
$$
E_3(2)F\mathrm{等价于将矩阵}F\mathrm{的第三行乘以}2.
$$



$$
设A,B为n\mathrm{阶可逆矩阵},则\left(\;\;\;\;\right).\;
$$
$$
A.
AB=BA \quad B.\mathrm{存在}n\mathrm{阶矩阵}P,使P^{-1}AP=B \quad C.\mathrm{存在}n\mathrm{阶矩阵}P,\mathrm{使得}P^TAP=B \quad D.\mathrm{可以用初等变换把}A\mathrm{变成}B \quad E. \quad F. \quad G. \quad H.
$$
$$
\begin{array}{l}\mathrm{由于矩阵乘法一般不满足交换律},\mathrm{因此}AB\mathrm{不一定等于}BA;\mathrm{由于可逆矩阵可以表示成一系列初等矩阵的乘积},\\即\;\;\\A=P_1P_2⋯ P_n,B=Q_1Q_2⋯ Q_n,故A\left(P_1P_2⋯ P_n\right)^{-1}=E,即\;B=Q_1Q_2⋯ Q_nA\left(P_1P_2⋯ P_n\right)^{-1}\\\mathrm{因此两个可逆的矩阵可用初等变换进行互换}.\end{array}
$$



$$
F=\begin{pmatrix}1&0&3\\2&1&-2\end{pmatrix},\;E\left(1,\;2\right)是3\mathrm{阶初等方阵},FE\left(1,\;2\right)\mathrm{则等于}\left(\right).\;
$$
$$
A.
\begin{pmatrix}2&1&-2\\1&0&3\end{pmatrix} \quad B.\begin{pmatrix}0&1&3\\1&2&-2\end{pmatrix} \quad C.\begin{pmatrix}2&0&6\\2&1&-2\end{pmatrix} \quad D.\begin{pmatrix}2&0&3\\4&1&-2\end{pmatrix} \quad E. \quad F. \quad G. \quad H.
$$
$$
FE\left(1,\;2\right)\mathrm{表示交换矩阵}F的1,\;2列,即FE\left(1,\;2\right)=\begin{pmatrix}0&1&3\\1&2&-2\end{pmatrix}
$$



$$
F=\begin{pmatrix}0&1&2\\3&2&1\\1&0&-2\end{pmatrix},\;E_2\left(3\right)是3\mathrm{阶初等方阵},则FE_2\left(3\right)\mathrm{等于}\left(\right).\;
$$
$$
A.
\begin{pmatrix}0&1&2\\9&6&3\\1&0&-2\end{pmatrix} \quad B.\begin{pmatrix}0&3&2\\3&6&1\\1&0&-2\end{pmatrix} \quad C.\begin{pmatrix}0&2&1\\3&1&2\\1&-2&0\end{pmatrix} \quad D.\begin{pmatrix}0&1&4\\3&2&2\\1&0&-4\end{pmatrix} \quad E. \quad F. \quad G. \quad H.
$$
$$
FE_2\left(3\right)\mathrm{表示将矩阵}F\mathrm{的第}2\mathrm{列乘以}3,即FE_2\left(3\right)=\begin{pmatrix}0&3&2\\3&6&1\\1&0&-2\end{pmatrix}
$$



$$
F=\begin{pmatrix}2&1&-3\\0&2&-1\\3&0&4\end{pmatrix},\;P=E_{23}\left(3\right)\mathrm{是初等方阵},则FP\mathrm{等于}\left(\right).\;
$$
$$
A.
\begin{pmatrix}2&1&-3\\0&2&-1\\3&6&1\end{pmatrix} \quad B.\begin{pmatrix}2&1&-3\\3&0&4\\0&2&-1\end{pmatrix} \quad C.\begin{pmatrix}2&-3&1\\0&-1&2\\3&4&0\end{pmatrix} \quad D.\begin{pmatrix}2&1&0\\0&2&5\\3&0&4\end{pmatrix} \quad E. \quad F. \quad G. \quad H.
$$
$$
FP\mathrm{表示将矩阵}F\mathrm{的第}2\mathrm{列乘以}3\mathrm{加到第}3列,即FP=\begin{pmatrix}2&1&0\\0&2&5\\3&0&4\end{pmatrix}
$$



$$
设F=\begin{pmatrix}-1&3&2\\1&0&-3\\4&2&1\end{pmatrix},\;P=E_{13}\left(2\right)=\begin{pmatrix}1&0&2\\0&1&0\\0&0&1\end{pmatrix}是3\mathrm{阶初等方阵},则PF\mathrm{等于}\left(\right).\;
$$
$$
A.
\begin{pmatrix}7&7&4\\1&0&-3\\4&2&1\end{pmatrix} \quad B.\begin{pmatrix}3&3&2\\-5&0&-3\\6&2&1\end{pmatrix} \quad C.\begin{pmatrix}8&3&2\\1&0&-3\\10&2&1\end{pmatrix} \quad D.\begin{pmatrix}2&3&-7\\1&0&-3\\4&2&0\end{pmatrix} \quad E. \quad F. \quad G. \quad H.
$$
$$
\begin{array}{l}\mathrm{根据初等变换与初等矩阵的性质定理可知},PF\mathrm{表示将矩阵的第}3\mathrm{行乘以}2\mathrm{加到第}1行,即\\PF=\begin{pmatrix}7&7&4\\1&0&-3\\4&2&1\end{pmatrix}.\end{array}
$$



$$
设A,B\mathrm{均为}n\mathrm{阶矩阵},A与B\mathrm{等价},\mathrm{则下列命题中错误的是}.
$$
$$
A.
若\left|A\right|>0,则\left|B\right|>0 \quad B.若\left|A\right|\neq0,则B\mathrm{也可逆} \quad C.若A与E\mathrm{等价},则B与E\mathrm{等价} \quad D.\mathrm{存在可逆矩阵}P,Q,\mathrm{使得}PAQ=B \quad E. \quad F. \quad G. \quad H.
$$
$$
\begin{array}{l}\mathrm{由等价矩阵的定义可知},A与B\mathrm{等价},\mathrm{则存在可逆矩阵}P,Q,\mathrm{使得}PAQ=B;\;\;\\\mathrm{由于等价矩阵有相同的秩},\mathrm{故若}\left|A\right|\neq0,则\left|B\right|\neq0,即B\mathrm{也可逆};\;\;\\\mathrm{且由等价的传递性可知}:若A与E\mathrm{等价},则B与E\mathrm{等价};\;\;\\A与B\mathrm{等价},若\left|A\right|>0,\mathrm{则只能得出}\left|B\right|\neq0\end{array}
$$



$$
\mathrm{用初等矩阵}E_{ji}\left(k\right)\mathrm{右乘矩阵}A,\;\mathrm{得到乘积}AE_{ji}\left(k\right),\mathrm{它相当于}\left(\right).
$$
$$
A.
\mathrm{用数}k\mathrm{乘以}A\mathrm{的第}i\mathrm{列加到第}j\mathrm{列上} \quad B.\mathrm{用数}i\mathrm{乘以}A\mathrm{的第}k\mathrm{列加到第}j\mathrm{列上} \quad C.\mathrm{用数}i\mathrm{乘以}A\mathrm{的第}j\mathrm{列加到第}k\mathrm{列上} \quad D.\mathrm{用数}k\mathrm{乘以}A\mathrm{的第}j\mathrm{列加到第}i\mathrm{列上} \quad E. \quad F. \quad G. \quad H.
$$
$$
AP\left(j,\;i\left(k\right)\right)\mathrm{表示用数}k\mathrm{乘以}A\mathrm{的第}j\mathrm{列加到第}i\mathrm{列上}
$$



$$
\begin{array}{l}设A\mathrm{是三阶方阵},将A\mathrm{的第一列与第二列交换得矩阵}B,\mathrm{再把矩阵}B\mathrm{的第二列加到第三列得矩阵}C,\mathrm{则满足}AQ=C\\\mathrm{的可逆矩阵}Q为\left(\;\;\;\right).\end{array}
$$
$$
A.
\begin{pmatrix}0&1&0\\1&0&0\\1&0&1\end{pmatrix} \quad B.\begin{pmatrix}0&1&0\\1&0&1\\0&0&1\end{pmatrix} \quad C.\begin{pmatrix}0&1&0\\1&0&0\\0&1&1\end{pmatrix} \quad D.\begin{pmatrix}0&1&1\\1&0&0\\0&0&1\end{pmatrix} \quad E. \quad F. \quad G. \quad H.
$$
$$
\begin{array}{l}将A\mathrm{的第一列与第二列交换得矩阵}B,\mathrm{对应的初等矩阵}Q_1=\begin{pmatrix}0&1&0\\1&0&0\\0&0&1\end{pmatrix},\mathrm{把矩阵}B\mathrm{的第二列加到第三列得矩阵}C,\\\mathrm{对应的初等矩阵}Q_2=\begin{pmatrix}1&0&0\\0&1&1\\0&0&1\end{pmatrix},\mathrm{所以}Q=Q_1Q_2=\begin{pmatrix}0&1&0\\1&0&0\\0&0&1\end{pmatrix}\begin{pmatrix}1&0&0\\0&1&1\\0&0&1\end{pmatrix}=\begin{pmatrix}0&1&1\\1&0&0\\0&0&1\end{pmatrix}.\end{array}
$$



$$
设A=\begin{pmatrix}a_{11}&a_{12}&a_{13}\\a_{21}&a_{22}&a_{23}\\a_{31}&a_{32}&a_{33}\end{pmatrix},\;B=\begin{pmatrix}a_{11}&a_{12}&a_{13}+a_{11}\\a_{21}&a_{22}&a_{23}+a_{21}\\a_{31}&a_{32}&a_{33}+a_{31}\end{pmatrix},\;P_1=\begin{pmatrix}1&0&1\\0&1&0\\0&0&1\end{pmatrix},\;P_2=\begin{pmatrix}1&0&0\\0&1&0\\1&0&1\end{pmatrix},\;则B=\left(\right).
$$
$$
A.
P_1A=B \quad B.P_2A=B \quad C.AP_1=B \quad D.AP_2=B \quad E. \quad F. \quad G. \quad H.
$$
$$
\mathrm{矩阵}B\mathrm{是将矩阵}A\mathrm{的第}1\mathrm{列加到第}3列,即B=AP_1.
$$



$$
\mathrm{若初等矩阵}\begin{pmatrix}1&0&0\\0&0&1\\0&1&0\end{pmatrix}A,\mathrm{相当于对矩阵}A\mathrm{施行}\left(\right)\mathrm{的初等变换}.
$$
$$
A.
r_2↔ r_3 \quad B.c_2↔ c_3 \quad C.r_1↔ r_3 \quad D.c_1↔ c_3 \quad E. \quad F. \quad G. \quad H.
$$
$$
\mathrm{以初等矩阵左乘矩阵}A\mathrm{相当于对}A\mathrm{施行相应行变换},\mathrm{因此是第}2\mathrm{行和第}3\mathrm{行的互换}.
$$



$$
设A是n\left(n\geq2\right)\mathrm{阶可逆方阵},将A\mathrm{的第一行与第二行交换得矩阵}B,A^*,B^*\mathrm{分别是}A与B\mathrm{的伴随矩阵},则\left(\;\;\;\;\;\;\;\;\;\right)
$$
$$
A.
\mathrm{交换}A^*\mathrm{第一列与第二列得}B^* \quad B.\mathrm{交换}A^*\mathrm{第一行与第二行得}B^* \quad C.\mathrm{交换}A^*\mathrm{第一列与第二列得}-B^* \quad D.\mathrm{交换}A^*\mathrm{第一行与第二行得}-B^* \quad E. \quad F. \quad G. \quad H.
$$
$$
\mathrm{设交换}A\mathrm{的第一行与第二行的初等矩阵为}P,则B=PA,\;\;B^*=\left|B\right|B^{-1}=\left|PA\right|\left(PA\right)^{-1}=-\left|A\right|A^{-1}P^{-1}=-A^* P
$$



$$
设A=\begin{pmatrix}2&4&0\\1&2&-1\end{pmatrix},\;B=\begin{pmatrix}2&1\\0&-1\\1&0\end{pmatrix},\;C=E_{23}\mathrm{是三阶初等方阵},则ACB=\left(\right).
$$
$$
A.
\begin{pmatrix}8&2\\4&2\end{pmatrix} \quad B.\begin{pmatrix}8&-2\\4&2\end{pmatrix} \quad C.\begin{pmatrix}4&2\\8&2\end{pmatrix} \quad D.\begin{pmatrix}4&-2\\8&2\end{pmatrix} \quad E. \quad F. \quad G. \quad H.
$$
$$
ACB=\left(AC\right)B=\begin{pmatrix}2&0&4\\1&-1&2\end{pmatrix}\begin{pmatrix}2&1\\0&-1\\1&0\end{pmatrix}=\begin{pmatrix}8&2\\4&2\end{pmatrix}
$$



$$
\mathrm{矩阵}A=\begin{pmatrix}2&1&1\\3&1&1\\2&7&8\end{pmatrix},\;则\begin{pmatrix}1&0&0\\0&0&1\\0&1&0\end{pmatrix}\begin{pmatrix}2&1&1\\3&1&1\\2&7&8\end{pmatrix}\mathrm{相当于对}A\mathrm{进行}\left(\right)\mathrm{的初等变换}.
$$
$$
A.
\mathrm{第一行与第二行互换} \quad B.\mathrm{第二行与第三行互换} \quad C.\mathrm{第一列与第二列互换} \quad D.\mathrm{第二列与第三列互换} \quad E. \quad F. \quad G. \quad H.
$$
$$
\mathrm{根据初等变换的有关内容},\mathrm{以初等矩阵左乘矩阵}A,\mathrm{相当于}A\mathrm{对进行初等行变换}.
$$



$$
设A是n\mathrm{阶可逆方阵},\mathrm{互换}A\mathrm{中第}i\mathrm{行和第}j\mathrm{行得到矩阵}B,则AB^{-1}=\left(\right).
$$
$$
A.
E_{ij} \quad B.E_i\left(-1\right) \quad C.E_j\left(-1\right) \quad D.E_{ij}\left(-1\right) \quad E. \quad F. \quad G. \quad H.
$$
$$
\mathrm{由题意及初等矩阵性质},有B=E_{ij}A,\mathrm{所以}AB^{-1}=A\left(E_{ij}A\right)^{-1}=E_{ij}.
$$



$$
\mathrm{下列各矩阵中},\mathrm{初等矩阵是}\left(\right).
$$
$$
A.
\begin{pmatrix}0&1&0\\0&0&1\\1&0&0\end{pmatrix} \quad B.\begin{pmatrix}1&0&0\\0&2&0\\0&0&1\end{pmatrix} \quad C.\begin{pmatrix}0&0&1\\0&1&1\\1&0&0\end{pmatrix} \quad D.\begin{pmatrix}0&0&1\\0&1&0\\1&0&-1\end{pmatrix} \quad E. \quad F. \quad G. \quad H.
$$
$$
\mathrm{根据定义},\mathrm{初等矩阵是单位矩阵经过一次初等变换后得到的矩阵},\mathrm{答案}\begin{pmatrix}1&0&0\\0&2&0\\0&0&1\end{pmatrix}\mathrm{符合},\mathrm{其它选项中的矩阵不符合条件}.
$$



$$
\mathrm{下列各矩阵中},\mathrm{初等矩阵是}\left(\right).
$$
$$
A.
\begin{pmatrix}1&0&-2\\0&1&0\\0&0&1\end{pmatrix} \quad B.\begin{pmatrix}0&0&1\\0&1&0\\3&0&0\end{pmatrix} \quad C.\begin{pmatrix}1&0&1\\0&1&0\\0&0&2\end{pmatrix} \quad D.\begin{pmatrix}0&1&1\\0&1&0\\0&0&-1\end{pmatrix} \quad E. \quad F. \quad G. \quad H.
$$
$$
\mathrm{根据定义},\mathrm{初等矩阵是单位矩阵经过一次初等变换后得到的矩阵},\mathrm{答案}\begin{pmatrix}1&0&-2\\0&1&0\\0&0&1\end{pmatrix}\mathrm{是初等矩阵},\mathrm{其它选项中的矩阵不符合条件}.
$$



$$
\mathrm{下列矩阵中},\mathrm{不是初等矩阵的是}\left(\right).
$$
$$
A.
\begin{pmatrix}0&0&1\\0&1&0\\1&0&0\end{pmatrix} \quad B.\begin{pmatrix}1&1\\0&1\end{pmatrix} \quad C.\begin{pmatrix}1&-2\\0&1\end{pmatrix} \quad D.\begin{pmatrix}0&0&1\\0&-1&0\\1&0&0\end{pmatrix} \quad E. \quad F. \quad G. \quad H.
$$
$$
\begin{array}{l}\mathrm{根据定义},\mathrm{初等矩阵是对单位矩阵施以一次初等变换得到的矩阵},\mathrm{答案}\begin{pmatrix}0&0&1\\0&-1&0\\1&0&0\end{pmatrix}\mathrm{是先交换}1,3\mathrm{两行},\mathrm{然后第二行乘以}-1,\\\mathrm{进行了两次变换},\mathrm{所以不是初等矩阵},\mathrm{其余选项中的矩阵都是初等矩阵}.\end{array}
$$



$$
A=\begin{pmatrix}1&2&3\\3&-1&2\end{pmatrix},\;E_{23}\mathrm{是二阶初等矩阵},则AE_{23}\mathrm{等于}\left(\right).\;
$$
$$
A.
\begin{pmatrix}3&-1&2\\1&2&3\end{pmatrix} \quad B.\begin{pmatrix}1&3&2\\3&2&-1\end{pmatrix} \quad C.\begin{pmatrix}3&2&-1\\1&3&2\end{pmatrix} \quad D.\begin{pmatrix}2&4&6\\3&-1&2\end{pmatrix} \quad E. \quad F. \quad G. \quad H.
$$
$$
\mathrm{用初等矩阵左乘矩阵相当于对矩阵作相应的初等行变换},AE_{23}\mathrm{等价于将矩阵的一}、\mathrm{二列互换}.
$$



$$
F=\begin{pmatrix}2&1&-3\\0&2&-1\\3&0&4\end{pmatrix},\;P=E_{23}\left(3\right)\mathrm{是初等方阵},则PF\mathrm{等于}\left(\right).
$$
$$
A.
\begin{pmatrix}2&1&-3\\0&2&-1\\3&6&1\end{pmatrix} \quad B.\begin{pmatrix}2&1&0\\0&2&5\\3&0&4\end{pmatrix} \quad C.\begin{pmatrix}2&1&-3\\9&2&11\\3&0&4\end{pmatrix} \quad D.\begin{pmatrix}2&-8&-3\\0&-1&-1\\3&12&4\end{pmatrix} \quad E. \quad F. \quad G. \quad H.
$$
$$
\mathrm{根据初等变换与初等矩阵的性质定理可知},PF\mathrm{表示将矩阵}F\mathrm{的第}3\mathrm{行乘以}3\mathrm{加到第}2行,\mathrm{即为答案}C.
$$



$$
F=\begin{pmatrix}-1&3&2\\1&0&-3\\4&2&1\end{pmatrix},\;P=E_{13}\left(2\right)=\begin{pmatrix}1&0&2\\0&1&0\\0&0&1\end{pmatrix}是3\mathrm{阶初等方阵},则FP\mathrm{等于}\left(\right).\;
$$
$$
A.
\begin{pmatrix}-1&3&0\\1&0&-1\\4&2&9\end{pmatrix} \quad B.\begin{pmatrix}3&3&2\\-5&0&-3\\6&2&1\end{pmatrix} \quad C.\begin{pmatrix}7&7&4\\1&0&-3\\4&2&1\end{pmatrix} \quad D.\begin{pmatrix}2&3&-7\\1&0&-3\\4&2&0\end{pmatrix} \quad E. \quad F. \quad G. \quad H.
$$
$$
\mathrm{根据初等变换与初等矩阵的性质定理可知},FP\mathrm{表示将矩阵的第}1\mathrm{列乘以}2\mathrm{加到第}3列,\mathrm{即为答案}A
$$



$$
设F\mathrm{可逆},且F\begin{pmatrix}a_{11}&a_{12}\\a_{21}&a_{22}\\a_{31}&a_{32}\end{pmatrix}=\begin{pmatrix}a_{31}&a_{32}\\a_{21}&a_{22}\\a_{11}&a_{12}\end{pmatrix},则F=(\;\;).
$$
$$
A.
\begin{pmatrix}1&0&0\\0&1&0\\0&0&1\end{pmatrix} \quad B.\begin{pmatrix}0&1&0\\1&0&0\\0&0&1\end{pmatrix} \quad C.\begin{pmatrix}1&0&0\\0&0&1\\0&1&0\end{pmatrix} \quad D.\begin{pmatrix}0&0&1\\0&1&0\\1&0&0\end{pmatrix} \quad E. \quad F. \quad G. \quad H.
$$
$$
\mathrm{矩阵}\begin{pmatrix}a_{31}&a_{32}\\a_{21}&a_{22}\\a_{11}&a_{12}\end{pmatrix}\mathrm{相当于是由矩阵}\begin{pmatrix}a_{11}&a_{12}\\a_{21}&a_{22}\\a_{31}&a_{32}\end{pmatrix}\mathrm{交换第一行和第三行得到的},\mathrm{根据初等矩阵的性质},\mathrm{答案为}D.
$$



$$
\mathrm{关于初等矩阵},\mathrm{下列说法正确的是}().\;
$$
$$
A.
\mathrm{其行列式的值为}±1; \quad B.\mathrm{相乘仍为初等矩阵}; \quad C.\mathrm{都与一个单位矩阵等价}; \quad D.\mathrm{相加仍为初等矩阵}. \quad E. \quad F. \quad G. \quad H.
$$
$$
\begin{array}{l}\mathrm{初等矩阵是由单位矩阵经过一次初等变换后得到的矩阵},\mathrm{因此可经过初等变换化为单位阵},\mathrm{由矩阵的等价知},初\\\mathrm{等矩阵与单位矩阵等价};\mathrm{初等矩阵的行列式可为}±1,\;k\left(k∈ R\right)\mathrm{三个值},\mathrm{且初等矩阵相加和相乘不一定为初等矩阵}.\end{array}
$$



$$
设F\mathrm{可逆},且F\begin{pmatrix}a_{11}&a_{12}\\a_{21}&a_{22}\\a_{31}&a_{32}\end{pmatrix}=\begin{pmatrix}a_{11}&a_{12}\\a_{21}+3a_{11}&a_{22}+3a_{12}\\a_{31}&a_{32}\end{pmatrix},则F=(\;\;\;\;\;).\;
$$
$$
A.
\begin{pmatrix}1&0&0\\0&3&0\\0&0&1\end{pmatrix} \quad B.\begin{pmatrix}1&0&0\\3&1&0\\0&0&1\end{pmatrix} \quad C.\begin{pmatrix}1&3&0\\0&1&0\\0&0&1\end{pmatrix} \quad D.\begin{pmatrix}0&0&1\\0&1&0\\1&0&0\end{pmatrix} \quad E. \quad F. \quad G. \quad H.
$$
$$
\mathrm{矩阵}\begin{pmatrix}a_{11}&a_{12}\\a_{21}+3a_{11}&a_{22}+3a_{12}\\a_{31}&a_{32}\end{pmatrix}\mathrm{是由矩阵}\begin{pmatrix}a_{11}&a_{12}\\a_{21}&a_{22}\\a_{31}&a_{32}\end{pmatrix}\mathrm{的第一行乘以}3\mathrm{加到第二行得到的},即F=\begin{pmatrix}1&0&0\\3&1&0\\0&0&1\end{pmatrix}
$$



$$
\mathrm{设矩阵}A=\begin{pmatrix}2&4&0\\1&2&-1\end{pmatrix},\;C=E_{23}\mathrm{是三阶初等矩阵},则AC=(\;\;\;)
$$
$$
A.
\begin{pmatrix}2&0&4\\1&-1&2\end{pmatrix} \quad B.\begin{pmatrix}4&0&2\\2&-1&1\end{pmatrix} \quad C.\begin{pmatrix}1&2&-1\\2&4&0\end{pmatrix} \quad D.\begin{pmatrix}0&4&2\\-1&2&1\end{pmatrix} \quad E. \quad F. \quad G. \quad H.
$$
$$
\mathrm{根据初等矩阵和初等变换的有关内容},AC\mathrm{就是将}A\mathrm{的第二列和第三列互换得到},\mathrm{答案为}A.
$$



$$
\mathrm{用初等矩阵}E_{23}\left(4\right)\mathrm{右乘矩阵}A,\mathrm{得到乘积}AE_{23}\left(4\right),\mathrm{它相当于}(\;\;).
$$
$$
A.
\mathrm{用数}4\mathrm{乘以}A\mathrm{的第}2\mathrm{列加到第}3\mathrm{列上} \quad B.\mathrm{用数}3\mathrm{乘以}A\mathrm{的第}4\mathrm{列加到第}2\mathrm{列上} \quad C.\mathrm{用数}4\mathrm{乘以}A\mathrm{的第}3\mathrm{列加到第}2\mathrm{列上} \quad D.\mathrm{用数}3\mathrm{乘以}A\mathrm{的第}2\mathrm{列加到第}4\mathrm{列上} \quad E. \quad F. \quad G. \quad H.
$$
$$
AE_{23}\left(4\right)\mathrm{表示用数}4\mathrm{乘以}A\mathrm{的第}2\mathrm{列加到第}3\mathrm{列上}
$$



$$
\begin{array}{l}\mathrm{矩阵}\begin{pmatrix}1&0&2&-1\\2&0&3&1\\3&0&4&-3\end{pmatrix}\mathrm{的标准形是}\left(\;\;\;\;\;\right)\end{array}
$$
$$
A.
\begin{pmatrix}1&0&0\\0&1&0\\0&0&1\end{pmatrix} \quad B.\begin{pmatrix}1&0&0&0\\0&1&1&0\\0&0&1&0\end{pmatrix} \quad C.\begin{pmatrix}1&0&0&0\\0&1&0&0\\0&0&1&0\end{pmatrix}; \quad D.\begin{pmatrix}1&0&0&0\\0&0&1&0\\0&0&0&1\end{pmatrix} \quad E. \quad F. \quad G. \quad H.
$$
$$
\begin{array}{lc}\begin{array}{l}\begin{pmatrix}1&0&2&-1\\2&0&3&1\\3&0&4&-3\end{pmatrix}\xrightarrow[{r_3+\left(-3\right)r_1}]{r_2+\left(-2\right)r_1}\begin{pmatrix}1&0&2&-1\\0&0&-1&3\\0&0&-2&0\end{pmatrix}\xrightarrow[{r_3×\left(-\frac12\right)}]{r_2×\left(-1\right)}\begin{pmatrix}1&0&2&-1\\0&0&1&-3\\0&0&1&0\end{pmatrix}\rightarrow\begin{pmatrix}1&0&0&0\\0&1&0&0\\0&0&1&0\end{pmatrix}.\\\end{array}&\\&\end{array}
$$



$$
F=\begin{pmatrix}1&2&3\\3&-1&2\end{pmatrix},\;E_{12}是2\mathrm{阶初等方阵},则E_{12}F\mathrm{等于}\left(\right).
$$
$$
A.
\begin{pmatrix}3&-1&2\\1&2&3\end{pmatrix} \quad B.\begin{pmatrix}2&1&3\\-1&3&2\end{pmatrix} \quad C.\begin{pmatrix}2&4&6\\3&-1&2\end{pmatrix} \quad D.\begin{pmatrix}1&3&2\\3&2&-1\end{pmatrix} \quad E. \quad F. \quad G. \quad H.
$$
$$
\mathrm{用初等矩阵左乘矩阵相当于对矩阵作相应的初等行变换},E_{12}F\mathrm{等价于将矩阵的一}、\mathrm{二行互换}.
$$



$$
F=\begin{pmatrix}1&2&3\\0&1&4\\2&3&0\end{pmatrix},\;E_3\left(2\right)是3\mathrm{阶初等方阵},则E_3\left(2\right)F\mathrm{等于}\left(\right).\;
$$
$$
A.
\begin{pmatrix}1&3&2\\0&4&1\\2&0&3\end{pmatrix} \quad B.\begin{pmatrix}1&2&3\\2&3&0\\0&1&4\end{pmatrix} \quad C.\begin{pmatrix}1&2&3\\0&1&4\\4&6&0\end{pmatrix} \quad D.\begin{pmatrix}1&2&6\\0&1&8\\2&3&0\end{pmatrix} \quad E. \quad F. \quad G. \quad H.
$$
$$
E_3\left(2\right)F\mathrm{等价于将矩阵}F\mathrm{的第三行乘以}2.
$$



$$
\begin{array}{l}\mathrm{设矩阵}A=\begin{pmatrix}a_{11}&a_{12}&a_{13}&a_{14}\\a_{21}&a_{22}&a_{23}&a_{24}\\a_{31}&a_{32}&a_{33}&a_{34}\\a_{41}&a_{42}&a_{43}&a_{44}\end{pmatrix},\;B=\begin{pmatrix}a_{14}&a_{13}&a_{12}&a_{11}\\a_{24}&a_{23}&a_{22}&a_{21}\\a_{34}&a_{33}&a_{32}&a_{31}\\a_{44}&a_{43}&a_{42}&a_{41}\end{pmatrix},\;\\P_1=\begin{pmatrix}0&0&0&1\\0&1&0&0\\0&0&1&0\\1&0&0&0\end{pmatrix},\;P_2=\begin{pmatrix}1&0&0&0\\0&0&1&0\\0&1&0&0\\0&0&0&1\end{pmatrix},\;\mathrm{其中}A\mathrm{可逆},则B^{-1}\mathrm{等于}\left(\right).\end{array}
$$
$$
A.
A^{-1}P_1P_2 \quad B.P_1A^{-1}P_2 \quad C.P_1P_2A^{-1} \quad D.P_2A^{-1}P_1 \quad E. \quad F. \quad G. \quad H.
$$
$$
\begin{array}{l}\mathrm{由于}B=AP_2P_1,\;故\\B^{-1}=\left(AP_2P_1\right)^{-1}=P_1^{-1}P_2^{-1}A^{-1},\\而P_1^{-1}=P_1,\;P_2^{-1}=P_2,\;\mathrm{所以}B^{-1}=P_1P_2A^{-1}.\end{array}
$$



$$
设E_{23}与E_{23}\left(1\right)\mathrm{都是}3\mathrm{阶初等方阵},且E_{23}\left(1\right)·F·E_{23}=\begin{pmatrix}1&3&-2\\2&1&-3\\-1&4&3\end{pmatrix},则F\mathrm{等于}\left(\right).\;
$$
$$
A.
\begin{pmatrix}1&3&-2\\-1&4&3\\3&-3&-6\end{pmatrix} \quad B.\begin{pmatrix}1&5&-2\\-1&1&3\\2&4&-3\end{pmatrix} \quad C.\begin{pmatrix}1&-2&3\\-1&3&-8\\-1&3&4\end{pmatrix} \quad D.\begin{pmatrix}1&-2&3\\3&-6&-3\\-1&3&4\end{pmatrix} \quad E. \quad F. \quad G. \quad H.
$$
$$
\begin{array}{l}\mathrm{由条件可知}\\F=E_{23}^{-1}\left(1\right)\begin{pmatrix}1&3&-2\\2&1&-3\\-1&4&3\end{pmatrix}E_{23}^{-1}=E_{23}\left(-1\right)\begin{pmatrix}1&3&-2\\2&1&-3\\-1&4&3\end{pmatrix}E_{23}\\\mathrm{即将}\begin{pmatrix}1&3&-2\\2&1&-3\\-1&4&3\end{pmatrix}第3\mathrm{行的}\left(-1\right)\mathrm{倍加到第}2行,\mathrm{再交换第}2,\;3\mathrm{列得}\begin{pmatrix}1&-2&3\\3&-6&-3\\-1&3&4\end{pmatrix}.\end{array}
$$



$$
设A=\begin{pmatrix}a_{11}&a_{12}&a_{13}\\a_{21}&a_{22}&a_{23}\\a_{31}&a_{32}&a_{33}\end{pmatrix},\;B=\begin{pmatrix}a_{21}&a_{22}+ka_{23}&a_{23}\\a_{31}&a_{32}+ka_{33}&a_{33}\\a_{11}&a_{12}+ka_{13}&a_{13}\end{pmatrix},\;P_1=\begin{pmatrix}0&1&0\\0&0&1\\1&0&0\end{pmatrix},\;P_2=\begin{pmatrix}1&0&0\\0&1&0\\0&k&1\end{pmatrix},\;则B=\left(\right).
$$
$$
A.
AP_1P_2 \quad B.P_1AP_2 \quad C.P_2AP_1 \quad D.AP_2P_1 \quad E. \quad F. \quad G. \quad H.
$$
$$
\mathrm{根据初等矩阵和初等变换的有关内容即可}
$$



$$
\mathrm{已知}A\begin{pmatrix}1&2&3\\0&1&-1\end{pmatrix}=\begin{pmatrix}1&2&3\\2&5&5\end{pmatrix},\;则A=\left(\right).
$$
$$
A.
\begin{pmatrix}1&2\\0&1\end{pmatrix} \quad B.\begin{pmatrix}1&0\\2&1\end{pmatrix} \quad C.\begin{pmatrix}2&0\\1&1\end{pmatrix} \quad D.\begin{pmatrix}1&2\\1&1\end{pmatrix} \quad E. \quad F. \quad G. \quad H.
$$
$$
\begin{array}{l}\mathrm{矩阵}\begin{pmatrix}1&2&3\\2&5&5\end{pmatrix}\mathrm{是由}\begin{pmatrix}1&2&3\\0&1&-1\end{pmatrix}\mathrm{的第一行乘以}2\mathrm{加到第二行得到的},\mathrm{由初等变换和初等矩阵的知识},\mathrm{可知}\\A=\begin{pmatrix}1&0\\2&1\end{pmatrix}.\end{array}
$$



$$
设A=\begin{pmatrix}2&4&0\\1&2&-1\end{pmatrix},\;B=\begin{pmatrix}2&1\\0&-1\\1&0\end{pmatrix},\;C=E_{23}\mathrm{是三阶初等方阵},则\left|ACB\right|=\left(\right).
$$
$$
A.
8 \quad B.4 \quad C.2 \quad D.1 \quad E. \quad F. \quad G. \quad H.
$$
$$
ACB=\left(AC\right)B=\begin{pmatrix}2&0&4\\1&-1&2\end{pmatrix}\begin{pmatrix}2&1\\0&-1\\1&0\end{pmatrix}=\begin{pmatrix}8&2\\4&2\end{pmatrix},\;\left|ACB\right|=8.
$$



$$
设\begin{pmatrix}0&1&0\\1&0&0\\0&0&1\end{pmatrix}A\begin{pmatrix}1&0&1\\0&1&0\\0&0&1\end{pmatrix}=\begin{pmatrix}1&2&3\\4&5&6\\7&8&9\end{pmatrix},则A=\left(\right).
$$
$$
A.
\begin{pmatrix}4&5&2\\1&2&6\\7&8&2\end{pmatrix} \quad B.\begin{pmatrix}4&5&6\\1&2&3\\7&8&9\end{pmatrix} \quad C.\begin{pmatrix}4&5&3\\1&2&6\\7&8&9\end{pmatrix} \quad D.\begin{pmatrix}4&5&2\\1&2&2\\7&8&2\end{pmatrix} \quad E. \quad F. \quad G. \quad H.
$$
$$
\begin{array}{l}\mathrm{注意到所给等式左端的两个矩阵是初等矩阵}E_{12}及E_{13}\left(1\right),\\\mathrm{把等式右端记作}B,\;\mathrm{根据初等矩阵与初等变换的关系},得B\begin{array}{c}r_1↔ r_2\\∼\\c_3-c_1\end{array}A,\;故\\B=\begin{pmatrix}1&2&3\\4&5&6\\7&8&9\end{pmatrix}\xrightarrow{r_1↔ r_2}\begin{pmatrix}4&5&6\\1&2&3\\7&8&9\end{pmatrix}\xrightarrow{c_3-c_1}\begin{pmatrix}4&5&2\\1&2&2\\7&8&2\end{pmatrix}=A.\end{array}
$$



$$
设A是n\mathrm{阶可逆方阵},\mathrm{互换}A\mathrm{中第}2\mathrm{行和第}3\mathrm{行得到矩阵}B,则AB^{-1}=\left(\right).
$$
$$
A.
E_{23} \quad B.E_2\left(-1\right) \quad C.E_3\left(-1\right) \quad D.E_{23}\left(-1\right) \quad E. \quad F. \quad G. \quad H.
$$
$$
\mathrm{由题意及初等矩阵性质},有B=E_{23}A,\mathrm{所以}AB^{-1}=A\left(E_{23}A\right)^{-1}=E_{23}.
$$



$$
\mathrm{在矩阵}A=\begin{pmatrix}2&1&1\\3&1&1\\2&7&8\end{pmatrix}\mathrm{的左边乘以初等矩阵}\begin{pmatrix}1&0&0\\0&0&1\\0&1&0\end{pmatrix}\mathrm{相当于进行}\left(\right)\mathrm{的初等变换}.
$$
$$
A.
\mathrm{第一行与第二行互换} \quad B.\mathrm{第二行与第三行互换} \quad C.\mathrm{第一列与第二列互换} \quad D.\mathrm{第二列与第三列互换} \quad E. \quad F. \quad G. \quad H.
$$
$$
\mathrm{根据初等变换的有关内容},\mathrm{以初等矩阵左乘矩阵}A,\mathrm{相当于对}A\mathrm{进行初等行变换}.
$$



$$
设A=\begin{pmatrix}a_{11}&a_{12}&a_{13}\\a_{21}&a_{22}&a_{23}\\a_{31}&a_{32}&a_{33}\end{pmatrix},\;B=\begin{pmatrix}a_{21}&a_{22}&a_{23}\\a_{11}&a_{12}&a_{13}\\a_{31}+a_{11}&a_{32}+a_{12}&a_{33}+a_{13}\end{pmatrix},\;P_1=\begin{pmatrix}0&1&0\\1&0&0\\0&0&1\end{pmatrix},\;P_2=\begin{pmatrix}1&0&0\\0&1&0\\1&0&1\end{pmatrix},\;\mathrm{则必有}\left(\right).
$$
$$
A.
AP_1P_2=B \quad B.AP_2P_1=B \quad C.P_1P_2A=B \quad D.P_2P_1A=B \quad E. \quad F. \quad G. \quad H.
$$
$$
\mathrm{因为}A\mathrm{的第}1\mathrm{行加到第}3行,\mathrm{再交换第}1,\;2行,\mathrm{从而得到}B,\;故A\mathrm{左乘}P_1,\;即P_1P_2A=B.
$$



$$
\mathrm{用初等变换求}A=\begin{pmatrix}2&3&1&-3&-7\\1&2&0&-2&-4\\3&-2&8&3&0\\2&-3&7&4&3\end{pmatrix}\mathrm{的标准形}\left(\;\;\;\;\right)
$$
$$
A.
\begin{pmatrix}1&0&0&0&0\\0&0&0&0&0\\0&0&0&0&0\\0&0&0&0&0\end{pmatrix} \quad B.\begin{pmatrix}1&0&0&0&0\\0&1&0&0&0\\0&0&0&0&0\\0&0&0&0&0\end{pmatrix} \quad C.\begin{pmatrix}1&0&0&0&0\\0&1&0&0&0\\0&0&1&0&0\\0&0&0&0&0\end{pmatrix} \quad D.\begin{pmatrix}1&0&0&0&0\\0&1&0&0&0\\0&0&1&0&0\\0&0&0&1&0\end{pmatrix} \quad E. \quad F. \quad G. \quad H.
$$
$$
\begin{array}{l}\begin{pmatrix}2&3&1&-3&-7\\1&2&0&-2&-4\\3&-2&8&3&0\\2&-3&7&4&3\end{pmatrix}\xrightarrow[\begin{array}{c}r_3-3r_2\\r_4-2r_2\end{array}]{r_1-2r_2}\begin{pmatrix}0&-1&1&1&1\\1&2&0&-2&-4\\0&-8&8&9&12\\0&-7&7&8&11\end{pmatrix}\\\xrightarrow[\begin{array}{c}r_3-8r_1\\r_4-7r_1\end{array}]{r_2+2r_1}\begin{pmatrix}0&-1&1&1&1\\1&0&2&0&-2\\0&0&0&1&4\\0&0&0&1&4\end{pmatrix}\xrightarrow[{r_4-r_3}]{\begin{array}{c}r_2×\left(-1\right)\\r_1↔ r_2\end{array}}\begin{pmatrix}1&0&2&0&-2\\0&1&-1&-1&-1\\0&0&0&1&4\\0&0&0&0&0\end{pmatrix}\\\xrightarrow{r_2+r_3}\begin{pmatrix}1&0&2&0&-2\\0&1&-1&0&3\\0&0&0&1&4\\0&0&0&0&0\end{pmatrix}\rightarrow\begin{pmatrix}1&0&0&0&0\\0&1&0&0&0\\0&0&1&0&0\\0&0&0&0&0\end{pmatrix}.\end{array}
$$



$$
\mathrm{用初等变换求}A=\begin{pmatrix}1&-1&3&-4&3\\3&-3&5&-4&1\\2&-2&3&-2&0\\3&-3&4&-2&-1\end{pmatrix}\mathrm{的标准型}\left(\;\;\;\;\right)
$$
$$
A.
\begin{pmatrix}1&0&0&0&0\\0&1&0&0&0\\0&0&0&0&0\\0&0&0&0&0\end{pmatrix} \quad B.\begin{pmatrix}1&0&0&0&0\\0&1&0&0&0\\0&0&1&0&0\\0&0&0&0&0\end{pmatrix} \quad C.\begin{pmatrix}1&0&0&0&0\\0&1&0&0&0\\0&0&1&0&0\\0&0&0&1&0\end{pmatrix} \quad D.\begin{pmatrix}1&0&0&0&0\\0&0&0&0&0\\0&0&0&0&0\\0&0&0&0&0\end{pmatrix} \quad E. \quad F. \quad G. \quad H.
$$
$$
\begin{array}{l}\begin{pmatrix}1&-1&3&-4&3\\3&-3&5&-4&1\\2&-2&3&-2&0\\3&-3&4&-2&-1\end{pmatrix}\xrightarrow[\begin{array}{c}r_3-2r_1\\r_4-3r_1\end{array}]{r_2-3r_1}\begin{pmatrix}1&-1&3&-4&3\\0&0&-4&8&-8\\0&0&-3&6&-6\\0&0&-5&10&-10\end{pmatrix}\\\xrightarrow[\begin{array}{c}r_3×\left(-\frac13\right)\\r_4×\left(-\frac15\right)\end{array}]{r_2×\left(-\frac14\right)}\begin{pmatrix}1&-1&3&-4&3\\0&0&1&-2&2\\0&0&1&-2&2\\0&0&1&-2&2\end{pmatrix}\xrightarrow[{}_{\begin{array}{c}r_3-r_2\\r_4-r_2\end{array}}]{r_1-3r_1}\begin{pmatrix}1&-1&0&2&-3\\0&0&1&-2&2\\0&0&0&0&0\\0&0&0&0&0\end{pmatrix}\\\rightarrow\begin{pmatrix}1&0&0&0&0\\0&1&0&0&0\\0&0&0&0&0\\0&0&0&0&0\end{pmatrix}.\end{array}
$$



$$
设F=\begin{pmatrix}2&1\\6&4\end{pmatrix},\;G=E_{12},\;H=E_2\left(2\right)\mathrm{都是}2\mathrm{阶初等方阵},则GF^{-1}H\mathrm{等于}\left(\right).\;
$$
$$
A.
\begin{pmatrix}6&4\\4&-2\end{pmatrix} \quad B.\begin{pmatrix}-3&2\\2&-1\end{pmatrix} \quad C.\begin{pmatrix}-1&4\\1&-3\end{pmatrix} \quad D.\begin{pmatrix}-6&4\\4&-2\end{pmatrix} \quad E. \quad F. \quad G. \quad H.
$$
$$
GF^{-1}H=E_{12}\begin{pmatrix}2&-\frac12\\-3&1\end{pmatrix}E_2\left(2\right)=\begin{pmatrix}-3&1\\2&-\frac12\end{pmatrix}E_2\left(2\right)=\begin{pmatrix}-3&2\\2&-1\end{pmatrix}.
$$



$$
\begin{array}{l}\mathrm{已知}A,B\mathrm{均是三阶矩阵},将A\mathrm{中第}3\mathrm{行的}-2\mathrm{倍加至第}2\mathrm{行得到矩阵}A_1,\;将B\mathrm{中第}2\mathrm{列加至第}1\mathrm{列得到矩阵}\;B_1\\\mathrm{又知}A_1B_1=\begin{pmatrix}1&1&1\\0&2&2\\0&0&3\end{pmatrix},则AB=\left(\;\;\;\;\;\;\right).\end{array}
$$
$$
A.
\begin{pmatrix}0&1&1\\-2&2&8\\0&0&3\end{pmatrix} \quad B.\begin{pmatrix}3&1&1\\2&2&8\\0&1&3\end{pmatrix} \quad C.\begin{pmatrix}1&0&1\\-2&4&8\\0&1&3\end{pmatrix} \quad D.\begin{pmatrix}1&2&1\\3&2&8\\0&1&3\end{pmatrix} \quad E. \quad F. \quad G. \quad H.
$$
$$
\begin{array}{l}\mathrm{由题设条件},令\;\\\;\;\;\;\;\;\;\;\;\;\;\;\;\;\;\;\;\;\;\;\;\;\;\;\;\;\;\;\;\;P=\begin{pmatrix}1&0&0\\0&1&-2\\0&0&1\end{pmatrix}\;,Q=\begin{pmatrix}1&0&0\\1&1&0\\0&0&1\end{pmatrix}\;,\\\;\;则\;\;\;\;\;\;A_1=PA,\;\;B_1=BQ⇒ A_1B_1=PABQ,\;\\\;\;\;\;\;\;\;\;\;\;\;\;AB=P^{-1}A_1B_1Q^{-1}\\\;\;\;\;\;\;\;\;\;\;\;\;\;\;\;\;\;\;\;\;\;\;\;\;=\;\;\begin{pmatrix}1&0&0\\0&1&2\\0&0&1\end{pmatrix}\;\begin{pmatrix}1&1&1\\0&2&2\\0&0&3\end{pmatrix}\begin{pmatrix}1&0&0\\-1&1&0\\0&0&1\end{pmatrix}=\begin{pmatrix}0&1&1\\-2&2&8\\0&0&3\end{pmatrix}.\end{array}
$$



$$
设A=\begin{pmatrix}a_{11}&a_{12}&a_{13}\\a_{21}&a_{22}&a_{23}\\a_{31}&a_{32}&a_{33}\end{pmatrix},\;B=\begin{pmatrix}a_{21}&a_{22}+ka_{23}&a_{23}\\a_{31}&a_{32}+ka_{33}&a_{33}\\a_{11}&a_{12}+ka_{13}&a_{13}\end{pmatrix},\;P_1=\begin{pmatrix}0&1&0\\0&0&1\\1&0&0\end{pmatrix},\;P_2=\begin{pmatrix}1&0&0\\0&1&0\\0&k&1\end{pmatrix},\;则A=\left(\right).
$$
$$
A.
P_1^{-1}BP_2^{-1} \quad B.P_2^{-1}BP_1^{-1} \quad C.P_1^{-1}P_2^{-1}B \quad D.BP_1^{-1}P_2^{-1} \quad E. \quad F. \quad G. \quad H.
$$
$$
\mathrm{由于}P_1^{-1}=\begin{pmatrix}0&0&1\\1&0&0\\0&1&0\end{pmatrix},\;P_2^{-1}=\begin{pmatrix}1&0&0\\0&1&0\\0&-k&1\end{pmatrix},故P_1^{-1}BP_2^{-1}=A.
$$



$$
\begin{array}{l}\mathrm{设有矩阵}A=\begin{pmatrix}3&0&1\\1&-1&2\\0&1&1\end{pmatrix},\;\mathrm{则下列计算正确的是}\left(\right).\\\left(1\right)E_{12}A=\begin{pmatrix}1&-1&2\\3&0&1\\0&1&1\end{pmatrix},\left(2\right)AE_{3\;1}\left(2\right)=\begin{pmatrix}5&0&1\\5&-1&2\\2&1&1\end{pmatrix}.\end{array}
$$
$$
A.
\left(1\right) \quad B.\left(2\right) \quad C.\left(1\right)\left(2\right) \quad D.\mathrm{两个都不正确} \quad E. \quad F. \quad G. \quad H.
$$
$$
\begin{array}{l}\mathrm{矩阵}A=\begin{pmatrix}3&0&1\\1&-1&2\\0&1&1\end{pmatrix},\;而\;E_{12}=\begin{pmatrix}0&1&0\\1&0&0\\0&0&1\end{pmatrix},\;E_{3\;1}\left(2\right)=\begin{pmatrix}1&0&0\\0&1&0\\2&0&1\end{pmatrix},\;\\则E_{12}A=\begin{pmatrix}0&1&0\\1&0&0\\0&0&1\end{pmatrix}\begin{pmatrix}3&0&1\\1&-1&2\\0&1&1\end{pmatrix}=\begin{pmatrix}1&-1&2\\3&0&1\\0&1&1\end{pmatrix},\\\mathrm{即用}E_{12}\mathrm{左乘}A,\mathrm{相当于交换矩阵}A\mathrm{的第}1\mathrm{与第}2行.又\\A_{3\;1}\left(2\right)=\begin{pmatrix}3&0&1\\1&-1&2\\0&1&1\end{pmatrix}\begin{pmatrix}1&0&0\\0&1&0\\2&0&1\end{pmatrix}=\begin{pmatrix}5&0&1\\5&-1&2\\2&1&1\end{pmatrix},\\\mathrm{即用}E_{3\;1}\left(2\right)\mathrm{右乘}A,\mathrm{相当于将矩阵}A\mathrm{的第}3\mathrm{列乘}2\mathrm{加于第}1列.\end{array}
$$



$$
设A=\begin{pmatrix}2&3&0\\1&-1&1\\-2&1&3\end{pmatrix},\;B=\begin{pmatrix}1&2&3\\2&1&0\\0&1&1\end{pmatrix},\;C=\begin{pmatrix}1&0&0\\1&1&0\\0&0&1\end{pmatrix}=E_{21}\left(1\right),则\left|ACB\right|=\left(\right).
$$
$$
A.
51 \quad B.-69 \quad C.-51 \quad D.69 \quad E. \quad F. \quad G. \quad H.
$$
$$
ACB=\begin{pmatrix}5&3&0\\0&-1&1\\-1&1&3\end{pmatrix}\begin{pmatrix}1&2&3\\2&1&0\\0&1&1\end{pmatrix},\;\left|ACB\right|=\begin{vmatrix}5&3&0\\0&-1&1\\-1&1&3\end{vmatrix}\begin{vmatrix}1&2&3\\2&1&0\\0&1&1\end{vmatrix}=-23×3=-69.
$$



$$
\begin{array}{l}设A=\begin{pmatrix}3&-1&1&-1\\1&-2&-2&1\\-2&3&-3&1\end{pmatrix},\;B=\begin{pmatrix}4&2&2\\2&3&1\\1&2&1\\3&3&2\end{pmatrix},\;C=\begin{pmatrix}-2&3&1\\1&-2&3\\3&1&-2\end{pmatrix},\;则\left(AB+2C\right)E_{13}=\left(\right)\\(\mathrm{其中}E_{13}\mathrm{是三阶初等方阵}).\;\end{array}
$$
$$
A.
\begin{pmatrix}4&6&8\\-3&6&9\\4&-6&4\end{pmatrix} \quad B.\begin{pmatrix}4&6&8\\3&6&-9\\-4&6&4\end{pmatrix} \quad C.\begin{pmatrix}6&8&4\\6&-9&3\\-6&4&4\end{pmatrix} \quad D.\begin{pmatrix}4&6&8\\3&6&-9\\-4&6&-4\end{pmatrix} \quad E. \quad F. \quad G. \quad H.
$$
$$
\left(AB+2C\right)E_{13}=\left[\begin{pmatrix}8&2&4\\1&-5&0\\-2&2&-2\end{pmatrix}+\begin{pmatrix}-4&6&2\\2&-4&6\\6&2&-4\end{pmatrix}\right]E_{13}=\begin{pmatrix}6&8&4\\6&-9&3\\-6&4&4\end{pmatrix}
$$



$$
设E_2\left(3\right),\;E_{12}\left(1\right)\mathrm{都是}3\mathrm{阶初等方阵},且E_{12}\left(1\right)·F·E_2\left(3\right)=\begin{pmatrix}2&-9&-3\\3&6&9\\1&3&3\end{pmatrix},则\;F\mathrm{等于}\left(\right).
$$
$$
A.
\begin{pmatrix}11&-9&-3\\-1&2&3\\-2&3&3\end{pmatrix} \quad B.\begin{pmatrix}1&-11&-6\\1&2&3\\1&3&3\end{pmatrix} \quad C.\begin{pmatrix}-1&-5&-12\\3&2&9\\1&1&3\end{pmatrix} \quad D.\begin{pmatrix}2&-3&-3\\1&5&12\\1&1&3\end{pmatrix} \quad E. \quad F. \quad G. \quad H.
$$
$$
\begin{array}{l}E_{12}\left(1\right)·F·E_2\left(3\right)=\begin{pmatrix}2&-9&-3\\3&6&9\\1&3&3\end{pmatrix},\;则\\F=E_{12}^{-1}\left(1\right)\begin{pmatrix}2&-9&-3\\3&6&9\\1&3&3\end{pmatrix}E_2^{-1}\left(3\right)=E_{12}\left(-1\right)\begin{pmatrix}2&-9&-3\\3&6&9\\1&3&3\end{pmatrix}E_2\left(\frac13\right)\\=\begin{pmatrix}-1&-15&-12\\3&6&9\\1&3&3\end{pmatrix}E\left(2\left(\frac13\right)\right)=\begin{pmatrix}-1&-5&-12\\3&2&9\\1&1&3\end{pmatrix}\end{array}
$$



$$
\begin{array}{l}设A=\begin{pmatrix}3&-1&1&-1\\1&-2&-2&1\\-2&3&-3&1\end{pmatrix},\;B=\begin{pmatrix}4&2&2\\2&3&1\\1&2&1\\3&3&2\end{pmatrix},\;C=\begin{pmatrix}-2&3&1\\1&-2&3\\3&1&-2\end{pmatrix},\;则\left(AB+2C\right)E_{23}=\left(\right)\\(\mathrm{其中}E_{23}\mathrm{是三阶初等方阵}).\;\end{array}
$$
$$
A.
\begin{pmatrix}4&6&8\\-3&6&9\\4&-6&4\end{pmatrix} \quad B.\begin{pmatrix}4&6&8\\3&6&-9\\-4&6&4\end{pmatrix} \quad C.\begin{pmatrix}4&6&8\\3&6&-9\\4&-6&4\end{pmatrix} \quad D.\begin{pmatrix}4&6&8\\3&6&-9\\-4&6&-4\end{pmatrix} \quad E. \quad F. \quad G. \quad H.
$$
$$
\left(AB+2C\right)E_{23}=\left[\begin{pmatrix}8&2&4\\1&-5&0\\-2&2&-2\end{pmatrix}+\begin{pmatrix}-4&6&2\\2&-4&6\\6&2&-4\end{pmatrix}\right]E_{23}=\begin{pmatrix}4&6&8\\3&6&-9\\4&-6&4\end{pmatrix}
$$



$$
设A=\begin{pmatrix}1&2&-3\\-2&4&1\\3&-1&2\end{pmatrix},\;B=\begin{pmatrix}2&4&0\\3&1&2\\0&2&3\end{pmatrix},\;C=E_{3\;1}\left(2\right),\;\mathrm{其中}C\mathrm{是三阶初等方阵},则ABC及CAB\mathrm{分别为}\left(\right).
$$
$$
A.
\begin{pmatrix}-2&0&-5\\30&-2&11\\11&15&4\end{pmatrix},\;\begin{pmatrix}8&0&-5\\8&-2&11\\19&15&-6\end{pmatrix} \quad B.\begin{pmatrix}8&0&-5\\8&-2&11\\19&15&-6\end{pmatrix},\;\begin{pmatrix}-2&0&-5\\30&-2&11\\11&15&4\end{pmatrix} \quad C.\begin{pmatrix}14&30&3\\8&-2&11\\3&15&4\end{pmatrix},\;\begin{pmatrix}8&0&11\\8&-2&27\\3&15&10\end{pmatrix} \quad D.\begin{pmatrix}8&0&11\\8&-2&27\\3&15&10\end{pmatrix},\;\begin{pmatrix}14&30&3\\8&-2&11\\3&15&4\end{pmatrix} \quad E. \quad F. \quad G. \quad H.
$$
$$
\begin{array}{l}AB=\begin{pmatrix}8&0&-5\\8&-2&11\\3&15&4\end{pmatrix},\;ABC=\begin{pmatrix}-2&0&-5\\30&-2&11\\11&15&4\end{pmatrix},\\CAB=\begin{pmatrix}8&0&-5\\8&-2&11\\19&15&-6\end{pmatrix}\end{array}
$$



$$
\begin{array}{l}设A为3\mathrm{阶矩阵},将A\mathrm{的第一行加到第三行得到}B,\mathrm{再将}B\mathrm{的第一行和第三行交换得到}C,记\\P_1=\begin{pmatrix}0&0&1\\0&1&0\\1&0&0\end{pmatrix},P_2\begin{pmatrix}1&0&0\\0&1&0\\1&0&1\end{pmatrix},\mathrm{则必有}(\;\;)\end{array}
$$
$$
A.
AP_1P_2=C \quad B.P_1P_2A=C \quad C.AP_2P_1=C \quad D.P_2P_1A=C \quad E. \quad F. \quad G. \quad H.
$$
$$
\mathrm{根据初等矩阵和初等变换的有关内容即可}
$$



$$
设\begin{pmatrix}5&2&0\\2&1&0\\0&0&1\end{pmatrix}X=\begin{pmatrix}5&2&0\\2&1&0\\0&-3&1\end{pmatrix},则X=(\;\;).
$$
$$
A.
\begin{pmatrix}1&0&0\\0&1&0\\0&-3&1\end{pmatrix} \quad B.\begin{pmatrix}1&0&-3\\0&1&0\\0&0&1\end{pmatrix} \quad C.\begin{pmatrix}1&0&0\\0&-3&0\\0&0&1\end{pmatrix} \quad D.\begin{pmatrix}-3&0&0\\0&1&0\\0&0&1\end{pmatrix} \quad E. \quad F. \quad G. \quad H.
$$
$$
\begin{array}{l}\mathrm{矩阵}\begin{pmatrix}5&2&0\\2&1&0\\0&-3&1\end{pmatrix}\mathrm{是由}\begin{pmatrix}5&2&0\\2&1&0\\0&0&1\end{pmatrix}\mathrm{的第三列乘以}-3\mathrm{加到第二列得到的},\mathrm{由初等变换和初等矩阵的知识},\mathrm{可知}\\X=\begin{pmatrix}1&0&0\\0&1&0\\0&-3&1\end{pmatrix}\end{array}
$$



$$
若X\begin{pmatrix}2&1&1\\3&1&1\\2&7&8\end{pmatrix}=\begin{pmatrix}2&7&8\\3&1&1\\2&1&1\end{pmatrix},则X=(\;).
$$
$$
A.
\begin{pmatrix}0&0&1\\0&1&0\\1&0&0\end{pmatrix} \quad B.\begin{pmatrix}1&0&1\\0&1&0\\1&0&0\end{pmatrix} \quad C.\begin{pmatrix}0&1&1\\0&1&0\\1&0&0\end{pmatrix} \quad D.\begin{pmatrix}0&1&0\\0&0&1\\1&0&0\end{pmatrix} \quad E. \quad F. \quad G. \quad H.
$$
$$
\mathrm{相当于把}\begin{pmatrix}2&1&1\\3&1&1\\2&7&8\end{pmatrix}\mathrm{交换第一行和第三行},\mathrm{所以}X=\begin{pmatrix}0&0&1\\0&1&0\\1&0&0\end{pmatrix}
$$



$$
\mathrm{若矩阵}\begin{pmatrix}1&a&-1&2\\1&-1&a&2\\1&0&-1&2\end{pmatrix}\mathrm{的秩为}2,则a\mathrm{的值为}(\;).
$$
$$
A.
0 \quad B.0或-1 \quad C.-1 \quad D.-1或1 \quad E. \quad F. \quad G. \quad H.
$$
$$
\begin{pmatrix}1&a&-1&2\\1&-1&a&2\\1&0&-1&2\end{pmatrix}\rightarrow\begin{pmatrix}1&a&-1&2\\0&-1-a&a+1&0\\0&-a&0&0\end{pmatrix},\mathrm{侧当}-a=0或a+1=0⇒ a=-1时,\mathrm{矩阵的秩为}2.
$$



$$
\mathrm{矩阵}A=\begin{pmatrix}1&0&0&1\\1&2&0&-1\\3&-1&0&4\\1&4&5&1\end{pmatrix}\mathrm{的秩为}(\;\;).
$$
$$
A.
1 \quad B.2 \quad C.3 \quad D.4 \quad E. \quad F. \quad G. \quad H.
$$
$$
\begin{array}{l}A=\begin{pmatrix}1&0&0&1\\1&2&0&-1\\3&-1&0&4\\1&4&5&1\end{pmatrix}\rightarrow\begin{pmatrix}1&0&0&1\\0&2&0&-2\\0&-1&0&1\\0&4&5&0\end{pmatrix}\rightarrow\begin{pmatrix}1&0&0&1\\0&1&0&-1\\0&0&5&4\\0&0&0&0\end{pmatrix}\\故R(A)=3.\end{array}
$$



$$
设A=\begin{pmatrix}1&2&3\\-3&0&1\\2&1&1\end{pmatrix},则R(A)=(\;).
$$
$$
A.
0 \quad B.2 \quad C.1 \quad D.3 \quad E. \quad F. \quad G. \quad H.
$$
$$
\begin{array}{l}A=\begin{pmatrix}1&2&3\\-3&0&1\\2&1&1\end{pmatrix}\rightarrow\begin{pmatrix}1&2&3\\0&6&10\\0&-3&-5\end{pmatrix}\rightarrow\begin{pmatrix}1&2&3\\0&3&5\\0&0&0\end{pmatrix},\\\mathrm{可见}R(A)=2.\end{array}
$$



$$
设A=\begin{pmatrix}1&-2&0&3\\2&-4&0&6\\-1&1&2&0\\0&-1&2&3\end{pmatrix},则R(A)=(\;).
$$
$$
A.
1 \quad B.2 \quad C.3 \quad D.4 \quad E. \quad F. \quad G. \quad H.
$$
$$
对A\mathrm{做初等行变换}:A\rightarrow\begin{pmatrix}1&-2&0&3\\0&-1&2&3\\0&0&0&0\\0&0&0&0\end{pmatrix},故R(A)=2.
$$



$$
设A=\begin{pmatrix}1&4&5&2\\2&1&3&0\\-1&3&2&2\end{pmatrix},\mathrm{则矩阵}A\mathrm{的秩为}(\;).
$$
$$
A.
1 \quad B.2 \quad C.3 \quad D.4 \quad E. \quad F. \quad G. \quad H.
$$
$$
A\rightarrow\begin{pmatrix}1&4&5&2\\0&7&7&4\\0&0&0&0\end{pmatrix},故R(A)=2.
$$



$$
\mathrm{矩阵}\begin{pmatrix}0&2&-3&1\\0&3&-4&3\\0&4&-7&-1\end{pmatrix}\mathrm{的秩为}(\;).
$$
$$
A.
2 \quad B.3 \quad C.1 \quad D.0 \quad E. \quad F. \quad G. \quad H.
$$
$$
\begin{pmatrix}0&2&-3&1\\0&3&-4&3\\0&4&-7&-1\end{pmatrix}\rightarrow\begin{pmatrix}0&2&-3&1\\0&0&1&3\\0&0&-1&-3\end{pmatrix}\rightarrow\begin{pmatrix}0&1&0&5\\0&0&1&3\\0&0&0&0\end{pmatrix}\rightarrow\begin{pmatrix}1&0&0&0\\0&1&0&0\\0&0&0&0\end{pmatrix},\mathrm{所以秩为}2
$$



$$
\mathrm{已知矩阵}A=\begin{pmatrix}1&2&3\\2&7&-1\\2&4&6\end{pmatrix},\mathrm{则矩阵}A\mathrm{的秩为}(\;)
$$
$$
A.
1 \quad B.2 \quad C.3 \quad D.0 \quad E. \quad F. \quad G. \quad H.
$$
$$
\mathrm{矩阵有一个二阶子式}\begin{vmatrix}1&2\\2&7\end{vmatrix}=3\neq0,\mathrm{而它的第一行和第三行成比例},\mathrm{即行列式等于}0,\mathrm{由秩的定义知},\mathrm{该矩阵的秩等于}2
$$



$$
\mathrm{已知矩阵}A=\begin{pmatrix}1&1&1\\1&2&3\\2&3&4\end{pmatrix},\mathrm{则矩阵}A\mathrm{的秩为}(\;)
$$
$$
A.
1 \quad B.2 \quad C.3 \quad D.0 \quad E. \quad F. \quad G. \quad H.
$$
$$
\begin{pmatrix}1&1&1\\1&2&3\\2&3&4\end{pmatrix}\rightarrow\begin{pmatrix}1&1&1\\0&1&2\\0&0&0\end{pmatrix},\mathrm{所以秩}=2
$$



$$
\mathrm{已知矩阵}A=\begin{pmatrix}1&0&3&3\\2&1&1&2\\-1&-1&2&1\end{pmatrix},\mathrm{则矩阵}A\mathrm{的秩为}(\;)
$$
$$
A.
1 \quad B.2 \quad C.3 \quad D.4 \quad E. \quad F. \quad G. \quad H.
$$
$$
A=\begin{pmatrix}1&0&3&3\\2&1&1&2\\-1&-1&2&1\end{pmatrix}\rightarrow\begin{pmatrix}1&0&3&3\\0&1&-5&-4\\0&0&0&0\end{pmatrix},\mathrm{所以秩}=2
$$



$$
\mathrm{设矩阵}A=\begin{pmatrix}k&1&1&1\\1&k&1&1\\1&1&k&1\\1&1&1&k\end{pmatrix}\mathrm{的秩}=4,则k\mathrm{满足}(\;).
$$
$$
A.
k\neq1 \quad B.k=1 \quad C.k=-3 \quad D.k\neq1且k\neq-3 \quad E. \quad F. \quad G. \quad H.
$$
$$
\begin{array}{l}\begin{array}{l}A=\begin{pmatrix}k&1&1&1\\1&k&1&1\\1&1&k&1\\1&1&1&k\end{pmatrix}⇾\begin{pmatrix}k+3&k+3&k+3&k+3\\1&k&1&1\\1&1&k&1\\1&1&1&k\end{pmatrix}=B\\当k=-3时,B⇾\begin{pmatrix}0&0&0&0\\1&-3&1&1\\1&1&-3&1\\1&1&1&-3\end{pmatrix}⇾\begin{pmatrix}1&1&1&-3\\0&-4&0&4\\0&0&-4&4\\0&0&0&0\end{pmatrix},\;当k\neq-3时,B⇾\begin{pmatrix}1&1&1&1\\1&k&1&1\\1&1&k&1\\1&1&1&k\end{pmatrix}⇾\begin{pmatrix}1&1&1&1\\0&k-1&0&0\\0&0&k-1&0\\0&0&0&k-1\end{pmatrix}\\当k=1时,\;R(A)=R(B)=1;\mathrm{故当}k\neq1且k\neq-3时,R(A)=R(B)=4\end{array}\\\\\\\\\end{array}
$$



$$
\mathrm{设矩阵}A=\begin{pmatrix}0&1&0&0\\0&0&1&0\\0&0&0&1\\0&0&0&0\end{pmatrix},则A^2\mathrm{的秩等于}(\;)
$$
$$
A.
4 \quad B.3 \quad C.2 \quad D.1 \quad E. \quad F. \quad G. \quad H.
$$
$$
\mathrm{因为}A^2=\begin{pmatrix}0&0&1&0\\0&0&0&1\\0&0&0&0\\0&0&0&0\end{pmatrix},故R(A^2)=2
$$



$$
\mathrm{已知矩阵}A=\begin{pmatrix}1&2&3\\3&6&10\\2&5&7\\1&2&4\end{pmatrix},则R(A)=(\;)
$$
$$
A.
1 \quad B.2 \quad C.3 \quad D.4 \quad E. \quad F. \quad G. \quad H.
$$
$$
A=\begin{pmatrix}1&2&3\\3&6&10\\2&5&7\\1&2&4\end{pmatrix}\rightarrow\begin{pmatrix}1&2&3\\0&0&1\\0&1&1\\0&0&1\end{pmatrix}\rightarrow\begin{pmatrix}1&2&3\\0&1&0\\0&0&1\\0&0&0\end{pmatrix},故R(A)=3.
$$



$$
设A=\begin{pmatrix}k&1&1&1\\1&k&1&1\\1&1&k&1\\1&1&1&k\end{pmatrix}\mathrm{的秩为}1,则k\mathrm{满足}(\;)
$$
$$
A.
k\neq1 \quad B.k=1 \quad C.k=3 \quad D.k=-3 \quad E. \quad F. \quad G. \quad H.
$$
$$
\begin{array}{l}A=\begin{pmatrix}k&1&1&1\\1&k&1&1\\1&1&k&1\\1&1&1&k\end{pmatrix}\rightarrow\begin{pmatrix}k+3&k+3&k+3&k+3\\1&k&1&1\\1&1&k&1\\1&1&1&k\end{pmatrix}=B\\当k=-3时,B\rightarrow\begin{pmatrix}0&0&0&0\\1&-3&1&1\\1&1&-3&1\\1&1&1&-3\end{pmatrix}\rightarrow\begin{pmatrix}1&1&1&-3\\0&-4&0&4\\0&0&-4&4\\0&0&0&0\end{pmatrix}\\当k\neq-3时,B\rightarrow\begin{pmatrix}1&1&1&1\\1&k&1&1\\1&1&k&1\\1&1&1&k\end{pmatrix}\rightarrow\begin{pmatrix}1&1&1&1\\0&k-1&0&0\\0&0&k-1&0\\0&0&0&k-1\end{pmatrix}\\当k=1时,\;R(A)=R(B)=1;当k\neq1,且k\neq-3时,\;R(A)=R(B)=4\end{array}
$$



$$
\mathrm{设矩阵}A=\begin{pmatrix}0&1&0&0\\0&0&1&0\\0&0&0&1\\0&0&0&0\end{pmatrix},则A^4\mathrm{的秩等于}(\;)
$$
$$
A.
0 \quad B.3 \quad C.2 \quad D.1 \quad E. \quad F. \quad G. \quad H.
$$
$$
\mathrm{因为}A^2=\begin{pmatrix}0&0&1&0\\0&0&0&1\\0&0&0&0\\0&0&0&0\end{pmatrix},A^4=\begin{pmatrix}0&0&0&0\\0&0&0&0\\0&0&0&0\\0&0&0&0\end{pmatrix},故R(A^4)=0
$$



$$
设A=\begin{pmatrix}1&λ&-1&2\\2&-1&λ&5\\1&10&-6&1\end{pmatrix}\mathrm{的秩等于}2,则λ\mathrm{满足}(\;)
$$
$$
A.
λ=3 \quad B.λ\neq3 \quad C.λ=-3 \quad D.λ=4 \quad E. \quad F. \quad G. \quad H.
$$
$$
\begin{array}{l}\mathrm{对矩阵施以初等行变换}:\begin{pmatrix}1&λ&-1&2\\2&-1&λ&5\\1&10&-6&1\end{pmatrix}\rightarrow\begin{pmatrix}1&10&-6&1\\0&-21&λ+12&3\\0&λ-10&5&1\end{pmatrix},\\\mathrm{由已知条件得},\frac{-21}{λ-10}=\frac{λ+12}5=\frac31,\mathrm{解得}:λ=3\end{array}
$$



$$
设A=\begin{pmatrix}1&-1&2\\2&1&3\\3&3k&1\end{pmatrix},且R(A)=3,则k\mathrm{满足}(\;)
$$
$$
A.
k\neq3 \quad B.k\neq-3 \quad C.k\neq4 \quad D.k\neq-4 \quad E. \quad F. \quad G. \quad H.
$$
$$
\begin{array}{l}\begin{pmatrix}1&-1&2\\2&1&3\\3&3k&1\end{pmatrix}\rightarrow\begin{pmatrix}1&-1&2\\0&3&-1\\0&3k+3&-5\end{pmatrix}\rightarrow\begin{pmatrix}1&-1&2\\0&3&-1\\0&0&k-4\end{pmatrix},\\当k-4\neq0,\mathrm{即当}k\neq4时,R(A)=3.\end{array}
$$



$$
设A=\begin{pmatrix}1&-1&2\\2&1&3\\4&k&1\end{pmatrix},且R(A)=2,则k\mathrm{满足}(\;)
$$
$$
A.
k=17 \quad B.k=-17 \quad C.k\neq17 \quad D.k\neq-17 \quad E. \quad F. \quad G. \quad H.
$$
$$
\begin{array}{l}\begin{pmatrix}1&-1&2\\2&1&3\\4&k&1\end{pmatrix}\rightarrow\begin{pmatrix}1&-1&2\\0&3&-1\\0&k+4&-7\end{pmatrix},\mathrm{因为}R(A)=2,\mathrm{所以}\frac3{k+4}=\frac17\\即k=17时,R(A)=2<3\end{array}
$$



$$
设A=\begin{pmatrix}1&0&1&1\\-3&3&7&1\\-1&3&9&3\\-5&3&5&-1\end{pmatrix},\mathrm{则矩阵}A\mathrm{的秩为}(\;).
$$
$$
A.
1 \quad B.2 \quad C.3 \quad D.4 \quad E. \quad F. \quad G. \quad H.
$$
$$
对A\mathrm{作初等行变换}:A\rightarrow\begin{pmatrix}1&0&1&1\\0&3&10&4\\0&0&0&0\\0&0&0&0\end{pmatrix},故R(A)=2.
$$



$$
\mathrm{矩阵}A=\begin{pmatrix}a&a+1&3a+1\\2&1&5\\3&2&8\end{pmatrix}\mathrm{的秩为}(\;).
$$
$$
A.
1 \quad B.2 \quad C.3 \quad D.与a\mathrm{有关} \quad E. \quad F. \quad G. \quad H.
$$
$$
\begin{array}{l}\begin{pmatrix}a&a+1&3a+1\\2&1&5\\3&2&8\end{pmatrix}\rightarrow\begin{pmatrix}2&1&5\\3&2&8\\a&a+1&3a+1\end{pmatrix}\rightarrow\begin{pmatrix}1&1&3\\2&1&5\\a&a+1&3a+1\end{pmatrix}\rightarrow\begin{pmatrix}1&1&3\\0&1&1\\0&0&0\end{pmatrix},\\\mathrm{故不论}a\mathrm{取何值},A\mathrm{的秩为}2.\end{array}
$$



$$
设A=\begin{pmatrix}1&-1&2\\2&1&3\\4&k&1\end{pmatrix},若R(A)=3,则k为(\;).
$$
$$
A.
k=17 \quad B.k\neq17 \quad C.k=4 \quad D.k\neq4 \quad E. \quad F. \quad G. \quad H.
$$
$$
\begin{array}{l}\begin{pmatrix}1&-1&2\\2&1&3\\4&k&1\end{pmatrix}∼\begin{pmatrix}1&-1&2\\0&3&-1\\0&k+4&-7\end{pmatrix}∼\begin{pmatrix}1&-1&2\\0&3&-1\\0&0&-7+\frac{k+4}3\end{pmatrix},\\当-7+\frac{k+4}3=0,即k=17时,R(A)=2<3;当k\neq17时,R(A)=3.\end{array}
$$



$$
设A=\begin{pmatrix}1&-2&-3&2&-4\\-3&7&-1&1&-3\\2&-5&4&-3&7\\-3&6&9&-6&-1\end{pmatrix},\mathrm{则矩阵}A\mathrm{的秩为}(\;).
$$
$$
A.
1 \quad B.2 \quad C.3 \quad D.4 \quad E. \quad F. \quad G. \quad H.
$$
$$
\begin{array}{l}对A\mathrm{做初等行变换}:A\rightarrow\begin{pmatrix}1&-2&-3&2&-4\\0&1&-10&7&-15\\0&0&0&0&0\\0&0&0&0&-13\end{pmatrix},\\\mathrm{故秩}(A)=3.\end{array}
$$



$$
设A=\begin{pmatrix}1&-3&2&2&1\\0&3&6&0&-3\\2&-3&-2&4&4\\3&-6&0&6&5\\-2&9&2&-4&-5\end{pmatrix},\mathrm{则矩阵}A\mathrm{的秩为}(\;)
$$
$$
A.
2 \quad B.3 \quad C.4 \quad D.5 \quad E. \quad F. \quad G. \quad H.
$$
$$
\begin{array}{l}对A\mathrm{做初等行变换}:A\rightarrow\begin{pmatrix}1&-3&2&2&1\\0&1&2&0&-1\\0&0&-12&0&5\\0&0&0&0&0\\0&0&0&0&0\end{pmatrix},\\故R(A)=3\end{array}
$$



$$
\mathrm{矩阵}B=\begin{pmatrix}1&2&3&4&5&6\\2&3&4&5&6&7\\3&4&5&6&7&8\\4&5&6&7&8&9\\5&6&7&8&9&10\end{pmatrix}\mathrm{的秩为}(\;).
$$
$$
A.
1 \quad B.2 \quad C.4 \quad D.5 \quad E. \quad F. \quad G. \quad H.
$$
$$
\begin{array}{l}B\frac{r_{i+1}-r_i}{i=4,3,2,1}\begin{pmatrix}1&2&3&4&5&6\\1&1&1&1&1&1\\1&1&1&1&1&1\\1&1&1&1&1&1\\1&1&1&1&1&1\end{pmatrix}\rightarrow\begin{pmatrix}0&1&2&3&4&5\\1&1&1&1&1&1\\0&0&0&0&0&0\\0&0&0&0&0&0\\0&0&0&0&0&0\end{pmatrix},\\\mathrm{显然}R(B)=2.\end{array}
$$



$$
\mathrm{已知矩阵}A=\begin{pmatrix}1&1&-6&-10\\2&5&k&19\\1&2&-1&k\end{pmatrix}\mathrm{的秩为}2,则k=(\;).
$$
$$
A.
0 \quad B.1 \quad C.2 \quad D.3 \quad E. \quad F. \quad G. \quad H.
$$
$$
\begin{array}{l}\mathrm{对矩阵}A\mathrm{作初等行变换},\mathrm{即得}\\A=\begin{pmatrix}1&1&-6&-10\\2&5&k&19\\1&2&-1&k\end{pmatrix}\xrightarrow[{r_3-r_1}]{r_2-2r_1}\begin{pmatrix}1&1&-6&-10\\0&3&k+12&39\\0&1&5&k+10\end{pmatrix}\xrightarrow{r_2+(-3)r_3}\begin{bmatrix}1&1&-6&-10\\0&0&k-3&9-3k\\0&1&5&k+10\end{bmatrix},则\left\{\begin{array}{l}k-3=0\\9-3k=0\end{array}\mathrm{解得}k=3\right.\\\mathrm{由题设}A\mathrm{的秩为}2,则A\mathrm{的所有三阶子式为}0,\mathrm{而且至少有一个二阶子式不为}0.\mathrm{显然}\begin{vmatrix}1&1\\0&3\end{vmatrix}\neq0,且\\\begin{vmatrix}1&1&-6\\0&3&k+12\\0&1&5\end{vmatrix}=0⇒ k=3.\end{array}
$$



$$
设A=\begin{pmatrix}1&-1&1&2\\3&λ&-1&2\\5&3&μ&6\end{pmatrix},若R(A)=2,则(\;).
$$
$$
A.
λ=5,μ=1 \quad B.λ=1,μ=5 \quad C.λ=-5,μ=1 \quad D.λ=5,μ=-1 \quad E. \quad F. \quad G. \quad H.
$$
$$
\begin{array}{l}A\xrightarrow[{r_3-5r_1}]{r_2-3r_1}\begin{pmatrix}1&-1&1&2\\0&λ+3&-4&-4\\0&8&μ-5&-4\end{pmatrix}\xrightarrow{r_3-r_2}\begin{pmatrix}1&-1&1&2\\0&λ+3&-4&-4\\0&5-λ&μ-1&0\end{pmatrix},\\因R(A)=2,故\left\{\begin{array}{l}5-λ=0\\μ-1=0\end{array}\right.⇒\left\{\begin{array}{l}λ=5\\μ=1\end{array}\right..\end{array}
$$



$$
\mathrm{矩阵}A=\begin{pmatrix}3&2&0&5&0\\3&-2&3&6&-1\\2&0&1&5&-3\\1&6&-4&-1&4\end{pmatrix}\mathrm{的秩为}(\;).
$$
$$
A.
1 \quad B.2 \quad C.3 \quad D.4 \quad E. \quad F. \quad G. \quad H.
$$
$$
\begin{array}{l}对A\mathrm{作初等行变换},\mathrm{变成行阶梯形矩阵}.\\A\xrightarrow{r_1↔ r_4}\begin{pmatrix}1&6&-4&-1&4\\3&-2&3&6&-1\\2&0&1&5&-3\\3&2&0&5&0\end{pmatrix}\xrightarrow{r_2-r_4}\begin{pmatrix}1&6&-4&-1&4\\0&-4&3&1&-1\\2&0&1&5&-3\\3&2&0&5&0\end{pmatrix}\\\\\xrightarrow[{r_4-3r_1}]{r_3-2r_1}\begin{pmatrix}1&6&-4&-1&4\\0&-4&3&1&-1\\0&-12&9&7&-11\\0&-16&12&8&-12\end{pmatrix}\xrightarrow[{r_4-4r_2}]{r_3-3r_2}\begin{pmatrix}1&6&-4&-1&4\\0&-4&3&1&-1\\0&0&0&4&-8\\0&0&0&4&-8\end{pmatrix}\\\xrightarrow{r_4-r_3}\begin{pmatrix}1&6&-4&-1&4\\0&-4&3&1&-1\\0&0&0&4&-8\\0&0&0&0&0\end{pmatrix}\\\mathrm{由行阶梯形矩阵有三个非零行知}R(A)=3.\end{array}
$$



$$
\mathrm{设矩阵}A=\begin{pmatrix}1&λ&-1&2\\2&-1&λ&5\\1&10&-6&1\end{pmatrix},\mathrm{其中}λ\neq3,\mathrm{则矩阵}A\mathrm{的秩为}(\;).
$$
$$
A.
1 \quad B.2 \quad C.3 \quad D.4 \quad E. \quad F. \quad G. \quad H.
$$
$$
\begin{array}{l}对A\mathrm{作初等行变换},得\\A\rightarrow\begin{pmatrix}0&λ-10&5&1\\0&-21&λ+12&3\\1&10&-6&1\end{pmatrix}\rightarrow\begin{pmatrix}1&10&-6&1\\0&1&\frac{λ+12}{-21}&\frac1{-7}\\0&0&\frac{(λ+5)(λ-3)}{21}&\frac{λ-3}7\end{pmatrix}\\\;\;\;\;\;\;\;\;\;\;\;\;\;\;\;\;\;\;\;\;\;\;\;\;\;\;\;\;\;\;\;\;\;\;\;\;\;\;\;\;\;\;\\\;\;当λ=3时,R(A)=2;\;当λ\neq3时,R(A)=3.\end{array}
$$



$$
设A=\begin{pmatrix}1&2&3\\λ&0&1\\2&1&1\end{pmatrix},则R(A)\mathrm{的最小值为}\;(\;).
$$
$$
A.
1 \quad B.2 \quad C.3 \quad D.\mathrm{无法确定} \quad E. \quad F. \quad G. \quad H.
$$
$$
\begin{array}{l}\begin{vmatrix}A\end{vmatrix}=\begin{vmatrix}1&2&3\\λ&0&1\\2&1&1\end{vmatrix}=λ+3,当λ\neq-3时,\begin{vmatrix}A\end{vmatrix}\neq0,R(A)=3;\\当λ=-3时,A=\begin{pmatrix}1&2&3\\-3&0&1\\2&1&1\end{pmatrix}\rightarrow\begin{pmatrix}1&2&3\\0&6&10\\0&-3&-5\end{pmatrix}\rightarrow\begin{pmatrix}1&2&3\\0&6&10\\0&0&0\end{pmatrix},R(A)=2.\end{array}
$$



$$
\mathrm{矩阵}A=\begin{pmatrix}0&16&-7&-5&5\\1&-5&2&1&-1\\-1&-11&5&4&-4\\2&6&-3&-3&7\end{pmatrix}\mathrm{的秩}R(A)=(\;).
$$
$$
A.
1 \quad B.2 \quad C.3 \quad D.4 \quad E. \quad F. \quad G. \quad H.
$$
$$
\begin{array}{l}A=\begin{pmatrix}0&16&-7&-5&5\\1&-5&2&1&-1\\-1&-11&5&4&-4\\2&6&-3&-3&7\end{pmatrix}∼\begin{pmatrix}1&-5&2&1&-1\\0&16&-7&-5&5\\-1&-11&5&4&-4\\2&6&-3&-3&7\end{pmatrix}\\\;\;\;∼\begin{pmatrix}1&-5&2&1&-1\\0&16&-7&-5&5\\0&-16&7&5&-5\\0&16&-7&-5&9\end{pmatrix}∼\begin{pmatrix}1&-5&2&1&-1\\0&16&-7&-5&5\\0&0&0&0&4\\0&0&0&0&0\end{pmatrix}\\故R(A)=3.\end{array}
$$



$$
\mathrm{若矩阵}A=\begin{pmatrix}a+1&4&7\\2a-1&-1&2\\1&5&6\end{pmatrix}\mathrm{的秩最小},则a\mathrm{的取值为}(\;).
$$
$$
A.
a\neq-1 \quad B.a=-1 \quad C.a\neq2 \quad D.a=2 \quad E. \quad F. \quad G. \quad H.
$$
$$
\begin{array}{l}\begin{pmatrix}a+1&4&7\\2a-1&-1&2\\1&5&6\end{pmatrix}\rightarrow\begin{pmatrix}1&5&6\\a+1&4&7\\2a-1&-1&2\end{pmatrix}\rightarrow\begin{pmatrix}1&5&6\\0&5a+1&6a-1\\0&5a-2&6a-4\end{pmatrix}\\\;\;\;\;\;\;\;\;\;\;\;\;\;\;\;\;\;\;\rightarrow\begin{pmatrix}1&5&6\\0&5a+1&6a-1\\0&-3&-3\end{pmatrix}\rightarrow\begin{pmatrix}1&5&6\\0&1&1\\0&5a+1&6a-1\end{pmatrix}\rightarrow\begin{pmatrix}1&5&6\\0&1&1\\0&0&a-2\end{pmatrix},\\a\neq2,A\mathrm{的秩为}3;a=2,A\mathrm{的秩为}2.\end{array}
$$



$$
\mathrm{若矩阵}A=\begin{pmatrix}a&a+1&3-a\\2&1&-5\\a&-1&a+1\end{pmatrix}\mathrm{的秩为}3,则a\mathrm{的取值为}(\;).
$$
$$
A.
a\neq-2且a\neq-0.8 \quad B.a\neq-2 \quad C.a\neq-0.8 \quad D.a\neq-2或a\neq-0.8 \quad E. \quad F. \quad G. \quad H.
$$
$$
\begin{array}{l}\begin{array}{l}\begin{pmatrix}a&a+1&3-a\\2&1&-5\\a&-1&a+1\end{pmatrix}\rightarrow\begin{pmatrix}1&\frac12&-\frac52\\a&a+1&3-a\\a&-1&a+1\end{pmatrix}\rightarrow\begin{pmatrix}1&\frac12&-\frac52\\0&a+2&2-2a\\0&-2-a&7a+2\end{pmatrix}\rightarrow\begin{pmatrix}1&\frac12&-\frac52\\0&a+2&2-2a\\0&0&5a+4\end{pmatrix},\\当a\neq-2且a\neq-0.8时,A\mathrm{的秩为}3;\\\;当a=-2或a=-0.8时,A\mathrm{的秩为}2.\end{array}\\\end{array}
$$



$$
\mathrm{矩阵}A=\begin{pmatrix}x&y&y&y\\y&x&y&y\\y&y&x&y\\y&y&y&x\end{pmatrix}\mathrm{的秩的最小值和最大值分别为}(\;).
$$
$$
A.
1,4 \quad B.2,4 \quad C.0,4 \quad D.1,3 \quad E. \quad F. \quad G. \quad H.
$$
$$
\begin{array}{l}\mathrm{对矩阵施以初等行变换}\\\;\;\;\;\;\;\;\;\;\;\;\;\;\;\;\;\;\;\;\;\;\;\;A\rightarrow\begin{pmatrix}x+3y&x+3y&x+3y&x+3y\\y&x&y&y\\y&y&x&y\\y&y&y&x\end{pmatrix}\rightarrow\begin{pmatrix}1&1&1&1\\y&x&y&y\\y&y&x&y\\y&y&y&x\end{pmatrix}\rightarrow\begin{pmatrix}1&1&1&1\\0&x-y&0&0\\0&0&x-y&0\\0&0&0&x-y\end{pmatrix},\\(1)当x-y=0,x+3y=0即x=y=0时,A=O,\mathrm{所以}R(A)=0;\\(2)当x-y\neq0,x+3y\neq0时,\mathrm{所以}R(A)=4;\\(3)当x-y\neq0,x+3y=0时,\mathrm{所以}R(A)=3;\\(3)当x-y=0,x+3y\neq0时,\mathrm{所以}R(A)=1;\\\end{array}
$$



$$
设A=\begin{pmatrix}1&2&3&1\\2&-1&k&2\\0&1&1&3\\1&-1&0&4\\2&0&2&5\end{pmatrix}\mathrm{的秩等于}3,则k=(\;\;\;\;\;)
$$
$$
A.
-1 \quad B.1 \quad C.2 \quad D.-2 \quad E. \quad F. \quad G. \quad H.
$$
$$
\mathrm{定义法},\mathrm{因为}A\mathrm{的秩为},\mathrm{故其四阶子式}\begin{vmatrix}1&2&3&1\\2&-1&k&2\\0&1&1&3\\1&-1&0&4\end{vmatrix}=0,\mathrm{解得}:k=1.\;
$$



$$
设A=\begin{pmatrix}3&-2&λ&-16\\2&-3&0&1\\1&-1&1&-3\\3&μ&1&-2\end{pmatrix}\mathrm{的秩等于}4,则λ,μ\mathrm{满足}(\;\;\;\;)
$$
$$
A.
λ\neq5,μ\neq-4 \quad B.λ=5,μ=-4 \quad C.λ\neq5,μ=-4 \quad D.λ=5,μ\neq-4 \quad E. \quad F. \quad G. \quad H.
$$
$$
\begin{array}{l}对A\mathrm{作初等变换},得A\rightarrow\begin{pmatrix}1&-3&1&-1\\0&7&-2&-1\\0&0&λ-5&0\\0&0&0&μ+4\end{pmatrix},\\\mathrm{由条件得},当λ\neq5,μ\neq-4时,R(A)=4.\end{array}
$$



$$
设A=\begin{pmatrix}3&-2&λ&-16\\2&-3&0&1\\1&-1&1&-3\\3&μ&1&-2\end{pmatrix}\mathrm{的秩等于}2,则λ,μ\mathrm{满足}(\;\;\;\;)
$$
$$
A.
λ\neq5,μ\neq-4 \quad B.λ=5,μ=-4 \quad C.λ\neq5,μ=-4 \quad D.λ=5,μ\neq-4 \quad E. \quad F. \quad G. \quad H.
$$
$$
对A\mathrm{作初等变换},得A\rightarrow\begin{pmatrix}1&-3&1&-1\\0&7&-2&-1\\0&0&λ-5&0\\0&0&0&μ+4\end{pmatrix},\mathrm{由条件得},当λ=5,μ=-4时,R(A)=2.
$$



$$
设A=\begin{pmatrix}3&-2&λ&-16\\2&-3&0&1\\1&-1&1&-3\\3&μ&1&-2\end{pmatrix}\mathrm{的秩等于}3,则λ,μ\mathrm{满足}(\;\;\;\;)
$$
$$
A.
λ\neq5,μ\neq-4 \quad B.λ=5,μ=-4 \quad C.λ\neq5,μ=-4 \quad D.λ=5,μ\neq-4或λ\neq5,μ=-4 \quad E. \quad F. \quad G. \quad H.
$$
$$
\begin{array}{l}对A\mathrm{作初等变换},得A\rightarrow\begin{pmatrix}1&-3&1&-1\\0&7&-2&-1\\0&0&λ-5&0\\0&0&0&μ+4\end{pmatrix},\mathrm{由条件得},当λ=5,μ\neq-4或λ\neq5,μ=-4时,\\R(A)=3.\end{array}
$$



$$
\mathrm{已知}\;A=\begin{pmatrix}1&1&1&1\\0&1&-1&b\\2&3&a&4\\3&5&1&7\end{pmatrix},且R(A)=3,则a,b\;\mathrm{满足}(\;\;\;\;\;)
$$
$$
A.
a\neq1,b=2\;或\;\;\;\;a=1,b\neq2\; \quad B.a=1,b=1\; \quad C.a\neq2,b=1\;或\;\;\;\;a=1,b\neq2\; \quad D.a=1,b=2\;或\;\;\;\;a\neq1,b\neq2\; \quad E. \quad F. \quad G. \quad H.
$$
$$
A=\begin{pmatrix}1&1&1&1\\0&1&-1&b\\2&3&a&4\\3&5&1&7\end{pmatrix}\rightarrow\begin{pmatrix}1&1&1&1\\0&1&-1&b\\0&0&a-1&2-b\\0&0&0&4-2b\end{pmatrix},\mathrm{由于}R(A)=3,\mathrm{所以}a\neq1,b=2\;或\;\;\;\;a=1,b\neq2\;
$$



$$
\mathrm{设矩阵}A=\begin{pmatrix}0&1&0&0\\0&0&1&0\\0&0&0&1\\0&0&0&0\end{pmatrix},则A^k(k\geq4)\mathrm{的秩等于}(\;\;\;\;\;)\;
$$
$$
A.
0 \quad B.1 \quad C.2 \quad D.3 \quad E. \quad F. \quad G. \quad H.
$$
$$
\mathrm{因为}A^2=\begin{pmatrix}0&0&1&0\\0&0&0&1\\0&0&0&0\\0&0&0&0\end{pmatrix},\;A^3=\begin{pmatrix}0&0&0&1\\0&0&0&0\\0&0&0&0\\0&0&0&0\end{pmatrix},A^4=O,\;故R(A^k)=0\;\;\;\;(k\geq4)
$$



$$
\mathrm{若矩阵}A=\begin{pmatrix}2a&3a-1&a+1\\a+1&a+1&3a-1\\2&2a&a+1\end{pmatrix}\mathrm{的秩为}3,则a\mathrm{的取值为}\;(\;).
$$
$$
A.
a\neq1且a\neq\frac1{11} \quad B.a\neq1 \quad C.a\neq\frac1{11} \quad D.a\neq1或a\neq\frac1{11} \quad E. \quad F. \quad G. \quad H.
$$
$$
\begin{array}{l}\begin{pmatrix}2a&3a-1&a+1\\a+1&a+1&3a-1\\2&2a&a+1\end{pmatrix}\rightarrow\begin{pmatrix}4&a-1&0\\1-11a&0&0\\2&0&a-1\end{pmatrix},\\当a\neq1且a\neq\frac1{11}时,R(A)=3;当a=\frac1{11}时,R(A)=2;当a=1时,R(A)=1.\end{array}
$$



$$
\mathrm{若矩阵}A=\begin{pmatrix}3a&2a+1&a+1\\2a-1&2a-1&a-1\\4a-1&3a&2a\end{pmatrix}\mathrm{的秩为}2,则a\mathrm{的取值为}(\;).\;
$$
$$
A.
a=-1 \quad B.a=0或a=1 \quad C.a^2\neq1 \quad D.a^2=1 \quad E. \quad F. \quad G. \quad H.
$$
$$
\begin{array}{l}\left|A\right|=\begin{vmatrix}3a&2a+1&a+1\\2a-1&2a-1&a-1\\4a-1&3a&2a\end{vmatrix}=a(a-1)^2,当a\neq0或a\neq1时,\left|A\right|\neq0,\mathrm{此时}R(A)=3;\\当a=0时,A=\begin{pmatrix}0&1&1\\-1&-1&-1\\-1&0&0\end{pmatrix}\rightarrow\begin{pmatrix}1&0&0\\0&1&1\\0&0&0\end{pmatrix},R(A)=2;\\当a=1时,A=\begin{pmatrix}3&3&2\\1&1&0\\3&3&2\end{pmatrix}\rightarrow\begin{pmatrix}1&1&0\\0&0&1\\0&0&0\end{pmatrix},R(A)=2.\\\mathrm{综合上述},当a\neq0或a\neq1时,\mathrm{此时}R(A)=3;当a=0或a=1时,\mathrm{此时}R(A)=2.\end{array}
$$



$$
\mathrm{设矩阵}A=\begin{pmatrix}1&λ&-1&2\\2&-1&λ&5\\1&10&-6&1\end{pmatrix},\mathrm{其中}λ\mathrm{为参数},\mathrm{则矩阵}A\mathrm{的秩的最大值与最小值分别为}\;(\;).
$$
$$
A.
1,3 \quad B.2,3 \quad C.1,2 \quad D.3,3 \quad E. \quad F. \quad G. \quad H.
$$
$$
\begin{array}{l}\mathrm{对矩阵施以初等行变换}:\begin{pmatrix}1&λ&-1&2\\2&-1&λ&5\\1&10&-6&1\end{pmatrix}\rightarrow\begin{pmatrix}1&10&-6&1\\0&-21&λ+12&3\\0&λ-10&5&1\end{pmatrix}\\令\frac{-21}{10-λ}=\frac{λ+12}5=\frac31,\mathrm{解得}:λ=3,\mathrm{此时}\begin{pmatrix}1&λ&-1&2\\2&-1&λ&5\\1&10&-6&1\end{pmatrix}\rightarrow\begin{pmatrix}1&10&-6&1\\0&-7&5&1\\0&0&0&0\end{pmatrix},\\\mathrm{故当}λ=3时,R(A)\mathrm{为最小},R(A)_{min}=2,当λ\neq3时,R(A)\mathrm{为最大},R(A)_{max}=3.\end{array}
$$



$$
\mathrm{设矩阵}A=\begin{pmatrix}3&-2&λ&-16\\2&-3&0&1\\1&-1&1&-3\\3&μ&1&-2\end{pmatrix},\mathrm{其中}λ,μ\mathrm{为参数},\mathrm{则矩阵}A\mathrm{秩的最大值和最小值分别为}(\;).\;
$$
$$
A.
2,4 \quad B.2,3 \quad C.3,4 \quad D.1,4 \quad E. \quad F. \quad G. \quad H.
$$
$$
\begin{array}{l}对A\mathrm{作行}˴\mathrm{列的初等变换},得\\A\rightarrow\begin{pmatrix}1&3&1&-1\\0&7&2&1\\0&0&λ-5&0\\0&0&0&μ+4\end{pmatrix},\\当λ=5,μ=-4时,R(A)\mathrm{的最小值是}2;\\\;当λ\neq5,μ\neq-4时,R(A)\mathrm{的最大值是}4.\end{array}
$$



$$
设a_1,⋯,a_m,b_1,⋯,b_n(m<\;n)\;\mathrm{均为非零实数},A=\begin{pmatrix}a_1b_1&a_1b_2&⋯&a_1b_n\\a_2b_1&a_2b_2&⋯&a_2b_n\\⋯&⋯&⋯&⋯\\a_mb_1&a_mb_2&⋯&a_mb_n\end{pmatrix},则A\mathrm{的秩为}.\;
$$
$$
A.
n \quad B.m \quad C.1 \quad D.\mathrm{大于}1\mathrm{而小于}n\mathrm{的某一整数} \quad E. \quad F. \quad G. \quad H.
$$
$$
\begin{array}{l}\mathrm{用初等行变换求矩阵的秩}:\\A=\begin{pmatrix}a_1b_1&a_1b_2&⋯&a_1b_n\\a_2b_1&a_2b_2&⋯&a_2b_n\\⋯&⋯&⋯&⋯\\a_mb_1&a_mb_2&⋯&a_mb_n\end{pmatrix}\rightarrow\begin{pmatrix}b_1&b_2&⋯&b_n\\b_1&b_2&⋯&b_n\\⋯&⋯&⋯&⋯\\b_1&b_2&⋯&b_n\end{pmatrix}\rightarrow\begin{pmatrix}b_1&b_2&⋯&b_n\\0&0&⋯&0\\⋯&⋯&⋯&⋯\\0&0&⋯&0\end{pmatrix},\\故R(A)=1.\end{array}
$$



$$
\mathrm{矩阵}A=\begin{pmatrix}a+x&b+x&c+x&d+x\\a-y&b-y&c-y&d-y\\a+z&b+z&c+z&d+z\end{pmatrix}\mathrm{的秩为}(\;),\mathrm{其中}a,b,c,d,x,y,z\mathrm{互不相同}.\;
$$
$$
A.
1 \quad B.2 \quad C.3 \quad D.4 \quad E. \quad F. \quad G. \quad H.
$$
$$
\begin{array}{l}\mathrm{矩阵}A=\begin{pmatrix}a+x&b+x&c+x&d+x\\a-y&b-y&c-y&d-y\\a+z&b+z&c+z&d+z\end{pmatrix}\rightarrow\begin{pmatrix}a+x&b+x&c+x&d+x\\-(x+y)&-(x+y)&-(x+y)&-(x+y)\\z-x&z-x&z-x&z-x\end{pmatrix}\\\;\;\;\;\;\;\;\;\;\;\;\;\;\;\;\;\;\;\;\;\;\;\;\;\;\;\;\;\;\;\;\;\;\;\;\;\;\;\;\;\rightarrow\begin{pmatrix}a-d&b-d&c-d&d+x\\0&0&0&-(x+y)\\0&0&0&z-x\end{pmatrix}\;\\由a,b,c,d,x,y,z\mathrm{互不相同},则A\rightarrow\begin{pmatrix}a-d&b-d&c-d&d+x\\0&0&0&0\\0&0&0&1\end{pmatrix}\;,R(A)=2.\;\;\;\;\;\;\;\;\;\\\end{array}
$$



$$
\mathrm{矩阵}A=\begin{pmatrix}0&a_1&0&⋯&0\\0&0&a_2&⋯&0\\⋯&⋯&⋯&⋯&⋯\\0&0&0&⋯&a_{n-1}\\a_n&0&0&⋯&0\end{pmatrix}\mathrm{满足}(\;\;)\mathrm{什么条件时},R(A)=n.\;
$$
$$
A.
a_1a_2⋯ a_n=0 \quad B.a_1=a_2=⋯=a_n \quad C.a_1\neq a_2\neq⋯\neq a_n \quad D.a_1a_2⋯ a_n\neq0 \quad E. \quad F. \quad G. \quad H.
$$
$$
\mathrm{若要使得}R(A)=n,\mathrm{则有}\left|A\right|=(-1)^{n+1}a_1a_2⋯ a_n\neq0即a_1a_2⋯ a_n\neq0
$$



$$
设A,B\mathrm{均为}n\mathrm{阶方阵},且A=\frac12(B+E),则A^2=A\mathrm{当且仅当}(\;).\;
$$
$$
A.
B^2=E \quad B.B^2=O \quad C.B^2=B \quad D.B=A \quad E. \quad F. \quad G. \quad H.
$$
$$
\begin{array}{l}若A^2=A,则\\\;\;\;\;\;\;\;\;\;\;\;\;\;\;\;\;A^2=\frac14(B+E)(B+E)=\frac14(B^2+2B+E)=\frac12(B+E),\\\;\;\;\;\;\;\;\;\;\;\;\;\;\;\;B^2+2B+E=2B+2E⇒ B^2=E;\\若B^2=E,则\\\;\;\;\;\;\;\;\;\;\;\;\;\;\;\;\;A^2=\frac14(B^2+2B+E)=\frac14(E+2B+E)=\frac12(B+E)=A,\\\end{array}
$$



$$
\text{设}A\text{是}4×3\mathrm{型矩阵},B=\begin{pmatrix}1&0&2\\0&2&0\\-1&0&3\end{pmatrix},\text{若}R\left(A\right)=2,\text{则}R\left(AB\right)=().
$$
$$
A.
4 \quad B.3 \quad C.2 \quad D.1 \quad E. \quad F. \quad G. \quad H.
$$
$$
\mathrm{由于}\left|B\right|=\begin{vmatrix}1&0&2\\0&2&0\\-1&0&3\end{vmatrix}=10\neq0,⇒ B\mathrm{可逆},\mathrm{所以}\left(AB\right)=R\left(A\right)=2.
$$



$$
\mathrm{要断言矩阵}A\mathrm{的秩为}r,\mathrm{只须条件}(\;)\mathrm{满足即可}.
$$
$$
A.
A\mathrm{中有}r\mathrm{阶子式不为}0 \quad B.A\mathrm{中任何}r+1\mathrm{阶子式为}0 \quad C.A\mathrm{中不为}0\mathrm{的子式的阶数小于等于}r \quad D.A\mathrm{中不为}0\mathrm{的子式的最高阶数等于}r \quad E. \quad F. \quad G. \quad H.
$$
$$
\mathrm{要断言矩阵}A\mathrm{的秩为}r,\mathrm{只需满足}A\mathrm{中有}r\mathrm{阶子式不为零},且r+1\mathrm{阶子式全为零};\mathrm{或不为零的子式的最高阶数等于}r.
$$



$$
设A与B\mathrm{都是}n\mathrm{阶方阵},则\;R(A+B)(\;)
$$
$$
A.
\geq max\{R(A),R(B)\} \quad B.\leq min\{R(A),R(B)\} \quad C.>R(A)+R(B) \quad D.\leq R(A)+R(B) \quad E. \quad F. \quad G. \quad H.
$$
$$
\mathrm{由矩阵秩的性质可知}:R(A+B)\leq R(A)+R(B)
$$



$$
若n\mathrm{阶方阵}A\mathrm{的秩为}r,\mathrm{则结论}(\;)\mathrm{成立}.\;
$$
$$
A.
\left|A\right|\neq0 \quad B.\left|A\right|=0 \quad C.r>n \quad D.r\leq n \quad E. \quad F. \quad G. \quad H.
$$
$$
\mathrm{由矩阵秩的性质可知}r\leq n,若r=n,则\left|A\right|\neq0;若r<\;n,则\left|A\right|=0.
$$



$$
设A是3\mathrm{阶方阵},且A^2=O,\mathrm{下列各式中},\mathrm{不一定成立的是}(\;).\;
$$
$$
A.
A=O \quad B.R(A)<3 \quad C.A^3=O \quad D.\left|A\right|=0 \quad E. \quad F. \quad G. \quad H.
$$
$$
\begin{array}{l}A^2=O⇒ A^3=A^2A=O,且\left|A^2\right|=\left|A\right|^2=0,即\left|A\right|=0,\\\mathrm{由秩的性质},0=R(A^2)=R(AA)\geq R(A)+R(A)-3,故\;R(A)\leq\frac32<3,\\\mathrm{答案}A=O\mathrm{无法确定}.如A=\begin{pmatrix}1&1\\-1&-1\end{pmatrix}\end{array}
$$



$$
设A=\begin{pmatrix}1&2&3&4&5\\0&1&-1&3&2\\0&0&0&3&2\\0&0&0&0&0\\0&0&0&0&0\end{pmatrix},\mathrm{则矩阵的秩为}(\;).\;
$$
$$
A.
0 \quad B.2 \quad C.3 \quad D.4 \quad E. \quad F. \quad G. \quad H.
$$
$$
\mathrm{矩阵}A\mathrm{中存在一个三阶子式}\begin{vmatrix}1&2&4\\0&1&3\\0&0&3\end{vmatrix}=3\neq0,\mathrm{而所有的四阶子式为零},\mathrm{由矩阵秩的定义可知},R(A)=3.
$$



$$
\mathrm{设矩阵}A\mathrm{与矩阵}B\mathrm{等价},\mathrm{则下列说法正确的是}(\;).\;
$$
$$
A.
A\mathrm{的秩小于}B\mathrm{的秩} \quad B.A\mathrm{的秩大于}B\mathrm{的秩} \quad C.A\mathrm{的秩等于}B\mathrm{的秩} \quad D.A与B\mathrm{的行列式相等} \quad E. \quad F. \quad G. \quad H.
$$
$$
\begin{array}{l}\mathrm{若矩阵}A\mathrm{经过有限次初等变换变成矩阵}B,\mathrm{则称矩阵}A\mathrm{与矩阵}B\mathrm{等价},\mathrm{由于初等变换不改变矩阵的秩},\mathrm{故矩阵}A\mathrm{与矩阵}B\mathrm{的秩相等};\;\;\\\mathrm{又由于初等矩阵的行列式不一定等于}1,\mathrm{故矩阵}A\mathrm{与矩阵}B\mathrm{的行列式不一定相等}.\end{array}
$$



$$
\mathrm{矩阵}B=\begin{pmatrix}2&-1&0&3&-2\\0&3&1&-2&5\\0&0&0&4&-3\\0&0&0&0&0\end{pmatrix}\mathrm{的秩为}(\;).
$$
$$
A.
1 \quad B.2 \quad C.3 \quad D.4 \quad E. \quad F. \quad G. \quad H.
$$
$$
\begin{array}{l}∵\;B\mathrm{是一个行阶梯形矩阵},\mathrm{其非零行只有}3行,\\\;∴ B\mathrm{的所有四阶子式全为零}.\\而\;\;\;\;\;\;\begin{vmatrix}2&-1&3\\0&3&-2\\0&0&4\end{vmatrix}\;\;=24\neq0\;,\\∴ R(B)=3.\end{array}
$$



$$
\mathrm{矩阵}A=\begin{pmatrix}1&2&3\\2&3&-5\\4&7&1\end{pmatrix}\mathrm{的秩为}(\;).
$$
$$
A.
0 \quad B.1 \quad C.2 \quad D.3 \quad E. \quad F. \quad G. \quad H.
$$
$$
\begin{array}{l}在A中,\begin{vmatrix}1&3\\2&-5\end{vmatrix}\neq0.\\又\;∵\;A的3\mathrm{阶子式只有一个}\left|A\right|,且\\\left|A\right|=\begin{vmatrix}1&2&3\\2&3&-5\\4&7&1\end{vmatrix}=\begin{vmatrix}1&2&3\\0&-1&-11\\0&-1&-11\end{vmatrix}=0,\\∴ R(A)=2.\end{array}
$$



$$
\mathrm{矩阵}A=\begin{pmatrix}1&-5&6&-2\\2&-1&3&-2\\-1&-4&3&0\end{pmatrix}\mathrm{的秩为}(\;).
$$
$$
A.
1 \quad B.2 \quad C.3 \quad D.4 \quad E. \quad F. \quad G. \quad H.
$$
$$
\begin{array}{l}\begin{vmatrix}1&-5&6\\2&-1&3\\-1&-4&3\end{vmatrix}=\begin{vmatrix}1&-5&-2\\2&-1&-2\\-1&-4&0\end{vmatrix}=\begin{vmatrix}1&6&-2\\2&3&-2\\-1&3&0\end{vmatrix}=\begin{vmatrix}-5&6&-2\\-1&3&-2\\-4&3&0\end{vmatrix}=0,\\又A\mathrm{有一个不等于零的二阶子式}\begin{vmatrix}1&-5\\2&-1\end{vmatrix}=9,\;\mathrm{所以}\;R(A)=2.\end{array}
$$



$$
\mathrm{奇异方阵经过}(\;)后,\mathrm{矩阵的秩有可能改变}.
$$
$$
A.
\mathrm{初等变换} \quad B.\mathrm{左乘初等矩阵} \quad C.左˴\mathrm{右同乘初等矩阵} \quad D.\mathrm{和一个单位矩阵相加} \quad E. \quad F. \quad G. \quad H.
$$
$$
\begin{array}{l}\mathrm{因初等变换以及矩阵与初等矩阵相乘都不改变矩阵的秩},\mathrm{不管这个矩阵是不是奇异阵}.\;\;\\\mathrm{而奇异方阵和一个单位矩阵相加有可能改变矩阵的秩},如\\A=\begin{pmatrix}-1&0&0\\0&-1&0\\0&0&0\end{pmatrix},R(A)=2,而\\A+E=\begin{pmatrix}0&&\\&0&\\&&1\end{pmatrix},R(A)=1.\end{array}
$$



$$
若A是m× s\mathrm{型矩阵},B是s× n\mathrm{型矩阵},\mathrm{那么}R(AB)\;(\;).
$$
$$
A.
<\;R(A) \quad B.<\;R(B) \quad C.\leq min\{R(A),R(B)\} \quad D.\mathrm{以上都不对} \quad E. \quad F. \quad G. \quad H.
$$
$$
\mathrm{根据矩阵的秩的性质可得},R(AB)\leq min\{R(A),R(B)\}
$$



$$
\mathrm{在一个矩阵上添加两行或两列后},\mathrm{所得到的矩阵的秩}(\;).
$$
$$
A.
\mathrm{不变} \quad B.\mathrm{增加}1 \quad C.\mathrm{增加}2 \quad D.\mathrm{以上都有可能} \quad E. \quad F. \quad G. \quad H.
$$
$$
\mathrm{根据矩阵的秩的定义},\mathrm{显然各种选项的情况都有可能}.
$$



$$
设A\mathrm{为四阶方阵},\mathrm{且矩阵}A\mathrm{的秩}R(A)=3,则R(A^*)\;=(\;).
$$
$$
A.
1 \quad B.2 \quad C.3 \quad D.4 \quad E. \quad F. \quad G. \quad H.
$$
$$
当R(A)=n-1时,R(A^*)=1.\mathrm{本题中}R(A)=4-1=3,故R(A^*)=1.
$$



$$
\mathrm{设四阶方阵}A\mathrm{的秩为}2,\mathrm{则其伴随矩阵}A^*\mathrm{的秩为}(\;).\;
$$
$$
A.
4 \quad B.2 \quad C.1 \quad D.0 \quad E. \quad F. \quad G. \quad H.
$$
$$
R(A)=2<\;n-1=3,\mathrm{故伴随矩阵的定义可知}R(A^*)=0.
$$



$$
设A=\begin{pmatrix}1&-1&1&5\\9&3&1&1\\1&1&1&-1\end{pmatrix},\mathrm{则矩阵}A\mathrm{的秩为}(\;\;).\;
$$
$$
A.
1 \quad B.2 \quad C.3 \quad D.4 \quad E. \quad F. \quad G. \quad H.
$$
$$
\mathrm{因为}\begin{vmatrix}1&-1&1\\9&3&1\\1&1&1\end{vmatrix}=16\neq0,故R(A)=3.
$$



$$
设A=\begin{pmatrix}1&1&2&-1\\1&0&1&0\\2&1&3&2\end{pmatrix},\mathrm{则矩阵}A\mathrm{的秩为}(\;\;).\;
$$
$$
A.
1 \quad B.2 \quad C.3 \quad D.4 \quad E. \quad F. \quad G. \quad H.
$$
$$
\mathrm{因为}\begin{vmatrix}1&2&-1\\0&1&0\\1&3&2\end{vmatrix}=3\neq0,故R(A)=3.
$$



$$
设A=\begin{pmatrix}1&1&0&0\\0&0&1&1\\-2&0&2&2\\0&-3&0&3\end{pmatrix},\mathrm{则矩阵}A\mathrm{的秩为}(\;\;).\;
$$
$$
A.
1 \quad B.2 \quad C.3 \quad D.4 \quad E. \quad F. \quad G. \quad H.
$$
$$
\mathrm{因为}\left|A\right|=-6\neq0,故R(A)=4.
$$



$$
\mathrm{若一个}n\mathrm{阶方阵}A\mathrm{的行列式值不为零},\mathrm{则对}A\mathrm{进行若干次矩阵的初等变换后},\mathrm{其行列式的值}(\;\;\;\;\;)
$$
$$
A.
\mathrm{保持不变}; \quad B.\mathrm{保持正负号不变}; \quad C.\mathrm{可以变成任何值} \quad D.\mathrm{保持不为}0 \quad E. \quad F. \quad G. \quad H.
$$
$$
\mathrm{由于}A\mathrm{的行列式值不为零},即A\mathrm{是可逆的},\mathrm{满秩的},\mathrm{而初等变换是不改变矩阵的秩的},\mathrm{所以变换后行列式仍然不等于}0
$$



$$
\begin{array}{l}A是5×4\mathrm{型矩阵},R(A)=r<4,\;则A\mathrm{中必}\;(\;\;)\;\\\end{array}
$$
$$
A.
\mathrm{没有等于}0的r-1\mathrm{阶子式},\mathrm{至少有一个}r\mathrm{阶子式不为}0; \quad B.\mathrm{一定有等于}0的r\mathrm{阶子式},\mathrm{没有不等于}0的r+1\mathrm{阶子式}; \quad C.\begin{array}{l}\mathrm{一定有不等于}0的r\mathrm{阶子式},\mathrm{没有不等于}0的r+1\mathrm{阶子式};\\\end{array} \quad D.\mathrm{任何}r\mathrm{阶子式不等于}0,\mathrm{任何}r+1\mathrm{阶子式都等于}0. \quad E. \quad F. \quad G. \quad H.
$$
$$
\mathrm{根据矩阵秩的定义可知}
$$



$$
\mathrm{已知矩阵}A=\begin{pmatrix}1&2&-2\\4&t&3\\3&-1&1\end{pmatrix}\mathrm{的秩等于}2,则\;t=(\;)
$$
$$
A.
2 \quad B.3 \quad C.-3 \quad D.0 \quad E. \quad F. \quad G. \quad H.
$$
$$
\begin{pmatrix}1&2&-2\\4&t&3\\3&-1&1\end{pmatrix}\rightarrow\begin{pmatrix}1&2&-2\\0&t-8&11\\0&-7&7\end{pmatrix},\mathrm{所以}t-8=-11\;,即t=-3
$$



$$
若\left|A\right|=0,\mathrm{则对}A\mathrm{进行若干次矩阵的初等行变换后},\mathrm{其行列式的值}(\;\;)\;
$$
$$
A.
\mathrm{保持不变}; \quad B.\mathrm{保持正负号不变}; \quad C.\mathrm{可以变成任何值}; \quad D.\mathrm{保持为}0 \quad E. \quad F. \quad G. \quad H.
$$
$$
\mathrm{由于}A\mathrm{的行列式值为零},\mathrm{即它不是满秩的},\mathrm{而初等变换是不改变矩阵的秩的},\mathrm{所以变换后行列式仍然等于}0
$$



$$
若\left|A_{n× n}\right|=0,\mathrm{则对}A\mathrm{进行若干次矩阵的初等行变换后},R(A)(\;\;\;\;)
$$
$$
A.
=n \quad B.\leq n \quad C.<\;n \quad D.=0 \quad E. \quad F. \quad G. \quad H.
$$
$$
\mathrm{由于}A\mathrm{的行列式值为零},即R(A)<\;n\;,\mathrm{而初等变换是不改变矩阵的秩的}
$$



$$
\mathrm{已知}R(A)=r_1\;,C\mathrm{是可逆矩阵},B=AC,记R(B)=r_2,则(\;\;)
$$
$$
A.
r_1<\;r_2 \quad B.r_1>r_2 \quad C.r_1\geq r_2 \quad D.r_1=r_2 \quad E. \quad F. \quad G. \quad H.
$$
$$
\mathrm{由秩的性质},\mathrm{可逆矩阵左乘或右乘一个矩阵},\mathrm{秩不发生变化}
$$



$$
\mathrm{若矩阵}A\mathrm{有一个}r\mathrm{阶子不为零},\mathrm{则下列结论正确的是}\;(\;\;)
$$
$$
A.
R(A)<\;r \quad B.R(A)\leq r \quad C.R(A)>r \quad D.R(A)\geq r \quad E. \quad F. \quad G. \quad H.
$$
$$
\mathrm{根据秩的定义可知}
$$



$$
\mathrm{若方阵}A˴B\mathrm{等价},\mathrm{则下列说法正确的是}(\;\;)
$$
$$
A.
\left|A\right|=\left|B\right| \quad B.A˴B\mathrm{都与单位矩阵等价} \quad C.A˴B\mathrm{具有相同的秩}\; \quad D.AB=E \quad E. \quad F. \quad G. \quad H.
$$
$$
\mathrm{等价的矩阵秩相同}
$$



$$
设A是4\mathrm{阶方阵},且\;R(A)=2,B\mathrm{是可逆矩阵},则R(AB)=(\;\;\;\;).
$$
$$
A.
2 \quad B.4 \quad C.\leq2 \quad D.\leq4 \quad E. \quad F. \quad G. \quad H.
$$
$$
\mathrm{由秩的性质},\mathrm{可逆矩阵左乘或右乘一个矩阵},\mathrm{其秩不发生变化}
$$



$$
设A˴B\mathrm{为两个矩阵},若AB=E,\mathrm{则下列结论一定正确的是}(\;\;\;)
$$
$$
A.
R(A)=R(B) \quad B.B\mathrm{一定是}A\mathrm{的逆矩阵} \quad C.R(E)\geq max\{R(A),R(B)\} \quad D.R(E)\leq R(B) \quad E. \quad F. \quad G. \quad H.
$$
$$
A˴B\mathrm{不一定是同阶矩阵},\mathrm{所以第二个选项不符合};\mathrm{从秩的性质可知},\mathrm{答案选}D
$$



$$
设A是4×3\mathrm{型矩阵},且R(A)=2而B=\begin{pmatrix}1&0&2\\0&2&0\\-1&0&3\end{pmatrix},则\;R(AB)=(\;)
$$
$$
A.
4 \quad B.3 \quad C.2 \quad D.1 \quad E. \quad F. \quad G. \quad H.
$$
$$
由B=\begin{pmatrix}1&0&2\\0&2&0\\-1&0&3\end{pmatrix},得\left|B\right|=10\neq0,\mathrm{所以}R(AB)=R(A)=2
$$



$$
\mathrm{若矩阵}\;A=\begin{pmatrix}1&0&0\\0&k&0\\1&-1&1\end{pmatrix}\mathrm{的秩等于}3,则k\mathrm{满足}(\;\;\;\;)
$$
$$
A.
k=0 \quad B.k\neq0 \quad C.k=1 \quad D.k\neq1 \quad E. \quad F. \quad G. \quad H.
$$
$$
\left|A\right|=k,R(A)=3\;故\left|A\right|=k\neq0
$$



$$
\mathrm{若矩阵}A=\begin{pmatrix}1&0&5&-1\\0&-2&6&3\\0&0&0&0\end{pmatrix}\;,则\;R(A)=(\;)
$$
$$
A.
1 \quad B.2 \quad C.3 \quad D.4 \quad E. \quad F. \quad G. \quad H.
$$
$$
\mathrm{由于这是一个行阶梯矩阵},\mathrm{所以}R(A)\mathrm{等于非零行的行数}2
$$



$$
若A˴B\mathrm{等价},则()
$$
$$
A.
\left|A\right|=\left|B\right| \quad B.R(A)=R(B) \quad C.\left|A\right|\neq\left|B\right| \quad D.\left|A\right|=-\left|B\right| \quad E. \quad F. \quad G. \quad H.
$$
$$
\mathrm{等价的矩阵具有相同的秩}
$$



$$
若A˴B\mathrm{为两个矩阵},\mathrm{已知}A是n× m\mathrm{型矩阵},且BA=O,\;\;则(\;\;\;\;)\;
$$
$$
A.
A=O或B=O \quad B.AB=O \quad C.R(A)+R(B)\leq n \quad D.\left|A\right|=0或\left|B\right|=0 \quad E. \quad F. \quad G. \quad H.
$$
$$
\mathrm{根据秩的性质可得}
$$



$$
设A^* 是n\mathrm{阶奇异矩阵}A\mathrm{的伴随矩阵},且A^*\neq O,则\;R(A)=(\;)
$$
$$
A.
1 \quad B.n \quad C.n-1 \quad D.\mathrm{不能确定} \quad E. \quad F. \quad G. \quad H.
$$
$$
\mathrm{由于}A\mathrm{是奇异矩阵},\mathrm{所以}R(A)\leq n-1,又A^*\neq O,\mathrm{因此}R(A)\geq n-1,\mathrm{综合以上},R(A)=n-1
$$



$$
设A为4×3\mathrm{型矩阵},且R(A)=2,而B=\begin{pmatrix}1&0&0\\0&2&0\\-1&0&3\end{pmatrix},则\;R(AB)=(\;).
$$
$$
A.
1 \quad B.2 \quad C.3 \quad D.4 \quad E. \quad F. \quad G. \quad H.
$$
$$
\left|B\right|=6\neq0,故R(B)=3,\mathrm{而且}B\mathrm{是可逆矩阵},\mathrm{不改变矩阵的秩},有R(AB)=R(A)=2
$$



$$
设A是m× n\mathrm{型矩阵},B是n× m\mathrm{型矩阵},则(\;).
$$
$$
A.
当m>n时,\mathrm{必有行列式}\left|AB\right|\neq0 \quad B.当m>n时,\mathrm{必有行列式}\left|AB\right|=0 \quad C.当n>m时,\mathrm{必有行列式}\left|AB\right|\neq0 \quad D.当n>m时,\mathrm{必有行列式}\left|AB\right|=0 \quad E. \quad F. \quad G. \quad H.
$$
$$
\begin{array}{l}AB为m× m\mathrm{阶矩阵},当m>\;n时,\\R(AB)\leq\;min\{R(A),R(B)\}\leq\;min\{m,n\}\leq\;n<\;m,则\left|AB\right|=0.\end{array}
$$



$$
\mathrm{矩阵}\;\begin{pmatrix}0&1&1&-1&2\\0&2&-2&-2&0\\0&-1&-1&1&1\\1&1&0&1&-1\end{pmatrix}\mathrm{的秩为}(\;).
$$
$$
A.
1 \quad B.2 \quad C.3 \quad D.4 \quad E. \quad F. \quad G. \quad H.
$$
$$
由\begin{vmatrix}0&1&1&2\\0&2&-2&0\\0&-1&-1&1\\1&1&0&-1\end{vmatrix}=\begin{vmatrix}1&1&2\\2&-2&0\\-1&-1&1\end{vmatrix}=12\neq0,\mathrm{故矩阵的秩为}4.
$$



$$
设A为m× n\mathrm{型矩阵},B为n× m\mathrm{型矩阵},E为m\mathrm{阶单位矩阵},若AB=E,则(\;).
$$
$$
A.
R(A)=m,R(B)=m \quad B.R(A)=m,R(B)=n \quad C.R(A)=n,R(B)=m \quad D.R(A)=n,R(B)=n \quad E. \quad F. \quad G. \quad H.
$$
$$
\begin{array}{l}\\R(AB)=R(E)=m.\\\mathrm{因为}R(AB)\leq R(A)且R(AB)\leq R(B),\mathrm{所以}R(A)\geq m\mathrm{又显然}\\R(A)\leq m,R(B)\leq m,故R(A)=R(B)=m\end{array}
$$



$$
设A,B\mathrm{均为}n\mathrm{阶方阵},且R(A)=R(B),\mathrm{则下列式子成立的是}\;(\;).
$$
$$
A.
R(A-B)=0 \quad B.R(A+B)=2R(A) \quad C.R(A,B)\geq2R(A) \quad D.R(A,B)\leq R(A)+R(B) \quad E. \quad F. \quad G. \quad H.
$$
$$
\mathrm{根据矩阵的常用性质可知},R(A,B)\leq R(A)+R(B),\mathrm{其余选项都不正确}.
$$



$$
\mathrm{已知三阶方阵}A=\begin{pmatrix}1&0&-1\\2&λ&1\\1&2&1\end{pmatrix},B\mathrm{是秩为}2\mathrm{的三阶方阵},且R(AB)=1,则λ=(\;)
$$
$$
A.
2 \quad B.1 \quad C.3 \quad D.0 \quad E. \quad F. \quad G. \quad H.
$$
$$
\begin{array}{l}\mathrm{由题设可知}\left|A\right|=0,\mathrm{否则}A\mathrm{可逆},\mathrm{于是有}A=P_1P_2⋯ P_1,\mathrm{其中}P_1,P_2,⋯,P_1\mathrm{为初等矩阵},\mathrm{从而}\\R(AB)=R(B)=2\;,\mathrm{与已知条件}R(AB)=1\mathrm{矛盾}.\;则\\\left|A\right|=\begin{vmatrix}1&0&-1\\2&λ&1\\1&2&1\end{vmatrix}=2λ-6=0⇒λ=3\end{array}
$$



$$
\mathrm{设三阶矩阵}A=\begin{pmatrix}x&1&1\\1&x&1\\1&1&x\end{pmatrix}\;,当x=1和x=-2时,\mathrm{矩阵}A\mathrm{的秩分别为}(\;).\;
$$
$$
A.
1和2 \quad B.1和3 \quad C.2和3 \quad D.3和2 \quad E. \quad F. \quad G. \quad H.
$$
$$
\begin{array}{l}\mathrm{直接从矩阵秩的行列式定义出发讨论}.\;\mathrm{由于}\;\\\;\;\;\;\;\;\;\;\;\;\;\;\;\;\;\;\;\;\;\;\;\;\;\;\;\;\;\;\;\;\;\;\;\;\;\;\;\;\;\;\;\begin{vmatrix}x&1&1\\1&x&1\\1&1&x\end{vmatrix}=(x+2)(x-1)^2,\\故\;\;1当x\neq1且x\neq-2时,\left|A\right|\;\neq0,R(A)=3;\\\;\;\;\;\;\;2当x=1时,\left|A\right|\;=0,且A=\begin{pmatrix}1&1&1\\1&1&1\\1&1&1\end{pmatrix},\mathrm{显然},R(A)=1;\\\;\;\;\;\;\;3当x=-2时,\left|A\right|\;=0,且A=\begin{pmatrix}-2&1&1\\1&-2&1\\1&1&-2\end{pmatrix},\mathrm{这时有二阶子式}\\\;\;\;\;\;\;\;\;\;\;\;\;\;\;\;\;\;\;\;\;\;\;\;\;\;\;\;\;\;\;\;\;\;\;\;\;\;\;\;\begin{vmatrix}-2&1\\1&-2\end{vmatrix}\neq0,\mathrm{显然},R(A)=2;\end{array}
$$



$$
\begin{array}{l}\mathrm{下列两个命题中正确的是}(\;).\\(1)设A为n\mathrm{阶矩阵},则R(A+E)+R(A-E)\geq n.\\(2)设A为n\mathrm{阶非奇异矩阵},B为n× m\mathrm{矩阵},则R(AB)=R(B).\end{array}
$$
$$
A.
(1) \quad B.(2) \quad C.(1)(2) \quad D.\mathrm{两个都不正确} \quad E. \quad F. \quad G. \quad H.
$$
$$
\begin{array}{l}(1)因(A+E)+(E-A)=2E,则R(A+E)+R(E-A)\geq R(2E)=n,\;\;\\而R(E-A)=R(A-E),\mathrm{所以}R(A+E)+R(A-E)\geq n.\;\\\;(2)\mathrm{因为}A\mathrm{非奇异},\mathrm{故可表示成若干初等矩阵之积},A=P_1P_2⋯ P_s,\;\\\;P_i(i=1,2,⋯,s)\mathrm{皆为初等矩阵}.AB=P_1P_2⋯ P_sB,\;\\\;即AB是B\mathrm{经过}s\mathrm{次初等行变换后得出的}.\mathrm{因而}R(AB)=R(B).\end{array}
$$



$$
设A是m× n(m<\;n)\mathrm{矩阵},C是n\mathrm{阶可逆矩阵},R(A)=r,R(AC)=r_1,则(\;).\;
$$
$$
A.
n>r_1>r \quad B.r_1>r>n \quad C.r=r_1 \quad D.r_1=n \quad E. \quad F. \quad G. \quad H.
$$
$$
\begin{array}{l}c是n\mathrm{阶可逆矩阵},\mathrm{故可表示成若干初等矩阵之积}C=p_1P_2⋯ P_s,\mathrm{其中}P_i(i=1,2,⋯,s)\mathrm{皆为初等矩阵}.\;\;\\AC=AP_1P_2⋯ P_s,即AC是A\mathrm{经过}s\mathrm{次初等行变换后得出的},故R(AC)=R(A),即r=r_1.\;\end{array}
$$



$$
设A˴B\mathrm{都是}n\mathrm{阶非零矩阵},且AB=O,则A和B\mathrm{的秩}(\;).\;
$$
$$
A.
\mathrm{必有一个为}0 \quad B.\mathrm{都小于}n \quad C.\mathrm{一个小于}n\mathrm{一个等于}n \quad D.\mathrm{都等于}n \quad E. \quad F. \quad G. \quad H.
$$
$$
\mathrm{由矩阵的性质可知}AB=O,则R(A)+R(B)\leq n,即A和B\mathrm{的秩都小于}n.
$$



$$
\mathrm{在矩阵}A\mathrm{中增加一列而得到矩阵}B,设A˴B\mathrm{的秩分别为}r_1˴r_2,\mathrm{则它们之间的关系必为}(\;).\;
$$
$$
A.
r_1=r_2 \quad B.r_1=r_2-1 \quad C.r_1>r_2 \quad D.r_1\leq r_2 \quad E. \quad F. \quad G. \quad H.
$$
$$
\begin{array}{l}设A\mathrm{的某个}r_1\mathrm{阶子式}D_r\neq0.\;\mathrm{矩阵}B\mathrm{是由矩阵}A\mathrm{增加一列得到的},\mathrm{所以在}B\mathrm{中能找到与}D_r\mathrm{相同的}r_1\mathrm{阶子式}\;\overline{D_r},\\\mathrm{由于}\\\;\;\;\;\;\;\;\;\;\;\;\;\;\;\;\;\;\;\;\;\;\;\;\;\;\;\;\;\;\;\;\;\;\;\;\;\;\;\;\;\;\;\;\;\;\;\;\;\;\;\;\;\;\;\;\;\;\;\;\;\;\;\;\;\;\;\;\;\;\;\;\;\overline{D_r}=D_r\neq0,\\\mathrm{故而}\;R(B)\;\geq R(A),即r_1\leq r_2.\\\end{array}
$$



$$
设A\mathrm{是五阶方阵},且A\mathrm{的秩为}3,则A\mathrm{的伴随矩阵的秩为}(\;).
$$
$$
A.
0 \quad B.1 \quad C.2 \quad D.3 \quad E. \quad F. \quad G. \quad H.
$$
$$
n\mathrm{阶矩阵}A,当R(A)<\;n-1时,A\mathrm{中所有的}n-1\mathrm{阶子式即伴随矩阵的元素全为零},即A^*=0,\mathrm{所以}R(A^*)=0
$$



$$
设A为n\mathrm{阶非零方阵},且A\neq E,A^2=A(E为n\mathrm{阶单位矩阵}),则(\;).\;
$$
$$
A.
A\mathrm{的秩为}n \quad B.A\mathrm{的秩为}0 \quad C.A\mathrm{的秩小于}n,\mathrm{但不为}0 \quad D.A\mathrm{的秩大于}n \quad E. \quad F. \quad G. \quad H.
$$
$$
\begin{array}{l}A^2=A⇒ A(A-E)=O,\mathrm{由矩阵秩的性质可得}R(A)+R(A-E)\leq n,又\;\;\\\;\;\;\;\;\;\;\;\;\;\;\;\;\;\;\;\;\;\;\;\;\;\;\;\;\;\;\;\;\;\;\;\;\;\;\;\;\;\;\;\;\;\;\;\;\;\;\;\;\;\;A\neq E⇒ A-E\neq O⇒ R(A-E)\geq1,则R(A)\leq n-1.\;\;\\\;\;\;\;\;\;\;\;\;\;\;\;\;\;\;\;\;\;\;\;\;\;\;\;\;\;\;\;\;\;\;\;\;\;\;\;\;\;\;\;\;\;\;\;\;\;\;\;\;\;\;\;\;\;\;\;\;\;\;\;\;\;\;\;\;\;\;\;\;\;\;\;\;\;\;\;\;A\neq O⇒ R(A)\geq1.\;\;\\故A\mathrm{的秩小于}n,\mathrm{但不为}0.\end{array}
$$



$$
设A为3\mathrm{阶方阵},且R(A)=1,则(\;).\;
$$
$$
A.
R(A^*)=3 \quad B.R(A^*)=2 \quad C.R(A^*)=1 \quad D.R(A^*)=0 \quad E. \quad F. \quad G. \quad H.
$$
$$
R(A)=1,则A\mathrm{中的二阶子式全部为零},\mathrm{由伴随矩阵的定义可知}A^*=O,则R(A^*)=0.
$$



$$
若n\mathrm{阶方阵}A,B,C\mathrm{满足}AB=CB,\mathrm{则必有}(\;).
$$
$$
A.
A=C \quad B.B=O \quad C.若A,B,C\mathrm{皆可逆},则\frac1{\left|A\right|}=\frac1{\left|C\right|} \quad D.R(AB)=R(C) \quad E. \quad F. \quad G. \quad H.
$$
$$
\begin{array}{l}\mathrm{由于矩阵的乘法不满足消去律},\mathrm{因此由}AB=CB\mathrm{不能得出}A=C;若A,B,C\mathrm{皆可逆},\mathrm{则由}AB=CB\mathrm{可得}\\AB· B^{-1}=CB· B^{-1}⇒ A=C,故\left|A\right|=\left|C\right|,且\frac1{\left|A\right|}=\frac1{\left|C\right|}.\end{array}
$$



$$
\mathrm{已知}n\mathrm{阶方阵}A\mathrm{经初等变换可以化成}B,则(\;).
$$
$$
A.
\left|A\right|=\left|B\right| \quad B.\begin{array}{l}若A\mathrm{可逆},则A^{-1}=B^{-1}\\\end{array} \quad C.AX=0与BX=0\mathrm{同解} \quad D.R(A)=R(B) \quad E. \quad F. \quad G. \quad H.
$$
$$
\begin{array}{l}\mathrm{初等变换不改变矩阵的秩},\mathrm{所以}R(A)=R(B);\;\;\\n\mathrm{阶方阵}A\mathrm{经初等变换可以化成}B,则PAQ=B,\mathrm{其中}PQ\mathrm{为初等矩阵},\mathrm{但初等矩阵的行列式可为}±1,k,\\\mathrm{因此}\left|A\right|,\left|B\right|\mathrm{不一定相等},\mathrm{且在可逆情况下的逆矩阵也不一定相等};\;\;\\\mathrm{方程组}AX=0与BX=0\mathrm{也不一点同解},\mathrm{因为求解方程组时是将系数矩阵进行初等行变换},\mathrm{不包括列变换}.\end{array}
$$



$$
\mathrm{设四阶方阵}A\mathrm{的秩为}3,\mathrm{则其伴随矩阵}A^*\mathrm{的秩为}(\;).\;
$$
$$
A.
2 \quad B.1 \quad C.4 \quad D.3 \quad E. \quad F. \quad G. \quad H.
$$
$$
\begin{array}{l}n阶(n\geq2)\mathrm{方阵}A\mathrm{与其伴随矩阵秩的关系如下}:\\\;\;\;\;\;\;\;\;\;\;\;\;\;\;\;\;\;\;\;\;\;\;\;\;\;R(A^*)=\left\{\begin{array}{ccc}n,&当&R(A)=n\\1,&当&R(A)=n-1\\0,&当&R(A)<\;n-1\end{array}\right.\\\mathrm{题中四阶方阵}A\mathrm{的秩为}3=4-1,故R(A^*)=1.\\\end{array}
$$



$$
设n\mathrm{阶方阵}A与B\mathrm{的的秩相等},\mathrm{则下列成立的是}(\;).\;
$$
$$
A.
\mathrm{必存在}n\mathrm{阶可逆矩阵}P,Q,\mathrm{使得}PAQ=B \quad B.\mathrm{必存在}n\mathrm{阶可逆矩阵}P,\mathrm{使得}P^{-1}AP=B \quad C.\mathrm{必存在}n\mathrm{阶可逆矩阵}P,\mathrm{使得}P^TAP=B \quad D.\mathrm{必有}\left|A\right|=\left|B\right| \quad E. \quad F. \quad G. \quad H.
$$
$$
\begin{array}{l}\mathrm{由于任何矩阵}A\mathrm{经过有限次初等变换},\mathrm{可以化为标准形矩阵}\begin{pmatrix}E_r&0\\0&0\end{pmatrix},\mathrm{其中}R(A)=r,\mathrm{则由条件可知}A与B\mathrm{有相同的标准形},\mathrm{即存在初等矩阵}P_i,Q_i,H_i,G_i,\\有\;\\\;\;\;\;\;\;\;\;\;\;\;\;\;\;\;\;\;\;\;\;\;\;\;\;\;\;\;\;\;\;\;\;\;\;\;\;\;\;\;\;\;\;\;\;\;\;\;\;\;\;\;\;\;\;\;\;\;\;\;\;\;\;\;\;\;\;P_1⋯ P_sAQ_1⋯ Q_t=\begin{pmatrix}E_r&0\\0&0\end{pmatrix}=H_1⋯ H_sBG_1⋯ G_t,\;\;\\\;\;\;\;\;\;\;\;\;\;\;\;\;\;\;\;\;\;\;\;\;\;\;\;\;\;\;\;\;\;\;\;\;\;\;\;\;\;\;\;\;\;\;\;\;\;\;\;\;\;\;\;\;\;\;\;\;\;\;\;\;\;\;\;\;\;H_s\;^{-1}\;⋯ H_1\;^{-1}\;P_1⋯ P_sAQ_1⋯ Q_tG_t\;^{-1}\;\;⋯ G_1^{-1}=B\;,\;\;\\\mathrm{又由于初等矩阵的逆矩阵也是初等矩阵},\mathrm{且可逆矩阵可表示为一系列初等矩阵的乘积},令\\P=H_s\;^{-1}\;⋯ H_1\;^{-1}\;P_1⋯ P_s,Q=Q_1⋯ Q_tG_t\;^{-1}\;\;⋯ G_1^{-1},\mathrm{则存在可逆矩阵}P,Q,\mathrm{使得}PAQ=B.\end{array}
$$



$$
设A为5×4\mathrm{型矩阵},A=\begin{pmatrix}1&2&3&1\\2&-1&k&2\\0&1&1&3\\1&-1&0&4\\2&0&2&5\end{pmatrix},且A\mathrm{的秩为}3\;则k=(\;).
$$
$$
A.
0 \quad B.1 \quad C.-1 \quad D.2 \quad E. \quad F. \quad G. \quad H.
$$
$$
\begin{array}{l}\mathrm{因为}A\mathrm{的秩为}3\;\mathrm{故其四阶子式}\\\;\;\;\;\;\;\;\;\;\;\;\;\;\;\;\;\;\;\;\;\;\;\;\;\;\;\;\;\;\;\;\;\;\;\;\begin{vmatrix}1&2&3&1\\2&-1&k&2\\0&1&1&3\\1&-1&0&4\end{vmatrix}=0,\\\mathrm{解得}k=1.\end{array}
$$



$$
\mathrm{若矩阵}A=\begin{pmatrix}1&2&4\\2&λ&1\\1&1&0\end{pmatrix},\mathrm{为使矩阵}A\mathrm{的秩有最小值},则λ\mathrm{应为}(\;).
$$
$$
A.
2 \quad B.1 \quad C.\frac94 \quad D.\frac12 \quad E. \quad F. \quad G. \quad H.
$$
$$
\begin{array}{l}\mathrm{矩阵}A\mathrm{已有一个二阶子式}\begin{vmatrix}1&4\\2&1\end{vmatrix}=-7\neq0,故A\mathrm{的秩的最小值为}2,\mathrm{从而}A\mathrm{的三阶子式为零},即\;\;\\\begin{vmatrix}1&2&4\\2&λ&1\\1&1&0\end{vmatrix}=0⇒λ=\frac94.\end{array}
$$



$$
当x,y\mathrm{满足条件}(\;)时,\mathrm{矩阵}A=\begin{pmatrix}1&1&1&1&1\\3&2&1&-3&x\\0&1&2&6&3\\5&4&3&-1&y\end{pmatrix}\mathrm{的秩为}2.\;
$$
$$
A.
x=2,y=0 \quad B.x=0,y=2 \quad C.x=3,y=1 \quad D.x=1,y=3 \quad E. \quad F. \quad G. \quad H.
$$
$$
\begin{array}{l}\mathrm{由题设},\mathrm{矩阵}A\mathrm{的秩为}2,而A\mathrm{中至少有一个}2\mathrm{阶子式}\begin{vmatrix}1&1\\3&2\end{vmatrix}=-1\neq0,则A\mathrm{中所有三阶子式和四阶子式均为零},\\\mathrm{选取包含}x,y\mathrm{的三阶和四阶子式},\mathrm{也可是两个三阶或四阶子式},\mathrm{令它们为}0,\mathrm{解方程组即得}x=0,y=2.\end{array}
$$



$$
设A=\begin{pmatrix}1&2&3\\λ&0&1\\2&1&1\end{pmatrix},则R(A)\mathrm{的最小值为}(\;).\;
$$
$$
A.
1 \quad B.2 \quad C.3 \quad D.\mathrm{无法确定} \quad E. \quad F. \quad G. \quad H.
$$
$$
\begin{array}{l}\left|A\right|=\begin{vmatrix}1&2&3\\λ&0&1\\2&1&1\end{vmatrix}=λ+3,当λ\neq-3时,\left|A\right|\neq0,R(A)=3;\\当λ=-3时,A=\begin{pmatrix}1&2&3\\-3&0&1\\2&1&1\end{pmatrix}\rightarrow\begin{pmatrix}1&2&3\\0&6&10\\0&-3&-5\end{pmatrix}\rightarrow\begin{pmatrix}1&2&3\\0&6&10\\0&0&0\end{pmatrix},R(A)=2.\end{array}
$$



$$
设A为n\mathrm{阶非零方阵},且A\neq E,A^2=A(E为n\mathrm{阶单位矩阵}),则(\;).
$$
$$
A.
A\mathrm{的秩为}n \quad B.A\mathrm{的秩为}0 \quad C.A\mathrm{的秩小于}n,\mathrm{但不为}0 \quad D.A\mathrm{的秩大于}n \quad E. \quad F. \quad G. \quad H.
$$
$$
\begin{array}{l}A^2=A⇒ A(A-E)=O,\mathrm{由矩阵秩的性质可得}R(A)+R(A-E)\leq n,又\;\;\\\;\;\;\;\;\;\;\;\;\;\;\;\;\;\;\;\;\;\;\;\;\;\;\;\;\;\;\;\;\;\;\;\;\;\;\;\;\;\;\;\;\;\;\;\;\;\;\;\;\;\;A\neq E⇒ A-E\neq O⇒ R(A-E)\geq1,则R(A)\leq n-1.\;\;\\\;\;\;\;\;\;\;\;\;\;\;\;\;\;\;\;\;\;\;\;\;\;\;\;\;\;\;\;\;\;\;\;\;\;\;\;\;\;\;\;\;\;\;\;\;\;\;\;\;\;\;\;\;\;\;\;\;\;\;\;\;\;\;\;\;\;\;\;\;\;\;\;\;\;\;\;\;A\neq O⇒ R(A)\geq1.\;\;\\故A\mathrm{的秩小于}n,\mathrm{但不为}0.\end{array}
$$



$$
\mathrm{设矩阵}A=\begin{pmatrix}k&1&1&1\\1&k&1&1\\1&1&k&1\\1&1&1&k\end{pmatrix}\mathrm{的秩等于}3,则\;k=(\;)
$$
$$
A.
3 \quad B.-3 \quad C.2 \quad D.-2 \quad E. \quad F. \quad G. \quad H.
$$
$$
\begin{array}{l}A=\begin{pmatrix}k&1&1&1\\1&k&1&1\\1&1&k&1\\1&1&1&k\end{pmatrix}\rightarrow\begin{pmatrix}k+3&k+3&k+3&k+3\\1&k&1&1\\1&1&k&1\\1&1&1&k\end{pmatrix}=B,\\当k=-3时,B\rightarrow\begin{pmatrix}0&0&0&0\\1&-3&1&1\\1&1&-3&1\\1&1&1&-3\end{pmatrix}\rightarrow\begin{pmatrix}1&1&1&-3\\0&-4&0&4\\0&0&-4&4\\0&0&0&0\end{pmatrix}\\当k\neq-3时,B\rightarrow\begin{pmatrix}1&1&1&1\\1&k&1&1\\1&1&k&1\\1&1&1&k\end{pmatrix}\rightarrow\begin{pmatrix}1&1&1&1\\0&k-1&0&0\\0&0&k-1&0\\0&0&0&k-1\end{pmatrix}\\当k=1时,\;R(A)=R(B)=1;当k\neq1时R(A)=R(B)=4\end{array}
$$



$$
\mathrm{已知}A=\begin{pmatrix}1&1&1&1\\0&1&-1&b\\2&3&1&4\\3&5&1&7\end{pmatrix}\;,且R(A)=3,则b\mathrm{满足}(\;\;\;\;\;)
$$
$$
A.
b=2 \quad B.b\neq2 \quad C.b\neq-2 \quad D.b\neq1 \quad E. \quad F. \quad G. \quad H.
$$
$$
A=\begin{pmatrix}1&1&1&1\\0&1&-1&b\\2&3&1&4\\3&5&1&7\end{pmatrix}\;\rightarrow\begin{pmatrix}1&1&1&1\\0&1&-1&b\\0&0&0&2-b\\0&0&0&0\end{pmatrix}\;,\mathrm{由于}R(A)=3\mathrm{所以}b\neq2\;
$$



$$
设n(n\geq3)\mathrm{阶矩阵}A=\begin{pmatrix}1&a&a&⋯&a\\a&1&a&⋯&a\\a&a&1&⋯&a\\⋯&⋯&⋯&⋯&⋯\\a&a&a&⋯&1\end{pmatrix},若R(A)=n-1,则a\mathrm{的值为}\;(\;).
$$
$$
A.
1 \quad B.n-1 \quad C.\frac1{1-n} \quad D.\frac1n \quad E. \quad F. \quad G. \quad H.
$$
$$
\begin{array}{l}由R(A)=n-1知,\left|A\right|=0,\mathrm{把矩阵}A\mathrm{的各列加到第列后提取公因子},\mathrm{再把得到的矩阵化为三角阵可知}:\;\;\\\;\;\;\;\;\;\;\;\;\;\;\;\;\;\;\;\;\;\;\;\;\;\;\;\;\;\;\;\;\;\;\;\;\;\;\;\;\;\;\;\;\;\;\;\;\;\;\;\;\;\;\;\;\;\;\;\;\;\;\;\;\;\;\;\;\;\;\;\left|A\right|\;=\left[1+(n-1)a\right](1-a)^{n-1},\;\;\\又\left|A\right|=0,若a=1,则R(A)=1\neq n-1,\mathrm{故有}1+(n-1)a=0,即a=\frac1{1-n}.\end{array}
$$



$$
设A为4×3\mathrm{型矩阵},且R(A)=2,又B=\;\begin{pmatrix}4&0&2\\0&2&0\\1&0&3\end{pmatrix},则R(AB)-R(A)=(\;)
$$
$$
A.
0 \quad B.1 \quad C.2 \quad D.3 \quad E. \quad F. \quad G. \quad H.
$$
$$
B\mathrm{可逆},则B\mathrm{可表示为一系列初等矩阵的乘积},\mathrm{又初等变换不改变矩阵的秩},故R(AB)=R(A).
$$



$$
\begin{array}{l}\mathrm{下列命题中正确的有}(\;).\\(1)设A为m× n\mathrm{型矩阵},b为m×1\mathrm{型矩阵},则\;R(A)\leq R(Ab);\\(2)\mathrm{从矩阵}A\mathrm{中划去一行得到矩阵}B,\;则\;R(A)\geq R(B);\\(3)\mathrm{在秩是}r\mathrm{的矩阵中},\mathrm{没有等于}0的r-1\mathrm{阶子式},\mathrm{可能存在等于}0的r\mathrm{阶子式}.\end{array}
$$
$$
A.
0个 \quad B.1个 \quad C.2个 \quad D.3个 \quad E. \quad F. \quad G. \quad H.
$$
$$
\begin{array}{l}(1)\mathrm{由题设条件和矩阵的秩的概念},\mathrm{易见}\\\;\;\;\;\;\;\;\;\;\;\;\;\;\;\;\;\;\;\;\;\;\;\;\;\;\;\;\;\;\;\;\;\;\;\;\;\;\;\;\;R(A)\leq R(Ab)\;\;\leq R(A)+1.\;\;\\(2)设\;R(B)=r且b\mathrm{的某个}r\mathrm{阶子式}D_r\neq0.\;\mathrm{矩阵}B\mathrm{是由矩阵}A\mathrm{划去一行得到的},\mathrm{所以在}A\mathrm{中能找到与}D_r\mathrm{相同的}r\mathrm{阶子式}\;{\overline D}_r\mathrm{由于}\;\;\\\;\;\;\;\;\;\;\;\;\;\;\;\;\;\;\;\;\;\;\;\;\;\;\;\;\;\;\;\;\;\;\;\;\;\;\;\;\;\;\;\;\;\;\;\;\;\;\;\;\;\;\;\;\;\;\;{\overline D}_r=D_r\neq0,\\\mathrm{故而}\;\;R(A)\geq R(B).\\(3)\mathrm{在秩}r\mathrm{是的矩阵中},\mathrm{可能存在等于}0的r-1\mathrm{阶子式},\mathrm{也可能存在等于}0的r\mathrm{阶子式}.\\\mathrm{例如}A=\begin{pmatrix}1&0&0&0\\0&1&0&0\\0&0&1&0\\0&0&0&0\\0&0&0&0\end{pmatrix},R(A)=3,\mathrm{同时存在等于}0的r\mathrm{阶子式}.\mathrm{因此正确的为}(1)(2)\end{array}
$$



$$
\mathrm{已知}Q=\begin{pmatrix}1&2&3\\2&4&t\\3&6&9\end{pmatrix},P\mathrm{为三阶非零矩阵},\mathrm{且满足}PQ=O,则(\;).\;
$$
$$
A.
当t=6时,R(P)=1 \quad B.当t=6时,R(P)=2 \quad C.当t\neq6时,R(P)=1 \quad D.当t\neq6时,R(P)=2 \quad E. \quad F. \quad G. \quad H.
$$
$$
\begin{array}{l}PQ=O,\mathrm{则由矩阵秩的性质可知}:R(P)+R(Q)\leq3,又P\neq O⇒ R(P)\geq1.\;\;\\当t=6时,R(Q)=1,故1\leq R(P)\leq2;\;\\当t\neq6时,R(Q)=2,故R(P)=1.\end{array}
$$



$$
设A^* 为n\mathrm{阶方阵}(n\geq2),A\mathrm{的伴随矩阵},\mathrm{则下列选项正确的是}(\;).
$$
$$
A.
若A\mathrm{的秩为}1,则A^*\mathrm{的秩也为}1 \quad B.若A\mathrm{的秩为}n-1,则A^*\mathrm{的秩也为}n-1 \quad C.若A\mathrm{为满秩方阵},则A^*\mathrm{也是满秩方阵} \quad D.若A\mathrm{为非满秩方阵},则A^*\mathrm{是满秩方阵} \quad E. \quad F. \quad G. \quad H.
$$
$$
\begin{array}{l}设A为n阶(n\geq2)\mathrm{方阵},则R(A^*)=\left\{\begin{array}{ccc}n&当&R(A)=n\\1&当&R(A)=n-1\\0&当&R(A)<\;n-1\end{array}\right..\;\;\\若A\mathrm{为非零矩阵},\mathrm{且当}R(A)<\;n-1时,A\mathrm{的所有}n-1\mathrm{阶子式即}A^*\mathrm{的任一元素均为零},\mathrm{于是}A^*=O.\end{array}
$$



$$
设\begin{pmatrix}x_1&x_2\\-3&4\end{pmatrix}\begin{pmatrix}3&-2\\y_1&y_2\end{pmatrix}=5\begin{pmatrix}1&0\\-1&2\end{pmatrix},则\left(y_{1,}y_2\right)=\left(\;\;\;\right).
$$
$$
A.
\left(1,2\right) \quad B.\left(2,1\right) \quad C.\left(1,1\right) \quad D.\left(1,-1\right) \quad E. \quad F. \quad G. \quad H.
$$
$$
\begin{array}{l}设\begin{pmatrix}x_1&x_2\\-3&4\end{pmatrix}\begin{pmatrix}3&-2\\y_1&y_2\end{pmatrix}=5\begin{pmatrix}1&0\\-1&2\end{pmatrix},则\\\;\;\;\;\;\;\;\;\;\;\;\;\;\;\;\;\;\;\;\;\;\;\;\;\;\;\;\;\;\;\;\;\;\;\;\;\;\;\;\;\;\;\;\;\;\;\;\;\;\;\;\;\;\;\;\;\;\;\;\left\{\begin{array}{l}-9+4y_1=-5\\6+4y_2=10\end{array}\right.⇒\left\{\begin{array}{l}y_1=1\\y_2=1\end{array}\right.\\\end{array}
$$



$$
设\begin{pmatrix}x_1&x_2\\-3&4\end{pmatrix}\begin{pmatrix}3&-2\\y_1&y_2\end{pmatrix}=\begin{pmatrix}1&0\\3&2\end{pmatrix},则\left(y_{1,}y_2\right)=\left(\;\;\;\right).
$$
$$
A.
\left(3,1\right) \quad B.\left(3,-1\right) \quad C.\left(1,-1\right) \quad D.\left(1,2\right) \quad E. \quad F. \quad G. \quad H.
$$
$$
\begin{array}{l}设\begin{pmatrix}x_1&x_2\\-3&4\end{pmatrix}\begin{pmatrix}3&-2\\y_1&y_2\end{pmatrix}=\begin{pmatrix}1&0\\3&2\end{pmatrix},则\\\;\;\;\;\;\;\;\;\;\;\;\;\;\;\;\;\;\;\;\;\;\;\;\;\;\;\;\;\;\;\;\;\;\;\;\;\;\;\;\;\;\;\;\;\;\;\;\;\;\;\;\;\;\;\;\;\;\;\;\;\;\;\;\;\left\{\begin{array}{l}-9+4y_1=3\\6+4y_2=2\end{array}\right.⇒\left\{\begin{array}{l}y_1=3\\y_2=-1\end{array}\right.\\\;\;\;\;\;\;\;\;\;\;\;\;\;\;\;\;\;\;\;\;\;\;\;\;\;\;\;\;\;\;\;\;\;\;\;\;\;\;\;\;\;\;\;\;\;\;\;\;\;\;\;\;\;\;\;\;\;\;\;\;\;\;\;\end{array}
$$



$$
设\begin{pmatrix}1&0\\-1&1\end{pmatrix}\begin{pmatrix}x_1&y_1\\x_2&y_2\end{pmatrix}=\begin{pmatrix}1&-1\\0&1\end{pmatrix},则\begin{pmatrix}x_1&y_1\\x_2&y_2\end{pmatrix}=\left(\;\;\;\right)
$$
$$
A.
\begin{pmatrix}0&1\\-1&1\end{pmatrix} \quad B.\begin{pmatrix}1&-1\\1&0\end{pmatrix} \quad C.\begin{pmatrix}1&-1\\1&1\end{pmatrix} \quad D.\begin{pmatrix}1&1\\0&-1\end{pmatrix} \quad E. \quad F. \quad G. \quad H.
$$
$$
\begin{pmatrix}1&0\\-1&1\end{pmatrix}^{-1}\begin{pmatrix}1&-1\\0&1\end{pmatrix}=\begin{pmatrix}1&0\\1&1\end{pmatrix}\begin{pmatrix}1&-1\\0&1\end{pmatrix}=\begin{pmatrix}1&-1\\1&0\end{pmatrix}
$$



$$
设\begin{pmatrix}x_1&x_2\\y_1&y_2\end{pmatrix}\begin{pmatrix}1&-1\\0&1\end{pmatrix}=\begin{pmatrix}2&1\\1&0\end{pmatrix},则\begin{pmatrix}x_1&x_2\\y_1&y_2\end{pmatrix}=\left(\;\;\;\right).
$$
$$
A.
\begin{pmatrix}2&3\\1&1\end{pmatrix} \quad B.\begin{pmatrix}2&1\\3&1\end{pmatrix} \quad C.\begin{pmatrix}1&1\\1&0\end{pmatrix} \quad D.\begin{pmatrix}1&1\\0&1\end{pmatrix} \quad E. \quad F. \quad G. \quad H.
$$
$$
\begin{pmatrix}x_1&x_2\\y_1&y_2\end{pmatrix}=\begin{pmatrix}2&1\\1&0\end{pmatrix}\begin{pmatrix}1&-1\\0&1\end{pmatrix}^{-1}=\begin{pmatrix}2&1\\1&0\end{pmatrix}\begin{pmatrix}1&1\\0&1\end{pmatrix}=\begin{pmatrix}2&3\\1&1\end{pmatrix}.
$$



$$
设\begin{pmatrix}1&2\\-1&1\end{pmatrix}\begin{pmatrix}x_1&y_1\\x_2&y_2\end{pmatrix}=\begin{pmatrix}6&-3\\0&3\end{pmatrix},则\begin{pmatrix}x_1&y_1\\x_2&y_2\end{pmatrix}=\left(\;\;\;\right).
$$
$$
A.
\begin{pmatrix}2&-3\\2&0\end{pmatrix} \quad B.\begin{pmatrix}1&-3\\2&0\end{pmatrix} \quad C.\begin{pmatrix}1&3\\1&0\end{pmatrix} \quad D.\begin{pmatrix}1&-3\\1&0\end{pmatrix} \quad E. \quad F. \quad G. \quad H.
$$
$$
\begin{pmatrix}x_1&y_1\\x_2&y_2\end{pmatrix}=\begin{pmatrix}1&2\\-1&1\end{pmatrix}^{-1}\begin{pmatrix}6&-3\\0&3\end{pmatrix}=\begin{pmatrix}\frac13&-\frac23\\\frac13&\frac13\end{pmatrix}\begin{pmatrix}6&-3\\0&3\end{pmatrix}=\begin{pmatrix}2&-3\\2&0\end{pmatrix}
$$



$$
\begin{pmatrix}1&2\\3&-1\end{pmatrix}\begin{pmatrix}x_1&y_1\\x_2&y_2\end{pmatrix}=\begin{pmatrix}1&11\\17&5\end{pmatrix},则\left(x_1,y_1\right)=\;\left(\;\;\;\right).
$$
$$
A.
\left(-2,4\right) \quad B.\left(5,3\right) \quad C.\left(5,-2\right) \quad D.\left(3,4\right) \quad E. \quad F. \quad G. \quad H.
$$
$$
\begin{array}{l}\begin{pmatrix}1&2\\3&-1\end{pmatrix}\begin{pmatrix}x_1&y_1\\x_2&y_2\end{pmatrix}=\begin{pmatrix}1&11\\17&5\end{pmatrix},则\\\;\;\;\;\;\;\;\;\;\;\;\;\;\;\;\;\;\;\;\;\;\;\;\;\;\;\;\;\;\;\;\;\begin{pmatrix}x_1&y_1\\x_2&y_2\end{pmatrix}=\begin{pmatrix}1&2\\3&-1\end{pmatrix}^{-1}\;\begin{pmatrix}1&11\\17&5\end{pmatrix}=-\frac17\begin{pmatrix}-1&-2\\-3&1\end{pmatrix}\begin{pmatrix}1&11\\17&5\end{pmatrix}=\begin{pmatrix}5&3\\-2&4\end{pmatrix}\end{array}
$$



$$
\begin{pmatrix}1&2\\3&-1\end{pmatrix}\begin{pmatrix}x_1&y_1\\x_2&y_2\end{pmatrix}=\begin{pmatrix}1&11\\17&5\end{pmatrix},则\left(x_2,y_2\right)=\;\left(\;\;\;\right).
$$
$$
A.
\left(-2,4\right) \quad B.\left(5,3\right) \quad C.\left(5,-2\right) \quad D.\left(3,4\right) \quad E. \quad F. \quad G. \quad H.
$$
$$
\begin{array}{l}\begin{pmatrix}1&2\\3&-1\end{pmatrix}\begin{pmatrix}x_1&y_1\\x_2&y_2\end{pmatrix}=\begin{pmatrix}1&11\\17&5\end{pmatrix},则\\\;\;\;\;\;\;\;\;\;\;\;\;\;\;\;\;\;\;\;\;\;\;\;\;\;\;\;\;\;\;\;\begin{pmatrix}x_1&y_1\\x_2&y_2\end{pmatrix}=\begin{pmatrix}1&2\\3&-1\end{pmatrix}^{-1}\begin{pmatrix}1&11\\17&5\end{pmatrix}=-\frac17\begin{pmatrix}-1&-2\\-3&1\end{pmatrix}\begin{pmatrix}1&11\\17&5\end{pmatrix}=\begin{pmatrix}5&3\\-2&4\end{pmatrix}\end{array}
$$



$$
\mathrm{已知}A=\begin{pmatrix}3&-1&2&0\\1&5&2&1\end{pmatrix},B=\begin{pmatrix}5&-1&3&1\\7&5&4&2\end{pmatrix}且A+2X=B+X,则X=(\;).
$$
$$
A.
\begin{pmatrix}2&0&1&1\\5&0&2&1\end{pmatrix} \quad B.\begin{pmatrix}2&0&1&1\\6&0&2&1\end{pmatrix} \quad C.\begin{pmatrix}2&0&1&1\\6&0&1&1\end{pmatrix} \quad D.\begin{pmatrix}2&0&2&1\\6&0&2&1\end{pmatrix} \quad E. \quad F. \quad G. \quad H.
$$
$$
\begin{array}{l}X=B-A=\begin{pmatrix}5&-1&3&1\\7&5&4&2\end{pmatrix}-\begin{pmatrix}3&-1&2&0\\1&5&2&1\end{pmatrix}\\=\begin{pmatrix}2&0&1&1\\6&0&2&1\end{pmatrix}\end{array}
$$



$$
设A=\begin{pmatrix}1&2&3\\0&1&1\end{pmatrix},B=\begin{pmatrix}1&0&1\\2&1&1\end{pmatrix},若Y\;\mathrm{满足}(2A+Y)+2(B-Y)=O,则Y=(\;).
$$
$$
A.
\begin{pmatrix}4&4&8\\4&4&4\end{pmatrix} \quad B.\begin{pmatrix}8&4&8\\4&4&4\end{pmatrix} \quad C.\begin{pmatrix}4&4&8\\4&8&4\end{pmatrix} \quad D.\begin{pmatrix}4&4&8\\4&4&8\end{pmatrix} \quad E. \quad F. \quad G. \quad H.
$$
$$
Y=2A+2B=\begin{pmatrix}4&4&8\\4&4&4\end{pmatrix}
$$



$$
设\begin{pmatrix}1&2\\3&-4\end{pmatrix}\begin{pmatrix}x_1&x_2\\y_1&1\end{pmatrix}=\begin{pmatrix}0&1\\10&t\end{pmatrix},则x_2=\left(\;\;\;\right).
$$
$$
A.
1 \quad B.-1 \quad C.2 \quad D.-2 \quad E. \quad F. \quad G. \quad H.
$$
$$
\begin{array}{l}\begin{pmatrix}x_1&x_2\\y_1&1\end{pmatrix}=\begin{pmatrix}1&2\\3&-4\end{pmatrix}^{-1}\begin{pmatrix}0&1\\10&t\end{pmatrix}=-\frac1{10}\begin{pmatrix}-4&-2\\-3&1\end{pmatrix}\begin{pmatrix}0&1\\10&t\;\end{pmatrix},则\\\left\{\begin{array}{l}-\frac1{10}\left(-3+t\right)=1\\-\frac1{10}\left(-4-2t\right)=x_2\end{array}\right.⇒\left\{\begin{array}{l}t=-7\\x_2=-1\end{array}\right.\end{array}
$$



$$
设\begin{pmatrix}1&2\\3&-4\end{pmatrix}\begin{pmatrix}x_1&x_2\\y_1&1\end{pmatrix}=\begin{pmatrix}0&1\\10&t\end{pmatrix},则t=\left(\;\;\;\right).
$$
$$
A.
1 \quad B.-1 \quad C.7 \quad D.-7 \quad E. \quad F. \quad G. \quad H.
$$
$$
\begin{array}{l}\begin{pmatrix}x_1&x_2\\y_1&1\end{pmatrix}=\begin{pmatrix}1&2\\3&-4\end{pmatrix}^{-1}\begin{pmatrix}0&1\\10&t\end{pmatrix}=-\frac1{10}\begin{pmatrix}-4&-2\\-3&1\end{pmatrix}\begin{pmatrix}0&1\\10&t\;\end{pmatrix},则\\\left\{\begin{array}{l}-\frac1{10}\left(-3+t\right)=1\\-\frac1{10}\left(-4-2t\right)=x_2\end{array}\right.⇒\left\{\begin{array}{l}t=-7\\x_2=-1\end{array}\right.\end{array}
$$



$$
设A=\begin{pmatrix}1&-2\\0&1\end{pmatrix},g\left(x\right)=\begin{vmatrix}x&-1\\-3&x+1\end{vmatrix},则g\left(A\right)=\left(\;\;\;\right).
$$
$$
A.
\begin{pmatrix}-1&-6\\0&-1\end{pmatrix} \quad B.\begin{pmatrix}1&6\\0&1\end{pmatrix} \quad C.\begin{pmatrix}-1&6\\0&-1\end{pmatrix} \quad D.\begin{pmatrix}1&-6\\0&1\end{pmatrix} \quad E. \quad F. \quad G. \quad H.
$$
$$
\begin{array}{l}g\left(x\right)=\begin{vmatrix}x&-1\\-3&x+1\end{vmatrix}=x^2+x-3,g\left(A\right)=A^2+A-3E,则\\\;\;\;\;\;\;\;\;\;\;\;\;\;\;\;\;\;\;\;\;\;\;\;\;\;\;\;\;g\left(A\right)=\begin{pmatrix}1&-2\\0&1\end{pmatrix}^2+\begin{pmatrix}1&-2\\0&1\end{pmatrix}-3\begin{pmatrix}1&0\\0&1\end{pmatrix}=\begin{pmatrix}-1&-6\\0&-1\end{pmatrix}\end{array}
$$



$$
\mathrm{矩阵方程}\begin{pmatrix}2&1\\1&2\end{pmatrix}X=\begin{pmatrix}1&2\\-1&4\end{pmatrix},\mathrm{则二阶矩阵}X为\left(\;\;\;\right).
$$
$$
A.
\begin{pmatrix}1&0\\-1&2\end{pmatrix} \quad B.\begin{pmatrix}1&0\\1&2\end{pmatrix} \quad C.\begin{pmatrix}1&4\\-1&2\end{pmatrix} \quad D.\begin{pmatrix}2&0\\-1&1\end{pmatrix} \quad E. \quad F. \quad G. \quad H.
$$
$$
\begin{array}{l}\\X=\begin{pmatrix}2&1\\1&2\end{pmatrix}^{-1}\begin{pmatrix}1&2\\-1&4\end{pmatrix}=\frac13\begin{pmatrix}2&-1\\-1&2\end{pmatrix}\begin{pmatrix}1&2\\-1&4\end{pmatrix}=\begin{pmatrix}1&0\\-1&2\end{pmatrix}\end{array}
$$



$$
设A=\begin{pmatrix}1&2\\0&1\end{pmatrix},B=\begin{pmatrix}1&1\\1&0\end{pmatrix},C=\begin{pmatrix}1&0\\1&1\end{pmatrix},\mathrm{满足}AX+B=3C,则X=\left(\;\;\;\right).
$$
$$
A.
\begin{pmatrix}-2&-7\\2&3\end{pmatrix} \quad B.\begin{pmatrix}-2&7\\2&-3\end{pmatrix} \quad C.\begin{pmatrix}2&-7\\2&-3\end{pmatrix} \quad D.\begin{pmatrix}2&7\\-2&-3\end{pmatrix} \quad E. \quad F. \quad G. \quad H.
$$
$$
\begin{array}{l}AX=3C-B=\begin{pmatrix}2&-1\\2&3\end{pmatrix},\\X=A^{-1}\begin{pmatrix}2&-1\\2&3\end{pmatrix}=\begin{pmatrix}1&-2\\0&1\end{pmatrix}\begin{pmatrix}2&-1\\2&3\end{pmatrix}=\begin{pmatrix}-2&-7\\2&3\end{pmatrix}\end{array}
$$



$$
设AX+E=A^2+X,且A=\begin{pmatrix}1&0&1\\0&2&0\\1&0&1\end{pmatrix},则X=\left(\;\;\;\right).
$$
$$
A.
\begin{pmatrix}2&0&1\\0&3&0\\0&0&2\end{pmatrix} \quad B.\begin{pmatrix}2&0&0\\0&3&0\\1&0&2\end{pmatrix} \quad C.\begin{pmatrix}2&0&1\\0&3&0\\1&0&2\end{pmatrix} \quad D.\begin{pmatrix}2&0&1\\0&1&0\\1&0&2\end{pmatrix} \quad E. \quad F. \quad G. \quad H.
$$
$$
\left(A-E\right)X=A^2-E,而A-E=\begin{pmatrix}0&0&1\\0&1&0\\1&0&0\end{pmatrix}\mathrm{可逆},故X=\left(A-E\right)^{-1}\left(A^2-E\right)=A+E=\begin{pmatrix}2&0&1\\0&3&0\\1&0&2\end{pmatrix}
$$



$$
设A=\begin{pmatrix}1&2\\3&4\end{pmatrix},B=\begin{pmatrix}1&2\\0&1\end{pmatrix},且AB+X=2B;则X=\left(\;\;\;\right).
$$
$$
A.
\begin{pmatrix}1&0\\3&8\end{pmatrix} \quad B.\begin{pmatrix}1&0\\-3&-8\end{pmatrix} \quad C.\begin{pmatrix}1&4\\3&10\end{pmatrix} \quad D.\begin{pmatrix}1&4\\-3&-10\end{pmatrix} \quad E. \quad F. \quad G. \quad H.
$$
$$
由AB+X=2B得,X=2B-AB=2\begin{pmatrix}1&2\\0&1\end{pmatrix}-\begin{pmatrix}1&2\\3&4\end{pmatrix}\begin{pmatrix}1&2\\0&1\end{pmatrix}=\begin{pmatrix}1&0\\-3&-8\end{pmatrix}
$$



$$
设A=\begin{pmatrix}1&2\\3&4\end{pmatrix},B=\begin{pmatrix}1&0\\2&1\end{pmatrix},且A+XB=2B;则X=\left(\;\;\;\right).
$$
$$
A.
\begin{pmatrix}5&-2\\5&2\end{pmatrix} \quad B.\begin{pmatrix}5&2\\5&-2\end{pmatrix} \quad C.\begin{pmatrix}5&-2\\5&-2\end{pmatrix} \quad D.\begin{pmatrix}-5&-2\\-5&-2\end{pmatrix} \quad E. \quad F. \quad G. \quad H.
$$
$$
由A+XB=2B得,X=\left(2B-A\right)B^{-1}=\left[2\begin{pmatrix}1&0\\2&1\end{pmatrix}-\begin{pmatrix}1&2\\3&4\end{pmatrix}\right]\begin{pmatrix}1&0\\2&1\end{pmatrix}^{-1}=\begin{pmatrix}1&-2\\1&-2\end{pmatrix}\begin{pmatrix}1&0\\-2&1\end{pmatrix}=\begin{pmatrix}5&-2\\5&-2\end{pmatrix}
$$



$$
设A=\begin{pmatrix}-1&5\\3&-3\end{pmatrix},B=\begin{pmatrix}1&0\\-1&1\end{pmatrix},AB+X=2B,则X=\left(\;\;\;\right).
$$
$$
A.
\begin{pmatrix}8&-5\\-8&5\end{pmatrix} \quad B.\begin{pmatrix}-8&5\\8&-5\end{pmatrix} \quad C.\begin{pmatrix}8&-5\\8&-5\end{pmatrix} \quad D.\begin{pmatrix}-8&-5\\8&-5\end{pmatrix} \quad E. \quad F. \quad G. \quad H.
$$
$$
由AB+X=2B得,X=2B-AB=\begin{pmatrix}8&-5\\-8&5\end{pmatrix}.
$$



$$
设f\left(A\right)=A^2-5A,且A=\begin{pmatrix}2&1\\3&3\end{pmatrix},则f\left(A\right)=\left(\;\;\;\right).
$$
$$
A.
\begin{pmatrix}3&0\\0&3\end{pmatrix} \quad B.\begin{pmatrix}-3&0\\0&-3\end{pmatrix} \quad C.\begin{pmatrix}-3&0\\0&3\end{pmatrix} \quad D.\begin{pmatrix}3&0\\0&-3\end{pmatrix} \quad E. \quad F. \quad G. \quad H.
$$
$$
f\left(A\right)=\begin{pmatrix}2&1\\3&3\end{pmatrix}^2-5\begin{pmatrix}2&1\\3&3\end{pmatrix}=\begin{pmatrix}7&5\\15&12\end{pmatrix}-\begin{pmatrix}10&5\\15&15\end{pmatrix}=\begin{pmatrix}-3&0\\0&-3\end{pmatrix}
$$



$$
设AX=B,\mathrm{其中},A=\begin{pmatrix}3&5\\1&2\end{pmatrix},B=\begin{pmatrix}4&-1&2\\3&0&-1\end{pmatrix},则X=\left(\;\;\;\right)
$$
$$
A.
\begin{pmatrix}-7&-2&9\\5&1&-5\end{pmatrix} \quad B.\begin{pmatrix}7&2&-9\\5&1&-5\end{pmatrix} \quad C.\begin{pmatrix}-7&-2&9\\-5&1&5\end{pmatrix} \quad D.\begin{pmatrix}-7&-2&9\\5&-1&5\end{pmatrix} \quad E. \quad F. \quad G. \quad H.
$$
$$
X=A^{-1}B=\begin{pmatrix}2&-5\\-1&3\end{pmatrix}\begin{pmatrix}4&-1&2\\3&0&-1\end{pmatrix}=\begin{pmatrix}-7&-2&9\\5&1&-5\end{pmatrix}
$$



$$
\begin{pmatrix}1&2\\3&4\end{pmatrix}X=\begin{pmatrix}5&3\\3&6\end{pmatrix},则X=\left(\;\;\;\;\right)
$$
$$
A.
\begin{pmatrix}7&0\\6&\frac32\end{pmatrix} \quad B.\begin{pmatrix}-7&0\\6&-\frac32\end{pmatrix} \quad C.\begin{pmatrix}7&0\\-6&\frac32\end{pmatrix} \quad D.\begin{pmatrix}-7&0\\6&\frac32\end{pmatrix} \quad E. \quad F. \quad G. \quad H.
$$
$$
\mathrm{由于}\begin{pmatrix}1&2\\3&4\end{pmatrix}\mathrm{可逆},故A=\begin{pmatrix}1&2\\3&4\end{pmatrix}^{-1}\begin{pmatrix}5&3\\3&6\end{pmatrix}=\begin{pmatrix}-7&0\\6&\frac32\end{pmatrix}
$$



$$
\mathrm{已知}\begin{pmatrix}1&-1&-1\\2&-1&-3\\3&2&-5\end{pmatrix}\begin{pmatrix}x_1\\x_2\\x_3\end{pmatrix}=\begin{pmatrix}2\\1\\0\end{pmatrix},则x_2=\left(\;\;\;\;\right)
$$
$$
A.
5 \quad B.1 \quad C.0 \quad D.3 \quad E. \quad F. \quad G. \quad H.
$$
$$
\mathrm{由矩阵的乘法}:\left\{\begin{array}{l}\begin{array}{c}x_1-x_2-x_3=2\\2x_1-x_2-3x_3=1\end{array}\\\;3x_1+2x_2-5x_3=0\end{array}\right.\;得x_2=0
$$



$$
\mathrm{已知}\begin{pmatrix}1&-1&-1\\2&-1&-3\\3&2&-5\end{pmatrix}\begin{pmatrix}x_1\\x_2\\x_3\end{pmatrix}=\begin{pmatrix}2\\1\\0\end{pmatrix},则x_1=\left(\;\;\;\;\right)
$$
$$
A.
2 \quad B.5 \quad C.4 \quad D.9 \quad E. \quad F. \quad G. \quad H.
$$
$$
\mathrm{由矩阵的乘法}:\left\{\begin{array}{l}\begin{array}{c}x_1-x_2-x_3=2\\2x_1-x_2-3x_3=1\end{array}\\\;3x_1+2x_2-5x_3=0\end{array}\right.\;得x_1=5
$$



$$
\mathrm{已知}\begin{pmatrix}1&-1&-1\\2&-1&-3\\3&2&-5\end{pmatrix}\begin{pmatrix}x_1\\x_2\\x_3\end{pmatrix}=\begin{pmatrix}2\\1\\0\end{pmatrix},则x_3=\left(\;\;\;\;\right)
$$
$$
A.
3 \quad B.0 \quad C.5 \quad D.1 \quad E. \quad F. \quad G. \quad H.
$$
$$
\mathrm{由矩阵的乘法}:\left\{\begin{array}{l}\begin{array}{c}x_1-x_2-x_3=2\\2x_1-x_2-3x_3=1\end{array}\\\;3x_1+2x_2-5x_3=0\end{array}\right.\;得x_3=3
$$



$$
\begin{pmatrix}1&1&-1\\-2&1&1\\1&1&1\end{pmatrix}X=\begin{pmatrix}2\\3\\6\end{pmatrix},则X=\left(\;\;\;\right).
$$
$$
A.
\begin{pmatrix}1\\3\\2\end{pmatrix} \quad B.\begin{pmatrix}1\\2\\2\end{pmatrix} \quad C.\begin{pmatrix}3\\2\\1\end{pmatrix} \quad D.\begin{pmatrix}2\\3\\1\end{pmatrix} \quad E. \quad F. \quad G. \quad H.
$$
$$
\mathrm{由矩阵的乘法即得答案}X=\begin{pmatrix}1\\3\\2\end{pmatrix}
$$



$$
\begin{array}{l}设X\mathrm{可逆},且\begin{pmatrix}a_{11}&a_{12}\\a_{21}&a_{22}\\a_{31}&a_{32}\end{pmatrix}X=\begin{pmatrix}a_{11}+ka_{12}&a_{12}\\a_{21}+ka_{22}&a_{22}\\a_{31}+ka_{32}&a_{32}\end{pmatrix},则X=\left(\;\;\;\right).\\\\\end{array}
$$
$$
A.
\begin{pmatrix}k&0\\0&1\end{pmatrix} \quad B.\begin{pmatrix}1&0\\0&k\end{pmatrix} \quad C.\begin{pmatrix}1&k\\0&1\end{pmatrix} \quad D.\begin{pmatrix}1&0\\k&1\end{pmatrix} \quad E. \quad F. \quad G. \quad H.
$$
$$
\mathrm{矩阵}\begin{pmatrix}a_{11}+ka_{12}&a_{12}\\a_{21}+ka_{22}&a_{22}\\a_{31}+ka_{32}&a_{32}\end{pmatrix}\mathrm{相对于是矩阵}\begin{pmatrix}a_{11}&a_{12}\\a_{21}&a_{22}\\a_{31}&a_{32}\end{pmatrix}\mathrm{的第二列乘以数}k\mathrm{加到第一列得到的},\mathrm{根据初等矩阵的性质},\mathrm{答案为}D
$$



$$
\begin{pmatrix}1&4\\-1&2\end{pmatrix}X\begin{pmatrix}2&0\\-1&1\end{pmatrix}=\begin{pmatrix}3&1\\0&-1\end{pmatrix},则X=\left(\;\;\;\right).
$$
$$
A.
\begin{pmatrix}1&1\\\frac14&0\end{pmatrix} \quad B.\begin{pmatrix}1&1\\-\frac14&0\end{pmatrix} \quad C.\begin{pmatrix}1&-1\\\frac14&0\end{pmatrix} \quad D.\begin{pmatrix}-1&1\\\frac14&0\end{pmatrix} \quad E. \quad F. \quad G. \quad H.
$$
$$
\begin{array}{l}记\;A=\begin{pmatrix}1&4\\-1&2\end{pmatrix},B=\begin{pmatrix}2&0\\-1&1\end{pmatrix},C=\begin{pmatrix}3&1\\0&-1\end{pmatrix},即AXB=C,\\\left|A\right|=6,\left|B\right|=2,且A^{-1}=\begin{pmatrix}\frac13&-\frac23\\\frac16&\frac16\end{pmatrix},B^{-1}=\begin{pmatrix}\frac12&0\\\frac12&1\end{pmatrix},\\故X=A^{-1}CB^{-1}=\begin{pmatrix}1&1\\\frac14&0\end{pmatrix}\end{array}
$$



$$
设X\begin{pmatrix}1&0&0\\0&2&0\\0&0&3\end{pmatrix}=\begin{pmatrix}1&2&3\end{pmatrix},则X=\left(\;\;\;\;\right).
$$
$$
A.
\begin{pmatrix}-1&-1&1\end{pmatrix} \quad B.\begin{pmatrix}-1&-1&-1\end{pmatrix} \quad C.\begin{pmatrix}1&1&-1\end{pmatrix} \quad D.\begin{pmatrix}1&1&1\end{pmatrix} \quad E. \quad F. \quad G. \quad H.
$$
$$
\begin{array}{l}X\begin{pmatrix}1&0&0\\0&2&0\\0&0&3\end{pmatrix}=\begin{pmatrix}1&2&3\end{pmatrix}\\⇒ X=\begin{pmatrix}1&2&3\end{pmatrix}\begin{pmatrix}1&0&0\\0&2&0\\0&0&3\end{pmatrix}^{-1}=\begin{pmatrix}1&1&1\end{pmatrix}\end{array}
$$



$$
\mathrm{已知}A=\begin{pmatrix}1&1&-1\\0&1&1\\0&0&-1\end{pmatrix},且A^2-AB=E,则B=\left(\;\;\;\right)
$$
$$
A.
\begin{pmatrix}1&2&1\\0&0&0\\0&0&0\end{pmatrix} \quad B.\begin{pmatrix}0&2&1\\0&0&0\\0&0&0\end{pmatrix} \quad C.\begin{pmatrix}0&2&1\\0&1&0\\0&0&0\end{pmatrix} \quad D.\begin{pmatrix}0&2&1\\0&1&0\\0&0&-1\end{pmatrix} \quad E. \quad F. \quad G. \quad H.
$$
$$
\begin{array}{l}由A^2-AB=E得,AB=A^2-E,\;\;故B=A-A^{-1},\mathrm{由初等变换得}\;A^{-1}=\begin{pmatrix}1&-1&-2\\0&1&1\\0&0&-1\end{pmatrix},\mathrm{所以}\\B=A-A^{-1}=\begin{pmatrix}0&2&1\\0&0&0\\0&0&0\end{pmatrix}\end{array}
$$



$$
设2A+\begin{pmatrix}1&0&0\\0&1&0\\0&0&1\end{pmatrix}\begin{pmatrix}1&2&0\\2&0&0\\2&-2&1\end{pmatrix}=\begin{pmatrix}3&0&-2\\4&2&2\\-2&0&3\end{pmatrix},则A=(\:).
$$
$$
A.
\begin{pmatrix}1&-1&-1\\1&1&1\\-2&1&1\end{pmatrix} \quad B.\begin{pmatrix}-4&0&-5\\5&1&3\\-7&1&-4\end{pmatrix} \quad C.\begin{pmatrix}-4&0&-5\\5&-1&3\\7&-1&4\end{pmatrix} \quad D.\begin{pmatrix}-4&0&-5\\5&1&3\\7&-1&4\end{pmatrix} \quad E. \quad F. \quad G. \quad H.
$$
$$
A=\frac12\begin{bmatrix}\begin{pmatrix}3&0&-2\\4&2&2\\-2&0&3\end{pmatrix}-\begin{pmatrix}1&2&0\\2&0&0\\2&-2&1\end{pmatrix}\end{bmatrix}=\begin{pmatrix}1&-1&-1\\1&1&1\\-2&1&1\end{pmatrix}
$$



$$
\mathrm{已知矩阵方程}AX+E=A^2+X,\mathrm{其中}A=\begin{pmatrix}2&0&0\\0&2&0\\1&6&2\end{pmatrix},则X=\left(\;\;\;\right).
$$
$$
A.
\begin{pmatrix}2&0&0\\0&3&0\\1&6&3\end{pmatrix} \quad B.\begin{pmatrix}1&0&0\\0&1&0\\1&6&1\end{pmatrix} \quad C.\begin{pmatrix}3&0&0\\0&3&0\\1&6&3\end{pmatrix} \quad D.\begin{pmatrix}3&0&0\\0&3&0\\0&6&2\end{pmatrix} \quad E. \quad F. \quad G. \quad H.
$$
$$
\begin{array}{l}由AX+E=A^2+X,\left(A-E\right)X=A^2-E,而A-E=\begin{pmatrix}1&0&0\\0&1&0\\1&6&1\end{pmatrix}\mathrm{可逆},\\故X=A+E=\begin{pmatrix}3&0&0\\0&3&0\\1&6&3\end{pmatrix}\end{array}
$$



$$
\mathrm{矩阵}A=\begin{pmatrix}1&2&3\\2&2&1\\3&4&3\end{pmatrix},B=\begin{pmatrix}2&5\\3&1\\4&3\end{pmatrix},\mathrm{则矩阵方程}AX=B\mathrm{的解}X=\left(\;\;\;\right).
$$
$$
A.
\begin{pmatrix}3&2\\-2&-3\\1&3\end{pmatrix} \quad B.\begin{pmatrix}2&3\\-2&-3\\1&3\end{pmatrix} \quad C.\begin{pmatrix}1&3\\-2&-3\\2&1\end{pmatrix} \quad D.\begin{pmatrix}-3&-2\\2&3\\1&3\end{pmatrix} \quad E. \quad F. \quad G. \quad H.
$$
$$
\begin{array}{l}A\mathrm{可逆},则X=A^{-1}B,\\\;\;\;\;\;\;\;\left(A\;\;B\right)=\begin{pmatrix}1&2&3&2&5\\2&2&1&3&1\\3&4&3&4&3\end{pmatrix}\xrightarrow[{r_3-3r_1}]{r_2-2r_1}\begin{pmatrix}1&2&3&2&5\\0&-2&-5&-1&-9\\0&-2&-6&-2&-12\end{pmatrix}\\\;\;\;\;\;\;\xrightarrow[{r_3-r_2}]{r_1+r_2}\begin{pmatrix}1&0&-2&1&-4\\0&-2&-5&-1&-9\\0&0&-1&-1&-3\end{pmatrix}\xrightarrow[{r_2-5r_3}]{r_1-2r_3}\begin{pmatrix}1&0&0&3&2\\0&-2&0&4&6\\0&0&-1&-1&-3\end{pmatrix}\\\;\;\;\;\;\;\xrightarrow[{r_3\div\left(-1\right)}]{r_2\div\left(-2\right)}\begin{pmatrix}1&0&0&3&2\\0&1&0&-2&-3\\0&0&1&1&3\end{pmatrix},\\\;\;\;\;X=\begin{pmatrix}3&2\\-2&-3\\1&3\end{pmatrix}\\\end{array}
$$



$$
设\begin{pmatrix}5&2&0\\2&1&0\\0&0&1\end{pmatrix}X=\begin{pmatrix}5&2&0\\2&1&0\\0&-3&1\end{pmatrix},则X=\left(\;\;\;\;\right).
$$
$$
A.
\begin{pmatrix}1&0&0\\0&1&0\\0&-3&1\end{pmatrix} \quad B.\begin{pmatrix}1&0&-3\\0&1&0\\0&0&1\end{pmatrix} \quad C.\begin{pmatrix}1&0&0\\0&-3&0\\0&0&1\end{pmatrix} \quad D.\begin{pmatrix}-3&0&0\\0&1&0\\0&0&1\end{pmatrix} \quad E. \quad F. \quad G. \quad H.
$$
$$
\begin{array}{l}\mathrm{矩阵}\begin{pmatrix}5&2&0\\2&1&0\\0&-3&1\end{pmatrix}\mathrm{是由}\begin{pmatrix}5&2&0\\2&1&0\\0&0&1\end{pmatrix}\mathrm{的第三列乘以}-3\mathrm{第二列得到的},\mathrm{由初等变换和初等矩阵的知识},\mathrm{可知}\\X=\begin{pmatrix}1&0&0\\0&1&0\\0&-3&1\end{pmatrix}\end{array}
$$



$$
设AX+E=A^2+X,且A=\begin{pmatrix}1&0&1\\0&3&0\\1&0&1\end{pmatrix},则X=\left(\;\;\;\;\;\right).
$$
$$
A.
\begin{pmatrix}2&0&1\\0&1&0\\1&0&2\end{pmatrix} \quad B.\begin{pmatrix}2&0&0\\0&1&0\\1&0&2\end{pmatrix} \quad C.\begin{pmatrix}2&0&1\\0&3&1\\1&0&2\end{pmatrix} \quad D.\begin{pmatrix}2&0&1\\0&4&0\\1&0&2\end{pmatrix} \quad E. \quad F. \quad G. \quad H.
$$
$$
\begin{array}{l}\left(A-E\right)X=A^2-E,而A-E=\begin{pmatrix}0&0&1\\0&2&0\\1&0&0\end{pmatrix}\mathrm{可逆},\\故X=\left(A-E\right)^{-1}\left(A^2-E\right)=A+E=\begin{pmatrix}2&0&1\\0&4&0\\1&0&2\end{pmatrix}\end{array}
$$



$$
设A=\begin{pmatrix}-1&3&1\\1&1&0\\2&3&1\end{pmatrix},B\mathrm{是三阶矩阵},且AB+E=A^2-B,则B=\left(\;\;\;\right).
$$
$$
A.
\begin{pmatrix}2&3&1\\1&0&0\\2&3&0\end{pmatrix} \quad B.\begin{pmatrix}0&3&1\\1&2&0\\2&3&2\end{pmatrix} \quad C.\begin{pmatrix}-2&3&1\\1&0&0\\2&3&0\end{pmatrix} \quad D.\begin{pmatrix}0&3&1\\1&2&0\\2&3&-2\end{pmatrix} \quad E. \quad F. \quad G. \quad H.
$$
$$
\begin{array}{l}由\left(A+E\right)B=\left(A+E\right)\left(A-E\right),\\即\left|A+E\right|=\begin{vmatrix}0&3&1\\1&2&0\\2&3&2\end{vmatrix}=\begin{vmatrix}0&3&1\\1&2&0\\2&0&1\end{vmatrix}\neq0\\故B=A-E=\begin{pmatrix}-2&3&1\\1&0&0\\2&3&0\end{pmatrix}\end{array}
$$



$$
undefined
$$
$$
A.
\begin{pmatrix}1&0\\-3&0\\14&1\end{pmatrix} \quad B.\begin{pmatrix}1&0\\3&0\\14&1\end{pmatrix} \quad C.\begin{pmatrix}1&0\\-3&0\\14&-1\end{pmatrix} \quad D.\begin{pmatrix}-1&0\\3&0\\-14&1\end{pmatrix} \quad E. \quad F. \quad G. \quad H.
$$
$$
\begin{pmatrix}1&0&0&1&0\\6&2&0&0&0\\12&18&3&0&3\end{pmatrix}\rightarrow\begin{pmatrix}1&0&0&1&0\\0&1&0&-3&0\\0&0&1&14&1\end{pmatrix},\;\;\mathrm{所以}X=\begin{pmatrix}1&0\\-3&0\\14&1\end{pmatrix}
$$



$$
设AX=B+X,\mathrm{其中}A=\begin{pmatrix}1&1&1\\1&1&1\\-3&2&1\end{pmatrix},B=\begin{pmatrix}1&3&1\\2&-1&1\\-3&1&-1\end{pmatrix},求\left|X\right|.
$$
$$
A.
4 \quad B.-4 \quad C.-1 \quad D.1 \quad E. \quad F. \quad G. \quad H.
$$
$$
\left(A-E\right)X=B,\left|A-E\right|=\begin{vmatrix}0&1&1\\1&0&1\\-3&2&0\end{vmatrix}=-1\neq0,\left|X\right|=-\left|B\right|=-\begin{vmatrix}1&3&1\\2&-1&1\\-3&1&-1\end{vmatrix}=4.
$$



$$
设AX=B+X,\mathrm{其中}A=\begin{pmatrix}1&1&1\\2&2&-1\\-4&-3&1\end{pmatrix},B=\begin{pmatrix}1&2&1\\2&-1&1\\-1&3&2\end{pmatrix},则\left|X\right|=\left(\;\;\;\;\right).
$$
$$
A.
5 \quad B.-5 \quad C.-10 \quad D.10 \quad E. \quad F. \quad G. \quad H.
$$
$$
\left(A-E\right)X=B,\left|A-E\right|=2,\left|X\right|=\frac12\left|B\right|=\frac12\begin{vmatrix}1&2&1\\2&-1&1\\-1&3&2\end{vmatrix}=-5.
$$



$$
设A=\begin{pmatrix}1&0&3\\0&2&0\\1&0&1\end{pmatrix},AB+E=A^2+B,则B=\left(\;\;\;\;\right).
$$
$$
A.
\begin{pmatrix}2&0&3\\0&3&0\\1&0&2\end{pmatrix} \quad B.\begin{pmatrix}1&0&1\\0&3&0\\1&0&1\end{pmatrix} \quad C.\begin{pmatrix}2&0&0\\0&3&0\\0&0&2\end{pmatrix} \quad D.\begin{pmatrix}1&0&2\\0&3&0\\2&0&1\end{pmatrix} \quad E. \quad F. \quad G. \quad H.
$$
$$
\begin{array}{l}\mathrm{由方程}AB+E=A^2+B,\;\mathrm{合并含有未知矩阵}B\mathrm{的项得}\\\;\;\;\;\;\;\;\;\;\;\;\;\;\;\;\;\;\;\;\;\;\;\;\;\;\;\;\;\;\;\;\;\;\;\left(A-E\right)B=A^2-E=\left(A-E\right)\left(A+E\right),\\又\;\;\;\;\;\;\;\;\;\;\;\;\;\;\;\;\;\;\;\;\;\;\;\;\;\;\;\;\;\;\;A-E=\begin{pmatrix}0&0&3\\0&1&0\\1&0&0\end{pmatrix},\\\mathrm{其行列式}\left|A-E\right|=-1\neq0,故A-E\mathrm{可逆},用\left(A-E\right)^{-1}\mathrm{左乘上式两边},\mathrm{即得}\\\;\;\;\;\;\;\;\;\;\;\;\;\;\;\;\;\;\;\;\;\;\;\;\;\;\;\;\;\;\;B=A+E=\begin{pmatrix}2&0&3\\0&3&0\\1&0&2\end{pmatrix}\end{array}
$$



$$
\begin{array}{l}设f\left(x\right)=ax^2+bx+c,A为n\mathrm{阶矩阵},E为n\mathrm{阶单位矩阵}.\mathrm{定义}f\left(A\right)=aA^2+bA+cE,\mathrm{已知}\\f\left(x\right)=x^2-5x+3,A=\begin{pmatrix}2&-1\\-3&3\end{pmatrix},则f\left(A\right)=\left(\;\;\;\;\right).\end{array}
$$
$$
A.
\begin{pmatrix}0&0\\0&0\end{pmatrix} \quad B.\begin{pmatrix}5&-1\\-3&8\end{pmatrix} \quad C.\begin{pmatrix}1&-1\\-3&2\end{pmatrix} \quad D.\begin{pmatrix}3&0\\0&3\end{pmatrix} \quad E. \quad F. \quad G. \quad H.
$$
$$
\begin{array}{l}f\left(A\right)=\begin{pmatrix}2&-1\\-3&3\end{pmatrix}^2-5\begin{pmatrix}2&-1\\-3&3\end{pmatrix}+3\begin{pmatrix}1&0\\0&1\end{pmatrix}\\=\begin{pmatrix}7&-5\\-15&12\end{pmatrix}-5\begin{pmatrix}2&-1\\-3&3\end{pmatrix}+3\begin{pmatrix}1&0\\0&1\end{pmatrix}\\=\begin{pmatrix}0&0\\0&0\end{pmatrix}\end{array}
$$



$$
\mathrm{矩阵方程}\begin{pmatrix}1&2&3\\2&2&5\\3&5&1\end{pmatrix}\begin{pmatrix}x_1\\x_2\\x_3\end{pmatrix}=\begin{pmatrix}1\\2\\3\end{pmatrix},则x_2=\left(\;\;\;\;\right).
$$
$$
A.
0 \quad B.1 \quad C.2 \quad D.4 \quad E. \quad F. \quad G. \quad H.
$$
$$
\begin{pmatrix}1&2&3&1\\2&2&5&2\\3&5&1&3\end{pmatrix}\rightarrow\begin{pmatrix}1&2&3&1\\0&1&-7&0\\0&-2&1&0\end{pmatrix}\rightarrow\begin{pmatrix}1&0&0&1\\0&1&0&0\\0&0&1&0\end{pmatrix},\mathrm{从而}\left\{\begin{array}{l}\begin{array}{c}x_1=1\\x_2=0\end{array}\\x_3=0\end{array}\right.
$$



$$
\mathrm{若矩阵}A\mathrm{满足}2A+\begin{pmatrix}3&1&2\\1&-1&-2\\1&3&1\end{pmatrix}\begin{pmatrix}1&1&1\\3&1&3\\2&-2&1\end{pmatrix}=\begin{pmatrix}2&0&-2\\4&2&2\\-2&0&3\end{pmatrix},则A=(\;\;).
$$
$$
A.
\begin{pmatrix}-4&0&5\\5&-1&3\\7&-1&-4\end{pmatrix} \quad B.\begin{pmatrix}4&0&-5\\5&-1&3\\-7&1&-4\end{pmatrix} \quad C.\begin{pmatrix}-4&0&-5\\5&1&3\\-7&-1&4\end{pmatrix} \quad D.\begin{pmatrix}-4&0&-5\\5&-1&3\\-7&-1&-4\end{pmatrix} \quad E. \quad F. \quad G. \quad H.
$$
$$
A=\frac12\left(\begin{pmatrix}2&0&-2\\4&2&2\\-2&0&3\end{pmatrix}-\begin{pmatrix}10&0&8\\-6&4&-4\\12&2&11\end{pmatrix}\right)=\begin{pmatrix}-4&0&-5\\5&-1&3\\-7&-1&-4\end{pmatrix}.
$$



$$
设A=\begin{pmatrix}1&0&0\\0&\frac12&0\\0&0&1\end{pmatrix},B=\begin{pmatrix}2&1\\5&3\end{pmatrix},C=\begin{pmatrix}1&3\\2&0\\3&1\end{pmatrix},\mathrm{矩阵方程}AXB=C\mathrm{的解}X=\left(\;\;\;\right).
$$
$$
A.
\begin{pmatrix}-2&1\\10&-4\\-10&4\end{pmatrix} \quad B.\begin{pmatrix}2&1\\10&-4\\10&4\end{pmatrix} \quad C.\begin{pmatrix}-12&5\\12&-4\\4&-1\end{pmatrix} \quad D.\begin{pmatrix}2&1\\4&-10\\10&4\end{pmatrix} \quad E. \quad F. \quad G. \quad H.
$$
$$
A^{-1}=\begin{pmatrix}1&0&0\\0&2&0\\0&0&1\end{pmatrix},\;\;B^{-1}=\begin{pmatrix}3&-1\\-5&2\end{pmatrix},\mathrm{所以}X=\begin{pmatrix}1&0&0\\0&2&0\\0&0&1\end{pmatrix}\begin{pmatrix}1&3\\2&0\\3&1\end{pmatrix}\begin{pmatrix}3&-1\\-5&2\end{pmatrix}=\begin{pmatrix}-12&5\\12&-4\\4&-1\end{pmatrix}
$$



$$
设A=\begin{pmatrix}3&1&1\\0&3&1\\0&0&3\end{pmatrix},B=\begin{pmatrix}1&0\\0&2\\-1&1\end{pmatrix},\mathrm{若矩阵}X\mathrm{满足等式}AX=B+2X,则X=\left(\;\;\;\right).
$$
$$
A.
\begin{pmatrix}1&-2\\1&1\\-1&1\end{pmatrix} \quad B.\begin{pmatrix}0&-2\\1&1\\-1&1\end{pmatrix} \quad C.\begin{pmatrix}0&-2\\1&1\\1&1\end{pmatrix} \quad D.\begin{pmatrix}1&-2\\1&1\\1&1\end{pmatrix} \quad E. \quad F. \quad G. \quad H.
$$
$$
\begin{array}{l}\mathrm{因为}AX=B+2X,\mathrm{所以}\left(A-2E\right)X=B,且\left|A-2E\right|=\begin{vmatrix}1&1&1\\0&1&1\\0&0&1\end{vmatrix}=1\neq0,\mathrm{于是}\\(A-2E^{\;\;\;}B)=\begin{pmatrix}1&1&1&1&0\\0&1&1&0&2\\0&0&1&-1&1\end{pmatrix}\rightarrow\begin{pmatrix}1&0&0&1&-2\\0&1&0&1&1\\0&0&1&-1&1\end{pmatrix}.\mathrm{所以}X=\begin{pmatrix}1&-2\\1&1\\-1&1\end{pmatrix}\end{array}
$$



$$
设A=\begin{pmatrix}0&0&0\\0&-1&1\\0&1&-1\end{pmatrix},且AB=A-B,\mathrm{则矩阵}B=\left(\;\;\;\right).
$$
$$
A.
\begin{pmatrix}0&0&0\\0&-1&1\\0&1&-1\end{pmatrix} \quad B.\begin{pmatrix}0&0&0\\0&1&1\\0&-1&1\end{pmatrix} \quad C.\begin{pmatrix}0&0&0\\0&-1&1\\0&1&1\end{pmatrix} \quad D.\begin{pmatrix}0&0&0\\0&1&-1\\0&-1&1\end{pmatrix} \quad E. \quad F. \quad G. \quad H.
$$
$$
\begin{array}{l}\begin{array}{l}由AB=A-B得\left(A+E\right)B=A,因\left|A+E\right|=\begin{vmatrix}1&0&0\\0&0&1\\0&1&0\end{vmatrix}=-1\neq0,\mathrm{所以}A+E\mathrm{可逆},\;\mathrm{求得}\\\;\;\;\;\;\;\;\;\;\;\;\;\;\;\;\;\;\;\;\;\;\;\left(A+E\right)^{-1}=\begin{pmatrix}1&0&0\\0&0&1\\0&1&0\end{pmatrix},且B=\left(A+E\right)^{-1}\;A,\end{array}\\\;\;\;\;\;\;\;\;\;\;\;\;\;\;\;\;\;\;\;\;\;\;\;\;⇒ B=\begin{pmatrix}1&0&0\\0&0&1\\0&1&0\end{pmatrix}\begin{pmatrix}0&0&0\\0&-1&1\\0&1&-1\end{pmatrix}=\begin{pmatrix}0&0&0\\0&1&-1\\0&-1&1\end{pmatrix}.\end{array}
$$



$$
\mathrm{矩阵方程}AX-E=A+X,\mathrm{其中}A=\begin{pmatrix}1&0&1\\0&2&0\\1&0&1\end{pmatrix},E=\begin{pmatrix}1&0&0\\0&1&0\\0&0&1\end{pmatrix},则X=\left(\;\;\;\;\right).
$$
$$
A.
\begin{pmatrix}1&0&2\\0&3&0\\2&0&1\end{pmatrix} \quad B.\begin{pmatrix}1&0&1\\0&2&0\\2&0&1\end{pmatrix} \quad C.\begin{pmatrix}1&0&0\\0&1&0\\2&0&1\end{pmatrix} \quad D.\begin{pmatrix}0&0&1\\0&1&0\\1&0&0\end{pmatrix} \quad E. \quad F. \quad G. \quad H.
$$
$$
\begin{array}{l}A-E=\begin{pmatrix}0&0&1\\0&1&0\\1&0&0\end{pmatrix},A+E=\begin{pmatrix}2&0&1\\0&3&0\\1&0&2\end{pmatrix},\left(A-E\right)^{-1}=\begin{pmatrix}0&0&1\\0&1&0\\1&0&0\end{pmatrix}.\\\;\;\;X=\left(A-E\right)^{-1}\left(A+E\right)=\begin{pmatrix}0&0&1\\0&1&0\\1&0&0\end{pmatrix}\begin{pmatrix}2&0&1\\0&3&0\\1&0&2\end{pmatrix}=\begin{pmatrix}1&0&2\\0&3&0\\2&0&1\end{pmatrix}\end{array}
$$



$$
\mathrm{若三阶方阵}A,B\mathrm{满足}A^{-1}BA=6A+BA,且A=\begin{pmatrix}\frac13&0&0\\0&\frac14&0\\0&0&\frac17\end{pmatrix},则B=\left(\;\;\;\;\right)
$$
$$
A.
\begin{pmatrix}3&0&0\\0&1&0\\0&0&2\end{pmatrix} \quad B.\begin{pmatrix}1&0&0\\0&2&0\\0&0&3\end{pmatrix} \quad C.\begin{pmatrix}2&0&0\\0&1&0\\0&0&3\end{pmatrix} \quad D.\begin{pmatrix}3&0&0\\0&2&0\\0&0&1\end{pmatrix} \quad E. \quad F. \quad G. \quad H.
$$
$$
\begin{array}{l}因A^{-1}BA-BA=6A,故\left(A^{-1}-E\right)BA=6A,因A\mathrm{可逆},则\\\left(A^{-1}-E\right)B=6E,B=6\left(A^{-1}-E\right)^{-1}=\begin{pmatrix}3&0&0\\0&2&0\\0&0&1\end{pmatrix}\end{array}
$$



$$
\mathrm{若三阶方阵}A,B\mathrm{满足}A^{-1}BA=6A+BA,且A=\begin{pmatrix}\frac12&0&0\\0&\frac14&0\\0&0&\frac17\end{pmatrix},则B=\left(\;\;\;\;\right)
$$
$$
A.
\begin{pmatrix}6&0&0\\0&2&0\\0&0&1\end{pmatrix} \quad B.\begin{pmatrix}6&0&0\\0&2&0\\0&0&-1\end{pmatrix} \quad C.\begin{pmatrix}6&0&0\\0&-2&0\\0&0&1\end{pmatrix} \quad D.\begin{pmatrix}-6&0&0\\0&2&0\\0&0&1\end{pmatrix} \quad E. \quad F. \quad G. \quad H.
$$
$$
\begin{array}{l}因A^{-1}BA-BA=6A,故\left(A^{-1}-E\right)BA=6A,因A\mathrm{可逆},则\\\left(A^{-1}-E\right)B=6E,B=6\left(A^{-1}-E\right)^{-1}=\begin{pmatrix}6&0&0\\0&2&0\\0&0&1\end{pmatrix}\end{array}
$$



$$
设A=\begin{pmatrix}0&3&3\\1&1&0\\-1&2&3\end{pmatrix},AB=A+2B,则B=\left(\;\;\;\;\right).
$$
$$
A.
\begin{pmatrix}0&3&3\\1&-2&-3\\-1&-1&0\end{pmatrix} \quad B.\begin{pmatrix}0&3&3\\-1&2&3\\-1&-1&0\end{pmatrix} \quad C.\begin{pmatrix}0&3&3\\1&-2&-3\\1&1&0\end{pmatrix} \quad D.\begin{pmatrix}0&3&3\\-1&2&3\\1&1&0\end{pmatrix} \quad E. \quad F. \quad G. \quad H.
$$
$$
\begin{array}{l}AB=A+2B⇒\left(A-2E\right)B=A,故\\\;(A-2E\;^{}A)=\begin{pmatrix}-2&3&3&0&3&3\\1&-1&0&1&1&0\\-1&2&1&-&2&3\end{pmatrix}\rightarrow\begin{pmatrix}1&0&0&0&3&3\\0&1&0&-1&2&3\\0&0&1&1&1&0\end{pmatrix},\mathrm{所以}B=\begin{pmatrix}0&3&3\\-1&2&3\\1&1&0\end{pmatrix}\end{array}
$$



$$
设n\mathrm{阶矩阵}A,B\mathrm{满足}A+B=AB,\mathrm{已知}B=\begin{pmatrix}1&3&0\\2&1&0\\0&0&2\end{pmatrix},\mathrm{则矩阵}A=\left(\;\;\;\right).
$$
$$
A.
\begin{pmatrix}1&\frac12&0\\\frac13&1&0\\0&0&2\end{pmatrix} \quad B.\begin{pmatrix}1&\frac12&0\\\frac13&-1&0\\0&0&2\end{pmatrix} \quad C.\begin{pmatrix}1&\frac12&0\\-\frac13&1&0\\0&0&\frac12\end{pmatrix} \quad D.\begin{pmatrix}1&\frac12&0\\\frac13&-1&0\\0&0&\frac12\end{pmatrix} \quad E. \quad F. \quad G. \quad H.
$$
$$
\begin{array}{l}\mathrm{由条件可知}AB-A-B+E=E,则A\left(B-E\right)-\left(B-E\right),即\left(B-E\right)\left(A-E\right)=E,故A-E=\left(B-E\right)^{-1},则\\A=\left(B-E\right)^{-1}+E=\begin{pmatrix}0&\frac12&0\\\frac13&0&0\\0&0&1\end{pmatrix}+E=\begin{pmatrix}1&\frac12&0\\\frac13&1&0\\0&0&2\end{pmatrix}\end{array}
$$



$$
设A=\begin{pmatrix}10&-2\\0&4\\1&3\end{pmatrix},B=\begin{pmatrix}2&1\\5&3\end{pmatrix},C=\begin{pmatrix}1&3\\2&0\\3&1\end{pmatrix},若\left(X-A\right)B=C,\mathrm{则矩阵}X=\left(\;\;\;\;\right).
$$
$$
A.
\begin{pmatrix}-2&3\\6&2\\5&1\end{pmatrix} \quad B.\begin{pmatrix}-2&3\\6&1\\5&2\end{pmatrix} \quad C.\begin{pmatrix}-2&3\\6&2\\5&2\end{pmatrix} \quad D.\begin{pmatrix}-2&3\\6&2\\5&-2\end{pmatrix} \quad E. \quad F. \quad G. \quad H.
$$
$$
\begin{array}{l}X-A=CB^{-1}=\begin{pmatrix}1&3\\2&0\\3&1\end{pmatrix}\begin{pmatrix}3&-1\\-5&2\end{pmatrix}=\begin{pmatrix}-12&5\\6&-2\\4&-1\end{pmatrix},\\\;\;\;\;\;X=CB^{-1}+A=\begin{pmatrix}-12&5\\6&-2\\4&-1\end{pmatrix}+\begin{pmatrix}10&-2\\0&4\\1&3\end{pmatrix}=\begin{pmatrix}-2&3\\6&2\\5&2\end{pmatrix}\end{array}
$$



$$
设AX=B+X,\mathrm{其中}A=\begin{pmatrix}1&1&1\\1&1&1\\-3&2&1\end{pmatrix},B=\begin{pmatrix}1&3&1\\0&-1&0\\-3&1&-1\end{pmatrix},则\left|X\right|=\left(\;\;\;\right).
$$
$$
A.
3 \quad B.-3 \quad C.2 \quad D.-2 \quad E. \quad F. \quad G. \quad H.
$$
$$
设AX=B+X,\left|A-E\right|=\begin{vmatrix}0&1&1\\1&0&1\\-3&2&0\end{vmatrix}=-1\neq0,\mathrm{所以}\left|X\right|=-\left|B\right|=2
$$



$$
设X\begin{pmatrix}\frac12&0&0\\0&1&5\\0&1&6\end{pmatrix}=\begin{pmatrix}1&1&2\\0&0&-6\end{pmatrix},则X=\left(\;\;\;\;\right)
$$
$$
A.
\begin{pmatrix}0&1&0\\1&-2&-1\end{pmatrix} \quad B.\begin{pmatrix}1&0&0\\-1&2&-1\end{pmatrix} \quad C.\begin{pmatrix}2&4&-3\\0&6&-6\end{pmatrix} \quad D.\begin{pmatrix}0&1&0\\1&-2&1\end{pmatrix} \quad E. \quad F. \quad G. \quad H.
$$
$$
\begin{array}{l}令\;A=\begin{pmatrix}\frac12&0&0\\0&1&5\\0&1&6\end{pmatrix},B=\begin{pmatrix}1&1&2\\0&0&-6\end{pmatrix},\mathrm{因为}\left|A\right|=\frac12\neq0,故X=BA^{-1},\mathrm{由初等变换得}\\A^{-1}=\begin{pmatrix}2&0&0\\0&6&-5\\0&-1&1\end{pmatrix},X=BA^{-1}=\begin{pmatrix}2&4&-3\\0&6&-6\end{pmatrix}\end{array}
$$



$$
设A=\begin{pmatrix}0&0&1\\0&1&0\\1&2&1\end{pmatrix},B\mathrm{是三阶矩阵},且A^2+E=AB-2A+B,则B=\left(\;\;\;\;\right).
$$
$$
A.
\begin{pmatrix}1&0&1\\0&2&0\\1&2&2\end{pmatrix} \quad B.\begin{pmatrix}1&0&1\\0&2&0\\1&1&2\end{pmatrix} \quad C.\begin{pmatrix}-1&0&1\\0&2&0\\1&1&2\end{pmatrix} \quad D.\begin{pmatrix}-1&0&1\\0&2&0\\1&2&2\end{pmatrix} \quad E. \quad F. \quad G. \quad H.
$$
$$
\begin{array}{l}\mathrm{由原式得}\\\;\;\;\;\;\;\;\;\left(A+E\right)^2=\left(A+E\right)B,\left|A+E\right|=\begin{vmatrix}1&0&1\\0&2&0\\1&2&2\end{vmatrix}\neq0,\\\;故\;B=A+E=\begin{pmatrix}1&0&1\\0&2&0\\1&2&2\end{pmatrix}\end{array}
$$



$$
\mathrm{已知}A+B=AB,B=\begin{pmatrix}1&-3&0\\2&1&0\\0&0&2\end{pmatrix},则A=\left(\;\;\;\;\right).
$$
$$
A.
\begin{pmatrix}1&\frac12&0\\-\frac13&1&0\\0&0&2\end{pmatrix} \quad B.\begin{pmatrix}1&-\frac12&0\\-\frac13&1&0\\0&0&2\end{pmatrix} \quad C.\begin{pmatrix}1&-\frac12&0\\\frac13&-1&0\\0&0&2\end{pmatrix} \quad D.\begin{pmatrix}1&\frac12&0\\-\frac13&1&0\\0&0&-2\end{pmatrix} \quad E. \quad F. \quad G. \quad H.
$$
$$
\begin{array}{l}由A+B=AB⇒ A\left(B-E\right)=B,B-E=\begin{pmatrix}0&-3&0\\2&0&0\\0&0&1\end{pmatrix},\left|B-E\right|=6\neq0\\\mathrm{所以}A=B\left(B-E\right)^{-1},\mathrm{利用分块矩阵的知识},\left(B-E\right)^{-1}=\begin{pmatrix}0&\frac12&0\\-\frac13&0&0\\0&0&1\end{pmatrix},\\\mathrm{所以}A=B\left(B-E\right)^{-1}=\begin{pmatrix}1&\frac12&0\\-\frac13&1&0\\0&0&2\end{pmatrix}\end{array}
$$



$$
\mathrm{设矩阵}A,B\mathrm{满足}A^* BA=2BA-8E,\mathrm{其中}A=\begin{pmatrix}1&0&0\\0&-2&0\\0&0&1\end{pmatrix}则\;\;B=\left(\;\;\;\;\right).
$$
$$
A.
\begin{pmatrix}-2&0&0\\0&4&0\\0&0&-2\end{pmatrix} \quad B.\begin{pmatrix}2&0&0\\0&4&0\\0&0&-2\end{pmatrix} \quad C.\begin{pmatrix}2&0&0\\0&-2&0\\0&0&2\end{pmatrix} \quad D.\begin{pmatrix}2&0&0\\0&-4&0\\0&0&2\end{pmatrix} \quad E. \quad F. \quad G. \quad H.
$$
$$
\begin{array}{l}\mathrm{由原方程得},\left(A^*-2E\right)BA=-8E,\\即A^*-2E=\left|A\right|A^{-1}-2E=-2\begin{pmatrix}1&0&0\\0&-\frac12&0\\0&0&1\end{pmatrix}-2\begin{pmatrix}1&0&0\\0&1&0\\0&0&1\end{pmatrix}=\begin{pmatrix}-4&0&0\\0&-1&0\\0&0&-4\end{pmatrix},\\B=-8\left(A^*-2E\right)^{-1}A^{-1}=\begin{pmatrix}2&0&0\\0&-4&0\\0&0&2\end{pmatrix}\\\end{array}
$$



$$
设\;X\begin{pmatrix}1&1&-1\\0&1&1\\1&1&0\end{pmatrix}=\begin{pmatrix}1&-1&1\\1&1&0\\2&1&1\end{pmatrix},则X=\left(\;\;\;\right).
$$
$$
A.
\begin{pmatrix}-3&-2&4\\0&0&1\\-2&-1&4\end{pmatrix} \quad B.\begin{pmatrix}-3&-2&4\\0&0&1\\-4&-1&4\end{pmatrix} \quad C.\begin{pmatrix}-1&-2&4\\0&0&1\\-2&-1&4\end{pmatrix} \quad D.\begin{pmatrix}-3&-2&-4\\0&0&1\\-2&-1&4\end{pmatrix} \quad E. \quad F. \quad G. \quad H.
$$
$$
令B=\begin{pmatrix}1&1&-1\\0&1&1\\1&1&0\end{pmatrix},C=\begin{pmatrix}1&-1&1\\1&1&0\\2&1&1\end{pmatrix}则B^{-1}=\begin{pmatrix}-1&-1&2\\1&1&-1\\-1&0&1\end{pmatrix},\;\;\mathrm{所以}X=CB^{-1}=\begin{pmatrix}-3&-2&4\\0&0&1\\-2&-1&4\end{pmatrix}
$$



$$
设A=\begin{pmatrix}0&0&1\\1&1&0\\1&2&2\end{pmatrix},B\mathrm{是三阶矩阵},且A+E=AB-2A-B,则B=\left(\;\;\;\right)
$$
$$
A.
\begin{pmatrix}3&4&0\\2&1&2\\-4&-4&3\end{pmatrix} \quad B.\begin{pmatrix}3&4&0\\-2&-1&-2\\4&4&3\end{pmatrix} \quad C.\begin{pmatrix}3&4&0\\-2&-1&2\\4&4&3\end{pmatrix} \quad D.\begin{pmatrix}3&4&0\\2&-1&2\\4&4&3\end{pmatrix} \quad E. \quad F. \quad G. \quad H.
$$
$$
\begin{array}{l}由A+E=AB-2A-B⇒\left(A-E\right)B=3A+E,\\\;\;\;\;\;\;A-E=\begin{pmatrix}-1&0&1\\1&0&0\\1&2&1\end{pmatrix},\\且\left|A-E\right|=2\neq0,故B=\left(A-E\right)^{-1}\left(3A+E\right).\\\begin{pmatrix}-1&0&1&\vdots&1&0&3\\1&0&0&\vdots&3&4&0\\1&2&1&\vdots&3&6&7\end{pmatrix}\rightarrow\begin{pmatrix}1&0&-1&\vdots&-1&0&-3\\0&0&1&\vdots&4&4&3\\0&1&1&\vdots&2&3&5\end{pmatrix}\;\;\;\;\;\;\;\;\;\;\;\\\rightarrow\begin{pmatrix}1&0&0&\vdots&3&4&0\\0&0&1&\vdots&4&4&3\\0&1&0&\vdots&-2&-1&2\end{pmatrix}\;\;\;\;\;\;\;\;\;\;\;\;\;\;\;\;\;\;\;\;\;\;\;\;\;\;\;\;\;\;\;\;\;\;\;∴ B=\begin{pmatrix}3&4&0\\-2&-1&2\\4&4&3\end{pmatrix}\;\;\;\;\\\end{array}
$$



$$
\mathrm{设矩阵}A,B\mathrm{满足关系式}AB=2B+A,且A=\begin{pmatrix}3&0&1\\1&1&0\\0&1&4\end{pmatrix},则B=\left(\;\;\;\;\right).
$$
$$
A.
\begin{pmatrix}5&2&-2\\4&-3&-2\\-2&2&3\end{pmatrix} \quad B.\begin{pmatrix}5&-2&-2\\4&-3&-2\\-2&2&3\end{pmatrix} \quad C.\begin{pmatrix}5&-2&-2\\4&-3&-2\\2&2&-3\end{pmatrix} \quad D.\begin{pmatrix}5&-2&-2\\4&3&-2\\-2&2&-3\end{pmatrix} \quad E. \quad F. \quad G. \quad H.
$$
$$
\begin{array}{l}AB=2B+A⇒\left(A-2E\right)B=A,\\A-2E=\begin{pmatrix}1&0&1\\1&-1&0\\0&1&2\end{pmatrix},且\left|A-2E\right|=-1\neq0,故B=\left(A-2E\right)^{-1}A=\begin{pmatrix}5&-2&-2\\4&-3&-2\\-2&2&3\end{pmatrix}\end{array}
$$



$$
\mathrm{设矩阵}A=\begin{pmatrix}1&1&0\\0&2&0\\0&0&-1\end{pmatrix},且A^* BA=BA-2E,\mathrm{其中}E\mathrm{为单位矩阵},A^* 为A\mathrm{的伴随矩阵},\mathrm{则矩阵}B=\left(\;\;\;\right).
$$
$$
A.
\begin{pmatrix}\frac23&\frac16&0\\0&1&0\\0&0&1\end{pmatrix} \quad B.\begin{pmatrix}\frac23&\frac16&0\\0&1&0\\0&0&2\end{pmatrix} \quad C.\begin{pmatrix}\frac23&\frac16&0\\0&\frac12&0\\0&0&1\end{pmatrix} \quad D.\begin{pmatrix}\frac23&-\frac16&0\\0&\frac12&0\\0&0&2\end{pmatrix} \quad E. \quad F. \quad G. \quad H.
$$
$$
\begin{array}{l}\mathrm{因为}\left|A\right|=-2,A^{-1}=-\frac12A^*,\mathrm{在等式两边乘以}-\frac12,\\得A^{-1}BA=-\frac12BA+E,\mathrm{上式两端左乘}A,\mathrm{右乘}A^{-1},得\\\;\;\;\;\;\;\;\;B=-\frac12AB+E,\left(E+\frac12A\right)B=E,\\\mathrm{因此}B=\left(E+\frac12A\right)^{-1}=\begin{pmatrix}\frac32&\frac12&0\\0&2&0\\0&0&\frac12\end{pmatrix}^{-1}=\begin{pmatrix}\frac23&-\frac16&0\\0&\frac12&0\\0&0&2\end{pmatrix}\end{array}
$$



$$
\mathrm{设矩阵}A=\begin{pmatrix}1&0&0&0\\-2&3&0&0\\0&-4&5&0\\0&0&-6&7\end{pmatrix},E\mathrm{为四阶单位矩阵},且B=\left(E+A\right)^{-1}\left(E-A\right),则\left(E+B\right)^{-1}=\left(\;\;\;\right)
$$
$$
A.
\begin{pmatrix}1&0&0&0\\-1&2&0&0\\0&2&3&0\\0&0&3&-4\end{pmatrix} \quad B.\begin{pmatrix}-1&0&0&0\\1&-2&0&0\\0&2&-3&0\\0&0&3&-4\end{pmatrix} \quad C.\begin{pmatrix}1&0&0&0\\1&2&0&0\\0&2&3&0\\0&0&3&4\end{pmatrix} \quad D.\begin{pmatrix}1&0&0&0\\-1&2&0&0\\0&-2&3&0\\0&0&-3&4\end{pmatrix} \quad E. \quad F. \quad G. \quad H.
$$
$$
\begin{array}{l}\;\;\;\;\;\;\;E+B=E+\left(E+A\right)^{-1}\left(E-A\right)\\=\left(E+A\right)^{-1}\left(E+A\right)+\left(E+A\right)^{-1}\left(E-A\right)\\=\left(E+A\right)^{-1}\left(E+A+E-A\right)=2\left(E+A\right)^{-1},\\\left(E+B\right)^{-1}=\left[2\left(E+A\right)^{-1}\right]^{-1}=\frac12\left(E+A\right)\\=\frac12\begin{pmatrix}2&0&0&0\\-2&4&0&0\\0&-4&6&0\\0&0&-6&8\end{pmatrix}=\begin{pmatrix}1&0&0&0\\-1&2&0&0\\0&-2&3&0\\0&0&-3&4\end{pmatrix}\end{array}
$$



$$
\begin{array}{l}\mathrm{设四阶矩阵}B\mathrm{满足}\left[\left(\frac12A\right)^*\right]^{-1}BA^{-1}=2AB+12E,\mathrm{其中}E\mathrm{是四阶单位矩阵},而\;A=\begin{pmatrix}1&2&0&0\\1&3&0&0\\0&0&0&2\\0&0&-1&0\end{pmatrix},\mathrm{则矩阵}\\B=\left(\;\;\;\;\;\right).\end{array}
$$
$$
A.
\begin{pmatrix}2&-4&0&0\\-2&2&0&0\\0&0&2&2\\0&0&1&2\end{pmatrix} \quad B.\begin{pmatrix}2&-4&0&0\\-2&-2&0&0\\0&0&2&2\\0&0&-1&2\end{pmatrix} \quad C.\begin{pmatrix}2&4&0&0\\2&-2&0&0\\0&0&2&2\\0&0&-1&2\end{pmatrix} \quad D.\begin{pmatrix}2&-4&0&0\\-2&2&0&0\\0&0&2&2\\0&0&1&2\end{pmatrix} \quad E. \quad F. \quad G. \quad H.
$$
$$
\begin{array}{l}\mathrm{由拉普拉斯展开定理},有\left|A\right|=\begin{vmatrix}1&2\\1&3\end{vmatrix}·\begin{vmatrix}0&2\\-1&0\end{vmatrix}=2\\A\mathrm{是四阶可逆矩阵},故\left(\frac12A\right)^*=\left(\frac12\right)^3· A^*\\\mathrm{于是}\left[\left(\frac12A\right)^*\right]^{-1}=\left(\frac1{2^3}A^*\right)^{-1}=8\left(A^*\right)^{-1}=8·\frac A{\left|A\right|}=4A,\\\mathrm{原方程化简为}2ABA^{-1}=AB+6E,\mathrm{左乘}A^{-1},得B\left(2A^{-1}-E\right)=6A^{-1}\\故B=6A^{-1}\left(2A^{-1}-E\right)^{-1}=6\left[\left(2A^{-1}-E\right)A\right]^{-1}=6\left(2E-A\right)^{-1}.\\即B=6\left(2E-A\right)^{-1}=6\begin{pmatrix}1&-2&0&0\\-1&-1&0&0\\0&0&2&-2\\0&0&1&2\end{pmatrix}^{-1}=\begin{pmatrix}2&-4&0&0\\-2&-2&0&0\\0&0&2&2\\0&0&-1&2\end{pmatrix}\end{array}
$$



$$
\mathrm{设矩阵}A\mathrm{的伴随矩阵}A^*=\begin{pmatrix}1&0&0&0\\0&1&0&0\\0&0&1&0\\0&0&0&8\end{pmatrix},且ABA^{-1}=BA^{-1}+3E,\mathrm{其中}E\mathrm{为四阶单位矩阵},\mathrm{则矩阵}B=\left(\;\;\;\right).
$$
$$
A.
\begin{pmatrix}6&0&0&0\\0&6&0&0\\6&0&6&0\\0&0&0&-1\end{pmatrix} \quad B.\begin{pmatrix}-6&0&0&0\\0&-6&0&0\\0&0&-6&0\\0&0&0&1\end{pmatrix} \quad C.\begin{pmatrix}6&0&0&0\\0&6&0&0\\6&0&6&0\\1&3&0&-1\end{pmatrix} \quad D.\begin{pmatrix}6&0&0&0\\0&6&0&0\\0&0&6&0\\0&0&0&-1\end{pmatrix} \quad E. \quad F. \quad G. \quad H.
$$
$$
\begin{array}{l}\begin{array}{l}\begin{array}{l}由ABA^{-1}=BA^{-1}+3E\;\;\;\;\;\;\mathrm{可知}\left(A-E\right)B=3A,\;\;A^{-1}\left(A-E\right)B=3A^{-1}A,\;\;则(E-A^{-1})B=3E,\;\;\mathrm{因为}\left|A^*\right|=\left|A\right|^3=8\\\;\;⇒\left(E-\frac12A^*\right)B=3E\left(因A^{-1}=\frac12A^*\right)\\\;\;\;\;\;\;\;\;⇒\left(2E-A^*\right)B=6E\\⇒ B=6\left(2E-A^*\right)^{-1}\left(因\left(2E-A^*\right)\mathrm{可逆}\right)\\=\begin{pmatrix}6&0&0&0\\0&6&0&0\\0&0&6&0\\0&0&0&-1\end{pmatrix}\end{array}\\\end{array}\\\\\end{array}
$$



$$
\mathrm{已知}AB-B=A,\mathrm{其中}B=\begin{pmatrix}1&-2&0\\2&1&0\\0&0&2\end{pmatrix},则A=\left(\;\;\;\;\right)
$$
$$
A.
\begin{pmatrix}1&2&0\\2&1&0\\0&0&2\end{pmatrix} \quad B.\begin{pmatrix}1&2&0\\-2&1&0\\0&0&2\end{pmatrix} \quad C.\begin{pmatrix}1&\frac12&0\\\frac12&1&0\\0&0&2\end{pmatrix} \quad D.\begin{pmatrix}1&\frac12&0\\-\frac12&1&0\\0&0&2\end{pmatrix} \quad E. \quad F. \quad G. \quad H.
$$
$$
\begin{array}{l}由AB-B=A及A\left(B-E\right)=B,\mathrm{从而}A=B\left(B-E\right)^{-1},\mathrm{因为}\\\;\;\;\;\;\;\;\;\;\;\;\;\;\;\;\;\;\;\;\;\;\;\;\;\;\;\;B-E=\begin{pmatrix}1&-2&0\\2&1&0\\0&0&2\end{pmatrix}-\begin{pmatrix}1&0&0\\0&1&0\\0&0&1\end{pmatrix}=\begin{pmatrix}0&-2&0\\2&0&0\\0&0&1\end{pmatrix},\\故\left(B-E\right)^{-1}=\begin{pmatrix}C^{-1}&0\\0&D^{-1}\end{pmatrix},\mathrm{其中}C^{-1}=\begin{pmatrix}0&-2\\2&0\end{pmatrix}^{-1}=\begin{pmatrix}0&\frac12\\-\frac12&\end{pmatrix},D^{-1}=\left(1\right)^{-1}=\left(1\right),则\\\;\;\;A=B\left(B-E\right)^{-1}=\begin{pmatrix}1&-2&0\\2&1&0\\0&0&2\end{pmatrix}\begin{pmatrix}0&\frac12&0\\-\frac12&0&0\\0&0&1\end{pmatrix}=\begin{pmatrix}1&\frac12&0\\-\frac12&1&0\\0&0&2\end{pmatrix}\end{array}
$$



$$
设B=\frac14(A^2-A-2E),\mathrm{其中}A=\begin{pmatrix}2&-1&0\\1&0&0\\0&0&3\end{pmatrix},则B^{-1}=\left(\;\;\;\;\;\right).
$$
$$
A.
\begin{pmatrix}-3&1&0\\-1&-1&0\\0&0&1\end{pmatrix} \quad B.\begin{pmatrix}3&1&0\\-1&-1&0\\0&0&1\end{pmatrix} \quad C.\;\begin{pmatrix}-3&1&0\\-1&-1&0\\0&0&-1\end{pmatrix} \quad D.\begin{pmatrix}-3&1&0\\1&1&0\\0&0&1\end{pmatrix} \quad E. \quad F. \quad G. \quad H.
$$
$$
\begin{array}{l}B=\frac14\left(A-2E\right)\left(A+E\right),\mathrm{所以}B^{-1}=4(A+E)^{-1}(A-2E)^{-1}\\\;\;\;\;4\;(A+E)^{-1}=\begin{pmatrix}1&1&0\\-1&3&0\\0&0&1\end{pmatrix},\;\;(A-2E)^{-1}=\begin{pmatrix}-2&1&0\\-1&0&0\\0&0&1\end{pmatrix},\mathrm{所以}B^{-1}=\begin{pmatrix}-3&1&0\\-1&-1&0\\0&0&1\end{pmatrix}\end{array}
$$



$$
\mathrm{已知}A=\begin{pmatrix}-4&-3&1\\-5&-3&1\\6&4&-1\end{pmatrix},且A^2-AB=E,则B=\left(\;\;\;\;\right).
$$
$$
A.
\begin{pmatrix}-5&-2&1\\4&5&0\\4&2&-4\end{pmatrix} \quad B.\begin{pmatrix}-5&-2&1\\-4&-5&0\\4&2&-4\end{pmatrix} \quad C.\begin{pmatrix}5&2&1\\-4&-5&0\\4&2&4\end{pmatrix} \quad D.\begin{pmatrix}5&-2&1\\-4&5&0\\4&2&4\end{pmatrix} \quad E. \quad F. \quad G. \quad H.
$$
$$
\begin{array}{l}由A^2-AB=E⇒ A\left(A-B\right)=E⇒ A-B=A^{-1}\\⇒ B=A-A^{-1}=\begin{pmatrix}-4&-3&1\\-5&-3&1\\6&4&-1\end{pmatrix}-\begin{pmatrix}1&-1&0\\-1&2&1\\2&2&3\end{pmatrix}=\begin{pmatrix}-5&-2&1\\-4&-5&0\\4&2&-4\end{pmatrix}\end{array}
$$



$$
\mathrm{已知}A=\begin{pmatrix}1&1&-1\\-1&1&1\\1&-1&1\end{pmatrix},\mathrm{矩阵}X\;\mathrm{满足}A^* X=A^{-1}+2X,\mathrm{其中}A^* 是A\mathrm{的伴随矩阵},\mathrm{则矩阵}X=\left(\;\;\;\right).
$$
$$
A.
\frac14\begin{pmatrix}1&1&0\\0&1&1\\1&0&1\end{pmatrix} \quad B.\frac12\begin{pmatrix}1&1&0\\0&1&1\\1&0&1\end{pmatrix} \quad C.\begin{pmatrix}1&1&0\\0&1&1\\1&0&1\end{pmatrix} \quad D.\begin{pmatrix}1&0&1\\0&1&0\\1&0&1\end{pmatrix} \quad E. \quad F. \quad G. \quad H.
$$
$$
\begin{array}{l}由AA^*=\left|A\right|E,\mathrm{用矩阵}A\mathrm{左乘方程的两端},得\\\;\;\;\;\;\;\;\;\;\;\;\;\;\;\;\;\;\;\;\;\;\;\;\;\left|A\right|X=E+2AX⇒\left(\left|A\right|E-2A\right)X=E\\\;\;\;\;\;\;\;\;\;\;\;\;\;\;\;\;\;\;\;\;\;\;\;\;\;\;\;\;\;\;\;\;\;\;\;\;\;\;\;\;\;\;\;\;\;\;\;\;\;\;\;\;\;\;\;\;⇒ X=\left(\left|A\right|E-2A\right)\;\;^{-1}\;\;\;\;\\\mathrm{由于}\left|\;A\right|\;=\begin{vmatrix}1&1&-1\\-1&1&1\\1&-1&1\end{vmatrix}\;\;=4\\\;\;\;\;\;\;\;\;\;\;\;\;\;\;\;\;\;\;\;\;\;\;\left|A\right|\;E-2A=2\begin{pmatrix}1&-1&1\\1&1&-1\\-1&1&1\end{pmatrix},\\故\;\;\;\;\;\;\;\;\;\;\;X=\frac12\;\begin{pmatrix}1&-1&1\\1&1&-1\\-1&1&1\end{pmatrix}^{-1}=\frac14\begin{pmatrix}1&1&0\\0&1&1\\1&0&1\end{pmatrix}\end{array}
$$



$$
\mathrm{已知}A=diag\left(1,-2,1\right),A^* BA=2BA-8E,则B=\left(\;\;\;\;\;\right)
$$
$$
A.
diag\left(2,-4,2\right) \quad B.diag\left(0,1,0\right) \quad C.diag\left(\frac12,-1,\frac12\right) \quad D.diag\left(1-2,1\right) \quad E. \quad F. \quad G. \quad H.
$$
$$
\begin{array}{l}\mathrm{因所给矩阵方程中含有}A\mathrm{及其伴随矩阵}A^*,\mathrm{故可从公式}AA^*=\left|A\right|E\mathrm{着手}.\;\;\\用A\mathrm{左乘所给方程两边},得\\\;\;\;\;\;\;\;\;\;\;\;\;\;\;\;\;\;\;\;\;\;\;\;\;\;\;AA^* BA=2ABA-8A,\\又\left|A\right|=-2\neq0,故A\mathrm{是可逆矩阵},用A^{-1}\mathrm{右乘上式两边得}\\\;\;\;\;\;\;\;\left|A\right|B=2AB-8E⇒\left(2A+2E\right)B=8E⇒\left(A+E\right)B=4E.\\\mathrm{注意到}\;\;\;\;\;A+E=diag\left(1,-2,1\right)+diag\left(1,1,1\right)=diag\left(2,-1,2\right),\\\mathrm{是可逆矩阵},且\\\;\;\;\;\;\;\;\;\;\left(A+E\right)^{-1}=diag\left(\frac12,-1,\frac12\right),\\\mathrm{于是}\;\;\;B=4\left(A+E\right)^{-1}=diag\left(2,-4,2\right)\\\end{array}
$$



$$
\mathrm{矩阵}A=\begin{pmatrix}1&0&0\\1&1&0\\1&1&1\end{pmatrix},B=\begin{pmatrix}0&1&1\\1&0&1\\1&1&0\end{pmatrix},\mathrm{则矩阵方程}AXA+BXB=AXB+BXA+E\mathrm{的解}X=\left(\;\;\;\right)
$$
$$
A.
\begin{pmatrix}1&2&5\\0&1&2\\0&0&1\end{pmatrix} \quad B.\begin{pmatrix}1&2&0\\0&1&2\\0&0&1\end{pmatrix} \quad C.\begin{pmatrix}1&0&0\\2&1&0\\5&2&1\end{pmatrix} \quad D.\begin{pmatrix}1&0&0\\2&1&0\\0&2&1\end{pmatrix} \quad E. \quad F. \quad G. \quad H.
$$
$$
\begin{array}{l}\mathrm{由于}\;\;\;AXA+BXB-AXB-BXA=E,\\\mathrm{于是}\;\;\;\;AX\left(A-B\right)-BX\left(A-B\right)=E,\\即\;\;\;\;\;\;\;\;\;\;\left(A-B\right)X\left(A-B\right)=E\\\mathrm{因为}\left|A-B\right|\neq0,则\\\;\;\;\;\;\;\;\;\;\;\;\;\;X=\left[\left(A-B\right)^{-1}\right]^2=\begin{pmatrix}1&2&5\\0&1&2\\0&0&1\end{pmatrix}\end{array}
$$



$$
设P=\begin{pmatrix}1&2\\1&4\end{pmatrix},Λ=\begin{pmatrix}1&0\\0&2\end{pmatrix},AP=PΛ,求\;A^n.
$$
$$
A.
\begin{pmatrix}2-2^n&2^n-1\\2-2^{n+1}&2^{n+1}-1\end{pmatrix} \quad B.\begin{pmatrix}2-2^{n+1}&2^{n+1}-1\\2-2^n&2^n-1\end{pmatrix} \quad C.\begin{pmatrix}2^n-2&2^n-1\\2^{n+1}-2&2^{n+1}-1\end{pmatrix} \quad D.\begin{pmatrix}2^{n+1}-2&2^{n+1}-1\\2^n-2&2^n-1\end{pmatrix} \quad E. \quad F. \quad G. \quad H.
$$
$$
\begin{array}{l}\left|P\right|=2,P^{-1}=\frac12\begin{pmatrix}4&-2\\-1&1\end{pmatrix}.\\A=PΛ P^{-1},\;\;\;\;A^n=PΛ^nP^{-1},\\而Λ^n=\begin{pmatrix}1&0\\0&2^n\end{pmatrix},\\故\;\;\;A^n=\begin{pmatrix}1&2\\1&4\end{pmatrix}\begin{pmatrix}1&0\\0&2^n\end{pmatrix}\frac12\begin{pmatrix}4&-2\\-1&1\end{pmatrix}=\frac12\begin{pmatrix}1&2^{n+1}\\1&2^{n+2}\end{pmatrix}\begin{pmatrix}4&-2\\-1&1\end{pmatrix}\\\;\;\;\;=\frac12\begin{pmatrix}4-2^{n+1}&2^{n+1}-2\\4-2^{n+2}&2^{n+2}-2\end{pmatrix}=\begin{pmatrix}2-2^n&2^n-1\\2-2^{n+1}&2^{n+1}-1\end{pmatrix}\end{array}
$$



$$
设P^{-1}AP=Λ,\mathrm{其中}P=\begin{pmatrix}-1&-4\\1&1\end{pmatrix},\;Λ=\begin{pmatrix}-1&0\\0&2\end{pmatrix},则A^{11}=\left(\;\;\;\;\;\right)
$$
$$
A.
\frac13\begin{pmatrix}1+2^{11}&4+2^{11}\\-1-2^{11}&-4-2^{11}\end{pmatrix} \quad B.\frac13\begin{pmatrix}1+2^{13}&4+2^{13}\\-1-2^{11}&-4+2^{11}\end{pmatrix} \quad C.\frac13\begin{pmatrix}1+2^{13}&4+2^{13}\\-1+2^{11}&-4-2^{11}\end{pmatrix} \quad D.\frac13\begin{pmatrix}1+2^{13}&4+2^{13}\\-1-2^{11}&-4-2^{11}\end{pmatrix} \quad E. \quad F. \quad G. \quad H.
$$
$$
\begin{array}{l}由\;\;\;P^{-1}AP=∧⇒ A=P∧ P^{-1}⇒ A^{11}=P∧^{11}P^{-1},\\因\;\;\;\left|P\right|=3,P˟=\begin{pmatrix}1&4\\-1&-1\end{pmatrix},P^{-1}=\frac13\begin{pmatrix}1&4\\-1&-1\end{pmatrix},\\而\;\;\;\;\;\;∧^{11}=\begin{pmatrix}-1&0\\0&2\end{pmatrix}^{11}=\begin{pmatrix}-1&0\\0&2^{11}\end{pmatrix},\\故\;\;\;A^{11}=\frac13\begin{pmatrix}-1&-4\\1&1\end{pmatrix}\begin{pmatrix}-1&0\\0&2^{11}\end{pmatrix}\begin{pmatrix}1&4\\-1&-1\end{pmatrix}\\\;\;\;\;\;\;\;\;=\frac13\begin{pmatrix}1+2^{13}&4+2^{13}\\-1-2^{11}&-4-2^{11}\end{pmatrix}\end{array}
$$



$$
\begin{array}{l}设\left(2E-C^{-1}B\right)A^T=C^{-1},\mathrm{其中}A^T\mathrm{是四阶矩阵}A\mathrm{的转置矩阵},\\\;\;\;\;\;\;\;\;\;B=\begin{pmatrix}1&2&-3&-2\\0&1&2&-3\\0&0&1&2\\0&0&0&1\end{pmatrix},C=\begin{pmatrix}1&2&0&1\\0&1&2&0\\0&0&1&2\\0&0&0&1\end{pmatrix},\\\mathrm{则矩阵}A=\left(\;\;\;\;\right).\\\end{array}
$$
$$
A.
\begin{pmatrix}1&0&0&0\\-2&1&0&0\\1&-2&1&0\\0&1&-2&1\end{pmatrix} \quad B.\begin{pmatrix}1&0&0&0\\-2&1&0&0\\0&-2&1&0\\0&0&-2&1\end{pmatrix} \quad C.\begin{pmatrix}0&0&0&0\\-2&0&0&0\\1&-2&0&0\\0&1&-2&0\end{pmatrix} \quad D.\begin{pmatrix}1&1&0&0\\-2&1&1&0\\0&-2&1&1\\0&0&-2&1\end{pmatrix} \quad E. \quad F. \quad G. \quad H.
$$
$$
\begin{array}{l}\mathrm{用矩阵}C\mathrm{左乘方程的两端},得\\\;\;\;\;\;\;\;\;\;\;\left(2C-B\right)A^T=E,\\\mathrm{对上式两端取转置},得\\\;\;\;\;\;\;\;\;\;\;A\left(2C^T-B^T\right)=E\\\mathrm{因为}A\mathrm{是四阶方阵},故\\\;\;\;\;\;\;\;\;\;A=\left(2C^T-B^T\right)^{-1}=\begin{pmatrix}1&0&0&0\\2&1&0&0\\3&2&1&0\\4&3&2&1\end{pmatrix}^{-1}=\begin{pmatrix}1&0&0&0\\-2&1&0&0\\1&-2&1&0\\0&1&-2&1\end{pmatrix}\end{array}
$$



$$
设AP=P∧,\mathrm{其中}P=\begin{pmatrix}1&0&0\\0&1&2\\0&1&1\end{pmatrix},∧=\begin{pmatrix}-1&&\\&1&\\&&5\end{pmatrix},则A^8\left(5E-6A+A^2\right)=\left(\;\;\;\;\right)
$$
$$
A.
\begin{pmatrix}12&0&0\\0&0&0\\0&0&0\end{pmatrix} \quad B.\begin{pmatrix}1&1&1\\1&1&1\\1&1&1\end{pmatrix} \quad C.\begin{pmatrix}4&1&1\\1&4&1\\1&1&4\end{pmatrix} \quad D.\begin{pmatrix}1&0&0\\0&0&0\\0&0&0\end{pmatrix} \quad E. \quad F. \quad G. \quad H.
$$
$$
\begin{array}{l}∵\left|P\right|=-1\neq0,∴ P\mathrm{可逆},且\\\;\;\;\;\;\;\;\;\;p^{-1}=\begin{pmatrix}1&0&0\\0&-1&2\\0&1&-1\end{pmatrix}\\\;\;\;\;\;\;\;\;⇒ A=P∧ P^{-1}⇒ A^k=P∧^kp^{-1},\\\;\;\;\;\;\;\;\;ϕ\left(A\right)=5A^8-6A^9+A^{10}=P\left[5∧^8-6∧^9+∧^{10}\right]P^{-1}\\\;\;\;\;\;\;\;\;\;=P\begin{pmatrix}12&&\\&0&\\&&0\end{pmatrix}P^{-1}=\begin{pmatrix}1&0&0\\0&1&2\\0&1&1\end{pmatrix}\begin{pmatrix}12&0&0\\0&0&0\\0&0&0\end{pmatrix}\begin{pmatrix}1&0&0\\0&-1&2\\0&1&-1\end{pmatrix}\\\;\;\;\;\;\;\;\;\;\;=\begin{pmatrix}12&0&0\\0&0&0\\0&0&0\end{pmatrix}\\\;\;\;\end{array}
$$



$$
\mathrm{已知}A,B\mathrm{为三阶方阵},\mathrm{且满足}\left|A\right|=3,\left|B\right|=4,\left|A^{-1}+\boldsymbol B\right|=8,则\left|B^{-1}+A\right|=(\;\;)
$$
$$
A.
2 \quad B.3 \quad C.4 \quad D.6 \quad E. \quad F. \quad G. \quad H.
$$
$$
A\left(A^{-1}+\boldsymbol B\right)\boldsymbol B^{-1}\boldsymbol=\boldsymbol B^{\boldsymbol-\mathbf1}\boldsymbol+\boldsymbol A\boldsymbol,\boldsymbol 则\left|\mathbf B^{\boldsymbol-\mathbf1}\boldsymbol+\mathbf A\right|\boldsymbol=\mathbf6
$$



$$
设A^2+AB+8A=A^3+4E,A=\begin{pmatrix}1&-1&-1&-1\\-1&1&-1&-1\\-1&-1&1&-1\\-1&-1&-1&1\end{pmatrix},\mathrm{则四阶矩阵}B=\left(\;\;\;\;\right)
$$
$$
A.
-4E \quad B.4E \quad C.-2E \quad D.2E \quad E. \quad F. \quad G. \quad H.
$$
$$
\begin{array}{l}\mathrm{由条件可计算得}\\A^2=4E\left(\mathrm{所以}A\mathrm{可逆}\right),\;\;\;A^3=4A,\\\;\;\;\;\;\;\;\;AB=-4A,\;\;B=-4E\end{array}
$$



$$
\begin{array}{l}设A,B\mathrm{是两个三阶矩阵},\mathrm{满足关系}:\\\;\;\;\;\;\;A^2-AB-2B^2=A-2BA-B\\\mathrm{已知}\;B=\begin{pmatrix}-1&2&1\\0&1&-1\\1&3&2\end{pmatrix},且\left|A-B\right|\neq0,则A=\left(\;\;\;\;\right).\end{array}
$$
$$
A.
\begin{pmatrix}3&4&-2\\0&-1&2\\-2&6&3\end{pmatrix} \quad B.\begin{pmatrix}3&4&2\\0&-1&2\\2&6&-3\end{pmatrix} \quad C.\begin{pmatrix}3&-4&-2\\0&-1&2\\-2&-6&-3\end{pmatrix} \quad D.\begin{pmatrix}3&4&-2\\0&-1&2\\-2&6&-3\end{pmatrix} \quad E. \quad F. \quad G. \quad H.
$$
$$
\begin{array}{l}\mathrm{由所给关系得}\\\;\;\;\;\;\;\;\;\;\;\;\;\;\;\;\;\;\;\left(A+2B\right)\left(A-B\right)-\left(A-B\right)=O\\即\\\;\;\;\;\;\;\;\;\;\;\;\;\;\;\;\;\;\;\left(A+2B-E\right)\left(A-B\right)=O\\由\left|A-B\right|\neq0,知A=E-2B=\begin{pmatrix}3&-4&-2\\0&-1&2\\-2&-6&-3\end{pmatrix}\\\end{array}
$$



$$
设a=\begin{pmatrix}1\\2\\0\end{pmatrix},A=aa^T+E,\mathrm{矩阵}X\mathrm{满足}\frac13A^* X=A^{-1}+X,\mathrm{则矩阵}X=\left(\;\;\;\;\right)
$$
$$
A.
\begin{pmatrix}-\frac34&\frac12&0\\\frac12&0&0\\0&0&1\end{pmatrix} \quad B.\begin{pmatrix}\frac34&-\frac12&0\\-\frac12&0&0\\0&0&-1\end{pmatrix} \quad C.\begin{pmatrix}\frac34&-\frac12&0\\-\frac12&0&0\\0&0&1\end{pmatrix} \quad D.\begin{pmatrix}-\frac34&\frac12&0\\\frac12&0&0\\0&0&-1\end{pmatrix} \quad E. \quad F. \quad G. \quad H.
$$
$$
\begin{array}{l}\;\;\;\;\;\;\;\;\;\left|A\right|=\left|aa^T+E\right|=\begin{vmatrix}2&2&0\\2&5&0\\0&0&1\end{vmatrix}=6.\\由\frac13A^* X=A^{-1}+X,得\left(2E-A\right)X=E,由2E-A=\begin{pmatrix}0&-2&0\\-2&-3&0\\0&0&1\end{pmatrix}\;知\\\;\;\;\;\;\;\;\;\;\;\;X=\left(2E-A\right)^{-1}=\begin{pmatrix}\frac34&-\frac12&0\\-\frac12&0&0\\0&0&1\end{pmatrix}\end{array}
$$



$$
设A,B为n\mathrm{阶矩阵},\mathrm{且满足}2B^{-1}A=A-4E,\mathrm{其中}E为n\mathrm{阶单位矩阵},若A=\begin{pmatrix}1&-2&0\\1&2&0\\0&0&2\end{pmatrix},则B=\left(\;\;\;\right).
$$
$$
A.
\begin{pmatrix}0&2&0\\-1&-1&0\\0&0&-2\end{pmatrix} \quad B.\begin{pmatrix}1&2&0\\-1&-1&0\\0&0&2\end{pmatrix} \quad C.\begin{pmatrix}0&2&1\\-1&-1&2\\2&0&-2\end{pmatrix} \quad D.\begin{pmatrix}-2&0&0\\0&0&2\\0&-1&-1\end{pmatrix} \quad E. \quad F. \quad G. \quad H.
$$
$$
\begin{array}{l}\mathrm{由于}2B^{-1}A=A-4E,\mathrm{方程两边同时左乘}B,有\\2A=B(A-4E)\\\mathrm{所以}B=2A(A-4E)^{-1}\\\;\;\;\;\;A-4E=\begin{pmatrix}-3&-2&0\\1&-2&0\\0&0&-2\end{pmatrix},\;\left(A-4E\right)^{-1}=\begin{pmatrix}-\frac14&\frac14&0\\-\frac18&-\frac38&0\\0&0&-\frac12\end{pmatrix},\\故\;\;\;\;\;\;\;\;\;B=\begin{pmatrix}0&2&0\\-1&-1&0\\0&0&-2\end{pmatrix}\end{array}
$$



$$
\begin{array}{l}设\left(2E-C^{-1}B\right)A^T=C^{-1},\mathrm{其中}A^T\mathrm{是四阶矩阵}A\mathrm{的转置矩阵},\\B=\begin{pmatrix}1&2&-3&2\\0&1&2&-3\\0&0&1&2\\0&0&0&1\end{pmatrix},C=\begin{pmatrix}1&1&0&1\\0&1&2&0\\0&0&1&2\\0&0&0&1\end{pmatrix},\mathrm{则矩阵}A=\left(\;\;\;\;\right).\end{array}
$$
$$
A.
\begin{pmatrix}1&0&0&0\\0&1&0&0\\3&2&1&0\\0&3&2&1\end{pmatrix} \quad B.\begin{pmatrix}1&0&0&0\\0&1&0&0\\-3&2&1&0\\6&1&-2&1\end{pmatrix} \quad C.\begin{pmatrix}1&0&0&0\\0&1&0&0\\-3&-2&1&0\\-6&1&-2&1\end{pmatrix} \quad D.\begin{pmatrix}1&0&0&0\\0&1&0&0\\-3&-2&1&0\\6&1&-2&1\end{pmatrix} \quad E. \quad F. \quad G. \quad H.
$$
$$
\begin{array}{l}\mathrm{用矩阵}C\mathrm{左乘方程的两端},得\left(2C-B\right)A^T=E,\mathrm{对上式两端取转置},得A\left(2C^T-B^T\right)=E,\mathrm{因为}A\mathrm{是四阶方阵},\\故\\\;\;\;\;\;\;\;\;\;\;\;\;\;\;\;\;\;\;\;\;\;\;\;\;\;\;\;\;\;A\left(2C^T-B^T\right)=\begin{pmatrix}1&0&0&0\\0&1&0&0\\3&2&1&0\\0&3&2&1\end{pmatrix}^{-1}=\begin{pmatrix}1&0&0&0\\0&1&0&0\\-3&-2&1&0\\6&1&-2&1\end{pmatrix}\end{array}
$$



$$
设B=\begin{pmatrix}1&-1&0\\0&1&-1\\0&0&1\end{pmatrix},C=\begin{pmatrix}2&1&3\\0&2&1\\0&0&2\end{pmatrix},\mathrm{矩阵}X\mathrm{满足关系式}X\lbrack E-B^T{(C^{-1})}^T\rbrack C^T=E,则X=(\;\;)
$$
$$
A.
\begin{pmatrix}1&0&0\\-2&1&0\\-1&2&1\end{pmatrix} \quad B.\begin{pmatrix}1&0&0\\2&1&0\\-1&-2&1\end{pmatrix} \quad C.\begin{pmatrix}1&0&0\\-2&1&0\\1&-2&1\end{pmatrix} \quad D.\begin{pmatrix}1&0&0\\-2&1&0\\-1&-2&1\end{pmatrix} \quad E. \quad F. \quad G. \quad H.
$$
$$
由X\lbrack E-B^T{(C^{-1})}^T\rbrack C^T=E则X{(C-B)}^T=E\mathrm{所以}X={\lbrack{(C-B)}^{-1}\rbrack}^T,\\有\;C-B=\begin{pmatrix}1&2&3\\0&1&2\\0&0&1\end{pmatrix},{(C-B)}^{-1}=\begin{pmatrix}1&-2&1\\0&1&-2\\0&0&1\end{pmatrix},\mathrm{因此}X=\begin{pmatrix}1&0&0\\-2&1&0\\1&-2&1\end{pmatrix}
$$



$$
\mathrm{已知矩阵}X\;\mathrm{满足关系}:XA=B^T+3X,\;\mathrm{其中}A=\begin{pmatrix}4&-3\\2&1\end{pmatrix},B=\begin{pmatrix}2&3&0\\0&-1&4\end{pmatrix},则X=\left(\;\;\;\;\;\right).
$$
$$
A.
\begin{pmatrix}-1&\frac12\\-1&2\\-2&1\end{pmatrix} \quad B.\begin{pmatrix}-1&\frac32\\-1&2\\-2&1\end{pmatrix} \quad C.\begin{pmatrix}-1&\frac32\\-1&2\\2&-1\end{pmatrix} \quad D.\begin{pmatrix}-1&\frac32\\1&2\\-2&1\end{pmatrix} \quad E. \quad F. \quad G. \quad H.
$$
$$
\begin{array}{l}由XA=B^T+3X,\;得XA-3X=B^T,\;即X\left(A-3E\right)=B^T.\\A-3E=\begin{pmatrix}4&-3\\2&1\end{pmatrix}-\begin{pmatrix}3&0\\0&3\end{pmatrix}=\begin{pmatrix}1&-3\\2&-2\end{pmatrix},\\\mathrm{所以}\left(A-3E\right)^{-1}=\begin{pmatrix}1&-3\\2&-2\end{pmatrix}^{-1}=\frac14\begin{pmatrix}-2&3\\-2&1\end{pmatrix}.\\\mathrm{在等式}X\left(A-3E\right)=B^T\mathrm{两端右乘}\left(A-3E\right)^{-1},\;得\\X=B^T\left(A-3E\right)^{-1}=\begin{pmatrix}2&0\\3&-1\\0&4\end{pmatrix}·\frac14\begin{pmatrix}-2&3\\-2&1\end{pmatrix}=\frac14\begin{pmatrix}-4&6\\-4&8\\-8&4\end{pmatrix}=\begin{pmatrix}-1&\frac32\\-1&2\\-2&1\end{pmatrix}\end{array}
$$



$$
\mathrm{已知}A=\begin{pmatrix}1&1&-1\\0&1&1\\0&0&-1\end{pmatrix},B=\begin{pmatrix}2&0&1\\0&2&0\\0&0&2\end{pmatrix},且AXB=AX+A^2B-A^2+B,\mathrm{则矩阵}X=\left(\;\;\;\right).
$$
$$
A.
\begin{pmatrix}3&-1&-6\\0&3&3\\0&0&3\end{pmatrix} \quad B.\begin{pmatrix}3&-1&-6\\0&3&3\\0&0&-3\end{pmatrix} \quad C.\begin{pmatrix}3&1&-6\\0&3&3\\0&0&-3\end{pmatrix} \quad D.\begin{pmatrix}3&-1&6\\0&3&3\\0&0&-3\end{pmatrix} \quad E. \quad F. \quad G. \quad H.
$$
$$
\begin{array}{l}由AXB=AX+A^2B-A^2+B有AXB-AX-A^2B+A^2=B,\\则AX\left(B-E\right)-A^2\left(B-E\right)=B故\;A\left(X-A\right)\left(B-E\right)=B,\\因A,\;B-E\mathrm{均可逆},\mathrm{方程两边同时左乘}A^{-1},\mathrm{右乘}\left(B-E\right)^{-1},\\得X-A=A^{-1}B\left(B-E\right)^{-1},X=A^{-1}B\left(B-E\right)^{-1}+A,\\由A^{-1}=\begin{pmatrix}1&-1&-2\\0&1&1\\0&0&-1\end{pmatrix},B-E=\begin{pmatrix}1&0&1\\0&1&0\\0&0&1\end{pmatrix},\left(B-E\right)^{-1}=\begin{pmatrix}1&0&-1\\0&1&0\\0&0&1\end{pmatrix}\\故\\X=\begin{pmatrix}1&1&-1\\0&1&1\\0&0&-1\end{pmatrix}-\begin{pmatrix}1&-1&-2\\0&1&1\\0&0&-1\end{pmatrix}\begin{pmatrix}2&0&1\\0&2&0\\0&0&2\end{pmatrix}\begin{pmatrix}1&0&-1\\0&1&0\\0&0&1\end{pmatrix}\\\;\;\;=\begin{pmatrix}1&1&-1\\0&1&1\\0&0&-1\end{pmatrix}+\begin{pmatrix}2&-2&-5\\0&2&2\\0&0&-2\end{pmatrix}=\begin{pmatrix}3&-1&-6\\0&3&3\\0&0&-3\end{pmatrix}\end{array}
$$



$$
设A^2+AB-A=0,\mathrm{其中}A=\begin{pmatrix}1&-1&-1&-1\\-1&1&-1&-1\\-1&-1&1&-1\\-1&-1&-1&1\end{pmatrix},\mathrm{则矩阵}B=\left(\;\;\;\right).
$$
$$
A.
\begin{pmatrix}0&1&1&1\\1&0&1&1\\1&1&0&1\\1&1&1&0\end{pmatrix} \quad B.\begin{pmatrix}0&1&1&1\\1&1&1&1\\1&1&0&1\\1&1&1&1\end{pmatrix} \quad C.\begin{pmatrix}0&-1&-1&-1\\1&0&1&1\\1&1&0&1\\1&1&1&0\end{pmatrix} \quad D.\begin{pmatrix}0&1&1&1\\1&0&1&1\\1&1&0&1\\-1&-1&-1&0\end{pmatrix} \quad E. \quad F. \quad G. \quad H.
$$
$$
\begin{array}{l}A^2=4E,故A\mathrm{可逆},\mathrm{因此},由\\A\left(A+B-E\right)=O知A+B-E=O\\B=E-A=\begin{pmatrix}0&1&1&1\\1&0&1&1\\1&1&0&1\\1&1&1&0\end{pmatrix}\end{array}
$$



$$
\mathrm{矩阵}A=\begin{pmatrix}2&2&0\\2&1&3\\0&1&0\end{pmatrix},\mathrm{则矩阵方程}AX=A+X\mathrm{的解}X=(\;\;\;).
$$
$$
A.
\begin{pmatrix}-2&2&6\\2&0&-3\\2&-1&-3\end{pmatrix} \quad B.\begin{pmatrix}2&2&6\\2&0&3\\2&0&3\end{pmatrix} \quad C.\begin{pmatrix}-2&1&3\\1&0&-3\\1&-1&-3\end{pmatrix} \quad D.\begin{pmatrix}-1&2&6\\0&0&-3\\2&-1&-3\end{pmatrix} \quad E. \quad F. \quad G. \quad H.
$$
$$
\begin{array}{l}\mathrm{把所给方程变形为}(A-E)X=A,则X=(A-E)^{-1}A.\\(A-E\;\;\;A)=\begin{pmatrix}1&2&0&2&2&0\\2&0&3&2&1&3\\0&1&-1&0&1&0\end{pmatrix}\xrightarrow[{r_2↔ r_3}]{r_2-2r_1}\begin{pmatrix}1&2&0&2&2&0\\0&1&-1&0&1&0\\0&-4&3&-2&-3&3\end{pmatrix}\\\xrightarrow[{r_3\div(-1)}]{r_2+4r_2}\begin{pmatrix}1&2&0&2&2&0\\0&1&-1&0&1&0\\0&0&1&2&-1&-3\end{pmatrix}\xrightarrow[{}]{r_2+r_3}\begin{pmatrix}1&2&0&2&2&0\\0&1&0&2&0&-3\\0&0&1&2&-1&-3\end{pmatrix}\\\xrightarrow[{}]{r_1-2r_2}\begin{pmatrix}1&0&0&-2&2&6\\0&1&0&2&0&-3\\0&0&1&2&-1&-3\end{pmatrix},\\\mathrm{即得}\;\;\;\;\;X=\begin{pmatrix}-2&2&6\\2&0&-3\\2&-1&-3\end{pmatrix}.\end{array}
$$



$$
\mathrm{设三阶矩阵}A,\;B\mathrm{满足关系式}:A^{-1}BA=6A+BA,且A=\begin{pmatrix}\frac13&&\\&\frac14&\\&&\frac17\end{pmatrix},则B=\left(\;\;\;\right)
$$
$$
A.
\begin{pmatrix}3&0&0\\0&1&0\\0&0&2\end{pmatrix} \quad B.\begin{pmatrix}3&0&0\\0&2&0\\0&0&1\end{pmatrix} \quad C.\begin{pmatrix}1&0&0\\0&2&0\\0&0&3\end{pmatrix} \quad D.\begin{pmatrix}2&0&0\\0&1&0\\0&0&3\end{pmatrix} \quad E. \quad F. \quad G. \quad H.
$$
$$
由A^{-1}BA=6A+BA知:B=6\left(A^{-1}-E\right)^{-1}=\begin{pmatrix}3&0&0\\0&2&0\\0&0&1\end{pmatrix}
$$



$$
\mathrm{已知}A=\begin{pmatrix}1&1&1\\0&1&0\\2&2&3\end{pmatrix},且A^2-AB=E,则B=\left(\;\;\;\;\right).
$$
$$
A.
\begin{pmatrix}-2&-2&2\\0&0&0\\4&2&2\end{pmatrix} \quad B.\begin{pmatrix}-2&2&2\\0&0&0\\4&2&2\end{pmatrix} \quad C.\begin{pmatrix}-2&2&2\\0&0&0\\4&-2&2\end{pmatrix} \quad D.\begin{pmatrix}-2&2&2\\0&0&0\\4&2&-2\end{pmatrix} \quad E. \quad F. \quad G. \quad H.
$$
$$
\begin{array}{l}由A^2-AB=E⇒ A\left(A-B\right)=E⇒ A-B=A^{-1}\\⇒ B=A-A^{-1}=\begin{pmatrix}1&1&1\\0&1&0\\2&2&3\end{pmatrix}-\begin{pmatrix}3&-1&-1\\0&1&0\\-2&0&1\end{pmatrix}=\begin{pmatrix}-2&2&2\\0&0&0\\4&2&2\end{pmatrix}\end{array}
$$



$$
\mathrm{已知}A=diag\left(2,-2,1\right),A^* BA=4BA-8E,则B=\left(\;\;\;\;\;\right)
$$
$$
A.
diag\left(\frac23,-2,1\right) \quad B.diag\left(\frac23,2,1\right) \quad C.diag\left(\frac23,-2,2\right) \quad D.diag\left(\frac23,-1,1\right) \quad E. \quad F. \quad G. \quad H.
$$
$$
\begin{array}{l}\mathrm{因所给矩阵方程中含有}A\mathrm{及其伴随矩阵}A^*,\mathrm{故可从公式}AA^*=\left|A\right|E\mathrm{着手}.\;\;\\用A\mathrm{左乘所给方程两边},得\\\;\;\;\;\;\;\;\;\;\;\;\;\;\;\;\;\;\;\;\;\;\;\;\;\;\;AA^* BA=4ABA-8A,\\又\left|A\right|=-4\neq0,故A\mathrm{是可逆矩阵},用A^{-1}\mathrm{右乘上式两边得}\\\;\;\;\;\;\;\;\left|A\right|B=4AB-8E⇒\left(4A+4E\right)B=8E⇒\left(A+E\right)B=2E.\\\mathrm{注意到}\;\;\;\;\;A+E=diag\left(2,-2,1\right)+diag\left(1,1,1\right)=diag\left(3,-1,2\right),\\\mathrm{是可逆矩阵},且\\\;\;\;\;\;\;\;\;\;\left(A+E\right)^{-1}=diag\left(\frac13,-1,\frac12\right),\\\mathrm{于是}\;\;\;B=2\left(A+E\right)^{-1}=diag\left(\frac23,-2,1\right)\\\end{array}
$$



$$
\mathrm{已知}A=diag\left(2,-2,1\right),A^* BA=4BA-8E,则B^{-1}=\left(\;\;\;\;\;\right)
$$
$$
A.
diag\left(\frac12,\frac{-1}2,1\right) \quad B.diag\left(\frac23,2,1\right) \quad C.diag\left(\frac32,-1,1\right) \quad D.diag\left(\frac32,\frac{-1}2,1\right) \quad E. \quad F. \quad G. \quad H.
$$
$$
\begin{array}{l}\mathrm{因所给矩阵方程中含有}A\mathrm{及其伴随矩阵}A^*,\mathrm{故可从公式}AA^*=\left|A\right|E\mathrm{着手}.\;\;\\用A\mathrm{左乘所给方程两边},得\\\;\;\;\;\;\;\;\;\;\;\;\;\;\;\;\;\;\;\;\;\;\;\;\;\;\;AA^* BA=4ABA-8A,\\又\left|A\right|=-4\neq0,故A\mathrm{是可逆矩阵},用A^{-1}\mathrm{右乘上式两边得}\\\;\;\;\;\;\;\;\left|A\right|B=4AB-8E⇒\left(4A+4E\right)B=8E⇒\left(A+E\right)B=2E.\\\mathrm{注意到}\;\;\;\;\;A+E=diag\left(2,-2,1\right)+diag\left(1,1,1\right)=diag\left(3,-1,2\right),\\\\\\\mathrm{于是}\;\;\;B^{-1}=\frac{A+E}2=diag\left(\frac32,\frac{-1}2,1\right)\\\end{array}
$$



$$
设P^{-1}AP=Λ,\mathrm{其中}P=\begin{pmatrix}-1&-4\\1&1\end{pmatrix},\;Λ=\begin{pmatrix}-1&0\\0&2\end{pmatrix},则A^{13}=\left(\;\;\;\;\;\right)
$$
$$
A.
\frac13\begin{pmatrix}1+2^{15}&4-2^{15}\\-1-2^{13}&-4-2^{13}\end{pmatrix} \quad B.\frac13\begin{pmatrix}1+2^{15}&4+2^{13}\\-1-2^{13}&-4-2^{13}\end{pmatrix} \quad C.\frac13\begin{pmatrix}1+2^{15}&4+2^{15}\\-1+2^{13}&-4-2^{13}\end{pmatrix} \quad D.\frac13\begin{pmatrix}1+2^{15}&4+2^{15}\\-1-2^{13}&-4-2^{13}\end{pmatrix} \quad E. \quad F. \quad G. \quad H.
$$
$$
\begin{array}{l}由\;\;\;P^{-1}AP=Λ⇒ A=PΛ P^{-1}⇒ A^{13}=PΛ^{13}P^{-1},\\因\;\;\;\left|P\right|=3,P^{-1}=\frac13\begin{pmatrix}1&4\\-1&-1\end{pmatrix},\\而\;\;\;\;\;\;Λ^{13}=\begin{pmatrix}-1&0\\0&2\end{pmatrix}^{13}=\begin{pmatrix}-1&0\\0&2^{13}\end{pmatrix},\\故\;\;\;A^{13}=\frac13\begin{pmatrix}-1&-4\\1&1\end{pmatrix}\begin{pmatrix}-1&0\\0&2^{13}\end{pmatrix}\begin{pmatrix}1&4\\-1&-1\end{pmatrix}\\\;\;\;\;\;\;\;\;=\frac13\begin{pmatrix}1+2^{15}&4+2^{15}\\-1-2^{13}&-4-2^{13}\end{pmatrix}\end{array}
$$



$$
设P^{-1}AP=Λ,\mathrm{其中}P=\begin{pmatrix}-1&-4\\1&1\end{pmatrix},\;Λ=\begin{pmatrix}-1&0\\0&2\end{pmatrix},则A^{10}=\left(\;\;\;\;\;\right)
$$
$$
A.
\frac13\begin{pmatrix}-1-2^{12}&-4+2^{12}\\1-2^{10}&4-2^{10}\end{pmatrix} \quad B.\frac13\begin{pmatrix}-1+2^{12}&-4-2^{12}\\1-2^{10}&4-2^{10}\end{pmatrix} \quad C.\frac13\begin{pmatrix}1+2^{12}&-4+2^{12}\\1-2^{10}&4-2^{10}\end{pmatrix} \quad D.\frac13\begin{pmatrix}-1+2^{12}&-4+2^{12}\\1-2^{10}&4-2^{10}\end{pmatrix} \quad E. \quad F. \quad G. \quad H.
$$
$$
\begin{array}{l}由\;\;\;P^{-1}AP=Λ⇒ A=PΛ P^{-1}⇒ A^{10}=PΛ^{10}P^{-1},\\因\;\;\;\left|P\right|=3,P^{-1}=\frac13\begin{pmatrix}1&4\\-1&-1\end{pmatrix},\\而\;\;\;\;\;\;Λ^{10}=\begin{pmatrix}-1&0\\0&2\end{pmatrix}^{10}=\begin{pmatrix}1&0\\0&2^{10}\end{pmatrix},\\故\;\;\;A^{13}=\frac13\begin{pmatrix}-1&-4\\1&1\end{pmatrix}\begin{pmatrix}1&0\\0&2^{10}\end{pmatrix}\begin{pmatrix}1&4\\-1&-1\end{pmatrix}\\\;\;\;\;\;\;\;\;=\frac13\begin{pmatrix}-1+2^{12}&-4+2^{12}\\1-2^{10}&4-2^{10}\end{pmatrix}\end{array}
$$



$$
\mathrm{已知}A=diag\left(2,-1,4\right),A^* BA=4BA-8E,则B^{-1}=\left(\;\;\;\;\;\right)
$$
$$
A.
diag\left(2,\frac12,3\right) \quad B.diag\left(2,1,3\right) \quad C.diag\left(2,\frac12,2\right) \quad D.diag\left(3,\frac12,1\right) \quad E. \quad F. \quad G. \quad H.
$$
$$
\begin{array}{l}\mathrm{因所给矩阵方程中含有}A\mathrm{及其伴随矩阵}A^*,\mathrm{故可从公式}AA^*=\left|A\right|E\mathrm{着手}.\;\;\\用A\mathrm{左乘所给方程两边},得\\\;\;\;\;\;\;\;\;\;\;\;\;\;\;\;\;\;\;\;\;\;\;\;\;\;\;AA^* BA=4ABA-8A,\\又\left|A\right|=-8\neq0,故A\mathrm{是可逆矩阵},用A^{-1}\mathrm{右乘上式两边得}\\\;\;\;\;\;\;\;\left|A\right|B=4AB-8E⇒\left(4A+8E\right)B=8E⇒\left(A+2E\right)B=2E.\\\mathrm{注意到}\;\;\;\;\;A+2E=diag\left(4,1,6\right),\\\\\\\mathrm{于是}\;\;\;B^{-1}=\frac{A+2E}2=diag\left(2,\frac12,3\right)\\\end{array}
$$



$$
设B=\frac14(A^2-A-2E),\mathrm{其中}A=\begin{pmatrix}0&1&0\\1&2&0\\0&0&3\end{pmatrix},则B^{-1}=\left(\;\;\;\;\;\right).
$$
$$
A.
\begin{pmatrix}-2&2&0\\2&2&0\\0&0&-1\end{pmatrix} \quad B.\begin{pmatrix}-2&2&0\\2&-2&0\\0&0&1\end{pmatrix} \quad C.\;\begin{pmatrix}-2&2&0\\-2&2&0\\0&0&1\end{pmatrix} \quad D.\begin{pmatrix}-2&2&0\\2&2&0\\0&0&1\end{pmatrix} \quad E. \quad F. \quad G. \quad H.
$$
$$
\begin{array}{l}B=\frac14\left(A-2E\right)\left(A+E\right),\mathrm{所以}B^{-1}=4(A+E)^{-1}(A-2E)^{-1}\\\;\;\;\;4\;(A+E)^{-1}=\begin{pmatrix}6&-2&0\\-2&2&0\\0&0&1\end{pmatrix},\;\;(A-2E)^{-1}=\begin{pmatrix}0&1&0\\1&2&0\\0&0&1\end{pmatrix},\mathrm{所以}B^{-1}=\begin{pmatrix}-2&2&0\\2&2&0\\0&0&1\end{pmatrix}\end{array}
$$



$$
设B=\frac14(A^2-A-2E),\mathrm{其中}A=\begin{pmatrix}2&1&0\\1&0&0\\0&0&3\end{pmatrix},则B^{-1}=\left(\;\;\;\;\;\right).
$$
$$
A.
\begin{pmatrix}2&2&0\\2&-2&0\\0&0&1\end{pmatrix} \quad B.\begin{pmatrix}2&-2&0\\2&-2&0\\0&0&4\end{pmatrix} \quad C.\begin{pmatrix}2&2&0\\2&-2&0\\0&0&2\end{pmatrix} \quad D.\begin{pmatrix}2&2&0\\-2&-2&0\\0&0&4\end{pmatrix} \quad E. \quad F. \quad G. \quad H.
$$
$$
\begin{array}{l}B=\frac14\left(A-2E\right)\left(A+E\right),\mathrm{所以}B^{-1}=4(A+E)^{-1}(A-2E)^{-1}\\\;\;\;\;4\;(A+E)^{-1}=\begin{pmatrix}2&-2&0\\-2&6&0\\0&0&1\end{pmatrix},\;\;(A-2E)^{-1}=\begin{pmatrix}2&1&0\\1&0&0\\0&0&1\end{pmatrix},\mathrm{所以}B^{-1}=\begin{pmatrix}2&2&0\\2&-2&0\\0&0&1\end{pmatrix}\end{array}
$$



$$
\mathrm 设A=\begin{pmatrix}1&2&1\\2&1&2\end{pmatrix},B=\begin{pmatrix}4&3&2\\-2&1&-2\end{pmatrix},且(2A-X) 2(B-X)=O,则X=()
$$
$$
A.
\begin{pmatrix}\frac{10}3&\frac{10}3&2\\0&\frac43&0\end{pmatrix} \quad B.\begin{pmatrix}\frac{20}3&\frac{20}3&2\\0&\frac43&0\end{pmatrix} \quad C.\begin{pmatrix}\frac{10}3&\frac{10}3&2\\0&\frac{10}3&0\end{pmatrix} \quad D.\begin{pmatrix}\frac53&\frac53&2\\0&\frac43&0\end{pmatrix} \quad E. \quad F. \quad G. \quad H.
$$
$$
由\left(2A-X\right) 2\left(B-X\right)=0得,X=\frac23\left(A B\right)=\begin{pmatrix}\frac{10}3&\frac{10}3&2\\0&\frac43&0\end{pmatrix}
$$



$$
\mathrm{向量}β=\left(3,\;5,\;-6\right)^T\mathrm{可表示为向量组}α_1=\left(1,\;0,\;1\right)^T,\;α_2=\left(1,\;1,\;1\right)^T,\;α_3=\left(0,\;-1,\;-1\right)^T\mathrm{的线性组合表达式为}\left(\right).\;
$$
$$
A.
β=-11α_1+14α_2+9α_3 \quad B.β=11α_1-8α_2+14α_3 \quad C.β=-11α_1+14α_2+8α_3 \quad D.β=11α_1-8α_2+9α_3 \quad E. \quad F. \quad G. \quad H.
$$
$$
\begin{array}{l}\mathrm{对矩阵}A=\left(α_1^T,\;α_2^T,\;α_3^T,\;β^T\right)\mathrm{仅施以初等行变换}\\A=\left(α_1^T,\;α_2^T,\;α_3^T,\;β^T\right)=\begin{pmatrix}1&1&0&3\\0&1&-1&5\\1&1&-1&-6\end{pmatrix}\overrightarrow{r_3-r_1}\begin{pmatrix}1&1&0&3\\0&1&-1&5\\0&0&-1&-9\end{pmatrix}\\\overset{r_2-r_3}{\underset{}{\overrightarrow{\left(-1\right)r_3}}}\begin{pmatrix}1&1&0&3\\0&1&0&14\\0&0&1&9\end{pmatrix}\overset{}{\overrightarrow{r_1-r_2}}\begin{pmatrix}1&0&0&-11\\0&1&0&14\\0&0&1&9\end{pmatrix},\;\;\mathrm{所以}\;\;\;β=-11α_1+14α_2+9α_3\end{array}
$$



$$
\mathrm{如果向量}β\mathrm{可由向量组}α_1,\;α_2,\;⋯,\;α_s\mathrm{线性表示},\mathrm{则下列结论中正确的是}\left(\right).
$$
$$
A.
\mathrm{存在一组不全为零的数}k_1,\;k_2,\;⋯,\;k_s,\mathrm{使等式}β=k_1α_1+k_2α_2+⋯+k_sα_s\mathrm{成立} \quad B.\mathrm{存在一组全不为零的数}k_1,\;k_2,\;⋯,\;k_s,\mathrm{使等式}β=k_1α_1+k_2α_2+⋯+k_sα_s\mathrm{成立} \quad C.\mathrm{存在一组数}k_1,\;k_2,\;⋯,\;k_s,\mathrm{使等式}β=k_1α_1+k_2α_2+⋯+k_sα_s\mathrm{成立} \quad D.对β\mathrm{的线性表达式唯一} \quad E. \quad F. \quad G. \quad H.
$$
$$
\begin{array}{l}\mathrm{向量}β\mathrm{能由向量组}α_1,\;α_2,\;⋯,\;α_s\mathrm{线性表示},\mathrm{仅要求存在一组数}k_1,\;k_2,\;⋯,\;k_s,\\\mathrm{使得}β=k_1α_1+k_2α_2+⋯+k_sα_s\mathrm{成立}.\end{array}
$$



$$
\mathrm{将向量}β=\left(4,\;-1\right)^T\mathrm{表成向量}α_1=\left(1,\;2\right)^T,\;α_2=\left(2,\;3\right)^T\mathrm{的线性组合为}\;\left(\right).
$$
$$
A.
-14α_1+9α_2 \quad B.2α_1+α_2 \quad C.-10α_1+7α_2 \quad D.-14α_1+7α_2 \quad E. \quad F. \quad G. \quad H.
$$
$$
\begin{array}{l}设β=x_1α_1+x_2α_2,则\left\{\begin{array}{l}x_1+2x_2=4\\2x_1+3x_2=-1\end{array}\right.⇒ x_1=-14,\;x_2=9.\;\;\\或\begin{pmatrix}1&2&4\\2&3&-1\end{pmatrix}\rightarrow\begin{pmatrix}1&2&4\\0&-1&-9\end{pmatrix}\rightarrow\begin{pmatrix}1&0&-14\\0&1&9\end{pmatrix},故-14α_1+9α_2.\end{array}
$$



$$
\mathrm{已知向量}α_1=\left(1,\;1,\;0\right)^T,\;α_2=\left(1,\;0,\;1\right)^T,\;α_3=\left(0,\;1,\;1\right)^T,\;β=\left(2,\;0,\;0\right)^T.\mathrm{若用}α_1,\;\;α_2,\;α_3\mathrm{的线性组合来表示}β,则β=\left(\right).
$$
$$
A.
α_1+α_2-α_3 \quad B.2α_1-α_2-α_3 \quad C.α_1+α_2-2α_3 \quad D.2α_1-α_2-2α_3 \quad E. \quad F. \quad G. \quad H.
$$
$$
\begin{array}{l}设β=k_1α_1+k_2α_2+k_3α_3,\mathrm{对矩阵}\left(α_1,\;α_2,\;α_3,\;β\right)\mathrm{施以初等行变换}:\\\;\;\begin{pmatrix}1&1&0&2\\1&0&1&0\\0&1&1&0\end{pmatrix}\rightarrow\begin{pmatrix}1&1&0&2\\0&-1&1&-2\\0&0&2&-2\end{pmatrix}\rightarrow\begin{pmatrix}1&0&1&0\\0&1&-1&2\\0&0&1&-1\end{pmatrix}\rightarrow\begin{pmatrix}1&0&0&1\\0&1&0&1\\0&0&1&-1\end{pmatrix},\;\;\\\mathrm{由最后一个矩阵可知},取k_1=1,\;k_2=1,\;k_3=-1\mathrm{可使}β=α_1+α_2-α_3\end{array}
$$



$$
\mathrm{已知向量}α_1=\left(1,\;0,\;0\right)^T,\;α_2=\left(0,\;1,\;2\right)^T,\;α_3=\left(0,\;0,\;1\right)^T,\;是R^3\mathrm{的一个基},则α=\left(1,\;2,\;3\right)^T,\mathrm{可用该基线性表示为}α=\left(\right)\;.
$$
$$
A.
α_1+2α_2-α_3 \quad B.α_1+2α_2+α_3 \quad C.α_1-2α_2-α_3 \quad D.α_1-2α_2+α_3 \quad E. \quad F. \quad G. \quad H.
$$
$$
\begin{array}{l}设k_1α_1+k_2α_2+k_3α_3=α,\mathrm{则对矩阵}\left(α_1,\;α_2,\;α_3,\;α\right)\mathrm{施以初等行变换},得\\\begin{pmatrix}1&0&0&1\\0&1&0&2\\0&2&1&3\end{pmatrix}\;\rightarrow\begin{pmatrix}1&0&0&1\\0&1&0&2\\0&0&1&-1\end{pmatrix}\;,\;\;\\故α=α_1+2α_2-α_3.\end{array}
$$



$$
\mathrm{向量}β=\left(3,\;4\right)^T\mathrm{用向量}α_1=\left(1,\;2\right)^T,\;α_2=\left(-1,\;0\right)^T\mathrm{线性表示式为}\;\left(\right).
$$
$$
A.
β=2α_1-α_2 \quad B.β=4α_1+α_2 \quad C.β=α_1-2α_2 \quad D.β=5α_1+2α_2 \quad E. \quad F. \quad G. \quad H.
$$
$$
\begin{array}{l}\begin{pmatrix}1&-1&3\\2&0&4\end{pmatrix}\rightarrow\begin{pmatrix}1&-1&3\\0&2&-2\end{pmatrix}\rightarrow\begin{pmatrix}1&0&2\\0&1&-1\end{pmatrix}\\则β=2α_1-α_2\end{array}
$$



$$
设α_1=\left(1,\;2,\;1\right)^T,\;α_2=\left(2,\;9,\;0\right)^T,\;α_3=\left(3,\;3,\;4\right)^T,\;β=\left(5,\;-1,\;9\right)^T,用α_1,\;\;α_2,\;α_3\mathrm{的线性组合表示向量}β 为\left(\right).
$$
$$
A.
β=2α_1-3α_2+3α_3 \quad B.β=-2α_1-α_2+3α_3 \quad C.β=α_1-α_2+2α_3 \quad D.β=α_1-α_2+α_3 \quad E. \quad F. \quad G. \quad H.
$$
$$
\begin{array}{l}\mathrm{对矩阵}\left(α_1,\;α_2,\;α_3,\;β\right)\mathrm{施以初等行变换}:\\\;\;\begin{pmatrix}1&2&3&5\\2&9&3&-1\\1&0&4&9\end{pmatrix}\rightarrow\begin{pmatrix}1&0&4&9\\0&2&-1&-4\\0&1&-1&3\end{pmatrix}\rightarrow\begin{pmatrix}1&0&0&1\\0&1&0&-1\\0&0&1&2\end{pmatrix},\;\;\\\mathrm{由条件易知}:β=α_1-α_2+2α_3\end{array}
$$



$$
设α_1=\left(1,\;9,\;1\right)^T,\;α_2=\left(-1,\;3,\;1\right)^T,\;α_3=\left(1,\;1,\;1\right)^T,用α_1,α_2,α_3\mathrm{的线性组合来表示向量}β=\left(5,\;1,\;-1\right)^T为\left(\right).
$$
$$
A.
β=α_1-3α_2+α_3 \quad B.β=α_1-2α_2+2α_3 \quad C.β=2α_1-α_2+2α_3 \quad D.β=3α_1-α_2+α_3 \quad E. \quad F. \quad G. \quad H.
$$
$$
\begin{array}{l}\mathrm{将矩阵}\left(α_1,\;α_2,\;α_3,\;β\right)\mathrm{施以初等行变换}:\\\;\;\begin{pmatrix}1&-1&1&5\\9&3&1&1\\1&1&1&-1\end{pmatrix}\rightarrow\begin{pmatrix}1&-1&1&5\\0&-6&-8&10\\0&2&0&-6\end{pmatrix}\rightarrow\begin{pmatrix}1&0&0&1\\0&1&0&-3\\0&0&1&1\end{pmatrix},\;\;\\故β=α_1-3α_2+α_3\end{array}
$$



$$
设α_1=\left(-1,\;1,\;2\right)^T,\;α_2=\left(3,\;2,\;1\right)^T,\;α_3=\left(2,\;-3,\;-2\right)^T,\mathrm{并用}α_1,α_2,α_3\mathrm{的线性组合来表示向量}α=\left(1,\;-9,\;-3\right)^T为\left(\right).
$$
$$
A.
α=2α_1-α_2+α_3 \quad B.α=-2α_1-3α_2+4α_3 \quad C.α=-2α_1-α_2+α_3 \quad D.α=2α_1-α_2+3α_3 \quad E. \quad F. \quad G. \quad H.
$$
$$
\begin{array}{l}\mathrm{将矩阵}\left(α_1,\;α_2,\;α_3,\;α\right)\mathrm{施以初等行变换}:\\\;\;\begin{pmatrix}-1&3&2&1\\1&2&-3&-9\\2&1&-2&-3\end{pmatrix}\rightarrow\begin{pmatrix}-1&3&2&1\\0&1&10&29\\0&2&3&7\end{pmatrix}\rightarrow\begin{pmatrix}1&0&0&2\\0&1&0&-1\\0&0&1&3\end{pmatrix},\;\;\\故α=2α_1-α_2+3α_3\end{array}
$$



$$
设α_1=\left(-\frac35,\;\frac45,\;0\right),\;α_2=\left(\frac45,\;\frac35,\;0\right),\;α_3=\left(0,\;0,\;1\right),\;β=\left(3,\;-7,\;4\right),用α_1,\;\;α_2,\;α_3\mathrm{的线性组合表示向量}β 为\;\left(\right).
$$
$$
A.
β=-\frac{37}5α_1-\frac95α_2+4α_3 \quad B.β=-\frac{37}5α_1+\frac95α_2+4α_3 \quad C.β=\frac{37}5α_1-\frac95α_2+4α_3 \quad D.β=\frac{37}5α_1+\frac95α_2+4α_3 \quad E. \quad F. \quad G. \quad H.
$$
$$
\begin{array}{l}设β=x_1α_1+x_2α_2+x_3α_3,则\\\left\{\begin{array}{c}-\frac35x_1+\frac45x_2=3\\\begin{array}{l}\frac45x_1+\frac35x_2=-7\\x_3=4\end{array}\end{array}\right.⇒ x_1=-\frac{37}5,\;x_2=-\frac95,\;x_3=4\\故β=-\frac{37}5α_1-\frac95α_2+4α_3\end{array}
$$



$$
若γ=\left(0,\;k\right)^T\mathrm{能由}α=\left(1+k,\;1\right)^T,\;β=\left(1,\;1+k\right)^T\mathrm{唯一线性表示},则k\neq\left(\right).\;
$$
$$
A.
0与-2 \quad B.1 \quad C.-1 \quad D.0与-1 \quad E. \quad F. \quad G. \quad H.
$$
$$
\begin{vmatrix}1+k&1\\1&1+k\end{vmatrix}=k\left(k+2\right)\neq0,\;α,\;β\mathrm{线性无关},α,\;β,\;γ\mathrm{线性相关}.
$$



$$
\begin{array}{l}\mathrm{已知向量}α_1=\left(1,\;2,\;-1,\;-2\right)^T,\;α_2=\left(2,\;3,\;0,\;-1\right)^T,\;α_3=\left(1,\;2,\;1,\;4\right)^T,\;α_4=\left(1,\;3,\;-1,\;0\right)^T,\;α=\left(7,\;14,\;-1,\;2\right)^T,把α 表\\成α_1,\;α_2,\;α_3,\;α_4\mathrm{的线性组合为}\;\left(\right).\end{array}
$$
$$
A.
α=2α_1+α_3+2α_4 \quad B.α=α_1+2α_3+α_4 \quad C.α=2α_2+α_3+2α_4 \quad D.α=α_1+2α_2+α_3+α_4 \quad E. \quad F. \quad G. \quad H.
$$
$$
\begin{array}{l}即\;\left(α_1,\;α_2,\;α_3,\;α_4,\;α\right)\rightarrow\begin{pmatrix}1&2&1&1&7\\0&-1&0&1&0\\0&2&2&0&6\\0&3&6&2&16\end{pmatrix}\rightarrow\begin{pmatrix}1&0&0&0&0\\0&1&0&0&2\\0&0&1&0&1\\0&0&0&1&2\end{pmatrix}\\故α=2α_2+α_3+2α_4.\end{array}
$$



$$
\mathrm{向量}γ=\left(a,0\right)^T\mathrm{可以由}α=\left(a,1\right)^T和β=\left(0,a-2\right)^T\mathrm{唯一线性表示},则a\left(\;\;\;\;\;\;\;\right).
$$
$$
A.
α=0 \quad B.α=2 \quad C.α\neq0且α\neq2 \quad D.α=0或α=2 \quad E. \quad F. \quad G. \quad H.
$$
$$
\begin{vmatrix}a&0\\1&a-2\end{vmatrix}=a\left(a-2\right)\neq0,\;α,\;β\mathrm{线性无关},α,\;β,\;γ\mathrm{线性相关}.
$$



$$
设α_1=\left(2,\;1,\;0,\;3\right)^T,\;α_2=\left(3,\;-1,\;5,\;2\right)^T,\;α_3=\left(-1,\;0,\;2,\;1\right)^T,\;β=\left(2,\;3,\;-7,\;3\right)^T,\mathrm{向量}β\mathrm{可由}α_1,\;\;α_2,\;α_3\mathrm{线性表示为}\left(\right).
$$
$$
A.
β=2α_1-α_2-α_3 \quad B.β=2α_1-α_2+α_3 \quad C.β=2α_1+α_2-α_3 \quad D.β=2α_1+α_2+α_3 \quad E. \quad F. \quad G. \quad H.
$$
$$
\begin{array}{l}设c_1α_1+c_2α_2+c_3α_3=β,令A=(\begin{array}{c}α_1,\;α_2,\;α_3,\;β\end{array}),\mathrm{对矩阵}A\mathrm{作初等行变换}:\\A\rightarrow\begin{pmatrix}1&-1&0&3\\0&5&-2&7\\0&5&1&-6\\0&5&-1&-4\end{pmatrix}\rightarrow\begin{pmatrix}1&0&0&2\\0&1&0&-1\\0&0&1&-1\\0&0&0&0\end{pmatrix},\;\;\\故c_1=2,\;c_2=-1,\;c_3=-1.\end{array}
$$



$$
设α_1=\left(2,\;3,\;5\right)^T,\;α_2=\left(3,\;7,\;8\right)^T,\;α_3=\left(1,\;-6,\;1\right)^T,\;若β=\left(7,\;-2,\;λ\right)^T\mathrm{可由}α_1,\;\;α_2,\;α_3\mathrm{线性表示},则λ=\left(\right).
$$
$$
A.
15 \quad B.16 \quad C.14 \quad D.17 \quad E. \quad F. \quad G. \quad H.
$$
$$
\begin{array}{l}若β=\left(7,\;-2,\;λ\right)^T\mathrm{可由}α_1,\;\;α_2,\;α_3\mathrm{线性表示}\;,\mathrm{则向量组}A:α_1,α_2,α_3\mathrm{与向量组}B:α_1,α_2,α_3,\;β\mathrm{等价},\;\\\mathrm{所以}\;R(A)=R(B),\;即\\B=\begin{pmatrix}2&3&1&7\\3&7&-6&-2\\5&8&1&λ\end{pmatrix}\rightarrow\begin{pmatrix}2&3&1&7\\0&1&-3&-5\\0&0&0&λ-15\end{pmatrix},\\\mathrm{故当}λ=15时,\;R(A)=R(B),即\;β\mathrm{可由}α_1,\;\;α_2,\;α_3\mathrm{线性表示}.\end{array}
$$



$$
\mathrm{已知}α_1=\begin{pmatrix}2\\0\\0\end{pmatrix},\;α_2=\begin{pmatrix}0\\0\\-3\end{pmatrix},\mathrm{下列向量中是}α_1,\;α_2\mathrm{的线性组合的是}\left(\right).
$$
$$
A.
β=\begin{pmatrix}-3\\0\\4\end{pmatrix} \quad B.β=\begin{pmatrix}0\\1\\0\end{pmatrix} \quad C.β=\begin{pmatrix}1\\1\\0\end{pmatrix} \quad D.β=\begin{pmatrix}0\\-1\\1\end{pmatrix} \quad E. \quad F. \quad G. \quad H.
$$
$$
\begin{array}{l}\mathrm{由线性组合的定义可知},若β 是α_1,\;α_2\mathrm{的线性组合},则β=k_1α_1+k_2α_2,\\即\begin{pmatrix}2k_1\\0\\0\end{pmatrix}+\begin{pmatrix}0\\0\\-3k_2\end{pmatrix}=\begin{pmatrix}2k_1\\0\\-3k_2\end{pmatrix}=β,\mathrm{中间元素为}0\mathrm{的只有选项}A.\;\;\\\mathrm{本题亦可用代入法},\mathrm{将选项中的答案进行代入},\;\;\\\mathrm{可知}\begin{pmatrix}-3\\0\\4\end{pmatrix}=-\frac32\begin{pmatrix}2\\0\\0\end{pmatrix}-\frac43\begin{pmatrix}0\\0\\-3\end{pmatrix},\mathrm{其余选项都不能表示}.\end{array}
$$



$$
\begin{array}{l}设α_1=\left(-1,\;3,\;2,\;0\right)^T,\;α_2=\left(2,\;0,\;4,\;-1\right)^T,\;α_3=\left(7,\;1,\;1,4\right)^T,\;α_4=\left(6,\;3,\;1,\;2\right)^T,若\\c_1α_1+c_2α_2+c_3α_3+c_4α_4=\left(0,\;5,\;6,\;-3\right)^T,则c_1,\;c_2,\;c_3,\;c_4\mathrm{分别为}\;\left(\right).\end{array}
$$
$$
A.
c_1=1,\;c_2=1,\;c_3=-1,\;c_4=1 \quad B.c_1=1,\;c_2=2,\;c_3=3,\;c_4=4 \quad C.c_1=1,\;c_2=3,\;c_3=4,\;c_4=1 \quad D.c_1=2,\;c_2=1,\;c_3=4,\;c_4=3 \quad E. \quad F. \quad G. \quad H.
$$
$$
\begin{array}{l}令β=\left(0,\;5,\;6,\;-3\right),A=\left[α_1,\;α_2,\;α_3,\;α_4\right],对A\mathrm{作初等行变换}:\\A\rightarrow\begin{pmatrix}1&0&0&0&1\\0&1&0&0&1\\0&0&1&0&-1\\0&0&0&1&1\end{pmatrix},\\\mathrm{得唯一解}c_1=1,\;c_2=1,\;c_3=-1,\;c_4=1.\end{array}
$$



$$
\begin{array}{l}\mathrm{设有向量}\\\;α_1=\begin{pmatrix}1+λ\\1\\1\end{pmatrix}\;,\;α_2=\begin{pmatrix}1\\1+λ\\1\end{pmatrix},\;α_3=\begin{pmatrix}1\\1\\1+λ\end{pmatrix},\;β=\begin{pmatrix}0\\λ\\λ^2\end{pmatrix},\;\;\\\mathrm{则当}λ\mathrm{取何值时},β\mathrm{可由}α_1,\;α_2,\;α_3\mathrm{线性表示},\mathrm{且表达式唯一}\left(\right)\;.\end{array}
$$
$$
A.
λ=0 \quad B.λ=-3 \quad C.λ\neq-3且λ\neq0 \quad D.λ\neq0 \quad E. \quad F. \quad G. \quad H.
$$
$$
\begin{array}{l}\left|α_1α_2α_3\right|=λ^2\left(λ+3\right).\\\left(1\right)当λ\neq0且λ\neq-3时,β\mathrm{可由}α_1,\;α_2,\;α_3\mathrm{唯一地线性表示}.\;\;\\\left(2\right)当λ=0时,β\mathrm{可由}α_1,\;α_2,\;α_3\mathrm{线性表示},\mathrm{但表达式不唯一}.\;\;\\\left(3\right)当λ=-3时,β\mathrm{不能由}α_1,\;α_2,\;α_3\mathrm{线性表示}.\end{array}
$$



$$
\mathrm{设有向量}α_1=\begin{pmatrix}1\\0\\2\\3\end{pmatrix},\;α_2=\begin{pmatrix}1\\1\\3\\5\end{pmatrix},\;α_3=\begin{pmatrix}1\\-1\\a+2\\1\end{pmatrix},\;α_4=\begin{pmatrix}1\\2\\4\\a+8\end{pmatrix},\;β=\begin{pmatrix}1\\1\\b+3\\5\end{pmatrix}若β\mathrm{不能由}α_1,\;α_2,\;α_3,\;α_4\mathrm{线性表示},则a,b为\;\left(\right).
$$
$$
A.
a=-1,\;b\neq0 \quad B.a\neq-1,\;b=0 \quad C.a=1,\;b\neq0 \quad D.a\neq1,\;b=0 \quad E. \quad F. \quad G. \quad H.
$$
$$
\begin{array}{l}\mathrm{设线性方程组}x_1α_1+x_2α_2+x_3α_3+x_4α_4=β,\;\mathrm{于是有}\\\;\;\left(α_1α_2α_3α_4\vertβ\right)⇒\begin{pmatrix}1&1&1&1&\vert&1\\0&1&-1&2&\vert&1\\0&0&a+1&0&\vert&b\\0&0&0&a+1&\vert&0\end{pmatrix},\;\;\\当a=-1且b\neq0时,β\mathrm{不能由}α_1,\;α_2,\;α_3,\;α_4\mathrm{线性表示}.\end{array}
$$



$$
\mathrm{设有向量}α_1=\begin{pmatrix}1\\0\\2\\3\end{pmatrix},\;α_2=\begin{pmatrix}1\\1\\3\\5\end{pmatrix},\;α_3=\begin{pmatrix}1\\-1\\a+2\\1\end{pmatrix},\;α_4=\begin{pmatrix}1\\2\\4\\a+8\end{pmatrix},\;β=\begin{pmatrix}1\\1\\b+3\\5\end{pmatrix}若β\mathrm{能由}α_1,\;α_2,\;α_3,\;α_4\mathrm{线性表示},则a,b为\;\left(\right).
$$
$$
A.
a=-1,\;b\neq0 \quad B.a\neq1,\;b=0 \quad C.b\neq0 \quad D.a\neq-1 \quad E. \quad F. \quad G. \quad H.
$$
$$
\begin{array}{l}\begin{array}{l}解1:\mathrm{因为}\beta\mathrm{能由}α_1,\;\alpha_2,\;α_3,\;α_4\mathrm{唯一线性表示},\mathrm{所以}α_1,\;\alpha_2,\;α_3,\;α_4\mathrm{线性无关},故\\\;\;\left|α_1\alpha_2α_3α_4\right|⇒\begin{vmatrix}1&1&1&1\\0&1&-1&2\\0&0&a+1&0\\0&0&0&a+1\end{vmatrix}=\left(a+1\right)^2\neq0,\;\;\\即a\neq-1时,b\mathrm{可取任意值}.\end{array}\\\begin{array}{l}解2:\mathrm{设线性方程组}x_1α_1+x_2α_2+x_3α_3+x_4α_4=\beta,\;\mathrm{于是有}\\\;\;\left(α_1α_2α_3α_4\vertβ\right)⇒\begin{pmatrix}1&1&1&1&\vert&1\\0&1&-1&2&\vert&1\\0&0&a+1&0&\vert&b\\0&0&0&a+1&\vert&0\end{pmatrix},\;\;\\当a\neq-1时,\mathrm{表达式唯一}.\end{array}\\\\\end{array}
$$



$$
\mathrm{向量}β_1=\left(4,\;3,\;-1,\;11\right)^T\mathrm{表示成向量组}α_1=\left(1,\;2,\;-1,\;5\right)^T,α_2=\left(2,\;-1,\;1,\;1\right)^T\mathrm{的线性组合的表达式为}\left(\right).
$$
$$
A.
β_1=2α_1+α_2 \quad B.β_1=-2\alpha_1+3α_2 \quad C.β_1=α_1+\frac32α_2 \quad D.β_1=-α_1+\frac52α_2 \quad E. \quad F. \quad G. \quad H.
$$
$$
\begin{array}{l}设k_1α_1+k_2α_2=β_1,\mathrm{对矩阵}\left(α_1,\;α_2,\;β_1\right)\mathrm{施以初等行变换}:\\\begin{pmatrix}1&2&4\\2&-1&3\\-1&1&-1\\5&1&11\end{pmatrix}\;\rightarrow\begin{pmatrix}1&2&4\\0&-5&-5\\0&3&3\\0&-9&-9\end{pmatrix}\rightarrow\begin{pmatrix}1&2&4\\0&1&1\\0&0&0\\0&0&0\end{pmatrix}\rightarrow\begin{pmatrix}1&0&2\\0&1&1\\0&0&0\\0&0&0\end{pmatrix},\\\mathrm{由上面的初等变换可取}k_1=2,\;k_2=1使β_1=2α_1+α_2.\end{array}
$$



$$
\begin{array}{l}\mathrm{下列命题中的向量}β\mathrm{能由其余向量线性表示的有}\left(\right)个.\;\\\left(1\right)α_1=\left(1,\;2\right)^T,\;α_2=\left(-1,\;0\right)^T,\;β=\left(3,\;4\right)^T;\\\left(2\right)\;α_1=\left(1,\;0,\;2\right)^T,\;α_2=\left(2,\;-8,\;0\right)^T,\;β=\left(1,\;2,\;-1\right)^T;\\\left(3\right)\;β=\left(2,\;-1,\;5,\;1\right)^T,\;ε_1=\left(1,\;0,\;0,\;0\right)^T,\;ε_2=\left(0,\;1,\;0,\;0\right)^T,\;ε_3=\left(0,\;0,\;1,\;0\right),\;ε_4=\left(0,\;0,\;0,\;1\right)^T.\end{array}
$$
$$
A.
0个 \quad B.1个 \quad C.2个 \quad D.3个 \quad E. \quad F. \quad G. \quad H.
$$
$$
\begin{array}{l}\left(1\right)\mathrm{矩阵}A=\left(α_1,\;α_2\right)\mathrm{与矩阵}B=\left(α_1,\;α_2,\;β\right)\mathrm{的秩相等},\mathrm{都为}2,\mathrm{因此}β\mathrm{可由其余向量线性表示}:\\β=2\;α_1-α_2;\;\\\left(2\right)\mathrm{矩阵}A=\left(α_1,\;α_2\right)\mathrm{的秩为}2,\mathrm{矩阵}B=\left(α_1,\;α_2,\;β\right)\mathrm{的秩为}3,\mathrm{所以}β\mathrm{不能由其余向量线性表示};\;\;\\\left(3\right)\mathrm{由于}ε_1,ε_2,ε_3,ε_4\mathrm{为单位向量},\mathrm{所以}β=2ε_1-ε_2+5ε_3+ε_4\end{array}
$$



$$
\begin{array}{l}\mathrm{已知}α_1=\left(\begin{array}{c}\frac1{\sqrt3}\end{array},\;\frac1{\sqrt3},\;\frac1{\sqrt3}\right)^T,\;α_2=\left(\begin{array}{c}-\frac1{\sqrt2}\end{array},\;\frac1{\sqrt2},\;0\right)^T,\;α_3=\left(\begin{array}{c}-\frac1{\sqrt6}\end{array},\;-\frac1{\sqrt6},\;\frac2{\sqrt6}\right)^T,\\\mathrm{若用此向量组来线性表示向量}α=\left(1,\;-1,\;-1\right)^T,则\left(\right).\;\end{array}
$$
$$
A.
α=-\frac1{\sqrt3}α_1-\sqrt2α_2-\frac2{\sqrt6}α_3 \quad B.α=-\frac1{\sqrt3}α_1-\sqrt2α_2+\frac2{\sqrt6}α_3 \quad C.α=-\frac1{\sqrt3}α_1+\sqrt2α_2-\frac2{\sqrt6}α_3 \quad D.α=\frac1{\sqrt3}α_1-\sqrt2α_2-\frac2{\sqrt6}α_3 \quad E. \quad F. \quad G. \quad H.
$$
$$
\begin{array}{l}设k_1α_1+k_2α_2+k_3α_3=α,\mathrm{则对矩阵}\left(α_1,\;α_2,\;α_3,\;α\right)\mathrm{进行初等行变换},得\\\begin{pmatrix}\frac1{\sqrt3}&-\frac1{\sqrt2}&-\frac1{\sqrt6}&1\\\frac1{\sqrt3}&\frac1{\sqrt2}&-\frac1{\sqrt6}&-1\\\frac1{\sqrt3}&0&\frac2{\sqrt6}&-1\end{pmatrix}\rightarrow\begin{pmatrix}\frac1{\sqrt3}&-\frac1{\sqrt2}&-\frac1{\sqrt6}&1\\0&\frac2{\sqrt2}&0&-2\\0&\frac1{\sqrt2}&\frac3{\sqrt6}&-2\end{pmatrix}\\\rightarrow\begin{pmatrix}\frac1{\sqrt3}&0&0&-\frac13\\0&\frac1{\sqrt2}&0&-1\\0&0&\frac3{\sqrt6}&-1\end{pmatrix}\rightarrow\begin{pmatrix}1&0&0&-\frac1{\sqrt3}\\0&1&0&-\sqrt2\\0&0&1&-\frac2{\sqrt6}\end{pmatrix},\\\mathrm{则由最后的矩阵可知}\alpha=-\frac1{\sqrt3}α_1-\sqrt2α_2-\frac2{\sqrt6}α_3.\end{array}
$$



$$
\mathrm{向量}α=\left(2,\;1,\;3\right)^T\mathrm{用向量}α_1=\left(0,\;\frac1{\sqrt2},\;\frac1{\sqrt2}\right)^T,\;α_2=\left(1,\;0,\;0\right)^T,\;α_3=\left(0,\;\frac1{\sqrt2},\;\frac{-1}{\sqrt2}\right)^T,\;\mathrm{线性表示为}\left(\right).
$$
$$
A.
α=2\sqrt2α_1+2α_2-\sqrt2α_3 \quad B.α=2\sqrt2α_1+2α_2+\sqrt2α_3 \quad C.α=\sqrt2α_1+2α_2-2\sqrt2α_3 \quad D.α=\sqrt2α_1+2α_2+2\sqrt2α_3 \quad E. \quad F. \quad G. \quad H.
$$
$$
\begin{array}{l}解1:设k_1α_1+k_2α_2+k_3α_3=α,\mathrm{则对矩阵}\left(α_1,\;α_2,\;\alpha_3,\;α\right)\mathrm{进行初等行变换}:\\\begin{pmatrix}0&1&0&2\\\frac1{\sqrt2}&0&\frac1{\sqrt2}&1\\\frac1{\sqrt2}&0&\frac{-1}{\sqrt2}&3\end{pmatrix}\rightarrow\begin{pmatrix}0&1&0&2\\\frac1{\sqrt2}&0&\frac1{\sqrt2}&1\\0&0&-\sqrt2&2\end{pmatrix}\rightarrow\begin{pmatrix}1&0&1&\sqrt2\\0&1&0&2\\0&0&-\sqrt2&2\end{pmatrix}\rightarrow\begin{pmatrix}1&0&0&2\sqrt2\\0&1&0&2\\0&0&1&-\sqrt2\end{pmatrix}\\\mathrm{由最后一个矩阵可知},α=2\sqrt2α_1+2α_2-\sqrt2α_3.\\解2:\mathrm{设矩阵}A=\left(α_1,\;α_2,\;α_3\right),k_1α_1+k_2α_2+k_3α_3=α,即A\begin{pmatrix}k_1\\k_2\\k_3\end{pmatrix}=α,\;\mathrm{因为矩阵}A\mathrm{为正交矩阵},\mathrm{所以}A^TA=E.\\故\begin{pmatrix}k_1\\k_2\\k_3\end{pmatrix}=A^Tα=\begin{pmatrix}0&\frac1{\sqrt2}&\frac1{\sqrt2}\\1&0&0\\0&\frac1{\sqrt2}&-\frac1{\sqrt2}\end{pmatrix}\begin{pmatrix}2\\1\\3\end{pmatrix}=\begin{pmatrix}2\sqrt2\\2\\-\sqrt2\end{pmatrix},\\\mathrm{所以}α=2\sqrt2α_1+2α_2-\sqrt2α_3.\end{array}
$$



$$
\begin{array}{l}β=\left(1,\;2,\;1,\;1\right)^T,\;α_1=\left(1,\;1,\;1,\;1\right)^T,\;α_2=\left(1,\;1,\;-1,\;-1\right)^T,\;α_3=\left(1,\;-1,\;1,\;-1\right)^T,\;α_4=\left(1,\;-1,\;-1,\;1\right)^T,\mathrm{将向量}β\mathrm{表成向}\\量α_1,\;α_2,\;α_3,\;α_4\mathrm{的线性组合为}\left(\right).\end{array}
$$
$$
A.
β=\frac54α_1+\frac14α_2-\frac14α_3-\frac14α_4 \quad B.β=\frac54α_1+\frac14α_2+\frac14α_3-\frac14α_4 \quad C.β=\frac54α_1+\frac14α_2-\frac14α_3+\frac14\alpha_4 \quad D.β=\frac54α_1-\frac14α_2-\frac14\alpha_3+\frac14α_4 \quad E. \quad F. \quad G. \quad H.
$$
$$
\begin{array}{l}设β=k_1α_1+k_2α_2+k_3α_3+k_4α_4,即\\\left\{\begin{array}{c}k_1+k_2+k_3+k_4=1,\\k_1+k_2-k_3-k_4=2,\\k_1-k_2+k_3-k_4=1,\\k_1-k_2-k_3+k_4=1.\end{array}\right.\\\mathrm{解得}k_1=\frac54,\;k_2=\frac14,\;k_3=-\frac14,\;k_4=-\frac14,\\故β=\frac54α_1+\frac14α_2-\frac14α_3-\frac14α_4\end{array}
$$



$$
\begin{array}{l}设α_1=\left(1,\;0,\;0,\;3\right)^T,\;α_2=\left(1,\;1,\;-1,\;2\right)^T,\;α_3=\left(1,\;2,\;a-3,\;1\right)^T,\;α_4=\left(1,\;2,\;-2,\;a\right)^T,\;β=\left(0,\;1,\;b,\;-1\right)^T,若β\mathrm{能由}\\α_1,\;α_2,\;α_3,\;α_4\mathrm{线性表示且表示式惟一},则a,b\mathrm{的值为}(\;).\end{array}
$$
$$
A.
a\neq1 \quad B.a=1 \quad C.a\neq-1,\;b\neq-1 \quad D.a=-1,\;b=-1 \quad E. \quad F. \quad G. \quad H.
$$
$$
\begin{array}{l}解1:\mathrm{因为}β\mathrm{能由}α_1,\;α_2,\;α_3,\;α_4\mathrm{唯一线性表示},\mathrm{所以}α_1,\;α_2,\;α_3,\;α_4\mathrm{线性无关},故\\\left|α_1α_2α_3α_4\right|=\begin{vmatrix}1&0&-1&-1\\0&1&2&2\\0&0&a-1&0\\0&0&0&a-1\end{vmatrix}=\left(a-1\right)^2\neq0,\\即a\neq1,\;b\mathrm{可取任意值}.\\解2:\mathrm{对向量组}\left(α_1,\;α_2,\;α_3,\;α_4,\;β\right)\mathrm{实施初等行变换}:\\A=\begin{pmatrix}1&1&1&1&0\\0&1&2&2&1\\0&-1&a-3&-2&b\\3&2&1&a&-1\end{pmatrix}\rightarrow\begin{pmatrix}1&0&-1&-1&-1\\0&1&2&2&1\\0&0&a-1&0&b+1\\0&0&0&a-1&0\end{pmatrix}=B,\\当a\neq1,\;b∈ R时,R\left(A\right)=R\left(B\right)=4,\mathrm{此时}β\mathrm{可由}α_1,\;α_2,\;α_3,\;α_4\mathrm{线性表示且表达式惟一}.\end{array}
$$



$$
\mathrm{已知向量}β=\left(-1,\;2,\;μ\right)^T\mathrm{可由}α_1=\left(1,\;-1,\;2\right)^T,\;α_2=\left(0,\;1,\;-1\right)^T,\;α_3=\left(2,\;-3,\;λ\right)^T\mathrm{唯一地线性表示},则\left(\right).
$$
$$
A.
λ\neq5 \quad B.λ=5 \quad C.μ=λ \quad D.μ\neqλ \quad E. \quad F. \quad G. \quad H.
$$
$$
\begin{array}{l}\mathrm{因为}β\mathrm{可由}α_1,\;α_2,\;α_3\mathrm{唯一地线性表示},\mathrm{所以}α_1,\;α_2,\;α_3\mathrm{必线性无关},即\\\mathrm{必须有}λ-5\neq0,即λ\neq5,α_1,\;α_2,\;α_3\mathrm{线性无关}.\end{array}
$$



$$
\begin{array}{l}\mathrm{已知向量}α_1=\left(1,\;-1,\;2,\;3\right)^T,\;α_2=\left(0,\;2,\;5,\;8\right)^T,\;α_3=\left(2,\;2,\;0,\;-1\right)^T,\;α=\left(-1,\;7,\;-1,\;-2\right)^T,\\把α\mathrm{表成}α_1,\;\;α_2,\;α_3\mathrm{的线性组合为}\;\left(\right).\end{array}
$$
$$
A.
α=-3α_1+α_2+α_3 \quad B.α=3α_1+α_2+α_3 \quad C.α=-3α_1-α_2+α_3 \quad D.α=3α_1+α_2-α_3 \quad E. \quad F. \quad G. \quad H.
$$
$$
\begin{array}{l}\left(α_1,\;\;α_2,\;α_3,\;α\right)\rightarrow\begin{pmatrix}1&0&2&-1\\0&2&4&6\\0&5&-4&1\\0&8&-7&1\end{pmatrix}\rightarrow\begin{pmatrix}1&0&2&-1\\0&1&2&3\\0&0&1&1\\0&0&0&0\end{pmatrix}\rightarrow\begin{pmatrix}1&0&0&-3\\0&1&0&1\\0&0&1&1\\0&0&0&0\end{pmatrix}\\⇒α=-3α_1+α_2+α_3\end{array}
$$



$$
α,\;β,\;γ\mathrm{为某个向量空间的向量},k,\;m,\;l\mathrm{为实数},km\neq0,且kα+lβ+mγ=0,\mathrm{则有}\left(\right).
$$
$$
A.
α,\;β 与α,\;γ\mathrm{等价} \quad B.α,\;β 与β,\;γ\mathrm{等价} \quad C.α,\;γ 与β,\;γ\mathrm{等价} \quad D.β 与γ\mathrm{等价} \quad E. \quad F. \quad G. \quad H.
$$
$$
\begin{array}{l}由km\neq0⇒ k\neq0,\;m\neq0,\mathrm{由条件可知}:α=-\frac lkβ-\frac mkγ,γ=-\frac kmα-\frac lmβ,\\\;\;\mathrm{因此}α,\;β\mathrm{可由}β,\;γ\mathrm{线性表出},β,\;γ\mathrm{可由}α,\;β\mathrm{线性表出},α,\;β 与β,\;γ\mathrm{可相互线性表出},\mathrm{即等价}.\;\;\\\mathrm{又由于}l\mathrm{不确定是否非零},\mathrm{因此}β\mathrm{不可由}α,\;γ\mathrm{线性表出}.\mathrm{故选项中只有}α,\;β 与β,\;γ\mathrm{等价}.\end{array}
$$



$$
\begin{array}{l}\mathrm{已知向量}γ_1,\;γ_2\mathrm{由向量}β_1,\;β_2,\;β_3\mathrm{线性表示的表示式为}γ_1=3β_1-β_2+β_3,γ_2=β_1+2β_2+4β_3,\\\;\mathrm{向量}β_1,\;β_2,\;β_3\mathrm{由向量}α_1,\;α_2,\;α_3\mathrm{线性表示的表示式为}\\β_1\;=2α_1+α_2-5α_3,\;β_2\;=α_1+3α_2+α_3,\;\;β_3\;=-α_1+4\alpha_2-α_3,\;\\\mathrm{则向量由}γ_1,\;γ_2\mathrm{向量}α_1,\;α_2,\;α_3\mathrm{的线性表示式为}\left(\right).\end{array}
$$
$$
A.
\left(γ_1,\;γ_2\right)=\left(α_1α_2α_3\right)\begin{pmatrix}4&0\\4&23\\-17&-7\end{pmatrix}; \quad B.\left(γ_1,\;γ_2\right)=\left(α_1α_2α_3\right)\begin{pmatrix}4&0\\4&23\\17&7\end{pmatrix}; \quad C.\left(γ_1,\;γ_2\right)=\left(α_1α_2α_3\right)\begin{pmatrix}4&0\\-4&23\\17&-7\end{pmatrix}; \quad D.\left(γ_1,\;γ_2\right)=\left(α_1α_2α_3\right)\begin{pmatrix}4&0\\4&-23\\-17&7\end{pmatrix}. \quad E. \quad F. \quad G. \quad H.
$$
$$
\begin{array}{l}∵\begin{pmatrix}γ_1&γ_2\end{pmatrix}=\begin{pmatrix}β_1&β_2&β_3\end{pmatrix}\begin{pmatrix}3&1\\-1&2\\1&4\end{pmatrix},\;\begin{pmatrix}β_1&β_2&β_3\end{pmatrix}=\begin{pmatrix}α_1&α_2&α_3\end{pmatrix}\begin{pmatrix}2&1&-1\\1&3&4\\-5&2&-1\end{pmatrix}\\∴\begin{pmatrix}γ_1&γ_2\end{pmatrix}=\begin{pmatrix}α_1&α_2&α_3\end{pmatrix}\begin{pmatrix}2&1&-1\\1&3&4\\-5&2&-1\end{pmatrix}\begin{pmatrix}3&1\\-1&2\\1&4\end{pmatrix}=\begin{pmatrix}α_1&α_2&α_3\end{pmatrix}\begin{pmatrix}4&0\\4&23\\-17&-7\end{pmatrix}.\end{array}
$$



$$
\begin{array}{l}\mathrm{已知向量组}B:β_1,\;β_2,\;β_3\mathrm{由向量组}A:α_1,\;α_2,\;α_3\mathrm{线性表示的表示式为}\;\;\;\;\;\;\;\;\;\;\;\\β_1=α_1-α_2+α_3,β_2=α_1+α_2-\alpha_3,β_3=-α_1+\alpha_2+α_3\\\mathrm{则将向量组}A\mathrm{的向量用向量}B\mathrm{的向量可线性表示为}\left(\right).\end{array}
$$
$$
A.
α_1=\frac12\left(β_1+β_2\right),α_2=\frac12\left(β_2+β_3\right),α_3=\frac12\left(β_1+β_3\right) \quad B.α_1=\frac12\left(β_1+β_2\right),α_2=\frac32\left(β_2+β_3\right),α_3=\frac12\left(β_1+β_3\right) \quad C.α_1=\frac12\left(β_1+β_2\right),α_2=\frac12\left(β_2+β_3\right),α_3=\frac32\left(β_1+β_3\right) \quad D.α_1=\frac32\left(β_1+β_2\right),α_2=\frac12\left(β_2+β_3\right),α_3=\frac12\left(β_1+β_3\right) \quad E. \quad F. \quad G. \quad H.
$$
$$
\begin{array}{l}\mathrm{类似于解方程组},把α_1,\;α_2,\;α_3\mathrm{看成未知量},\mathrm{利用消元法求解}.\\β_1=α_1-α_2+α_3\;\;\;\;\;\;\;\;①\\β_2=\alpha_1+α_2-α_3\;\;\;\;\;\;\;\;②\\β_3=-α_1+α_2+α_3\;\;\;\;\;③\\①+②有\;\;\;\;\;\;2α_1=β_1+β_2,\\①+③有\;\;\;\;\;\;2α_3=β_1+β_3,\\②+③有\;\;\;\;\;\;2α_3=β_2+β_3,\\\mathrm{所以}α_1=\frac12\left(β_1+β_2\right),α_2=\frac12\left(β_2+β_3\right),α_3=\frac12\left(β_1+β_3\right)\\\mathrm{或由}\begin{pmatrix}β_1\\β_2\\β_3\end{pmatrix}=A\begin{pmatrix}α_1\\α_2\\α_3\end{pmatrix},\;A=\begin{pmatrix}1&-1&1\\1&1&-1\\-1&1&1\end{pmatrix}⇒\begin{pmatrix}α_1\\α_2\\α_3\end{pmatrix}=A^{-1}\begin{pmatrix}β_1\\β_2\\β_3\end{pmatrix}.\end{array}
$$



$$
\begin{array}{l}\mathrm{设有向量}\\α_1=\begin{pmatrix}1\\1\\0\end{pmatrix},\;α_2=\begin{pmatrix}5\\3\\2\end{pmatrix},\;α_3=\begin{pmatrix}1\\3\\-1\end{pmatrix},\;α_4=\begin{pmatrix}-2\\2\\-3\end{pmatrix},\\A\mathrm{是三阶矩阵},\mathrm{且有}Aα_1=α_2,\;Aα_2=α_3,\;Aα_3=α_4,\;则Aα_4=\left(\right).\end{array}
$$
$$
A.
\begin{pmatrix}7\\5\\2\end{pmatrix} \quad B.\begin{pmatrix}7\\2\\5\end{pmatrix} \quad C.\begin{pmatrix}7\\5\\-2\end{pmatrix} \quad D.\begin{pmatrix}7\\-2\\5\end{pmatrix} \quad E. \quad F. \quad G. \quad H.
$$
$$
\begin{array}{l}\mathrm{由于}\left(α_1,\;α_2,\;\alpha_3,\;α_4\right)\rightarrow\begin{pmatrix}1&0&0&2\\0&1&0&-1\\0&0&1&1\end{pmatrix},\\\mathrm{则有}α_4=2α_1-\;α_2+α_3,\;\mathrm{于是}\\Aα_4=A\left(2α_1-\;α_2+α_3\right)=2Aα_1-Aα_2+Aα_3\\\;\;\;\;\;\;=2α_2-α_3+α_4=\begin{pmatrix}7\\5\\2\end{pmatrix}.\end{array}
$$



$$
\begin{array}{l}\mathrm{设向量}β\mathrm{可由向量组}α_1,\;α_2,\;⋯,\;α_m\mathrm{线性表示},\mathrm{但不能由向量组}\left(\text{I}\right)α_1,\;α_2,\;⋯,\;\alpha_{m-1}\mathrm{线性表示},\mathrm{记向量组}\\\left(\text{II}\right)α_1,\;α_2,\;⋯,\;α_{m-1},\;β,\mathrm{则正确说法为}\left(\right).\end{array}
$$
$$
A.
α_m\mathrm{不能由向量组}\left(\text{I}\right)\mathrm{线性表示},\mathrm{也不能由}\left(\text{II}\right)\mathrm{线性表示} \quad B.α_m\mathrm{不能由向量组}\left(\text{I}\right)\mathrm{线性表示},\mathrm{但可由}\left(\text{II}\right)\mathrm{线性表示} \quad C.α_m\mathrm{可由向量组}\left(\text{I}\right)\mathrm{线性表示},\mathrm{也可由}\left(\text{II}\right)\mathrm{线性表示} \quad D.α_m\mathrm{可由向量组}\left(\text{I}\right)\mathrm{线性表示},\mathrm{但不可由}\left(\text{II}\right)\mathrm{线性表示} \quad E. \quad F. \quad G. \quad H.
$$
$$
\begin{array}{l}由β\mathrm{可由向量组}α_1,\;α_2,\;⋯,\;α_m\mathrm{线性表示可得}β=k_1α_1+k_2α_2+⋯+k_mα_m,\;\mathrm{显然}k_m\neq0,\;\mathrm{否则}\\β=k_1α_1+k_2α_2+⋯+k_{m-1}α_{m-1},与β\mathrm{不能由向量组}α_1,\;α_2,\;⋯,\;α_{m-1}\mathrm{线性表示矛盾},\mathrm{因此}\\α_m=-\frac{k_1}{k_m}α_1-⋯-\frac{k_{m-1}}{k_m}α_{m-1}+\frac1{k_m}β.\\即α_m\mathrm{可由向量组}\left(\text{II}\right)\mathrm{线性表出};\\\mathrm{下面用反证法可证}α_m\mathrm{不能由向量组}\left(\text{I}\right)\mathrm{线性表出}:\\若α_m\mathrm{可由向量组}\left(\text{I}\right)\mathrm{线性表出},即α_m=k_1α_1+k_2α_2+⋯+k_{m-1}α_{m-1},\;\mathrm{代入}\\β=k_1α_1+k_2α_2+⋯+k_mα_m\mathrm{可得}β\mathrm{由向量组}α_1,\;α_2,\;⋯,\;α_{m-1}\mathrm{线性表出的表达式},\mathrm{与题意矛盾}.\\\mathrm{因此}α_m\mathrm{可由向量组}\left(\text{II}\right)\mathrm{线性表出},\mathrm{但不可由向量组}\left(\text{I}\right)\mathrm{线性表出}.\end{array}
$$



$$
\begin{array}{l}\mathrm{设向量组}α_1=\left(a,\;2,\;10\right)^T,\;α_2=\left(-2,\;1,\;5\right)^T,\;α_3=\left(-1,\;1,\;4\right)^T,\;β=\left(1,\;b,\;c\right)^T,\;若β\mathrm{可由}α_1,\;α_2,\;α_3\mathrm{线性表示},则a,\;b,\;c\\\mathrm{满足的条件为}\left(\right).\end{array}
$$
$$
A.
a\neq-4 \quad B.a=-4 \quad C.a\neq-4且3b-c=1 \quad D.a=-4或3b-c=1 \quad E. \quad F. \quad G. \quad H.
$$
$$
\begin{array}{l}\left[α_1,\;α_2,\;α_3,\;β\right]=\begin{pmatrix}a&-2&-1&1\\2&1&1&b\\10&5&4&c\end{pmatrix}\xrightarrow{r_1\rightarrow r_2}\begin{pmatrix}2&1&1&b\\a&-2&-1&1\\10&5&4&c\end{pmatrix}\\\xrightarrow[{r_3-5r_{}}]{r_2-\frac a2r_1}\begin{pmatrix}2&1&1&b\\0&-2-\frac a2&-1-\frac a2&1-\frac{ab}2\\0&0&-1&c-5b\end{pmatrix}\\当-2-\frac a2\neq0,即a\neq-4时,β\mathrm{可由}α_1,\;α_2,\;α_3\mathrm{线性表示},\mathrm{且表示式惟一}.\end{array}
$$



$$
设A=(α,\;γ_1,\;γ_2),\;B=(β,\;γ_1,\;γ_2)\mathrm{均是三阶方阵},α,\;β,\;γ_1,\;γ_2\mathrm{是三维列向量},若\left|A\right|=2,\left|B\right|=3,则\left|A+B\right|=(\;).
$$
$$
A.
5 \quad B.10 \quad C.20 \quad D.40 \quad E. \quad F. \quad G. \quad H.
$$
$$
\left|A+B\right|=\left|α+β,2γ_1,2γ_2\right|=4\left|α,γ_1,γ_2\right|+4\left|β,γ_1,γ_2\right|=4×2+4×3=20
$$



$$
设A=(α,\;γ_1,\;γ_2),\;B=(β,\;γ_1,\;γ_2)\mathrm{均是三阶方阵},α,\;β,\;γ_1,\;γ_2\mathrm{是三维列向量},若\left|A\right|=2,\left|B\right|=3,则\left|A+2B\right|=(\;).
$$
$$
A.
6 \quad B.8 \quad C.54 \quad D.72 \quad E. \quad F. \quad G. \quad H.
$$
$$
\left|A+2B\right|=\left|α+2β,3γ_1,3γ_2\right|=9(\left|α,γ_1,γ_2\right|+2\left|β,γ_1,γ_2\right|)=9×8=72
$$



$$
设α=\left(2,\;0,\;-1,\;3\right)^T,β=\left(1,\;7,\;4,\;-2\right)^T,γ=\left(0,\;1,\;0,\;1\right)^T.\;则2\alpha+β-3γ=\left(\right).\;
$$
$$
A.
\left(5,\;4,\;2,\;1\right)^T \quad B.\left(5,\;3,\;0,\;2\right)^T \quad C.\left(5,\;4,\;0,\;1\right)^T \quad D.\left(5,\;4,\;2,\;0\right)^T \quad E. \quad F. \quad G. \quad H.
$$
$$
2α+β-3γ=2\left(2,\;0,\;-1,\;3\right)^T+\left(1,\;7,\;4,\;-2\right)^T-3\left(0,\;1,\;0,\;1\right)^T=\left(5,\;4,\;2,\;1\right)^T.
$$



$$
设\;ν_1=\left(1,\;1,\;0\right)^T,\;ν_2=\left(0,\;1,\;1\right)^T,\;ν_3=\left(3,\;4,\;0\right)^T,\;则3ν_1+2ν_2-ν_3=\left(\right).
$$
$$
A.
\left(0,\;1,\;2\right)^T \quad B.\left(0,\;2,\;1\right)^T \quad C.\left(1,\;0,\;2\right)^T \quad D.\left(1,\;2,\;0\right)^T \quad E. \quad F. \quad G. \quad H.
$$
$$
\begin{array}{l}3ν_1+2ν_2-ν_3=3\left(1,\;1,\;0\right)^T+2\left(0,\;1,\;1\right)^T-\left(3,\;4,\;0\right)^T\\\;\;\;\;\;\;\;\;\;\;\;\;\;\;\;\;\;\;\;\;\;\;\;=\left(3×1+2×0-3,\;3×1+2×1-4,\;3×0+2×1-0\right)^T\\\;\;\;\;\;\;\;\;\;\;\;\;\;\;\;\;\;\;\;\;\;\;\;\;=\left(0,\;1,\;2\right)^T\end{array}
$$



$$
\begin{array}{l}设3\left(α_1-α\right)+2(α_2+α)=5\left(α_3+α\right),\mathrm{其中}α_1=\left(2,\;5,\;1,\;3\right)^T,\;α_2=\left(10,\;1,\;5,\;10\right)^T,\;\\α_3=\left(4,\;1,\;-1,\;1\right)^T,则α=\left(\right).\end{array}
$$
$$
A.
\left(1,\;2,\;3,\;4\right)^T \quad B.\left(2,\;3,\;4,\;5\right)^T \quad C.\left(1,\;3,\;5,\;6\right)^T \quad D.\left(5,\;3,\;4,\;1\right)^T \quad E. \quad F. \quad G. \quad H.
$$
$$
\begin{array}{l}由3α_1-3α+2α_2+2α=5α_3+5α 得,α=\frac16\left(3α_1+2α_2-5α_3\right)=\frac16\left(6,\;12,\;18,\;24\right)^T=\left(1,\;2,\;3,\;4\right)^T.\\\end{array}
$$



$$
\mathrm{已知向量}α_1=\;\left(1,\;2,\;3\right)^T,\;α_2=\;\left(3,\;2,\;1\right)^T,\;则3α_1+2α_2=\left(\right)
$$
$$
A.
\left(9,\;10,\;11\right)^T \quad B.\left(9,\;12,\;11\right)^T \quad C.\left(10,\;9,\;11\right)^T \quad D.\left(10,\;12,\;11\right)^T \quad E. \quad F. \quad G. \quad H.
$$
$$
3α_1+2α_2=3\;\left(1,\;2,\;3\right)^T+2\left(3,\;2,\;1\right)^T=\left(3,\;6,\;9\right)^T+\left(6,\;4,\;2\right)^T=\left(9,\;10,\;11\right)^T.
$$



$$
\begin{array}{l}设α_1,α_2,α_3,β_1,β_2\;\mathrm{都是}4\mathrm{维列向量},且4\mathrm{阶行列式}\left|α_1,α_2,α_3,β_1\right|=m,\;\left|α_1,α_2,α_3,β_2\right|=n,则4\mathrm{阶行列式}\\\left|α_1,α_2,α_3,(β_1+β_2)\right|=\left(\right).\end{array}
$$
$$
A.
m+n \quad B.m-n \quad C.n-m \quad D.-m-n \quad E. \quad F. \quad G. \quad H.
$$
$$
\left|α_1,α_2,α_3,(β_1+β_2)\right|=\left|α_1,α_2,α_3,β_1\right|+\left|α_1,α_2,α_3,β_2\right|=m+n
$$



$$
\begin{array}{l}设α_1,α_2,α_3,β_1,β_2\;\mathrm{都是}4\mathrm{维列向量},且4\mathrm{阶行列式}\left|α_1,α_2,α_3,β_1\right|=m,\;\left|α_1,α_2,α_3,β_2\right|=n,则4\mathrm{阶行列式}\\\left|α_1,α_3,(β_1+β_2),α_2\right|=\left(\right).\end{array}
$$
$$
A.
m+n \quad B.m-n \quad C.n-m \quad D.-m-n \quad E. \quad F. \quad G. \quad H.
$$
$$
\begin{array}{l}4\mathrm{阶行列式}\\\left|α_1,α_3,(β_1+β_2),α_2\right|=\left|α_1,α_3,β_1,α_2\right|+\left|α_1,α_3,β_2,α_2\right|=m+n\end{array}
$$



$$
\begin{array}{l}设α_1,\alpha_2,α_3,β_1,β_2\;\mathrm{都是}4\mathrm{维列向量},且4\mathrm{阶行列式}\left|α_1,α_2,α_3,β_1\right|=m,\;\left|α_1,α_2,α_3,β_2\right|=n,则4\mathrm{阶行列式}\\\left|α_1,α_2,(β_1+β_2),α_3\right|=\left(\right).\end{array}
$$
$$
A.
m+n \quad B.m-n \quad C.n-m \quad D.-m-n \quad E. \quad F. \quad G. \quad H.
$$
$$
\left|α_1,α_2,(β_1+β_2),α_3\right|=\left|α_1,α_2,β_1,α_3\right|+\left|α_1,α_2,β_2,α_3\right|=-\left|α_1,α_2,α_3,β_1\right|-\left|α_1,α_2,α_3,β_2\right|=-m-n
$$



$$
\begin{array}{l}设α_1,α_2,α_3,β_1,β_2\;\mathrm{都是}4\mathrm{维列向量},且4\mathrm{阶行列式}\left|α_1,α_2,α_3,β_1\right|=m,\;\left|α_1,α_2,α_3,β_2\right|=n,则4\mathrm{阶行列式}\\\left|α_1,(β_1+β_2),α_3,α_2\right|=\left(\right).\end{array}
$$
$$
A.
m+n \quad B.m-n \quad C.n-m \quad D.-m-n \quad E. \quad F. \quad G. \quad H.
$$
$$
\left|α_1,(β_1+β_2),α_3,α_2\right|=\left|α_1,β_1,α_3,α_2\right|+\left|α_1,β_2,α_3,α_2\right|=-\left|α_1,α_2,α_3,β_1\right|-\left|α_1,α_2,α_3,β_2\right|=-m-n
$$



$$
设A=\left(α,\;γ_1,\;γ_2\right),\;B=\left(β,\;γ_1,\;γ_2\right)\mathrm{均是三阶方阵},α,\;β,\;γ_1,\;γ_2\mathrm{是三维列向量},若\left|A\right|=2,\left|B\right|=3,则\;\left|2A+B\right|=\left(\right).
$$
$$
A.
27 \quad B.72 \quad C.36 \quad D.63 \quad E. \quad F. \quad G. \quad H.
$$
$$
\left|2A+B\right|=\left|2α+β,\;3γ_1,\;3γ_2\right|=18\left|α,\;γ_1,\;γ_2\right|+9\left|β,\;γ_1,\;γ_2\right|=18×2+9×3=63
$$



$$
设A=\left(α,\;γ_1,\;γ_2\right),\;B=\left(β,\;γ_1,\;γ_2\right)\mathrm{均是三阶方阵},\alpha,\;β,\;γ_1,\;γ_2\mathrm{是三维列向量},若\left|A\right|=2,\left|B\right|=3,则\;\left|2A-B\right|=\left(\right).
$$
$$
A.
-1 \quad B.7 \quad C.1 \quad D.5 \quad E. \quad F. \quad G. \quad H.
$$
$$
\left|2A-B\right|=\left|2α-β,\;γ_1,\;γ_2\right|=2\left|α,\;γ_1,\;γ_2\right|-\left|β,\;γ_1,\;γ_2\right|=2×2-3=1
$$



$$
设A=\left(α,\;γ_1,\;γ_2\right),\;B=\left(β,\;γ_1,\;γ_2\right)\mathrm{均是三阶方阵},α,\;β,\;γ_1,\;γ_2\mathrm{是三维列向量},若\left|A\right|=2,\left|B\right|=3,则\;\left|3A-2B\right|=\left(\right).
$$
$$
A.
-1 \quad B.0 \quad C.1 \quad D.6 \quad E. \quad F. \quad G. \quad H.
$$
$$
\left|3A-2B\right|=\left|3α-2β,\;γ_1,\;γ_2\right|=3\left|α,\;γ_1,\;γ_2\right|-2\left|β,\;γ_1,\;γ_2\right|=3×2-2×3=0
$$



$$
设A=\left(α,\;γ_1,\;γ_2\right),\;B=\left(β,\;γ_1,\;γ_2\right)\mathrm{均是三阶方阵},α,\;β,\;γ_1,\;γ_2\mathrm{是三维列向量},若\left|A\right|=2,\left|B\right|=3,则\;\left|A-2B\right|=\left(\right).
$$
$$
A.
-1 \quad B.-4 \quad C.1 \quad D.8 \quad E. \quad F. \quad G. \quad H.
$$
$$
\left|A-2B\right|=\left|α-2β,\;-γ_1,\;-γ_2\right|=\left|α,\;γ_1,\;γ_2\right|-2\left|β,\;-γ_1,\;-γ_2\right|=2-6=-4
$$



$$
设α=\left(1,\;2,\;-1\right)^T,\;β=\left(1,\;-1,\;1\right)^T,\;γ=\left(-1,\;2,\;1\right)^T,则\alpha+2β-γ=\left(\right)\;.
$$
$$
A.
\left(4,\;2,\;2\right)^T \quad B.\left(4,\;-2,\;0\right)^T \quad C.\left(2,\;-2,\;0\right)^T \quad D.\left(2,\;2,\;0\right)^T \quad E. \quad F. \quad G. \quad H.
$$
$$
略
$$



$$
设α=\left(2,\;1,\;0\right)^T,\;β=\left(0,\;1,\;2\right)^T,\;γ=\left(-1,\;2,\;1\right)^T,则2α+3β-2γ=\left(\right)\;.
$$
$$
A.
\left(6,\;1,\;4\right)^T \quad B.\left(6,\;1,\;5\right)^T \quad C.\left(6,\;0,\;1\right)^T \quad D.\left(6,\;1,\;6\right)^T \quad E. \quad F. \quad G. \quad H.
$$
$$
略
$$



$$
设α=\left(2,\;1,\;0\right)^T,\;β=\left(0,\;1,\;2\right)^T,\;γ=\left(-1,\;2,\;1\right)^T,且2α-γ=δ-3β,则δ=\left(\right)\;.
$$
$$
A.
\left(5,\;3,\;5\right)^T \quad B.\left(5,\;3,\;3\right)^T \quad C.\left(5,\;5,\;3\right)^T \quad D.\left(5,\;5,\;5\right)^T \quad E. \quad F. \quad G. \quad H.
$$
$$
略
$$



$$
设α=\left(1,\;1,\;0,\;2\right)^T,\;β=\left(0,\;1,\;1,\;-3\right)^T,\;γ=\left(3,\;4,\;0,\;5\right)^T,则3α+2β-γ=\left(\right)\;.
$$
$$
A.
\left(0,\;1,\;2,\;-5\right)^T \quad B.\left(1,\;2,\;0,\;5\right)^T \quad C.\left(0,\;1,\;2,\;5\right)^T \quad D.\left(0,\;2,\;1,\;-5\right)^T \quad E. \quad F. \quad G. \quad H.
$$
$$
略
$$



$$
\begin{array}{l}\mathrm{设向量}α_1=\left(2,\;5,\;1,\;3\right)^T,\;α_2=\left(10,\;1,\;5,\;10\right)^T,\;α_3=\left(4,\;1,\;-1,\;1\right)^T,\mathrm{向量}α\mathrm{满足}3\left(α_1-α\right)+2\left(α_2+\alpha\right)=5\left(α_3+α\right),\\则\;α=\left(\right).\end{array}
$$
$$
A.
\left(1,\;2,\;3,\;4\right)^T \quad B.\left(1,\;2,\;3,\;5\right)^T \quad C.\left(0,\;1,\;2,\;3\right)^T \quad D.\left(0,\;2,\;3,\;4\right)^T \quad E. \quad F. \quad G. \quad H.
$$
$$
由3\left(α_1-α\right)+2\left(α_2+α\right)=5\left(α_3+α\right),得6α=3α_1+2\alpha_2-5α_3=6\left(1,\;2,\;3,\;4\right)^T.
$$



$$
\begin{array}{l}\mathrm{设向量}\alpha_1=\left(2,\;5,\;1,\;3\right)^T,\;α_2=\left(10,\;1,\;5,\;10\right)^T,\;α_3=\left(4,\;1,\;-1,\;1\right)^T,,\mathrm{向量}α\mathrm{满足}3\left(α_1-α\right)-2\left(α_2+\alpha\right)=2α_3-6α,\\则\;\alpha=\left(\right).\end{array}
$$
$$
A.
\left(22,\;-11,\;5,\;13\right)^T \quad B.\left(11,\;22,\;3,\;5\right)^T \quad C.\left(22,\;11,\;5,\;15\right)^T \quad D.\left(22,\;-11,\;5,\;-13\right)^T \quad E. \quad F. \quad G. \quad H.
$$
$$
由3\left(α_1-α\right)-2\left(α_2+α\right)=2α_3-6α,得α=2α_3-3α_1+2α_2=\left(22,\;-11,\;5,\;13\right)^T.
$$



$$
\mathrm{已知}α_1=\begin{pmatrix}3\\1\\5\\2\end{pmatrix},\;α_2=\begin{pmatrix}10\\5\\1\\10\end{pmatrix},\;α_3=\begin{pmatrix}1\\-1\\1\\4\end{pmatrix},若3\left(α_1-β\right)+2\left(α_2-β\right)=5\left(α_3+β\right),则\;β=\left(\right)
$$
$$
A.
\begin{pmatrix}2.4\\1.8\\1.2\\0.6\end{pmatrix} \quad B.\begin{pmatrix}2.4\\1.8\\0.2\\0.6\end{pmatrix} \quad C.\begin{pmatrix}2.4\\1.8\\1.2\\1.6\end{pmatrix} \quad D.\begin{pmatrix}2.4\\0.8\\1.2\\0.6\end{pmatrix} \quad E. \quad F. \quad G. \quad H.
$$
$$
3α_1-3β+2α_2-2β=5α_3+5β,\;β=\frac1{10}\left(3α_1+2α_2-5α_3\right)=\begin{pmatrix}2.4\\1.8\\1.2\\0.6\end{pmatrix}
$$



$$
\mathrm{设向量}α_1=\left(4,\;1,\;3,\;-2\right),\;α_2=\left(1,\;2,\;-3,\;2\right),\;α_3=\left(16,\;9,\;1,\;-3\right),则3α_1+5α_2-α_3=(\;).
$$
$$
A.
\left(1,\;4,\;-7,\;7\right) \quad B.\left(1,\;4,\;7,\;-7\right) \quad C.\left(1,\;4,\;-5,\;7\right) \quad D.\left(1,\;4,\;5,\;-7\right) \quad E. \quad F. \quad G. \quad H.
$$
$$
3α_1+5\alpha_2-α_3=3\left(4,\;1,\;3,\;-2\right)+5\left(1,\;2,\;-3,\;2\right)-\left(16,\;9,\;1,\;-3\right)=\left(1,\;4,\;-7,\;7\right)
$$



$$
\begin{array}{l}\mathrm{设向量}α_1=\left(2,\;5,\;1,\;3\right),\;α_2=\left(10,\;1,\;5,\;10\right),\;α_3=\left(4,\;1,\;-1,\;1\right),\mathrm{向量}α\mathrm{满足}3\left(α_1-α\right)+2\left(\;α_2+α\right)=2\left(α_3-α\right),\\则\;α=\left(\right).\end{array}
$$
$$
A.
\left(-18,\;-15,\;-15,\;-27\right) \quad B.\left(18,\;-15,\;15,\;-27\right) \quad C.\left(-18,\;15,\;-15,\;-27\right) \quad D.\left(18,\;-15,\;-15,\;27\right) \quad E. \quad F. \quad G. \quad H.
$$
$$
\begin{array}{l}3\left(α_1-α\right)+2\left(\;α_2+α\right)=2\left(α_3-α\right)⇒α=2α_3-3α_1-2α_2,即\\α=2\left(4,\;1,\;-1,\;1\right)-3\left(2,\;5,\;1,\;3\right)-2\left(10,\;1,\;5,\;10\right)=\left(-18,\;-15,\;-15,\;-27\right)\end{array}
$$



$$
\begin{array}{l}设α_1,α_2,α_3,β_1,β_2\;\mathrm{都是}4\mathrm{维列向量},且4\mathrm{阶行列式}\left|α_1,\;α_2,\;α_{3,\;}β_1\right|=m,\;\left|α_1,\;α_2,\;β_2,\;α_3\right|=n,则4\mathrm{阶行列式}\\\left|α_3,\;α_2,\;α_1,\;β_1+β_2\right|=\left(\right).\end{array}
$$
$$
A.
m+n \quad B.-\left(m+n\right) \quad C.n-m \quad D.m-n \quad E. \quad F. \quad G. \quad H.
$$
$$
\begin{array}{l}\left|α_3,\;α_2,\;α_1,\;β_1+β_2\right|=\left|α_3,\;α_2,\;α_{1,\;}β_1\right|+\left|α_3,\;α_2,\;α_1,\;β_2\right|=-\left|α_1,\;α_2,\;α_3,\;β_1\right|-\left|α_1,\;α_2,\;α_3,\;β_2\right|\\=-\left|α_1,\;α_2,\;α_3,\;β_1\right|+\left|α_1,\;α_2,\;β_1,\;α_3\right|=-m+n\end{array}
$$



$$
\begin{array}{l}设A,B\mathrm{都是三阶矩阵},且A=\begin{pmatrix}α\\2γ_2\\3γ_3\end{pmatrix},\;B=\begin{pmatrix}β\\γ_2\\γ_3\end{pmatrix},\mathrm{其中}α,\;\beta,\;γ_2,\;γ_3\mathrm{均为三维行向量},\left|A\right|=12,\;\left|B\right|=3,\mathrm{则行列式}\\\left|A+B\right|=\left(\right).\end{array}
$$
$$
A.
10 \quad B.20 \quad C.60 \quad D.72 \quad E. \quad F. \quad G. \quad H.
$$
$$
\begin{array}{l}\left|A\right|=6\begin{vmatrix}α\\γ_2\\γ_3\end{vmatrix}=12⇒\begin{vmatrix}α\\γ_2\\γ_3\end{vmatrix}=2,\left|B\right|=\begin{vmatrix}β\\γ_2\\γ_3\end{vmatrix}=3,\\故\left|A+B\right|=\;\begin{vmatrix}α+β\\3γ_2\\4γ_3\end{vmatrix}=12\begin{vmatrix}α+β\\γ_2\\γ_3\end{vmatrix}=12\begin{vmatrix}α\\γ_2\\γ_3\end{vmatrix}+12\begin{vmatrix}β\\γ_2\\γ_3\end{vmatrix}=12\left(2+3\right)=60.\;\end{array}
$$



$$
\begin{array}{l}设A,B\mathrm{都是三阶矩阵},且A=\begin{pmatrix}α\\2γ_2\\2γ_3\end{pmatrix},\;B=\begin{pmatrix}β\\γ_2\\γ_3\end{pmatrix},\mathrm{其中}α,\;β,\;γ_2,\;γ_3\mathrm{均为三维行向量},\left|A\right|=12,\;\left|B\right|=2,\mathrm{则行列式}\\\left|A+B\right|=\left(\right).\end{array}
$$
$$
A.
18 \quad B.45 \quad C.72 \quad D.9 \quad E. \quad F. \quad G. \quad H.
$$
$$
\begin{array}{l}\left|A\right|=4\begin{vmatrix}\begin{array}{c}α\\γ_2\\γ_3\end{array}\end{vmatrix}=12⇒\begin{vmatrix}\begin{array}{c}α\\γ_2\\γ_3\end{array}\end{vmatrix}=3,\left|B\right|=\begin{vmatrix}\begin{array}{c}β\\γ_2\\γ_3\end{array}\end{vmatrix}=2,\\故\left|A+B\right|=\;\begin{vmatrix}\begin{array}{c}α+β\\3γ_2\\3γ_3\end{array}\end{vmatrix}\begin{array}{c}\\\\\end{array}=9\begin{vmatrix}\begin{array}{c}α+β\\γ_2\\γ_3\end{array}\end{vmatrix}=9\begin{vmatrix}\begin{array}{c}α\\γ_2\\γ_3\end{array}\end{vmatrix}+9\begin{vmatrix}\begin{array}{c}β\\γ_2\\γ_3\end{array}\end{vmatrix}=9\left(3+2\right)=45.\;\end{array}
$$



$$
\begin{array}{l}设A,B\mathrm{都是三阶矩阵},且A=\begin{pmatrix}α\\2γ_2\\3γ_3\end{pmatrix},\;B=\begin{pmatrix}β\\γ_2\\2γ_3\end{pmatrix},\mathrm{其中}\alpha,\;β,\;γ_2,\;γ_3\mathrm{均为三维行向量},\left|A\right|=12,\;\left|B\right|=2,\mathrm{则行列式}\\\left|A+B\right|=\left(\right).\end{array}
$$
$$
A.
15 \quad B.30 \quad C.45 \quad D.60 \quad E. \quad F. \quad G. \quad H.
$$
$$
\begin{array}{l}\left|A\right|=6\begin{vmatrix}\begin{array}{c}α\\γ_2\\γ_3\end{array}\end{vmatrix}=12⇒\begin{vmatrix}\begin{array}{c}α\\γ_2\\γ_3\end{array}\end{vmatrix}=2,\left|B\right|=\begin{vmatrix}\begin{array}{c}β\\γ_2\\2γ_3\end{array}\end{vmatrix}=2⇒\begin{vmatrix}\begin{array}{c}β\\γ_2\\γ_3\end{array}\end{vmatrix}=1,\\故\left|A+B\right|=\;\begin{vmatrix}\begin{array}{c}α+β\\3γ_2\\5γ_3\end{array}\end{vmatrix}=15\begin{vmatrix}\begin{array}{c}α+β\\γ_2\\γ_3\end{array}\end{vmatrix}=15\begin{vmatrix}\begin{array}{c}α\\γ_2\\γ_3\end{array}\end{vmatrix}+15\begin{vmatrix}\begin{array}{c}β\\γ_2\\γ_3\end{array}\end{vmatrix}=15\left(2+1\right)=45.\;\end{array}
$$



$$
\begin{array}{l}设A,B\mathrm{都是三阶矩阵},且A=\begin{pmatrix}α\\2γ_2\\3γ_3\end{pmatrix},\;B=\begin{pmatrix}β\\γ_2\\γ_3\end{pmatrix},\mathrm{其中}α,\;β,\;γ_2,\;γ_3\mathrm{均为三维行向量},\left|A\right|=12,\;\left|B\right|=2,\mathrm{则行列式}\\\left|A+B\right|=\left(\right).\end{array}
$$
$$
A.
18 \quad B.48 \quad C.60 \quad D.24 \quad E. \quad F. \quad G. \quad H.
$$
$$
\begin{array}{l}\left|A\right|=6\begin{vmatrix}\begin{array}{c}α\\γ_2\\γ_3\end{array}\end{vmatrix}=12⇒\begin{vmatrix}\begin{array}{c}α\\γ_2\\γ_3\end{array}\end{vmatrix}=2,\left|B\right|=\begin{vmatrix}\begin{array}{c}β\\γ_2\\γ_3\end{array}\end{vmatrix}=2,\\故\left|A+B\right|=\;\begin{vmatrix}\begin{array}{c}α+β\\3γ_2\\4γ_3\end{array}\end{vmatrix}=12\begin{vmatrix}\begin{array}{c}α+β\\γ_2\\γ_3\end{array}\end{vmatrix}=12\begin{vmatrix}\begin{array}{c}α\\γ_2\\γ_3\end{array}\end{vmatrix}+12\begin{vmatrix}\begin{array}{c}β\\γ_2\\γ_3\end{array}\end{vmatrix}=12\left(2+2\right)=48.\;\end{array}
$$



$$
\mathrm{设四阶方阵}A=\left(ζ,\;α,\;β,\;γ\right),\;B=\left(η,\;β,\;γ,\;α\right),\mathrm{已知行列式}\left|A\right|=3,\;\left|B\right|=2,\mathrm{则行列式}\;\left|A+B\right|=\left(\right)
$$
$$
A.
5 \quad B.8 \quad C.10 \quad D.12 \quad E. \quad F. \quad G. \quad H.
$$
$$
\begin{array}{l}\left|A+B\right|=\left|ζ+η,\;α+β,\;β+γ,\;γ+α\right|\\=\left|ζ,\;α,\;β,\;γ\right|\;+\;\left|ζ,\;β,\;γ,\;α\right|+\left|η,\;α,\;β,\;γ\right|\;+\;\left|η,\;β,\;γ,\;α\right|\\=2\left|ζ,\;α,\;β,\;γ\right|+2\left|η,\;α,\;β,\;γ\right|=10\end{array}
$$



$$
\mathrm{设四阶方阵}A=\left(ζ,\;α,\;β,\;γ\right),\;B=\left(η,\;β,\;γ,\;α\right),\mathrm{已知行列式}\left|A\right|=1,\;\left|B\right|=3,\mathrm{则行列式}\;\left|A+B\right|=\left(\right)
$$
$$
A.
4 \quad B.8 \quad C.12 \quad D.16 \quad E. \quad F. \quad G. \quad H.
$$
$$
\begin{array}{l}\left|A+B\right|=\left|ζ+η,\;α+β,\;β+γ,\;γ+α\right|\\=\left|ζ,\;α,\;β,\;γ\right|\;+\;\left|ζ,\;β,\;γ,\;α\right|+\left|η,\;α,\;β,\;γ\right|\;+\;\left|η,\;β,\;γ,\;α\right|\\=2\left|ζ,\;α,\;β,\;γ\right|+2\left|η,\;α,\;β,\;γ\right|=2×4=8\end{array}
$$



$$
\mathrm{设四阶方阵}A=\left(ζ,\;α,\;β,\;γ\right),\;B=\left(η,\;β,\;γ,\;α\right),\mathrm{已知行列式}\left|A\right|=2,\;\left|B\right|=3,\mathrm{则行列式}\;\left|A+B\right|=\left(\right)
$$
$$
A.
5 \quad B.8 \quad C.10 \quad D.12 \quad E. \quad F. \quad G. \quad H.
$$
$$
\begin{array}{l}\left|A+B\right|=\left|ζ+η,\;α+β,\;β+γ,\;γ+α\right|\\=\left|ζ,\;α,\;β,\;γ\right|\;+\;\left|ζ,\;β,\;γ,\;α\right|+\left|η,\;α,\;β,\;γ\right|\;+\;\left|η,\;β,\;γ,\;α\right|\\=2\left|ζ,\;α,\;β,\;γ\right|+2\left|η,\;α,\;β,\;γ\right|=2×5=10\end{array}
$$



$$
\mathrm{设四阶方阵}A=\left(ζ,\;α,\;β,\;γ\right),\;B=\left(η,\;α,\;β,\;γ\right),\mathrm{已知行列式}\left|A\right|=2,\;\left|B\right|=4,\mathrm{则行列式}\;\left|A+B\right|=\left(\right)
$$
$$
A.
4 \quad B.8 \quad C.10 \quad D.48 \quad E. \quad F. \quad G. \quad H.
$$
$$
\begin{array}{l}\left|A+B\right|=\left|ζ+η,\;2α,\;2β,\;2γ\right|=8\left|ζ+η,\;α,\;β,\;γ\right|\end{array}=8\left(\left|A\right|+\left|B\right|\right)=48
$$



$$
\mathrm{设四阶方阵}A=\left(ζ,\;α,\;β,\;γ\right),\;B=\left(η,\;β,\;γ,\;α\right),\mathrm{已知行列式}\left|A\right|=1,\;\left|B\right|=5,\mathrm{则行列式}\;\left|A+B\right|=\left(\right)
$$
$$
A.
5 \quad B.6 \quad C.8 \quad D.12 \quad E. \quad F. \quad G. \quad H.
$$
$$
\begin{array}{l}\left|A+B\right|=\left|ζ+η,\;α+β,\;β+γ,\;γ+α\right|\\=\left|ζ,\;α,\;β,\;γ\right|\;+\;\left|ζ,\;β,\;γ,\;α\right|+\left|η,\;α,\;β,\;γ\right|\;+\;\left|η,\;β,\;γ,\;α\right|\\=2\left|ζ,\;α,\;β,\;γ\right|+2\left|η,\;α,\;β,\;γ\right|=2×6=12\end{array}
$$



$$
\begin{array}{l}\mathrm{设向量}α_1=\left(2,\;5,\;1,\;3\right),\;α_2=\left(10,\;1,\;5,\;10\right),\;α_3=\left(4,\;1,\;-1,\;1\right),\mathrm{向量}α\mathrm{满足}3\left(α_1-α\right)+2\left(\;α_2+α\right)=2\left(α_3+α\right),\\则\;α=\left(\right).\end{array}
$$
$$
A.
\left(6,\;5,\;5,\;9\right) \quad B.\left(6,\;-5,\;5,\;9\right) \quad C.\left(\mathbf6\boldsymbol,\boldsymbol\;\mathbf5\boldsymbol,\boldsymbol\;\boldsymbol-\mathbf5\boldsymbol,\boldsymbol\;\mathbf9\right) \quad D.\left(6,\;-5,\;-5,\;9\right) \quad E. \quad F. \quad G. \quad H.
$$
$$
\begin{array}{l}由3\left(α_1-α\right)+2\left(\;α_2+α\right)=2\left(α_3+α\right)⇒-3α=2α_3-3α_1-2α_2,\mathrm{即得}\\α=-\frac13\left(-18,\;-15,\;-15,\;-27\right)=\left(6,\;5,\;5,\;9\right)\end{array}
$$



$$
\mathrm{已知}2α+3β=\left(1,\;2,\;3,\;4\right)^T,\;α+2β=\left(1,\;2,\;2,\;-1\right)^T,则β 为\;\left(\right).
$$
$$
A.
β=\left(1,\;2,\;1,\;-6\right)^{{}^T} \quad B.β=\left(1,\;2,\;-1,\;6\right)^T \quad C.β=\left(1,\;2,\;-1,\;-6\right)^T \quad D.β=\left(1,\;-2,\;-1,\;6\right)^T \quad E. \quad F. \quad G. \quad H.
$$
$$
\begin{array}{l}\mathrm{由条件可知}2α+3β=\left(1,\;2,\;3,\;4\right)^T,\;α+2β=\left(1,\;2,\;2,\;-1\right)^T,则\;\;\\β=2\left(α+2β\right)-\left(2α+3β\right)=\left(1,\;2,\;1,\;-6\right)^T.\end{array}
$$



$$
\begin{array}{l}设α_1,α_2,α_3,β_1,β_2\;\mathrm{都是}4\mathrm{维列向量},且4\mathrm{阶行列式}\left|α_1,\;α_2,\;α_3,\;β_1\right|=m,\;\left|α_1,\;α_2,\;β_2,\;α_3\right|=n,则4\mathrm{阶行列式}\\\left|α_1,\;α_2,\;α_3,\;β_1+β_2\right|=\left(\right).\end{array}
$$
$$
A.
m+n \quad B.m-n \quad C.n-m \quad D.-m-n \quad E. \quad F. \quad G. \quad H.
$$
$$
\left|α_1,α_2,α_3,\left(β_1+\beta_2\right)\right|=\left|α_1,α_2,α_3,β_1\right|-\left|α_1,α_2,β_2,α_3\right|=m-n
$$



$$
\mathrm{行列式}\left|α_1,\;α_2,\;α_3,\;β_1\right|=m,\;\left|α_1,\;α_3,\;β_2,\;α_2\right|=n,\mathrm{则行列式}\left|α_1,\;α_3,\;\left(β_1+β_2\right),\;α_2\right|\mathrm{的值为}\left(\right).
$$
$$
A.
m+n \quad B.m-n \quad C.n-m \quad D.-m-n \quad E. \quad F. \quad G. \quad H.
$$
$$
\left|α_1,\;α_3,\;\left(β_1+β_2\right),\;α_2\right|=\left|α_1,\;α_3,\;β_1,\;α_2\right|+\left|α_1,\;α_3,\;β_2,\;α_2\right|=m+n
$$



$$
\begin{array}{l}\mathrm{设向量}α_1=\left(2,\;5,\;1,\;3\right),\;α_2=\left(10,\;1,\;5,\;10\right),\;α_3=\left(4,\;1,\;-1,\;1\right),\mathrm{向量}α\mathrm{满足}α_1+α+\;α_2=2\left(α_3+α\right),\\则\;α=\left(\right).\end{array}
$$
$$
A.
\left(4,\;4,\;8,\;9\right) \quad B.\left(4,\;4,\;8,\;7\right) \quad C.\left(4,\;4,\;8,\;10\right) \quad D.\left(4,\;4,\;8,\;11\right) \quad E. \quad F. \quad G. \quad H.
$$
$$
\begin{array}{l}由α_1+α+\;α_2=2\left(α_3+α\right),得\\α=α_1+\;α_2-2α_3=\left(4,\;4,\;8,\;9\right)+\left(10,\;1,\;5,\;10\right)-2\left(4,\;1,\;-1,\;1\right)=\left(4,\;4,\;8,\;11\right)\end{array}
$$



$$
\mathrm{行列式}\left|α_1,\;α_2,\;α_3,\;β_1\right|=m,\;\left|α_1,\;α_3,\;α_2,\;β_2\right|=n,\mathrm{则行列式}\left|α_1,\;α_3,\;\left(β_1+β_2\right),\;α_2\right|\mathrm{的值为}\left(\right).
$$
$$
A.
m-n \quad B.m+n \quad C.mn \quad D.m+2n \quad E. \quad F. \quad G. \quad H.
$$
$$
\begin{array}{l}\mathrm{依题意得}\\\left|α_1,\;α_3,\;\left(β_1+β_2\right),\;α_2\right|=\left|α_1,\;α_3,\;β_1,\;α_2\right|+\left|α_1,\;α_3,\;β_2,\;α_2\right|\\=-\left|α_1,\;α_2,\;β_1,\;α_3\right|-\left|α_1,\;α_3,\;α_2,\;β_2\right|\\=\left|α_1,\;α_2,\;α_3,\;β_1\right|-\left|α_1,\;α_3,\;α_2,\;β_2\right|=m-n\end{array}
$$



$$
\mathrm{设四阶方阵}A=\left(ζ,\;α,\;β,\;γ\right),\;B=\left(η,\;β,\;γ,\;α\right),\mathrm{已知行列式}\left|A\right|=1,\;\left|B\right|=2,\mathrm{则行列式}\;\left|A+B\right|=\left(\right)
$$
$$
A.
8 \quad B.4 \quad C.12 \quad D.6 \quad E. \quad F. \quad G. \quad H.
$$
$$
\begin{array}{l}\left|A+B\right|=\left|ζ+η,\;α+β,\;β+γ,\;γ+\alpha\right|\\=\left|ζ,\;α,\;β,\;γ\right|\;+\;\left|ζ,\;β,\;γ,\;α\right|+\left|η,\;α,\;β,\;γ\right|\;+\;\left|η,\;β,\;γ,\;α\right|\\=2\left|ζ,\;α,\;β,\;γ\right|+2\left|η,\;α,\;β,\;γ\right|=2×3=6\end{array}
$$



$$
\begin{array}{l}设A,B\mathrm{都是三阶矩阵},且A=\begin{pmatrix}α\\2γ_2\\3γ_3\end{pmatrix},\;B=\begin{pmatrix}β\\γ_2\\γ_3\end{pmatrix},\mathrm{其中}α,\;β,\;γ_2,\;γ_3\mathrm{均为三维行向量},\left|A\right|=15,\;\left|B\right|=3,\mathrm{则行列式}\\\left|A-B\right|=\left(\right).\end{array}
$$
$$
A.
3 \quad B.0 \quad C.-1 \quad D.1 \quad E. \quad F. \quad G. \quad H.
$$
$$
\begin{array}{l}\left|A\right|=6\begin{vmatrix}\begin{array}{c}α\\γ_2\\γ_3\end{array}\end{vmatrix}=15⇒\begin{vmatrix}\begin{array}{c}α\\γ_2\\γ_3\end{array}\end{vmatrix}=\frac52,\left|B\right|=\begin{vmatrix}\begin{array}{c}β\\γ_2\\γ_3\end{array}\end{vmatrix}=3,\\故\left|A-B\right|=\;\begin{vmatrix}\begin{array}{c}α-β\\γ_2\\2γ_3\end{array}\end{vmatrix}=2\begin{vmatrix}\begin{array}{c}\alpha-β\\γ_2\\γ_3\end{array}\end{vmatrix}=2\begin{vmatrix}\begin{array}{c}α\\γ_2\\γ_3\end{array}\end{vmatrix}-2\begin{vmatrix}\begin{array}{c}β\\γ_2\\γ_3\end{array}\end{vmatrix}=2×\frac52-2×3=-1.\;\end{array}
$$



$$
\mathrm{已知}2α+3β=\left(1,\;2,\;3,\;4\right),\;α+2β=\left(1,\;2,\;2,\;-1\right),则α,\;β\mathrm{分别为}\;\left(\right).
$$
$$
A.
α=\left(-1,\;-2,\;0,\;11\right);\;β=\left(1,\;2,\;1,\;-6\right) \quad B.α=\left(-1,\;-2,\;0,\;11\right);\;β=\left(1,\;2,\;-1,\;6\right) \quad C.\alpha=\left(-1,\;2,\;0,\;11\right);\;β=\left(1,\;2,\;1,\;-6\right) \quad D.α=\left(-1,\;2,\;0,\;11\right);\;β=\left(1,\;2,\;-1,\;6\right) \quad E. \quad F. \quad G. \quad H.
$$
$$
\begin{array}{l}\mathrm{由条件可知}2α+3β=\left(1,\;2,\;3,\;4\right),\;α+2β=\left(1,\;2,\;2,\;-1\right),\;则\\β=2\left(α+2β\right)\;-\left(2α+3β\right)=\left(1,\;2,\;1,\;-6\right);α=\left(1,\;2,\;2,\;-1\right)-2β=\left(-1,\;-2,\;0,\;11\right)\;.\end{array}
$$



$$
\mathrm{已知}2α+3β=\left(1,\;2,\;3,\;4\right),\;α-2β=\left(1,\;2,\;2,\;-1\right),则β 为\;\left(\right).
$$
$$
A.
\frac17\left(-1,\;-2,\;-1,\;6\right) \quad B.\frac15\left(-1,\;-2,\;-1,\;6\right) \quad C.\frac15\left(1,\;2,\;1,\;-6\right) \quad D.\frac17\left(1,\;2,\;1,\;-6\right) \quad E. \quad F. \quad G. \quad H.
$$
$$
\begin{array}{l}\mathrm{由条件可知}2α+3β=\left(1,\;2,\;3,\;4\right),\;α-2β=\left(1,\;2,\;2,\;-1\right),\;则\\7β=\left(2α+3β\right)-\left(2α-4β\right)=\left(-1,\;-2,\;-1,\;6\right)\;.\end{array}
$$



$$
设α_1,\;α_2,\;β_1,\;β_2,\;γ\mathrm{都是三维行向量},\mathrm{且行列式}\begin{vmatrix}α_1\\β_1\\γ\end{vmatrix}=\begin{vmatrix}α_1\\β_2\\γ\end{vmatrix}=\begin{vmatrix}α_2\\β_1\\γ\end{vmatrix}=\begin{vmatrix}α_2\\β_2\\γ\end{vmatrix}=3,则\begin{vmatrix}α_1+α_2\\β_1+β_2\\2γ\end{vmatrix}=\left(\right).
$$
$$
A.
12 \quad B.16 \quad C.18 \quad D.24 \quad E. \quad F. \quad G. \quad H.
$$
$$
\begin{vmatrix}α_1+α_2\\β_1+β_2\\2γ\end{vmatrix}=2\begin{vmatrix}α_1\\β_1\\γ\end{vmatrix}+2\begin{vmatrix}α_1\\β_2\\γ\end{vmatrix}+2\begin{vmatrix}α_2\\β_1\\γ\end{vmatrix}+2\begin{vmatrix}α_2\\β_2\\γ\end{vmatrix}=2×12=24
$$



$$
设α_1,\;α_2,\;β_1,\;β_2,\;γ\mathrm{都是三维行向量},\mathrm{且行列式}\begin{vmatrix}α_1\\β_1\\γ\end{vmatrix}=\begin{vmatrix}α_1\\β_2\\γ\end{vmatrix}=\begin{vmatrix}α_2\\β_1\\γ\end{vmatrix}=\begin{vmatrix}α_2\\β_2\\γ\end{vmatrix}=1,则\begin{vmatrix}α_1+α_2\\β_1+β_2\\2γ\end{vmatrix}=\left(\right).
$$
$$
A.
2 \quad B.4 \quad C.6 \quad D.8 \quad E. \quad F. \quad G. \quad H.
$$
$$
\begin{vmatrix}α_1+α_2\\β_1+β_2\\2γ\end{vmatrix}=2\begin{vmatrix}α_1\\β_1\\γ\end{vmatrix}+2\begin{vmatrix}α_1\\β_2\\γ\end{vmatrix}+2\begin{vmatrix}α_2\\β_1\\γ\end{vmatrix}+2\begin{vmatrix}α_2\\β_2\\γ\end{vmatrix}=2×4=8
$$



$$
\begin{array}{l}设A,B\mathrm{都是三阶矩阵},且A=\begin{pmatrix}α\\2γ_2\\3γ_3\end{pmatrix},\;B=\begin{pmatrix}β\\γ_2\\γ_3\end{pmatrix},\mathrm{其中}α,\;β,\;γ_2,\;γ_3\mathrm{均为三维行向量},\left|A\right|=27,\;\left|B\right|=3,\mathrm{则行列式}\\\left|A-B\right|=\left(\right).\end{array}
$$
$$
A.
3 \quad B.2 \quad C.12 \quad D.10 \quad E. \quad F. \quad G. \quad H.
$$
$$
\begin{array}{l}\left|A\right|=6\begin{vmatrix}\begin{array}{c}α\\γ_2\\γ_3\end{array}\end{vmatrix}=27⇒\begin{vmatrix}\begin{array}{c}α\\γ_2\\γ_3\end{array}\end{vmatrix}=\frac92,\left|B\right|=\begin{vmatrix}\begin{array}{c}β\\γ_2\\γ_3\end{array}\end{vmatrix}=3,\\故\left|A-B\right|=\;\begin{vmatrix}\begin{array}{c}α-β\\γ_2\\2γ_3\end{array}\end{vmatrix}=2\begin{vmatrix}\begin{array}{c}α-β\\γ_2\\γ_3\end{array}\end{vmatrix}=2\begin{vmatrix}\begin{array}{c}α\\γ_2\\γ_3\end{array}\end{vmatrix}-2\begin{vmatrix}\begin{array}{c}β\\γ_2\\γ_3\end{array}\end{vmatrix}=2×\frac92-2×3=3.\;\end{array}
$$



$$
\begin{array}{l}设A,B\mathrm{都是三阶矩阵},且A=\begin{pmatrix}α\\2γ_2\\3γ_3\end{pmatrix},\;B=\begin{pmatrix}β\\γ_2\\γ_3\end{pmatrix},\mathrm{其中}α,\;β,\;γ_2,\;γ_3\mathrm{均为三维行向量},\left|A\right|=21,\;\left|B\right|=3,\mathrm{则行列式}\\\left|A-B\right|=\left(\right).\end{array}
$$
$$
A.
-1 \quad B.13 \quad C.4 \quad D.1 \quad E. \quad F. \quad G. \quad H.
$$
$$
\begin{array}{l}\left|A\right|=6\begin{vmatrix}\begin{array}{c}α\\γ_2\\γ_3\end{array}\end{vmatrix}=21⇒\begin{vmatrix}\begin{array}{c}α\\γ_2\\γ_3\end{array}\end{vmatrix}=\frac72,\left|B\right|=\begin{vmatrix}\begin{array}{c}β\\γ_2\\γ_3\end{array}\end{vmatrix}=3,\\故\left|A-B\right|=\;\begin{vmatrix}\begin{array}{c}α-β\\γ_2\\2γ_3\end{array}\end{vmatrix}=2\begin{vmatrix}\begin{array}{c}α-β\\γ_2\\γ_3\end{array}\end{vmatrix}=2\begin{vmatrix}\begin{array}{c}α\\\gamma_2\\γ_3\end{array}\end{vmatrix}-2\begin{vmatrix}\begin{array}{c}β\\γ_2\\γ_3\end{array}\end{vmatrix}=2×\frac72-2×3=1.\;\end{array}
$$



$$
\begin{array}{l}设A,B\mathrm{都是三阶矩阵},且A=\begin{pmatrix}α\\2γ_2\\3γ_3\end{pmatrix},\;B=\begin{pmatrix}β\\γ_2\\γ_3\end{pmatrix},\mathrm{其中}α,\;β,\;γ_2,\;γ_3\mathrm{均为三维行向量},\left|A\right|=12,\;\left|B\right|=3,\mathrm{则行列式}\\\left|A-B\right|=\left(\right).\end{array}
$$
$$
A.
-2 \quad B.2 \quad C.8 \quad D.10 \quad E. \quad F. \quad G. \quad H.
$$
$$
\begin{array}{l}\left|A\right|=6\begin{vmatrix}\begin{array}{c}\alpha\\γ_2\\γ_3\end{array}\end{vmatrix}=12⇒\begin{vmatrix}\begin{array}{c}α\\γ_2\\γ_3\end{array}\end{vmatrix}=2,\left|B\right|=\begin{vmatrix}\begin{array}{c}β\\γ_2\\γ_3\end{array}\end{vmatrix}=3,\\故\left|A-B\right|=\;\begin{vmatrix}\begin{array}{c}α-β\\γ_2\\2γ_3\end{array}\end{vmatrix}=2\begin{vmatrix}\begin{array}{c}α-β\\γ_2\\γ_3\end{array}\end{vmatrix}=2\begin{vmatrix}\begin{array}{c}α\\γ_2\\γ_3\end{array}\end{vmatrix}-2\begin{vmatrix}\begin{array}{c}β\\γ_2\\γ_3\end{array}\end{vmatrix}=2×2-2×3=-2.\;\end{array}
$$



$$
\begin{array}{l}设A,B\mathrm{都是三阶矩阵},且A=\begin{pmatrix}α\\2γ_2\\3γ_3\end{pmatrix},\;B=\begin{pmatrix}β\\γ_2\\γ_3\end{pmatrix},\mathrm{其中}α,\;β,\;γ_2,\;γ_3\mathrm{均为三维行向量},\left|A\right|=18,\;\left|B\right|=3,\mathrm{则行列式}\\\left|A-B\right|=\left(\right).\end{array}
$$
$$
A.
3 \quad B.6 \quad C.0 \quad D.12 \quad E. \quad F. \quad G. \quad H.
$$
$$
\begin{array}{l}\left|A\right|=6\begin{vmatrix}\begin{array}{c}α\\γ_2\\γ_3\end{array}\end{vmatrix}=18⇒\begin{vmatrix}\begin{array}{c}α\\γ_2\\γ_3\end{array}\end{vmatrix}=3,\left|B\right|=\begin{vmatrix}\begin{array}{c}β\\γ_2\\γ_3\end{array}\end{vmatrix}=3,\\故\left|A-B\right|=\;\begin{vmatrix}\begin{array}{c}α-β\\γ_2\\2γ_3\end{array}\end{vmatrix}=2\begin{vmatrix}\begin{array}{c}α-β\\γ_2\\γ_3\end{array}\end{vmatrix}=2\begin{vmatrix}\begin{array}{c}α\\γ_2\\γ_3\end{array}\end{vmatrix}-2\begin{vmatrix}\begin{array}{c}β\\γ_2\\γ_3\end{array}\end{vmatrix}=2×3-2×3=0.\;\end{array}
$$



$$
\begin{array}{l}设A,B\mathrm{都是三阶矩阵},且A=\begin{pmatrix}α\\2γ_2\\3γ_3\end{pmatrix},\;B=\begin{pmatrix}β\\γ_2\\γ_3\end{pmatrix},\mathrm{其中}α,\;β,\;γ_2,\;γ_3\mathrm{均为三维行向量},\left|A\right|=36,\;\left|B\right|=3,\mathrm{则行列式}\\\left|A-B\right|=\left(\right).\end{array}
$$
$$
A.
3 \quad B.6 \quad C.9 \quad D.18 \quad E. \quad F. \quad G. \quad H.
$$
$$
\begin{array}{l}\left|A\right|=6\begin{vmatrix}\begin{array}{c}α\\γ_2\\γ_3\end{array}\end{vmatrix}=36⇒\begin{vmatrix}\begin{array}{c}α\\γ_2\\γ_3\end{array}\end{vmatrix}=6,\left|B\right|=\begin{vmatrix}\begin{array}{c}β\\γ_2\\γ_3\end{array}\end{vmatrix}=3,\\故\left|A-B\right|=\;\begin{vmatrix}\begin{array}{c}α-β\\γ_2\\2γ_3\end{array}\end{vmatrix}=2\begin{vmatrix}\begin{array}{c}α-β\\γ_2\\γ_3\end{array}\end{vmatrix}=2\begin{vmatrix}\begin{array}{c}α\\γ_2\\γ_3\end{array}\end{vmatrix}-2\begin{vmatrix}\begin{array}{c}β\\γ_2\\γ_3\end{array}\end{vmatrix}=2×6-2×3=6.\;\end{array}
$$



$$
\begin{array}{l}设A,B\mathrm{都是三阶矩阵},且A=\begin{pmatrix}α\\2γ_2\\3γ_3\end{pmatrix},\;B=\begin{pmatrix}β\\γ_2\\γ_3\end{pmatrix},\mathrm{其中}α,\;β,\;γ_2,\;γ_3\mathrm{均为三维行向量},\left|A\right|=24,\;\left|B\right|=3,\mathrm{则行列式}\\\left|A-B\right|=\left(\right).\end{array}
$$
$$
A.
1 \quad B.2 \quad C.14 \quad D.7 \quad E. \quad F. \quad G. \quad H.
$$
$$
\begin{array}{l}\left|A\right|=6\begin{vmatrix}\begin{array}{c}α\\γ_2\\γ_3\end{array}\end{vmatrix}=24⇒\begin{vmatrix}\begin{array}{c}α\\γ_2\\γ_3\end{array}\end{vmatrix}=4,\left|B\right|=\begin{vmatrix}\begin{array}{c}β\\γ_2\\γ_3\end{array}\end{vmatrix}=3,\\故\left|A-B\right|=\;\begin{vmatrix}\begin{array}{c}α-β\\γ_2\\2γ_3\end{array}\end{vmatrix}=2\begin{vmatrix}\begin{array}{c}α-β\\γ_2\\γ_3\end{array}\end{vmatrix}=2\begin{vmatrix}\begin{array}{c}α\\γ_2\\γ_3\end{array}\end{vmatrix}-2\begin{vmatrix}\begin{array}{c}β\\γ_2\\γ_3\end{array}\end{vmatrix}=2×4-2×3=2.\;\end{array}
$$



$$
设α_1,\;α_2,\;β_1,\;β_2,\;γ\mathrm{都是三维行向量},\mathrm{且行列式}\begin{vmatrix}α_1\\β_1\\γ\end{vmatrix}=\begin{vmatrix}α_1\\β_2\\γ\end{vmatrix}=\begin{vmatrix}α_2\\β_1\\γ\end{vmatrix}=\begin{vmatrix}α_2\\β_2\\γ\end{vmatrix}=\frac32,则\begin{vmatrix}α_1+α_2\\β_1+β_2\\2γ\end{vmatrix}=\left(\right).
$$
$$
A.
12 \quad B.16 \quad C.18 \quad D.24 \quad E. \quad F. \quad G. \quad H.
$$
$$
\begin{vmatrix}α_1+\alpha_2\\β_1+β_2\\2γ\end{vmatrix}=2\begin{vmatrix}α_1\\β_1\\γ\end{vmatrix}+2\begin{vmatrix}α_1\\β_2\\γ\end{vmatrix}+2\begin{vmatrix}α_2\\β_1\\γ\end{vmatrix}+2\begin{vmatrix}α_2\\β_2\\γ\end{vmatrix}=2×4×\frac32=12
$$



$$
设α_1,\;α_2,\;β_1,\;β_2,\;γ\mathrm{都是三维行向量},\mathrm{且行列式}\begin{vmatrix}α_1\\β_1\\γ\end{vmatrix}=\begin{vmatrix}α_1\\β_2\\γ\end{vmatrix}=\begin{vmatrix}α_2\\β_1\\γ\end{vmatrix}=\begin{vmatrix}α_2\\β_2\\γ\end{vmatrix}=\frac14,则\begin{vmatrix}α_1+α_2\\β_1+β_2\\2γ\end{vmatrix}=\left(\right).
$$
$$
A.
2 \quad B.4 \quad C.8 \quad D.12 \quad E. \quad F. \quad G. \quad H.
$$
$$
\begin{vmatrix}\alpha_1+α_2\\β_1+β_2\\2γ\end{vmatrix}=2\begin{vmatrix}α_1\\β_1\\γ\end{vmatrix}+2\begin{vmatrix}α_1\\β_2\\γ\end{vmatrix}+2\begin{vmatrix}α_2\\β_1\\γ\end{vmatrix}+2\begin{vmatrix}α_2\\β_2\\γ\end{vmatrix}=2×4×\frac14=2
$$



$$
\mathrm{已知}3α+2β=\left(1,\;2,\;3,\;4\right),\;2α+3β=\left(-1,\;18,\;7,\;6\right),则α,β 为\;\left(\right).
$$
$$
A.
\left(1,\;-6,\;-1,\;0\right),\;\left(-1,\;10,\;3,\;2\right) \quad B.\left(1,\;6,\;-1,\;-6\right),\;\left(1,\;10,\;3,\;2\right) \quad C.\left(1,\;-6,\;-1,\;6\right),\;\left(-1,\;10,\;2,\;3\right) \quad D.\left(1,\;-6,\;-1,\;6\right),\;\left(-1,\;10,\;3,\;2\right) \quad E. \quad F. \quad G. \quad H.
$$
$$
\begin{array}{l}\mathrm{由条件}3α+2β=\left(1,\;2,\;3,\;4\right),\;2α+3β=\left(-1,\;18,\;7,\;6\right),\;则\\\;5α=3\left(1,\;2,\;3,\;4\right)-2\left(-1,\;18,\;7,\;6\right)=\left(1,\;-6,\;-1,\;0\right).\\5β=\;3\left(-1,\;18,\;7,6\right)-2\left(1,\;2,\;3,\;4\right)=\left(-1,\;10,\;3,\;2\right)\;.\;\;\;\;\;\;\;\;\;\;\;\;\;\;\;\;\;\;\end{array}
$$



$$
\mathrm{已知}3α+2β=\left(1,\;2,\;3,\;4\right),\;2α+3β=\left(1,\;-2,\;2,\;-1\right),则β 为\;\left(\right).
$$
$$
A.
\frac17\left(1,\;10,\;0,\;11\right) \quad B.\frac15\left(1,\;-10,\;0,\;-11\right) \quad C.\frac15\left(1,\;-10,\;-1,\;11\right) \quad D.\frac17\left(1,\;-10,\;0,\;-11\right) \quad E. \quad F. \quad G. \quad H.
$$
$$
\begin{array}{l}\begin{array}{l}\mathrm{由条件}3α+2β=\left(1,\;2,\;3,\;4\right),\;2α+3β=\left(1,\;-2,\;2,\;-1\right),\;则\end{array}\\α-β=\left(0,\;4,\;1,\;5\right)⇒5β=\left(2α+3β\right)-\left(2α-2β\right)\\=\left(1,\;-2,\;2,\;-1\right)-\left(0,\;8,\;2,\;10\right)=\left(1,\;-10,\;0,\;-11\right)\end{array}
$$



$$
\mathrm{设四阶方阵}A=\left(ζ,\;α,\;β,\;γ\right),\;B=\left(η,\;α,\;β,\;γ\right),\mathrm{已知行列式}\left|A\right|=2,\;\left|B\right|=4,\mathrm{则行列式}\;\left|A+B\right|=\left(\right)
$$
$$
A.
6 \quad B.8 \quad C.10 \quad D.12 \quad E. \quad F. \quad G. \quad H.
$$
$$
\begin{array}{l}\begin{array}{l}\left|A+B\right|=\left|ζ+η,\;α+β,\;β+\gamma,\;γ+α\right|=\left|ζ,\;α,\;β,\;γ\right|\end{array}+\left|ζ,\;β,\;γ,\;α\right|+\left|η,\;α,\;β,\;γ\right|+\left|η,\;β,\;\gamma,\;α\right|\\=2\left|ζ,\;α,\;β,\;γ\right|+2\left|η,\;α,\;β,\;γ\right|=2×6=12\end{array}
$$



$$
\mathrm{设四阶方阵}A=\left(ζ,\;α,\;β,\;γ\right),\;B=\left(η,\;α,\;β,\;γ\right),\mathrm{已知行列式}\left|A\right|=1,\;\left|B\right|=3,\mathrm{则行列式}\;\left|A+B\right|=\left(\right)
$$
$$
A.
4 \quad B.6 \quad C.32 \quad D.12 \quad E. \quad F. \quad G. \quad H.
$$
$$
\begin{array}{l}\begin{array}{l}\left|A+B\right|=\left|ζ+η,\;α+β,\;β+γ,\;γ+α\right|=\left|ζ,\;α,\;β,\;γ\right|\end{array}+\left|ζ,\;β,\;γ,\;α\right|+\left|η,\;α,\;β,\;γ\right|+\left|η,\;β,\;γ,\;α\right|\\=2\left|ζ,\;α,\;β,\;γ\right|+2\left|η,\;α,\;β,\;γ\right|=2×4=8\end{array}
$$



$$
\begin{array}{l}设3\left(α_1-α\right)+4(α_2+α)=5\left(α_3+α\right),\mathrm{其中}α_1=\left(2,\;5,\;1,\;3\right)^T,\;α_2=\left(10,\;1,\;5,\;10\right)^T,\;\\α_3=\left(4,\;1,\;-1,\;1\right)^T,则α=\left(\right).\end{array}
$$
$$
A.
\frac14\left(26,\;14,\;28,\;44\right)^T \quad B.\frac14\left(26,\;15,\;28,\;44\right)^T \quad C.\frac14\left(26,\;14,\;28,\;45\right)^T \quad D.\frac14\left(26,\;14,\;29,\;44\right)^T \quad E. \quad F. \quad G. \quad H.
$$
$$
\begin{array}{l}由3\left(α_1-α\right)+4(α_2+α)=5\left(α_3+α\right)\mathrm{整理得}\\α=\frac14\left(3α_1+4α_2-5α_3\right)\\=\frac14\left[3\left(2,\;5,\;1,\;3\right)^T+4\left(10,\;1,\;5,\;10\right)^T-5\left(4,\;1,\;-1,\;1\right)^T\right]\\=\frac14\left(26,\;14,\;28,\;44\right)^T\end{array}
$$



$$
设α_1,\;α_2,\;β_1,\;β_2,\;γ\mathrm{都是三维行向量},\mathrm{且行列式}\begin{vmatrix}α_1\\β_1\\γ\end{vmatrix}=\begin{vmatrix}α_1\\β_2\\γ\end{vmatrix}=\begin{vmatrix}α_2\\β_1\\γ\end{vmatrix}=\begin{vmatrix}α_2\\β_2\\γ\end{vmatrix}=2,则\begin{vmatrix}α_1+α_2\\β_1+β_2\\2γ\end{vmatrix}=\left(\right).
$$
$$
A.
4 \quad B.8 \quad C.16 \quad D.12 \quad E. \quad F. \quad G. \quad H.
$$
$$
\begin{vmatrix}α_1+α_2\\β_1+β_2\\2γ\end{vmatrix}=2\begin{vmatrix}α_1\\β_1\\γ\end{vmatrix}+2\begin{vmatrix}α_1\\β_2\\γ\end{vmatrix}+2\begin{vmatrix}α_2\\β_1\\γ\end{vmatrix}+2\begin{vmatrix}α_2\\β_2\\γ\end{vmatrix}=2×8=16
$$



$$
设α_1,\;α_2,\;β_1,\;β_2,\;γ\mathrm{都是三维行向量},\mathrm{且行列式}\begin{vmatrix}α_1\\β_1\\γ\end{vmatrix}=\begin{vmatrix}α_1\\β_2\\γ\end{vmatrix}=\begin{vmatrix}α_2\\β_1\\γ\end{vmatrix}=\begin{vmatrix}α_2\\β_2\\γ\end{vmatrix}=\frac52,则\begin{vmatrix}α_1+α_2\\β_1+β_2\\2γ\end{vmatrix}=\left(\right).
$$
$$
A.
20 \quad B.18 \quad C.16 \quad D.10 \quad E. \quad F. \quad G. \quad H.
$$
$$
\begin{vmatrix}α_1+α_2\\β_1+β_2\\2γ\end{vmatrix}=2\begin{vmatrix}α_1\\β_1\\γ\end{vmatrix}+2\begin{vmatrix}α_1\\β_2\\γ\end{vmatrix}+2\begin{vmatrix}α_2\\β_1\\γ\end{vmatrix}+2\begin{vmatrix}α_2\\β_2\\γ\end{vmatrix}=2×4×\frac52=20
$$



$$
\mathrm{向量组}α_{1,}α_{2,}...,α_s(s\geq2)\mathrm{线性相关的充要条件是}()
$$
$$
A.
α_{1,}α_{2,}...,α_s\mathrm{都不是零向量} \quad B.α_{1,}α_{2,}...,α_s\mathrm{中至少有一个向量可由其余向量线性表示} \quad C.α_{1,}α_{2,}...,α_s\mathrm{任意两个向量成比例} \quad D.α_{1,}α_{2,}...,α_s\mathrm{中每一个向量都可由其余向量线性表示} \quad E. \quad F. \quad G. \quad H.
$$
$$
\mathrm{由线性相关的判断定理可知}:\mathrm{向量组线性相关的充要条件是向量组中至少有一个向量可由其余向量线性表示}.
$$



$$
n\mathrm{维向量}α_1,\;α_2\;,...,α_s\mathrm{线性无关的充分条件是}()
$$
$$
A.
α_1,\;α_2\;,...,α_s\mathrm{都不是零向量} \quad B.α_1,\;α_2\;,...,α_s\mathrm{中任意两个向量都不成比例} \quad C.α_1,\;α_2\;,...,α_s\mathrm{中任一个向量都不能由其余向量线性表示} \quad D.s<\;n \quad E. \quad F. \quad G. \quad H.
$$
$$
\begin{array}{l}\mathrm{若向量组}α_1,\;α_2\;,...,α_s\mathrm{中任一个向量都不能由其余向量线性表示},\mathrm{则可推出}α_1,\;α_2\;,...,α_s\mathrm{线性无关},\mathrm{因此构成充分条件};\\\mathrm{由线性相关性的定义可知其它选项都不成立}.\end{array}
$$



$$
\mathrm{一个}n\mathrm{维向量组}α_1,\;α_2\;,...,α_s(s>1)\mathrm{线性相关的充要条件是}()
$$
$$
A.
\mathrm{含有零向量} \quad B.\mathrm{有两个向量的对应分量成比例} \quad C.\mathrm{有一个向量是其余向量的线性组合} \quad D.\mathrm{每一个向量是其余向量的线性组合} \quad E. \quad F. \quad G. \quad H.
$$
$$
\mathrm{向量组线性相关的充要条件是向量组中至少有一个向量可由其余向量线性表示},\mathrm{即至少有一个向量是其余向量的线性组合}.
$$



$$
\mathrm{对于向量组}α_1,α_2,...,α_r,\mathrm{因为有}0α_1+0α_2+...+0α_r=0,则α_1,α_2,...,α_r()
$$
$$
A.
\mathrm{全为零向量} \quad B.\mathrm{线性相关} \quad C.\mathrm{线性无关} \quad D.\mathrm{是任意的} \quad E. \quad F. \quad G. \quad H.
$$
$$
\mathrm{无论向量组}α_1,α_2,...,α_r\mathrm{是线性相关或线性无关},\mathrm{都有}0α_1+0α_2+...+0\alpha_r=0,\mathrm{因此向量组是任意的}
$$



$$
\mathrm{向量组}α_{1,}α_{2,}...,α_n\mathrm{线性无关的充分必要条件是}()
$$
$$
A.
\mathrm{向量组}α_{1,}α_{2,}...,α_n\mathrm{中任意两个向量的分量不成比例} \quad B.\mathrm{向量组}α_{1,}α_{2,}...,α_n\mathrm{都不含零向量} \quad C.\mathrm{向量组}α_{1,}α_{2,}...,α_n\mathrm{中有部分向量线性无关} \quad D.\mathrm{向量组}α_{1,}α_{2,}...,α_n\mathrm{中任一向量不能由其它}n-1\mathrm{个向量线性表示} \quad E. \quad F. \quad G. \quad H.
$$
$$
\begin{array}{l}\mathrm{向量组线性相关的充要条件是向量组中至少有一个向量能由其余向量线性表示},\mathrm{故线性无关的充要条件是任一向量}\\\mathrm{不能由其余向量线性表示}.\end{array}
$$



$$
n\mathrm{维向量组}α_{1,}α_{2,}...,α_s(3\leq s\leq n)\mathrm{线性无关的充分必要条件是}()
$$
$$
A.
\mathrm{存在一组不全为}0\mathrm{的常数}k_1,...,k_s,使k_1α_1+,...,+k_sα_s\neq0 \quad B.\mathrm{该组中任意两向量都线性无关} \quad C.\mathrm{该组中存在一个向量},\mathrm{它不能用其余向量线性表出} \quad D.\mathrm{该组中任意一个向量都不能用其余向量线性表出} \quad E. \quad F. \quad G. \quad H.
$$
$$
\begin{array}{l}\mathrm{向量组线性无关的充要条件是该组中任意一个向量都不能由其余向量线性表出};\\\;\mathrm{或由表达式}k_1α_1+,...,+k_sα_s=0\mathrm{可得}k_1=k_{2=}...k_s=0,.\end{array}
$$



$$
\mathrm{设向量组}α_1=\left(λ,1,1\right)^T,α_2=\left(1,λ,1\right)^T,α_3=\left(1,1,λ\right)^T\mathrm{线性相关},\mathrm{则必有}()
$$
$$
A.
λ=0或\;λ=1 \quad B.λ=-1或\;λ=2 \quad C.λ=1或\;λ=2 \quad D.λ=1或\;λ=-2 \quad E. \quad F. \quad G. \quad H.
$$
$$
\begin{array}{l}(α_1,\;α_2,\;α_3)=\begin{pmatrix}λ&1&1\\1&λ&1\\1&1&λ\end{pmatrix}\rightarrow\begin{pmatrix}1&1&λ\\1&λ&1\\λ&1&1\end{pmatrix}\rightarrow\begin{pmatrix}1&1&λ\\0&λ-1&1-λ\\0&1-λ&1-λ^2\end{pmatrix}\rightarrow\begin{pmatrix}1&1&λ\\0&λ-1&1-λ\\0&0&\left(λ+2\right)\left(λ-1\right)\end{pmatrix},\\\mathrm{由于向量组线性相关},\mathrm{因此矩阵}(α_1,\;α_2,\;α_3)\mathrm{的秩小于}3,\mathrm 故\left(\mathrmλ+2\right)\left(\mathrmλ-1\right)=0⇒\mathrmλ=-2\mathrm{或λ}=1.\\\mathrm{另解}:\begin{vmatrix}\mathrmλ&1&1\\1&\mathrmλ&1\\1&1&\mathrmλ\end{vmatrix}=\mathrmλ^3+2-3\mathrmλ=\mathrmλ^3-\mathrmλ-2\mathrmλ+2=\left(\mathrmλ-1\right)^2\left(\mathrmλ+2\right)=0,∴\mathrmλ=-2\mathrm{或λ}=1.\\\end{array}
$$



$$
\mathrm{向量}α=\left(k,\;1,\;1\right)^T,\;β=\left(-1,\;k,\;k\right)^T,\;γ=\left(k,\;1,\;2\right)^T,(\mathrm{其中}k∈ R)\mathrm{的关系是}\left(\right).
$$
$$
A.
\mathrm{线性相关} \quad B.\mathrm{线性无关} \quad C.γ\mathrm{可由}α,β\mathrm{线性表示} \quad D.α 是β,γ\mathrm{的线性组合} \quad E. \quad F. \quad G. \quad H.
$$
$$
\begin{vmatrix}k&-1&k\\1&k&1\\1&k&2\end{vmatrix}=k^2+1\neq0
$$



$$
设α=\left(k,\;1,\;1\right)^T,\;β=\left(-1,\;k,\;k\right)^T,\;γ=\left(k,\;1,\;2\right)^T,\;A=\left(α,\;β,\;γ\right),\mathrm{则满足向量组}\left(α,\;β,\;γ\right)\mathrm{线性相关的实数}k\mathrm{的个数为}\left(\right).\;
$$
$$
A.
0 \quad B.1 \quad C.2 \quad D.3 \quad E. \quad F. \quad G. \quad H.
$$
$$
\mathrm{满足}\begin{vmatrix}k&-1&k\\1&k&1\\1&k&2\end{vmatrix}=k^2+1=0\mathrm{的实数不存在}.
$$



$$
若α_1,\alpha_2,⋯,α_m\left(m⩾2\right)\mathrm{线性相关},\mathrm{那么向量组内}\left(\right)\mathrm{可由向量组的其余向量线性表示}.
$$
$$
A.
\mathrm{任何一个向量} \quad B.\mathrm{没有一个向量} \quad C.\mathrm{至少有一个向量} \quad D.\mathrm{至多有一个向量} \quad E. \quad F. \quad G. \quad H.
$$
$$
\mathrm{向量组线性相关的充要条件是向量组中至少有一个向量可由其余向量线性表示}.
$$



$$
若α_1,α_2,⋯,α_m是m个n\mathrm{维向量},\mathrm{则命题}“若α_1,α_2,⋯,α_m\mathrm{线性无关}”\mathrm{与命题}()\mathrm{不等价}.
$$
$$
A.
α_1,α_2,⋯,α_m\mathrm{中没有零向量} \quad B.若∑_{i=1}^mk_iα_i=0,\mathrm{则必有}k_1=k_2=⋯=k_m=0 \quad C.\mathrm{不存在不全为零的数}k_1,k_2,⋯,k_m,\mathrm{使得}∑_{i=1}^mk_iα_i=0 \quad D.\mathrm{对任意一组不全为零的数}k_1,k_2,⋯,k_m,\mathrm{必有}∑_{i=1}^mk_iα_i\neq0 \quad E. \quad F. \quad G. \quad H.
$$
$$
\begin{array}{l}\mathrm{向量组中有零向量},\mathrm{则向量组线性相关},\mathrm{但没有零向量也不一定线性无关},\\如:α_1=(1,0,0),α_2=(0.1,0),α_3=(1,1,0),\mathrm{因此向量组不含零向量与向量组线性无关不等价};\\\mathrm{由向量组线性无关的定义可知其余选项都与向量组线性无关等价}.\end{array}
$$



$$
\mathrm{再加一个向量}β 后,\mathrm{向量组}β,α_1,⋯,α_s\mathrm{线性无关是向量组}α_1,α_{2,}⋯α_s\mathrm{线性无关的}()\mathrm{条件}
$$
$$
A.
\mathrm{充分} \quad B.\mathrm{必要} \quad C.\mathrm{充分必要} \quad D.\mathrm{既不充分也不必要} \quad E. \quad F. \quad G. \quad H.
$$
$$
\mathrm{线性无关组的部分组也线性无关},\mathrm{因此向量组}β,α_1,⋯,α_s\mathrm{线性无关可推出向量组}α_1,\alpha_{2,}⋯α_s\mathrm{线性无关},\mathrm{反之不然},\mathrm{所以是充分条件}
$$



$$
α_1=\left(1,1,1\right)^T,\;α_2=\left(a,0,b\right)^T,\;α_3=\left(1,3,2\right)^T,\;若α_1,α_2,α_3\mathrm{线性相关},则a,b\mathrm{满足}().
$$
$$
A.
a=2b \quad B.a=b \quad C.a=-2b \quad D.a=-b \quad E. \quad F. \quad G. \quad H.
$$
$$
\mathrm{由线性相关性的性质可知}\left|α_1,α_2,α_3\right|=0,即\begin{vmatrix}1&a&1\\1&0&3\\1&b&2\end{vmatrix}=0⇒ a=2b
$$



$$
α_1=\left(1,1,1\right)^T,\;α_2=\left(a,0,b\right)^T,\;\alpha_3=\left(1,3,2\right)^T,\;若α_1,α_2,α_3\mathrm{线性无关},则a,b\mathrm{满足}().
$$
$$
A.
a\neq2b \quad B.a=b \quad C.a\neq-2b \quad D.a=-b \quad E. \quad F. \quad G. \quad H.
$$
$$
\mathrm{由线性相关性的性质可知}\left|α_1,α_2,\alpha_3\right|\neq0,既\begin{vmatrix}1&a&1\\1&0&3\\1&b&2\end{vmatrix}\neq0⇒ a\neq2b
$$



$$
\mathrm{向量组}α_1=\left(0,4,2\right)^T,\;α_2=\left(2,3,1\right)^T,\;α_3=\left(1-k,2,3\right)^T\mathrm{线性相关},\mathrm{则实数}k=().
$$
$$
A.
k=9 \quad B.k=6 \quad C.k=-9 \quad D.k=-6 \quad E. \quad F. \quad G. \quad H.
$$
$$
\begin{array}{l}\mathrm{因为}α_1,α_2,α_3\mathrm{线性相关},\\则\left|α_1,α_2,α_3\right|=0⇒\begin{vmatrix}0&2&1-k\\4&3&2\\2&1&3\end{vmatrix}=0⇒ k-9=0⇒ k=9.\end{array}
$$



$$
设α_1=\left(1,k,0\right)^T,\;\alpha_2=\left(0,1,k\right)^T,\;α_3=\left(k,0,1\right)^T,\left(k∈ R\right)\;,\;\mathrm{如果向量组}α_1,α_2,α_3\mathrm{线性无关},则().
$$
$$
A.
k\neq-1 \quad B.k\neq0 \quad C.k=0 \quad D.k=-1 \quad E. \quad F. \quad G. \quad H.
$$
$$
\mathrm{由条件可知}\left|α_1,α_2,α_3\right|\neq0,即\begin{vmatrix}1&0&k\\k&1&0\\0&k&1\end{vmatrix}=1+k^3\neq0⇒ k\neq-1.
$$



$$
\mathrm{已知向量组}α_1=\left(1,2,3\right)^T,\;α_2=\left(1,2,0\right)^T,\;α_3=\left(t,1,-2\right)^T\mathrm{线性相关},则().
$$
$$
A.
t=\frac12 \quad B.t=-\frac12 \quad C.t=2 \quad D.t=-2 \quad E. \quad F. \quad G. \quad H.
$$
$$
\mathrm{由条件可知}\left|α_1,α_2,α_3\right|=0,即\begin{vmatrix}1&1&t\\2&2&1\\3&0&-2\end{vmatrix}=0⇒3\left(1-2t\right)=0⇒ t=\frac12
$$



$$
\mathrm{如果}α=\left(1,2,3\right)^T,\;β=\left(3,-1,2\right)^T,\;γ=\left(2,3,m\right)^T\mathrm{线性相关},则m=().
$$
$$
A.
3 \quad B.-2 \quad C.5 \quad D.1 \quad E. \quad F. \quad G. \quad H.
$$
$$
\left|α,β,γ\right|=0,即\begin{vmatrix}1&3&2\\2&-1&3\\3&2&m\end{vmatrix}=7(m-5)=0⇒ m=5
$$



$$
\mathrm{设向量组}α_1=\left(1-t,3,0\right)^T,\;α_2=\left(0,2-t,2\right)^T,\;α_3=\left(-1-t,5,0\right)^T\mathrm{是线性相关},则().
$$
$$
A.
t=4 \quad B.t=-4 \quad C.t=-2 \quad D.t=2 \quad E. \quad F. \quad G. \quad H.
$$
$$
\mathrm{由条件可知}\left|α_1,α_2,α_3\right|=0,即\begin{vmatrix}1-t&0&-1-t\\3&2-t&5\\0&2&0\end{vmatrix}=4t-16=0⇒ t=4
$$



$$
\mathrm{向量组}α_1=\begin{pmatrix}a\\1\\1\end{pmatrix},α_2=\begin{pmatrix}1\\a\\-1\end{pmatrix},α_3=\begin{pmatrix}1\\-1\\a\end{pmatrix}\mathrm{线性相关},则a=().
$$
$$
A.
-1 \quad B.2 \quad C.-1\mathrm{或者}2 \quad D.0 \quad E. \quad F. \quad G. \quad H.
$$
$$
\begin{array}{l}∵α_1,α_2,α_3\mathrm{线性相关}\Leftrightarrow\begin{vmatrix}a&1&1\\1&a&-1\\1&-1&a\end{vmatrix}=0,\\∴ 由\begin{vmatrix}a&1&1\\1&a&-1\\1&-1&a\end{vmatrix}=\left(a+1\right)^2\left(a-2\right)=0⇒ a=2或a=-1\end{array}
$$



$$
\mathrm{已知向量组}α_1=\left(1,2,3\right)^T,\;α_2=\left(2,3,1\right)^T,\;α_3=\left(1,3,t\right)^T\mathrm{线性相关},则t=().
$$
$$
A.
8 \quad B.2 \quad C.3 \quad D.1 \quad E. \quad F. \quad G. \quad H.
$$
$$
\mathrm{由条件可知}\left|α_1,α_2,α_3\right|=\begin{vmatrix}1&2&1\\2&3&3\\3&1&t\end{vmatrix}=0⇒8-t=0⇒ t=8.
$$



$$
\mathrm{设向量组}α=\left(1,0,5\right)^T,\;β=\left(0,-7,3\right)^T,\;γ=\left(0,0,0\right)^T,\mathrm{则下列结论正确的是}().
$$
$$
A.
α,β,γ\mathrm{线性无关} \quad B.α,β,γ\mathrm{线性相关} \quad C.γ\mathrm{不能由}α,β\mathrm{线性表示} \quad D.α\mathrm{可由}β,γ\mathrm{线性表示} \quad E. \quad F. \quad G. \quad H.
$$
$$
\mathrm{显然},\;α,β\mathrm{线性无关},γ\mathrm{为零向量},\;\;α,β,γ\mathrm{线性相关}.
$$



$$
\mathrm{设向量组}α_1=\left(1,1,0\right)^T,\;α_2=\left(1,t,3\right)^T,\;α_3=\left(1,3,t\right)^T\mathrm{线性无关},则t\mathrm{的正确取值为}().
$$
$$
A.
t\neq3 \quad B.t\neq-1 \quad C.t\neq3或\;t\neq1 \quad D.t\neq3且\;t\neq-2 \quad E. \quad F. \quad G. \quad H.
$$
$$
\begin{vmatrix}1&1&1\\1&t&3\\0&3&t\end{vmatrix}=t^2-t-6=\left(t-3\right)\left(t+2\right)\neq0⇒ t\neq3且t\neq-2.
$$



$$
\mathrm{若向量组}α_1=\left(1,1,1\right)^T,\;α_2=\left(1,n,0\right)^T,\;α_3=\left(1,2,3\right)^T\mathrm{线性无关},则n\mathrm{应满足}().
$$
$$
A.
n\neq\frac12 \quad B.n=\frac12 \quad C.n=1 \quad D.n\neq1 \quad E. \quad F. \quad G. \quad H.
$$
$$
\begin{vmatrix}1&1&1\\1&n&2\\1&0&3\end{vmatrix}=2n-1\neq0⇒ n\neq\frac12
$$



$$
\mathrm{若向量组}α_1=\left(1,3,1\right)^T,\;α_2=\left(1,2,0\right)^T,\;α_3=\left(1,4,x\right)^T\mathrm{线性相关},则x=().
$$
$$
A.
0 \quad B.1 \quad C.2 \quad D.3 \quad E. \quad F. \quad G. \quad H.
$$
$$
\begin{vmatrix}1&1&1\\3&2&4\\1&0&x\end{vmatrix}=2-x=0
$$



$$
\mathrm{若向量组}α_1=\left(1,-1,1\right)^T,\;α_2=\left(2,4,a\right)^T,\;α_3=\left(-3,-3,5\right)^T\mathrm{的秩为}2,则a=().
$$
$$
A.
6 \quad B.2 \quad C.-2 \quad D.-6 \quad E. \quad F. \quad G. \quad H.
$$
$$
\begin{vmatrix}1&2&-3\\-1&4&-3\\1&a&5\end{vmatrix}=36+6a=0⇒ a=-6.
$$



$$
\mathrm{若向量组}α_1=\left(1,2,3\right)^T,\;α_2=\left(3,-1,2\right)^T,\;α_3=\left(2,3,k\right)^T\mathrm{线性相关},则k=().
$$
$$
A.
5 \quad B.6 \quad C.7 \quad D.9 \quad E. \quad F. \quad G. \quad H.
$$
$$
\begin{vmatrix}1&3&2\\2&-1&3\\3&2&k\end{vmatrix}=35-7k=0⇒ k=5
$$



$$
\mathrm{设向量组}α_1=\left(1,1,1\right)^T,\;α_2=\left(1,3,-1\right)^T,\;α_3=\left(5,3,t\right)^T\mathrm{是线性相关},则().
$$
$$
A.
t\neq7 \quad B.t=7 \quad C.t=-7 \quad D.t=-8 \quad E. \quad F. \quad G. \quad H.
$$
$$
\begin{array}{l}\mathrm{对下列矩阵施以初等行变换}\begin{pmatrix}1&1&5\\1&3&3\\1&-1&t\end{pmatrix}\rightarrow\begin{pmatrix}1&1&5\\0&2&-2\\0&-2&t-5\end{pmatrix}\rightarrow\begin{pmatrix}1&1&5\\0&1&-1\\0&0&t-7\end{pmatrix},\\当t\neq7时,\mathrm{向量组线性无关};当t=7时,\mathrm{向量组线性相关}.\end{array}
$$



$$
当t\mathrm{满足条件}()时,\mathrm{向量组}α_1=\left(1,1,0\right)^T,\;α_2=\left(1,3,-1\right)^T,\;α_3=\left(5,3,t\right)^T\mathrm{线性无关}.
$$
$$
A.
t=1 \quad B.t=-1 \quad C.t\neq1 \quad D.t\neq-1 \quad E. \quad F. \quad G. \quad H.
$$
$$
\begin{vmatrix}1&1&5\\1&3&3\\0&-1&t\end{vmatrix}=2(t-1)\neq0⇒ t\neq1
$$



$$
\mathrm{满足向量组}α_1=\left(k,1,1\right)^T,\;α_2=\left(-1,k,k\right)^T,\;α_3=\left(k,1,2\right)^T\mathrm{线性相关的实值}k\mathrm{的个数为}().
$$
$$
A.
0 \quad B.1 \quad C.2 \quad D.3 \quad E. \quad F. \quad G. \quad H.
$$
$$
\begin{vmatrix}k&-1&k\\1&k&1\\1&k&2\end{vmatrix}=k^2+1=0⇒ k=± i,\mathrm{非实数}.
$$



$$
\mathrm{满足向量组}α_1=\left(\;k,1,0\right)^T,\;α_2=\left(1,k,0\right)^T,\;α_3=\left(0,1,k\right)^T\mathrm{线性相关}k\mathrm{的的所有可能取值个数为}().
$$
$$
A.
0 \quad B.1 \quad C.2 \quad D.3 \quad E. \quad F. \quad G. \quad H.
$$
$$
\begin{vmatrix}k&1&0\\1&k&1\\0&0&k\end{vmatrix}=k(k^2-1)=0⇒ k=±1\mathrm{或者}k=0.
$$



$$
设α_1=\left(1,k,0\right)^T,\;α_2=\left(0,1,k\right)^T,\;α_3=\left(k,0,1\right)^T,\left(k∈ R\right)\;,\;\mathrm{如果向量组}α_1,α_2,α_3\mathrm{线性相关},则().
$$
$$
A.
k\neq1 \quad B.k\neq0 \quad C.k=0 \quad D.k=-1 \quad E. \quad F. \quad G. \quad H.
$$
$$
\mathrm{由条件可知}\left|α_1,α_2,α_3\right|=0,即\begin{vmatrix}1&0&k\\k&1&0\\0&k&1\end{vmatrix}=1+k^3=0⇒ k=-1.
$$



$$
\mathrm{如果向量}β\mathrm{可由向量组}α_1,α_2,...,α_s\;\mathrm{线性表示},则()
$$
$$
A.
β\mathrm{的线性表示式不唯一}; \quad B.\mathrm{存在一组不全为零的数}k_1,k_2,⋯,k_s,\mathrm{使得等式}β=k_1α_1+k_2α_2+⋯+k_sα_s\mathrm{成立}; \quad C.α_1,α_2,...,α_s\;\mathrm{中至少有一个向量可由其余}s-1\mathrm{个向量及}β\mathrm{线性表示}; \quad D.\mathrm{向量组}α_1,α_2,...,α_s,β\;\mathrm{线性相关}. \quad E. \quad F. \quad G. \quad H.
$$
$$
\mathrm{因为向量}β\mathrm{可由向量组}α_1,α_2,...,α_s\;\mathrm{线性表示},\;\mathrm{所以向量组}β,α_1,α_2,...,α_s\;\mathrm{线性相关}.
$$



$$
\mathrm{若向量组}α_1=\left(1,2,-1,1\right),α_2=\left(2,0,t,0\right),α_3=\left(0,-4,5,-2\right)\mathrm{线性相关},则t=().\;
$$
$$
A.
0 \quad B.1 \quad C.2 \quad D.3 \quad E. \quad F. \quad G. \quad H.
$$
$$
\begin{array}{l}由α_1,α_2,α_3\mathrm{线性相关}⇒ R\left(a_1,α_2,α_3\right)<3,\\\mathrm{所以}\begin{pmatrix}1&2&-1&1\\2&0&t&0\\0&-4&5&-2\end{pmatrix}\rightarrow\begin{pmatrix}1&2&-1&1\\0&-4&t+2&-2\\0&0&t-3&0\end{pmatrix},t=3\end{array}
$$



$$
\mathrm{已知向量组}\alpha_1=\left(k,2,1\right)^T,\;α_2=\left(2,k,0\right)^T,\;α_3=\left(1,-1,1\right)^T,\;\mathrm{若向量组}α_1,α_2,α_3\mathrm{线性无关},则k为().
$$
$$
A.
k\neq-2且k\neq3 \quad B.k\neq-2或k\neq3 \quad C.k\neq2且k\neq-3 \quad D.k\neq2或k\neq-3 \quad E. \quad F. \quad G. \quad H.
$$
$$
\left|α_1,α_2,α_3\right|=\begin{vmatrix}k&2&1\\2&k&-1\\1&0&1\end{vmatrix}=k^2-k-6=\left(k-3\right)\left(k+2\right)\neq0得k\neq-2且k\neq3.
$$



$$
n\mathrm{维向量组}α_1,α_2,α_3\;\mathrm{线性相关的充要条件是则}()
$$
$$
A.
\mathrm{存在唯一一组不全为零的数}k_1,k_2,k_3使k_1α_1+k_2α_2+k_3α_3=0 \quad B.n<3 \quad C.α_1,α_2,α_3\;\mathrm{中至少有一个向量能由其余向量线性表示} \quad D.α_1,α_2,α_3\mathrm{中任意两个向量均线性无关}\; \quad E. \quad F. \quad G. \quad H.
$$
$$
\mathrm{向量组线性相关的充要条件是向量组中至少有一个向量能由其余向量线性表示}
$$



$$
\mathrm{已知}n\mathrm{维向量组}α_1,α_2,...,α_m\;(m>2)\mathrm{线性无关},则()
$$
$$
A.
\mathrm{对任意一组数}k_1,k_2,⋯,k_m\mathrm{都有}k_1α_1+k_2α_2+⋯+k_mα_m=0 \quad B.m>n \quad C.\mathrm{对任意}n\mathrm{维向量}β,有α_1,α_2,...,α_m,β\mathrm{线性相关} \quad D.α_1,α_2,...,{α_m}_{}\left(m>2\right)\mathrm{中任意两个向量均线性无关} \quad E. \quad F. \quad G. \quad H.
$$
$$
\begin{array}{l}n\mathrm{维向量组}a_1,a_2,...,a_m\mathrm{线性无关},\mathrm{可得}m⩽ n,\mathrm{因为当向量组中所含向量的个数大于向量的维数时},\mathrm{此向量组必线性相关}.\\\mathrm{且根据定义可知若}k_1α_1+k_2α_2+⋯+k_mα_m=0,则\;k_1=k_2=⋯=k_m=0;\\\mathrm{线性无关组中的部分组仍线性无关},故α_1,α_2,...,α_m\mathrm{任意两个向量均线性无关}.\end{array}
$$



$$
\mathrm{已知向量组}α_1=\left(1,1,2,1\right)^T,\;α_2=\left(1,0,0,2\right)^T,\;α_3=\left(-1,-4,-8,k\right)^T\mathrm{线性相关},则k为().
$$
$$
A.
k=2 \quad B.k=-2 \quad C.k=1 \quad D.k=-1 \quad E. \quad F. \quad G. \quad H.
$$
$$
\begin{array}{l}A=\begin{pmatrix}1&1&-1\\1&0&-4\\2&0&-8\\1&2&k\end{pmatrix}⇒\begin{pmatrix}1&1&-1\\0&-1&-3\\0&-2&-6\\0&1&k+1\end{pmatrix}⇒\begin{pmatrix}1&1&-1\\0&-1&-3\\0&0&k-2\\0&0&0\end{pmatrix},\\\mathrm{可见},当k=2时,秩\left(A\right)=2<3,\mathrm{这是向量组}α_1,α_2,α_3\mathrm{才是线性相关}.\end{array}
$$



$$
\;\mathrm{设向量组}α_1=\left(6,k+1,3\right)^T,\;α_2=\left(k,2,-2\right)^T,\;α_3=\left(k,1,0\right)^T,\;若α_1,α_2,α_3\mathrm{线性相关},则k为().
$$
$$
A.
k=-4或k=\frac32 \quad B.k\neq-4或k\neq\frac32 \quad C.k=-4或k\neq\frac32 \quad D.k\neq-4或k=\frac32 \quad E. \quad F. \quad G. \quad H.
$$
$$
\begin{array}{l}\begin{array}{l}\left|A\right|=\begin{vmatrix}6&k&k\\k+1&2&1\\3&-2&0\end{vmatrix}=12-5k-2k^2=\left(k+4\right)\left(3-2k\right),\\当k=-4\;或k=\frac32时,\left|A\right|=0,α_1,α_2,α_3\mathrm{线性相关};\end{array}\\\begin{array}{lc}当k\neq-4\;且k\neq\frac32时,\left|A\right|\neq0,α_1,α_2,α_3\mathrm{线性无关};&\end{array}\end{array}
$$



$$
\;\mathrm{设向量组}α_1=\left(6,k+1,3\right)^T,\;α_2=\left(k,2,-2\right)^T,\;α_3=\left(k,1,0\right)^T,\;若α_1,α_2,α_3\mathrm{线性无关},则k为().
$$
$$
A.
k=-4或k=\frac32 \quad B.k\neq-4且k\neq\frac32 \quad C.k=-4或k\neq\frac32 \quad D.k\neq-4或k=\frac32 \quad E. \quad F. \quad G. \quad H.
$$
$$
\begin{array}{l}\begin{array}{l}\left|A\right|=\begin{vmatrix}6&k&k\\k+1&2&1\\3&-2&0\end{vmatrix}=12-5k-2k^2=\left(k+4\right)\left(3-2k\right),\\当k=-4\;或k=\frac32时,\left|A\right|=0,α_1,α_2,α_3\mathrm{线性相关};\end{array}\\\begin{array}{lc}当k\neq-4\;且k\neq\frac32时,\left|A\right|\neq0,α_1,α_2,α_3\mathrm{线性无关};&\end{array}\end{array}
$$



$$
设n\mathrm{维向量组}a_1,a_2,...,a_m\;\mathrm{线性无关},则()
$$
$$
A.
\mathrm{组中增加一个任意向量后也线性无关} \quad B.\mathrm{组中去掉一个向量后仍线性无关} \quad C.\mathrm{存在不全为}0\mathrm{的数}k_1,k_2,...,k_m,使∑_{i=1}^mk_iα_i=0 \quad D.\mathrm{组中至少一个向量可由其余向量线性表示} \quad E. \quad F. \quad G. \quad H.
$$
$$
\begin{array}{l}\mathrm{由于线性无关组中的部分组也线性无关},\mathrm{因此组中去掉一个向量后仍线性无关};\\\mathrm{若有表达式}∑_{i=1}^mk_iα_i=0,k_1=k_2=⋯=k_m=0,\mathrm{因此任何一个向量都不能由其余向量线性表示}.\end{array}
$$



$$
\mathrm{向量组}α=\left(0,4,2-t\right),β=\left(2,3-t,1\right),γ=\left(1-t,2,3\right)\mathrm{线性相关},则t=().
$$
$$
A.
6 \quad B.9 \quad C.-3 \quad D.-1 \quad E. \quad F. \quad G. \quad H.
$$
$$
\begin{array}{l}\mathrm{由于向量组线性相关},则\\A=\left|α,β,γ\right|=\begin{vmatrix}0&2&1-t\\4&3-t&2\\2-t&1&3\end{vmatrix}=t^3-6t^2+3t-18=\left(t-6\right)\left(t^2+3\right)=0,\\故t=6.\end{array}
$$



$$
\mathrm{已知}n\mathrm{维向量组}α_1,α_2,...,α_m(m>2)\;\mathrm{线性无关},则()
$$
$$
A.
\mathrm{对任意一组数}k_1,k_2,⋯,k_m\mathrm{都有}k_1α_1+k_2α_2+⋯+k_mα_m=0 \quad B.m>n \quad C.\mathrm{对任意}n\mathrm{维向量}β,有α_1,α_2,...,α_m,β\mathrm{线性相关} \quad D.α_1,α_2,...,α_m\left(m>2\right)\mathrm{中任意两个向量均线性无关} \quad E. \quad F. \quad G. \quad H.
$$
$$
\begin{array}{l}n\mathrm{维向量组}α_1,α_2,...,α_m\mathrm{线性无关},\mathrm{可得}m⩽ n,\mathrm{因为当向量组中所含向量的个数大于向量的维数时},\mathrm{此向量组必线性相关}.\\\mathrm{且根据定义可知若}k_1α_1+k_2α_2+⋯+k_mα_m=0,则\;k_1=k_2=⋯=k_n=0;\\\mathrm{线性无关组中的部分组仍线性无关},故α_1,α_2,...,α_m\mathrm{任意两个向量均线性无关}.\end{array}
$$



$$
\mathrm{向量组}I:a_1,...,a_s(s⩾3)\mathrm{线性相关的必要条件是}().
$$
$$
A.
I\mathrm{中每个向量都可以用其余的向量线性表出} \quad B.I\mathrm{中至少有一个向量可用其余的向量线性表出} \quad C.I\mathrm{中只有一个向量能用其余的向量线性表出} \quad D.I\mathrm{的任何部分组都线性相关} \quad E. \quad F. \quad G. \quad H.
$$
$$
\begin{array}{l}若A⇒ B,则B是A\mathrm{的必要条件},A是B\mathrm{的充分条件}.\\\mathrm{向量组}I\mathrm{线性相关}⇒ I\mathrm{中至少有一个向量可用其余向量线性表出},\mathrm{满足答案}.\\I\mathrm{的任何部分组线性相关}⇒\mathrm{向量组}I\mathrm{线性相关},\mathrm{是充分条件}.\end{array}
$$



$$
设n\mathrm{维向量组}α_1,α_2,...,α_m\;\mathrm{线性无关},则()
$$
$$
A.
\mathrm{组中增加一个任意向量后也线性无关} \quad B.\mathrm{组中去掉一个向量后仍线性无关} \quad C.\mathrm{存在不全为}0\mathrm{的数}k_1,k_2,...,k_m,使∑_{i=1}^mk_iα_i=0 \quad D.\mathrm{组中至少一个向量可由其余向量线性表示} \quad E. \quad F. \quad G. \quad H.
$$
$$
\begin{array}{l}\mathrm{由于线性无关组中的部分组也线性无关},\mathrm{因此组中去掉一个向量后仍线性无关};\\\mathrm{若有表达式}∑_{i=1}^mk_iα_i=0,k_1=k_2=⋯=k_m=0,\mathrm{因此任何一个向量都不能由其余向量线性表示}.\end{array}
$$



$$
设A是n\mathrm{阶矩阵},且A\mathrm{的行列式}\left|A\right|=0,则A中().\;
$$
$$
A.
\mathrm{必有一列元素为零} \quad B.\mathrm{必有两列元素对应成比例} \quad C.\mathrm{必有一列向量是其余列向量的线性组合} \quad D.\mathrm{任一列向量是其余列向量的线性组合} \quad E. \quad F. \quad G. \quad H.
$$
$$
由\left|A\right|=0\mathrm{可知},\mathrm{矩阵}A\mathrm{的行向量或列向量都线性相关},\mathrm{即必有一列}(行)\mathrm{向量是其余列}(行)\mathrm{向量的线性组合}.
$$



$$
\mathrm{下列命题中},\mathrm{正确的是}().
$$
$$
A.
\mathrm{任意}n个n+1\mathrm{维向量线性相关} \quad B.\mathrm{任意}n个n+1\mathrm{维向量线性无关} \quad C.\mathrm{任意}n+1个n\mathrm{维向量线性相关} \quad D.\mathrm{任意}n+1个n\mathrm{维向量线性无关} \quad E. \quad F. \quad G. \quad H.
$$
$$
\begin{array}{l}\mathrm{当向量组中所含向量的个数大于向量的维数时},\mathrm{此向量组必线性相关},\mathrm{故任意}n+1个n\mathrm{维向量线性相关};\\\mathrm{任意}n个n+1\mathrm{维向量可线性相关也可线性无关}.\end{array}
$$



$$
\begin{array}{l}\mathrm{下列向量组中线性无关的个数有}().\\(1)α_1=\left(0,0,0\right)^T,\;α_2=\left(-2,2,0\right)^T,\;α_3=\left(3,-5,2\right)^T\\(2)α_1=\left(1,1,1\right)^T,\;α_2=\left(3,-1,2\right)^T,\;α_3=\left(2,2,-1\right)^T\\(3)α_1=\left(1,0,0,2,5\right)^T,\;α_2=\left(0,1,0,3,4\right)^T,\;α_3=\left(0,0,1,4,7\right)^T,\;α_4=\left(0,0,0,1,12\right)^T\end{array}
$$
$$
A.
0个 \quad B.1个 \quad C.2个 \quad D.3个 \quad E. \quad F. \quad G. \quad H.
$$
$$
\begin{array}{l}\begin{array}{l}\begin{array}{l}(1)\mathrm{因为}α_1\mathrm{为零向量},\mathrm{所以}α_1,α_2,α_3\mathrm{线性相关}.\\(2)\left|α_1,α_2,α_3\right|=\begin{vmatrix}1&3&2\\1&-1&2\\1&2&-1\end{vmatrix}=12\neq0,\mathrm{所以}α_1,α_2,α_3\mathrm{线性无关}\\(3)β_1=\left(1,0,0,2\right)^T,β_2=\left(0,1,0,3\right)^T,β_3=\left(0,0,1,4\right)^T,β_4=\left(0,0,0,1\right)^T,\left|β_{1,}β_2,β_3,β_4\right|=1\neq0,\end{array}\\\mathrm{所以}β_1,β_2,β_3,β_4\mathrm{线性无关},而α_1,α_2,α_3,α_4\mathrm{是在}β_1,β_2,β_3,β_4\mathrm{分别添加序号为}4\mathrm{的分量得到的向量组},\end{array}\\\mathrm{所以}α_1,α_2,α_3,α_4\mathrm{线性无关}.\\故(2)(3)\mathrm{线性无关}.\\\end{array}
$$



$$
设α_1=\left(1,-1,1,0\right)^T,\;α_2=\left(1,1,-1,0\right)^T,\;α_3=\left(-1,1,1,λ\right)^T,\;则α_1,α_2,α_3().
$$
$$
A.
\mathrm{必相互正交} \quad B.\mathrm{必线性无关} \quad C.\mathrm{必线性相关} \quad D.\mathrm{相关与否与}λ\mathrm{的值有关} \quad E. \quad F. \quad G. \quad H.
$$
$$
\begin{array}{l}\left(α_1,α_2,α_3\right)=\begin{pmatrix}1&1&-1\\-1&1&1\\1&-1&1\\0&0&λ\end{pmatrix}\rightarrow\begin{pmatrix}1&1&-1\\0&2&0\\0&-2&2\\0&0&λ\end{pmatrix}\rightarrow\begin{pmatrix}1&1&-1\\0&1&0\\0&0&1\\0&0&λ\end{pmatrix},\\\mathrm{由于}R\left(α_1,α_2,α_3\right)=3,\mathrm{故向量组线性无关}.\\\mathrm{由于}\left\langleα_1,α_3\right\rangle=-1\neq0,\mathrm{所以}α_1,α_3\mathrm{不正交}.\end{array}
$$



$$
设n\mathrm{维向量组}α_1,α_2,⋯,α_m\mathrm{线性相关},\mathrm{则组中}\left(\right).
$$
$$
A.
\mathrm{任一向量可由其余向量线性表示} \quad B.\mathrm{某一向量可由其余向量线性表示} \quad C.\mathrm{去掉任一向量之后},\mathrm{仍线性相关} \quad D.\mathrm{添上一向量之后},\mathrm{就会线性无关} \quad E. \quad F. \quad G. \quad H.
$$
$$
\begin{array}{l}\mathrm{若向量组中的部分组线性相关},\mathrm{则整个向量组线性相关},\mathrm{因此题中向量组添上一个向量后仍线性相关};\\\mathrm{且向量组线性相关的充要条件是至少存在一个向量可由其余向量线性表示}.\end{array}
$$



$$
\mathrm{设矩阵}A_{m× n}\mathrm{的秩为}R\left(A\right)=m<\;n,E_m为m\mathrm{阶单位矩阵},\mathrm{下述结论中},\mathrm{正确的是}().\;
$$
$$
A.
A\mathrm{的任意个}m\mathrm{列向量必线性无关} \quad B.A\mathrm{的任一个}m\mathrm{阶子式不等于零} \quad C.A\mathrm{通过初等行变换必可化为}\left(E_m,0\right)\mathrm{的形式} \quad D.A\mathrm{的列向量组中存在}m\mathrm{个向量线性无关} \quad E. \quad F. \quad G. \quad H.
$$
$$
\begin{array}{l}\mathrm{矩阵}A_{m× n}\mathrm{的秩为}R\left(A\right)=m<\;n,\mathrm{根据矩阵秩的定义可知}A\mathrm{存在一个}m\mathrm{阶子式不为零},\\即A\mathrm{的列向量组中存在}m\mathrm{个向量线性无关},\mathrm{通过初等变换化为标准形}\left(E_m,0\right)\mathrm{的形式}.\\\end{array}
$$



$$
\begin{array}{l}\mathrm{向量组}β_1=(1,1,0)^T,β_2=(1,1,1)^T,β_3=(2,a,b)^T与α_1=(0,1,1)^T,α_2=(1,2,1)^T,α_3=(1,0,-1)^T\mathrm{的秩相同},\\\text{且}β_3\text{可}由α_1,α_2,α_3\mathrm{线性表示},则a,b\mathrm{的值为}(\;).\;\end{array}
$$
$$
A.
a=2,b=0 \quad B.a=0,b=2 \quad C.a=-2,b=0 \quad D.a=0,b=-2 \quad E. \quad F. \quad G. \quad H.
$$
$$
\begin{array}{l}线α_1,α_2\mathrm{性无关},而α_3=α_2-2α_1,\mathrm{所以向量组}α_1,α_2,α_3\mathrm{的秩为}2,\text{且}α_1,α_2\mathrm{是一个最大无关组}.\\\mathrm{由于向量组}β_1,β_2,β_3\text{与}α_1,α_2,α_3\mathrm{的秩相同},\mathrm{即秩为}2,\mathrm{所以}β_1,β_2,β_3\mathrm{线性相关},\text{即}\\\begin{vmatrix}1&1&2\\1&1&a\\0&1&b\end{vmatrix}=2-a=0,\mathrm{于是}a=2,\mathrm{从而}β_3=(2,2,b)^T.\\又β_3\mathrm{可由}α_1,α_2,α_3\mathrm{线性表示},\mathrm{从而可由}α_1,α_2\mathrm{线性表示},\mathrm{所以}α_1,α_2,β_3\mathrm{线性相关},\text{即}\\\begin{vmatrix}0&1&2\\1&2&2\\1&1&b\end{vmatrix}=-b=0\;,\mathrm{于是}b=0,\mathrm{从而}β_3=(2,2,0)^T.\end{array}
$$



$$
\mathrm{向量组}α_1=\left(1,1,2,1\right)^T,\;α_2=\left(1,0,0,2\right)^T,\;α_3=\left(-1,-4,-8,k\right)^T\mathrm{线性相关},则k=().
$$
$$
A.
k=2 \quad B.k=-2 \quad C.k=1 \quad D.k=-1 \quad E. \quad F. \quad G. \quad H.
$$
$$
\begin{array}{l}\begin{array}{l}A=\begin{pmatrix}1&1&-1\\1&0&-4\\2&0&-8\\1&2&k\end{pmatrix}⇒\begin{pmatrix}1&1&-1\\0&-1&-3\\0&-2&-6\\0&1&k+1\end{pmatrix}⇒\begin{pmatrix}1&1&-1\\0&-1&-3\\0&0&k-2\\0&0&0\end{pmatrix},\\\mathrm{可见},当k=2时,秩\left(A\right)=2<3,\mathrm{这时向量组}α_1,α_2,α_3\mathrm{线性相关的}.\end{array}\\\end{array}
$$



$$
\mathrm{若向量组}α_1=\left(1,0,2,3\right)^T,\;α_2=\left(1,1,3,5\right)^T,\;α_3=\left(1,-1,a+2,1\right)^T\;α_4=\left(1,2,4,a+9\right)^T\mathrm{线性相关},\;则a\mathrm{的值}().
$$
$$
A.
a=-1或a=-2 \quad B.a\neq-1或a=-2 \quad C.a=-1或a\neq-2 \quad D.a\neq-1且a\neq-2 \quad E. \quad F. \quad G. \quad H.
$$
$$
\begin{array}{l}\;\mathrm{因为}\begin{vmatrix}1&1&1&1\\0&1&-1&2\\2&3&a+2&4\\3&5&1&a+9\end{vmatrix}=\left(a+1\right)\left(a+2\right),\\\mathrm{所有}a=-1或a=-2时,\mathrm{向量组线性先关},\mathrm{否则线性无关}.\end{array}
$$



$$
若α_1,α_2,⋯,α_m是m个n\mathrm{维向量},\mathrm{则命题}“若α_1,α_2,⋯,α_m\mathrm{线性无关}”\mathrm{与命题}()\mathrm{不等价}.
$$
$$
A.
α_1,α_2,⋯,α_m\mathrm{中没有零向量} \quad B.若∑_{i=1}^mk_ia_i=0,\mathrm{则必有}k_1=k_2=⋯=k_m=0 \quad C.\mathrm{不存在不全为零的数}k_1,k_2,⋯,k_m,\mathrm{使得}∑_{i=1}^mk_ia_i=0 \quad D.\mathrm{对任意一组不全为零的数}k_1,k_2,⋯,k_m,\mathrm{必有}∑_{i=1}^mk_ia_i\neq0 \quad E. \quad F. \quad G. \quad H.
$$
$$
\begin{array}{l}\mathrm{向量组中有零向量},\mathrm{则向量组线性相关},\mathrm{但没有零向量也不一定线性无关},\\如:α_1=(1,0,0),α_2=(0.1,0),α_3=(1,1,0),\mathrm{因此向量组不含零向量与向量组线性无关不等价};\\\mathrm{由向量组线性无关的定义可知其余选项都与向量组线性无关等价}.\end{array}
$$



$$
\mathrm{设有}m\mathrm{维列向量组}α_1,α_2,⋯,α_n\mathrm{线性相关},\mathrm{则列向量组}α_1,α_2,⋯α_n().
$$
$$
A.
\;当m<\;n时,\mathrm{一定线性相关} \quad B.\;当m<\;n时,\mathrm{一定线性无关} \quad C.\;当m>n时,\mathrm{一定线性相关} \quad D.\;当m>n时,\mathrm{一定线性无关} \quad E. \quad F. \quad G. \quad H.
$$
$$
\mathrm{由线性相关定义},当m<\;n时,\mathrm{向量个数即矩阵列数多于行数},n\mathrm{个向量一定线性相关}.
$$



$$
\begin{array}{l}\begin{array}{l}\mathrm{下列向量组中线性相关的个数有}().\\(1)α_1=\left(2,1,-3,-2\right)^T,\;α_2=\left(-4,-2,6,4\right)^T\\(2)α_1=\left(1,0,5\right)^T,\;α_2=\left(0,3,2\right)^T,\;\alpha_3=\left(0,0,0\right)^T\\(3)α_1=\left(1,2,1\right)^T,\;α_2=\left(2,1,3\right)^T,\;α_3=\left(1,0,1\right)^T\end{array}\\\;(4)α_1=\left(1,2,1,0\right)^T,α_2=\left(2,1,3,-5\right)^T,α_3=\left(1,0,1,2\right)^T\end{array}
$$
$$
A.
1 \quad B.2 \quad C.3 \quad D.4 \quad E. \quad F. \quad G. \quad H.
$$
$$
(1)(2)\mathrm{线性相关},(3)(4)\mathrm{线性无关}
$$



$$
\begin{array}{l}\mathrm{下列向量组中线性无关的向量组个数为}().\\(1)α=\left(1,2,1\right)^T,\;β=\left(2,1,3\right)^T,\;γ=\left(1,1,1\right)^T\\(2)α=\left(1,0,2\right)^T,\;β=\left(0,1,2\right)^T,\;γ=\left(1,2,0\right)^T\\(3)α=\left(1,1,1,0\right)^T,\;β=\left(1,1,0,1\right)^T,\;γ=\left(1,0,1,1\right)^T\end{array}
$$
$$
A.
3 \quad B.2 \quad C.1 \quad D.0 \quad E. \quad F. \quad G. \quad H.
$$
$$
(1)(2)\mathrm{用行列式},(3)\mathrm{用初等变换},(1)(2)(3)\mathrm{均线性无关}.
$$



$$
\begin{array}{l}\mathrm{下列向量组中线性无关的向量组个数为}().\\(1)α=\left(1,0,1\right)^T,\;β=\left(1,1,0\right)^T,\;γ=\left(1,1,1\right)^T\\(2)α=\left(1,0,0,0\right)^T,\;β=\left(1,1,0,0\right)^T,\;γ=\left(1,1,1,0\right)^T\\(3)α=\left(1,2,3,4\right)^T,\;β=\left(2,3,4,5\right)^T,\;γ=\left(3,4,5,6\right)^T\end{array}
$$
$$
A.
1 \quad B.2 \quad C.3 \quad D.0 \quad E. \quad F. \quad G. \quad H.
$$
$$
(1)(2)\;\mathrm{线性无关},(3)\mathrm{线性相关}
$$



$$
\begin{array}{lc}\begin{array}{l}\mathrm{下列向量组中},\mathrm{线性相关的有}().\\(1)α_1=\left(1,0,2,0\right)^T,\;α_2=\left(0,1,-1,0\right)^T,α_3=\left(1,0,2,1\right)^T\\(2)α_1=\left(1,3,1\right)^T,\;α_2=\left(0,-7,3\right)^T,\;α_3=\left(-1,2,5\right)^T,α_4=\left(1,1,1\right)^T\\(3)α_1=\left(1,0,2,-1\right)^T,\;α_2=\left(3,0,6,-3\right)^T,\;\alpha_3=\left(2,1,-5,3\right)^T,\;α_4=(9,2,0,-6)^{{}^T}\end{array}&\;\end{array}
$$
$$
A.
(1) \quad B.(1)(2) \quad C.(2)(3) \quad D.(1)(3) \quad E. \quad F. \quad G. \quad H.
$$
$$
\begin{array}{l}\mathrm{向量组}(1)\mathrm{有三阶非零子式},\mathrm{其秩为}3,\mathrm{列满秩},\mathrm{所以}(1)\mathrm{线性无关};\\\mathrm{向量组}(2)\mathrm{向量个数大于维数},(2)\mathrm{线性相关};\\\mathrm{向量组}(3){\mathrm{中α}}_1,{\mathrmα}_2\mathrm{的分量对应成比例},{\mathrm{所以α}}_1,{\mathrmα}_2\mathrm{线性相关},\mathrm 故(3)\mathrm{线性相关}.\end{array}
$$



$$
\begin{array}{l}\mathrm{下列向量组中线性无关的向量组个数为}().\\(1)α=\left(1,1,1\right)^T,\;β=\left(1,2,3\right)^T,\;γ=\left(2,3,4\right)^T\\(2)α=\left(1,0,0,1\right)^T,\;β=\left(0,1,0,2\right)^T,\;γ=\left(0,0,1,3\right)^T\\(3)α=\left(0,0,1\right)^T,\;β=\left(1,1,1\right)^T,\;γ=\left(1,2,3\right)^T\end{array}
$$
$$
A.
1 \quad B.2 \quad C.3 \quad D.0 \quad E. \quad F. \quad G. \quad H.
$$
$$
(1)(3)\mathrm{用行列式},(2)\mathrm{看前三个分量即可},(2)(3)\;\mathrm{两个向量组线性无关}.
$$



$$
\begin{array}{l}\mathrm{下列向量组中},\mathrm{线性相关的为}().\\(1)α=\left(1,1,1\right)^T,\;β=\left(1,2,3\right)^T,\;γ=\left(1,3,4\right)^T\\(2)α=\left(1,1,2,3\right)^T,\;β=\left(1,2,3,4\right)^T,\;γ=\left(2,3,4,5\right)^T\\(3)α=\left(1,2,3,4\right)^T,\;β=\left(2,3,4,5\right)^T,\;γ=\left(3,4,5,6\right)^T\end{array}
$$
$$
A.
(3) \quad B.(2) \quad C.(2)(3) \quad D.(1)(3) \quad E. \quad F. \quad G. \quad H.
$$
$$
(1)(2)\mathrm{线性无关},(3)\mathrm{线性相关},\mathrm{使用初等行变换}.
$$



$$
\begin{array}{l}\mathrm{下列向量组中},\mathrm{线性无关的为}().\\(1)α=\left(1,1,1\right)^T,\;β=\left(1,2,3\right)^T,\;γ=\left(1,3,4\right)^T\\(2)α=\left(1,1,2,3\right)^T,\;β=\left(1,2,3,4\right)^T,\;γ=\left(2,3,4,5\right)^T\\(3)α=\left(1,2,3,4\right)^T,\;β=\left(2,3,4,5\right)^T,\;γ=\left(3,4,5,6\right)^T\end{array}
$$
$$
A.
(1)(3) \quad B.(1)(2)(3) \quad C.(2)(3) \quad D.(1)(2) \quad E. \quad F. \quad G. \quad H.
$$
$$
(1)(2)\mathrm{线性无关},(3)\mathrm{线性相关},\mathrm{使用初等行变换}.
$$



$$
\begin{array}{l}\mathrm{下列向量组中},\mathrm{线性无关的向量组有}().\\(1)α=\left(1,0,0,2,1\right)^T,\;β=\left(0,1,0,3,0\right)^T,\;γ=\left(0,0,1,5,6\right)^T\\(2)α=\left(1,1,2\right)^T,\;β=\left(1,3,3\right)^T,\;γ=\left(2,2,1\right)^T,δ=\left(3,1,1\right)^T\\(3)α=\left(1,2,3,\right)^T,\;β=\left(2,2,4\right)^T,\;γ=\left(3,1,3\right)^T\end{array}
$$
$$
A.
(1) \quad B.(3) \quad C.(1)(3) \quad D.(1)(2)(3) \quad E. \quad F. \quad G. \quad H.
$$
$$
\begin{array}{l}(1)\mathrm{看前三个分量知其线性无关};\\(2)\mathrm{向量个数大于向量维数},\mathrm{线性相关};\\(3)\mathrm{向量组成的矩阵之行列式不等于零},\mathrm{线性无关}.\end{array}
$$



$$
设A是n\mathrm{阶矩阵},且R(A)=r,\mathrm{则下列结论正确的是}(\;\;\;)
$$
$$
A.
若\vert A\vert=0,则r\leq n; \quad B.若r<\;n,则A\mathrm{中任意列向量是其余列向量的线性组合}; \quad C.A\mathrm{中任意}r\mathrm{个行向量都线性无关}; \quad D.若\vert A\vert=0,则A\mathrm{中任意}r+1\mathrm{个行向量都线性相关}. \quad E. \quad F. \quad G. \quad H.
$$
$$
若\vert A\vert=0,则R(A)=r\;<\;n,\;\mathrm{所以}A\mathrm{中任意}r+1\mathrm{个行向量都线性相关}.
$$



$$
\mathrm{设向量组}α_1=\left[0,0,1,k\right]^T,\;α_2=\left[0,k,1,0\right]^T,\;α_3=\left[1,1,0,0\right]^T,\;α_4=\left[k,0,0,1\right]^T,\mathrm{若向量组}α_1,α_2,α_3,α_4\mathrm{线性无关},则k\mathrm{的值为}().
$$
$$
A.
k\neq0且k\neq1 \quad B.k=0或k=1 \quad C.k=0 \quad D.k\neq1 \quad E. \quad F. \quad G. \quad H.
$$
$$
\begin{array}{l}4\mathrm{维向量组}α_1,α_2,α_3,α_4\mathrm{相性无关当且}4\mathrm{阶行列式}\left|α_1,α_2,α_3,α_4\right|\neq0.\;而\\\left|α_1,α_2,α_3,α_4\right|=\begin{vmatrix}0&0&1&k\\0&k&1&0\\1&1&0&0\\k&0&0&1\end{vmatrix}=-\begin{vmatrix}1&1&0&0\\0&k&1&0\\0&0&1&k\\k&0&0&1\end{vmatrix}=-\begin{vmatrix}1&1&0&0\\0&k&1&0\\0&0&1&k\\0&-k&0&1\end{vmatrix}\\\;\;\;\;\;\;\;\;\;=-\begin{vmatrix}1&1&0&0\\0&k&1&0\\0&0&1&k\\0&0&1&1\end{vmatrix}=-\begin{vmatrix}1&1&0&0\\0&k&1&0\\0&0&1&k\\0&0&0&1-k\end{vmatrix}=k\left(k-1\right).\\\mathrm{所以},\mathrm{当且仅当}k\neq0\mathrm{而且}k\neq1时,\left|α_1,α_2,α_3,α_4\right|\neq0.\;\;\mathrm{此时向量组}α_1,α_2,α_3,α_4\mathrm{相性无关}.\\\\\\\end{array}
$$



$$
\begin{array}{l}\mathrm{若向量组}α_1=\left(1,2,-1,1\right),\;α_2=\left(2,0,t,0\right),\;α_3=\left(0,-4,5,-2\right)\mathrm{线性相关},\;则t=().\\\mathrm{若向量组}β_1=\left(1,2,-1,1\right)^T,\;β_2=\left(2,0,t,0\right)^T,\;β_3=\left(0,-4,5,-2\right)^T\mathrm{线性相关},\;则t=().\end{array}
$$
$$
A.
3,2 \quad B.3,3 \quad C.2,3 \quad D.2,-3 \quad E. \quad F. \quad G. \quad H.
$$
$$
\begin{array}{l}由α_1,α_2,α_3\mathrm{线性相关}⇒ R\left(α_1,α_2,α_3\right)<\;3⇒ R\left(α_1,α_2,α_3\right)=2.\\\mathrm{所以},\begin{pmatrix}1&2&-1&1\\2&0&t&0\\0&-4&5&-2\end{pmatrix}\rightarrow\begin{pmatrix}1&2&-1&1\\0&-4&t+2&-2\\0&0&t-3&0\end{pmatrix}⇒ t=3;\\\begin{pmatrix}1&2&0\\2&0&-4\\-1&t&5\\1&0&-2\end{pmatrix}\rightarrow\begin{pmatrix}1&2&0\\0&1&1\\0&t+2&5\\0&0&0\end{pmatrix}\rightarrow\begin{pmatrix}1&2&0\\0&1&1\\0&t-3&0\\0&0&0\end{pmatrix}⇒ t=3.\end{array}
$$



$$
\mathrm{已知}A是m× n\mathrm{型矩阵},β 是n\mathrm{维列向量},\mathrm{下列结论正确的是}(\;\;)\;
$$
$$
A.
若m<\;n,则A\mathrm{的列向量组线性无关}; \quad B.若m<\;n,则A\mathrm{的列向量组线性相关}; \quad C.若A\mathrm{的列向量组线性相关},\mathrm{则矩阵}\begin{pmatrix}A\\β^T\end{pmatrix}\mathrm{的列向量组也线性相关}; \quad D.\;\mathrm{若矩阵}\begin{pmatrix}A\\β^T\end{pmatrix}\mathrm{的列向量组线性无关},则A\mathrm{的列向量组也线性无关}. \quad E. \quad F. \quad G. \quad H.
$$
$$
若m<\;n,则A\mathrm{的列向量组所含向量的个数大于其向量的维数},\mathrm{所以}A\mathrm{的列向量组线性相关}.
$$



$$
设A是m× n\mathrm{型矩阵},B是n× m\mathrm{型矩阵},E是n\mathrm{阶单位矩阵},m>\;n,\mathrm{已知}BA=E,\mathrm{下列结论正确的是}(\;\;)
$$
$$
A.
A\;\mathrm{是可逆矩阵}; \quad B.A\mathrm{的秩}R(A)<\;n; \quad C.A\mathrm{的列向量组线性无关}; \quad D.A\mathrm{的列向量组线性相关}. \quad E. \quad F. \quad G. \quad H.
$$
$$
\mathrm{因为}n=R\left(E\right)\leq R\left(A\right)\leq n,\mathrm{所以}R\left(A\right)=n,故A\mathrm{的列向量组线性无关}.
$$



$$
\mathrm{已知}A是m× n\mathrm{型矩阵},β 是n\mathrm{维列向量},\mathrm{下列结论正确的是}(\;\;)\;
$$
$$
A.
若m<\;n,则A\mathrm{的列向量组线性无关}; \quad B.若m=\;n,则A\mathrm{的列向量组线性相关}; \quad C.若A\mathrm{的列向量组线性相关},\mathrm{则矩阵}\begin{pmatrix}A\\β^T\end{pmatrix}\mathrm{的列向量组也线性相关}; \quad D.\;\mathrm{若矩阵}\begin{pmatrix}A\\β^T\end{pmatrix}\mathrm{的列向量组线性相关},则A\mathrm{的列向量组线性无关}. \quad E. \quad F. \quad G. \quad H.
$$
$$
\;\mathrm{因为}\begin{pmatrix}A\\β^T\end{pmatrix}\mathrm{的列向量组是}m+1\mathrm{维的},\mathrm{若其线性相关},\mathrm{则去掉一个分量后},\mathrm{得到的}A\mathrm{的列向量组也线性相关}.
$$



$$
\mathrm{设向量组}α_1=\left(1,1,1,3\right)^T,\;α_2=\left(-1,-3,5,1\right)^T,\;α_3=\left(3,2,-1,p+2\right)^T,α_4=\left(-2,,-6,10,p\right)^T.\mathrm{若该向量组线性无关},则p\mathrm{的值为}(\;).\;
$$
$$
A.
p\neq2 \quad B.p=2 \quad C.p=1 \quad D.p\neq1 \quad E. \quad F. \quad G. \quad H.
$$
$$
\begin{array}{l}\mathrm{对矩阵}\left(α_1,α_2,α_3,α_4\right)\mathrm{作初等行变换}\\A=\begin{pmatrix}1&-1&3&-2\\1&-3&2&-6\\1&5&-1&10\\3&1&p+2&p\end{pmatrix}\rightarrow\begin{pmatrix}1&-1&3&-2\\0&-2&-1&-4\\0&6&-4&12\\0&4&p-7&p+6\end{pmatrix}\\\;\;\rightarrow\begin{pmatrix}1&-1&3&-2\\0&-2&-1&-4\\0&0&-7&0\\0&0&p-9&p-2\end{pmatrix}\rightarrow\begin{pmatrix}1&-1&3&-2\\0&-2&-1&-4\\0&0&1&0\\0&0&0&p-2\end{pmatrix}\\当p\neq2时,\mathrm{向量组}α_1,α_2,α_3,α_4\mathrm{线性无关};当p=2时,\mathrm{向量组}α_1,α_2,α_3,α_4\mathrm{相关}.\end{array}
$$



$$
设α_1=\begin{pmatrix}b\\b+1\\2b+1\end{pmatrix},\;α_2=\begin{pmatrix}1\\b+1\\3\end{pmatrix},\;\alpha_3=\begin{pmatrix}1\\2\\b+2\end{pmatrix}\mathrm{线性相关}\;,\;\mathrm{且向量}β=\begin{pmatrix}1\\2\\3\end{pmatrix}\mathrm{可由}α_1,α_2,α_3\mathrm{线性表示},则b为().
$$
$$
A.
b=-1 \quad B.b=-2或b=1 \quad C.b=-2 \quad D.b=1 \quad E. \quad F. \quad G. \quad H.
$$
$$
\begin{array}{l}记A=\left(α_1,α_2,α_3\right)\mathrm{则有}α_1,α_2,α_3\mathrm{线性相关},知\\\left|A\right|=\begin{vmatrix}b&1&1\\b+1&b+1&2\\2b+1&3&b+2\end{vmatrix}=\begin{vmatrix}b&1&1\\1&b&1\\1&1&b\end{vmatrix}=\left(b+2\right)\left(b-1\right)^2=0.\\当b=-2时,\\\left(A\vdotsβ\right)=\left(α_1,α_2,α_3\vdotsβ\right)=\begin{pmatrix}-2&1&1&\vdots&1\\-1&-1&2&\vdots&2\\-3&3&0&\vdots&3\end{pmatrix}\xrightarrow 行\begin{pmatrix}1&0&-1&\vdots&0\\0&1&-1&\vdots&0\\0&0&0&\vdots&1\end{pmatrix},\\故β\mathrm{不能由}α_1,α_2,α_3\mathrm{线性表示}.\\当b=1时,\\\left(A\vdotsβ\right)=\left(α_1,α_2,α_3\vdotsβ\right)=\begin{pmatrix}1&1&1&\vdots&1\\2&2&2&\vdots&2\\3&3&3&\vdots&3\end{pmatrix}\xrightarrow 行\begin{pmatrix}1&1&1&\vdots&1\\0&0&0&\vdots&0\\0&0&0&\vdots&0\end{pmatrix},\\故β\mathrm{可由}α_1,α_2,α_3\mathrm{线性表示},故b=1.\\\end{array}
$$



$$
\mathrm{设向量组}α_1,α_2,α_3\mathrm{线性无关},\mathrm{向量组}α_2,α_3,α_4\mathrm{线性相关},则().
$$
$$
A.
α_4\mathrm{不能被}α_1,α_2,α_3\mathrm{线性表示}\; \quad B.α_4\mathrm{必能被}α_1,α_2,α_3\mathrm{线性表示}\; \quad C.α_1\mathrm{可能被}α_2,α_3,α_4\mathrm{线性表示}\; \quad D.α_1\mathrm{必能被}α_2,α_3,α_4\mathrm{线性表示}\; \quad E. \quad F. \quad G. \quad H.
$$
$$
\begin{array}{l}α_1,α_2,α_3\mathrm{线性无关},则α_2,α_3\mathrm{线性无关},又α_2,α_3,α_4\mathrm{线性相关},故α_4\mathrm{可由}α_2,α_3\mathrm{唯一线性表示},则α_4\mathrm{必能被}α_1,α_2,α_3\mathrm{线性表示}.\\\mathrm{不能确定}α_1\mathrm{是否能被}α_2,α_3,α_4\mathrm{线性表示}.\end{array}
$$



$$
\mathrm{向量组}α_i=\left(x_i,y_i,z_i\right)^T(i=1,2,3)\mathrm{线性相关},\mathrm{且该向量组的秩为}2\mathrm{的充要条件为}.
$$
$$
A.
\mathrm{三个向量中存在不共线的两向量} \quad B.\mathrm{三向量共面} \quad C.\mathrm{三向量共面},\mathrm{且至少有两向量不共线} \quad D.\mathrm{三向量相互平行} \quad E. \quad F. \quad G. \quad H.
$$
$$
\;\mathrm{三向量共面},\mathrm{且至少有两向量不共线等价于}A(α_1,α_2,α_3)=2.
$$



$$
\begin{array}{l}\begin{array}{lc}\begin{array}{l}\mathrm{下列向量组中},\mathrm{线性相关的个数为}().\\(1)α_1=\left(1,2,3,-1\right)^T,\;α_2=\left(3,2,1,-1\right)^T,α_3=\left(2,3,1,1\right)^T,α_4=(2,2,2,-1)^T,α_5=(5,5,2,0)^T\\(2)α_1=\left(3,1,2,-4\right)^T,\;α_2=\left(1,0,5,2\right)^T,\;α_3=\left(-1,2,0,3\right)^T\\(3)α_1=\left(1,0,-1\right)^T,\;α_2=\left(-2,2,0\right)^T,\;α_3=\left(3,-5,2\right)^T\end{array}&\;\end{array}\\\;\;(4)α_1=\left(1,1,3,1\right)^T,\;α_2=\left(3,-1,2,4\right)^T,\;α_3=\left(2,2,7,-1\right)^T\end{array}
$$
$$
A.
4个 \quad B.1个 \quad C.2个 \quad D.3个 \quad E. \quad F. \quad G. \quad H.
$$
$$
\begin{array}{l}(1)\mathrm{由结论}:n+1\mathrm{个维列向量是线性相关的},\mathrm{可知原向量组是线性相关的}.\\(2)\left[α_1,α_2,α_3\right]=\begin{pmatrix}3&1&-1\\1&0&2\\2&5&0\\4&2&3\end{pmatrix}\rightarrow\begin{pmatrix}1&0&0\\0&1&0\\0&0&1\\0&0&0\end{pmatrix}=A,R(A)=3=\mathrm{列数},故α_1,α_2,α_3\mathrm{是线性无关的}.\\(3)\begin{vmatrix}1&-2&3\\0&2&-5\\-1&0&2\end{vmatrix}=\begin{vmatrix}1&-2&3\\0&2&-5\\0&-2&5\end{vmatrix}=0,\mathrm{所以}α_1,α_2,α_3\mathrm{线性相关}.\\(4)\begin{pmatrix}1&3&2\\1&-1&2\\3&2&7\\1&4&-1\end{pmatrix}\rightarrow\begin{pmatrix}1&3&2\\0&-4&0\\0&-7&1\\0&1&-3\end{pmatrix}\rightarrow\begin{pmatrix}1&3&2\\0&1&0\\0&0&1\\0&0&0\end{pmatrix}\\\mathrm{因此向量组构成的矩阵的秩为}3,\mathrm{所以}α_1,α_2,α_3\mathrm{线性无关}.\end{array}
$$



$$
\mathrm{已知三阶矩阵}A\mathrm{与三维列向量}x\mathrm{满足}A^3x=3Ax-2A^2x\;,\mathrm{且向量组}x,Ax,A^2x\mathrm{线性无关},\;\mathrm{若存在矩阵}P=\left(x,Ax,A^2x\right)使AP=PB,则\;\left|B\right|=().
$$
$$
A.
0 \quad B.1 \quad C.2 \quad D.3 \quad E. \quad F. \quad G. \quad H.
$$
$$
\begin{array}{l}\mathrm{记向量}y=Ax,x=A^2x,\mathrm{则所给矩阵}P\mathrm{可写为}P=\left(x,y,z\right),\;\mathrm{并且由分块矩阵乘法规则},有\;AP=A\left(x,y,z\right)=\left(Ax,Ay,Az\right).\\因Ax=y,Ay=z,Az=A^3x=3Ax-2A^2x=3y-3z,故\\AP=\left(y,z,3y-2z\right)=\left(x,y,z\right)\begin{pmatrix}0&0&0\\1&0&3\\0&1&-2\end{pmatrix}=P\begin{pmatrix}0&0&0\\1&0&3\\0&1&-2\end{pmatrix},\\\mathrm{于是}B=P^{-1}AP=\begin{pmatrix}0&0&0\\1&0&3\\0&1&-2\end{pmatrix},则\left|B\right|=0.\end{array}
$$



$$
设α_1=\begin{pmatrix}1\\0\\0\\λ_1\end{pmatrix},α_2=\begin{pmatrix}1\\2\\0\\λ_2\end{pmatrix},α_3=\begin{pmatrix}-1\\2\\3\\λ_4\end{pmatrix},α_4=\begin{pmatrix}-2\\1\\2\\λ_4\end{pmatrix},\mathrm{其中}λ_1,λ_2,λ_3,λ_4\mathrm{是任意实数},\mathrm{则有}\;().
$$
$$
A.
α_1,α_2,α_3\mathrm{总线性相关} \quad B.α_1,α_2,α_3,α_4\mathrm{总线性相关} \quad C.α_1,α_2,α_3\mathrm{总线性无关} \quad D.α_1,α_2,α_3,α_4\mathrm{总线性无关} \quad E. \quad F. \quad G. \quad H.
$$
$$
\mathrm{因为向量组}\begin{pmatrix}1\\0\\0\end{pmatrix},\begin{pmatrix}1\\2\\0\end{pmatrix},\begin{pmatrix}-1\\2\\3\end{pmatrix}\mathrm{线性无关},\mathrm{且线性无关组的向量增加分量仍线性无关},\mathrm{因此}α_1,α_2,α_3\mathrm{总线性无关}.
$$



$$
\mathrm{向量组}I:α_1...,α_s(s⩾3)\mathrm{性相关的必要条件是}().
$$
$$
A.
I\mathrm{中每个向量都可以用其余的向量线性表出} \quad B.I\mathrm{中至少有一个向量可用其余的向量线性表出} \quad C.I\mathrm{中只有一个向量能用其余的向量线性表出} \quad D.I\mathrm{的任何部分组都线性相关} \quad E. \quad F. \quad G. \quad H.
$$
$$
\begin{array}{l}若A⇒ B,则B是A\mathrm{的必要条件},A是B\mathrm{的充分条件}.\\\mathrm{向量组}I\mathrm{线性相关}⇒ I\mathrm{中至少有一个向量可用其余向量线性表出},\mathrm{满足答案}.\\I\mathrm{的任何部分组线性相关}⇒\mathrm{向量组}I\mathrm{线性相关},\mathrm{是充分条件}.\end{array}
$$



$$
\mathrm{设向量组}(Ⅰ)为α_1=\begin{pmatrix}α_{11}\\α_{21}\\α_{31}\end{pmatrix},α_2=\begin{pmatrix}α_{12}\\α_{22}\\α_{32}\end{pmatrix},\alpha_3=\begin{pmatrix}α_{13}\\α_{23}\\α_{33}\end{pmatrix},\mathrm{向量组}(Ⅱ)为β_1=\begin{pmatrix}α_{11}\\α_{21}\\α_{31}\\α_{41}\end{pmatrix},β_2=\begin{pmatrix}α_{12}\\α_{22}\\α_{32}\\α_{42}\end{pmatrix},β_3=\begin{pmatrix}α_{13}\\α_{23}\\α_{33}\\α_{43}\end{pmatrix},则(\;).
$$
$$
A.
(Ⅰ)\mathrm{组线性相关}⇒(Ⅱ)\mathrm{组线性相关} \quad B.(Ⅰ)\mathrm{组线性无关}⇒(Ⅱ)\mathrm{组线性无关} \quad C.(Ⅱ)\mathrm{组线性无关}⇒(Ⅰ)\mathrm{组线性无关} \quad D.(Ⅰ)\mathrm{组线性无关}⇒(Ⅱ)\mathrm{组线性相关} \quad E. \quad F. \quad G. \quad H.
$$
$$
\begin{array}{l}\mathrm{线性相关组增加向量不改线性相关性},\mathrm{线性无关组增加分量不改线性无关性},故\\(Ⅰ)\mathrm{组线性无关}⇒(Ⅱ)\mathrm{租线性无关}\end{array}
$$



$$
\mathrm{若向量组}α,β,γ\mathrm{线性无关},\mathrm{则向量组}α+β,β+γ,γ+\alpha(\;).
$$
$$
A.
\mathrm{线性相关} \quad B.\mathrm{线性无关} \quad C.\mathrm{任意} \quad D.\mathrm{无法判断} \quad E. \quad F. \quad G. \quad H.
$$
$$
\begin{array}{l}(α+β,β+γ,γ+α)=(α,β,γ)\begin{pmatrix}1&0&1\\1&1&0\\0&1&1\end{pmatrix},记B=AK,因\begin{vmatrix}K\end{vmatrix}=2\neq0,知K\mathrm{可逆},\mathrm{由矩阵的秩的性质知}\\r(B)=r(A)=3.\\故α+β,β+γ,γ+α\mathrm{也线性无关}\end{array}
$$



$$
\begin{array}{l}\mathrm{设向量组}\\M:\alpha_1=(a_1,\;a_2,\;a_3)^T,α_2=(b_1,\;b_2,\;b_3)^T,α_3=(c_1,c_2,c_3)^T;\\N:α_1=(a_1,\;a_2,\;a_3,a_4)^T,α_2=(b_1,\;b_2,\;b_3,b_4)^T,α_3=(c_1,c_2,c_3,c_4)^T.\\\mathrm{则有}(\;).\\\end{array}
$$
$$
A.
M\mathrm{组线性相关},则N\mathrm{组线性相关} \quad B.M\mathrm{组线性}\;\mathrm{无关},则N\mathrm{组线性}\;\mathrm{无关} \quad C.N\mathrm{组线性}\;\mathrm{无关},则M\mathrm{组线性无关} \quad D.M\mathrm{组线性无关的充要条件是}N\mathrm{组线性无关} \quad E. \quad F. \quad G. \quad H.
$$
$$
\begin{array}{l}\mathrm{由于无关组增加分量扔线性无关},\mathrm{因此}M\mathrm{组线性无关}.则N\mathrm{组线性无关};\\设M:\alpha_1=(1,0,0)^T,α_2=(0,1,0)^T,α_3=(1,1,0)^T;\\N:α_1=(1,0,0,0)^T,α_2=(0,1,0,0)^T,α_3=(1,1,0,1)^T;\\\mathrm{则可通过此例说明其他选项都是不正确}.\\\end{array}
$$



$$
\mathrm{设向量组}(Ⅰ),\mathrm{向量组}(Ⅱ)是(Ⅰ)\mathrm{的部分组},\mathrm{则下列判断正确的是}(\;).
$$
$$
A.
若(Ⅰ)\mathrm{线性相关},则(Ⅱ)\mathrm{也线性相关} \quad B.若(Ⅰ)\mathrm{线性}\;\mathrm{无关},则(Ⅱ)\mathrm{也线性无关} \quad C.若(Ⅱ)\mathrm{线性}\;\mathrm{无关},则(Ⅰ)\mathrm{也线性无关} \quad D.(Ⅰ)\mathrm{的相关性与}(Ⅱ)\mathrm{的相关性没有联系} \quad E. \quad F. \quad G. \quad H.
$$
$$
\mathrm{向量组与部分组的关系为}:\mathrm{若向量组线性无关},\mathrm{则部分组扔线性无关};\mathrm{若部分组线性相关},\mathrm{则向量组也线性相关}.
$$



$$
\mathrm{设有向量组}\;α,β,γ,δ,\mathrm{其中}\alpha,β,γ\mathrm{线性无关},\mathrm{则正确选项为}(\;).
$$
$$
A.
α,γ\mathrm{线性无关} \quad B.α,β,γ,δ\mathrm{线性无关} \quad C.α,β,γ,δ\mathrm{线性相关} \quad D.β,γ,δ\mathrm{线性相关} \quad E. \quad F. \quad G. \quad H.
$$
$$
\mathrm{由向量组与其部分组的关系可得},
$$



$$
\mathrm{设向量组}(Ⅰ)为α_1=\begin{pmatrix}1\\0\\0\end{pmatrix},α_2=\begin{pmatrix}0\\1\\0\end{pmatrix},α_3=\begin{pmatrix}0\\0\\1\end{pmatrix},\mathrm{向量组}(Ⅱ)为β_1=\begin{pmatrix}1\\0\\0\\1\end{pmatrix},β_2=\begin{pmatrix}0\\1\\0\\2\end{pmatrix},β_3=\begin{pmatrix}0\\0\\1\\3\end{pmatrix},则(\;).
$$
$$
A.
(Ⅰ)\mathrm{组线性相关},\;(Ⅱ)\mathrm{组也线性相关}; \quad B.(Ⅰ)\mathrm{组线性无关},(Ⅱ)\mathrm{组也线性无关} \quad C.(Ⅰ)\mathrm{组线性相关},(Ⅱ)\mathrm{组线性无关}; \quad D.(Ⅰ)\mathrm{组线性无关},(Ⅱ)\mathrm{组线性相关}. \quad E. \quad F. \quad G. \quad H.
$$
$$
\begin{array}{l}\mathrm{显然}(Ⅰ)\mathrm{组线性无关},\mathrm{线性无关组增加分量不改线性无关性},\\故(Ⅱ)\mathrm{组线性无关}.\end{array}
$$



$$
\begin{array}{l}\mathrm{设向量组}(Ⅰ)为α_1=\begin{pmatrix}1\\1\\1\end{pmatrix},α_2=\begin{pmatrix}0\\1\\1\end{pmatrix},α_2=\begin{pmatrix}0\\0\\1\end{pmatrix},\\\mathrm{向量组}(Ⅱ)为β_1=\begin{pmatrix}1\\1\\1\\1\end{pmatrix},β_2=\begin{pmatrix}0\\1\\1\\2\end{pmatrix},β_3=\begin{pmatrix}0\\0\\1\\3\end{pmatrix},则(\;).\end{array}
$$
$$
A.
(Ⅰ)\;\mathrm{组线性相关},\;(Ⅱ)\;\mathrm{组也线性相关}; \quad B.(Ⅰ)\;\mathrm{组线性无关},(Ⅱ)\;\mathrm{组也线性无关}; \quad C.(Ⅰ)\;\mathrm{组线性无关},\;(Ⅱ)\mathrm{组线性相关}; \quad D.(Ⅰ)\;\mathrm{组线性相关},(Ⅱ)\;\mathrm{组线性无关}. \quad E. \quad F. \quad G. \quad H.
$$
$$
\begin{array}{l}(I)\mathrm{组线性无关},\mathrm{线性无关组增加分量不改变线性无关性},\mathrm{所以}(Ⅱ)\mathrm{组线性无关}.\\\end{array}
$$



$$
\begin{array}{l}\mathrm{设向量组}(Ⅰ)为α_1=\begin{pmatrix}1\\0\\0\end{pmatrix},α_2=\begin{pmatrix}0\\1\\0\end{pmatrix},α_3=\begin{pmatrix}0\\0\\1\end{pmatrix},\\\mathrm{向量组}(Ⅱ)为β_1=\begin{pmatrix}1\\0\\0\\2\\1\end{pmatrix},β_2=\begin{pmatrix}0\\1\\0\\3\\4\end{pmatrix},β_3=\begin{pmatrix}0\\0\\1\\5\\6\end{pmatrix},则(\;).\end{array}
$$
$$
A.
(Ⅰ)\mathrm{组线性相关},(Ⅱ)\mathrm{组也线性相关}; \quad B.(Ⅰ)\;\mathrm{组线性无关},(Ⅱ)\;\mathrm{组也线性无关}; \quad C.(Ⅰ)\mathrm{组线性无关},(Ⅱ)\mathrm{组线性相关}; \quad D.(Ⅰ)\mathrm{组线性相关},(Ⅱ)\mathrm{组线性无关}. \quad E. \quad F. \quad G. \quad H.
$$
$$
\begin{array}{l}(Ⅰ)\mathrm{组线性无关},\;\;\mathrm{线性无关组增加分量不改变线性无关性},\\故(Ⅱ)\mathrm{组也线性无关}.\end{array}
$$



$$
\mathrm{设向量组}Ⅰ:α_1,α_2,···α_s\mathrm{与向量组}Ⅱ:α_1,α_2,···α_s,α_{s+1},···,α_{s+t},\mathrm{则不正确}\;\mathrm{说法为}(\;).
$$
$$
A.
若Ⅱ\mathrm{线性无关},\mathrm{必有}Ⅰ\mathrm{线性无关} \quad B.若Ⅰ\mathrm{线性相关},\mathrm{必有}Ⅱ\mathrm{线性相关} \quad C.\mathrm{向量组}Ⅰ\mathrm{是向量组}Ⅱ\mathrm{的部分组} \quad D.若Ⅰ\mathrm{线性无关},\mathrm{必有}Ⅱ\mathrm{线性无关} \quad E. \quad F. \quad G. \quad H.
$$
$$
\mathrm{由向量组与部分之间关系的性质可得}.
$$



$$
\begin{array}{l}\mathrm{设向量组}α_1,α_2,α_3\mathrm{线性无关},\mathrm{向量组}\\\;\;\;\;\;\;\;\;\;\;\;\;\;\;\;\;\;\;\;\;\;\;\;\;\;\;\;\;\;\;\;\;\;\;\;\;\;\;\;\;\;\;\;\;\;β_1=-α_1+tα_2,β_2=-α_2+mα_3,β_3=-α_3+sα_1,\\\mathrm{线性无关},\mathrm{则常数}t,m,s\mathrm{满足}(\;).\end{array}
$$
$$
A.
tms\neq1 \quad B.tms=1 \quad C.tms\neq-1 \quad D.tms\neq-2 \quad E. \quad F. \quad G. \quad H.
$$
$$
\begin{array}{l}设x_1β_1+x_2β_2+x_3β_3=0,即\\\;\;\;\;\;\;\;\;\;\;\;\;\;\;\;\;\;\;\;\;\;\;\;\;\;\;\;\;\;\;\;\;\;\;\;\;\;\;\;\;\;\;(-x_1+sx_3)α_1+(tx_1-x_2)α_1+(mx_2+x_3)α_3=0.\\由α_1,α_2,α_3\mathrm{线性无关得}\\\;\;\;\;\;\;\;\;\;\;\;\;\;\;\;\;\;\;\;\;\;\;\;\;\;\;\;\;\;\;\;\;\;\;\;\;\;\;\left\{\begin{array}{c}-x_1+sx_3=0\\tx_1-x_2=0\\mx_2-x_3=0\end{array}\right.,\\当\begin{vmatrix}-1&0&\;s\\t&-1&0\\0&m&-1\end{vmatrix}=-1+tms\neq0\mathrm{时以上方程组只有零解},即tms\neq1时,β_1,β_2,β_3\mathrm{线性无关}.\end{array}
$$



$$
\mathrm{设向量组}α_1,α_2,α_3,α_4\mathrm{线性无关},若β_1=α_1-α_2,β_2=α_2-α_3,β_3=α_3-α_4,β_4=α_4,则β_1,β_2,β_3,β_4(\;).
$$
$$
A.
\mathrm{线性相关} \quad B.\mathrm{线性无关} \quad C.\mathrm{任意关系} \quad D.\mathrm{无法判断} \quad E. \quad F. \quad G. \quad H.
$$
$$
\begin{array}{l}\mathrm{不妨设}α_1,α_2,α_3,α_4,β_1,β_2,β_3,β_4\mathrm{均为列向量},\mathrm{由条件知},\\(β_1,β_2,β_3,β_4)=(α_1,α_2,α_3,α_4)\begin{pmatrix}1&0&0&0\\-1&1&0&0\\0&-1&1&0\\0&0&-1&1\end{pmatrix}.\\\mathrm{令矩阵}A=(α_1,\alpha_2,α_3,α_4),B=(β_1,β_2,β_3,β_4),\\\mathrm{于是},有B=AP.\mathrm{由于}α_1,α_2,α_3,α_4\mathrm{线性无关},\vert P\vert=1\neq0,故β_1,β_2,β_3,β_4\mathrm{线性无关}.\end{array}
$$



$$
\begin{array}{l}\mathrm{设向量组}α_1,α_2,α_3,β\mathrm{线性无关},\mathrm{已知}\\β_1=α_1+β,β_2=α_2+2β,β_3=α_3+3β,\;则β_1,β_2,β_3,β(\;).\end{array}
$$
$$
A.
\mathrm{线性相关} \quad B.\mathrm{线性无关} \quad C.\mathrm{任意关系} \quad D.\mathrm{无法判断} \quad E. \quad F. \quad G. \quad H.
$$
$$
\begin{array}{l}\mathrm{不妨设}α_1,α_2,α_3,α_4,β_1,β_2,β_3,β_4\mathrm{均为列向量},\mathrm{由条件知},\\(β_1,β_2,β_3,β)=(α_1,α_2,α_3,β)\begin{pmatrix}1&0&0&0\\0&1&0&0\\0&0&1&0\\1&2&3&1\end{pmatrix}.\\\mathrm{令矩阵}A=(α_1,α_2,α_3,β),B=(β_1,β_2,β_3,β),\\\mathrm{于是},有B=AP.\mathrm{由于}α_1,α_2,α_3,β\mathrm{线性无关},\vert P\vert=1\neq0,故β_1,β_2,β_3,β\mathrm{线性无关}.\end{array}
$$



$$
\mathrm{若向量组}α_1,α_2,α_3\mathrm{线性无关},\mathrm{则向量组}β_1=α_1,β_2=α_1+α_2,β_3=α_2+α_3(\;).
$$
$$
A.
\mathrm{线性相关} \quad B.\mathrm{线性无关} \quad C.\mathrm{无法判断} \quad D.\begin{array}{l}\mathrm{任意关系}\\\end{array} \quad E. \quad F. \quad G. \quad H.
$$
$$
\begin{array}{l}由\;\mathrm{题意}(β_1,β_2,β_3)=(α_1,α_2,α_3)\begin{pmatrix}1&1&0\\0&1&1\\0&0&1\end{pmatrix},\\记B=AK,因\vert K\vert=1\neq0,由α_1,α_2,α_3\mathrm{线性无关知},β_1,β_2,β_3\mathrm{线性无关}\end{array}
$$



$$
\begin{array}{l}\mathrm{设向量组}α_1,α_2,α_3\mathrm{线性无关}\;\mathrm{是向量组}β_1=α_1,β_2=α_1+α_2,β_3=α_1+α_2+α_3\mathrm{线性无关的}(\;)\mathrm{条件}\\\end{array}
$$
$$
A.
\mathrm{充分} \quad B.\mathrm{必要} \quad C.\mathrm{充要} \quad D.\mathrm{既不充分也不必要} \quad E. \quad F. \quad G. \quad H.
$$
$$
\begin{array}{l}由\;\mathrm{题意}(β_1,β_2,β_3)=(α_1,α_2,α_3)\begin{pmatrix}1&1&1\\0&1&1\\0&0&1\end{pmatrix},\\记B=AK,因\vert K\vert=1\neq0,\mathrm{由故}α_1,α_2,α_3\mathrm{线性无关是}β_1,β_2,β_3\mathrm{线性无关的充要条件}.\end{array}
$$



$$
\mathrm{向量组}α_1,α_2,α_3\mathrm{线性无关是向量组}β_1=α_1+α_2,β_2=α_2+α_3,β_3=α_3+α_1\mathrm{线性无关的}(\;)\mathrm{条件}.
$$
$$
A.
\mathrm{充分} \quad B.\mathrm{必要} \quad C.\mathrm{充要} \quad D.\mathrm{既不充分也不必要} \quad E. \quad F. \quad G. \quad H.
$$
$$
\begin{array}{l}由\;\mathrm{题意}(β_1,β_2,β_3)=(α_1,α_2,α_3)\begin{pmatrix}1&0&1\\1&1&0\\0&1&1\end{pmatrix},\\记B=AK,因\vert K\vert=2\neq0,\mathrm{所以}α_1,α_2,α_3\mathrm{线性无关是}β_1,β_2,β_3\mathrm{线性无关的充要条件}\end{array}
$$



$$
\mathrm{若向量组}α_1,α_2,α_3\mathrm{线性无关},\mathrm{则向量组}β_1=α_1-α_2,β_2=α_2-α_3,β_3=α_3-α_1(\;).
$$
$$
A.
\mathrm{线性相关} \quad B.\mathrm{线性无关} \quad C.\mathrm{无法判断} \quad D.\mathrm{任意关系} \quad E. \quad F. \quad G. \quad H.
$$
$$
\begin{array}{l}\begin{array}{l}由\;\mathrm{题意}(β_1,β_2,β_3)=(α_1,α_2,α_3)\begin{pmatrix}1&0&-1\\-1&1&0\\0&-1&1\end{pmatrix},\\记B=AK,因\vert K\vert=0,R(K)=2,R(A)=3⇒ R(B)\leq min\{R(A),R(K)\}\end{array}\\\mathrm{故向量组}β_1,β_2,β_3\mathrm{线性相关}\\\end{array}
$$



$$
\mathrm{向量组}α_1,α_2,α_3\mathrm{线性无关是向量组}β_1=α_1+α_2+α_3,β_2=α_2+α_3,β_3=α_3\mathrm{线性无关的}(\;)\mathrm{条件}.
$$
$$
A.
\mathrm{充要} \quad B.\mathrm{充分} \quad C.\mathrm{必要} \quad D.\mathrm{既不充分也不必要} \quad E. \quad F. \quad G. \quad H.
$$
$$
\begin{array}{l}由\;\mathrm{题意}(β_1,β_2,β_3)=(α_1,α_2,α_3)\begin{pmatrix}1&0&0\\1&1&0\\1&1&1\end{pmatrix},\\记B=AK,因\vert K\vert=1\neq0,由α_1,α_2,α_3\mathrm{线性无关}\;是β_1,β_2,β_3\mathrm{线性无关的充要条件}.\end{array}
$$



$$
\mathrm{向量组}α_1,α_2,α_3\mathrm{线性无关是向量组}β_1=-α_1,β_2=α_1-α_2,β_3=α_2-α_3\mathrm{线性无关的}(\;)\mathrm{条件}.
$$
$$
A.
\mathrm{充要} \quad B.\mathrm{充分} \quad C.\mathrm{必要} \quad D.\mathrm{既不充分也不必要} \quad E. \quad F. \quad G. \quad H.
$$
$$
\begin{array}{l}由\;\mathrm{题意}(β_1,β_2,β_3)=(α_1,α_2,α_3)\begin{pmatrix}-1&1&0\\0&-1&1\\0&0&-1\end{pmatrix},\\记B=AK,因\vert K\vert=-1\neq0,由α_1,α_2,α_3\mathrm{线性无关是}β_1,β_2,β_3\mathrm{线性无关的充要条件}.\end{array}
$$



$$
\begin{array}{l}\mathrm{已知}α_1=\begin{pmatrix}1\\1\\1\end{pmatrix},α_2=\begin{pmatrix}0\\2\\5\end{pmatrix},α_3=\begin{pmatrix}2\\4\\7\end{pmatrix},\mathrm{则向量组}α_1,α_2,α_3及α_1,α_2\mathrm{的线性相关性的说法正确的是}(\;).\\\end{array}
$$
$$
A.
\mathrm{线性相关};\mathrm{线性无关} \quad B.\mathrm{线性无关};\mathrm{线性无关} \quad C.\mathrm{线性相关};\mathrm{线性相关} \quad D.\mathrm{线性无关};\mathrm{线性相关} \quad E. \quad F. \quad G. \quad H.
$$
$$
\begin{array}{l}\mathrm{对矩阵}A=(α_1,α_2,α_3)\mathrm{施行初等行变换变成行阶梯形矩阵},\mathrm{可同时看出矩阵}A及B=(α_1,α_2)\mathrm{的秩},\mathrm{依定理即可}\\\mathrm{得到结论}.\\(α_1,α_2,α_3)=\begin{pmatrix}1&0&2\\1&2&4\\1&5&7\end{pmatrix}\xrightarrow[{r_3-r_1}]{r_2-r_1}\begin{pmatrix}1&0&2\\0&2&2\\0&5&5\end{pmatrix}\xrightarrow{r_3-\frac52r_2}\begin{pmatrix}1&0&2\\0&2&2\\0&0&0\end{pmatrix},\\\mathrm{易见},R(A)=2,R(B)=2,\mathrm{故向量组}α_1,α_2,α_3\mathrm{线性相关},\mathrm{向量组}α_1,α_2\mathrm{线性无关}.\\\mathrm{或因}α_2=α_3-2α_1知α_1,α_2,α_3\mathrm{线性相关},而α_1,α_2\mathrm{的对应分类不成比例},α_1,α_2\mathrm{线性无关}.\end{array}
$$



$$
\begin{array}{l}\mathrm{设向量组}α_1,α_2,α_3\mathrm{线性无关},\mathrm{已知}\\\;\;\;\;\;\;\;\;\;\;\;\;\;\;\;\;\;\;\;\;\;\;\;\;β_1=k_1α_1+α_2+k_1α_3,\\\;\;\;\;\;\;\;\;\;\;\;\;\;\;\;\;\;\;\;\;\;\;\;\;\;β_2=α_1+k_2α_2+(k_2+1)α_3,\\\;\;\;\;\;\;\;\;\;\;\;\;\;\;\;\;\;\;\;\;\;\;\;\;β_3=α_1+α_2+α_3\\若β_1,β_2,β_3\mathrm{线性相关},则k_1,k_2\mathrm{的值为}(\;).\end{array}
$$
$$
A.
k_1=3 \quad B.k_1=2 \quad C.k_2=-1 \quad D.k_1=1或k_2=0 \quad E. \quad F. \quad G. \quad H.
$$
$$
\begin{array}{l}\mathrm{不妨设}α_1,α_2,α_3,β_1,β_2,β_3\mathrm{均为列向量},\mathrm{由条件知},\\(β_1,β_2,β_3)=(α_1,α_2,α_3)\begin{pmatrix}k_1&1&1\\1&k_2&1\\k_1&k_2+1&1\end{pmatrix}.\\\mathrm{令矩阵}A=(α_1,α_2,α_3),B=(β_1,β_2,β_3),P=\begin{pmatrix}k_1&1&1\\1&k_2&1\\k_1&k_2+1&1\end{pmatrix}\\\mathrm{于是},有B=AP.\mathrm{由于}α_1,α_2,α_3\mathrm{线性无关},r(A)=3,而\vert P\vert=(1-k_1)k_2,\mathrm{故当}k_1=1或k_2=0时,r(B)<3,β_1,β_2,β_3\mathrm{线性相关}.\end{array}
$$



$$
设A是4×5\mathrm{型矩阵},且A\mathrm{的行向量组线性无关},\mathrm{则有}(\;).
$$
$$
A.
A\mathrm{的列向量组线性无关} \quad B.\mathrm{矩阵}B=(A,b)\mathrm{的行向量组线性无关} \quad C.\mathrm{矩阵}B=(A,b)\mathrm{的任意}4\mathrm{个列向量线性无关} \quad D.\mathrm{矩阵}B=(A,b)\mathrm{的行及列向量组都线性无关} \quad E. \quad F. \quad G. \quad H.
$$
$$
\begin{array}{l}\mathrm{因为}“\mathrm{线性无关组正价分量仍线性无关}”,\mathrm{矩阵}B的4\mathrm{个行向量是由}A得4\mathrm{个行向量证加一个分向量得到的},\mathrm{故由}\\A\mathrm{的航向了线性无关可知}B\mathrm{的行向量组也线性无关}.\end{array}
$$



$$
\begin{array}{l}\mathrm{设向量}α_1,α_2,α_3,α_4\mathrm{均为}4\mathrm{维列向量},且β=α_1+α_2+α_3+α_4,\\\mathrm{则向量组}β-α_1,β-α_2,β-α_3,β-α_4\mathrm{线性无关是}α_1,α_2,α_3,α_4\mathrm{线性无关}(\;)\mathrm{条件}.\end{array}
$$
$$
A.
\mathrm{充分} \quad B.\mathrm{必要} \quad C.\mathrm{充要} \quad D.\mathrm{既不充分也不必要} \quad E. \quad F. \quad G. \quad H.
$$
$$
\begin{array}{l}令A=(α_1,α_2,α_3,\alpha_4),B=(β-α_1,β-α_2,β-α_3,β-α_4),则\\B=A\begin{pmatrix}0&1&1&1\\1&0&1&1\\1&1&0&1\\1&1&1&0\end{pmatrix},\mathrm{记为}B=AP,\\因\vert P\vert\neq0⇒ R(A)=R(B),\mathrm{二者可相互线性表示},\mathrm{等价},\mathrm{充要条件}.\end{array}
$$



$$
\mathrm{设向量}α_1,α_2,α_3\mathrm{线性无关},\mathrm{若向量组}tα_2-α_1,mα_3-α_2,α_1-\alpha_3\mathrm{线性相关},\mathrm{则常数}m,t\mathrm{的值为}(\;).
$$
$$
A.
tm=1 \quad B.tm\neq1 \quad C.tm\neq-1 \quad D.tm=-1 \quad E. \quad F. \quad G. \quad H.
$$
$$
\begin{array}{l}\mathrm{设存在}3\mathrm{个数}k_1,k_2,k_3,使\\\;\;\;\;\;\;\;\;\;\;\;\;\;\;\;\;\;\;\;\;\;\;\;\;\;\;\;\;\;\;\;\;\;\;\;\;k_1(tα_2-α_1)+\;k_2(mα_3-α_2)+\;k_3(α_1-α_3)=0,\\\;\;\;\;\;\;\;\;\;\;\;\;\;\;\;\;\;\;\;\;\;\;\;\;\;\;\;\;\;\;\;\;\;\;\;\;(k_3-k_1)α_1+\;(tk_1-k_2)α_2+\;(mk_2-k_3)α_3=0,\\\mathrm{因为}α_1,α_2,α_3\mathrm{线性无关},\mathrm{所以}\\\;\;\;\;\;\;\;\;\;\;\;\;\;\;\;\;\;\;\;\;\;\;\;\;\;\;\;\;\;\;\;\;\;\;\;\left\{\begin{array}{c}-k_1+k_3=0\\tk_1-k_2=0\\mk_2-k_3=0\end{array}\right.,D=\begin{vmatrix}-1&0&1\\t&-1&0\\0&m&-1\end{vmatrix}=tm-1.\\(1)当tm-1\neq0时,即tm\neq1时,\mathrm{方程组只有零解}k_1=k_2=k_3=0,\mathrm{因此},tα_2-α_1,mα_3-α_2,α_1-α_3\mathrm{线性无关},\\(2)当tm-1=0时,即tm=1时,\mathrm{方程组有非零解},\mathrm{因此},tα_2-α_1,mα_3-α_2,α_1-α_3\mathrm{线性相关},\end{array}
$$



$$
\begin{array}{l}\mathrm{设列向量组}α_1,α_2,α_3\mathrm{线性无关},\mathrm{已知}\\\;\;\;\;\;\;\;\;\;\;\;\;\;\;\;\;\;\;\;\;\;\;\;\;β_1=k_1α_1+α_2+k_1α_3,β_2=α_1+k_2α_2+(k_2+1)α_3,\;β_3=α_1+α_2+α_3\\若β_1,β_2,β_3\mathrm{线性无关},则k_1,k_2\mathrm{的值为}(\;).\end{array}
$$
$$
A.
k_1=1或\;k_2=0 \quad B.k_1=1 \quad C.k_2=0 \quad D.k_1\neq1且k_2\neq0 \quad E. \quad F. \quad G. \quad H.
$$
$$
\begin{array}{l}\begin{array}{l}\mathrm{有条件知},(β_1,β_2,β_3)=(α_1,α_2,α_3)\begin{bmatrix}k_1&1&1\\1&k_2&1\\k_1&k_2+1&1\end{bmatrix},令A=(α_1,α_2,α_3),B=(β_1,β_2,β_3),P=\begin{bmatrix}k_1&1&1\\1&k_2&1\\k_1&k_2+1&1\end{bmatrix}\\则B=AP,\mathrm{由于}α_1,α_2,α_3\mathrm{线性无关},\mathrm{所以}R(A)=3,若β_1,β_2,β_3\mathrm{线性无关},则\begin{vmatrix}k_1&1&1\\1&k_2&1\\k_1&k_2+1&1\end{vmatrix}=k_2(1-k_1)\neq0,故\end{array}\\k_1\neq1且k_2\neq0\end{array}
$$



$$
\mathrm{若向量组}α_1,α_2,α_3,α_4\mathrm{线性无关},\mathrm{则向量组}β_1=α_1+α_2,β_2=α_2+\alpha_3,β_3=α_3+α_4,β_4=α_4+α_1(\;).
$$
$$
A.
\mathrm{线性无关} \quad B.\mathrm{线性相关} \quad C.\mathrm{任意关系} \quad D.\mathrm{无法判断} \quad E. \quad F. \quad G. \quad H.
$$
$$
\begin{array}{l}\begin{array}{l}\begin{array}{l}(β_1,β_2,β_3,β_4)=(α_1α_2α_3,α_4)\begin{pmatrix}1&0&0&1\\1&1&0&0\\0&1&1&0\\0&0&1&1\end{pmatrix},\\记B=AK,因\vert K\vert=0,R(K)=3,R(A)=4,\end{array}\\\mathrm{由矩阵的秩的性质知}R(B)=min(R(A),R(K))=3.\\\mathrm{故向量组}β_1,β_2,β_3,β_4\mathrm{线性相关}.\end{array}\\\end{array}
$$



$$
\mathrm{若向量组}α_1,α_2,α_3,α_4\mathrm{线性无关},\mathrm{则向量组}β_1=α_1,β_2=α_1+α_2,β_3=α_2+α_3,β_4=α_3+α_4(\;).
$$
$$
A.
\mathrm{线性无关} \quad B.\mathrm{线性相关} \quad C.\mathrm{无法判断} \quad D.\mathrm{任意关系} \quad E. \quad F. \quad G. \quad H.
$$
$$
\begin{array}{l}\begin{array}{l}由\;\mathrm{题意}(β_1,β_2,β_3,β_4)=(α_1,α_2,α_3,α_4)\begin{pmatrix}1&1&0&0\\0&1&1&0\\0&0&1&1\\0&0&0&1\end{pmatrix},\\记B=AK,因\vert K\vert=1\neq0,由α_1,α_2,α_3,α_4\mathrm{线性无关知}β_1,β_2,β_3,β_4\mathrm{线性无关}\end{array}\end{array}
$$



$$
\mathrm{若向量组}α_1,α_2,α_3,α_4\mathrm{线性无关},\mathrm{则向量组}β_1=α_1,β_2=α_1-α_2,β_3=α_2-α_3,β_4=α_3-α_4(\;).
$$
$$
A.
\mathrm{线性相关} \quad B.\mathrm{线性无关} \quad C.\mathrm{无法判断} \quad D.\mathrm{任意关系} \quad E. \quad F. \quad G. \quad H.
$$
$$
\begin{array}{l}\begin{array}{l}由\;\mathrm{题意}(β_1,β_2,β_3,β_4)=(α_1,α_2,α_3,α_4)\begin{pmatrix}1&1&0&0\\0&-1&1&0\\0&0&-1&1\\0&0&0&-1\end{pmatrix},\\记B=AK,因\vert K\vert=-1\neq0,由α_1,α_2,α_3,α_4\mathrm{线性无关知}β_1,β_2,β_3,β_4\mathrm{线性无关}\end{array}\end{array}
$$



$$
\mathrm{若向量组}α_1,α_2,α_3,α_4\mathrm{线性无关},\mathrm{则向量组}β_1=-α_1,β_2=α_1-α_2,β_3=α_2-α_3,β_4=α_3-α_4(\;).
$$
$$
A.
\mathrm{线性相关} \quad B.\mathrm{线性无关} \quad C.\mathrm{无法判断} \quad D.\mathrm{任意关系} \quad E. \quad F. \quad G. \quad H.
$$
$$
\begin{array}{l}\begin{array}{l}由\;\mathrm{题意}(β_1,β_2,β_3,β_4)=(α_1,α_2,α_3,α_4)\begin{pmatrix}-1&1&0&0\\0&-1&1&0\\0&0&-1&1\\0&0&0&-1\end{pmatrix},\\记B=AK,因\vert K\vert=1\neq0,由α_1,α_2,α_3,α_4\mathrm{线性无关知}β_1,β_2,β_3,β_4\mathrm{线性无关}\end{array}\end{array}
$$



$$
\mathrm{若向量组}α_1,α_2,α_3,α_4\mathrm{线性无关},\mathrm{则向量组}β_1=α_1,β_2=α_1+α_2,β_3=α_1+α_2+α_3,β_4=α_1+α_2+α_3+α_4(\;).
$$
$$
A.
\mathrm{线性无关} \quad B.\mathrm{线性相关} \quad C.\mathrm{无法判断} \quad D.\mathrm{任意关系} \quad E. \quad F. \quad G. \quad H.
$$
$$
\begin{array}{l}\begin{array}{l}由\;\mathrm{题意}(β_1,β_2,β_3,β_4)=(α_1,α_2,α_3,α_4)\begin{pmatrix}1&1&1&1\\0&1&1&1\\0&0&1&1\\0&0&0&1\end{pmatrix},\\记B=AK,因\vert K\vert=1\neq0,由α_1,α_2,α_3,α_4\mathrm{线性无关知}β_1,β_2,β_3,β_4\mathrm{线性无关}\end{array}\end{array}
$$



$$
\begin{array}{l}\mathrm{设向量组}Ⅰ:α_1,α_2,···α_s\mathrm{与向量组}Ⅱ:α_1,α_2,···α_s,α_{s+1},···,,α_{s+t},\mathrm{则以下说法中正确的有}(\;)项.\\(1)若Ⅰ\mathrm{线性无关},\mathrm{必有}Ⅱ\mathrm{线性无关}\;\;\;\;(2)若Ⅰ\mathrm{线性无关},\mathrm{必有}Ⅱ\mathrm{线性相关}\\(3)若Ⅱ\mathrm{线性无关},\mathrm{必有}Ⅰ\mathrm{线性无关}\;\;\;\;\;(4)若Ⅱ\mathrm{线性相关},\mathrm{必有}Ⅰ\mathrm{线性无关}\\(5)若Ⅰ\mathrm{线性相关},\mathrm{必有}Ⅱ\mathrm{线性相关}\;\;\;\;\;(6)若Ⅱ\mathrm{线性相关},\mathrm{必有}Ⅰ\mathrm{线性相关}\\\end{array}
$$
$$
A.
1 \quad B.2 \quad C.3 \quad D.4 \quad E. \quad F. \quad G. \quad H.
$$
$$
\mathrm{由向量组与部分之间关系的性质可得}.
$$



$$
\begin{array}{l}\mathrm{下列说法正确的是}(\;).\\(1)\mathrm{设向量组}α_1,α_2,α_3\mathrm{线性无关},\mathrm{则向量组}\\β_1=α_1+2α_2+3α_3,β_2=α_2+2α_3,β_3=α_3\mathrm{线性无关};\\(2)\mathrm{设向量组}α_1,α_2,α_3\mathrm{线性无关},\mathrm{则向量组}\\β_1=α_1+α_2+α_3,β_2=α_2+α_3,β_3=α_3\mathrm{线性无关};\end{array}
$$
$$
A.
(1)(2) \quad B.(1) \quad C.(2) \quad D.\mathrm{都不正确} \quad E. \quad F. \quad G. \quad H.
$$
$$
\begin{array}{l}(1)\mathrm{由题}(β_1,β_2,β_3)=(α_1,α_2,α_3)\begin{pmatrix}1&0&0\\2&1&0\\3&2&1\end{pmatrix},B=AP,P\mathrm{可逆}.\\(2)\mathrm{由题}(β_1,β_2,β_3)=(α_1,α_2,α_3)\begin{pmatrix}1&0&0\\1&1&0\\1&1&1\end{pmatrix},B=AP,P\mathrm{可逆}.\\故(1)(2)\mathrm{都正确}.\end{array}
$$



$$
\begin{array}{l}\mathrm{已知向量组}α_1,α_2,α_3\mathrm{线性无关},\mathrm{则下列向量组中}(\;)\mathrm{线性无关}.\\(1)α_1+α_2,α_2+α_3,α_3+α_1\\(2)α_1,α_1+α_2,α_2+α_3\\(3)α_1,α_1+α_2,α_1+α_2+α_3\\(4)α_1-α_2,α_2-α_3,α_3-α_1\end{array}
$$
$$
A.
(1)(2) \quad B.(1)(3) \quad C.(2)(3) \quad D.(1)(2)(3) \quad E. \quad F. \quad G. \quad H.
$$
$$
\begin{array}{l}\mathrm{前三个向量组可以依次表示为}\\(β_1,β_2,β_3)=(α_1,α_2,α_3)\begin{pmatrix}1&0&1\\1&1&0\\0&1&1\end{pmatrix},(β_1,β_2,β_3)=(\alpha_1,α_2,α_3)\begin{pmatrix}1&1&0\\0&1&1\\0&0&1\end{pmatrix},\\(β_1,β_2,β_3)=(α_1,α_2,α_3)\begin{pmatrix}1&1&1\\0&1&1\\0&0&1\end{pmatrix},\mathrm{前三个向量组均线性无关},\\\mathrm{只有第四个向量组}α_1-α_2+α_2-α_3+α_3-α_1=0,\mathrm{线性相关}.\end{array}
$$



$$
\begin{array}{l}\mathrm{已知向量组}α_1,α_2,α_3\mathrm{线性无关},\mathrm{则下列向量组中}\left(\;\;\;\;\right)\mathrm{线性相关}.\;\;\\\left(1\right)\;α_1+α_2,α_2+α_3,α_3+α_1\\\left(2\right)\;α_1,α_1+α_2,α_2+α_3\\\left(3\right)\;α_1,α_1+α_2,α_1+α_2+α_3\\\left(4\right)\;α_1-α_2,α_2-α_3,α_3-α_1\\\end{array}
$$
$$
A.
\left(1\right) \quad B.\left(2\right) \quad C.\left(3\right) \quad D.\left(4\right) \quad E. \quad F. \quad G. \quad H.
$$
$$
\begin{array}{l}\mathrm{前三个向量组可以依次表示为}\;\\\left(β_1,β_2,β_3\right)=\left(α_1,α_2,α_3\right)\begin{pmatrix}1&0&1\\1&1&0\\0&1&1\end{pmatrix},\left(β_1,β_2,β_3\right)=\left(α_1,α_2,α_3\right)\begin{pmatrix}1&1&1\\0&1&1\\0&0&1\end{pmatrix},\\\left(β_1,β_2,β_3\right)=\left(α_1,α_2,α_3\right)\begin{pmatrix}1&1&1\\0&1&1\\0&0&1\end{pmatrix}\;,\mathrm{前三个向量组均线性无关}.\;\;\\\mathrm{只有第四个向量组}α_1-α_2+α_2-α_3+α_3-α_1=0,\mathrm{线性相关}.\end{array}
$$



$$
\begin{array}{l}\mathrm{下列说法正确的是}(\;).\;\\\;(1)\mathrm{设向量组}\alpha_1,α_2,α_3\mathrm{线性无关},\mathrm{则向量组}\\β_1=α_1+α_2,β_2=α_1-α_2,β_3=α_1-2α_2+α_3\mathrm{线性无关}.\;\;\\(2)\mathrm{设向量组}α_1,α_2,α_3\mathrm{线性无关},\mathrm{则向量组}\;\;\\β_1=α_1+2α_2+3α_3,β_2=-α_1+α_3,β_3=3α_1+3α_2+4α_3\mathrm{线性无关}.\end{array}
$$
$$
A.
\;(1)\;(2) \quad B.\;(1) \quad C.\;(2) \quad D.\;(1)\;(2)\mathrm{都不正确} \quad E. \quad F. \quad G. \quad H.
$$
$$
\begin{array}{l}(1)\mathrm{由题}\left(β_1,β_2,β_3\right)=\left(α_1,α_2,α_3\right)\begin{pmatrix}1&1&1\\1&-1&-2\\0&0&1\end{pmatrix},B=AP,P\mathrm{可逆}.\;\;\\(2)\mathrm{由题}\left(β_1,β_2,β_3\right)=\left(α_1,α_2,α_3\right)\begin{pmatrix}1&-1&3\\2&0&3\\3&1&4\end{pmatrix},B=AP,P\mathrm{可逆}.\end{array}
$$



$$
\mathrm{已知向量组}α_1,α_2,α_3,α_4\mathrm{线性无关},\mathrm{则下列向量组中线性相关的是}\left(\;\;\;\right).\;
$$
$$
A.
α_1+α_2,α_2+α_3,α_3+α_4,α_4+α_1 \quad B.α_1,α_1+α_2,α_2+α_3,α_3+α_4 \quad C.α_1+α_2,α_2+α_3,α_3+α_4,α_4 \quad D.α_1,α_1+\alpha_2,α_1+α_2+α_3,α_1+α_2+α_3+α_4, \quad E. \quad F. \quad G. \quad H.
$$
$$
\begin{array}{l}\mathrm{因为}\left(α_1+α_2\right)-\left(α_2+α_3\right)+\left(α_3+α_4\right)-\left(α_4+α_1\right)=0,\mathrm{所以四个向量线性相关}.\;\;\\\mathrm{其余三个依次表示为}\;\;\\\left(β_1,β_2,β_3,β_4\right)=\left(α_1,α_2,α_3,α_4\right)\begin{pmatrix}1&1&0&0\\0&1&1&0\\0&0&1&1\\0&0&0&1\end{pmatrix},\\\left(β_1,β_2,β_3,β_4\right)=\left(α_1,α_2,α_3,α_4\right)\begin{pmatrix}1&0&0&0\\1&1&0&0\\0&1&1&0\\0&0&1&1\end{pmatrix},\;\;\\\left(β_1,β_2,β_3,β_4\right)=\left(α_1,α_2,α_3,α_4\right)\begin{pmatrix}1&1&1&1\\0&1&1&1\\0&0&1&1\\0&0&0&1\end{pmatrix},\mathrm{均线性无关}.\\\end{array}
$$



$$
\begin{array}{l}\mathrm{已知向量组}α_1,α_2,α_3,α_4\mathrm{线性无关},\mathrm{则下列向量组中}\left(\;\;\;\;\right)\mathrm{个向量组线性相关}.\;\;\\\left(1\right)\;α_1+α_2,α_2+α_3,α_3+α_4,\alpha_4+α_1\\\left(2\right)\;\;α_1,α_1+α_2,α_2+α_3,α_3+α_4\\\left(3\right)\;\;α_1+α_2,α_2+α_3,α_3+α_4,α_4\\\left(4\right)\;α_1,\;\;α_1+α_2,\;α_1+α_2+α_3,\;α_1+α_2+α_3+α_4\end{array}
$$
$$
A.
1 \quad B.2 \quad C.3 \quad D.4 \quad E. \quad F. \quad G. \quad H.
$$
$$
\begin{array}{l}\mathrm{因为}\left(α_1+α_2\right)-\left(α_2+α_3\right)+\left(α_3+α_4\right)-\left(α_4+α_1\right)=0,\mathrm{所以四个向量线性相关}.\;\;\\\mathrm{其余三个依次表示为}\;\;\\\left(β_1,β_2,β_3,\;\;β_4\right)=\left(α_1,α_2,α_3,\;\;α_4\;\right)\begin{pmatrix}1&1&0&0\\0&1&1&0\\0&0&1&1\\0&0&0&1\end{pmatrix},\\\left(β_1,β_2,β_3,\;\;β_4\right)=\left(\alpha_1,α_2,α_3,\;\;α_4\;\right)\begin{pmatrix}1&0&0&0\\1&1&0&0\\0&1&1&0\\0&0&1&1\end{pmatrix},\;\;\\\left(β_1,β_2,β_3,\;\;β_4\right)=\left(α_1,α_2,α_3,\;\;α_4\;\right)\begin{pmatrix}1&1&1&1\\0&1&1&1\\0&0&1&1\\0&0&0&1\end{pmatrix},\mathrm{均线性无关}.\end{array}
$$



$$
\begin{array}{l}\mathrm{已知向量组}α_1,α_2,α_3,α_4\mathrm{线性无关},\mathrm{则下列向量组中线性相关的向量组有}\left(\;\;\;\right)个.\;\\\begin{array}{l}\left(1\right)\;α_1+α_2,α_2+α_3,α_3+α_4,α_4+α_1\\\left(2\right)\;\;α_1-α_2,α_2-α_3,α_3-α_4,α_4-α_1\\\left(3\right)\;\;α_1+α_2,α_2+α_3,α_3-α_4,α_4-α_1\\\left(4\right)\;α_1+α_2,\;α_2+α_3,α_3+α_4,α_4-α_1\end{array}\end{array}
$$
$$
A.
1 \quad B.2 \quad C.3 \quad D.4 \quad E. \quad F. \quad G. \quad H.
$$
$$
\begin{array}{l}\mathrm{因为}\left(α_1+α_2\right)-\left(α_2+α_3\right)+\left(α_3+α_4\right)-\left(α_4+α_1\right)=0,\;\;\;\;\;\\\;\;\;\;\;\;\;\;\;\left(α_1-α_2\right)+\left(α_2-α_3\right)+\left(α_3-α_4\right)+\left(α_4-α_1\right)=0,\\\;\;\;\;\;\;\;\;\;\;\left(α_1+α_2\right)-\left(α_2+α_3\right)+\left(α_3-α_4\right)+\left(α_4-α_1\right)=0\;,\;\;\\\mathrm{因此只有向量组}α_1+α_2,α_2+α_3,α_3+α_4,α_4-α_1\mathrm{线性无关},\mathrm{前三组线性相关}.\end{array}
$$



$$
\begin{array}{l}\mathrm{已知向量组线性无关},\mathrm{则下列向量组中线性无关的向量组有}\left(\;\;\;\;\right)个.\\\begin{array}{l}\left(1\right)\;α_1+α_2,α_2+α_3,α_3+α_4,α_4+α_1\\\left(2\right)\;\;α_1-α_2,α_2-α_3,α_3-α_4,α_4-α_1\\\left(3\right)\;\;α_1+α_2,α_2+α_3,α_3-α_4,α_4-α_1\\\left(4\right)\;α_1+α_2,\;α_2+α_3,α_3+α_4,α_4-α_1\end{array}\end{array}
$$
$$
A.
1 \quad B.2 \quad C.3 \quad D.4 \quad E. \quad F. \quad G. \quad H.
$$
$$
\begin{array}{l}\mathrm{因为}\left(α_1+α_2\right)-\left(α_2+α_3\right)+\left(α_3+α_4\right)-\left(α_4+α_1\right)=0,\;\;\;\;\;\\\;\;\;\;\;\;\;\;\left(α_1-α_2\right)+\left(α_2-α_3\right)+\left(α_3-α_4\right)+\left(α_4-α_1\right)=0,\\\;\;\;\;\;\;\;\;\left(α_1+α_2\right)-\left(α_2+α_3\right)+\left(α_3-α_4\right)+\left(α_4-α_1\right)=0\;,\;\;\\\mathrm{因此只有向量组}α_1+α_2,α_2+α_3,α_3+α_4,α_4-α_1\mathrm{线性无关},\mathrm{前三组线性相关}.\end{array}
$$



$$
\begin{array}{l}\mathrm{已知向量组}α_1,\;α_2,α_3,α_4\mathrm{线性无关},\mathrm{则下列向量组中线性无关的向量组有}\left(\;\;\;\;\right)个.\\\begin{array}{l}\left(1\right)\;α_1+α_2,α_2+α_3,α_3+α_4,α_4+α_1\\\left(2\right)\;α_1,\;α_1+α_2,α_2+α_3,α_3+α_4\\\left(3\right)\;\;α_1+α_2,\alpha_2+α_3,α_3+α_4,α_4\\\left(4\right)\;α_1,α_1+α_2,α_1+\;α_2+α_3,α_1+\;α_2+α_3+α_4\end{array}\end{array}
$$
$$
A.
1 \quad B.2 \quad C.3 \quad D.4 \quad E. \quad F. \quad G. \quad H.
$$
$$
\begin{array}{l}\mathrm{因为}\left(α_1+α_2\right)-\left(α_2+α_3\right)+\left(α_3+α_4\right)-\left(α_4+α_1\right)=0,\mathrm{所以四个向量线性相关}.\;\;\\\mathrm{其余三个依次表示为}\;\;\\\left(β_1,β_2,β_3\right)=\left(α_1,α_2,α_3\right)\begin{pmatrix}1&1&0&0\\0&1&1&0\\0&0&1&1\\0&0&0&1\end{pmatrix},\\\left(β_1,β_2,β_3\right)=\left(α_1,α_2,α_3\right)\begin{pmatrix}1&0&0&0\\1&1&0&0\\0&1&1&0\\0&0&1&1\end{pmatrix},\;\;\\\left(β_1,β_2,β_3\right)=\left(α_1,α_2,α_3\right)\begin{pmatrix}1&1&1&1\\0&1&1&1\\0&0&1&1\\0&0&0&1\end{pmatrix},\mathrm{均线性无关}.\end{array}
$$



$$
\begin{array}{l}设n\mathrm{维列向量组}α_1,α_2,⋯,α_m\left(m<\;n\right)\mathrm{线性无关},则n\mathrm{维列向量组}β_1,β_2,⋯,β_m\mathrm{线性无关的充分条件有}\;\left(\;\;\;\;\right)个\\(1)\mathrm{向量组}α_1,α_2,⋯,α_m\mathrm{可由向量组}β_1,β_2,⋯,β_m\mathrm{线性表示}\\(2)\mathrm{向量组}β_1,β_2,⋯,β_m\mathrm{可由向量组}α_1,α_2,⋯,α_m\mathrm{线性表示}\;\\(3)\mathrm{向量组}α_1,α_2,⋯,α_m\mathrm{与向量组}β_1,β_2,⋯,β_m\mathrm{等秩}\;\;\\(4)\mathrm{向量组}α_1,α_2,⋯,α_m\mathrm{与向量组}β_1,β_2,⋯,β_m\mathrm{等价}\end{array}
$$
$$
A.
1 \quad B.2 \quad C.3 \quad D.4 \quad E. \quad F. \quad G. \quad H.
$$
$$
\begin{array}{l}\mathrm{充分性}:\mathrm{若向量组}α_1,α_2,⋯,α_m\mathrm{线性无关},且α_1,α_2,⋯,α_m与β_1,β_2,⋯,β_m\mathrm{等价},\mathrm{由于等价向量组有相同的}\\秩,则r\left(β_1,β_2,⋯,β_m\right)=r\left(α_1,α_2,⋯,α_m\right),即β_1,β_2,⋯,β_m\mathrm{也线性无关}.\end{array}
$$



$$
\begin{array}{l}\mathrm{设三维列向量}α_1,α_2,α_3\mathrm{线性无关},A\mathrm{是三阶矩阵},\mathrm{且有}\\Aα_1=α_1+2α_2+3α_3,Aα_2=2α_2+3α_3,Aα_3=3α_2-4α_3,\\则\left|A\right|=\left(\;\;\;\;\right)\end{array}
$$
$$
A.
-17 \quad B.17 \quad C.-8 \quad D.8 \quad E. \quad F. \quad G. \quad H.
$$
$$
\begin{array}{l}\mathrm{由条件知}\;\\\;\;\;\;\;\;\;A\left(α_1,α_2,α_3\right)\;=\left(Aα_1,Aα_2,Aα_3\right)\;\;\;\;\\\;\;\;\;\;\;\;\;\;\;\;\;\;\;\;\;\;\;\;\;\;\;\;\;\;\;\;\;\;\;\;\;\;\;\;=\left(α_1+2α_2+3α_3,2α_2+3α_3,3\alpha_2-4α_3\right)\;\\\;\;\;\;\;\;\;\;\;\;\;\;\;\;\;\;\;\;\;\;\;\;\;\;\;\;\;\;\;\;\;\;\;\;\;\;=\;\left(α_1,α_2,α_3\right)\begin{pmatrix}1&0&0\\2&2&3\\3&3&-4\end{pmatrix}\;\;\;\;\;\;\;\;\;\;\;\;\;\;\;\;\;\;\;\;\;\;\;\;\;\;\;\;\;\;\;\;\;\;\;\;\\\mathrm{由于}α_1,α_2,α_3\mathrm{线性无关},则\left|α_1,α_2,α_3\right|\neq0\;故\;\\\;\;\;\;\;\;\;\;\;\;\;\;\;\;\;\;\;\;\;\;\;\;\;\;\left|A\right|=\begin{vmatrix}1&0&0\\2&2&3\\3&3&-4\end{vmatrix}=-17\\\;\;\;\;\;\;\;\;\;\;\;\;\;\;\;\;\;\;\;\;\;\;\;\;\;\;\;\;\;\;\;\;\;\;\;\;\;\;\;\;\;\;\;\;\;\;\;\;\;\;\;\;\;\;\;\;\;\;\;\;\;\;\;\;\;\;\;\;\;\;\;\;\;\;\;\;\;\;\;\;\;\;\;\;\;\;\;\;\;\;\;\;\;\;\;\;\;\;\;\;\;\;\;\;\;\;\;\;\;\;\;\;\;\;\;\;\;\;\end{array}
$$



$$
\begin{array}{l}\mathrm{设向量组}α_1,α_2,α_3\mathrm{线性相关},\mathrm{向量组}α_2,α_3,α_4\mathrm{线性无关},\mathrm{则下列说法正确的是}\left(\;\;\;\right).\;\\(1)α_1\mathrm{能由}α_2,α_3\mathrm{线性表示};\;\;\\(2)α_4\mathrm{不能由}α_1,α_2,α_3\mathrm{线性表示}.\end{array}
$$
$$
A.
(1) \quad B.(2) \quad C.(1)(2) \quad D.\mathrm{都不正确} \quad E. \quad F. \quad G. \quad H.
$$
$$
\begin{array}{l}(1)因α_2,α_3,α_4\mathrm{线性无关},故α_2,α_3\mathrm{线性无关},而α_1,α_2,α_3\mathrm{线性相关},\mathrm{从而}α_1\mathrm{能由}α_2,α_3\mathrm{线性表示}.\;\\(2)\;\mathrm{假设}α_4\mathrm{能由}α_1,α_2,α_3\mathrm{表示},\mathrm{而由}(1)知α_1\mathrm{能由}α_2,α_3\mathrm{表示},\mathrm{因此}α_4\mathrm{能由}α_2,α_3\mathrm{线性表示},这α_2,α_3,α_4\mathrm{与线性无关矛盾}.\;\;\\\mathrm{因此}(1)(2)\mathrm{都正确}.\end{array}
$$



$$
\mathrm{已知向量组}α_1,α_2,α_3,α_4\mathrm{线性无关},\mathrm{则下列向量组中线性无关的是}\left(\;\;\right).
$$
$$
A.
α_1+α_2,α_2+α_3,α_3+α_4,α_4+α_1 \quad B.α_1-α_2,α_2-α_3,α_3-α_4,α_4-α_1 \quad C.α_1+α_2,α_2+α_3,α_3-α_4,α_4-α_1 \quad D.α_1+α_2,α_2+α_3,α_3+α_4,α_4-α_1 \quad E. \quad F. \quad G. \quad H.
$$
$$
\begin{array}{l}\mathrm{因为}\\\begin{array}{l}\left(α_1+α_2\right)-\left(α_2+α_3\right)+\left(α_3+α_4\right)-\left(α_4+α_1\right)=0,\;\;\;\;\;\\\left(α_1-α_2\right)+\left(α_2-α_3\right)+\left(α_3-α_4\right)+\left(α_4-α_1\right)=0,\\\left(α_1+α_2\right)-\left(α_2+α_3\right)+\left(α_3-α_4\right)+\left(α_4-α_1\right)=0\;,\;\;\end{array}\\\mathrm{因此只有向量组}α_1+α_2,α_2+α_3,α_3+α_4,α_4-α_1\mathrm{线性无关}\end{array}
$$



$$
\begin{array}{l}设α,β,γ 与ξ,η,ζ\mathrm{为两个六维向量组},\mathrm{若存在着两组不全为零的数}a,b,c与k,l,m使\\\left(a+k\right)α+\left(b+l\right)β+\left(c+m\right)γ+\left(a-k\right)ξ+\left(b-l\right)η+\left(c-m\right)ζ=0,则\left(\;\;\;\right).\end{array}
$$
$$
A.
α,β,γ 和ξ,η,ζ\mathrm{都线性相关} \quad B.α,β,γ 和ξ,η,ζ\mathrm{都线性无关} \quad C.α+ξ,β+η,γ+ζ,α-ξ,β-η,γ-ζ\mathrm{线性相关} \quad D.α+ξ,β+η,γ+ξ,α-ξ,β-η,γ-ζ\mathrm{线性无关} \quad E. \quad F. \quad G. \quad H.
$$
$$
\begin{array}{l}将\left(a+k\right)α+\left(b+l\right)β+\left(c+m\right)γ+\left(a-k\right)ξ+\left(b-l\right)η+\left(c-m\right)ζ=0\mathrm{展开},\mathrm{再按系数}a,b,c与k,l,m\mathrm{合并得}\;\;\;\\\;\;\;\;a\left(α+ξ\right)+b\left(β+η\right)+c\left(γ+ζ\right)+k\left(α-ξ\right)+l\left(β-η\right)+m\left(γ-ζ\right)=0\\\mathrm{由于}a,b,c与k,l,m\mathrm{不全为零},故α+ξ,β+η,γ+ζ,α-ξ,β-η,γ-ζ\mathrm{线性相关}.\end{array}
$$



$$
若α_1,α_2\mathrm{线性无关},α_1+β,α_2+β\mathrm{线性相关},\mathrm{则向量}β 由α_1,α_2\mathrm{线性表示的表示式为}\left(\;\;\;\right).
$$
$$
A.
β=kα_1-\left(1+k\right)α_2 \quad B.β=kα_1+\left(1+k\right)α_2 \quad C.β=kα_1-\left(1-k\right)α_2 \quad D.β=kα_1+\left(1-k\right)α_2 \quad E. \quad F. \quad G. \quad H.
$$
$$
\begin{array}{l}\begin{array}{l}因α_1+β,α_2+β\mathrm{线性相关},\mathrm{故存在不全为零的常数}k_1,k_2\;,使\;\;\;\\\;\;\;\;\;\;\;\;\;\;\;\;\;\;\;\;k_1\left(α_1+β\right)\;+\;k_2\left(α_2+β\right)=0\\\;\;\;\;\;\;\;\;\;\;\;\;\;\;\;\;\;⇒\left(k_1+k_2\right)\;β=-k_1α_1\;-k_2α_2\;.\;\;\;\;\;\;\;\;\;\;\;\;\;(*)\;\;\;\\因α_1,α_2\mathrm{线性无关},故\;k_1+k_2\neq0,\mathrm{不然},由(*)\;\mathrm{式得}\;\;\;\;\\\;\;\;\;\;\;\;\;\;\;\;\;\;\;\;\;\;\;\;k_1α_1\;-k_2α_2=0⇒ k_1=k_2=0\\\mathrm{所以}\;\;\;β=\frac{k_1}{k_1+k_2}α_1-\frac{k_2}{k_1+k_2}α_2,k_1,k_2∈ R,\;k_1+k_2\neq0\end{array}\\=kα_1-\left(1+k\right)α_2,\;k=-\frac{k_1}{k_1+k_2},\;k∈ R\end{array}
$$



$$
\begin{array}{l}\mathrm{下列命题中},\mathrm{正确的有}\left(\;\;\right)\;\\\left(1\right)α_1,α_2\mathrm{线性相关},β_1,β_2\mathrm{也线性相关},则α_1+β_1,α_2\;+β_2\mathrm{一定线性相关};\;\;\;\\\left(2\right)β_1=α_1,β_2=α_1+α_2,…,β_r=α_1+α_2+…+α_r,\mathrm{且向量组}α_1,α_2,…,α_r\mathrm{线性无关},\mathrm{则向量组}\\β_1,β_2,…,β_r\mathrm{线性无关}.\;\;\\\left(3\right)\mathrm{向量组}α_1,α_2,…,α_s\mathrm{线性相关},\mathrm{且其中任意}s-1\mathrm{个向量都线性无关},\mathrm{则必存在一组全部不为零的数}\;\\k_1,k_2,…,k_s,使k_1α_1+k_2α_2+…+k_sα_s=0\end{array}
$$
$$
A.
0个 \quad B.1个 \quad C.2个 \quad D.3个 \quad E. \quad F. \quad G. \quad H.
$$
$$
\begin{array}{l}\left(1\right)\mathrm{的说法不正确}\;\;\\例α_1=\begin{pmatrix}1\\0\end{pmatrix},α_2=\begin{pmatrix}2\\0\end{pmatrix},β_1=\begin{pmatrix}0\\2\end{pmatrix},β_2=\begin{pmatrix}0\\3\end{pmatrix}⇒α_1+β_1=\begin{pmatrix}1\\2\end{pmatrix},\;α_2+β_2=\begin{pmatrix}2\\3\end{pmatrix}\;\;\;\;\\\;\;\;\;\;\;\;\;\;⇒α_1,α_2\mathrm{线性相关},β_1,β_2\mathrm{也线性相关},但α_1+β_1,\;α_2+β_2\mathrm{线性无关}.\;\;\\\left(2\right)\mathrm{的说法正确}\;\;\;设A=\left(α_1,α_2,…,\;α_r\right),B=\left(β_1,β_2,…,\;β_r\right)\;\;\\\mathrm{依题}\;B=AP\\P=\begin{pmatrix}1&…&…&1\\0&1&…&1\\\vdots&…&…&\vdots\\0&…&0&1\end{pmatrix},\left|P\right|=1\neq0⇒ R\left(A\right)=R\left(B\right).\;\;\;\\\mathrm{因向量组}α_1,α_2,…,α_r\mathrm{线性无关},\mathrm{所以}β_1,β_2,…,β_r\mathrm{线性无关}.\;\;\\\left(3\right)\mathrm{的说法也正确}.\;\;\;\;\mathrm{由线性}α_1,α_2,…,α_r\mathrm{相关},\\\mathrm{故存在不全为零的数}\;k_1,k_2,…,k_s,使\;k_1α_1+k_2α_2+…+k_sα_s=0,\\若\;k_1,k_2,…,k_s\mathrm{全部不为零},\mathrm{即得证}.\;\;\\若\;k_1,k_2,…,k_s\mathrm{中至少有一个为零},\mathrm{不妨设}k_1=0\;\mathrm{则有不全为零的}k_2,…,k_s,使\;\;\\\;\;\;\;\;\;\;\;\;\;k_1α_1+k_2α_2+…+k_sα_s=0⇒α_2,…,α_s\;\;\mathrm{线性相关}.\;\;\;\\\mathrm{与题设矛盾}.\mathrm{因此}k_1,k_2,…,k_s\mathrm{全部不为零}.\end{array}
$$



$$
\begin{array}{l}\mathrm{下列命题中不正确的有}\left(\;\;\;\right).\;\\\left(1\right)\;\mathrm{向量组}A:α_1=\left(1,2,1,3\right)^T,α_2=\left(4,-1,-5,-6\right)^{T,}与\;\mathrm{向量组}\\B:β_1=\left(-1,3,4,7\right)^T,β_2=\left(2,-1,-3,-4\right)^T\mathrm{等价};\;\;\\\left(2\right)α_1,α_2,…,α_n\mathrm{是一组}n\mathrm{维向量},\mathrm{已知}n\mathrm{维单位坐标向量}ε_1,ε_2,…,ε_n\mathrm{能由它们线性表示},则\\α_1,α_2,…,α_n\mathrm{线性无关};\;\;\\\left(3\right)α_1,α_2,…,α_n\mathrm{是一组}n\mathrm{维向量},\mathrm{它们线性无关的充分必要条件是任一}n\mathrm{维向量都可由它们线性表示}.\end{array}
$$
$$
A.
0个 \quad B.1个 \quad C.2个 \quad D.3个 \quad E. \quad F. \quad G. \quad H.
$$
$$
\begin{array}{l}\;\left(1\right)\left(α_1,α_2,β_1,β_1\right)\;=\begin{pmatrix}1&4&-1&2\\2&-1&3&-1\\3&-5&4&-3\\4&-6&7&-4\end{pmatrix}\;\xrightarrow[{r_4-3r_1}]{\begin{array}{c}r_2-2r_1\\r_3-r_1\end{array}}\;\;\begin{pmatrix}1&4&-1&2\\0&-9&5&-5\\0&-9&5&-5\\0&-18&10&-10\end{pmatrix}\;\;\\\;\;\;\;\;\;\;\;\;\;\;\;\;\;\;\;\;\;\;\;\;\;\;\;\;\;\;\;\;\;\;\;\xrightarrow[{r_2\div\left(-9\right)}]{\begin{array}{c}r_3-r_2\\r_4-2r_2\end{array}}\;\;\begin{pmatrix}1&4&-1&2\\0&1&-5/9&5/9\\0&0&0&0\\0&0&0&0\end{pmatrix}\;\xrightarrow{r_1-4r_2}\;\begin{pmatrix}1&0&\textstyle\frac{11}9&-{\textstyle\frac29}\\0&1&-{\textstyle\frac59}&\textstyle\frac59\\0&0&0&0\\0&0&0&0\end{pmatrix}\;\;\;\;\;\;\;\;\;\;\;\;\;\;\;\;\;\;\;\;\;\;\;\;\;\;\;\\\mathrm{易见向量组}A\mathrm{与向量组}B\mathrm{有相同的秩},\mathrm{故向量组}A\mathrm{与向量组}B\mathrm{等价}.\;\;\\\left(2\right)\;\mathrm{提示}:\mathrm{将条件中的线性表示式转化为矩阵形式},\mathrm{再利用向量组}α_1,α_2,…,α_n\mathrm{构成的矩阵行列式非零即可得到结论};\;\;\\\left(3\right)\mathrm{充分性显然};\mathrm{必要性以一组}n\mathrm{维单位向量组为中介},\mathrm{证明此单位向量组也能由}α_1,α_2,…,α_n\mathrm{线性表示},\mathrm{再利用}(2)\mathrm{的结论即可}.\;\;\\因\left(1\right)\left(2\right)\left(3\right)\mathrm{此都正确}.\end{array}
$$



$$
\begin{array}{l}\mathrm{下列说法正确的为}\;\left(\;\;\;\right)\\\left(1\right)A是n× m\mathrm{矩阵},B是m× n\mathrm{矩阵},且n<\;m,E是n\mathrm{阶单位矩阵},若AB=E,则B\mathrm{的列向量组线性无关};\;\\\left(2\right)\;设A为4×3\mathrm{矩阵},B为3×3\mathrm{矩阵},且AB=O\;\mathrm{其中}A=\begin{pmatrix}1&1&-1\\1&2&1\\2&3&0\\0&-1&-2\end{pmatrix}\;,则B\mathrm{的列向量组线性无关}.\end{array}
$$
$$
A.
\left(1\right) \quad B.\left(2\right) \quad C.\left(1\right)\left(2\right) \quad D.\mathrm{都不正确} \quad E. \quad F. \quad G. \quad H.
$$
$$
\begin{array}{l}\left(1\right)\mathrm{因为}R\left(B\right)\leq min\left\{\left.m,n\right\}\right.\leq n,又R\left(B\right)\geq R\left(AB\right)=\left(E\right)=n,故R\left(B\right)\;=n,\mathrm{所以}B\mathrm{的列向量组线无关};\;\;\\\left(2\right)由AB=O\;知R\left(A\right)+R\left(B\right)\leq3\;\;\;\;又R\left(A\right)=2,\mathrm{所以}R\left(B\right)\leq3-2=1\;\;故B\mathrm{的列向量一定线性相关}.\;\;\\\mathrm{或反证}\left(2\right):设B\mathrm{的列向量线性无关},则R\left(B\right)=3,AB=O⇒ ABB^{-1}=O⇒ A=O,\mathrm{与已知矛盾},\;\\\;\mathrm{因此}\left(1\right)\left(2\right)\mathrm{都正确}.\end{array}
$$



$$
\begin{array}{l}\;\mathrm{下列说法正确的是}(\;\;\;).\;\\(1)\mathrm{设向量组}α_1,α_2,α_31\mathrm{线性无关},则β_1=α_1+2α_2+3α_3,β_2=-α_1+α_3,β_3=3α_1+3α_2+4α_3\mathrm{线性无关};\;\;\\(2)\mathrm{设向量组}α,β,γ,\mathrm{线性无关},\mathrm{则向量组}α+β,α-β,α-2β+γ\mathrm{线性无关}.\end{array}
$$
$$
A.
\left(1\right) \quad B.\left(2\right) \quad C.\left(1\right)\left(2\right) \quad D.\mathrm{都不正确} \quad E. \quad F. \quad G. \quad H.
$$
$$
\begin{array}{l}\begin{array}{l}\begin{array}{l}(1)设k_1β_1+k_2β_2+k_3β_3=0,即\;\;\\\left(k_1-k_2+3k_3\right)α_1+\left(2k_1+3k_3\right)α_2+\left(3k_1+k_2+4k_3\right)α_3=0.\\由α_1,α_2,α_3\mathrm{线性无关知}:\left\{\begin{array}{l}\begin{array}{c}k_1-k_2+3k_3=0\\2k_1+3k_3=0\end{array}\\\begin{array}{c}3k_1+k_2+4k_3=0\end{array}\end{array}\right..\;\\\mathrm{该方程组只有零解}k_1=k_2=k_3=0,故β_1,β_2,β_3\mathrm{线性无关}.\;\\(2)设k_1\left(α+β\right)+k_2\left(α-β\right)+k_3\left(α-2β+γ\right)=0,即\;\\\;\;\;\;\left(k_1+k_2+k_3\right)α+\left(k_1-k_2-k_3\right)β+k_3γ=0.\\由α,β,γ\mathrm{线性无关得}\left\{\begin{array}{l}\begin{array}{c}k_1+k_2+k_3=0\\k_1-k_2-2k_3=0\end{array}\\\begin{array}{c}k_3=0\end{array}\end{array}\right.,\;\end{array}\\\mathrm{它只有零解},k_1=k_2=k_3=0,\mathrm{故所讨论的向量组线性无关}.\;\\\mathrm{另解}:\left(1\right),\left(β_1,β_2,β_3\right)=\left(α_1,α_2,α_3\right)\begin{pmatrix}1&-1&3\\2&0&3\\3&3&4\end{pmatrix},\mathrm{记为}B=AP,\left|P\right|\neq0⇒ P\mathrm{可逆},\mathrm{二者等价等秩},\mathrm{即得线性无关}.\;\end{array}\\\;\left(2\right)\left(α+β,α-β,α-2β+γ\right)=\left(α,β,γ\right)\begin{pmatrix}1&1&1\\1&-1&-2\\0&0&1\end{pmatrix},记B_1=A_1P_1,\left|P_1\right|\neq0,P_1\mathrm{可逆},\mathrm{二者等价等秩},\mathrm{即得线性无关}.\\\end{array}
$$



$$
\begin{array}{l}\mathrm{已知向量组}α_1,α_2,α_3,α_4\mathrm{线性无关},\mathrm{则下列向量组中}\left(\;\;\;\right)\mathrm{线性无关}.\\\left(1\right)α_1+α_2+\;α_3+\;α_4,α_2+\;α_3+\;\alpha_4,α_3+\;\alpha_4,\;α_4\\\left(2\right)α_1,α_1+α_2,α_2+\;α_3,α_3+\;α_4\\\left(3\right)α_1+α_2,α_2+\;α_3,α_3+\;α_4,\;α_4\\\left(4\right)α_1,α_1+α_2,α_1+α_2+\;α_3,α_1+α_2+α_3+\;α_4\\\end{array}
$$
$$
A.
\left(1\right)\left(4\right) \quad B.\left(2\right)\left(3\right) \quad C.\left(1\right)\left(2\right)\left(3\right) \quad D.\left(1\right)\left(2\right)\left(3\right)\left(4\right) \quad E. \quad F. \quad G. \quad H.
$$
$$
\begin{array}{l}\mathrm{四个向量组可以依次表示为}\;\;\\\left(β_1,β_2,β_3,β_4\right)=\left(α_1,α_2,α_3,α_4\right)\begin{pmatrix}1&0&0&0\\1&1&1&0\\1&1&1&0\\1&1&1&1\end{pmatrix},\\\left(β_1,β_2,β_3,β_4\right)=\left(α_1,α_2,α_3,α_4\right)\begin{pmatrix}1&1&0&0\\0&1&1&0\\0&0&1&0\\0&0&0&1\end{pmatrix},\\\left(β_1,β_2,β_3,β_4\right)=\left(α_1,α_2,α_3,α_4\right)\begin{pmatrix}1&0&0&0\\1&1&0&0\\0&1&1&0\\0&0&1&1\end{pmatrix},\\\left(β_1,β_2,β_3,β_4\right)=\left(α_1,α_2,α_3,α_4\right)\begin{pmatrix}1&1&1&1\\0&1&1&1\\0&0&1&1\\0&0&0&1\end{pmatrix},\;\;\\\mathrm{四个向量组均线性无关}.\end{array}
$$



$$
\mathrm{设向量组}A\mathrm{的秩为}r_1,\mathrm{向量组}B\mathrm{的秩为}r_2,A\mathrm{组可由}B\mathrm{组线性表示},则A与B\mathrm{的关系为}().\;
$$
$$
A.
r_1\geq r_2 \quad B.r_1\leq r_2 \quad C.r_1=r_2 \quad D.\mathrm{不能确定} \quad E. \quad F. \quad G. \quad H.
$$
$$
A\mathrm{组可由}B\mathrm{组线性表示},则r_1\leq r_2.
$$



$$
\text{设}A\text{为}3×4\mathrm{矩阵},\text{且}R\left(A\right)=2,\mathrm{则下列结论中},\mathrm{不正确的是}().
$$
$$
A.
A\mathrm{的所有}3\mathrm{阶子式都为零} \quad B.A\mathrm{的所有}2\mathrm{阶子式都不为零} \quad C.A\mathrm{的列向量组线性相关} \quad D.A\mathrm{的行向量组线性相关} \quad E. \quad F. \quad G. \quad H.
$$
$$
\begin{array}{l}\mathrm{根据矩阵秩的定义可知},\mathrm{原矩阵的所有}3\mathrm{阶子式全为零},\mathrm{且存在一个}2\mathrm{阶子式不为零};\mathrm{且矩阵的行}、\mathrm{列向量}\\\mathrm{组的秩都为}2,\mathrm{因此都线性相关}.\end{array}
$$



$$
\text{设}α_1=\left(1,1,-2\right)^T,\alpha_2=\left(0,0,1\right)^T,α_3=\left(1,-1,0\right)^T,α_4=\left(3,-1,-1\right)^T,则α_{1,}α_{2,}α_{3,}α_4\mathrm{向量组的秩为}().\;
$$
$$
A.
2 \quad B.4 \quad C.1 \quad D.3 \quad E. \quad F. \quad G. \quad H.
$$
$$
\left(α_{1,}α_{2,}α_{3,}α_4\right)=\begin{pmatrix}1&0&1&3\\1&0&-1&-1\\-2&1&0&-1\end{pmatrix}\rightarrow\begin{pmatrix}1&0&1&3\\0&1&2&5\\0&0&1&2\end{pmatrix}\mathrm{故向量组的秩为}3.
$$



$$
\text{设}α_1=\left(1,2,3\right)^T,α_2=\left(4,5,6\right)^T,α_3=\left(7,8,9\right)^T,\mathrm{则向量组}α_{1,}α_{2,}α_{3,}().\;
$$
$$
A.
\mathrm{其秩为}2 \quad B.\mathrm{线性无关} \quad C.\mathrm{其秩为}0 \quad D.\mathrm{其秩为}1 \quad E. \quad F. \quad G. \quad H.
$$
$$
\begin{array}{l}\begin{array}{l}将α_{1,}α_{2,}α_{3,}\mathrm{构成的矩阵进行初等行变换}\\\left(α_{1,}α_{2,}α_3\right)=\begin{pmatrix}1&4&7\\2&5&8\\3&6&9\end{pmatrix}\rightarrow\begin{pmatrix}1&4&7\\0&1&2\\0&0&0\end{pmatrix}\end{array}\\\mathrm{故向量组线性相关},\mathrm{且秩为}2.\end{array}
$$



$$
\text{设}α_1=\left(1,1,0\right)^T,α_2=\left(3,0,-9\right)^T,α_3=\left(1,2,3\right)^T,α_4=\left(1,-1,-6\right)^T,\mathrm{则向量组}α_{1,}α_{2,}α_{3,}α_4\mathrm{的秩为}().\;
$$
$$
A.
1 \quad B.2 \quad C.3 \quad D.4 \quad E. \quad F. \quad G. \quad H.
$$
$$
\begin{pmatrix}1&3&1&1\\1&0&2&-1\\0&-9&3&-6\end{pmatrix}\rightarrow\begin{pmatrix}1&3&1&1\\0&3&-1&2\\0&0&0&0\end{pmatrix},\mathrm{故向量组的秩为}2.
$$



$$
\mathrm{已知向量组}α_1=\left(1,2,-1,1\right)^{},α_2=\left(2,0,t,0\right)^{},α_3=\left(0,-4,5,-2\right)^{},\mathrm{秩为}2,\text{则}\textit{t}\text{=().}
$$
$$
A.
1 \quad B.2 \quad C.3 \quad D.4 \quad E. \quad F. \quad G. \quad H.
$$
$$
\begin{array}{l}\left(α_{1,}α_{2,}α_3\right)=\begin{pmatrix}1&2&-1&1\\2&0&t&0\\0&-4&5&-2\end{pmatrix}\rightarrow\begin{pmatrix}1&2&-1&1\\0&-4&t+2&-2\\0&-4&5&-2\end{pmatrix}\rightarrow\begin{pmatrix}1&2&-1&1\\0&-4&t+2&-2\\0&0&3-t&0\end{pmatrix},\\\mathrm{由于}r\left(α_{1,}α_{2,}α_3\right)=2⇒3-t=0,\text{即}\textit{t}\text{=3}.\end{array}
$$



$$
\begin{array}{l}\mathrm{设向量组}\left(Ⅰ\right)α_{1,}α_{2,}...,α_{s,};\left(Ⅱ\right)β_1,β_2,...,β_t\mathrm{的秩分别是}r_1\text{和}r_2,\text{且}\left(Ⅰ\right)\mathrm{中每个向量都可由}\left(Ⅱ\right)\mathrm{线性表示},\\\text{则}r_1\text{和}r_2\mathrm{的关系是}().\;\end{array}
$$
$$
A.
r_1\leq r_2 \quad B.r_1=r_2 \quad C.r_1\geq r_2 \quad D.r_1\;<\;r_2 \quad E. \quad F. \quad G. \quad H.
$$
$$
\left(Ⅰ\right)\mathrm{中每个向量都可由}\left(Ⅱ\right)\mathrm{线性表示},\mathrm{则向量组}\left(Ⅰ\right)\mathrm{可由}\left(Ⅱ\right)\mathrm{线性表示},\text{则R}\left(Ⅰ\right)\text{≤}R\left(Ⅱ\right),\text{即}r_1\leq r_2.
$$



$$
\mathrm{已知向量组}\left(Ⅰ\right)α_1=\left(1,2,-1\right)^T,α_2=\left(2,-3,1\right)^T,α_3=\left(4,1,-1\right)^T,\mathrm{如果向量组}\left(Ⅱ\right)\mathrm{与向量组}\left(Ⅰ\right)\mathrm{等价},\;\;\mathrm{则向量组}\left(Ⅱ\right)\mathrm{的秩为}().
$$
$$
A.
2 \quad B.3 \quad C.1 \quad D.\mathrm{无法确定} \quad E. \quad F. \quad G. \quad H.
$$
$$
\begin{array}{l}\left(α_{1,}α_{2,}α_3\right)=\begin{pmatrix}1&2&4\\2&-3&1\\-1&1&-1\end{pmatrix}\rightarrow\begin{pmatrix}1&2&4\\0&-7&-7\\0&3&3\end{pmatrix}\rightarrow\begin{pmatrix}1&2&4\\0&1&1\\0&0&0\end{pmatrix},\\\mathrm{故向量组}\left(Ⅰ\right)\mathrm{的秩为}2,\mathrm{由于等价向量组的秩相等},\mathrm{故向量组}\left(Ⅱ\right)\mathrm{的秩也为}2.\end{array}
$$



$$
n\mathrm{维基本单位向量}e_1,e_2,...,e_n,\mathrm{均可由向量组}α_1,α_2,...,α_r\mathrm{线性表出},\mathrm{则向量个数}n,r\mathrm{的关系为}().
$$
$$
A.
n\leq r \quad B.n\geq r \quad C.n\;<\;r \quad D.n>r \quad E. \quad F. \quad G. \quad H.
$$
$$
\begin{array}{l}\mathrm{由条件可知}R\left(e_1,e_2,...e_n\right)\leq R\left(a_1,a_2,...a_r\right),\text{又}R\left(e_1,e_2,...e_n\right)=n,\;\;\\\text{故n≤}R\left(a_1,a_2,...a_r\right)\leq r⇒ n\leq r.\end{array}
$$



$$
\mathrm{向量组}α=\left(1,2,3,4\right)^T,β=\left(2,3,4,5\right)^T,γ=\left(3,4,5,6\right)^T,δ=\left(4,5,6,7\right)^T\mathrm{的秩是}().
$$
$$
A.
2 \quad B.1 \quad C.3 \quad D.4 \quad E. \quad F. \quad G. \quad H.
$$
$$
\begin{array}{l}\left(α,β,γ,δ\right)=\begin{pmatrix}1&2&3&4\\2&3&4&5\\3&4&5&6\\4&5&6&7\end{pmatrix}\rightarrow\begin{pmatrix}1&2&3&4\\1&1&1&1\\1&1&1&1\\1&1&1&1\end{pmatrix}\rightarrow\begin{pmatrix}1&2&3&4\\1&1&1&1\\0&0&0&0\\0&0&0&0\end{pmatrix}\rightarrow\begin{pmatrix}1&2&3&4\\0&-1&-2&-3\\0&0&0&0\\0&0&0&0\end{pmatrix}\\\mathrm{故向量组的秩为}2.\end{array}
$$



$$
\text{已知向量组}α_1=\left(1,0,0,-3,4\right)^T,α_2=\left(0,1,0,6,15\right)^T,α_3=\left(0,0,1,6,7\right)^T,\text{则该向量组的秩为}().
$$
$$
A.
1 \quad B.2 \quad C.3 \quad D.4 \quad E. \quad F. \quad G. \quad H.
$$
$$
\begin{array}{l}令A=(α_1,α_2,α_3)=\begin{pmatrix}1&0&0\\0&1&0\\0&0&1\\-3&6&6\\4&15&7\end{pmatrix}\;,则R(A)\;\leq min\left\{\left.3,5\right\}\right.=3;\;\\\mathrm{又矩阵}A\mathrm{中存在一个三阶子式}\begin{vmatrix}1&0&0\\0&1&0\\0&0&1\end{vmatrix}=1\neq0,则R(A)\geq3,\\\;故R(A)=3,\mathrm{即向量组的秩为}3.\end{array}
$$



$$
\begin{array}{l}\text{设}α_1=\left(2,1,1,-1\right)^T,α_2=\left(1,2,1,3\right)^T,α_3=\left(1,1,2,5\right)^T,\mathrm{如果向量组}β_1,β_2,β_3,β_4\mathrm{与向量组}α_1,α_2,α_3\mathrm{等价},\\\mathrm{则向量组}β_1,β_2,β_3,β_4\mathrm{的秩等于}().\end{array}
$$
$$
A.
1 \quad B.2 \quad C.3 \quad D.4 \quad E. \quad F. \quad G. \quad H.
$$
$$
\begin{array}{l}\mathrm{等价向量组有相同的秩},\mathrm{下面用初等行变换求的秩}:\\\begin{pmatrix}2&1&1\\1&2&1\\1&1&2\\-1&3&5\end{pmatrix}\rightarrow\begin{pmatrix}1&1&2\\1&2&1\\2&1&1\\-1&3&5\end{pmatrix}\rightarrow\begin{pmatrix}1&1&2\\0&1&-1\\0&-1&-3\\0&4&7\end{pmatrix}\rightarrow\begin{pmatrix}1&1&2\\0&1&-1\\0&0&-4\\0&0&11\end{pmatrix}\rightarrow\begin{pmatrix}1&1&2\\0&1&-1\\0&0&1\\0&0&0\end{pmatrix},\\\mathrm{故向量组}R\left(β_1,β_2,β_3,β_4\right)=R\left(α_1,α_2,α_3\right)=3.\\\end{array}
$$



$$
\begin{array}{l}\text{设}α_1=\begin{pmatrix}0\\3\\1\end{pmatrix},α_2=\begin{pmatrix}2\\4\\0\end{pmatrix},α_3=\begin{pmatrix}2\\7\\1\end{pmatrix},\alpha_4=\begin{pmatrix}4\\11\\1\end{pmatrix},\mathrm{则此向量组的秩为}\;().\\\end{array}
$$
$$
A.
1 \quad B.2 \quad C.3 \quad D.4 \quad E. \quad F. \quad G. \quad H.
$$
$$
\begin{array}{l}\mathrm{对下列矩阵施行初等行变换}:\\\left(α_1,α_2,α_3,α_4\right)=\begin{pmatrix}0&2&2&4\\3&4&7&11\\1&0&1&1\end{pmatrix}\rightarrow\begin{pmatrix}1&0&1&1\\3&4&7&11\\0&2&2&4\end{pmatrix}\rightarrow\begin{pmatrix}1&0&1&1\\0&4&4&8\\0&2&2&4\end{pmatrix}\rightarrow\begin{pmatrix}1&0&1&1\\0&1&1&2\\0&0&0&0\end{pmatrix},\\\mathrm{故向量组的秩为}2.\end{array}
$$



$$
设n\mathrm{阶方阵}A\mathrm{的秩}r\;<\;n,\mathrm{则在}A的n\mathrm{个行向量中},().
$$
$$
A.
\mathrm{必有}r\mathrm{个行向量线性无关} \quad B.\mathrm{任意}r\mathrm{个行向量均可构成最大无关组} \quad C.\mathrm{任意}r\mathrm{个行向量均线性无关} \quad D.\mathrm{任一行向量均可由其它}r\mathrm{个行向量线性表示} \quad E. \quad F. \quad G. \quad H.
$$
$$
\mathrm{方阵的}A\mathrm{秩为}r,\mathrm{则必有}r\mathrm{个行向量线性无关}
$$



$$
\mathrm{向量组}α_1=\left(1,2,2\right)^T,α_2=\left(2,4,4\right)^T,α_3=\left(1,0,3\right)^T,α_4=\left(0,4,-2\right)^T\mathrm{的秩为}()\;.
$$
$$
A.
1 \quad B.2 \quad C.3 \quad D.4 \quad E. \quad F. \quad G. \quad H.
$$
$$
\begin{array}{l}\left(α_1,α_2,α_3,α_4\right)=\begin{pmatrix}1&2&1&0\\2&4&0&4\\2&4&3&-2\end{pmatrix}\rightarrow\begin{pmatrix}1&2&1&0\\0&0&-2&4\\0&0&1&-2\end{pmatrix}\rightarrow\begin{pmatrix}1&2&1&0\\0&0&-2&4\\0&0&0&0\end{pmatrix},\\\mathrm{故向量组的秩为}2.\end{array}
$$



$$
\text{设}α_1=\left(1,0,2,1\right)^{},α_2=\left(2,0,1,-1\right)^{},α_3=\left(1,1,0,1\right)^{},α_4=\left(4,1,3,1\right)\mathrm{则向量组}α_1,α_2,α_3,\alpha_4\mathrm{的秩为}().\;
$$
$$
A.
1 \quad B.2 \quad C.3 \quad D.4 \quad E. \quad F. \quad G. \quad H.
$$
$$
\text{令}A=\begin{pmatrix}α_1\\α_2\\α_3\\α_4\end{pmatrix},对A\mathrm{作初等行变换}:A\rightarrow\begin{pmatrix}1&0&2&1\\0&1&-2&0\\0&0&1&1\\0&0&0&0\end{pmatrix},故α_1,α_2,α_3,α_4\mathrm{的秩为}3
$$



$$
\text{设}α_1=\left(1,1,-2\right)^T,α_2=\left(0,0,1\right)^T,\alpha_3=\left(1,-1,0\right)^T,α_4=\left(3,-1,-1\right)^T\mathrm{则向量组}α_1,α_2,α_3,α_4\mathrm{的秩为}().\;
$$
$$
A.
2 \quad B.4 \quad C.1 \quad D.3 \quad E. \quad F. \quad G. \quad H.
$$
$$
\begin{array}{l}\left(α_1,α_2,α_3,\alpha_4\right)=\begin{pmatrix}1&0&1&3\\1&0&-1&-1\\-2&1&0&-1\end{pmatrix}\rightarrow\begin{pmatrix}1&0&1&3\\0&1&2&5\\0&0&1&2\end{pmatrix},\\\mathrm{故向量组的秩为}3.\end{array}
$$



$$
\text{设}m× n\mathrm{型矩阵}A\text{的}m\mathrm{个行向量线性无关},\text{则}A^T\mathrm{矩阵的秩为}().
$$
$$
A.
m \quad B.n \quad C.r \quad D.\mathrm{无法确定} \quad E. \quad F. \quad G. \quad H.
$$
$$
R\left(A\right)=A\mathrm{的行向量组的秩}=A\mathrm{的列向量组的秩},R\left(A_{m× n}\right)=m=R\left({A^T}_{m× n}\right).
$$



$$
\mathrm{设有向量组}α_1=\begin{pmatrix}a\\1\\0\end{pmatrix},α_2=\begin{pmatrix}1\\a\\0\end{pmatrix},α_3=\begin{pmatrix}0\\1\\a\end{pmatrix},\mathrm{那么}a=()时,R\left(α_1,α_2,α_3\right)\;<\;3.\;
$$
$$
A.
0 \quad B.1 \quad C.±1 \quad D.1\text{或}0\text{或-1} \quad E. \quad F. \quad G. \quad H.
$$
$$
\text{由}R\left(α_1,α_2,α_3\right)\;<\;3,\text{知}\begin{vmatrix}a&1&0\\1&a&1\\0&0&a\end{vmatrix}\text{=}a\left(a+1\right)\left(a-1\right)=0⇒ a=0\text{或}a=1\text{或}a=-1.
$$



$$
\mathrm{设有向量组}α_1=\begin{pmatrix}a\\1\\1\end{pmatrix},\alpha_2=\begin{pmatrix}1\\a\\1\end{pmatrix},α_3=\begin{pmatrix}1\\1\\a\end{pmatrix},当R\left(α_1,\alpha_2,α_3\right)\;<\;3时,a=().\;
$$
$$
A.
-2 \quad B.1 \quad C.-1\text{或}2 \quad D.-2\text{或}1 \quad E. \quad F. \quad G. \quad H.
$$
$$
\text{由}R\left(α_1,α_2,α_3\right)\;<\;3,\text{知}\begin{vmatrix}a&1&1\\1&a&1\\1&1&a\end{vmatrix}\text{=}a^3-3a+2=\left(a+2\right)\left(a-1\right)^2=0⇒ a=1\text{或}a=-2.
$$



$$
\mathrm{向量组}α=\left(1,2,-1,1\right)^T,β=\left(2,0,t,0\right)^T,γ=\left(0,-4,5,-2\right)^T\mathrm{的秩为}2,\text{则}t=().
$$
$$
A.
1 \quad B.2 \quad C.3 \quad D.0 \quad E. \quad F. \quad G. \quad H.
$$
$$
\begin{pmatrix}1&2&0\\2&0&-4\\-1&t&5\\1&0&-2\end{pmatrix}\rightarrow\begin{pmatrix}1&2&0\\0&-4&-4\\0&t+2&5\\0&-2&-2\end{pmatrix}\rightarrow\begin{pmatrix}1&2&0\\0&1&1\\0&t+2&5\\0&0&0\end{pmatrix}\rightarrow\begin{pmatrix}1&2&0\\0&1&1\\0&0&3-t\\0&0&0\end{pmatrix},\mathrm{由于秩为}2,\mathrm{所以}3-t=0.即t=3
$$



$$
\mathrm{向量组}α_1=\left(1,1,1,0\right)^T,α_3=\left(1,1,2,1\right)^T,α_3=\left(1,2,3,4\right)^T,α_4=\left(2,3,4,5\right)^T\mathrm{的秩为}().
$$
$$
A.
4 \quad B.1 \quad C.2 \quad D.3 \quad E. \quad F. \quad G. \quad H.
$$
$$
\left(α_1,α_2,\alpha_3,α_4\right)=\begin{pmatrix}1&1&1&2\\1&1&2&3\\1&2&3&4\\0&1&4&5\end{pmatrix}\rightarrow\begin{pmatrix}1&1&1&2\\0&1&2&2\\0&0&1&1\\0&0&0&1\end{pmatrix},\mathrm{故秩等于}4
$$



$$
\mathrm{向量组}α=\left(1,1,2\right)^T,β=\left(1,2,3\right)^T,γ=\left(1,1,λ+1\right)^T\mathrm{的秩为}2,\text{则}λ=().
$$
$$
A.
2 \quad B.1 \quad C.0 \quad D.-1 \quad E. \quad F. \quad G. \quad H.
$$
$$
\begin{vmatrix}1&1&1\\1&2&1\\2&3&λ+1\end{vmatrix}=λ-1=0⇒λ=1.
$$



$$
\mathrm{向量组}α=\left(1,1,0\right)^T,β=\left(1,3,-1\right)^T,γ=\left(5,3,t\right)^T\mathrm{的秩为}3,\text{则t}=().
$$
$$
A.
t=1 \quad B.t\neq1 \quad C.t=-1 \quad D.t\neq-1 \quad E. \quad F. \quad G. \quad H.
$$
$$
\begin{vmatrix}1&1&5\\1&3&3\\0&-1&t\end{vmatrix}=2\left(t-1\right)\neq0⇒ t\neq1
$$



$$
\mathrm{向量组}α_1=\left(1,2,1\right)^T,α_2=\left(0,2,0\right)^T,α_3=\left(1,1,1\right)^T,α_3=\left(2,1,3\right)^T\;\;\mathrm{的秩及最大线性无关组是}().
$$
$$
A.
3;α_1,α_2,α_4 \quad B.3;α_1,α_2,α_3 \quad C.2;α_1,α_2 \quad D.2;α_2,α_4 \quad E. \quad F. \quad G. \quad H.
$$
$$
\begin{pmatrix}1&0&1&2\\2&2&1&1\\1&0&1&3\end{pmatrix}\rightarrow\begin{pmatrix}1&0&1&0\\0&2&-1&-3\\0&0&0&1\end{pmatrix},\mathrm{故秩为}3,\mathrm{最大无关组为}α_1,α_2,α_4\text{或}α_1,α_3,α_4
$$



$$
\mathrm{已知不含零向量的向量组}α_1,α_2...,α_m(m>1)\mathrm{的最大线性无关组是唯一的},\mathrm{则该向量组的秩为}(\;).\;
$$
$$
A.
1 \quad B.m \quad C.m-1 \quad D.\mathrm{无法确定} \quad E. \quad F. \quad G. \quad H.
$$
$$
\begin{array}{l}\mathrm{根据不含零向量的向量组}α_1,α_2...,\alpha_m\mathrm{的最大线性无关组是唯一的可知},α_1,α_2...,α_m\mathrm{线性无关}(\mathrm{否则最大线性无关组不唯一})\\,\mathrm{故该向量组的秩为}m\end{array}
$$



$$
\begin{array}{l}\mathrm{如果向量组}\left(Ⅰ\right):α_{1,}α_{2,}...,α_{r,}\mathrm{与向量组}\left(Ⅱ\right):β_1,β_2,...,β_s\mathrm{等价},\text{且}\left(Ⅰ\right)\mathrm{线性无关},\\\text{则}r\text{与}s\mathrm{的大小关系是}().\;\end{array}
$$
$$
A.
r\leq s \quad B.r=s \quad C.r\geq s \quad D.r\;<\;s \quad E. \quad F. \quad G. \quad H.
$$
$$
\mathrm{由于等价向量组的秩相等},\mathrm{又组}\left(Ⅰ\right)\mathrm{线性无关},\text{故}R\left(Ⅱ\right)=R\left(Ⅰ\right)=r,\text{又}R\left(Ⅱ\right)\leq s,\text{故}r\leq s.
$$



$$
\text{已知向量组}α_1=\left(1,-a,1,1\right)^T,α_2=\left(1,1,-a,1\right)^T,α_3=\left(1,1,1,-a\right)^T\mathrm{的秩为}3,\text{则}a\mathrm{的取值范围是}(\;).\;
$$
$$
A.
a\neq-1 \quad B.a=-1 \quad C.a=1 \quad D.a\neq1 \quad E. \quad F. \quad G. \quad H.
$$
$$
\begin{array}{l}\left(α_{1,}α_{2,}α_3\right)=\begin{pmatrix}1&1&1\\-a&1&1\\1&-a&1\\1&1&-a\end{pmatrix}\rightarrow\begin{pmatrix}1&1&1\\0&1+a&1+a\\0&-a-1&0\\0&0&-a-1\end{pmatrix}\rightarrow\begin{pmatrix}1&1&1\\0&a+1&0\\0&0&a+1\\0&0&0\end{pmatrix}\\\mathrm{由于向量组的秩为}3,\text{则}a+1\neq0⇒ a\neq-1\end{array}
$$



$$
\mathrm{已知向量组}α_1=\left(1,1,2,-2\right)^T,α_2=\left(1,3,-x,x\right)^T,\alpha_3=\left(1,-1,6,-6\right)^T\mathrm{秩为}2,\text{则}\textit{x}\text{=().}
$$
$$
A.
0 \quad B.1 \quad C.2 \quad D.3 \quad E. \quad F. \quad G. \quad H.
$$
$$
\begin{array}{l}\mathrm{对下列矩阵进行初等行变换}:\\\left(α_{1,}α_{2,}α_3\right)=\begin{pmatrix}1&1&1\\1&3&-1\\2&-x&6\\-2&x&-6\end{pmatrix}\rightarrow\begin{pmatrix}1&1&1\\0&2&-2\\0&-x-2&4\\0&x+2&-4\end{pmatrix}\rightarrow\begin{pmatrix}1&1&1\\0&1&-1\\0&0&2-x\\0&0&x-2\end{pmatrix}\rightarrow\begin{pmatrix}1&1&1\\0&1&-1\\0&0&2-x\\0&0&0\end{pmatrix}\\\mathrm{由于向量组的秩为}2,\text{故}2-x=0⇒ x=2.\\\end{array}
$$



$$
\mathrm{已知向量}α=\left(1,0,-1,2\right)^T,β=(0,1,0,2)^T,A=αβ^T,\text{则 }\textit{R}\left(A\right)\textit{=}\text{()}
$$
$$
A.
2 \quad B.3 \quad C.1 \quad D.0 \quad E. \quad F. \quad G. \quad H.
$$
$$
A=αβ^T,\mathrm{由矩阵的秩的性质可知},R\left(A\right)\leq min\{R\left(α\right),R\left(β^T\right)\}=1;\text{又}A\mathrm{非零},\text{则}R\left(A\right)\geq1,\text{故}R\left(A\right)=1.
$$



$$
\mathrm{设向量组的秩为}r,则\;()
$$
$$
A.
\mathrm{该向量组所含向量的个数必大于}r \quad B.\mathrm{该向量组中任何}r\mathrm{个向量必线性无关},\mathrm{任何}r+1\mathrm{个向量必线性相关} \quad C.\mathrm{该向量组中有}r\mathrm{个向量线性无关},有r+1\mathrm{个向量线性相关} \quad D.\mathrm{该向量组中有}r\mathrm{个向量线性无关},\mathrm{任何}r+1\mathrm{个向量必线性相关} \quad E. \quad F. \quad G. \quad H.
$$
$$
\begin{array}{l}\mathrm{设该向量组构成的矩阵为}A,则r(A)=r,\mathrm{由矩阵的秩的定义可得}\\\mathrm{存在}r\mathrm{阶子式所在的}r\mathrm{个行}(列)\mathrm{向量线性无关},\mathrm{任何}r+1\mathrm{个行}(列)\mathrm{向量线性相关}.\end{array}
$$



$$
\begin{array}{l}\mathrm{已知向量组}Ⅰ:α_1=\left(1,1,0,3\right)^T,α_2=\left(2,2,2,7\right)^T,α_3=\left(-1,-1,0,-1\right)^T;\mathrm{又知向量组}Ⅱ\mathrm{是线性无关的且与}Ⅰ\mathrm{等价的}\\\mathrm{向量组},\mathrm{则组}Ⅱ\mathrm{中向量个数为}\;()\end{array}
$$
$$
A.
1 \quad B.2 \quad C.3 \quad D.\mathrm{无法确定} \quad E. \quad F. \quad G. \quad H.
$$
$$
\begin{array}{l}\mathrm{对下列矩阵施以初等行变换}:\\\left(α_1,α_2,α_3\right)=\begin{pmatrix}1&2&-1\\1&2&-1\\0&2&0\\3&7&-1\end{pmatrix}\rightarrow\begin{pmatrix}1&2&-1\\0&0&0\\0&2&0\\0&1&2\end{pmatrix}\rightarrow\begin{pmatrix}1&2&-1\\0&1&0\\0&0&2\\0&0&0\end{pmatrix},\\故r(Ⅰ)=3,\mathrm{又等价向量组的秩相等},故r(Ⅱ)=3,\mathrm{又向量组线性无关},\mathrm{因此向量组}Ⅱ\mathrm{中有}3\mathrm{个向量}.\end{array}
$$



$$
\text{设}A\text{为}4×5\mathrm{矩阵},\text{则()}.\;
$$
$$
A.
A\mathrm{的秩至少是}4 \quad B.A\mathrm{的列向量组线性相关} \quad C.A\mathrm{的列向量组线性无关} \quad D.A\mathrm{中存在}4\mathrm{阶非零子式} \quad E. \quad F. \quad G. \quad H.
$$
$$
\begin{array}{l}\mathrm{矩阵}A\mathrm{可写成列向量组的形式为}:A=(\alpha_1,α_2,α_3,α_4,α_5),\text{又}0\leq R\left(A\right)\leq min\{4,5\},\text{即}R(A)\leq4<\;5,\\故A\mathrm{的列向量线性相关}.\end{array}
$$



$$
设n\mathrm{维向量组}α_1,α_2,⋯,α_m\mathrm{的秩为}r,\text{则()}.
$$
$$
A.
\mathrm{该向量组中任意}r\mathrm{个向量组线性无关} \quad B.\mathrm{该向量组中任意}r+1\mathrm{个向量}(\mathrm{若有的话})\mathrm{都线性相关} \quad C.\mathrm{该向量组中存在惟一的最大线性无关组} \quad D.\mathrm{该向量组当}m>r\mathrm{时有若干个最大线性无关组} \quad E. \quad F. \quad G. \quad H.
$$
$$
\begin{array}{l}\mathrm{由向量组秩的定义和最大无关组的定义可知},\mathrm{该向量组中存在}r\mathrm{个线性无关的向量组},\mathrm{且任意}r+1\mathrm{个向量都线性相关},\\\mathrm{且向量组的最大无关组可能不止一个}\;,\mathrm{因此不一定唯一},\mathrm{也并非一定有若干个}.\end{array}
$$



$$
\mathrm{已知向量}α_1,α_2,⋯,α_m\mathrm{组线性相关},\text{则()}.\;
$$
$$
A.
\mathrm{该向量组的任何部分组必线性相关} \quad B.\mathrm{该向量组的任何部分组必线性无关} \quad C.\mathrm{该向量组的秩小于}m \quad D.\mathrm{该向量组的最大线性无关组是唯一的} \quad E. \quad F. \quad G. \quad H.
$$
$$
\begin{array}{l}\mathrm{若向量组的部分组线性相关},\mathrm{则整个向量组线性相关};\mathrm{而线性无关组的部分组也线性无关};\mathrm{向量的的极大无关组可能不止一个},\mathrm{因此不唯一}.\;\;\\\mathrm{向量组线性相关的充要条件是该向量组的秩小于向量的个数}.\end{array}
$$



$$
\mathrm{设矩阵}A_{m× n}\mathrm{的秩为}R\left(A\right)=m\;<\;n,I_m为m\mathrm{阶单位矩阵},\mathrm{下述结论中},\mathrm{正确的是}().\;
$$
$$
A.
A\mathrm{的任意}m\mathrm{个列向量必线性无关} \quad B.A\mathrm{的任一个}m\mathrm{阶子式不等于零} \quad C.A\mathrm{通过初等行变换必可化为}\left(I_m,0\right)\mathrm{的形式} \quad D.A\mathrm{通过初等变换必可化为}\left(I_m,0\right)\mathrm{的形式} \quad E. \quad F. \quad G. \quad H.
$$
$$
\begin{array}{l}\mathrm{根据矩阵秩的定义可知}A\mathrm{存在一个}m\mathrm{阶子式不为零},\text{即}A\mathrm{的列向量组中存在}m\mathrm{个向量线性无关},\\\mathrm{通过初等变换化为标准形}\left(I_m,0\right)\mathrm{的形式}.\end{array}
$$



$$
\text{设}n\mathrm{阶方阵}A\mathrm{的秩}r\;<\;n,\mathrm{则在}A的n\mathrm{个行向量中},().\;
$$
$$
A.
\mathrm{必有}r\mathrm{个行向量线性无关} \quad B.\mathrm{任意}r\mathrm{个行向量均可构成最大无关组} \quad C.\mathrm{任意}r\mathrm{个行向量均线性无关} \quad D.\mathrm{任一行向量均可由其它}r\mathrm{个行向量线性表示} \quad E. \quad F. \quad G. \quad H.
$$
$$
\mathrm{方阵}A\mathrm{的秩为}r,\mathrm{则必有}r\mathrm{个行向量线性无关}
$$



$$
\begin{array}{l}\mathrm{设向量组}Ⅰ:α_{1,}α_{2,}...,α_r,\mathrm{可由向量组}Ⅱ:β_1,β_2,...,β_s\mathrm{线性表示},\mathrm{下列命题正确的是}().\\\end{array}
$$
$$
A.
\mathrm{若向量组}Ⅰ\mathrm{线性无关},\text{则}r\leq s \quad B.\mathrm{若向量组}Ⅰ\mathrm{线性相关},\text{则}r>s \quad C.\mathrm{若向量组}Ⅱ\mathrm{线性无关},\text{则}r\leq s \quad D.\mathrm{若向量组}Ⅱ\mathrm{线性相关},\text{则}r\;<\;s \quad E. \quad F. \quad G. \quad H.
$$
$$
\begin{array}{l}\mathrm{向量组}I\mathrm{可由向量组}II\mathrm{线性表示},\text{则}R\left(I\right)\leq R\left(II\right),\mathrm{若向量组}I\mathrm{线性无关},则r=R\left(I\right)\leq R\left(II\right)\leq s⇒ r\leq s;\\\mathrm{若向量组}I\mathrm{线相关},\text{则}R\left(I\right)\leq R\left(II\right)\leq s,\text{且}R\left(I\right)\;<\;r,\mathrm{不能得到}r,s\mathrm{的关系},\mathrm{向量组}II\mathrm{线性相}(\text{无})\mathrm{关的情况也类似}.\end{array}
$$



$$
\mathrm{矩阵}A\mathrm{适合条件}()\text{时},\mathrm{它的秩为}r.\;
$$
$$
A.
A\mathrm{中任何}r+1\mathrm{列线性相关} \quad B.A\mathrm{中任何}r\mathrm{列线性相关} \quad C.A\mathrm{中有}r\mathrm{列线性无关} \quad D.A\mathrm{中有}r\mathrm{列线性无关},\mathrm{任意}r+1\mathrm{列都线性相关} \quad E. \quad F. \quad G. \quad H.
$$
$$
\begin{array}{l}\mathrm{根据矩阵秩的定义可知},\mathrm{矩阵}A\mathrm{的秩为}r,\text{则}A\mathrm{中有}r\mathrm{列线性无关},\mathrm{且任何}r+1\mathrm{列线性相关};\\\end{array}
$$



$$
\begin{array}{l}\mathrm{设向量组}Ⅰ:α_{1,}α_{2,}...,α_{r,}\text{和}Ⅱ:β_1,β_2,...,β_s.\mathrm{若向量组}Ⅰ\mathrm{可由}Ⅱ\mathrm{表示},\text{且}r>s,\mathrm 则().\;\\\end{array}
$$
$$
A.
Ⅱ\mathrm{线性相关} \quad B.Ⅱ\mathrm{线性无关} \quad C.Ⅰ\mathrm{线性相关} \quad D.Ⅰ\mathrm{线性无关} \quad E. \quad F. \quad G. \quad H.
$$
$$
\begin{array}{l}\mathrm{向量组}I\mathrm{可由}II\mathrm{表示},\text{则}R\left(I\right)\leq R\left(II\right)\leq s\;<\;r,即R\left(I\right)\;<\;r\;,\mathrm{故向量组}I\mathrm{线性相关};\mathrm{由于}R\left(II\right)\leq s,\\\mathrm{所以不能确定向量组}II\mathrm{是否线性相关或线性无关}.\end{array}
$$



$$
\mathrm{设向量组}α_1,α_2,α_3,α_4\mathrm{线性无关},\text{又}β_1=α_1+α_2,β_2=α_2-α_3,β_3=α_3+α_4,β_4=α_4+α_1,\;\;\mathrm{则向量组}β_1,β_2,β_3,β_4().
$$
$$
A.
\mathrm{秩为}1 \quad B.\mathrm{秩为}4 \quad C.\mathrm{秩为}2 \quad D.\mathrm{秩为}3 \quad E. \quad F. \quad G. \quad H.
$$
$$
\begin{array}{l}\mathrm{由题设条件可知}\left(β_1,β_2,β_3,β_4\right)=\left(α_1,α_2,α_3,\alpha_4\right)\begin{pmatrix}1&0&0&1\\1&1&0&0\\0&-1&1&0\\0&0&1&1\end{pmatrix},\mathrm{记作}B=AK.\\因\left|K\right|=2\neq0,知K\mathrm{可逆},\mathrm{由矩阵秩的性质知}:R\left(B\right)=R\left(A\right)⇒ R\left(B\right)=4.\end{array}
$$



$$
\mathrm{若向量组}α_1=\left(1,2,-1,1\right)^T,α_2=\left(2,0,t,0\right)^T\;,α_3=\left(0,-4,5,-2\right)^T\;,α_4\;=\left(3,-2,t+4,-1\right)^T\;\mathrm{的秩为}3,\text{则}()
$$
$$
A.
t\neq3 \quad B.t=3 \quad C.t\neq1 \quad D.t=1 \quad E. \quad F. \quad G. \quad H.
$$
$$
\begin{array}{l}\left(α_1,α_2,α_3,α_4\right)=\begin{pmatrix}1&2&0&3\\2&0&-4&-2\\-1&t&5&t+4\\1&0&-2&-1\end{pmatrix}\rightarrow\begin{pmatrix}1&2&0&3\\0&-4&-4&-8\\0&t+2&5&t+7\\0&-2&-2&-4\end{pmatrix}\rightarrow\begin{pmatrix}1&2&0&3\\0&1&1&2\\0&0&3-t&3-t\\0&0&0&0\end{pmatrix}\\\mathrm{显然},α_1,α_2\mathrm{线性无关},\text{且}t\neq3时,\text{则}R\left(α_1,α_2,α_3,α_4\right)=3,α_1,α_2,α_3\mathrm{是最大无关组}.\end{array}
$$



$$
当λ=(),时,\mathrm{向量组}α_1=\left(λ,1,1\right)^T,α_2=\left(1,λ,1\right)^T,α_3=\left(1,1,λ\right)^T\mathrm{线性相关},且R(α_1,α_2,α_3)=2.
$$
$$
A.
-2\text{或}1 \quad B.-2 \quad C.1 \quad D.0 \quad E. \quad F. \quad G. \quad H.
$$
$$
\begin{array}{l}\mathrm{由向量组线性相关},\text{则}\begin{vmatrix}λ&1&1\\1&λ&1\\1&1&λ\end{vmatrix}=0⇒(λ+2)(λ-1)^2=0,\text{则}λ=-2\text{或}1.\;\\\text{当}λ=1\text{时},\mathrm{向量组线性相关},R(α_1,α_2,α_3)=1;\\\text{当}\textit{λ}=-2\text{时},\mathrm{向量组线性相关},R(α_1,α_2,α_3)=2.\end{array}
$$



$$
\mathrm{已知}α_1=\left(2,3,4,5\right)^T,α_2=\left(3,4,\;5,6\right)^T,\alpha_3=\left(4,5,6,7\right)^T,α_4=\left(5,6,7,8\right)^T,则R(α_1,α_2,α_3,α_4)=().
$$
$$
A.
1 \quad B.2 \quad C.3 \quad D.4 \quad E. \quad F. \quad G. \quad H.
$$
$$
\begin{array}{l}\begin{pmatrix}2&3&4&5\\3&4&5&6\\4&5&6&7\\5&6&7&8\end{pmatrix}\rightarrow\begin{pmatrix}2&3&4&5\\1&1&1&1\\1&1&1&1\\1&1&1&1\end{pmatrix}\rightarrow\begin{pmatrix}1&2&3&4\\0&1&1&1\\0&0&0&0\\0&0&0&0\end{pmatrix},\mathrm{可见}R\left(α_1,α_2,α_3,α_4\right)=2\\\end{array}
$$



$$
\begin{array}{l}\mathrm{已知向量组}Ⅰ:α_1=\left(1,1,0,3\right)^T,α_2=\left(2,2,2,7\right)^T,α_3=\left(-1,-1,0,-1\right)^T;\mathrm{又知向量组}Ⅱ\mathrm{是线性无关的且与}Ⅰ\mathrm{等价的}\\\mathrm{向量组},\mathrm{则组}Ⅱ\mathrm{中向量个数为}\;()\end{array}
$$
$$
A.
1 \quad B.2 \quad C.3 \quad D.4 \quad E. \quad F. \quad G. \quad H.
$$
$$
\begin{array}{l}\mathrm{对下列矩阵施以初等行变换}:\\\left(α_1,α_2,α_3\right)=\begin{pmatrix}1&2&-1\\1&2&-1\\0&2&0\\3&7&-1\end{pmatrix}\rightarrow\begin{pmatrix}1&2&-1\\0&0&0\\0&2&0\\0&1&2\end{pmatrix}\rightarrow\begin{pmatrix}1&2&-1\\0&1&0\\0&0&2\\0&0&0\end{pmatrix},\\故r(Ⅰ)=3,\mathrm{又等价向量组的秩相等},故r(Ⅱ)=3,\mathrm{又向量组线性无关},\mathrm{因此向量组}Ⅱ\mathrm{中有}3\mathrm{个向量}.\end{array}
$$



$$
\mathrm{矩阵}A\mathrm{适合条件}()\text{时},\mathrm{它的秩为}r.\;
$$
$$
A.
A\mathrm{中任何}r+1\mathrm{列线性相关} \quad B.A\mathrm{中任何}r\mathrm{列线性相关} \quad C.A\mathrm{中有}r\mathrm{列线性无关} \quad D.A\mathrm{中线性无关的列向量最多有}r个 \quad E. \quad F. \quad G. \quad H.
$$
$$
\begin{array}{l}\mathrm{根据矩阵秩的定义可知},\mathrm{矩阵}A\mathrm{的秩为}r,\text{则}A\mathrm{中有}r\mathrm{列线性无关},\mathrm{且任何}r+1\mathrm{列线性相关};\\\mathrm{或线性无关的列向量最多有}r个.\end{array}
$$



$$
\mathrm{设向量组}α_1,α_2,α_3,α_4\mathrm{线性无关},\text{又}β_1=α_1+α_2,β_2=α_2-α_3,β_3=\alpha_3+α_4,β_4=α_4+α_1,\;\;\mathrm{则向量组}β_1,β_2,β_3,β_4().
$$
$$
A.
\mathrm{秩为}1 \quad B.\mathrm{秩为}2 \quad C.\mathrm{秩为}3 \quad D.\mathrm{秩为}4 \quad E. \quad F. \quad G. \quad H.
$$
$$
\begin{array}{l}\mathrm{由题设条件可知}\left(β_1,β_2,β_3,β_4\right)=\left(α_1,α_2,\alpha_3α_4\right)\begin{pmatrix}1&0&0&1\\1&1&0&0\\0&-1&1&0\\0&0&1&1\end{pmatrix},\mathrm{记作}B=AK.\\因\left|K\right|=2\neq0,知K\mathrm{可逆},\mathrm{由矩阵秩的性质知}:R\left(B\right)=R\left(A\right)⇒ R\left(B\right)=4.\end{array}
$$



$$
\begin{array}{l}\mathrm{已知向量组}(1):α_1,α_2,\alpha_3;\mathrm{向量组}(2):α_1,α_2,α_3,α_4;\mathrm{向量组}(3):α_1,α_2,α_3,α_5.\;\mathrm{如果各向量组的秩都为}3,\mathrm{则向量组}\\α_1,α_2,α_3,α_4,\alpha_5\mathrm{的秩为}(\;).\end{array}
$$
$$
A.
2 \quad B.3 \quad C.4 \quad D.5 \quad E. \quad F. \quad G. \quad H.
$$
$$
\begin{array}{l}\mathrm{因为}R\lbrackα_1,α_2,α_3\rbrack=3,\mathrm{所以}\alpha_1,α_2,α_3\mathrm{线性无关};\\\text{又}R\lbrackα_1,α_2,α_3,α_4\rbrack=3,\mathrm{因此}α_1,α_2,α_3,\alpha_4\mathrm{线性相关}.\;\text{则}α_4\mathrm{可用}α_1,α_2,α_3\mathrm{线性表示}.\;\\又α_1,α_2,α_3,α_5\mathrm{的秩为}3,\mathrm{从而}α_5\mathrm{也可用}α_1,α_2,α_3\mathrm{线性表示},\\故α_1,α_2,α_3\mathrm{为向量组}α_1,α_2,α_3,α_4,\alpha_5\mathrm{的最大无关组},即α_1,α_2,α_3,α_4,α_5\mathrm{的秩为}3\end{array}
$$



$$
\mathrm{向量组}α_1=(0,0,1,2,-1)^T,α_2=(1,3,-2,2,-1)^T,α_3=(2,6,-4,5,7)^T,α_4=(-1,-3,4,0,-19)^T\mathrm{的秩为}(\;).
$$
$$
A.
1 \quad B.2 \quad C.3 \quad D.4 \quad E. \quad F. \quad G. \quad H.
$$
$$
\begin{array}{l}A=\left(α_1,α_2,α_3,α_4\right)=\begin{pmatrix}0&1&2&-1\\0&3&6&-3\\1&-2&-4&4\\2&2&5&0\\-1&-1&7&-19\end{pmatrix}\rightarrow\begin{pmatrix}1&-2&-4&4\\0&1&2&-1\\0&0&1&-2\\0&0&0&0\\0&0&0&0\end{pmatrix},\\R(A)=3,\mathrm{所以向量组的秩为}3.\end{array}
$$



$$
\mathrm{向量组}α_1=(1,2,3,-1)^T,α_2=(3,2,1,-1)^T,α_3=(2,3,1,1)^T,α_4=(2,2,2,-1)^T,α_5=(5,5,2,0)^T\mathrm{的秩为}(\;).
$$
$$
A.
3 \quad B.4 \quad C.5 \quad D.2 \quad E. \quad F. \quad G. \quad H.
$$
$$
\begin{array}{l}(α_1,α_2,α_3,α_4,α_5)=\begin{pmatrix}1&3&2&2&5\\2&2&3&2&5\\3&1&1&2&2\\-1&-1&1&-1&0\end{pmatrix}\rightarrow\begin{pmatrix}0&2&3&1&5\\0&0&5&0&5\\0&-2&4&-1&2\\-1&-1&1&-1&0\end{pmatrix}\rightarrow\begin{pmatrix}0&2&3&1&5\\0&0&1&0&1\\0&0&7&0&7\\-1&-1&1&-1&0\end{pmatrix}\\\rightarrow\begin{pmatrix}-1&-1&1&-1&0\\0&2&3&1&5\\0&0&1&0&1\\0&0&0&0&0\end{pmatrix}\\\mathrm{所以向量组的秩}3,α_1,α_2,α_3\mathrm{是一个最大线性无关组}.\end{array}
$$



$$
\mathrm{向量组}α=\left(1,2,1,0\right)^T,β=\left(2,1,3,-5\right)^T,γ=\left(1,0,1,2\right)^T,\mathrm{则向量组}α,β,γ\mathrm{一定}().\;
$$
$$
A.
\mathrm{线性相关} \quad B.\mathrm{线性无关} \quad C.R\left(α,β,γ\right)=2 \quad D.\mathrm{线性相关性无法确定} \quad E. \quad F. \quad G. \quad H.
$$
$$
因A=\left(α,β,γ\right)\mathrm{有三阶非零子式}.\begin{vmatrix}1&2&1\\2&1&0\\1&3&1\end{vmatrix}=2\neq0,\mathrm{所以它们线性无关}
$$



$$
\begin{array}{l}\mathrm{设向量组}α_1,α_2,α_3,α_4\mathrm{线性无关},\text{又}β_1=α_1+α_2,β_2=α_2+α_3,β_3=α_3+α_4,β_4=α_4+α_1,\;\\\;\mathrm{则向量组}β_1,β_2,β_3,β_4\mathrm{的秩为}().\end{array}
$$
$$
A.
1 \quad B.2 \quad C.3 \quad D.4 \quad E. \quad F. \quad G. \quad H.
$$
$$
\begin{array}{l}\mathrm{由题设条件可知}\left(β_1,β_2,β_3,β_4\right)=\left(α_1,α_2,\alpha_3,α_4\right)\begin{pmatrix}1&0&0&1\\1&1&0&0\\0&1&1&0\\0&0&1&1\end{pmatrix},\mathrm{记作}B=AK.\\\text{因}K\rightarrow\begin{pmatrix}1&0&0&1\\0&1&0&-1\\0&0&1&1\\0&0&0&0\end{pmatrix},\;\;K\mathrm{有三阶非零子式},\mathrm{由矩阵秩的定义知}:R\left(B\right)=R\left(A\right)⇒ R\left(B\right)=3.\end{array}
$$



$$
当k=()\text{时},\mathrm{向量组}α_1=\left(k,1,1\right)^T,α_2=\left(1,k,1\right)^T,α_3=\left(1,1,k\right)^T\mathrm{线性相关},\text{且}R(α_1,α_2,α_3)=1.
$$
$$
A.
-2\text{或}1 \quad B.1 \quad C.-2 \quad D.-1\text{或}2 \quad E. \quad F. \quad G. \quad H.
$$
$$
\begin{array}{l}\mathrm{由向量组线性相关},\text{则}\begin{vmatrix}k&1&1\\1&k&1\\1&1&k\end{vmatrix}=(k-1)^2(k+2)=0⇒ k=1\text{或}k=-2.\\\text{当k}=1\text{时},\mathrm{向量组线性相关},R(α_1,α_2,α_3)=1;\\\end{array}
$$



$$
\mathrm{已知向量}α_1=(1,2,3),α_2=(4,5,6),α_3=(7,8,9),\mathrm{则向量组}α_1,α_1+α_2,α_1+α_2+α_3\mathrm{的秩为}(\;).
$$
$$
A.
1 \quad B.2 \quad C.3 \quad D.\mathrm{无法确定} \quad E. \quad F. \quad G. \quad H.
$$
$$
\begin{array}{l}令A=\begin{pmatrix}α_1\\α_2\\α_3\end{pmatrix}=\begin{pmatrix}1&2&3\\4&5&6\\7&8&9\end{pmatrix},\mathrm{则由}A\xrightarrow{\text{行}}\begin{pmatrix}1&2&3\\0&-3&-6\\0&0&0\end{pmatrix}\mathrm{可得秩}(A)=2,\\\text{故}α_1,α_2,α_3\mathrm{的秩为}2.\\因α_1,α_1+α_2,α_1+α_2+α_3\mathrm{可由}α_1,α_2,α_3\mathrm{线性表示},\\\mathrm{所以}α_1,α_1+α_2,α_1+α_2+α_3\mathrm{的秩小于等于}α_1,α_2,α_3\mathrm{的秩},又α_1,α_2\mathrm{线性无关},\\\text{故}α_1,α_1+α_2\mathrm{也线性无关}.\;\\\mathrm{于是}α_1,α_1+α_2,α_1+α_2+α_3\mathrm{的秩为}2,\text{且}α_1,α_1+α_2\mathrm{就是它的一个最大无关组}.\\\end{array}
$$



$$
\mathrm{设有向量组}α_1=(1,-1,2,4)^T,α_2=(0,3,1,2)^T,α_3=(3,0,7,14)^T,α_4=(1,-2,2,0)^T\;\;,α_3=(2,1,5,10)^T,\mathrm{则该向量组的秩为}()
$$
$$
A.
1 \quad B.2 \quad C.3 \quad D.4 \quad E. \quad F. \quad G. \quad H.
$$
$$
\begin{array}{l}\mathrm{此题可以采用排除法},\text{因}-3α_1-α_2+α_3=0,-2α_1-α_2+α_5=0,\;\;\\\mathrm{由于最大线性无关组中的向量是线性无关的},\mathrm{因此只有线性无关组}α_1,α_2,α_4\mathrm{是最大无关组},\mathrm{秩为}3.\;\;\\\mathrm{或对}A=(α_1,α_2,α_3,α_4,α_5)\mathrm{施行初等行变换化为行阶梯形矩阵}:\\A\rightarrow\begin{pmatrix}1&1&-2&1&4\\0&3&3&-1&3\\0&1&1&0&1\\0&2&2&-4&2\end{pmatrix}\rightarrow\begin{pmatrix}1&0&-1&0&4\\0&1&1&0&1\\0&0&0&1&0\\0&0&0&0&0\end{pmatrix},\\\mathrm{因此只有线性无关组}α_1,α_2,α_4\mathrm{是其最大无关组},\mathrm{秩为}3.\end{array}
$$



$$
\mathrm{已知两个向量组}α_1=\begin{pmatrix}1\\2\\3\end{pmatrix},α_2=\begin{pmatrix}1\\0\\1\end{pmatrix},β_1=\begin{pmatrix}-1\\2\\t\end{pmatrix},β_2=\begin{pmatrix}4\\1\\5\end{pmatrix},\mathrm{若两向量组等价},\text{则}t\mathrm{的值为}(\;).\;
$$
$$
A.
t=1 \quad B.t=-1 \quad C.t\neq1 \quad D.t\neq-1 \quad E. \quad F. \quad G. \quad H.
$$
$$
\begin{array}{l}(α_1,α_2,β_1,β_2)=\begin{pmatrix}1&1&-1&4\\2&0&2&1\\3&1&t&5\end{pmatrix}\rightarrow\begin{pmatrix}1&1&-1&4\\0&1&-2&\frac72\\0&0&t-1&0\end{pmatrix}\rightarrow\begin{pmatrix}1&0&1&\frac12\\0&1&-2&\frac72\\0&0&t-1&0\end{pmatrix}\\\text{故}t=1\text{时},R(α_1,α_2,β_1,β_2)=R(α_1,α_2)=R(β_1,β_2)=2,\mathrm{两向量组可相互表出},\mathrm{即两向量组等价}.\end{array}
$$



$$
\begin{array}{l}\mathrm{设向量组}A:α_1,α_2\;,α_3:\mathrm{向量组}B:α_1,α_2\;,\alpha_3\;,α_4;\;\mathrm{向量组}\;\;C:α_1,α_2\;,α_3\;,α_5;\;\\\;若R(α_1,α_2\;,α_3)=R(α_1,\alpha_2\;,α_3\;,α_4)\;=3,R(α_1,\alpha_2\;,α_3\;,α_5)=4,\;\\\mathrm{则向量组}α_1,α_2\;,α_3\;,α_5-α_4\mathrm{的秩为}()\end{array}
$$
$$
A.
1 \quad B.2 \quad C.3 \quad D.4 \quad E. \quad F. \quad G. \quad H.
$$
$$
\begin{array}{l}\mathrm{向量组}A\mathrm{线性无关};\mathrm{向量组}B\mathrm{线性相关},且α_1,α_2,α_3\mathrm{是一个最大无关组},\\即,α_4\mathrm{可由}α_1,α_2,α_3\mathrm{线性表示},\mathrm{向量组}C\mathrm{线性无关},且α_5\mathrm{不能由}α_1,α_2,α_3\mathrm{线性表示},\\则α_1,α_2,α_3,α_5-\alpha_4\mathrm{中任一向量不能由其余向量线性表示},\mathrm{即线性无关},\mathrm{所以秩为}4.\end{array}
$$



$$
设n\mathrm{维向量组}M:α_1,α_2,...,α_s,与N:β_1,β_2,...β_t,\mathrm{的秩都是}r,则\;(\;).
$$
$$
A.
\mathrm{向量组}M与N\mathrm{等价} \quad B.R(α_1,...,α_s,β_1,...β_t)=2r \quad C.若s=t=r,则M与N\mathrm{等价} \quad D.若M\mathrm{可由}N\mathrm{线性表出},则M与N\mathrm{等价} \quad E. \quad F. \quad G. \quad H.
$$
$$
\begin{array}{l}\mathrm{两个向量组等价指的是它们能相互线性表示},\mathrm{等秩的向量组不一定能互相表示},\;\\\;\mathrm{例如}:α_1=(1,0)^T,α_2=(2,0)^T与;\;β_1=(0,1)^T,β_2=(0,2)^T\;\;;\\\;\;\;\;\;\;\;\;\;\;\;\;\;\;\;\;\;\;\;\;\;\;\;\;\;\;\;r\leq R(α_1,α_2,...,α_s,N:β_1,β_2,...β_t)\leq2r\;\;,\;\;\\当M与N\mathrm{等价时},\mathrm{达到左边的等号},当M与N\mathrm{中没有一个向量能由另一向量组线性表示时达到右边的等号};\;\\\;若s=t=r\mathrm{只能说明}M与N\mathrm{向量组都线性无关};\;\;\\M与N\mathrm{有相同的秩},\mathrm{又若}M\mathrm{可由}N\mathrm{线性表出},\mathrm{则它们的最大无关组可相互线性表出},\mathrm{又向量组可由其}\;\;\\\mathrm{最大无关组线性表出},\mathrm{由等价的传递性可知}M与N\mathrm{可相互线性表出},\mathrm{即等价}.\end{array}
$$



$$
\mathrm{向量组}α_1,α_2,⋯,α_n\mathrm{的秩不为零的充分必要条件是}\;(\;).
$$
$$
A.
α_1,\alpha_2,⋯,α_n\mathrm{中至少有一个非零向量} \quad B.α_1,α_2,⋯,α_n\mathrm{全是非零向量} \quad C.\alpha_1,α_2,⋯,α_n\mathrm{中有一个线性相关组} \quad D.α_1,α_2,⋯,α_n\mathrm{线性无关} \quad E. \quad F. \quad G. \quad H.
$$
$$
\begin{array}{l}α_1,α_2,⋯,α_n\mathrm{全是非零向量可推出其秩不为零},\mathrm{但反之不然},\mathrm{因此只是充分条件};\\α_1,α_2,⋯,α_n\mathrm{若全是零向量},\mathrm{满足有一个线性相关组},\mathrm{但向量组的秩也为零};\\α_1,α_2,⋯,α_n\mathrm{线性无关可推出向量组的秩非零},\mathrm{但反之不然}.\\\mathrm{因此},\mathrm{此题只有选项}α_1,α_2,⋯,α_n\mathrm{中至少有一个非零向量是}α_1,α_2,⋯,α_n\mathrm{的秩非零的充要条件}.\end{array}
$$



$$
\begin{array}{l}\mathrm{下列命题中正确的有}(\;).\\(1)设A为n\mathrm{阶矩阵},R(A)=r<\;n,\;\mathrm{则矩阵}A\mathrm{的任意}r\mathrm{个列向量线性无关}.\\(2)\mathrm{设向量组}α_1,α_2,⋯α_s\mathrm{线性无关},\mathrm{且可由向量组}β_1,β_2,⋯β_t\mathrm{线性表示},\mathrm{则必有}s<\;t.\\(3)设A为m× n\mathrm{阶矩阵},\mathrm{如果矩阵}A的n\mathrm{个列向量线性无关},\mathrm{那么}R(A)=n.\\(4)\mathrm{如果向量组}α_1,α_2,⋯α_s\mathrm{的秩为}s,\mathrm{则向量组}α_1,α_2,⋯α_s\mathrm{中任一部分组都线性无关}.\end{array}
$$
$$
A.
1个 \quad B.2个 \quad C.3个 \quad D.4个 \quad E. \quad F. \quad G. \quad H.
$$
$$
\begin{array}{l}(1)\mathrm{不正确}.\\\mathrm{例如}A=\begin{pmatrix}1&0&⋯&0\\0&0&⋯&0\\\vdots&\vdots&&\vdots\\0&0&⋯&0\end{pmatrix},R(A)=1,但A\mathrm{中后}n-1\mathrm{个向量},\mathrm{每一个均线性相关},\mathrm{则结论不成立}.\\(2)\mathrm{不正确}.\\\mathrm{还可能为}s=t,\;如α_1=\begin{pmatrix}1\\1\end{pmatrix},α_2=\begin{pmatrix}0\\1\end{pmatrix},\beta_1=\begin{pmatrix}1\\0\end{pmatrix},β_2=\begin{pmatrix}0\\2\end{pmatrix}.\\(3)\mathrm{正确}.\\\mathrm{矩阵}A\mathrm{的秩等于}A\mathrm{的行向量组的秩},\mathrm{也等于}A\mathrm{的列向量组的秩}.\\(4)\mathrm{正确}.\\\mathrm{因为如果一个向量组有线性相关的部分组},\mathrm{则这个向量组线性相关}.\end{array}
$$



$$
\begin{array}{l}\mathrm{下列向量组中},\mathrm{秩不等于}3\mathrm{的有}\;(\;).\\(1)α_1=(1,1,1)^T,α_2=(1,1,0)^T,α_3=(1,0,0)^T,α_4=(1,2,-3)^T;\\(2)α_1=(2,1,1,1)^T,α_2=(-1,1,7,10)^T,α_3=(3,1,-1,-2)^T,α_4=(8,5,9,11)^T;\\(3)α_1=(1,1,3,1),α_2=(-1,1,-1,3),α_3=(5,-2,8,-9),α_4=(-1,3,1,7);\end{array}
$$
$$
A.
0个 \quad B.1个 \quad C.2个 \quad D.3个 \quad E. \quad F. \quad G. \quad H.
$$
$$
\begin{array}{l}(1)A=(α_1,α_2,α_3,α_4)=\begin{pmatrix}1&1&1&1\\1&1&0&2\\1&0&0&-3\end{pmatrix}\xrightarrow[{r_2-r_3}]{r_1-r_2}\begin{pmatrix}0&0&1&-1\\0&1&0&5\\1&0&0&-3\end{pmatrix},\\\mathrm{向量组的秩为}3;\;\;\\\mathrm{同理依次可求出向量组}(2)(3)\mathrm{的秩都等于}2\\\end{array}
$$



$$
\begin{array}{l}\mathrm{设向量组}\\\;\;\;\;\;\;\;\;\;\;\;\;\;\;\;\;α_1=\begin{pmatrix}a\\3\\1\end{pmatrix},\;α_2=\begin{pmatrix}2\\b\\3\end{pmatrix},\;α_3=\begin{pmatrix}1\\2\\1\end{pmatrix},\;α_4=\begin{pmatrix}2\\3\\1\end{pmatrix}\\\mathrm{的秩为}2\;则a˴b\mathrm{的值分别为}(\;).\end{array}
$$
$$
A.
a=2,b=5 \quad B.a=2,b=4 \quad C.a=3,b=5 \quad D.a=3,b=4 \quad E. \quad F. \quad G. \quad H.
$$
$$
\begin{array}{l}(α_3,α_4,α_1,α_2)=\begin{pmatrix}1&2&a&2\\2&3&3&b\\1&1&1&3\end{pmatrix}\xrightarrow[{r_2-r_1}]{r_2-2r_1}\left(\begin{array}{cccc}1&2&a&2\\0&-1&3-2a&b\\0&-1&1-a&1\end{array}-4\right)\\\;\;\;\;\;\;\;\;\;\;\;\;\;\;\;\;\;\;\;\;\;\;\xrightarrow{r_3-r_2}\left(\begin{array}{cccc}1&2&a&2\\0&-1&3-2a&b\\0&0&a-2&5-b\end{array}-4\right),\\\mathrm{于是}R(\alpha_1,α_2,α_3,α_4)=2⇒ R(α_3,α_4,α_1,α_2)=2⇒ a=2\;且b=5.\\\end{array}
$$



$$
\mathrm{设矩阵}A_{m× n}\mathrm{的秩}R(A)=m<\;n,B为n\mathrm{阶方阵},则(\;).\;
$$
$$
A.
A_{m× n}\mathrm{的任意阶子式均不为零} \quad B.当R(B)=n\mathrm{时有}R(AB)=m \quad C.A_{m× n}\mathrm{的任意个}m\mathrm{列向量均线性无关} \quad D.\left|A^TA\right|\neq0 \quad E. \quad F. \quad G. \quad H.
$$
$$
\begin{array}{l}阵A_{m× n}\mathrm{的秩}R(A)=m<\;n,则A_{m× n}\mathrm{中存在}m\mathrm{阶子式不为零},\mathrm{且存在}m\mathrm{个列向量线性无关};\\\mathrm{由于}{A^T}_{n× m}A_{m× n}为n\mathrm{阶矩阵},且R(A^TA)\leq min\{R(A^T),R(A)\}=m<\;n,故\left|A^TA\right|=0;\\B为n\mathrm{阶方阵},\mathrm{当秩}R(B)=n时,B\mathrm{可表示为一系列初等矩阵的乘积},\mathrm{又初等变换不改变矩阵的秩},\;\;\\\mathrm{因此}R(AB)=m.\end{array}
$$



$$
\begin{array}{l}\mathrm{下列命题中的向量组}A,B,\mathrm{等价的有}(\;).\\(1)\mathrm{向量组}B\mathrm{能由向量组}A\mathrm{线性表示},\mathrm{且它们的秩相等};\\(2)A=(α_1,α_2)=\begin{pmatrix}2&3\\0&-2\\-1&1\\3&-1\end{pmatrix},B=(β_1,β_2)=\begin{pmatrix}-5&4\\6&-4\\-5&3\\9&-5\end{pmatrix};\\(3)A:α_1=\begin{pmatrix}0\\1\\1\end{pmatrix},α_2=\begin{pmatrix}1\\1\\0\end{pmatrix};B:β_1=\begin{pmatrix}-1\\0\\1\end{pmatrix},β_2=\begin{pmatrix}1\\2\\1\end{pmatrix},β_3=\begin{pmatrix}3\\2\\-1\end{pmatrix},\\\\\\\end{array}
$$
$$
A.
0个 \quad B.1个 \quad C.2个 \quad D.3个 \quad E. \quad F. \quad G. \quad H.
$$
$$
\begin{array}{l}(1)\mathrm{设向量组}A和B\mathrm{的秩都为}s.\;因B\mathrm{组能由}A\mathrm{组线性表示},故A\mathrm{组和}B\mathrm{组合并而成的向量组}(A,B)\mathrm{能由}A\mathrm{组线性表示}.\;而A\mathrm{组是}(A,B)\mathrm{组的}\\\mathrm{部分组},故A\mathrm{组总能由}(A,B)\mathrm{组线性表示}.\;\mathrm{所以}(A,B)\mathrm{组与}A\mathrm{组等价},\mathrm{因此}(A,B)\mathrm{组的秩也为}s.\;\\\;\;\;\;\;\;\;\mathrm{又因}B\mathrm{组的秩为}s,故B\mathrm{组的极大无关组}B_0含s\mathrm{个向量},\mathrm{因此}B_0\mathrm{组也是}(A,B)\mathrm{组的极大无关组},\mathrm{从而}(A,B)\mathrm{组与}B_0\mathrm{组等价}.\;\\由A\mathrm{组与}(A,B)\mathrm{组等价},(A,B)与B_0\mathrm{等价},\mathrm{推知}A\mathrm{组与}B\mathrm{组等价}.\\(2)\mathrm{可用初等行变换求解矩阵方程}\;\;(β_1,β_2)=(α_1,α_2)X,(α_1,α_2)=\;(β_1,β_2)Y,\mathrm{即两向量组可互相表出},\mathrm{即等价};\\(3)(B,A)=\begin{pmatrix}-1&1&3&0&1\\0&2&2&1&1\\1&1&-1&1&0\end{pmatrix}\rightarrow\begin{pmatrix}1&1&-1&1&0\\0&2&2&1&1\\0&0&0&0&0\end{pmatrix},\\\mathrm{即知}R(B)=R(B,A)=2,\;且\;R(A)\leq2,而α_1与α_2\mathrm{不成比例},故R(A)=2\;.\mathrm{因此},\mathrm{向量组}A与B\mathrm{等价}.\;\;\mathrm{因此}(1)(2)(3)\mathrm{中的向量都等价}.\\\end{array}
$$



$$
\mathrm{矩阵}A=\begin{pmatrix}2&-1&-1&1&2\\1&1&-2&1&4\\4&-6&2&-2&4\\3&6&-9&7&9\end{pmatrix},\mathrm{用列向量组可表示为}A=(α_1,α_2,α_3,α_4α_5),\mathrm{则关于列向量组的秩和最大无关组},\mathrm{下列说法正确的是}(\;).\;
$$
$$
A.
\mathrm{秩为}3,\mathrm{最大无关组为}α_1,α_2,α_4 \quad B.\mathrm{秩为}3,\mathrm{最大无关组为}α_1,α_2,α_3 \quad C.\mathrm{秩为}2,\mathrm{最大无关组为}α_1,α_2 \quad D.\mathrm{秩为}2,\mathrm{最大无关组为}α_3,α_4 \quad E. \quad F. \quad G. \quad H.
$$
$$
\begin{array}{l}对A\mathrm{施行初等行变换化为行阶梯形矩阵}:\\\;\;\;\;\;\;\;\;\;\;\;\;\;\;\;\;\;\;\;\;\;\;\;\;\;\;\;\;\;\;\;A\rightarrow\begin{pmatrix}1&1&-2&1&4\\0&1&-1&1&0\\0&0&0&1&-3\\0&0&0&0&0\end{pmatrix}\rightarrow\begin{pmatrix}1&0&-1&0&4\\0&1&-1&0&3\\0&0&0&1&-3\\0&0&0&0&0\end{pmatrix},\\知R(A)=3,\mathrm{故列向量组的最大无关组含}\;3\;\mathrm{个向量}.\;\mathrm{而三个非零行的非零首元在第}1,2,4\mathrm{三列},故α_1,α_2,α_4\mathrm{为列向量组的}\\\mathrm{一个最大无关组}.\end{array}
$$



$$
\mathrm{设向量组}α_1,α_2,...,α_s,\mathrm{的秩是}r,\;\mathrm{则在其中任意选取}m\mathrm{个向量所构成向量组的秩}\;().
$$
$$
A.
\geq r+m-s \quad B.\leq r+m-s \quad C.\geq s+m-r \quad D.\leq s-m-r \quad E. \quad F. \quad G. \quad H.
$$
$$
\begin{array}{l}\text{在}α_1,α_2,...,α_s\mathrm{中任取}m\mathrm{个向量},\mathrm{设其秩是}q,且\alpha_{i_1},α_{i_2},...,α_{i_q}\mathrm{是其最大线性无关组}.\text{由}\\于R(α_1,α_2,...,α_s)=r\;\;\mathrm{因此对}α_{i_1},α_{i_2},...,α_{i_q}\mathrm{可再扩充}r-q\mathrm{个向量使之成为}α_1,α_2,...,α_s\mathrm{的最大线性无关组},\\\mathrm{而这}r-q\mathrm{个向量只能取自这}m\mathrm{个向量之外},故\;r-q\leq s-m\;即q\geq r+m-s.\end{array}
$$



$$
\mathrm{设向量组}A:α_1,α_2,...,α_S\mathrm{的秩为}r_1\;\mathrm{向量组}B:β_1,β_1,...,β_t\mathrm{的秩}r_2,\;\mathrm{向量组}C:α_1,α_2,...,α_S,β_1,β_1,...,β_t\mathrm{的秩}r_3,\;\text{则 }()
$$
$$
A.
\text{max}\{r_1,r_2\}\leq r_3\leq r_1+r_2 \quad B.r_3\leq\text{max}\{r_1,r_2\}\leq r_1+r_2 \quad C.\text{max}\{r_1,r_2\}\leq r_1+r_2\leq r_3 \quad D.\mathrm{无法判断} \quad E. \quad F. \quad G. \quad H.
$$
$$
\begin{array}{l}\text{设}A,B,C\mathrm{的最大线性无关组分别为}\;A',B',C',\mathrm{含有的向量个数}(\text{秩})\mathrm{分别为}r_1,r_2,r_3,\;\text{则}A,B,C\mathrm{分别与}A',B',C'\mathrm{等价},\\\mathrm{易知}A,B\mathrm{均可由}C\mathrm{线性表示},\text{则}\\\;\;\;\;\;\;\;\;\;\;\;\;\;\;\;\;\;\;\;\;\;\;\;\;\;秩(C)\geq 秩(A)\;,秩(C)\geq 秩(B)\;,\\\text{即}\;\text{max}\{r_1,r_2\}\leq r_3\;\\\;\;\;\;\;\;设A'与B'\mathrm{中的向量共同构成向量组}D,\;则A,B\mathrm{均可由}D\mathrm{线性表示},即C\mathrm{可由}D\mathrm{线性表示},\mathrm{从而}C'\mathrm{可由}D\mathrm{线性表示},\\\mathrm{所以秩}(C')\leq\text{秩}(D),D为r_1+r_2\mathrm{阶矩阵},\mathrm{所以}\;\\\;\;\;\;\;\;\;\;\;\;\;\;\;\;\;\;\;\;\;\;\;\;\;\;\;\text{秩}(D)\leq r_1+r_2\;即r_3\leq r_1+r_2.\end{array}
$$



$$
\begin{array}{l}\mathrm{向量组}B:β_1,...,β_r\mathrm{能由向量组}A:α_1,...,α_s\mathrm{线性表示为}(β_1,...,β_r)\;=(α_1,...,α_s)K,\\\;\mathrm{其中}K\text{为}s× r\mathrm{矩阵},\text{且}A\mathrm{组线性无关}.\;\text{则}B\mathrm{组线性无关的充分必要条件是矩阵}K\mathrm{的秩}R(K)\;=()\end{array}
$$
$$
A.
r \quad B.s \quad C.r-s \quad D.s-r \quad E. \quad F. \quad G. \quad H.
$$
$$
\begin{array}{l}\text{若}B\mathrm{组线性无关},\text{令}\;B=(β_1,...,β_r),A=(\alpha_1,...,α_s),\\\mathrm{则有}B=AK,\;\mathrm{由定理知}\;R(B)=R(AK)\leq\text{min}\left\{R(\textit{A}\text{),}R(K)\right\}\text{≤}\textit{R}\text{(}\textit{K}\text{),}\\\text{由}B\text{组}:β_1,β_2,...,β_r\mathrm{线性无关知}\;R(B)=r\;\text{故}\;R(K)\;\geq r,\mathrm{又知}K为s× r\mathrm{阶矩阵},\text{则}\\\;\;\;\;\;\;\;\;\;\;\;\;\;\;\;\;\;\;\;\;\;\;\;\;\;\;\;\;\;\;\;\;\;R(K)\leq\text{min}\left\{\textit{r}\text{,}s\right\}\text{,}\\\mathrm{由于向量组}B\mathrm{能由向量组}A\mathrm{线性表示},\text{则}r\leq s.\;\;∴\text{min}\left\{r,s\right\}\text{=}\textit{r.}\\\text{综上所述知 }\textit{r}\text{≤}\;R(K)\leq r\text{即}R(K)=r\textit{ }\text{ 此即充分性.      }\\\text{必要条件  若}R(K)=r\text{令}{\textit{χ}}_\textit1{\textit{β}}_1\text{+}χ_2{\textit{β}}_2\text{+⋯+}χ_1{\textit{β}}_1\textit{=}\text{0,利用条件中的线性表示式代入矩阵}\textit{K}\text{,}\\\text{  通过}R(K)=r\text{和变形线性方程组解的判断可知 }{\textit{χ}}_\textit1{\textit{=χ}}_\mathit2{\textit{=...χ}}_\mathit r\text{所以}{\textit{β}}_1{\textit{,β}}_\mathit2{\textit{...β}}_\mathit r\text{线性无关.}\\\\\end{array}
$$



$$
\mathrm{向量组}α_1=\left(1,2,3,4\right)^T,α_2=\left(2,3,4,5\right)^T,α_3=\left(3,4,5,6\right)^T,α_4=\left(4,5,6,7\right)^T\;\;\mathrm{的秩及最大线性无关组是}().
$$
$$
A.
2;α_1,α_2 \quad B.3;α_1,α_2,α_3 \quad C.1;α_1 \quad D.4;α_1,α_2,α_3,α_4 \quad E. \quad F. \quad G. \quad H.
$$
$$
\begin{pmatrix}1&2&3&4\\2&3&4&5\\3&4&5&6\\4&5&6&7\end{pmatrix}\rightarrow\begin{pmatrix}1&0&1&2\\0&1&1&1\\0&0&0&0\\0&0&0&0\end{pmatrix},⇒ R\left(α_1,α_2,α_3,α_4\right)=2.\mathrm{由于任何两个向量线性无关},\mathrm{故都可以作为最大无关组}
$$



$$
\mathrm{向量组}α_1=\left(1,0,0\right)^T,α_2=\left(0,1,0\right)^T,α_3=\left(0,0,0\right)^T,α_4=\left(1,1,0\right)^T\mathrm{的最大无关向量组为}().
$$
$$
A.
α_{1\;\;},α_{2\;\;},α_3 \quad B.α_{1\;\;},α_{2\;\;},α_4 \quad C.α_{1\;\;},α_{2\;\;} \quad D.α_{3\;\;},α_4 \quad E. \quad F. \quad G. \quad H.
$$
$$
\mathrm{显然}α_4=α_1+α_2,\mathrm{故向量组的秩为}2,又α_1,α_2\mathrm{线性无关},\mathrm{故构成一组最大无关组};\;\;\mathrm{由于}α_3,α_4\mathrm{线性相关},\mathrm{所以不是最大无关组}.
$$



$$
\mathrm{向量组}α_1=\left(1,2,1,3\right)^T,α_2=\left(4,-1,-5,-6\right)^T,α_3=\left(1,-3,-4,-7\right)^T\mathrm{的最大无关组为}\;().
$$
$$
A.
α_1,α_2,α_3 \quad B.α_1,α_2 \quad C.α_1 \quad D.α_3 \quad E. \quad F. \quad G. \quad H.
$$
$$
\begin{array}{l}\\\mathrm{对向量组构成的矩阵进行初等行变换得}\\\;\;\;\;\;\;\;\;\;\;\;\;\;\;\;\;\;(α_1,α_2,α_3)=\begin{pmatrix}1&4&1\\2&-1&-3\\1&-5&-4\\3&-6&-7\end{pmatrix}\rightarrow\begin{pmatrix}1&4&1\\0&9&5\\0&0&0\\0&0&0\end{pmatrix}\\\mathrm{因此向量组的秩为}2,又α_1,α_2\mathrm{线性无关},\mathrm{因此构成向量组的最大无关组}.\\\end{array}
$$



$$
\mathrm{向量组}α_{i1},α_{i2},⋯,α_{it}\mathrm{和向量组}α_{j1},α_{j2},⋯,α_{jt}\mathrm{分别是}α_1,α_2,⋯,α_n\mathrm{的最大无关组},\mathrm{且向量组的秩等于}r,\mathrm{则有}()
$$
$$
A.
r=n \quad B.t=n \quad C.r=t \quad D.r\neq t \quad E. \quad F. \quad G. \quad H.
$$
$$
\mathrm{向量组的最大无关组所含向量个数是相同的},\mathrm{且都小于等于向量组中向量的个数},\mathrm{因此}r=t\leq\;n
$$



$$
若α_1,α_2,⋯,α_r\mathrm{是向量组}α_1,α_2,⋯,α_r,⋯,α_n\mathrm{最大无关组},\mathrm{则论断不正确的是}()
$$
$$
A.
α_n\mathrm{可由}α_1,α_2,⋯,α_r\mathrm{线性表示} \quad B.α_1\mathrm{可由}α_{r+1},α_{r+2},⋯,α_n\mathrm{线性表示} \quad C.α_1\mathrm{可由}α_1,α_2,⋯,α_r\mathrm{线性表示} \quad D.α_n\mathrm{可由}α_{r+1},α_{r+2},⋯,α_n\mathrm{线性表示} \quad E. \quad F. \quad G. \quad H.
$$
$$
\mathrm{根据极大线性无关组的定义可知}α_1,α_n\mathrm{都可由}α_1,α_2,⋯,α_r\mathrm{线性表示},又α_n=0α_{r+1}+0α_{r+2}+α_n,\mathrm{所以用排除法可得出答案}.
$$



$$
\mathrm{向量组}α_1=\left(1,0,0\right)^T,α_2=\left(1,1,0\right)^T,α_3=\left(0,0,0\right)^T,α_4=\left(1,1,1\right)^T\mathrm{的最大无关向量组为}().
$$
$$
A.
α_{1\;\;},α_{2\;\;},α_3 \quad B.α_{1\;\;},α_{2\;\;},α_4 \quad C.α_{1\;\;},α_{2\;\;} \quad D.α_{3\;\;},α_4 \quad E. \quad F. \quad G. \quad H.
$$
$$
\mathrm{显然向量组的秩为}3,又α_1,α_2,α_4\mathrm{线性无关},\mathrm{故构成一组最大无关组}.
$$



$$
\mathrm{向量组}α_1=\left(1,1,1,1\right)^T,α_2=\left(0,1,1,2\right)^T,α_3=\left(0,0,1,3\right)^T\mathrm{的最大无关组为}\;().
$$
$$
A.
α_1,α_2,α_3 \quad B.α_1,\alpha_2 \quad C.α_1 \quad D.α_3 \quad E. \quad F. \quad G. \quad H.
$$
$$
\begin{array}{l}\\\mathrm{对向量组构成的矩阵进行初等行变换得}\\\;\;\;\;\;\;\;\;\;\;\;\;\;\;\;\;\;(α_1,α_2,α_3)=\begin{pmatrix}1&0&0\\1&1&0\\1&1&1\\1&2&3\end{pmatrix}\rightarrow\begin{pmatrix}1&0&0\\0&1&0\\0&0&1\\0&0&0\end{pmatrix}\\\mathrm{因此向量组的秩为}3,又α_1,α_2,α_3\mathrm{线性无关},\mathrm{因此构成向量组的最大无关组}.\\\end{array}
$$



$$
\begin{array}{l}\mathrm{设矩阵}A=\begin{pmatrix}1&2&3&0&1\\2&4&6&1&4\\3&6&9&0&4\end{pmatrix},\mathrm{用列向量组可表示为}A=(α_1,α_2,α_3,α_4,α_5),\mathrm{则关于列向量组的秩和最大无关组},\mathrm{下列说法正确的是}()\\(1)\mathrm{秩为}3,\mathrm{最大无关组为}α_1,α_4,α_5\;\;\;\;\;(2)\mathrm{秩为}3,\mathrm{最大无关组为}α_1,α_2,α_4\\(3)\mathrm{秩为}3,\mathrm{最大无关组为}α_1,α_2,α_5\;\;\;\;\;(4)\mathrm{秩为}3,\mathrm{最大无关组为}α_1,α_2,α_3\end{array}
$$
$$
A.
(1) \quad B.(2) \quad C.(3) \quad D.(4) \quad E. \quad F. \quad G. \quad H.
$$
$$
A=\begin{pmatrix}1&2&3&0&1\\2&4&6&1&4\\3&6&9&0&4\end{pmatrix}\rightarrow\begin{pmatrix}1&2&3&0&1\\0&0&0&1&2\\0&0&0&0&1\end{pmatrix}\mathrm{向量组的秩为}3,\mathrm{最大无关组为}α_1,α_4,α_5
$$



$$
\mathrm{向量组}α=\left(1,3,4,5\right)^T,β=\left(0,1,3,1\right)^T,γ=\left(1,2,1,4\right)^T\mathrm{的秩和最大无关组为}\;().
$$
$$
A.
3;α,β,γ \quad B.2;α,β \quad C.1;α \quad D.0 \quad E. \quad F. \quad G. \quad H.
$$
$$
(α,β,γ)=\begin{pmatrix}1&0&1\\3&1&2\\4&3&1\\5&1&4\end{pmatrix}\rightarrow\begin{pmatrix}1&0&1\\0&1&-1\\0&0&0\\0&0&0\end{pmatrix}\mathrm{所以}R(α,β.γ)=2,α,β\mathrm{为一个最大无关组}.
$$



$$
\mathrm{向量组}α_1=\left(1,0,-1,4\right)^T,α_2=\left(1,1,0,3\right)^T,α_3=\left(-2,0,2,-8\right)^T,α_4=\left(-1,0,0,1\right)^T\mathrm{的最大线性无关组为}\;().
$$
$$
A.
α_1,α_2,α_4 \quad B.α_1,α_2,α_3 \quad C.α_1,α_2 \quad D.α_2,α_4 \quad E. \quad F. \quad G. \quad H.
$$
$$
\begin{array}{l}\mathrm{对下列矩阵施以初等变换}:(α_1,α_2,α_3,α_4)=\begin{pmatrix}1&1&-2&-1\\0&1&0&0\\-1&0&2&0\\4&3&-8&1\end{pmatrix}\rightarrow\begin{pmatrix}1&1&-2&-1\\0&1&0&0\\0&1&0&-1\\0&-1&0&5\end{pmatrix}\rightarrow\begin{pmatrix}1&1&-2&-1\\0&1&0&0\\0&0&0&1\\0&0&0&0\end{pmatrix}\\\mathrm{从最后一个矩阵可看出向量组的秩为}3,\mathrm{故线性相关},\mathrm{且最大无关组为}α_1,α_2,α_4.\end{array}
$$



$$
\mathrm{向量组}α_1=\left(1,1,4,2\right)^T,α_2=\left(1,-1,-2,4\right)^T,α_3=\left(-3,2,3,-11\right)^T,α_4=\left(1,3,10,0\right)^T\mathrm{的最大线性无关组为}\;().
$$
$$
A.
α_1,α_2 \quad B.α_1,α_2,α_3 \quad C.α_2,α_3,α_4 \quad D.α_1,α_2,α_3,α_4 \quad E. \quad F. \quad G. \quad H.
$$
$$
\begin{array}{l}把α_i\mathrm{写成列向量},\mathrm{构成矩阵}A,\;\mathrm{再做初等变换化}A\mathrm{为阶梯型}:\begin{pmatrix}1&1&-3&1\\1&-1&2&3\\4&-2&3&10\\2&4&-11&0\end{pmatrix}\rightarrow\begin{pmatrix}1&1&-3&1\\0&-2&5&2\\0&-6&15&6\\0&2&-5&-2\end{pmatrix}\rightarrow\begin{pmatrix}1&1&-3&1\\0&-2&5&2\\0&0&0&0\\0&0&0&0\end{pmatrix}\\\mathrm{那么阶梯形矩阵中每一行第一个非零元所在的列对应的列向量}α_1,α_2\mathrm{就是最大线性无关组}.\end{array}
$$



$$
\begin{array}{l}\mathrm{向量组}α_1=\left(1,-1,2,4\right)^T,α_2=\left(0,3,1,2\right)^T,α_3=\left(3,0,7,14\right)^T,α_4=\left(1,-2,2,0\right)^T,α_5=\left(2,1,5,10\right)^T,\\\mathrm{则该向量组的最大线性无关组为}\;().\end{array}
$$
$$
A.
α_1,α_2,α_3 \quad B.α_1,α_2,α_4 \quad C.α_1,α_2,α_5 \quad D.α_1,α_2,α_4,α_5 \quad E. \quad F. \quad G. \quad H.
$$
$$
\begin{array}{l}\mathrm{此题可以采用排除法},因-3α_1-α_2+α_3=0,-2α_1-α_2+α_5=0,\\\mathrm{由于最大线性无关组中的向量是线性无关的},\mathrm{因此只有线性无关组}α_1,α_2,α_4\mathrm{是最大无关组}.\\或(α_1,α_2,α_3,α_4,α_5)\rightarrow\begin{pmatrix}1&0&3&1&2\\0&3&3&-1&3\\0&1&1&0&1\\0&2&2&-4&2\end{pmatrix}\rightarrow\begin{pmatrix}1&0&1&0&0\\0&1&1&0&1\\0&0&0&1&0\\0&0&0&0&0\end{pmatrix},\mathrm{所以}α_1,α_2,α_4\mathrm{是最大无关组}.\end{array}
$$



$$
\begin{array}{l}设α_1=\left(1,-1,2,4\right)^T,α_2=\left(0,3,1,2\right)^T,\alpha_3=\left(3,0,7,14\right)^T,α_4=\left(1,2,2,0\right)^T,α_5=\left(1,2,3,6\right)^T,\\\mathrm{则向量组}α_1,α_2,α_3,α_4,α_5\mathrm{的最大线性无关组为}\;().\end{array}
$$
$$
A.
α_1,α_2,α_3 \quad B.α_1,α_2,α_4 \quad C.α_1,α_2,α_5 \quad D.α_1,α_2,α_4,α_5 \quad E. \quad F. \quad G. \quad H.
$$
$$
\begin{array}{l}\mathrm{由于}α_3=3α_1+α_2\;,\mathrm{则向量组}α_1,α_2,α_3\mathrm{线性相关};\\又α_5=α_1+α_2,\mathrm{因此向量组}α_1,α_2,α_5\mathrm{以及}α_1,α_2,α_4,α_5\mathrm{线性相关};\\\mathrm{故由排除法可知向量组}α_1,α_2,α_4\mathrm{是最大无关组}.\end{array}
$$



$$
\begin{array}{l}\mathrm{向量组}α_1=\left(1,0,2,-1\right)^T,α_2=\left(1,2,0,1\right)^T,α_3=\left(0,1,-1,1\right)^T,α_4=\left(1,1,1,0\right)^T,α_5=\left(3,2,4,0\right)^T,\\\mathrm{则这个向量组的最大线性无关组可取}().\end{array}
$$
$$
A.
α_1,α_2 \quad B.α_1,α_2,α_3 \quad C.α_1,α_2,α_4 \quad D.α_1,α_2,α_5 \quad E. \quad F. \quad G. \quad H.
$$
$$
\begin{array}{l}\begin{array}{l}(α_1,α_2,α_3,α_4,α_5)\rightarrow\begin{pmatrix}1&1&0&1&3\\0&2&1&1&2\\2&0&-1&1&4\\-1&1&1&0&0\end{pmatrix}\rightarrow\begin{pmatrix}1&1&0&1&3\\0&2&1&1&2\\0&0&0&0&1\\0&0&0&0&0\end{pmatrix},\mathrm{故向量组的秩为}3,\mathrm{一组最大无关组为}α_1,α_2,α_5\end{array}\\\mathrm{而向量组}α_1,α_2,α_3和α_1,α_2,α_4\mathrm{均线性相关}.\end{array}
$$



$$
\begin{array}{l}\mathrm{向量组}(1)α_1=\begin{pmatrix}1\\2\\-1\\4\end{pmatrix},α_2=\begin{pmatrix}9\\100\\10\\4\end{pmatrix},α_3=\begin{pmatrix}-2\\-4\\2\\-8\end{pmatrix}\;,\mathrm{向量组}\;(2)α_1=(1,2,1,3)^T,α_2=(4,-1,-5,-6)^T,α_3=(1,-3,-4,-7)^T\\\mathrm{的最大无关组分别为}()\end{array}
$$
$$
A.
α_1,α_2;α_1,α_2 \quad B.α_1,α_3;α_1,α_2 \quad C.α_1,α_3;α_1,α_2,α_3 \quad D.α_1,α_2;α_1,α_2,α_3 \quad E. \quad F. \quad G. \quad H.
$$
$$
\begin{array}{l}(1)-2α_1=α_3⇒α_1,α_3\mathrm{线性相关},\mathrm{向量组的秩为}2,\mathrm{一组最大线性无关组为}α_1,α_2.\\(2)(α_1,α_2,α_3)\rightarrow\begin{pmatrix}1&4&1\\0&-9&-5\\0&-9&-5\\0&-18&-10\end{pmatrix}\rightarrow\begin{pmatrix}1&4&1\\0&9&5\\0&0&0\\0&0&0\end{pmatrix}\mathrm{向量组的秩为}2,\mathrm{最大线性无关组为}α_1,α_2.\end{array}
$$



$$
\mathrm{向量组}α_1=\left(1,1,1,4\right)^T,α_2=\left(2,1,3,5\right)^T,α_3=\left(1,-1,3,-2\right)^T,α_4=\left(3,1,5,6\right)^T\mathrm{的最大无关组为}().
$$
$$
A.
α_1,α_2 \quad B.α_1,α_2,α_3 \quad C.α_2,α_3,α_4 \quad D.\alpha_1,α_2,α_3,α_4 \quad E. \quad F. \quad G. \quad H.
$$
$$
\begin{array}{l}(α_1,α_2,α_3,α_4)=\begin{pmatrix}1&2&1&3\\1&1&-1&1\\1&3&3&5\\4&5&-2&6\end{pmatrix}\rightarrow\begin{pmatrix}1&2&1&3\\0&-1&-2&-2\\0&1&2&2\\0&-3&-6&-6\end{pmatrix}\rightarrow\begin{pmatrix}1&2&1&2\\0&1&2&2\\0&0&0&0\\0&0&0&0\end{pmatrix}\rightarrow\begin{pmatrix}1&0&-3&-1\\0&1&2&2\\0&0&0&0\\0&0&0&0\end{pmatrix}\\\mathrm{得到一个最大无关组}α_1,α_2,且α_3=-3α_1+2α_2,α_4=-α_1+2α_2\end{array}
$$



$$
\mathrm{下列说法中},\mathrm{向量组}α_1,α_2,⋯,α_s\mathrm{必定线性相关的是}\;(\;\;\;).
$$
$$
A.
β_1,β_2,⋯,β_{s-1}\mathrm{可由}α_1,α_2,⋯,α_s\mathrm{线性表示} \quad B.R(α_1,⋯,α_s,β_1,⋯,β_{s-1})=R(β_1,⋯,β_{s-1}) \quad C.R(α_1,⋯,α_s)=R(α_1,⋯,α_s,β) \quad D.α_1=\begin{pmatrix}β_1\\γ_1\end{pmatrix},α_2=\begin{pmatrix}β_2\\γ_2\end{pmatrix},⋯α_s=\begin{pmatrix}β_s\\γ_s\end{pmatrix},\mathrm{其中}γ_1,γ_2…γ_s\mathrm{线性相关} \quad E. \quad F. \quad G. \quad H.
$$
$$
\begin{array}{l}由R(α_1,⋯,α_s,β_1,⋯,β_{s-1})=R(β_1,⋯,β_{s-1}),\mathrm{可得}α_1,α_2,⋯,α_s\mathrm{可由},β_1,β_2⋯,β_{s-1}\mathrm{线性表示},\\又s-1<\;s,\mathrm{根据线性相关定理可知},α_1,α_2,⋯,α_s\mathrm{线性相关}.\;\mathrm{其余选项中的无法推出}.\end{array}
$$



$$
\begin{array}{l}\mathrm{下列矩阵中},\mathrm{列向量的最大无关组由第}1,2,3\mathrm{列向量构成的为}()\\(1)\begin{pmatrix}1&1&0\\2&0&4\\2&3&-2\end{pmatrix}\;\;\;\;\;(2)\begin{pmatrix}25&31&17&43\\75&94&53&132\\75&94&54&134\\25&32&20&48\end{pmatrix}\end{array}
$$
$$
A.
(1) \quad B.(1)(2) \quad C.(1)(2)\mathrm{都不是} \quad D.(2) \quad E. \quad F. \quad G. \quad H.
$$
$$
\begin{array}{l}(1)\begin{pmatrix}1&1&0\\2&0&4\\2&3&-2\end{pmatrix}\xrightarrow[{r_3-2r_1}]{r_2-2r_1}\begin{pmatrix}1&1&0\\0&-2&4\\0&1&-2\end{pmatrix}\xrightarrow[{r_3\div(-2)}]{r_2+2r_3}\begin{pmatrix}1&1&0\\0&0&0\\0&-\frac12&1\end{pmatrix},\mathrm{第一列和第三列向量是矩阵的列向量组的一个最大无关组}.\\(2)\begin{pmatrix}25&31&17&43\\75&94&53&132\\75&94&54&134\\25&32&20&48\end{pmatrix}\xrightarrow[\begin{array}{c}r_3-3r_1\\r_4-r_1\end{array}]{r_2-3r_1}\begin{pmatrix}25&31&17&43\\0&1&2&3\\0&1&3&5\\0&1&3&5\end{pmatrix}\xrightarrow[{r_3-r2}]{r_4-r_3}\begin{pmatrix}25&31&17&43\\0&1&2&3\\0&0&1&2\\0&0&0&0\end{pmatrix}\mathrm{所以第}1,2,3\mathrm{列构成一个最大无关组}.\end{array}
$$



$$
\begin{array}{l}\mathrm{设矩阵}A=\begin{pmatrix}1&2&3&0&1\\2&1&1&2&1\\1&3&4&0&1\end{pmatrix},\mathrm{用列向量组可表示为}A=(α_1,α_2,α_3,α_4,α_5),\mathrm{则关于列向量组的秩和最大无关组},\mathrm{下列说法不正确的是}()\end{array}
$$
$$
A.
\;\mathrm{秩为}3,\mathrm{最大无关组为}α_1,α_2,α_4\; \quad B.\;\mathrm{秩为}3,\mathrm{最大无关组为}α_1,α_2,α_3 \quad C.\;\mathrm{秩为}3,\mathrm{最大无关组为}α_1,α_2,α_5 \quad D.\;\mathrm{秩为}2,\mathrm{最大无关组为}α_1,α_2 \quad E. \quad F. \quad G. \quad H.
$$
$$
A=\begin{pmatrix}1&2&3&0&1\\2&1&1&2&1\\1&3&4&0&1\end{pmatrix}\rightarrow\begin{pmatrix}1&2&3&0&1\\0&-3&-5&2&-1\\0&1&1&0&0\end{pmatrix}\rightarrow\begin{pmatrix}1&2&3&0&1\\0&1&1&0&0\\0&0&-2&2&-1\end{pmatrix},\mathrm{所以}R(A)=3,\mathrm{最大无关组可以为}α_1,α_2,α_3或\;α_1,α_2,α_4或\;α_1,α_2,α_5.
$$



$$
\begin{array}{l}\mathrm{设矩阵}A=\begin{pmatrix}0&1&0&1&-1\\-1&1&1&2&0\\-1&0&-1&5&-3\end{pmatrix},\mathrm{用列向量组可表示为}A=(α_1,α_2,α_3,α_4,α_5),\mathrm{则关于列向量组的秩和最大无关组},\mathrm{下列说法不正确的是}()\end{array}
$$
$$
A.
\mathrm{秩为}3,\mathrm{最大无关组为}α_1,α_2,α_4 \quad B.\;\mathrm{秩为}3,\mathrm{最大无关组为}α_1,α_2,α_3 \quad C.\mathrm{秩为}3,\mathrm{最大无关组为}α_1,α_2,α_5\; \quad D.\mathrm{秩为}2,\mathrm{最大无关组为}α_1,α_2 \quad E. \quad F. \quad G. \quad H.
$$
$$
\begin{array}{l}A=\begin{pmatrix}0&1&0&1&-1\\-1&1&1&2&0\\-1&0&-1&5&-3\end{pmatrix}\rightarrow\begin{pmatrix}-1&1&1&2&0\\0&1&0&1&-1\\0&-1&-2&3&-3\end{pmatrix}\rightarrow\begin{pmatrix}-1&1&1&2&0\\0&1&0&1&-1\\0&0&-2&4&-4\end{pmatrix},\mathrm{所以}R(A)=3,\mathrm{最大无关组可以为}α_1,α_2,α_3或α_1,α_2,α_4\;\;或α_1,α_2,α_5\;,\\\mathrm{故答案}D\mathrm{不正确}.\end{array}
$$



$$
\begin{array}{l}\mathrm{设矩阵}A=\begin{pmatrix}1&1&3&1&4\\2&1&4&1&5\\1&1&3&2&6\end{pmatrix},\mathrm{用列向量组可表示为}A=(α_1,α_2,α_3,α_4,α_5),\mathrm{则关于列向量组的秩和最大无关组},\mathrm{下列说法不正确的是}()\end{array}
$$
$$
A.
\mathrm{秩为}3,\mathrm{最大无关组为}α_1,α_3,α_4\; \quad B.\mathrm{秩为}3,\mathrm{最大无关组为}α_1,α_2,α_4 \quad C.\mathrm{秩为}3,\mathrm{最大无关组为}α_1,α_2,α_5\; \quad D.\mathrm{秩为}3,\mathrm{最大无关组为}α_1,α_2,α_3 \quad E. \quad F. \quad G. \quad H.
$$
$$
A=\begin{pmatrix}1&1&3&1&4\\2&1&4&1&5\\1&1&3&2&6\end{pmatrix}\rightarrow\begin{pmatrix}1&1&3&1&4\\0&-1&-2&-1&-3\\0&0&0&1&2\end{pmatrix},\mathrm{秩为}3\;,\mathrm{最大无关组为}α_1,α_3,α_4或\alpha_1,α_2,α_4或α_1,α_2,α_5,\mathrm{故答案}D\mathrm{不正确}.
$$



$$
设\;α_1=(1,2,-1,0)^T,\;\;\;α_2=(1,1,0,2)^T,\;\;\;α_3=(2,1,1,a)^T,\;\;\;\mathrm{若由}\;α_1,\;α_2,α_3\mathrm{形成的向量空间维数是}2,则\;a=(\;).
$$
$$
A.
2 \quad B.6 \quad C.3 \quad D.5 \quad E. \quad F. \quad G. \quad H.
$$
$$
\begin{array}{l}\mathrm{应填}6.\\(α_1,α_2,α_3)=\begin{pmatrix}1&1&2\\2&1&1\\-1&0&1\\0&2&a\end{pmatrix}\rightarrow\begin{pmatrix}1&1&2\\0&-1&-3\\0&1&3\\0&2&a\end{pmatrix}\rightarrow\begin{pmatrix}1&1&2\\0&1&3\\0&0&0\\0&0&a-6\end{pmatrix},\\\mathrm{因为由}α_1,α_2,α_3\mathrm{组成向量组的秩为}2,\mathrm{所以}a=6.\end{array}
$$



$$
\mathrm{已知}α_1,α_2,α_3,α_4\mathrm{是线性无关的四维向量},V=\{β=k_1α_1+k_2α_2+k_3α_3+k_4α_4\vert k_1,k_2,k_3,k_4∈ R\},则V\mathrm{的维数是}(\;).
$$
$$
A.
1 \quad B.2 \quad C.3 \quad D.4 \quad E. \quad F. \quad G. \quad H.
$$
$$
\begin{array}{l}V\mathrm{是向量组}\alpha_1,α_2,α_3,α_4\mathrm{的线性组合},\mathrm{即由}α_1,α_2,α_3,α_4\mathrm{生成的向量空间},\\则V\mathrm{的维数等于向量组}α_1,α_2,α_3,α_4\mathrm{的秩},即V是4\mathrm{维向量空间}.\end{array}
$$



$$
设α,β\mathrm{为线性无关的}n\mathrm{维向量},则V=\{x=λα+μβ\vertλ,μ∈ R\}\mathrm{的维数是}\;(\;).
$$
$$
A.
1 \quad B.2 \quad C.0 \quad D.1或2 \quad E. \quad F. \quad G. \quad H.
$$
$$
\mathrm{由生成向量空间的性质易知}dimV=R(α,β)=2
$$



$$
\mathrm{若向量组}α_1=\begin{pmatrix}2\\1\\-2\end{pmatrix},α_2=\begin{pmatrix}0\\3\\1\end{pmatrix},α_3=\begin{pmatrix}0\\0\\k-2\end{pmatrix}\mathrm{能构成}R^3\mathrm{的一个基},则\;(\;).
$$
$$
A.
k\neq2 \quad B.k\neq1 \quad C.k\neq3 \quad D.k\neq0 \quad E. \quad F. \quad G. \quad H.
$$
$$
\begin{array}{l}\mathrm{由条件可知}α_1,α_2,\alpha_3\mathrm{线性无关},则\;\\\;\;\;\;\;\;\;\;\;\;\;\;\;\;\;\;\;\;\;\;\;\;\;\;\;\;\;\;\;\;\;\;\;\;\;\;\;\;\;\;\;\;\;\;\;\;\;\;\;\;\;\;\;\;\;\;\begin{vmatrix}2&0&0\\1&3&0\\-2&1&k-2\end{vmatrix}=6(k-2)\neq0⇒ k\neq2\;.\end{array}
$$



$$
\mathrm{设向量组}α_1=\begin{pmatrix}1\\1\\1\end{pmatrix},α_2=\begin{pmatrix}1\\2\\3\end{pmatrix},α_3=\begin{pmatrix}1\\3\\t\end{pmatrix}\mathrm{能构成}R^3\mathrm{的一个基},则\;(\;).
$$
$$
A.
t\neq5\; \quad B.t\neq-5 \quad C.t\neq-6 \quad D.t=5\; \quad E. \quad F. \quad G. \quad H.
$$
$$
\begin{array}{l}\mathrm{由条件可知}α_1,α_2α_3\mathrm{线性无关},则\;\\\;\end{array}\begin{vmatrix}1&1&1\\1&2&3\\1&3&t\end{vmatrix}\neq0,即t-5\neq0⇒ t\neq5\;.
$$



$$
\mathrm{设向量}α_1=(1,-1,0)^T,α_2=(2,1,3)^T,α_3=(3,0,3)^T,\mathrm{则由}α_1,α_2,α_3\mathrm{生成的向量空间的维数是}(\;).
$$
$$
A.
1 \quad B.2 \quad C.3 \quad D.0 \quad E. \quad F. \quad G. \quad H.
$$
$$
\begin{array}{l}\mathrm{由向量组生成的向量空间的维数等于该向量组的秩},\mathrm{下面用初等行变换求向量组的秩}:\\\;\;\;\;\;\;\;\;\;\;\;\;\;\;\;\;\;\;\;\;\;\;\;\;\;\;\;\;\;\;\;\;\;\;\;\;\;\;\;\;\;\;(α_1,α_2,α_3)\;=\begin{pmatrix}1&2&3\\-1&1&0\\0&3&3\end{pmatrix}\rightarrow\begin{pmatrix}1&2&3\\0&3&3\\0&3&3\end{pmatrix}\rightarrow\begin{pmatrix}1&2&3\\0&1&1\\0&0&0\end{pmatrix},\\\mathrm{故向量组}α_1,α_2,α_3\mathrm{的秩为}2,\mathrm{即生成向量空间的维数为}2.\end{array}
$$



$$
\begin{array}{l}\mathrm{设向量空间}W=\{(x_1,2x_2,3x_1)^T\vert x_1,x_2∈ R\},则W\mathrm{的维数等于}(\;).\\\end{array}
$$
$$
A.
1 \quad B.2 \quad C.3 \quad D.0 \quad E. \quad F. \quad G. \quad H.
$$
$$
\begin{array}{l}\mathrm{由条件可知向量空间}W\mathrm{是由}x_1,x_2\mathrm{组成的三维向量的集合},\mathrm{因此令}x_1=1,x_2=0和x_1=0,x_2=1,\mathrm{则向量组}\\(1,0,3)^T,(0,2,0)^T是W\mathrm{的一组基},则W\mathrm{的维数为}2.\end{array}
$$



$$
设R^4\mathrm{的子空间}W=\{(a,b,c,a+c)^T\vert a,b,c∈ R\},则W\mathrm{的维数为}(\;).\;
$$
$$
A.
1 \quad B.2 \quad C.3 \quad D.4 \quad E. \quad F. \quad G. \quad H.
$$
$$
\begin{array}{l}由W\mathrm{的结构可知},W\mathrm{主要由}a,b,c\mathrm{组成},\mathrm{分别令}(a,b,c)^T=(1,0,0)^T,(0,1,0)^T,(0,0,1)^T时,\mathrm{对应}W\mathrm{中的向量}a_1=(1,0,0,1)^T,a_2=(0,1,0,0)^T,\\a_3=(0,0,1,1)^T\mathrm{构成}W\mathrm{的一组基},即dim(W)=3.\end{array}
$$



$$
设R^4\mathrm{的子空间}W=\{(a,b,a-b,a+b)^T\vert a,b∈ R\},则W\mathrm{的维数为}(\;).\;
$$
$$
A.
1 \quad B.2 \quad C.3 \quad D.4 \quad E. \quad F. \quad G. \quad H.
$$
$$
\begin{array}{l}由W\mathrm{的结构可知},W\mathrm{主要由}a,b\mathrm{组成},\mathrm{分别令}(a,b)=(1,0),(0,1)时,\mathrm{对应}W\mathrm{中的向量}a_1=(1,0,1,1)^T,a_2=(0,1,-1,1)^T,\\\mathrm{构成}W\mathrm{的一组基},即dim(W)=2.\end{array}
$$



$$
设V=\{x=(x_1,x_2,x_3)\vert x_1+x_2+x_3=0,且x_1,x_2,x_3∈ R\},则(\;).
$$
$$
A.
V是1\mathrm{维向量空间} \quad B.V是2\mathrm{维向量空间} \quad C.V是3\mathrm{维向量空间} \quad D.V\mathrm{不是向量空间} \quad E. \quad F. \quad G. \quad H.
$$
$$
\mathrm{由题设可知}V=\{x=(x_1,x_2,-x_1-x_1)\vert x_1,x_2∈ R\},\mathrm{易知}(1,0,-1),(0,1,-1)为V\mathrm{的一组基},\mathrm{所以是}2\mathrm{维的}
$$



$$
设α_1=(1,0,1)^T,α_2=(0,1,1)^T,α_3=(1,1,0)^T\mathrm{生成的向量空间为}V,则V\mathrm{的维数为}(\;).\;
$$
$$
A.
1 \quad B.2 \quad C.3 \quad D.0 \quad E. \quad F. \quad G. \quad H.
$$
$$
\begin{array}{l}\mathrm{对矩阵}(α_1,α_2,α_3)\mathrm{实施初等行变换}\;\;\\\begin{bmatrix}1&0&1\\0&1&1\\1&1&0\end{bmatrix}\rightarrow\begin{bmatrix}1&0&1\\0&1&1\\0&1&-1\end{bmatrix}\rightarrow\begin{bmatrix}1&0&1\\0&1&1\\0&0&-2\end{bmatrix}\rightarrow\begin{bmatrix}1&0&0\\0&1&0\\0&0&1\end{bmatrix},\;\;\\\mathrm{故向量组}α_1,α_2,α_3\mathrm{的秩等于}3,\mathrm{所以}V\mathrm{的维数等于}3.\end{array}
$$



$$
\mathrm{设向量组}α_1=\begin{pmatrix}1\\-1\\1\end{pmatrix},α_2=\begin{pmatrix}1\\-2\\2\end{pmatrix},α_3=\begin{pmatrix}1\\a\\5\end{pmatrix}\mathrm{能构成}R^3\mathrm{的一个基},则\;(\;).
$$
$$
A.
a=5 \quad B.a=-5 \quad C.a\neq5 \quad D.a\neq-5 \quad E. \quad F. \quad G. \quad H.
$$
$$
\mathrm{由条件可知向量组}α_1,α_2,α_3\mathrm{线性无关},则\begin{vmatrix}1&1&1\\-1&-2&a\\1&2&5\end{vmatrix}=a+5\neq0⇒ a\neq-5.
$$



$$
设α_1=(2,1,0)^T,α_2=(0,1,2)^T,α_3=(-1,2,k)^T\mathrm{能构成}R^3\mathrm{的一个基},则\;(\;).
$$
$$
A.
k=5 \quad B.k\neq5 \quad C.k=-5 \quad D.k\neq-5 \quad E. \quad F. \quad G. \quad H.
$$
$$
\mathrm{由条件可知向量组}α_1,α_2,α_3\mathrm{线性无关},则\begin{vmatrix}2&0&-1\\1&1&2\\0&2&k\end{vmatrix}=2k-10\neq0⇒ k\neq5.
$$



$$
设α_1=(1,0,0)^T,α_2=(1,1,0)^T,α_3=(1,1,1)^T\mathrm{生成的向量空间为}V,则V\mathrm{的维数为}\;(\;).
$$
$$
A.
1 \quad B.2 \quad C.3 \quad D.0 \quad E. \quad F. \quad G. \quad H.
$$
$$
\mathrm{因为}\begin{vmatrix}1&1&1\\0&1&1\\0&0&1\end{vmatrix}=1\neq0⇒ R(α_1,α_2,α_3)=3.\mathrm{向量空间的维数就是向量组的秩}.\;
$$



$$
设α_1=(1,0,1)^T,α_2=(0,1,0)^T,α_3=(1,2,2)^T\mathrm{生成的向量空间为}V,则V\mathrm{的维数为}\;(\;).
$$
$$
A.
0 \quad B.1 \quad C.2 \quad D.3 \quad E. \quad F. \quad G. \quad H.
$$
$$
\mathrm{因为}\begin{vmatrix}1&1&1\\0&1&2\\1&0&2\end{vmatrix}=\begin{vmatrix}1&0&1\\0&1&2\\0&0&1\end{vmatrix}=1\neq0⇒ R(α_1,α_2,α_3)=3.
$$



$$
\begin{array}{l}\mathrm{设向量组}α_1=(1,1,1,-1)^T,α_2=(-1,-1,-1,1)^T,α_3=(-1,1,-2,3)^T,α_4=(1,-3,3,-5)^T.V\mathrm{为由}α_1,α_2,α_3,α_4\mathrm{生成的向量空间},\\则V\mathrm{的维数为}(\;).\;\end{array}
$$
$$
A.
1 \quad B.2 \quad C.4 \quad D.3 \quad E. \quad F. \quad G. \quad H.
$$
$$
\mathrm{由向量组生成的向量空间的维数为此向量组的秩},\mathrm{由初等行变换可求出}R(α_1,α_2,α_3,α_4)=2,即dimV(α_1,α_2,α_3,α_4)=R(α_1,α_2,α_3,α_4)=2.
$$



$$
\mathrm{由向量}α_1=(1,0,0,0)^T,α_2=(1,2,0,0)^T,α_3=(1,2,3,0)^T,α_4=(1,2,3,4)^T\mathrm{生成的向量空间的一组基为}(\;).
$$
$$
A.
α_1,α_2,α_3,\alpha_4 \quad B.α_1,α_2,α_3 \quad C.α_1,α_2 \quad D.α_1 \quad E. \quad F. \quad G. \quad H.
$$
$$
\mathrm{矩阵}A=(α_1,α_2,α_3,α_4)\mathrm{是三角形矩阵},α_1,α_2,α_3,α_4\mathrm{生成的向量空间的一个基是向量组本身}.
$$



$$
\mathrm{由向量}α_1=(1,-1,0,0)^T,α_2=(0,1,-1,0)^T,α_3=(0,0,1,-1)^T,α_4=(0,0,0,1)^T\mathrm{生成的向量空间的一组基为}(\;).
$$
$$
A.
α_1,α_2,α_3,α_4 \quad B.α_1,α_2,α_3 \quad C.α_1,α_2 \quad D.α_1 \quad E. \quad F. \quad G. \quad H.
$$
$$
\mathrm{矩阵}A=(α_1,α_2,α_3,α_4)\mathrm{是三角形矩阵},α_1,α_2,α_3,α_4\mathrm{生成的向量空间的一个基是向量组本身}.
$$



$$
\mathrm{由向量}α_1=(1,2,3,4)^T,α_2=(2,3,4,5)^T,α_3=(3,4,5,6)^T,α_4=(4,5,6,7)^T\mathrm{生成的向量空间的一组基为}(\;).
$$
$$
A.
α_1,α_2,α_3,α_4 \quad B.α_1,α_2,α_3 \quad C.α_1,α_2 \quad D.α_1 \quad E. \quad F. \quad G. \quad H.
$$
$$
\begin{array}{l}\mathrm{矩阵}A=(α_1,α_2,α_3,α_4)\rightarrow\begin{pmatrix}1&2&3&4\\0&1&2&3\\0&0&0&0\\0&0&0&0\end{pmatrix},\;\;\\\mathrm{所以}α_1,α_2,α_3,α_4\mathrm{生成的向量空间的一个基是}α_1,α_2.\end{array}
$$



$$
设U=\{α=(x_1,x_2,x_3)^T\vert x_1-x_2=0\},则U\mathrm{的维数为}(\;).\;
$$
$$
A.
0 \quad B.1 \quad C.2 \quad D.3 \quad E. \quad F. \quad G. \quad H.
$$
$$
\begin{array}{l}\mathrm{因为}U\mathrm{中任一向量}α=\lbrack x_1,x_2,x_3\rbrack^T\mathrm{满足}x_1-x_2=0,即x_1=x_2,\mathrm{所以}\;\;\\α=\lbrack x_1,x_2,x_3\rbrack^T=\lbrack x_1,x_2,x_3\rbrack^T=x_2\lbrack1,1,0\rbrack^T+x_3\lbrack0,0,1\rbrack^T\\\mathrm{由于向量}\lbrack1,1,0\rbrack^T和\lbrack0,0,1\rbrack^T\mathrm{是线性无关的},\mathrm{故它是}U\mathrm{的一个基},dimU=2.\end{array}
$$



$$
设R^4\mathrm{的子空间}W=\{(a,b,a+b,c)\vert a,b,c∈ R\},则W\mathrm{的维数为}(\;).\;
$$
$$
A.
1 \quad B.2 \quad C.3 \quad D.4 \quad E. \quad F. \quad G. \quad H.
$$
$$
\begin{array}{l}由W\mathrm{的结构可知},W\mathrm{主要由}a,b,c\mathrm{组成},\mathrm{分别令}(a,b,c)=(1,0,0),(0,1,0),(0,1,1)时,\mathrm{对应}W\mathrm{中的向量}α_1=(1,0,1,0),α_2=(0,1,1,0),\\α_3=(0,0,0,1)\mathrm{构成}W\mathrm{的一组基},即dim(W)=3.\end{array}
$$



$$
设W=\{(a,b,c,d)\vert a+b+c+d=0,a,b,c,d∈ R\}是R^4\mathrm{的一个空间},则W\mathrm{的维数为}(\;).
$$
$$
A.
1 \quad B.2 \quad C.3 \quad D.4 \quad E. \quad F. \quad G. \quad H.
$$
$$
\begin{array}{l}由W\mathrm{的结构可知},W\mathrm{主要是由}a,b,c\mathrm{组成},W=\{(a,b,c,-a-b-c)\vert a,b,c∈ R\},且\\(a,b,c,-a-b-c)=a(1,0,0,-1)+b(0,1,0,-1)+c(0,0,1,-1),\mathrm{显然}\\(1,0,0,-1),(0,1,0,-1),(0,0,1,-1)\mathrm{线性无关},\mathrm{故是}W\mathrm{的一组基},故dim(W)=3\end{array}
$$



$$
\mathrm{设向量空间}U=\{x=(x_1,x_2,x_3)^T\vert x_2=2x_3\},则U\mathrm{的维数为}(\;).
$$
$$
A.
\mathrm{无限维} \quad B.1 \quad C.2 \quad D.3 \quad E. \quad F. \quad G. \quad H.
$$
$$
\begin{array}{l}\mathrm{因为}U\mathrm{中的任一向量}x=(x_1,x_2,x_3)^T\mathrm{满足方程}x_2=2x_3,而x_1和x_3\mathrm{是独立的},\mathrm{可取}α_1=(1,0,0)^T,α_2=(0,2,1{)^T,}\mathrm{易知}α_1,α_2\mathrm{线性无关},\\而U\mathrm{中任一向量}x=(x_1,x_2,x_3)^T\mathrm{可由}α_1,α_2\mathrm{线性表示}\;\;\;\\\;\;\;\;\;\;\;\;\;\;\;\;\;\;\;\;\;\;\;\;\;\;x=\begin{pmatrix}x_1\\x_2\\x_3\end{pmatrix}=\begin{pmatrix}x_1\\2x_3\\x_3\end{pmatrix}=x_1\begin{pmatrix}1\\0\\0\end{pmatrix}+x_3\begin{pmatrix}0\\2\\1\end{pmatrix},\;\;\\故α_1,α_2是R\mathrm{的一个基},dimU=2,(x_1,x_3)\mathrm{是任一向量}x=(x_1,x_2,x_3)^T\mathrm{在基底}α_1,α_2\mathrm{下的坐标}.\end{array}
$$



$$
设V\mathrm{是由向量}α_1=(1,2,3,4)^T,α_2=(-1,0,1,2{)^T,α_3=(3,3,4,7)^T}\mathrm{生成的向量空间},则V\mathrm{的一组基为}(\;).\;
$$
$$
A.
\alpha_1,α_2,α_3 \quad B.α_1,α_2 \quad C.\alpha_1,α_3 \quad D.α_2,α_3 \quad E. \quad F. \quad G. \quad H.
$$
$$
\begin{array}{l}\mathrm{由向量组生成的向量空间的维数等于向量组的秩},\mathrm{用初等行变换来求向量组的秩和最大无关组}:\\(α_1,α_2,α_3)=\begin{pmatrix}1&-1&3\\2&0&3\\3&1&4\\4&2&7\end{pmatrix}\rightarrow\begin{pmatrix}1&-1&3\\0&2&-3\\0&4&-5\\0&6&-5\end{pmatrix}\rightarrow\begin{pmatrix}1&-1&3\\0&2&-3\\0&0&1\\0&0&0\end{pmatrix},\;\;\\\mathrm{故向量组的秩为}3,\mathrm{即向量空间}V\mathrm{的维数也为}3,\mathrm{最大无关组为}α_1,α_2,α_3.\end{array}
$$



$$
在R^4中,\mathrm{由向量}α_1=(2,1,3,1)^T,α_2=(1,2,0,1)^T,α_3=(-1,1,-3,0)^T,α_4=(1,1,1,1)^T\mathrm{生成的向量空间的一组基为}(\;).\;
$$
$$
A.
α_1,α_2 \quad B.α_1,α_3 \quad C.α_1,α_2,α_3 \quad D.α_1,α_2,\alpha_4 \quad E. \quad F. \quad G. \quad H.
$$
$$
\begin{array}{l}\mathrm{由初等行变换可求出}R(α_1,α_2,α_3,\alpha_4)=3,\;\;\\\;(α_1,α_2,α_3,α_4)\rightarrow\begin{pmatrix}1&1&0&0\\0&1&1&0\\0&0&0&1\\0&0&0&0\end{pmatrix}\\\mathrm{即向量空间的一组基为}α_1,α_2,α_4.\end{array}
$$



$$
设V\mathrm{是由向量}α_1=(1,2,3,1)^T,α_2=(1,1,1,2)^T,α_3=(3,3,4,5)^T\mathrm{生成的向量空间},则V\mathrm{的一组基为}(\;).
$$
$$
A.
α_1,α_2,α_3 \quad B.α_1,α_2 \quad C.α_1,α_3 \quad D.α_2,α_3 \quad E. \quad F. \quad G. \quad H.
$$
$$
\begin{array}{l}\mathrm{由向量组生成的向量空间的维数等于向量组的秩},\mathrm{下面进行初等行变换}:\\(α_1,α_2,α_3)=\begin{pmatrix}1&1&3\\2&1&3\\3&1&4\\1&2&5\end{pmatrix}\rightarrow\begin{pmatrix}1&1&3\\0&-1&-3\\0&-2&-5\\0&1&2\end{pmatrix}\rightarrow\begin{pmatrix}1&-1&3\\0&-1&-3\\0&0&1\\0&0&-1\end{pmatrix}\rightarrow\begin{pmatrix}1&-1&3\\0&-1&-3\\0&0&1\\0&0&0\end{pmatrix},\;\;\\\mathrm{故向量组的秩为}3,\mathrm{即向量空间}V\mathrm{的一组基为}α_1,α_2,α_3.\end{array}
$$



$$
\begin{array}{l}\mathrm{设向量组}α_1=(1,1,1,-1)^T,α_2=(-1,-1,-1,1)^T,α_3=(-1,1,-2,3)^T,α_4=(1,-3,3,-5)^T.V\mathrm{为由}\\α_1,α_2,α_3,α_4\mathrm{生成的向量空间},则V\mathrm{的一组基为}(\;).\;\end{array}
$$
$$
A.
α_1,α_2 \quad B.α_1,α_3 \quad C.α_1,α_2,α_3 \quad D.α_1,α_2,α_4 \quad E. \quad F. \quad G. \quad H.
$$
$$
\begin{array}{l}\mathrm{由初等行变换}\\(α_1,α_2,α_3,α_4)\rightarrow\begin{pmatrix}1&-1&-1&1\\0&0&1&-2\\0&0&0&0\\0&0&0&0\end{pmatrix}.\\\mathrm{得向量空间的一组基为}α_1,α_3.\\\end{array}
$$



$$
\begin{array}{l}设α_1=(1,1,0)^T,\;α_2=(0,1,1)^T,\;α_3=(0,0,1)^T\mathrm{生成向量空间}V_1,\;β_1=(1,0,1)^T,\;β_2=(0,1,0)^T,\;β_3=(1,1,2)^T\mathrm{生成向量空间}V_2,\\\mathrm{则下列关系中不正确的是}(\;\;).\end{array}
$$
$$
A.
V_1=V_2 \quad B.V_1\neq V_2 \quad C.V_1=R^3 \quad D.V_2=R^3 \quad E. \quad F. \quad G. \quad H.
$$
$$
\mathrm{易知},\mathrm{向量组}α_1,α_2,α_3\mathrm{与向量组}β_1,β_2,β_3\mathrm{等价},\mathrm{且线性无关},\mathrm{故它们生成的向量空间就是}R^3.
$$



$$
\begin{array}{l}设e_1=(1,0,0)^T,\;e_2=(0,1,0)^T,\;e_3=(0,0,1)^T\mathrm{生成向量空间}V_1,\;α_1=(1,0,1)^T,\;\alpha_2=(0,1,1)^T,\;α_3=(1,1,0)^T\mathrm{生成向量空间}V_2,\\\mathrm{则下列命题中不正确的是}(\;\;).\end{array}
$$
$$
A.
V_1=V_2 \quad B.V_1\neq V_2 \quad C.V_1=R^3 \quad D.V_2=R^3 \quad E. \quad F. \quad G. \quad H.
$$
$$
\mathrm{易知},\mathrm{向量组}α_1,α_2,α_3\mathrm{与向量组}e_1,e_2,e_3\mathrm{等价},\mathrm{且生成的向量空间就是}R^3.
$$



$$
\begin{array}{l}设e_1=(1,0,0)^T,\;e_2=(0,1,0)^T,\;e_3=(0,0,1)^T\mathrm{生成向量空间}V_1,\;α_1=(1,0,1)^T,\;α_2=(0,1,0)^T,\;α_3=(1,2,2)^T\mathrm{生成向量空间}V_2,\\\mathrm{则下列命题中不正确的是}(\;\;).\end{array}
$$
$$
A.
V_1=V_2 \quad B.V_1\neq V_2 \quad C.V_1=R^3 \quad D.V_2=R^3 \quad E. \quad F. \quad G. \quad H.
$$
$$
\mathrm{向量组}α_1,α_2,α_3\mathrm{与向量组}e_1,e_2,e_3\mathrm{等价},\mathrm{且生成的向量空间就是}R^3.
$$



$$
\begin{array}{l}设α_1=(1,2,-1)^T,\;α_2=(1,-1,1)^T,\;α_3=(-1,2,1)^T\mathrm{生成向量空间}V_1,\;β_1=(1,0,1)^T,\;β_2=(0,1,1)^T,\;β_3=(1,1,0)^T\mathrm{生成向量空间}V_2,\\\mathrm{则下列命题中正确的是}(\;\;).\end{array}
$$
$$
A.
V_1=V_2 \quad B.V_1\neq V_2 \quad C.V_1=R^2 \quad D.V_2\neq R^3 \quad E. \quad F. \quad G. \quad H.
$$
$$
\mathrm{向量组}α_1,\alpha_2,α_3\mathrm{与向量组}β_1,β_2,β_3\mathrm{等价},\mathrm{都等价于单位坐标向量组}e_1,e_2,e_3.
$$



$$
\begin{array}{l}设e_1=(1,0,0)^T,\;e_2=(0,1,0)^T,\;e_3=(0,0,1)^T\mathrm{生成向量空间}V_1,\;α_1=(1,0,1)^T,\;α_2=(0,1,0)^T,\;α_3=(1,1,2)^T\mathrm{生成向量空间}V_2,\\\mathrm{则下列命题中不正确的是}(\;\;).\end{array}
$$
$$
A.
V_1=V_2 \quad B.V_1\neq V_2 \quad C.V_1=R^3 \quad D.V_2=R^3 \quad E. \quad F. \quad G. \quad H.
$$
$$
\mathrm{易知},\mathrm{向量组}\alpha_1,α_2,α_3\mathrm{与向量组}e_1,e_2,e_3\mathrm{等价},\mathrm{且生成的向量空间就是}R^3.
$$



$$
\begin{array}{l}设α_1=(1,2,-1)^T,\;α_2=(1,-1,1)^T,\;α_3=(-1,2,1)^T\mathrm{生成向量空间}V_1,\;β_1=(2,1,0)^T,\;β_2=(0,1,2)^T,\;β_3=(-1,2,1)^T\mathrm{生成向量空间}V_2,\\\mathrm{则下列关系式中不正确的是}(\;\;).\end{array}
$$
$$
A.
V_1=V_2 \quad B.V_1\neq V_2 \quad C.V_1=R^3 \quad D.V_2=R^3 \quad E. \quad F. \quad G. \quad H.
$$
$$
\mathrm{向量组}α_1,α_2,α_3\mathrm{与向量组}β_1,β_2,β_3\mathrm{等价},\mathrm{且它们的秩等于}3.
$$



$$
\begin{array}{l}设α_1=(1,0,1)^T,\;α_2=(0,1,1)^T,\;α_3=(1,1,0)^T\mathrm{生成向量空间}V_1,\;β_1=(2,1,0)^T,\;β_2=(0,1,2)^T,\;β_3=(-1,2,1)^T\mathrm{生成向量空间}V_2,\\\mathrm{则下列命题中不正确的是}(\;\;).\end{array}
$$
$$
A.
V_1=R^3 \quad B.V_2=R^3 \quad C.V_1\neq V_2 \quad D.V_1=V_2 \quad E. \quad F. \quad G. \quad H.
$$
$$
\mathrm{向量组}α_1,α_2,α_3\mathrm{与向量组}β_1,β_2,β_3\mathrm{等价},\mathrm{都等价于单位坐标向量组}e_1,e_2,e_3.
$$



$$
\begin{array}{l}设α_1=(1,0,1)^T,\;α_2=(0,1,1)^T,\;α_3=(1,1,0)^T\mathrm{生成向量空间}V_1,\;β_1=(1,0,2)^T,\;β_2=(0,1,2)^T,\;β_3=(1,2,1)^T\mathrm{生成向量空间}V_2,\\\mathrm{则下列命题中不成立的是}(\;\;).\end{array}
$$
$$
A.
V_1=R^3 \quad B.V_2=R^3 \quad C.V_1\neq V_2 \quad D.V_1=V_2 \quad E. \quad F. \quad G. \quad H.
$$
$$
\mathrm{向量组}α_1,α_2,α_3\mathrm{与向量组}β_1,β_2,β_3\mathrm{等价},\mathrm{都等价于向量组}e_1,e_2,e_3.
$$



$$
\begin{array}{l}\mathrm{下列集合是向量空间的有}(\;\;\;).\\(1)\;V=\{x=(0,x_2,⋯,x_n)^T\vert x_2,⋯,x_n∈ R\};\\(2)\;V=\{x=(1,x_2,⋯,x_n)^T\vert x_2,⋯,x_n∈ R\};\\(3)\;V=\{ξ=λα+μβ,\;λ,\;\;μ∈ R\},\;\mathrm{其中}α,β\mathrm{为两个已知的}n\mathrm{维向量};\end{array}
$$
$$
A.
1个 \quad B.2个 \quad C.3个 \quad D.0个 \quad E. \quad F. \quad G. \quad H.
$$
$$
\mathrm{验证集合是否对向量的加法和数乘两种运算封闭即可},\mathrm{从而判断出}(2)\mathrm{对数乘运算不封闭},\mathrm{不是向量空间}.
$$



$$
\begin{array}{l}\mathrm{下列命题中正确的有}(\;\;\;).\\(1)\;\mathrm{向量组}α_1,⋯,α_m\mathrm{与向量组}β_1,⋯β_s\mathrm{等价},则\\\;\{λ_1α_1+λ_2\alpha_2+⋯+λ_mα_m\vertλ_i∈ R\}\;=\;\{μ_1β_1+μ_2β_2+⋯μ_sβ_s\vertμ_i∈ R\};\\(2)\;R^3\mathrm{中过原点的平面是}R^3\mathrm{的子空间};\\(3)\;\mathrm{集合}H=\{(s,t,0)\vert(s,t)∈ R^2\}是R^3\mathrm{的子空间}.\end{array}
$$
$$
A.
1个 \quad B.2个 \quad C.3个 \quad D.0个 \quad E. \quad F. \quad G. \quad H.
$$
$$
\begin{array}{l}\left(1\right)∀ξ∈ V_1,\;则ξ\mathrm{可由}α_1,⋯,α_m\mathrm{线性表示}.故ξ\mathrm{可由}β_1,⋯,β_s\mathrm{线性表示}⇒ξ∈ V_2.⇒ V_1⊂ V_2.\\\mathrm{类似地可证}:\;V_2⊂ V_1.\mathrm{所以}V_1=V_2;\\(2)R^3\mathrm{中过原点的平面可以看作集合}\\\;\;\;\;\;\;\;\;\;V=\{(α,β,γ)∈ R^3\vertα x+β y+γ z=0,\;\mathrm{其中}(x,y,z)∈ R^3\},\\\mathrm{从而可证明它对加法和数乘运算封闭},\mathrm{从而满足子空间的定义};\\(3)\mathrm{易证对加法和数乘运算封闭},\mathrm{因此}H是R^3\mathrm{的子空间}.\\\mathrm{因此}(1)(2)(3)\mathrm{都正确}.\end{array}
$$



$$
\begin{array}{l}\begin{array}{l}\mathrm{下列集合为向量空间的有}().\\(1)\;V_1=\{x=(x_1,x_2,⋯,x_n)^T\vert x_1,⋯,x_n∈ R\mathrm{满足}x_1+x_2+⋯+x_n=0\};\\(2)\;V_2=\{x=(x_1,x_2,⋯,x_n)^T\vert x_1,⋯,x_n∈ R\mathrm{满足}x_1+x_2+⋯+x_n=1\};\\(3)\;R^3\mathrm{中与向量}(0,0,1)\mathrm{不平行的全体向量所组成的集合}.\end{array}\end{array}
$$
$$
A.
0个 \quad B.1个 \quad C.2个 \quad D.3个 \quad E. \quad F. \quad G. \quad H.
$$
$$
\begin{array}{l}(1)V_1\mathrm{是向量空间},\mathrm{加法和数乘运算都封闭};\\(2)V_2\mathrm{不是向量空间},\mathrm{因为}\\\;\;\;\;\;\;\;\;\;\;(a_1+b_1)+\;(a_2+b_2)+⋯+\;(a_n+b_n)=(b_1+b_2+⋯+b_n)+(a_1+a_2+⋯+a_n)=1+1=2,\\故α+β\not∈ V_2\;即V_2\mathrm{关于加法不封闭},\mathrm{从而}V_2\mathrm{不是子空间}.\\(3)R^3\mathrm{中与向量}(0,0,1)\mathrm{不平行的全体向量所组成的集合不构成向量空间}.\\\;\;\;\;\begin{array}{l}\;\;\;∵\mathrm{对向量}α_1=(0,k,0),α_2=(0,-k,1)(k\neq0),\\α_1,α_2\mathrm{均不平行于}(0,0,1),\;\mathrm{不平行的全体向量所组成的集合对加法不封闭}.\;\mathrm{故所给向量集合不构成向量空间}.\;\;\\\mathrm{因此},\mathrm{只有}(1)\mathrm{是向量空间}.\end{array}\end{array}
$$



$$
\begin{array}{l}\mathrm{下列集合中为向量空间的有}().\\(1)\;W_1=\{(a,1,1)\vert a∈ R\};\\(2)\;W_2=\{(a,b,c)\vert b=a+c,a,c∈ R\};\\(3)\;W_3=\{(a,b,c)\vert b=a+c+1,a,c∈ R\};\end{array}
$$
$$
A.
0个 \quad B.1个 \quad C.2个 \quad D.3个 \quad E. \quad F. \quad G. \quad H.
$$
$$
\begin{array}{l}(1)\mathrm{不是},\mathrm{因为任取}a=(a,1,1)∈ W_1,则2a=(2a,2,2)\not∈ W_1;\\(2)是,\mathrm{易证}W_2\mathrm{关于向量的加法和数乘运算封闭},故W_2\mathrm{是向量空间};\\(3)\mathrm{不是},\mathrm{任取}a=(a,b,c)∈ W_3,b=a+c+1,因2b=2a+2c+2,则2a=(2a,2b,c)\not∈ W_3,故W_3\mathrm{不是向量空间}.\end{array}
$$



$$
\begin{array}{l}\mathrm{下列集合为向量空间的是}().\\(1)V=\{γ\vertγ=kα+lβ,k,l∈ R\};\\(2)A\mathrm{为数域}P上m× n\mathrm{型矩阵},有V=\{x=(x_1,x_2,⋯,x_n)\vert Ax=0\};\\(3)V=\{γ\vertγ=k_1α_1+k_2α_2++⋯+k_mα_m,k_1,k_2,⋯,k_m∈ R\};\end{array}
$$
$$
A.
1个 \quad B.2个 \quad C.3个 \quad D.0个 \quad E. \quad F. \quad G. \quad H.
$$
$$
\mathrm{三个向量集合均对向量加法和数乘运算封闭},故V\mathrm{构成向量空间}.
$$



$$
设α_1=(0,1,2),α_2=(1,3,5),α_3=(2,1,0)\mathrm{生成向量空间}V_1;\;β_1=(1,2,3),β_2=(-1,0,1)\mathrm{生成向量空间}V_2;\;\mathrm{则下列命题中正确选项为}()
$$
$$
A.
V_1=V_2 \quad B.V_1\neq V_2 \quad C.V_1=R^3 \quad D.V_2=R^3 \quad E. \quad F. \quad G. \quad H.
$$
$$
\begin{array}{l}\mathrm{可以证明向量组}α_1,α_2,α_3,\mathrm{与向量组}β_1,β_2\mathrm{等价}.\end{array}
$$



$$
\begin{array}{l}\mathrm{下列集合是向量空间的有}()个.\\(1)V=\{x=(x_1,x_2,⋯,x_n)\vert x_1+x_2+⋯+x_n=0,x_i∈ R\}\\(2)V=\{x=(x_1,x_2,⋯,x_n)\vert x_1+2x_2+⋯+nx_n=1,x_i∈ R\}\\(3)V=\{x=(x_1,0,⋯,0,x_n)\vert x_1,x_i∈ R,n\geq2\}\\(4)V=\{x=(x_1,x_2,⋯,x_n)\vert x_1^2+x_2^2+⋯+x_n^2=1,x_i∈ R\}\end{array}
$$
$$
A.
0 \quad B.1 \quad C.2 \quad D.3 \quad E. \quad F. \quad G. \quad H.
$$
$$
\mathrm{验证向量空间对加法与数乘运算的封闭性},(1),(3)\mathrm{符合}.
$$



$$
\begin{array}{l}设α_1=(1,1,0,0),α_2=(1,0,1,1),α_3=(1,3,-2,-2)\mathrm{生成向量空间}V_1;\;\\β_1=(2,-1,3,3),β_2=(0,1-1,-1)\mathrm{生成向量空间}V_2;\;\mathrm{则下列命题正确的是}()\end{array}
$$
$$
A.
V_1=V_2 \quad B.V_1\neq V_2 \quad C.V_1=R^3 \quad D.V_2=R^3 \quad E. \quad F. \quad G. \quad H.
$$
$$
\begin{array}{l}\begin{array}{l}\mathrm{可以证明向量组}\alpha_1,α_2,α_3\mathrm{与向量组}β_1,β_2\mathrm{等价}.\\令A=\begin{pmatrix}α_1\\α_2\\α_3\end{pmatrix},对A\mathrm{初等行变换}:\\A\rightarrow\begin{pmatrix}1&1&0&0\\0&-1&1&1\\0&2&-2&-2\end{pmatrix}\rightarrow\begin{pmatrix}1&1&0&0\\0&-1&1&1\\0&0&0&0\end{pmatrix}\end{array}\\\rightarrow\begin{pmatrix}2&2&0&0\\0&1&-1&-1\\0&0&0&0\end{pmatrix}\rightarrow\begin{pmatrix}2&-1&3&3\\0&1&-1&-1\\0&0&0&0\end{pmatrix}=\begin{pmatrix}β_1\\β_2\\0\end{pmatrix}\\故\{α_1,α_2,α_3\}与\{β_1,β_2\}\mathrm{等价}\\\\\\\end{array}
$$



$$
\mathrm{已知向量}a_1=\begin{pmatrix}1\\-1\\-2\end{pmatrix},a_2=\begin{pmatrix}5\\-4\\-7\end{pmatrix},a_3=\begin{pmatrix}-3\\1\\0\end{pmatrix}和x=\begin{pmatrix}-4\\3\\a\end{pmatrix},\mathrm{如果}x\mathrm{属于由}a_1,a_2,a_3\mathrm{生成的}R^3\mathrm{的子空间},则a\mathrm{的值为}()\;
$$
$$
A.
a=5 \quad B.a=4 \quad C.a=1 \quad D.a=0 \quad E. \quad F. \quad G. \quad H.
$$
$$
\begin{array}{l}\mathrm{如果}x\mathrm{属于由}a_1,a_2,a_3\mathrm{生成的}R^3\mathrm{的子空间当且仅当存在}x_1,x_2,x_3\mathrm{使得}\\\;\;\;\;\;\;\;\;\;\;\;\;\;\;\;\;\;\;\;\;\;\begin{pmatrix}1&5&-3\\-1&-4&1\\-2&-7&0\end{pmatrix}\begin{pmatrix}x_1\\x_2\\x_3\end{pmatrix}=\begin{pmatrix}-4\\3\\a\end{pmatrix}\\\mathrm{施行初等行变换}\\\;\;\;\;\;\;\;\;\;\;\;\;\;\;\;\;\;\;\;\;\;\begin{pmatrix}1&5&-3&-4\\-1&-4&1&3\\-2&-7&0&a\end{pmatrix}\rightarrow\;\;\begin{pmatrix}1&5&-3&-4\\0&1&-2&-1\\0&0&0&a-5\end{pmatrix}\\故a-5=0,\mathrm{因此}x\mathrm{属于由}a_1,a_2,a_3\mathrm{生成的}R^3\mathrm{的子空间当且仅当}a=5\end{array}
$$



$$
\begin{array}{l}\mathrm{向量空间}R^2\mathrm{的一个基}α_1=(1,0)^T,α_2=(1,2)^T,\;若R^2\mathrm{的一向量}x\mathrm{在基}α_1,α_2\mathrm{的坐标为}(-2,3)^T,则x=();\\\mathrm{又已知向量}y=(4,5)^T,则y\mathrm{在基}α_1,α_2\mathrm{的坐标为}()\end{array}
$$
$$
A.
(1,6)^T;\begin{pmatrix}\frac32,&\frac52\end{pmatrix} \quad B.(6,1)^T;\begin{pmatrix}\frac52,&\frac32\end{pmatrix} \quad C.(1,-6)^T;\begin{pmatrix}\frac23,&\frac25\end{pmatrix} \quad D.(-6,1)^T;\begin{pmatrix}\frac53,&\frac23\end{pmatrix} \quad E. \quad F. \quad G. \quad H.
$$
$$
\begin{array}{l}x=(-2)\begin{pmatrix}1\\0\end{pmatrix}+3\begin{pmatrix}1\\2\end{pmatrix}=\begin{pmatrix}1\\6\end{pmatrix}\\设y\mathrm{在基}α_1,α_2\mathrm{的坐标为}(λ_1,λ_2),则\\\;\;\;\;\;\;\;\;\;\;\;\;\;\;λ_1\begin{pmatrix}1\\0\end{pmatrix}+λ_2\begin{pmatrix}1\\2\end{pmatrix}=\begin{pmatrix}4\\5\end{pmatrix}或\begin{pmatrix}1&1\\0&2\end{pmatrix}\begin{pmatrix}λ_1\\λ_2\end{pmatrix}=\begin{pmatrix}4\\5\end{pmatrix}\\故\;\;\;\;λ_1=\frac32,\;λ_2=\frac52\end{array}
$$



$$
\mathrm{向量}x=(3,12,7)^T\mathrm{属于由}v_1=(3,6,3)^T,v_2=(-1,0,1)^T\mathrm{生成的向量空间},则x在v_1,v_2\mathrm{中的坐标为}\;()
$$
$$
A.
(2,\;3) \quad B.(3,\;2) \quad C.(3,\;6) \quad D.(6,\;3) \quad E. \quad F. \quad G. \quad H.
$$
$$
\begin{array}{l}设λ_1,λ_2是x在v_1,v_2\mathrm{中的坐标},则\\\;\;\;\;\;\;\;λ_1\begin{pmatrix}3\\6\\2\end{pmatrix}+λ_2\begin{pmatrix}-1\\0\\1\end{pmatrix}=\begin{pmatrix}3\\12\\7\end{pmatrix}\\\mathrm{利用行变换可得}\\\;\;\;\;\;\;\;\begin{pmatrix}3&-1&3\\6&0&12\\2&1&7\end{pmatrix}\rightarrow\begin{pmatrix}1&0&2\\0&1&3\\0&0&0\end{pmatrix}\\\mathrm{因此}\;λ_1=2,\;λ_2=3\end{array}
$$



$$
\mathrm{三维线性空间一组基}α_1=(1,1,0)^T,\;α_2=(1,0,1)^T,\;α_3=(0,1,1)^T,则β=(2,0,0)^T\mathrm{在上述基底下的坐标是}()
$$
$$
A.
(1,1,\;-1) \quad B.(1,\;-1,\;-1) \quad C.(-1,\;1,\;-1) \quad D.(1,\;-1,\;1) \quad E. \quad F. \quad G. \quad H.
$$
$$
\begin{array}{l}设β=x_1α_1+x_2α_2+x_3α_3=(α_1α_2α_3)\begin{pmatrix}x_1\\x_2\\x_3\end{pmatrix},\;即\begin{pmatrix}2\\0\\0\end{pmatrix}=\begin{pmatrix}1&1&0\\1&0&1\\0&1&1\end{pmatrix}\begin{pmatrix}x_1\\x_2\\x_3\end{pmatrix}\\则\left\{\begin{array}{c}x_1\;+\;x_2\;=\;2\\x_1\;+\;x_3\;=\;0\\x_2\;+\;x_3\;=\;0\end{array}\right.⇒ x_1=1,\;x_2=1,x_3=-1\end{array}
$$



$$
设α_1=(1,2,1),α_1=(2,9,0),α_1=(3,3,4)是R^3\mathrm{的一个基},\mathrm{向量}β\mathrm{在这个基下的坐标是}(-1,3,2),则β=()
$$
$$
A.
(11,\;31,\;7) \quad B.(12,\;30,\;6) \quad C.(13,\;31,\;6) \quad D.(12,\;31,\;7) \quad E. \quad F. \quad G. \quad H.
$$
$$
β=-α_1+3α_2+2α_3=(11,\;31,\;7)
$$



$$
设\;α_1=(1,0,0),\;α_2=(2,2,0),\;α_3=(3,3,3)是R^3\mathrm{的一个基},\mathrm{则向量}β=(2,-1,3)\mathrm{此基下的坐标为}()
$$
$$
A.
(3,-2,1) \quad B.(3,2,1) \quad C.(3,-2,-1) \quad D.(3,2,-1) \quad E. \quad F. \quad G. \quad H.
$$
$$
设β=x_1α_1+x_2α_2+x_3α_3,\;\;\mathrm{由条件可得}x_1=3,\;\;x_2=-2,\;x_3=1\;,\;\;\mathrm{所求坐标}(3,\;-2,\;1)
$$



$$
设α_1=(-\frac1{\sqrt2},\;0,\;\frac1{\sqrt2}),\;α_2=(\frac1{\sqrt6},\;-\frac2{\sqrt6},\;\frac1{\sqrt6}),\;α_3=(\frac1{\sqrt3},\;\frac1{\sqrt3},\;\frac1{\sqrt3}),\;β=(-1,\;2,\;1),则β 在R^3\mathrm{的基}α_1,\;α_2,\;α_3\mathrm{下的坐标为}()\;
$$
$$
A.
(\sqrt2,\;\frac4{\sqrt6},\;\frac2{\sqrt3}) \quad B.(\sqrt2,\;-\frac4{\sqrt6},\;-\frac2{\sqrt3}) \quad C.(\sqrt2,\;\frac4{\sqrt6},\;-\frac2{\sqrt3}) \quad D.(\sqrt2,\;-\frac4{\sqrt6},\;\frac2{\sqrt3}) \quad E. \quad F. \quad G. \quad H.
$$
$$
设β=x_1α_1+x_2α_2+x_3α_3,\mathrm{则易求得}β\mathrm{在给定基下的坐标为}(\sqrt2,\;-\frac4{\sqrt6},\;\frac2{\sqrt3})
$$



$$
设R^3\mathrm{中的基}α_1=(1,0,1)^T,\;α_2=(1,1,0)^T,\;α_3=(0,1,1)^T,\;则α=(1,2,7)^T\mathrm{在此基下的坐标为}()
$$
$$
A.
(3,-2,4) \quad B.(3,-1,3) \quad C.(3,2,-4) \quad D.(3,1,-3) \quad E. \quad F. \quad G. \quad H.
$$
$$
\begin{array}{l}\begin{pmatrix}1&1&0&1\\0&1&1&2\\1&0&1&7\end{pmatrix}\rightarrow\begin{pmatrix}1&10&0&1\\0&1&1&2\\0&-1&1&6\end{pmatrix}\rightarrow\begin{pmatrix}1&1&0&1\\0&1&1&2\\0&0&2&8\end{pmatrix}\\\rightarrow\begin{pmatrix}1&1&0&1\\0&1&1&2\\0&0&1&4\end{pmatrix}\rightarrow\begin{pmatrix}1&1&0&1\\0&1&0&-2\\0&0&1&4\end{pmatrix}\rightarrow\begin{pmatrix}1&0&0&3\\0&1&0&-2\\0&0&1&4\end{pmatrix}\\\mathrm{故所求坐标为}(3,-2,4)\end{array}
$$



$$
\mathrm{向量}β=(1,3,0)^T\mathrm{在向量空间}R^3\mathrm{的一个基}\alpha_1=(1,0,1)^T,α_2=(0,1,0)^T,α_3=(1,2,2)^T\mathrm{下的坐标为}()
$$
$$
A.
(2,5,-1) \quad B.(2,-1,5) \quad C.(2,5,-2) \quad D.(2,5,1) \quad E. \quad F. \quad G. \quad H.
$$
$$
\begin{array}{l}\mathrm{令矩阵}A=(α_1,\;α_2,\;α_3),\;对(Aβ)\mathrm{进行初等行变换}:\\\;\;\;\;\;\;\;\;\;\;\;\;\;\;\;\;(Aβ)=\begin{pmatrix}1&0&1&1\\0&1&2&3\\1&0&2&0\end{pmatrix}\rightarrow\begin{pmatrix}1&0&1&1\\0&1&2&3\\0&0&1&-1\end{pmatrix}\rightarrow\begin{pmatrix}1&0&0&2\\0&1&0&5\\0&0&1&-1\end{pmatrix}\\则\;β=2α_1+5α_2-α_3\\\end{array}
$$



$$
\mathrm{向量}β=(3,2,1)^T\mathrm{在向量空间}R^3\mathrm{的一个基}α_1=(1,0,0)^T,α_2=(1,1,0)^T,α_3=(1,1,1)^T\mathrm{下的坐标为}()
$$
$$
A.
(1,1,1) \quad B.(1,2,3) \quad C.(3,2,1) \quad D.(-1,-1,-1) \quad E. \quad F. \quad G. \quad H.
$$
$$
\begin{array}{l}\mathrm{令矩阵}A=(α_1,\;α_2,\;α_3),\;对(A\vertβ)\mathrm{进行初等行变换}:\\\;\;\;\;\;\;\;\;\;\;\;\;\;\;\;\;(A\vertβ)=\begin{pmatrix}1&1&1&3\\0&1&1&2\\0&0&1&1\end{pmatrix}\rightarrow\begin{pmatrix}1&0&1&1\\0&1&0&1\\0&0&1&1\end{pmatrix}\\则\;β=α_1+α_2+α_3\\\end{array}
$$



$$
\mathrm{向量}β=(1,3,2)^T\mathrm{在向量空间}R^3\mathrm{的一个基}α_1=(1,0,1)^T,α_2=(0,1,0)^T,α_3=(1,2,2)^T\mathrm{下的坐标为}()
$$
$$
A.
(1,1,-1)^T \quad B.(0,-1,1)^T \quad C.(0,1,1)^T \quad D.(0,-1,-1)^T \quad E. \quad F. \quad G. \quad H.
$$
$$
\begin{array}{l}\mathrm{令矩阵}A=(α_1,\;α_2,\;α_3),\;对(Aβ)\mathrm{进行初等行变换}:\\\;\;\;\;\;\;\;\;\;\;\;\;\;\;\;\;(Aβ)=\begin{pmatrix}1&0&1&1\\0&1&2&3\\1&0&2&2\end{pmatrix}\rightarrow\begin{pmatrix}1&0&1&1\\0&1&2&3\\0&0&1&1\end{pmatrix}\rightarrow\begin{pmatrix}1&0&0&0\\0&1&0&1\\0&0&1&1\end{pmatrix}\\则\;β=α_2+α_3\\\end{array}
$$



$$
\mathrm{向量}β=(1,-3,3)^T\mathrm{在向量空间}R^3\mathrm{的一个基}α_1=(1,1,1)^T,α_2=(-1,1,-2)^T,α_3=(-1,-1,1)^T\mathrm{下的坐标为}(\;).
$$
$$
A.
(0,3,-2) \quad B.(3,-2,0) \quad C.(0,-1,-2) \quad D.(-1,-2,0) \quad E. \quad F. \quad G. \quad H.
$$
$$
\begin{array}{l}\mathrm{令矩阵}A=(α_1,α_2,α_3),对(A\;β)\mathrm{进行初等行变换}:\\\;\;\;\;\;\;\;\;\;\;\;(A\;β)=\begin{pmatrix}1&-1&-1&1\\1&1&-1&-3\\1&-2&1&3\end{pmatrix}\rightarrow\begin{pmatrix}1&-1&-1&1\\0&2&0&-4\\0&-1&2&2\end{pmatrix}\rightarrow\begin{pmatrix}1&0&0&3\\0&1&0&-2\\0&0&1&0\end{pmatrix}\\则β=-α_1-2α_2.\\\end{array}
$$



$$
α=(2,-1,2)^T\mathrm{在基}α_1=(1,0,3)^T,α_2=(2,1,1)^T,α_3=(1,1,1)^T\mathrm{下的坐标为}(\;).
$$
$$
A.
(1,2,-3) \quad B.(1,-1,3) \quad C.(2,-2,4) \quad D.(3,1,-3) \quad E. \quad F. \quad G. \quad H.
$$
$$
\begin{array}{l}设α=x_1α_1+x_2α_2+x_3α_3即\begin{pmatrix}2\\-1\\2\end{pmatrix}=x_1\begin{pmatrix}1\\0\\3\end{pmatrix}+x_2\begin{pmatrix}2\\1\\1\end{pmatrix}+x_3\begin{pmatrix}1\\1\\1\end{pmatrix},\\\mathrm{整理后},\mathrm{得方程组}\\\;\;\;\;\;\;\;\;\;\;\;\;\;\;\;\;\;\;\;\;\;\;\;\;\;\;\;\;\;\;\;\;\;\;\;\;\;\;\;\;\;\;\;\;\;\;\left\{\begin{array}{c}x_1+2x_2+x_3=2\\x_2+x_3=-1\\3x_1+x_2+x_3=2\end{array}\right..\\\mathrm{解得}x_1=1,x_2=2,x_3=-3,即x=(1,2,-3)是α\mathrm{基向量下的坐标}\end{array}
$$



$$
\mathrm{向量}β=(0,1,2)^T\mathrm{在向量空间}R^3\mathrm{的一个基}α_1=(1,2,1)^T,α_2=(0,1,1)^T,α_3=(1,1,1)^T\mathrm{下的坐标为}(\;).
$$
$$
A.
(-1,2,1) \quad B.(1,2,1) \quad C.(-1,2,-1) \quad D.(1,2,-1) \quad E. \quad F. \quad G. \quad H.
$$
$$
\begin{array}{l}设α=x_1α_1+x_2α_2+x_3α_3=(α_1α_2α_3)\begin{pmatrix}x_1\\x_2\\x_3\end{pmatrix},即\begin{pmatrix}0\\1\\2\end{pmatrix}=\begin{pmatrix}1&0&1\\2&1&1\\1&1&1\end{pmatrix}\begin{pmatrix}x_1\\x_2\\x_3\end{pmatrix}\\则\;\left\{\begin{array}{c}x_1+x_3=2\\2x_1+x_2+x_3=1⇒ x_1=-1,x_2=2,x_3=1\\x_1+x_2+x_3=2\end{array}\right.\end{array}
$$



$$
\mathrm{向量}β=(1,2,3)^T\mathrm{在向量空间}R^3\mathrm{的一个基}α_1=(1,1,1)^T,α_2=(1,2,1)^T,α_3=(1,1,2)^T\mathrm{下的坐标为}(\;).
$$
$$
A.
(-2,1,2) \quad B.(2,1,2) \quad C.(-2,1,-2) \quad D.(2,1,-2) \quad E. \quad F. \quad G. \quad H.
$$
$$
\begin{array}{l}设β=x_1α_1+x_2\alpha_2+x_3a_3=(α_1α_2α_3)\begin{pmatrix}x_1\\x_2\\x_3\end{pmatrix},即\begin{pmatrix}1\\2\\3\end{pmatrix}=\begin{pmatrix}1&1&1\\1&2&1\\1&1&2\end{pmatrix}\begin{pmatrix}x_1\\x_2\\x_3\end{pmatrix}\\则\;\left\{\begin{array}{c}x_1+x_2+x_3=1\\x_1+2x_2+x_3=2⇒ x_1=-2,x_2=1,x_3=2\\x_1+x_2+2x_3=3\end{array}\right.\end{array}
$$



$$
\mathrm{向量}β=(1,3,2)^T\mathrm{在三维空间}R^3\mathrm{的一组基}α_1=(1,0,1)^T,α_2=(1,1,0)^T,α_3=(0,1,1)^T\mathrm{下的坐标为}(\;).
$$
$$
A.
(0,1,2)^T \quad B.(1,2,0)^T \quad C.(0,2,1)^T \quad D.(0,1,1)^T \quad E. \quad F. \quad G. \quad H.
$$
$$
\begin{array}{l}设β=x_1α_1+x_2α_2+x_3α_3=(a_1a_2a_3)\begin{pmatrix}x_1\\x_2\\x_3\end{pmatrix},即\\\begin{pmatrix}1\\3\\2\end{pmatrix}=\begin{pmatrix}1&1&0\\0&1&1\\1&0&1\end{pmatrix}\begin{pmatrix}x_1\\x_2\\x_3\end{pmatrix}⇒ x_1=0,x_2=1,x_3=2.\end{array}
$$



$$
\begin{array}{l}设R^3\mathrm{的两组基为}(1)α_1=(1,1,1)^T,α_2=(0,1,1)^T,α_3=(0,0,1)^T;\\\;\;\;\;\;\;\;\;\;\;\;\;\;\;\;\;\;\;\;\;\;\;\;\;\;\;\;\;\;(2)β_1=(1,0,1)^T,β_2=(0,1,-1)^T,β_3=(1,2,0)^T;\\则α_1,α_2,α_3到β_1,β_2,β_3\mathrm{的过渡}Q为(\;).\end{array}
$$
$$
A.
\begin{pmatrix}1&0&1\\-1&1&1\\1&-2&-2\end{pmatrix} \quad B.\begin{pmatrix}1&0&1\\1&-1&1\\1&-2&-2\end{pmatrix} \quad C.\begin{pmatrix}1&0&1\\-1&1&-1\\1&-2&-2\end{pmatrix} \quad D.\begin{pmatrix}1&0&1\\-1&1&1\\1&2&-2\end{pmatrix} \quad E. \quad F. \quad G. \quad H.
$$
$$
\begin{array}{l}B=AP⇒ P=A^{-1}B,\;\;\lbrack A\vdots B\rbrack\xrightarrow{\mathrm{初等行变换}}\lbrack E\vdots A^{-1}B\rbrack,\\令\;\;\;\;\;\;\;\;\;\;\;\;\;\;\;A=(α_1,α_2,α_3)=\begin{pmatrix}1&0&0\\1&1&0\\1&1&1\end{pmatrix}\;\;\;\;\;\;\;\;B=(β_1,β_2,β_3)=\begin{pmatrix}1&0&0\\0&1&0\\0&-1&1\end{pmatrix}\\\lbrack A\vdots B\rbrack=\left(\begin{array}{c}1\\1\\1\end{array}\begin{array}{cccccc}0&0&\vdots&1&0&1\\1&0&\vdots&0&1&2\\1&1&\vdots&1&-1&0\end{array}\right)\rightarrow\left(\begin{array}{c}1\\0\\0\end{array}\begin{array}{cccccc}0&0&\vdots&1&0&1\\1&0&\vdots&-1&1&1\\1&1&\vdots&0&-1&-1\end{array}\right)\rightarrow\left(\begin{array}{c}1\\0\\0\end{array}\begin{array}{cccccc}0&0&\vdots&1&0&1\\1&0&\vdots&-1&1&1\\0&1&\vdots&0&-2&-2\end{array}\right)\\\mathrm{故过渡矩阵}P=\begin{pmatrix}1&0&1\\-1&1&1\\1&-2&-2\end{pmatrix}.\\\end{array}
$$



$$
\begin{array}{l}设R^3\mathrm{的两组基为}(1)α_1=(1,1,1)^T,α_2=(0,1,1)^T,α_3=(0,0,1)^T;\\\;\;\;\;\;\;\;\;\;\;\;\;\;\;\;\;\;\;\;\;\;\;\;\;\;\;\;\;\;(2)β_1=(1,0,1)^T,β_2=(0,1,1)^T,β_3=(1,1,0)^T;\\则α_1,α_2,α_3到β_1,β_2,β_3\mathrm{的过渡}Q为(\;).\end{array}
$$
$$
A.
\begin{pmatrix}1&0&1\\-1&1&0\\1&0&-1\end{pmatrix} \quad B.\begin{pmatrix}1&0&1\\1&-1&0\\1&0&-1\end{pmatrix} \quad C.\begin{pmatrix}1&0&1\\-1&0&1\\1&-1&0\end{pmatrix} \quad D.\begin{pmatrix}1&0&-1\\-1&1&0\\1&1&-1\end{pmatrix} \quad E. \quad F. \quad G. \quad H.
$$
$$
\begin{array}{l}\mathrm{解题思路}:B=AQ⇒ Q=A^{-1}B,\;\;\lbrack A\vdots B\rbrack\xrightarrow{\mathrm{初等行变换}}\lbrack E\vdots A^{-1}B\rbrack,\\令\;\;\;\;\;\;\;\;\;\;\;\;\;\;\;A=(α_1,α_2,α_3)=\begin{pmatrix}1&0&0\\1&1&0\\1&1&1\end{pmatrix}\;\;\;\;\;\;\;\;B=(β_1,β_2,β_3)=\begin{pmatrix}1&0&1\\0&1&1\\1&1&0\end{pmatrix}\\\lbrack A\vdots B\rbrack=\left(\begin{array}{c}1\\1\\1\end{array}\begin{array}{cccccc}0&0&\vdots&1&0&1\\1&0&\vdots&0&1&1\\1&1&\vdots&1&1&0\end{array}\right)\rightarrow\left(\begin{array}{c}1\\0\\0\end{array}\begin{array}{cccccc}0&0&\vdots&1&0&1\\1&0&\vdots&-1&1&0\\0&1&\vdots&1&0&-1\end{array}\right)\\\mathrm{故过渡矩阵}Q=\begin{pmatrix}1&0&1\\-1&1&0\\1&0&-1\end{pmatrix}.\\\end{array}
$$



$$
设α_1=\begin{pmatrix}-\frac35&\frac45&0\end{pmatrix},α_2=\begin{pmatrix}\frac45&\frac35&0\end{pmatrix},α_2=\begin{pmatrix}0&0&1\end{pmatrix}为R^3\mathrm{的一组基},\mathrm{则向量}α=(1,-1,2)\mathrm{在此基下的坐标为}(\;).\;
$$
$$
A.
\begin{pmatrix}-\frac75&,\frac15,&2\end{pmatrix} \quad B.\begin{pmatrix}-\frac75&,-\frac15,&2\end{pmatrix} \quad C.\begin{pmatrix}\frac75&,-\frac15,&2\end{pmatrix} \quad D.\begin{pmatrix}\frac75&,\frac15,&2\end{pmatrix} \quad E. \quad F. \quad G. \quad H.
$$
$$
\begin{array}{l}设α=\;x_1α_1+x_2α_2+x_3\alpha_3,则\\\;\;\;\;\;\;\;\;\;\;\;\;\;\;\;\;\;\;\;\;\;\;\;\;\;\;\;\;\;\;\;\;\;\;\left\{\begin{array}{c}-\frac35x_1+\frac45x_2=1\\\frac45x_1+\frac35x_2=-1\\x_3=2\end{array}\right.⇒ x_1=-\frac75,x_2=\frac15,x_3=2.\\故α=-\frac75α_1+\frac15α_2+2a_3.\end{array}
$$



$$
\begin{array}{l}设R^3\mathrm{中的一组基}ξ_1=(1,-2,1)^T,ξ_2=(0,1,1)^T,ξ_3=(3,2,1)^T,\mathrm{向量}α\mathrm{在基}ξ_1,ξ_2,ξ_3\mathrm{下的坐标}(x_1x_2x_3),\mathrm{在另}\\\mathrm{一组基}η_1,η_2,η_3\mathrm{下的坐标为}(y_1,y_2,y_3),\mathrm{具有}\\\;\;\;\;\;\;\;\;\;\;\;\;\;\;\;\;\;\;\;\;\;\;\;\;\;\;\;\;\;\;\;\;\;\;\;\;\;\;\;\;\;\;\;\;\;\;\;\;\;\;\;\;\;\;\;\;\;\;\;\;\;\;\;\;\;\;\;\;y_1=x_1-x_2-x_3,y_2=-x_1+x_2,y_3=x_1+2x_3,\\\mathrm{则由基}η_1,η_2,η_3\mathrm{到基}ξ_1,ξ_2,ξ_3\mathrm{的过渡矩阵为}(\;).\end{array}
$$
$$
A.
\begin{pmatrix}1&-1&-1\\-1&1&0\\1&0&2\end{pmatrix} \quad B.\begin{pmatrix}1&-1&-1\\1&1&0\\1&1&2\end{pmatrix} \quad C.\begin{pmatrix}-1&1&-1\\-1&1&1\\0&2&2\end{pmatrix} \quad D.\begin{pmatrix}-1&1&-1\\-1&2&0\\1&3&2\end{pmatrix} \quad E. \quad F. \quad G. \quad H.
$$
$$
\begin{array}{l}\mathrm{由题设知}\begin{pmatrix}y_1\\y_2\\y_3\end{pmatrix}=\begin{pmatrix}1&-1&-1\\-1&1&0\\1&0&2\end{pmatrix}\begin{pmatrix}x_1\\x_2\\x_3\end{pmatrix}=C\begin{pmatrix}x_1\\x_2\\x_3\end{pmatrix}.\\\mathrm{因为}\\\;\;\;\;\;\;\;\;\;\;\;\;\;\;\;α=(ξ_1,ξ_2,ξ_3)\begin{pmatrix}x_1\\x_2\\x_3\end{pmatrix}=(η_1,η_2,η_3)\begin{pmatrix}y_1\\y_2\\y_3\end{pmatrix}=(η_1,η_2,η_3)C\begin{pmatrix}x_1\\x_2\\x_3\end{pmatrix},\\\mathrm{所以}\;\;\;\;\;\;\;\;\;\;\;\;\;\;\;\;\;\;\;\;\;(ξ_1,ξ_2,ξ_3)=\;\;\;\;(η_1,η_2,η_3)C.\\\mathrm{即由基}η_1,η_2,η_3\mathrm{到基}ξ_1,ξ_2,ξ_3\mathrm{的过渡矩阵为}\\\;\;\;\;\;\;\;\;\;\;\;\;\;\;\;\;\;\;\;\;\;\;\;\;\;\;\;\;\;\;\;\;\;\;\;\;\;\;\;\;\;\;\;\;\;\;\;\;\;\;\;\;\;\;\;\;\;\;\;\;\;\;\;C=\begin{pmatrix}1&-1&-1\\-1&1&0\\1&0&2\end{pmatrix}\end{array}
$$



$$
\mathrm{向量}v_1=(5,0,7)^T,v_2=(-9,-8,-13)^T\mathrm{在基}α_1=(1,-1,0)^T,α_2=(2,1,3)^T,α_3=(3,1,2)^T\mathrm{下的坐标分别为}(\;).
$$
$$
A.
(2,3,-1);(3,-3,-2) \quad B.(2,1,-3);(3,-3,-2) \quad C.(2,3,-1);(3,1,2) \quad D.(2,1,-3);(3,1,2) \quad E. \quad F. \quad G. \quad H.
$$
$$
\begin{array}{l}设v_1=k_1α_1+k_2α_2+k_3α_3,则\\\;\;\;\;\;\;\;\;\;\;\;\;\;\;\;\;\;\;\;\;\;\;\;\;\;\;\;\;\;\;\;\;\;\left\{\begin{array}{c}k_1+2k_2+3k_3=5\\-k_1+k_2+k_3=0\\3k_2+2k_3=7\end{array}\right.⇒\left\{\begin{array}{c}k_1=2\\k_2=3\\k_3=-1\end{array}\right.,\\故\;\;\;\;\;\;\;\;\;\;\;\;\;\;\;\;\;\;\;\;\;\;\;\;\;v_1=2α_1+3α_2-α_3.\\设v_2=λ_1α_1+λ_2α_2+λ_3α_3,则\\\;\;\;\;\;\;\;\;\;\;\;\;\;\;\;\;\;\;\;\;\;\;\;\;\;\;\;\;\;\;\;\;\;\;\;\;\;\;\;\left\{\begin{array}{c}λ_1+2λ_2+3λ_3=-9\\-λ_1+λ_2+λ_3=-8\\3λ_2+2λ_3=-13\end{array}\right.⇒\left\{\begin{array}{c}λ_1=3\\λ_2=-3\\λ_3=-2\end{array}\right.,\\故\;\;\;\;\;\;\;\;\;\;\;\;\;\;\;\;\;\;\;\;v_2=3α_1-3α_2-2α_3.\end{array}
$$



$$
\begin{array}{l}R^3\mathrm{中的两组基为}\\\;\;\;\;\;\;\;\;\;\;\;\;\;\;\;\;\;\;\;\;\;\;\;ζ_1=(1,0,0)^T,ζ_2=(-1,1,0)^T,ζ_3=(1,-2,1)^T;η_1=(2,0,0)^T,η_2=(-2,1,0)^T,η_3=(4,-4,1)^T.\\则ζ_1,ζ_2,ζ_3到η_1,η_2,η_3\mathrm{的过渡矩阵为}(\;).\end{array}
$$
$$
A.
\begin{pmatrix}2&-1&1\\0&1&-2\\0&0&1\end{pmatrix} \quad B.\begin{pmatrix}2&1&1\\0&1&2\\0&0&1\end{pmatrix} \quad C.\begin{pmatrix}2&1&0\\0&1&2\\0&0&1\end{pmatrix} \quad D.\begin{pmatrix}2&-1&0\\0&1&-2\\0&0&1\end{pmatrix} \quad E. \quad F. \quad G. \quad H.
$$
$$
\begin{array}{l}设(η_1,η_2,η_3)=(ζ_1,ζ_2,ζ_3)A,\mathrm{则过渡矩阵}\\\;\;\;\;\;\;\;\;\;\;\;\;\;\;\;\;\;\;\;\;\;\;\;\;\;\;\;\;\;\;A=(ζ_1,ζ_2,ζ_3)^{-1}(η_1,η_2,η_3)=\begin{pmatrix}1&-1&1\\0&1&-2\\0&0&1\end{pmatrix}^{-1}\begin{pmatrix}2&-2&4\\0&1&-4\\0&0&1\end{pmatrix}=\begin{pmatrix}2&-1&1\\0&1&-2\\0&0&1\end{pmatrix}\end{array}
$$



$$
\begin{array}{l}\begin{array}{l}设R^3\mathrm{的两组基为}ζ_1=(1,-1,0,0)^T,ζ_2=(0,1,-1,0)^T,ζ_3=(0,0,1,-1)^T,ζ_4=(0,0,0,1)^T;\\\;\;\;\;\;\;\;\;\;\;\;\;\;\;\;\;\;\;\;\;\;\;\;\;\;\;\;\;\;η_1=(1,0,0,0)^T,η_2=(1,2,0,0)^T,η_3=(1,2,3,0)^T,η_4=(1,2,3,4)^T;\end{array}\\\;\;\mathrm{已知向量}α\mathrm{在基}ζ_1ζ_2ζ_3ζ_4\mathrm{下的坐标是}(1,2,3,4),\mathrm{则向量}α\mathrm{在基}\;η_1\;η_2\;η_3\;η_4\mathrm{下的坐标为}().\end{array}
$$
$$
A.
(1/2,1/6,1/12,1/4) \quad B.(1/2,1/6,1/12,1/16) \quad C.(1/2,1/4,1/6,1/12) \quad D.(1/2,1/6,1/12,1/14) \quad E. \quad F. \quad G. \quad H.
$$
$$
\begin{array}{l}\mathrm{由题设知}α=1·ζ_1+2ζ_2+3ζ_3+4ζ_4=(1,1,1,1)^T,设α=x_1η_1+x_2η_2+x_3η_3+x_4η_4.\mathrm{作初等变换},有\\\;\;\;\;\;\;\;\;\;\;\;\;\;\;\;\;\;\;\;\;\;\;\;\;\;\;\;\;\;\;\;\;\;\;\;\;\;\;\;\;\;\;\;\;\;\;(η_1η_2η_3η_4\;α)\rightarrow\begin{pmatrix}1&0&0&0&1/2\\0&1&0&0&1/6\\0&0&1&0&1/12\\0&0&0&1&1/4\end{pmatrix},\;\;\\\mathrm{解得}x_1=1/2,x_2=1/6,x_3=1/12,x_4=1/4.故α 在η_1,η_2,η_3,η_4\mathrm{下的坐标是}(1/2,1/6,1/12,1/4).\end{array}
$$



$$
\begin{array}{l}R^3\mathrm{中的两组基为}\\\;\;\;\;\;\;\;\;\;\;\;\;\;\;\;\;\;\;\;\;\;\;\;ζ_1=(1,0,0)^T,ζ_2=(-1,1,0)^T,ζ_3=(1,-2,1)^T;η_1=(2,0,0)^T,η_2=(-2,1,0)^T,η_3=(4,-4,1)^T.\\\mathrm{已知向量}α=(2,3,-1)^T,\mathrm{则向量}α\mathrm{分别在基}η_1,η_2,η_3和ζ_1,ζ_2,ζ_3\mathrm{下的坐标为}(\;).\end{array}
$$
$$
A.
\begin{pmatrix}2\\-1\\-1\end{pmatrix};\begin{pmatrix}4\\1\\-1\end{pmatrix} \quad B.\begin{pmatrix}2\\1\\-1\end{pmatrix};\begin{pmatrix}4\\1\\-1\end{pmatrix} \quad C.\begin{pmatrix}2\\-1\\-1\end{pmatrix};\begin{pmatrix}4\\-1\\-1\end{pmatrix} \quad D.\begin{pmatrix}2\\-1\\1\end{pmatrix};\begin{pmatrix}4\\-1\\1\end{pmatrix} \quad E. \quad F. \quad G. \quad H.
$$
$$
\begin{array}{l}\begin{array}{l}设(η_1,η_2,η_3)=(ζ_1,ζ_2,ζ_3)A,\mathrm{则过渡矩阵}\\\;\;\;\;\;\;\;\;\;\;\;\;\;\;\;\;\;\;\;\;\;\;\;\;\;\;\;\;\;\;A=(ζ_1,ζ_2,ζ_3)^{-1}(η_1,η_2,η_3)=\begin{pmatrix}1&-1&1\\0&1&-2\\0&0&1\end{pmatrix}^{-1}\begin{pmatrix}2&-2&4\\0&1&-4\\0&0&1\end{pmatrix}=\begin{pmatrix}2&-1&1\\0&1&-2\\0&0&1\end{pmatrix}\end{array}\\设α\mathrm{在基}η_1,η_2,η_3\mathrm{下的坐标为}(y_1,y_2,y_3)则\\\;\;\;\;\;\;\;\;\;\;\;\;\;\;\;\;\;\;\;\;\;\;\;\;\;\;\;\;\;\;\;\;\;\;\;\;\;\;\;\;\;\;\;\;\;\;\;\;\;\;\;\;\;\;\;\;\;\;\;\;\;\;\;\;\;\;\;\;\;\;\;\;\;\;\;\;\;\;\;\;\;\;\;\;y_1η_1+y_2η_2+y_3η_3=α.\\\mathrm{由此},\mathrm{得到线性方程组}\\\;\;\;\;\;\;\;\;\;\;\;\;\;\;\;\;\;\;\;\;\;\;\;\;\;\;\;\;\;\;\;\;\;\;\;\;\;\;\;\;\;\;\;\;\;\;\;\;\;\;\;\;\;\;\;\;\;\;\;\;\;\;\;\;\;\;\;\;\;\;\;\;\;\;\left\{\begin{array}{c}2y_1-2y_2+4y_3=2\\y_2-4y_3=3\\y_3=-1\end{array}\right.,\\\mathrm{解得}y_1=2,y_2=-1,y_3=-1,\mathrm{所以}α\mathrm{在基}η_1,η_2,η_3\mathrm{下的坐标为}(2,-1,-1).\\设α\mathrm{在基}ζ_1,ζ_2,ζ_3\mathrm{下的坐标为}(x_1,x_2,x_3).\mathrm{由坐标变换公式},得\\\;\;\;\;\;\;\;\;\;\;\;\;\;\;\;\;\;\;\;\;\;\;\;\;\;\;\;\;\;\;\;\;\;\;\;\;\;\;\;\;\;\;\;\;\begin{pmatrix}x_1\\x_2\\x_3\end{pmatrix}=A\begin{pmatrix}y_1\\y_2\\y_3\end{pmatrix}=\begin{pmatrix}2&-1&1\\0&1&-2\\0&0&1\end{pmatrix}\begin{pmatrix}2\\-1\\-1\end{pmatrix}=\begin{pmatrix}4\\1\\-1\end{pmatrix}.\\\end{array}
$$



$$
\begin{array}{l}R^3\mathrm{中的两组基为}\\\;\;\;\;\;\;\;\;\;\;\;\;\;\;\;\;\;\;\;\;\;\;\;ζ_1=(1,0,0)^T,ζ_2=(-1,1,0)^T,ζ_3=(1,-2,1)^T;η_1=(2,0,0)^T,η_2=(-2,1,0)^T,η_3=(4,-4,1)^T.\\\mathrm{则在这两组基下有相同坐标的非零向量}(\;).\end{array}
$$
$$
A.
α=k(1,1,0)^T,(k\neq0) \quad B.\alpha=k(1,0,1)^T,(k\neq0) \quad C.α=k(0,1,1)^T,(k\neq0) \quad D.α=k(1,1,1)^T,(k\neq0) \quad E. \quad F. \quad G. \quad H.
$$
$$
\begin{array}{l}\begin{array}{l}设(η_1,η_2,η_3)=(ζ_1,ζ_2,ζ_3)A,\mathrm{则过渡矩阵}\\\;\;\;\;\;\;\;\;\;\;\;\;\;\;\;\;\;\;\;\;\;\;\;\;\;\;\;\;\;\;A=(ζ_1,ζ_2,ζ_3)^{-1}(η_1,η_2,η_3)=\begin{pmatrix}1&-1&1\\0&1&-2\\0&0&1\end{pmatrix}^{-1}\begin{pmatrix}2&-2&4\\0&1&-4\\0&0&1\end{pmatrix}=\begin{pmatrix}2&-1&1\\0&1&-2\\0&0&1\end{pmatrix}\end{array}.\\\mathrm{设所求向量}α\mathrm{的坐标为}(x_1,x_2,x_3),\mathrm{由题设知},\\\;\;\;\;\;\;\;\;\;\;\;\;\;\;\;\;\;\;\;\;\;\;\;\;\;\;\;\;\;α=(ζ_1,ζ_2,ζ_3)\begin{pmatrix}x_1\\x_2\\x_3\end{pmatrix}=(η_1,η_2,η_3)\begin{pmatrix}x_1\\x_2\\x_3\end{pmatrix}=(ζ_1,ζ_2,ζ_3)A\begin{pmatrix}x_1\\x_2\\x_3\end{pmatrix}.\\\mathrm{于是}A\begin{pmatrix}x_1\\x_2\\x_3\end{pmatrix}=\begin{pmatrix}x_1\\x_2\\x_3\end{pmatrix},即(A-E)\begin{pmatrix}x_1\\x_2\\x_3\end{pmatrix}=0.\mathrm{解线性方程组}\begin{pmatrix}1&-1&1\\0&0&-2\\0&0&0\end{pmatrix}\begin{pmatrix}x_1\\x_2\\x_3\end{pmatrix}=0得\\\;\;\;\;\;\;\;\;\;\;\;\;\;\;\;\;\;\;\;\;\;\;\;\;\;\;\;\;α=(x_1,x_2,x_3)^T=k(1,1,0)^T,(k\mathrm{为非零任意常数}).\end{array}
$$



$$
\begin{array}{l}R^3\mathrm{中的两组基为}\\\;\;\;\;\;\;\;\;\;\;\;\;\;\;\;\;\;\;\;\;\;\;\;ζ_1=(1,0,0)^T,ζ_2=(-1,1,0)^T,ζ_3=(1,-2,1)^T;η_1=(2,0,0)^T,η_2=(-2,1,0)^T,η_3=(4,-4,1)^T.\\若α\mathrm{在基}η_1,η_2,η_3\mathrm{下的坐标为}(\;2,-1,1),则α 在ζ_1,ζ_2,ζ_3\mathrm{下的坐标为}(\;).\end{array}
$$
$$
A.
(4,1,-1)^T \quad B.(4,1,1)^T \quad C.(4,-1,1)^T \quad D.(4,-1,-1)^T \quad E. \quad F. \quad G. \quad H.
$$
$$
\begin{array}{l}\begin{array}{l}设(η_1,η_2,η_3)=(ζ_1,ζ_2,ζ_3)A,\mathrm{则过渡矩阵}\\\;\;\;\;\;\;\;\;\;\;\;\;\;\;\;\;\;\;\;\;\;\;\;\;\;\;\;\;\;\;A=(ζ_1,ζ_2,ζ_3)^{-1}(η_1,η_2,η_3)=\begin{pmatrix}1&-1&1\\0&1&-2\\0&0&1\end{pmatrix}^{-1}\begin{pmatrix}2&-2&4\\0&1&-4\\0&0&1\end{pmatrix}=\begin{pmatrix}2&-1&1\\0&1&-2\\0&0&1\end{pmatrix}\end{array}\\设α\mathrm{在基}ζ_1,ζ_2,ζ_3\mathrm{下的坐标为}(x_1,x_2,x_3).\mathrm{由坐标变换公式},得\\\;\;\;\;\;\;\;\;\;\;\;\;\;\;\;\;\;\;\;\;\;\;\;\;\;\;\;\;\;\;\;\;\;\;\;\;\;\;\;\;\;\;\;\;\begin{pmatrix}x_1\\x_2\\x_3\end{pmatrix}=\begin{pmatrix}2&-1&1\\0&1&-2\\0&0&1\end{pmatrix}\begin{pmatrix}2\\-1\\-1\end{pmatrix}=\begin{pmatrix}4\\1\\-1\end{pmatrix}.\\\end{array}
$$



$$
\mathrm{向量}β=(1,1,3)^T\mathrm{在向量空间}R^3\mathrm{的一个基}α_1=(-2,4,1)^T,α_2=(-1,3,5)^T,α_1=(2,-3,1)^T\mathrm{下的坐标为}(\;).
$$
$$
A.
(4,-1,4)^T \quad B.(2,-1,2)^T \quad C.(-4,-1,-4)^T \quad D.(-2,-1,-2)^T \quad E. \quad F. \quad G. \quad H.
$$
$$
\begin{array}{l}\mathrm{令矩阵}A=(α_1,α_2,α_3),对(A\;β)\mathrm{进行初等行变换}:\\\;\;\;\;\;\;\;\;\;\;\;\;\;\;\;\;\;\;\;\;\;\;\;\;\;\;\;\;\;\;\;\;\;\;\;\;(A\;β)=\begin{pmatrix}-2&-1&2&1\\4&3&-3&1\\1&5&1&3\end{pmatrix}\;\\\rightarrow\begin{pmatrix}1&1/2&-1&-1/2\\4&3&-3&1\\1&5&1&3\end{pmatrix}\rightarrow\begin{pmatrix}1&1/2&-1&-1/2\\0&1&1&3\\0&0&1&4\end{pmatrix}\rightarrow\begin{pmatrix}1&0&0&4\\0&1&0&-1\\0&0&1&4\end{pmatrix}\\则β=4α_1-α_2+4α_3.\end{array}
$$



$$
\begin{array}{l}\mathrm{已知}R^3\mathrm{的两个基为}\\\;\;\;\;\;\;\;\;\;\;\;\;\;\;α_1=\begin{pmatrix}1\\1\\1\end{pmatrix},α_2=\begin{pmatrix}1\\0\\-1\end{pmatrix},α_3=\begin{pmatrix}1\\0\\1\end{pmatrix},\\及\;\;\;\;\;\;\;\;\;\;\;β_1=\begin{pmatrix}1\\2\\1\end{pmatrix},β_2=\begin{pmatrix}2\\3\\4\end{pmatrix},β_3=\begin{pmatrix}3\\4\\3\end{pmatrix},\\\mathrm{则由基}α_1,α_2,α_3\mathrm{到基}β_1,β_2,\beta_3\mathrm{的过渡矩阵}P=(\;).\end{array}
$$
$$
A.
\begin{pmatrix}2&3&4\\0&-1&0\\-1&0&-1\end{pmatrix} \quad B.\begin{pmatrix}4&3&6\\0&1&0\\-1&0&-1\end{pmatrix} \quad C.\begin{pmatrix}1&3&0\\0&-1&0\\-1&2&-1\end{pmatrix} \quad D.\begin{pmatrix}2&0&4\\1&-1&1\\-1&0&-1\end{pmatrix} \quad E. \quad F. \quad G. \quad H.
$$
$$
\begin{array}{l}\mathrm{直接由过渡矩阵的定义来解},\mathrm{即矩阵}\;A=(α_1,α_2,α_3),B=(β_1,β_2,β_3).\\因α_1,α_2,α_3与β_1,β_2,β_3\mathrm{均为}R^3\mathrm{中的基},故A和B\mathrm{均为三阶可逆阵},\mathrm{由过渡矩阵定义},\\\;\;\;\;\;\;\;\;\;\;\;\;\;\;\;\;\;\;\;\;\;\;\;\;\;\;\;\;\;\;(β_1,β_2,β_3)=(α_1,α_2,α_3)P或B=AP,\\得\;\;\;\;\;\;\;\;\;\;\;\;\;\;\;\;\;\;\;\;\;\;\;\;\;\;\;\;\;\;\;\;\;\;\;\;\;\;\;\;\;\;\;\;\;\;\;\;\;\;\;\;\;\;\;\;\;\;\;\;\;\;\;\;\;\;\;\;\;P=A^{-1}B.\\\mathrm{利用初等行变换的方法},\mathrm{即可求得}P\mathrm{如下}:\\\;\;\;\;\;\;\;\;\;\;\;\;\;\;\;\;\;\;\;\;\;\;\;\;\;\;\;\;\;\;\;\;\;\;\;\;\;\;\;\;\;\;\;(α_1,α_2,α_3,β_1,β_2,β_3)\\\;\;\;\;\;\;\;\;\;\;\;\;\;\;\;\;\;\;\;\;\;\;\;\;\;\;\;\;\;\;\;\;\;\;\;\;\;\;\;=\begin{pmatrix}1&1&1&1&2&3\\1&0&0&2&3&4\\1&-1&1&1&4&3\end{pmatrix}\xrightarrow{\mathrm{初等行变换}}\begin{pmatrix}1&0&0&2&3&4\\0&1&0&0&-1&0\\0&0&1&-1&0&-1\end{pmatrix},\\\mathrm{从而}\;\;\;\;\;\;\;\;\;\;\;P=A^{-1}B=\begin{pmatrix}2&3&4\\0&-1&0\\-1&0&-1\end{pmatrix}.\end{array}
$$



$$
\begin{array}{l}在R^3\mathrm{中已知向量}α=(4,12,6)\mathrm{在基底}ε_1=(1,0,0),ε_2=(0,1,0),ε_3=(0,0,1)\;\mathrm{上的坐标为}(4,12,6),则α\mathrm{在基底}\\e_1=(-2,1,3),e_2=(-1,0,1),e_3=(-2,-5,-1)\mathrm{上的}\;\mathrm{坐标为}(\;).\end{array}
$$
$$
A.
α=7e_1-16e_2-e_3 \quad B.α=7e_1-16e_2+e_3 \quad C.α=6e_1-10e_2-6e_3 \quad D.α=5e_1-16e_2+e_3 \quad E. \quad F. \quad G. \quad H.
$$
$$
\begin{array}{l}\mathrm{首先通过初等行变换可求得}e_1,e_2,e_3到ε_1,ε_2,ε_3\mathrm{的过渡矩阵}\\\begin{pmatrix}-2&-1&-2&1&0&0\\1&0&-5&0&1&0\\3&1&-1&0&0&1\end{pmatrix}\rightarrow\begin{pmatrix}\frac52&-\frac32&\frac52\\-7&4&-6\\\frac12&-\frac12&\frac12\end{pmatrix},\\α 有e_1,e_2,e_3\mathrm{上的坐标为}\\\;\;\;\;\;\;\;\;\;\;\;\;\;\;\;\;\;\;\;\;\;\;\;\;\;\;\;\;\;\;\;\;\;\;\begin{pmatrix}\frac52&-\frac32&\frac52\\-7&4&-6\\\frac12&-\frac12&\frac12\end{pmatrix}\begin{pmatrix}4\\12\\6\end{pmatrix}=\begin{pmatrix}7\\-16\\-1\end{pmatrix}\\即α=7e_1-16e_2-e_3.\end{array}
$$



$$
\begin{array}{l}\mathrm{已知}ε_1=(1,1,0),ε_2=(0,1,1),ε_3=(1,-1,2)和η_1(1,0,1),η_2(0,1,1),η_3(1,1,4)为R^{3\;}\mathrm{的两个基},若β 在\\ε_1,ε_2,ε_3\mathrm{下的坐标为}(1,1,1),则β 在η_1,η_2,η_3\mathrm{下的坐标为}(\;).\end{array}
$$
$$
A.
\begin{pmatrix}2\\1\\0\end{pmatrix} \quad B.\begin{pmatrix}2\\-1\\0\end{pmatrix} \quad C.\begin{pmatrix}-2\\1\\0\end{pmatrix} \quad D.\begin{pmatrix}-2\\-1\\0\end{pmatrix} \quad E. \quad F. \quad G. \quad H.
$$
$$
\begin{array}{l}\mathrm{先求从}η_1,η_2,η_3到ε_1,ε_2,ε_3\mathrm{的过渡矩阵}P\\\;\;\;\;\;\left(\begin{array}{c}1\\0\\1\end{array}\begin{array}{cccccc}0&1&\vdots&1&0&1\\1&1&\vdots&1&1&-1\\1&4&\vdots&0&1&2\end{array}\right)\rightarrow\;\left(\begin{array}{c}1\\0\\0\end{array}\begin{array}{cccccc}0&1&\vdots&1&0&1\\1&1&\vdots&1&1&-1\\1&3&\vdots&-1&0&1\end{array}\right)\rightarrow\;\left(\begin{array}{c}1\\0\\0\end{array}\begin{array}{cccccc}0&1&\vdots&1&0&1\\1&1&\vdots&1&1&-1\\0&2&\vdots&-2&0&2\end{array}\right)\\\;\;\;\;\;\rightarrow\;\left(\begin{array}{c}1\\0\\0\end{array}\begin{array}{cccccc}0&1&\vdots&1&0&1\\1&1&\vdots&1&1&-1\\0&1&\vdots&-1&0&1\end{array}\right)\rightarrow\;\left(\begin{array}{c}1\\0\\0\end{array}\begin{array}{cccccc}0&0&\vdots&2&0&0\\1&0&\vdots&2&1&-2\\0&1&\vdots&-1&0&1\end{array}\right)\\\mathrm{故过渡矩阵为}P=\begin{pmatrix}2&0&0\\2&1&-2\\-1&0&1\end{pmatrix}.设β 在η_1,η_2,η_3\mathrm{下的坐标为}(y_1,y_2,y_3),则\\\;\;\;\;\;\;\begin{pmatrix}y_1\\y_2\\y_3\end{pmatrix}=P\begin{pmatrix}1\\1\\1\end{pmatrix}=\begin{pmatrix}2&0&0\\2&1&-2\\-1&0&1\end{pmatrix}\begin{pmatrix}1\\1\\1\end{pmatrix}=\begin{pmatrix}2\\1\\0\end{pmatrix}.\\\;\end{array}
$$



$$
\begin{array}{l}在R^3中,\mathrm{由基}α_1=(1,0,0)^T,α_2=(1,1,0)^T,α_3=(1,1,1)^T\mathrm{通过过渡矩阵}A\begin{pmatrix}1&-1&0\\0&1&-1\\0&0&1\end{pmatrix}\mathrm{所得到的新基为}\\β_1,β_2,β_3,则α=-α_1-2α_2+5α_3在β_1,β_2,β_3\mathrm{下的坐标为}(\;).\end{array}
$$
$$
A.
\begin{pmatrix}2\\3\\5\end{pmatrix} \quad B.\begin{pmatrix}2\\3\\-1\end{pmatrix} \quad C.\begin{pmatrix}2\\-3\\5\end{pmatrix} \quad D.\begin{pmatrix}2\\-3\\1\end{pmatrix} \quad E. \quad F. \quad G. \quad H.
$$
$$
\begin{array}{l}(β_1,β_2,β_3)=(α_1,α_2,α_3)A=(α_1,α_2,α_3)\begin{pmatrix}1&-1&0\\0&1&-1\\0&0&1\end{pmatrix}\\\;\;\;\;\;\;\;\;\;\;\;\;\;\;\;\;=(α_1,-α_1+α_2,-α_2+α_3)\\故β_1=(1,0,0)^T,β_2=(-1,1,0)^T,β_1=(0,-1,1)^T.\\设α=-α_1-2α_2+5α_3=(α_1,α_2,α_3)\begin{pmatrix}-1\\-2\\5\end{pmatrix}=(β_1,β_2,β_3)\begin{pmatrix}x_1\\x_2\\x_3\end{pmatrix}\\\mathrm{所以}\begin{pmatrix}x_1\\x_2\\x_3\end{pmatrix}=A^{-1}\begin{pmatrix}-1\\-2\\5\end{pmatrix}=\begin{pmatrix}1&1&1\\0&1&1\\0&0&1\end{pmatrix}\begin{pmatrix}-1\\-2\\5\end{pmatrix}=\begin{pmatrix}2\\3\\5\end{pmatrix}.\end{array}
$$



$$
\begin{array}{l}设R^3\mathrm{的两组基为}(1)α_1=(1,2,-1)^T,α_2=(1,-1,1)^T,α_3=(-1,2,1)^T;\\\;\;\;\;\;\;\;\;\;\;\;\;\;\;\;\;\;\;\;\;\;\;\;\;\;\;\;\;\;(2)β_1=(2,1,0)^T,β_2=(0,1,2)^T,β_3=(-1,2,1)^T;\\则α_1,α_2,α_3到β_1,β_2,β_3\mathrm{的过渡}P为(\;).\end{array}
$$
$$
A.
\begin{pmatrix}1&0&0\\1&1&0\\0&1&1\end{pmatrix} \quad B.\begin{pmatrix}1&0&1\\1&1&0\\0&1&1\end{pmatrix} \quad C.\begin{pmatrix}1&0&1\\-1&1&0\\1&-2&1\end{pmatrix} \quad D.\begin{pmatrix}1&0&-1\\-1&1&1\\1&2&-2\end{pmatrix} \quad E. \quad F. \quad G. \quad H.
$$
$$
\begin{array}{l}\begin{array}{l}B=AP⇒ P=A^{-1}B,\;\;\lbrack A\vdots B\rbrack\xrightarrow{\mathrm{初等行变换}}\lbrack E\vdots A^{-1}B\rbrack,\\令\;\;\;\;\;\;\;\;\;\;\;\;\;\;\;A=(α_1,α_2,α_3)=\begin{pmatrix}1&1&-1\\2&-1&2\\-1&1&1\end{pmatrix}\;\;\;\;\;\;\;\;B=(β_1,β_2,β_3)=\begin{pmatrix}2&0&-1\\1&1&2\\0&2&1\end{pmatrix}\\\lbrack A\vdots B\rbrack=\left(\begin{array}{c}1\\2\\-1\end{array}\begin{array}{cccccc}1&-1&\vdots&2&0&-1\\-1&2&\vdots&1&1&2\\1&1&\vdots&0&2&1\end{array}\right)\rightarrow\left(\begin{array}{c}1\\0\\0\end{array}\begin{array}{cccccc}1&-1&\vdots&2&0&-1\\-3&4&\vdots&-3&1&4\\1&0&\vdots&1&1&0\end{array}\right)\\\rightarrow\left(\begin{array}{c}1\\0\\0\end{array}\begin{array}{cccccc}0&0&\vdots&1&0&0\\1&0&\vdots&1&1&0\\0&1&\vdots&0&1&1\end{array}\right)\\\mathrm{故过渡矩阵}P=\begin{pmatrix}1&0&0\\1&1&0\\0&1&1\end{pmatrix}.\end{array}\\\\\\\end{array}
$$



$$
\begin{array}{l}设R^3\mathrm{的两组基为}(1)α_1=(1,0,1)^T,α_2=(0,1,0)^T,α_3=(1,2,2)^T;\\\;\;\;\;\;\;\;\;\;\;\;\;\;\;\;\;\;\;\;\;\;\;\;\;\;\;\;\;\;(2)β_1=(1,0,0)^T,β_2=(1,1,0)^T,β_3=(1,1,1)^T;\\则α_1,α_2,α_3到β_1,β_2,β_3\mathrm{的过渡}Q为(\;).\end{array}
$$
$$
A.
\begin{pmatrix}2&2&1\\2&3&1\\-1&-1&0\end{pmatrix} \quad B.\begin{pmatrix}2&2&1\\2&3&1\\-1&1&0\end{pmatrix} \quad C.\begin{pmatrix}2&2&1\\2&3&1\\1&1&0\end{pmatrix} \quad D.\begin{pmatrix}2&2&-1\\2&3&-1\\1&1&0\end{pmatrix} \quad E. \quad F. \quad G. \quad H.
$$
$$
\begin{array}{l}\begin{array}{l}B=AP⇒ P=A^{-1}B,\;\;\lbrack A\vdots B\rbrack\xrightarrow{\mathrm{初等行变换}}\lbrack E\vdots A^{-1}B\rbrack,\\令\;\;\;\;\;\;\;\;\;\;\;\;\;\;\;A=(α_1,α_2,α_3)=\begin{pmatrix}1&0&1\\0&1&2\\1&0&2\end{pmatrix}\;\;\;\;\;\;\;\;B=(β_1,β_2,β_3)=\begin{pmatrix}1&1&1\\0&1&1\\0&0&1\end{pmatrix}\\\lbrack A\vdots B\rbrack=\left(\begin{array}{c}1\\0\\1\end{array}\begin{array}{cccccc}0&1&\vdots&1&1&1\\1&2&\vdots&0&1&1\\0&2&\vdots&0&0&1\end{array}\right)\rightarrow\left(\begin{array}{c}1\\0\\0\end{array}\begin{array}{cccccc}0&1&\vdots&1&1&1\\1&2&\vdots&0&1&1\\0&1&\vdots&-1&-1&0\end{array}\right)\\\rightarrow\left(\begin{array}{c}1\\0\\0\end{array}\begin{array}{cccccc}0&0&\vdots&2&2&1\\1&0&\vdots&2&3&1\\0&1&\vdots&-1&-1&0\end{array}\right)\\\mathrm{故过渡矩阵}P=\begin{pmatrix}2&2&1\\2&3&1\\-1&-1&0\end{pmatrix}.\end{array}\\\\\\\end{array}
$$



$$
\begin{array}{l}设R^3\mathrm{的两组基为}(1)α_1=(1,0,1)^T,α_2=(0,1,0)^T,α_3=(0,1,1)^T;\\\;\;\;\;\;\;\;\;\;\;\;\;\;\;\;\;\;\;\;\;\;\;\;\;\;\;\;\;\;(2)β_1=(1,0,0)^T,β_2=(1,1,0)^T,β_3=(1,1,1)^T;\\则α_1,α_2,α_3到β_1,β_2,β_3\mathrm{的过渡}Q为(\;).\end{array}
$$
$$
A.
\begin{pmatrix}1&1&1\\1&2&1\\-1&-1&0\end{pmatrix} \quad B.\begin{pmatrix}1&2&1\\1&3&1\\-1&-1&0\end{pmatrix} \quad C.\begin{pmatrix}1&1&1\\1&3&1\\-1&-1&0\end{pmatrix} \quad D.\begin{pmatrix}1&1&1\\1&2&1\\1&1&0\end{pmatrix} \quad E. \quad F. \quad G. \quad H.
$$
$$
\begin{array}{l}\begin{array}{l}B=AQ⇒ Q=A^{-1}B,\;\;\lbrack A\vdots B\rbrack\xrightarrow{\mathrm{初等行变换}}\lbrack E\vdots A^{-1}B\rbrack,\\令\;\;\;\;\;\;\;\;\;\;\;\;\;\;\;A=(α_1,α_2,α_3)=\begin{pmatrix}1&0&0\\0&1&1\\1&0&1\end{pmatrix}\;\;\;\;\;\;\;\;B=(β_1,β_2,β_3)=\begin{pmatrix}1&1&1\\0&1&1\\0&0&1\end{pmatrix}\\\lbrack A\vdots B\rbrack=\left(\begin{array}{c}1\\0\\1\end{array}\begin{array}{cccccc}0&0&\vdots&1&1&1\\1&1&\vdots&0&1&1\\0&1&\vdots&0&0&1\end{array}\right)\rightarrow\left(\begin{array}{c}1\\0\\0\end{array}\begin{array}{cccccc}0&0&\vdots&1&1&1\\1&1&\vdots&0&1&1\\0&1&\vdots&-1&-1&0\end{array}\right)\\\rightarrow\left(\begin{array}{c}1\\0\\0\end{array}\begin{array}{cccccc}0&0&\vdots&1&1&1\\1&0&\vdots&1&2&1\\0&1&\vdots&-1&-1&0\end{array}\right)\\\mathrm{故过渡矩阵}P=\begin{pmatrix}1&1&1\\1&2&1\\-1&-1&0\end{pmatrix}.\end{array}\\\\\\\end{array}
$$



$$
\begin{array}{l}设R^3\mathrm{中的任一向量}α 在α_1,α_2,α_3\mathrm{下的坐标为}(x_1,x_2,x_3),\mathrm{在基}β_1,β_2,β_3\mathrm{下的坐标为}(y_1,y_2,y_3),\mathrm{且两组坐标有关}\\系:y_1=x_1,y_2=x_2-x_1,y_3=x_3-x_2,R^3\mathrm{的基变换公式为}(β_1,β_2,β_3)=(α_1,α_2,α_3)P,则P=(\;).\end{array}
$$
$$
A.
\begin{pmatrix}1&0&0\\1&1&0\\1&1&1\end{pmatrix} \quad B.\begin{pmatrix}1&0&0\\-1&1&0\\1&-1&1\end{pmatrix} \quad C.\begin{pmatrix}1&0&0\\1&-1&0\\1&1&-1\end{pmatrix} \quad D.\begin{pmatrix}1&1&1\\0&1&1\\0&0&1\end{pmatrix} \quad E. \quad F. \quad G. \quad H.
$$
$$
\begin{array}{l}由y_1=x_1,y_2=x_2-x_1,y_3=x_3-x_2得x_1=y_1,x_2=x_1+y_2=y_1+y_2,x_3=x_2+y_3=y_1+y_2+y_3,\mathrm{从而}\\α=(β_1,β_2,β_3)\begin{pmatrix}y_1\\y_2\\y_3\end{pmatrix}=(α_1,α_2,α_3)\begin{pmatrix}x_1\\x_2\\x_3\end{pmatrix}=(α_1,α_2,α_3)\begin{pmatrix}1&0&0\\1&1&0\\1&1&1\end{pmatrix}\begin{pmatrix}y_1\\y_2\\y_3\end{pmatrix},\mathrm{所以}P=\begin{pmatrix}1&0&0\\1&1&0\\1&1&1\end{pmatrix}.\end{array}
$$



$$
\begin{array}{l}设R^3\mathrm{中的任一向量}α 在α_1,α_2,α_3\mathrm{下的坐标为}(x_1,x_2,x_3),\mathrm{在基}β_1,β_2,β_3\mathrm{下的坐标为}(y_1,y_2,y_3),\mathrm{且两组坐标有关}\\系:y_1=x_1,y_2=x_1-x_2,y_3=x_2-x_3,R^3\mathrm{的基变换公式为}(β_1,β_2,β_3)=(α_1,α_2,α_3)P,则P=(\;).\end{array}
$$
$$
A.
\begin{pmatrix}1&0&0\\1&-1&0\\1&-1&-1\end{pmatrix} \quad B.\begin{pmatrix}1&0&0\\1&1&0\\1&1&1\end{pmatrix} \quad C.\left(-\begin{array}{ccc}1&0&0\\1&-1&0\\-1&-1&-1\end{array}\right) \quad D.\begin{pmatrix}1&0&0\\1&-1&0\\0&1&-1\end{pmatrix} \quad E. \quad F. \quad G. \quad H.
$$
$$
\begin{array}{l}由y_1=x_1,y_2=x_1-x_2,y_3=x_2-x_3得x_1=y_1,x_2=x_1-y_2=y_1-y_2,x_3=x_2-y_3=y_1-y_2-y_3,\mathrm{从而}\\α=(β_1,β_2,β_3)\begin{pmatrix}y_1\\y_2\\y_3\end{pmatrix}=(α_1,α_2,α_3)\begin{pmatrix}x_1\\x_2\\x_3\end{pmatrix}=(α_1,α_2,α_3)\begin{pmatrix}1&0&0\\1&-1&0\\1&-1&-1\end{pmatrix}\begin{pmatrix}y_1\\y_2\\y_3\end{pmatrix},\mathrm{所以}P=\begin{pmatrix}1&0&0\\1&-1&0\\1&-1&-1\end{pmatrix}.\end{array}
$$



$$
\begin{array}{l}设R^3\mathrm{中的任一向量}α 在α_1,α_2,α_3\mathrm{下的坐标为}(x_1,x_2,x_3),\mathrm{在基}β_1,β_2,β_3\mathrm{下的坐标为}(y_1,y_2,y_3),\mathrm{且两组坐标有关}\\系:y_1=x_1,y_2=x_1+x_2,y_3=x_2+x_3,R^3\mathrm{的基变换公式为}(β_1,β_2,β_3)=(α_1,α_2,α_3)P,则P=(\;).\end{array}
$$
$$
A.
\begin{pmatrix}1&0&0\\-1&1&0\\1&-1&1\end{pmatrix} \quad B.\begin{pmatrix}1&0&0\\1&1&0\\1&1&1\end{pmatrix} \quad C.\begin{pmatrix}1&0&0\\-1&1&0\\-1&-1&1\end{pmatrix} \quad D.\begin{pmatrix}1&0&0\\-1&1&0\\0&-1&1\end{pmatrix} \quad E. \quad F. \quad G. \quad H.
$$
$$
\begin{array}{l}\mathrm{解题思路}:由y_1=x_1,y_2=x_1+x_2,y_3=x_2+x_3得x_1=y_1,x_2=y_2-x_1=y_2-y_1,x_3=y_3-x_2=y_3-y_2+y_1,\mathrm{从而}\\α=(β_1,β_2,β_3)\begin{pmatrix}y_1\\y_2\\y_3\end{pmatrix}=(α_1,α_2,\alpha_3)\begin{pmatrix}x_1\\x_2\\x_3\end{pmatrix}=(α_1,α_2,α_3)\begin{pmatrix}1&0&0\\-1&1&0\\1&-1&1\end{pmatrix}\begin{pmatrix}y_1\\y_2\\y_3\end{pmatrix},\mathrm{所以}P=\begin{pmatrix}1&0&0\\-1&1&0\\1&-1&1\end{pmatrix}.\end{array}
$$



$$
\begin{array}{l}设ζ_1,ζ_2,ζ_3是R^3\mathrm{的一组基},且α_1=ζ_1+ζ_2-2ζ_3,α_2=ζ_1-ζ_2-ζ_3,α_3=ζ_1+ζ_3,是R^3\mathrm{的另外一组基},\mathrm{则向量}\\β=6ζ_1-ζ_2-ζ_3\mathrm{在基}α_1,α_2,α_3\mathrm{中的坐标为}\;(\;).\end{array}
$$
$$
A.
(1,2,3) \quad B.(3,2,1) \quad C.(2,1,3) \quad D.(2,4,3) \quad E. \quad F. \quad G. \quad H.
$$
$$
\begin{array}{l}设k_1α_1+k_2α_2+k_3α_3=β,\mathrm{由条件可得到线性方程组}\\\left\{\begin{array}{c}k_1+k_2+k_3=6\\k_1-k_2=-1\\-2k_1-k_2+k_3=-1\end{array}\right.,\\\mathrm{解得}k_1=1,k_2=2,k_3=3.故β\mathrm{关于}α_1,α_2,α_3\mathrm{的坐标是}(1,2,3).\end{array}
$$



$$
设α_1=(1,1,-1)^T,α_2=(-2,-1,2)^T,\mathrm{向量}α=(2,λ,μ)^T与α_1,α_2\mathrm{都正交},则\;λ=\left(\;\;\;\right).
$$
$$
A.
1 \quad B.2 \quad C.0 \quad D.3 \quad E. \quad F. \quad G. \quad H.
$$
$$
\begin{array}{l}\mathrm{根据条件可得}α_1^Tα=0,α_2^Tα=0,即\;\;\;\;\\\;\;\;\;\;\;\;\;\;\;\;\;\;\;\;\;\;\;\;\;\;\;\;\;\;\;\;\;\;\;\;\;\;\left\{\begin{array}{l}2+λ-μ=0\\-4-λ+2μ=0\end{array}\right.⇒λ=0,μ=2.\;\;\;\;\;\;\;\;\;\;\;\;\;\;\;\;\;\;\;\;\;\;\end{array}
$$



$$
设α=(x,0,{\textstyle\frac12})^T,β=(0,2y,0)^T\mathrm{均为单位向量},则x,y\mathrm{分别为}\left(\right)
$$
$$
A.
\textstyle\frac{±\sqrt3}2,±\frac12 \quad B.\textstyle\frac{±\sqrt3}2,\frac12 \quad C.\textstyle\frac{\sqrt3}2,±\frac12 \quad D.\textstyle\frac{\sqrt3}2,\frac12 \quad E. \quad F. \quad G. \quad H.
$$
$$
\begin{array}{l}\mathrm{由于}α,β\mathrm{都为单位向量},\mathrm{则其长度都为}1,即\;\;;\\\;\;\;\;\;\;\;\;\;\;\;\;\;\;\;\;\;\;\;\;\;\;\;\;\;\;\;\;\;\;\;\;\;\;\;\;\;\;\;\;\;\;\;\;\;\;\;\;\;\;\;\;\;\;\;\;\;\;\;\;\;\;\;\;\;x^2+{\textstyle\frac14}=1⇒ x={\textstyle\frac{±\sqrt3}2};\left(2y\right)^2=1⇒ y={\textstyle±}{\textstyle\frac12}\end{array}
$$



$$
\mathrm{设向量}α=\left(-1,a,3,1\right)^T与β=\left(2,-1,1,b\right)^T\mathrm{正交},则a与b\mathrm{的关系式是}\left(\;\right)
$$
$$
A.
a-b=1 \quad B.a+b=1 \quad C.a+b=-1 \quad D.a-b=-1 \quad E. \quad F. \quad G. \quad H.
$$
$$
α,β\mathrm{正交}⇒α^Tβ=0,即-1×2+a×\left(-1\right)+3×1+1× b=0,,则b-a=-1
$$



$$
\mathrm{向量}\begin{pmatrix}2\\1\\0\\3\end{pmatrix}\mathrm{与向量}\begin{pmatrix}1\\-2\\1\\k\end{pmatrix}\mathrm{的内积为}2,则\;k=\left(\;\;\right)
$$
$$
A.
\textstyle\frac23 \quad B.\textstyle\frac13 \quad C.\textstyle-\frac13 \quad D.\textstyle-\frac23 \quad E. \quad F. \quad G. \quad H.
$$
$$
\begin{pmatrix}2&1&0&3\end{pmatrix}\begin{pmatrix}1\\-2\\1\\k\end{pmatrix}=2⇒2-2+3k=2⇒ k={\textstyle\frac23}
$$



$$
\mathrm{设向量}α=(-3,4,-2,4)^T,则\left\|α\right\|=\left(\;\;\;\right)
$$
$$
A.
3\sqrt5 \quad B.5\sqrt3 \quad C.5 \quad D.7 \quad E. \quad F. \quad G. \quad H.
$$
$$
∥α∥=\sqrt{(-3)^2+4^2+(-2)^2+4^2}=\sqrt{45}=3\sqrt5
$$



$$
\mathrm{如果向量}α=(1,-2,2,-1)^T\mathrm{与向量}β=(1,1,k,3)^T\mathrm{正交},则\;k=()
$$
$$
A.
2 \quad B.0 \quad C.1 \quad D.3 \quad E. \quad F. \quad G. \quad H.
$$
$$
\begin{array}{l}\mathrm{由条件可知}\alpha^Tβ=0,即\\\;\;\;\;\;\;\;\;\;\;\;\;\;\;\;\;\;\;\;\;\;\;\;\;\;\;\;\;\;\;\;\;\;\;\;\;\;\;\;\;\;\;\;\;\;\;\;\;\;\;\;1×1+(-2)×1+2× k+(-1)×3=0⇒ k=2\end{array}
$$



$$
\mathrm{设向量}α=\left(1,a,b\right)^{\;T}\mathrm{与向量}α_1=\left(2,2,2\right)^{\;\;T},α_2=\left(3,1,3\right)^{\;T}\mathrm{都正交},则a,b\mathrm{分别为}\left(\;\;\;\right)
$$
$$
A.
a=0,b=-1 \quad B.a=0,b=1 \quad C.a=1,b=0 \quad D.a=-1,b=0 \quad E. \quad F. \quad G. \quad H.
$$
$$
\begin{array}{l}\\\begin{array}{l}\mathrm{由条件可知}\\\;\;\;\;\;\;\;\;\;\;\;\;\;\;\;\;\;\;\;\;\;\;\;\;\;\;\;\;\;\;\;\;\;\;\;\;\;\;\;\;\;\;\;\;\;\left\{\begin{array}{l}\left\langleα,α_1\right\rangle=2+2a+2b=0\\\left\langleα,α_2\right\rangle=3+a+3b=0\end{array}\right.⇒\left\{\begin{array}{l}a=0\\b=-1\end{array}\right.\end{array}\end{array}
$$



$$
\mathrm{向量}α_1=(1,5,k,-1)^T,α_2=(2k,3,-2,k)^T\mathrm{正交},则k为(\;).\;
$$
$$
A.
15 \quad B.-15 \quad C.5 \quad D.-5 \quad E. \quad F. \quad G. \quad H.
$$
$$
由\left\langleα_1,α_2\right\rangle=0,得2k+15-2k-k=0,k=15
$$



$$
设α_1=(2k,k-1,0,3)^T,α_2=(5,-3,k,k+1)^T\mathrm{正交},则k为(\;).\;
$$
$$
A.
k=\frac35 \quad B.k=-\frac35 \quad C.k=-\frac15 \quad D.k=\frac15 \quad E. \quad F. \quad G. \quad H.
$$
$$
由\left\langleα_1,α_2\right\rangle=0,得\;10k-3k+3+3k+3=0,k=-\frac35
$$



$$
\mathrm{若向量}α=(-1,0,3,1)^T与β=(2,-1,1,b)^T\mathrm{正交},则b=\left(\;\;\;\right)\;
$$
$$
A.
3 \quad B.-3 \quad C.1 \quad D.-1 \quad E. \quad F. \quad G. \quad H.
$$
$$
\mathrm{根据定义有}:\;-1×2+0×(-1)+3×1+1× b=0⇒ b=-1
$$



$$
\mathrm{若向量}α=(-1,0,-1,1)^T与β=(2,-1,1,b)^T\mathrm{内积为}3,则b=\left(\;\;\;\right)\;
$$
$$
A.
6 \quad B.-6 \quad C.3 \quad D.-3 \quad E. \quad F. \quad G. \quad H.
$$
$$
(-1)×2+0×(-1)+(-1)×1+1× b=3⇒ b=6
$$



$$
设α_1=(1,0,-1)^T,α_2=(1,2,-2)^T,\;\;则α_1,α_2\mathrm{的夹角为}(\;\;\;\;)\;
$$
$$
A.
\textstyleθ=-{\displaystyle\fracπ6} \quad B.\textstyleθ={\displaystyle\fracπ6} \quad C.\textstyleθ=-{\displaystyle\fracπ4} \quad D.\textstyleθ={\displaystyle\fracπ4} \quad E. \quad F. \quad G. \quad H.
$$
$$
\textstyle\cosθ=\frac3{\sqrt2×3}=\frac{\sqrt2}2,\mathrm{所以}θ={\displaystyle\fracπ4}
$$



$$
设α_1=(1,1,-1)^T,α_2=(-2,-1,2)^{T,}则α_1,α_2\mathrm{夹角的余弦值为}(\;\;\;\;)
$$
$$
A.
\textstyle\frac5{3\sqrt3} \quad B.\textstyle\frac{-5}{3\sqrt3} \quad C.\textstyle\frac2{3\sqrt3} \quad D.\textstyle\frac{-2}{3\sqrt3}\;\; \quad E. \quad F. \quad G. \quad H.
$$
$$
\boldsymbol\;\boldsymbol\;\mathbf{\textstyle\cos}\mathbf{\textstyleθ}\mathbf{\textstyle=}\mathbf{\textstyle\frac{-2-1-2}{3\sqrt3}}\mathbf{\textstyle=}\mathbf{\textstyle\frac{-5}{3\sqrt3}}
$$



$$
在R^n\mathrm{空间中},\mathrm{向量}a\mathrm{与任意向量}β\mathrm{的内积都等于零的充分必要条件是}\left\|a\right\|=\left(\;\;\right)
$$
$$
A.
0 \quad B.1 \quad C.-1 \quad D.2 \quad E. \quad F. \quad G. \quad H.
$$
$$
\mathrm{由题设可知},\mathrm{向量}a\mathrm{与其本身的内积也等于零},\mathrm{即向量}a\mathrm{的长度等于零},则a=0;\;\;\mathrm{反之},若a=0,\mathrm{则内积}\left\langle a,a\right\rangle=0,\mathrm{因此构成充要条件}.
$$



$$
\;\mathrm{与向量}α_1=(2,2,2)^T,α_2=(3,1,3)^T\mathrm{都正交的一个向量}α=(1,λ,μ)^T,则μ=(\;\;).
$$
$$
A.
0 \quad B.1 \quad C.-1 \quad D.2 \quad E. \quad F. \quad G. \quad H.
$$
$$
\begin{array}{l}\mathrm{由题设可知}\\\;\;\;\;\;\;\;\;\;\;\;\;\;\;\;\;\;\;\;\;\;\;\;\;\left\{\begin{array}{l}2+2λ+2μ=0\\3+λ+3μ=0\end{array}\right.⇒\left\{\begin{array}{l}λ=0\\μ=-1\end{array}\right.\end{array}
$$



$$
\textstyle\boldsymbol 与{\boldsymbolα}_\mathbf1\boldsymbol=\boldsymbol(\mathbf1\boldsymbol,\mathbf2\boldsymbol,\boldsymbol-\mathbf1\boldsymbol)^\mathbf T\boldsymbol,{\boldsymbolα}_\mathbf2\boldsymbol=\boldsymbol(\mathbf4\boldsymbol,\mathbf0\boldsymbol,\mathbf2\boldsymbol)^\mathbf T\mathbf{都正交的所有向量}\boldsymbolβ\boldsymbol=\left(\boldsymbol\;\boldsymbol\;\right)^{}
$$
$$
A.
\textstyle\boldsymbol(\mathbf1\boldsymbol,\boldsymbol-\mathbf3\boldsymbol,\boldsymbol-\mathbf5\boldsymbol)^\mathbf T\boldsymbol\; \quad B.\textstyle\boldsymbol\;\boldsymbol k\boldsymbol(\mathbf1\boldsymbol,\boldsymbol-\mathbf3\boldsymbol,\boldsymbol-\mathbf5\boldsymbol)^\mathbf T\boldsymbol k \quad C.\textstyle\boldsymbol(\mathbf2\boldsymbol,\boldsymbol-\mathbf3\boldsymbol,\boldsymbol-\mathbf4\boldsymbol)^\mathbf T \quad D.\textstyle\boldsymbol k\boldsymbol(\mathbf2\boldsymbol,\boldsymbol-\mathbf3\boldsymbol,\boldsymbol-\mathbf4\boldsymbol)^\mathbf T \quad E. \quad F. \quad G. \quad H.
$$
$$
\begin{array}{l}设β=(x_1,x_2,x_3),\mathrm{则由}α_1^Tβ=0,α_2^Tβ=0\mathrm{可得}\\\left\{\begin{array}{l}x_1+2x_2-x_3=0\\4x_1+2x_3=0\end{array}\right.⇒β=k(2,-3,-4)^T\end{array}
$$



$$
\textstyle\boldsymbol 设{\boldsymbolα}_\mathbf1\boldsymbol=\boldsymbol(\mathbf1\boldsymbol,\mathbf1\boldsymbol,\boldsymbol-\mathbf1\boldsymbol)^\mathbf T\boldsymbol,{\boldsymbolα}_\mathbf2\boldsymbol=\boldsymbol(\mathbf0\boldsymbol,\boldsymbol-\mathbf1\boldsymbol,\mathbf2\boldsymbol)^\mathbf T\boldsymbol,\mathbf{向量}\boldsymbolα\boldsymbol=\boldsymbol(\mathbf2\boldsymbol,\boldsymbolλ\boldsymbol,\boldsymbolμ\boldsymbol)^\mathbf T\boldsymbol 与{\boldsymbolα}_\mathbf1\boldsymbol,{\boldsymbolα}_\mathbf2\mathbf{都正交}\boldsymbol,\mathbf{则有}\boldsymbol\;\left(\boldsymbol\;\right)
$$
$$
A.
\textstyle\boldsymbolλ\boldsymbol=\mathbf4\boldsymbol,\boldsymbolμ\boldsymbol=\boldsymbol-\mathbf2\boldsymbol\; \quad B.\textstyle\boldsymbolλ\boldsymbol=\boldsymbol-\mathbf4\boldsymbol,\boldsymbolμ\boldsymbol=\mathbf2 \quad C.\textstyle\boldsymbol\;\boldsymbolλ\boldsymbol=\mathbf4\boldsymbol,\boldsymbolμ\boldsymbol=\mathbf2 \quad D.\textstyle\boldsymbolλ\boldsymbol=\boldsymbol-\mathbf4\boldsymbol,\boldsymbolμ\boldsymbol=\boldsymbol-\mathbf2 \quad E. \quad F. \quad G. \quad H.
$$
$$
\begin{array}{l}\\\textstyle\begin{array}{l}\mathbf{根据题意知}\boldsymbol:\left\{\begin{array}{l}\mathbf2\boldsymbol+\mathbfλ\boldsymbol-\mathbfμ\boldsymbol=\mathbf0\\\boldsymbol-\mathbfλ\boldsymbol+\mathbf2\mathbfμ\boldsymbol=\mathbf0\end{array}\right.\boldsymbol,\mathbf{所以λ}\boldsymbol=\boldsymbol-\mathbf4\boldsymbol,\mathbfμ\boldsymbol=\boldsymbol-\mathbf2\\\end{array}\end{array}
$$



$$
\boldsymbol 设\boldsymbolα\boldsymbol=\boldsymbol(\mathbf0\boldsymbol,\boldsymbol y\boldsymbol,\boldsymbol-{\textstyle\frac{\mathbf1}{\sqrt{\mathbf2}}}\boldsymbol)^\mathbf T\boldsymbol,\boldsymbolβ\boldsymbol=\boldsymbol(\boldsymbol x\boldsymbol,\mathbf0\boldsymbol,\mathbf0\boldsymbol)^\mathbf T\boldsymbol,\boldsymbol 若\boldsymbolα\boldsymbol,\boldsymbolβ\mathbf{为标准正交向量组}\boldsymbol,\boldsymbol 则\boldsymbol x\boldsymbol 和\boldsymbol y\mathbf{分别为}\boldsymbol(\boldsymbol\;\boldsymbol\;\boldsymbol)\boldsymbol.
$$
$$
A.
\boldsymbol x\boldsymbol=\mathbf1\boldsymbol,\boldsymbol y\boldsymbol={\textstyle\frac{\sqrt{\mathbf2}}{\mathbf2}}\boldsymbol\; \quad B.\boldsymbol\;\boldsymbol x\boldsymbol=\boldsymbol±\mathbf1\boldsymbol,\boldsymbol y\boldsymbol=\boldsymbol±{\textstyle\frac{\sqrt{\mathbf2}}{\mathbf2}} \quad C.\boldsymbol\;\boldsymbol x\boldsymbol=\boldsymbol-\mathbf1\boldsymbol,\boldsymbol y\boldsymbol=\boldsymbol-{\textstyle\frac{\sqrt{\mathbf2}}{\mathbf2}} \quad D.\boldsymbol x\boldsymbol=\mathbf1\boldsymbol,\boldsymbol y\boldsymbol=\boldsymbol-{\textstyle\frac{\sqrt{\mathbf2}}{\mathbf2}} \quad E. \quad F. \quad G. \quad H.
$$
$$
\mathbf{由已知}\boldsymbol y^\mathbf2\boldsymbol+\frac{\mathbf1}{\mathbf2}\boldsymbol=\mathbf1\boldsymbol,\boldsymbol 则\boldsymbol y\boldsymbol=\boldsymbol±\frac{\sqrt{\mathbf2}}{\mathbf2}\boldsymbol,\boldsymbol x^\mathbf2\boldsymbol=\mathbf1\boldsymbol,\boldsymbol x\boldsymbol=\boldsymbol±\mathbf1\boldsymbol,\boldsymbol 由\boldsymbolα^\mathbf T\boldsymbolβ\boldsymbol=\mathbf0\boldsymbol 知\boldsymbol x\boldsymbol,\boldsymbol y\mathbf{为任意值}\boldsymbol,\mathbf{综上得}\boldsymbol x\boldsymbol=\boldsymbol±\mathbf1\boldsymbol,\boldsymbol y\boldsymbol=\boldsymbol±\frac{\sqrt{\mathbf2}}{\mathbf2}
$$



$$
\boldsymbol 设\boldsymbolα\boldsymbol,\boldsymbolβ\mathbf{都是}\boldsymbol n\mathbf{维的单位列向量}\boldsymbol,\boldsymbol 则\boldsymbolα\boldsymbol-\boldsymbolβ\boldsymbol 与\boldsymbolα\boldsymbol+\boldsymbolβ\mathbf{的内积为}\left(\boldsymbol\;\boldsymbol\;\right)
$$
$$
A.
1 \quad B.0 \quad C.-1 \quad D.2 \quad E. \quad F. \quad G. \quad H.
$$
$$
\begin{array}{l}\left\langle\left(\mathbfα\boldsymbol-\mathbfβ\right)\boldsymbol,\left(\mathbfα\boldsymbol+\mathbfβ\right)\right\rangle\boldsymbol=\left\langle\mathbfα\boldsymbol,\mathbfα\right\rangle\boldsymbol-\left\langle\mathbfβ\boldsymbol,\mathbfα\right\rangle\boldsymbol+\left\langle\mathbfα\boldsymbol,\mathbfβ\right\rangle\boldsymbol-\left\langle\mathbfβ\boldsymbol,\mathbfβ\right\rangle\\\boldsymbol\;\mathbf 由\left\langle\mathbfα\boldsymbol,\mathbfβ\right\rangle\boldsymbol=\left\langle\mathbfβ\boldsymbol,\mathbfα\right\rangle\mathbf 知\boldsymbol,\mathbf{故所求的内积为}\boldsymbol:\\\boldsymbol\;\boldsymbol\;\boldsymbol\;\boldsymbol\;\boldsymbol\;\boldsymbol\;\boldsymbol\;\boldsymbol\;\boldsymbol\;\boldsymbol\;\boldsymbol\;\boldsymbol\;\boldsymbol\;\boldsymbol\;\boldsymbol\;\boldsymbol\;\boldsymbol\;\boldsymbol\;\boldsymbol\;\boldsymbol\;\boldsymbol\;\boldsymbol\;\boldsymbol\;\boldsymbol\;\boldsymbol\;\boldsymbol\;\boldsymbol\;\boldsymbol\;\boldsymbol\;\boldsymbol\;\boldsymbol\;\boldsymbol\;\boldsymbol\;\boldsymbol\;\boldsymbol\;\boldsymbol\;\boldsymbol\;\boldsymbol\;\boldsymbol\;\boldsymbol\;\boldsymbol\;\boldsymbol\;\boldsymbol\;\boldsymbol\;\boldsymbol\;\boldsymbol\;\boldsymbol\;\boldsymbol\;\boldsymbol\;\boldsymbol\;\boldsymbol\;\boldsymbol\;\boldsymbol\;\boldsymbol\;\boldsymbol\;\boldsymbol\;\boldsymbol\;\boldsymbol\;\boldsymbol\;\boldsymbol\;\boldsymbol\;\boldsymbol\;\boldsymbol\;\boldsymbol\;\left\langle\mathbfα\boldsymbol,\mathbfα\right\rangle\boldsymbol-\left\langle\mathbfβ\boldsymbol,\mathbfβ\right\rangle\boldsymbol=\left\|\mathbfα\right\|^\mathbf2\boldsymbol-\left\|\mathbfβ\right\|^\mathbf2\boldsymbol=\mathbf1\boldsymbol-\mathbf1\boldsymbol=\mathbf0\end{array}
$$



$$
\mathbf{向量}\boldsymbolα\boldsymbol=\begin{pmatrix}\mathbf1\\\mathbf0\\\boldsymbol-\mathbf1\\\mathbf0\end{pmatrix}\boldsymbol,\boldsymbolβ\boldsymbol=\begin{pmatrix}\mathbf0\\\mathbf1\\\mathbf0\\\mathbf2\end{pmatrix}\mathbf{之间的夹角为}\left(\boldsymbol\;\boldsymbol\;\boldsymbol\;\right)
$$
$$
A.
\mathbf{90}^\boldsymbol. \quad B.\boldsymbol\;\boldsymbol\;\mathbf0^\boldsymbol. \quad C.\boldsymbol\;\mathbf{60}^\boldsymbol. \quad D.\mathbf{180}^\boldsymbol. \quad E. \quad F. \quad G. \quad H.
$$
$$
\begin{array}{l}\boldsymbol\;\boldsymbol\;\boldsymbol\;\boldsymbol\;\boldsymbol\;\boldsymbol\;\boldsymbol\;\boldsymbol\;\boldsymbol\;\boldsymbol\;\boldsymbol\;\boldsymbol\;\boldsymbol\;\boldsymbol\;\boldsymbol\;\boldsymbol\;\boldsymbol\;\boldsymbol\;\boldsymbol\;\boldsymbol\;\boldsymbol\;\boldsymbol\;\boldsymbol\;\boldsymbol\;\boldsymbol\;\boldsymbol\;\boldsymbol\;\boldsymbol\;\left\|\mathbfα\right\|\boldsymbol=\sqrt{\mathbf1^\mathbf2\boldsymbol+\mathbf0^\mathbf2\boldsymbol+\left(\boldsymbol-\mathbf1\right)^\mathbf2\boldsymbol+\mathbf0^\mathbf2}\boldsymbol=\sqrt{\mathbf2}\boldsymbol,\\\boldsymbol\;\mathbf{同理}\boldsymbol\;\left\|\mathbfβ\right\|\boldsymbol=\sqrt{\mathbf5}\boldsymbol,\boldsymbol\;\boldsymbol\;\\\mathbf 又\left[\mathbfα\boldsymbol,\mathbfβ\right]\boldsymbol\;\boldsymbol=\mathbf1\boldsymbol·\mathbf0\boldsymbol+\mathbf0\boldsymbol·\mathbf1\boldsymbol+\left(\boldsymbol-\mathbf1\right)\boldsymbol·\mathbf0\boldsymbol+\mathbf0\boldsymbol·\mathbf2\boldsymbol=\mathbf0\boldsymbol,\boldsymbol\;\\\boldsymbol\;\mathbf{所以}\boldsymbol\;\mathbf{cosθ}\boldsymbol={\textstyle\frac{\left\langle\mathbfα\boldsymbol,\mathbfβ\right\rangle}{\left\|\mathbfα\right\|\left\|\mathbfβ\right\|}}\boldsymbol=\mathbf0\boldsymbol⇒\mathbfθ\boldsymbol=\mathbf{90}^\boldsymbol.\end{array}
$$



$$
在R^3\mathrm{中与向量}α=(1,1,1)^T\mathrm{正交的全体向量可表示为}()
$$
$$
A.
V=\left\{(-k_1-k_2,k_1,k_2)^T│k_1,k_2∈ R\right\} \quad B.V=\left\{(k_1+k_2,k_1,k_2)^T│k_1,k_2∈ R\right\} \quad C.V=\left\{(k_1-k_2,k_1,k_2)^T│k_1,k_2∈ R\right\} \quad D.V=\left\{(k_2-k_1,k_1,k_2)^T│k_1,k_2∈ R\right\} \quad E. \quad F. \quad G. \quad H.
$$
$$
\begin{array}{l}设β=(b_1,b_2,b_3)^T与α\mathrm{正交}⇒\left\langleα,β\right\rangle=b_1+b_2+b_3.=0\;\;\\令\;\;b_2=k_1,b_3=k_2⇒ b_1=-k_1-k_2\\\mathrm{于是与}α=(1,1,1)^T\mathrm{正交的全体向量为}\;\;\\\;\;\;\;\;\;\;\;\;\;\;\;\;\;\;\;\;\;\;\;\;V=\left\{(-k_1-k_2,k_1,k_2)^T│k_1,k_2∈ R\right\}\end{array}
$$



$$
\begin{array}{l}\\\begin{array}{l}\mathrm{设向量}α_1=(-1,1,2,-1)^T,{\boldsymbolα}_\mathbf2\boldsymbol=\boldsymbol(\mathbf0\boldsymbol,\mathbf3\boldsymbol,\mathbf8\boldsymbol,\boldsymbol-\mathbf2\boldsymbol)^\mathbf T\boldsymbol,{\boldsymbolα}_\mathbf3\boldsymbol=\boldsymbol(\mathbf3\boldsymbol,\mathbf1\boldsymbol,\mathbf2\boldsymbol,\mathbf2\boldsymbol)^\mathbf T\boldsymbol,则\\\left\|3α_1-α_2+α_3\right\|=\left(\;\;\right)\end{array}\end{array}
$$
$$
A.
\textstyle\sqrt2 \quad B.\textstyle2 \quad C.\textstyle\sqrt{10} \quad D.1 \quad E. \quad F. \quad G. \quad H.
$$
$$
\begin{array}{l}3α_1-α_2+α_3=3(-1,1,2,-1)^T-(0,3,8,-2)^T+(3,1,2,2)^T=(0,1,0,1)^T\\\;则\left\|3α_1-α_2+α_3\right\|=\sqrt{1^2+1^2}=\sqrt2.\end{array}
$$



$$
设α_1,α_2∈ R^n,\left\|α_1\right\|=\left\|α_2\right\|=1,\left\langleα_1,α_2\right\rangle={\textstyle\frac14},则\left\|α_1+α_2\right\|=\left(\;\;\right)
$$
$$
A.
\frac{\sqrt{10}}2 \quad B.\frac{\sqrt5}2 \quad C.\frac52 \quad D.5 \quad E. \quad F. \quad G. \quad H.
$$
$$
\begin{array}{l}\left\|α_1+α_2\right\|^2=\left\langleα_1+α_2,α_1+α_2\right\rangle=\left\langleα_1,α_1\right\rangle+2\left\langleα_1,α_2\right\rangle+\left\langleα_2,α_2\right\rangle\\=\left\|α_1\right\|^2+2\left\langleα_1,α_2\right\rangle+\left\|α_2\right\|^2=1+2×{\textstyle\frac14}+1={\textstyle\frac52}\\故\left\|α_1+α_2\right\|={\textstyle\frac{\sqrt{10}}2}\end{array}
$$



$$
\mathrm{与向量}α_1=(1,1,-1,1)^T,α_2=(1,-1,1,1)^T,α_3=(1,1,1,1)^T\mathrm{都正交的单位向量为}().\;
$$
$$
A.
\textstyle±\frac1{\sqrt2}(1,0,0,-1)^T\; \quad B.\textstyle\;\frac1{\sqrt2}(1,0,0,-1)^T\; \quad C.\textstyle±\frac1{\sqrt5}(2,0,0,1)^T \quad D.\textstyle\frac1{\sqrt5}(2,0,0,1)^T \quad E. \quad F. \quad G. \quad H.
$$
$$
\begin{array}{l}\mathrm{设向量}α=\left(x{}_1,x_2,x_3,x_4\right)与α_1,α_2,α_3\mathrm{都正交},则α\mathrm{应满足方程}\\\;\;\;\;\;\;\;\;\;\;\;\;\;\;\;\;\;\;\;\;\;\;\;\;\;\;\;\;\;\;\;\;\;\;\;\;\;\;\;\;\;\;\;\;\;\;\;\;\;\;\;\;\;α\;_i^Tα=0\;\;\;\;\left(i=1,2,3\right)\;,\;\;\;\;\;\;\;\;\;\;\;\;\;\;\;\;\;\;\;\;\;\;\;\\\mathrm{即满足方程组}\;\;\left\{\begin{array}{l}\begin{array}{c}x_1+x_2-x_3+x_4=0\\x_1-x_2+x_3+x_4=0\end{array}\\\begin{array}{c}x_1+x_2+x_3+x_4\end{array}=0\end{array}\right.\\\mathrm{它的基础解系为}\;ζ=±1·(1,0,0,-1)^T\;,\\\mathrm{单位化},\mathrm{得向量}±{\textstyle\frac1{\sqrt2}}(1,0,0,-1)^T,\;\mathrm{即为所求}.\end{array}
$$



$$
\boldsymbol\;\boldsymbol 设{\boldsymbolα}_\mathbf1\boldsymbol=\boldsymbol(\mathbf2\boldsymbol,\boldsymbol-\mathbf2\boldsymbol,\mathbf1\boldsymbol)^\mathbf T\boldsymbol,{\boldsymbol\alpha}_\mathbf2\boldsymbol=\boldsymbol(\mathbf0\boldsymbol,\mathbf1\boldsymbol,\boldsymbol-\mathbf1\boldsymbol)^\mathbf T\boldsymbol,\boldsymbol 且\boldsymbolβ\boldsymbol={\boldsymbol k}_\mathbf1{\boldsymbolα}_\mathbf1\boldsymbol+{\boldsymbol k}_\mathbf2{\boldsymbolα}_\mathbf2\mathbf{是与}{\boldsymbolα}_\mathbf1\mathbf{正交的单位向量}\boldsymbol,\boldsymbol 则\boldsymbolβ\boldsymbol 为\boldsymbol(\boldsymbol\;\boldsymbol\;\boldsymbol\;\boldsymbol)\boldsymbol.
$$
$$
A.
\textstyle\frac{\mathbf1}{\mathbf3}\begin{pmatrix}\mathbf2\\\mathbf1\\\boldsymbol-\mathbf2\end{pmatrix}\boldsymbol\;\boldsymbol\; \quad B.\textstyle\boldsymbol±\frac{\mathbf1}{\mathbf3}\begin{pmatrix}\mathbf2\\\mathbf1\\\boldsymbol-\mathbf2\end{pmatrix}\boldsymbol\; \quad C.\textstyle\mathbf3\begin{pmatrix}\mathbf2\\\mathbf1\\\boldsymbol-\mathbf2\end{pmatrix} \quad D.\textstyle\boldsymbol\;\boldsymbol±\mathbf3\begin{pmatrix}\mathbf2\\\mathbf1\\\boldsymbol-\mathbf2\end{pmatrix} \quad E. \quad F. \quad G. \quad H.
$$
$$
\begin{array}{l}由\;\;\\α_1^Tβ=k_1α_1^Tα_1+k_2α_1^Tα_2=0,\;\;\\得9k_1-3k_2=0,k_2-3k_1=0.令k_1=t故\;\;\\β=t\begin{pmatrix}2\\-2\\1\end{pmatrix}+3t\begin{pmatrix}0\\1\\-1\end{pmatrix}=t\begin{pmatrix}2\\1\\-2\end{pmatrix}.\;\;\\且\left\|β\right\|^2=t^2\lbrack2^2+1^2+(-2)^2\rbrack=9t^2=1,\;\;\\故t=±\frac13.\mathrm{故得}k_1=\frac13,k_2=1或k_1=-\frac13,k_2=-1\mathrm{为所求}.故\;\;\;\\β=\frac13\begin{pmatrix}2\\-2\\1\end{pmatrix}+\begin{pmatrix}0\\1\\-1\end{pmatrix}=\frac13\begin{pmatrix}2\\1\\-2\end{pmatrix}\\或\\β=-\frac13\begin{pmatrix}2\\-2\\1\end{pmatrix}-\begin{pmatrix}0\\1\\-1\end{pmatrix}=-\frac13\begin{pmatrix}2\\1\\-2\end{pmatrix}.\\\end{array}
$$



$$
\mathrm{已知}α_1=\left[1,1,1\right]^T,\mathrm{若存在向量}α_2,α_3,使α_1,α_2,α_3\mathrm{正交},则(\;\;).
$$
$$
A.
α_2=\left[-1,1,0\right]^T,α_3=\left[-1,0,1\right]^T \quad B.α_2=\left[-1,1,0\right]^T,α_3=\left[-\frac12,-\frac12,1\right]^T \quad C.α_2=\left[1,1,0\right]^T,α_3=\left[-1,0,1\right]^T \quad D.α_2=\left[-1,1,0\right]^T,α_3=\left[\frac12,\frac12,1\right]^T \quad E. \quad F. \quad G. \quad H.
$$
$$
\begin{array}{l}\mathrm{由题设可知}α_2,α_3\mathrm{应满足方程}α_1^Tx=0,设x=\left[x_1,x_2,x_3\right],即x_1+x_2+x_3=0,\mathrm{它的基础解系含向量的个数为}\\3-1=2,\mathrm{其解为}\;\\\;\;\;\;\;\;\;\;\;\;\;\;\;\;\;\;\;\;\;\;\;\;\;\;\;\;\;\;\;\;\;\;\;\;\;\;\;\;\;\;\;\begin{pmatrix}x_1\\x_2\\x_3\end{pmatrix}=c_1\begin{pmatrix}-1\\1\\0\end{pmatrix}+c_2\begin{pmatrix}-1\\0\\1\end{pmatrix},\;\;\\\mathrm{其中均为}c_1,c_2\mathrm{常数},\mathrm{基础解系为}\;\\\;\;\;\;\;\;\;\;\;\;\;\;\;\;\;\;\;\;\;\;\;\;\;\;\;\;\;\;\;\;\;\;\;\;\;\;\;\;\;\;\;\;\;\;\;\;\;\;\;\;\;\;\;\;\;\;\;\;\;\;\;\;\;\;\;\;\;ξ_1=\left[-1,1,0\right]^T\;,\;ξ_2=\left[-1,0,1\right]^T,\;\;\\\mathrm{可验证}ξ_1,ξ_2与α_1\mathrm{正交},\mathrm{只要将}ξ_1,ξ_2\mathrm{正交化就可以了}.令α_2=ξ_1=\left[-1,1,0\right]^T,则\;\;\\\;\;\;\;\;\;\;\;\;\;\;\;\;\;\;\;\;\;\;\;\;\;\;\;\;\;\;\;\;\;\;\;\;\;\;\;\;\;\;\;\;\;\;\;\;\;\;\;\;\;\;\;\;\;\;\;\;\;\;\;\;\;\;\;\;\;\;\;α_3\;=ξ_2-\frac{\left\langleξ_1,ξ_2\right\rangle}{\left\langleξ_1,ξ_1\right\rangle}ξ_1=\begin{pmatrix}-1\\0\\1\end{pmatrix}-{\textstyle\frac12}\begin{pmatrix}-1\\1\\0\end{pmatrix}=\begin{pmatrix}-{\textstyle\frac12}\\-{\textstyle\frac12}\\1\end{pmatrix},\;\;\\\mathrm{求得}α_2=\begin{pmatrix}-1\\1\\0\end{pmatrix},α_3=\begin{pmatrix}-{\textstyle\frac12}\\-{\textstyle\frac12}\\1\end{pmatrix},\mathrm{使得}α_1,α_2,α_3\mathrm{两两正交}.\end{array}
$$



$$
设α=(1,0,1)^T,β=(1,-1,0)^T,则α×β=\left(\;\;\;\;\right)
$$
$$
A.
(1,0,-1)^T \quad B.(1,1,-1)^T \quad C.(1,-1,-1)^T \quad D.(0,1,-1)^T \quad E. \quad F. \quad G. \quad H.
$$
$$
α×β=\begin{vmatrix}i&j&κ\\1&0&1\\1&-1&0\end{vmatrix}=i+j-k=(1,1,-1)^T
$$



$$
设α=(2,1,-1)^T,β=(1,-1,2)^T,则α×β=\left(\;\right)
$$
$$
A.
(-1,5,3)^T \quad B.(1,5,-3)^T \quad C.(1,-5,-3)^T \quad D.(1,-5,3)^T \quad E. \quad F. \quad G. \quad H.
$$
$$
α×β=\begin{vmatrix}i&j&k\\2&1&-1\\1&-1&2\end{vmatrix}=i-5j-3k=(1,-5,-3)^T
$$



$$
\;设α=(1,0,1)^T,β=(1,-1,0)^T,γ=(-2,\;\;0\;,\;1)^T,则α×\left(β+γ\right)=\left(\;\;\right)
$$
$$
A.
(1,2,-1)^T \quad B.(1,-2,1)^T \quad C.(1,-2,-1)^T \quad D.(-1,2,1)^T \quad E. \quad F. \quad G. \quad H.
$$
$$
\begin{array}{l}α×\left(β+γ\right)=\begin{vmatrix}i&j&k\\1&0&1\\-1&-1&1\end{vmatrix}=i-2j-k=(1,-2,-1)^T\\\end{array}
$$



$$
设α=\left(1,0,1\right)^T,β=\left(1,-1,0\right)^T,则\left(-2α\right)×β=\left(\;\;\;\right)
$$
$$
A.
\left(2,-2,2\right)^T \quad B.\left(2,2,-2\right)^T \quad C.\left(-4,-4,4\right)^T \quad D.\left(-2,-2,2\right)^T \quad E. \quad F. \quad G. \quad H.
$$
$$
\left(-2α\right)×β=(-2)×\begin{vmatrix}i&j&k\\1&0&1\\1&-1&0\end{vmatrix}=\left(-2,-2,2\right)^T
$$



$$
设α=\left(1,0,1\right)^T,β=\left(1,-1,0\right)^T,则\;β×\left(-2α\right)=\left(\;\;\;\right)
$$
$$
A.
\left(2,2,-2\right)^T \quad B.\left(-2,-2,2\right)^T \quad C.\left(2,-2,-2\right)^T \quad D.\left(-2,2,-2\right)^T \quad E. \quad F. \quad G. \quad H.
$$
$$
β×\left(-2α\right)=-2\begin{vmatrix}i&j&k\\1&-1&0\\1&0&1\end{vmatrix}=\left(2,2,-2\right)^T
$$



$$
设α=(1,0,1)^T,β=(1,-1,0)^T,γ=(-2,0,1)^T,则\left(α+γ\right)×β=\left(\;\;\right)
$$
$$
A.
(2,-2,1)^T \quad B.(-2,-2,1)^T \quad C.(-2,2,1)^T \quad D.(2,2,1)^T \quad E. \quad F. \quad G. \quad H.
$$
$$
\left(α+γ\right)×β=\begin{vmatrix}i&j&k\\-1&0&2\\1&-1&0\end{vmatrix}=(2,2,1)^T
$$



$$
设α=\left(1,0,1\right)^T,β=\left(1,-1,0\right)^T,\mathrm{则与}α,β\mathrm{都垂直的向量是}(\;\;).
$$
$$
A.
\left(3,-3,-3\right)^T \quad B.\left(1,-1,-1\right)^T \quad C.\left(3,3,-3\right)^T \quad D.\left(-1,3,-1\right)^T \quad E. \quad F. \quad G. \quad H.
$$
$$
α×β=\begin{vmatrix}i&j&k\\1&0&1\\1&-1&0\end{vmatrix}=\left(1,1,-1\right)^T,与α,β\mathrm{都垂直的向量与}\left(1,1,-1\right)^T\mathrm{对应成比例}.
$$



$$
设α=\left(1,1,0\right)^T,β=\left(1,0,1\right)^T,\mathrm{则以}α,β\mathrm{为邻边的平行四边形的面积为}(\;).
$$
$$
A.
\sqrt2 \quad B.2 \quad C.\sqrt5 \quad D.\sqrt3 \quad E. \quad F. \quad G. \quad H.
$$
$$
α×β=\begin{vmatrix}i&j&k\\1&1&0\\1&0&1\end{vmatrix}=\left(1,-1,-1\right)^T,\mathrm{所以围得面积为}\sqrt{1^2+\left(-1\right)^2+\left(-1\right)^2}=\sqrt3
$$



$$
设α=\left(1,0,1\right)^T,β=\left(1,-1,0\right)^T,\mathrm{则与}α,β\;\mathrm{都垂直的向量为}(\;\;).
$$
$$
A.
\left(-1,1,1\right)^T \quad B.\left(-1,1,-1\right)^T \quad C.\left(1,1,-1\right)^T \quad D.\left(1,-1,-1\right)^T \quad E. \quad F. \quad G. \quad H.
$$
$$
α×β=\begin{vmatrix}i&j&k\\1&0&1\\1&-1&0\end{vmatrix}=\left(1,1,-1\right)^T
$$



$$
设α=\left(2,1,-1\right)^T,β=\left(1,-1,2\right)^T,则\;α×β=\left(\;\;\right)
$$
$$
A.
\left(1,5,-3\right)^T \quad B.\left(1,-5,-3\right)^T \quad C.\left(1,-5,3\right)^T \quad D.\left(-1,-5,-3\right)^T \quad E. \quad F. \quad G. \quad H.
$$
$$
α×β=\begin{vmatrix}i&j&k\\2&1&-1\\1&-1&2\end{vmatrix}=\left(1,-5,-3\right)^T
$$



$$
设α,β,γ\mathrm{为非零向量},且α·β=0,α×γ=0,则
$$
$$
A.
α⁄⁄β 且β⟂γ, \quad B.α⟂β 且β⁄⁄γ, \quad C.α⁄⁄γ 且β⟂\alpha, \quad D.α⟂γ 且β⁄⁄γ, \quad E. \quad F. \quad G. \quad H.
$$
$$
α⁄⁄γ 且β⟂α,
$$



$$
设α=\left(1,0,1\right)^T,β=\left(1,-1,0\right)^T,γ=\left(-2,0,1\right)^T,则α×\left(β-γ\right)=\left(\;\;\right)
$$
$$
A.
\left(1,-4,1\right)^T \quad B.\left(-1,4,1\right)^T \quad C.\left(1,4,-1\right)^T \quad D.\left(-1,4,-1\right)^T \quad E. \quad F. \quad G. \quad H.
$$
$$
α×\left(β-γ\right)=\begin{vmatrix}i&j&k\\1&0&1\\3&-1&-1\end{vmatrix}=\left(1,4,-1\right)^T
$$



$$
\;设\left\|α\right\|=4,\left\|β\right\|=3,且α·β=6\sqrt3,则\left\|α×β\right\|=\left(\;\;\;\right)
$$
$$
A.
2\sqrt3 \quad B.3 \quad C.4 \quad D.6 \quad E. \quad F. \quad G. \quad H.
$$
$$
\left\|\alpha×β\right\|=4×3×\sinθ,而\cosθ={\textstyle\frac{6\sqrt3}{4×3}}={\textstyle\frac{\sqrt3}2},\mathrm{所以}\left\|α×β\right\|=4×3×{\textstyle\frac12}=6
$$



$$
设α=(1,0,-1)^T,β=(2,2,1)^T,γ=(1,1,-1)^T,则\;2α×\left(β-γ\right)=\left(\;\;\;\;\right)
$$
$$
A.
(2,6,2)^T \quad B.(-2,-6,-2)^T \quad C.(-2,-6,2)^T \quad D.(2,-6,2)^T \quad E. \quad F. \quad G. \quad H.
$$
$$
\;2α×\left(β-γ\right)=2\begin{vmatrix}i&j&k\\1&0&-1\\1&1&2\end{vmatrix}=(2,-6,2)^T
$$



$$
设\left\|α\right\|=2,\left\|β\right\|=\sqrt2,且α·β=2,则\left\|α×β\right\|=\left(\;\;\;\right)
$$
$$
A.
4 \quad B.\sqrt2 \quad C.1 \quad D.2 \quad E. \quad F. \quad G. \quad H.
$$
$$
\mathrm{因为}α·β=2,2=2×\sqrt2\cosθ,\mathrm{所以}θ={\textstyle\fracπ4},\mathrm{所以}\left\|α×β\right\|=2×\sqrt2\sinθ=2
$$



$$
设α=(a,-1,2)^T,β=(-2,0,b)^T,且α×β\;//(1,2,-1)^T,\mathrm{则有}(\;\;).
$$
$$
A.
a=-4,b=-2 \quad B.a=-4,b=2 \quad C.a=4,b=-2 \quad D.a=4,b=2 \quad E. \quad F. \quad G. \quad H.
$$
$$
α×β=\begin{vmatrix}i&j&k\\a&-1&2\\-2&0&b\end{vmatrix}=(-b,-ab-4,-2)^T,又α×β⁄⁄(1,2,-1)^{T,},\mathrm{故有}{\textstyle\frac{-b}1}={\textstyle\frac{-ab-4}2}={\textstyle\frac{-2}{-1}}⇒ a=4,b=-2
$$



$$
\;设α=\left(1,0,1\right)^T,β=\left(1,-1,0\right)^T,\mathrm{则两向量}α,β\mathrm{的夹角的正弦值为}(\;\;).
$$
$$
A.
\textstyle\frac{\sqrt3}6 \quad B.\textstyle\frac32 \quad C.\textstyle\frac{\sqrt3}4 \quad D.\textstyle\frac{\sqrt3}2 \quad E. \quad F. \quad G. \quad H.
$$
$$
\mathrm{因为}\left\|α×β\right\|=\left\|α\right\|·\left\|β\right\|\sinθ,\mathrm{因为}\begin{vmatrix}i&j&k\\1&0&1\\1&-1&0\end{vmatrix}=\left(1,1,-1\right)^T,\mathrm{所以sin}θ={\textstyle\frac{\sqrt3}2}
$$



$$
设α=(2,1,-1)^T,β=(1,-1,2)^T,则\;-2α×2β=\left(\;\;\right)
$$
$$
A.
(-4,-20,12)^T \quad B.(-4,20,-12)^T \quad C.(-4,20,12)^T \quad D.(-4,16,12)^T \quad E. \quad F. \quad G. \quad H.
$$
$$
\begin{array}{l}\;-2α×2β=-4\begin{vmatrix}i&j&k\\2&1&-1\\1&-1&2\end{vmatrix}=(-4,20,12)^T\\\end{array}
$$



$$
\;设\left\|α\right\|=\sqrt2,\left\|β\right\|=\sqrt2,且α·β=1,则\left\|α×β\right\|=\left(\;\;\;\;\right)
$$
$$
A.
\sqrt3 \quad B.\sqrt2 \quad C.1 \quad D.\textstyle\frac{\sqrt3}2 \quad E. \quad F. \quad G. \quad H.
$$
$$
\begin{array}{l}\;\mathrm{因为}\;α·β=\sqrt2×\sqrt2\cosθ,\mathrm{所以cos}θ={\textstyle\frac12}\\\left\|α×β\right\|=\sqrt2×\sqrt2\sinθ=\sqrt3\end{array}
$$



$$
设α=(1,b,-1)^T,β=(-1,a,2)^T,且α×β//(1,-1,3)^T,\mathrm{则有}(\;\;).\;
$$
$$
A.
a=-5,b=-2 \quad B.a=5,b=2 \quad C.a=6,b=-2 \quad D.a=5,b=-2 \quad E. \quad F. \quad G. \quad H.
$$
$$
\mathrm{因为}α×β\boldsymbol 与\boldsymbolα\boldsymbol,\boldsymbolβ\mathbf{都垂直}\boldsymbol,\mathbf{所以}\left\{\begin{array}{l}\mathbf1\boldsymbol-\mathbf b\boldsymbol-\mathbf3\boldsymbol=\mathbf0\\\boldsymbol-\mathbf1\boldsymbol-\mathbf a\boldsymbol+\mathbf6\boldsymbol=\mathbf0\end{array}\right.\mathbf{所以a}\boldsymbol=\mathbf5\boldsymbol,\mathbf b\boldsymbol=\boldsymbol-\mathbf2
$$



$$
设α=(3,b,-1)^T,β=(-1,a,2)^T,且α×β//(1,-1,1)^T,\mathrm{则有}(\;\;).\;
$$
$$
A.
a=-1,b=-2 \quad B.a=-1,b=2 \quad C.a=1,b=-2 \quad D.a=1,b=2 \quad E. \quad F. \quad G. \quad H.
$$
$$
α×β 与α,β\mathrm{都垂直},\mathrm{因为}\left\{\begin{array}{l}3-b-1=0\\-1-a+2=0\end{array}\right.\mathrm{所以}a=1,b=2
$$



$$
设α=(3,-5,8)^T,β=(-1,1,z)^T,\mathrm{满足}\left|\vertα+β\right|\vert=\left|\vertα-β\vert\right|,则\;z=\left(\;\;\right)
$$
$$
A.
1 \quad B.2 \quad C.-1 \quad D.-2 \quad E. \quad F. \quad G. \quad H.
$$
$$
\begin{array}{l}\mathrm{因为}\left|\vertα+β\vert\right|=\left|\vertα-β\right|\vert,α+β=(2,-4,8+z)^T,α-β=(4,-6,8-z)^T\\\mathrm{所以}2^2+\left(-4\right)^2+\left(8+z\right)^2=4^2+\left(-6\right)^2+\left(8-z\right)^2,\mathrm{所以}z=1\end{array}
$$



$$
设α=(2,-1,2)^T,β=(3,0,1)^T,\mathrm{则以}α,β\mathrm{为边的平行四边形的面积为}(\;\;).\;
$$
$$
A.
,3\sqrt3 \quad B.\sqrt5 \quad C.5 \quad D.\sqrt{26} \quad E. \quad F. \quad G. \quad H.
$$
$$
α×β=\begin{vmatrix}i&j&k\\2&-1&2\\3&0&1\end{vmatrix}=(-1,4,3)^T,\mathrm{所以面积为}\left\|α×β\right\|=\sqrt{\left(-1\right)^2+4^2+3^3}=\sqrt{26}
$$



$$
设α=(1,a,-1)^T,β=(0,1,2)^T,\mathrm{若以}α,β\mathrm{为边的平行四边形的面积为}\sqrt{54},则(\;\;).\;
$$
$$
A.
a=3 \quad B.a=-4或a=3 \quad C.a=-4 \quad D.a=-3 \quad E. \quad F. \quad G. \quad H.
$$
$$
\begin{array}{l}\mathrm{因为}α×β=\begin{vmatrix}i&j&k\\1&a&-1\\0&1&2\end{vmatrix}=(2a+1,-2,1)^T\\\left(2a+1\right)^2+\left(-2\right)^2+1^2=54\\\mathrm{所以}a=-4或a=3\end{array}
$$



$$
设α=(a,b,-1)^T,β=(-1,a,b)^T,且α×β//(1,-1,3)^T,\mathrm{则有}(\;\;).
$$
$$
A.
a=-5,b=2 \quad B.a=5,b=-2 \quad C.a=5,b=2 \quad D.a=-5,b=-2 \quad E. \quad F. \quad G. \quad H.
$$
$$
\begin{array}{l}解1:设γ=(1,-1,3)^T,\mathrm{因为}α×β⁄⁄γ,\mathrm{所以}α·γ=0,β·γ=0,即\;\;\;\\\;\;\;\;\;\;\;\;\;\;\;\;\;\;\mathrm{所以}\left\{\begin{array}{l}a-b-3=0\\-1-a+3b=0\end{array}\right.\\\;\;\;\;\;\;\;\;\;\;\;\;\;\;\;\;\;\mathrm{解得}:a=5,b=2\\解2:α×β=\begin{vmatrix}i&j&k\\a&b&-1\\-1&a&b\end{vmatrix}=(b^2+a,-ab+1,a^2+b)^T,又α×β⁄⁄(1,-1,3)^T,\mathrm{故有}\\{\textstyle\frac{b^2+a}1}={\textstyle\frac{-ab+1}{-1}}={\textstyle\frac{a^2+b}3}\mathrm{所以}a=5,b=2\end{array}
$$



$$
\mathrm{已知三维向量空间中两个向量}α_1=\begin{pmatrix}1\\1\\1\end{pmatrix},α_2=\begin{pmatrix}1\\-2\\1\end{pmatrix}\mathrm{正交},若α_1,α_2,α_3\mathrm{构成三维空间的一个正交基},则α_3=\left(\right).
$$
$$
A.
\left(-1,0,1\right)^T \quad B.\left(-1,1,0\right)^T \quad C.\left(0,-1,1\right)^T \quad D.\left(-1,0,0\right)^T \quad E. \quad F. \quad G. \quad H.
$$
$$
\begin{array}{l}设α_3=\left(χ_1,χ_2,χ_3\right)^{\;T}\neq0,\mathrm{且分别与}α_1,α_2\mathrm{正交},则\;\\\;\;\;\;\;\;\;\;\;\;\;\;\;\;\;\;\;\;\;\;\left\langleα_1,α_3\right\rangle=\left\langleα_2,α_3\right\rangle=0\;\;\;\;\;\;\;\;\;\;\;\;\;\;\;\;\;\\即\;\;\;\;\;\;\;\;\;\;\;\;\;\;\;\;\;\left\{\begin{array}{l}\;\left\langleα_1,α_3\right\rangle=χ_1+χ_2+χ_3=0\\\left\langleα_2,α_3\right\rangle=χ_1-2χ_2+χ_3=0\end{array}\right.\;\;\;\\\;\mathrm{解之得}\;\;\;\;\;\;\;\;\;χ_1=-χ_3,χ_2=0\;\;\;\;\;\;\;\;\;\;\\令χ_3=1⇒α_3=\begin{pmatrix}χ_1\\χ_2\\{\boldsymbolχ}_\mathbf3\end{pmatrix}\;=\begin{pmatrix}-1\\0\\1\end{pmatrix}\;\;\;\\\mathrm{由上可知}α_1,α_2,α_3\mathrm{构成三维空间的一个正交基}.\end{array}
$$



$$
\begin{array}{l}\mathrm{下列矩阵为正交矩阵的是}\left(\;\;\;\;\right).\;\\\left(1\right)\;\begin{pmatrix}1&-{\textstyle\frac12}&\textstyle\frac13\\-{\textstyle\frac12}&1&\textstyle\frac12\\\textstyle\frac13&\textstyle\frac12&-1\end{pmatrix};\;\;\left(2\right)\;\begin{pmatrix}\textstyle\frac19&-{\textstyle\frac89}&\textstyle-\frac49\\-{\textstyle\frac89}&\textstyle\frac19&\textstyle-\frac49\\\textstyle-\frac49&\textstyle-\frac49&\textstyle\frac79\end{pmatrix}\;\;\;\;\;\;\;.\end{array}
$$
$$
A.
\left(1\right)\; \quad B.\left(2\right)\; \quad C.\left(1\right)\;\left(2\right)\; \quad D.\mathrm{都不是} \quad E. \quad F. \quad G. \quad H.
$$
$$
\begin{array}{l}(1)\mathrm{考察矩阵的第一列和第二列},\\\;\mathrm{因为}\;\;\;1×\left(-{\textstyle\frac12}\right)+\left(-{\textstyle\frac12}\right)×1+{\textstyle\frac13}×{\textstyle\frac12}\neq0,\;\;\;\;\;\;\\\mathrm{所以它不是正交矩阵};\;\\(2)\mathrm{由正交矩阵的定义},\\\mathrm{因为}\;\;\begin{pmatrix}\textstyle\frac19&-{\textstyle\frac89}&\textstyle-\frac49\\-{\textstyle\frac89}&\textstyle\frac19&\textstyle-\frac49\\\textstyle-\frac49&\textstyle-\frac49&\textstyle\frac79\end{pmatrix}\;\begin{pmatrix}\textstyle\frac19&-{\textstyle\frac89}&\textstyle-\frac49\\-{\textstyle\frac89}&\textstyle\frac19&\textstyle-\frac49\\\textstyle-\frac49&\textstyle-\frac49&\textstyle\frac79\end{pmatrix}^T=\begin{pmatrix}1&0&0\\0&1&0\\0&0&1\end{pmatrix}\;,\;\;\;\;\;\;\;\\\mathrm{所以它是正交矩阵}.\end{array}
$$



$$
\begin{array}{l}\mathrm{下列两个矩阵中是正交矩阵的为}\left(\;\;\;\right)\\\left(1\right)\;\;\begin{pmatrix}3&-3&1\\-3&1&3\\1&3&-3\end{pmatrix};\;\;\;\;\left(2\right)\;\;\begin{pmatrix}\textstyle\frac23&\textstyle\frac23&\textstyle\frac13\\\textstyle\frac23&-{\textstyle\frac13}&-{\textstyle\frac23}\\\textstyle\frac13&-{\textstyle\frac23}&\textstyle\frac{\mathbf2}{\mathbf3}\end{pmatrix}\;\;\;\;\;\;\;\;\;.\end{array}
$$
$$
A.
\left(1\right) \quad B.\left(2\right) \quad C.\left(1\right)\left(2\right) \quad D.\mathrm{两个都不是} \quad E. \quad F. \quad G. \quad H.
$$
$$
\begin{array}{l}\left(1\right)\mathrm{第一个行向量非单位向量},\mathrm{故不是正交矩阵};\;\\\left(2\right)\mathrm{该方阵每一个行向量均是单位向量},\mathrm{且两两正交},\mathrm{故为正交矩阵}.\end{array}
$$



$$
设A\mathrm{为正交矩阵},α_i,α_j\mathrm{分别是}A\mathrm{的第}i,j列\left(i\neq j\right),则\left\langleα_i,α_j\right\rangle 和\left\langleα_j,α_j\right\rangle\mathrm{分别为}\left(\;\;\;\right)
$$
$$
A.
0;1 \quad B.0;0 \quad C.1;1 \quad D.1;0 \quad E. \quad F. \quad G. \quad H.
$$
$$
\begin{array}{l}\mathrm{正交矩阵的列向量都是单位正交向量组},\mathrm{则不同列的向量正交},即\left\langleα_i,α_j\right\rangle=0;\;\;\\\mathrm{又由于列向量为单位向量},则\left\langleα_j,α_j\right\rangle=\left\|α_j\right\|^2=1\end{array}
$$



$$
\mathrm{设矩阵}A=\begin{pmatrix}\textstyle\frac23&\textstyle\frac1{\sqrt2}&\textstyle\frac1{\sqrt{18}}\\a&b&\textstyle\frac{-4}{\sqrt{18}}\\\textstyle\frac23&\textstyle\frac{-1}{\sqrt2}&\textstyle\frac1{\sqrt{18}}\end{pmatrix}\mathrm{为正交矩阵},则a,b\mathrm{分别为}\left(\;\;\;\;\right)
$$
$$
A.
a={\textstyle\frac13},b=0 \quad B.a={\textstyle±\frac13},b=0 \quad C.a={\textstyle0},b={\textstyle\frac1{\sqrt2}} \quad D.a={\textstyle0},b=±{\textstyle\frac1{\sqrt2}} \quad E. \quad F. \quad G. \quad H.
$$
$$
\mathrm{正交矩阵的列向量是正交的单位向量组},\mathrm{因此}\left({\textstyle\frac23}\right)^2+a^2+\left({\textstyle\frac23}\right)^2=1,\left({\textstyle\frac1{\sqrt2}}\right)^2+b^2+\left({\textstyle-\frac1{\sqrt2}}\right)^2=1,\mathrm{且第一列向量与第三列向量正交},则a={\textstyle\frac13},b=0.
$$



$$
\mathrm{已知}A=\begin{pmatrix}\textstyle\frac1{\sqrt3}&0&χ\\\textstyle\frac1{\sqrt3}&\textstyle\frac{-1}{\sqrt2}&\textstyle\frac1{\sqrt6}\\\textstyle\frac1{\sqrt3}&\textstyle\frac1{\sqrt2}&\textstyle\frac1{\sqrt6}\end{pmatrix}^T\mathrm{是正交矩阵},则χ=\left(\;\;\;\right)
$$
$$
A.
-{\textstyle\frac2{\sqrt6}} \quad B.\textstyle\frac2{\sqrt6} \quad C.\textstyle±\frac2{\sqrt6} \quad D.0 \quad E. \quad F. \quad G. \quad H.
$$
$$
\begin{array}{l}\mathrm{正交矩阵的列向量相互正交},\mathrm{考察矩阵的第一列和第三列},因\;\;\\\;\;\;\;\;\;\;\;\;\;{\textstyle\frac1{\sqrt3}}χ+\;{\textstyle\frac1{\sqrt3}}×\;{\textstyle\frac1{\sqrt6}}+\;{\textstyle\frac1{\sqrt3}}×\;{\textstyle\frac1{\sqrt6}}=0⇒χ=-{\textstyle\frac2{\sqrt6}}.\end{array}
$$



$$
\mathrm{已知}A=\begin{pmatrix}\textstyle\frac{-2}{\sqrt5}&\textstyle\frac2{3\sqrt5}&χ\\\textstyle\frac1{\sqrt5}&\textstyle\frac4{3\sqrt5}&\textstyle\frac23\\0&\textstyle\frac5{3\sqrt5}&-{\textstyle\frac23}\end{pmatrix}\mathrm{是正交矩阵},则χ=\left(\;\;\;\right)
$$
$$
A.
\textstyle\frac13 \quad B.\textstyle-\frac13 \quad C.\textstyle\frac23 \quad D.\textstyle-\frac23 \quad E. \quad F. \quad G. \quad H.
$$
$$
\begin{array}{l}\mathrm{正交矩阵的列向量组是单位正交向量组},\mathrm{考察矩阵的第一列和第三列}:\;\\\;\;\;\;\;-{\textstyle\frac2{\sqrt5}}χ+{\textstyle\frac1{\sqrt5}}×{\textstyle\frac23}=0⇒χ={\textstyle\frac13}\;.\end{array}
$$



$$
\mathrm{设向量}α=\left(1,1,-1\right)^T,β=\left(-2,-1,2\right)^T,\mathrm{有一向量}γ=\left(2,a,b\right)^T,与α 和β\mathrm{都正交},则a=\left(\;\;\;\right)
$$
$$
A.
1 \quad B.2 \quad C.0 \quad D.3 \quad E. \quad F. \quad G. \quad H.
$$
$$
\begin{array}{l}γ=\left(2,a,b\right)^T与α 和β\mathrm{都正交},\mathrm{故有}\left\langleγ,α\right\rangle=\left\langleγ,β\right\rangle=0,即\\\left\{\begin{array}{l}2+a-b=0\\-4-a+2b=0\end{array}\right.⇒ a=0,b=2\end{array}
$$



$$
\mathrm{设向量}α=(1,1,-1)^T,β=(-2,-1,2)^T,\mathrm{有一向量}γ=(2,a,b)^T,与α 和β\mathrm{都正交},则\;b=\left(\;\;\;\right)
$$
$$
A.
1 \quad B.2 \quad C.3 \quad D.0 \quad E. \quad F. \quad G. \quad H.
$$
$$
\begin{array}{l}γ=(2,a,b)^T与α 和β\mathrm{都正交},\mathrm{故有}\left\langleγ,α\right\rangle=\left\langleγ,β\right\rangle=0,即\\\left\{\begin{array}{l}2+a-b=0\\-4-a+2b=0\end{array}\right.⇒ a=0,b=2\end{array}
$$



$$
\mathrm{与向量}α_1=\left(1,0,1\right)^T,α_2=\left(0,1,1\right)^T\mathrm{等价的标准正交向量组是}(\;\;\;\;)
$$
$$
A.
\textstyle\frac1{\sqrt2}\left(1,0,1\right)^T,\frac1{\sqrt6}\left(-1,2,1\right)^T \quad B.\textstyle\frac1{\sqrt2}\left(-1,0,1\right)^T,\frac1{\sqrt6}\left(-1,2,1\right)^T \quad C.\textstyle\frac1{\sqrt2}\left(1,0,-1\right)^T,\frac1{\sqrt6}\left(-1,2,1\right)^T \quad D.\textstyle\frac1{\sqrt2}\left(1,0,1\right)^T,\frac1{\sqrt6}\left(1,-2,1\right)^T \quad E. \quad F. \quad G. \quad H.
$$
$$
\begin{array}{l}\mathrm{由施密特正交化方法},\mathrm{先正交化},\mathrm{即令}β_1=α_1=(1,0,1)^T\;,\\β_2=α_2-{\textstyle\frac{\left\langleα_2,β_1\right\rangle}{\left\langleβ_1,β_1\right\rangle}}β_1=(0,1,1)^T-{\textstyle\frac12}(1,0,1)^T={\textstyle\frac12}{\textstyle(}{\textstyle-}{\textstyle1}{\textstyle,}{\textstyle2}{\textstyle,}{\textstyle1}{\textstyle)}{\textstyle{}^T}{\textstyle,}\\\mathrm{再将其单位化得},η_1={\textstyle\frac{β_1}{\left\|β_1\right\|}}={\textstyle\frac1{\sqrt2}}(1,0,1)^T,η_2={\textstyle\frac{β_2}{\left\|β_2\right\|}}={\textstyle\frac1{\sqrt6}}(-1,2,1)^T\end{array}
$$



$$
\mathrm{与向量}α_1=\left(1,1,1\right)^T,α_2=\left(0,1,1\right)^T,\mathrm{等价的标准正交向量组是}(\;\;\;\;)\;
$$
$$
A.
{\textstyle\frac1{\sqrt2}}\left(1,1,1\right)^T,\;\;\frac1{\sqrt6}\left(-1,2,1\right)^T \quad B.\frac1{\sqrt3}\left(1,1,1\right)^T,\;\;\frac1{\sqrt6}\left(-2,1,1\right)^T \quad C.\frac1{\sqrt3}\left(1,1,1\right)^T,\;\;\frac1{\sqrt2}\left(0,1,1\right)^T \quad D.\frac1{\sqrt3}\left(1,1,1\right)^T,\;\;\frac1{\sqrt3}\left(0,-1,1\right)^T \quad E. \quad F. \quad G. \quad H.
$$
$$
\begin{array}{l}\mathrm{由施密特正交化方法},\mathrm{先正交化},\mathrm{即令}\;β_1=α_1=\left(1,1,1\right)^T,\;β_2=α_2-{\textstyle\frac{\left\langle\;α_2,\;β_1\right\rangle}{\left\langleβ_1,\;β_1\right\rangle}}β_1={\textstyle\frac13}\left(-2,1,1\right)^T,\\\mathrm{再将其单位化得},η_1={\textstyle\frac{β_1}{\left\|β_1\right\|}}={\textstyle\frac1{\sqrt3}}\left(1,1,1\right)^T,η_2={\textstyle\frac{β_2}{\left\|β_2\right\|}}={\textstyle\frac1{\sqrt6}}\left(-2,1,1\right)^T\end{array}
$$



$$
\mathrm{与向量组}α_1=\left(1,1,1\right)^T,α_2=\left(1,2,3\right)^T,α_3=\left(1,4,9\right)^T,\mathrm{都等价的标准正交向量组是}(\;\;\;)
$$
$$
A.
\textstyle\frac1{\sqrt3}\left(1,1,1\right)^T,\frac1{\sqrt2}\left(1,0,-1\right)^T,\frac1{\sqrt6}\left(1,-2,1\right)^T \quad B.\textstyle\frac1{\sqrt3}\left(1,1,1\right)^T,\frac1{\sqrt2}\left(-1,0,1\right)^T,\frac1{\sqrt6}\left(1,-2,-1\right)^T \quad C.\textstyle\frac1{\sqrt3}\left(1,1,1\right)^T,\frac1{\sqrt2}\left(-1,0,1\right)^T,\frac1{\sqrt6}\left(1,-2,1\right)^T \quad D.\textstyle\frac1{\sqrt3}\left(1,1,1\right)^T,\frac1{\sqrt2}\left(-1,0,1\right)^T,\frac1{\sqrt6}\left(-1,2,1\right)^T \quad E. \quad F. \quad G. \quad H.
$$
$$
\begin{array}{l}\mathrm{由施密特正交化方法},\mathrm{先正交化},\mathrm{即令}β_1=α_1=\left(1,1,1\right)^T,\\β_2=α_2-{\textstyle\frac{\left\langleα_2,β_1\right\rangle}{\left\langleβ_1,β_1\right\rangle}}β_1=\left(1,2,3\right)^T-{\textstyle\frac63}\left(1,1,1\right)^T=\left(-1,0,1\right)^T,\\β_3=α_3-{\textstyle\frac{\left\langleα_3,β_1\right\rangle}{\left\langleβ_1,β_1\right\rangle}}β_1-\frac{\left\langleα_3,β_2\right\rangle}{\left\langleβ_2,β_2\right\rangle}β_2=\left(1,4,9\right)^T-{\textstyle\frac{14}3}\left(1,1,1\right)^T-{\textstyle4}\left(-1,0,1\right)^T={\textstyle\frac13}\left(1,-2,1\right)^T,\\\mathrm{再将其单位化得},η_1={\textstyle\frac{β_1}{\left\|β_1\right\|}}={\textstyle\frac1{\sqrt3}}\left(1,1,1\right)^T,η_2={\textstyle\frac{β_2}{\left\|β_2\right\|}}={\textstyle\frac1{\sqrt2}}\left(-1,0,1\right)^T,\\η_3={\textstyle\frac{β_3}{\left\|β_3\right\|}}={\textstyle\frac1{\sqrt6}}\left(1,-2,1\right)^T\end{array}
$$



$$
设A=\begin{pmatrix}1&4&8\\4&7&-4\\8&-4&1\end{pmatrix},若B=kA\mathrm{为正交矩阵},则\;k\mathrm{的值为}(\;\;\;\;\;).
$$
$$
A.
±9 \quad B.9 \quad C.±{\textstyle\frac19} \quad D.\textstyle\frac19 \quad E. \quad F. \quad G. \quad H.
$$
$$
\begin{array}{l}\mathrm{由于矩阵}A\mathrm{的列向量相互正交},且\sqrt{1^2+4^2+8^2}=9,\sqrt{4^2+7^2+\left(-4\right)^2}=9,取k=±{\textstyle\frac19},则\\B=kA=±{\textstyle\frac19}\begin{pmatrix}1&4&8\\4&7&-4\\8&-4&1\end{pmatrix}\mathrm{为正交}\end{array}
$$



$$
设A=\begin{pmatrix}2&3&6\\3&-6&2\\6&2&-3\end{pmatrix},若B=kA\mathrm{为一正交矩阵},则k\mathrm{的值为}(\;\;\;\;).
$$
$$
A.
7 \quad B.±{\textstyle\frac17} \quad C.±7 \quad D.\textstyle\frac17 \quad E. \quad F. \quad G. \quad H.
$$
$$
取k=±{\textstyle\frac1{\sqrt{2^2+3^2+6^2}}}=±{\textstyle\frac17},记B=±{\textstyle\frac17}\begin{pmatrix}2&3&6\\3&-6&2\\6&2&-3\end{pmatrix},则B\mathrm{为正交阵}.
$$



$$
\mathrm{设矩阵}A={\textstyle\frac12}\begin{pmatrix}1&2a&1\\-1&\sqrt2&2b\\\sqrt2&2c&-\sqrt2\end{pmatrix}\mathrm{为正交矩阵},则a,b,c\mathrm{的值分别为}(\;\;\;\;).
$$
$$
A.
a={\textstyle\frac1{\sqrt2}},b=-{\textstyle\frac12},c=0 \quad B.a={\textstyle0},b=-{\textstyle\frac12},c={\textstyle\frac12} \quad C.a={\textstyle\frac1{\sqrt2}},b={\textstyle0},c={\textstyle\frac12} \quad D.a=-{\textstyle\frac1{\sqrt2}},b=-{\textstyle\frac12},c=1 \quad E. \quad F. \quad G. \quad H.
$$
$$
\begin{array}{l}A\mathrm{为正交矩阵},A\mathrm{的列向量组相互正交},由\;\;\;\\\;\;\;\;\;\;\;\;\;\;\;\;\;\;\;\;\;\;\;\;\;\;\;\;\;\;\;\;\;\;\;\;\;\;\;\;\;\;\;\;\;\;\;\;\;\;\;\;\;\;\;\;\;\;\;\;\;\;\;\;\;\;\;\;\;\;\;\;\;\;\;\;\;\;\;\;\;\left\{\begin{array}{l}\begin{array}{c}2a-\sqrt2+2\sqrt2c=0\\1-2b-2=0\end{array}\\\begin{array}{c}\mathbf2\boldsymbol a\boldsymbol+\mathbf2\sqrt{\mathbf2}\boldsymbol b\boldsymbol-\mathbf2\sqrt{\mathbf2}\boldsymbol c\boldsymbol=\mathbf0\end{array}\end{array}\right.\;\;\;\;\;\;\;\;\;\;\;\;\;\;\;\;\;\;\;\;\;\;\;\;\;\;\;\;\;\;\;\;\;\;\;\;\;\;\;\;\;\;\;\;\;\;\;\;\\\\\mathrm{可得}a={\textstyle\frac1{\sqrt2}},b=-{\textstyle\frac12},c=0.\mathrm{此时}A\mathrm{的各列为单位向量},故A\mathrm{为正交矩阵}.\end{array}
$$



$$
\mathrm{已知}A=\begin{pmatrix}x&\textstyle\frac12\\\textstyle\frac12&y\end{pmatrix}\mathrm{是正交矩阵},且x>0,则\begin{pmatrix}x\\y\end{pmatrix}=\left(\;\;\right)
$$
$$
A.
x={\textstyle\frac{\sqrt3}2},y=-{\textstyle\frac{\sqrt3}2} \quad B.x=-{\textstyle\frac{\sqrt3}2},y={\textstyle\frac{\sqrt3}2} \quad C.x={\textstyle\frac12},y={\textstyle\frac{\sqrt3}2} \quad D.x={\textstyle\frac12},y=-{\textstyle\frac{\sqrt3}2} \quad E. \quad F. \quad G. \quad H.
$$
$$
\begin{array}{l}\mathrm{由于}A\mathrm{是正交矩阵},AA^T=A^TA=E,即\;\;\\\;\;\;\;\;\;\;\;\;\;\;\begin{pmatrix}x&\textstyle\frac12\\\textstyle\frac12&y\end{pmatrix}\begin{pmatrix}x&\textstyle\frac12\\\textstyle\frac12&y\end{pmatrix}=\begin{pmatrix}1&\textstyle0\\0&\textstyle1\end{pmatrix}⇒\left\{\begin{array}{l}\begin{array}{c}x^2+{\textstyle\frac14}=1\\{\textstyle\frac12}x+{\textstyle\frac12}y=0\end{array}\\{\textstyle\frac14}\begin{array}{c}+y^2=1\end{array}\end{array}\right.,\\\;\;\mathrm{所以}\left\{\begin{array}{l}χ={\textstyle\frac{\sqrt3}2}\\y=-{\textstyle\frac{\sqrt3}2}\end{array}\right.;或\begin{array}{l}χ={\textstyle\frac{-\sqrt3}2}\\y={\textstyle\frac{\sqrt3}2}\end{array}(\mathrm{舍去}).\end{array}
$$



$$
\begin{array}{l}\mathrm{已知}α_1=\left({\textstyle\frac1{\sqrt3}},{\textstyle\frac1{\sqrt3}},{\textstyle\frac1{\sqrt3}}\right),α_2=\left({\textstyle-\frac1{\sqrt2},\frac1{\sqrt2},0}\right),α_3=\left({\textstyle-\frac1{\sqrt6},-\frac1{\sqrt6},\frac2{\sqrt6}}\right)是R^3\mathrm{的一个标准正交基},\\\mathrm{若用这个基来线性表示}R^3\mathrm{中的向量}α=\left({\textstyle1,-1,-1}\right),则α=\left(\;\;\;\right)\end{array}
$$
$$
A.
-{\textstyle\frac1{\sqrt3}}α_1-\sqrt2α_2-{\textstyle\frac2{\sqrt6}}α_3 \quad B.{\textstyle\frac1{\sqrt3}}α_1-\sqrt2α_2-{\textstyle\frac2{\sqrt6}}α_3 \quad C.-{\textstyle\frac1{\sqrt3}}α_1+\sqrt2α_2+{\textstyle\frac2{\sqrt6}}α_3 \quad D.{\textstyle\frac1{\sqrt3}}α_1+\sqrt2α_2-{\textstyle\frac2{\sqrt6}}α_3 \quad E. \quad F. \quad G. \quad H.
$$
$$
\begin{array}{l}设κ_1α_1+κ_2α_2+κ_3α_3=α,\mathrm{则对矩阵}\left(α_1,α_2,α_3,α\right)\mathrm{进行初等行变换},得\;\\\;\;\begin{pmatrix}\textstyle\frac1{\sqrt3}&-{\textstyle\frac1{\sqrt2}}&-{\textstyle\frac1{\sqrt6}}&1\\\textstyle\frac1{\sqrt3}&\textstyle\frac1{\sqrt2}&-{\textstyle\frac1{\sqrt6}}&-1\\\textstyle\frac1{\sqrt3}&0&\textstyle\frac2{\sqrt6}&-1\end{pmatrix}\rightarrow\begin{pmatrix}\textstyle\frac1{\sqrt3}&-{\textstyle\frac1{\sqrt2}}&-{\textstyle\frac1{\sqrt6}}&1\\\textstyle0&\textstyle\frac2{\sqrt2}&0&-2\\\textstyle0&\textstyle\frac1{\sqrt2}&\textstyle\frac3{\sqrt6}&-2\end{pmatrix}\rightarrow\begin{pmatrix}\textstyle\frac1{\sqrt3}&0&0&\textstyle-\frac13\\\textstyle0&\textstyle\frac1{\sqrt2}&0&-1\\\textstyle0&\textstyle0&\textstyle\frac3{\sqrt6}&-1\end{pmatrix}\rightarrow\begin{pmatrix}\textstyle1&0&0&\textstyle-\frac1{\sqrt3}\\\textstyle0&\textstyle1&0&-\sqrt2\\\textstyle0&\textstyle0&\textstyle1&-{\textstyle\frac2{\sqrt6}}\end{pmatrix},\;\\\;\mathrm{则由最后的矩阵可知}α=-{\textstyle\frac1{\sqrt3}}α_1-\sqrt2α_2-{\textstyle\frac2{\sqrt6}}α_3.\end{array}
$$



$$
若α_1=(\frac23,{\textstyle\frac13}{\textstyle,}{\textstyle\frac23})^T,α_2=({\textstyle\frac13}{\textstyle,}{\textstyle\frac23}{\textstyle,}{\textstyle-}{\textstyle\frac23})^T,及α_3\mathrm{是两两正交的单位向量组},则α_3=\left(\;\;\;\right)
$$
$$
A.
±(-{\textstyle\frac23},{\textstyle\frac23},{\textstyle\frac13})^T \quad B.(\boldsymbol-{\textstyle\frac{\mathbf2}{\mathbf3}}\boldsymbol,{\textstyle\frac{\mathbf2}{\mathbf3}}\boldsymbol,{\textstyle\frac{\mathbf1}{\mathbf3}})^\mathbf T \quad C.(\frac23,-{\textstyle\frac23},-{\textstyle\frac13})^T \quad D.±(-{\textstyle\frac23},{\textstyle\frac13},{\textstyle\frac23})^T \quad E. \quad F. \quad G. \quad H.
$$
$$
\begin{array}{l}设α_3=(x_1,x_2,x_3)^T,\mathrm{由条件可知}α_1^Tα_3=0,α_2^Tα_3=0即\;\;\\\;\;\;\;\;\;\;\;\;\;\;\;\;\;\;\;\;\;\;\;\;\;\;\;\;\;\;\;\;\;\;\;\;\;\;\;\;\;\;\;\;\;\;\;\;\;\;\;\;\;\;\;\;\;\;\;\;\;\;\;\;\;\;\;\;\;\;\;\;\;\;\;\;\;\;\;\;\;\;\;\;\;\;\;\;\;\;\;\;\;\;\;\;\;\;\left\{\begin{array}{l}{\textstyle\frac23}x_1+{\textstyle\frac13}{\textstyle x}{\textstyle{}_2}{\textstyle+}{\textstyle\frac23}{\textstyle x}{\textstyle{}_3}{\textstyle=}{\textstyle0}\\\textstyle\frac13x_1+\frac23x_2-\frac23x_3=0\end{array}\right.,\;\;\\令x{\textstyle{}_3}{\textstyle=}{\textstyle k},\mathrm{解得}α_3=k\left(-2,2,1\right)^T,又\left\|α_3\right\|=1⇒ k^2\left(4+4+1\right)=1⇒ k=±{\textstyle\frac13},则\\α_3=±(-{\textstyle\frac23},{\textstyle\frac23},{\textstyle\frac13})^T\end{array}
$$



$$
\mathrm{如果}α_1=\left(1,0,0\right)^T,α_2=\left(0,{\textstyle\frac1{\sqrt2}},{\textstyle\frac1{\sqrt2}}\right)^T及α_3是R^3\mathrm{中两两正交的单位向量组},则\;α_3=\left(\;\;\;\;\right)
$$
$$
A.
±(0,-{\textstyle\frac1{\sqrt2}},{\textstyle\frac1{\sqrt2}})^T \quad B.±(\frac1{\sqrt2},-{\textstyle\frac1{\sqrt2}},{\textstyle\frac1{\sqrt2}})^T \quad C.±(-\frac1{\sqrt2},0,{\textstyle\frac1{\sqrt2}})^T \quad D.±(\frac1{\sqrt2},-{\textstyle\frac1{\sqrt2}},{\textstyle0})^T \quad E. \quad F. \quad G. \quad H.
$$
$$
\begin{array}{l}设α_3=(x_1,x_2,x_3)^T,\mathrm{则由条件可知}α_1^Tα_3=0,α_2^Tα_3=0,即\;\;\\\;\;\;\;\;\;\;\;\;\;\;\;\;\;\;\;\;\;\;\;\;\;\;\;\;\;\;\;\;\;\;\;\;\;\;\;\;\;\;\;\;\;\;\;\;\;\;\;\;\;\;\;\;\;\left\{\begin{array}{l}x_1=0\\{\textstyle\frac1{\sqrt2}}x_2+{\textstyle\frac1{\sqrt2}}x{\textstyle{}_3}{\textstyle=}{\textstyle0}\end{array}\right.,\mathrm{解之得}α_3=k(0,-1,1)^T,\mathrm{其中}k∈ R,\;\;\\由α_3\mathrm{为单位向量},则\left\|α_3\right\|=1⇒ k=±{\textstyle\frac1{\sqrt2}},则α_3=±(0,-{\textstyle\frac1{\sqrt2}},{\textstyle\frac1{\sqrt2}})^T.\end{array}
$$



$$
\mathrm{设方阵}A\mathrm{为正交矩阵},且A^T=A^*,\mathrm{其中}A^* 是A\mathrm{的伴随矩阵},则A\mathrm{的行列式等于}\left(\;\;\;\;\right)
$$
$$
A.
1 \quad B.-1 \quad C.0 \quad D.±1 \quad E. \quad F. \quad G. \quad H.
$$
$$
\begin{array}{l}A\mathrm{为正交矩阵},则AA^T=E,又A^T=A^*,则AA^*=E;\;\;\\\mathrm{由伴随矩阵的性质可知}AA^*=\left|A\right|E,即\left|\mathbf A\right|\boldsymbol E\boldsymbol=\boldsymbol E\boldsymbol⇒\left|\mathbf A\right|\boldsymbol=\mathbf1.\end{array}
$$



$$
设α=(0,y,-{\textstyle\frac1{\sqrt2}})^T,β=(x,0,0)^T,若α,β\mathrm{是标准正交向量组},则\left(\;\;\;\;\right)
$$
$$
A.
x\mathrm{任意},y=-{\textstyle\frac1{\sqrt2}} \quad B.x\mathrm{任意},y={\textstyle\frac12} \quad C.x=±1,y=±{\textstyle\frac1{\sqrt2}} \quad D.x=1,y=-{\textstyle\frac1{\sqrt2}} \quad E. \quad F. \quad G. \quad H.
$$
$$
\mathrm{标准正交向量组中的向量必须是单位向量及向量长度为}1,\mathrm{且两两正交},\mathrm{则可得}x=±1,y=±{\textstyle\frac1{\sqrt2}},
$$



$$
α_1,α_2,α_3\mathrm{是一个标准正交组},则\left\|4α_1-7α_2+4α_3\right\|=\left(\;\;\right).
$$
$$
A.
9 \quad B.81 \quad C.36 \quad D.18 \quad E. \quad F. \quad G. \quad H.
$$
$$
\begin{array}{l}\left\|4α_1-7α_2+4α_3\right\|^2=\left\langle4α_1-7α_2+4α_3,4α_1-7α_2+4α_3\right\rangle\;\;\;\;\;\;\;\;\;\;\\\;\;\;\;\;\;\;\;\;\;\;\;\;\;\;\;\;\;\;\;\;\;\;\;\;\;\;\;\;=4^2+\left(-7\right)^2+4^2\;=81\;\;\;\;\;\;\;\;\;\;\;\;\;\;\;\;\;\;\;,\;\;\\故\;\left\|4α_1-7α_2+4α_3\right\|=\sqrt{81}\;\;=9\;\end{array}
$$



$$
设A\mathrm{为正交矩阵},且A^T=-A^*,\mathrm{其中}A^* 是A\mathrm{的伴随矩阵},则A\mathrm{的行列式等于}(\;\;\;\;).
$$
$$
A.
1 \quad B.-1 \quad C.0 \quad D.±1 \quad E. \quad F. \quad G. \quad H.
$$
$$
\begin{array}{l}A\mathrm{为正交矩阵},则AA^T=E,又A^T=-A^*,则AA^*=-E;\;\;\\\mathrm{由伴随矩阵的性质可知}AA^*=\left|A\right|E,即\left|A\right|E=-E⇒\left|A\right|=-1.\end{array}
$$



$$
设n\mathrm{元向量组}α_1,α_2,⋯,α_m\mathrm{是正交向量组},则m与n\mathrm{的大小关系为}(\;\;\;\;).
$$
$$
A.
m\leq n \quad B.m\geq n \quad C.m=n \quad D.m>n \quad E. \quad F. \quad G. \quad H.
$$
$$
\mathrm{因为}n\mathrm{元向量组}α_1,α_2,⋯,α_m\mathrm{是正交向量组},\mathrm{所以向量组}α_1,α_2,⋯,α_m\mathrm{是线性无关的向量组},\mathrm{因此}m\leq n,\mathrm{因为}n+1个n\mathrm{维的向量必线性相关}.
$$



$$
设A,B\mathrm{均是}m\mathrm{阶正交矩阵},且\left|A\right|=1,\left|B\right|=-1,则\left|A+B\right|\mathrm{的值为}(\;\;\;\;).
$$
$$
A.
1 \quad B.-1 \quad C.±1 \quad D.0 \quad E. \quad F. \quad G. \quad H.
$$
$$
\begin{array}{l}\left|A+B\right|=\left|AB^TB+AA^TB\right|=\left|A\right|\left|B^T+A^T\right|\left|B\right|=\left|\mathbf A\right|\left|\mathbf B\boldsymbol+\mathbf A\right|\left|\mathbf B\right|=-\left|A+B\right|,\\故\left|A+B\right|=0\end{array}
$$



$$
\mathrm{下列矩阵中不是正交矩阵的为}(\;\;\;\;).
$$
$$
A.
\begin{pmatrix}\textstyle\frac{\sqrt3}3&\textstyle\frac{\sqrt3}3&\textstyle\frac{\sqrt3}3\\0&-{\textstyle\frac1{\sqrt2}}&\textstyle\frac1{\sqrt2}\\-{\textstyle\frac2{\sqrt6}}&\textstyle\frac1{\sqrt6}&\textstyle\frac1{\sqrt6}\end{pmatrix} \quad B.\begin{pmatrix}1&0&0&0\\0&\textstyle\frac1{\sqrt3}&\textstyle\frac1{\sqrt2}&0\\0&\textstyle\frac1{\sqrt3}&0&1\\0&\textstyle\frac1{\sqrt3}&-{\textstyle\frac1{\sqrt2}}&0\end{pmatrix} \quad C.\begin{pmatrix}\textstyle\frac12&\textstyle\frac12&\textstyle\frac12&\textstyle\frac12\\\textstyle\frac12&-{\textstyle\frac56}&\textstyle\frac16&\textstyle\frac16\\\textstyle\frac12&\textstyle\frac16&\textstyle\frac16&-{\textstyle\frac56}\\\textstyle\frac12&\textstyle\frac16&-{\textstyle\frac56}&\textstyle\frac16\end{pmatrix} \quad D.\begin{pmatrix}\textstyle\frac67&\textstyle\frac27&\textstyle\frac37\\\textstyle\frac27&\textstyle\frac37&-{\textstyle\frac67}\\-{\textstyle\frac37}&\textstyle\frac67&\textstyle\frac27\end{pmatrix} \quad E. \quad F. \quad G. \quad H.
$$
$$
\mathrm{因为}\begin{pmatrix}1&0&0&0\\0&\textstyle\frac1{\sqrt3}&\textstyle\frac1{\sqrt2}&0\\0&\textstyle\frac1{\sqrt3}&0&1\\0&\textstyle\frac1{\sqrt3}&-{\textstyle\frac1{\sqrt2}}&0\end{pmatrix}\mathrm{的第}2\mathrm{列与第}4\mathrm{列不正交},\mathrm{故不是正交矩阵}.
$$



$$
\mathrm{下列矩阵中不是正交矩阵的是}(\;\;\;\;).
$$
$$
A.
\begin{pmatrix}0&1&0\\\textstyle\frac1{\sqrt2}&0&\textstyle\frac1{\sqrt2}\\\textstyle\frac1{\sqrt2}&0&\textstyle-\frac1{\sqrt2}\end{pmatrix} \quad B.\begin{pmatrix}\textstyle\frac1{\sqrt3}&\textstyle\frac1{\sqrt3}&\textstyle\frac1{\sqrt3}\\-{\textstyle\frac2{\sqrt6}}&\textstyle\frac1{\sqrt6}&\textstyle\frac1{\sqrt6}\\0&-\frac1{\sqrt2}&\textstyle\frac1{\sqrt2}\end{pmatrix} \quad C.\begin{pmatrix}1&-{\textstyle\frac12}&\textstyle\frac13\\-{\textstyle\frac12}&1&\textstyle\frac12\\\textstyle\frac13&\textstyle\frac12&-1\end{pmatrix} \quad D.\begin{pmatrix}\textstyle\frac19&-{\textstyle\frac89}&-{\textstyle\frac49}\\-{\textstyle\frac89}&\textstyle\frac19&-{\textstyle\frac49}\\-{\textstyle\frac49}&-{\textstyle\frac49}&\textstyle\frac79\end{pmatrix} \quad E. \quad F. \quad G. \quad H.
$$
$$
\mathrm{矩阵}\begin{pmatrix}1&-{\textstyle\frac12}&\textstyle\frac13\\-{\textstyle\frac12}&1&\textstyle\frac12\\\textstyle\frac13&\textstyle\frac12&-1\end{pmatrix}\mathrm{中第一个列向量非单位向量},\mathrm{故不是正交阵}.
$$



$$
设A是n\mathrm{阶正交矩阵},且\left|A\right|<0,则\left|A+E\right|=\left(\;\;\right).
$$
$$
A.
0 \quad B.1 \quad C.-1 \quad D.±1 \quad E. \quad F. \quad G. \quad H.
$$
$$
\begin{array}{l}\;\left|A+E\right|=\left|A+AA^T\right|=\left|A\left(E+A^T\right)\right|=\left|A\right|\left|\left(E+A^T\right)\right|=\left|A\right|\left|E+A\right|\\⇒\left(1-\left|A\right|\right)\left|A+E\right|=0⇒\left|A+E\right|=0\left(因1-\left|A\right|>0\right).\end{array}
$$



$$
设A=\begin{pmatrix}1&2&2&0\\2&-1&0&2\\2&0&-1&-2\\0&2&-2&1\end{pmatrix},若B=kA\mathrm{为正交阵},则k=\left(\;\;\;\right)
$$
$$
A.
\textstyle\frac13 \quad B.-{\textstyle\frac13} \quad C.±{\textstyle\frac13} \quad D.±3 \quad E. \quad F. \quad G. \quad H.
$$
$$
\mathrm{矩阵}A\mathrm{的列向量相互正交},\mathrm{当取}k=±{\textstyle\frac1{\sqrt{1+4+4}}}=±{\textstyle\frac13},即B=±{\textstyle\frac13}A时B\mathrm{是正交阵}
$$



$$
\mathrm{已知两个单位正交向量}ζ_1=(\frac19,-{\textstyle\frac89},-{\textstyle\frac49})^T,ζ_2=({\textstyle-}{\textstyle\frac89}{\textstyle,}{\textstyle\frac19}{\textstyle,}{\textstyle-}{\textstyle\frac49})^T,以ζ_1,ζ_2,ζ_3\mathrm{为列向量构成的矩阵}\mathbb{Q}\mathrm{是正交矩阵},\mathrm{则列向量}ζ_3为(\;\;\;).
$$
$$
A.
±(\frac49,{\textstyle\frac49},{\textstyle\frac79})^T \quad B.({\textstyle\frac49}{\textstyle,}{\textstyle\frac49}{\textstyle,}{\textstyle\frac79})^T \quad C.({\textstyle\frac49}{\textstyle,}{\textstyle\frac49}{\textstyle,}{\textstyle-}{\textstyle\frac79})^T \quad D.±({\textstyle\frac49}{\textstyle,}{\textstyle\frac49}{\textstyle,}{\textstyle-}{\textstyle\frac79})^T \quad E. \quad F. \quad G. \quad H.
$$
$$
\begin{array}{l}设ζ_3=\left(x_1,x_2,x_3\right),\mathrm{则由}ζ_1·ζ_3=0,ζ_2·ζ_3=0,得\;\;\\\left\{\begin{array}{l}\textstyle\frac19x_1-\frac89x_2-\frac49x_3=0\\-{\textstyle\frac89}{\textstyle x}{\textstyle{}_1}{\textstyle+}{\textstyle\frac19}{\textstyle x}{\textstyle{}_2}{\textstyle-}{\textstyle\frac49}{\textstyle x}{\textstyle{}_3}{\textstyle=}{\textstyle0}\end{array}\right.,\;\;\\\mathrm{解得}x{\textstyle{}_1}\;=-{\textstyle\frac47}x_3,x{\textstyle{}_2}\;=-{\textstyle\frac47}x_3\;.\\\mathrm{代入}\left|ζ_3\right|^2=x_1^2+x_2^2+x_3^2=1,得x_3=±{\textstyle\frac79},\mathrm{所以}ζ_3=±({\textstyle\frac49}{\textstyle,}{\textstyle\frac49}{\textstyle,}{\textstyle-}{\textstyle\frac79})^T.\end{array}
$$



$$
设A=\begin{pmatrix}a&b&c&x_1\\b&-a&d&x_2\\c&-d&-a&x_3\\d&c&-b&x_4\end{pmatrix},a,b,c,d\mathrm{是不全为}0\mathrm{的实数},若B=kA\mathrm{为正交矩阵},则x_i\left(i=1,2,3,4\right)及k\mathrm{分别为}(\;\;\;).
$$
$$
A.
x_1=d,x_2=-c,x_3=b,x_4=-a,k=±{\textstyle\frac1{\sqrt{a^2+b^2+c^2+d^2}}} \quad B.x_1=-d,x_2=-c,x_3=b,x_4=-a,k=±{\textstyle\frac1{\sqrt{a^2+b^2+c^2+d^2}}} \quad C.x_1=d,x_2=-c,x_3=b,x_4=-a,k={\textstyle\frac1{\sqrt{a^2+b^2+c^2+d^2}}} \quad D.x_1=-d,x_2=-c,x_3=b,x_4=-a,k={\textstyle\frac1{\sqrt{a^2+b^2+c^2+d^2}}} \quad E. \quad F. \quad G. \quad H.
$$
$$
\begin{array}{l}若kA\mathrm{为正交矩阵},\mathrm{则它的行向量成正交向量组},\mathrm{由此有}\;\\ab+b\left(-a\right)+cd+x_1x_2=0\;,cd+x_1x_2=0,\;\;\\\mathrm{同理有}-bd+x_1x_3=0,ad+x_1x_4=0,\;\;\\取x_1=d,x_2=-c,x_3=b,x_4=-a.\;\;\\\mathrm{经验证}:A的4\mathrm{个列向量两两正交},\mathrm{且它们的模都是}\sqrt{a^2+b^2+c^2+d^2},取k=±{\textstyle\frac1{\sqrt{a^2+b^2+c^2+d^2}}},则\\是B=±{\textstyle\frac1{\sqrt{a^2+b^2+c^2+d^2}}}\begin{pmatrix}a&b&c&d\\b&-a&d&-c\\c&-d&-a&b\\d&c&-b&-a\end{pmatrix}\mathrm{正交阵}.\end{array}
$$



$$
\text{方程}x^2+y^2+z^2=0\text{在空间表示()}
$$
$$
A.
\text{球面} \quad B.\text{一点} \quad C.\text{圆锥面} \quad D.\mathrm{圆柱面} \quad E. \quad F. \quad G. \quad H.
$$
$$
\text{由题目可知}x=y=z=0\text{,所以表示的为一点.}
$$



$$
\text{方程}z^2-x^2-y^2=0\text{在空间表示().}
$$
$$
A.
\text{柱面} \quad B.\text{圆锥面} \quad C.\text{旋转双曲面} \quad D.\text{平面} \quad E. \quad F. \quad G. \quad H.
$$
$$
\text{根据圆锥面的定义可知.}
$$



$$
\text{方程}z^2-x^2=1\text{在空间表示}
$$
$$
A.
\text{双曲线} \quad B.\text{母线平行}z\text{轴的双曲柱面} \quad C.\text{母线平行}y\text{轴的双曲柱面} \quad D.\text{母线平行}x\text{轴的双曲柱面} \quad E. \quad F. \quad G. \quad H.
$$
$$
\text{根据双曲柱面的定义可知.}
$$



$$
\text{方程}x^2=2y\text{在空间表示的是().}
$$
$$
A.
\text{抛物线} \quad B.\text{抛物柱面} \quad C.\text{母线平行}x\text{轴的柱面} \quad D.\text{旋转抛物面} \quad E. \quad F. \quad G. \quad H.
$$
$$
\text{由抛物柱面的定义可知.}
$$



$$
\text{方程}x^2+y^2+z^2-2x-4y-4z-7=0\text{表示()曲面.}
$$
$$
A.
\text{以}\left(1,2,2\right)\text{为球心,半径为}4\text{的球面} \quad B.\text{以}\left(1,-2,2\right)\text{为球心,半径为}4\text{的球面} \quad C.\text{以}\left(1,-2,-2\right)\text{为球心,半径为}4\text{的球面} \quad D.\text{以}\left(-1,-2,2\right)\text{为球心,半径为}4\text{的球面} \quad E. \quad F. \quad G. \quad H.
$$
$$
\begin{array}{l}\text{这类缺}xy,yz,zx\text{项的二次方程总可以经配方处理化为球面的标准方程.对本题,将方程写成}\\\left(x-1\right)^2+\left(y-2\right)^2+\left(z-2\right)^2=16\text{, 可见题设方程表示以}\left(1,2,2\right)\text{为球心,半径为4的球面.}\end{array}
$$



$$
\text{方程}x^2+\frac{y^2}9-\frac{z^2}{25}=-1\text{是().}
$$
$$
A.
\text{单叶双曲面} \quad B.\text{双叶双曲面} \quad C.\text{椭球面} \quad D.\text{双曲抛物面} \quad E. \quad F. \quad G. \quad H.
$$
$$
\text{由双叶双曲面的定义可知,形如}\frac{x^2}{a^2}+\frac{y^2}{b^2}-\frac{z^2}{c^2}=-1\text{的曲面称为双叶双曲面.}
$$



$$
\text{方程}x^2+\frac{y^2}9-\frac{z^2}{25}=1\text{是().}
$$
$$
A.
\text{单叶双曲面} \quad B.\text{双叶双曲面} \quad C.\text{椭球面} \quad D.\text{双曲抛物面} \quad E. \quad F. \quad G. \quad H.
$$
$$
\text{由双叶双曲面的定义可知,形如}\frac{x^2}{a^2}+\frac{y^2}{b^2}-\frac{z^2}{c^2}=1\text{的曲面称为单叶双曲面.}
$$



$$
\text{方程}\frac{x^2}4+y^{{}^2}=z^2\text{表示的是().}
$$
$$
A.
\text{二次锥面} \quad B.\text{椭球面} \quad C.\text{双曲面} \quad D.\text{双曲线} \quad E. \quad F. \quad G. \quad H.
$$
$$
\text{由二次锥面的定义可得形如}\frac{x^2}{a^2}+\frac{y^2}{b^2}-\frac{z^2}{c^2}=0\text{的曲面叫二次锥面.}
$$



$$
x^2-\frac{y^2}4+z^2=-1\text{表示的曲面是().}
$$
$$
A.
\text{单叶双曲面} \quad B.\text{双叶双曲面} \quad C.\text{双曲柱面} \quad D.\text{锥面} \quad E. \quad F. \quad G. \quad H.
$$
$$
\text{根据双叶双曲面的定义可知}x^2-\frac{y^2}4+z^2=-1\text{为双叶双曲面.}
$$



$$
\text{方程}x^2=1\text{在空间表示().}
$$
$$
A.
\text{两点} \quad B.\text{母线平行}x\text{轴的柱面} \quad C.\text{平行}yOz\text{平面的两个平面} \quad D.\text{旋转曲面} \quad E. \quad F. \quad G. \quad H.
$$
$$
x^2=1\text{,解得}x=±1\text{,此表示平行}yOz\text{平面的两个平面.}
$$



$$
\text{方程}\frac{y^2}4+\frac{z^2}9=1\text{在空间表示()}
$$
$$
A.
\text{椭圆柱面} \quad B.\text{双曲柱面} \quad C.\text{抛物柱面} \quad D.\text{椭圆} \quad E. \quad F. \quad G. \quad H.
$$
$$
\text{由椭圆柱面定义可知}
$$



$$
\text{方程}\frac{y^2}4\text{-}\frac{z^2}9=1\text{在空间表示()}
$$
$$
A.
\text{双曲线} \quad B.\text{双曲柱面} \quad C.\text{抛物柱面} \quad D.\text{二次锥面} \quad E. \quad F. \quad G. \quad H.
$$
$$
\text{双曲柱面}
$$



$$
\text{方程}\frac{x^2}2+\frac{y^2}4+\frac{z^2}9=1\text{在空间表示().}
$$
$$
A.
\text{二次锥面} \quad B.\text{椭球面} \quad C.\text{单叶双曲面} \quad D.\text{椭圆柱面} \quad E. \quad F. \quad G. \quad H.
$$
$$
\text{是椭球面}
$$



$$
\text{方程}x^2=9z\text{在空间表示().}
$$
$$
A.
\text{二次锥面} \quad B.\text{抛物柱面} \quad C.\text{抛物线} \quad D.\text{双曲柱面} \quad E. \quad F. \quad G. \quad H.
$$
$$
\text{是抛物柱面}
$$



$$
\text{方程}y^2=4x\text{在空间表示()}
$$
$$
A.
\text{母线平行}z\text{的抛物柱面} \quad B.\text{母线平行}x\text{的抛物柱面} \quad C.\text{母线平行}y\text{的抛物柱面} \quad D.\text{母线平行}xoy\text{面的抛物柱面} \quad E. \quad F. \quad G. \quad H.
$$
$$
\text{含两个变量,所以是母线平行}z\text{的抛物柱面}
$$



$$
\text{方程}y^2=9z\text{在空间表示()}
$$
$$
A.
\text{母线平行}y\text{轴的抛物柱面} \quad B.\text{母线平行}z\text{轴的抛物柱面} \quad C.\text{母线平行}yoz\text{面的抛物柱面} \quad D.\text{母线平行}x\text{轴的抛物柱面} \quad E. \quad F. \quad G. \quad H.
$$
$$
\text{母线平行}x\text{轴的抛物柱面}
$$



$$
\text{方程}\frac{x^2}4+\frac{y^2}3-\frac{z^2}5=-1\text{在空间中表示(  ). }
$$
$$
A.
\text{单叶双曲面} \quad B.\text{双曲柱面} \quad C.\text{二次锥面} \quad D.\text{双叶双曲面} \quad E. \quad F. \quad G. \quad H.
$$
$$
\text{平方项有2个负号,所以是双叶双曲面}
$$



$$
\text{方程}\frac{x^2}5+\frac{y^2}3-\frac{z^2}4=0\text{在空间中表示(  ). }
$$
$$
A.
\text{二次锥面} \quad B.\text{单叶双曲面} \quad C.\text{双叶双曲面} \quad D.\text{2条直线} \quad E. \quad F. \quad G. \quad H.
$$
$$
\text{二次锥面}
$$



$$
\text{方程}\frac{x^2}5+\frac{y^2}3-\frac{z^2}4=1\text{在空间中表示(  ). }
$$
$$
A.
\text{双叶双曲面} \quad B.\text{单叶双曲面} \quad C.\text{二次锥面} \quad D.\text{双曲柱面} \quad E. \quad F. \quad G. \quad H.
$$
$$
\text{单叶双曲面}
$$



$$
\text{方程}\frac{x^2}4-\frac{y^2}9=z\text{在空间中表示(  ). }
$$
$$
A.
\text{双曲柱面} \quad B.\text{双曲抛物面(马鞍面)} \quad C.\text{二次锥面} \quad D.\text{单叶双曲面} \quad E. \quad F. \quad G. \quad H.
$$
$$
\text{双曲抛物面(马鞍面)}
$$



$$
\text{方程}\frac{x^2}4-\frac{y^2}9=3\text{在空间表示().}
$$
$$
A.
\text{母线平行}y\text{轴的双曲柱面} \quad B.\text{母线平行}x\text{轴的双曲柱面} \quad C.\text{母线平行}xoy\mathrm 面\text{的双曲柱面} \quad D.\text{母线平行}z\text{轴的双曲柱面} \quad E. \quad F. \quad G. \quad H.
$$
$$
\text{母线平行}z\text{轴的双曲柱面}
$$



$$
\text{方程}\frac{x^2}3-\frac{y^2}5=z\text{在空间表示().}
$$
$$
A.
\text{二次锥面} \quad B.\text{双曲抛物面} \quad C.\text{双曲线} \quad D.\text{单叶双曲面} \quad E. \quad F. \quad G. \quad H.
$$
$$
\text{双曲抛物面}
$$



$$
\text{方程}8x^2+4y^2-z^2=-64\text{在空间中表示().}
$$
$$
A.
\text{单叶双曲面} \quad B.\text{椭球面} \quad C.\text{双叶双曲面} \quad D.\text{二次锥面} \quad E. \quad F. \quad G. \quad H.
$$
$$
\text{化为}-\frac{x^2}8-\frac{y^2}{16}+\frac{z^2}{64}=1\text{,是双叶双曲面}
$$



$$
\text{方程}8x^2+4y^2-3z^2=0\text{在空间中表示().}
$$
$$
A.
\text{单叶双曲面} \quad B.\text{双叶双曲面} \quad C.\text{椭球面} \quad D.\text{二次锥面} \quad E. \quad F. \quad G. \quad H.
$$
$$
\frac{x^2}{\displaystyle\frac38}+\frac{y^2}{\displaystyle\frac34}=z^2\text{,是二次锥面}
$$



$$
\text{方程}8x^2+4y^2-z^2=64\text{在空间中表示().}
$$
$$
A.
\text{双叶双曲面} \quad B.\text{单叶双曲面} \quad C.\text{二次锥面} \quad D.\text{椭球面} \quad E. \quad F. \quad G. \quad H.
$$
$$
\text{正确答案是:单叶双曲面}
$$



$$
\text{方程}8x^2-4y^2+3z^2=0\text{在空间中表示().}
$$
$$
A.
\text{单叶双曲面} \quad B.\text{双叶双曲面} \quad C.\text{二次锥面} \quad D.\text{两条直线} \quad E. \quad F. \quad G. \quad H.
$$
$$
\text{二次锥面}
$$



$$
\begin{array}{l}\text{下列曲面方程在空间中表示柱面的是(  ).}\\\text{(1)}z^2=2x,\text{(2)}z=\sqrt{x^2+y^2}\\\text{(3)}\frac{x^2}4-\frac{y^2}3=2,\text{(4)}x^2+y^2=2\end{array}
$$
$$
A.
\text{(1)(2)} \quad B.\text{(2)(3)} \quad C.(2)(4) \quad D.(1)(3)(4) \quad E. \quad F. \quad G. \quad H.
$$
$$
\mathrm{根据柱面方程的定义},(1)(3)(4)\mathrm{是柱面}.
$$



$$
\text{方程}16x^2+4y^2+z^2=64\text{表示().}
$$
$$
A.
\text{椭球面} \quad B.\text{二次锥面} \quad C.\text{单叶双曲面} \quad D.\text{抛物柱面} \quad E. \quad F. \quad G. \quad H.
$$
$$
\text{椭球面}
$$



$$
\begin{array}{l}\text{下列曲面方程在空间中表示二次锥面的是(  ).}\\\text{(1)}z^2=2x\text{,(2)}z=\sqrt{x^2+2y^2}\text{,}\\\text{(3)}4x^2+3y^2-36z^2=0\text{,(4)}\frac{x^2}2+y^2=3z\\\end{array}
$$
$$
A.
\text{(1)(3)} \quad B.\text{(2)(3)} \quad C.\text{(3)(4)} \quad D.\text{(2)(4)} \quad E. \quad F. \quad G. \quad H.
$$
$$
\text{(2)(3)是二次锥面}
$$



$$
\text{方程}8x^2-4y^2+3z^2=-5\text{在空间中表示().}
$$
$$
A.
\text{单叶双曲面} \quad B.\text{双叶双曲面} \quad C.\text{二次锥面} \quad D.\text{椭球面} \quad E. \quad F. \quad G. \quad H.
$$
$$
\text{可化为}-\frac{8x^2}5+\frac{4y^2}5-\frac{3z^2}5=1\text{,所以是双叶双曲面}
$$



$$
\begin{array}{l}\text{下列曲面方程在空间中表示柱面的是(  ).}\\\text{(1)}z^2=2y\text{,(2)}z=\sqrt{x^2+y^2}\text{,}\\\text{,(3)}\frac{x^2}4-\frac{y^2}3=2\text{,(4)}x^2-y^2=1\\\end{array}
$$
$$
A.
\text{(1)(2)} \quad B.\text{(2)(3)} \quad C.\text{(2)(4)} \quad D.\text{(1)(3)(4)} \quad E. \quad F. \quad G. \quad H.
$$
$$
\text{根据柱面方程的定义(1)(3)(4)是柱面}
$$



$$
\text{方程}\frac{x^2}3+\frac{y^2}5=9z\text{在空间表示().}
$$
$$
A.
\text{椭球面} \quad B.\text{二次锥面} \quad C.\text{椭圆抛物面} \quad D.\text{双叶双曲面} \quad E. \quad F. \quad G. \quad H.
$$
$$
\text{是椭圆抛物面}
$$



$$
\begin{array}{l}\text{下列曲面方程在空间中表示柱面的是(  ).}\\\text{(1)}z^2=2x\text{,(2)}z=\sqrt{x^2+y^2}\text{,}\\\text{(3)}4x^2-4y^2+36z^2=144\text{,(4)}x^2+y^2=2\\\end{array}
$$
$$
A.
\text{(1)(2)} \quad B.\text{(2)(3)} \quad C.\text{(1)(4)} \quad D.\text{(2)(4)} \quad E. \quad F. \quad G. \quad H.
$$
$$
\text{(1)(4)是柱面}
$$



$$
\text{方程}3z=y^2+\frac{x^2}4\text{的图形是(  ). }
$$
$$
A.
\text{椭球面} \quad B.\text{双曲面} \quad C.\text{旋转抛物面} \quad D.\text{椭圆抛物面} \quad E. \quad F. \quad G. \quad H.
$$
$$
\text{是椭圆抛物面}
$$



$$
\text{下面方程在空间中表示柱面的是(  ).}
$$
$$
A.
2x^2+3y^2=z^2 \quad B.2x^2+3y^2=5 \quad C.2x^2+3y^2+z^2=1 \quad D.x^2+\frac{y^2}2-z^2=1 \quad E. \quad F. \quad G. \quad H.
$$
$$
2x^2+3y^2=5\mathrm{是柱面}
$$



$$
\begin{array}{l}\text{下面方程的图形是柱面的是(  ).}\\\text{(1)}\frac{x^2}9+\frac{z^2}4=1\text{,(2)}\frac{x^2}9+\frac{y^2}4+z^2=1,\\\text{(3)}-\frac{x^2}9+\frac{z^2}4=1\text{,(4)},y^2-z=0\end{array}
$$
$$
A.
\text{(2)(3)} \quad B.\text{(1)(2)} \quad C.\text{(1)(3)(4)} \quad D.\text{(2)(4)} \quad E. \quad F. \quad G. \quad H.
$$
$$
\text{(1)(3)(4)}
$$



$$
\text{方程}16x^2+4y^2-z^2=64\text{表示().}
$$
$$
A.
\text{锥面} \quad B.\text{单叶双曲面} \quad C.\text{双叶双曲面} \quad D.\text{椭圆抛物面} \quad E. \quad F. \quad G. \quad H.
$$
$$
\text{方程变形为}\frac{x^2}4+\frac{y^2}{16}-\frac{z^2}{64}=1\text{,为单叶双曲面}
$$



$$
\text{方程}\frac{x^2}4-\frac{y^2}9+\frac{z^2}5=0\text{在空间表示().}
$$
$$
A.
\text{单叶双曲面} \quad B.\text{双叶双曲面} \quad C.\text{双曲抛物面} \quad D.\text{二次锥面} \quad E. \quad F. \quad G. \quad H.
$$
$$
\text{二次锥面}
$$



$$
\text{方程}\frac{x^2}2\text{+}\frac{y^2}2\text{-}\frac{z^2}3\text{=0表示旋转曲面,它的旋转轴是 ()}
$$
$$
A.
\text{x轴} \quad B.\text{y轴} \quad C.\text{z轴} \quad D.\text{直线x=y=z} \quad E. \quad F. \quad G. \quad H.
$$
$$
\text{根据旋转曲面的特征来判断,因为}x^2,y^2\text{的系数相同,与}z^2\text{的系数不同}
$$



$$
\text{xOy平面上曲线}x^2-4y^2=9\text{绕y轴旋转一周所得旋转曲面方程为().}
$$
$$
A.
-x^2+z^2-4y^2=9 \quad B.-x^2+z^2+4y^2=9 \quad C.x^2-z^2-4y^2=9 \quad D.x^2+z^2-4y^2=9 \quad E. \quad F. \quad G. \quad H.
$$
$$
\text{由曲面的定义可知旋转曲面的方程为}x^2+z^2-4y^2=9.
$$



$$
\text{将}xOz\text{坐标面上的抛物线}z^2=5x\text{绕}x\text{轴旋转一周所生成的旋转曲面的方程为().}
$$
$$
A.
y^2+z^2=5x \quad B.y^2+z^2=-5x \quad C.y^2-z^2=5x \quad D.y^2-z^2=-5x \quad E. \quad F. \quad G. \quad H.
$$
$$
\begin{array}{l}\text{对方程}z^2=5x,x\text{不变,将}z\text{改成}\left(±\sqrt{y^2+z^2}\right)\text{,于是得所求旋转曲面的方程为:}\left(±\sqrt{y^2+z^2}\right)^2=5x\text{,即}y^2+z^2=5x\text{.}\\\text{此为旋转抛物面的方程.}\end{array}
$$



$$
\text{将}xOz\text{坐标面上的圆}x^2+z^2=9\text{绕}z\text{轴旋转一周,则所生成的旋转曲面的方程为().}
$$
$$
A.
x^2+y^2+z^2=9 \quad B.x^2+y^2+z^2=8 \quad C.x^2+y^2+z^2=16 \quad D.-x^2-y^2+z^2=9 \quad E. \quad F. \quad G. \quad H.
$$
$$
\begin{array}{l}\text{在方程}x^2+z^2=9\text{中,}z\text{不变,将}x\text{改成}\left(±\sqrt{x^2+y^2}\right)\text{,得所求旋转曲面的方程为:}\\\left(±\sqrt{x^2+y^2}\right)^2+z^2=9\text{,即}x^2+y^2+z^2=9\text{,}\\\text{显见这是以原点为球心,半径为3的球面方程.}\end{array}
$$



$$
\text{以曲线}Γ\text{:}\left\{\begin{array}{l}f\left(y,z\right)=0\\x=0\end{array}\right.\text{为母线,以}Oz\text{轴为旋转轴的旋转曲面的方程是().}
$$
$$
A.
f\left(±\sqrt{x^2+y^2},z\right)=0 \quad B.f\left(\sqrt{x^2+y^2},z\right)=0 \quad C.f\left(\sqrt{x^2+y^2},x\right)=0 \quad D.f\left(±\sqrt{x^2+y^2},y\right)=0 \quad E. \quad F. \quad G. \quad H.
$$
$$
\text{由题意可得:曲线以}Oz\text{轴为旋转轴的旋转曲面的方程为:}f\left(±\sqrt{x^2+y^2},z\right)=0
$$



$$
zOx\text{平面上曲线}z^{2=}5x\text{绕}z\text{轴旋转而成的旋转曲面方程为().}
$$
$$
A.
z^4=25\left(x^2+y^2\right) \quad B.z^4=25\left(x^2-y^2\right) \quad C.z^4=5\left(x^2+y^2\right) \quad D.z=25\left(x^2+y^2\right) \quad E. \quad F. \quad G. \quad H.
$$
$$
\text{原方程}z\text{不变,}x\text{变成}z^2=±5\sqrt{x^2+y^2}\text{,即}z^4=25\left(x^2+y^2\right)
$$



$$
\text{曲面}z=\sqrt{x^2+y^2}\text{是().}
$$
$$
A.
zOx\text{平面上曲线}z=x\text{绕}z\text{轴旋转而成的旋转曲面} \quad B.yOz\text{平面上曲线}z=\left|y\right|\text{绕}z\text{轴旋转而成的旋转曲面} \quad C.zOx\text{平面上曲线}z=x\text{绕}x\text{轴旋转而成的旋转曲面} \quad D.yOz\text{平面上曲线}z=\left|y\right|\text{绕}y\text{轴旋转而成的旋转曲面} \quad E. \quad F. \quad G. \quad H.
$$
$$
\text{图面为二次锥面,为}zOy\text{平面上曲线}z=\left|y\right|\text{绕}z\text{轴旋转而成的旋转曲面}
$$



$$
\text{将}xOz\text{坐标面上的曲线}x^2-z^2=9\text{绕}x\text{轴旋转一周,则生成的旋转曲面方程为().}
$$
$$
A.
x^2+y^2-z^2=9 \quad B.x^2-y^2-z^2=9 \quad C.-x^2-y^2+z^2=9 \quad D.x^2-y^2+z^2=-9 \quad E. \quad F. \quad G. \quad H.
$$
$$
\text{根据旋转曲面的方程的定义知所求曲面为}x^2-y^2-z^2=9
$$



$$
\text{将}xOz\text{坐标面上的圆}x^2+z^2=9\text{绕}z\text{轴旋转一周,则生成的旋转曲面方程为().}
$$
$$
A.
x^2-y^2+z^2=9 \quad B.-x^2-y^2+z^2=9 \quad C.x^2+y^2+z^2=-9 \quad D.x^2+y^2+z^2=9 \quad E. \quad F. \quad G. \quad H.
$$
$$
\text{根据旋转曲面的定义得:}x^2+y^2+z^2=9
$$



$$
\text{将}yoz\text{坐标面上的曲线}y^2-3z^2=8\text{绕}y\text{轴旋转一周,则生成的旋转曲面方程为(  ). }
$$
$$
A.
3x^2+y^2-3z^2=8 \quad B.3x^2-y^2+3z^2=-8 \quad C.3x^2+y^2+3z^2=8 \quad D.-3x^2-y^2-3z^2=8 \quad E. \quad F. \quad G. \quad H.
$$
$$
\text{根据旋转曲面的定义知旋转曲面的方程为:}3x^2-y^2+3z^2=-8
$$



$$
\text{将x}oz\text{坐标面上的直线}z=3x\text{绕}z\text{轴旋转一周,则生成的旋转曲面方程为(  ). }
$$
$$
A.
9x^2+9y^2=z^2 \quad B.3x^2+3y^2=z^2 \quad C.9z^2+9y^2=x^2 \quad D.9x^2+9y^2=-z^2 \quad E. \quad F. \quad G. \quad H.
$$
$$
\text{根据旋转曲面的定义知旋转曲面的方程为:}9x^2+9y^2=z^2
$$



$$
\text{将坐标面}yoz\text{上的曲线}f\left(y,z\right)=0\text{绕}y\text{轴旋转一周得到的旋转曲面方程是().}
$$
$$
A.
f\left(y,\sqrt{x^2+z^2}\right)=0 \quad B.f\left(y,±\sqrt{y^2+z^2}\right)=0 \quad C.f\left(y,±\sqrt{x^2+z^2}\right)=0 \quad D.f\left(±\sqrt{x^2+z^2},z\right)=0 \quad E. \quad F. \quad G. \quad H.
$$
$$
\text{根据旋转曲面的定义得: 旋转一周得到的旋转曲面方程}f\left(y,±\sqrt{x^2+z^2}\right)=0
$$



$$
\text{将}xOy\text{坐标面上的双曲线}4x^2-9y^2=36\text{绕}y\text{轴旋转一周,则所生成的旋转曲面的方程为().}
$$
$$
A.
4x^2-9y^2+4z^2=36 \quad B.4x^2-9y^2-4z^2=36 \quad C.-4x^2+9y^2+4z^2=36 \quad D.-4x^2+9y^2-4z^2=36 \quad E. \quad F. \quad G. \quad H.
$$
$$
\text{绕}y\text{轴旋转一周生成的曲面方程为}4x^2-9y^2+4z^2=36
$$



$$
\text{求}xOy\text{平面上曲线}\frac{x^2}9+\frac{y^2}4=1\text{绕}x\text{轴旋转而成的旋转曲面方程为(  ). }
$$
$$
A.
\frac{x^2}9+\frac{y^2}4+\frac{z^2}9=1 \quad B.\frac{x^2}9+\frac{y^2}4-\frac{z^2}4=1 \quad C.\frac{x^2}9+\frac{y^2}4+\frac{z^2}4=1 \quad D.\frac{x^2}4+\frac{y^2}9+\frac{z^2}4=1 \quad E. \quad F. \quad G. \quad H.
$$
$$
\text{根据旋转曲面的定义得:}\frac{x^2}9+\frac{y^2}4+\frac{z^2}4=1
$$



$$
\text{下列各曲线中,绕}y\text{轴旋转而成椭球面}3x^2+2y^2+3z^2=1\text{的曲线是().}
$$
$$
A.
\left\{\begin{array}{l}2x^2+3y^2=1\\y=0\end{array}\right. \quad B.\left\{\begin{array}{l}3y^2+2z^2=1\\x=0\end{array}\right. \quad C.\left\{\begin{array}{l}3x^2+2y^2=1\\z=0\end{array}\right. \quad D.\left\{\begin{array}{l}3x^2+3z^2=1\\y=0\end{array}\right. \quad E. \quad F. \quad G. \quad H.
$$
$$
3x^2+2y^2+3z^2=1\text{在}yOx\text{上的投影为}\left\{\begin{array}{l}3x^2+2y^2=1\\z=0\end{array}\right.\text{,此线绕}y\text{轴旋转即成椭球面}3x^2+2y^2+3z^2=1
$$



$$
\text{旋转曲面}\frac{x^2}9-\frac{y^2}9-\frac{z^2}9=1\text{的旋转轴是(  ). }
$$
$$
A.
x\text{轴} \quad B.y\text{轴} \quad C.\boldsymbol z\text{轴} \quad D.\text{直线}x=y=z \quad E. \quad F. \quad G. \quad H.
$$
$$
x\text{轴}
$$



$$
\text{坐标面}yoz\text{上的直线}z=y+a\text{绕}z\text{轴旋转一周得到的旋转曲面方程为(  ). }
$$
$$
A.
\left(z-a\right)^2=x^2-y^2 \quad B.z^2=x^2+\left(y+a\right)^{{}^2} \quad C.\left(z+a\right)^2=x^2+y^2 \quad D.\left(z-a\right)^2=x^2+y^2 \quad E. \quad F. \quad G. \quad H.
$$
$$
\text{根据旋转曲面的定义知:}\left(z-a\right)^2=x^2+y^2
$$



$$
\text{将}xoz\text{坐标面上的曲线}z^2=6x\text{绕}x\mathrm{轴旋转一周},\mathrm{则生成的旋转曲面的方程为}\;(\;\;)\;
$$
$$
A.
6y^2+z^2-6x=0 \quad B.-y^2-z^2-6x=0 \quad C.y^2+z^2-6x=0 \quad D.y^2-z^2-6x=0 \quad E. \quad F. \quad G. \quad H.
$$
$$
\text{根据旋转曲面的定义知所求旋转曲面的方程为:}y^2+z^2-6x=0
$$



$$
\text{旋转曲面}x^2-y^2-z^2=1\text{是().}
$$
$$
A.
xOy\text{平面上的双曲线绕}x\mathrm{轴旋转所得} \quad B.xOz\text{平面上的双曲线绕}z\text{轴旋转所得} \quad C.xOy\text{平面上的椭圆绕}x\text{轴旋转所得} \quad D.xOz\text{平面上的椭圆绕}x\text{轴旋转所得} \quad E. \quad F. \quad G. \quad H.
$$
$$
\begin{array}{l}x^2-y^2-z^2=1\text{变形为}y^2+z^2-x^2=-1\text{,}\\\text{此为双叶双曲面,可看作是}xOy\text{平面上的双曲线绕}x\text{轴旋转所得.}\end{array}
$$



$$
\text{曲面}\frac{x^2}4+\frac{y^2}9+\frac{z^2}9=1\text{是().}
$$
$$
A.
\text{球面} \quad B.xOy\text{平面上的曲线}\frac{x^2}4+\frac{y^2}9=1\text{绕}y\text{轴旋转而成旋转椭球面} \quad C.xOz\text{平面上的曲线}\frac{x^2}4+\frac{z^2}9=1\text{绕x轴旋转而成旋转椭球面} \quad D.\text{柱面} \quad E. \quad F. \quad G. \quad H.
$$
$$
\text{曲面方程为}\frac{x^2}4+\frac{y^2+z^2}9=1\text{,是}xOz\text{平面上的曲线}\frac{x^2}4+\frac{z^2}9=1\text{绕}x\text{轴旋转而成的椭球面.}
$$



$$
\text{曲面}2x^2+4y^2+4z^2=1\text{是().}
$$
$$
A.
\text{球面} \quad B.xOy\text{平面上曲线}2x^2+4y^2=1\text{绕}y\text{轴旋转而成} \quad C.\mathrm{柱面} \quad D.xOz\text{平面上曲线}2x^2+4z^2=1\text{绕}x\text{轴旋转而成} \quad E. \quad F. \quad G. \quad H.
$$
$$
\text{曲面是二次锥面,是}xOz\text{平面上曲线}2x^2+4z^2=1\text{绕}x\text{轴旋转而成.}
$$



$$
\text{将}xOy\text{坐标面上的双曲线}4x^2-9y^2=36\text{绕}x\text{轴旋转一周,则所生成的旋转曲面的方程为().}
$$
$$
A.
4x^2-9y^2-9z^2=36 \quad B.4x^2+9y^2+9z^2=36 \quad C.4x^2+9y^2-4z^2=36 \quad D.x^2+9y^2+4z^2=36 \quad E. \quad F. \quad G. \quad H.
$$
$$
\text{绕}x\text{轴旋转一周生成的曲面方程为 }4x^2-9y^2-9z^2=36
$$



$$
\text{下列关于旋转曲面}\frac{x^2}4+\frac{y^2}9+\frac{z^2}9=1\text{形成叙述正确的是().}
$$
$$
A.
\frac{x^2}4+\frac{z^2}9=1\text{绕}x\text{轴旋转一周而成的旋转椭球面} \quad B.\frac{x^2}4+\frac{z^2}9=1\text{绕}y\text{轴旋转一周而成的旋转椭球面} \quad C.\frac{x^2}4-\frac{z^2}9=1\text{绕}x\text{轴旋转一周而成的旋转椭球面} \quad D.\frac{x^2}4+\frac{z^2}9=1\text{绕z轴旋转一周而成的旋转椭球面} \quad E. \quad F. \quad G. \quad H.
$$
$$
\begin{array}{l}\text{将方程改写为}\frac{x^2}4+\frac{y^2+z^2}9=1\text{,可看作}\frac{x^2}4+\frac{y^2}9=1\text{绕}x\text{轴旋转一周而成的旋转椭球面;或}\\\frac{x^2}4+\frac{z^2}9=1\text{绕}x\text{轴旋转一周而成的旋转椭球面.}\end{array}
$$



$$
\text{关于旋转曲面}x^2-\frac{y^2}4+z^2=1\text{形成叙述正确的是()}
$$
$$
A.
\text{双曲线}x^2-\frac{y^2}4=1\text{绕}y\text{轴旋转一周而成的旋转单叶双曲面} \quad B.\text{双曲线}x^2+\frac{y^2}4=1\text{绕}y\text{轴旋转一周而成的旋转单叶双曲面} \quad C.\text{双曲线}x^2-\frac{y^2}4=1\text{绕}x\text{轴旋转一周而成的旋转单叶双曲面} \quad D.\text{双曲线}x^2-\frac{y^2}4=1\text{绕}z\text{轴旋转一周而成的旋转单叶双曲面} \quad E. \quad F. \quad G. \quad H.
$$
$$
\text{原方程改写为}\left(x^2+z^2\right)-\frac{y^2}4=1\text{,可看作}xOy\text{平面上的双曲线}x^2-\frac{y^2}4=1\text{绕}y\text{轴旋转一周而成的旋转}\mathrm{单叶双曲面}
$$



$$
\text{旋转曲面}x^2-y^2-z^2=1\text{是().}
$$
$$
A.
xOy\text{平面上的双曲线}x^2-y^2=1\text{绕}y\text{轴旋转所得} \quad B.xOz\text{平面上的双曲线}x^2-z^2=1\text{绕}z\text{轴旋转所得} \quad C.xOy\text{平面上的双曲线}x^2-y^2=1\text{绕}x\text{轴旋转所得} \quad D.xOy\text{平面上的椭圆}x^2+y^2=1\text{绕}x\text{轴旋转所得} \quad E. \quad F. \quad G. \quad H.
$$
$$
x^2-y^2-z^2=1\text{确定的曲面为双叶双曲面,可看作是}xOy\text{平面上的双曲线}x^2-y^2=1\text{绕}x\text{轴旋转所得.}
$$



$$
\text{旋转曲面}x^2-y^2-z^2=1\text{是().}
$$
$$
A.
xOy\text{平面上的双曲线绕}x\text{轴旋转所得} \quad B.xOy\text{平面上的双曲线绕}z\text{轴旋转所得} \quad C.xOy\text{平面上的椭圆绕}x\text{轴旋转所得} \quad D.xOy\text{平面上的椭圆绕y轴旋转所得} \quad E. \quad F. \quad G. \quad H.
$$
$$
x^2-y^2-z^2=1\text{变形为}y^2+z^2-x^2=-1\text{,  此为双叶双曲面,可看作是}xOy\text{平面上的双曲线绕}x\text{轴旋转所得.}
$$



$$
\text{曲面}x^2+y^2=9z^2\text{是().}
$$
$$
A.
\text{球面} \quad B.xOy\text{平面上曲线}x^2=9y^2\text{绕}x\text{轴旋转而成的} \quad C.xOz\text{平面上曲线}x^2=9z^2\text{绕}y\text{轴旋转而成的} \quad D.yOz\text{平面上曲线y}=3z\text{绕}z\text{轴旋转而成的} \quad E. \quad F. \quad G. \quad H.
$$
$$
\text{该曲面是圆锥面,是由}yOz\text{平面上曲线}y=3z\text{绕}z\text{轴旋转而成的.}
$$



$$
xOy\text{平面上曲线}x^2-4y^2=9\text{绕}x\text{轴旋转一周所得旋转曲面方程为()}
$$
$$
A.
x^2-4y^2+4z^2=9 \quad B.x^2-4y^2-z^2=9 \quad C.x^2-4y^2-4z^2=9 \quad D.x^2-4y^2+z^2=9 \quad E. \quad F. \quad G. \quad H.
$$
$$
xOy\text{平面上曲线}x^2-4y^2=9\text{绕}x\text{轴旋转,}x\text{不变,把}y^2\text{变为}y^2+z^2\text{,即得旋转曲面方程}x^2-4y^2-4z^2=9
$$



$$
\mathrm{双曲面}x^2-y^2/4-z^2/9=1\mathrm{与平面}y=4\mathrm{交线为}(\;).
$$
$$
A.
\mathrm{双曲线} \quad B.\mathrm{椭圆} \quad C.\mathrm{抛物线} \quad D.\mathrm{一对相交直线} \quad E. \quad F. \quad G. \quad H.
$$
$$
将y=4\mathrm{代入到双曲面方程得}x^2-z^2/9=5,\mathrm{所以相交的曲线为双曲线}.
$$



$$
\mathrm{双曲面}x^2-y^2/4-z^2/9=1与yOz\mathrm{平面}(\;).
$$
$$
A.
\mathrm{交于一双曲线} \quad B.\mathrm{交于一对相交直线} \quad C.\mathrm{不交} \quad D.\mathrm{交于一椭圆} \quad E. \quad F. \quad G. \quad H.
$$
$$
\mathrm{平面}yOz是x=0,\mathrm{代入到双曲面}x^2-y^2/4-z^2/9=1\mathrm{中等式不成立},\mathrm{所以双曲面与平面}yOz\mathrm{不相交}.
$$



$$
\mathrm{锥面}x^2+y^2/25=z^2/16与xoy\mathrm{平面的交线为}(\;).
$$
$$
A.
\mathrm{一对相交直线} \quad B.\mathrm{一点} \quad C.\mathrm{椭圆} \quad D.\mathrm{双曲线} \quad E. \quad F. \quad G. \quad H.
$$
$$
xOy\mathrm{平面为}z=0\mathrm{代入}x^2+y^2/25=z^2/16\mathrm{锥面得}x=y=0,\mathrm{所以相交于原点}.
$$



$$
\mathrm{曲面}x^2-y^2=z在xOz\mathrm{平面上的投影曲线是}\;(\;).
$$
$$
A.
x^2=z \quad B.\left\{\begin{array}{l}y^2=-z\\x=0\end{array}\right. \quad C.\left\{\begin{array}{l}x^2-y^2=0\\z=0\end{array}\right. \quad D.\left\{\begin{array}{l}x^2=z\\y=0\end{array}\right. \quad E. \quad F. \quad G. \quad H.
$$
$$
xOz\mathrm{平面为}y=0,\mathrm{代入曲面方程得}x^2=z,\mathrm{所以题目所求方程为}\left\{\begin{array}{l}x^2=z\\y=0\end{array}\right..
$$



$$
\mathrm{方程}\left\{\begin{array}{l}x^2+y^2+z^2=25\\x=3\end{array}\right.\mathrm{所表示的曲线是}(\;).\;
$$
$$
A.
\mathrm{曲线是平面}x=3\mathrm{上的一个圆}y^2+z^2=16 \quad B.\mathrm{曲线是平面}x=3\mathrm{上的一个圆}y^2+z^2=8 \quad C.\mathrm{曲线是平面}x=-3\mathrm{上的一个圆}y^2+z^2=16 \quad D.\mathrm{曲线是平面}x=3\mathrm{上的一个圆}y^2+z^2=9 \quad E. \quad F. \quad G. \quad H.
$$
$$
\begin{array}{l}\mathrm{曲线是平面}x=3\mathrm{上的一个圆}y^2+z^2=16,\\\mathrm{它由平面}x=3\mathrm{截球面}x^2+y^2+z^2=25\mathrm{而得},\mathrm{圆心在点}(3,0,0),\mathrm{半径为}4.\end{array}
$$



$$
\mathrm{方程}\left\{\begin{array}{l}x^2+4y^2+9z^2=36\\y=1\end{array}\right.\mathrm{所表示的曲线为}(\;).\;
$$
$$
A.
\mathrm{曲线是平面}y=1\mathrm{截椭球面}x^2+4y^2+9z^2=36\mathrm{而得的椭圆}\left\{\begin{array}{l}x^2+9z^2=32\\y=1\end{array}\right. \quad B.\mathrm{曲线是平面}y=1\mathrm{截椭球面}x^2+4y^2+9z^2=36\mathrm{而得的椭圆}\left\{\begin{array}{l}x^2+9z^2=16\\y=-1\end{array}\right. \quad C.\mathrm{曲线是平面}y=1\mathrm{截椭球面}x^2+4y^2+9z^2=36\mathrm{而得的椭圆}\left\{\begin{array}{l}x^2+9z^2=16\\y=1\end{array}\right. \quad D.\mathrm{曲线是平面}y=1\mathrm{截椭球面}x^2+4y^2+9z^2=36\mathrm{而得的椭圆}\left\{\begin{array}{l}x^2+3z^2=32\\y=1\end{array}\right. \quad E. \quad F. \quad G. \quad H.
$$
$$
\begin{array}{l}\mathrm{曲线是平面}y=1\mathrm{截椭球面}x^2+4y^2+9z^2=36\mathrm{而得的椭圆}\left\{\begin{array}{l}x^2+9z^2=32\\y=1\end{array}\right..\\\mathrm{它在平面上}y=1,\mathrm{椭圆中心在点}(0,1,0)处,长,\mathrm{短半轴分别与}x\mathrm{轴和}y\mathrm{轴平行},\mathrm{其长分别为}4\sqrt[{}]2与4/3\sqrt2.\end{array}
$$



$$
\mathrm{方程}\left\{\begin{array}{l}x^2-4y^2+z^2=25\\x=-3\end{array}\right.\mathrm{所表示的曲线是}(\;).\;
$$
$$
A.
\mathrm{曲线是平面}x=-3\mathrm{截单叶双曲面}x^2-4y^2+z^2=25\mathrm{而得的双曲线}\left\{\begin{array}{l}z^2-4y^2=16\\x=-3\end{array}\right. \quad B.\mathrm{曲线是平面}x=3\mathrm{截单叶双曲面}x^2-4y^2+z^2=25\mathrm{而得的双曲线}\left\{\begin{array}{l}z^2-4y^2=16\\x=3\end{array}\right. \quad C.\mathrm{曲线是平面}x=-3\mathrm{截单叶双曲面}x^2-4y^2+z^2=25\mathrm{而得的双曲线}\left\{\begin{array}{l}z^2-4y^2=25\\x=-3\end{array}\right. \quad D.\mathrm{曲线是平面}x=-3\mathrm{截单叶双曲面}x^2-4y^2+z^2=25\mathrm{而得的双曲线}\left\{\begin{array}{l}z^2-4y^2=5\\x=-3\end{array}\right. \quad E. \quad F. \quad G. \quad H.
$$
$$
\begin{array}{l}\mathrm{曲线是平面}x=-3\mathrm{截单叶双曲面}x^2-4y^2+z^2=25\mathrm{而得的双曲线}\left\{\begin{array}{l}z^2-4y^2=16\\x=-3\end{array}\right..\;\;\\\mathrm{它在平面}x=-3上,\mathrm{双曲线中心在点}(-3,0,0)处,\mathrm{实轴与虚轴分别平行于}x\mathrm{轴和}y轴,\mathrm{半实轴长为}4,\mathrm{半虚轴长为}2.\end{array}
$$



$$
\mathrm{双曲面}x^2/9+y^2/16-z^2/49=1与y=4\mathrm{交线为}(\;).\;
$$
$$
A.
\mathrm{双曲线} \quad B.\mathrm{椭圆} \quad C.\mathrm{抛物线} \quad D.\mathrm{一对相交直线} \quad E. \quad F. \quad G. \quad H.
$$
$$
将y=4\mathrm{代入到}x^2/9+y^2/16-z^2/49=1\mathrm{中得}x=±\frac37z,\mathrm{所以相交的曲线为一对相交直线}.
$$



$$
\mathrm{锥面}x^2+y^2/16=z^2\mathrm{与平面}yOz\mathrm{的交线为}(\;).
$$
$$
A.
\mathrm{椭圆} \quad B.\mathrm{双曲线} \quad C.\mathrm{一对相交直线} \quad D.\mathrm{一点} \quad E. \quad F. \quad G. \quad H.
$$
$$
yOz\mathrm{平面为}x=0,\mathrm{代入到锥面方程中得}y=±4z\mathrm{为一对相交直线}
$$



$$
\mathrm{曲面}x^2+y^2+z^2=25与z=3\mathrm{的交线为}(\;).
$$
$$
A.
圆 \quad B.\mathrm{椭圆} \quad C.点 \quad D.\mathrm{两条直线} \quad E. \quad F. \quad G. \quad H.
$$
$$
\mathrm{是圆}
$$



$$
\mathrm{曲面}x^2+4y^2+9z^2=36与y=1\mathrm{的交线为}(\;).
$$
$$
A.
圆 \quad B.\mathrm{椭圆} \quad C.点 \quad D.\mathrm{两条直线} \quad E. \quad F. \quad G. \quad H.
$$
$$
\mathrm{椭圆}
$$



$$
\mathrm{曲面}x^2-4y^2+z^2=25与z=-3\mathrm{的交线为}(\;).
$$
$$
A.
圆 \quad B.\mathrm{椭圆} \quad C.\mathrm{双曲线} \quad D.\mathrm{两条直线} \quad E. \quad F. \quad G. \quad H.
$$
$$
\mathrm{双曲线}
$$



$$
\mathrm{曲面}y^2+z^2-4x+8=0与y=4\mathrm{的交线为}(\;).
$$
$$
A.
\mathrm{椭圆} \quad B.\mathrm{双曲线} \quad C.\mathrm{抛物线} \quad D.\mathrm{两条直线} \quad E. \quad F. \quad G. \quad H.
$$
$$
\mathrm{抛物线}
$$



$$
\mathrm{曲面}9x^2+9y^2=10z与yOz\mathrm{平面的交线为}(\;).\;
$$
$$
A.
\left\{\begin{array}{l}y^2=10/9z\\x=0\end{array}\right. \quad B.\left\{\begin{array}{l}y^2=z\\x=0\end{array}\right. \quad C.\left\{\begin{array}{l}y^2=-10/9z\\x=0\end{array}\right. \quad D.\left\{\begin{array}{l}y^2=10/9z\\x=10\end{array}\right. \quad E. \quad F. \quad G. \quad H.
$$
$$
\begin{array}{l}\mathrm{根据旋转曲面的定义},\mathrm{可见方程}9x^2+9y^2=10z\mathrm{表示旋转抛物面}.\\yOz\mathrm{平面的方程为}x=0,\mathrm{代入得到}y^2=10/9z,\mathrm{从而所求交线的方程为}\left\{\begin{array}{l}y^2=10/9z\\x=0\end{array}\right..\end{array}
$$



$$
\mathrm{曲面}x^2+y^2+z^2=a^2与z=1/2a\mathrm{的交线为}(\;).\;
$$
$$
A.
\mathrm{椭圆线} \quad B.\mathrm{圆线} \quad C.\mathrm{双曲线} \quad D.\mathrm{抛物线} \quad E. \quad F. \quad G. \quad H.
$$
$$
x^2+y^2+1/4a^2=a^2,x^2+y^2=3/4a^2,\mathrm{是圆线}
$$



$$
\mathrm{由曲面}z=\sqrt{2-x^2-y^2}及z=x^2+y^2\mathrm{所围的立体在}xOy\mathrm{面上的投影为}(\;).\;
$$
$$
A.
\left\{\begin{array}{l}x^2+y^2=1\\z=0\end{array}\right. \quad B.\left\{\begin{array}{l}x^2+y^2=\sqrt2\\z=0\end{array}\right. \quad C.\left\{\begin{array}{l}x^2+y^2\leq\sqrt2\\z=0\end{array}\right. \quad D.\left\{\begin{array}{l}x^2+y^2\leq1\\z=0\end{array}\right. \quad E. \quad F. \quad G. \quad H.
$$
$$
\mathrm{消去}z,得x^2+y^2=1,\;\mathrm{所以投影为}\left\{\begin{array}{l}x^2+y^2\leq1\\z=0\end{array}\right.
$$



$$
\mathrm{旋转抛物面}z=x^2+y^2(0\leq z\leq4)\mathrm{在坐标面}xOy\mathrm{上的投影为}(\;).
$$
$$
A.
\left\{\begin{array}{l}x^2+y^2\leq4\\z=0\end{array}\right. \quad B.\left\{\begin{array}{l}x^2+y^2\leq16\\z=0\end{array}\right. \quad C.\left\{\begin{array}{l}x^2-y^2\leq4\\z=0\end{array}\right. \quad D.\left\{\begin{array}{l}x-y^2\leq16\\z=0\end{array}\right. \quad E. \quad F. \quad G. \quad H.
$$
$$
\begin{array}{l}\mathrm{从方程组}\left\{\begin{array}{l}z=x^2+y^2\\z=4\end{array}\right.\mathrm{消去}z\mathrm{得到向}xOy\mathrm{面的投影柱面方程为}:x^2+y^2=4,\mathrm{故该立体在面上的投影为}\left\{\begin{array}{l}x^2+y^2\leq4\\z=0\end{array}\right.;\\z=x^2+y^2与yOz\mathrm{面的交线为}\left\{\begin{array}{l}z=y^2(0\leq z\leq4)\\x=0\end{array}\right.,\\\mathrm{故该立体在}yOz\mathrm{面上的投影为}\left\{\begin{array}{l}y^2\leq z\leq4\\x=0\end{array}\right.;\\\mathrm{而该立体在}xOz\mathrm{面上的投影为}\left\{\begin{array}{l}x^2\leq z\leq4\\y=0\end{array}\right..\end{array}
$$



$$
\mathrm{曲线}\left\{\begin{array}{l}z=f(x,y)\\z=g(x,y)\end{array}\right.\mathrm{关于}xOy\mathrm{面的投影柱面方程是}(\;).
$$
$$
A.
f(x,y)=g(x,y) \quad B.f(x,y)=-g(x,y) \quad C.\left\{\begin{array}{l}f(x,y)=g(x,y)\\z=0\end{array}\right. \quad D.\left\{\begin{array}{l}f(x,y)=g(x,y)\\z=1\end{array}\right. \quad E. \quad F. \quad G. \quad H.
$$
$$
f(x,y)=g(x.y)
$$



$$
\mathrm{圆锥面}z=\sqrt{x^2+y^2}\mathrm{与旋转抛物面}z=2-x^2-y^2\mathrm{所围立体在}xOy\mathrm{上的投影区域为}(\;).\;
$$
$$
A.
\left\{\begin{array}{l}x^2+y^2=1\\z=0\end{array}\right. \quad B.\left\{\begin{array}{l}x^2+y^2\leq1\\z=1\end{array}\right. \quad C.\left\{\begin{array}{l}x^2+y^2\leq1\\z=0\end{array}\right. \quad D.\left\{\begin{array}{l}x^2+y^2\leq1\\z=-2\end{array}\right. \quad E. \quad F. \quad G. \quad H.
$$
$$
\mathrm{两方程联立},消z得x^2+y^2=1,\mathrm{所以投影区域为}\left\{\begin{array}{l}x^2+y^2\leq1\\z=0\end{array}\right.
$$



$$
\mathrm{椭圆抛物面}z=x^2+2y^2\mathrm{与柱面}z=2-x^2\mathrm{所围立体在}xOy\mathrm{上的投影区域为}(\;).\;
$$
$$
A.
x^2+y^2\leq1 \quad B.\left\{\begin{array}{l}x^2+y^2=1\\z=0\end{array}\right. \quad C.x^2+y^2=1 \quad D.\left\{\begin{array}{l}x^2+y^2\leq1\\z=0\end{array}\right. \quad E. \quad F. \quad G. \quad H.
$$
$$
由\left\{\begin{array}{l}z=x^2+2y^2\\z=2-x^2\end{array}\right.得:x^2+y^2=1,\mathrm{所以在}xOy\mathrm{面上投影区域为}\left\{\begin{array}{l}x^2+y^2\leq1\\z=0\end{array}\right.
$$



$$
\mathrm{球面}x^2+y^2+z^2=4\mathrm{与椭圆抛物面}x^2+y^2=3z\mathrm{所围立体在上}xoy\mathrm{面上的投影区域为}(\;).\;
$$
$$
A.
x^2+y^2\leq3 \quad B.x^2+y^2=3 \quad C.\left\{\begin{array}{l}x^2+y^2\leq3\\z=0\end{array}\right. \quad D.\left\{\begin{array}{l}x^2+y^2\leq3\\z=1\end{array}\right. \quad E. \quad F. \quad G. \quad H.
$$
$$
由\left\{\begin{array}{l}x^2+y^2+z^2=4\\x^2+y^2=3z\end{array}\right.得:z=-4\mathrm{舍去},z=1,即x^2+y^2=3,\mathrm{所以在}xOy\mathrm{面上的投影区域为}\left\{\begin{array}{l}x^2+y^2\leq3\\z=0\end{array}\right.
$$



$$
\mathrm{椭圆抛物面}z=x^2+y^2\mathrm{与椭球面}2x^2+2y^2+z^2=8\mathrm{所围立体在}xOy\mathrm{上的投影区域为}(\;).\;
$$
$$
A.
\left\{\begin{array}{l}x^2+y^2=2\\z=2\end{array}\right. \quad B.\left\{\begin{array}{l}x^2+y^2=2\\z=0\end{array}\right. \quad C.\left\{\begin{array}{l}x^2+y^2\leq2\\z=2\end{array}\right. \quad D.\left\{\begin{array}{l}x^2+y^2\leq2\\z=0\end{array}\right. \quad E. \quad F. \quad G. \quad H.
$$
$$
由\left\{\begin{array}{l}2x^2+2y^2+z^2=8\\z=x^2+y^2\end{array}\right.得:z=2,z=-4(\mathrm{舍去}),即x^2+y^2\leq2,\mathrm{所以投影区域为}\left\{\begin{array}{l}x^2+y^2\leq2\\z=0\end{array}\right.
$$



$$
\mathrm{方程组}\left\{\begin{array}{l}x^2+y^2=4\\x+y=1\end{array}\right.\mathrm{在空间表示}\;(\;).
$$
$$
A.
圆 \quad B.\mathrm{椭圆} \quad C.\mathrm{圆柱面} \quad D.\mathrm{两条直线} \quad E. \quad F. \quad G. \quad H.
$$
$$
\mathrm{方程组}\left\{\begin{array}{l}x^2+y^2=4\\x+y=1\end{array}\right.\mathrm{解得}\left\{\begin{array}{l}x=1/2-\sqrt7/2\\y=1/2+\sqrt7/2\end{array}\right.或\left\{\begin{array}{l}y=1/2-\sqrt7/2\\x=1/2+\sqrt7/2\end{array}\right.,\mathrm{所以该方程表示两条直线}.
$$



$$
\mathrm{曲线}\left\{\begin{array}{l}x^2+y^2+z^2=4\\y=z\end{array}\right.在yOz\mathrm{平面上的投影曲线是}(\;)\;
$$
$$
A.
\left\{\begin{array}{l}y=z\\x=0\end{array}\right.(\vert y\vert\leq\sqrt2) \quad B.\left\{\begin{array}{l}y=z\\x=0\end{array}\right.(\vert y\vert\leq\sqrt3) \quad C.\left\{\begin{array}{l}y=z\\x=1\end{array}\right.(\vert y\vert\leq\sqrt2) \quad D.\left\{\begin{array}{l}y=z\\x=1\end{array}\right.(\vert y\vert\leq\sqrt3) \quad E. \quad F. \quad G. \quad H.
$$
$$
\mathrm{由曲线方程把}x\mathrm{消去},\mathrm{直接可得}\;\left\{\begin{array}{l}y=z\\x=0\end{array}\right.(\vert y\vert\leq\sqrt2).
$$



$$
\mathrm{球面}x^2+y^2+z^2=R^2与x+z=a(0<\;a\;<\;R)\mathrm{交线在}xOy\mathrm{平面上投影曲线的方程是}\;(\;).
$$
$$
A.
\left\{\begin{array}{l}x^2+y^2+(a-x)^2=R^2\\z=0\end{array}\right. \quad B.\left\{\begin{array}{l}x^2+y^2+2ax=R^2\\z=0\end{array}\right. \quad C.\left\{\begin{array}{l}x^2+y^2+(a-y)^2=R^2\\z=0\end{array}\right. \quad D.\left\{\begin{array}{l}x^2+y^2+2ay=R^2\\z=0\end{array}\right. \quad E. \quad F. \quad G. \quad H.
$$
$$
\mathrm{由题意可得空间曲线方程为}:\left\{\begin{array}{l}x^2+y^2+z^2=R^2\\x+z=a\end{array}\right.,\mathrm{消去}z\mathrm{得到}:\left\{\begin{array}{l}x^2+y^2+(a-x)^2=R^2\\z=0\end{array}\right.,\mathrm{即为所求}.
$$



$$
\mathrm{曲面}x^2+4y^2+z^2=4\mathrm{与平面}x+z=a\mathrm{的交线在}yOz\mathrm{平面上投影方程是}\;(\;).
$$
$$
A.
\left\{\begin{array}{l}4y^2+z^2+2ay=4\\x=0\end{array}\right. \quad B.\left\{\begin{array}{l}4y^2+z^2+2az=4\\x=0\end{array}\right. \quad C.\left\{\begin{array}{l}(a-z)^2+4y^2+z^2=4\\x=0\end{array}\right. \quad D.\left\{\begin{array}{l}(a-y)^2+4y^2+z^2=4\\x=0\end{array}\right. \quad E. \quad F. \quad G. \quad H.
$$
$$
\mathrm{由题意可得曲线方程为}:\left\{\begin{array}{l}x^2+4y^2+z^2=4\\x+z=a\end{array}\right.,\mathrm{消去}x得:\left\{\begin{array}{l}(a-z)^2+4y^2+z^2=4\\x=0\end{array}\right.\;,\mathrm{即为所求}
$$



$$
\mathrm{直线}\left\{\begin{array}{l}x+y+z=a\\x+cy=b\end{array}\right.在yOz\mathrm{平面上投影是}\;(\;).
$$
$$
A.
\left\{\begin{array}{l}(1-c)y+z=a-b\\x=0\end{array}\right. \quad B.\left\{\begin{array}{l}(1-c)z+y=a-b\\x=0\end{array}\right. \quad C.\left\{\begin{array}{l}(1-c)y+z=a+b\\x=0\end{array}\right. \quad D.\left\{\begin{array}{l}(1+c)y+z=a-b\\x=0\end{array}\right. \quad E. \quad F. \quad G. \quad H.
$$
$$
\mathrm{由直线的方程},\mathrm{消去}x,\mathrm{可得}:\left\{\begin{array}{l}(1-c)y+z=a-b\\x=0\end{array}\right.,\mathrm{即为所求}.
$$



$$
\mathrm{曲面}x^2+y^2+z^2=a^2与x^2+y^2=2az(a>\;0)\mathrm{的交线是}\;(\;).
$$
$$
A.
\mathrm{抛物线} \quad B.\mathrm{双曲线} \quad C.\mathrm{圆周} \quad D.\mathrm{椭圆} \quad E. \quad F. \quad G. \quad H.
$$
$$
\mathrm{联立}\left\{\begin{array}{l}x^2+y^2+z^2=a^2\\x^2+y^2=2az\end{array}\right.\mathrm{解得}z=(-1+\sqrt2)a,x^2+y^2=\lbrack1-(1-\sqrt2)^2\rbrack a^2,\mathrm{所以交线为一个圆周}.
$$



$$
\mathrm{曲线}\left\{\begin{array}{l}y^2+z^2-2x=0\\z=3\end{array}\right.\mathrm{关于}xoy\mathrm{面的投影柱面方程是}(\;).
$$
$$
A.
\left\{\begin{array}{l}y^2=2x\\z=0\end{array}\right. \quad B.y^2=2x-9 \quad C.\left\{\begin{array}{l}y^2=2x-9\\z=0\end{array}\right. \quad D.y^2=2x+9 \quad E. \quad F. \quad G. \quad H.
$$
$$
\mathrm{由题设方程组中消去变量}z\mathrm{后得}y^2+9-2x=0,\mathrm{所以}y^2=2x-9\mathrm{即为在}xOy\mathrm{面上的投影方程}.
$$



$$
\mathrm{曲线}\left\{\begin{array}{l}z=2-x^2-y^2\\z=(x-1)^2+(y-1)^2\end{array}\right.\mathrm{在坐标面}xOy\mathrm{上的投影曲线的方程为}(\;).\;
$$
$$
A.
\left\{\begin{array}{l}z=0\\x^2+y^2=x+y\end{array}\right. \quad B.\left\{\begin{array}{l}z=0\\x^2-y^2=x+y\end{array}\right. \quad C.\left\{\begin{array}{l}z=0\\x^2+y^2=x-y\end{array}\right. \quad D.\left\{\begin{array}{l}z=0\\x^2-2y^2=x+y\end{array}\right. \quad E. \quad F. \quad G. \quad H.
$$
$$
\begin{array}{l}\mathrm{由曲线方程消去}z,\mathrm{得此曲线在}xOy\mathrm{坐标面上的投影}\\\mathrm{柱面方程}:\;\;\;2-x^2-y^2=(x-1)^2+(y-1)^2,\\\mathrm{整理化简得}\;\;\;x^2+y^2=x+y,\\\mathrm{故曲线在}xOy\mathrm{面上的投影曲线方程为}\left\{\begin{array}{l}z=0\\x^2+y^2=x+y\end{array}\right.\end{array}
$$



$$
\mathrm{曲面}x^2+y^2+z^2=a^2与x^2+y^2=2az\mathrm{的交线为}(\;).\;
$$
$$
A.
\mathrm{圆周} \quad B.\mathrm{椭圆} \quad C.\mathrm{抛物线} \quad D.\mathrm{双曲线} \quad E. \quad F. \quad G. \quad H.
$$
$$
\mathrm{曲面}x^2+y^2+z^2=a^2\mathrm{为一个球面},x^2+y^2=2az\mathrm{为椭圆抛物面},\mathrm{因此交线为圆周}.
$$



$$
\mathrm{曲线}\left\{\begin{array}{l}z=2-x^2-y^2\\z=(x-1)^2+(y-1)^2\end{array}\right.在xOy\mathrm{面上的投影曲线的方程为}(\;).\;
$$
$$
A.
\left\{\begin{array}{l}z=1\\x^2+y^2=2x+y\end{array}\right. \quad B.\left\{\begin{array}{l}z=0\\x^2+y^2=2x+y\end{array}\right. \quad C.\left\{\begin{array}{l}z=1\\x^2+y^2=x+y\end{array}\right. \quad D.\left\{\begin{array}{l}z=0\\x^2+y^2=x+y\end{array}\right. \quad E. \quad F. \quad G. \quad H.
$$
$$
\begin{array}{l}\mathrm{由曲线的表达式我们可以消掉}z,\\\mathrm{得到}:(x-1)^2+(y-1)^2=2-x^2-y^2,\\\mathrm{整理得}:x^2+y^2=x+y,\\\mathrm{则此时曲线在}xOy\mathrm{面上的投影曲线的方程为}:\left\{\begin{array}{l}z=0\\x^2+y^2=x+y\end{array}\right.\end{array}
$$



$$
\mathrm{锥面}z=\sqrt{x^2+y^2}\mathrm{与柱面}z^2=2x\mathrm{所围立体在}xOy\mathrm{面上的投影为}\;(\;).
$$
$$
A.
\left\{\begin{array}{l}(x-1)^2+y^2\leq1\\z=0\end{array}\right. \quad B.\left\{\begin{array}{l}(x-1)^2+y^2\leq1\\z=3\end{array}\right. \quad C.\left\{\begin{array}{l}(x-1)^2+y^2\geq1\\z=0\end{array}\right. \quad D.\left\{\begin{array}{l}(x-1)^2+y^2=1\\z=0\end{array}\right. \quad E. \quad F. \quad G. \quad H.
$$
$$
\begin{array}{l}\mathrm{锥面与柱面的交线为}:\left\{\begin{array}{l}z=\sqrt{x^2+y^2}\\z^2=2x\end{array}\right.,\\\mathrm{则由曲线的表达式可消去}z,\mathrm{得到}:x^2+y^2=2x,\\\mathrm{整理得}:(x-1)^2+y^2=1,\\\mathrm{得曲线在面上的投影曲线的方程为}:\left\{\begin{array}{l}(x-1)^2+y^2=1\\z=0\end{array}\right.\;,\;\\\mathrm{故所围立体在}xOy\mathrm{面上的投影应为圆在}xOy\mathrm{面上所围的部分}:\left\{\begin{array}{l}(x-1)^2+y^2\leq1\\z=0\end{array}\right.\end{array}
$$



$$
\mathrm{曲线}\left\{\begin{array}{l}z=x^2+2y^2\\z=2-x^2\end{array}\right.\mathrm{关于}xOy\mathrm{平面的投影柱面为}(\;).\;
$$
$$
A.
x^2+y^2=1 \quad B.\left\{\begin{array}{l}x^2+y^2=2\\z=0\end{array}\right. \quad C.\left\{\begin{array}{l}x^2+y^2=1\\z=0\end{array}\right. \quad D.\left\{\begin{array}{l}x^2+y^2=2\\z=1\end{array}\right. \quad E. \quad F. \quad G. \quad H.
$$
$$
\mathrm{由曲线的表达式消去}z,\mathrm{得到投影柱面}:x^2+y^2=1
$$



$$
\mathrm{直线}\frac{x-1}1=\frac{y+3}{-2}=\frac{z-1}{-1}在xoy\mathrm{平面上的投影直线方程为}\;(\;).
$$
$$
A.
\left\{\begin{array}{l}2x+y-1=0\\z=0\end{array}\right. \quad B.\left\{\begin{array}{l}2x+y+1=0\\z=1\end{array}\right. \quad C.\left\{\begin{array}{l}2x+y-1=0\\z=1\end{array}\right. \quad D.\left\{\begin{array}{l}2x+y+1=0\\z=0\end{array}\right. \quad E. \quad F. \quad G. \quad H.
$$
$$
\mathrm{由直线方程可得}:z-1=-(x-1),z-1=\frac{y+3}2,\mathrm{因此消掉}z得:x-1=\frac{y+3}{-2},\mathrm{则直线在}xOy\mathrm{平面上的投影直线方程为}:\left\{\begin{array}{l}x-1=\frac{y+3}{-2}\\z=0\end{array}\right.或\left\{\begin{array}{l}2x+y+1=0\\z=0\end{array}\right.
$$



$$
\mathrm{假定直线}L在yoz\mathrm{平面上的投影方程为}\left\{\begin{array}{l}2y-3z=1\\x=0\end{array}\right.,\mathrm{而在}zOx\mathrm{平面上的投影方程为}\left\{\begin{array}{l}x+z=2\\y=0\end{array}\right.,\mathrm{则直线}L在xOy\mathrm{面上的投影方程为}(\;).
$$
$$
A.
\left\{\begin{array}{l}3x+2y=7\\z=0\end{array}\right. \quad B.\left\{\begin{array}{l}3x+2y=6\\z=0\end{array}\right. \quad C.\left\{\begin{array}{l}3x-2y=7\\z=0\end{array}\right. \quad D.\left\{\begin{array}{l}3x+5y=7\\z=0\end{array}\right. \quad E. \quad F. \quad G. \quad H.
$$
$$
\begin{array}{l}\mathrm{依题设},\mathrm{所求直线}L\mathrm{方程为}\left\{\begin{array}{l}2y-3z=1\\x+z=2\end{array}\right.,\\\mathrm{消去}z,\mathrm{得到直线在}xOy\mathrm{面上的投影直线方程为}\left\{\begin{array}{l}3x+2y=7\\z=0\end{array}\right..\end{array}
$$



$$
\mathrm{双曲面}x^2-y^2/4-z^2/9=1\mathrm{与平面}y=4\mathrm{交线为}(\;).
$$
$$
A.
\mathrm{双曲线} \quad B.\mathrm{椭圆} \quad C.\mathrm{抛物线} \quad D.\mathrm{一对相交直线} \quad E. \quad F. \quad G. \quad H.
$$
$$
将y=4\mathrm{代入到双曲面方程得}x^2-z^2/9=5,\mathrm{所以相交的曲线为双曲线}.
$$



$$
\mathrm{双曲面}x^2-y^2/4-z^2/9=1\mathrm{与平面}y=4\mathrm{交线为}(\;).
$$
$$
A.
\mathrm{双曲线} \quad B.\mathrm{椭圆} \quad C.\mathrm{抛物线} \quad D.\mathrm{一对相交直线} \quad E. \quad F. \quad G. \quad H.
$$
$$
将y=4\mathrm{代入到双曲面方程得}x^2-z^2/9=5,\mathrm{所以相交的曲线为双曲线}.
$$



$$
\begin{array}{l}\mathrm{设空间直线的标准方程是}\frac x0=\frac y1=\frac z2,\mathrm{该直线过原点},且().\\\end{array}
$$
$$
A.
\mathrm{垂直于}x轴 \quad B.\mathrm{垂直于}z轴,\mathrm{但不平行于}x轴 \quad C.\mathrm{垂直于}y轴,\mathrm{但不平行于}x轴 \quad D.\mathrm{平行于}x轴 \quad E. \quad F. \quad G. \quad H.
$$
$$
\mathrm{该直线的方向向量为}\{0,1,2\},x\mathrm{轴的方向向量为}\{1,0,0\},\mathrm{向量的数量积为零},\mathrm{所以该直线于}x\mathrm{轴垂直}
$$



$$
\mathrm{过两点}M_1(2,-1,5)和M_2(-1,0,6)\mathrm{的直线方程为}(\;).
$$
$$
A.
\frac{x-2}{-3}=\frac{\displaystyle y+1}{\displaystyle1}=\frac{\displaystyle z-5}{\displaystyle1} \quad B.\frac{x-2}4=\frac{\displaystyle y+1}{\displaystyle1}=\frac{\displaystyle z-5}{\displaystyle1} \quad C.\frac{x-2}{-3}=\frac{\displaystyle y-1}{\displaystyle1}=\frac{\displaystyle z+5}{\displaystyle1} \quad D.\frac{x-2}{-3}=\frac{\displaystyle y+1}{\displaystyle0}=\frac{\displaystyle z-5}{\displaystyle1} \quad E. \quad F. \quad G. \quad H.
$$
$$
\begin{array}{l}\mathrm{可取方向向量}\overrightarrow s\;=\overrightarrow{M_1M_2}\;=\{-3,1,1\},\\\mathrm{于是所求直线方程为}\frac{x-2}{-3}=\frac{\displaystyle y+1}{\displaystyle1}=\frac{\displaystyle z-5}{\displaystyle1}.\end{array}
$$



$$
\mathrm{直线}\left\{\begin{array}{l}5x+y-3z-7=0\\2x+y-3z-7=0\end{array}\right.(\;).
$$
$$
A.
\mathrm{垂直}yOz\mathrm{平面} \quad B.在yOz\mathrm{平面内} \quad C.\mathrm{平行}x轴 \quad D.在xOy\mathrm{平面内} \quad E. \quad F. \quad G. \quad H.
$$
$$
\mathrm{由直线方程可知},x=0,\mathrm{所以直线是在}yOz\mathrm{平面内}
$$



$$
\begin{array}{l}\mathrm{直线}\left\{\begin{array}{l}2x-y+3z-4=0\\x+y+z+1=0\end{array}\right.\;\mathrm{的标准方程为}(\;)\\\end{array}
$$
$$
A.
\frac{x-1}{-4}=\frac{\displaystyle y+2}{\displaystyle-1}=\frac{\displaystyle z}{\displaystyle3} \quad B.\frac{x-1}{-4}=\frac{\displaystyle y+2}{\displaystyle1}=\frac{\displaystyle z}{\displaystyle3} \quad C.\frac{x-1}{-4}=\frac{\displaystyle y+2}{\displaystyle1}=\frac{\displaystyle z}{\displaystyle-3} \quad D.\frac{x-1}4=\frac{\displaystyle y+2}{\displaystyle0}=\frac{\displaystyle z}{\displaystyle-3} \quad E. \quad F. \quad G. \quad H.
$$
$$
\begin{array}{l}\mathrm{直线的方向向量}s=\begin{vmatrix}i&j&k\\2&-1&3\\1&1&1\end{vmatrix}=\{-4,1,3\},\mathrm{再在直线上取定}\;\;x_0=1,⇒ y_0\;=-2,z_0=0.\\\mathrm{所得直线的标准方程为}\frac{x-1}{-4}=\frac{\displaystyle y+2}{\displaystyle1}=\frac{\displaystyle z}{\displaystyle3}\;.\end{array}
$$



$$
\mathrm{过点}(-1,2,0)\mathrm{且与平面}2x+y-z=0\mathrm{垂直的直线方程为}()
$$
$$
A.
\frac{x+1}2=\frac{\displaystyle y-2}{\displaystyle1}=\frac{\displaystyle z}{\displaystyle-1} \quad B.\frac{x-1}2=\frac{\displaystyle y-2}{\displaystyle1}=\frac{\displaystyle z}{\displaystyle-1} \quad C.\frac{x+1}2=\frac{\displaystyle y+2}{\displaystyle1}=\frac{\displaystyle z}{\displaystyle-1} \quad D.\frac{x+1}2=\frac{\displaystyle y-2}{\displaystyle-1}=\frac{\displaystyle z}{\displaystyle1} \quad E. \quad F. \quad G. \quad H.
$$
$$
\mathrm{所求直线的方向向量}\;\;s=\{2,1,-1\}\;,\mathrm{由直线的标准式方程可得}\frac{x+1}2=\frac{\displaystyle y-2}{\displaystyle1}=\frac{\displaystyle z}{\displaystyle-1}.
$$



$$
\mathrm{下列方程中表示的是空间的一条直线的是}()
$$
$$
A.
x+y+z+1=0 \quad B.x+y+1=0 \quad C.\left\{\begin{array}{l}x=1\\y=1\end{array}\right. \quad D.z=a \quad E. \quad F. \quad G. \quad H.
$$
$$
\mathrm{答案}\;\;\;A,B,D\;\mathrm{均表示平面}
$$



$$
\;\mathrm{过点}(1,0,1)\mathrm{且垂直于平面}x+3y-2z+2=0\mathrm{的直线方程为}
$$
$$
A.
\frac{x+1}1=\frac{\displaystyle y}{\displaystyle3}=\frac{\displaystyle z+1}{\displaystyle-2} \quad B.\frac{x-1}1=\frac{\displaystyle y}{\displaystyle3}=\frac{\displaystyle z-1}{\displaystyle2} \quad C.\frac{x+1}1=\frac{\displaystyle y}{\displaystyle-3}=\frac{\displaystyle z-1}{\displaystyle-2} \quad D.\frac{x-1}1=\frac{\displaystyle y}{\displaystyle3}=\frac{\displaystyle z-1}{\displaystyle-2} \quad E. \quad F. \quad G. \quad H.
$$
$$
\begin{array}{l}\mathrm{所求直线的方向向量可取作已知平面的法向量}\overrightarrow n=\{1,3,-2\},\mathrm{又过点}(1,0,1),\\\mathrm{故直线方程为}\frac{x-1}1=\frac{\displaystyle y}{\displaystyle3}=\frac{\displaystyle z-1}{\displaystyle-2}\end{array}
$$



$$
\mathrm{过点}(1,-3,0)\mathrm{与平面}2x+3y-z+7=0\mathrm{垂直的直线方程为}()
$$
$$
A.
\frac{x-1}1=\frac{\displaystyle y}{\displaystyle3}=\frac{\displaystyle z-1}{\displaystyle-2} \quad B.\frac{x-1}1=\frac{\displaystyle y+3}{\displaystyle-3}=\frac{\displaystyle z}{\displaystyle1} \quad C.\frac{x-1}2=\frac{\displaystyle y+3}{\displaystyle3}=\frac{\displaystyle z}{\displaystyle-1} \quad D.\frac{x-1}2=\frac{\displaystyle y-3}{\displaystyle3}=\frac{\displaystyle z}{\displaystyle-1} \quad E. \quad F. \quad G. \quad H.
$$
$$
\begin{array}{l}\mathrm{所求直线的方向向量可取作已知平面的法向量}\overrightarrow n=\{2,3,-1\},\mathrm{又过点}(1,-3,0),\\\mathrm{股直线方程为}\frac{x-1}2=\frac{\displaystyle y+3}{\displaystyle3}=\frac{\displaystyle z}{\displaystyle-1}.\end{array}
$$



$$
\mathrm{过点}(0,1,-3)\mathrm{与平面}3x-y+4z-8=0\mathrm{垂直的直线方程为}(\;)
$$
$$
A.
\frac{x-1}1=\frac{\displaystyle y+3}{\displaystyle3}=\frac{\displaystyle z}{\displaystyle-1} \quad B.\frac x3=\frac{\displaystyle y-1}{\displaystyle-1}=\frac{\displaystyle z+3}{\displaystyle4} \quad C.\frac x3=\frac{\displaystyle y+1}{\displaystyle-1}=\frac{\displaystyle z+3}{\displaystyle4} \quad D.\frac x3=\frac{\displaystyle y-1}{\displaystyle-1}=\frac{\displaystyle z-3}{\displaystyle4} \quad E. \quad F. \quad G. \quad H.
$$
$$
\begin{array}{l}\mathrm{所求直线的方向向量可取作已知平面的法向量}\overrightarrow n=\{3,-1,4\},\mathrm{又过点}(0,1,-3),\\\mathrm{故直线方程为}\frac x3=\frac{\displaystyle y-1}{\displaystyle-1}=\frac{\displaystyle z+3}{\displaystyle4}.\end{array}
$$



$$
\mathrm{过点}(2,3,-5)\mathrm{且与直线}\frac{x-2}{-1}=\frac y3=\frac{z+1}4\mathrm{平行的直线的标准方程是}(\;)
$$
$$
A.
\frac{x+2}{-1}=\frac{y-3}3=\frac{z+5}4 \quad B.\frac{x-2}{-1}=\frac{y-3}3=\frac{z+5}{-4} \quad C.\frac{x-2}{-1}=\frac{y+3}3=\frac{z+5}4 \quad D.\frac{x-2}{-1}=\frac{y-3}3=\frac{z+5}4 \quad E. \quad F. \quad G. \quad H.
$$
$$
\begin{array}{l}\mathrm{两条直线平行},\mathrm{方向向量平行},\mathrm{故所求直线的方向向量可取}\overrightarrow n=\{-1,3,4\},\mathrm{又过点}(2,3,-5),\\\mathrm{所以直线的标准方程为}\frac{x-2}{-1}=\frac{y-3}3=\frac{z+5}4\end{array}
$$



$$
\mathrm{过点}M_1(3,-2,1),M_2(-1,0,2)\mathrm{的直线方程为}(\;).
$$
$$
A.
\frac{x+1}4=\frac y{-2}=\frac{z-2}{-1} \quad B.\frac{x+1}4=\frac y2=\frac{z-2}{-1} \quad C.\frac{x+1}4=\frac y{-2}=\frac{z-2}1 \quad D.\frac{x+1}{-4}=\frac y2=\frac{z-2}{-1} \quad E. \quad F. \quad G. \quad H.
$$
$$
\begin{array}{l}n=\{3-(-1),-2-0,1-2\}=\{4,-2,-1\},点M_2\mathrm{在直线上},\mathrm{则直线方程式为}\frac{x+1}4=\frac y{-2}=\frac{z-2}{-1}.\\\end{array}
$$



$$
若\left\{\begin{array}{l}2x+3y-z+D=0\\2x-2y+2z-6=0\end{array}\right.与x\mathrm{轴有交点},则D=(\;).
$$
$$
A.
-6 \quad B.6 \quad C.-3 \quad D.3 \quad E. \quad F. \quad G. \quad H.
$$
$$
\mathrm{根据题意当}y=z=0时,\mathrm{方程有解}\;,则:\;2x+D=0,2x-6=0\;,\mathrm{得到}D=-6.
$$



$$
\mathrm{过点}(0,-3,2)\mathrm{且与两点}P_1(3,4,-7),P_2(2,7,-6)\mathrm{的连线平行的直线的对称式方程为}(\;).
$$
$$
A.
\frac x{-1}=\frac{y+3}3=\frac{z-2}1 \quad B.\frac x1=\frac{y+3}3=\frac{z-2}1 \quad C.\frac x{-1}=\frac{y+3}{-3}=\frac{z-2}1 \quad D.\frac x1=\frac{y+3}3=\frac{z-2}{-1} \quad E. \quad F. \quad G. \quad H.
$$
$$
\begin{array}{l}\overrightarrow{P_1P_2}=\{-1,3,1\},\mathrm{由题意可知所求直线的方向向量也为}:\{-1,3,1\},\\\mathrm{又点}(0,-3,2)\mathrm{在直线上},\mathrm{则对称式方程为}:\frac x{-1}=\frac{y+3}3=\frac{z-2}1\end{array}
$$



$$
\mathrm{过点}(-3,2,5)\mathrm{且与两平面}x-4z=3和2x-y-5z=1\mathrm{的交线平行的直线方程为}().
$$
$$
A.
\frac{x+3}4=\frac{y-2}3=\frac{z-5}1 \quad B.\frac{x+3}4=\frac{y-2}{-3}=\frac{z-5}1 \quad C.\frac{x+3}4=\frac y3=\frac{z-5}1 \quad D.\frac{x+3}{-4}=\frac{y-2}3=\frac{z+5}{-1} \quad E. \quad F. \quad G. \quad H.
$$
$$
\begin{array}{l}\mathrm{设所求直线的方向向量为}\overrightarrow s=\{m,n,p\},\\\mathrm{根所题意知}\overrightarrow s⟂\overrightarrow{n_1},\overrightarrow s⟂\overrightarrow{n_2},\\\;\;\;\;\;\;\;\;\;\;\;\;\;\;\;\;\;\;\;\;\;\;\;\;\;\;\;\;\;\;\;\;\;\;\;\;\;\;\;\;\;\;\;\;取\overrightarrow s=\overrightarrow{n_1}×\overrightarrow{n_2}=\begin{vmatrix}\overrightarrow i&\overrightarrow j&\overrightarrow k\\1&0&-4\\2&-1&-5\end{vmatrix}=\{-4,-3,-1\},\\\mathrm{所求直线方程}\;\frac{x+3}4=\frac{y-2}3=\frac{z-5}1\end{array}
$$



$$
\mathrm{过点}(0,2,4)\mathrm{且与两平面}x+2z=1和y-3z=2\mathrm{平行的直线方程为}().
$$
$$
A.
\frac x{-2}=\frac{y-2}3=\frac{z-4}1 \quad B.\frac x{-2}=\frac{y-2}3=\frac{z+4}1 \quad C.\frac x{-2}=\frac{y-2}{-3}=\frac{z-4}1 \quad D.\frac x{-2}=\frac{y+2}3=\frac{z+4}1 \quad E. \quad F. \quad G. \quad H.
$$
$$
\begin{array}{l}\mathrm{该直线的方向向量}\overrightarrow s\mathrm{与两平面的法向量}\overrightarrow{n_1}与\overrightarrow{n_2}\mathrm{都垂直}.\mathrm{因为}\\\overrightarrow s=\overrightarrow{n_1}×\overrightarrow{n_2}=\begin{vmatrix}\overrightarrow i&\overrightarrow j&\overset{\boldsymbol\rightarrow}{\mathbf k}\\1&0&2\\0&1&-3\end{vmatrix}=\{-2,3,1\},\\\mathrm{故所求直线的方程为}\frac x{-2}=\frac{y-2}3=\frac{z-4}1\end{array}
$$



$$
\mathrm{设有直线}l_1:\frac{x-1}1=\frac{y-5}{-2}=\frac{z+8}1与l_2:\left\{\begin{array}{l}x-y=6\\2y+z=3\end{array}\right.,则l_1与l_2\mathrm{的夹角为}(\;).
$$
$$
A.
\fracπ6 \quad B.\fracπ4 \quad C.\fracπ3 \quad D.\fracπ2 \quad E. \quad F. \quad G. \quad H.
$$
$$
\begin{array}{l}\mathrm{两直线的方向向量分别为}\{1,-2,1\}和\{1,1,-2\},\\\mathrm{所以可得cos}θ=\frac{\begin{vmatrix}1-2-2\end{vmatrix}}{\sqrt{1+1+4}\sqrt{1+1+4}}=\frac12,\mathrm{夹角为}θ=\fracπ3.\end{array}
$$



$$
点A(-1,2,0)\mathrm{在平面}x+2y-z+3=0\mathrm{上的投影为}().
$$
$$
A.
(2,0,-1) \quad B.(2,0,1) \quad C.(-2,0,-1) \quad D.(-2,0,1) \quad E. \quad F. \quad G. \quad H.
$$
$$
\begin{array}{l}\mathrm{平面的}x+2y-z+3=0\mathrm{的法向量为}\{1,2,-1\},\mathrm{则过点}A\;\mathrm{且垂直于该平面的直线方程为}\\\frac{x+1}1=\frac{y-2}2=\frac z{-1},\mathrm{该直线于平面的交点即为点}A\mathrm{在平面上的投影}.\mathrm{联立两方程解的交点为}(-2,0,1)\end{array}
$$



$$
\mathrm{一直线过点}A(2,-3,4),\mathrm{且和}y\mathrm{轴垂直相交},\mathrm{则其方程为}(\;).
$$
$$
A.
\frac{x-2}2=\frac{\displaystyle y+3}{\displaystyle0}=\frac{\displaystyle z-4}{\displaystyle4} \quad B.\frac{x-2}2=\frac{\displaystyle y+3}{\displaystyle0}=\frac{\displaystyle z+4}{\displaystyle4} \quad C.\frac{x+2}2=\frac{\displaystyle y+3}{\displaystyle0}=\frac{\displaystyle z-4}{\displaystyle4} \quad D.\frac{x-2}2=\frac{\displaystyle y+3}{\displaystyle1}=\frac{\displaystyle z-4}{\displaystyle3} \quad E. \quad F. \quad G. \quad H.
$$
$$
\begin{array}{l}\mathrm{因为直线和}y\mathrm{轴垂直相交},\\\mathrm{所以交点为}B(0,-3,0)\\取\overrightarrow s=\overrightarrow{BA}=\{2,0,4\}\\\mathrm{所求直线方程}\\\frac{x-2}2=\frac{\displaystyle y+3}{\displaystyle0}=\frac{\displaystyle z-4}{\displaystyle4}\end{array}
$$



$$
\mathrm{已知两直线的方程分别为}l_1:\frac x{-7}=\frac{y-{\displaystyle\textstyle\frac{27}7}}{-2}=\frac{z+{\displaystyle\textstyle\frac97}}{10};l_2:x=y-1=\frac{z+3}{-1}\;则(\;).\;\;
$$
$$
A.
l_1与l_2\mathrm{平行},\mathrm{但不重合} \quad B.l_1与l_2\mathrm{仅有一个交点} \quad C.l_1与l_2\mathrm{重合} \quad D.l_1与l_2\mathrm{是异面直线} \quad E. \quad F. \quad G. \quad H.
$$
$$
\begin{array}{l}\;l_1\mathrm{方向向量为}\{-7,-2,10\},l_2\mathrm{的方向向量为}\{1,1,-1\},\mathrm{所以两直线不平行},\\-7·1+(-2)·1+10·(-1)\neq0,\mathrm{所以也不垂直},\mathrm{可以解得有一交点为}(4,5,-7),\mathrm{所以}l_1与l_2\mathrm{仅有一个交点}.\end{array}
$$



$$
\mathrm{设空间直线方程为}\left\{\begin{array}{l}x=0\\2y-z=0\end{array}\right.\mathrm{则该直线必}(\;).
$$
$$
A.
\mathrm{过原点且垂直于}x轴 \quad B.\mathrm{过原点且垂直于}y轴 \quad C.\mathrm{过原点且垂直于}z轴 \quad D.\mathrm{过原点且平行于}x轴 \quad E. \quad F. \quad G. \quad H.
$$
$$
\mathrm{由题意得},\mathrm{该直线肯定过原点},又x=0,\mathrm{所以该直线在平面}yOz内,\mathrm{所以垂直于}x轴
$$



$$
\mathrm{用参数方程表示直线}\left\{\begin{array}{l}2x-y-3z+2=0\\x+2y-z-6=0\end{array}\right.\mathrm{下列正确的是}(\;).
$$
$$
A.
\left\{\begin{array}{c}x=7t+2/5\\y=-t+14/5\\z=5t\end{array}\right. \quad B.\left\{\begin{array}{c}x=7t+5\\y=-t+14/5\\z=5t\end{array}\right. \quad C.\left\{\begin{array}{c}x=7t+2/5\\y=t+14/5\\z=t\end{array}\right. \quad D.\left\{\begin{array}{c}x=7t+2/5\\y=t+14/5\\z=5t\end{array}\right. \quad E. \quad F. \quad G. \quad H.
$$
$$
\begin{array}{l}\mathrm{设直线的方向向量为}n,\mathrm{则可取}\\n=n_1× n_2=\begin{vmatrix}i&j&k\\2&-1&-3\\1&2&-1\end{vmatrix}=7i-j+5k.\\\mathrm{再在直线上取一点},\mathrm{例如},\mathrm{可令}z=0,得\\\left\{\begin{array}{l}2x-y+2=0\\x+2y-6=0\end{array}\right.⇒ x=\frac25,y=\frac{14}5,\\\mathrm{于是},\mathrm{直线的对称式方程}\frac{x-2/5}7=\frac{y-14/5}{-1}=\frac{z-0}5,\\\mathrm{参数方程为}\left\{\begin{array}{c}x=7t+2/5\\y=-t+14/5\\z=5t\end{array}\right.\end{array}
$$



$$
\mathrm{直线}\left\{\begin{array}{l}x-2y+z-9=0\\3x-6y+z-27=0\end{array}\right.\mathrm{的对称式方程为}(\;).
$$
$$
A.
\frac{x-9}4=\frac{y-1}2=\frac z1 \quad B.\frac{x-9}4=\frac{y-1}2=\frac z0 \quad C.\frac{x-9}4=\frac y2=\frac z1 \quad D.\frac{x-9}4=\frac y2=\frac z0 \quad E. \quad F. \quad G. \quad H.
$$
$$
\begin{array}{l}\mathrm{设直线方向向量为}n,\mathrm{则可取}:\\n=n_1× n_2=\begin{vmatrix}i&j&k\\1&-2&1\\3&-6&1\end{vmatrix}=4i+2j,\\\mathrm{在直线上取一点},令z=0,\mathrm{则带入直线}\left\{\begin{array}{l}x-2y+z-9=0\\3x-6y+z-27=0\end{array}\right.,\mathrm{得到}:x=9,y=0,\mathrm{则对称式方程为}\\\frac{x-9}4=\frac y2=\frac z0\end{array}
$$



$$
\mathrm{过点}(0,2,4)\mathrm{且与两平面}x+2z=1和y-3z=2\mathrm{平行的直线方程是}(\;).
$$
$$
A.
\frac x{-2}=\frac{y-2}3=\frac{z-4}1 \quad B.\frac x2=\frac{y-2}3=\frac{z-4}1 \quad C.\frac x2=\frac{y+2}3=\frac{z+4}1 \quad D.\frac x{-2}=\frac{y-2}{-3}=\frac{z-4}1 \quad E. \quad F. \quad G. \quad H.
$$
$$
\begin{array}{l}\mathrm{该直线的方向向量}n\mathrm{与两平面的法向量}n_1,n_2\mathrm{都垂直},\mathrm{因为}\\n=n_1× n_2=\begin{vmatrix}i&j&k\\1&0&2\\0&1&-3\end{vmatrix}=\{-2,3,1\},\\\mathrm{故所求直线方程为}:\frac x{-2}=\frac{y-2}3=\frac{z-4}1\end{array}
$$



$$
\mathrm{直线}\left\{\begin{array}{l}x=3z-5\\y=2z-8\end{array}\right.\mathrm{的对称式方程为}(\;).\;\;
$$
$$
A.
\frac{x+5}3=\frac{y+8}2=\frac z1 \quad B.\frac{x-5}3=\frac{y-8}2=\frac z1 \quad C.\frac{x+5}3=\frac{y+8}{-2}=\frac z1 \quad D.\frac{x-5}{-3}=\frac{y-8}{-2}=\frac z1 \quad E. \quad F. \quad G. \quad H.
$$
$$
\begin{array}{l}\mathrm{设直线的方向向量为}n,则:\\n=n_1× n_2=\begin{vmatrix}i&j&k\\1&0&-3\\0&1&-2\end{vmatrix}=3i+2j+k,\mathrm{在直线上任取一点},令z=0,\mathrm{由直线方程}\left\{\begin{array}{l}x=3z-5\\y=2z-8\end{array}\right.得\\x=-5,y=-8,\mathrm{对称式方程为}:\frac{x+5}3=\frac{y+8}2=\frac z1.\end{array}
$$



$$
\mathrm{过点}(-1,2,1)\mathrm{且平行直线}\left\{\begin{array}{l}x+y-2z-1=0\\x+2y-z+1=0\end{array}\right.\mathrm{的直线方程为}(\;).
$$
$$
A.
\frac{x+1}3=\frac{y-2}1=\frac{z-1}1 \quad B.\frac{x+1}3=\frac{y-2}{-1}=\frac{z-1}1 \quad C.\frac{x+1}{-3}=\frac{y-2}{-1}=\frac{z-1}1 \quad D.\frac{x+1}3=\frac{y-2}{-1}=\frac{z-1}{-1} \quad E. \quad F. \quad G. \quad H.
$$
$$
\begin{array}{l}\mathrm{直线}\left\{\begin{array}{l}x+y-2z-1=0\\x+2y-z+1=0\end{array}\right.\mathrm{的方向向量为}:\\n_1=\begin{vmatrix}i&j&k\\1&1&-2\\1&2&-1\end{vmatrix}=3i-j+k,\mathrm{由题意可得}n=n_1=\{3,-1,1\},\mathrm{则直线}\\\mathrm{方程为}\frac{x+1}3=\frac{y-2}{-1}=\frac{z-1}1\end{array}
$$



$$
\mathrm{过点}(0,2,4)\mathrm{且与平面}x+2z=1及y-3z=2\mathrm{都平行的直线方程为}(\;).
$$
$$
A.
\frac x{-2}=\frac{y-2}3=\frac{\displaystyle z-4}1 \quad B.\frac x2=\frac{y-2}3=\frac{\displaystyle z-4}1 \quad C.\frac x{-2}=\frac{y-2}3=\frac{\displaystyle z-4}{-1} \quad D.\frac x2=\frac{y-2}{-3}=\frac{\displaystyle z-4}1 \quad E. \quad F. \quad G. \quad H.
$$
$$
\begin{array}{l}\mathrm{该直线方向}n\mathrm{与两平面法向量都垂直},则:\\n=n_1× n_2=\begin{vmatrix}i&j&k\\1&0&2\\0&1&-3\end{vmatrix}=\{-2,3,1\},\\\mathrm{故所求直线方程为}:\frac x{-2}=\frac{y-2}3=\frac{\displaystyle z-4}1(或\left\{\begin{array}{l}x+2z-8=0\\y-3z+10=0\end{array}\right.).\end{array}
$$



$$
\mathrm{过点}(2,2,4)\mathrm{且与平面}x+z=1及x+y-3z=2\mathrm{都平行的直线方程为}(\;).
$$
$$
A.
\frac{x-2}{-2}=\frac{y-2}4=\frac{\displaystyle z-4}1 \quad B.\frac{x-2}{-1}=\frac{y-2}{-4}=\frac{\displaystyle z-4}1 \quad C.\frac{x-2}{-1}=\frac{y-2}4=\frac{\displaystyle z-4}1 \quad D.\frac{x-2}{-1}=\frac{y-2}4=\frac{\displaystyle z-4}{-2} \quad E. \quad F. \quad G. \quad H.
$$
$$
\begin{array}{l}\mathrm{该直线方向向量}n\mathrm{与两平面的法向量垂直},则\\n=n_1× n_2=\begin{vmatrix}i&j&k\\1&0&1\\1&1&-3\end{vmatrix}=-i+4j+k,\mathrm{则直线方程为}:\frac{x-2}{-1}=\frac{y-2}4=\frac{\displaystyle z-4}1(或\left\{\begin{array}{l}x+z=6\\x+y-3z=-8\end{array}\right.).\\\end{array}
$$



$$
\mathrm{直线}\frac{x+1}3=\frac{y+3}2=\frac{\displaystyle z}1与\frac x1=\frac{y+5}2=\frac{\displaystyle z-2}7\mathrm{的关系是}(\;).
$$
$$
A.
\mathrm{平行} \quad B.\mathrm{相交} \quad C.\mathrm{垂直不相交} \quad D.\mathrm{不垂直不相交} \quad E. \quad F. \quad G. \quad H.
$$
$$
\begin{array}{l}s_1=\{3,2,1\},s_2=\{1,2,7\},\\s_1· s_2=14\neq0,\mathrm{所以不垂直},\mathrm{且他们也不平行}\\又\begin{vmatrix}3&2&1\\1&2&7\\-1-0&-3-(-5)&0-2\end{vmatrix}=-60\neq0,\mathrm{所以不相交}\end{array}
$$



$$
\mathrm{过点}(3,4,-4)\mathrm{且方向角为}\fracπ3,\fracπ4,\fracπ3\mathrm{的直线的对称式方程为}(\;).
$$
$$
A.
\frac{x-3}1=\frac{y-4}{\sqrt2}=\frac{z+4}1 \quad B.\frac{x-3}1=\frac{y-4}{-\sqrt2}=\frac{z+4}1 \quad C.\frac{x-3}1=\frac{y-4}{\sqrt2}=\frac{z+4}{-1} \quad D.\frac{x-3}{-1}=\frac{y-4}{\sqrt2}=\frac{z+4}1 \quad E. \quad F. \quad G. \quad H.
$$
$$
\begin{array}{l}\mathrm{设所求直线的方向向量为}:\{A,B,C\},\mathrm{则由题意得到}:\\\cos\fracπ3=\frac A{\sqrt{A^2+B^2+C^2}}=\frac12,\cos\fracπ4=\frac B{\sqrt{A^2+B^2+C^2}}=\frac{\sqrt2}2\\\cos\fracπ3=\frac C{\sqrt{A^2+B^2+C^2}}=\frac12,\\\mathrm{则方向向量为}:\left\{\frac12\right.,\begin{array}{c}\frac{\sqrt2}2,\end{array}\left.\frac12\right\},\mathrm{对称式方程为}:\frac{x-3}{\displaystyle\frac12}=\frac{y-4}{\displaystyle\frac{\sqrt2}2}=\frac{z+4}{\displaystyle\frac12},\\\mathrm{整理得}:\frac{x-3}1=\frac{y-4}{\sqrt2}=\frac{z+4}1.\end{array}
$$



$$
\mathrm{两平面}x-2y-z=3,2x-4y-2z=5\mathrm{各自与平面}x+y-3z=0\mathrm{的交线是}(\;).
$$
$$
A.
\mathrm{相交的} \quad B.\mathrm{平行的} \quad C.\mathrm{异面的} \quad D.\mathrm{重合的} \quad E. \quad F. \quad G. \quad H.
$$
$$
\begin{array}{l}\mathrm{两平面}x-2y-z=3\mathrm{与平面}x+y-3z=0\mathrm{的交线为}\\\left\{\begin{array}{l}x-2y-z=3\\x+y-3z=0\end{array}\right.,\mathrm{其对称式方程为}\frac{x-1}1=\frac{y+1}{\displaystyle\frac27}=\frac z{\displaystyle\frac37}\\\mathrm{同理可得}2x-4y-2z=5\mathrm{与平面}x+y-3z=0\mathrm{的交线方程为}\\\frac{x-1}1=\frac{y+{\displaystyle\frac{11}{14}}}{\displaystyle\frac27}=\frac{z-{\displaystyle\frac1{14}}}{\displaystyle\frac37},\mathrm{所以两线是平行的}.\end{array}
$$



$$
\mathrm{直线}L:\left\{\begin{array}{l}x+y-z-1=0\\x-y+z+1=0\end{array}\right.,\mathrm{在平面}{\textstyle\prod_{}}:x+y+z=0\mathrm{上的投影直线的方程为}(\;).
$$
$$
A.
\left\{\begin{array}{l}y-z=1\\x+y+z=0\end{array}\right. \quad B.\left\{\begin{array}{l}y+z=1\\x+y+z=0\end{array}\right. \quad C.\left\{\begin{array}{l}y-z=1\\x+2y-z=0\end{array}\right. \quad D.\left\{\begin{array}{l}y-z=2\\x+y+z=0\end{array}\right. \quad E. \quad F. \quad G. \quad H.
$$
$$
\begin{array}{l}设{\textstyle\prod_1}\mathrm{是过直线}L\mathrm{且垂直于平面}\;{\textstyle\prod_\;}\mathrm{的平面},则{\textstyle\prod_1}与{\textstyle\prod_\;}\mathrm{的交线即为}L在{\textstyle\prod_\;}\mathrm{上的投影直线},\mathrm{下面求}{\textstyle\prod_1}\mathrm{的方程}.\\\;设{\textstyle\prod_\;的}\mathrm{法向量为}\overrightarrow n,\mathrm{直线}L\mathrm{的方向向量为}\overrightarrow v,{\textstyle\prod_1}\mathrm{的法向量为}\overrightarrow{n_1},\mathrm{则有}\overrightarrow{n_1}⟂\overrightarrow n,\\\mathrm{故可取}\overrightarrow{n_1}=\overrightarrow n×\overrightarrow v=\begin{vmatrix}\overrightarrow i&\overrightarrow j&\overrightarrow k\\1&1&1\\0&-2&-2\end{vmatrix}=\overrightarrow{2j}-\overrightarrow{2k},\\\mathrm{在直线}L\mathrm{上取一点}P(0,1,0),\mathrm{得平面}{\textstyle\prod_1}\mathrm{的方程为}2(y-1)-2z=0,\\\mathrm{再把它与平面}x+y+z=0\mathrm{联立},得\left\{\begin{array}{l}y-z=1\\x+y+z=0\end{array}\right.\\\mathrm{此即为所求投影直线方程}.\end{array}
$$



$$
\mathrm{已知点}P(1,3,-4),\mathrm{平面}{\textstyle\prod_{}}\mathrm{的方程为}3x+y-2z=0,\mathrm{则与点}P\mathrm{关于平面}{\textstyle\prod_{}}\mathrm{对称的}Q\mathrm{点的坐标是}(\;).
$$
$$
A.
(5,-1,0) \quad B.(5,1,0) \quad C.(-5,-1,0) \quad D.(-5,1,0) \quad E. \quad F. \quad G. \quad H.
$$
$$
\begin{array}{l}过P\mathrm{做垂直与平面}π\mathrm{的直线}L:\left\{\begin{array}{c}x=1+3λ\\y=3+λ\\z=-4-2λ\end{array}\right.·\\\mathrm{求直线}L\mathrm{与平面}π\mathrm{的交点}M\mathrm{之坐标}(x_0,y_0,z_0),由\\3x+y-2z=3(1+3λ)+(3+λ)-2(-4-2λ)=14λ+14=0\\\;\;\;\;\;\;\;\;\;\;\;\;\;\;\;\;\;\;⇒λ=-1⇒ x_0=-2,y_0=2,z_0=-2.\\\mathrm{令与点}P\mathrm{关于平面}π\mathrm{对称点}Q\mathrm{的坐标为}(x_1,y_1,z_1),\mathrm{利用中点公式有}\\\frac{1+x_1}2=-2,\frac{3+y_1}2=2,\frac{-4+z_1}2=-2,\\\;\;\;\;\;\;\;\;\;\;\;⇒ x_1=-5,y_1=1,z_1=0,\\即Q\mathrm{点坐标为}(-5,1,0).\end{array}
$$



$$
\mathrm{若两直线}\frac{x-1}1=\frac{y+1}2=\frac{z-1}λ,\frac{x+1}1=\frac{y-1}1=\frac z1\mathrm{相交},则λ=(\;).
$$
$$
A.
1 \quad B.\frac23 \quad C.-\frac54 \quad D.\frac54 \quad E. \quad F. \quad G. \quad H.
$$
$$
s_1=\{1,2,λ\},s_2=\{1,1,1\},\mathrm{由两线相交得}\begin{vmatrix}-1-1&1+1&0-1\\1&2&λ\\1&1&1\end{vmatrix}=\begin{vmatrix}-2&2&-1\\1&2&λ\\1&1&1\end{vmatrix}=0,λ=\frac54.
$$



$$
\begin{array}{l}\mathrm{设有两条直线}L_1:\left\{\begin{array}{l}x+y-1=0\\x-y+z+1=0\end{array}\right.,L_2:\left\{\begin{array}{l}2x-y+z-1=0\\x+y-z+1=0\end{array}\right.\mathrm{及平面}\;π:x+y+z=0,\mathrm{则在}π\mathrm{上且与直线}L_1和L_2\mathrm{相交的}\\\mathrm{直线方程为}(\;).\end{array}
$$
$$
A.
x=y=z \quad B.\frac x1=\frac{y-{\displaystyle\frac12}}2=\frac{z+{\displaystyle\frac12}}{-3} \quad C.\frac x1=\frac{y+{\displaystyle\frac12}}2=\frac{z-{\displaystyle\frac12}}{-3} \quad D.\frac x2=\frac{y-{\displaystyle1}}3=\frac{z-2}1 \quad E. \quad F. \quad G. \quad H.
$$
$$
\begin{array}{l}L_1:\left\{\begin{array}{l}x+y-1=0\\x-y+z+1=0\end{array}\;,\;L_2:\left\{\begin{array}{l}2x-y+z-1=0\\x+y-z+1=0\end{array}\right.\right.,\mathrm{与平面交点分别为}(\frac12,\frac12,-1)和(0,-\frac12,\frac12),\\\mathrm{两点所确定的直线}\frac x1=\frac{y+{\displaystyle\frac12}}2=\frac{z-{\displaystyle\frac12}}{-3}\mathrm{即为所求}.\end{array}
$$



$$
\mathrm{用对称方程及参数方程表示直线}\left\{\begin{array}{l}x+y+z+1=0\\2x-y+3z+4=0\end{array}\right.,\mathrm{下面叙述正确的是}(\;).
$$
$$
A.
\mathrm{对称式方程}\frac{x-1}4=\frac{y-0}{-1}=\frac{z+2}{-3},\mathrm{参数方程}\left\{\begin{array}{c}x=1+4t\\y=-t\\z=-2-3t\end{array}\right. \quad B.\mathrm{对称式方程}\frac{x-1}4=\frac{y-0}1=\frac{z+2}3,\mathrm{参数方程}\left\{\begin{array}{c}x=1+4t\\y=-t\\z=-2-3t\end{array}\right. \quad C.\mathrm{对称式方程}\frac{x-1}4=\frac{y-0}{-1}=\frac{z+2}{-3},\mathrm{参数方程}\left\{\begin{array}{c}x=1-4t\\y=-t\\z=-2-3t\end{array}\right. \quad D.\mathrm{对称式方程}\frac{x-1}3=\frac{y-0}{-1}=\frac{z+2}{-3},\mathrm{参数方程}\left\{\begin{array}{c}x=1+3t\\y=-t\\z=-2-3t\end{array}\right. \quad E. \quad F. \quad G. \quad H.
$$
$$
\begin{array}{l}\mathrm{在直线上任取一点}(x_0,\;y_0,\;z_0),\\\mathrm{例如},取x_0=1\rightarrow\left\{\begin{array}{l}y_0+z_0+2=0\\y_0-3z_0-6=0\end{array}\right.\rightarrow y_0=0,z_0=-2,\\\mathrm{得点坐标}(1,0,-2),\mathrm{因所求直线与两平面的法向量者垂直},\mathrm{可取}\\\;\;\;\;\;\;\;\;\;\;\;\;\;\;\;\;\;\;\;\;\;\;\;\;\;\;\;\;\;\;\;\;\;\;\;\;\;\;\;\;\;\;\;\;\;\;\;\;\;\;\;\;\;\;\;\;\;\;\overrightarrow s=\overrightarrow{n_1}×\overrightarrow{n_2}=\begin{vmatrix}\overrightarrow i&\overrightarrow j&\overrightarrow k\\1&1&1\\2&-1&3\end{vmatrix}=\{4,-1,-3\},\\\mathrm{对称式方程}\frac{x-1}4=\frac{y-0}{-1}=\frac{z+2}{-3},\mathrm{参数方程}\left\{\begin{array}{c}x=1+4t\\y=-1\\z=-2-3t\end{array}\right..\end{array}
$$



$$
\mathrm{过点}M(2,3,5),\mathrm{且平行于平面}5x-3y+2z-10=0\mathrm{的平面是}(),
$$
$$
A.
5x+3y+2z-11=0 \quad B.5x-3y+2z+11=0 \quad C.5x-3y+2z-11=0 \quad D.5x+3y+2z+11=0 \quad E. \quad F. \quad G. \quad H.
$$
$$
\begin{array}{l}\mathrm{根据两平面平行的性质},\mathrm{可得所求平面的方程为}5x-3y+2z+D=0\;\;,\\\mathrm{且过点}M(2,3,5),得D=-11,\;\;\mathrm{所以所求方程为}5x-3y+2z-11=0.\end{array}
$$



$$
\mathrm{平面}3x-5z+1=0(),
$$
$$
A.
\mathrm{平行于}zOx\mathrm{平面} \quad B.\mathrm{平行于}y轴 \quad C.\mathrm{垂直于}y轴 \quad D.\mathrm{垂直于}x轴 \quad E. \quad F. \quad G. \quad H.
$$
$$
\mathrm{平面法向量为}\{3,0,-5\}\mathrm{垂直于}y轴,\mathrm{所以平面平行于}y轴.
$$



$$
\mathrm{平面是}3x-3y-6=0是(),
$$
$$
A.
\mathrm{平行于}xOy\mathrm{平面} \quad B.\mathrm{平行于}z轴,\mathrm{但不通过}z轴 \quad C.\mathrm{垂直于}y轴 \quad D.\mathrm{通过}z轴 \quad E. \quad F. \quad G. \quad H.
$$
$$
\mathrm{平面的法向量为}\{3,-3,0\},\mathrm{所以平面是平行于}z轴,\mathrm{又因为不过原点},\mathrm{所以不通过}z轴.
$$



$$
\mathrm{过点}M(2,4,-3)\mathrm{且与平面}2x+3y-5z=5\mathrm{平行的平面方程为}().
$$
$$
A.
2x+3y-5z=31 \quad B.2x+3y+5z=31 \quad C.x+3y-5z=17 \quad D.x-3y-5z=17 \quad E. \quad F. \quad G. \quad H.
$$
$$
\begin{array}{l}\mathrm{因为所求平面和已知平面平行},\mathrm{而已知平面的法向量为}\overrightarrow n=\{2,3,-5\}.\\\;\mathrm{设所求平面的法向量为}\overrightarrow n,则\overrightarrow n∥\overrightarrow{n_1},\mathrm{故可取}\overrightarrow n=\overrightarrow{n_1},\\\mathrm{于是},\mathrm{所求平面方程为}\;2(x-2)+3(y-4)-5(z+3)=0\;,\;\;\\即\;2x+3y-5z=31\;.\end{array}
$$



$$
\mathrm{已知三平面的方程分别为}:π_1:x-5y+2z+1=0;π_2:3x-2y+5z+8=0;π_3:2x-10y+4z-9=0\mathrm{则必有}().
$$
$$
A.
π_1与π_2\mathrm{平行} \quad B.π_1与π_3\mathrm{平行} \quad C.π_2与π_3\mathrm{垂直} \quad D.π_2与π_3\mathrm{平行} \quad E. \quad F. \quad G. \quad H.
$$
$$
\mathrm{显然}π_1与π_3\mathrm{的平行},\mathrm{其它选项均不正确}.
$$



$$
点(1,2,1)\mathrm{到平面}x+y+z+2=0\mathrm{的距离为}().
$$
$$
A.
1 \quad B.2 \quad C.2\sqrt3 \quad D.\sqrt3 \quad E. \quad F. \quad G. \quad H.
$$
$$
\begin{array}{l}\mathrm{由距离公式},得\\d=\frac{\left|1×1+2×1+1×1+2\right|}{\sqrt{1^2+1^2+1^2}}=2\sqrt3.\end{array}
$$



$$
\mathrm{平面}Ax+By+Cz+D=0过x轴,则().
$$
$$
A.
A=D=0 \quad B.B=0,C\neq0 \quad C.B\neq0,C=0 \quad D.A=0,C=0 \quad E. \quad F. \quad G. \quad H.
$$
$$
\mathrm{平面}Ax+By+Cz+D=0\mathrm{过轴},\mathrm{可得平面平行于}x\mathrm{轴且过原点}.\mathrm{所以可得}A=D=0;
$$



$$
\mathrm{平面}3x-5y+1=0().
$$
$$
A.
\mathrm{平行于}zOx\mathrm{平面} \quad B.\mathrm{平行于}z轴 \quad C.\mathrm{垂直于}y轴 \quad D.\mathrm{垂直于}x轴 \quad E. \quad F. \quad G. \quad H.
$$
$$
\mathrm{平面法向量为}\{3,-5,0\}\mathrm{垂直于}z轴,\mathrm{所以平面平行于}z轴.
$$



$$
点M(1,-2,1)\mathrm{到平面}x-2y+2z-10=0\mathrm{的距离为}(),
$$
$$
A.
1 \quad B.2 \quad C.3 \quad D.\frac13 \quad E. \quad F. \quad G. \quad H.
$$
$$
\mathrm{根据点到平面的距离公式可得}d=\frac{\left|1+4+2-10\right|}{\sqrt{1+4+4}}=1.
$$



$$
\mathrm{设平面}2x+y+kz+1=0\mathrm{垂直于平面}-2x+y+z+2=0,则k=().
$$
$$
A.
1 \quad B.2 \quad C.3 \quad D.0 \quad E. \quad F. \quad G. \quad H.
$$
$$
\mathrm{由题意可得}:2·(-2)+1·1+k=0,\mathrm{解得}:k=3.
$$



$$
\mathrm{设平面}2x+y+kz+1=0\mathrm{平行于平面}4x+ky+4z-3=0,则k=().
$$
$$
A.
0 \quad B.1 \quad C.2 \quad D.3 \quad E. \quad F. \quad G. \quad H.
$$
$$
\mathrm{由题意得}:\frac24=\frac1k=\frac k4,\mathrm{解得}:k=2.
$$



$$
\mathrm{设平面}2x+y+kz+1=0\mathrm{到原点}O\mathrm{的距离为}\frac13,则k=().
$$
$$
A.
±2 \quad B.2 \quad C.±1 \quad D.1 \quad E. \quad F. \quad G. \quad H.
$$
$$
\mathrm{原点}O=(0,0,0),则:d=\frac{\left|0+0+0+1\right|}{\sqrt{2^2+1+k^2}}=\frac13,\mathrm{解得}:k=±2.
$$



$$
\mathrm{已知原点到平面}2x-y+kz=6\mathrm{的距离等于}2,则k\mathrm{的值为}().
$$
$$
A.
±2 \quad B.±1 \quad C.2 \quad D.1 \quad E. \quad F. \quad G. \quad H.
$$
$$
d=\frac{\left|0-0+0-6\right|}{\sqrt{2^2+(-1)^2+k^2}}=2,\mathrm{解得}:k=±2.
$$



$$
\mathrm{平面}3x-3y=0是\;().
$$
$$
A.
\mathrm{平行于}xOy\mathrm{平面} \quad B.\mathrm{平行于}z轴,\mathrm{但不通过}z轴 \quad C.\mathrm{垂直于}y轴 \quad D.\mathrm{通过}z轴 \quad E. \quad F. \quad G. \quad H.
$$
$$
\mathrm{平面的法线为}\{1,-1,0\},\mathrm{所以平面是平行于}z轴,\mathrm{又因为经过原点},\mathrm{所以通过}z轴.
$$



$$
\mathrm{过点}(1,2,-1)\mathrm{且与直线}\frac{x-1}2=y+1=\frac{z-1}{-1}\mathrm{垂直的平面方程为}(\;\;\;)
$$
$$
A.
2x+y-z-5=0 \quad B.2x+y-z+5=0 \quad C.2x-y-z-5=0 \quad D.2x-y+z-5=0 \quad E. \quad F. \quad G. \quad H.
$$
$$
\begin{array}{l}\mathrm{平面的法向量}n=\{2,1,-1\},\\\mathrm{由点法式方程得}:2·(x-1)+1·(y-2)-1·(z+1)=0\;\;\;\\\mathrm{整理得},2x+y-z-5=0.\end{array}
$$



$$
\mathrm{经过}z\mathrm{轴和点}(1,1,1)\mathrm{的平面方程的为}(\;\;\;)
$$
$$
A.
x+y=0 \quad B.x-z=0 \quad C.x-y=0 \quad D.z-y=0 \quad E. \quad F. \quad G. \quad H.
$$
$$
\begin{array}{l}\mathrm{平面经过}z轴,\mathrm{故可设平面的方程为}Ax+By=0\;\mathrm{又过点}(1,1,1),\\得A=-B,\mathrm{整理得}x-y=0,\end{array}
$$



$$
\mathrm{过点}(1,0,-1)\mathrm{且垂直于直线}\left\{\begin{array}{l}x+2y-z-1=0\\2x-y+z+1=0\end{array}\right.\mathrm{的平面方程为}(\;\;\;)
$$
$$
A.
x+3y-5z-6=0 \quad B.x-3y+5z-6=0 \quad C.x-3y-5z+6=0 \quad D.x-3y-5z-6=0 \quad E. \quad F. \quad G. \quad H.
$$
$$
\begin{array}{l}\mathrm{直线的方向向量}s=\begin{vmatrix}i&j&k\\1&2&-1\\2&-1&1\end{vmatrix}=\left\{1,-3,-5\right\},\\\mathrm{也就是所求平面的法向量},\;\;\;\\\mathrm{由点法式方程得}:\;1·(x-1)-3·(y-0)-5·(z+1)=0,\;\\\mathrm{整理得},x-3y-5z-6=0.\end{array}
$$



$$
\mathrm{通过}x\mathrm{轴且与平面}2x+y-z=0\mathrm{垂直的平面方程为}\;(\;\;\;\;\;)
$$
$$
A.
y-z=0 \quad B.y+z=0 \quad C.x-z=0 \quad D.x-y=0 \quad E. \quad F. \quad G. \quad H.
$$
$$
\begin{array}{l}\mathrm{平面经过}x轴,\mathrm{故可设平面的方程为}By+Cz=0,\\\;\mathrm{已知平面的法向量}n=\left\{2,1,-1\right\},\;\;\mathrm{由条件知},2·0+1· B+(-1)· C=0,⇒ B=C,\\\mathrm{故所求平面为}y+z=0.\end{array}
$$



$$
\mathrm{过点}(2,-3,1)\mathrm{和平面}y+3z+4=0\mathrm{平行的平面方程为}(\;\;\;\;)
$$
$$
A.
y-3z=0. \quad B.y+3z=0. \quad C.3y+z=0. \quad D.3y-z=0. \quad E. \quad F. \quad G. \quad H.
$$
$$
\mathrm{由点法式方程得}:0·(x-2)+1·(y+3)+3·(z-1)=0\;\mathrm{整理得},y+3z=0
$$



$$
\mathrm{过点}(4,0,-2)\mathrm{和点}(5,1,7)\mathrm{且平行于}z\mathrm{轴的平面方程为}(\;\;\;\;\;)
$$
$$
A.
x+y-4=0 \quad B.x-y+4=0 \quad C.x-y-4=0 \quad D.x+y+4=0 \quad E. \quad F. \quad G. \quad H.
$$
$$
\begin{array}{l}\mathrm{可设平面的方程为}Ax+By+D=0,\\\mathrm{又经过点}(4,0,-2)\mathrm{和点}(5,1,7),得\;\left\{\begin{array}{l}4A+D=0\\5A+B+D=0\end{array}\right.,⇒ B=-A,D=-4A,\\\mathrm{故所求平面为}x-y-4=0\end{array}
$$



$$
\mathrm{过点}(1,0,1)\mathrm{且垂直于直线}\left\{\begin{array}{l}x+2y-z-1=0\\2x-y+z+1=0\end{array}\right.\mathrm{的平面方程为}(\;\;\;)\;
$$
$$
A.
x-3y-5z+4=0 \quad B.x+3y-5z-6=0 \quad C.x-3y+5z-6=0 \quad D.x-3y-5z+6=0 \quad E. \quad F. \quad G. \quad H.
$$
$$
\begin{array}{l}\mathrm{直线的方向向量}s=\begin{vmatrix}i&j&k\\1&2&-1\\2&-1&1\end{vmatrix}=\left\{1,-3,-5\right\},\mathrm{也就是所求平面的法向量},\;\;\;\\\mathrm{由点法式方程得}:1·(x-1)-3·(y-0)-5·(z-1)=0\;\;\mathrm{整理得},x-3y-5z+4=0.\end{array}
$$



$$
\mathrm{过点}(2,-3,1)\mathrm{与平面}x+4y-3z+7=0\mathrm{平行的平面方程为}(\;\;\;)
$$
$$
A.
x+4y-3z+13=0 \quad B.x+4y-3z-13=0 \quad C.x+4y+3z+13=0 \quad D.x-4y+3z-13=0 \quad E. \quad F. \quad G. \quad H.
$$
$$
\mathrm{由点法式方程得}:1·(x-2)+4·(y+3)-3·(z-1)=0\;\mathrm{整理得},x+4y-3z+13=0
$$



$$
\mathrm{过点}(1,5,-2)\mathrm{与平面}2x-7y+z-8=0\mathrm{平行的平面方程为}(\;\;\;)\;
$$
$$
A.
2x-7y+z-35=0 \quad B.2x-7y-z+35=0 \quad C.2x-7y+z+5=0 \quad D.2x-7y+z+35=0 \quad E. \quad F. \quad G. \quad H.
$$
$$
\mathrm{由点法式方程得}:2·(x-1)-7·(y-5)+1·(z+2)=0,\;\mathrm{整理得},2x-7y+z+35=0
$$



$$
\mathrm{若平面}x+2y-kz=1\mathrm{与平面}y-z=3成\fracπ4角,则k=().
$$
$$
A.
\frac14 \quad B.\frac34 \quad C.-\frac14 \quad D.-\frac34 \quad E. \quad F. \quad G. \quad H.
$$
$$
\begin{array}{l}n_1=\{1,2,-k\},n_2=\{0,1,-1\},则:\cos\fracπ4=\frac{\left|2+k\right|}{\sqrt{1+4+k}^2\sqrt{1+1}}=\frac{\sqrt2}2,\;\\\mathrm{解得}:k=\frac14.\end{array}
$$



$$
点(3,1,-1)\mathrm{到平面}22x+4y-20z-45=0\mathrm{的距离等于}().\;
$$
$$
A.
\frac32 \quad B.\frac12 \quad C.1 \quad D.2 \quad E. \quad F. \quad G. \quad H.
$$
$$
d=\frac{\left|22·3+4·1-20·(-1)-45\right|}{\sqrt{22^2+4^2+(-20)^2}}=\frac32
$$



$$
\mathrm{平面}19x-4y+8z+21=0和19x-4y+8z+42=0\mathrm{之间的距离等于}().
$$
$$
A.
1 \quad B.2 \quad C.3 \quad D.4 \quad E. \quad F. \quad G. \quad H.
$$
$$
\mathrm{由于}:\frac{19}{19}=\frac{-4}{-4}=\frac88=1,\mathrm{这两平面平行},又\frac{21}{42}=\frac12,\mathrm{因此相差一个单位},\mathrm{得两平面间距离为}1.
$$



$$
\mathrm{过点}(1,2,1)\mathrm{与向量}S_1=i-2j-3k,S_2=-j-k\mathrm{平行的平面方程为}().
$$
$$
A.
x-y+z=0 \quad B.x-y-z=0 \quad C.x+y+z=0 \quad D.x+y-z=0 \quad E. \quad F. \quad G. \quad H.
$$
$$
\begin{array}{l}\mathrm{设所求平面方程的法向量为}\overrightarrow n=\{A,B,C\},\mathrm{则由题意得}:A-2B-3C=0\;,-B-C=0,\\\mathrm{可得}\overrightarrow n=\{1,-1,1\},\mathrm{又点}(1,2,1)\mathrm{在平面上},\mathrm{得平面方程为}:(x-1)-(y-2)+(z-1)=0,\mathrm{整理得}:x-y+z=0.\end{array}
$$



$$
\mathrm{平面}2x-3y-z+12=0在x轴,y轴,z\mathrm{轴上的截距分别为}().
$$
$$
A.
-6,4,12 \quad B.6,4,12 \quad C.-6,-4,12 \quad D.6,-4,12 \quad E. \quad F. \quad G. \quad H.
$$
$$
\begin{array}{l}当y=z=0时,\mathrm{那么}:x=-6,\mathrm{则平面在轴上截距为}:a=-6;\\当x=z=0时,\mathrm{那么}:y=4,\mathrm{则平面在轴上截距为}:b=4;\;\\当x=y=0时,\mathrm{那么}:z=12,\mathrm{则平面在轴上截距为}:c=12\end{array}
$$



$$
\mathrm{过点}M_0(1,0,2)\mathrm{且与连接坐标原点及点}M_0\mathrm{的线段}OM_0\mathrm{垂直平面方程是}().
$$
$$
A.
x+2z-5=0 \quad B.x+2z+5=0 \quad C.x+y+2z-5=0 \quad D.x+2z-3=0 \quad E. \quad F. \quad G. \quad H.
$$
$$
\begin{array}{l}\mathrm{注意到}\overrightarrow{OM_0}\mathrm{即为所求平面的法向量},由\overrightarrow n=\;\overrightarrow{OM_0}=±\{1,0,2\},\\\;\mathrm{根据点法式平面方程},\mathrm{所求平面方程为}\;±\lbrack(x-1)+0(y-0)+2(z-2)\rbrack=0,\\即x+2z-5=0.\end{array}
$$



$$
\mathrm{平行于}x轴,\mathrm{且过点}P(3,-1,2)及Q(0,1,0)\mathrm{的平面方程是}().
$$
$$
A.
y+z=1 \quad B.y-z=1 \quad C.-y+z=1 \quad D.y+z=0 \quad E. \quad F. \quad G. \quad H.
$$
$$
\begin{array}{l}\mathrm{平行于}x轴,\mathrm{则可设平面为}By+Cz+D=0,\mathrm{代入两点得}\left\{\begin{array}{l}-B+2C+D=0\\B+D=0\end{array}\right.,\\\mathrm{解得}B=C=-D,\mathrm{所以该平面方程为}y+z=1.\end{array}
$$



$$
\mathrm{平面}Ax+By+Cz+D=0过z轴,则\;
$$
$$
A.
A=D=0 \quad B.B=0,C\neq0 \quad C.C=D=0 \quad D.B=C=0 \quad E. \quad F. \quad G. \quad H.
$$
$$
\mathrm{平面}Ax+By+Cz+D=0过z轴,\mathrm{可得平面平行于}z\mathrm{轴且过原点}.\mathrm{所以可得}C=D=0;
$$



$$
\mathrm{过点}M_0(2,-1,3)\mathrm{且与连接坐标原点及点}M_0\mathrm{的线段}OM_0\mathrm{垂直的平面方程是}()
$$
$$
A.
2x-y+3z-12=0 \quad B.2x-y+3z-14=0 \quad C.2x-y+3z+10=0 \quad D.2x-y+3z+11=0 \quad E. \quad F. \quad G. \quad H.
$$
$$
\begin{array}{l}\mathrm{注意到}\overrightarrow{OM_0}\mathrm{即为所求平面的法向量},由\;\overrightarrow n=\overrightarrow{OM_0}=±\{2,-1,3\},\;\mathrm{根据点法式平面方程},\mathrm{所求平面方程为}\;,\\±\lbrack2(x-2)-(y+1)+3(z-3)\rbrack=0,即2x-y+3z-14=0.\end{array}
$$



$$
\mathrm{已知平面}Ax+By+Cz+D=0\mathrm{过点}(k,k,0)与(2k,2k,0),k\neq0\mathrm{且垂直于}xOy\mathrm{平面},\mathrm{则其系数满足}()
$$
$$
A.
A=-B,C=D=0 \quad B.B=-C,A=D=0 \quad C.C=-A,B=D=0 \quad D.C=A,B=D=0 \quad E. \quad F. \quad G. \quad H.
$$
$$
\begin{array}{l}\mathrm{两点代入平面方程得}\left\{\begin{array}{l}Ak+Bk+D=0\\2Ak+2Bk+D=0\end{array}\right.\mathrm{可得}\left\{\begin{array}{l}D=0\\A=-B\end{array}\right.,\mathrm{平面又垂直于}xOy\mathrm{平面},\mathrm{所以}C=0,\mathrm{即系数满足}\\A=-B,C=D=0.\end{array}
$$



$$
\mathrm{两平面}-x+2y-z+1=0,y+3z-1=0\mathrm{的位置关系为}().
$$
$$
A.
\mathrm{两平面相交},\mathrm{夹角为}θ=arc\cos\frac1{\sqrt{60}} \quad B.\mathrm{两平面相交},\mathrm{夹角为}θ=\fracπ6 \quad C.\mathrm{两平面相交},\mathrm{夹角为}θ=\fracπ3 \quad D.\mathrm{两平面平行} \quad E. \quad F. \quad G. \quad H.
$$
$$
\begin{array}{l}\overrightarrow{n_1}=\{-1,2,-1\},\overrightarrow{n_2}=\{0,1,3\}且\cosθ=\frac{\left|-1×0+2×1-1×3\right|}{\sqrt{(-1)^2+2^2+(-1)^2·\sqrt{1^2+3^2}}}=\frac1{\sqrt{60}},\\\mathrm{故两平面相交},\mathrm{夹角为}θ=arc\cos\frac1{\sqrt{60}}.\end{array}
$$



$$
\mathrm{两平面}2x-y+z-1=0,-4x+2y-2z-1=0\;\mathrm{的位置关系为}().
$$
$$
A.
\mathrm{两平面平行但不重合} \quad B.\mathrm{两平面重合} \quad C.\mathrm{两平面相交},\mathrm{夹角为}θ=arc\cos\frac1{\sqrt{60}} \quad D.\mathrm{两平面相交},\mathrm{夹角为}θ=arc\cos\frac2{\sqrt{60}} \quad E. \quad F. \quad G. \quad H.
$$
$$
\overrightarrow{n_1}=\{2,-1,1\},\overrightarrow{n_2}=\{-4,2,-2\}且\;\;\;\frac2{-4}\;=\frac{-1}2\;=\frac1{-2}\;,\;\;又M(1,1,0)∈{\textstyle\prod_1},M(1,1,0)\not∈{\textstyle\prod_2},\mathrm{故两平面平行但不重合}.
$$



$$
\mathrm{求两平行平面}{\textstyle\prod_1}:10x+2y-2z-5=0和{\textstyle\prod_2}{\textstyle:}{\textstyle5}{\textstyle x}{\textstyle+}{\textstyle y}{\textstyle-}{\textstyle z}{\textstyle-}{\textstyle1}{\textstyle=}{\textstyle0}\mathrm{之间的距离}d=().
$$
$$
A.
\frac{\sqrt3}6 \quad B.\frac{5\sqrt3}6 \quad C.\frac{\sqrt3}5 \quad D.\frac{\sqrt3}{16} \quad E. \quad F. \quad G. \quad H.
$$
$$
\begin{array}{l}\mathrm{可在平面}\prod\nolimits_2\mathrm{上任取一点},\mathrm{该点到平面}\prod\nolimits_1\mathrm{的距离即为这两平面间的距离}.\;\\\mathrm{为此},\mathrm{在平面上取点}(0,1,0),则\;d=\frac{\left|10×0+2×1+)-(-2)×0-5\right|}{\sqrt{10^2+2^2+(-2)^2}}=\frac3{\sqrt{108}}=\frac{\sqrt3}6\;.\end{array}
$$



$$
\mathrm{平行于}x轴,\mathrm{且过点}P(3,-1,2)及Q(0,1,0)\mathrm{的平面方程是}().
$$
$$
A.
y+z=1 \quad B.y-z=1 \quad C.-y+z=1 \quad D.y+z=0 \quad E. \quad F. \quad G. \quad H.
$$
$$
\begin{array}{l}\mathrm{平行于}x轴,\mathrm{则可设平面为}By+Cz+D=0,\\\mathrm{代入两点得}\left\{\begin{array}{l}-B+2C+D=0\\B+D=0\end{array}\right.,\mathrm{解得}B=C=-D,\mathrm{所以该平面方程为}y+z=1.\end{array}
$$



$$
\mathrm{过点}M_0(2,9,-6)\mathrm{且与连接坐标原点及点}M_0\mathrm{的线段}OM_0\mathrm{垂直的平面方程是}().
$$
$$
A.
2x+9y-6z-121=0 \quad B.2x+9y-6z+121=0 \quad C.2x+9y-6z-124=0 \quad D.x+9y-6z-121=0 \quad E. \quad F. \quad G. \quad H.
$$
$$
\begin{array}{l}\mathrm{注意到}\overrightarrow{OM_0}\mathrm{即为所求平面的法向量},由\;\overrightarrow n=\overrightarrow{OM_0}=±\{2,9,-6\},\\\;\mathrm{根据点法式平面方程},\mathrm{所求平面方程为}\;±\lbrack2(x-2)+9(y-9)-6(z+6)\rbrack=0,\\即2x+9y-6z-121=0.\end{array}
$$



$$
\mathrm{通过直线}x=2t-1,y=3t+2,z=2t-3\mathrm{和直线}x=2t+3,y=3t-3,z=2t+1\mathrm{的平面方程为}().
$$
$$
A.
x-z-2=0 \quad B.x+z=0 \quad C.x-2y+z=0 \quad D.2x+3y+2z=-2 \quad E. \quad F. \quad G. \quad H.
$$
$$
\begin{array}{l}\mathrm{两直线的方向向量都为}s_1=\{2,3,2\},\mathrm{此为两平行直线},\\\mathrm{分别取两直线上的点}(-1,2,-3)和(3,-3,1),\mathrm{两点所在直线的方向向量为}s_2=\{4,-5,4\},\\\mathrm{平面的法向量为}s_1× s_2=\{1,0,-1\},\mathrm{可解得平面方程为}x-z-2=0\end{array}
$$



$$
\mathrm{过点}M_1(1,1,2),M_2(3,2,3),M_3(2,0,3)\mathrm{三点的平面方程为}().
$$
$$
A.
2x-y-3z+5=0 \quad B.2x-y-z-5=0 \quad C.2x-3y-13z+5=0 \quad D.2x+y-3z+5=0 \quad E. \quad F. \quad G. \quad H.
$$
$$
\begin{array}{l}\mathrm{设所求平面的法向量}\overrightarrow n,\mathrm{因为}\;\overrightarrow{M_1M_2}=\overrightarrow{2i}\;+\overrightarrow j+\overrightarrow k,\;\overrightarrow{M_1M_3}=\overrightarrow i\;-\overrightarrow j+\overrightarrow k,\;\\\;\mathrm{故可取}\overrightarrow n=\;\overrightarrow{M_1M_2}×\overrightarrow{M_1M_3}=\begin{vmatrix}\overrightarrow i&\overrightarrow j&\overrightarrow k\\2&1&1\\1&-1&1\end{vmatrix}=\overrightarrow{2i}-\overrightarrow j-\overrightarrow{3k},\;\;\mathrm{又平面过点}M_1(1,1,2),\\\mathrm{从而所求平面方程为}2(x-1)-(y-1)-3(z-2)=0,\;\;即2x-y-3z+5=0.\end{array}
$$



$$
\mathrm{平面过原点}O,\mathrm{且垂直于平面}{\textstyle\prod_1}\;:x+2y+3z-2=0,{\textstyle\prod_2}:6x-y+5z+2=0\;\;\mathrm{则此平面方程为}().
$$
$$
A.
x+y-z=0 \quad B.x-y-z=0 \quad C.x+2y-z=0 \quad D.x+y-2z=0 \quad E. \quad F. \quad G. \quad H.
$$
$$
\begin{array}{l}\mathrm{设所求的平面的法向量}\overrightarrow n,\mathrm{依题设有}\overrightarrow n⟂\overrightarrow{n_1},\overrightarrow n⟂\overrightarrow{n_2},\;\;\mathrm{从而}\overrightarrow n=\overrightarrow{n_1}×\overrightarrow{n_2},\\\mathrm{其中}\overrightarrow{n_1}=\overrightarrow i+\overrightarrow{2j}+\overrightarrow{3k},\overrightarrow{n_2}=\overrightarrow{6i}-\overrightarrow j+\overrightarrow{5k},\;∴\overrightarrow n=\overrightarrow{n_1}×\overrightarrow{n_2}=\;\begin{vmatrix}\overrightarrow i&\overrightarrow j&\overrightarrow k\\1&2&3\\6&-1&5\end{vmatrix}=\overrightarrow{13i}+\overrightarrow{13j}-\overrightarrow{13k},\;\;\\\mathrm{由点法式得所求平面方程}13x+13y-13z=0.\;\;即x+y-z=0.\end{array}
$$



$$
\mathrm{过点}A(2,-1,4),B(-1,3,-2)和C(0,2,3)\mathrm{的平面方程为}().
$$
$$
A.
14x+9y-z-15=0 \quad B.14x-9y-z-33=0 \quad C.14x-9y+z-15=0 \quad D.x+9y-z-15=0 \quad E. \quad F. \quad G. \quad H.
$$
$$
\begin{array}{l}解\;\;\overrightarrow{AB}=\{-3,4,-6\},\;\;\overrightarrow{AC}=\{-2,3,-1\}\;,\;\;\\取\;\overrightarrow n\;=\;\overrightarrow{AB}×\overrightarrow{AC}\;=\begin{vmatrix}\overrightarrow i&\overrightarrow j&\overrightarrow k\\-3&4&-6\\-2&3&-1\end{vmatrix}\;=\overrightarrow{14}+\overrightarrow{9j}-\overrightarrow k,\;\;\\\mathrm{所以平面方程为}\;\;14(x-2)+9(y+1)-(z-4)=0\;\;.\;\;\mathrm{化简得}\;14x+9y-z-15=0\;.\end{array}
$$



$$
\mathrm{通过}x\mathrm{轴和点}(4,-3,-1)\mathrm{的平面方程为}().
$$
$$
A.
y-3z=0 \quad B.y+3z=0 \quad C.2x+y-3z=0 \quad D.x+2y-3z=0 \quad E. \quad F. \quad G. \quad H.
$$
$$
\begin{array}{l}\mathrm{设所求平面的一般方程为}\;Ax+By+Cz+D=0\;,\;\;\mathrm{因为所求平面通过}x轴,\\\mathrm{且法向量垂直于}x轴,\mathrm{于是法向量在}x\mathrm{轴上的投影为零},\\即A=0,\mathrm{又平面通过原点},\mathrm{所以}D=0,\mathrm{从而方程成为}\;By+Cz=0\;\;\\\mathrm{又因平面过点}(4,-3,-1),\mathrm{因此有}\;-3B-C=0\;,即C=-3B.\;\;\mathrm{以此代入方程}(1),\\\mathrm{再除以}B(B\neq0),\mathrm{便得到所求方程为}\;y-3z=0\;.\end{array}
$$



$$
\mathrm{设平面过原点及点}(6,-3,2),\mathrm{且与平面}4x-y+2z=8\mathrm{垂直},\mathrm{则此平面方程为}().
$$
$$
A.
2x+2y-3z=0 \quad B.2x-2y+3z=0 \quad C.2x+2y-z=0 \quad D.2x-y-3z=0 \quad E. \quad F. \quad G. \quad H.
$$
$$
\begin{array}{l}\mathrm{设平面为}Ax+By+Cz+D=0\;\;\;\;,\;\;\mathrm{由平面过原点知}D=0,\;\;\\\mathrm{由平面过点}(6,-3,2)知.\;6A-3B+2C=0\;,∵\{A,B,C\}⟂\{4,-1,2\},∵4A-B+2C=0\rightarrow A=B=-\frac23C,\;\;\\\mathrm{所求平面方程为}\;2x+2y-3z=0.\;\;\;\;\;\;\;\;\;\;\;\;\;\;\;\;\end{array}
$$



$$
\begin{array}{l}\mathrm{已知三角形的顶点为点}A(2,1,5),B(0,4,-1)和C(3,4,-7),\mathrm{通过点}M(2,-6,3)\mathrm{作一平面}π,使π\mathrm{平行于}\\\triangle ABC\mathrm{所在平面},\mathrm{则方程为}\;().\end{array}
$$
$$
A.
6x+10y+3z+39=0 \quad B.2x+3y+4z-7=0 \quad C.3x+2y-z+18=0 \quad D.4y-x+z-10=0 \quad E. \quad F. \quad G. \quad H.
$$
$$
\begin{array}{l}\mathrm{考察三角形所在平面}\overrightarrow{AB}=\{-2,3,-6\},\overrightarrow{AC}=\{1,3,-12\},\\\mathrm{所以}\overrightarrow n=\overrightarrow{AB}×\overrightarrow{AC}=\{-18,-30,-9\},\mathrm{且过点}A(2,1,5)\mathrm{所以该平面为}\;6(x-2)+10(y-1)+3(z-5)=0,\\\mathrm{整理得}6x+10y+3z-37=0,\mathrm{所以平面}π\mathrm{的方程设为}6x+10y+3z+D=0,\\\mathrm{代入点}M(2,-6,3)\mathrm{解得}D=39,\mathrm{所以}π\mathrm{的方程为}6x+10y+3z+39=0.\end{array}
$$



$$
\mathrm{经过两平面}4x-y+3z-1=0,x+5y-z+2=0\mathrm{的交线作平面π},\mathrm{并使π与}y\mathrm{轴平行},\mathrm{则平面π的方程为}().
$$
$$
A.
14x-21z-3=0 \quad B.21x-14z+3=0 \quad C.21x+14z-3=0 \quad D.21x+14z+3=0 \quad E. \quad F. \quad G. \quad H.
$$
$$
\begin{array}{l}\mathrm{经过两平面}4x-y+3z-1=0,x+5y-z+2=0\mathrm{的交线方程}\left\{\begin{array}{l}4x-y+3z-1=0\\x+5y-z+2=0\end{array}\right.\\\mathrm{通过此直线的平面束方程为}4x-y+3z-1+λ(x+5y-z+2)=0,\\即(4+λ)x+(5λ-1)y+(3-λ)z-1+2λ=0\mathrm{该平面与}y\mathrm{轴平行},\\\mathrm{则有}5λ-1=0,λ=1/5,\mathrm{所以平面方程为}21x+14z-3=0.\end{array}
$$



$$
\mathrm{过点}M_1(1,1,2),M_2(3,2,3),M_3(2,0,3)\mathrm{三点的平面方程为}().
$$
$$
A.
2x-y-3z+5=0 \quad B.2x-y-z+5=0 \quad C.2x-y-3z+1=0 \quad D.2x+y-3z+3=0 \quad E. \quad F. \quad G. \quad H.
$$
$$
\begin{array}{l}\mathrm{设所求平面方程向量为}n,\mathrm{因为}\overrightarrow{M_1M_2}=2i+j+k,\overrightarrow{M_1M_3}=i-j+k\\\mathrm{故可取}:n=\overrightarrow{M_1M_2}×\overrightarrow{M_1M_3}=\begin{vmatrix}i&j&k\\2&1&1\\1&-1&1\end{vmatrix}=2i-j-3k,\mathrm{又平面过点}M_1(1,1,2),\\\mathrm{从而所求平面方程为}:2(x-1)-(y-1)-3(z-2)=0,即:2x-y-3z+5=0.\end{array}
$$



$$
\mathrm{过点}M_0(2,9,-6)\mathrm{且与连接坐标原点及点}M_0\mathrm{的线段}OM_0\mathrm{垂直平面方程是}().
$$
$$
A.
2x+9y-6z-121=0 \quad B.2x+9y-6z+121=0 \quad C.2x+9y-6z-124=0 \quad D.x+9y-6z-121=0 \quad E. \quad F. \quad G. \quad H.
$$
$$
\begin{array}{l}\mathrm{注意到}\overrightarrow{OM_0}\mathrm{即为所求平面的法向量},由\;\overrightarrow n=\overrightarrow{OM_0}=±\{2,9,-6\},\;\\\mathrm{根据点法式平面方程},\mathrm{所求平面方程为}\;±\lbrack2(x-2)+9(y-9)-6(z+6)\rbrack=0,\\即2x+9y-6z-121=0.\end{array}
$$



$$
\mathrm{过直线}L:\left\{\begin{array}{l}x+2y-z-6=0\\x-2y+z=0\end{array}\right.\mathrm{作平面}{\textstyle\prod_{}},\mathrm{使它垂直于平面}{\textstyle\prod_1}:x+2y+z=0,\mathrm{则平面}{\textstyle\prod_{}}\mathrm{的方程为}().
$$
$$
A.
3x-2y+z+6=0 \quad B.3x-2y+z-6=0 \quad C.3x-2y-z-6=0 \quad D.x-2y-z-6=0 \quad E. \quad F. \quad G. \quad H.
$$
$$
\begin{array}{l}\mathrm{设过直线的平面束的方程为}(x+2y-z-6)+λ(x-2y+z)=0\;\;,\;\\\;即(1+λ)x+2(1-λ)y+(λ-1)z-6=0.\;\;\\\mathrm{现要在上述平面束中找出一个平面}{\textstyle\prod_{}},\mathrm{使它垂直于题设平面}{\textstyle\prod_1},\\\mathrm{因平面}{\textstyle\prod_{}}\mathrm{垂直于平面}{\textstyle\prod_1},\mathrm{故平面}{\textstyle\prod_{}}\mathrm{的法向量}\overrightarrow n(λ)\mathrm{垂直于平面}{\textstyle\prod_1}\mathrm{的法向量}\overrightarrow{n_1}=\{1,2,1\},\\\mathrm{于是}\overset{}{\overrightarrow n(λ)}·\overrightarrow n=0,即\;\;1·(1+λ)+4(1-λ)+(λ-1)=0\;,\;\;\\\mathrm{解得}λ=2,\mathrm{故所求平面方程为}\;\;π:3x-2y+z-6=0\;\;\;\;.\;\;\\\end{array}
$$



$$
\mathrm{过点}(1,2,1)\mathrm{且与两直线}\left\{\begin{array}{l}x+2y-z+1=0\\x-y+z-1=0\end{array}\right.\;和\left\{\begin{array}{l}2x-y+z=0\\x-y+z=0\end{array}\right.\mathrm{都平行的平面方程是}().
$$
$$
A.
x-y+z=0 \quad B.x-y-z=0 \quad C.x+y+z-4=0 \quad D.2x-y+z=0 \quad E. \quad F. \quad G. \quad H.
$$
$$
\begin{array}{l}\mathrm{注意到所求平面的法向量}\overrightarrow n\mathrm{与两直线的方向向量}\overrightarrow{s_1}\overrightarrow{s_2}\mathrm{都垂直},\mathrm{故可设}\overrightarrow n=\overrightarrow{s_1}×\overrightarrow{s_2}.\mathrm{因为}\\\overrightarrow{s_1}\;=\begin{vmatrix}\overrightarrow i&\overrightarrow j&\overrightarrow k\\1&2&-1\\1&-1&1\end{vmatrix}\;=\{1,-2,-3\},\overrightarrow{s_2}\;=\begin{vmatrix}\overrightarrow i&\overrightarrow j&\overrightarrow k\\2&-1&1\\1&-1&1\end{vmatrix}\;=\{0,-1,-1\},\;∴\overrightarrow n=\overrightarrow{s_1}×\overrightarrow{s_2}=\begin{vmatrix}\overrightarrow i&\overrightarrow j&\overrightarrow k\\1&-2&-3\\0&-1&-1\end{vmatrix}\;=\{-1,1,-1\}\;.\;\\\;\mathrm{故所求平面方程为}-(x-1)+(y-2)-(z-1)=0,\;\;即x-y+z=0.\end{array}
$$



$$
\mathrm{适合哪一组条件的平面是存在且唯一的}().
$$
$$
A.
\mathrm{过一已知点},\mathrm{且与两条已知异面直线平行} \quad B.\mathrm{垂直于已知平面},\mathrm{且经过一已知直线} \quad C.\mathrm{与两条已知直线垂直},\mathrm{又经过一已知点} \quad D.\mathrm{过一已知点},\mathrm{并与另一已知直线平行} \quad E. \quad F. \quad G. \quad H.
$$
$$
\mathrm{过一已知点},\mathrm{且与两条已知异面直线平行};\mathrm{过此点分别作两异面直线的平行线},\mathrm{所得到两条直线所在的平面即为所求},\mathrm{是唯一的}.
$$



$$
\mathrm{平行于平面}x+y+z=100\mathrm{且与球面}x^2+y^2+z^2=4\mathrm{相切的平面方程为}().
$$
$$
A.
x+y+z+2\sqrt3=0或x+y+z-2\sqrt3=0 \quad B.x+y+2z+2\sqrt3=0或x+y+2z-2\sqrt3=0 \quad C.x+y+z+2\sqrt3=0或x+2y+z-2\sqrt3=0 \quad D.x+y+z+2=0或x+y+z-2\sqrt3=0 \quad E. \quad F. \quad G. \quad H.
$$
$$
\begin{array}{l}\mathrm{平行于}x+y+z=100\mathrm{的平面方程可设为}\;\;:{\textstyle\prod_{}}:x+y+z+D=0\\\;\mathrm{因为}{\textstyle\prod_{}}与x^2+y^2+z^2=4\mathrm{相切},\mathrm{所以}\;\;{\frac{\left|x+y+z+D\right|}{\sqrt{1^2+1^2+1^2}}}_{(x,y,z)=(0,0,0)}=2,\;\;\\即\left|D\right|=2\sqrt3.\;\;\\\mathrm{所以要求的平面方程为}x+y+z+2\sqrt3=0或x+y+z-2\sqrt3=0.\end{array}
$$



$$
\mathrm{经过两点}M_1(3,-2,9)和M_2(-6,0,-4)\mathrm{且与平面}2x-y+4z-8=0\mathrm{垂直的平面的方程为}().
$$
$$
A.
x-2y-z+2=0 \quad B.x-2y-z-2=0 \quad C.x+2y-z+21=0 \quad D.3x-2y-5z+2=0 \quad E. \quad F. \quad G. \quad H.
$$
$$
\begin{array}{l}\mathrm{设所求的平面方程为}Ax+By+Cz+D=0\;\;\\\;\mathrm{由于点}M_1和M_2\mathrm{在平面上},故\;\;3A-2B+9C+D=0,-6A-4C+D=0\;.\;\;\\\mathrm{又由于所求平面与平面}2x-y+4z-8=0\mathrm{垂直},\\\mathrm{由两平面垂直条件有}\;2A-B+4C=0\;.\;\;\\\mathrm{从上面三个方程中解出}A,B,C,得A=\frac D{2,}\;\;B=-D,C=-\frac D2,\;\\\;\mathrm{代入所设方程},\mathrm{并约去因子}\frac D2,\mathrm{得所求的平面方程}\;x-2y-z+2=0\;.\end{array}
$$



$$
\mathrm{已知动点与平面}yOz\mathrm{的距离为}4\mathrm{个单位},\mathrm{且与定点}A(5,2,-1)\mathrm{的距离为}3\mathrm{个单位},\mathrm{则动点的轨迹是}().
$$
$$
A.
\mathrm{圆柱面} \quad B.\mathrm{平面}x=4\mathrm{上的圆} \quad C.\mathrm{平面}x=4\mathrm{上的椭圆} \quad D.\mathrm{椭圆柱面} \quad E. \quad F. \quad G. \quad H.
$$
$$
\begin{array}{l}\mathrm{动点与}yOz\mathrm{平面的距离为}4\mathrm{个单位},\\\mathrm{则可得出该点是平面}x=±4\mathrm{上的点},\\\mathrm{与定点}A(5,2,-1)\mathrm{的距离为}3\mathrm{个单位},\mathrm{设该点为}(±4,y,z),\\当x=-4时,\mathrm{此点与}A(5,2,-1)\mathrm{距离不可能为}3,\mathrm{所以该点为}(4,y,z),\\\mathrm{有等式}1+(y-2)^2+(z+1)^2=9,\mathrm{其为平面}x=4\mathrm{上的圆}\end{array}
$$



$$
\mathrm{两平面}x-2y-z=3,2x-4y-2z=5\mathrm{各自与平面}x+y-3z=0\mathrm{的交线是}().
$$
$$
A.
\mathrm{相交的} \quad B.\mathrm{平行的} \quad C.\mathrm{异面的} \quad D.\mathrm{重合的} \quad E. \quad F. \quad G. \quad H.
$$
$$
\begin{array}{l}\mathrm{两平面}x-2y-z=3\mathrm{与平面}x+y-3z=0\mathrm{的交线为}\;\left\{\begin{array}{l}x-2y-z=3\\x+y-3z=0\end{array}\right.,\\\mathrm{其对称式方程为}\frac{x-1}1=\frac{y+1}{\displaystyle\frac27}=\frac z{\displaystyle\frac37},\;\\\mathrm{同理可得}2x-4y-2z=5\mathrm{与平面}x+y-3z=0\mathrm{的的交线方程为}\;\;\frac{x-1}1=\frac{y+{\displaystyle\frac{11}{14}}}{\displaystyle\frac27}=\frac{z-{\displaystyle\frac1{14}}}{\displaystyle\frac37},\\\mathrm{所以两线是平行的}.\end{array}
$$



$$
\mathrm{经过两平面}4x-y+3z-1=0,x+5y-z+2=0\mathrm{的交线作平面π},\mathrm{并使π与}y\mathrm{轴平行},\mathrm{则平面π的方程为}().
$$
$$
A.
14x-21z-3=0 \quad B.21x-14z+3=0 \quad C.21x+14z-3=0 \quad D.21x+14z+3=0 \quad E. \quad F. \quad G. \quad H.
$$
$$
\begin{array}{l}\mathrm{经过两平面}4x-y+3z-1=0,x+5y-z+2=0\mathrm{的交线方程}\left\{\begin{array}{l}4x-y+3z-1=0\\x+5y-z+2=0\end{array}\right.\\\mathrm{通过此直线的平面束方程为}4x-y+3z-1+λ(x+5y-z+2)=0,\\即(4+λ)x+(5λ-1)y+(3-λ)z-1+2λ=0\;\mathrm{该平面与}y\mathrm{轴平行},\mathrm{则有}5λ-1=0,λ=\frac15,\\\mathrm{所以平面方程为}21x+14z-3=0.\end{array}
$$



$$
\mathrm{平行于平面}6x+y+6z+5=0\mathrm{而与三个坐标面所围成的四面体体积}v\mathrm{为一个单位的平面方程为}().
$$
$$
A.
6x+y+6z=6 \quad B.6x+y+6z=1 \quad C.x-y+6z=6 \quad D.6x-3y+z=5 \quad E. \quad F. \quad G. \quad H.
$$
$$
\begin{array}{l}\frac x1+\frac y6+\frac z1=1解\;\;\mathrm{设平面方程为}\frac xa+\frac yb+\frac zc=1,∵ V=1,∴\frac13·\frac12abc=1.\;\;\\\mathrm{由所求平面与已知平面平行得}\;\frac{\displaystyle\frac1a}6\;=\frac{\displaystyle\frac1b}1=\frac{\displaystyle\frac1c}6,\\\;\mathrm{向量平行的充要条件}\;\;\\令\frac1{6a}=\frac1b=\frac1{6c}=t\rightarrow a=\frac1{6t},b=\frac1t,c=\frac1{6t}.\;\;\\由1=\frac16·\frac1{6t}·\frac1t·\frac1{6t}\rightarrow t=\frac16.\;∴ a=1,b=6,c=1\;.\;\;\mathrm{所求平面方程为}\frac x1+\frac y6+\frac z1=1,\\即6x+y+6z=6.\;\\\end{array}
$$



$$
点M\left(1,2,1\right)\mathrm{到平面}x+2y+2z-10=0\mathrm{的距离为}\left(\;\;\;\right)
$$
$$
A.
1 \quad B.±1 \quad C.-1 \quad D.\textstyle\frac13 \quad E. \quad F. \quad G. \quad H.
$$
$$
\mathrm{根据点到平面的距离公式可得}d={\textstyle\frac{\left|1+4+2-10\right|}{\sqrt{1+4+4}}}=1.
$$



$$
\mathrm{直线}\left\{\begin{array}{l}5x+y-3z-7=0\\2x+y-3z-7=0\end{array}\right.\left(\;\;\;\right)
$$
$$
A.
\mathrm{垂直}yOz\mathrm{平面} \quad B.在yOz\mathrm{平面内} \quad C.\mathrm{平行}x轴 \quad D.在xOy\mathrm{平面内} \quad E. \quad F. \quad G. \quad H.
$$
$$
\mathrm{由直线方程可知},x=0,\mathrm{所以直线是在}yOz\mathrm{平面内}
$$



$$
\mathrm{直线}\left\{\begin{array}{l}3x+2z≡0\\5x-1=0\end{array}\right.\left(\;\right)
$$
$$
A.
\mathrm{平行}y轴 \quad B.\mathrm{垂直}y轴 \quad C.\mathrm{平行}x轴 \quad D.\mathrm{平行}zOx\mathrm{平面} \quad E. \quad F. \quad G. \quad H.
$$
$$
\mathrm{由题意得}\left\{\begin{array}{l}x={\textstyle\frac15}\\z=-{\textstyle\frac3{10}}\end{array}\right.,\mathrm{所以是平行于}y轴;
$$



$$
\mathrm{直线}l_1:x-1=y=-\left(z+1\right),l_2:x=-\left(y-1\right)={\textstyle\frac{z+1}0}\mathrm{相对关系是}\left(\;\;\right)
$$
$$
A.
\mathrm{平行} \quad B.\mathrm{重合} \quad C.\mathrm{垂直} \quad D.\mathrm{异面} \quad E. \quad F. \quad G. \quad H.
$$
$$
\mathrm{直线}l_1\mathrm{的方向向量为}\left\{1,1,-1\right\},\mathrm{直线}l_2\mathrm{的方向向量为}\left\{1,-1,0\right\},\mathrm{两向量的数量积为零},\mathrm{所以两直线是垂直的}.
$$



$$
\mathrm{直线}{\textstyle\frac{x+3}{-2}}={\textstyle\frac{y+4}{-7}}={\textstyle\frac z3}\mathrm{与平面}4x-2y-2z=3\mathrm{的关系是}\left(\;\;\right)
$$
$$
A.
\mathrm{平行},\mathrm{但直线不在平面上} \quad B.\mathrm{直线在平面上} \quad C.\mathrm{垂直相交} \quad D.\mathrm{相交但不垂直} \quad E. \quad F. \quad G. \quad H.
$$
$$
\begin{array}{l}\mathrm{因为}Am+Bn+Cp=\left(-2\right)·4+\left(-2\right)·\left(-7\right)+\left(-2\right)·3=0,\;\\\;\mathrm{所以直线与平面平行},\mathrm{直线上的一点}\left(-3,-4,0\right)\mathrm{不在平面上},\mathrm{所以直线不在平面上}.\end{array}
$$



$$
\mathrm{空间直线}{\textstyle\frac{x+2}3}={\textstyle\frac{y-2}1}={\textstyle\frac{z+1}{-5}}\mathrm{与平面}4x+3y+3z+1=0\mathrm{的位置关系是}\left(\;\;\;\right)
$$
$$
A.
\mathrm{互相垂直} \quad B.\mathrm{互相平行} \quad C.\mathrm{不平行也不垂直} \quad D.\mathrm{直线在平面上} \quad E. \quad F. \quad G. \quad H.
$$
$$
3·4+1·3-5·3=0,\mathrm{又因为点}\left(-2,2,-1\right)\mathrm{不在平面内},\mathrm{所以是相互平行的}.
$$



$$
在yOz\mathrm{平面上},\mathrm{且垂直于向量}a=\left\{5,4,3\right\}\mathrm{的单位向量}b=\left(\;\;\;\;\right)
$$
$$
A.
\left\{0,-{\textstyle\frac34},1\right\} \quad B.\left\{0,{\textstyle\frac35},{\textstyle\frac45}\right\} \quad C.\left\{0,±{\textstyle\frac35},∓{\textstyle\frac45}\right\} \quad D.\left\{0,{\textstyle-3},{\textstyle4}\right\} \quad E. \quad F. \quad G. \quad H.
$$
$$
\begin{array}{l}设yOz\mathrm{平面的向量为}c=\left\{0,m,n\right\},a· c=4m+3n=0,m:n=-3:4,\;\;\\\mathrm{所以}c=\left\{0,-3,4\right\}或\boldsymbol c\boldsymbol=\left\{\mathbf0\boldsymbol,\mathbf3\boldsymbol,\boldsymbol-\mathbf4\right\},\mathrm{其单位向量为}\left\{0,±{\textstyle\frac35},∓{\textstyle\frac45}\right\}.\end{array}
$$



$$
\mathrm{两条平行直线}l_1:x=t+1,y=2t-1,z=t\;\;\;l_2:x=t+2,y=2t-1,z=t+1\mathrm{的距离是}\left(\;\;\;\right)
$$
$$
A.
\sqrt2 \quad B.\textstyle\frac23\sqrt3 \quad C.\textstyle\frac23 \quad D.2 \quad E. \quad F. \quad G. \quad H.
$$
$$
L_1\mathrm{的一点为}M\left(1,-1,0\right),L_2\mathrm{的方向向量为}s=\left\{1,2,1\right\},L_2\mathrm{上的一点}M_0\left(2,-1,1\right),\mathrm{所以两直线的距离为}d={\textstyle\frac{\left|\overset\rightharpoonup{MM_0}× s\right|}{\left|s\right|}}={\textstyle\frac23}{\textstyle\sqrt3}.
$$



$$
\mathrm{若平面}kx+y-2z=1\mathrm{与直线}{\textstyle\frac x2}={\textstyle\frac{y-1}4}={\textstyle\frac{z+2}3}\mathrm{平行},则k=\left(\;\;\;\;\right)
$$
$$
A.
1 \quad B.2 \quad C.3 \quad D.0 \quad E. \quad F. \quad G. \quad H.
$$
$$
\mathrm{由题意可得}:2k+4-6=0,\mathrm{解得}:k=1.
$$



$$
\mathrm{直线}{\textstyle\frac x1}={\textstyle\frac{y+2}3}={\textstyle\frac{z+7}5}\mathrm{与平面}3x+y-9z+17=0\mathrm{的交点为}\left(\;\;\;\;\right)
$$
$$
A.
\left(0,-2,-7\right) \quad B.\left(3,7,8\right) \quad C.\left(1,1,-2\right) \quad D.\left(2,4,3\right) \quad E. \quad F. \quad G. \quad H.
$$
$$
\mathrm{解方程组}:\left\{\begin{array}{l}\begin{array}{c}x={\textstyle\frac{y+2}3}\\x={\textstyle\frac{z+7}5}\end{array}\\3x+y-9z+17=0\end{array}\right.\mathrm{解得}:x=2,y=4,z=3.
$$



$$
\mathrm{直线}{\textstyle\frac{x-12}4}={\textstyle\frac{y-9}3}={\textstyle\frac{z-1}1}\mathrm{与平面}3x+5y-z-2=0\mathrm{的交点为}\left(\;\;\;\;\;\right)
$$
$$
A.
\left(0,0,-2\right) \quad B.\left(0,0,2\right) \quad C.\left(12,9,1\right) \quad D.\left(4,3,-1\right) \quad E. \quad F. \quad G. \quad H.
$$
$$
\mathrm{解方程组}:\left\{\begin{array}{l}\begin{array}{c}{\textstyle\frac{x-12}4}={\textstyle\frac{z-1}1}\\{\textstyle\frac{y-9}3}={\textstyle\frac{z-1}1}\end{array}\\\begin{array}{c}3x+5y-z-2=0\end{array}\end{array}\right.\mathrm{解得}:x=0,y=0,z=-2.
$$



$$
\mathrm{直线}l_1:\left\{\begin{array}{l}\begin{array}{c}x=-4+t\\y=3-2t\end{array}\\\begin{array}{c}z=2+3t\end{array}\end{array}\right.\mathrm{与直线}l_2:\left\{\begin{array}{l}\begin{array}{c}x=-3+2t\\y=-1-4t\end{array}\\\begin{array}{c}z=5+6t\end{array}\end{array}\right.\mathrm{之间的位置关系是}\left(\;\;\;\right)
$$
$$
A.
\mathrm{平行} \quad B.\mathrm{垂直} \quad C.\mathrm{相交} \quad D.\mathrm{重合} \quad E. \quad F. \quad G. \quad H.
$$
$$
\mathrm{由两直线的参数表达式可得}:s_1=\left\{1,-2,3\right\},s_2=\left\{2,-4,6\right\},得:{\textstyle\frac12}={\textstyle\frac{-2}{-4}}={\textstyle\frac36},\mathrm{显然不重合},\mathrm{因此两直线平行}.
$$



$$
\mathrm{直线}l:{\textstyle\frac{x+2}3}={\textstyle\frac{y-2}{-1}}={\textstyle\frac{z+3}2}\mathrm{和平面}n:2x+3y+3z-8=0\mathrm{的交点是}\left(\;\;\;\right)
$$
$$
A.
\left(3,{\textstyle\frac13},{\textstyle\frac13}\right) \quad B.\left(-1,{\textstyle1},1\right) \quad C.\left(1,-1,1\right) \quad D.\left(1,1,-1\right) \quad E. \quad F. \quad G. \quad H.
$$
$$
\mathrm{联立方程}\left\{\begin{array}{l}{\textstyle\frac{x+2}3}={\textstyle\frac{y-2}{-1}}={\textstyle\frac{z+3}2}\\2x+3y+3z=8\end{array}\right.\mathrm{解得}\left\{\begin{array}{l}\begin{array}{c}x=3\\y={\textstyle\frac13}\end{array}\\z={\textstyle\frac13}\end{array}\right.,\mathrm{所以交点为}\left(3,{\textstyle\frac13},{\textstyle\frac13}\right).
$$



$$
\mathrm{设直线}L_1:{\textstyle\frac{x-1}1}={\textstyle\frac{y-5}{-2}}={\textstyle\frac{z+8}1}与L_2:\left\{\begin{array}{l}x-y=6\\2y+z=3\end{array}\right.,则L_1与L_2\mathrm{的夹角为}(\;\;)
$$
$$
A.
\textstyle\fracπ6 \quad B.\textstyle\fracπ4 \quad C.\textstyle\fracπ3 \quad D.\textstyle\fracπ2 \quad E. \quad F. \quad G. \quad H.
$$
$$
\begin{array}{l}\mathrm{因为}L_1\mathrm{的方向是}\overset\rightharpoonup{n_1}=\left(1,-2,1\right),L_2\mathrm{的方向是}\overset\rightharpoonup{n_2}=\left(-1,-1,2\right)\\\mathrm{所以}L_1与L_2\mathrm{的夹角}θ\mathrm{的余弦cos}θ={\textstyle\frac{\overset\rightharpoonup{n_1}·\overset\rightharpoonup{n_2}}{\left|\overset\rightharpoonup{n_1}\right|\left|\overset\rightharpoonup{n_2}\right|}}={\textstyle\frac12},θ={\textstyle\fracπ3}.\end{array}
$$



$$
\mathrm{直线}l:{\textstyle\frac{x+3}{-2}}={\textstyle\frac{y+4}{-7}}={\textstyle\frac z3}\mathrm{与平面}π:4x-2y-z-3=0\mathrm{的位置关系是}\left(\;\;\;\right)
$$
$$
A.
l与π\mathrm{平行} \quad B.l在π 上 \quad C.l与π\mathrm{相交} \quad D.l与π\mathrm{垂直} \quad E. \quad F. \quad G. \quad H.
$$
$$
\begin{array}{l}{\textstyle\frac{-2}4}{\textstyle\neq}{\textstyle\frac{-7}{-2}},\mathrm{所以}l与π\mathrm{不垂直},\left(-2\right)·4+\left(-7\right)·\left(-2\right)+3·\left(-1\right){\textstyle\neq}{\textstyle0},\;\;\\\mathrm{所以}l与π\mathrm{不平行},\mathrm{直线上的点}\left(-3,-4,0\right)\mathrm{不在平面上},\mathrm{所以位置关系是相交}.\end{array}
$$



$$
\mathrm{设直线}L:\left\{\begin{array}{l}x+3y+2z+1=0\\2x-y-10z+3=0\end{array}\right.,\mathrm{设平面}π:4x-2y+z-2=0,\mathrm{则直线}L(\;\;)
$$
$$
A.
\mathrm{平行于π} \quad B.在\mathrmπ 上 \quad C.\mathrm{垂直于π} \quad D.与π\mathrm{相交},\mathrm{但不垂直}. \quad E. \quad F. \quad G. \quad H.
$$
$$
\begin{array}{l}L\mathrm{由平面}A:x+3y+2z+1=0\mathrm{与平面}B:2x-y-10z+3=0\mathrm{相交而成}\;\;\\\left(1,3,2\right)\left(4,-2,1\right)=4-6+2=0\\\mathrm{所以}A与π\mathrm{垂直}\;\\\left(2,-1,-10\right)\left(4,-2,1\right)=8+2-10=0\\\mathrm{所以}B与\mathrm{π垂直}\;\\\mathrm{所以π}与A,B\mathrm{的交线}L\mathrm{垂直}.\end{array}
$$



$$
\mathrm{两直线}\left\{\begin{array}{l}x+2y-z=7\\-2x+y+z=7\end{array}\right.与\left\{\begin{array}{l}3x+6y-3z=8\\2x-y-z=0\end{array}\right.\mathrm{的位置关系为}\left(\;\;\;\right).
$$
$$
A.
\mathrm{平行} \quad B.\mathrm{垂直} \quad C.\mathrm{相交不垂直} \quad D.\mathrm{异面} \quad E. \quad F. \quad G. \quad H.
$$
$$
\begin{array}{l}\mathrm{设两直线的方向向量为}s_1与s_2,\;\;\\则s_1=\begin{vmatrix}\xrightarrow[i]{}&\xrightarrow[j]{}&\xrightarrow[k]{}\\1&2&-1\\-2&1&1\end{vmatrix},\;\;\;\;\;\;\;\;s_2=\begin{vmatrix}\xrightarrow[i]{}&\xrightarrow[j]{}&\xrightarrow[k]{}\\1&2&-1\\2&-1&-1\end{vmatrix},\\\;\;\;\;\;\;\;\;=\left\{3,1,5\right\}\;\;\;\;\;\;\;\;\;\;\;\;\;\;\;\;\;\;\;\;\;\;=\left\{-3,-1,-5\right\}\;\;\;\;\\\;\;\;\;\;\;\;\;\;\;\;\;\;\;\;\;\;\;\;\;\;\;\;\;\;\;\;\;\;\;\;\;\;\;\;\;\;\;\;\;\;\;\;\;\;\;\;=-\left\{3,1,5\right\}\;\;\;\;\;\\\mathrm{所以}\;\;\xrightarrow[s_1]{}⁄⁄\;\xrightarrow[s_2]{}\;\;\;\;\;\;\;\;\;\;\;\;\;\;\;\;\;\;\;\;\;\;\;\;\;\;\;\;\;\;\;\;\;\;\;\;\;\;\;\;\;\;\;\;\;\;\;\;\;\;\;\;\;\;\;\;\;\;\;\;\;\;\;\;\;\;\;\;\;\;\;\;\;\;\;\;\;\;\;\;\;\;\;\;\;\;\;\;\;\;\;\;\;\;\;\;\;\;\;\;\;\;\;\;\;\;\;\;\;\;\;\;\;\;\;\;\;\;\;\;\end{array}
$$



$$
\mathrm{两条平行直线}L_1:x=t+1,y=2t-1,z=t\;\;L_2:x=t+2,y=2t-1,z=t+1\mathrm{的距离是}\left(\;\;\;\;\right)
$$
$$
A.
\sqrt2 \quad B.\textstyle\frac23\sqrt3 \quad C.\textstyle\frac23 \quad D.2 \quad E. \quad F. \quad G. \quad H.
$$
$$
\begin{array}{l}L_1\mathrm{的一点为}M\left(1,-1,0\right),L_2\mathrm{的方向向量为}s=\left\{1,2,1\right\},\mathrm{上的一点}M_0\left(2,-1,1\right),\mathrm{所以两直线的距离为}\\d={\textstyle\frac{\left|\overset\rightharpoonup{MM_0}× s\right|}{\left|s\right|}}≡{\textstyle\frac23}{\textstyle\sqrt3}.\end{array}
$$



$$
\mathrm{已知点}A\left(6,6,-1\right),点B\left(-2,-6,3\right),则AB与xOy\mathrm{面交点的坐标是}\left(\;\;\;\right)
$$
$$
A.
\left(4,-3,1\right) \quad B.\left(4,-3,0\right) \quad C.\left(4,3,1\right) \quad D.\left(4,3,0\right) \quad E. \quad F. \quad G. \quad H.
$$
$$
\begin{array}{l}AB\mathrm{的对称式方程为}{\textstyle\frac{x-6}8}={\textstyle\frac{y-6}{12}}={\textstyle\frac{z+1}{-4}},xOy\mathrm{面为}z=0,\;\\\;\mathrm{代入直线方程得}\left\{\begin{array}{l}x≡4\\y=3\end{array}\right.\mathrm{所以交点的坐标为}\left(4,3,0\right).\end{array}
$$



$$
\mathrm{与两直线}l_1:\left\{\begin{array}{l}\begin{array}{c}x=1\\y=t-1\end{array}\\\begin{array}{c}z=2+t\end{array}\end{array}\right.及l_2:{\textstyle\frac{x+1}1}={\textstyle\frac{y+2}2}={\textstyle\frac{z+1}1}\mathrm{都平行且过原点的平面方程为}\left(\;\;\;\right)
$$
$$
A.
x-y+z=0 \quad B.x-y-z=0 \quad C.x-2y+z=0 \quad D.x+y-z=0 \quad E. \quad F. \quad G. \quad H.
$$
$$
\begin{array}{l}\mathrm{设平面方程为}:Ax+By+Cz+D=0,\mathrm{由题意得}\;\;\\\mathrm{直线的方向向量为}:\left\{0,1,1\right\},\\\;\mathrm{直线的方向向量为}:\left\{1,2,1\right\},\\\;\mathrm{然后由平行得到}:B+C=0,A+2B+C=0,\;\\\mathrm{平面过原点得}:D=0,\mathrm{得到}A=C,B=-C,\\\;\mathrm{因此平面方程为}:x-y+z=0\end{array}
$$



$$
\mathrm{要使直线}{\textstyle\frac{x-a}3}={\textstyle\frac y{-2}}={\textstyle\frac{z+1}a}\mathrm{在平面}3x+4y-az=3a-1内,则a=\left(\;\;\;\right)
$$
$$
A.
-1 \quad B.-2 \quad C.1 \quad D.2 \quad E. \quad F. \quad G. \quad H.
$$
$$
\begin{array}{l}\mathrm{由题意可得直线的方向向量为}:\left\{3,-2,a\right\},\mathrm{平面的法向量为}:\left\{3,4,-a\right\},,则:\\3·3+\left(-2\right)·4-a· a=0,\mathrm{解得}a=±1.\;\\当a=1时,\mathrm{直线方程为}:{\textstyle\frac{x-1}3}={\textstyle\frac y{-2}}={\textstyle\frac{z+1}1},\mathrm{任意直线上一点}\left(4,-2,0\right),\mathrm{但是不满足平面方程}\\3x+4y-z=2,\mathrm{因此不符合条件};\;\\当a=-1时,\mathrm{直线方程为}:{\textstyle\frac{x+1}3}={\textstyle\frac y{-2}}={\textstyle\frac{z+1}{-1}},\mathrm{任意直线上一点}\left(2,-2,-2\right),\mathrm{满足平面方程}\\3x+4y+z=-4,\mathrm{因此}a=-1\end{array}
$$



$$
设n\mathrm{元齐次线性方程组}Ax=0,若r(A)=r<\;n,\mathrm{则基础解系}()
$$
$$
A.
\mathrm{惟一存在} \quad B.\mathrm{含有}r\mathrm{个向量} \quad C.\mathrm{含有}n-r\mathrm{个向量} \quad D.\mathrm{含有无穷多个向量} \quad E. \quad F. \quad G. \quad H.
$$
$$
\begin{array}{l}n\mathrm{元齐次线性方程组}Ax=0,若r(A)=r<\;n,\mathrm{则基础解系中含有}n-r\mathrm{个向量}.\\\mathrm{由于基础解系不唯一},\mathrm{因此不能确定基础解系的个数}.\end{array}
$$



$$
\mathrm{齐次线性方程组}x_1+2x_2+⋯+nx_n=0\mathrm{的解空间的维数为}()
$$
$$
A.
1 \quad B.2 \quad C.3 \quad D.n-1 \quad E. \quad F. \quad G. \quad H.
$$
$$
\begin{array}{l}\mathrm{解空间的维数为齐次线性方程组基础解系中向量的个数},\mathrm{由于方程组的系数矩阵}A\mathrm{的秩}r(A)=1,\mathrm{则基础解系中}\\\mathrm{的向量个数为}n-r(A)=n-1.\mathrm{故解空间维数为}n-1.\end{array}
$$



$$
\mathrm{齐次线性方程组}\left\{\begin{array}{l}x_1+3x_3+4x_4-5x_5=0\\x_2-2x_3-3x_4+x_5=0\end{array}\right.\mathrm{的解空间的维数是}()
$$
$$
A.
5 \quad B.2 \quad C.3 \quad D.4 \quad E. \quad F. \quad G. \quad H.
$$
$$
\begin{array}{l}\mathrm{原方程组可化为}\left\{\begin{array}{l}x_1=-3x_3-4x_4+5x_5\\x_2=2x_3+3x_4-x_5\end{array}\right.\mathrm{其中}x_3,x_4,x_5\mathrm{是自由未知量},\mathrm{分别取}\begin{pmatrix}x_3\\x_4\\x_5\end{pmatrix}=\begin{pmatrix}1\\0\\0\end{pmatrix},\begin{pmatrix}0\\1\\0\end{pmatrix},\begin{pmatrix}0\\0\\1\end{pmatrix}\mathrm{得到原方程组的基础解系}:\\η_1=\begin{pmatrix}-3\\2\\1\\0\\0\end{pmatrix},η_2=\begin{pmatrix}-4\\3\\0\\1\\0\end{pmatrix},η_3=\begin{pmatrix}5\\-1\\0\\0\\1\end{pmatrix}\mathrm{所以原齐次线性方程组解空间的维数是}3.\end{array}
$$



$$
设V=\{x=(x_1,x_2,x_3)\vert x_1+x_2+x_3=0,且x_1,x_2,x_3∈ R\},则()
$$
$$
A.
V是1\mathrm{维向量空间} \quad B.V是2\mathrm{维向量空间} \quad C.V是3\mathrm{维向量空间} \quad D.V\mathrm{不是向量空间} \quad E. \quad F. \quad G. \quad H.
$$
$$
\begin{array}{l}\mathrm{由题设可知}V\mathrm{是由线性方程组}x_1+x_2+x_3=0\mathrm{的解空间},\mathrm{线性方程组的基础解系中所含解向量的个数为}n-R(A)=3-1=2,\\\mathrm{因此解空间的维数为}2.\end{array}
$$



$$
\mathrm{齐次方程组}\left\{\begin{array}{l}a_1x_1+a_2x_2+⋯+a_nx_n=0\\b_1x_1+b_2x_2+⋯+b_nx_n=0\end{array}\right.\mathrm{的基础解系中含有}n-1\mathrm{个解向量},\mathrm{则必有}(\;)\mathrm{成立},(a_i\neq0,i=1,2,⋯,n)
$$
$$
A.
a_1=a_2=⋯=a_n \quad B.b_1=b_2=⋯=b_n \quad C.\begin{vmatrix}a_1&a_2\\b_1&b_2\end{vmatrix}\neq0 \quad D.\frac{a_i}{b_i}=m\neq0,i=1,2,⋯,n \quad E. \quad F. \quad G. \quad H.
$$
$$
\mathrm{方程组的基础解系中含有}n-1\mathrm{解向量},\mathrm{则系数矩阵的秩为}n-(n-1)=1\mathrm{即系数矩阵的两行对应成比例},即\frac{a_i}{b_i}=m\neq0,i=1,2,⋯,n
$$



$$
\mathrm{设矩阵}A=\begin{pmatrix}1&1&1&1&1\\3&2&1&1&-3\\0&1&2&2&6\\5&4&3&3&-1\end{pmatrix},x=\begin{pmatrix}x_1\\\vdots\\x_5\end{pmatrix},\mathrm{则齐次线性方程组}Ax=0\mathrm{的解的空间}V\mathrm{的维数为}()
$$
$$
A.
1 \quad B.2 \quad C.3 \quad D.4 \quad E. \quad F. \quad G. \quad H.
$$
$$
A=\begin{pmatrix}1&1&1&1&1\\3&2&1&1&-3\\0&1&2&2&6\\5&4&3&3&-1\end{pmatrix}\rightarrow\begin{pmatrix}1&1&1&1&1\\0&-1&-2&-2&-6\\0&0&0&0&0\\0&0&0&0&0\end{pmatrix}\;R(A)=2,故dimV=5-2=3
$$



$$
\mathrm{齐次线性方程组}\left\{\begin{array}{c}3x_1+2x_2-5x_3+4x_4=0\\3x_1-x_2+3x_3-3x_4=0\\3x_1+5x_2-13x_3+11x_4=0\end{array}\right.\mathrm{解空间的维数为}()
$$
$$
A.
1 \quad B.2 \quad C.3 \quad D.4 \quad E. \quad F. \quad G. \quad H.
$$
$$
\mathrm{设方程组的系数矩阵为}A,则R(A)=2,\mathrm{故解空间的维数是}4-2=2.
$$



$$
\mathrm{齐次线性方程组}\left\{\begin{array}{c}x_1+x_2+x_3+x_4=0\\2x_1+3x_2+x_3+x_4=0\\4x_1+5x_2+3x_3+3x_4=0\end{array}\right.\mathrm{解空间的维数为}()
$$
$$
A.
1 \quad B.2 \quad C.3 \quad D.4 \quad E. \quad F. \quad G. \quad H.
$$
$$
A=\begin{pmatrix}1&1&1&1\\2&3&1&1\\4&5&3&3\end{pmatrix}\rightarrow⋯\rightarrow\begin{pmatrix}1&1&1&1\\0&1&-1&-1\\0&0&0&0\end{pmatrix},R(A)=2<\;n=4\mathrm{基础解系的解向量个数为}2.
$$



$$
设A=\begin{pmatrix}1&2&1&2\\0&1&c&c\\1&c&0&1\end{pmatrix},\mathrm{且方程组}Ax=0\mathrm{的解空间的维数为}2,则c\mathrm{的值为}()
$$
$$
A.
1 \quad B.2 \quad C.3 \quad D.4 \quad E. \quad F. \quad G. \quad H.
$$
$$
A\rightarrow\begin{pmatrix}1&2&1&2\\0&1&c&c\\0&c-2&-1&-1\end{pmatrix}\rightarrow\begin{pmatrix}1&2&1&2\\0&1&c&c\\0&0&-(c-1)^2&-(c-1)^2\end{pmatrix},\mathrm{欲使}R(A)=2,\mathrm{只有}(c-1)^2=0,\mathrm{可得}c=1.
$$



$$
设A=\begin{pmatrix}1&2&1&2\\0&1&c&c\\1&c&0&1\end{pmatrix},\mathrm{且方程组}Ax=0\mathrm{的解空间的维数为}1,则c\mathrm{的值满足}()
$$
$$
A.
c=1 \quad B.c\neq1 \quad C.c\mathrm{取任意值} \quad D.c\neq2 \quad E. \quad F. \quad G. \quad H.
$$
$$
A\rightarrow\begin{pmatrix}1&2&1&2\\0&1&c&c\\0&c-2&-1&-1\end{pmatrix}\rightarrow\begin{pmatrix}1&0&1-2c&2-2c\\0&1&c&c\\0&0&-(c-1)^2&-(c-1)^2\end{pmatrix},\mathrm{欲使}R(A)=3,\mathrm{只有}(c-1)^2\neq0,\mathrm{可得}c\neq1.
$$



$$
\mathrm{设解空间}V=\{X=\begin{pmatrix}x_1\\x_2\\\vdots\\x_n\end{pmatrix}\vert x_1+x_2+⋯+x_n=0\},\mathrm{则维数为}()
$$
$$
A.
1 \quad B.2 \quad C.n \quad D.n-1 \quad E. \quad F. \quad G. \quad H.
$$
$$
\mathrm{解空间的维数为齐次线性方程组基础解系中向量的个数},\mathrm{由于方程组的系数矩阵}A\mathrm{的秩}r(A)=1,\mathrm{则基础解系中的向量个数为}n-r(A)=n-1
$$



$$
\mathrm{设解空间}V=\{X=\begin{pmatrix}x_1\\x_2\\\vdots\\x_n\end{pmatrix}\vert nx_1+(n-1)x_2+⋯+x_n=0\},则V\mathrm{的维数为}()
$$
$$
A.
1 \quad B.2 \quad C.3 \quad D.n-1 \quad E. \quad F. \quad G. \quad H.
$$
$$
\begin{array}{l}\mathrm{解空间的维数为齐次线性方程组基础解系中向量的个数},\mathrm{由于方程组的系数矩阵}A\mathrm{的秩}r(A)=1,\mathrm{则基础解系中的向量个数为}n-r(A)=n-1\\故dimV=n-1.\end{array}
$$



$$
\mathrm{齐次线性方程组}\left\{\begin{array}{l}x_1+2x_3+4x_4+5x_5=0\\x_1+2x_2+4x_4+5x_5=0\end{array}\right.\mathrm{的解空间的维数是}()
$$
$$
A.
5 \quad B.2 \quad C.3 \quad D.4 \quad E. \quad F. \quad G. \quad H.
$$
$$
\mathrm{因为系数矩阵的秩为}2,\mathrm{所以方程组解空间的维数为}5-2=3
$$



$$
\mathrm{设解空间}V=\{X=\begin{pmatrix}x_1\\x_2\\x_3\\x_4\end{pmatrix}\vert x_1+x_2+x_3+x_4=0\},\mathrm{则维数为}()
$$
$$
A.
1 \quad B.2 \quad C.3 \quad D.4 \quad E. \quad F. \quad G. \quad H.
$$
$$
\begin{array}{l}\mathrm{解空间的维数为齐次线性方程组基础解系中向量的个数},\mathrm{由于方程组的系数矩阵}A\mathrm{的秩}r(A)=1,\mathrm{则基础解系}\mathrm{中的向量个数为}n-r(A)=4-1=3,\\故dimV=3.\end{array}
$$



$$
\mathrm{齐次线性方程组}\left\{\begin{array}{c}x_1+x_2+x_3+x_4=0\\x_1+λ x_2+x_3-x_4=0\\x_1+x_2+λ x_3-x_4=0\end{array}\right.\mathrm{的基础解系中只含有一个解向量},则λ\mathrm{满足}()
$$
$$
A.
λ\neq1 \quad B.λ=1 \quad C.λ\neq2 \quad D.λ\mathrm{为任意值} \quad E. \quad F. \quad G. \quad H.
$$
$$
\begin{array}{l}A=\begin{pmatrix}1&1&1&1\\1&λ&1&-1\\1&1&λ&-1\end{pmatrix}\rightarrow\begin{pmatrix}1&1&1&1\\0&λ-1&0&-2\\0&0&λ-1&-2\end{pmatrix}\mathrm{由此可见},当λ\neq1时,r(A)=3,\mathrm{即当}λ\neq1时,\\\mathrm{齐次线性方程组的基础解系中只有一个解向量}.\end{array}
$$



$$
设α_1,α_2,α_3\mathrm{是齐次线性方程组}Ax=0\mathrm{的基础解系},\mathrm{则下列不是基础解系的是}()
$$
$$
A.
α_1+α_2,α_2+2α_3,α_3+3α_1 \quad B.α_1-α_2,α_2-α_3,α_3-α_1 \quad C.α_1+α_2,α_2+α_3,α_3+α_1 \quad D.α_1+2α_2,α_2+2α_3,α_3+2α_1 \quad E. \quad F. \quad G. \quad H.
$$
$$
\begin{array}{l}由k_1(α_1+α_2)+k_2(α_2+2α_3)+k_3(α_3+3α_1)=0⇒\left\{\begin{array}{l}k_1+3k_3=0\\\begin{array}{c}k_1+k_2=0\\2k_2+k_3=0\end{array}\end{array}\right.⇒ k_1=k_2=k_3=0\\而α_1+α_2,α_2+2α_3,α_3+3α_1\mathrm{线性无关},\mathrm{同理可证},\mathrm{选项}2\mathrm{是线性相关的}.\end{array}
$$



$$
5\mathrm{元齐次线性方程组}Ax=0\mathrm{的解空间维数为}3,则A\mathrm{的列向量组的秩为}()
$$
$$
A.
1 \quad B.2 \quad C.3 \quad D.4 \quad E. \quad F. \quad G. \quad H.
$$
$$
\mathrm{由基础解系的性质可知}:r(A)=5-3=2;\;\mathrm{由于矩阵的秩等于其行}(列)\mathrm{向量组的秩},故A\mathrm{的列向量组的秩也等于}2.
$$



$$
5\mathrm{元齐次线性方程组}Ax=O\mathrm{的解空间维数为}2,则A\mathrm{的非零子式的最高阶数为}()
$$
$$
A.
1 \quad B.2 \quad C.3 \quad D.4 \quad E. \quad F. \quad G. \quad H.
$$
$$
\mathrm{由基础解系的性质可知}:r(A)=5-2=3;\;\mathrm{由于矩阵的秩等于其行}(列)\mathrm{向量组的秩},故A\mathrm{的列向量组的秩也等于}3.
$$



$$
设α_1,α_2,α_3\mathrm{是齐次线性方程组}Ax=\mathbf0\mathrm{的基础解系},\mathrm{则下列不是基础解系的是}()
$$
$$
A.
α_1+2α_2,α_2+3α_3,α_3+4α_1 \quad B.α_1-α_2,α_2-α_3,α_3-α_1 \quad C.α_1+α_2,α_2+α_3,α_3+α_1 \quad D.α_1+2α_2,α_2+2α_3,α_3+2α_1 \quad E. \quad F. \quad G. \quad H.
$$
$$
\begin{array}{l}由k_1(α_1+α_2)+k_2(α_2+2α_3)+k_3(α_3+4α_1)=0⇒\left\{\begin{array}{l}k_1+4k_3=0\\\begin{array}{c}k_1+k_2=0\\2k_2+k_3=0\end{array}\end{array}\right.⇒ k_1=k_2=k_3=0\\而α_1+α_2,α_2+2α_3,α_3+4α_1\mathrm{线性无关},\mathrm{同理可证},\mathrm{选项}2\mathrm{是线性相关的}.\end{array}
$$



$$
设α_1,α_2,α_3\mathrm{是齐次线性方程组}Ax=O\mathrm{的基础解系},\mathrm{则下列不是基础解系的是}()
$$
$$
A.
α_1+3α_2,α_2+3α_3,α_3+3α_1 \quad B.α_1,α_1+α_2,α_1+α_2+α_3 \quad C.α_1-α_2,α_2-α_3,α_3-α_{1\;\;\;} \quad D.α_1+2α_2,α_2+2α_3,α_3+2α_1 \quad E. \quad F. \quad G. \quad H.
$$
$$
\begin{array}{l}由k_1(α_1+α_2)+k_2(α_2+2α_3)+k_3(α_3+3α_1)=0⇒\left\{\begin{array}{l}k_1+3k_3=0\\\begin{array}{c}k_1+k_2=0\\2k_2+k_3=0\end{array}\end{array}\right.⇒ k_1=k_2=k_3=0\\得α_1+α_2,α_2+2α_3,α_3+3α_1\mathrm{线性无关},\mathrm{同理可证},\mathrm{选项}C\mathrm{是线性相关的}.\end{array}
$$



$$
5\mathrm{元齐次线性方程组}Ax=0\mathrm{的解空间维数为}3,则A\mathrm{的非零子式的最高阶数为}()
$$
$$
A.
1 \quad B.2 \quad C.3 \quad D.4 \quad E. \quad F. \quad G. \quad H.
$$
$$
\mathrm{由基础解系的性质可知}:r(A)=5-3=2;\;\mathrm{矩阵的秩等于非零子式的最高阶数也等于}2.
$$



$$
\mathrm{设齐次线性方程组}x_1+2x_2+⋯+(n-1)x_{n-1}+nx_n=O,\mathrm{下列命题错误的是}()
$$
$$
A.
\;\mathrm{解空间的维数为}n-1 \quad B.\mathrm{基础解系为}(-2,1,⋯,0,0)^T,(-3,0,1,⋯,0)^T,(-n,0,⋯,0,1)^T \quad C.\mathrm{基础解系为}(1,0,⋯,0,-2)^T,(0,1,⋯,0,-3)^T,(0,0,⋯,1,-n)^T \quad D.\mathrm{系数矩阵的秩为}1 \quad E. \quad F. \quad G. \quad H.
$$
$$
\begin{array}{l}\mathrm{原方程组即为}x_1=-2x_2-3x_3-⋯-nx_n,\\取x_2=1,x_3=⋯=x_n=0,得x_1=-2\\取x_3=1,x_2=x_4=⋯=x_n=0,得x_1=-3;⋯\;⋯;\\取x_n=1,x_2=⋯=x_{n-1}=0,得x_1=-n\\\end{array}
$$



$$
\begin{array}{l}\mathrm{设齐次线性方程组}x_1+2x_2+⋯+(n-1)x_{n-1}+nx_n=0,\;\mathrm{则正确说法的个数是}()\\(1)\mathrm{解空间的维数为}n-1\\(2)\mathrm{基础解系为}(-2,1,⋯,0,0)^T,(-3,0,1,⋯,0)^T,(-n,0,⋯,0,1)^T\\(3)\mathrm{基础解系为}(1,0,⋯,0,-2)^T,(0,1,⋯,0,-3)^T,(0,0,⋯,1,-n)^T\\(4)\mathrm{系数矩阵的秩为}n-1\end{array}
$$
$$
A.
1 \quad B.2 \quad C.3 \quad D.4 \quad E. \quad F. \quad G. \quad H.
$$
$$
\begin{array}{l}\begin{array}{l}取x_2=1,x_3=⋯=x_n=0,得x_1=-2;\\取x_3=1,x_2=x_4=⋯=x_n=0,得x_1=-3;⋯\;⋯;\\取x_n=1,x_2=x_2=⋯=x_{n-1}=0,得x_1=-n\end{array}\\\mathrm{所以}(1)(2)\mathrm{正确}.\end{array}
$$



$$
\mathrm{对齐次线性方程组}Ax=0\mathrm{的系数矩阵}A\mathrm{施行初等行变换得}\begin{pmatrix}1&0&0&3\\0&1&0&1\\0&0&1&2\\1&0&0&3\end{pmatrix},\mathrm{则原方程组基础解系为}()
$$
$$
A.
\begin{pmatrix}-3\\-1\\-2\\1\end{pmatrix} \quad B.\begin{pmatrix}3\\-1\\-2\\1\end{pmatrix} \quad C.\begin{pmatrix}-3\\1\\-2\\1\end{pmatrix} \quad D.\begin{pmatrix}-3\\-1\\2\\1\end{pmatrix} \quad E. \quad F. \quad G. \quad H.
$$
$$
\begin{array}{l}\mathrm{由条件可知}A\rightarrow\begin{pmatrix}1&0&0&3\\0&1&0&1\\0&0&1&2\\1&0&0&3\end{pmatrix}\rightarrow\begin{pmatrix}1&0&0&3\\0&1&0&1\\0&0&1&2\\0&0&0&0\end{pmatrix},r(A)=3,\mathrm{方程组的基础解系中只含有一个向量},\mathrm{得到与原方程组同解的方程组}\\\left\{\begin{array}{c}x_1+3x_4=0\\x_2+x_4=0\\x_3+2x_4=0\end{array}\right.令x_4=1,\mathrm{则得基础解系}\begin{pmatrix}-3\\-1\\-2\\1\end{pmatrix}\end{array}
$$



$$
\mathrm{已知}ξ_1,ξ_2,ξ_3,ξ_4是Ax=O\mathrm{的基础解系},\mathrm{则此方程组的基础解系还可以选用}()
$$
$$
A.
ξ_1+ξ_2,ξ_2+ξ_3,ξ_3+ξ_4,ξ_4+ξ_1 \quad B.与ξ_1,ξ_2,ξ_3,ξ_4\mathrm{等价的向量组}α_1,α_2,α_3,α_4 \quad C.与ξ_1,ξ_2,ξ_3,ξ_4\mathrm{等秩的向量组}α_1,α_2,α_3,α_4 \quad D.ξ_1+ξ_2,ξ_2+ξ_3,ξ_3-ξ_4,ξ_4-ξ_1 \quad E. \quad F. \quad G. \quad H.
$$
$$
\begin{array}{l}\begin{array}{l}因α_1,α_2,α_3,α_4与ξ_1,ξ_2,ξ_3,ξ_4\mathrm{等价},\mathrm{所以}α_1,α_2,α_3,α_4\mathrm{可由}ξ_1,ξ_2,ξ_3,ξ_4\mathrm{线性表示},\mathrm{于是}α_1,α_2,α_3,α_4\mathrm{都是}Ax=0\mathrm{的解向量},\mathrm{且等价向量组有相同}\\\mathrm{的秩},\mathrm{因此}α_1,α_2,α_3,α_4\mathrm{线性无关},\mathrm{根据基础解系的定义可知},α_1,α_2,α_3,α_4\mathrm{是方程组}Ax=0\mathrm{的基础解系}.故(B)\mathrm{正确}.\end{array}\\\mathrm{向量组等秩不能得到等价},故(C)\mathrm{错误}.\end{array}
$$



$$
\mathrm{齐次线性方程组}\left\{\begin{array}{c}x_1+2x_2+x_3-x_4=0\\3x_1+6x_2-x_3-3x_4=0\\5x_1+10x_2+x_3-5x_4=0\end{array}\right.\mathrm{的一个基础解系为}()
$$
$$
A.
ξ_1=\begin{pmatrix}-2\\1\\0\\0\end{pmatrix},ξ_2=\begin{pmatrix}1\\0\\0\\1\end{pmatrix} \quad B.ξ_1=\begin{pmatrix}-2\\1\\0\\0\end{pmatrix},ξ_2=\begin{pmatrix}1\\0\\-1\\0\end{pmatrix} \quad C.ξ=\begin{pmatrix}1\\0\\0\\1\end{pmatrix} \quad D.ξ_1=\begin{pmatrix}-2\\1\\0\\0\end{pmatrix},ξ_2=\begin{pmatrix}1\\0\\0\\1\end{pmatrix},ξ_3=\begin{pmatrix}1\\0\\-1\\0\end{pmatrix} \quad E. \quad F. \quad G. \quad H.
$$
$$
\begin{array}{l}A=\begin{pmatrix}1&2&1&-1\\3&6&-1&-3\\5&10&1&-5\end{pmatrix}\rightarrow\begin{pmatrix}1&2&1&-1\\0&0&-4&0\\0&0&-4&0\end{pmatrix}\rightarrow\begin{pmatrix}1&2&0&-1\\0&0&1&0\\0&0&0&0\end{pmatrix},\mathrm{同解方程组为}\left\{\begin{array}{l}x_1+2x_2-x_4=0\\x_3=0\end{array}\right.\\\mathrm{基础解系为}ξ_1=\begin{pmatrix}-2\\1\\0\\0\end{pmatrix},ξ_2=\begin{pmatrix}1\\0\\0\\1\end{pmatrix}\end{array}
$$



$$
\mathrm{对齐次数方程组}Ax=O\mathrm{的系数矩阵施行初等行变换得}\begin{pmatrix}0&0&0&0\\0&1&0&1\\0&0&1&2\\1&0&0&3\end{pmatrix}\mathrm{则原方程组基础解系为}()
$$
$$
A.
\begin{pmatrix}-3\\-1\\-2\\1\end{pmatrix} \quad B.\begin{pmatrix}3\\-1\\-2\\1\end{pmatrix} \quad C.\begin{pmatrix}-3\\1\\-2\\1\end{pmatrix} \quad D.\begin{pmatrix}-3\\-1\\2\\1\end{pmatrix} \quad E. \quad F. \quad G. \quad H.
$$
$$
\begin{array}{l}\mathrm{由条件可知}A\rightarrow\begin{pmatrix}0&0&0&0\\0&1&0&1\\0&0&1&2\\1&0&0&3\end{pmatrix}\rightarrow\begin{pmatrix}1&0&0&3\\0&1&0&1\\0&0&1&2\\0&0&0&0\end{pmatrix},r(A)=3\mathrm{方程组的基础解系中只含有一个向量},\mathrm{得到与原方程组同解的方程组}\\\left\{\begin{array}{c}x_1+3x_4=0\\x_2+x_4=0\\x_3+2x_4=0\end{array}\right.令x_4=1,\mathrm{则得基础解系为}\begin{pmatrix}-3\\-1\\-2\\1\end{pmatrix}\end{array}
$$



$$
5\mathrm{元齐次线性方程组}Ax=O\mathrm{的解空间维数为}3,则A的3\mathrm{阶子式等于}()
$$
$$
A.
1 \quad B.2 \quad C.3 \quad D.0 \quad E. \quad F. \quad G. \quad H.
$$
$$
\mathrm{由基础解系的性质可知}:r(A)=5-3=2;\;\mathrm{所以所有的三阶子式都为零}.
$$



$$
设A=\begin{pmatrix}1&1&1&1\\2&2&2&2\end{pmatrix},\mathrm{则齐次线性方程组}Ax=O\mathrm{的解空间的一组基为}()
$$
$$
A.
η_1=\begin{pmatrix}-1\\1\\0\\0\end{pmatrix},η_2=\begin{pmatrix}-1\\0\\1\\0\end{pmatrix},η_3=\begin{pmatrix}-1\\0\\0\\1\end{pmatrix} \quad B.η_1=\begin{pmatrix}-1\\1\\0\\0\end{pmatrix},η_2=\begin{pmatrix}-1\\0\\1\\0\end{pmatrix} \quad C.η_1=\begin{pmatrix}-1\\1\\0\\0\end{pmatrix},η_2=\begin{pmatrix}-1\\0\\0\\1\end{pmatrix} \quad D.η_1=\begin{pmatrix}-1\\0\\0\\1\end{pmatrix},η_2=\begin{pmatrix}-1\\0\\1\\0\end{pmatrix} \quad E. \quad F. \quad G. \quad H.
$$
$$
\begin{array}{l}\mathrm{根据解空间的定义},\mathrm{解空间的基可取为方程组的基础解系},\mathrm{对系数矩阵施以初等行变换得}A=\begin{pmatrix}1&1&1&1\\2&2&2&2\end{pmatrix}\rightarrow\begin{pmatrix}1&1&1&1\\0&0&0&0\end{pmatrix}\\\mathrm{即得与原方程组同解的方程组}x_1+x_2+x_3+x_4=0,令\begin{pmatrix}x_1\\x_2\\x_3\end{pmatrix}=\begin{pmatrix}1\\0\\0\end{pmatrix},\begin{pmatrix}0\\1\\0\end{pmatrix},\begin{pmatrix}0\\0\\1\end{pmatrix}\\\mathrm{即得基础解系}η_1=\begin{pmatrix}-1\\1\\0\\0\end{pmatrix},η_2=\begin{pmatrix}-1\\0\\1\\0\end{pmatrix},η_3=\begin{pmatrix}-1\\0\\0\\1\end{pmatrix}\end{array}
$$



$$
\mathrm{设齐次线性方程组}nx_1+(n-1)x_2+⋯+2x_{n-1}+x_n=0,\mathrm{下列命题错误的是}()
$$
$$
A.
\mathrm{解空间的维数为}n-1 \quad B.\mathrm{基础解系为}(1,0,⋯,0,-n)^T,(0,1,⋯,0,-n+1)^T,(0,0,⋯,1,-2)^T \quad C.\mathrm{基础解系为}(1,0,⋯,0,-2)^T,(0,1,⋯,0,-3)^T,(0,0,⋯,1,-n)^T \quad D.\mathrm{系数矩阵的秩为}1 \quad E. \quad F. \quad G. \quad H.
$$
$$
\begin{array}{l}\begin{array}{l}\mathrm{原方程组即为}x_n=-nx_1-(n-1)x_2-⋯-2x_{n-1},\\取x_1=1,x_2=x_3=⋯=x_{n-1}=0,得x_n=-n\\取x_2=1,x_1=x_3=x_4=⋯=x_{n-1}=0,得x_n=-(n-1)=-n+1;⋯\;⋯;\end{array}\\取x_{n-1}=1,x_1=x_2=⋯=x_{n-2}=0,得x_n=-2\end{array}
$$



$$
\mathrm{设齐次线性方程组}\left\{\begin{array}{c}x_1+x_2+x_3+x_4=0\\2x_1+3x_2+x_3+x_4=0\\4x_1+5x_2+3x_3+3x_4=0\end{array}\right.,\mathrm{则下列错误的是}()
$$
$$
A.
\mathrm{方程组的一个基础解系为}:α_1=(-2,1,1,0)^T,α_2=(-2,1,0,1)^T \quad B.\mathrm{方程组必有非零解} \quad C.\mathrm{解空间的维数为}2 \quad D.\mathrm{非零解的个数为}2个 \quad E. \quad F. \quad G. \quad H.
$$
$$
\begin{array}{l}\mathrm{作初等变换}:A=\begin{pmatrix}1&1&1&1\\2&3&1&1\\4&5&3&3\end{pmatrix}\rightarrow⋯\rightarrow\begin{pmatrix}1&1&1&1\\0&1&-1&-1\\0&0&0&0\end{pmatrix}R(A)=2<\;n=4,\mathrm{基础解系的解向量个数为}2.\\\mathrm{一般解为}\left\{\begin{array}{l}x_1=-2x_3-2x_4\\x_2=x_3+x_4\end{array}\right.,x_3,x_4\mathrm{是自由未知量}.\\令x_3=1,x_4=0,则x_1=-2,x_2=1;\\令x_3=0,x_4=1,则x_1=-2,x_2=1;\\\mathrm{于是得原方程组的一个基础解系为}:α_1=(-2,1,1,0)^T,α_2=(-2,1,0,1)^T,\mathrm{其通解是}k_1α_1+k_2α_2\end{array}
$$



$$
设ζ_1,ζ_2,ζ_3是Ax=0\mathrm{的一个基础解系},则()\mathrm{也是该方程组的一个基础解系}
$$
$$
A.
ζ_1,ζ_2,ζ_3\mathrm{的一个等价向量组} \quad B.ζ_1,ζ_2,ζ_3\mathrm{的一个等秩向量组} \quad C.ζ_1,ζ_1+ζ_2,ζ_1+ζ_2+ζ_3 \quad D.ζ_1-ζ_2,ζ_2-ζ_3,ζ_3-ζ_1 \quad E. \quad F. \quad G. \quad H.
$$
$$
\begin{array}{l}与ζ_1,ζ_2,ζ_3\mathrm{等价向量组不一定是基础解系},\mathrm{等秩的向量组中的向量不一定是}Ax=0\mathrm{的解};\\ζ_1-ζ_2,ζ_2-ζ_3,ζ_3-ζ_1\mathrm{线性相关},\mathrm{不能构成基础解系};\\\mathrm{向量组}ζ_1,ζ_1+ζ_2,ζ_1+ζ_2+ζ_3\mathrm{线性无关},\mathrm{且其中的向量都是}Ax=0\mathrm{的解},\mathrm{故可构成}Ax=0\mathrm{的一个基础解系}.\end{array}
$$



$$
设A为n\mathrm{阶阵},A\mathrm{有一个}n-3\mathrm{阶子式不为零},且α_1,α_2,α_3是Ax=0\mathrm{的三个线性无关的解向量},则Ax=0\mathrm{的基础解系为}()
$$
$$
A.
α_1+α_2,α_2+α_3,α_3+α_1 \quad B.α_2-α_1,α_3-α_2,α_1-α_3 \quad C.2α_2-α_1,\frac12α_3-α_2,α_1-α_3 \quad D.α_1+α_2+α_3,α_3-α_2,-α_1-2α_3 \quad E. \quad F. \quad G. \quad H.
$$
$$
\begin{array}{l}\mathrm{由条件可知}Ax=0\mathrm{的基础解系中所含解向量的个数为}n-r(A)=n-(n-3)=3\mathrm{且基础解系中的三个解向量是线性无关的},\mathrm{适合条件的只有向量组}\\α_1+α_2,α_2+α_3,α_3+α_1\end{array}
$$



$$
设ζ_1,ζ_2,ζ_3\mathrm{是齐次线性方程组}Ax=O\mathrm{的基础解系},则()\mathrm{不是}Ax=O\mathrm{的基础解系}.
$$
$$
A.
ζ_1+ζ_2,ζ_2+ζ_3,ζ_3+ζ_1 \quad B.ζ_1+2ζ_2,ζ_2+2ζ_3,ζ_3+2ζ_1 \quad C.ζ_1,ζ_1+ζ_2,ζ_1+ζ_2+ζ_3 \quad D.ζ_1-ζ_2,ζ_2-ζ_3,ζ_3-ζ_1 \quad E. \quad F. \quad G. \quad H.
$$
$$
\mathrm{选项中只有}ζ_1-ζ_2,ζ_2-ζ_3,ζ_3-ζ_1\mathrm{线性相关},\mathrm{因此}(D)\mathrm{不是基础解系}.
$$



$$
A是n\mathrm{阶矩阵},\mathrm{对于齐次线性方程组}Ax=O\mathrm{如果每个}n\mathrm{维向量都是方程组的解},则r(A)=()
$$
$$
A.
0 \quad B.1 \quad C.n-1 \quad D.n \quad E. \quad F. \quad G. \quad H.
$$
$$
\begin{array}{l}\mathrm{每个}n\mathrm{维向量都是解},\mathrm{因而有}n\mathrm{个线性无关组的解},\mathrm{那么解空间的维数是}n,\mathrm{又因解空间维数是}n-r(A),故n=n-r(A),即r(A)=0.\\\mathrm{故应填}0.\end{array}
$$



$$
设A为n\mathrm{阶方阵},A^* 为A\mathrm{的伴随矩阵},若A^* x=O\mathrm{的解空间维数为}n-1,则Ax=O\mathrm{的基础解系中所含向量个数}k\mathrm{必满足}()
$$
$$
A.
k=0 \quad B.k=1 \quad C.k>1 \quad D.k=n \quad E. \quad F. \quad G. \quad H.
$$
$$
\begin{array}{l}\mathrm{由条件可知}R(A^*)=n-(n-1)=1\\故A\mathrm{中有}n-1\mathrm{阶子式不为零},即R(A)=n-1,\mathrm{因此基础解系所含向量个数}k=1.\end{array}
$$



$$
A是n\mathrm{阶矩阵},\mathrm{对于齐次线性方程组}Ax=O,\mathrm{如果每个}n\mathrm{维向量都是方程组的解},则A=()
$$
$$
A.
\mathrm{零矩阵} \quad B.\mathrm{单位矩阵} \quad C.\mathrm{可逆矩阵} \quad D.\mathrm{非零矩阵} \quad E. \quad F. \quad G. \quad H.
$$
$$
\begin{array}{l}\mathrm{每个}n\mathrm{维向量都是解},\mathrm{因而有}n\mathrm{个线性无关组的解},\mathrm{那么解空间的维数是}n,\mathrm{又因解空间维数是}n-r(A),故n=n-r(A),即r(A)=0.\\\mathrm{故为零矩阵}.\end{array}
$$



$$
设A=(α_1,α_2,α_3,α_4)是4\mathrm{阶矩阵},A^* 为A\mathrm{的伴随矩阵},若(1,0,1,0)^T\mathrm{是方程组}Ax=O\mathrm{的一个基础解系},则A^* x=O\mathrm{的基础解系可为}()
$$
$$
A.
α_1,α_3 \quad B.α_1,α_2 \quad C.α_1,α_2,α_3 \quad D.α_2,α_3,α_4 \quad E. \quad F. \quad G. \quad H.
$$
$$
\begin{array}{l}\mathrm{因为}Ax=0\mathrm{的基础解系中含一个线性无关的向量},\mathrm{所以}r(A)=3,\mathrm{于是}r(A^*)=1.故A^* x=0\mathrm{的基础解系中含有}3\mathrm{个线性无关的向量}.\\又A^* A=\left|A\right|E=0,\mathrm{所以}A\mathrm{的列向量组中含}A^* x=O\mathrm{的基础解系},\mathrm{因为}(1,0,1,0)^T是Ax=O\mathrm{的基础解系},\mathrm{所以}α_1+α_3=0,故\\α_1,α_2,α_4与α_2,α_3,α_4\mathrm{线性无关},\mathrm{显然}α_2,α_3,α_4是A^* x=O\mathrm{的一个基础解系}.\end{array}
$$



$$
\mathrm{设线性方程组}\left\{\begin{array}{c}x_1+x_2+x_3=0\\x_1+2x_2+ax_3=0\\x_1+4x_2+a^2x_3=0\end{array}\right.\mathrm{解空间的维数为}1,则()
$$
$$
A.
a\neq1,a\neq2 \quad B.a=1 \quad C.a=2 \quad D.a=1或a=2 \quad E. \quad F. \quad G. \quad H.
$$
$$
\begin{array}{l}\mathrm{方程组的系数行列式}\begin{vmatrix}1&1&1\\1&2&a\\1&4&a^2\end{vmatrix}=(a-1)(a-2)\\当a\neq1,a\neq2时,\mathrm{方程组只有零解},\\当a=1时,\mathrm{对方程组的系数矩阵施以初等行变换}\begin{pmatrix}1&1&1\\1&2&1\\1&4&1\end{pmatrix}\rightarrow\begin{pmatrix}1&0&1\\0&1&0\\0&0&0\end{pmatrix},x=k\begin{pmatrix}-1\\0\\1\end{pmatrix}\mathrm{其中}k\mathrm{为任意常数}.\\当a=2时,\mathrm{对线性方程组的系数矩阵施以初等行变换}\begin{pmatrix}1&1&1\\1&2&2\\1&4&4\end{pmatrix}\rightarrow\begin{pmatrix}1&0&0\\0&1&1\\0&0&0\end{pmatrix},\mathrm{因此通解为}x=k\begin{pmatrix}0\\-1\\1\end{pmatrix}\mathrm{其中}k\mathrm{为任意常数}.\end{array}
$$



$$
设n\mathrm{阶矩阵}A\mathrm{的伴随矩阵}A^*\neq O,若ζ_1,ζ_2\mathrm{是非齐次线性方程组}Ax=b\mathrm{的互不相等的两个解},\mathrm{则对应的齐次线性方程组}Ax=O\mathrm{的基础解系}()
$$
$$
A.
\mathrm{不存在} \quad B.\mathrm{仅含一个非零解向量} \quad C.\mathrm{含有两个线性无关的解向量} \quad D.\mathrm{含有三个线性无关的解向量} \quad E. \quad F. \quad G. \quad H.
$$
$$
\begin{array}{l}由A^*\neq O知A^*\mathrm{有非零元素},\mathrm{于是}A\mathrm{有非零的}n-1\mathrm{阶子式};\mathrm{由非齐次方程组}Ax=b\mathrm{的解不惟一},知\left|A\right|=0.\mathrm{综上得}:r(A)=n-1\\\mathrm{于是}Ax=0\mathrm{的基础解系仅含}n-(n-1)即1\mathrm{个非零向量}.\end{array}
$$



$$
\mathrm{若齐次线性方程组}\left\{\begin{array}{c}x_1+x_2+x_3+x_4=0\\x_1+λ x_2+x_3-x_4=0\\x_1+x_2+λ x_3-x_4=0\end{array}\right.\mathrm{的基础解系中只含有一个解向量},则λ\mathrm{的值为}()
$$
$$
A.
λ\neq-2 \quad B.λ\neq2 \quad C.λ\neq-1 \quad D.λ\neq1 \quad E. \quad F. \quad G. \quad H.
$$
$$
\begin{array}{l}\mathrm{因为齐次线性方程组基础解系中含有}n-R(A)\mathrm{个解向量},\mathrm{现已知}n-R(A)=1,即4-R(A)=1,\mathrm{所以}R(A)=3,\mathrm{利用初等变换可将系数矩阵化为}\\A=\begin{pmatrix}1&1&1&1\\1&λ&1&-1\\1&1&λ&-1\end{pmatrix}\rightarrow\begin{pmatrix}1&1&1&1\\0&λ-1&0&-2\\0&0&λ-1&-2\end{pmatrix}\mathrm{由此可见},当λ\neq1时,R(A)=3,\mathrm{即当}λ\neq1时,\mathrm{齐次线性方程组的基础解系中只有一个解向量}.\end{array}
$$



$$
设η_1,η_2,η_3为Ax=0\mathrm{的基础解系},则\;λη_1-η_2,η_2-η_3,η_3-η_1\mathrm{也是}Ax=0\mathrm{的基础解系的充要条件是}()
$$
$$
A.
λ\neq1 \quad B.λ=1 \quad C.λ=-1 \quad D.λ=1或λ=-1 \quad E. \quad F. \quad G. \quad H.
$$
$$
\begin{array}{l}\mathrm{由基础解系的定义可知}:λη_1-η_2,η_2-η_3,η_3-η_1是Ax=0\mathrm{的基础解系}\Leftrightarrowλη_1-η_2,η_2-η_3,η_3-η_1\mathrm{线性无关}\\\mathrm{由于}\left(λη_1-η_2,η_2-η_3,η_3-η_1\right)=(η_1,η_2,η_3)\begin{pmatrix}λ&0&-1\\-1&1&0\\0&-1&1\end{pmatrix}=(η_1,η_2,η_3)K\\若\left|K\right|\neq0⇒λ-1\neq0,即λ\neq1\mathrm{时矩阵}K\mathrm{可逆},则r(λη_1-η_2,η_2-η_3,η_3-η_1)=r(η_1,η_2,η_3)=3\\\mathrm{即向量组}λη_1-η_2,η_2-η_3,η_3-η_1\mathrm{线性无关}.\end{array}
$$



$$
设A为n(n\geq2)\mathrm{阶方阵},A^* 为A\mathrm{的伴随矩阵},若Ax=0\mathrm{的解空间维数为}1,则A^* x=0\mathrm{的基础解系中所含向量个数}k\mathrm{必满足}()
$$
$$
A.
k=0 \quad B.k=1 \quad C.k=n \quad D.k=n-1 \quad E. \quad F. \quad G. \quad H.
$$
$$
R(A)=n-1,R(A^*)=1,\mathrm{因此基础解系所含向量个数}k=n-R(A^*)=n-1
$$



$$
设A为n\mathrm{阶阵},A\mathrm{有一个}n-3\mathrm{阶子式不为零},且α_1,α_2,α_3是Ax=0\mathrm{的三个线性无关的解向量},则Ax=0\mathrm{的基础解系为}()
$$
$$
A.
α_1+2α_2,α_2+2α_3,α_3+2α_1 \quad B.α_2-α_1,α_3-α_2,α_1-α_3 \quad C.2α_2-α_1,\frac12α_3-α_2,α_1-α_3 \quad D.α_1+α_2+α_3,α_3-α_2,-α_1-2α_3 \quad E. \quad F. \quad G. \quad H.
$$
$$
\begin{array}{l}\mathrm{由条件可知}Ax=0\mathrm{的基础解系中所含解向量的个数为}n-r(A)=n-(n-3)=3\mathrm{且基础解系中的三个解向量是线性无关的},\mathrm{适合条件的只有向量组}\\α_1+2α_2,α_2+2α_3,α_3+2α_1\end{array}
$$



$$
设A=(α_1,α_2,α_3,α_4)是4\mathrm{阶矩阵},A^* 为A\mathrm{的伴随矩阵},若(1,0,2,0)^T\mathrm{是方程组}Ax=O\mathrm{的一个基础解系},则A^* x=O\mathrm{的基础解系可为}()
$$
$$
A.
α_1,α_3 \quad B.α_1,α_2 \quad C.α_1,α_2,α_3 \quad D.α_1,α_2,α_4 \quad E. \quad F. \quad G. \quad H.
$$
$$
\begin{array}{l}\mathrm{因为}Ax=0\mathrm{的基础解系中含一个线性无关的向量},\mathrm{所以}r(A)=3,\mathrm{于是}r(A*)=1.故A^* x=0\mathrm{的基础解系中含有}3\mathrm{个线性无关的向量}.\\又A^* A=\left|A\right|E=0,\mathrm{所以}A\mathrm{的列向量组中含}A^* x=0\mathrm{的基础解系},\mathrm{因为}(1,0,2,0)^T是AX=0\mathrm{的基础解系},\mathrm{所以}α_1+2α_3=0,故\\α_1,α_2,α_4与α_2,α_3,α_4\mathrm{线性无关},\mathrm{显然}α_1,α_2,α_4是A^* x=0\mathrm{的一个基础解系}.\end{array}
$$



$$
设A=\begin{pmatrix}1&2&5\\1&t&-2\\t&7&8\\1&4&-9\end{pmatrix}\mathrm{且方程组}Ax=O\mathrm{的基础解系中只含有一个线性无关的解向量},则t\mathrm{的值为}()
$$
$$
A.
t=3 \quad B.t=-3 \quad C.t\neq3 \quad D.t\mathrm{为任意值} \quad E. \quad F. \quad G. \quad H.
$$
$$
A\rightarrow\begin{pmatrix}1&2&5\\0&t-2&-7\\0&7-2t&8-5t\\0&2&-14\end{pmatrix}\rightarrow\begin{pmatrix}1&2&5\\0&1&-7\\0&0&7(t-3)\\0&0&-19(t-3)\end{pmatrix}\mathrm{由于基础解系中只含一个向量},则r(A)=3-1=2,\mathrm{必有}t=3.
$$



$$
设n\mathrm{阶矩阵}A\mathrm{的伴随矩阵}A^*\neq O,若ζ_1,ζ_2,ζ_3,ζ_4\mathrm{是非齐次线性方程组}Ax=b\mathrm{的互不相等的解},\mathrm{则对应的齐次线性方程组}Ax=O\mathrm{的基础解系}()
$$
$$
A.
\mathrm{不存在} \quad B.\mathrm{仅含一个非零解向量} \quad C.\mathrm{含有两个线性无关的解向量} \quad D.\mathrm{含有三个线性无关的解向量} \quad E. \quad F. \quad G. \quad H.
$$
$$
\begin{array}{l}由A^*\neq O知A^*\mathrm{有非零元素},\mathrm{于是}A\mathrm{有非零的}n-1\mathrm{阶子式};\mathrm{由非齐次方程组}Ax=b\mathrm{的解不惟一},知\left|A\right|=0.\mathrm{综上得}:r(A)=n-1\\\mathrm{于是}Ax=0\mathrm{的基础解系仅含}n-(n-1)即1\mathrm{个非零向量}.\end{array}
$$



$$
\mathrm{设方程组}\left\{\begin{array}{c}a_{11}x_1+⋯+a_{1n}x_n=0\\⋯⋯⋯⋯⋯\\a_{n1}x_1+⋯+a_{nn}x_n=0\end{array}\right.\mathrm{的系数矩阵}A\mathrm{的秩为}n-1,且A\mathrm{中某元素}a_{kj}\mathrm{的代数余子式}A_{kj}\neq0,\mathrm{则基础解系为}()
$$
$$
A.
(A_{k1},A_{k2},⋯,A_{kj},⋯,A_{kn})^T \quad B.(A_{1k},A_{2k},⋯,A_{kj},⋯,A_{nk})^T \quad C.(A_{11},A_{22},⋯,A_{kk},⋯,A_{nn})^T \quad D.(M_{k1},M_{k2},⋯,M_{kj},⋯,M_{kn})^T \quad E. \quad F. \quad G. \quad H.
$$
$$
\begin{array}{l}由R(A)=(n-1)知:\mathrm{基础解系中仅含}1\mathrm{个解向量},且\left|A\right|=0.\\\mathrm{因为}AA^*=\left|A\right|E=0,故A\mathrm{的伴随矩阵}A^*\mathrm{每一列都是方程组}Ax=0\mathrm{的解},即n\mathrm{维列向量}(A_{i1},A_{i2},⋯,A_{in})^T,i=1,⋯,n\mathrm{是方程组的解},\\\mathrm{其中任意一个非零向量可构成方程组的基础解系},\mathrm{又因为}A_{kj}\neq0,\mathrm{所以向量}(A_{k1},A_{k2},⋯,A_{kj},⋯,A_{kn})^T\mathrm{为非零向量},\mathrm{可为基础解系}.\end{array}
$$



$$
\mathrm{设矩阵}A\mathrm{的伴随矩阵的行列式为零},\mathrm{且矩阵}A\mathrm{的一个元素}a_{kj}\mathrm{的代数余子式}A_{kj}\neq0,则Ax=O\mathrm{基础解系为}()
$$
$$
A.
(A_{k1},A_{k2},⋯,A_{kj},⋯,A_{kn})^T \quad B.(A_{1k},A_{2k},⋯,A_{kj},⋯,A_{nk})^T \quad C.(A_{11},A_{22},⋯,A_{kk},⋯,A_{nn})^T \quad D.(M_{k1},M_{k2},⋯,M_{kj},⋯,M_{kn})^T \quad E. \quad F. \quad G. \quad H.
$$
$$
\begin{array}{l}由R(A)=(n-1)知:\mathrm{基础解系中仅含}1\mathrm{个解向量},且\left|A\right|=0.\\\mathrm{因为}AA^*=\left|A\right|E=0,故A\mathrm{的伴随矩阵}A^*\mathrm{每一列都是方程组}Ax=0\mathrm{的解},即n\mathrm{维列向量}(A_{i1},A_{i2},⋯,A_{in})^T,i=1,⋯,n\mathrm{是方程组的解},\\\mathrm{其中任意一个非零向量可构成方程组的基础解系},\mathrm{又因为}A_{kj}\neq0,\mathrm{所以向量}(A_{k1},A_{k2},⋯,A_{kj},⋯,A_{kn})^T\mathrm{为非零向量},\mathrm{可为基础解系}.\end{array}
$$



$$
\mathrm{设矩阵}A=\begin{pmatrix}1&1&-3&t\\3&t&-3&4\\1&5&-9&-8\end{pmatrix},\mathrm{齐次线性方程组}Ax=0\mathrm{的基础解系含有}2\mathrm{个线性无关的解向量},则t为()
$$
$$
A.
t=-1 \quad B.t=1 \quad C.t=8 \quad D.t\neq-1 \quad E. \quad F. \quad G. \quad H.
$$
$$
\begin{array}{l}\mathrm{作初等行变换},得A\rightarrow\begin{pmatrix}1&1&-3&t\\0&1&-\frac32&-2-\frac t4\\0&0&\frac32(t+1)&\frac14(t-8)(t+1)\end{pmatrix}\mathrm{由于齐次线性方程组}Ax=0\mathrm{的基础解系含有}2\mathrm{个线性无关的解向量},\\即n-r(A)=4-r(A)=2,\mathrm{从而知道}r(A)=2,\mathrm{因此}t=-1.\end{array}
$$



$$
设A是4\mathrm{阶矩阵},ξ_1,ξ_2\mathrm{是齐次线性方程组}Ax=0\mathrm{的一个基础解系},则A\mathrm{的伴随矩阵}A^*=()
$$
$$
A.
A^*=O \quad B.\left|A\right|A^{-1} \quad C.A \quad D.A^{-1} \quad E. \quad F. \quad G. \quad H.
$$
$$
\begin{array}{l}\mathrm{由于}A是n\mathrm{阶矩阵},ξ_1与ξ_2\mathrm{是齐次线性方程组}Ax=0\mathrm{的两个线性无关的解},\mathrm{因此齐次线性方程组}Ax=0\mathrm{的基础解系中至少有两个解向量}.\mathrm{所以齐次线性}\\\mathrm{方程组}Ax=0\mathrm{的系数矩阵}A\mathrm{的秩至多为}2.\mathrm{这表明}A\mathrm{的任意一个}3\mathrm{阶子式皆为}0.而A\mathrm{的伴随矩阵}A^*\mathrm{中的每一个元素都是矩阵}A\mathrm{的一个}3\mathrm{阶子式},\\即A\mathrm{的伴随矩阵}A^*\mathrm{中每一个元素都是}0.\mathrm{所以}A\mathrm{的伴随矩阵是零矩阵},即A^*=O.\end{array}
$$



$$
设A为4\mathrm{阶方阵},\mathrm{伴随矩阵}A^*\mathrm{的行列式为零},α_1,α_2,α_3是A\mathrm{的三个线性无关的列向量},则A^* x=O\mathrm{的基础解系为}()
$$
$$
A.
α_1+α_2,α_2+α_3,α_3+α_1 \quad B.α_2-α_1,α_3-α_2,α_1-α_3 \quad C.2α_2-α_1,\frac12α_3-α_2,α_1-α_3 \quad D.α_1+α_2+α_3,α_3-α_2,-α_1-2α_3 \quad E. \quad F. \quad G. \quad H.
$$
$$
\begin{array}{l}\mathrm{由条件可知}A* x=0\mathrm{的基础解系中所含解向量的个数为}4-r(A*)=4-1=3,\mathrm{所以},α_1,α_2,α_3\mathrm{为一个基础解系},\mathrm{故适合条件的只有向量组}\\α_1+α_2,α_2+α_3,α_3+α_1\end{array}
$$



$$
设A\mathrm{是行列式为零的}4\mathrm{阶方阵},α_1,α_2,α_3是A\mathrm{的三个线性无关的列向量},则A^* x=O\mathrm{的基础解系为}()
$$
$$
A.
α_1+α_2,α_2+α_3,α_3+α_1 \quad B.α_2-α_1,α_3-α_2,α_1-α_3 \quad C.2α_2-α_1,\frac12α_3-α_2,α_1-α_3 \quad D.α_1+α_2+α_3,α_3-α_2,-α_1-2α_3 \quad E. \quad F. \quad G. \quad H.
$$
$$
\begin{array}{l}\mathrm{由条件可知}A* x=0\mathrm{的基础解系中所含解向量的个数为}4-r(A*)=4-1=3,\mathrm{所以},α_1,α_2,α_3\mathrm{为一个基础解系},\mathrm{故适合条件的只有向量组}\\α_1+α_2,α_2+α_3,α_3+α_1\end{array}
$$



$$
\mathrm{设奇异矩阵}A\mathrm{的一个元素}a_{kj}\mathrm{的代数余子式}A_{kj}\neq0,则Ax=O\mathrm{基础解系为}()
$$
$$
A.
(A_{k1},A_{k2},⋯,A_{kj},⋯,A_{kn})^T \quad B.(A_{1k},A_{2k},⋯,A_{kj},⋯,A_{nk})^T \quad C.(A_{11},A_{22},⋯,A_{kk},⋯,A_{nn})^T \quad D.(M_{k1},M_{k2},⋯,M_{kj},⋯,M_{kn})^T \quad E. \quad F. \quad G. \quad H.
$$
$$
\begin{array}{l}由R(A)=(n-1)知:\mathrm{基础解系中仅含}1\mathrm{个解向量},且\left|A\right|=0.\\\mathrm{因为}AA^*=\left|A\right|E=0,故A\mathrm{的伴随矩阵}A^*\mathrm{每一列都是方程组}Ax=0\mathrm{的解},即n\mathrm{维列向量}(A_{i1},A_{i2},⋯,A_{in})^T,i=1,⋯,n\mathrm{是方程组的解},\\\mathrm{其中任意一个非零向量可构成方程组的基础解系},\mathrm{又因为}A_{kj}\neq0,\mathrm{所以向量}(A_{k1},A_{k2},⋯,A_{kj},⋯,A_{kn})^T\mathrm{为非零向量},\mathrm{可为基础解系}.\end{array}
$$



$$
设A=(a_{ij})_{n× n},且\left|A\right|=0,但A\mathrm{中某元素}a_{kl}\mathrm{的代数余子式}A_{kl}\neq0,则Ax=0\mathrm{的基础解系中的解向量的个数是}()
$$
$$
A.
1 \quad B.k \quad C.l \quad D.n \quad E. \quad F. \quad G. \quad H.
$$
$$
\begin{array}{l}\mathrm{代数余子式}A_{kl}\neq0,\mathrm{即存在}n-1\mathrm{阶子式不为零},又\left|A\right|=0,\mathrm{所有}n\mathrm{阶子式都为零},\mathrm{由矩阵秩的定义可知}r(A)=n-1,\mathrm{则基础解系中解向量的个数为}\\n-r(A)=n-(n-1)=1\end{array}
$$



$$
\begin{array}{l}\mathrm{已知}α_1,α_2,⋯,α_s\mathrm{是线性方程组}AX=0\mathrm{的一个基础解系},β_1=t_1α_1+t_2α_2,β_2=t_1α_2+t_2α_3,⋯,β_s=t_1α_s+t_2α_1,\\\mathrm{其中}t_1,t_2\mathrm{都是实数},若β_1,β_2,⋯,β_s\mathrm{也为}Ax=O\mathrm{的一个基础解系},则t_1,t_2\mathrm{满足}()\end{array}
$$
$$
A.
t_1^{s+1}+(-1)^{s+1}t_2^s\neq0 \quad B.t_1^{s+1}+(-1)^st_2^s\neq0 \quad C.t_1^s+(-1)^{s+1}t_2^{s+1}\neq0 \quad D.t_1^s+(-1)^{s+1}t_2^s\neq0 \quad E. \quad F. \quad G. \quad H.
$$
$$
\begin{array}{l}因β_i(i=1,2,⋯,s)为α_1,α_2,⋯,α_s\mathrm{的线性组合},\mathrm{所以}β_i(1\leq i\leq s)\mathrm{均为}Ax=O\mathrm{的解}.设k_1β_1+k_2β_2+⋯+k_sβ_s=0\;\;\;①\\即(t_1k_1+t_2k_s)α_1+(t_2k_1+t_1k_2)α_2+⋯+(t_2k_{s-1}+t_1k_s)α_s=0.\\\mathrm{由于}α_1,α_2,⋯,α_s\mathrm{线性无关},\mathrm{因此有}\left\{\begin{array}{c}t_1k_1+t_2k_s=0\\t_2k_1+t_1k_2=0\\⋯\;⋯\;⋯\\t_2k_{s-1}+t_1k_s=0\end{array}\right.\;\;\;\;②\\\mathrm{方程组}②\mathrm{的系数行列式}\begin{vmatrix}t_1&0&0&⋯&0&t_2\\t_2&t_1&0&⋯&0&0\\0&t_2&t_1&⋯&0&0\\\vdots&\vdots&\vdots&&\vdots&\vdots\\0&0&0&⋯&t_2&t_1\end{vmatrix}=t_1^s+(-1)^{s+1}t_2^s,\mathrm{所以当}t_1^s+(-1)^{s+1}t_2^s\neq0,\\\mathrm{方程组}②\mathrm{只有零解};k_1=k_2=⋯=k_s=0,\mathrm{从而}β_1,β_2,⋯,β_s\mathrm{线性无关},\mathrm{它们也为}AX=0\mathrm{的一个基础解系}.\end{array}
$$



$$
设A为n(n\geq2)\mathrm{阶方阵},A^* 为A\mathrm{的伴随矩阵},\mathrm{若对任意}n\mathrm{维向量}α,\mathrm{均有}A^*α=0,则r(A)\mathrm{满足}()
$$
$$
A.
r(A)\leq1 \quad B.r(A)\geq1 \quad C.r(A)\leq n-1 \quad D.r(A)<\;n-1 \quad E. \quad F. \quad G. \quad H.
$$
$$
\begin{array}{l}\mathrm{由条件可知},n\mathrm{维的单位向量组也是}A^*α=0\mathrm{的解},则A^*α=0\mathrm{的基础解系中包含}n\mathrm{个向量},即r(A^*)=n-n=0,即A^*=0.\\故A\mathrm{中任意的}n-1\mathrm{阶子式均为零},即r(A)<\;n-1.\end{array}
$$



$$
\mathrm{要使}ζ_1=(2,1,0)^T,ζ_2=(3,0,1)^T\mathrm{都是线性方程组}Ax=O\mathrm{的解},\mathrm{只要系数矩阵}A为()
$$
$$
A.
\begin{pmatrix}2&0&1\\3&2&4\end{pmatrix} \quad B.\begin{pmatrix}1&-2&-3\end{pmatrix} \quad C.\begin{pmatrix}2&0&3\\1&2&4\end{pmatrix} \quad D.\begin{pmatrix}1&-2&0\\1&0&-3\\4&0&2\end{pmatrix} \quad E. \quad F. \quad G. \quad H.
$$
$$
因ζ_1,ζ_2\mathrm{线性无关},\mathrm{故方程组}Ax=0\mathrm{的基础解系中解向量的个数}k\geq2,又k=n-r(A)=3-r(A)\geq2,则r(A)\leq1.
$$



$$
\mathrm{要使}ζ_1=(-1,2,0)^T,ζ_2=(3,0,1)^T\mathrm{都是线性方程组}Ax=O\mathrm{的解},\mathrm{只要系数矩阵}A为()
$$
$$
A.
\begin{pmatrix}2&1&0\\-1&0&3\\0&0&4\end{pmatrix} \quad B.\begin{pmatrix}0&1&0\\-2&-1&6\end{pmatrix} \quad C.\begin{pmatrix}2&1&-6\\-4&-2&12\end{pmatrix} \quad D.\begin{pmatrix}1&0&0\\2&0&0\\1&1&1\end{pmatrix} \quad E. \quad F. \quad G. \quad H.
$$
$$
\begin{array}{l}因ζ_1,ζ_2\mathrm{线性无关},\mathrm{故方程组}Ax=0\mathrm{的基础解系中解向量的个数}k\geq2,又k=n-r(A)=3-r(A)\geq2,则r(A)\leq1.\\\mathrm{选项中只有矩阵}\begin{pmatrix}2&1&-6\\-4&-2&12\end{pmatrix}\mathrm{的秩等于}1,\mathrm{故可为系数矩阵}.\end{array}
$$



$$
设A=(a_{ij})_{n× n},且\left|A\right|=0,但A\mathrm{中某元素}a_{kl}\mathrm{的代数余子式}A_{kl}\neq0,则A^* x=O\mathrm{的基础解系中的解向量的个数是}()
$$
$$
A.
1 \quad B.2 \quad C.n-1 \quad D.n \quad E. \quad F. \quad G. \quad H.
$$
$$
\begin{array}{l}\begin{array}{l}\mathrm{代数余子式}A_{kl}\neq0,\mathrm{即存在}n-1\mathrm{阶子式不为零},又\left|A\right|=0,\mathrm{所有}n\mathrm{阶子式都为零},\mathrm{由矩阵秩的定义可知}r(A)=n-1,则A^* x=0\mathrm{基础解系中}\\\mathrm{解向量的个数为}n-r(A^*)=n-1.\end{array}\\\end{array}
$$



$$
设A=(a_1,α_2,⋯,α_n)_{n× n},且\left|A\right|=0,但A\mathrm{中代数余子式}A_{11}\neq0,则A^* x=O\mathrm{的基础解系是}()
$$
$$
A.
a_2,α_3,⋯,α_n \quad B.a_1,α_3,⋯,α_n \quad C.a_1,α_2,⋯,α_{n-1} \quad D.a_1,α_2,α_3,⋯,α_n \quad E. \quad F. \quad G. \quad H.
$$
$$
\begin{array}{l}\begin{array}{l}\mathrm{代数余子式}A_{11}\neq0,即α_2,⋯,α_n\mathrm{线性无关},又\left|A\right|=0,\mathrm{可知}r(A)=n-1,则A^* x=0\mathrm{基础解系中}\\\mathrm{解向量的个数为}n-r(A^*)=n-1,\mathrm{即为}α_2,⋯,α_n.\end{array}\\\end{array}
$$



$$
设A为n(n\geq2)\mathrm{阶方阵},A^* 为A\mathrm{的伴随矩阵},\mathrm{若对任意}n\mathrm{维向量}α,\mathrm{均有}A^*α=0,\mathrm{则齐次线性方程组}Ax=O\mathrm{的基础解系中所含向量个数}k\mathrm{必满足}()
$$
$$
A.
k=0 \quad B.k=1 \quad C.k>1 \quad D.k=n \quad E. \quad F. \quad G. \quad H.
$$
$$
\begin{array}{l}\mathrm{由条件可知},n\mathrm{维的单位向量组也是}A^*α=0\mathrm{的解},则A^*α=0\mathrm{的基础解系中包含}n\mathrm{个向量},即r(A^*)=n-n=0,即A^*=0.\\故A\mathrm{中任意的}n-1\mathrm{阶子式均为零},即r(A)<\;n-1,\mathrm{因此基础解系所含向量个数}k=n-r(A)>1.\end{array}
$$



$$
\mathrm{齐次线性方程组}x_1+2x_2+⋯+nx_n=0\mathrm{自由未知量的个数为}()
$$
$$
A.
1 \quad B.2 \quad C.3 \quad D.n-1 \quad E. \quad F. \quad G. \quad H.
$$
$$
\begin{array}{l}\mathrm{齐次线性方程组自由未知量的个数等于基础解系中向量的个数},\mathrm{由于方程组的系数矩阵}A\mathrm{的秩}r(A)=1,\mathrm{则基础解系中}\\\mathrm{的向量个数为}n-r(A)=n-1.\mathrm{故自由未知量的个数为}n-1.\end{array}
$$



$$
\mathrm{齐次线性方程组}\left\{\begin{array}{l}x_1+3x_3+4x_4-5x_5=0\\2x_1-ax_3+8x_4+bx_5=0\end{array}\right.\mathrm{的解空间的维数是}4,\mathrm 则a\mathrm 和b\mathrm{的取值分别是}()
$$
$$
A.
6,-10 \quad B.6,10 \quad C.-6,10 \quad D.-6,-10 \quad E. \quad F. \quad G. \quad H.
$$
$$
\mathrm{解空间的维数是}4,\mathrm{所以系数矩阵的秩为}1,\mathrm{所以两个方程的未知量的系数对应成比例},\mathrm{故选}D
$$



$$
设V=\{x=(x_1,x_2,…,x_n)^T\vert nx_1+(n-1)x_2+⋯+x_n=0,且x_1,x_2,…,x_n∈ R\},则()
$$
$$
A.
V是1\mathrm{维向量空间} \quad B.V是2\mathrm{维向量空间} \quad C.V是n-1\mathrm{维向量空间} \quad D.V\mathrm{不是向量空间} \quad E. \quad F. \quad G. \quad H.
$$
$$
\begin{array}{l}\mathrm{由题设可知}V\mathrm{是由线性方程组}nx_1+(n-1)x_2+⋯+x_n=0\mathrm{的解空间},\mathrm{线性方程组的基础解系中所含解向量的个数为}n-R(A)=n-1,\\\mathrm{因此解空间的维数为}n-1.\end{array}
$$



$$
设A=(α_1,α_2,α_3,α_4)是4\mathrm{阶矩阵},A^* 为A\mathrm{的伴随矩阵},若(2,0,5,0)^T\mathrm{是方程组}Ax=O\mathrm{的一个基础解系},则A^* x=O\mathrm{的基础解系可为}()
$$
$$
A.
α_1,α_3 \quad B.α_1,α_2 \quad C.α_1,α_2,α_4 \quad D.α_1,α_2,α_3 \quad E. \quad F. \quad G. \quad H.
$$
$$
\begin{array}{l}\mathrm{因为}Ax=0\mathrm{的基础解系中含一个线性无关的向量},\mathrm{所以}r(A)=3,\mathrm{于是}r(A^*)=1.故A^* x=0\mathrm{的基础解系中含有}3\mathrm{个线性无关的向量}.\\又A^* A=\left|A\right|E=0,\mathrm{所以}A\mathrm{的列向量组中含}A^* x=O\mathrm{的基础解系},\mathrm{因为}(1,0,1,0)^T是Ax=O\mathrm{的基础解系},\mathrm{所以}2α_1+5α_3=0,故\\α_1,α_2,α_4与α_2,α_3,α_4\mathrm{线性无关},\mathrm{显然}α_1,α_2,α_4是A^* x=O\mathrm{的一个基础解系}.\end{array}
$$



$$
设n\mathrm{阶矩阵}A\mathrm{的伴随矩阵}A^*\neq O,\vert A^*\vert=0,\mathrm{则对应的齐次线性方程组}Ax=O\mathrm{的基础解系}()
$$
$$
A.
\mathrm{不存在} \quad B.\mathrm{仅含一个非零解向量} \quad C.\mathrm{含有两个线性无关的解向量} \quad D.\mathrm{含有三个线性无关的解向量} \quad E. \quad F. \quad G. \quad H.
$$
$$
\begin{array}{l}由A*\neq O知A*\mathrm{有非零元素},\mathrm{于是}A\mathrm{有非零的}n-1\mathrm{阶子式};由\left|A*\right|=0,知\left|A\right|=0.\mathrm{综上得}:r(A)=n-1\\\mathrm{于是}Ax=0\mathrm{的基础解系仅含}n-(n-1)即1\mathrm{个非零向量}.\end{array}
$$



$$
\begin{array}{l}\mathrm{若齐次线性方程组}\left\{\begin{array}{c}λ x_1+x_2+x_3=0\\x_1+λ x_2+x_3=0\\x_1+x_2+λ x_3=0\end{array}\right.\mathrm{仅有零解},则\left(\right).\\\end{array}
$$
$$
A.
\;λ\neq±1 \quad B.\;λ\neq±2 \quad C.\;λ\neq1\;且\;λ\neq-2 \quad D.\;λ\neq-1 \quad E. \quad F. \quad G. \quad H.
$$
$$
\begin{array}{l}\begin{array}{l}\;\;\mathrm{齐次线性方程组仅有零解},\mathrm{则系数矩阵的秩为}3,\mathrm{由题可知}\\\;\;\;\;\;\;\;\;\;\;\;\begin{vmatrix}λ&1&1\\1&λ&1\\1&1&λ\end{vmatrix}\neq0⇒\left(λ-1\right)^2\left(λ+2\right)\neq0,故\;λ\neq1\;且\;λ\neq-2.\end{array}\\\\\\\end{array}
$$



$$
\mathrm{齐次线性方程组}Ax=O\mathrm{有非零解的充要条件是}\left(\;\;\;\;\;\right).\;
$$
$$
A.
\mathrm{系数矩阵}A\mathrm{的任意两个列向量线性相关} \quad B.\mathrm{系数矩阵}A\mathrm{的任意两个列向量线性无关} \quad C.\mathrm{系数矩阵}A\mathrm{中至少有一个列向量是其余列向量的线性组合} \quad D.\mathrm{系数矩阵}A\mathrm{中任一列向量是其余向量的线性组合} \quad E. \quad F. \quad G. \quad H.
$$
$$
\mathrm{齐次线性方程组}Ax=0\mathrm{有非零解的充要条件是}r(A_{m× n})<\;n,\mathrm{即系数矩阵}A\mathrm{的列向量组线性相关},故A\mathrm{中至少有一个列向量是其余列向量的线性组合}.
$$



$$
\begin{array}{l}\mathrm{齐次线性方程组}\left\{\begin{array}{c}x_1+x_2+x_3=0\\2x_1-x_2+ax_3=0\\x_1-2x_2+3x_3=0\end{array}\right.\mathrm{有非零解的充分必要条件是}a=\left(\;\;\;\;\right).\\\end{array}
$$
$$
A.
\;1 \quad B.\;2 \quad C.\;3 \quad D.\;4 \quad E. \quad F. \quad G. \quad H.
$$
$$
\begin{array}{l}\mathrm{齐次线性方程组有非零解的充分必要条件是系数矩阵}A_{m× n}\mathrm{的秩}r(A)<\;n.\\\mathrm{对方阵而言},则\left|A\right|=0,则\begin{vmatrix}1&1&1\\2&-1&a\\1&-2&3\end{vmatrix}=0⇒ a=4.\end{array}
$$



$$
\mathrm{方程组}\left\{\begin{array}{c}x_1+2x_2+3x_3=0\\3x_1+6x_2+10x_3=0\\2x_1+5x_2+7x_3=0\\x_1+2x_2+4x_3=0\end{array}\right.\;\;\mathrm{的解的情形是}\left(\right).
$$
$$
A.
\;\mathrm{无解} \quad B.\;\mathrm{有唯一解} \quad C.\;\mathrm{基础解系中有一个向量} \quad D.\;\mathrm{基础解系中有两个向量} \quad E. \quad F. \quad G. \quad H.
$$
$$
\begin{array}{l}\;\;\mathrm{对方程组的系数矩阵进行初等行变换得}\\\;\;\;\;\;\;\;\;\;\;\;\;\;\;\;\;\;\;\;\;\;\;\;\;\;\;\;\;\;\;\;\;\;\;\;\;\;\;\;\;\;\;\;\;\;\;\;\;\;\;\;\;\;\;\;\;\;\;\begin{pmatrix}1&2&3\\3&6&10\\2&5&7\\1&2&4\end{pmatrix}\rightarrow\;\begin{pmatrix}1&2&3\\0&0&1\\0&1&1\\0&0&1\end{pmatrix}\rightarrow\;\begin{pmatrix}1&2&3\\0&1&0\\0&0&1\\0&0&0\end{pmatrix}\\\;\;由r(A)=3\mathrm{可知},\mathrm{线性方程组仅有唯一零解}.\\\\\\\\\end{array}
$$



$$
\mathrm{方程组}\;\;\left\{\begin{array}{c}x_1+2x_2-3x_3=0\\2x_1+5x_2+2x_3=0\\3x_1-x_2-4x_3=0\\7x_1+8x_2-8x_3=0\end{array}\right.\mathrm{的解的情形是}\left(\right).
$$
$$
A.
\;\mathrm{无解} \quad B.\;\mathrm{基础解系中有一个向量} \quad C.\;\mathrm{仅有零解} \quad D.\;\mathrm{基础解系中有两个向量} \quad E. \quad F. \quad G. \quad H.
$$
$$
\begin{array}{l}\;\;\mathrm{对系数矩阵}A\mathrm{进行初等行变换},有\\\\\;\;\;\;\;\;\;\;\;\;\;\;\;\;\;\;\;\;\;\;\;\;\;\;\begin{pmatrix}1&2&-3\\2&5&2\\3&-1&-4\\7&8&-8\end{pmatrix}\rightarrow\begin{pmatrix}1&2&-3\\0&1&8\\0&-7&5\\0&-6&13\end{pmatrix}\rightarrow\begin{pmatrix}1&2&-3\\0&1&8\\0&0&1\\0&0&0\end{pmatrix},\\\\\mathrm{由于}r(A)=3,\mathrm{即方程组只有唯一零解}.\end{array}
$$



$$
\mathrm{若线性方程组}\left\{\begin{array}{c}2x+y+z=0\\kx+y+z=0\\x-y+z=0\end{array}\right.有2\mathrm{个不同的解},则\left(\right).
$$
$$
A.
\;k=1 \quad B.\;k=2 \quad C.\;k=-1 \quad D.\;k=-2 \quad E. \quad F. \quad G. \quad H.
$$
$$
\begin{array}{l}\;\;\mathrm{齐次线性方程组有非零解},即r(A)<\;3\Leftrightarrow\left|A\right|=0,则\\\;\;\;\;\;\;\;\;\;\;\;\;\;\;\;\;\;\;\;\;\;\;\begin{vmatrix}2&1&1\\k&1&1\\1&-1&1\end{vmatrix}=2(2-k)=0⇒ k=2.\end{array}
$$



$$
\mathrm{若齐次线性方程组}\left\{\begin{array}{c}x_1+x_2+x_3=0\\x_1+λ x_2+x_3=0\\x_1+x_2+λ x_3=0\end{array}\right.\mathrm{只有零解},则\;λ\;\mathrm{满足}\left(\right).
$$
$$
A.
\;λ=1 \quad B.\;λ\neq1 \quad C.\;λ\neq0 \quad D.\;λ\neq-2 \quad E. \quad F. \quad G. \quad H.
$$
$$
\begin{array}{l}\mathrm{齐次线性方程组只有零解},\mathrm{则其系数矩阵}A\mathrm{的秩等于}n,\mathrm{有题设可知}\left|A\right|\neq0,即\\\;\;\;\;\;\;\;\;\;\;\;\;\;\;\;\;\;\;\;\;\;\;\;\;\;\;\;\;\;\;\;\;\;\;\;\;\;\;\;\;\;\;\;\;\;\;\begin{vmatrix}1&1&1\\1&λ&1\\1&1&λ\end{vmatrix}=(λ-1)^2\neq0⇒λ\neq1.\end{array}
$$



$$
设A是m× n\mathrm{矩阵},\mathrm{齐次线性方程组}Ax=0\mathrm{仅有零解的充分必要条件是系数矩阵的秩}R(A)\;\left(\right).\;
$$
$$
A.
\;\mathrm{小于}m \quad B.\;\mathrm{小于}n \quad C.\;\mathrm{等于}m \quad D.\;\mathrm{等于}n \quad E. \quad F. \quad G. \quad H.
$$
$$
n\mathrm{元齐次线性方程组仅有零解的充要条件是其系数矩阵的秩等于}n.
$$



$$
\mathrm{齐次方程组}\left\{\begin{array}{c}2x_1-x_2+3x_3=0\\x_1-3x_2+4x_3=0\\-x_1+2x_2+ax_3=0\end{array}\right.\mathrm{有无穷多个的解},则\;a\;\mathrm{的值为}(\;).\;
$$
$$
A.
\;-3 \quad B.\;3 \quad C.\;-1 \quad D.\;1 \quad E. \quad F. \quad G. \quad H.
$$
$$
\begin{array}{l}\mathrm{系数矩阵行列式}D=\begin{vmatrix}2&-1&3\\1&-3&4\\-1&2&a\end{vmatrix}=-5(a+3).\\当a=-3\mathrm{时方程组有非零解}.\end{array}
$$



$$
\mathrm{齐次方程组}\left\{\begin{array}{c}ax_1+x_2+x_3=0\\x_1+bx_2+x_3=0\\x_1+2bx_2+x_3=0\end{array}\right.\mathrm{有非零解},则a,b\mathrm{的值为}(\;).\;
$$
$$
A.
\;b=0\;或\;a=1 \quad B.\;b=0 \quad C.\;a=1 \quad D.\;b=0\;且\;a=1 \quad E. \quad F. \quad G. \quad H.
$$
$$
\begin{array}{l}\;\;\mathrm{系数行列式}:\begin{vmatrix}a&1&1\\1&b&1\\1&2b&1\end{vmatrix}=b(1-a).\\\;\;\mathrm{所以当}b=0或a=1\mathrm{时方程组有非零解}.\end{array}
$$



$$
设a\mathrm{为满足}\left|a\right|\neq1\mathrm{的实数},\mathrm{若齐次线性方程组}\begin{pmatrix}2-a&2\\\frac12&2-a\end{pmatrix}x=0\mathrm{有非零解},则(\;\;\;\;).\;
$$
$$
A.
\;a=3 \quad B.\;a=3\;或\;a=1 \quad C.\;a=1 \quad D.\;a\neq3 \quad E. \quad F. \quad G. \quad H.
$$
$$
\begin{array}{l}\mathrm{齐次线性方程组有非零解},则r\left(A\right)\;<\;2,\;故\left|A\right|=0,\;\\\begin{vmatrix}2-a&2\\\frac12&2-a\end{vmatrix}=0⇒ a^2-4a+3=0,\\故a=3或a=1(舍)\end{array}
$$



$$
\mathrm{若齐次线性方程组}\left\{\begin{array}{c}λ x_1+x_2+x_3=0\\x_1+λ x_2+x_3=0\\x_1+x_2+λ x_3=0\end{array}\right.\mathrm{有非零解},则\left(\right).\;
$$
$$
A.
\;λ=±1 \quad B.\;λ=±2 \quad C.\;λ=1\;或\;\;λ=-2 \quad D.\;λ=-1 \quad E. \quad F. \quad G. \quad H.
$$
$$
\begin{array}{l}\mathrm{齐次线性方程组有非零解},\mathrm{则系数矩阵的秩小于}n,\mathrm{由题可知}\\\;\;\;\;\;\;\;\;\;\;\;\;\;\;\;\;\;\;\;\;\;\;\;\;\;\;\;\;\;\;\;\;\;\begin{vmatrix}λ&1&1\\1&λ&1\\1&1&λ\end{vmatrix}=0⇒(λ-1)^2(λ+2)=0,故λ=1\;或\;λ=-2.\end{array}
$$



$$
\mathrm{若齐次线性方程组}\left\{\begin{array}{c}λ x_1+x_2+x_3=0\\x_1+λ x_2+x_3=0\\x_1+x_2+λ x_3=0\end{array}\right.\mathrm{有两个线性无关的解},则\left(\right).\;
$$
$$
A.
\;λ=-1或λ=2 \quad B.\;λ=2 \quad C.\;λ=1\;或\;\;λ=-2 \quad D.\;λ=1 \quad E. \quad F. \quad G. \quad H.
$$
$$
\begin{array}{l}\mathrm{齐次线性方程组有有两个线性无关的解},\mathrm{则系数矩阵的秩等于}1,\mathrm{由题可知}\\\;\;\;\;\;\;\;\;\;\;\;\;\;\;\;\;\;\;\;\;\;\;\;\;\;\;\;\;\;\;\;\;\;\;\;\;\;\;\;\;\;\;\;\begin{vmatrix}λ&1&1\\1&λ&1\\1&1&λ\end{vmatrix}=0⇒(λ-1)^2(λ+2)=0,故λ=1\;或\;λ=-2(舍).\end{array}
$$



$$
设A=\begin{pmatrix}1&2&-2\\4&t&3\\3&-1&1\end{pmatrix},β\mathrm{是非零的}3\mathrm{维列向量},且Aβ=0,则t\mathrm{的值为}(\;\;\;).\;
$$
$$
A.
\;t=-3 \quad B.\;t=3 \quad C.\;t\neq-3 \quad D.\;t=-7 \quad E. \quad F. \quad G. \quad H.
$$
$$
\begin{array}{l}\mathrm{由已知条件}⇒\left|A\right|=0(\mathrm{否则},\mathrm{可得}β=0),\;\;\\由\vert A\vert=0⇒7t+21=0⇒ t=-3.\end{array}
$$



$$
\mathrm{设线性方程组}\left\{\begin{array}{c}x_1+x_2+x_3=0\\x_1+2x_2+ax_3=0\\x_1+4x_2+a^2x_3=0\end{array}\right.\;\mathrm{解空间的维数等于}1,则(\;\;)\;
$$
$$
A.
\;a\neq1,a\neq2 \quad B.\;a=1 \quad C.\;a=2 \quad D.\;a=1\;或\;a=2 \quad E. \quad F. \quad G. \quad H.
$$
$$
\begin{array}{l}\;\;\mathrm{系数行列式}\;\begin{vmatrix}1&1&1\\1&2&a\\1&4&a^2\end{vmatrix}\;=(a-1)(a-2).\\\;\;当a=1\;或\;a=2时,\mathrm{方程组有非零解}.\end{array}
$$



$$
\mathrm{设齐次线性方程组}\left\{\begin{array}{c}(1+a)x_1+x_2+x_3+x_4=0\\2x_1+(2+a)x_2+2x_3+2x_4=0\\3x_1+3x_2+(3+a)x_3+3x_4=0\\4x_1+4x_2+4x_3+(4+a)x_4=0\end{array}\right.\;\;\mathrm{的解空间维数为}1,则\;a=(\;\;)
$$
$$
A.
\;-10 \quad B.\;1 \quad C.\;2 \quad D.\;3 \quad E. \quad F. \quad G. \quad H.
$$
$$
\begin{array}{l}\;\;\mathrm{方程组的系数行列式}\;\left|A\right|=\begin{vmatrix}1+a&1&1&1\\2&2+a&2&2\\3&3&3+a&3\\4&4&4&4+a\end{vmatrix}=(a+10)a^3.\\\;\;\mathrm{由于原方程组有非零解},故\left|A\right|=0\mathrm{推知}a=0\;或\;a=-10\\\;\;当a=0时,R(A)=1\\当a=-10时,R(A)=3\end{array}
$$



$$
\mathrm{设线性方程组}\left\{\begin{array}{c}x_1+x_2+x_3=0\\x_1+2x_2+ax_3=0\\x_1+4x_2+a^2x_3=0\end{array}\right.\;\mathrm{只有零解},则(\;\;)
$$
$$
A.
\;a\neq1且a\neq2 \quad B.\;a=1 \quad C.\;a=2 \quad D.\;a=1\;或\;a=2 \quad E. \quad F. \quad G. \quad H.
$$
$$
\begin{array}{l}\;\;\mathrm{系数行列式}\;\begin{vmatrix}1&1&1\\1&2&a\\1&4&a^2\end{vmatrix}\;=(a-1)(a-2).\\\;\;当a\neq1\;且a\neq2时,\mathrm{方程组只有零解}.\end{array}
$$



$$
\mathrm{设齐次线性方程组}\left\{\begin{array}{c}(1+a)x_1+x_2+x_3+x_4=0\\2x_1+(2+a)x_2+2x_3+2x_4=0\\3x_1+3x_2+(3+a)x_3+3x_4=0\\4x_1+4x_2+4x_3+(4+a)x_4=0\end{array}\right.\;\;\mathrm{的解空间维数为}3,则\;a=(\;\;)
$$
$$
A.
\;0 \quad B.-\;10 \quad C.\;2 \quad D.\;3 \quad E. \quad F. \quad G. \quad H.
$$
$$
\begin{array}{l}\begin{array}{l}\;\;\mathrm{方程组的系数行列式}\;\left|A\right|=\begin{vmatrix}1+a&1&1&1\\2&2+a&2&2\\3&3&3+a&3\\4&4&4&4+a\end{vmatrix}=(a+10)a^3.\\\;\;\mathrm{由于原方程组有非零解},故\left|A\right|=0\mathrm{推知}a=0\;或\;a=-10\end{array}\\\;\;当a=0时,R(A)=1\\当a=-10时,R(A)=3\end{array}
$$



$$
\begin{array}{l}若4\mathrm{元线性方程组}Ax=0\mathrm{的同解方程组是}\left\{\begin{array}{l}x_1=-3x_3\\x_2=0\end{array}\right.\\\mathrm{则系数矩阵的秩}r(A)\mathrm{和自由未知量的个数分别是}(\;\;\;\;)\end{array}
$$
$$
A.
\;r(A)=2,\mathrm{自由未知量的个数应为}2个 \quad B.\;r(A)=1,\mathrm{自由未知量的个数应为}3个 \quad C.\;r(A)=3,\mathrm{自由未知量的个数应为}1个 \quad D.\;\mathrm{不确定} \quad E. \quad F. \quad G. \quad H.
$$
$$
\;\;\mathrm{由同解方程组可知该元线性方程组的系数矩阵}A\mathrm{的秩}r(A)=2,\mathrm{自由未知量的个数应为}n-r,即4-2=2(个).
$$



$$
设A=\begin{pmatrix}1&2&-2\\4&t&3\\3&-1&1\end{pmatrix},B\mathrm{是非零的}3\mathrm{阶矩阵},且AB=O,则t\mathrm{的值为}(\;\;\;).\;
$$
$$
A.
\;t=-3 \quad B.\;t=3 \quad C.\;t\neq-3 \quad D.\;t=-7 \quad E. \quad F. \quad G. \quad H.
$$
$$
\begin{array}{l}\mathrm{由已知条件}⇒\mathrm{齐次线性方程组}\boldsymbol A\boldsymbol x=\mathbf0\mathrm{有非零解},\\\mathrm{所以}\vert A\vert=0⇒ t=-3.\end{array}
$$



$$
\mathrm{线性方程组}\left\{\begin{array}{c}x_1+x_2+x_3=0\\ax_1+bx_2+cx_3=0\\a^2x_1+b^2x_2+c^2x_3=0\end{array}\right.\mathrm{仅有零解的充要条件是}(\;\;\;)\;
$$
$$
A.
\;a\neq b,b\neq c \quad B.\;b\neq c,c\neq a \quad C.\;c\neq a,a\neq b \quad D.\;a\neq b,b\neq c,c\neq a \quad E. \quad F. \quad G. \quad H.
$$
$$
\begin{array}{l}\;\;D=\begin{vmatrix}1&1&1\\a&b&c\\a^2&b^2&c^2\end{vmatrix}=(b-a)(c-b)(c-a),\\\;\;当a\neq b,b\neq c,c\neq a时,D\neq0,\mathrm{方程组仅有零解}.\end{array}
$$



$$
\mathrm{设齐次线性方程组组}\left\{\begin{array}{c}x_1+x_2+x_3+ax_4=0\\x_1+2x_2+x_3+x_4=0\\x_1+x_2-3x_3+x_4=0\\x_1+x_2+x_3+bx_4=0\end{array}\right.\mathrm{有非零解},则a,b\mathrm{应满足的条件是}(\;\;\;\;\;)
$$
$$
A.
\;a-b=0 \quad B.\;a-b=1 \quad C.\;a-b=2 \quad D.\;a-b=3 \quad E. \quad F. \quad G. \quad H.
$$
$$
\begin{array}{l}\mathrm{对系数矩阵施以初等行变换得}\begin{pmatrix}1&1&1&a\\1&2&1&1\\1&1&-3&1\\1&1&1&b\end{pmatrix}\rightarrow\begin{pmatrix}1&1&1&a\\0&1&0&1-a\\0&0&-4&1-a\\0&0&0&a-b\end{pmatrix},\\\mathrm{由于方程组有非零解},则r(A)\;<\;n=4,故a-b=0.\end{array}
$$



$$
\mathrm{方程组}\left\{\begin{array}{c}2x+ky+z=0\\(k-1)x-y+2z=0\\4x+y+4z=0\end{array}\right.\mathrm{有非零解},则k\mathrm{的值为}(\;).\;
$$
$$
A.
\;\;k=1\;或\;k=\frac94 \quad B.\;\;k=1\; \quad C.\;\;k=\frac94 \quad D.\;\;k\neq1\;且\;k\neq\frac94 \quad E. \quad F. \quad G. \quad H.
$$
$$
\begin{array}{l}\;\;\mathrm{方程组要有非零解},\mathrm{必须}:\begin{vmatrix}2&k&1\\k-1&-1&2\\4&1&4\end{vmatrix}=0.\\\mathrm{解得}k=1\;或\;k=\frac94.\\\end{array}
$$



$$
\mathrm{若方程组}\left\{\begin{array}{c}3λ x_1+3x_2+2x_3=0\\x_1+x_2-x_3=0\\(4λ-1)x_1+3x_2+2x_3=0\end{array}\right.\mathrm{有非零解},则λ 为(\;).\;
$$
$$
A.
\;λ=1 \quad B.\;λ=-1 \quad C.\;λ\neq1 \quad D.\;λ=-2 \quad E. \quad F. \quad G. \quad H.
$$
$$
\mathrm{方程组有非零解}⇒\vert A\vert=0,而\left|A\right|=5-5λ,\mathrm{故当且仅当}λ=1\mathrm{时方程组有非零解}.
$$



$$
\mathrm{设线性方程组}\left\{\begin{array}{c}(λ+3)x_1+x_2+2x_3=0\\λ x_1+(λ-1)x_2+x_3=0\\3(λ+1)x_1+λ x_2+(λ+3)x_3=0\end{array}\right.\mathrm{有非零解},则λ 为\left(\right).
$$
$$
A.
\;λ=1 \quad B.\;λ=0 \quad C.\;λ=1\;或\;λ=0 \quad D.\;λ\neq0\;且\;λ\neq1 \quad E. \quad F. \quad G. \quad H.
$$
$$
\begin{array}{l}\;\;\mathrm{方程组的系数行列式为}\begin{vmatrix}λ+3&1&2\\λ&λ-1&1\\3(λ+1)&λ&λ+3\end{vmatrix}=λ^2(λ-1),\\\;\;\mathrm{则当}λ=0\;或\;=1时,\mathrm{方程组有非零解};\end{array}
$$



$$
\mathrm{齐次线性方程组}\left\{\begin{array}{c}2x_1-x_2+3x_3=0\\3x_1-4x_2+7x_3=0\\1x_1-2x_2+ax_3=0\end{array}\right.\mathrm{有非零解},则a=\left(\right).
$$
$$
A.
\;1 \quad B.\;2 \quad C.\;3 \quad D.\;4 \quad E. \quad F. \quad G. \quad H.
$$
$$
\begin{array}{l}\;\;A=\begin{pmatrix}2&-1&3\\3&-4&7\\1&-2&a\end{pmatrix}\xrightarrow{r_1↔ r_3}\begin{pmatrix}1&-2&a\\3&-4&7\\2&-1&3\end{pmatrix}\xrightarrow{\begin{array}{c}r_2-3r_1\\r_3-2r_1\end{array}}\begin{pmatrix}1&-2&a\\0&2&7-3a\\0&3&3-2a\end{pmatrix}\xrightarrow{r_3-\frac32r_2}\begin{pmatrix}1&-2&a\\0&2&7-3a\\0&0&\frac52a-\frac{15}2\end{pmatrix},\\\;\;当a=3时,r(A)=2,\;\mathrm{齐次线性方程组有非零解}.\\\end{array}
$$



$$
设n\mathrm{阶方阵}A,B\mathrm{满足}AB=O,B\neq O,\mathrm{则必有}\left(\right)
$$
$$
A.
\;A=O \quad B.\;A\mathrm{为可逆方阵} \quad C.\;\left|B\right|\neq0 \quad D.\;\left|A\right|=0 \quad E. \quad F. \quad G. \quad H.
$$
$$
\begin{array}{l}AB=O⇒\left|AB\right|=\left|A\right|\left|B\right|=0,则\left|A\right|=0或\left|B\right|=0,\mathrm{不能单独得出}\left|A\right|=0且B\neq O,\mathrm{不能得出}\left|B\right|\neq0,\mathrm{例如}\\B=\begin{pmatrix}0&0&⋯&1\\0&0&⋯&0\\⋯&⋯&⋯&⋯\\0&0&⋯&0\end{pmatrix}\\由AB=O,B\neq O\mathrm{可得}Ax=0\mathrm{有非零解},\mathrm{根据线性方程组解的判断定理可知}r(A)<\;n⇒\vert A\vert=0.\end{array}
$$



$$
\mathrm{设矩阵}A=\begin{pmatrix}0&0&1\\2&1&0\\1&1&-1\end{pmatrix},B\mathrm{是三阶矩阵},且AB=O,\mathrm{则矩阵}B\mathrm{的秩为}\left(\right).\;
$$
$$
A.
\;2 \quad B.\;1 \quad C.\;0 \quad D.\;3 \quad E. \quad F. \quad G. \quad H.
$$
$$
\begin{array}{l}\;\;\mathrm{由题设可知}\begin{vmatrix}0&0&1\\2&1&0\\-1&1&-1\end{vmatrix}=\begin{vmatrix}2&1\\-1&1\end{vmatrix}=3\neq0⇒ r(A)=3,\\\;\;\mathrm{由线性方程组的判定定理可知齐次线性方程组}Ax=0\mathrm{只有零解},故AB=0⇒ B=0⇒ r(B)=0.\end{array}
$$



$$
设A\mathrm{是一个}n\mathrm{阶方阵},\mathrm{存在一个}n\mathrm{阶非零矩阵}B,\mathrm{使得}AB=O\mathrm{的充要条件是}().\;
$$
$$
A.
\;\left|A\right|>0\;且\;\left|A\right|\neq n \quad B.\;\left|A\right|=n \quad C.\;\left|A\right|<0 \quad D.\left|A\right|=0 \quad E. \quad F. \quad G. \quad H.
$$
$$
\mathrm{存在一个}n\mathrm{阶非零矩阵}B\mathrm{使得}AB=O\mathrm{等价于}Ax=0\mathrm{有非零解}B,故Ax=0\mathrm{有非零解的充要条件是}r(A)<\;n,即\left|A\right|=0.
$$



$$
\mathrm{方程组}Ax=0\mathrm{仅有零解的充分必要条件是}\left(\right).\;
$$
$$
A.
\;A\mathrm{的行向量组线性无关} \quad B.\;A\mathrm{的列向量组线性无关} \quad C.\;A\mathrm{的行向量组线性相关} \quad D.\;A\mathrm{的列向量组线性相关} \quad E. \quad F. \quad G. \quad H.
$$
$$
\;\;\mathrm{线性方程组}Ax=0\mathrm{仅有零解的充分必要条件是}A\mathrm{列满秩},\mathrm{即矩阵}A\mathrm{的列向量组线性无关}.
$$



$$
\mathrm{若线性方程组}\left\{\begin{array}{c}2x+y+z=0\\kx+y+z=0\\x-y+z=0\end{array}\right.\mathrm{有非零解},则\left(\right).
$$
$$
A.
\;k=1 \quad B.\;k=2 \quad C.\;k=-1 \quad D.\;k=-2 \quad E. \quad F. \quad G. \quad H.
$$
$$
\begin{array}{l}\;\;\mathrm{齐次线性方程组有非零解},即r(A)<\;3\Leftrightarrow\left|A\right|=0,则\\\;\;\;\;\;\;\;\;\;\;\;\;\;\;\;\;\;\;\;\;\;\;\;\;\;\;\;\;\;\;\;\;\;\;\;\begin{vmatrix}2&1&1\\k&1&1\\1&-1&1\end{vmatrix}=2(2-k)=0⇒ k=2.\end{array}
$$



$$
设A=\begin{pmatrix}a&1&1&2\\2&a+1&2a&3a+1\\1&1&1&2\end{pmatrix},\mathrm{且存在}3\mathrm{阶非零矩阵}B使BA=O,则a\mathrm{的值为}(\;).
$$
$$
A.
\;1 \quad B.\;3 \quad C.\;2 \quad D.\;0 \quad E. \quad F. \quad G. \quad H.
$$
$$
\begin{array}{l}\;\;由BA=O得A^TB^T=O.对B^T\mathrm{按列分块}B^T=\begin{bmatrix}β_1,&β_2,&β_3\end{bmatrix},则\\\;\;\;\;\;\;\;\;\;\;\;\;\;\;\;\;\;\;\;\;\;\;\;\;\;\;\;\;\;\;\;\;\;\;\;\;A^TB^T=A^T\begin{bmatrix}β_1,&β_2,&β_3\end{bmatrix}=\begin{bmatrix}A^Tβ_1,&A^Tβ_2,&A^Tβ_3\end{bmatrix}=O\\\;\;\mathrm{所以}A^Tβ_1=O,A^Tβ_2=O,A^Tβ_3=O.因B\neq O,\mathrm{故存在}β_j\neq0(j=1,2,3),即β_j\mathrm{是齐次线性方程组}A^Tx=O\\\;\;\mathrm{的一个非零解}.\mathrm{故秩}(A^T)<3.又\\\;\;\;\;\;\;\;\;\;\;\;\;\;\;\;\;\;\;\;\;\;\;\;\;\;\;\;\;\;\;\;\;\;\;\;A=\begin{pmatrix}a&1&1&2\\2&a+1&2a&3a+1\\1&1&1&2\end{pmatrix}\xrightarrow 行\;\;\;\;\;\begin{pmatrix}1&1&1&2\\0&a-1&2a-2&3a-3\\0&0&a-1&a-1\end{pmatrix}\\\;\;\mathrm{故当}a=1时,秩(A^T)=秩(A)=1<3.\end{array}
$$



$$
\begin{array}{l}\mathrm{齐次线性方程组}\left\{\begin{array}{c}λ x_1+x_2+λ^2x_3=0\\x_1+λ x_2+x_3=0\\x_1+x_2+λ x_3=0\end{array}\right.\\\mathrm{的系数矩阵记为}A,\mathrm{若存在三阶矩阵}B\neq O,\mathrm{使得}AB=O,则\left(\right)\end{array}
$$
$$
A.
λ=-2且\left|B\right|=0 \quad B.λ=-2且\left|B\right|\neq0 \quad C.λ=1且\left|B\right|=0 \quad D.λ=1且\left|B\right|\neq0 \quad E. \quad F. \quad G. \quad H.
$$
$$
\begin{array}{l}\;\;\mathrm{对方程组的系数矩阵作初等行变换},有\\\;\;\;\;\;\;\;\;\;\;\;\;\;\;\;\;\;\;\;\;\;\;\;\;\;\;\;\;\;\;\;\;\;\;\;\;\;\;A=\;\begin{pmatrix}λ&1&λ^2\\1&λ&1\\1&1&λ\end{pmatrix}\;\rightarrow\begin{pmatrix}0&1-λ&0\\0&λ-1&1-λ\\1&1&λ\end{pmatrix},\\\;\;\mathrm{由此可见},当λ\neq1时,r(A)=3,\mathrm{齐次线性方程组只有零解},\mathrm{与题意矛盾},故λ=1.\\\;\;当λ=1时,r(A)=1,\mathrm{齐次线性方程组有非零解},\mathrm{根据条件有}AB=O,\mathrm{由矩阵秩的性质得}r(B)\leq2⇒\left|B\right|=0.\end{array}
$$



$$
设A,B\mathrm{均为}n\mathrm{阶方阵},且\left|AB\right|=2,\mathrm{则方程组}Ax=O与Bx=O\mathrm{的非零解的个数为}\_\_\_\_\_\_\_.
$$
$$
A.
\;0 \quad B.\;1 \quad C.\;2 \quad D.\;∞ \quad E. \quad F. \quad G. \quad H.
$$
$$
\;\;\mathrm{由于}\left|AB\right|=\vert A\vert\vert B\vert=2,则\vert A\vert\neq0且\vert B\vert\neq0,即r(A)=r(B)=n,\mathrm{因此方程组}Ax=0与Bx=0\mathrm{都只有零解}.
$$



$$
\mathrm{若齐次线性方程组}\left\{\begin{array}{c}ax_1+x_2+x_3=0\\x_1+ax_2+x_3=0\\x_1+x_2+ax_3=0\end{array}\right.\mathrm{的解空间维数为}1,则a\mathrm{的值为}\left(\right).\;
$$
$$
A.
\;1 \quad B.\;-2 \quad C.\;1\;或\;-2 \quad D.\;\mathrm{任意值} \quad E. \quad F. \quad G. \quad H.
$$
$$
\begin{array}{l}\;\;A=\begin{pmatrix}a&1&1\\1&a&1\\1&1&a\end{pmatrix}\xrightarrow{r_1↔ r_3}\begin{pmatrix}1&1&a\\1&a&1\\a&1&1\end{pmatrix}\xrightarrow[{r_3-ar_1}]{r_2-r_1}\begin{pmatrix}1&1&a\\0&a-1&1-a\\0&1-a&1-a^2\end{pmatrix}\xrightarrow{r_3+r_2}\begin{pmatrix}1&1&a\\0&a-1&1-a\\0&0&(1-a)(2+a)\end{pmatrix}.\\\;\;当a=1时,r(A)=1,\;\mathrm{齐次线性方程组有非零解};\\\;\;当a=-2时,\;r(A)=2,\mathrm{齐次线性方程组有非零解}.\end{array}
$$



$$
设A,B\mathrm{均为}n\mathrm{阶方阵},且\left|AB\right|=0,\mathrm{则方程组}Ax=O与Bx=O\mathrm{的非零解的个数为}(\;\;).
$$
$$
A.
\;0 \quad B.\;1 \quad C.\;2 \quad D.\;∞ \quad E. \quad F. \quad G. \quad H.
$$
$$
\;\mathrm{由于}\left|AB\right|=\left|A\right|\left|B\right|=0,则\left|A\right|=0或\left|B\right|=0,\mathrm{因此方程组}Ax=O与Bx=O\mathrm{有无数多个非零解}
$$



$$
\mathrm{若齐次线性方程组}\left\{\begin{array}{c}ax_1+x_2+x_3=0\\x_1+ax_2+x_3=0\\x_1+x_2+ax_3=0\end{array}\right.\mathrm{的解空间维数为}2,则a\mathrm{的值为}\left(\right).\;
$$
$$
A.
\;1 \quad B.\;2 \quad C.\;1\;或\;-2 \quad D.\;\mathrm{任意值} \quad E. \quad F. \quad G. \quad H.
$$
$$
\begin{array}{l}\;\;A=\begin{pmatrix}a&1&1\\1&a&1\\1&1&a\end{pmatrix}\xrightarrow{r_1↔ r_3}\begin{pmatrix}1&1&a\\1&a&1\\a&1&1\end{pmatrix}\xrightarrow[{r_3-ar_1}]{r_2-r_1}\begin{pmatrix}1&1&a\\0&a-1&1-a\\0&1-a&1-a^2\end{pmatrix}\xrightarrow{r_3+r_2}\begin{pmatrix}1&1&a\\0&a-1&1-a\\0&0&(1-a)(2+a)\end{pmatrix}.\\\;\;当a=1时,r(A)=1,\;\mathrm{齐次线性方程组有非零解};\\\;\;当a=-2时,\;r(A)=2,\mathrm{齐次线性方程组有非零解}.\end{array}
$$



$$
a_1x_1+a_2x_2+⋯+a_nx_n=0\;⋯⋯(1)与\;b_1x_1+b_2x_2+⋯+b_nx_n=0\;⋯⋯(2)\mathrm{同解的充要条件是}(\;\;\;)
$$
$$
A.
\;\mathrm{两个方程}(1)与(2)\mathrm{的系数成比例} \quad B.\;\mathrm{两个方程}(1)与(2)\mathrm{的系数不成比例} \quad C.\;\mathrm{两个方程}(1)与(2)\mathrm{的系数都为零} \quad D.\;\mathrm{两个方程}(1)与(2)\mathrm{的系数可取任意值} \quad E. \quad F. \quad G. \quad H.
$$
$$
\begin{array}{l}\;\;\mathrm{充分性}:当\frac{a_1}{b_1}=\frac{a_2}{b_2}=\frac{a_n}{bn}=k时,\mathrm{方程组}(1)\mathrm{与方程组}(2)\mathrm{即为同一个方程组},\mathrm{故有同解}.\\\;\;\mathrm{必要性}:当(1)与(2)\mathrm{有同解时},则(1)\mathrm{与方程组}\left\{\begin{array}{l}a_1x_1+⋯+a_nx_n=0\\b_1x_1+⋯+b_nx_n=0\end{array}\right.⋯⋯(3)\\\;\;\mathrm{同解},\mathrm{则系数矩阵}\begin{bmatrix}a_1&a_2&⋯&a_n\end{bmatrix}与\begin{pmatrix}a_1&a_2&⋯&a_n\\b_1&b_2&⋯&b_n\end{pmatrix}\mathrm{同秩且为}1,\mathrm{所以行向量}\begin{pmatrix}a_1&a_2&⋯&a_n\end{pmatrix}与\begin{pmatrix}b_1&b_2&⋯&b_n\end{pmatrix}\mathrm{相关},\\\;\;\mathrm{即成比例}.\end{array}
$$



$$
\mathrm{若齐次线性方程组}\left\{\begin{array}{c}ax_1+x_2+x_3=0\\x_1+ax_2+x_3=0\\x_1+x_2+ax_3=0\end{array}\right.\mathrm{的解空间维数大于}0,则a\mathrm{的值为}\left(\right).\;
$$
$$
A.
1 \quad B.-2 \quad C.1或-2 \quad D.\mathrm{任意值} \quad E. \quad F. \quad G. \quad H.
$$
$$
\begin{array}{l}\;\;A=\begin{pmatrix}a&1&1\\1&a&1\\1&1&a\end{pmatrix}\xrightarrow{r_1↔ r_3}\begin{pmatrix}1&1&a\\1&a&1\\a&1&1\end{pmatrix}\xrightarrow[{r_3-ar_1}]{r_2-r_1}\begin{pmatrix}1&1&a\\0&a-1&1-a\\0&1-a&1-a^2\end{pmatrix}\xrightarrow{r_3+r_2}\begin{pmatrix}1&1&a\\0&a-1&1-a\\0&0&(1-a)(2+a)\end{pmatrix}.\\\;\;当a=1时,r(A)=1,\;\mathrm{齐次线性方程组有非零解};\\\;\;当a=-2时,\;r(A)=2,\mathrm{齐次线性方程组有非零解}.\end{array}
$$



$$
设A是m× n\mathrm{矩阵},B是n× m\mathrm{矩阵},\mathrm{则线性方程组}(AB)x=0,\left(\right)
$$
$$
A.
\;当n>m\mathrm{时仅有零解} \quad B.\;当n>m\mathrm{时必有非零解} \quad C.\;当m>n\mathrm{时仅有零解} \quad D.\;当m>n\mathrm{时必有非零解} \quad E. \quad F. \quad G. \quad H.
$$
$$
\begin{array}{l}\;\;\mathrm{应选}(D)\\\;\;AB\;是m× m\mathrm{矩阵},\mathrm{因而线性方程组}(AB)x=0\mathrm{的未知数个数为}m.\\\;\;\mathrm{另一方面},由\;\;r(A)\leq min(m,n),r(B)\leq min(m,n),r(AB)\leq min(r(A),r(B))\\\;\;及(D)\mathrm{中的条件}m>n得\;r(AB)\leq min(m,n)<\;m,\\\;\;\mathrm{于是},\mathrm{由有解判别定理知}(AB)x=0\mathrm{必有非零解}.即(D)\mathrm{正确},\mathrm{同时得知}(C)\mathrm{是错误的}.\\\;\;\mathrm{同样的推理得可知}:(A)\mathrm{是错误的}.\\\;\;\mathrm{因为此时}r(AB)\leq m,当r(AB)<\;m\mathrm{时就只有非零解}.(B)\mathrm{是错误的}.\\\;\;\mathrm{因为此时}r(AB)\leq m,当r(AB)=m\mathrm{时就只有零解}.\\\;\;\mathrm{故选}(D).\end{array}
$$



$$
\mathrm{若线性方程组}\left\{\begin{array}{c}2x+y+z=0\\kx+y+z=0\\x-y+z=0\end{array}\right.\mathrm{只有零解},则\left(\right).
$$
$$
A.
\;k\neq1 \quad B.\;k\neq2 \quad C.\;k\neq-1 \quad D.\;k\neq-2 \quad E. \quad F. \quad G. \quad H.
$$
$$
\begin{array}{l}\;\;\mathrm{齐次线性方程组只有零解},即r(A)=\;3\Leftrightarrow\left|A\right|\neq0,则\\\;\;\;\;\;\;\;\;\;\;\;\;\;\;\;\;\;\;\;\;\;\;\begin{vmatrix}2&1&1\\k&1&1\\1&-1&1\end{vmatrix}=2(2-k)⇒ k\neq2.\end{array}
$$



$$
\begin{array}{l}\mathrm{齐次线性方程组}\left\{\begin{array}{c}λ x_1+x_2+λ^2x_3=0\\x_1+λ x_2+x_3=0\\x_1+x_2+λ x_3=0\end{array}\right.\\\mathrm{的系数矩阵记为}A,\mathrm{若存在三阶矩阵}B\neq0,\mathrm{使得}BA=O,则\left(\right)\end{array}
$$
$$
A.
λ=-2\;\;且\;\left|B\right|=0 \quad B.λ=-2\;且\;\left|B\right|\neq0 \quad C.λ=1\;且\;\left|B\right|=0 \quad D.λ=1\;且\;\left|B\right|\neq0 \quad E. \quad F. \quad G. \quad H.
$$
$$
\begin{array}{l}\;\;\mathrm{对方程组的系数矩阵作初等行变换},有\\\;\;\;\;\;\;\;\;\;\;\;\;\;\;\;\;\;\;\;\;\;\;\;\;\;\;\;\;\;\;\;\;\;\;\;\;\;\;A=\;\begin{pmatrix}λ&1&λ^2\\1&λ&1\\1&1&λ\end{pmatrix}\;\rightarrow\begin{pmatrix}0&1-λ&0\\0&λ-1&1-λ\\1&1&λ\end{pmatrix},\\\;\;\mathrm{由此可见},当λ\neq1时,r(A)=3,\mathrm{齐次线性方程组只有零解},\mathrm{与题意矛盾},故λ=1.\\\;\;当λ=1时,r(A)=1,\mathrm{齐次线性方程组有非零解},\mathrm{根据条件有}BA=O,\mathrm{由矩阵秩的性质得}r(B)\leq2⇒\left|B\right|=0.\end{array}
$$



$$
设A为n\mathrm{阶方阵},且Ax=0\mathrm{有无穷多解},则A^TAx=0\left(\right).
$$
$$
A.
\;\mathrm{有无穷多组解} \quad B.\;\mathrm{仅有零解} \quad C.\;\mathrm{有有限多组解} \quad D.\;\mathrm{无解} \quad E. \quad F. \quad G. \quad H.
$$
$$
\;\;Ax=0\mathrm{有无穷多解},则r(A)<\;n,又r(A^TA)=r(A)<\;n,故n\mathrm{元线性方程组}A^TAx=O\mathrm{也有无穷多解}.
$$



$$
设A是n\mathrm{阶方阵且}A\neq O,\mathrm{则存在一个}n\mathrm{阶非零矩阵}B使AB=O\mathrm{的充要条件是}(\;\;\;).
$$
$$
A.
\;\left|A\right|=0 \quad B.\left|A\right|\neq0 \quad C.\;\left|B\right|=0 \quad D.\;\left|B\right|\neq0 \quad E. \quad F. \quad G. \quad H.
$$
$$
\begin{array}{l}\mathrm{存在非零矩阵}B\mathrm{使得}AB=O\mathrm{的充要条件是齐次线性方程组}Ax=0\mathrm{有非零解};\;而Ax=0\mathrm{有非零解的充要条件是系数行列式}\left|A\right|=0.\\\end{array}
$$



$$
设A为n\mathrm{阶方阵},且Ax=0\mathrm{解空间维数为零},则A^TAx=0\;\;\;\;(\;\;\;\;\;).\;
$$
$$
A.
\;\mathrm{有无穷多组解} \quad B.\;\mathrm{仅有零解} \quad C.\;\mathrm{有有限多组解} \quad D.\;\mathrm{无解} \quad E. \quad F. \quad G. \quad H.
$$
$$
Ax=0\mathrm{只有零解},则\left|A\right|\neq0,又\left|A^TA\right|=\left|A\right|\left|A^T\right|=\vert A\vert^2\neq0,故n\mathrm{元线性方程组}A^TAx=0\mathrm{也只有零解}.
$$



$$
\mathrm{若齐次线性方程组}\left\{\begin{array}{c}ax_1+x_2+x_3=0\\x_1+ax_2+x_3=0\\x_1+x_2+ax_3=0\end{array}\right.\mathrm{的任意两个非零解都成比例},则a\mathrm{的值为}\left(\right).\;
$$
$$
A.
\;1 \quad B.\;-2 \quad C.\;1\;或\;-2 \quad D.\;\mathrm{任意值} \quad E. \quad F. \quad G. \quad H.
$$
$$
\begin{array}{l}\begin{array}{l}\;\;A=\begin{pmatrix}a&1&1\\1&a&1\\1&1&a\end{pmatrix}\xrightarrow{r_1↔ r_3}\begin{pmatrix}1&1&a\\1&a&1\\a&1&1\end{pmatrix}\xrightarrow[{r_3-ar_1}]{r_2-r_1}\begin{pmatrix}1&1&a\\0&a-1&1-a\\0&1-a&1-a^2\end{pmatrix}\xrightarrow{r_3+r_2}\begin{pmatrix}1&1&a\\0&a-1&1-a\\0&0&(1-a)(2+a)\end{pmatrix}.\\\;\;\mathrm{因为任意两个非零解都成比例},\mathrm{所以基础解系只有一个解向量},故\;r(A)=2,\;\;\\\mathrm{所以},a=-2.\end{array}\\\\\end{array}
$$



$$
\mathrm{若齐次线性方程组}\left\{\begin{array}{c}ax_1+x_2+x_3=0\\x_1+ax_2+x_3=0\\x_1+x_2+ax_3=0\end{array}\right.\mathrm{的系数矩阵的伴随矩阵为}B,\mathrm{且满足}R(B)=1,则a\mathrm{的值为}\left(\right).\;
$$
$$
A.
\;-2 \quad B.\;2 \quad C.\;1\;或\;-2 \quad D.1\;或\;2 \quad E. \quad F. \quad G. \quad H.
$$
$$
\begin{array}{l}\;\;A=\begin{pmatrix}a&1&1\\1&a&1\\1&1&a\end{pmatrix}\xrightarrow{r_1↔ r_3}\begin{pmatrix}1&1&a\\1&a&1\\a&1&1\end{pmatrix}\xrightarrow[{r_3-ar_1}]{r_2-r_1}\begin{pmatrix}1&1&a\\0&a-1&1-a\\0&1-a&1-a^2\end{pmatrix}\xrightarrow{r_3+r_2}\begin{pmatrix}1&1&a\\0&a-1&1-a\\0&0&(1-a)(2+a)\end{pmatrix}.\\\;\;由R(B)=1,\mathrm{所以}R(A)=2,即a=-2.\;\\\;\end{array}
$$



$$
设A是n\mathrm{阶方阵且}A\neq O,\mathrm{则存在一个}n\mathrm{阶非零矩阵}B使AB=O\mathrm{的充要条件是}(\;\;\;).
$$
$$
A.
\;\left|A\right|=0 \quad B.\;\left|A\right|\neq0 \quad C.\;\left|B\right|=0 \quad D.\;\left|B\right|\neq0 \quad E. \quad F. \quad G. \quad H.
$$
$$
\begin{array}{l}设B=\lbrack B_1,B_2,⋯,B_n\rbrack\neq0,∴ B_1,B_2,⋯,B_n\mathrm{不全为}0,\\由AB=O⇒ AB_j=0(j=1,⋯,n),即B_j为Ax=0\mathrm{的解},即Ax=0\mathrm{有非零解},\mathrm{所以}\left|A\right|=0.\\\mathrm{反之}:若\left|A\right|=0,则Ax=0\mathrm{有非零解},\mathrm{不妨设一非零解为}η_0=\begin{pmatrix}k_1\\\vdots\\k_n\end{pmatrix}\neq0,\\\mathrm{于是可令}n\mathrm{阶矩阵}B为:B=\begin{pmatrix}k_1&0&⋯&0\\k_1&0&⋯&0\\⋯&⋯&⋯&⋯\\k_n&0&0&0\end{pmatrix},\mathrm{即满足}AB=O\mathrm{为所找矩阵}.\end{array}
$$



$$
\mathrm{已知}A=\begin{pmatrix}a^2&b^2&c^2&d^2\\a&b&c&d\\1&1&1&1\end{pmatrix},\mathrm{其中}a<\;b<\;c<\;d,\mathrm{则下列说法中},\mathrm{错误的是}\left(\right).\;
$$
$$
A.
\;A^Tx=0\mathrm{只有零解} \quad B.\;\mathrm{存在}B\neq O,使AB=O \quad C.\;\left|A^TA\right|\neq0 \quad D.\;\left|A^TA\right|=0 \quad E. \quad F. \quad G. \quad H.
$$
$$
\begin{array}{l}\mathrm{矩阵}A\mathrm{存在三阶子式}\begin{vmatrix}a^2&b^2&c^2\\a&b&c\\1&1&1\end{vmatrix}=-(b-a)(c-a)(c-b)\neq0,\mathrm{且不存在四阶子式},故R(A_{3×4})=3<4,则\\\;Ax=0\mathrm{有非零解},\mathrm{因此存在}B\neq0,使AB=0;\\\mathrm{由于}R(A_{3×4})=3=R({A^T}_{4×3}),故A^Tx=0\mathrm{只有零解},\\又R(A^TA)\leq min\{R(A),R(A^T)\}=3,而A^TA\mathrm{为四阶矩阵},故\left|A^TA\right|=0.\\\end{array}
$$



$$
\mathrm{如果齐次线性方程组}\left\{\begin{array}{c}a_{11}x_1+a_{12}x_2+⋯+a_{1n}x_n=0\\a_{21}x_1+a_{22}x_2+⋯+a_{2n}x_n=0\\⋯\;⋯\\a_{s1}x_1+a_{s2}x_2+⋯+a_{sn}x_n=0\end{array}\right.\mathrm{有非零解},\mathrm{那么}.\left(\right)\;
$$
$$
A.
\;s<\;n \quad B.\;s=n \quad C.\;s>n \quad D.\;\mathrm{三种情况都有可能} \quad E. \quad F. \quad G. \quad H.
$$
$$
\;\;\mathrm{齐次线性方程组有非零解的充要条件是系数矩阵}A\mathrm{的秩小于}n,又r(A)\leq min\{s,n\},\mathrm{因此无法确定}s,n\mathrm{的关系},\mathrm{故各种情况都有可能}.
$$



$$
\mathrm{若齐次线性方程组}\left\{\begin{array}{c}ax_1+x_2+x_3=0\\x_1+ax_2+x_3=0\\x_1+x_2+ax_3=0\end{array}\right.有2\mathrm{个线性无关的解},则a\mathrm{的值为}().\;
$$
$$
A.
\;1 \quad B.\;-2 \quad C.\;1\;或\;-2 \quad D.\;\mathrm{任意值} \quad E. \quad F. \quad G. \quad H.
$$
$$
\;\;\begin{array}{l}\;\;A=\begin{pmatrix}a&1&1\\1&a&1\\1&1&a\end{pmatrix}\xrightarrow{r_1↔ r_3}\begin{pmatrix}1&1&a\\1&a&1\\a&1&1\end{pmatrix}\xrightarrow[{r_3-ar_1}]{r_2-r_1}\begin{pmatrix}1&1&a\\0&a-1&1-a\\0&1-a&1-a^2\end{pmatrix}\xrightarrow{r_3+r_2}\begin{pmatrix}1&1&a\\0&a-1&1-a\\0&0&(1-a)(2+a)\end{pmatrix}.\\\;\;当a=1时,r(A)=1,\;\mathrm{齐次线性方程组有非零解};\\\;\;\mathrm{且基础解系有两个线性无关解向量}\end{array}
$$



$$
设A为m× n\mathrm{矩阵},R(A)=m,\mathrm{齐次线性方程组}Ax=0\mathrm{只有零解},\overline A\mathrm{为非齐次线性方程组}Ax=b\mathrm{的增广矩阵},\mathrm{则下列说法不正确的是}().
$$
$$
A.
\;m=n \quad B.\;r(\overline{A)}=r(A) \quad C.\;\overline A\mathrm{的行向量组线性无关} \quad D.\;\overline A\mathrm{的列向量组线性无关} \quad E. \quad F. \quad G. \quad H.
$$
$$
\begin{array}{l}\;\;\mathrm{齐次方程组}Ax=0\mathrm{只有零解},则R(A)=n,故m=n,即A为n\mathrm{阶矩阵};\\\;\;又n=R(A)\leq R(\overline A)=R(A,b)\leq min\{n,n+1\}=n,即n\leq R(\overline A)\leq n⇒ R(\overline A)=n=R(A);\\\;\;R(\overline A)=n,且\overline A为n×(n+1)\mathrm{阶矩阵},\mathrm{故行向量线性无关},\mathrm{列向量组线性相关}.\end{array}
$$



$$
设A为n\mathrm{阶方阵},且Ax=0\mathrm{有无穷多解},则AA^Tx=0().
$$
$$
A.
\;\mathrm{有无穷多组解} \quad B.\;\mathrm{仅有零解} \quad C.\;\mathrm{有有限多组解} \quad D.\;\mathrm{无解} \quad E. \quad F. \quad G. \quad H.
$$
$$
\;\;Ax=0\mathrm{有无穷多解},则R(A)<\;n,又R(AA^T)\leq min\{R(A),R(A^T)\}=R(A)<\;n,故n\mathrm{元线性方程组}AA^Tx=0\mathrm{也有无穷多解}.
$$



$$
\begin{array}{l}\mathrm{向量组}α_1,α_2,⋯,α_r\mathrm{线性相关},则α_1+t_1β,α_2+t_2β,⋯,α_r+t_rβ(r\geq2)\mathrm{的线性相关性为}\;\left(\right)\\(\mathrm{其中}t_1,t_2,⋯,t_r\mathrm{不全为零},β\mathrm{为与}α_i\mathrm{同维数的任意向量})\end{array}
$$
$$
A.
\;\mathrm{线性相关} \quad B.\;\mathrm{线性无关} \quad C.\;\mathrm{与向量}β\mathrm{有关} \quad D.\;与t_i\mathrm{的值有关} \quad E. \quad F. \quad G. \quad H.
$$
$$
\begin{array}{l}\begin{array}{l}\;\;\mathrm{因为}α_1,α_2,⋯,α_r\mathrm{线性相关},\mathrm{所以存在}k_1,k_2,⋯,k_r\mathrm{不全为零的数使}k_1α_1+k_2α_2+⋯+k_rα_r=0.\\\;\;\mathrm{考虑线性方程组}k_1x_1+k_2x_2+⋯+k_rx_r=0为\;r\geq2,\mathrm{它必有非零解}.\\\;\;设\;(t_1,t_2,⋯,t_r)\mathrm{为任一非零解},\mathrm{则对任意向量}β,\mathrm{都有}\end{array}\\\;\;\;k_1α_1+k_2α_2+⋯+k_rα_r+(k_1t_1+k_2t_2+⋯+k_rt_r)β=0\\\;\;\;k_1(α_1+t_1β)+k_2(α_2+t_2β)+⋯+k_r(α_r+t_rβ)=0.\\\;\;由\;k_1,k_2,⋯,k_r\;\mathrm{不全为零得知}\;α_1+t_1β\;,α_2+t_2β,⋯,α_r+t_rβ\mathrm{线性相关}.\;\\\\\end{array}
$$



$$
设a,b,c,d\mathrm{是不全为零的实数},\mathrm{则方程组}\left\{\begin{array}{c}ax_1+bx_2+cx_3+dx_4=0\\bx_1-ax_2+dx_3-cx_4=0\\cx_1-dx_2-ax_3+bx_4=0\\dx_1+cx_2-bx_3-ax_4=0\end{array}\right.\mathrm{解空间的维数是}(\;\;)
$$
$$
A.
\;0 \quad B.\;1 \quad C.\;2 \quad D.\;3 \quad E. \quad F. \quad G. \quad H.
$$
$$
\begin{array}{l}\;\;\mathrm{系数矩阵}A=\begin{pmatrix}a&b&c&d\\b&-a&d&-c\\c&-d&-a&b\\d&c&-b&-a\end{pmatrix},令p=a^2+b^2+c^2+d^2,则\\\;\;\;\;\;\;\;AA^T=\begin{pmatrix}p&0&0&0\\0&p&0&0\\0&0&p&0\\0&0&0&p\end{pmatrix}⇒\left|AA^T\right|=(a^2+b^2+c^2+d^2)^4,\\\;\;\;\;\;但\left|A\right|=\left|A^T\right|,\left|AA\right|=\left|A\right|\left|A^T\right|=\left|A\right|^2,故\;\left|A\right|=(a^2+b^2+c^2+d^2)^2>0,\mathrm{原方程组只有零解}.\end{array}
$$



$$
设a,b,c,d\mathrm{是不全为零的实数},\mathrm{则方程组}\left\{\begin{array}{c}ax_1+bx_2+cx_3+dx_4=0\\bx_1-ax_2+dx_3-cx_4=0\\cx_1-dx_2-ax_3+bx_4=0\\dx_1+cx_2-bx_3-ax_4=0\end{array}\right.(\;\;)
$$
$$
A.
\;\mathrm{只有零解} \quad B.\;\mathrm{有非零解} \quad C.\;\mathrm{解的情况与参数有关} \quad D.\;\mathrm{解的情况无法确定} \quad E. \quad F. \quad G. \quad H.
$$
$$
\begin{array}{l}\;\;\mathrm{系数矩阵}A=\begin{pmatrix}a&b&c&d\\b&-a&d&-c\\c&-d&-a&b\\d&c&-b&-a\end{pmatrix},令p=a^2+b^2+c^2+d^2,则\\\;\;\;\;\;\;\;AA^T=\begin{pmatrix}p&0&0&0\\0&p&0&0\\0&0&p&0\\0&0&0&p\end{pmatrix}⇒\left|AA^T\right|=(a^2+b^2+c^2+d^2)^4,\\\;\;\;\;\;但\left|A\right|=\left|A^T\right|,\left|AA^T\right|=\left|A\right|\left|A^T\right|=\left|A\right|^2,故\;\left|A\right|=(a^2+b^2+c^2+d^2)^2>0,\mathrm{原方程组只有零解}.\end{array}
$$



$$
\mathrm{线性方程组}Ax=0,\mathrm{其解空间由向量组}α_1=(1,-1,1,0)^T,α_2=(1,1,0,1)^T,α_3=(2,0,1,1)^T\mathrm{所生成},则A=\left(\;\;\;\right)\;.
$$
$$
A.
\begin{pmatrix}1&-3&-7&2\\2&-6&-14&4\end{pmatrix} \quad B.\begin{pmatrix}2&1&0&-1\end{pmatrix} \quad C.\begin{pmatrix}1&-1&1&-1\\1&1&1&0\\1&-1&-1&1\end{pmatrix} \quad D.\begin{pmatrix}1&0&-1&-1\\0&1&1&-1\end{pmatrix} \quad E. \quad F. \quad G. \quad H.
$$
$$
\begin{array}{l}dimV(α_1,α_2,α_3)=2,α_1,α_2\mathrm{是解空间}V(α_1,α_2,α_3)\mathrm{的一组基}.\mathrm{于是},\mathrm{解空间}V=\{α=c_1α_1+c_2α_2\vert c_i∈ R,i=1,2\}\\\mathrm{设所有齐次线性方程为}α_1x_1+α_2x_2+α_3x_3+α_4x_4=0\mathrm{将基代入得齐次线性方程组}\left\{\begin{array}{l}α_1-α_2+α_3=0\\α_1+α_2+α_4=0\end{array}\right.\\\mathrm{解得此方程组的基础解系为}ξ_1=(1,0,-1,-1)^T,ξ_2=(0,1,1,-1)^T\mathrm{所求齐次线性方程组为}\left\{\begin{array}{l}x_1-x_3-x_4=0\\x_2+x_3-x_4=0\end{array}\right.\end{array}
$$



$$
设A为n\mathrm{阶方阵},且r(A^*)=1,\mathrm{向量}α_1,α_2\mathrm{是齐次线性方程组}Ax=O\mathrm{的两个不同的解},则Ax=O\mathrm{的通解为}()
$$
$$
A.
kα_2 \quad B.kα_1 \quad C.k(α_1-α_2) \quad D.k(α_1+\alpha_2) \quad E. \quad F. \quad G. \quad H.
$$
$$
\mathrm{由条件},R(A)=n-1,\mathrm{可知},\mathrm{只有}(α_1-α_2)\mathrm{才可构成齐次线性方程组}Ax=0\mathrm{的基础解系}.
$$



$$
设n\mathrm{阶矩阵}A\mathrm{的各行元素之和均为零},\mathrm{伴随矩阵}A^*\neq O,\mathrm{则线性方程组}Ax=O\mathrm{的全部解为}()
$$
$$
A.
\mathrm{只有零解} \quad B.k(1,1,⋯,1)^T(k∈ R) \quad C.(1,1,⋯,1)^T\mathrm{和零解} \quad D.(1,1,⋯,1)^T \quad E. \quad F. \quad G. \quad H.
$$
$$
\begin{array}{l}\mathrm{由条件可知},\mathrm{线性方程组}Ax=0\mathrm{的基础解系有}n-r=n-(n-1)=1\mathrm{个线性无关解},\mathrm{也就是由一个非零解构成},\mathrm{又由题设可知}(1,1,⋯,1)^T\mathrm{是齐次}\\\mathrm{方程组的一个非零解},\mathrm{所以}k(1,1,⋯,1)^T(k∈ R)\mathrm{为齐次方程组}Ax=0\mathrm{的通解}.\end{array}
$$



$$
设A=\begin{pmatrix}-1&2&-1\\1&-1&0\\-2&1&1\end{pmatrix},则Ax=0\mathrm{的通解为}(),k∈ R
$$
$$
A.
x=k\begin{pmatrix}1\\1\\1\end{pmatrix} \quad B.x=k\begin{pmatrix}1\\0\\0\end{pmatrix} \quad C.x=k\begin{pmatrix}1\\1\\0\end{pmatrix} \quad D.x=k\begin{pmatrix}0\\1\\1\end{pmatrix} \quad E. \quad F. \quad G. \quad H.
$$
$$
\begin{array}{l}\mathrm{对系数矩阵施以初等变换的}A=\begin{pmatrix}-1&2&-1\\1&-1&0\\-2&1&1\end{pmatrix}\rightarrow\begin{pmatrix}1&-1&0\\0&1&-1\\0&-1&1\end{pmatrix}\rightarrow\begin{pmatrix}1&-1&0\\0&1&-1\\0&0&0\end{pmatrix}\\\mathrm{即得与原方程组同解的方程组}\left\{\begin{array}{l}x_1-x_2=0\\x_2-x_3=0\end{array}\right.令x_3=k,则x=k\begin{pmatrix}1\\1\\1\end{pmatrix},k∈ R.\end{array}
$$



$$
\mathrm{方程组}\left\{\begin{array}{l}x_1+2x_2-x_3=0\\2x_1+4x_2+7x_3=0\end{array}\;\right.\mathrm{的通解为}()
$$
$$
A.
x=k\begin{pmatrix}2\\-1\\0\end{pmatrix} \quad B.x=k\begin{pmatrix}2\\1\\0\end{pmatrix} \quad C.x=k\begin{pmatrix}-2\\-1\\0\end{pmatrix} \quad D.x=k\begin{pmatrix}-2\\-1\\1\end{pmatrix} \quad E. \quad F. \quad G. \quad H.
$$
$$
\begin{array}{l}\mathrm{将系数矩阵进行初等行变换},得\;\begin{pmatrix}1&2&-1\\2&4&7\end{pmatrix}\rightarrow\begin{pmatrix}1&2&-1\\0&0&9\end{pmatrix}\\\mathrm{即得到与原方程组同解的方程组}\left\{\begin{array}{l}x_1+2x_2-x_3=0\\9x_3=0\end{array}\right.令x_2=k,则x=k\begin{pmatrix}2\\-1\\0\end{pmatrix}\end{array}
$$



$$
\mathrm{齐次方程组}\left\{\begin{array}{c}2x_1+3x_2+4x_3=0\\x_1+2x_2+2x_3=0\\x_1+x_2+2x_3=0\end{array}\right.\mathrm{的通解为}()
$$
$$
A.
\left\{\begin{array}{c}x_1=2t\\x_2=0\\x_3=-t\end{array}\right. \quad B.\left\{\begin{array}{c}x_1=2t\\x_2=1\\x_3=-t\end{array}\right. \quad C.\left\{\begin{array}{c}x_1=t\\x_2=0\\x_3=2t\end{array}\right. \quad D.\left\{\begin{array}{c}x_1=t\\x_2=1\\x_3=-2t\end{array}\right. \quad E. \quad F. \quad G. \quad H.
$$
$$
\begin{array}{l}\mathrm{对系数矩阵施以初等行变换得}\begin{pmatrix}2&3&4\\1&2&2\\1&1&2\end{pmatrix}\rightarrow\begin{pmatrix}1&1&2\\0&1&0\\0&1&0\end{pmatrix}\rightarrow\begin{pmatrix}1&1&2\\0&1&0\\0&0&0\end{pmatrix}\\\mathrm{即得与原方程组同解的方程组}\left\{\begin{array}{l}x_1+x_2+2x_3=0\\x_2=0\end{array}\right.令x_3=-t,t\mathrm{为任意常数},则\left\{\begin{array}{c}x_1=2t\\x_2=0\\x_3=-t\end{array}\right.\end{array}
$$



$$
设A为n\mathrm{阶方阵},且R(A)=n-1,α_1,α_2\mathrm{是非齐次线性方程组}Ax=b\mathrm{的两个不同的解向量},则Ax=0\mathrm{的通解为}()
$$
$$
A.
kα_2 \quad B.kα_1 \quad C.k(α_1-α_2) \quad D.k(α_1+α_2) \quad E. \quad F. \quad G. \quad H.
$$
$$
\mathrm{由条件可知},\mathrm{只有}(α_1-α_2)\mathrm{才会构成齐次线性方程组}Ax=0\mathrm{的基础解系}.
$$



$$
设n\mathrm{阶矩阵}\boldsymbol A\mathrm{的各行元素之和均为零},\;\mathrm 且\boldsymbol A\mathrm{的秩为}n-1,\mathrm{则线性方程组}\boldsymbol A\boldsymbol x=\mathbf0\mathrm{的全部解为}()
$$
$$
A.
\mathrm{只有零解} \quad B.k(1,1,⋯,1)^T(k∈ R) \quad C.(1,1,⋯,1)^T\mathrm{和零解} \quad D.(1,1,⋯,1)^T \quad E. \quad F. \quad G. \quad H.
$$
$$
\begin{array}{l}\mathrm{由条件可知},\mathrm{线性方程组}Ax=0\mathrm{的基础解系有}n-r=n-(n-1)=1\mathrm{个线性无关解},\mathrm{也就是由一个非零解构成},\\\mathrm{又由题设可知}(1,1,⋯,1)^T\mathrm{是齐次方程组的一个非零解},\mathrm{所以}k(1,1,⋯,1)^T(k∈ R)\mathrm{为齐次方程组}Ax=0\mathrm{的通解}.\end{array}
$$



$$
\mathrm{齐次线性方程组}\left\{\begin{array}{c}x_1+2x_2+x_3-x_4=0\\3x_1+6x_2-x_3-3x_4=0\\5x_1+10x_2+x_3-5x_4=0\end{array}\right.\mathrm{的通解为}()
$$
$$
A.
\begin{pmatrix}x_1\\x_2\\x_3\\x_4\end{pmatrix}=k_1\begin{pmatrix}-2\\1\\0\\0\end{pmatrix}+k_2\begin{pmatrix}1\\0\\0\\1\end{pmatrix},k_1,k_2∈ R \quad B.\begin{pmatrix}x_1\\x_2\\x_3\\x_4\end{pmatrix}=k_1\begin{pmatrix}-2\\1\\0\\0\end{pmatrix}+k_2\begin{pmatrix}1\\0\\1\\1\end{pmatrix},k_1,k_2∈ R \quad C.\begin{pmatrix}x_1\\x_2\\x_3\\x_4\end{pmatrix}=k_1\begin{pmatrix}-2\\1\\3\\0\end{pmatrix}+k_2\begin{pmatrix}1\\0\\0\\1\end{pmatrix},k_1,k_2∈ R \quad D.\begin{pmatrix}x_1\\x_2\\x_3\\x_4\end{pmatrix}=k_1\begin{pmatrix}-2\\1\\0\\0\end{pmatrix}+k_2\begin{pmatrix}2\\0\\1\\1\end{pmatrix},k_1,k_2∈ R \quad E. \quad F. \quad G. \quad H.
$$
$$
\begin{array}{l}\mathrm{对系数矩阵实施初等行变换}:\begin{pmatrix}1&2&1&-1\\3&6&-1&-3\\5&10&1&-5\end{pmatrix}\rightarrow\begin{pmatrix}1&2&0&-1\\0&0&1&0\\0&0&0&0\end{pmatrix},\mathrm{即得}\left\{\begin{array}{c}x_1=-2x_2+x_4\\x_2=x_2\\x_3=0\\x_4=x_4\end{array}\right.\\\mathrm{故方程组的解}\begin{pmatrix}x_1\\x_2\\x_3\\x_4\end{pmatrix}=k_1\begin{pmatrix}-2\\1\\0\\0\end{pmatrix}+k_2\begin{pmatrix}1\\0\\0\\1\end{pmatrix},k_1,k_2∈ R\end{array}
$$



$$
\begin{array}{l}A为2×3\mathrm{阶矩阵},R(A)=2,\mathrm{已知非齐次线性方程组}Ax=b\mathrm{有解}α_1,α_2,且α_1=(1,2,1)^T,α_1+α_2=(1,-1,1)^T\\\mathrm{则对应的齐次线性方程组}Ax=0\mathrm{的通解为}()\end{array}
$$
$$
A.
k(1,5,1)^T,k\mathrm{为任意常数} \quad B.k(0,3,0)^T,k\mathrm{为任意常数} \quad C.k(2,1,2)^T,k\mathrm{为任意常数} \quad D.\mathrm{零解} \quad E. \quad F. \quad G. \quad H.
$$
$$
\begin{array}{l}R(A)=2,则Ax=0\mathrm{的基础解系中含}3-2=1\mathrm{个向量},2α_1-(α_1+α_2)=(1,5,1)^T为Ax=0\mathrm{的解},\mathrm{可取为基础解系},\\\mathrm{因此通解为}k(1,5,1)^T,k\mathrm{为任意常数}\end{array}
$$



$$
\begin{array}{l}A为3\mathrm{阶矩阵},r(A^*)=1,\mathrm{已知非齐次线性方程组}Ax=b\mathrm{有解}α_1,α_2,且α_1=(1,2,1)^T,α_1+α_2=(1,-1,1)^T\\\mathrm{则对应的齐次线性方程组}Ax=0\mathrm{的通解为}()\end{array}
$$
$$
A.
k(1,5,1)^T,k\mathrm{为任意常数} \quad B.k_1(0,3,0)^T+k_2(1,5,1)^T,k_1,k_2\mathrm{为任意常数} \quad C.k(2,1,2)^T,k\mathrm{为任意常数} \quad D.\mathrm{零解} \quad E. \quad F. \quad G. \quad H.
$$
$$
\begin{array}{l}r(A)=2,则Ax=0\mathrm{的基础解系中含}3-2=1\mathrm{个向量},2α_1-(α_1+α_2)=(1,5,1)^T为Ax=0\mathrm{的解},\mathrm{可取为基础解系},\\\mathrm{因此}Ax=0\mathrm{的通解为}k(1,5,1)^T,k\mathrm{为任意常数}\end{array}
$$



$$
\mathrm{已知}3\mathrm{阶矩阵}A\mathrm{的秩为}2,\mathrm{矩阵}B=\begin{pmatrix}1&2&4\\2&4&8\\3&6&12\end{pmatrix},且AB=O,\mathrm{则下列不是线性方程组}Ax=0\mathrm{通解的是}()
$$
$$
A.
x=k_1(1,2,3)^T,k_1\mathrm{为任意常数} \quad B.x=k_1(4,8,12)^T,k_1\mathrm{为任意常数} \quad C.x=k_1(2,4,6)^T,k_1\mathrm{为任意常数} \quad D.x=k_1(1,2,4)^T,k_1\mathrm{为任意常数} \quad E. \quad F. \quad G. \quad H.
$$
$$
R(A)=2,\mathrm{基础解系只有一个向量},B\mathrm{的任意列都可以}.
$$



$$
\mathrm{已知}3\mathrm{阶非零矩阵}A\mathrm{满足}AB=O,\mathrm{其中}B=\begin{pmatrix}1&2&3\\2&4&6\\3&6&k\end{pmatrix},\;k\neq9,则r(A)\mathrm{满足}()
$$
$$
A.
R(A)=1 \quad B.R(A)=2 \quad C.R(A)=3 \quad D.R(A)=0 \quad E. \quad F. \quad G. \quad H.
$$
$$
由AB=O知,B\mathrm{的每一列均为}Ax=0\mathrm{的解},且r(A)+r(B)\leq3.r(B)=2,\mathrm{于是}r(A)\leq1,\mathrm{显然}r(A)\geq1,故r(A)=1
$$



$$
\begin{array}{l}设A为n\mathrm{阶实矩阵},A^T是A\mathrm{的转置矩阵},\mathrm{则对于线性方程组}(I)Ax=O(II)A^TAx=O\mathrm{正确的个数是}()\\(1)(I)\mathrm{的解是}(II)\mathrm{的解}\\(2)\;(II)\mathrm{的解是}(I)\mathrm{的解}\\(3)\;(I)\mathrm{的解不是}(II)\mathrm{的解}\;\;\\(4)\;(II)\mathrm{的解不是}(I)\mathrm{的解}\end{array}
$$
$$
A.
1 \quad B.2 \quad C.3 \quad D.4 \quad E. \quad F. \quad G. \quad H.
$$
$$
\begin{array}{l}\mathrm{显然}(I)\mathrm{的解是}(II)\mathrm{的解},\mathrm{因为若有}α,使Aα=0,\mathrm{则有}A^TAα=0;\\若A^TAα=0,\mathrm{则有}α^TA^TAα=0,即(Aα)^TAα=0,又A\mathrm{为实矩阵},\mathrm{故必有}Aα=0即(II)\mathrm{的解也是}(I)\mathrm{的解}.\end{array}
$$



$$
\begin{array}{l}设A为3\mathrm{阶矩阵},r(A^*)=1,\mathrm{已知非齐次线性方程组}Ax=b\mathrm{有解}α_1,α_2,且α_1=(1,1,1)^T,α_1+α_2=(1,-1,1)^T\\\mathrm{则对应的齐次线性方程组}Ax=0\mathrm{的通解为}()\end{array}
$$
$$
A.
k(1,3,1)^T,k\mathrm{为任意常数} \quad B.k(0,3,0)^T,k\mathrm{为任意常数} \quad C.k(2,1,2)^T,k\mathrm{为任意常数} \quad D.\mathrm{零解} \quad E. \quad F. \quad G. \quad H.
$$
$$
\begin{array}{l}r(A)=2,则Ax=0\mathrm{的基础解系中含}3-2=1\mathrm{个向量},2α_1-(α_1+α_2)=(1,3,1)^T为Ax=0\mathrm{的解},\mathrm{可取为基础解系},\\\mathrm{因此}Ax=0\mathrm{的通解为}k(1,3,1)^T,k\mathrm{为任意常数}\end{array}
$$



$$
\begin{array}{l}设n\mathrm{阶矩阵}A的n\mathrm{个列向量为}a_i\left(i=1,\;2,\;...,\;n\right),\;而n\mathrm{阶矩阵}B的n\mathrm{个列向量为}\\a_1+\;a_2,a_2+\;a_3,...,a_{n-1}+\;a_n,a_n+\;a_1\;\mathrm{则当}R\left(A\right)=n时,\mathrm{齐次线性方程组}Bx=0\mathrm{的解为}(\;).\end{array}
$$
$$
A.
当n\mathrm{为奇数时},Bx=0\;\mathrm{只有零解} \quad B.当n\mathrm{为奇数时},Bx=0\mathrm{有非零解} \quad C.\mathrm{方程组}Bx=0\mathrm{只有零解},与n\mathrm{奇偶性无关} \quad D.\mathrm{方程组}Bx=0\mathrm{有非零解},与n\mathrm{奇偶性无关} \quad E. \quad F. \quad G. \quad H.
$$
$$
\begin{array}{l}\mathrm{易知}B=AC,\mathrm{其中}\\C=\begin{pmatrix}1&0&0&...&0&0\;\;1\\1&1&0&...&0&0\;\;0\\0&1&1&...&0&0\;\;0\\\vdots&\vdots&\vdots&...&\vdots&\vdots\;\;\vdots\\0&0&0&...&1&1\;\;0\\0&0&0&...&0&1\;\;1\end{pmatrix},\;\left|C\right|=1+\left(-1\right)^{n+1}\\当n\mathrm{为奇数时},\left|C\right|\neq0,C\mathrm{为满秩矩阵},\mathrm{因而有}\\R\left(AC\right)=R\left(A\right)=R\left(B\right)=n\left(\mathrm{未知数个数}\right)\\故Bx=O\mathrm{只有零解}.\;\;\\当n\mathrm{为偶数时},\left|C\right|=0,\mathrm{从而}\left|B\right|\;=0故R\left(B\right)<\;n,\mathrm{于是}Bx=O\mathrm{有非零解}.\end{array}
$$



$$
\mathrm{方程组}\left\{\begin{array}{c}x_1-x_2-x_3+x_4=0\\x_1-x_2+x_3-3x_4=0\\x_1-x_2-2x_3+3x_4=0\end{array}\right.\mathrm{的通解为}()
$$
$$
A.
x=k_1\begin{pmatrix}1\\1\\0\\0\end{pmatrix}+k_2\begin{pmatrix}1\\0\\2\\1\end{pmatrix},k_1,k_2\mathrm{为任意常数} \quad B.x=k_1\begin{pmatrix}1\\1\\0\\0\end{pmatrix}+k_2\begin{pmatrix}1\\0\\2\\1\end{pmatrix}+k_3\begin{pmatrix}1\\0\\1\\0\end{pmatrix},k_1,k_2,k_3\mathrm{为任意常数} \quad C.x=k\begin{pmatrix}1\\1\\0\\0\end{pmatrix}k\mathrm{为任意常数} \quad D.x=k\begin{pmatrix}1\\0\\2\\1\end{pmatrix}k\mathrm{为任意常数} \quad E. \quad F. \quad G. \quad H.
$$
$$
\begin{array}{l}A\rightarrow\begin{pmatrix}1&-1&0&-1\\0&0&1&-2\\0&0&0&0\end{pmatrix},\mathrm{即得与原方程组同解的方程组}\left\{\begin{array}{l}x_1-x_2-x_4=0\\x_3-2x_4=0\end{array}\right.\\令x_2,x_4\mathrm{为自由未知量},则x=k_1\begin{pmatrix}1\\1\\0\\0\end{pmatrix}+k_2\begin{pmatrix}1\\0\\2\\1\end{pmatrix},k_1,k_2\mathrm{为任意常数}\end{array}
$$



$$
\mathrm{方程组}\left\{\begin{array}{c}2x_1-2x_2-x_3=0\\2x_1+4x_2+5x_3=0\\-4x_1+x_2-x_3=0\\2x_1-5x_2-4x_3=0\end{array}\right.\mathrm{的通解为}()
$$
$$
A.
x=k\begin{pmatrix}-1\\-2\\2\end{pmatrix},k∈ R \quad B.x=k_1\begin{pmatrix}-1\\-2\\2\end{pmatrix}+k_2\begin{pmatrix}-1\\1\\1\end{pmatrix},k_1,k_2∈ R \quad C.x=k\begin{pmatrix}-1\\1\\1\end{pmatrix},k∈ R \quad D.x=k\begin{pmatrix}-1\\2\\1\end{pmatrix},k∈ R \quad E. \quad F. \quad G. \quad H.
$$
$$
\begin{array}{l}A\rightarrow\begin{pmatrix}2&-2&-1\\2&4&5\\-4&1&-1\\2&-5&-4\end{pmatrix}\rightarrow\begin{pmatrix}1&-1&-\frac12\\0&1&1\\0&0&0\\0&0&0\end{pmatrix}\mathrm{即得与原方程组同解的方程组}\left\{\begin{array}{l}x_1-x_2-\frac12x_3=0\\x_2+x_3=0\end{array}\right.\\令x_3=2,则x=k\begin{pmatrix}-1\\-2\\2\end{pmatrix},k\mathrm{为任意常数}.\end{array}
$$



$$
\mathrm{已知}A=\begin{pmatrix}1&2&3\\2&t&6\\3&6&9\end{pmatrix},B是3\mathrm{阶矩阵且秩}r(B)=2,若AB=O,则t=()
$$
$$
A.
1 \quad B.2 \quad C.3 \quad D.4 \quad E. \quad F. \quad G. \quad H.
$$
$$
\begin{array}{l}B\mathrm{的每一列是齐次方程组}Ax=O\mathrm{的解},\mathrm{由于秩}r(B)=2,\mathrm{因此}Ax=0\mathrm{至少有}\;2\;\mathrm{个线性无关的解}.\;\\\mathrm{那么其基础解系中解向量的个数至少是}2,\;\mathrm{于是由}n-r(A)=3-r(A)\geq2.\mathrm{得知}r(A)\leq1,\mathrm{显然}r(A)\geq1,\mathrm{故必有}r(A)=1.\\\mathrm{那么矩阵}A\mathrm{中任两列而成比例},即\frac12=\frac2t=\frac36\end{array}
$$



$$
\mathrm{已知}3\mathrm{阶矩阵}A\mathrm{的秩为}2,\mathrm{矩阵}B=\begin{pmatrix}1&2&3\\2&4&6\\3&6&9\end{pmatrix},且AB=O,\mathrm{则下列不是}Ax=0\mathrm{通解的是}()
$$
$$
A.
x=k_1(1,2,3)^T,k_1\mathrm{为任意常数} \quad B.x=k_1(3,6,9)^T,k_1\mathrm{为任意常数} \quad C.x=k_1(2,4,6)^T,k_1\mathrm{为任意常数} \quad D.x=k_1(1,1,1)^T,k_1\mathrm{为任意常数} \quad E. \quad F. \quad G. \quad H.
$$
$$
R(A)=2,\mathrm{所以}B\mathrm{的每一列都是基础解系}.
$$



$$
\mathrm{已知}3\mathrm{阶非零矩阵}A\mathrm{满足}AB=O,\mathrm{其中}B=\begin{pmatrix}1&2&3\\2&4&6\\3&6&8\end{pmatrix},则R(A)\mathrm{满足}()
$$
$$
A.
R(A)=1 \quad B.R(A)=2 \quad C.R(A)=3 \quad D.R(A)=0 \quad E. \quad F. \quad G. \quad H.
$$
$$
由AB=O知,B\mathrm{的每一列均为}Ax=0\mathrm{的解},且r(A)+r(B)\leq3.r(B)=2,\mathrm{于是}r(A)\leq1,\mathrm{显然}r(A)\geq1,故r(A)=1
$$



$$
\begin{array}{l}设A为n\mathrm{阶实矩阵},A^T是A\mathrm{的转置矩阵},\mathrm{则对于线性方程组}(I)Ax=O和(II)A^TAx=O\;\mathrm{必有}()\end{array}
$$
$$
A.
(II)\mathrm{的解是}(I)\mathrm{的解},(I)\mathrm{的解也是}(II)\mathrm{的解} \quad B.(II)\mathrm{的解是}(I)\mathrm{的解},但(I)\mathrm{的解不是}(II)\mathrm{的解} \quad C.\;(I)\mathrm{的解不是}(II)\mathrm{的解}\;,\;(II)\mathrm{的解也不是}(I)\mathrm{的解} \quad D.(I)\mathrm{的解是}(II)\mathrm{的解}\;,\;但(II)\mathrm{的解不是}(I)\mathrm{的解} \quad E. \quad F. \quad G. \quad H.
$$
$$
\begin{array}{l}\mathrm{显然}(I)\mathrm{的解是}(II)\mathrm{的解},\mathrm{因为若有}α,使Aα=0,\mathrm{则有}A^TAα=0;\\若A^TAα=0,\mathrm{则有}α^TA^TAα=0,即(Aα)^TAα=0,又A\mathrm{为实矩阵},\mathrm{故必有}Aα=0即(II)\mathrm{的解也是}(I)\mathrm{的解}.\end{array}
$$



$$
\mathrm{设齐次线性方程组}\left\{\begin{array}{c}(1+a)x_1+x_2+x_3+x_4=0\\2x_1+(2+a)x_2+2x_3+2x_4=0\\3x_1+3x_2+(3+a)x_3+3x_4=0\\4x_1+4x_2+4x_3+(4+a)x_4=0\end{array}\right.\mathrm{的解空间维数为}3,\mathrm{则通解为}()
$$
$$
A.
x=k_1η_1+k_2η_2+k_3η_3,η_1=(-1,1,0,0)^T,η_2=(-1,0,1,0)^T,η_3=(-1,0,0,1)^T \quad B.x=k_1η_1+k_2η_2+k_3η_3,η_1=(1,1,0,0)^T,η_2=(1,0,1,0)^T,η_3=(1,0,0,1)^T \quad C.x=k_1η_1+k_2η_2+k_3η_3,η_1=(0,1,0,0)^T,η_2=(0,0,1,0)^T,η_3=(0,0,0,1)^T \quad D.x=k_1η_1+k_2η_2+k_3η_3,η_1=(1,1,0,0)^T,η_2=(2,0,1,0)^T,η_3=(3,0,0,1)^T \quad E. \quad F. \quad G. \quad H.
$$
$$
\begin{array}{l}\mathrm{方程组的系数行列式}\left|A\right|=\begin{vmatrix}1+a&1&1&1\\2&2+a&2&2\\3&3&3+a&3\\4&4&4&4+a\end{vmatrix}=(a+10)a^3.\mathrm{由于原方程组有非零解},故\left|A\right|=0,\mathrm{推知}a=0或a=-10\\当a=0时,\mathrm{对系数矩阵}A\mathrm{作初等变换},有A=\begin{pmatrix}1&1&1&1\\2&2&2&2\\3&3&3&3\\4&4&4&4\end{pmatrix}\rightarrow\begin{pmatrix}1&1&1&1\\0&0&0&0\\0&0&0&0\\0&0&0&0\end{pmatrix}\mathrm{故方程组的同解方程组为}x_1+x_2+x_3+x_4=0\\\mathrm{其基础解系为}η_1=(-1,1,0,0)^T,η_2=(-1,0,1,0)^T,η_3=(-1,0,0,1)^T\mathrm{于是所求方程组的通解为}x=k_1η_1+k_2η_2+k_3η_3,k_1,k_2,k_3\mathrm{为任意常数}.\end{array}
$$



$$
\mathrm{设齐次线性方程组}\left\{\begin{array}{c}(1+a)x_1+x_2+x_3+x_4=0\\2x_1+(2+a)x_2+2x_3+2x_4=0\\3x_1+3x_2+(3+a)x_3+3x_4=0\\4x_1+4x_2+4x_3+(4+a)x_4=0\end{array}\right.\mathrm{的解空间维数为}1,\mathrm{则通解为}()
$$
$$
A.
x=k(1,2,3,4)^T,k\mathrm{为任意常数} \quad B.x=k(1,1,1,1)^T,k\mathrm{为任意常数} \quad C.x=k(4,3,2,1)^T,k\mathrm{为任意常数} \quad D.x=k(0,1,2,3)^T,k\mathrm{为任意常数} \quad E. \quad F. \quad G. \quad H.
$$
$$
\begin{array}{l}\begin{array}{l}\mathrm{方程组的系数行列式}\left|A\right|=\begin{vmatrix}1+a&1&1&1\\2&2+a&2&2\\3&3&3+a&3\\4&4&4&4+a\end{vmatrix}=(a+10)a^3\mathrm{由于原方程组有非零解},故\left|A\right|=0,\mathrm{推知}a=0或a=-10\\当a=0时,\mathrm{对系数矩阵}A\mathrm{作初等变换},有A=\begin{pmatrix}1&1&1&1\\2&2&2&2\\3&3&3&3\\4&4&4&4\end{pmatrix}\rightarrow\begin{pmatrix}1&1&1&1\\0&0&0&0\\0&0&0&0\\0&0&0&0\end{pmatrix}\\当a=-10时,对A\mathrm{作初等变换},有A=\begin{pmatrix}-9&1&1&1\\2&-8&2&2\\3&3&-7&3\\4&4&4&-6\end{pmatrix}\rightarrow\begin{pmatrix}-9&1&1&1\\20&-1&0&0\\30&0&-10&0\\40&0&0&-10\end{pmatrix}\rightarrow\begin{pmatrix}-9&1&1&1\\-2&1&0&0\\-3&0&1&0\\-4&0&0&1\end{pmatrix}\rightarrow\begin{pmatrix}0&0&0&0\\-2&1&0&0\\-3&0&1&0\\-4&0&0&1\end{pmatrix}\end{array}\\\mathrm{故方程组的同解方程组为}\left\{\begin{array}{c}x_2=2x_1\\x_3=3x_1\\x_4=4x_1\end{array}\right.,\mathrm{由此得基础解系为}η=(1,2,3,4)^T,\mathrm{于是所求方程组的通解为}x=kη,\mathrm{其中}k\mathrm{为任意常数}.\end{array}
$$



$$
\mathrm{设线性方程组}\left\{\begin{array}{c}x_1+x_2+x_3=0\\x_1+2x_2+ax_3=0\\x_1+4x_2+a^2x_3=0\end{array}\right.,\mathrm{则下列错误的说法是}()
$$
$$
A.
当a\neq1且a\neq2时,\mathrm{方程组只有零解} \quad B.当a=1时,\mathrm{通解为}x=k\begin{pmatrix}-1\\0\\1\end{pmatrix},\mathrm{其中}k\mathrm{为任意常数} \quad C.当a=2时,\mathrm{通解为}x=k\begin{pmatrix}0\\-1\\1\end{pmatrix},\mathrm{其中}k\mathrm{为任意常数} \quad D.当a=1或a=2时,\mathrm{系数矩阵的秩均为}1 \quad E. \quad F. \quad G. \quad H.
$$
$$
\begin{array}{l}\mathrm{方程组的系数行列式}\begin{vmatrix}1&1&1\\1&2&a\\1&4&a^2\end{vmatrix}=(a-1)(a-2);当a\neq1,a\neq2时,\mathrm{方程组只有零解}.\\当a=1时,\mathrm{对方程组的系数矩阵施以初等行变换}\begin{pmatrix}1&1&1\\1&2&1\\1&4&1\end{pmatrix}\rightarrow\begin{pmatrix}1&0&1\\0&1&0\\0&0&0\end{pmatrix},x=k\begin{pmatrix}-1\\0\\1\end{pmatrix},\mathrm{其中}k\mathrm{为任意常数};\\\\当a=2时,\mathrm{对方程组的系数矩阵施以初等行变换}\begin{pmatrix}1&1&1\\1&2&2\\1&4&4\end{pmatrix}\rightarrow\begin{pmatrix}1&0&0\\0&1&1\\0&0&0\end{pmatrix}\mathrm{因此通解为}x=k\begin{pmatrix}0\\-1\\1\end{pmatrix},\mathrm{其中}k\mathrm{为任意常数}.\\\end{array}
$$



$$
\mathrm{若齐次线性方程组}\left\{\begin{array}{c}ax_1+x_2+x_3=0\\x_1+ax_2+x_3=0\\x_1+x_2+ax_3=0\end{array}\right.\mathrm{的解空间维数为}2,则a\mathrm{的值和通解分别为}()
$$
$$
A.
1,\begin{pmatrix}x_1\\x_2\\x_3\end{pmatrix}=c\begin{pmatrix}1\\1\\1\end{pmatrix} \quad B.-2,\begin{pmatrix}x_1\\x_2\\x_3\end{pmatrix}=c\begin{pmatrix}1\\1\\1\end{pmatrix} \quad C.1,\begin{pmatrix}x_1\\x_2\\x_3\end{pmatrix}=c_1\begin{pmatrix}-1\\1\\0\end{pmatrix}+c_2\begin{pmatrix}-1\\0\\1\end{pmatrix} \quad D.-2,\begin{pmatrix}x_1\\x_2\\x_3\end{pmatrix}=c_1\begin{pmatrix}-1\\1\\0\end{pmatrix}+c_2\begin{pmatrix}-1\\0\\1\end{pmatrix} \quad E. \quad F. \quad G. \quad H.
$$
$$
\begin{array}{l}A\rightarrow\begin{pmatrix}a&1&1\\1&a&1\\1&1&a\end{pmatrix}\xrightarrow{r_1\overset{}↔ r_2}\begin{pmatrix}1&1&a\\1&a&1\\a&1&1\end{pmatrix}\xrightarrow[{r_3-ar_1}]{r_2-r_1}\begin{pmatrix}1&1&a\\0&a-1&1-a\\0&1-a&1-a^2\end{pmatrix}\xrightarrow{r_3+r_2}\begin{pmatrix}1&1&a\\0&a-1&1-a\\0&0&(1-a)(2+a)\end{pmatrix}\\当a=1时,R(A)=1,\mathrm{齐次线性方程组有非零解};\mathrm{解空间维数是}2.\\\mathrm{通解为}\begin{pmatrix}x_1\\x_2\\x_3\end{pmatrix}=c_1\begin{pmatrix}-1\\1\\0\end{pmatrix}+c_2\begin{pmatrix}-1\\0\\1\end{pmatrix}\end{array}
$$



$$
\mathrm{若齐次线性方程组}\left\{\begin{array}{c}ax_1+x_2+x_3=0\\x_1+ax_2+x_3=0\\x_1+x_2+ax_3=0\end{array}\right.\mathrm{的解空间维数为}1,则a\mathrm{的值和通解分别为}()
$$
$$
A.
1,\begin{pmatrix}x_1\\x_2\\x_3\end{pmatrix}=k\begin{pmatrix}1\\1\\1\end{pmatrix},k\mathrm{为任意常数}. \quad B.-2,\begin{pmatrix}x_1\\x_2\\x_3\end{pmatrix}=k\begin{pmatrix}1\\1\\1\end{pmatrix},k\mathrm{为任意常数}. \quad C.1,\begin{pmatrix}x_1\\x_2\\x_3\end{pmatrix}=k_1\begin{pmatrix}-1\\1\\0\end{pmatrix}+k_2\begin{pmatrix}-1\\0\\1\end{pmatrix},k_1,k_2\mathrm{为任意常数}. \quad D.-2,\begin{pmatrix}x_1\\x_2\\x_3\end{pmatrix}=k_1\begin{pmatrix}-1\\1\\0\end{pmatrix}+k_2\begin{pmatrix}-1\\0\\1\end{pmatrix},k_1,k_2\mathrm{为任意常数}. \quad E. \quad F. \quad G. \quad H.
$$
$$
\begin{array}{l}\begin{array}{l}A\rightarrow\begin{pmatrix}a&1&1\\1&a&1\\1&1&a\end{pmatrix}\xrightarrow{r_1\overset{}↔ r_2}\begin{pmatrix}1&1&a\\1&a&1\\a&1&1\end{pmatrix}\xrightarrow[{r_3-ar_1}]{r_2-r_1}\begin{pmatrix}1&1&a\\0&a-1&1-a\\0&1-a&1-a^2\end{pmatrix}\xrightarrow{r_3+r_2}\begin{pmatrix}1&1&a\\0&a-1&1-a\\0&0&(1-a)(2+a)\end{pmatrix}\\当a=1时,R(A)=1,\mathrm{齐次线性方程组有非零解};\mathrm{解空间维数是}2.\\当a=-2时,R(A)=2,\mathrm{齐次线性方程组有非零解};\mathrm{解空间维数是}1.\end{array}\\当a=-2时,\mathrm{系数矩阵经初等行变换}\begin{pmatrix}1&1&-2\\0&-3&3\\0&0&0\end{pmatrix}\xrightarrow{r_2\div(-3)}\begin{pmatrix}1&1&-2\\0&1&-1\\0&0&0\end{pmatrix}\xrightarrow{r_3-r_2}\begin{pmatrix}1&0&-1\\0&1&-1\\0&0&0\end{pmatrix}\\得\left\{\begin{array}{c}x_1=x_3\\x_2=x_3\\x_3=x_3\end{array}\right.(x_3\mathrm{可任意取值}).\mathrm{所求方程组的解为}\begin{pmatrix}x_1\\x_2\\x_3\end{pmatrix}=k\begin{pmatrix}1\\1\\1\end{pmatrix},\;\left(k∈ R\right).\end{array}
$$



$$
设A,B为n\mathrm{阶矩阵},(AB)x=0与Bx=0\mathrm{同解的充要条件是}()
$$
$$
A.
R(AB)=R(B) \quad B.R(AB)=n \quad C.R(B)=n \quad D.R(AB)\leq R(B) \quad E. \quad F. \quad G. \quad H.
$$
$$
\begin{array}{l}\begin{array}{l}\mathrm{必要性},\mathrm{因为}(AB)x=0与Bx=0\mathrm{有相同的解},\mathrm{所以有相同的基础解系},\mathrm{即它们的基础解系中含有相同的解向量}n-R(B)=n-R(AB),\\\mathrm{所以}R(AB)=R(B).\\\mathrm{充分性},设R(AB)=R(B),\mathrm{并设}Bx=0\mathrm{的一个基础解系为}ξ_1,ξ_2,⋯,ξ_k,\mathrm{下面证明}ξ_1,ξ_2,⋯,ξ_k\mathrm{也是}(AB)x=0\mathrm{的一个基础解系}.\mathrm{首先因为}\\ABξ_1=A(Bξ_1)=A0=0(i=1,2,⋯,k),\mathrm{所以}ξ_i(i=1,2,⋯,k)\mathrm{是方程组}(AB)x=O\mathrm{的解}.\\\mathrm{因为}ξ_1,ξ_2,⋯,ξ_k\mathrm{线性无关},且R(AB)=R(B),\mathrm{所以}(AB)x=0与Bx=0\mathrm{的基础解系中含有相同数量的解向量},故ξ_1,ξ_2,⋯,ξ_k是(AB)x=0\mathrm{的基础解系},\end{array}\\(AB)x=0与Bx=0\mathrm{有完全相同的解}.\end{array}
$$



$$
\mathrm{方程}x_1-4x_2+2x_3-5x_4=6\mathrm{的通解为}()
$$
$$
A.
k_1(4,1,0,0)^T+k_2(-2,0,1,0)^T+k_3(5,0,0,1)^T+(6,0,0,0)^T,k_1,k_2,k_3∈ R \quad B.k_1(4,1,0,0)^T+k_2(-2,0,1,0)^T+k_3(5,0,0,1)^T+(1,0,0,0)^T,k_1,k_2,k_3∈ R \quad C.k_1(0,1,0,0)^T+k_2(-2,0,1,0)^T+k_3(5,0,0,1)^T+(6,0,0,0)^T,k_1,k_2,k_3∈ R \quad D.k_1(4,1,0,0)^T+k_2(2,0,1,0)^T+k_3(5,0,0,1)^T+(6,0,0,0)^T,k_1,k_2,k_3∈ R \quad E. \quad F. \quad G. \quad H.
$$
$$
\begin{array}{l}\mathrm{其增广矩阵}B=(1,-4,2,-5,6),因r(A)=r(B)=1<4,\mathrm{故该方程有无穷多解}.\\\mathrm{特解为}η^*=(6,0,0,0)^T,\mathrm{对应齐次线性方程组的一个基础解系含}n-r(A)=4-1=3\mathrm{个解向量}:\\ξ_1=(4,1,0,0)^T,ξ_2=(-2,0,1,0)^T,ξ_3=(5,0,0,1)^T,\mathrm{故所给方程的通解为}k_1ξ_1+k_2ξ_2+k_3ξ_3+η^*,\mathrm{其中}k_1,k_2,k_3\mathrm{为常数}.\end{array}
$$



$$
\begin{array}{l}设A是n\mathrm{阶方阵},Ax=O\mathrm{只有零解},\mathrm{则只有零解的方程组个数是}()\\(1)A^kx=O\;\;\;\;(2)A^Tx=O\;\;\;\;\;(3)\;A^TAx=O\;\;\;\;\;\;(4)A^* x=O\;\end{array}
$$
$$
A.
1 \quad B.2 \quad C.3 \quad D.4 \quad E. \quad F. \quad G. \quad H.
$$
$$
Ax=0\mathrm{只有零解},\mathrm{所以矩阵}A\mathrm{可逆},\mathrm{从而}A^k,A^T,A^TA,A^*\mathrm{都可逆}.
$$



$$
设α_1=\begin{pmatrix}\;a_1\\a_2\\a_3\end{pmatrix},α_2=\begin{pmatrix}b_1\\b_2\\b_3\end{pmatrix},α_3=\begin{pmatrix}c_1\\c_2\\c_3\end{pmatrix},\mathrm{则三直线}a_ix+b_iy+c_i=0(i=1,2,3)\mathrm{交于一点的充要条件是}().
$$
$$
A.
α_1,α_2,α_3\mathrm{线性相关} \quad B.α_1,α_2\mathrm{线性无关} \quad C.R\left(α_1,α_2,α_3\right)=R\left(α_1,α_2\right) \quad D.α_1,α_2,α_3\mathrm{线性相关},α_1,α_2\mathrm{线性无关} \quad E. \quad F. \quad G. \quad H.
$$
$$
\mathrm{三条直线交于一点的充要条件}R\left(α_1,α_2,α_3\right)=R\left(α_1,α_2\right)=2.
$$



$$
设β_1,\;β_2\mathrm{是非齐次线性方程组}Ax=b\mathrm{的两个不同的解},a_1,\;a_2是Ax=0\mathrm{的基础解系},k_1,\;k_2\mathrm{为任意常数},则Ax=b\mathrm{的通解为}\left(\right).\;
$$
$$
A.
k_1a_1+k_2\left(a_1+a_2\right)+\frac12\left(β_1-β_2\right),\;\;k_1,k_2∈ R \quad B.k_1a_1+k_2\left(a_1-a_2\right)+\frac12\left(β_1+β_2\right),\;\;k_1,k_2∈ R \quad C.k_1a_1+k_2\left(β_1+β_2\right)+\frac12\left(β_1-β_2\right),\;\;k_1,k_2∈ R \quad D.k_1a_1+k_2\left(β_1-β_2\right)+\frac12\left(β_1+β_2\right),\;\;k_1,k_2∈ R \quad E. \quad F. \quad G. \quad H.
$$
$$
\begin{array}{l}β_1,\;β_2\mathrm{是非齐次线性方程组的}Ax=b\mathrm{两个不同的解},\\则\frac12\left(β_1+β_2\right)为Ax=b\mathrm{的一个解},且\frac12\left(β_1-β_2\right)为Ax=0\mathrm{的解};\;\;\\a_1,\;a_2是Ax=0\mathrm{的基础解系},则a_1,\;a_2\mathrm{线性无关},又a_1,\;a_1-\;a_2\mathrm{也线性无关},\\\mathrm{因此也构成}Ax=0\mathrm{的基础解系},故Ax=b\mathrm{的通解可表示为}\\k_1a_1+k_2\left(a_1-a_2\right)+\frac12\left(β_1+β_2\right)\;.\;\;\\附:a_1,\;β_1-β_2\mathrm{由于不一定线性无关},\mathrm{因此不能构成基础解系}.\end{array}
$$



$$
设A为n\mathrm{阶方阵},\mathrm 且R\left(A\right)=n-1,a_1,\;a_2\mathrm{是非齐次线性方程组}Ax=b\mathrm{的两个不同的解向量},\mathrm 则Ax=O\mathrm{的通解为}\left(\right)
$$
$$
A.
ka_2\;,k∈ R \quad B.ka_1,k∈ R \quad C.k\left(a_1-a_2\right),k∈ R \quad D.k\left(a_1+a_2\right),k∈ R \quad E. \quad F. \quad G. \quad H.
$$
$$
\mathrm{由线性方程组解的性质定理可知},\mathrm{只有}\left(a_1-a_2\right)\mathrm{才是齐次线性方程组}Ax=0\mathrm{的解}.\mathrm{由知}R\left(A\right)=n-1,\mathrm{基础解系只含一个解向量}.
$$



$$
\begin{array}{l}设a_1,\;a_2,\;a_3\mathrm{是四元非齐次线性方程组}Ax=b\mathrm{的三个解向量},且R\left(A\right)=3,a_1=\left(1,\;2,\;3,\;4\right)^T,a_2+a_3=\left(0,\;1,\;2,\;3\right)^T.\;\\C\mathrm{表示任意常数},\mathrm{则线性方程组}Ax=b\mathrm{的通解}x=\left(\right)\end{array}
$$
$$
A.
\begin{pmatrix}1\\2\\3\\4\end{pmatrix}+C\begin{pmatrix}1\\1\\1\\1\end{pmatrix} \quad B.\begin{pmatrix}1\\2\\3\\4\end{pmatrix}+C\begin{pmatrix}0\\1\\2\\3\end{pmatrix} \quad C.\begin{pmatrix}1\\2\\3\\4\end{pmatrix}+C\begin{pmatrix}2\\3\\4\\5\end{pmatrix} \quad D.\begin{pmatrix}1\\2\\3\\4\end{pmatrix}+C\begin{pmatrix}3\\4\\5\\6\end{pmatrix} \quad E. \quad F. \quad G. \quad H.
$$
$$
\begin{array}{l}\\\mathrm{由于}r\left(A\right)=3,\mathrm{因此}Ax=0\mathrm{的基础解系中所含解的向量个数为}n-r\left(A\right)=4-3=1,\mathrm{根据条件可选取}\;\;\\α=2α_1-\left(α_2+α_3\right)=\begin{pmatrix}2\\3\\4\\5\end{pmatrix},β=α_1=\begin{pmatrix}1\\2\\3\\4\end{pmatrix}.\end{array}
$$



$$
\begin{array}{l}\mathrm{设四元非齐次线性方程组}Ax=b\mathrm{的系数矩阵的秩为}3,\mathrm{已知}η_1,\;η_2,\;η_3,\mathrm{为它的三个解向量},且\;\;\\η_1+η_2=\begin{pmatrix}2\\3\\4\\2\end{pmatrix},η_2-η_3=\begin{pmatrix}1\\0\\2\\1\end{pmatrix},\;\;\mathrm{则方程组的通解为}(\;).\end{array}
$$
$$
A.
X=k\begin{pmatrix}2\\3\\4\\2\end{pmatrix}+\begin{pmatrix}1\\0\\2\\1\end{pmatrix},\;k∈ R \quad B.X=k\begin{pmatrix}1\\0\\2\\1\end{pmatrix}+\begin{pmatrix}2\\3\\4\\2\end{pmatrix},\;k∈ R \quad C.X=k\begin{pmatrix}1\\0\\2\\1\end{pmatrix}+\begin{pmatrix}1\\\textstyle\frac32\\2\\1\end{pmatrix},\;k∈ R \quad D.X=k\begin{pmatrix}1\\\textstyle\frac32\\2\\1\end{pmatrix}+\begin{pmatrix}1\\0\\2\\1\end{pmatrix},\;k∈ R \quad E. \quad F. \quad G. \quad H.
$$
$$
\begin{array}{l}η_2-η_3=\begin{pmatrix}1\\0\\2\\1\end{pmatrix}\mathrm{即为齐次方程组的基础解系}.\;\;\\又∵ A\left(η_1+η_2\right)=Aη_1+Aη_2=b+b=2b,\;\;\\∴η={\textstyle\frac{η_1+η_2}2}=\begin{pmatrix}1\\\textstyle\frac32\\2\\1\end{pmatrix}\mathrm{为非齐次方程组的一个解},\;∴\;\mathrm{通解为}X=k\begin{pmatrix}1\\0\\2\\1\end{pmatrix}+\begin{pmatrix}1\\\textstyle\frac32\\2\\1\end{pmatrix}.\end{array}
$$



$$
\begin{array}{l}\mathrm{设三元非齐次线性方程组}Ax=b\mathrm{的系数矩阵}A\mathrm{的秩为}2,\mathrm{且它的三个解向量}ξ_1,\;ξ_2,\;ξ_3\;\mathrm{满足}\\ξ_1+ξ_2=\left(3,\;1,\;-1\right)^T,\;ξ_1+ξ_3=\left(2,\;0,\;-2\right)^T,\;\;则Ax=b\mathrm{的通解为}(\;).\end{array}
$$
$$
A.
k\left(1,\;1,\;1\right)^T+\left(1,\;0,\;-1\right)^T,\;k∈ R \quad B.k\left(1,\;0,\;-1\right)^T+\left(1,\;1,\;1\right)^T,\;k∈ R \quad C.k\left(1,\;0,\;1\right)^T+\left(1,\;0,\;-1\right)^T,\;k∈ R \quad D.k\left(1,\;0,\;-1\right)^T+\left(1,\;0,\;1\right)^T,\;k∈ R \quad E. \quad F. \quad G. \quad H.
$$
$$
\begin{array}{l}记\\ξ=\left(ξ_1+ξ_2\right)-\left(ξ_1+ξ_3\right)=\left(1,\;1,\;1\right)^T,\\η={\textstyle\frac12}\left(ξ_1+ξ_3\right)=\left(1,\;0,\;-1\right)^T,\\\mathrm{则易验证}η 是Ax=b\mathrm{的解向量},ξ\mathrm{是的}Ax=0\mathrm{解向量},由R\left(A\right)=2知ξ 是Ax=0\mathrm{的基础解系},\\\mathrm{从而}Ax=b\mathrm{的通解为}x=η+kξ,k\mathrm{为任意常数}.\end{array}
$$



$$
\mathrm{设四元非齐次线性方程组系数矩阵的秩为}3,\mathrm{已知}η_1,\;η_2,\;η_3\mathrm{是它的三个解向量},且η_1=\begin{pmatrix}4\\1\\0\\2\end{pmatrix},η_2+η_3=\begin{pmatrix}1\\0\\1\\2\end{pmatrix},\mathrm{则通解为}(\;).
$$
$$
A.
k\left(7,\;2,\;1,\;2\right)^T+\left(4,\;1,\;0,\;2\right)^T\;\;\left(k∈ R\right) \quad B.k\left(4,\;1,\;0,\;2\right)^T\;+\left(7,\;2,\;1,\;2\right)^T\;\left(k∈ R\right) \quad C.k\left(4,\;1,\;0,\;2\right)^T\;+\left(-7,\;-2,\;1,\;-2\right)^T\;\left(k∈ R\right) \quad D.k\left(-7,\;-2,\;1,\;-2\right)^T\;+\left(4,\;1,\;0,\;2\right)\;^T\left(k∈ R\right) \quad E. \quad F. \quad G. \quad H.
$$
$$
\begin{array}{l}\mathrm{因为}η_1,\;η_2,\;η_3是Ax=b\mathrm{的解},\mathrm{所以}\\Aη_1=b,\;Aη_2=b,\;Aη_3=b\\A\left(η_2+η_3-2η_1\right)=Aη_2+Aη_3-A\left(2η_1\right)=b+b-2b=0\\\mathrm{所以}η_2+η_3-2η_1=\left(-7,\;-2,\;1,\;-2\right)^T\mathrm{是对应齐次方程的解}.又n-r=4-3=1,\mathrm{故可取}η_2+η_3-2η_1\mathrm{为基础解系};\mathrm{所以原方程的通解为}\\k\left(-7,\;-2,\;1,\;-2\right)^T\;+\left(4,\;1,\;0,\;2\right)\;\left(k∈ R\right)\;.\end{array}
$$



$$
\begin{array}{l}\mathrm{已知三元非齐次方程组}Ax=b\mathrm{的三个特解分别为}\;\;\\μ_1=\left(2,\;1,\;0\right)^T,μ_2=\left(1,\;1,\;0\right)^T,μ_3=\left(1,\;0,\;1\right)^T\;\;\;\\且R\left(A\right)=1,\mathrm{则方程组}Ax=b\mathrm{的全部解为}(\;).\end{array}
$$
$$
A.
\begin{pmatrix}2\\1\\0\end{pmatrix}+c_1\begin{pmatrix}1\\0\\0\end{pmatrix}+c_2\begin{pmatrix}1\\1\\-1\end{pmatrix} \quad B.c_1\begin{pmatrix}2\\1\\0\end{pmatrix}+c_2\begin{pmatrix}1\\1\\0\end{pmatrix}+\begin{pmatrix}1\\1\\-1\end{pmatrix} \quad C.\begin{pmatrix}0\\1\\-1\end{pmatrix}+c_1\begin{pmatrix}1\\0\\0\end{pmatrix}+c_2\begin{pmatrix}1\\1\\-1\end{pmatrix} \quad D.c_1\begin{pmatrix}0\\1\\-1\end{pmatrix}+c_2\begin{pmatrix}1\\0\\0\end{pmatrix}+\begin{pmatrix}1\\1\\-1\end{pmatrix} \quad E. \quad F. \quad G. \quad H.
$$
$$
\begin{array}{l}\mathrm{由已知条件}\\ν_1=μ_1-μ_2=\left(1,\;0,\;0\right)^T,ν_2=μ_1-μ_3=\left(1,\;1,\;-1\right)^T,\;\;\mathrm{必是}Ax=b\mathrm{导出组}Ax=0\mathrm{的解},\\又R\left(A\right)=1,\mathrm{所以}Ax=0\mathrm{的基础解系含有两个线性无关的解向量不难验证},ν_1和ν_2\mathrm{是线性无关的},\mathrm{所以原方程组全部解为}\\μ_1+c_1ν_1+c_2ν_2=\begin{pmatrix}2\\1\\0\end{pmatrix}+c_1\begin{pmatrix}1\\0\\0\end{pmatrix}+c_2\begin{pmatrix}1\\1\\-1\end{pmatrix}.\end{array}
$$



$$
\begin{array}{l}\mathrm{设四元非齐次线性方程组}Ax=b\mathrm{的系数矩阵}A\mathrm{的秩为}3,\mathrm{已知}η_1,\;η_2,\;η_3,\;η_4\mathrm{是它的四个解向量},且\;\\η_1=\begin{pmatrix}0\\1\\0\\1\end{pmatrix},η_2+η_3+η_4=\begin{pmatrix}1\\5\\2\\8\end{pmatrix},\mathrm{则其通解为}(\;).\end{array}
$$
$$
A.
k\begin{pmatrix}1\\2\\2\\5\end{pmatrix}+\begin{pmatrix}0\\1\\0\\1\end{pmatrix},k∈ R \quad B.k\begin{pmatrix}0\\1\\0\\1\end{pmatrix}+\begin{pmatrix}1\\2\\2\\5\end{pmatrix},k∈ R \quad C.k\begin{pmatrix}0\\1\\0\\1\end{pmatrix}+\begin{pmatrix}\textstyle\frac13\\\textstyle\frac53\\\textstyle\frac23\\\textstyle\frac83\end{pmatrix},k∈ R \quad D.k_1\begin{pmatrix}0\\1\\0\\1\end{pmatrix}+k_2\begin{pmatrix}1\\2\\2\\5\end{pmatrix},k_1,k_2∈ R \quad E. \quad F. \quad G. \quad H.
$$
$$
\begin{array}{l}\mathrm{对应齐次方程基础解系所含解向量个数为}n-r=4-3=1,又\\A\left(η_4+η_2+η_3-3η_1\right)=Aη_4+Aη_2+Aη_3-3Aη_1=b+b+b-3b=0\\令\;a=\left(η_2+η_3+η_4-3η_1\right)=\begin{pmatrix}1\\5\\2\\8\end{pmatrix}-3\begin{pmatrix}0\\1\\0\\1\end{pmatrix}=\begin{pmatrix}1\\2\\2\\5\end{pmatrix}\;,\;\\\;故a\mathrm{为对应齐次方程组的非零解},\mathrm{即可作为基础解系}.\mathrm{所以通解为}\;\;\\x=ka+η_1=k\begin{pmatrix}1\\2\\2\\5\end{pmatrix}+\begin{pmatrix}0\\1\\0\\1\end{pmatrix},\left(k∈ R\right).\end{array}
$$



$$
\begin{array}{l}\mathrm{设有四元线性方程组}Ax=b,\mathrm{系数矩阵}A\mathrm{的秩为}3,\mathrm{又已知}β_1,\;β_2,\;β_3\mathrm{为三个解},且\\β_1=\left(2,\;0,\;0,\;2\right)^T,β_2+β_3=\left(0,\;2,\;2,\;0\right)^T\;\;,则Ax=b\mathrm{的通解为}(\;).\end{array}
$$
$$
A.
\begin{pmatrix}2\\0\\0\\2\end{pmatrix}+k\begin{pmatrix}2\\-1\\-1\\2\end{pmatrix},\;k∈ R \quad B.k\begin{pmatrix}2\\0\\0\\2\end{pmatrix}+\begin{pmatrix}2\\-1\\-1\\2\end{pmatrix},\;k∈ R \quad C.k\begin{pmatrix}2\\0\\0\\2\end{pmatrix}+\begin{pmatrix}0\\1\\1\\0\end{pmatrix},\;k∈ R \quad D.\begin{pmatrix}0\\2\\2\\0\end{pmatrix}+k\begin{pmatrix}2\\-1\\-1\\2\end{pmatrix},\;k∈ R \quad E. \quad F. \quad G. \quad H.
$$
$$
\begin{array}{l}\mathrm{因为}{\textstyle\frac12}\left(β_2+β_3\right)Ax=b\mathrm{的解},故β_1-{\textstyle\frac12}{\textstyle\left(β_2+β_3\right)}为Ax=0\mathrm{的解},\mathrm{又秩}\left(A\right)=3,且\\a=β_1-{\textstyle\frac12}{\textstyle\left(β_2+β_3\right)}=\left(2,\;0,\;0,\;2\right)^T-\left(0,\;1,\;1,\;0\right)^T=\left(2,\;-1,\;-1,\;2\right)^T\neq0,\;\;\\\mathrm{所以}a是Ax=0\mathrm{的基础解系},故Ax=b\mathrm{的通解为}x=β_1+ka=\;\begin{pmatrix}2\\0\\0\\2\end{pmatrix}+k\begin{pmatrix}2\\-1\\-1\\2\end{pmatrix},\mathrm{其中}k\mathrm{为任意实数}.\end{array}
$$



$$
\begin{array}{l}\mathrm{设四元非齐次线性方程组}Ax=b\mathrm{的系数矩阵秩为}2,\mathrm{已知}η_1,\;η_2,\;η_3,\;η_4\mathrm{为它的四个解向量},且\\η_1=\begin{pmatrix}1\\1\\0\\1\end{pmatrix},η_2+η_3=\begin{pmatrix}1\\2\\3\\4\end{pmatrix},η_4=\begin{pmatrix}1\\0\\1\\1\end{pmatrix},\;\;\mathrm{则其通解为}(\;).\end{array}
$$
$$
A.
k_1\begin{pmatrix}0\\1\\-1\\0\end{pmatrix}+k_2\begin{pmatrix}1\\0\\-3\\-2\end{pmatrix}+\begin{pmatrix}1\\1\\0\\1\end{pmatrix},k_1,\;k_2\mathrm{为任意常数}. \quad B.k\begin{pmatrix}0\\1\\-1\\0\end{pmatrix}+\begin{pmatrix}1\\1\\0\\1\end{pmatrix},k_1,\;k_2\mathrm{为任意常数}. \quad C.k_1\begin{pmatrix}1\\0\\1\\1\end{pmatrix}+k_2\begin{pmatrix}1\\1\\0\\1\end{pmatrix}+\begin{pmatrix}1\\2\\3\\4\end{pmatrix},k_1,\;k_2\mathrm{为任意常数}. \quad D.k\begin{pmatrix}0\\1\\-1\\0\end{pmatrix}+\begin{pmatrix}\textstyle\frac12\\1\\\textstyle\frac32\\2\end{pmatrix}+\begin{pmatrix}1\\1\\0\\1\end{pmatrix},k_1,\;k_2\mathrm{为任意常数}. \quad E. \quad F. \quad G. \quad H.
$$
$$
\begin{array}{l}\mathrm{齐次方程组两个解为}a_1=η_1-η_4=\begin{pmatrix}0\\1\\-1\\0\end{pmatrix}.\\\mathrm{又因}A\left(2η_1-η_2-η_3\right)\;=2Aη_1-Aη_2-Aη_3\;=0得\\a_2=2η_1-η_2-η_3=\begin{pmatrix}2\\2\\0\\2\end{pmatrix}-\begin{pmatrix}1\\2\\3\\4\end{pmatrix}=\begin{pmatrix}1\\0\\-3\\-2\end{pmatrix}.\;\;\mathrm{又因}a_1,\;a_2\mathrm{线性无关},\mathrm{故为齐次方程组的基础解系}.\;\\\;\mathrm{得通解为}X=k_1a_1+k_2a_2+η_1=k_1\begin{pmatrix}0\\1\\-1\\0\end{pmatrix}+k_2\begin{pmatrix}1\\0\\-3\\-2\end{pmatrix}+\begin{pmatrix}1\\1\\0\\1\end{pmatrix}.\\\end{array}
$$



$$
\begin{array}{l}设A是m×4\mathrm{矩阵},A\mathrm{的列向量组的秩为}2,b是m×1\mathrm{的非零矩阵},x=\left(x_1,\;...,\;x_4\right)^T,若a_1,\;a_2,\;a_3是Ax=b\mathrm{的解向量},\\\mathrm{且设}a_1=\begin{pmatrix}1\\1\\1\\1\end{pmatrix},a_1+a_2=\begin{pmatrix}1\\2\\3\\4\end{pmatrix}\;,a_2+a_3=\begin{pmatrix}1\\0\\4\\3\end{pmatrix}\;\mathrm{则方程组}Ax=b\mathrm{的通解为}(\;).\end{array}
$$
$$
A.
k_1\begin{pmatrix}-1\\0\\1\\2\end{pmatrix}+k_2\begin{pmatrix}0\\-2\\1\\-1\end{pmatrix}+\begin{pmatrix}1\\1\\1\\1\end{pmatrix},\left(k_1,\;k_2∈ R\right). \quad B.k_1\begin{pmatrix}-1\\0\\1\\2\end{pmatrix}+k_2\begin{pmatrix}0\\-2\\1\\-1\end{pmatrix}+\begin{pmatrix}1\\2\\3\\4\end{pmatrix},\left(k_1,\;k_2∈ R\right). \quad C.k_1\begin{pmatrix}1\\2\\3\\4\end{pmatrix}+k_2\begin{pmatrix}1\\0\\4\\3\end{pmatrix}+\begin{pmatrix}1\\1\\1\\1\end{pmatrix},\left(k_1,\;k_2∈ R\right). \quad D.k_1\begin{pmatrix}0\\1\\2\\3\end{pmatrix}+k_2\begin{pmatrix}0\\-2\\1\\-1\end{pmatrix}+\begin{pmatrix}\frac12\\0\\2\\\textstyle\frac32\end{pmatrix},\left(k_1,\;k_2∈ R\right). \quad E. \quad F. \quad G. \quad H.
$$
$$
\begin{array}{l}记\;b_1=a_2-a_1=\left(a_1+a_2\right)-2a_1=\left(-1,\;0,\;1,\;2\right)^T,\\b_2=a_3-a_1=\left(a_3+a_2\right)-\left(a_1+a_2\right)=\left(0,\;-2,\;1,\;-1\right)^T,\\则Ab_1=0,\;Ab_2=0,又b_1,\;b_2\mathrm{线性无关},且A\mathrm{的秩为}2,故b_1,\;b_2\mathrm{为对应齐次方程组的基础解系},故A\mathrm{的通解为}:\\X=\begin{pmatrix}x_1\\x_2\\x_3\\x_4\end{pmatrix}\;=k_1b_1+k_2b_2+a_1=k_1\begin{pmatrix}-1\\0\\1\\2\end{pmatrix}+k_2\begin{pmatrix}0\\-2\\1\\-1\end{pmatrix}+\begin{pmatrix}1\\1\\1\\1\end{pmatrix},\left(k_1,\;k_2∈ R\right).\end{array}
$$



$$
\begin{array}{l}\mathrm{设三元非齐次线性方程组的系数矩阵为}A,\mathrm{且行向量两两成比例},\\\mathrm{已知}a_1,\;a_2,\;a_3\mathrm{是三个解向量},\mathrm{其中}\\a_1+\;a_2=\left(1,\;2,\;3\right)^T,\;a_3+\;a_1=\left(1,\;0,\;-1\right)^T,\;a_2+\;a_3=\left(0,\;-1,\;1\right)^T\\\mathrm{则该线性方程组的通解为}(\;).\end{array}
$$
$$
A.
\begin{array}{l}x={\textstyle\frac12\left(1,\;2,\;3\right)^T}+k_1{\textstyle\left(1,\;3,\;2\right)}{\textstyle{}^T}+k_2{\textstyle\left(0,\;2,\;4\right)}{\textstyle{}^T}\left(k_1,\;k_2∈ R\right)\\\end{array} \quad B.x=\left(1,0,-1\right)^{T{\textstyle\;}}+k_1{\textstyle\left(1,\;3,\;2\right)}{\textstyle{}^T}+k_2{\textstyle\left(0,\;2,\;4\right)}{\textstyle{}^T}\left(k_1,\;k_2∈ R\right) \quad C.x={\textstyle\frac12\left(1,\;2,\;3\right)^T}+k_1{\textstyle\left(1,\;3,\;2\right)}{\textstyle{}^T},\;\;\;\left(k_1∈ R\right) \quad D.x={\textstyle\frac12\left(1,\;0,\;-1\right)^T}+k_1{\textstyle\left(0,\;2,\;4\right)}{\textstyle{}^T},\;\;\left(k_1∈ R\right) \quad E. \quad F. \quad G. \quad H.
$$
$$
\begin{array}{l}\mathrm{由于系数矩阵的秩为}1,\mathrm{故对应齐次线性方程组的基础解系中含}3-1=2\mathrm{个向量},\mathrm{可取为}\\a_1-\;a_3\;=\left(a_1+\;a_2\;\right)-\left(a_2+\;a_3\;\right)=\left(1,\;3,\;2\right){\textstyle{}^T},a_2-\;a_3\;=\left(a_1+\;a_2\;\right)-\left(a_3+\;a_1\;\right)=\left(0,\;2,\;4\right){\textstyle{}^T},\mathrm{为对应齐次方程的基础解系},\;\;\\\;\;\\\mathrm{因此所求方程组的通解为}A\\\end{array}
$$



$$
\begin{array}{l}设A\mathrm{是秩为}3的5×4\mathrm{矩阵},a_1,\;a_2,\;a_3\mathrm{是非齐次线性方程组}Ax=b\mathrm{的三个不同的解},若\;\\a_1+a_2+2a_3=\left(2,\;0,\;0,\;0\right)^T,\;\;3a_1+a_2=\left(2,\;4,\;6,\;8\right)^T,\\\mathrm{则下列不是方程组}Ax=b\mathrm{的通解的是}\left(\right)\end{array}
$$
$$
A.
\left({\textstyle\frac12},\;0,\;0,\;0\right)^T+k\left(0,\;2,\;3,\;4\right)^T\;\left(k∈ R\right) \quad B.\left({\textstyle\frac12,\;1,\;\frac32,\;2}\right)^T+k\left(0,\;2,\;3,\;4\right)^T\;\left(k∈ R\right) \quad C.\left({\textstyle2,\;0,\;0,\;0}\right)^T+k\left(0,\;2,\;3,\;4\right)^T\;\left(k∈ R\right) \quad D.\left({\textstyle\frac12,\;\frac12,\;\frac34,\;1}\right)^T+k\left(0,\;2,\;3,\;4\right)^T\;\left(k∈ R\right) \quad E. \quad F. \quad G. \quad H.
$$
$$
\begin{array}{l}\mathrm{由于秩}r\left(A\right)\;=3\mathrm{所以齐次方程组}Ax=0\mathrm{的解空间维数是}4-r\left(A\right)=1\mathrm{因为}\\a_1+a_2+2a_3-\left(3a_1+a_2\right)=2\left(a_3-a_1\right)=\left(0,\;-4,\;-6,\;-8\right)^T\\而a_3-a_1是Ax=0\mathrm{的解},\mathrm{即其基础解系}.\;由\\A\left(a_1+a_2+2a_3\right)\;=\;\;Aa_1+Aa_2+2Aa_3=\;4b\;\;\;\;\;\;\;\;\\知\;{\textstyle\frac14}\left(a_1+a_2+2a_3\right)\;\mathrm{是方程组}Ax=b\mathrm{的一个解},\;\mathrm{那么根据方程组的解的结构知其通解是}:\\\left({\textstyle\frac12,\;0,\;0,\;0}\right)^T+k\left(0,\;2,\;3,\;4\right)^T\;.\;\;\;\;\;\;\;\;\;\;\;\\\;\mathrm{同理可知}:\frac14\left(3a_1+a_2\right)\;,\frac18\left[a_1+a_2+2a_3+\left(3a_1+a_2\right)\right]\mathrm{都是方程组}Ax=b\mathrm{的一个解}.\;\;\\\mathrm{因此}\left({\textstyle\frac12,\;1,\;\frac32,\;2}\right)^T+k\left(0,\;2,\;3,\;4\right)^T\;\left(k∈ R\right),\left({\textstyle\frac12,\;\frac12,\;\frac34,\;1}\right)^T+k\left(0,\;2,\;3,\;4\right)^T\;\left(k∈ R\right)\mathrm{都是方程组的解},\\\mathrm{但选项中}\left({\textstyle2,\;0,\;0,\;0}\right)^T+k\left(0,\;2,\;3,\;4\right)^T\;\left(k∈ R\right)\mathrm{不是方程组的解}.\end{array}
$$



$$
\mathrm{方程}x_1-4x_2+2x_3-5x_4=6\mathrm{的通解为}\left(\right)\;
$$
$$
A.
k_1\left(4,\;1,\;0,\;0\right)^T+k_2\left(-2,\;0,\;1,\;0\right)^T+k_3\left(5,\;0,\;0,\;1\right)^T+\left(6,\;0,\;0,\;0\right)^T,\;\;k_1,k_2,k_3∈ R \quad B.k_1\left(4,\;1,\;0,\;0\right)^T+k_2\left(-2,\;0,\;1,\;0\right)^T+k_3\left(5,\;0,\;0,\;1\right)^T+\left(1,\;0,\;0,\;0\right)^T,\;\;k_1,k_2,k_3∈ R \quad C.k_1\left(0,\;1,\;0,\;0\right)^T+k_2\left(-2,\;0,\;1,\;0\right)^T+k_3\left(5,\;0,\;0,\;1\right)^T+\left(6,\;0,\;0,\;0\right)^T,\;\;k_1,k_2,k_3∈ R \quad D.k_1\left(4,\;1,\;0,\;0\right)^T+k_2\left(2,\;0,\;1,\;0\right)^T+k_3\left(5,\;0,\;0,\;1\right)^T+\left(6,\;0,\;0,\;0\right)^T,\;\;k_1,k_2,k_3∈ R \quad E. \quad F. \quad G. \quad H.
$$
$$
\begin{array}{l}\mathrm{其增广矩阵}B=\left(1,\;-4,\;2,\;-5,\;6\right),因R\left(A\right)=R\left(B\right)=1<\;4,\mathrm{故该方程有无穷多解},\;\\\;\mathrm{特解为}η^*=\left(6,\;0,\;0,\;0\right)^T,\mathrm{对应齐次线性方程组的一个基础解系含}n-r\left(A\right)=4-1=3\mathrm{个解向量}:\\ξ_1=\left(4,\;1,\;0,\;0\right)^T,\;\;ξ_2=\left(-2,\;0,\;1,\;0\right)^T,\;\;ξ_3=\left(5,\;0,\;1,\;0\right)^T,\mathrm{故所给方程的通解为}\\k_1ξ_{1\;}+\;k_2ξ_2\;+\;k_3ξ_3\;+\;η^*\;\;,\mathrm{其中}k_1,\;k_2,\;\;k_3\mathrm{为常数}.\end{array}
$$



$$
A=(α_1,α_2,⋯,α_n)是n\mathrm{阶矩阵},\mathrm{对于齐次线性方程组}Ax=O,如r(A)=n-1,\mathrm{且代数余子式}A_{11}\neq0,则A^* x=O\mathrm{的通解是}(\;\;\;\;)\;
$$
$$
A.
\;k_1α_1+\;k_2α_2+⋯+\;k_{n-1}α_{n-1}(k_1,k_2,⋯\;k_{n-1}\;∈ R) \quad B.\;k_1α_2+\;k_2α_3+⋯+\;k_{n-1}α_n,\;\;(k_1,k_2,⋯\;k_{n-1}\;∈ R) \quad C.\;kα_1,\;\;(k∈ R) \quad D.\;kα_n,\;\;(k∈ R) \quad E. \quad F. \quad G. \quad H.
$$
$$
\begin{array}{l}\;\;对A^* x=0,从r(A)=n-1知r(A^*)=1,\mathrm{那么}A^* x=0\mathrm{的解空间是}n-r(A^*)=n-1维,从A^* A=0知A\mathrm{的每一列都是}A^* x=0\mathrm{的解},\\\;\;\mathrm{由代数余子式}A_{11}\neq0,知n-1\mathrm{维向量}\\\;\;\;\;\;\;\;\;\;\;\;\;\;\;\;\;\;\;\;\;\;\;\;\;\;\;\;\;\;\;\;\;\;\;\;\;\;\;\;\;\;\;(a_{22},a_{32},⋯,a_{n2})^T,\;(a_{23},a_{33},⋯,a_{n3})^T,⋯,\;(a_{2n},a_{3n},⋯,a_{nn})^T\\\;\;\mathrm{线性无关},\mathrm{那么延伸为}n\mathrm{维向量}\\\;\;\;\;\;\;\;\;\;\;\;\;\;\;\;\;\;\;\;\;\;\;\;\;\;\;\;\;\;\;\;\;\;\;\;\;\;\;\;\;\;\;(a_{12},a_{22},⋯,a_{n2})^T,\;(a_{13},a_{23},⋯,a_{n3})^T,⋯,\;(a_{1n},a_{2n},⋯,a_{nn})^T\\\;\;\mathrm{仍线性无关},\mathrm{即是}A^* x=0\mathrm{的基础解系},\mathrm{通解其线性组合}.\end{array}
$$



$$
设A为n\mathrm{阶方阵},A\mathrm{的行向量组的秩为}n-1,a_1,\;a_2\mathrm{是非齐次线性方程组}Ax=b\mathrm{的两个不同的解向量},则Ax=b\mathrm{的通解为}\left(\right)
$$
$$
A.
a_1+ka_2\;,\;\;\;k∈ R \quad B.a_2+ka_1\;,\;k∈ R \quad C.a_1+k\left(a_1-a_2\right),\;\;\;k∈ R \quad D.a_1+k\left(a_1+a_2\right),\;\;\;k∈ R \quad E. \quad F. \quad G. \quad H.
$$
$$
\left(a_1-a_2\right)\mathrm{是齐次线性方程组}Ax=0\mathrm{的基础解系},\mathrm{由解的结构理论可知}
$$



$$
设ν_1,ν_2,...,ν_r是Ax=0\mathrm{的基础解系},a_1,a_2,...,a_n为A的n\mathrm{个列向量},若β=a_1+a_2+...+a_n,\mathrm{则方程组}Ax=β\mathrm{的通解为}(\;\;\;).\;
$$
$$
A.
x=k_1ν_1+k_2ν_2+...+k_rν_r+\begin{pmatrix}1\\1\\\vdots\\1\end{pmatrix},\;\;k_1,k_2,...,k_r∈ R: \quad B.x=k_1ν_1+k_2ν_2+...+k_rν_r+\begin{pmatrix}1\\2\\\vdots\\n\end{pmatrix}\;\;,\;\;k_1,k_2,...,k_r∈ R: \quad C.x=k_1ν_1+k_2ν_2+...+k_rν_r+\begin{pmatrix}n\\n-1\\\vdots\\1\end{pmatrix},\;\;\;k_1,k_2,...,k_r∈ R: \quad D.x=k_1ν_1+k_2ν_2+...+k_rν_r,\;\;\;\;k_1,k_2,...,k_r∈ R. \quad E. \quad F. \quad G. \quad H.
$$
$$
\begin{array}{l}由β=a_1+a_2+...+a_n\mathrm{可知}\\β=\left(a_1,a_2,...,a_n\right)\;\begin{pmatrix}1\\1\\\vdots\\1\end{pmatrix}\;=A\;\begin{pmatrix}1\\1\\\vdots\\1\end{pmatrix}\;,\;\;\\\mathrm{因此向量}\;\begin{pmatrix}1\\1\\\vdots\\1\end{pmatrix}\;\mathrm{是方程组}Ax=β\mathrm{的解向量},故Ax=β\mathrm{的通解为}\\x=k_1ν_1+k_2ν_2+...+k_rν_r+\begin{pmatrix}1\\1\\\vdots\\1\end{pmatrix}\;\;,k_1,k_2,...,k_r∈ R.\end{array}
$$



$$
\mathrm{方程组}x_1+x_2+...+x_n=1\mathrm{的通解为}\;\left(\right)
$$
$$
A.
k_1\left(-1,\;1,\;0,\;...,\;0\right)^T+k_2\left(-1,\;0,\;1,\;...,\;0\right)^T+...+k_{n-1}\left(-1,\;0,\;0,\;...,\;1\right)^T+\left(1,\;0,\;...,\;0\right)^T,\;\;k_1,k_2,...,k_{n-1}∈ R \quad B.k_1\left(0,\;1,\;0,\;...,\;0\right)^T+k_2\left(0,\;0,\;1,\;...,\;0\right)^T+...+k_{n-1}\left(0,\;0,\;0,\;...,\;1\right)^T+\left(1,\;0,\;...,\;0\right)^T,\;\;k_1,k_2,...,k_{n-1}∈ R \quad C.k_1\left(0,\;1,\;0,\;...,\;0\right)^T+k_2\left(0,\;0,\;1,\;...,\;0\right)^T+...+k_{n-1}\left(0,\;0,\;0,\;...,\;1\right)^T+\left(0,\;0,\;...,\;0\right)^T,\;\;k_1,k_2,...,k_{n-1}∈ R \quad D.k_1\left(-1,\;1,\;0,\;...,\;0\right)^T+k_2\left(-1,\;0,\;1,\;...,\;0\right)^T+...+k_{n-1}\left(-1,\;0,\;0,\;...,\;1\right)^T+\left(0,\;0,\;...,\;0\right)^T,\;\;k_1,k_2,...,k_{n-1}∈ R \quad E. \quad F. \quad G. \quad H.
$$
$$
\mathrm{齐次的基础解系}\left(-1,\;1,\;0,\;...,\;0\right)^T,\;\left(-1,\;0,\;1,\;...,\;0\right)^T,\;...,\;\left(-1,\;0,\;0,\;...,\;1\right)^T,\;\mathrm{非齐次的一个特解}\left(1,\;0,\;...,\;0\right)^T
$$



$$
\mathrm{设方程组}\left\{\begin{array}{c}ax+ay+\left(a+1\right)z=a\\ax+ay+\left(a-1\right)z=a\\\left(a+1\right)x+ay+\left(2a+3\right)z=1\end{array}\right.\mathrm{有两个不同的解},\mathrm{则其所有解为}(\;\;\;\;\;)
$$
$$
A.
k\begin{pmatrix}0\\1\\0\end{pmatrix}+\begin{pmatrix}1\\0\\0\end{pmatrix} \quad B.k\begin{pmatrix}1\\0\\1\end{pmatrix}+\begin{pmatrix}1\\0\\0\end{pmatrix} \quad C.k\begin{pmatrix}1\\1\\0\end{pmatrix}+\begin{pmatrix}1\\0\\0\end{pmatrix} \quad D.k\begin{pmatrix}1\\1\\1\end{pmatrix}+\begin{pmatrix}1\\0\\0\end{pmatrix} \quad E. \quad F. \quad G. \quad H.
$$
$$
\begin{array}{l}\left|A\right|=\begin{vmatrix}a&a&a+1\\a&a&a-1\\a+1&a&2a+3\end{vmatrix}=-2a.\;\;\\当a\neq0时,R\left(A\right)=R\left(\widetilde A\right)=3,\mathrm{此时方程组有唯一解}\left\{\begin{array}{c}x=1-a\\y=a\\z=0\end{array}\right.;\;\;\\当a=0时,\;\widetilde A\;=\begin{pmatrix}0&0&1&0\\0&0&-1&0\\1&0&3&1\end{pmatrix}\rightarrow\begin{pmatrix}1&0&0&1\\0&0&1&0\\0&0&0&0\end{pmatrix},\;\;\\\mathrm{此时原方程组有无穷多组解}k\begin{pmatrix}0\\1\\0\end{pmatrix}+\begin{pmatrix}1\\0\\0\end{pmatrix}:k\mathrm{取任意值}.\end{array}
$$



$$
设A=\begin{pmatrix}1&-1&-1\\-1&1&1\\0&-4&-2\end{pmatrix},\;ξ_1=\begin{pmatrix}-1\\1\\-2\end{pmatrix}\mathrm{则满足}Aξ_2=ξ_1\;\mathrm{的所有向量为}(\;\;\;).
$$
$$
A.
ξ_2=\begin{pmatrix}-{\textstyle\frac12}\\\textstyle\frac12\\0\end{pmatrix}+k\begin{pmatrix}\textstyle\frac12\\\textstyle-\frac12\\1\end{pmatrix},\;k\mathrm{为任意常数}. \quad B.ξ_2=\begin{pmatrix}-{\textstyle\frac12}\\\textstyle\frac12\\1\end{pmatrix}+k\begin{pmatrix}\textstyle\frac12\\\textstyle\frac12\\1\end{pmatrix},\;k\mathrm{为任意常数}. \quad C.ξ_2=\begin{pmatrix}-{\textstyle\frac12}\\\textstyle-\frac12\\0\end{pmatrix}+k\begin{pmatrix}\textstyle-\frac12\\\textstyle\frac12\\1\end{pmatrix},\;k\mathrm{为任意常数}. \quad D.ξ_2=\begin{pmatrix}-{\textstyle\frac12}\\\textstyle\frac12\\0\end{pmatrix}+k\begin{pmatrix}\textstyle\frac12\\\textstyle-\frac12\\-1\end{pmatrix},\;k\mathrm{为任意常数}. \quad E. \quad F. \quad G. \quad H.
$$
$$
\begin{array}{l}\mathrm{解方程}Aξ_2=ξ_1,\\\begin{pmatrix}1&-1&-1&-1\\-1&1&1&1\\0&-4&-2&-2\end{pmatrix}\rightarrow\begin{pmatrix}1&-1&-1&-1\\0&1&\frac12&\frac12\\0&0&0&0\end{pmatrix}\rightarrow\begin{pmatrix}1&0&-\frac12&-\frac12\\0&1&\frac12&\frac12\\0&0&0&0\end{pmatrix}.\\故ξ_2=\begin{pmatrix}-{\textstyle\frac12}\\\textstyle\frac12\\0\end{pmatrix}+k\begin{pmatrix}\textstyle\frac12\\\textstyle-\frac12\\1\end{pmatrix},\;k\mathrm{为任意常数}.\end{array}
$$



$$
设A=\begin{pmatrix}1&1&-1\\0&-2&0\\-1&1&1\end{pmatrix},\;b=\begin{pmatrix}1\\1\\-2\end{pmatrix}\;\;\mathrm{方程组}Ax=b\mathrm{的通解为}\left(\right)
$$
$$
A.
x=k\begin{pmatrix}1\\0\\1\end{pmatrix}\;+\begin{pmatrix}\frac32\\\textstyle-\frac12\\0\end{pmatrix}\;\;\left(k∈ R\right) \quad B.x=k\begin{pmatrix}-1\\0\\1\end{pmatrix}\;+\begin{pmatrix}-\frac32\\\textstyle\frac12\\0\end{pmatrix}\;\;\left(k∈ R\right) \quad C.x=k\begin{pmatrix}1\\0\\0\end{pmatrix}\;+\begin{pmatrix}\frac32\\\textstyle\frac12\\0\end{pmatrix}\;\;\left(k∈ R\right) \quad D.x=k\begin{pmatrix}1\\0\\-1\end{pmatrix}\;+\begin{pmatrix}\frac32\\\textstyle-\frac12\\0\end{pmatrix}\;\;\left(k∈ R\right) \quad E. \quad F. \quad G. \quad H.
$$
$$
\begin{array}{l}\left(A\vert b\right)\rightarrow\begin{pmatrix}1&1&-1&1\\0&2&0&-1\\0&0&0&0\end{pmatrix}\rightarrow\begin{pmatrix}1&0&-1&\frac32\\0&1&0&-\frac12\\0&0&0&0\end{pmatrix}\;\\\mathrm{原方程组等价于方程组}\\\left\{\begin{array}{l}x_1-x_3={\textstyle\frac32}\\x_2=-{\textstyle\frac12}\end{array}\right.\;令x_3=1\mathrm{可求出对应齐次线性方程组的基础解系为}\\η=\left(1,\;0,\;1\right)\;^T\;\;\mathrm{求特解}:\\令x_3=0,\;得x_1={\textstyle\frac32},\;x_2=-{\textstyle\frac12}\\\mathrm{故所求通解为}x=k\begin{pmatrix}1\\0\\1\end{pmatrix}\;+\begin{pmatrix}\frac32\\\textstyle-\frac12\\0\end{pmatrix}\;\;\left(k\mathrm{为任意常数}\right).\end{array}
$$



$$
设A=\begin{pmatrix}1&a&0&0\\0&1&a&0\\0&0&1&a\\-1&0&0&1\end{pmatrix},b=\begin{pmatrix}1\\-1\\0\\0\end{pmatrix}.\;\mathrm{已知非齐次线性方程组}Ax=b\mathrm{有无穷多解},\mathrm{则参数}a和Ax=b\mathrm{的通解分别为}(\;\;\;).
$$
$$
A.
a=-1,\;k\begin{pmatrix}1\\1\\1\\1\end{pmatrix}+\begin{pmatrix}0\\-1\\0\\0\end{pmatrix} \quad B.a=-1,\;k\begin{pmatrix}1\\-1\\0\\0\end{pmatrix}+\begin{pmatrix}1\\1\\1\\1\end{pmatrix} \quad C.a=1,\;k\begin{pmatrix}1\\1\\1\\1\end{pmatrix}+\begin{pmatrix}0\\-1\\0\\0\end{pmatrix} \quad D.a=1,\;k\begin{pmatrix}0\\-1\\0\\0\end{pmatrix}+\begin{pmatrix}1\\1\\1\\1\end{pmatrix} \quad E. \quad F. \quad G. \quad H.
$$
$$
\begin{array}{l}\begin{pmatrix}1&a&0&0&1\\0&1&a&0&-1\\0&0&1&a&0\\-1&0&0&1&0\end{pmatrix}\rightarrow\begin{pmatrix}-1&0&0&1&0\\0&1&a&0&-1\\0&0&1&a&0\\1&a&0&0&1\end{pmatrix}\rightarrow\begin{pmatrix}-1&0&0&1&0\\0&1&a&0&-1\\0&0&1&a&0\\0&0&-a^2&1&1+a\end{pmatrix}\rightarrow\begin{pmatrix}-1&0&0&1&0\\0&1&a&0&-1\\0&0&1&a&0\\0&0&0&1+a^3&1+a\end{pmatrix},\;\;\\\mathrm{可知要使得原线性方程组有无穷多解},\mathrm{则有}1+a^3=0及1+a=0,\mathrm{可知}a=-1.\;\;\\\mathrm{原线性方程组增广矩阵为},\mathrm{进一步化为行最简形得}\\\begin{pmatrix}1&0&0&-1&0\\0&1&0&-1&-1\\0&0&1&-1&0\\0&0&0&0&0\end{pmatrix},\mathrm{可知导出组的基础解系为}\begin{pmatrix}1\\1\\1\\1\end{pmatrix},\mathrm{非齐次方程的特解为}\begin{pmatrix}0\\-1\\0\\0\end{pmatrix},\mathrm{故其通解为}\;k\begin{pmatrix}1\\1\\1\\1\end{pmatrix}+\begin{pmatrix}0\\-1\\0\\0\end{pmatrix}\end{array}
$$



$$
设A=\begin{pmatrix}1&-1&0&0\\0&1&-1&0\\0&0&1&-1\\-1&0&0&1\end{pmatrix},b=\begin{pmatrix}1\\-1\\0\\0\end{pmatrix}.\;则\;Ax=b\mathrm{的通解为}(\;\;\;).
$$
$$
A.
\;k\begin{pmatrix}1\\1\\1\\1\end{pmatrix}+\begin{pmatrix}0\\-1\\0\\0\end{pmatrix},k∈ R \quad B.\;k\begin{pmatrix}0\\-1\\0\\0\end{pmatrix}+\begin{pmatrix}1\\1\\1\\1\end{pmatrix},k∈ R \quad C.\;k\begin{pmatrix}1\\1\\1\\1\end{pmatrix}+\begin{pmatrix}0\\1\\0\\0\end{pmatrix},k∈ R \quad D.k\begin{pmatrix}0\\1\\0\\0\end{pmatrix}+\begin{pmatrix}1\\1\\1\\1\end{pmatrix},k∈ R \quad E. \quad F. \quad G. \quad H.
$$
$$
\begin{array}{l}\mathrm{增广矩阵为化为行最简形得}\begin{pmatrix}1&0&0&-1&0\\0&1&0&-1&-1\\0&0&1&-1&0\\0&0&0&0&0\end{pmatrix},\mathrm{可知导出组的基础解系为}\begin{pmatrix}1\\1\\1\\1\end{pmatrix},\\\mathrm{非齐次方程的特解为}\begin{pmatrix}0\\-1\\0\\0\end{pmatrix},\mathrm{故其通解为}k\begin{pmatrix}1\\1\\1\\1\end{pmatrix}+\begin{pmatrix}0\\-1\\0\\0\end{pmatrix}\end{array}
$$



$$
设4\mathrm{元非齐次线性方程组},Ax=b\mathrm{有三个线性无关的解}:η_1,η_2,η_3,且r\left(A\right)=2,\mathrm{则下列不是方程组的通解的是}(\;\;\;\;).\;
$$
$$
A.
x=c_1\left(η_1-η_2\right)+c_2\left(η_2-η_3\right)+η_3\;,\;c_1,c_2∈ R \quad B.x=c_1\left(η_1-η_2\right)+c_2\left(η_1-η_3\right)+η_3,\;\;c_1,c_2∈ R \quad C.x=c_1\left(η_1-η_3\right)-c_2\left(η_2-η_3\right)+η_3,\;c_1,c_2∈ R \quad D.x=c_1\left(η_1-η_3\right)+c_2\left(η_3-η_1\right)+η_3,\;c_1,c_2∈ R \quad E. \quad F. \quad G. \quad H.
$$
$$
η_1-η_3和η_3-η_1\mathrm{线性相关},故x=c_1\left(η_1-η_3\right)+c_2\left(η_3-η_1\right)+η_3\mathrm{不是通解}
$$



$$
\mathrm{若非齐次线性方程组增广矩阵经初等行变换化为}\begin{pmatrix}0&1&0&-1&1\\0&0&1&4&3\end{pmatrix},\mathrm{那么该方程组的通解是}\_\_\_\_\_\_\_\_\_\_\_.\;
$$
$$
A.
k_1\begin{pmatrix}1\\0\\0\\0\end{pmatrix}+k_2\begin{pmatrix}0\\1\\-4\\1\end{pmatrix}+\begin{pmatrix}0\\1\\3\\0\end{pmatrix},\;(k_1,\;k_2∈ R) \quad B.k_1\begin{pmatrix}1\\0\\0\\0\end{pmatrix}+k_2\begin{pmatrix}0\\1\\4\\1\end{pmatrix}+\begin{pmatrix}0\\1\\-3\\0\end{pmatrix},\;(k_1,\;k_2∈ R) \quad C.k_1\begin{pmatrix}0\\0\\0\\1\end{pmatrix}+k_2\begin{pmatrix}1\\1\\-4\\0\end{pmatrix}+\begin{pmatrix}0\\1\\3\\0\end{pmatrix},\;(k_1,\;k_2∈ R) \quad D.k_1\begin{pmatrix}1\\0\\0\\0\end{pmatrix}+k_2\begin{pmatrix}0\\1\\-4\\1\end{pmatrix}+\begin{pmatrix}0\\-1\\3\\0\end{pmatrix},\;\;(k_1,\;k_2∈ R) \quad E. \quad F. \quad G. \quad H.
$$
$$
\begin{array}{l}\mathrm{根据条件可得与原方程组同解的方程组}\\\left\{\begin{array}{l}x_2-x_4=1\\x_3+4x_4=3\end{array}\right.,即\left\{\begin{array}{l}x_2=1+x_4\\x_3=3-4x_4\end{array}\right.,\;\;令x_1=k_1,x_4=k_2则\;\\\left\{\begin{array}{c}x_1=k_1\\x_2=1+k_2\\x_3=3-4k_2\\x_4=k_2\end{array}\right.⇒\begin{pmatrix}x_1\\x_2\\x_3\\x_4\end{pmatrix}\;=k_1\begin{pmatrix}1\\0\\0\\0\end{pmatrix}+k_2\begin{pmatrix}0\\1\\-4\\1\end{pmatrix}\;+\begin{pmatrix}0\\1\\3\\0\end{pmatrix}\;.(k_1,\;k_2∈ R)\end{array}
$$



$$
设η_1,η_2,...,η_s\mathrm{是非齐次线性方程组}Ax=b\mathrm{的解},若k_1η_1+k_2η_2+...+k_sη_s是Ax=b\mathrm{的解},则(\;\;\;\;)\;
$$
$$
A.
k_1+k_2+...+k_s=1 \quad B.k_1+k_2+...+k_s=s \quad C.k_1+k_2+...+k_s=0 \quad D.k_1+k_2+...+k_s\mathrm{的值不确定} \quad E. \quad F. \quad G. \quad H.
$$
$$
\begin{array}{l}若k_1η_1+k_2η_2+...+k_sη_s是Ax=b\mathrm{的解},则A\left(k_1η_1+k_2η_2+...+k_sη_s\right)=b,即\\A\left(k_1η_1+k_2η_2+...+k_sη_s\right)=k_1Aη_1+k_2Aη_2+...+k_sAsη_s=k_1b+k_2b+...+k_sb=\left(k_1+k_2+...+k_s\right)b=b,\;\;\\故k_1+k_2+...+k_s=1.\end{array}
$$



$$
\mathrm{设线性方程组}\left\{\begin{array}{c}\left(λ+3\right)x_1+x_2+2x_3=λ\\λ x_1+\left(λ-1\right)x_2+x_3=λ\\3\left(λ+1\right)x_1+λ x_2+\left(λ+3\right)x_3=3\end{array}\right.\;\;\mathrm{有无穷多解},\mathrm{则通解为}(\;\;\;\;\;)
$$
$$
A.
\begin{pmatrix}x_1\\x_2\\x_3\end{pmatrix}=c\begin{pmatrix}-1\\2\\1\end{pmatrix}+\begin{pmatrix}1\\-3\\0\end{pmatrix}\;\left(c∈ R\right) \quad B.\begin{pmatrix}x_1\\x_2\\x_3\end{pmatrix}=c\begin{pmatrix}0\\-1\\1\end{pmatrix}+\begin{pmatrix}1\\-3\\0\end{pmatrix}\;\left(c∈ R\right) \quad C.\begin{pmatrix}x_1\\x_2\\x_3\end{pmatrix}=c\begin{pmatrix}-1\\2\\1\end{pmatrix}+\begin{pmatrix}1\\-3\\1\end{pmatrix}\;\left(c∈ R\right) \quad D.\begin{pmatrix}x_1\\x_2\\x_3\end{pmatrix}=c\begin{pmatrix}-1\\-2\\1\end{pmatrix}+\begin{pmatrix}1\\-3\\0\end{pmatrix}\;\left(c∈ R\right) \quad E. \quad F. \quad G. \quad H.
$$
$$
\begin{array}{l}\mathrm{方程组的系数行列式为}\\\begin{vmatrix}λ+3&1&2\\λ&λ-1&1\\3\left(λ+1\right)&λ&λ+3\end{vmatrix}=λ^2\left(λ-1\right)\\\mathrm{则当}λ\neq0且λ\neq1时,\mathrm{方程组有唯一解};\;\\当λ=0时,\mathrm{方程组的增广矩阵作初等行变换}:\\\widetilde A=\begin{pmatrix}3&1&2&0\\0&-1&1&0\\3&0&3&3\end{pmatrix}\xrightarrow{r_3-r_1}\begin{pmatrix}3&1&2&0\\0&-1&1&0\\0&-1&1&3\end{pmatrix}\xrightarrow{r_3-r_2}\begin{pmatrix}3&1&2&0\\0&-1&1&0\\0&0&0&3\end{pmatrix}.\;\\因\;r\left(A\right)=2,\;r\left(\widetilde A\right)=3,\;\mathrm{所以方程组无解};\;\\当λ=1时,\mathrm{增广矩阵为}\\\widetilde A=\begin{pmatrix}4&1&2&0\\1&0&1&1\\6&1&4&3\end{pmatrix}\xrightarrow[{r_1↔ r_2}]{r_3-r_1}\begin{pmatrix}1&0&1&1\\4&1&2&1\\2&0&2&2\end{pmatrix}\xrightarrow[{r_3-2r_1}]{r_2-4r_1}\begin{pmatrix}1&0&1&1\\0&1&-2&-3\\0&0&0&0\end{pmatrix}\;.\;\\因r\left(A\right)\;=r\left(\widetilde A\right)=2<\;3\mathrm{所以方程组有无穷多个解},\mathrm{其通解为}\\\left\{\begin{array}{c}x_1=1-x_3\\x_2=-3+2x_3\\x_3=x_3\end{array}\right.,即\begin{pmatrix}x_1\\x_2\\x_3\end{pmatrix}=c\begin{pmatrix}-1\\2\\1\end{pmatrix}+\begin{pmatrix}1\\-3\\0\end{pmatrix}\;\left(c∈ R\right)\end{array}
$$



$$
\begin{array}{l}\mathrm{设方程组}\left\{\begin{array}{c}x_1+ax_2+x_3=3\\x_1+2ax_2+x_3=4\\x_1+x_2+bx_3=4\end{array}\right.,\mathrm{则下列说法正确的个数是}(\;\;\;\;)\;\\\;(1)当a\neq0且b\neq1时,\mathrm{方程组有唯一解};\;\\\;(2)当a=0时,\mathrm{无解};\;\\\;(3)当b=1时,\mathrm{无穷多解},\mathrm{通解为}\;x=\begin{pmatrix}2\\2\\0\end{pmatrix}+k\begin{pmatrix}-1\\0\\1\end{pmatrix}\;\left(c∈ R\right)\\(4)当b=1,a\neq{\textstyle\frac12}时,\mathrm{无解};\;\\\;(5)当b=1,a={\textstyle\frac12}时,\mathrm{无穷多解},\mathrm{通解为}\;x=\begin{pmatrix}2\\2\\0\end{pmatrix}+k\begin{pmatrix}-1\\0\\1\end{pmatrix}\;\left(c∈ R\right)\end{array}
$$
$$
A.
2 \quad B.3 \quad C.4 \quad D.5 \quad E. \quad F. \quad G. \quad H.
$$
$$
\begin{array}{l}D=\begin{vmatrix}1&a&1\\1&2a&1\\1&1&b\end{vmatrix}=a\left(b-1\right),\;\;\\(1)当a\neq0且b\neq1时,D\neq0,\mathrm{方程组有唯一解};\;\\(2)当a=0时,\;\;\\\overline A=\begin{pmatrix}1&0&1&3\\1&0&1&4\\1&1&b&4\end{pmatrix}\rightarrow\begin{pmatrix}1&0&1&3\\0&0&0&1\\0&1&b-1&1\end{pmatrix}\rightarrow\begin{pmatrix}1&0&1&3\\0&1&b-1&1\\0&0&0&1\end{pmatrix},\;R\left(A\right)=2\;,R\left(\overline A\right)=3,\mathrm{方程组无解};\\\;(3)当b=1时,\\\overline A=\begin{pmatrix}1&a&1&3\\1&2a&1&4\\1&1&b&4\end{pmatrix}\rightarrow\begin{pmatrix}1&0&1&2\\0&1&0&2\\0&0&0&1-2a\end{pmatrix}\;,\;\;\\当a\neq{\textstyle\frac12}时,\;R\left(A\right)=2,R\left(\overline A\right)=3,\mathrm{方程组无解};\;\;\\当a={\textstyle\frac12}时,\;R\left(\overline A\right)=R\left(A\right)=2,\mathrm{方程组有无穷多解};\\\mathrm{同解方程组为}\left\{\begin{array}{l}x_1+x_3=2\\x_2=2\end{array}\right.\;\;,\;\;\mathrm{易求得通解为}x=\begin{pmatrix}2\\2\\0\end{pmatrix}+k\begin{pmatrix}-1\\0\\1\end{pmatrix}\;\left(c∈ R\right).\end{array}
$$



$$
\begin{array}{l}设\left\{\begin{array}{c}x_1+x_2+kx_3=4\\-x_1+kx_2+x_3=k^2\\x_1-x_2+2x_3=-4\end{array}\right.,\mathrm{则正确结论的个数是}(\;\;\;\;)\;\;\\(1)当k=-1时,\mathrm{无解};\;\;\\(2)当k=4时,x=c\begin{pmatrix}-3\\-1\\1\end{pmatrix}+\begin{pmatrix}0\\4\\0\end{pmatrix},\mathrm{其中}c\mathrm{为任意常数};\;\;\\(3)当k=4时,\mathrm{无解};\;\;\\(4)当k=-1时,x=c\begin{pmatrix}-3\\-1\\1\end{pmatrix}+\begin{pmatrix}0\\4\\0\end{pmatrix},\mathrm{其中}c\mathrm{为任意常数}.\end{array}
$$
$$
A.
2 \quad B.1 \quad C.3 \quad D.4 \quad E. \quad F. \quad G. \quad H.
$$
$$
\begin{array}{l}\mathrm{方程组的系数行列式}\\\left|A\right|\;=\begin{vmatrix}1&1&k\\-1&k&1\\1&-1&2\end{vmatrix}=\left(1+k\right)\left(4-k\right).\;\;\\\mathrm{故当}k\neq-1且k\neq4时,\mathrm{有唯一解};\;\;\\当k=-1时,\\\widetilde A=\begin{pmatrix}1&1&-1&4\\-1&-1&1&1\\1&-1&2&-4\end{pmatrix}\rightarrow\begin{pmatrix}1&1&-1&4\\0&2&-3&8\\0&0&0&5\end{pmatrix},\;\;\mathrm{故方程组无解};\;\;\\当k=4时,\\\;\widetilde A=\begin{pmatrix}1&1&4&4\\-1&4&1&16\\1&-1&2&-4\end{pmatrix}\rightarrow\begin{pmatrix}1&0&3&0\\0&1&1&4\\0&0&0&0\end{pmatrix}.\;\;\mathrm{方程组有无穷多解},\mathrm{且解为}x==c\begin{pmatrix}-3\\-1\\1\end{pmatrix}+\begin{pmatrix}0\\4\\0\end{pmatrix},\mathrm{其中}c\mathrm{为任意常数}.\end{array}
$$



$$
\mathrm{关于}\left\{\begin{array}{c}λ x+y+z+t=1\\x+λ y+z+t=λ\\x+y+λ z+t=λ^2\\x+y+z+λ t=λ^3\end{array}\right.\mathrm{的解的情形},\mathrm{下列说法正确的是}(\;\;\;\;)\;
$$
$$
A.
当λ=-3时,\mathrm{有唯一解}; \quad B.当λ=1时,\mathrm{有无穷多解},x=k_1\begin{pmatrix}-1\\1\\0\\0\end{pmatrix}+k_2\begin{pmatrix}-1\\0\\1\\0\end{pmatrix}+k_3\begin{pmatrix}-1\\0\\0\\1\end{pmatrix}+\begin{pmatrix}1\\0\\0\\0\end{pmatrix},\mathrm{其中}k_1,k_2,k_3∈ R \quad C.当λ=1时,\mathrm{无解} \quad D.当λ=-3时,\mathrm{有无穷多解},x=k_1\begin{pmatrix}-1\\1\\0\\0\end{pmatrix}+k_2\begin{pmatrix}-1\\0\\1\\0\end{pmatrix}+k_3\begin{pmatrix}-1\\0\\0\\1\end{pmatrix}+\begin{pmatrix}1\\0\\0\\0\end{pmatrix},\mathrm{其中}k_1,k_2,k_3∈ R \quad E. \quad F. \quad G. \quad H.
$$
$$
\begin{array}{l}\begin{array}{l}\mathrm{方程组的系数行列式}\\\left|A\right|\;=\begin{vmatrix}λ&1&1&1\\1&λ&1&1\\1&1&λ&1\\1&1&1&λ\end{vmatrix}=\left(λ+3\right)\left(λ-1\right)^3.\;\;\\\mathrm{故当}λ\neq1且λ\neq-3时,\mathrm{有唯一解};\;\;\\当λ=1时,\\\overline A=\begin{vmatrix}1&1&1&1&1\\0&0&0&0&0\\0&0&0&0&0\\0&0&0&0&0\end{vmatrix},\;\;\mathrm{有无穷多解},\mathrm{易求得解为}\;\;x=k_1\begin{pmatrix}-1\\1\\0\\0\end{pmatrix}+k_2\begin{pmatrix}-1\\0\\1\\0\end{pmatrix}+k_3\begin{pmatrix}-1\\0\\0\\1\end{pmatrix}+\begin{pmatrix}1\\0\\0\\0\end{pmatrix},\mathrm{其中}k_1,k_2,k_3∈ R\end{array}\\\end{array}
$$



$$
设A={\left(a_{ij}\right)}_{3×3}\mathrm{是实正交阵},\mathrm{已知}a_{33}=-1,则Ax=\left(0,\;0,\;1\right)^T\mathrm{的解为}(\;\;\;).\;
$$
$$
A.
x=\left(0,\;0,\;-1\right)^T \quad B.x=\left(1,\;1,\;-1\right)^T \quad C.x=\left(a_{13},\;a_{23},\;-1\right)^T \quad D.x=\left(a_{31},\;a_{32},\;-1\right)^T \quad E. \quad F. \quad G. \quad H.
$$
$$
又Ax=\left(0,\;0,\;1\right)^T,\mathrm{所以}x=A^{-1}\left(0,\;0,\;1\right)^T=A^T\left(0,\;0,\;1\right)^T=\left(a_{31},\;a_{32},\;a_{33}\right)\;\;\;\mathrm{又因为}A\mathrm{是实正交矩阵},\mathrm{所以}\;\;x=\left(0,\;0,\;-1\right)^T.
$$



$$
设A={\left(a_{ij}\right)}_{3×3}\mathrm{是实正交阵},\mathrm{已知}a_{11}=1,则Ax=\left(1,\;0,\;0\right)^T\mathrm{的解为}(\;\;\;).\;
$$
$$
A.
x=\left(1,\;0,\;0\right)^T \quad B.x=\left(1,\;1,\;1\right)^T \quad C.x=\left(1,\;a_{21},\;a_{31}\right)^T \quad D.x=\left(1,\;a_{12},\;a_{13}\right)^T \quad E. \quad F. \quad G. \quad H.
$$
$$
又Ax=\left(1,\;0,\;0\right)^T,\mathrm{所以}x=A^{-1}\left(1,\;0,\;0\right)^T=A^T\left(1,\;0,\;0\right)^T=\left(a_{11},\;a_{12},\;a_{13}\right)\;\;\;\mathrm{又因为}A\mathrm{是实正交矩阵},\mathrm{所以}\;\;x=\left(1,\;0,\;0\right)^T.
$$



$$
\begin{array}{l}设4\mathrm{元非齐次线性方程组}Ax=b\mathrm{的系数矩阵}A\mathrm{的伴随矩阵的秩为}1,\\\mathrm{已知它的三个非零解向量}η_1,η_2,η_3\mathrm{不以任何次序构成等差数列},\mathrm{则方程组的通解为}\;\left(\right)\end{array}
$$
$$
A.
x=η_1+c\left[η_1-\frac12\left(η_2+η_3\right)\right]\left(c\mathrm{为任意常数}\right) \quad B.x=η_1+c\left[η_1+\frac12\left(η_2+η_3\right)\right]\left(c\mathrm{为任意常数}\right) \quad C.x=η_1+c\left[\left(η_2+η_3\right)-η_1\right]\left(c\mathrm{为任意常数}\right) \quad D.x=η_1+c\left[\left(η_2+η_3\right)+η_1\right]\left(c\mathrm{为任意常数}\right) \quad E. \quad F. \quad G. \quad H.
$$
$$
\begin{array}{l}\mathrm{依题意},\mathrm{方程组}Ax=b\mathrm{的导出组的基础解系含}4-3=1\mathrm{个向量},\mathrm{于是},\mathrm{导出组的任何一个非零解都可作为其基础解系}.\;\;\;\\\mathrm{由于}η_1,η_2,η_3\mathrm{不以任何次序构成等差数列}\;,则\;\\η_1-\frac12\left(η_2+η_3\right)\neq0\;,\;\;\mathrm{是导出组的非零解},\mathrm{可作为其基础解系}.\;\mathrm{故方程组}Ax=b\mathrm{的通解为}\;x=η_1+c\left[η_1-\frac12\left(η_2+η_3\right)\right]\left(c\mathrm{为任意常数}\right)\end{array}
$$



$$
设A=\begin{pmatrix}1&1&λ\\0&λ-1&0\\λ&1&1\end{pmatrix}\;,\;\;b=\begin{pmatrix}a\\1\\1\end{pmatrix}\mathrm{已知线性方程组}Ax=b\mathrm{存在两个不同的解}.则λ,\;a\mathrm{的值及方程组}Ax=b\mathrm{的通解分别为}\left(\right)
$$
$$
A.
λ=-1,\;a=-2,\;x=k\begin{pmatrix}1\\0\\1\end{pmatrix}+\begin{pmatrix}\textstyle-\frac32\\-{\textstyle\frac12}\\0\end{pmatrix}\left(k∈ R\right) \quad B.λ=-1,\;a=2,\;x=k\begin{pmatrix}1\\0\\1\end{pmatrix}+\begin{pmatrix}\textstyle\frac32\\-{\textstyle\frac12}\\0\end{pmatrix}\left(k∈ R\right) \quad C.λ=-1,\;a=-2,\;x=k\begin{pmatrix}1\\0\\1\end{pmatrix}+\begin{pmatrix}\textstyle\frac32\\\textstyle\frac12\\0\end{pmatrix}\left(k∈ R\right) \quad D.λ=1,\;a=-2,\;x=k\begin{pmatrix}1\\0\\1\end{pmatrix}+\begin{pmatrix}\textstyle\frac32\\\textstyle-\frac12\\0\end{pmatrix}\left(k∈ R\right) \quad E. \quad F. \quad G. \quad H.
$$
$$
\begin{array}{l}\left(1\right)\mathrm{因为线性方程组}Ax=b\mathrm{存在两个不同解},\mathrm{所以}r\left(A\right)<\;3即\left|A\right|=0\mathrm{解得}λ=-1或\;λ=1\\当λ=-1时,\\\overline A=\begin{pmatrix}1&1&-1&a\\0&-2&0&1\\-1&1&1&1\end{pmatrix}\rightarrow\begin{pmatrix}1&1&-1&a\\0&2&0&-1\\0&2&0&a+1\end{pmatrix}\rightarrow\begin{pmatrix}1&1&-1&a\\0&2&0&-1\\0&0&0&a+2\end{pmatrix},\\\mathrm{因为}r\left(A\right)=r\left(\overline A\right)<\;3\mathrm{所以}a=-2\;\\当λ=1时,\\\overline A=\begin{pmatrix}1&1&1&a\\0&0&0&1\\1&1&1&1\end{pmatrix}\rightarrow\begin{pmatrix}1&1&1&a\\0&0&0&1\\0&0&0&1-a\end{pmatrix}\rightarrow\begin{pmatrix}1&1&1&a\\0&0&0&1\\0&0&0&0\end{pmatrix}\\\mathrm{显然}r\left(A\right)\neq r\left(\overline A\right)\mathrm{所以}λ\neq1故\;λ=-1,\;a=-2\;\\\\\left(2\right)\widetilde A\rightarrow\begin{pmatrix}1&1&-1&-2\\0&2&0&-1\\0&0&0&0\end{pmatrix}\rightarrow\begin{pmatrix}1&0&-1&\textstyle-\frac32\\0&1&0&-\frac12\\0&0&0&0\end{pmatrix},\;\\由r\left(A\right)=r\left(\overline A\right)=2<\;3\mathrm{知方程组有无穷多解},\mathrm{且原方程组等价于方程组}\\\left\{\begin{array}{l}x_1-x_3=-{\textstyle\frac32}\\x_2=-{\textstyle\frac12}\end{array}\right.\;\;\;\\令x_3=1\mathrm{可求出对应齐次线性方程组的基础解系为}\;η=\left(1,\;0,\;1\right)\;^T\;\\\mathrm{求特解}:令x_3=0\mathrm{得故}x_1=-{\textstyle\frac32},\;x_2=-{\textstyle\frac12}\mathrm{所求通解为}\;\;\;\\x=k\begin{pmatrix}1\\0\\1\end{pmatrix}+\begin{pmatrix}\textstyle-\frac32\\-{\textstyle\frac12}\\0\end{pmatrix}\left(k\mathrm{为任意常数}\right)\end{array}
$$



$$
设A=\begin{pmatrix}1&-1&-1\\-1&1&1\\0&-4&-2\end{pmatrix},\;ξ_1=\begin{pmatrix}-1\\1\\-2\end{pmatrix},\mathrm{则满足}A^2\;ξ_3=ξ_1\mathrm{的所有向量为}(\;\;\;).
$$
$$
A.
ξ_3=\begin{pmatrix}-{\textstyle\frac12}\\0\\0\end{pmatrix}+k_2\begin{pmatrix}-{\textstyle1}\\1\\0\end{pmatrix}+k3\begin{pmatrix}0\\0\\1\end{pmatrix} \quad B.ξ_3=\begin{pmatrix}-{\textstyle\frac12}\\1\\0\end{pmatrix}+k_2\begin{pmatrix}\textstyle1\\1\\0\end{pmatrix}+k_3\begin{pmatrix}0\\1\\1\end{pmatrix} \quad C.ξ_3=\begin{pmatrix}-{\textstyle\frac12}\\1\\1\end{pmatrix}+k_2\begin{pmatrix}-{\textstyle1}\\1\\0\end{pmatrix}+k_3\begin{pmatrix}0\\0\\1\end{pmatrix} \quad D.ξ_3=\begin{pmatrix}\textstyle\frac12\\0\\0\end{pmatrix}+k_2\begin{pmatrix}\textstyle1\\1\\0\end{pmatrix}+k_3\begin{pmatrix}0\\0\\1\end{pmatrix} \quad E. \quad F. \quad G. \quad H.
$$
$$
\begin{array}{l}\mathrm{解方程}A^2\;ξ_3=ξ_1\;\;\\A^2=\begin{pmatrix}2&2&0\\-2&-2&0\\4&4&0\end{pmatrix},\;\;\begin{pmatrix}2&2&0&-1\\-2&-2&0&1\\4&4&0&-2\end{pmatrix}\rightarrow\;\begin{pmatrix}1&1&0&-{\textstyle\frac12}\\0&0&0&0\\0&0&0&0\end{pmatrix}\\故ξ_3=\begin{pmatrix}-{\textstyle\frac12}\\0\\0\end{pmatrix}+k_2\begin{pmatrix}-{\textstyle1}\\1\\0\end{pmatrix}+k_3\begin{pmatrix}0\\0\\1\end{pmatrix}\mathrm{其中}k_2,\;k_3\mathrm{为任意常数}.\end{array}
$$



$$
\mathrm{设矩阵}A=\left(a_1,\;a_2,\;a_3,\;a_4\right),\mathrm{其中}a_2,\;a_3,\;a_4\mathrm{线性无关},a_1+a_3=2a_2,\mathrm{则方程组}Ax=a_2\mathrm{的通解为}(\;\;\;).
$$
$$
A.
x=c\begin{pmatrix}1\\-2\\1\\0\end{pmatrix}+\begin{pmatrix}0\\1\\0\\0\end{pmatrix},\;c∈ R \quad B.x=c\begin{pmatrix}1\\2\\-1\\1\end{pmatrix}+\begin{pmatrix}1\\1\\1\\1\end{pmatrix},\;c∈ R \quad C.x=c\begin{pmatrix}1\\-2\\1\\0\end{pmatrix}+\begin{pmatrix}1\\2\\3\\4\end{pmatrix},\;c∈ R \quad D.x=c\begin{pmatrix}1\\-2\\1\\1\end{pmatrix}+\begin{pmatrix}1\\1\\1\\1\end{pmatrix},\;c∈ R \quad E. \quad F. \quad G. \quad H.
$$
$$
\begin{array}{l}\mathrm{显然},\mathrm{这是一个四元方程}.\mathrm{先决定系数矩阵}A\mathrm{的秩}.因a_2,\;a_3,\;a_4\mathrm{线性无关},故R\left(A\right)\geq\;3;\;\;\\\mathrm{又因}a_1\mathrm{能由}a_2,\;a_3\mathrm{线性表示}⇒ a_1,\;a_2,\;a_3,\mathrm{线性相关}⇒ a_1,\;a_2,\;a_3,\;a_4\;\;\mathrm{线性相关}(\mathrm{部分相关则整体相关})⇒ R\left(A\right)\leq\;3.\;\;\\\mathrm{综合上面两个不等式},有R\left(A\right)=\;3,\mathrm{从而原方程的基础解系所含向量个数为}4-3=1.\\\mathrm{进一步}a_1=2a_2-a_3\Leftrightarrow a_1-2a_2+a_3=0\Leftrightarrow x=\begin{pmatrix}1&-2&1\;\;0\end{pmatrix}^T,\;\;\mathrm{是导出组}Ax=0\mathrm{的解}\;\Leftrightarrow x=\begin{pmatrix}1&-2&1\;\;0\end{pmatrix}^T\;\mathrm{是导出组的基础解解};\;\;\\又a_2=0a_1+1a_2+0a_3+0a_4\Leftrightarrow x=\begin{pmatrix}0&1&0\;\;0\end{pmatrix}^T\mathrm{是方程}Ax=β\mathrm{的解}.\mathrm{于是由非齐次线性方程解的结构定理},\mathrm{原方程的通解为}\\x=c\begin{pmatrix}1\\-2\\1\\0\end{pmatrix}+\begin{pmatrix}0\\1\\0\\0\end{pmatrix},\;c∈ R\;\;.\end{array}
$$



$$
\begin{array}{l}设A\mathrm{是秩为}3的5×4\mathrm{矩阵},a_1,\;a_2,\;a_3\mathrm{是非齐次线性方程组}Ax=b\mathrm{的三个解},若\\a_1+a_2+a_3=\begin{pmatrix}2,&0,&0,&0\end{pmatrix}^T,\;\;2a_1+a_2=\begin{pmatrix}2,&4,&6,&8\end{pmatrix}^T\\\mathrm{则方程组}Ax=b\mathrm{的通解是}(\;\;\;).\end{array}
$$
$$
A.
\textstyle\frac13\left(a_1+a_2+a_3\right)+k\left(a_1-a_3\right),k∈ R \quad B.\textstyle\frac13\left(a_1+a_2+a_3\right)+k\left(a_1+a_2\right),k∈ R \quad C.\textstyle\left(a_1+a_2+a_3\right)+k\left(a_1-a_3\right),k∈ R \quad D.\textstyle\left(a_1+a_2+a_3\right)+k\left(a_1-a_2\right),\;k∈ R \quad E. \quad F. \quad G. \quad H.
$$
$$
\begin{array}{l}\mathrm{由于秩}R\left(A\right)\;=3,\;\mathrm{所以齐次方程组}Ax=0\mathrm{的解空间维数是}4-R\left(A\right)\;=1\;\mathrm{因为}\\a_1+a_2+a_3{\textstyle-}{\textstyle\left(2a_1+a_2\right)}{\textstyle=}{\textstyle a}{\textstyle{}_3}{\textstyle-}{\textstyle a}{\textstyle{}_1}{\textstyle=}\begin{pmatrix}0,&-4,&-6,&-8\end{pmatrix}^T\\而a{\textstyle{}_3}{\textstyle-}{\textstyle a}{\textstyle{}_1}是Ax=0\mathrm{的解},\mathrm{即其基础解系}.\;由\\A\left(a_1+a_2+a_3\right)=Aa_1+Aa_2+Aa_3=3b,\;\;知{\textstyle\frac13}\left(a_1+a_2+a_3\right)\mathrm{是方程组}Ax=b\mathrm{的一个解},\mathrm{那么根据方程组的解的结构知其通解}\end{array}
$$



$$
设A=\begin{pmatrix}1&1&λ\\0&λ-1&0\\λ&1&1\end{pmatrix}\;,\;b=\begin{pmatrix}1\\1\\-2\end{pmatrix}\;,\;\mathrm{已知线性方程组}Ax=b\mathrm{存在两个不同的解}.则λ\mathrm{的值及方程组}Ax=b\mathrm{的通解分别为}\left(\right)
$$
$$
A.
λ=-1,\;x=k\begin{pmatrix}1\\0\\1\end{pmatrix}+\begin{pmatrix}\textstyle\frac32\\-{\textstyle\frac12}\\0\end{pmatrix}\;\left(k∈ R\right) \quad B.λ=-1,\;x=k\begin{pmatrix}1\\0\\-1\end{pmatrix}+\begin{pmatrix}\textstyle\frac32\\\textstyle\frac12\\0\end{pmatrix}\;\left(k∈ R\right) \quad C.λ=1,\;x=k\begin{pmatrix}1\\0\\0\end{pmatrix}+\begin{pmatrix}\textstyle\frac32\\\textstyle\frac12\\0\end{pmatrix}\;\left(k∈ R\right) \quad D.λ=1,\;x=k\begin{pmatrix}1\\0\\0\end{pmatrix}+\begin{pmatrix}\textstyle\frac32\\\textstyle-\frac12\\0\end{pmatrix}\;\left(k∈ R\right) \quad E. \quad F. \quad G. \quad H.
$$
$$
\begin{array}{l}\left(1\right)\mathrm{因为线性方程组}Ax=b\mathrm{存在两个不同解},\mathrm{所以}R\left(A\right)<\;3即\left|A\right|=0,\;\mathrm{解得}λ=-1或\;λ=1\\当λ=-1时,\\\overline A\rightarrow\begin{pmatrix}1&1&-1&1\\0&2&0&-1\\0&2&0&-1\end{pmatrix}\rightarrow\begin{pmatrix}1&1&-1&1\\0&2&0&-1\\0&0&0&0\end{pmatrix},\\当λ=1时,\\\overline A\rightarrow\begin{pmatrix}1&1&1&1\\0&0&0&1\\0&0&0&0\end{pmatrix},\;\;\mathrm{显然}R\left(A\right)\neq R\left(\overline A\right),\;\mathrm{所以}λ\neq1故\;λ=-1\;\\\left(2\right)\;\widetilde A\rightarrow\begin{pmatrix}1&1&-1&1\\0&2&0&-1\\0&0&0&0\end{pmatrix}\rightarrow\begin{pmatrix}1&0&-1&\textstyle\frac32\\0&1&0&-{\textstyle\frac12}\\0&0&0&0\end{pmatrix},\\由R\left(A\right)=R\left(\widetilde A\right)=2<\;3\mathrm{知方程组有无穷多解},\mathrm{且原方程组等价于方程组}\\\left\{\begin{array}{l}x_1-x_3={\textstyle\frac32}\\x_2=-{\textstyle\frac12}\end{array}\right.,\\令x_3={\textstyle1},\;\mathrm{可求出对应齐次线性方程组的基础解系为}η=\left(1,\;0,\;1\right)^T\;\;\;\\\mathrm{求特解}:令x_3=0得x_1={\textstyle\frac32},\;x_2=-{\textstyle\frac12}\mathrm{故所求通解为}\;\;\;λ=-1,\;x=k\begin{pmatrix}1\\0\\1\end{pmatrix}+\begin{pmatrix}\textstyle\frac32\\-{\textstyle\frac12}\\0\end{pmatrix}\;\left(k\mathrm{为任意常数}\right)\;\end{array}
$$



$$
\mathrm{设矩阵}A=\left(a_1,\;a_2,a_3,a_4\right),A\mathrm{有一个非零的三阶子式},且a_1+a_3=2a_2,\mathrm{则方程组}Ax=a_1\mathrm{的通解为}(\;\;\;).
$$
$$
A.
x=c\begin{pmatrix}1\\-2\\1\\0\end{pmatrix}+\begin{pmatrix}1\\0\\0\\0\end{pmatrix},\;c∈ R \quad B.x=c\begin{pmatrix}1\\2\\-1\\1\end{pmatrix}+\begin{pmatrix}1\\1\\1\\1\end{pmatrix},\;c∈ R \quad C.x=c\begin{pmatrix}1\\-2\\1\\0\end{pmatrix}+\begin{pmatrix}1\\2\\3\\4\end{pmatrix},\;c∈ R \quad D.x=c\begin{pmatrix}1\\-2\\1\\1\end{pmatrix}+\begin{pmatrix}1\\1\\1\\1\end{pmatrix},\;c∈ R \quad E. \quad F. \quad G. \quad H.
$$
$$
\begin{array}{l}\begin{array}{l}\mathrm{解法一}\;\mathrm{显然},\mathrm{这是一个四元方程}.\mathrm{先决定系数矩阵}A\mathrm{的秩}.因A\mathrm{有一个非零的三阶子式},故R\left(A\right)\geq\;3;\;\;\\\mathrm{又因}a_1\mathrm{能由}a_2,\;a_3\mathrm{线性表示}⇒ a_1,\;a_2,\;a_3\mathrm{线性相关}⇒ a_1,\;a_2,\;a_3,\;a_4\;\;\mathrm{线性相关}(\mathrm{部分相关则整体相关})⇒ R\left(A\right)\leq\;3.\;\;\\\mathrm{综合上面两个不等式},有R\left(A\right)=\;3,\mathrm{从而原方程的基础解系所含向量个数为}4-3=1.\\\mathrm{进一步}a_1=2a_2-a_3\Leftrightarrow a_1-2a_2+a_3=0\Leftrightarrow x=\begin{pmatrix}1,&-2,&1,\;\;0\end{pmatrix}^T,\;\;\mathrm{是导出组}Ax=0\mathrm{的解}\;\Leftrightarrow x=\begin{pmatrix}1,&-2,&1,\;\;0\end{pmatrix}^T\;\mathrm{是导出组的基础解解};\;\;\\又a_1=1a_1+0a_2+0a_3+0a_4\Leftrightarrow x=\begin{pmatrix}1,&0,&0,\;\;0\end{pmatrix}^T\mathrm{是方程}Ax=β\mathrm{的解}.\mathrm{于是由非齐次线性方程解的结构定理},\mathrm{原方程的通解为}\\x=c\begin{pmatrix}1\\-2\\1\\0\end{pmatrix}+\begin{pmatrix}1\\0\\0\\0\end{pmatrix},\;c∈ R\;\;.\end{array}\\\\\end{array}
$$



$$
\mathrm{设矩阵}A=\left(a_1,\;a_2,a_3,a_4\right),\mathrm{其中}a_2,a_3,a_4\mathrm{线性无关},a_1=2a_2-a_3,\mathrm{向量}β=a_1+a_2+a_3+a_4,\;\mathrm{则方程组}Ax=β\mathrm{的通解为}(\;\;\;).
$$
$$
A.
x=k\begin{pmatrix}1\\-2\\1\\0\end{pmatrix}+\begin{pmatrix}1\\1\\1\\1\end{pmatrix},\;k∈ R \quad B.x=k\begin{pmatrix}1\\2\\-1\\1\end{pmatrix}+\begin{pmatrix}1\\1\\1\\1\end{pmatrix},\;k∈ R \quad C.x=k\begin{pmatrix}1\\-2\\1\\0\end{pmatrix}+\begin{pmatrix}1\\2\\3\\4\end{pmatrix},\;k∈ R \quad D.x=k\begin{pmatrix}1\\-2\\1\\1\end{pmatrix}+\begin{pmatrix}1\\1\\1\\1\end{pmatrix},\;k∈ R \quad E. \quad F. \quad G. \quad H.
$$
$$
\begin{array}{l}\begin{array}{l}\mathrm{解法一}\;\mathrm{显然},\mathrm{这是一个四元方程}.\mathrm{先决定系数矩阵}A\mathrm{的秩}.因a_2,\;a_3,\;a_4\mathrm{线性无关},故R\left(A\right)\geq\;3;\;\;\\\mathrm{又因}a_1\mathrm{能由}a_2,\;a_3\mathrm{线性表示}⇒ a_1,\;a_2,\;a_3\mathrm{线性相关}⇒ a_1,\;a_2,\;a_3,\;a_4\;\;\mathrm{线性相关}(\mathrm{部分相关则整体相关})⇒ R\left(A\right)\leq\;3.\;\;\\\mathrm{综合上面两个不等式},有R\left(A\right)=\;3,\mathrm{从而原方程的基础解系所含向量个数为}4-3=1.\\\mathrm{进一步}a_1=2a_2-a_3\Leftrightarrow a_1-2a_2+a_3=0\Leftrightarrow x=\begin{pmatrix}1,&-2,&1,\;0\end{pmatrix}^T,\;\;\mathrm{是导出组}Ax=0\mathrm{的解}\;\Leftrightarrow x=\begin{pmatrix}1,&-2,&1\;,\;0\end{pmatrix}^T\;\mathrm{是导出组的基础解解};\;\;\\又β=a_1+a_2+a_3+a_4\Leftrightarrow x=\begin{pmatrix}1&1&1\;\;1\end{pmatrix}^T\mathrm{是方程}Ax=β\mathrm{的解}.\mathrm{于是由非齐次线性方程解的结构定理},\mathrm{原方程的通解为}\\x=k\begin{pmatrix}1\\-2\\1\\0\end{pmatrix}+\begin{pmatrix}1\\1\\1\\1\end{pmatrix},\;k∈ R\;\;.\end{array}\\\\\\\mathrm{解法二}\;\mathrm{由题设条件},\mathrm{有等式}a_1+a_2+a_3+a_4=β=Ax=\left(a_1,\;a_2,\;a_3,\;a_4\right)\;\begin{pmatrix}x_1\\x_2\\x_3\\x_4\end{pmatrix}\;.\;\;\\将a_1=2a_2-a_3\mathrm{代入上式左右两端},\mathrm{整理后得到}\\\left(2x_1+x_2-3\right)\;a_2\;+\left(-x_1+x_3\right)\;a_3+\left(x_4-1\right)\;a_4=0.\;\;\\\mathrm{因向量组}a_2,\;a_3,\;a_4\mathrm{线性无关},\mathrm{所以}x\mathrm{的分量应满足}\left\{\begin{array}{c}2x_1+x_2-3=0\\-x_1+x_3=0\\x_4-1=0\end{array}\right.\;\;.\;\;\\\mathrm{上式是关于变元}x_i\left(1\leq\;i\leq\;4\right)\mathrm{的非齐次线性方程组}.\mathrm{对其增广矩阵进行初等行变换}:\\\begin{pmatrix}2&1&0&0&3\\-1&0&1&0&0\\0&0&0&1&1\end{pmatrix}\xrightarrow{r_2↔ r_1}\begin{pmatrix}1&0&-1&0&0\\2&1&0&0&3\\0&0&0&1&1\end{pmatrix}\;\xrightarrow{r_2-2r_1}\begin{pmatrix}1&0&-1&0&0\\0&1&2&0&3\\0&0&0&1&1\end{pmatrix}\;,\;\;\\选x_3\mathrm{为自由变量},\mathrm{得通解为}x=k\begin{pmatrix}1\\-2\\1\\0\end{pmatrix}+\begin{pmatrix}1\\1\\1\\1\end{pmatrix},\;c∈ R.\end{array}
$$



$$
\mathrm{若非齐次线性方程组}Ax=b\mathrm{中方程个数少于未知量个数},\mathrm{那么}\left(\right).
$$
$$
A.
Ax=b\mathrm{必有无穷多解} \quad B.Ax=\mathrm{O必有非零解} \quad C.Ax=\mathrm{O仅有零解} \quad D.Ax=\mathrm{O一定无解} \quad E. \quad F. \quad G. \quad H.
$$
$$
\mathrm{非齐次线性方程组}A_{m× n}x=b中,\mathrm{方程个数为}m,\mathrm{未知量个数为}n,由m<\;n⇒ r\left(A\right)\leq\;m<\;n,故A_{m× n}x=0\mathrm{必有非零解}.
$$



$$
\mathrm{若线性方程组}Ax=b\mathrm{的系数矩阵}A是m× n的,且m<\;n,则\left(\right).
$$
$$
A.
Ax=b\mathrm{必有无穷多解} \quad B.Ax=b\mathrm{一定无解} \quad C.Ax=\mathrm{O必有非零解} \quad D.Ax=\mathrm{O只有零解} \quad E. \quad F. \quad G. \quad H.
$$
$$
由m<\;n⇒ r\left(A\right)\leq\;m<\;n,故A_{m× n}x=0\mathrm{必有非零解}.
$$



$$
\mathrm{非齐次线性方程组}A_{5×5}x=b,\mathrm{当下列}\left(\right)\mathrm{成立时},\mathrm{该方程组有无穷多解}.
$$
$$
A.
r\left(A\right)=5 \quad B.r\left(A\;b\right)=5 \quad C.r\left(A\right)=r\left(A\;b\right)=5 \quad D.r\left(A\right)=r\left(A\;b\right)=4 \quad E. \quad F. \quad G. \quad H.
$$
$$
\mathrm{线性方程组}A_{m× n}x=b\mathrm{有无穷多解的条件是}r\left(A\right)=r\left(A\;b\right)<\;n,\mathrm{选项中}r\left(A\right)=r\left(A\;b\right)=4<\;5\mathrm{满足条件}.
$$



$$
\mathrm{线性方程组}\left\{\begin{array}{l}ax-by=1\\bx+ay=0\end{array}\right.,若a\neq b,\mathrm{则方程组}\left(\right).\;
$$
$$
A.
\mathrm{无解} \quad B.\mathrm{有唯一解} \quad C.\mathrm{有无穷多解} \quad D.\mathrm{其解需要讨论多种情况} \quad E. \quad F. \quad G. \quad H.
$$
$$
\left(A\;b\right)=\begin{pmatrix}a&-b&1\\b&a&0\end{pmatrix}\rightarrow\begin{pmatrix}a&-b&1\\0&\frac{a^2+b^2}a&-\frac ba\end{pmatrix},\mathrm{由于}a\neq b,\mathrm{因此}a^2+b^2\neq0,故r\left(A\right)=r\left(A\;b\right)=2,\mathrm{方程组有唯一解}.
$$



$$
\mathrm{若齐次方程组}Ax=O\mathrm{有无穷多解},\mathrm{则非齐次方程组}Ax=b\left(\right).\;
$$
$$
A.
\mathrm{必有无穷多解} \quad B.\mathrm{可能有唯一解} \quad C.\mathrm{必无解} \quad D.\mathrm{有解时必有无穷多组解} \quad E. \quad F. \quad G. \quad H.
$$
$$
\begin{array}{l}Ax=0\mathrm{有无穷多解}⇒\mathrm r\left({\mathrm A}_{\mathrm m×\mathrm n}\right)<\;\mathrm n,\mathrm{若方程组}Ax=b\mathrm{有解},则r\left(A\right)=r\left(A\;B\right)<\;n,\mathrm{即有无穷多解};\\\mathrm{但直接由r}\left({\mathrm A}_{\mathrm m×\mathrm n}\right)<\;\mathrm{n无法判断Ax}=\mathrm{b是否有解}.\end{array}
$$



$$
\mathrm{线性方程组}A_{m× n}x=b\mathrm{有解必要条件是}\left(\right).\;
$$
$$
A.
b=0 \quad B.m<\;n \quad C.m=n \quad D.r\left(A\right)=r\left(A\;b\right) \quad E. \quad F. \quad G. \quad H.
$$
$$
\mathrm{非齐次线性方程组}A_{m× n}X=b\mathrm{有解的充要条件是}r\left(A\right)=r\left(A\;b\right),\mathrm{当然也是必要条件}.
$$



$$
\mathrm{若方程组}\left\{\begin{array}{c}x_1+2x_2-x_3+3x_4=λ\\x_1+x_2-3x_3+5x_4=5\\x_2+2x_3-2x_4=2λ\end{array}\right.\mathrm{有解},则λ=\left(\right).\;
$$
$$
A.
5 \quad B.-5 \quad C.-1 \quad D.1 \quad E. \quad F. \quad G. \quad H.
$$
$$
\begin{array}{l}\mathrm{对增广矩阵施以行初等变换},有\\\left(A\;b\right)=\begin{pmatrix}1&2&-1&3&λ\\1&1&-3&5&5\\0&1&2&-2&2λ\end{pmatrix}\rightarrow\begin{pmatrix}1&2&-1&3&λ\\0&-1&-2&2&5-λ\\0&1&2&-2&2λ\end{pmatrix}\rightarrow\begin{pmatrix}1&2&-1&3&λ\\0&-1&-2&2&5\;-λ\\0&0&0&0&λ+5\end{pmatrix},\\\mathrm{非齐次线性方程组有解},则r(A)=r(A\;b),\mathrm{因此}λ+5=0⇒λ=-5.\end{array}
$$



$$
\mathrm{方程组}\left\{\begin{array}{c}x_1-2x_x+x_3=-5\\x_1+5x_2-7x_3=2\\3x_1+x_2-5x_x=1\end{array}\right.\mathrm{的解的情况是}\left(\right).\;
$$
$$
A.
\mathrm{有唯一解} \quad B.\mathrm{有无穷多解},\mathrm{通解形式是}η+cν(c\mathrm{任意常数}) \quad C.\mathrm{无解} \quad D.\mathrm{有无穷多解},\mathrm{通解形式是}η+c_1ν_1+c_2ν_2(c_1,c_2\mathrm{任意常数}) \quad E. \quad F. \quad G. \quad H.
$$
$$
\mathrm{系数矩阵的秩等于}r\left(A\right)=2\mathrm{而增广矩阵的秩}r\left(\widetilde A\right)=3,即r\left(A\right)\neq r\left(\widetilde A\right),\mathrm{所以方程组无解}.
$$



$$
n\mathrm{元齐次线性方程组}Ax=0\mathrm{有非零解},则\;Ax=b\left(\right).
$$
$$
A.
\mathrm{必有无穷多解} \quad B.\mathrm{必有唯一解} \quad C.\mathrm{必定无解} \quad D.\mathrm{无法确定} \quad E. \quad F. \quad G. \quad H.
$$
$$
\mathrm{由线性方程组解的判断定理可知}r\left(A\right)\;<\;n,\mathrm{不能得出}r\left(A\right)和r\left(A\;b\right)\mathrm{的关系},\mathrm{无法确定}Ax=b\mathrm{的解}.
$$



$$
设A是m× n\mathrm{矩阵},\mathrm{非齐次线性方程组}Ax=b\mathrm{的导出组为}Ax=O,\;\mathrm{如果}m\;<\;n,\;则\;\left(\right).
$$
$$
A.
Ax=b\mathrm{必有无穷多解} \quad B.Ax=b\mathrm{必有唯一解} \quad C.Ax=O\mathrm{必有非零解} \quad D.Ax=O\mathrm{必有唯一解} \quad E. \quad F. \quad G. \quad H.
$$
$$
\mathrm{由于}0\leq\;r(A)\leq\;min\left\{m,n\right\},\mathrm{所以}r(A)=m<\;n,\mathrm{由解的判断定理可知}Ax=0\mathrm{必有非零解}.
$$



$$
\mathrm{适用于任一线性方程组的解法是}\left(\right).
$$
$$
A.
\mathrm{逆矩阵求法} \quad B.\mathrm{克莱姆法则} \quad C.\mathrm{消元法} \quad D.\mathrm{以上方法都行} \quad E. \quad F. \quad G. \quad H.
$$
$$
\mathrm{逆矩阵求法仅适用于系数矩阵可逆的情况},\mathrm{克莱姆法则同理},\mathrm{只有消元法适用于任一线性方程组的求解}.
$$



$$
\mathrm{方程组}\left\{\begin{array}{c}x_1+2x_2-x_3+3x_4=4\\x_1+x_2-3x_3+5x_4=5\\x_2+2x_3-2x_4=2λ\end{array}\right.\mathrm{有解的条件是}λ=\left(\right)\;.
$$
$$
A.
-\frac12 \quad B.\frac12 \quad C.-1 \quad D.1 \quad E. \quad F. \quad G. \quad H.
$$
$$
\mathrm{根据非齐次方程组有解的条件为}r(A)=r(Ab),\mathrm{经过对方程组的增广矩阵进行初等行变换后可计算得}λ=-\frac12.
$$



$$
设A=\;\begin{pmatrix}1&2&1\\2&3&a+2\\1&a&-2\end{pmatrix},b=\begin{pmatrix}1\\3\\0\end{pmatrix},x=\begin{pmatrix}x_1\\x_2\\x_3\end{pmatrix}\mathrm{线性方程组}Ax=b\mathrm{无解},则\;a=\left(\right).
$$
$$
A.
-1 \quad B.1 \quad C.-3 \quad D.3 \quad E. \quad F. \quad G. \quad H.
$$
$$
\begin{array}{l}\mathrm{对增广矩阵作初等行变换},有\;\\\begin{pmatrix}1&2&1&&1\\2&3&a+2&\vert&3\\1&a&-2&&0\end{pmatrix}⇒\begin{pmatrix}1&2&1&&1\\0&-1&a&\vert&1\\0&0&a^2-2a-3&&a-3\end{pmatrix},\\,\mathrm{此时方程组无解必然是}r\left(A\right)=2,r\left(\widetilde A\right)=3即a^2-2a-3=0,a-3\neq0\mathrm{故应填}a=-1\end{array}
$$



$$
\mathrm{方程组}\left\{\begin{array}{c}2x_1-x_2+x_3+x_4=1\\x_1+2x_2+x_3-x_4=-1\\x_1+7x_2+2x_3-4x_4=λ+1\end{array}\right.\mathrm{有解},则λ\mathrm{的值为}(\;).\;
$$
$$
A.
λ=1 \quad B.λ=-1 \quad C.λ=-5 \quad D.λ=5 \quad E. \quad F. \quad G. \quad H.
$$
$$
\begin{array}{l}∵\overline A=\begin{pmatrix}2&-1&1&1&1\\1&2&1&-1&-1\\1&7&2&-4&λ+1\end{pmatrix}\rightarrow\begin{pmatrix}1&2&1&-1&1\\0&-5&-1&3&3\\0&0&0&0&λ+5\end{pmatrix},\\∵ 当λ=-5时,R\left(\overline A\right)=R\left(A\right)=2,\mathrm{即方程组有解}.\end{array}
$$



$$
\mathrm{若线性方程组}\left\{\begin{array}{c}x_1+x_2+x_3+x_4=2\\2x_2+3x_3+4x_4=a-2\\\left(a^2-1\right)x_4=a\left(a-1\right)^2\end{array}\right.\mathrm{无解},则a\mathrm{的值为}(\;).
$$
$$
A.
±1 \quad B.0 \quad C.1 \quad D.-1 \quad E. \quad F. \quad G. \quad H.
$$
$$
\begin{array}{l}\left(A\vdots b\right)=\begin{pmatrix}1&1&1&1&\vdots&2\\0&2&3&4&\vdots&a-2\\0&0&0\;&a^2-1\;&\vdots&a\left(a-1\right)^2\end{pmatrix}.\\当a=-1时,有\left(a\vdots b\right)\rightarrow\begin{pmatrix}1&1&1&1&2\\0&2&3&4&-3\\0&0&0&0&-4\end{pmatrix}.有R\left(A\right)=2,R\left(A,b\right)=3\mathrm{故方程组无解}.\\\end{array}
$$



$$
\mathrm{方程组}\left\{\begin{array}{c}λ x_1+x_2=λ^2\\x_1+λ x_x=1\end{array}\right.\mathrm{有唯一解},则λ\mathrm{的值为}(\;).\;
$$
$$
A.
λ=-1 \quad B.λ=1 \quad C.λ\neq±1 \quad D.λ=±1 \quad E. \quad F. \quad G. \quad H.
$$
$$
\begin{array}{l}\begin{pmatrix}λ&1&λ^2\\1&λ&1\end{pmatrix}\rightarrow\begin{pmatrix}1&λ&1\\0&1-λ^2&λ^2-λ\end{pmatrix}\\当λ\neq±1时R\left(A\right)=R\left(\widetilde A\right)=2,\mathrm{有唯一解}.\;\;\\当λ=1时,R\left(A\right)=R\left(\widetilde A\right)=1,\mathrm{有无穷多解}.\;\;\\当λ=-1时,R\left(A\right)=1,R\left(\widetilde A\right)=2,\mathrm{无解}.\end{array}
$$



$$
\mathrm{线性方程组}\left\{\begin{array}{c}x_1+x_3=λ\\4x_1+x_2+2x_3=λ+2\\6x_1+x_2+4x_3=2λ+3\end{array}\right.\mathrm{有解},则λ\mathrm{的值为}(\;).\;
$$
$$
A.
λ=1 \quad B.λ\neq1 \quad C.λ=-1 \quad D.λ=-2 \quad E. \quad F. \quad G. \quad H.
$$
$$
\begin{array}{l}\mathrm{注意这个方程组系数矩阵中不含参数},\mathrm{所以只要把增广矩阵化为阶梯形}\\\overline A=\begin{pmatrix}1&0&1&λ\\4&1&2&λ+2\\6&1&4&2λ+3\end{pmatrix}\rightarrow\begin{pmatrix}1&0&1&λ\\0&1&-2&-3λ+2\\0&1&-2&-4λ+3\end{pmatrix}\rightarrow\begin{pmatrix}1&0&1&λ\\0&1&-2&-3λ+2\\0&0&0&-λ+1\end{pmatrix}\\\mathrm{可见},\mathrm{只有当}λ=1时R\left(A\right)=R\left(\overline A\right)=2,\mathrm{方程组有解}.\end{array}
$$



$$
\mathrm{若方程组}\left\{\begin{array}{c}x_1+x_2=-a_1\\x_2+x_3=a_2\\x_3+x_4=-a_3\\x_1+x_4=a_4\end{array}\right.\mathrm{有解},则a_1,a_2,a_3,a_4\mathrm{应满足条件为}\_\_\_\_\_\_\_\_\_\_.\;
$$
$$
A.
a_1+a_2+a_3+a_4=0 \quad B.a_1+a_2+a_3+a_4\neq0 \quad C.a_1+a_2=a_3+a_4 \quad D.a_1+a_2\neq a_3+a_4 \quad E. \quad F. \quad G. \quad H.
$$
$$
\begin{array}{l}\mathrm{对方程组的增广矩阵}\widetilde A\mathrm{施以初等行变换},得\;\;\\\begin{pmatrix}1&1&0&0&-a_1\\0&1&1&0&a_2\\0&0&1&1&-a_3\\1&0&0&1&a_4\end{pmatrix}\xrightarrow{\mathrm{初等行变换}}\begin{pmatrix}1&1&0&0&-a_1\\0&1&1&0&\;a_2\\0&0&1&1&-a_3\\0&0&0&0&a_4+a_1+\;a_2+a_3\end{pmatrix},\\\mathrm{由于方程组有解},则R(A)=R(\widetilde A),故a_1+a_2+a_3+a_4=0.\end{array}
$$



$$
\mathrm{若方程组}\left\{\begin{array}{c}2x_1-3x_2+6x_3-5x_4=3\\x_2-4x_3+x_4=1\\4x_1-5x_2+8x_3-9x_4=k\end{array}\right.\mathrm{有解},则k=\_\_\_\_\_\_\_.\;
$$
$$
A.
7 \quad B.-7 \quad C.6 \quad D.-6 \quad E. \quad F. \quad G. \quad H.
$$
$$
\begin{array}{l}\mathrm{对增广矩阵施以初等行变换}\\\begin{pmatrix}2&-3&6&-5&3\\0&1&-4&1&1\\4&-5&8&-9&k\end{pmatrix}\rightarrow\begin{pmatrix}2&-3&6&-5&3\\0&1&-4&1&1\\0&1&-4&1&k-6\end{pmatrix}\rightarrow\begin{pmatrix}2&-3&6&-5&3\\0&1&-4&1&1\\0&0&0&0&k-7\end{pmatrix},\\\mathrm{方程组有解},则k-7=0⇒ k=7.\end{array}
$$



$$
设A=\begin{pmatrix}1&2&1\\2&3&a+2\\1&a&-2\end{pmatrix},\;b=\begin{pmatrix}1\\3\\0\end{pmatrix},\;x=\begin{pmatrix}x_1\\x_2\\x_3\end{pmatrix}\mathrm{若线性方程组}Ax=b\mathrm{有解},则a\mathrm{满足}(\;\;\;).
$$
$$
A.
a=-1 \quad B.a\neq-1 \quad C.a\neq-3 \quad D.a\neq3 \quad E. \quad F. \quad G. \quad H.
$$
$$
\begin{array}{l}\mathrm{对增广矩阵作初等行变换},有\\\begin{pmatrix}1&2&1&1\\2&3&a+2&3\\1&a&-2&0\end{pmatrix}⇒\begin{pmatrix}1&2&1&1\\0&-1&a&1\\0&0&a^2-2a-3&a-3\end{pmatrix},\\\mathrm{此时方程组无解必然是}R(A)=2,R(\widetilde A)=3即\;a^2-2a-3=0,a-3\neq\;\mathrm{故应填}a=-1\end{array}
$$



$$
设\left\{\begin{array}{c}\left(2-λ\right)x_1+2x_2-2x_3=1\\2x_1+\left(5-λ\right)x_2-4x_3=2\\-2x_1-4x_2+\left(5-λ\right)x_3=-λ-1\end{array}\right.,\;\;\mathrm{若方程组有唯一解},则λ\mathrm{的值为}(\;).
$$
$$
A.
λ=10 \quad B.λ=1 \quad C.λ\neq1,λ\neq10 \quad D.λ=1或λ=10 \quad E. \quad F. \quad G. \quad H.
$$
$$
\begin{array}{l}\mathrm{系数矩阵的行列式}\\\left|A\right|=\begin{vmatrix}2-λ&2&-2\\2&5-λ&-4\\-2&-4&5-λ\end{vmatrix}=-\left(λ-1\right)^2\left(λ-10\right)\\当λ\neq1,λ\neq10时,R\left(A\right)=R\left(B\right)=3,\mathrm{方程组有唯一解}.\end{array}
$$



$$
\mathrm{设方程组}\left\{\begin{array}{c}ax+ay+\left(a+1\right)z=a\\ax+ay+\left(a-1\right)z=a\\\left(a+1\right)x+ay+\left(2a+3\right)z=1\end{array}\right.,\mathrm{若方程组有唯一解},则a\mathrm{的值满足}(\;).\;
$$
$$
A.
a\neq0 \quad B.a=0 \quad C.a\neq-1 \quad D.a\neq-1且a\neq0 \quad E. \quad F. \quad G. \quad H.
$$
$$
\begin{array}{l}\left|A\right|=\begin{vmatrix}a&a&a+1\\a&a&a-1\\a+1&a&2a+3\end{vmatrix}=-2a\\.\;\;当a\neq0时,r\left(A\right)=r\left(\widetilde A\right)=3,\mathrm{此时方程组有唯一解}\end{array}
$$



$$
\mathrm{设矩阵}A=\begin{pmatrix}1\;1\;a\\1\;a\;1\\a\;1\;1\end{pmatrix}\;,β=\begin{pmatrix}1\\1\\-2\end{pmatrix}\mathrm{已知线性方程组}Ax=β\;\mathrm{有解但不唯一},则\;a=\left(\right)
$$
$$
A.
-2 \quad B.2 \quad C.-1 \quad D.1 \quad E. \quad F. \quad G. \quad H.
$$
$$
\begin{array}{l}\mathrm{对增广矩阵施行初等行变换}\\\widetilde A=\left(A\;β\right)=\begin{pmatrix}1&1&a&1\\1&a&1&1\\a&1&1&-2\end{pmatrix}\xrightarrow{\begin{array}{c}r_2-r_1\\r_3-ar_1\end{array}}\begin{pmatrix}1&1&a&1\\0&a-1&1-a&0\\0&1-a&1-a^2&-2-a\end{pmatrix}\xrightarrow{r_3+r_2}\begin{pmatrix}1&1&a&1\\0&a-1&1-a&0\\0&0&2-a-a^2&-2-a\end{pmatrix}\\\mathrm{由于方程组有解但不唯一},\mathrm{则一定有}r\left(A\right)=r\left(A\right)<\;n=3\;\;则a=-2\end{array}
$$



$$
\mathrm{已知方程组}\left\{\begin{array}{c}x_1+2x_2+x_3=1\\2x_1+3x_2+(a+2)x_3=3\\x_1+ax_2-2x_3=0\end{array}\right.\mathrm{无解},则a=\left(\right)
$$
$$
A.
a=3 \quad B.a=-1 \quad C.a=-1或a=3 \quad D.\mathrm{任意数} \quad E. \quad F. \quad G. \quad H.
$$
$$
\begin{array}{l}\mathrm{对方程组的增广矩阵施以初等行变换},有\;\\\begin{pmatrix}1&2&1&1\\2&3&a+2&3\\1&a&-2&0\end{pmatrix}\rightarrow\begin{pmatrix}1&2&1&1\\0&-1&a&1\\0&a-2&-3&-1\end{pmatrix}\rightarrow\begin{pmatrix}1&2&1&1\\0&1&-a&-1\\0&0&\left(a-3\right)\left(a+1\right)&\left(a-3\right)\end{pmatrix}\\,当a=3时,R\left(A\right)=R\left(A\vert b\right)=2,\mathrm{方程组有无穷多解},\mathrm{与题意不符};当a=-1时,R\left(A\right)=2\neq R\left(A\vert b\right)=3,\mathrm{故方程组无解}.\end{array}
$$



$$
\mathrm{已知方程组}\left\{\begin{array}{c}x_1+2x_2+x_3=1\\2x_1+3x_2+\left(a+2\right)x_3=3\\x_1+ax_2-2x_3=0\end{array}\right.\mathrm{有无穷多解},则a=\left(\right)
$$
$$
A.
a=3 \quad B.a=-1 \quad C.a=-1或a=3 \quad D.\mathrm{任意数} \quad E. \quad F. \quad G. \quad H.
$$
$$
\begin{array}{l}\mathrm{对方程组的增广矩阵施以初等行变换},有\\\begin{pmatrix}1&2&1&1\\2&3&a+2&3\\1&a&-2&0\end{pmatrix}\rightarrow\begin{pmatrix}1&2&1&1\\0&-1&a&1\\0&a-2&-3&-1\end{pmatrix}\rightarrow\begin{pmatrix}1&2&1&1\\0&1&-a&-1\\0&0&\left(a-3\right)\left(a+1\right)&\left(a-3\right)\end{pmatrix}\\,\;\;当a=3时,r\left(A\right)=r\left(A\;b\right)=2,\mathrm{方程组有无穷多解},\end{array}
$$



$$
\mathrm{若方程组}\left\{\begin{array}{c}x_1+x_2+x_3+x_4=1\\3x_1+2x_2+2x_3+3x_4=a\\x_2+x_3+x_4=1\\5x_1+3x_2+3x_3+5x_4=b\end{array}\right.\mathrm{有解},则a,b\mathrm{可以为}\left(\right).\;
$$
$$
A.
a=3,\;b=4 \quad B.a=2,\;b=5 \quad C.a=3,\;b=5 \quad D.a=3,\;b=3 \quad E. \quad F. \quad G. \quad H.
$$
$$
\begin{array}{l}\mathrm{增广矩阵施以行初等变换},有\\\begin{pmatrix}1&1&1&1&1\\3&2&2&3&a\\0&1&1&1&1\\5&3&3&5&b\end{pmatrix}\rightarrow\begin{pmatrix}1&1&1&1&1\\0&-1&-1&0&a-3\\0&1&1&1&1\\0&-2&-2&0&b-5\end{pmatrix}\rightarrow\begin{pmatrix}1&1&1&1&1\\0&1&1&1&1\\0&0&0&1&a-2\\0&0&0&0&b-2a+1\end{pmatrix}\\,\;\;\mathrm{方程组有解},则r(A)=r(Ab),故b-2a+1=0.\;\mathrm{选项中只有}a=3,b=5\mathrm{符合条件}.\end{array}
$$



$$
\mathrm{若方程组}\left\{\begin{array}{c}ax_1+bx_2+2x_3=1\\\left(b-1\right)x_2+x_3=0\\ax_1+bx_2+\left(1-b\right)x_3=3-2b\end{array}\right.\;\mathrm{有唯一解},则a,b\mathrm{须满足}
$$
$$
A.
a\neq0,\;b\neq±1 \quad B.a\neq0,\;b\neq1 \quad C.a\neq0,\;b\neq-1 \quad D.a=0,\;b\neq1 \quad E. \quad F. \quad G. \quad H.
$$
$$
\begin{pmatrix}a&b&2&1\\0&b-1&1&0\\a&b&1-b&3-2b\end{pmatrix}\rightarrow\begin{pmatrix}a&b&2&1\\0&b-1&1&0\\0&0&-1-b&2-2b\end{pmatrix}\rightarrow\begin{pmatrix}a&1&1&1\\0&b-1&1&0\\0&0&1+b&2\left(b-1\right)\end{pmatrix},\;\;\mathrm{所以当}a\neq0,b\neq±1时,\mathrm{有唯一解}.
$$



$$
设n\mathrm{元非齐次线性方程组}Ax=b\mathrm{的导出组为}Ax=0,\;\mathrm{如果}Ax=0\mathrm{仅有零解},则\;Ax=b\left(\right)
$$
$$
A.
\mathrm{必有无穷多解} \quad B.\mathrm{必有唯一解} \quad C.\mathrm{必定无解} \quad D.\mathrm{无法确定} \quad E. \quad F. \quad G. \quad H.
$$
$$
Ax=O\mathrm{仅有零解},\mathrm{只能得到}R\left(A\right),\mathrm{不能得到}R\left(Ab\right).
$$



$$
\mathrm{若方程组}\left\{\begin{array}{c}x_1+2x_2-2x_3+2x_4=2\\x_2-x_3-x_4=1\\x_1+x_2-x_3+3x_4=a\\x_1-x_2+x_3+5x_4=b\end{array}\right.\mathrm{有无穷多个解},则a,\;b\mathrm{须满足}\left(\right).\;
$$
$$
A.
a=1,\;b=-1 \quad B.a=-1,\;b=-1 \quad C.a=-1,\;b=1 \quad D.a=1,\;b=1 \quad E. \quad F. \quad G. \quad H.
$$
$$
\begin{pmatrix}1&2&-2&2&2\\0&1&-1&-1&1\\1&1&-1&3&a\\1&-1&1&5&b\end{pmatrix}\rightarrow\begin{pmatrix}1&2&-2&2&2\\0&1&-1&-1&1\\0&0&0&0&a-1\\0&0&0&0&b+1\end{pmatrix},\;\;\mathrm{故当}a=1,b=-1时,\mathrm{有无穷多个解}.
$$



$$
\mathrm{若方程组}\left\{\begin{array}{c}x_1+x_2+x_3+x_4=1\\3x_1+2x_2+2x_3+3x_4=a\\x_2+x_3+x_4=1\\5x_1+3x_2+3x_3+5x_4=b\end{array}\right.\mathrm{有解},则a,b\mathrm{应为}\left(\right).\;
$$
$$
A.
a=3,\;b=4 \quad B.a=2,\;b=5 \quad C.a=3,\;b=5 \quad D.a=3,\;b=3 \quad E. \quad F. \quad G. \quad H.
$$
$$
\begin{array}{l}\mathrm{增广矩阵施以行初等变换},有\\\begin{pmatrix}1&1&1&1&1\\3&2&2&3&a\\0&1&1&1&1\\5&3&3&5&b\end{pmatrix}\rightarrow\begin{pmatrix}1&1&1&1&1\\0&-1&-1&0&a-3\\0&1&1&1&1\\0&-2&-2&0&b-5\end{pmatrix}\rightarrow\begin{pmatrix}1&1&1&1&1\\0&1&1&1&1\\0&0&0&1&a-2\\0&0&0&0&b-2a+1\end{pmatrix}\\,\;\;\mathrm{方程组有解},则r(A)=r(A,b),故b-2a+1=0.\;\mathrm{选项中只有}a=3,b=5\mathrm{符合条件}.\end{array}
$$



$$
\mathrm{若方程组}\left\{\begin{array}{c}x_1+2x_2-x_3=λ-1\\3x_2-x_3=λ-2\\λ x_2-x_3=\left(λ-3\right)\left(λ-4\right)+\left(λ-2\right)\end{array}\right.\mathrm{有无穷多解},则λ=\left(\right).\;
$$
$$
A.
1 \quad B.2 \quad C.3 \quad D.4 \quad E. \quad F. \quad G. \quad H.
$$
$$
\begin{array}{l}\mathrm{增广矩阵为},\\\begin{pmatrix}1&2&-1&λ-1\\0&3&-1&λ-2\\0&λ&-1&(λ-3)(λ-4)+(λ-2)\end{pmatrix}\;\;\rightarrow\begin{pmatrix}1&2&-1&λ-1\\0&3&-1&λ-2\\0&0&\frac13(λ-3)&(λ-3)(λ-4)+\frac13(λ-2)(3-λ)\end{pmatrix}\;\\\;\\当λ=3时,r\left(A\right)=r\left(A\;b\right)=2,\mathrm{方程组有无穷多解},\mathrm{满足条件}.\end{array}
$$



$$
n\mathrm{阶矩阵}A\mathrm{可逆的充要条件是}\left(\right).\;
$$
$$
A.
A\mathrm{的任一行向量都是非零向量} \quad B.A\mathrm{的任一列向量都是非零向量} \quad C.\mathrm{非齐次线性方程组}Ax=b\mathrm{有解} \quad D.当x\neq0\mathrm 时,Ax\neq O,\mathrm{其中}x=\left(x_1,x_2,...,x_n\right)^T \quad E. \quad F. \quad G. \quad H.
$$
$$
n\mathrm{阶矩阵}A\mathrm{可逆的条件是}\left|A\right|\neq0,故r\left(A\right)=n,即A\mathrm{的行}(列)\mathrm{向量都线性无关},且Ax=O\mathrm{只有零解}.
$$



$$
设A为m× n\mathrm{矩阵},b为m\mathrm{维非零列向量},\mathrm{则有}\left(\right).\;
$$
$$
A.
当m<\;n时,Ax=b\mathrm{有无穷多解} \quad B.当m\;<\;n时,Ax=0\mathrm{有非零解},\mathrm{且基础解系中含}n-m\mathrm{个线性无关解向量} \quad C.若A有n\mathrm{阶子式不为零},则Ax=b\mathrm{有无穷多解} \quad D.若A有n\mathrm{阶子式不为零},则Ax=0\mathrm{仅有零解} \quad E. \quad F. \quad G. \quad H.
$$
$$
\begin{array}{l}记\left(A\;b\right)=B,则\;r\left(A\right)=r\left(B\right)=n⇒ Ax=b\mathrm{有唯一解};\\\;r\left(A\right)=r\left(B\right)<\;n⇒ Ax=b\mathrm{有无穷多解};\;\;\\r\left(A\right)<\;n⇒ Ax=0\mathrm{有非零解};\\r\left(A\right)=\;n⇒ Ax=0\mathrm{只有零解}.\;\;\\若A有n\mathrm{阶子式不为零},又r\left(A\right)=min\left\{m,n\right\},故r\left(A\right)=\;n,则Ax=O\mathrm{仅有零解}.\end{array}
$$



$$
\mathrm{若方程组}\left\{\begin{array}{c}x_1+2x_2-x_3=4\\\left(λ-3\right)x_2+2x_3=2\\\left(λ-1\right)\left(λ-2\right)x_3=\left(λ-3\right)\left(λ-4\right)\end{array}\right.\mathrm{有唯一解},则λ\mathrm{可以取为}\left(\right).\;
$$
$$
A.
1 \quad B.2 \quad C.3 \quad D.4 \quad E. \quad F. \quad G. \quad H.
$$
$$
\begin{array}{l}\mathrm{原方程组的增广矩阵为}\begin{pmatrix}1&2&-1&4\\0&λ-3&2&2\\0&0&\left(λ-1\right)\left(λ-2\right)&\left(λ-3\right)\left(λ-4\right)\end{pmatrix},\mathrm{本题可用排除法}:\;\;\\当λ=1或λ=2时,R\left(A\right)=2\neq R\left(A\vert b\right)=3,\mathrm{方程组无解},\mathrm{不满足条件};\;\\当λ=3时,R\left(A\right)=R\left(A\vert b\right)=2,\mathrm{方程组有无穷多解},\mathrm{也不满足条件};\;\;\\当λ=4时,R\left(A\right)=R\left(A\vert b\right)=3,\mathrm{方程组有唯一解}.故λ=4.\end{array}
$$



$$
\mathrm{若方程组}\left\{\begin{array}{c}\left(λ+3\right)x_1+x_2+2x_3=λ\\λ x_1+\left(λ-1\right)x_2+x_3=λ\\3\left(λ+1\right)x_1+λ x_2+\left(λ+3\right)x_3=3\end{array}\right.\mathrm{有无穷多解},则λ=\left(\right).\;
$$
$$
A.
-2 \quad B.-1 \quad C.0 \quad D.1 \quad E. \quad F. \quad G. \quad H.
$$
$$
\begin{array}{l}\mathrm{方程组的系数行列式为}\\\begin{vmatrix}λ+3&1&2\\λ&λ-1&1\\3\left(λ+1\right)&λ&λ+3\end{vmatrix}=λ^2\left(λ-1\right),\\故λ=0或λ=1.\;\;\\当λ=0时,\mathrm{方程组的增广矩阵作初等行变换}:\\\widetilde A=\begin{pmatrix}3&1&2&0\\0&-1&1&0\\3&0&3&3\end{pmatrix}\xrightarrow{r_3-r_1}\begin{pmatrix}3&1&2&0\\0&-1&1&0\\0&-1&1&3\end{pmatrix}\xrightarrow{r_3-r_2}\begin{pmatrix}3&1&2&0\\0&-1&1&0\\0&0&0&3\end{pmatrix}\\.\;\;因r\left(A\right)\;=2,r\left(\widetilde A\right)\;=3\mathrm{所以方程组无解};\;\;\\当λ=1时,\mathrm{增广矩阵为}\\\widetilde A=\begin{pmatrix}4&1&2&1\\1&0&1&1\\6&1&4&3\end{pmatrix}\xrightarrow[{r_1↔ r_2}]{r_3-r_1}\begin{pmatrix}1&0&1&1\\4&1&2&1\\2&0&2&2\end{pmatrix}\xrightarrow[{r_3-2r_1}]{r_2-4r_1}\begin{pmatrix}1&0&1&1\\0&1&-2&-3\\0&0&0&0\end{pmatrix}\\.\;\;因\;r\left(A\right)\;=r\left(\widetilde A\right)\;=2<\;3\mathrm{所以方程组有无穷多个解}.\;\;故λ=1.\end{array}
$$



$$
设A是m× n\mathrm{矩阵},Ax=O\mathrm{是非齐次线性方程组}Ax=b\mathrm{对应的齐次线性方程组},\mathrm{那么}\left(\right).\;
$$
$$
A.
若Ax=0\mathrm{仅有零解},则Ax=b\mathrm{有无穷多解} \quad B.若Ax=0\mathrm{有非零解},则Ax=b\mathrm{有无穷多解} \quad C.若Ax=b\mathrm{有无穷多解},则Ax=0\mathrm{仅有零解} \quad D.若Ax=b\mathrm{有无穷多解},则Ax=0\mathrm{有非零解} \quad E. \quad F. \quad G. \quad H.
$$
$$
\begin{array}{l}\\若Ax=b\mathrm{有无穷多解},则r\left(A\right)=r\left(A\;b\right)<\;n,故Ax=O\mathrm{有非零解}.\end{array}
$$



$$
设A是m× n\mathrm{矩阵},且m<\;n,若A\mathrm{的行向量组线性无关},b为m\mathrm{维非零列向量},则\left(\right).\;
$$
$$
A.
Ax=b\mathrm{有无穷多解} \quad B.Ax=b\mathrm{仅有唯一解} \quad C.Ax=b\mathrm{无解} \quad D.Ax=b\mathrm{仅有零解} \quad E. \quad F. \quad G. \quad H.
$$
$$
\mathrm{由矩阵}A\mathrm{的行向量组线性无关可得},R(A)=R(A\vert b)=m,又m<\;n,故R(A)=R(A\vert b)<\;n,\mathrm{则非齐次线性方程组}Ax=b\mathrm{有无穷多解}.
$$



$$
设A,B是n\mathrm{阶方阵},x,y,b是n×1\mathrm{矩阵},\mathrm{则方程组}\begin{pmatrix}\mathrm O&B\\A&\mathrm O\end{pmatrix}\begin{pmatrix}x\\y\end{pmatrix}=\begin{pmatrix}0\\b\end{pmatrix}\mathrm{有解的充要条件是}(\;).\;
$$
$$
A.
r\left(A\right)=r\left(A\vdots b\right),r\left(B\right)\mathrm{任意}; \quad B.Ax=b\mathrm{有解},By=0\mathrm{有非零解} \quad C.\left|A\right|\neq0,b\mathrm{可由}B\mathrm{的列向量线性表出} \quad D.\left|B\right|\neq0,b\mathrm{可由}A\mathrm{的列向量线性表出} \quad E. \quad F. \quad G. \quad H.
$$
$$
\begin{array}{l}\begin{pmatrix}\mathrm O&B\\A&\mathrm O\end{pmatrix}\begin{pmatrix}x\\y\end{pmatrix}=\begin{pmatrix}0\\b\end{pmatrix}\mathrm{有解的充要条件是}By=0,Ax=b\mathrm{同时有解}.\;\mathrm{由于齐次线性方程组}By=0\mathrm{总是有解的},\mathrm{所以}r\left(B\right)\mathrm{任意};\\\mathrm{又非齐次方程组}Ax=b\mathrm{有解的充要条件是}r(A)=r(A\vdots b)\end{array}
$$



$$
\begin{array}{l}\mathrm{已知}n\mathrm{维向量}a_1,a_{2,}...,a_n中,前n-1\mathrm{个向量线性相关},后n-1\mathrm{个向量线性无关}.β=a_1+a_2+...+a_n,\\\mathrm{矩阵}A=\left(a_1,a_{2,}...,a_n\right)是n\mathrm{阶矩阵},\mathrm{则下列正确的是}(\;\;\;\;\;\;)\end{array}
$$
$$
A.
\mathrm{方程组}Ax=β\mathrm{必有无穷多解} \quad B.\mathrm{方程组}Ax=β\mathrm{必无解} \quad C.\mathrm{方程组}Ax=β\mathrm{必唯一解} \quad D.\mathrm{方程组}Ax=β\mathrm{解的情形不能确定} \quad E. \quad F. \quad G. \quad H.
$$
$$
\begin{array}{l}\mathrm{由于}\;A\begin{pmatrix}1\\1\\\vdots\\1\end{pmatrix}\;=(a_1,a_2,...,a_n)\begin{pmatrix}1\\1\\\vdots\\1\end{pmatrix}\;=β,\;\;\\\mathrm{所以方程组}Ax=β\mathrm{有解}.(1,1,...,1)^T\mathrm{是其一个解}.\mathrm{因为}a_1,a_2,...,a_{n-1}\mathrm{线性相关},a_2,a_3,...,a_n\mathrm{线性无关},\\故a_2,a_3,...,a_n\mathrm{线性无关且}a_1\mathrm{可由}a_2,a_3,...,a_n\mathrm{线性表示},\mathrm{那么秩}r\left(A\right)=n-1.\mathrm{于是}Ax=β\mathrm{有无穷多解}.\end{array}
$$



$$
\mathrm{已知非齐次线性方程组}\left\{\begin{array}{c}x_1+x_2+x_3+x_4=-1\\4x_1+3x_2+5x_3-x_4=-1\\ax_1+3x_2+bx_3-5x_4=1\end{array}\right.有3\mathrm{个线性无关的解}.\mathrm{则方程组系数矩阵}A\mathrm{的秩为}(\;\;\;\;)
$$
$$
A.
1 \quad B.2 \quad C.3 \quad D.4 \quad E. \quad F. \quad G. \quad H.
$$
$$
\begin{array}{l}设ζ_1,ζ_{2,}ζ_3\mathrm{是该线性方程组的}3\mathrm{个线性无关的解},则ζ_1-ζ_{2,}ζ_{1,}-ζ_3,\mathrm{是对应的齐次线性方程组}Ax=0\mathrm{的两个线性无关的解},\mathrm{因而}4-r\left(A\right)\geq\;2,\\即r\left(A\right)\leq\;2;\;\;又A\mathrm{有一个}2\mathrm{阶子式}\begin{vmatrix}1&1\\4&3\end{vmatrix}=-1\neq0,\mathrm{于是}r\left(A\right)\geq\;2.\mathrm{因此}r\left(A\right)=\;2.\end{array}
$$



$$
\mathrm{已知非齐次线性方程组}\left\{\begin{array}{c}x_1+x_2+x_3+x_4=-1\\4x_1+3x_2+5x_3-x_4=-1\\ax_1+x_2+3x_3+bx_4=1\end{array}\right.有3\mathrm{个线性无关的解},\mathrm{则下列正确的是}()
$$
$$
A.
a=2,\;b=-3 \quad B.a\neq2,\;b\neq-3 \quad C.a=2,\;b\neq-3 \quad D.a\neq2,\;b=-3 \quad E. \quad F. \quad G. \quad H.
$$
$$
\begin{array}{l}\mathrm{对增广矩阵施以初等行变换},有\\\overline A=\begin{pmatrix}1&1&1&1&-1\\4&3&5&-1&-1\\a&1&3&b&1\end{pmatrix}\rightarrow\begin{pmatrix}1&1&1&1&-1\\0&-1&1&-5&3\\0&1-a&3-a&b-a&1+a\end{pmatrix}\rightarrow\begin{pmatrix}1&0&2&-4&2\\0&1&-1&5&-3\\0&0&4-2a&4a+b-5&4-2a\end{pmatrix}\\\;\;,\;\;因r\left(A\right)=2,故4-2a=0,4a+b-5=0即a=2,\;b=-3.\end{array}
$$



$$
设A为m× n\mathrm{阶矩阵},\mathrm{证明}:\mathrm{矩阵方程}YA=E_n\mathrm{有解的充要条件为}(\;\;).
$$
$$
A.
r\left(A\right)=n \quad B.r\left(A\right)\neq n \quad C.r\left(A\right)=m \quad D.r\left(A\right)\neq m \quad E. \quad F. \quad G. \quad H.
$$
$$
\begin{array}{l}\begin{array}{l}\mathrm{方程}YA=E_n\mathrm{有解}\Leftrightarrow\mathrm{方程}A^TY^T=E_n\mathrm{有解}\;\\\Leftrightarrow r(A^T)=n\Leftrightarrow r(A)=n\;\\\;注:当m=n,即A为n\mathrm{阶方阵时},\mathrm{显然}AX=E及YA=E\mathrm{有解}\Leftrightarrow r(A)=n,\mathrm{并有}X=Y=A^{-1};\end{array}\\\;\;当m\neq n时,\mathrm{按题设条件的解}X和Y\mathrm{不是惟一的}.\end{array}
$$



$$
\mathrm{已知矩阵}A=\begin{pmatrix}a_{11}&a_{12}&...&a_{1n}\\a_{21}&a_{22}&...&a_{2n}\\\vdots&\vdots&&\vdots\\a_{n1}&a_{n2}&...&a_{nn}\end{pmatrix}\mathrm{可逆},\mathrm{则线性方程组}\left\{\begin{array}{c}a_{11}x_1+a_{12}x_2+...+a_{1(n-1)}x_{n-1}=a_{1n}\\a_{21}x_1+a_{22}x_2+...+a_{2(n-1)}x_{n-1}=a_{2n}\\\vdots\\a_{n1}x_1+a_{n2}x_2+...+a_{n(n-1)}x_{n-1}=a_{nn}\end{array}\right.\mathrm{解的情况是}(\;\;\;\;\;\;\;)
$$
$$
A.
\mathrm{有解} \quad B.\mathrm{无解} \quad C.\mathrm{有无穷解} \quad D.\mathrm{有唯一解} \quad E. \quad F. \quad G. \quad H.
$$
$$
\begin{array}{l}\mathrm{设线性方程组的增广矩阵为}A,\mathrm{因为}A\mathrm{可逆},故R\left(A\right)=n,\mathrm{而该方程组的系数阵为}n×\left(n-1\right)\mathrm{矩阵},\mathrm{其秩最大}\\为n-1,\mathrm{所以系数阵的秩与增广阵的秩不等},\mathrm{故方程组无解}.\end{array}
$$



$$
\mathrm{若非齐次线性方程组}Ax=b\mathrm{的导出组为}Ax=0.\mathrm{如果}Ax=b\mathrm{有唯一解},则Ax=0\;\left(\right).\;
$$
$$
A.
\mathrm{必有无穷多组解} \quad B.\mathrm{必有唯一解} \quad C.\mathrm{必定无解} \quad D.\mathrm{解是任意的} \quad E. \quad F. \quad G. \quad H.
$$
$$
Ax=b\mathrm{有唯一解},则r\left(A\right)=r\left(A\;b\right)=n,则Ax=O\mathrm{仅有唯一零解}.
$$



$$
设A是m× n\mathrm{矩阵},\mathrm{非齐次线性方程组}Ax=b\mathrm{的导出组为}Ax=O.\mathrm{如果}m\;<\;n,则\left(\right).\;
$$
$$
A.
Ax=b\mathrm{必有无穷多组解} \quad B.Ax=b\mathrm{必有唯一解} \quad C.Ax=O\mathrm{必有非零解} \quad D.Ax=O\mathrm{必有唯一解} \quad E. \quad F. \quad G. \quad H.
$$
$$
\begin{array}{l}\mathrm{由题设可知}R\left(A\right)\leq\;min\left\{m,\;n\right\}\leq\;m<\;n,即R\left(A\right)<\;n,\mathrm{故齐次线性方程组}Ax=O\mathrm{必有非零解};\\\;\mathrm{但不能推出}R\left(A\right)=R\left(A\vert b\right),\mathrm{故不能判断非齐次线性方程组}Ax=b\mathrm{是否有解}.\end{array}
$$



$$
\mathrm{线性方程组}\left\{\begin{array}{c}x_1+x_2+2x_3+3x_4=1\\x_1+3x_2+6x_3+x_4=3\\3x_1-x_2-px_3+15x_4=3\\x_1-5x_2-10x_3+12x_4=t\end{array}\right.,当p,\;t\mathrm{取何值时},\mathrm{方程组有唯一解}?\mathrm{方程组无解}?\left(\right).\;
$$
$$
A.
p\neq2;\;p=2,\;t\neq1 \quad B.p\neq2;\;p=2,\;t=1 \quad C.p=2;\;p\neq2,\;t=1 \quad D.p=2;\;p\neq2,\;t\neq1 \quad E. \quad F. \quad G. \quad H.
$$
$$
\begin{array}{l}B=\begin{pmatrix}1&1&2&3&\vdots&1\\1&3&6&1&\vdots&3\\3&-1&-p&15&\vdots&3\\1&-5&-10&12&\vdots&t\end{pmatrix}\rightarrow\begin{pmatrix}1&1&2&3&\vdots&1\\0&2&4&-2&\vdots&2\\0&-4&-p-6&6&\vdots&0\\0&-6&-12&9&\vdots&t-1\end{pmatrix}\;\rightarrow\begin{pmatrix}1&1&2&3&\vdots&1\\0&1&2&-1&\vdots&1\\0&0&-p+2&2&\vdots&4\\0&0&0&3&\vdots&t+5\end{pmatrix}\;\;\;\;\;\;\;\\(1)当p\neq2时,R\left(A\right)=R\left(B\right)=4,\mathrm{方程组有唯一解};\;\\\;(2)当p=2时,有\;\\B=\begin{pmatrix}1&1&2&3&\vdots&1\\0&1&2&-1&\vdots&1\\0&0&0&2&\vdots&4\\0&0&0&3&\vdots&t+5\end{pmatrix}\rightarrow\begin{pmatrix}1&1&2&3&\vdots&1\\0&1&2&-1&\vdots&1\\0&0&0&1&\vdots&2\\0&0&0&0&\vdots&t-1\end{pmatrix}\;;\;\;\\当t\neq1时,R\left(A\right)=3<\;R\left(B\right)=4,\mathrm{方程组无解}.\end{array}
$$



$$
\mathrm{设线性方程组}\left\{\begin{array}{c}ax_1+x_2+x_3=1\\x_1+ax_2+x_3=a\\x_1+x_2+ax_3=a^2\end{array}\right.\mathrm{有唯一解},则a\mathrm{的值为}(\;).\;
$$
$$
A.
a\neq1且a\neq-2 \quad B.a=1 \quad C.a=-2 \quad D.a=1或a=-2 \quad E. \quad F. \quad G. \quad H.
$$
$$
\begin{array}{l}\mathrm{对增广矩阵施以初等行变换}\\\begin{pmatrix}a&1&1&1\\1&a&1&a\\1&1&a&a^2\end{pmatrix}\rightarrow\begin{pmatrix}1&1&a&a^2\\0&a-1&1-a&a-a^2\\0&1-a&1-a^2&1-a^3\end{pmatrix}\rightarrow\begin{pmatrix}1&1&a&a^2\\0&a-1&1-a&a-a^2\\0&0&\left(1-a\right)\left(2+a\right)&\left(1-a\right)\left(1+a\right)^2\end{pmatrix},\\\\当\left(1-a\right)\left(2+a\right)\neq0时,即a\neq1且a\neq-2时,R\left(A\right)=R\left(\widetilde A\right)=3,\mathrm{方程组有唯一解};\;\;\\当\left(1-a\right)\left(2+a\right)=\left(1-a\right)\left(1+a\right)^2时,\mathrm{即当}a=1时,R\left(A\right)=R\left(\widetilde A\right)=2,\mathrm{方程组无穷多解};\;\;\\当\left(1-a\right)\left(2+a\right)=0,且\left(1-a\right)\left(1+a\right)^2\neq0即a=-2时,R\left(A\right)=2\neq R\left(\widetilde A\right)=3,\mathrm{方程组无解}\end{array}
$$



$$
\mathrm{设线性方程组}\left\{\begin{array}{c}ax_1+x_2+x_3=1\\x_1+ax_2+x_3=a\\x_1+x_2+ax_3=a^2\end{array}\right.\mathrm{无解},则a\mathrm{的值为}(\;).\;
$$
$$
A.
a\neq1且a\neq-2 \quad B.a=1 \quad C.a=-2 \quad D.a=1或a=-2 \quad E. \quad F. \quad G. \quad H.
$$
$$
\begin{array}{l}\mathrm{对增广矩阵施以初等行变换}\\\begin{pmatrix}a&1&1&1\\1&a&1&a\\1&1&a&a^2\end{pmatrix}\rightarrow\begin{pmatrix}1&1&a&a^2\\0&a-1&1-a&a-a^2\\0&1-a&1-a^2&1-a^3\end{pmatrix}\rightarrow\begin{pmatrix}1&1&a&a^2\\0&a-1&1-a&a-a^2\\0&0&\left(1-a\right)\left(2+a\right)&\left(1-a\right)(1+a^2)\end{pmatrix},\\\\当\left(1-a\right)\left(2+a\right)\neq0时,即a\neq1且a\neq-2时,R\left(A\right)=R\left(\widetilde A\right)=3,\mathrm{方程组有唯一解};\;\;\\当\left(1-a\right)\left(2+a\right)=\left(1-a\right)\left(1+a\right)^2时,\mathrm{即当}a=1时,\mathrm{方程组无穷多解};\;\;\\当\left(1-a\right)\left(2+a\right)=0,且\left(1-a\right)(1+a^2)\neq0即a=-2时,R\left(A\right)=2\neq R\left(\widetilde A\right)=3,\mathrm{方程组无解}\end{array}
$$



$$
设n\mathrm{元非齐次线性方程组}Ax=b\mathrm{有解},\mathrm{其中}A为\left(n+1\right)× n\mathrm{矩阵},则\left|A,\;b\right|=\left(\right).
$$
$$
A.
\left|A,\;b\right|=0 \quad B.\left|A,\;b\right|=n \quad C.\left|A,\;b\right|=n+1 \quad D.\mathrm{不确定} \quad E. \quad F. \quad G. \quad H.
$$
$$
设Ax=b\mathrm{有解},故r\left(A\right)=r\left(A\;b\right).因A为(n+1)× n\mathrm{矩阵},\mathrm{所以}r\left(A\right)<\;n+1,\mathrm{于是}r\left(A\;b\right)<\;\;n+1,\mathrm{那么}\left(A\;b\right)\mathrm{非满秩},故\vert A,b\vert=0.
$$



$$
\begin{array}{l}\mathrm{假设线性方程组}\\\left\{\begin{array}{c}a_{11}x_1+a_{12}x_2+...+a_{1n}x_n=b_1\\a_{21}x_1+a_{22}x_2+...+a_{2n}x_n=b_2\\\vdots\\a_{n1}x_1+a_{n2}x_2+...+a_{nn}x_n=b_n\end{array}\right.\\\;\mathrm{的系数阵}A\mathrm{的秩等于矩阵}\\B=\begin{pmatrix}a_{11}&a_{12}&...&a_{1n}&b_1\\a_{21}&a_{22}&...&a_{2n}&b_2\\\vdots&\vdots&&\vdots&\vdots\\a_{n1}&a_{n2}&...&a_{nn}&b_n\\b_1&b_2&...&b_n&0\end{pmatrix}\\\;\mathrm{的秩},\mathrm{则该方程组}(\;\;\;\;\;\;\;\;).\end{array}
$$
$$
A.
\mathrm{有解} \quad B.\mathrm{无解} \quad C.\mathrm{有唯一解} \quad D.\mathrm{有无穷多解} \quad E. \quad F. \quad G. \quad H.
$$
$$
\mathrm{因为系数阵的秩不超过增广矩阵的秩},即R\left(A\right)\leq\;R\left(\overline A\right),而B比\overline A\mathrm{多一行},故R\left(\overline A\right)\leq\;R\left(B\right),\mathrm{再由假设}R\left(A\right)=R\left(B\right),得R\left(A\right)=R\left(\overline A\right),\mathrm{从而这个线性方程组有解}.
$$



$$
\mathrm{设线性方程组}\left\{\begin{array}{c}a_1x+b_1y+c_1z=d_1\\a_2x+b_2y+c_2z=d_2\\a_3x+b_3y+c_3z=d_3\\a_4x+b_4y+c_4z=d_4\end{array}\right.\mathrm{有解},则\begin{vmatrix}a_1&b_1&c_1&d_1\\a_2&b_2&c_2&d_2\\a_3&b_3&c_3&d_3\\a_4&b_4&c_4&d_4\end{vmatrix}.
$$
$$
A.
\mathrm{等于零} \quad B.\mathrm{不等于零} \quad C.\mathrm{大于零} \quad D.\mathrm{小于零} \quad E. \quad F. \quad G. \quad H.
$$
$$
记A=\begin{pmatrix}a_1&b_1&c_1\\a_2&b_2&c_2\\a_3&b_3&c_3\\a_4&b_4&c_4\end{pmatrix},\;\;\widetilde A\;=\begin{pmatrix}a_1&b_1&c_1&d_1\\a_2&b_2&c_2&d_2\\a_3&b_3&c_3&d_3\\a_4&b_4&c_4&d_4\end{pmatrix}\mathrm{因已知方程组有解},故R\left(A\right)=R\left(\widetilde A\right),\mathrm{但由于}R\left(A\right)\leq\;3,故\left|\widetilde A\right|=0.
$$



$$
设A,\;B是n\mathrm{阶方阵},x,\;y,\;b是n×1\mathrm{矩阵},\mathrm{则方程组}\begin{pmatrix}A&\mathrm O\\\mathrm O&B\end{pmatrix}\begin{pmatrix}x\\y\end{pmatrix}=\begin{pmatrix}0\\b\end{pmatrix}\mathrm{无解的充要条件}\left(\right).\;
$$
$$
A.
r\left(A\right)=r\left(A\vdots b\right),\;r\left(B\right)\mathrm{任意}; \quad B.r\left(B\right)\neq r\left(B\vdots b\right),\;r\left(A\right)\mathrm{任意}; \quad C.By=b\mathrm{有解},Ax=0\mathrm{有非零解} \quad D.By=b\mathrm{有解},Ax=0\mathrm{只有零解} \quad E. \quad F. \quad G. \quad H.
$$
$$
\begin{array}{l}\begin{pmatrix}A&\mathrm O\\\mathrm O&B\end{pmatrix}\begin{pmatrix}x\\y\end{pmatrix}=\begin{pmatrix}0\\b\end{pmatrix}\mathrm{有解的充要条件是}By=b,\;Ax=0\mathrm{同时有解}.\;\mathrm{由于齐次线性方程组}Ax=0\mathrm{总是有解的},\mathrm{所以}r\left(A\right)\mathrm{任意};\\\mathrm{非齐次方程组}By=b\mathrm{无解的充要条件是}r(B)\neq r(B\vdots b)\end{array}
$$



$$
设A,\;B是n\mathrm{阶方阵},x,\;y,\;b是n×1\mathrm{矩阵},\mathrm{则方程组}\begin{pmatrix}A&\mathrm O\\\mathrm O&B\end{pmatrix}\begin{pmatrix}x\\y\end{pmatrix}=\begin{pmatrix}0\\b\end{pmatrix}\mathrm{有解的充要条件}\left(\right).\;
$$
$$
A.
r\left(A\right)=r\left(A\vdots b\right),\;r\left(B\right)\mathrm{任意}; \quad B.r\left(B\right)=r\left(B\vdots b\right),\;r\left(A\right)\mathrm{任意}; \quad C.By=b\mathrm{有解},Ax=0\mathrm{有非零解} \quad D.By=b\mathrm{有解},Ax=0\mathrm{只有零解} \quad E. \quad F. \quad G. \quad H.
$$
$$
\begin{array}{l}\begin{pmatrix}A&\mathrm O\\\mathrm O&B\end{pmatrix}\begin{pmatrix}x\\y\end{pmatrix}=\begin{pmatrix}0\\b\end{pmatrix}\mathrm{有解的充要条件是}By=b,\;Ax=0\mathrm{同时有解}.\;\mathrm{由于齐次线性方程组}Ax=0\mathrm{总是有解的},\mathrm{所以}r\left(A\right)\mathrm{任意};\\\mathrm{非齐次方程组}By=b\mathrm{有解的充要条件是}r(B)=r(B\vdots b)\mathrm{故选}B.\end{array}
$$



$$
设A,\;B是n\mathrm{阶方阵},x,\;y,\;b是n×1\mathrm{矩阵},\mathrm{则方程组}\begin{pmatrix}A&\mathrm O\\\mathrm O&B\end{pmatrix}\begin{pmatrix}x\\y\end{pmatrix}=\begin{pmatrix}0\\b\end{pmatrix}\mathrm{有解的充要条件}\left(\right).\;
$$
$$
A.
r\left(A\right)=r\left(A\vdots b\right),\;r\left(B\right)\mathrm{任意}; \quad B.By=b\mathrm{有解},Ax=0\mathrm{有非零解} \quad C.\left|A\right|\neq0,\;b\mathrm{可由}B\mathrm{的列向量线性表出} \quad D.r\left(A\right)\mathrm{任意},b\mathrm{可由}B\mathrm{的列向量线性表出} \quad E. \quad F. \quad G. \quad H.
$$
$$
\begin{array}{l}\begin{pmatrix}A&\mathrm O\\\mathrm O&B\end{pmatrix}\begin{pmatrix}x\\y\end{pmatrix}=\begin{pmatrix}0\\b\end{pmatrix}\mathrm{有解的充要条件是}By=b,\;Ax=0\mathrm{同时有解}.\;\mathrm{由于齐次线性方程组}Ax=0\mathrm{总是有解的},\mathrm{所以}r\left(A\right)\mathrm{任意};\\\mathrm{又非齐次方程组}By=b\mathrm{有解等价于}b\mathrm{可由}B\mathrm{的列向量线性表出},\mathrm{故选}D.\end{array}
$$



$$
n\mathrm{元线性非齐次方程组}Ax=b\mathrm{有唯一解的充分必要条件是}\left(\right).\;
$$
$$
A.
秩\left(A\right)=n \quad B.\left|A\right|\neq0 \quad C.秩\left(A\;b\right)=n \quad D.秩\left(A\right)=n,且b\mathrm{可由}A\mathrm{的列向量线性表示}. \quad E. \quad F. \quad G. \quad H.
$$
$$
\begin{array}{l}\mathrm{根据线性方程组解的判断定理可知}:n\mathrm{元线性非齐次方程组}Ax=b\mathrm{有唯一解的充分必要条件}r\left(A\right)=r\left(A\;b\right)=n,\\则\left(A\;b\right)\mathrm{的列向量组线性相关},故b\mathrm{可由}A\mathrm{的列向量线性表示}.\end{array}
$$



$$
\mathrm{非齐次线性方程组}\left\{\begin{array}{c}x_1+x_2+x_3=1\\ax_1+bx_2+cx_3=d\\a^2x_1+b^2x_2+c^2x_3=d^2\\a^3x_1+b^3x_2+c^3x_3=d^3\end{array}\right.:(\mathrm{其中}a,\;b,\;c,\;d\mathrm{两两不等})\mathrm{解的情况是}(\;\;\;).\;
$$
$$
A.
\mathrm{有解} \quad B.\mathrm{无解} \quad C.\mathrm{唯一解} \quad D.\mathrm{无穷多解} \quad E. \quad F. \quad G. \quad H.
$$
$$
\begin{array}{l}\mathrm{方程组的增广矩阵}\widetilde A\mathrm{的四阶子式是范德蒙行列式}:\\\begin{vmatrix}1&1&1&1\\a&b&c&d\\a^2&b^2&c^2&d^2\\a^3&b^3&c^3&d^3\end{vmatrix}=\left(b-a\right)\left(c-a\right)\left(d-a\right)\left(c-b\right)\left(d-b\right)\left(d-c\right)\\\;\mathrm{已知}a,\;b,\;c,\;d\mathrm{各不相同},\mathrm{所以此行列式不等于零},\mathrm{于是}R\left(\widetilde A\right)=4.\mathrm{而系数矩阵}A\mathrm{只有三列},R\left(A\right)=3,\mathrm{因此}R\left(\widetilde A\right)\neq R\left(A\right).\mathrm{从而此线性方程组无解}.\end{array}
$$



$$
\begin{array}{l}\mathrm{设线性方程组}\left\{\begin{array}{c}x+ay+a^2z=a^3\\x+by+b^2z=b^3\\x+cy+c^2z=c^3\end{array}\right.,\mathrm{下列说法正确的个数是}(\;\;\;)\;\\\\(1)当a\neq b\neq c时,R\left(A\right)=R\left(\overline A\right)=3\mathrm{方程组有唯一解};\;\;\\(2)当a\neq b=c时,R\left(A\right)=R\left(\overline A\right)=2,\mathrm{方程组有无穷多解},\mathrm{基础解系中解向量个数为}1;\;\;\\(3)当a=b\neq c时,R\left(A\right)=R\left(\overline A\right)=2,\mathrm{方程组有无穷多解},\mathrm{基础解系中解向量个数为}1;\;\;\\(4)当a=c\neq b时,R\left(A\right)=R\left(\overline A\right)=2,\mathrm{方程组有无穷多解},\mathrm{基础解系中解向量个数为}1;\;\;\\(5)当a=b=c时,R\left(A\right)=R\left(\overline A\right)=1\mathrm{有无穷多解},\mathrm{基础解系中解向量个数为}2.\end{array}
$$
$$
A.
5 \quad B.4 \quad C.3 \quad D.2 \quad E. \quad F. \quad G. \quad H.
$$
$$
\begin{array}{l}\mathrm{因为}\\\left|A\right|=\begin{vmatrix}1&a&a^2\\1&b&b^2\\1&c&c^2\end{vmatrix}=\left(a-b\right)\left(b-c\right)\left(c-a\right),\;\;\\当a\neq b\neq c时,\mathrm{方程组有唯一解};\;\;\\当a,\;b,\;c\mathrm{中有两个量相等时},R\left(A\right)=R\left(\overline A\right)=2,\mathrm{方程组有无穷多解},\mathrm{基础解系中解向量个数为}1;\;\;\\当a=b=c时,R\left(A\right)=R\left(\overline A\right)=1\mathrm{仍有无穷多解},\mathrm{基础解系中解向量个数为}2.\end{array}
$$



$$
\begin{array}{l}设\;\\\left\{\begin{array}{c}\left(2-λ\right)x_1+2x_2-2x_3=1\\2x_1+\left(5-λ\right)x_2-4x_3=2\\-2x_1-4x_2+\left(5-λ\right)x_3=-λ-1\end{array}\right.,\;\;\\\mathrm{若方程组有}2\mathrm{个解}η_1和η_2\mathrm{满足}η_1-η_2\neq0,则λ\mathrm{的值为}(\;).\end{array}
$$
$$
A.
λ=10 \quad B.λ=1 \quad C.λ\neq1,\;λ\neq10 \quad D.λ=1或λ=10 \quad E. \quad F. \quad G. \quad H.
$$
$$
\begin{array}{l}\mathrm{系数矩阵的行列式}\\\left|A\right|=\begin{vmatrix}2-λ&2&-2\\2&5-λ&-4\\-2&-4&5-λ\end{vmatrix}=-\left(λ-1\right)^2\left(λ-10\right)\\当λ\neq1,λ\neq10时,\mathrm{方程组有唯一解}.\;\;\\当λ=10时,\\B=\begin{pmatrix}-8&2&-2&1\\2&-5&-4&2\\-2&-4&-5&-11\end{pmatrix}∼\begin{pmatrix}2&-5&-4&2\\0&-18&-18&9\\0&-9&-9&-9\end{pmatrix}∼\begin{pmatrix}2&-5&-4&2\\0&1&1&1\\0&0&0&1\end{pmatrix},R\left(B\right)=3,\;R\left(A\right)=2\mathrm{故方程组无解}.\;\;\\当λ=1时,\\B=\begin{pmatrix}1&2&-2&1\\2&4&-4&2\\-2&-4&4&2\end{pmatrix}∼\begin{pmatrix}1&2&-2&1\\0&0&0&0\\0&0&0&0\end{pmatrix},\;\;R\left(B\right)=R\left(A\right)=1,\mathrm{方程组有无穷多解}.\end{array}
$$



$$
\mathrm{设线性方程组}\left\{\begin{array}{c}ax_1+x_2+x_3=1\\x_1+ax_2+x_3=a\\x_1+x_2+ax_3=a^2\end{array}\right.有4\mathrm{个互不相同的解},则a\mathrm{的值为}(\;\;\;\;\;).\;
$$
$$
A.
a\neq1且a\neq-2 \quad B.a=1 \quad C.a=-2 \quad D.a=1或a=-2 \quad E. \quad F. \quad G. \quad H.
$$
$$
\begin{array}{l}\mathrm{对增广矩阵施以初等行变换}\\\begin{pmatrix}a&1&1&1\\1&a&1&a\\1&1&a&a^2\end{pmatrix}\rightarrow\begin{pmatrix}1&1&a&a^2\\0&a-1&1-a&a-a^2\\0&1-a&1-a^2&1-a^3\end{pmatrix}\rightarrow\begin{pmatrix}1&1&a&a^2\\0&a-1&1-a&a-a^2\\0&0&\left(1-a\right)\left(2+a\right)&\left(1-a\right)\left(1+a\right)^2\end{pmatrix},\;\;\\当\left(1-a\right)\left(2+a\right)\neq0时,即a\neq1且a\neq-2时,r\left(A\right)=r\left(\widetilde A\right)=3,\mathrm{方程组有唯一解};\;\;\\当\left(1-a\right)\left(2+a\right)=\left(1-a\right)\left(1+a\right)^2=0时,\mathrm{即当}a=1时,r\left(A\right)=r\left(\widetilde A\right)=2,\mathrm{方程组无穷多解};\;\;\\当\left(1-a\right)\left(2+a\right)=0且\left(1-a\right)\left(1+a\right)^2\neq0,即a=-2时,r\left(A\right)=2\neq r\left(\widetilde A\right)=3,\mathrm{方程组无解}\end{array}
$$



$$
\mathrm{对任意正整数}M,\mathrm{方程组}\left\{\begin{array}{c}3ax+\left(2a+1\right)y+\left(a+1\right)z=a\\\left(4a-1\right)x+3ay+2az=1\\\left(2a-1\right)x+\left(2a+1\right)y+\left(a+1\right)z=a+1\end{array}\right.\mathrm{都能找到}M\mathrm{个互异的解},则a\mathrm{的值为}(\;).\;
$$
$$
A.
a=0或a=1 \quad B.a\neq0,\;a\neq±1 \quad C.a=-1 \quad D.a\neq0 \quad E. \quad F. \quad G. \quad H.
$$
$$
\begin{array}{l}\mathrm{系数矩阵的行列式}\\\left|A\right|=\begin{vmatrix}3a&2a+1&a+1\\4a-1&3a&2a\\2a-1&2a+1&a+1\end{vmatrix}=a\left(a+1\right)\left(a-1\right),\;\;\\当a\neq0,a\neq±1时,\mathrm{原方程组有唯一解}.\;\;\\当a=0时,\widetilde A=\begin{pmatrix}0&1&1&0\\-1&0&0&1\\-1&1&1&1\end{pmatrix}∼\begin{pmatrix}0&1&1&0\\1&0&0&-1\\0&0&0&0\end{pmatrix},\mathrm{原方程组有无穷多组解}.\;\;\\当a=1时,\widetilde A=\begin{pmatrix}3&3&2&1\\3&3&2&1\\1&3&2&2\end{pmatrix}∼\begin{pmatrix}2&0&0&-1\\0&-6&-4&-5\\0&0&0&0\end{pmatrix},\mathrm{原方程组有无穷多组解}.\;\;\\当a=-1时,\;\;\widetilde A=\begin{pmatrix}-3&-1&0&-1\\-5&-3&-2&1\\-3&-1&0&0\end{pmatrix}∼\begin{pmatrix}3&1&0&1\\-5&-3&-2&1\\0&0&0&1\end{pmatrix},\;\;R\left(A\right)=2,\;R\left(\widetilde A\right)=3.\mathrm{原方程组无解}.\end{array}
$$



$$
\mathrm{方程组}\left\{\begin{array}{c}x_1+2x_3+4x_4=a+2c\\2x_1+2x_2+4x_3+8x_4=2a+b\\-x_1-2x_2+x_3+2x_4=-a-b+c\\2x_1+7x_3+14x_4=3a+b+2c-d\end{array}\right.\mathrm{有解的充要条件是}(\;\;\;\;).
$$
$$
A.
a+b-c-d=0 \quad B.a+b+c+d=0 \quad C.a+b-c+d=0 \quad D.a+b+c-d=0 \quad E. \quad F. \quad G. \quad H.
$$
$$
\overline A\xrightarrow{\mathrm{经初等行变换}}\begin{pmatrix}1&0&2&4&a+2c\\0&2&0&0&b-4c\\0&0&3&6&-c\\0&0&0&0&a+b-c-d\end{pmatrix},\;\;而R\left(A\right)=R\left(\overline A\right)\mathrm{的充要条件是}a+b-c-d=0,\mathrm{故原方程组有解的充要条件为}a+b-c-d=0
$$



$$
\mathrm{线性方程组}\left\{\begin{array}{c}x_1+x_2+2x_3+3x_4=1\\x_1+3x_2+6x_3+x_4=3\\3x_1-x_2-px_3+15x_4=3\\x_1-5x_2-10x_3+12x_4=t\end{array}\right.,\mathrm{方程组有唯一解},则p,\;t为(\;\;\;\;)\;
$$
$$
A.
p\neq2,\;t\mathrm{任意} \quad B.p=2,t\mathrm{任意} \quad C.p=2,t=5 \quad D.p=2,t=-5 \quad E. \quad F. \quad G. \quad H.
$$
$$
\begin{array}{l}B=\begin{pmatrix}1&1&2&3&\vdots&1\\1&3&6&1&\vdots&3\\3&-1&-p&15&\vdots&3\\1&-5&-10&12&\vdots&t\end{pmatrix}\rightarrow\begin{pmatrix}1&1&2&3&\vdots&1\\0&2&4&-2&\vdots&2\\0&-4&-p-6&6&\vdots&0\\0&-6&-12&9&\vdots&t-1\end{pmatrix}\;\rightarrow\begin{pmatrix}1&1&2&3&\vdots&1\\0&1&2&-1&\vdots&1\\0&0&-p+2&2&\vdots&4\\0&0&0&3&\vdots&t+5\end{pmatrix}\;\;\;\\当p\neq2时,r\left(A\right)=r\left(B\right)=4,\mathrm{方程组有唯一解};\end{array}
$$



$$
\mathrm{方程组}\left\{\begin{array}{c}2x_1+(4-λ)x_2+7=0\\(2-λ)x_1+2x_2+3=0\\2x_1+5x_2+6-λ=0\end{array}\right.,当λ\mathrm{为何值时无解}(\;\;\;).
$$
$$
A.
\;λ=-1 \quad B.\;λ=1 \quad C.\;λ=12 \quad D.\;λ\neq1,λ\neq12,\;\;λ\neq-1 \quad E. \quad F. \quad G. \quad H.
$$
$$
\begin{array}{l}\;\;\overline A=\begin{pmatrix}2&4-λ&\vdots&-7\\2-λ&2&\vdots&-3\\2&5&\vdots&-6+λ\end{pmatrix}\rightarrow\begin{pmatrix}2&5&λ-6\\0&λ+1&λ+1\\0&5λ-6&λ^2-8λ+6\end{pmatrix}\\\;\;(1)当λ=-1时,\overline A=\begin{pmatrix}2&5&-7\\0&-11&15\\0&0&0\end{pmatrix}\rightarrow\begin{pmatrix}1&0&-\frac1{11}\\0&1&-\frac{15}{11}\\0&0&0\end{pmatrix},\mathrm{得唯一解}:x_1=-\frac1{11},x_2=-\frac{15}{11}.\\\;\;(2)当λ\neq-1时,\overline A=\begin{pmatrix}2&5&\vdots&λ-6\\0&1&\vdots&1\\0&0&\vdots&λ^2-13λ+12\end{pmatrix},\mathrm{故当}λ^2-13λ+12=0,即λ=1或λ=12时,\mathrm{方程组有解}.\end{array}
$$



$$
\mathrm{若线性方程组}\left\{\begin{array}{c}\left(λ+3\right)x_1+x_2+2x_3=λ\\λ x_1+\left(λ-1\right)x_2+x_3=λ\\3\left(λ+1\right)x_1+λ x_2+\left(λ+3\right)x_3=3\end{array}\right.\;有2\mathrm{个互不相同的解},则λ 为\left(\right).\;
$$
$$
A.
λ=1 \quad B.λ=0 \quad C.λ\neq1 \quad D.λ=-1 \quad E. \quad F. \quad G. \quad H.
$$
$$
\begin{array}{l}\mathrm{方程组的系数行列式为}\\\begin{vmatrix}λ+3&1&2\\λ&λ-1&1\\3\left(λ+1\right)&λ&λ+3\end{vmatrix}=λ^2\left(λ-1\right)\\\mathrm{则当}λ\neq0且λ\neq1时,\mathrm{方程组有唯一解};\;\;\\当λ=0时,\mathrm{方程组的增广矩阵作初等行变换}:\\\widetilde A\;=\begin{pmatrix}3&1&2&0\\0&-1&1&0\\3&0&3&3\end{pmatrix}\xrightarrow{r_3-r_1}\begin{pmatrix}3&1&2&0\\0&-1&1&0\\0&-1&1&3\end{pmatrix}\xrightarrow{r_3-r_2}\begin{pmatrix}3&1&2&0\\0&-1&1&0\\0&0&0&3\end{pmatrix}.\;\;\\因r\left(A\right)=2,\;r\left(\widetilde A\right)\;=3,\;\mathrm{所以方程组无解};\;\;\\当λ=1时,\mathrm{增广矩阵为}\\\widetilde A\;=\begin{pmatrix}4&1&2&1\\1&0&1&1\\6&1&4&3\end{pmatrix}\xrightarrow[{r_1↔ r_2}]{r_3-r_1}\begin{pmatrix}1&0&1&1\\4&1&2&1\\2&0&2&2\end{pmatrix}\xrightarrow[{r3-2r_1}]{r_3-4r_1}\begin{pmatrix}1&0&1&1\\0&1&-2&-3\\0&0&0&0\end{pmatrix}.\;\;\\因\;r\left(A\right)=r\left(\widetilde A\right)=2<\;3\mathrm{所以方程组有无穷多个解}.\end{array}
$$



$$
\mathrm{若方程组}A_{m× n}x=b\left(m\leq\;n\right)\mathrm{对于任意}m\mathrm{维列向量}b\mathrm{都有解},则\left(\right).\;
$$
$$
A.
r\left(A\right)=n \quad B.r\left(A\right)=m \quad C.r\left(A\right)>\;n \quad D.r\left(A\right)<\;m \quad E. \quad F. \quad G. \quad H.
$$
$$
\begin{array}{l}\mathrm{由题设可知对任意一个常数项矩阵}b,\mathrm{方程组}Ax=b\mathrm{均有解},则r(A)=r(A\;b)\leq\;n.\;\;\\当r(A)\;<\;m时,\mathrm{增加任意一列},\mathrm{其秩可能改变};\mathrm{当且仅当}r(A)\;=\;m时,\mathrm{增加一列其秩不变},且r(A)=r(A\;b)=m.\end{array}
$$



$$
\mathrm{设非齐次线性方程组}Ax=b中,\mathrm{系数矩阵}A={\left(a_{ij}\right)}_{m× n},且r\left(A\right)=r,则\left(\right).\;
$$
$$
A.
当m=n时,\mathrm{方程组}Ax=b\mathrm{有惟一解} \quad B.当r=n时,\mathrm{方程组}Ax=b\mathrm{有惟一解} \quad C.当r<\;n时,\mathrm{方程组有无穷多解} \quad D.当r=m时,\mathrm{方程组}Ax=b\mathrm{有解} \quad E. \quad F. \quad G. \quad H.
$$
$$
\begin{array}{l}\mathrm{对于方程组}Ax=b,记\left(A\;b\right)=B,则\\r\left(A\right)=r\left(B\right)=n\Leftrightarrow Ax=b\mathrm{有唯一解};\;\;\\r(A)=r(B)<\;n\Leftrightarrow Ax=b\mathrm{有无穷多解};\;\;\\r(A)\neq r(B)\Leftrightarrow Ax=b\mathrm{无解}.\;\;\\\mathrm{对于方程组}Ax=b,r(A)\leq min\left\{m,\;n\right\}\;,且r(B)\leq min\left\{m,\;n+1\right\},当r=m时,r(A)=r(B)=m,\mathrm{方程组}Ax=b\mathrm{有解};\;\;\\\mathrm{因为不能确定}m,\;n\mathrm{是否相等},\mathrm{所以不能确定是唯一解还是无穷解}.\end{array}
$$



$$
\begin{array}{l}设A为n\mathrm{阶方阵},且\left|A\right|=0,\mathrm{对于非齐次线性方程组}Ax=b,若D_1,\;\;D_2,\;...,\;D_n中(D_J\mathrm{表示将}\left|A\right|\mathrm{中第}j\mathrm{列换为}b\mathrm{后得到的行列式})\\\mathrm{至少有一个不等于零},\mathrm{则该方程组}\left(\right).\end{array}
$$
$$
A.
\mathrm{无解} \quad B.\mathrm{尚不能确定是否有解} \quad C.\mathrm{有唯一解} \quad D.\mathrm{有无穷多解} \quad E. \quad F. \quad G. \quad H.
$$
$$
\begin{array}{l}A为n\mathrm{阶方阵},且\vert A\vert=0,则r\left(A\right)<\;n;\;\\又D_j\mathrm{中至少有一个不等于零},\mathrm{则矩阵}A\mathrm{中存在}n-1\mathrm{个列向量与向量}b\mathrm{构成的向量组线性无关},\\即r\left(A\;b\right)\geq\;\;n,又r\left(A\right)<\;min\left\{n,n+1\right\},故r\left(A\;b\right)=n;\;由r\left(A\;b\right)\neq r\left(A\right)\mathrm{可得非齐次线性方程组}Ax=b\mathrm{无解}.\end{array}
$$



$$
\begin{array}{l}\begin{array}{l}设A为m× n\mathrm{矩阵},\mathrm{则下列命题正确的有}\left(\right).\;\\\;\left(1\right)\mathrm{矩阵方程}AX=E_m\mathrm{有解的充要条件为}r\left(A\right)=m;\;\;\\\;\left(2\right)\mathrm{矩阵方程}YA=E_n\mathrm{有解的充要条件为}r\left(A\right)=n;\;\;\;\end{array}\\\;\;\left(3\right)若AX=AY\;且r\left(A\right)=n\;则X=Y.\end{array}
$$
$$
A.
0个 \quad B.1个 \quad C.2个 \quad D.3个 \quad E. \quad F. \quad G. \quad H.
$$
$$
\begin{array}{l}\left(1\right)\mathrm{方阵}AX=E_m\mathrm{有解}\Leftrightarrow r(A)\;=r(A,\;E_m)\;\Leftrightarrow\;r(A)\;=m\;\;\\(\mathrm{必要性由不等式得到}m\leq\;r(A,\;E_m)\;=\;r(A)\;\leq\;m;\;\;\;\;\;\;\;\;\\\;\;\;\;\;\;\mathrm{充分性由不等式得到}m=\;\;r(A)\leq\;\;r(A,\;E_m)\;\leq\;m).\;\;\\\left(2\right)\mathrm{方程}YA=En\mathrm{有解}\Leftrightarrow\mathrm{方程}A^TY^T=E_n\mathrm{有解}\;\Leftrightarrow\;\;\;\;r(A^T)\;=n\;\;\;\;(由(1))\;\;\Leftrightarrow\;\;\;\;r(A)\;=n\;(\mathrm{矩阵秩的性质}②).\;\\\left(3\right)\;设X,\;Y为n× s\mathrm{阶矩阵}.X=\left(a_1,\;...,\;a_s\right)\;,\;Y=\left(β_1,\;...,\;β_s\right)\;.\;\;\\由\;AX=AY⇒ A\left(X-Y\right)=0⇒\;A\left(a_j-β_j\right)\;=0\;\left(j=1,\;2,\;...,\;s\right)\;\\∵ r\left(A\right)=n.\mathrm{方程组}A\left(a_j-β_j\right)\;=0,\;\mathrm{只有零解}⇒ a_j=β_j\;\;\;\left(j=1,\;2,\;...,\;s\right)\\∴ X=Y\\\;\mathrm{所以}\left(1\right)\left(2\right)\left(3\right)\mathrm{都正确}.\end{array}
$$



$$
设A为m× n\mathrm{矩阵},B为m× s,\mathrm{已知矩阵方程}AX=B\mathrm{有解},\mathrm{则必有}\left(\right).\;
$$
$$
A.
r\left(A\right)\leq\;r\left(B\right) \quad B.r\left(A\right)\geq\;r\left(B\right) \quad C.r\left(A\right)>\;0 \quad D.r\left(B\right)>\;0 \quad E. \quad F. \quad G. \quad H.
$$
$$
\begin{array}{l}\begin{array}{l}\mathrm{由非齐次线性方程组解的判断定理可知}:\mathrm{矩阵方程}AX=B\mathrm{有解},则r\left(A\right)=r\left(A,\;\;B\right);\\\mathrm{又由矩阵秩的性质可知}r\left(A,\;\;B\right)\geq\;max\left\{r\left(A\right),\;r\left(B\right)\right\},故r(A)\geq r(B).\end{array}\\\end{array}
$$



$$
\mathrm{方程组}\left\{\begin{array}{c}2x_1+(4-λ)x_2+7=0\\(2-λ)x_1+2x_2+3=0\\2x_1+5x_2+6-λ=0\end{array}\right.\mathrm{在下列哪种}λ\mathrm{值时无解}(\;\;\;).
$$
$$
A.
\;λ=-1 \quad B.\;λ=1 \quad C.\;λ=12 \quad D.\;λ=-12 \quad E. \quad F. \quad G. \quad H.
$$
$$
\begin{array}{l}\;\;\overline A=\begin{pmatrix}2&4-λ&\vdots&-7\\2-λ&2&\vdots&-3\\2&5&\vdots&-6+λ\end{pmatrix}\rightarrow\begin{pmatrix}2&5&λ-6\\0&λ+1&λ+1\\0&5λ-6&λ^2-8λ+6\end{pmatrix}\\\;\;(1)当λ=-1时,\overline A=\begin{pmatrix}2&5&-7\\0&-11&15\\0&0&0\end{pmatrix}\rightarrow\begin{pmatrix}1&0&-\frac1{11}\\0&1&-\frac{15}{11}\\0&0&0\end{pmatrix},\mathrm{得唯一解}:x_1=-\frac1{11},x_2=-\frac{15}{11}.\\\;\;(2)当λ\neq-1时,\overline A=\begin{pmatrix}2&5&\vdots&λ-6\\0&1&\vdots&1\\0&0&\vdots&λ^2-13λ+12\end{pmatrix},\mathrm{故当}λ^2-13λ+12=0,即λ=1或λ=12时,\mathrm{方程组有解}.\end{array}
$$



$$
设A=\begin{pmatrix}1&2&3\\0&1&2\\2&1&1\end{pmatrix},\mathrm{则使得方程组}Ax=b\mathrm{有解的所有向量}b是(\;\;\;).\;
$$
$$
A.
\mathrm{任意向量} \quad B.\mathrm{单位向量} \quad C.\mathrm{线性无关的向量} \quad D.\mathrm{线性相关的向量} \quad E. \quad F. \quad G. \quad H.
$$
$$
\mathrm{由于}\left|A\right|=\begin{vmatrix}1&2&3\\0&1&2\\2&1&1\end{vmatrix}=1\neq0,故r\left(A\right)=3,\mathrm{因此}min\left\{3,\;4\right\}\geq\;r\left(A\;b\right)\geq\;\;r\left(A\right)=3,则\;r\left(A\;b\right)=\;\;r\left(A\right)=3,\mathrm{所以}b\mathrm{可以为任意向量}.
$$



$$
\mathrm{线性方程组}\left\{\begin{array}{c}a_{11}x_1+...+a_{1n}x_n=b_1\\\vdots\\a_{n1}x_1+...+a_{nn}x_n=b_n\end{array}\right.\mathrm{对任何}b_1,\;b_{2,}\;...,\;b_n\mathrm{都有解的充要条件是}(\;\;\;\;)\;
$$
$$
A.
\mathrm{系数行列式}\begin{vmatrix}a_{11}&...&a_{1n}\\\vdots&\vdots&\vdots\\a_{n1}&...&a_{nn}\end{vmatrix}\neq0 \quad B.\mathrm{系数行列式}\begin{vmatrix}a_{11}&...&a_{1n}\\\vdots&\vdots&\vdots\\a_{n1}&...&a_{nn}\end{vmatrix}=0 \quad C.\mathrm{系数矩阵的秩等于增广矩阵的秩} \quad D.\mathrm{系数矩阵的秩小于增广矩阵的秩} \quad E. \quad F. \quad G. \quad H.
$$
$$
\begin{array}{l}\mathrm{必要性}:(\mathrm{反证法})\;\;\\设\begin{vmatrix}a_{11}&...&a_{1n}\\\vdots&\vdots&\vdots\\a_{n1}&...&a_{nn}\end{vmatrix}=0,\mathrm{则其行向量组}a_i=\left(a_{i1},\;...,\;a_{in}\right)\left(i=1,\;2,\;...,\;n\right)\mathrm{必线性相关},\mathrm{不妨设}a_n\mathrm{可以由}a_1,\;...,\;a_{n-1}\mathrm{线性表出},\\即\;\;a_n=k_1a_1+...+k_{n-1}a_{n-1},\;\;\mathrm{此时若取}b_n=-k_1b_1-k_2b_2-...-k_{n-1}b_{n-1}+1,\mathrm{则增广阵可化为}:\\\overline A\;\;=\begin{pmatrix}a_1&b_1\\\vdots&\vdots\\a_n&b_n\end{pmatrix}\rightarrow\begin{pmatrix}a_1&b_1\\\vdots&\vdots\\a_{n-1}&b_{n-1}\\0&1\end{pmatrix},\;\;\mathrm{即得到秩}\left(\overline A\;\right)\neq 秩\left(A\right),\\\mathrm{故方程组无解},\mathrm{这与已知矛盾},\mathrm{假设不成立}.\;\;\mathrm{充分性}:\mathrm{由克莱姆法则即可得到}.\end{array}
$$



$$
A是n\mathrm{阶矩阵},\mathrm{对于齐次线性方程组}Ax=O,如R(A)=n-1,\mathrm{且代数余子式}A_{11}\neq0,则Ax=O\mathrm{的通解是}(\;\;\;\;)
$$
$$
A.
\;k(A_{11},A_{12},⋯,A_{1n})^T,\;(k∈ R) \quad B.\;\;k(A_{11},A_{21},⋯,A_{n1})^T,\;\;(k∈ R) \quad C.\;k(1,1,⋯,1)^T,\;(k∈ R) \quad D.\;\mathrm{以上都不对} \quad E. \quad F. \quad G. \quad H.
$$
$$
\begin{array}{l}\;\;对Ax=0,从R(A)=n-1\mathrm{知解空间是}1\mathrm{维的},且\left|A\right|=0.\mathrm{因为}AA^*=O,A^*\mathrm{的每一列都是}Ax=0\mathrm{的解},\mathrm{现已知}\\\;\;A_{11}\neq0,故(A_{11},A_{12},⋯,A_{1n})^T是Ax=0\mathrm{的非零解},\mathrm{即是基础解系},\mathrm{所以通解是}\;k(A_{11},A_{12},⋯,A_{1n})^T,(k∈ R)\end{array}
$$



$$
\begin{array}{l}设P_1\left(x_1,\;y_1,\;z_1\right),\;P_2\left(x_2,\;y_2,\;z_2\right),\;P_3\left(x_3,\;y_3,\;z_3\right),\;P_4\left(x_4,\;y_4,\;z_4\right),\mathrm{是空间共面的四点},\mathrm{若记}\\A=\begin{pmatrix}x_1&y_1&z_1&1\\x_2&y_2&z_2&1\\x_3&y_3&z_3&1\\x_4&y_4&z_4&1\end{pmatrix}\\,\;\;则A\mathrm{的秩}R\left(A\right)\mathrm{满足}(\;\;\;\;\;\;\;\;)\;\end{array}
$$
$$
A.
R\left(A\right)\leq\;3 \quad B.R\left(A\right)\leq\;1 \quad C.R\left(A\right)\leq\;2 \quad D.\mathrm{无法确定} \quad E. \quad F. \quad G. \quad H.
$$
$$
\begin{array}{l}\mathrm{设所在平面的方程为}ax+by+cz+d=0\left(abcd\neq0\right),将P_1,\;P_2,\;P_3,\;P_4\mathrm{的坐标代入得}\;\;\\A\begin{pmatrix}a\\b\\c\\d\end{pmatrix}=\begin{pmatrix}x_1&y_1&z_1&1\\x_2&y_2&z_2&1\\x_3&y_3&z_3&1\\x_4&y_4&z_4&1\end{pmatrix}\begin{pmatrix}a\\b\\c\\d\end{pmatrix}=0\\.\;\;P_1,\;P_2,\;P_3,\;P_4\mathrm{四点共面},\mathrm{此方程组必有非零解},\mathrm{故必有}\left|A\right|=0,\mathrm{亦即}A\mathrm{的秩}r\leq\;3.\end{array}
$$



$$
\mathrm{若向量}β=\begin{pmatrix}1\\a\\a^2-5\\7\end{pmatrix}\mathrm{可由向量组}α_1=\begin{pmatrix}1\\2\\-1\\3\end{pmatrix},α_2=\begin{pmatrix}2\\5\\a\\8\end{pmatrix},α_3=\begin{pmatrix}-1\\0\\3\\1\end{pmatrix}\mathrm{线性表示},\mathrm{则向量组}α_1,α_2,α_3\mathrm{的秩为}(\;).
$$
$$
A.
1 \quad B.2 \quad C.3 \quad D.\mathrm{无法确定} \quad E. \quad F. \quad G. \quad H.
$$
$$
\begin{array}{l}\mathrm{由于}β\mathrm{可由}α_1,α_2,α_3\mathrm{线性表示},从\\\;\;\;\;\;\;\;\;\;\;\;\;\;\;\;\;\;\;\;\;\;\;\;\;(α_1,α_2,α_3β_1)\;\;\rightarrow\begin{pmatrix}1&2&-1&1\\0&1&2&a-2\\0&0&a+1&0\\0&0&0&4-a\end{pmatrix},\\\mathrm{因为}R(α_1,α_2,α_3)=R(α_1,α_2,α_3,β_1)\;\mathrm{故知}a=4,R(α_1,α_2,α_3)=3,\;\\\end{array}
$$



$$
β=(1,k,5)^T\mathrm{能由向量组}α_1=(1,-3,2)^T,α_2=(2,-1,1)^T\mathrm{线性表示},则k为(\;).\;
$$
$$
A.
k=-8 \quad B.k\neq-8 \quad C.k=-8或k=-2 \quad D.k=-2 \quad E. \quad F. \quad G. \quad H.
$$
$$
\begin{array}{l}\mathrm{设有实数}x_1,x_2,使β=x_1α_1+x_2α_2,即\left\{\begin{array}{c}x_1+2x_2=1\\-3x_1-x_2=k\\2x_1+x_2=5\end{array}\right..\;\;\\\mathrm{由此},\begin{pmatrix}1&2&1\\-3&-1&k\\2&1&5\end{pmatrix}\;\;\rightarrow\begin{pmatrix}1&2&1\\0&5&k+3\\0&-3&3\end{pmatrix}\rightarrow\begin{pmatrix}1&2&1\\0&1&-1\\0&5&k+3\end{pmatrix}\rightarrow\begin{pmatrix}1&2&1\\0&1&-1\\0&0&k+8\end{pmatrix}\\\mathrm{可见},\mathrm{当时}k=-8,β\mathrm{能由}α_1,α_2\mathrm{线性表示};\\\mathrm{当时}k\neq-8,β\mathrm{不能由}α_1,α_2\mathrm{线性表示}.\;\;\\或\begin{vmatrix}1&2&1\\-3&-1&k\\2&1&5\end{vmatrix}=\begin{vmatrix}1&2&1\\0&5&k+3\\0&-3&3\end{vmatrix}=3(k+8)=0⇒ k=-8.\end{array}
$$



$$
β=(1,k,5)^T\mathrm{不能由向量组}α_1=(1,-3,2)^T,α_2=(2,-1,1)^T\mathrm{线性表示},则k为(\;).\;
$$
$$
A.
k=-8 \quad B.k\neq-8 \quad C.k\neq-2 \quad D.k\neq-2且k\neq-8 \quad E. \quad F. \quad G. \quad H.
$$
$$
\begin{array}{l}\mathrm{设有实数}x_1,x_2,使β=x_1α_1+x_2α_2,即\left\{\begin{array}{c}x_1+2x_2=1\\-3x_1-x_2=k\\2x_1+x_2=5\end{array}\right..\;\;\\\mathrm{由此},\begin{pmatrix}1&2&1\\-3&-1&k\\2&1&5\end{pmatrix}\;\;\rightarrow\begin{pmatrix}1&2&1\\0&5&k+3\\0&-3&3\end{pmatrix}\rightarrow\begin{pmatrix}1&2&1\\0&1&-1\\0&5&k+3\end{pmatrix}\rightarrow\begin{pmatrix}1&2&1\\0&1&-1\\0&0&k+8\end{pmatrix}\\\mathrm{可见},\mathrm{当时}k=-8,β\mathrm{能由}α_1,α_2\mathrm{线性表示};\\\mathrm{当时}k\neq-8,β\mathrm{不能由}α_1,α_2\mathrm{线性表示}.\;\;\\或\begin{vmatrix}1&2&1\\-3&-1&k\\2&1&5\end{vmatrix}=\begin{vmatrix}1&2&1\\0&5&k+3\\0&-3&3\end{vmatrix}=3(k+8)\neq0⇒ k\neq-8.\end{array}
$$



$$
\begin{array}{l}设α_i=(a_i,b_i,c_i)^T,i=1,2,3,\mathrm{则平面上三条直线}\\a_1x+a_2y+a_3=0,b_1x+b_2y+b_3=0,c_1x+c_2y+c_3=0,\\\mathrm{交于一点的充分必要条件是}(\;).\end{array}
$$
$$
A.
\begin{vmatrix}a_1,a_2,a_3\end{vmatrix}=0 \quad B.\begin{vmatrix}a_1,a_2,a_3\end{vmatrix}\neq0 \quad C.r(a_1,a_2,a_3)=r(a_1,a_2) \quad D.a_1,a_2\mathrm{线性无关},但a_1,a_2,a_3\mathrm{线性相关} \quad E. \quad F. \quad G. \quad H.
$$
$$
\begin{array}{l}(1)\mathrm{三条直线交于一点的充要条件是方程组}\\\left\{\begin{array}{c}a_1x+a_2y+a_3=0\\b_1x+b_2y+b_3=0\\c_1x+c_2y+c_3=0\end{array}\right.\\\mathrm{有唯一解},即a_3\mathrm{可由}a_1,a_2\mathrm{线性表示法唯一}.\;故a_1,a_2\mathrm{线性无关},a_1,a_2,a_3\mathrm{线性相关}.\;\\(2)\begin{vmatrix}a_1,a_2,a_3\end{vmatrix}\neq0\mathrm{肯定错},\mathrm{它表示}a_1,a_2,a_3\mathrm{线性无关},\mathrm{于是}r(A)\not≡ r\widetilde{(A)},\mathrm{方程组无解}.\;\;\;\;\;\\(3)而\begin{vmatrix}a_1,a_2,a_3\end{vmatrix}=0,r(a_1,a_2,a_3)=r(a_1,a_2)\mathrm{均是交于一点的必要条件},\mathrm{仅行列式为}0\mathrm{不能排除其中有平行}\\\mathrm{直线},\mathrm{对于}r(a_1,a_2,a_3)=r(a_1,a_2),\mathrm{因为秩可能是}1,\mathrm{也就是有平行直线}.\mathrm{作为充要条件是不正确的}.\end{array}
$$



$$
β=(1,k,5)^T\mathrm{能由向量组}α_1=(1,-2,2)^T,α_2=(1,-1,1)^T\mathrm{线性表示},则k为(\;).\;
$$
$$
A.
k=5 \quad B.k=-5 \quad C.k\neq-5 \quad D.k=5或k=-5 \quad E. \quad F. \quad G. \quad H.
$$
$$
\begin{array}{l}\begin{vmatrix}1&1&1\\-2&-1&k\\2&1&5\end{vmatrix}=\begin{vmatrix}1&1&1\\0&1&k+2\\0&-1&3\end{vmatrix}=\begin{vmatrix}1&1&1\\0&1&k+2\\0&0&k+5\end{vmatrix}=k+5=0\end{array}
$$



$$
β=(1,k,5)^T\mathrm{不能由向量组}α_1=(1,-2,2)^T,α_2=(1,-1,1)^T\mathrm{线性表示},则k为(\;).\;
$$
$$
A.
k=5或k=-5 \quad B.k=-5 \quad C.k\neq-5 \quad D.k\neq5 \quad E. \quad F. \quad G. \quad H.
$$
$$
\begin{array}{l}\begin{vmatrix}1&1&1\\-2&-1&k\\2&1&5\end{vmatrix}=\begin{vmatrix}1&1&1\\0&1&k+2\\0&-1&3\end{vmatrix}=\begin{vmatrix}1&1&1\\0&1&k+2\\0&0&k+5\end{vmatrix}=k+5\neq0\end{array}
$$



$$
设α_1=\begin{pmatrix}a\\a\\a\end{pmatrix},α_2=\begin{pmatrix}a-1\\a\\a-1\end{pmatrix},α_3=\begin{pmatrix}1\\1\\a-1\end{pmatrix},β=\begin{pmatrix}1\\2\\1\end{pmatrix}.若β\mathrm{可由}α_1,α_2,α_3\mathrm{线性表示},\mathrm{但表达式不唯一},则a\mathrm{的值为}(\;\;).
$$
$$
A.
a=0 \quad B.a=2 \quad C.a\neq0且a\neq2 \quad D.a=0或a=2 \quad E. \quad F. \quad G. \quad H.
$$
$$
\begin{array}{l}设k_1α_1+k_2α_2+k_3α_3=β,\;由\\\left(α_1,\;α_2,\;α_3\vertβ\right)=\begin{pmatrix}a&a-1&1&\vdots&1\\a&a&1&\vdots&2\\a&a-1&a-1&\vdots&1\end{pmatrix}\xrightarrow 行\begin{pmatrix}a&a-1&1&\vdots&1\\0&1&0&\vdots&1\\0&0&a-2&\vdots&0\end{pmatrix}\\\mathrm{可知}:当a=2时,r\left(α_1,\;α_2,\;α_3\right)=r\left(α_1,\;α_2,\;α_3,\;β\right)=2,故β\mathrm{可由}α_1,\;α_2,\;α_3\mathrm{线性表示},\mathrm{且不唯一}.\end{array}
$$



$$
\mathrm{设有向量}α_1=(1,4,0,2)^T,α_2=(2,7,1,3)^T,α_3=(0,1,-1,a)^T,β=(3,10,b,4)^T,若β\mathrm{不能由}α_1,α_2,α_3\mathrm{线性表示},则a,b\mathrm{满足}(\;)\;
$$
$$
A.
b\neq2 \quad B.b=2,\;a=1 \quad C.a\neq1 \quad D.b=2,a\neq1 \quad E. \quad F. \quad G. \quad H.
$$
$$
\begin{array}{l}\mathrm{作初等行变换},得\;\\\;\;\;\;\;\;\;\;\;\;\;\;\;\;\;\;\;\;\;\;\;\;\;\;\;\;\;\;\;\;\;(α_1,α_2,α_3,β)\rightarrow\begin{pmatrix}1&2&0&3\\0&-1&1&-2\\0&0&a-1&0\\0&0&0&b-2\end{pmatrix},\;\;\\\mathrm{因此},当b\neq2,β\mathrm{不能由}α_1,α_2,α_3\mathrm{线性表示}.\end{array}
$$



$$
\begin{array}{l}\mathrm{设有向量}\\\;\;\;\;\;\;\;\;\;\;\;\;α_1=\begin{pmatrix}1+λ\\1\\1\end{pmatrix},α_2=\begin{pmatrix}1\\1+λ\\1\end{pmatrix},α_3=\begin{pmatrix}1\\1\\1+λ\end{pmatrix},β=\begin{pmatrix}0\\λ\\λ^2\end{pmatrix},\;\;\\\mathrm{则当}λ 取(\;)\mathrm{值时},β\mathrm{不能由}α_1,α_2,α_3\mathrm{线性表示}.\end{array}
$$
$$
A.
λ=0 \quad B.λ=-3 \quad C.λ\neq-3 \quad D.λ=0或λ=-3 \quad E. \quad F. \quad G. \quad H.
$$
$$
\begin{array}{l}\begin{vmatrix}α_1α_2α_3\end{vmatrix}\;\;=λ^2(λ+3).\\(1)当λ\neq0且λ\neq-3时,β\mathrm{可由}α_1,α_2,α_3\mathrm{唯一地线性表示}.\\(2)\mathrm{当时}λ=0,β\mathrm{可由}α_1,α_2,α_3\mathrm{线性表示},\mathrm{但表达式不唯一}.\;\\(3)\mathrm{当时}λ=-3,β\mathrm{不能由}α_1,α_2,α_3\mathrm{线性表示}.\end{array}
$$



$$
\mathrm{设矩阵}A_{m× n}\mathrm{的秩为}m,且m<\;n,E_m为m\mathrm{阶单位阵},\mathrm{则下列结论正确的是}(\;).\;
$$
$$
A.
A\mathrm{的任意}m\mathrm{个列向量必线性无关} \quad B.A\mathrm{的任意的一个}m\mathrm{阶子式不等于}0 \quad C.\mathrm{若矩阵}B\mathrm{满足}BA=O,则B=O \quad D.A\mathrm{通过初等行变换},\mathrm{必可以化为}(E_m,O)\mathrm{的形式} \quad E. \quad F. \quad G. \quad H.
$$
$$
\begin{array}{l}R(A^T)=m,\mathrm{由线性方程组的判断定理可知}A_{n× m}^Tx=O\mathrm{只有零解},\mathrm{则若}BA=O⇒ A^TB^T=O,即B^T=O⇒ B=O.\mathrm{矩阵的}A_{m× n}\mathrm{秩为}m,\\则A\mathrm{中存在}m\mathrm{个列向量线性无关},\mathrm{且可通过初等变换},\mathrm{包括行变换和列变换},\mathrm{化为标准形}(E_m,0)\mathrm{的形式}.\end{array}
$$



$$
\mathrm{设有向量}\;α_1=\begin{pmatrix}1\\4\\0\\2\end{pmatrix},α_2=\begin{pmatrix}2\\7\\1\\3\end{pmatrix},α_3=\begin{pmatrix}0\\1\\-1\\a\end{pmatrix},β=\begin{pmatrix}3\\10\\b\\4\end{pmatrix}.\;若β\mathrm{可由线性}α_1,α_2,α_3\mathrm{表示},\mathrm{且表达式不唯一},则\;\;a和b\mathrm{满足}(\;)
$$
$$
A.
b=2,a\neq1 \quad B.b=2,a=1 \quad C.b\neq2,a\neq1 \quad D.b\neq2,a=1 \quad E. \quad F. \quad G. \quad H.
$$
$$
\begin{array}{l}\mathrm{做初等行变换},得\\\;\;\;\;\;\;\;\;\;\;\;\;\;\;\;\;\;\;\;\;\;\;(α_1,α_2,α_3β)⇒\begin{pmatrix}1&2&0&3\\0&-1&1&-2\\0&0&a-1&0\\0&0&0&b-2\end{pmatrix};\\当b=2时,a\neq1时,β\mathrm{可由}α_1,α_2,α_3\mathrm{唯一地线性表示},且\;β=-α_1+2α_2.\\当b=2,a=1时,β\mathrm{可由}α_1,α_2,α_3\mathrm{线性表示},\mathrm{但表达式不唯一},\mathrm{表达式为}\\\;\;\;β=-(2k+1)α_1+(k+2)α_2+kα_3,\mathrm{其中}k\mathrm{为任意常数}.\end{array}
$$



$$
\mathrm{已知}4\mathrm{维向量}α_1=(1,2,-1,3)^T,α_2=(2,5,t,8)^T,α_3=(-1,0,3,1)^T,β=(1,t,t^2-5,7)^T,若β\mathrm{可被向量}α_1,α_2,α_3\mathrm{线性表出},则t\mathrm{值为}(\;).\;
$$
$$
A.
t=4 \quad B.t\neq4 \quad C.t=-1 \quad D.t=-1或t=4 \quad E. \quad F. \quad G. \quad H.
$$
$$
\begin{array}{l}(α_1,α_2,α_3,β)=\begin{pmatrix}1&2&-1&1\\2&5&0&t\\-1&t&3&t^2-5\\3&8&1&7\end{pmatrix}\rightarrow\begin{pmatrix}1&2&-1&1\\0&1&2&t-2\\0&t+2&2&t^2-4\\0&2&4&4\end{pmatrix}\\\rightarrow\begin{pmatrix}1&2&-1&1\\0&1&2&2\\0&1&2&t-2\\0&0&-2(t+1)&0\end{pmatrix}\rightarrow\begin{pmatrix}1&2&-1&1\\0&1&2&2\\0&0&-2(t+1)&0\\0&0&0&t-4\end{pmatrix}\\当t=4时,β\mathrm{可由}α_1,α_2,α_3\mathrm{线性表出},\mathrm{表达式为}β=-3α_1+2α_2.\end{array}
$$



$$
设α_1=\begin{pmatrix}a\\a\\a\end{pmatrix},α_2=\begin{pmatrix}a-1\\a\\a-1\end{pmatrix},α_3=\begin{pmatrix}1\\1\\a-1\end{pmatrix},β=\begin{pmatrix}1\\2\\1\end{pmatrix}若β\mathrm{不能由}α_1,α_2,α_3\mathrm{线性表示},则a\mathrm{的值为}(\;).
$$
$$
A.
a=0 \quad B.a=2 \quad C.a\neq0且a\neq2 \quad D.a=0或a=2 \quad E. \quad F. \quad G. \quad H.
$$
$$
\begin{array}{l}设k_1α_1+k_2α_2+k_3α_3=β,\;由\\\left(α_1,\;α_2,\;α_3\vertβ\right)=\begin{pmatrix}a&a-1&1&\vdots&1\\a&a&1&\vdots&2\\a&a-1&a-1&\vdots&1\end{pmatrix}\xrightarrow 行\begin{pmatrix}a&a-1&1&\vdots&1\\0&1&0&\vdots&1\\0&0&a-2&\vdots&0\end{pmatrix}\\\mathrm{可知}:当a=0时,r\left(α_1,\;α_2,\;α_3\right)=2,r\left(α_1,\;α_2,\;α_3,\;β\right)=3,故β\mathrm{不能由}α_1,\;α_2,\;α_3\mathrm{线性表示}.\end{array}
$$



$$
\begin{array}{l}\mathrm{设有向量}\\\;\;\;\;\;\;\;\;\;\;\;\;α_1=\begin{pmatrix}1+λ\\1\\1\end{pmatrix},α_2=\begin{pmatrix}1\\1+λ\\1\end{pmatrix},α_3=\begin{pmatrix}1\\1\\1+λ\end{pmatrix},β=\begin{pmatrix}0\\λ\\λ^2\end{pmatrix},\;\;\\\mathrm{则当}λ 取(\;)\mathrm{值时},β\mathrm{可由}α_1,α_2,α_3\mathrm{线性表示},\mathrm{且表达式不唯一}.\end{array}
$$
$$
A.
λ=0 \quad B.λ=-3 \quad C.λ=-3或λ=0 \quad D.λ\neq0 \quad E. \quad F. \quad G. \quad H.
$$
$$
\begin{array}{l}\begin{vmatrix}α_1α_2α_3\end{vmatrix}\;\;=λ^2(λ+3).\\(1)当λ\neq0且λ\neq-3时,β\mathrm{可由}α_1,α_2,α_3\mathrm{唯一地线性表示}.\\(2)\mathrm{当时}λ=0,β\mathrm{可由}α_1,α_2,α_3\mathrm{线性表示},\mathrm{但表达式不唯一}.\;\\(3)\mathrm{当时}λ=-3,β\mathrm{不能由}α_1,α_2,α_3\mathrm{线性表示}.\end{array}
$$



$$
\mathrm{设有向量}α_1=\begin{pmatrix}1\\4\\0\\2\end{pmatrix},α_2=\begin{pmatrix}2\\7\\1\\3\end{pmatrix},α_3=\begin{pmatrix}0\\1\\-1\\a\end{pmatrix},β=\begin{pmatrix}3\\10\\b\\4\end{pmatrix}\;.\;若β\mathrm{可由}α_1,α_2,α_3\mathrm{唯一线性表示},则a,b(\;)
$$
$$
A.
b=2,a\neq1 \quad B.b=2,\;a=1 \quad C.b\neq2,a\neq1 \quad D.b\neq2,\;a=1 \quad E. \quad F. \quad G. \quad H.
$$
$$
\begin{array}{l}\mathrm{做初等行变换},得\;\\\;\;(α_1,α_2,α_3β)\rightarrow\begin{pmatrix}1&2&0&3\\0&-1&1&-2\\0&1&-1&b\\0&-1&a&-2\end{pmatrix}\rightarrow\begin{pmatrix}1&2&0&3\\0&-1&1&-2\\0&0&a-1&0\\0&0&0&b-2\end{pmatrix};\;\\当b=2时,a\neq1时,β\mathrm{可由}α_1,α_2,α_3\mathrm{唯一地线性表示},且\;β=-α_1+2α_2\end{array}
$$



$$
\mathrm{设三维向量}α_1=(1,1,1)^T,α_2=(a,1,1)^T,α_3=(1,2,b)^T,β=(2,3,4)^T,若β\mathrm{可由}α_1,α_2,α_3\mathrm{线性表示},\mathrm{且表示式不唯一},则a,b\mathrm{值为}(\;).\;
$$
$$
A.
a=1,b=3 \quad B.a=1,b\neq3 \quad C.a\neq1,b=3 \quad D.a\neq1,b\neq3 \quad E. \quad F. \quad G. \quad H.
$$
$$
\begin{array}{l}\mathrm{实施初等行变换}\\\begin{pmatrix}1&a&1&\vdots&2\\1&1&2&\vdots&3\\1&1&b&\vdots&4\end{pmatrix}\rightarrow\begin{pmatrix}1&1&2&\vdots&3\\0&a-1&-1&\vdots&-1\\0&0&b-2&\vdots&1\end{pmatrix}\rightarrow\begin{pmatrix}1&1&2&\vdots&3\\0&a-1&b-3&\vdots&0\\0&0&b-2&\vdots&1\end{pmatrix}\\当a=1,b=3时,\mathrm{上面的矩阵变为}\\\begin{pmatrix}1&a&1&:&2\\1&1&2&\vdots&3\\1&1&b&\vdots&4\end{pmatrix}\rightarrow\begin{pmatrix}1&1&2&\vdots&3\\0&0&0&\vdots&0\\0&0&1&\vdots&1\end{pmatrix}\rightarrow\begin{pmatrix}1&1&0&\vdots&1\\0&0&1&\vdots&1\\0&0&0&\vdots&0\end{pmatrix}\\\mathrm{此时}β\mathrm{可由}α_1,α_2,α_3\mathrm{线性表示},\mathrm{但表示式不唯一}.\\\\\\\end{array}
$$



$$
\begin{array}{l}设α_1=(1,0,0,3)^T,α_2=(1,1,-1,2)^T,α_3=(1,2,a-3,1)^T,α_4=(1,2,-2,a)^T,β=(0,1,b,-1)^T,若β\mathrm{不能由}\\α_1,α_2,α_3,α_4\mathrm{线性表示},则a,b\mathrm{值为}(\;).\;\end{array}
$$
$$
A.
a\neq1,b=-1 \quad B.a=1,b\neq-1 \quad C.a\neq1,b\neq-1 \quad D.a=1,b=-1 \quad E. \quad F. \quad G. \quad H.
$$
$$
\begin{array}{l}\begin{array}{l}\mathrm{对向量组}(α_1,α_2,α_3,α_4,β)\mathrm{实施初等行变换}:\\(α_1,α_2,α_3,α_4,β)=\begin{pmatrix}1&1&1&1&0\\0&1&2&2&1\\0&-1&a-3&-2&b\\3&2&1&a&-1\end{pmatrix}\rightarrow\begin{pmatrix}1&0&-1&-1&-1\\0&1&2&2&1\\0&0&a-1&0&b+1\\0&0&0&a-1&0\end{pmatrix}\\当a=1时,R(α_1,α_2,α_3,α_4)=2.\mathrm{此时}\\(I)若b+1\neq0,即b\neq-1,则β\mathrm{不能由}α_1,α_2,α_3,α_4\mathrm{线性表示}.\end{array}\\(II)若b+1=0,即b=-1,则β\mathrm{能由}α_1,α_2,α_3,α_4\mathrm{线性表示但表示式不唯一}.\\\\\end{array}
$$



$$
\begin{array}{l}设α_1=(1,0,0,3)^T,α_2=(1,1,-1,2)^T,α_3=(1,2,a-3,1)^T,α_4=(1,2,-2,a)^T,β=(0,1,b,-1)^T,若β\mathrm{能由}\\α_1,α_2,α_3,α_4\mathrm{线性表示但不唯一},则a,b\mathrm{值为}(\;).\;\end{array}
$$
$$
A.
a\neq1,b=-1 \quad B.a=1,b\neq-1 \quad C.a\neq1,b\neq-1 \quad D.a=1,b=-1 \quad E. \quad F. \quad G. \quad H.
$$
$$
\begin{array}{l}\begin{array}{l}\mathrm{对向量组}(α_1,α_2,α_3,α_4,β)\mathrm{实施初等行变换}:\\(α_1,α_2,α_3,α_4,β)=\begin{pmatrix}1&1&1&1&0\\0&1&2&2&1\\0&-1&a-3&-2&b\\3&2&1&a&-1\end{pmatrix}\rightarrow\begin{pmatrix}1&0&-1&-1&-1\\0&1&2&2&1\\0&0&a-1&0&b+1\\0&0&0&a-1&0\end{pmatrix}\\当a=1时,R(α_1,α_2,α_3,α_4)=2.\mathrm{此时}\\(I)若b+1\neq0,即b\neq-1,则β\mathrm{不能由}α_1,α_2,α_3,α_4\mathrm{线性表示}.\end{array}\\(II)若b+1=0,即b=-1,则β\mathrm{能由}α_1,α_2,α_3,α_4\mathrm{线性表示但表示式不唯一}.\\\\\end{array}
$$



$$
\begin{array}{l}设α=\begin{pmatrix}a_1\\a_2\\a_3\end{pmatrix},β=\begin{pmatrix}b_1\\b_2\\b_3\end{pmatrix},γ=\begin{pmatrix}c_1\\c_2\\c_3\end{pmatrix}.\;a_i^2+b_i^2\neq0,\;i=1,\;2,\;3.\mathrm{则下列三条直线}\\\left\{\begin{array}{c}l_1:a_1x+b_1y+c_1=0\\l_2:a_2x+b_2y+c_2=0\\l_3:a_3x+b_3y+c_3=0\end{array}\right.,\\\mathrm{相交于一点的充要条件为}(\;)\end{array}
$$
$$
A.
\mathrm{向量组}α,β,γ\mathrm{线性相关而向量组}α,β\mathrm{线性无关}. \quad B.\mathrm{向量组}α,β,γ\mathrm{线性相关而向量组}α,β\mathrm{线性相关}. \quad C.\mathrm{向量组}α,β,γ\mathrm{线性无关而向量组}α,β\mathrm{线性相关}. \quad D.\mathrm{向量组}α,β,γ\mathrm{线性无关而向量组}α,β\mathrm{线性无关}. \quad E. \quad F. \quad G. \quad H.
$$
$$
\begin{array}{l}\begin{array}{l}记3×2\mathrm{矩阵}A=(α,β),\mathrm{则三直线}l_1,l_2,l_3\mathrm{相交于一点},\end{array}\\\Leftrightarrow\mathrm{非齐次方程组}A\begin{pmatrix}x\\y\end{pmatrix}=-γ\mathrm{有唯一解},\\\Leftrightarrow r(A)=r(A,\;-γ)=2,\\\begin{array}{l}\Leftrightarrow r(A)=r(A,γ)=2(因r(A,-γ)=r(A,γ))\\\Leftrightarrow\mathrm{向量组}α,β,γ\mathrm{线性相关而向量组}α,β\mathrm{线性无关}.\end{array}\\\\\end{array}
$$



$$
\begin{array}{l}\mathrm{设向量组}α_1=(a,2,10)^T,α_2=(-2,1,5)^T,α_3=(-1,1,4)^T,β=(1,b,c)^T,若β\mathrm{可由}α_1,α_2,α_3\mathrm{线性表示},\mathrm{但表示式不}\\\mathrm{唯一},\mathrm{则的}a,b,c\mathrm{满足的条件为}(\;).\;\end{array}
$$
$$
A.
a\neq-4且3b-c\neq1 \quad B.a=-4且3b-c\neq1 \quad C.a=-4且3b-c=1 \quad D.a=-4或3b-c\neq1 \quad E. \quad F. \quad G. \quad H.
$$
$$
\begin{array}{l}\begin{bmatrix}α_1,&α_2,&α_3,&β\end{bmatrix}=\begin{pmatrix}a&-2&-1&1\\2&1&1&b\\10&5&4&c\end{pmatrix}\xrightarrow{r_1-r_2}\begin{pmatrix}2&1&1&b\\a&-2&-1&1\\10&5&4&c\end{pmatrix}\\\xrightarrow[{r_3-5r_1}]{r_2-\frac a2r_1}\begin{pmatrix}2&1&1&b\\0&-2-\frac a2&-1-\frac a2&1-\frac{ab}2\\0&0&-1&c-5b\end{pmatrix}\\当-2-\frac a2=0,即a=-4,且3b-c=1时,有\\A_1=\begin{pmatrix}2&1&0&-b-1\\0&0&1&1+2b\\0&0&0&0\end{pmatrix},R(α_1,α_2,α_3,β)=R(A_1)=2<3,\\β\mathrm{可由}α_1,α_2,α_3\mathrm{线性表示},\mathrm{但表示法不唯一}.\end{array}
$$



$$
\begin{array}{l}\mathrm{设三阶实对称矩阵的}A\mathrm{特征值为}λ_1=-1,λ_2=2,λ_3=5,\mathrm{向量}x_1=\begin{pmatrix}2\\2\\1\end{pmatrix},x_2=\begin{pmatrix}2\\-1\\-2\end{pmatrix},\mathrm{分别为}A\mathrm{的对应于}λ_1=-1,λ_2=2\mathrm{的特征向量},\\则A是(\;).\end{array}
$$
$$
A.
\begin{pmatrix}1&-2&0\\-2&2&-2\\0&-2&3\end{pmatrix} \quad B.\begin{pmatrix}1&-2&1\\-2&2&-2\\1&-2&3\end{pmatrix} \quad C.\begin{pmatrix}1&-2&0\\-2&2&-1\\0&-1&3\end{pmatrix} \quad D.\begin{pmatrix}1&-2&0\\-2&2&-2\\0&-2&1\end{pmatrix} \quad E. \quad F. \quad G. \quad H.
$$
$$
\begin{array}{l}\begin{array}{l}设x_3为A\mathrm{对应于}λ_3=5\mathrm{的特征向量},\mathrm{实对称矩阵不同特征值对应的特征向量正交},则x_3x_1^T=0,x_3x_2^T=0,\mathrm{可建立方程组直接求解};\;\\则x_3={(-1,2,-2)}^T.x_1,x_2,x_3\mathrm{单位化后构成的矩阵为}\end{array}\\P=\begin{pmatrix}\frac23&\frac23&-\frac13\\\frac23&-\frac13&\frac23\\\frac13&-\frac23&-\frac23\end{pmatrix}\\由P^{-1}AP=\begin{pmatrix}-1&0&0\\0&2&0\\0&0&5\end{pmatrix},且P^{-1}=P^T\mathrm{则有}\\A=P\begin{pmatrix}-1&0&0\\0&2&0\\0&0&5\end{pmatrix}P^T=\begin{pmatrix}1&-2&0\\-2&2&-2\\0&-2&3\end{pmatrix}\\\end{array}
$$



$$
\begin{array}{l}\mathrm{设三阶实对称矩阵}A\mathrm{的特征值为}λ_1=6,λ_2=3,λ_3=3,\mathrm{向量}x_1=\begin{pmatrix}1\\1\\1\end{pmatrix}为A\mathrm{的对应于}λ_1=6\mathrm{的特征向量},\\则A是(\;).\end{array}
$$
$$
A.
\begin{pmatrix}4&1&1\\1&4&1\\1&1&4\end{pmatrix} \quad B.\begin{pmatrix}3&1&1\\1&3&1\\1&1&3\end{pmatrix} \quad C.\begin{pmatrix}2&1&1\\1&2&1\\1&1&2\end{pmatrix} \quad D.\begin{pmatrix}-3&1&1\\1&-3&1\\1&1&-3\end{pmatrix} \quad E. \quad F. \quad G. \quad H.
$$
$$
\begin{array}{l}\begin{array}{l}设x_2为A\mathrm{对应于}λ_2=λ_3=3\mathrm{的特征向量},\mathrm{实对称矩阵不同特征值对应的特征向量正交},则x_2x_1^T=0,\mathrm{可建立方程组直接求解};\;\\则x_2={(-1,0,1)}^T,x_3={(-1,1,0)}^T.将x_1\mathrm{单位化},x_2,x_3\mathrm{正交单位化后组成矩阵为}\end{array}\\P=\begin{pmatrix}\frac1{\sqrt3}&-\frac1{\sqrt2}&-\frac1{\sqrt6}\\\frac1{\sqrt3}&0&\frac2{\sqrt6}\\\frac1{\sqrt3}&\frac1{\sqrt2}&-\frac1{\sqrt6}\end{pmatrix}\\由P^{-1}AP=\begin{pmatrix}6&0&0\\0&3&0\\0&0&3\end{pmatrix},且P^{-1}=P^T\mathrm{则有}\\A=P\begin{pmatrix}6&0&0\\0&3&0\\0&0&3\end{pmatrix}P^T=\begin{pmatrix}4&1&1\\1&4&1\\1&1&4\end{pmatrix}\\\end{array}
$$



$$
\begin{array}{l}\mathrm{设三阶实对称矩阵的}A\mathrm{特征值为}λ_1=-1,λ_2=1,λ_3=1,\mathrm{向量}x_1=\begin{pmatrix}0\\1\\1\end{pmatrix}为A\mathrm{的对应于}λ_1=-1\mathrm{的特征向量},\\则A是(\;).\end{array}
$$
$$
A.
\begin{pmatrix}1&0&0\\0&0&-1\\0&-1&0\end{pmatrix} \quad B.\begin{pmatrix}1&0&0\\0&0&1\\0&-1&0\end{pmatrix} \quad C.\begin{pmatrix}1&0&0\\0&0&-1\\0&1&0\end{pmatrix} \quad D.\begin{pmatrix}1&0&0\\0&0&-1\\0&-1&1\end{pmatrix} \quad E. \quad F. \quad G. \quad H.
$$
$$
\begin{array}{l}\begin{array}{l}设x_2为A\mathrm{对应于}λ_2=λ_3=1\mathrm{的特征向量},\mathrm{实对称矩阵不同特征值对应的特征向量正交},则x_2x_1^T=0,\mathrm{可建立方程组直接求解};\;\\则x_2={(1,0,0)}^T,x_3={(0,-1,1)}^T.将x_1,x_2,x_3\mathrm{单位化后组成矩阵为}\end{array}\\P=\begin{pmatrix}0&1&0\\\frac1{\sqrt2}&0&-\frac1{\sqrt2}\\\frac1{\sqrt2}&0&\frac1{\sqrt2}\end{pmatrix}\\由P^{-1}AP=\begin{pmatrix}-1&0&0\\0&1&0\\0&0&1\end{pmatrix},且P^{-1}=P^T\mathrm{则有}\\A=P\begin{pmatrix}-1&0&0\\0&1&0\\0&0&1\end{pmatrix}P^T=\begin{pmatrix}1&0&0\\0&0&-1\\0&-1&0\end{pmatrix}\\\end{array}
$$



$$
\begin{array}{l}\mathrm{设三阶实对称矩阵的}A\mathrm{特征值为}λ_1=1,λ_2=2,λ_3=3,\mathrm{向量}x_1=\begin{pmatrix}-1\\-1\\1\end{pmatrix},x_2=\begin{pmatrix}1\\-2\\-1\end{pmatrix},\mathrm{分别为}A\mathrm{的对应于}λ_1=1,λ_2=2\mathrm{的特征向量},\\则A是(\;).\end{array}
$$
$$
A.
\frac16\begin{pmatrix}13&-2&5\\-2&10&2\\5&2&13\end{pmatrix} \quad B.\frac13\begin{pmatrix}13&-2&5\\-2&10&2\\5&2&13\end{pmatrix} \quad C.\frac16\begin{pmatrix}10&-2&5\\-2&10&2\\5&2&13\end{pmatrix} \quad D.\frac13\begin{pmatrix}10&-2&5\\-2&10&2\\5&2&13\end{pmatrix} \quad E. \quad F. \quad G. \quad H.
$$
$$
\begin{array}{l}\begin{array}{l}设x_3为A\mathrm{对应于}λ_3=3\mathrm{的特征向量},\mathrm{实对称矩阵不同特征值对应的特征向量正交},则x_3x_1^T=0,x_3x_2^T=0,\mathrm{可建立方程组直接求解};\;\\则x_3={(1,0,1)}^T.x_1,x_2,x_3\mathrm{单位化后构成的矩阵为}\end{array}\\P=\begin{pmatrix}\frac{-1}{\sqrt3}&\frac1{\sqrt6}&\frac1{\sqrt2}\\\frac{-1}{\sqrt3}&-\frac2{\sqrt6}&0\\\frac1{\sqrt3}&-\frac1{\sqrt6}&\frac1{\sqrt2}\end{pmatrix}\\由P^{-1}AP=\begin{pmatrix}1&0&0\\0&2&0\\0&0&3\end{pmatrix},且P^{-1}=P^T\mathrm{则有}\\A=P\begin{pmatrix}1&0&0\\0&2&0\\0&0&3\end{pmatrix}P^T=\frac16\begin{pmatrix}13&-2&5\\-2&10&2\\5&2&13\end{pmatrix}\\\end{array}
$$



$$
设A=\begin{pmatrix}3&-2\\-2&3\end{pmatrix},则A^{10}-5A^9=()
$$
$$
A.
\begin{pmatrix}1&2\\2&1\end{pmatrix} \quad B.\begin{pmatrix}-2&-2\\-2&-2\end{pmatrix} \quad C.\begin{pmatrix}-1&2\\2&-1\end{pmatrix} \quad D.\begin{pmatrix}2&2\\2&2\end{pmatrix} \quad E. \quad F. \quad G. \quad H.
$$
$$
\begin{array}{l}\mathrm{因为}A\mathrm{是实对称矩阵},\mathrm{所以可以找到正交相似变换矩阵}\\P=\begin{pmatrix}\frac1{\sqrt2}&-\frac1{\sqrt2}\\\frac1{\sqrt2}&\frac1{\sqrt2}\end{pmatrix},\mathrm{使得}P^TAP=\begin{pmatrix}1&0\\0&5\end{pmatrix},\mathrm{从而}A=P\begin{pmatrix}1&0\\0&5\end{pmatrix}P^T,A^k=P\begin{pmatrix}1&0\\0&5\end{pmatrix}^kP^T,故\\A^{10}-5A^9=P\begin{pmatrix}1&0\\0&5\end{pmatrix}^{10}P^T-5P\begin{pmatrix}1&0\\0&5\end{pmatrix}^9P^T=P\begin{pmatrix}-4&0\\0&0\end{pmatrix}P^T=\begin{pmatrix}-2&-2\\-2&-2\end{pmatrix}\end{array}
$$



$$
设n\mathrm{阶实对称矩阵}A,\mathrm{满足}A^2=A,且A\mathrm{的秩为}r,则\left|A-2E\right|=()
$$
$$
A.
2 \quad B.2^r \quad C.(-1)^r2^{n-r} \quad D.2^n \quad E. \quad F. \quad G. \quad H.
$$
$$
\begin{array}{l}因A^2=A,\mathrm{所以}A\mathrm{的特征值为}0或1,\mathrm{又因}A\mathrm{是实对称矩阵且秩为}r,\mathrm{故存在可逆矩阵}P\mathrm{使得}\\P^{-1}AP=\begin{pmatrix}E_r&0\\0&0\end{pmatrix}=∧,\mathrm{其中}E_r为r\mathrm{阶单位矩阵},\mathrm{从而}\\\left|A-2E\right|=\left|P∧ P^{-1}-2PP^{-1}\right|=\left|∧-2E\right|=\left|\begin{pmatrix}-E_r&0\\0&-2E_{n-r}\end{pmatrix}\right|=(-1)^r2^{n-r}\\\end{array}
$$



$$
设A为n\mathrm{阶实对称矩阵},且A^2=2A,A\mathrm{的秩为}r,\;则A\mathrm{的迹}tr(A)=()
$$
$$
A.
r \quad B.2r \quad C.3r \quad D.4r \quad E. \quad F. \quad G. \quad H.
$$
$$
\begin{array}{l}设\;λ 是A\mathrm{的一个特征值},\mathrm{因为}A^2=2A,\mathrm{所以}λ^2=2λ,即λ=2或0,又A\mathrm{为实对称矩阵},则\\A∼∧=\begin{pmatrix}λ_1&&&\\&λ_2&&\\&&⋱&\\&&&λ_n\end{pmatrix},\mathrm{于是}r(A)=r(∧)=r,故\;λ_1=λ_2=⋯=λ_r=2,λ_{r+1}=λ_{r+2}=⋯=λ_n=0\\\mathrm{所以}\;tr(A)=2r\end{array}
$$



$$
\mathrm{已知}5\mathrm{阶矩阵}A\mathrm{与对角矩阵相似},且3是A\mathrm{的二重特征值},则R\;(A\;-3E)=()
$$
$$
A.
1 \quad B.2 \quad C.3 \quad D.4 \quad E. \quad F. \quad G. \quad H.
$$
$$
R(A-3E)=5-2=3
$$



$$
\mathrm{当矩阵}A为\lbrack\;\;\;\;\rbrack 时,\mathrm{矩阵}A\mathrm{必与实对角阵相似}.
$$
$$
A.
\mathrm{降秩矩阵} \quad B.\mathrm{实满秩矩阵} \quad C.\mathrm{实对称矩阵} \quad D.\mathrm{正交矩阵} \quad E. \quad F. \quad G. \quad H.
$$
$$
\mathrm{实对称矩阵一定可以相似对角化}
$$



$$
设A为3\mathrm{阶实对称矩阵},A\mathrm{的秩为}2,且A\begin{pmatrix}1&1\\0&0\\-1&1\end{pmatrix}=\begin{pmatrix}-1&1\\0&0\\1&1\end{pmatrix},则A\mathrm{的所有特征值为}()
$$
$$
A.
-1,0,1 \quad B.0,1,2 \quad C.1,2,3 \quad D.1,\;1,\;1 \quad E. \quad F. \quad G. \quad H.
$$
$$
A为降秩矩阵,所以A有零特征值,又因为A\begin{pmatrix}1\\0\\-1\end{pmatrix}=\begin{pmatrix}-1\\0\\1\end{pmatrix},\mathrm{所以}A\mathrm{有一个特征值为}-1,\mathrm{同理由}A\begin{pmatrix}1\\0\\1\end{pmatrix}=\begin{pmatrix}1\\0\\1\end{pmatrix}\mathrm{可知}A\mathrm{有一个特征值为}1
$$



$$
设A为4\mathrm{阶是对称矩阵},\mathrm{且满足}A^2+A=0,若A\mathrm{的秩为}3,则A\mathrm{相似于}()
$$
$$
A.
\begin{pmatrix}1&0&0&0\\0&1&0&0\\0&0&1&0\\0&0&0&0\end{pmatrix} \quad B.\begin{pmatrix}1&0&0&0\\0&1&0&0\\0&0&-1&0\\0&0&0&0\end{pmatrix} \quad C.\begin{pmatrix}1&0&0&0\\0&-1&0&0\\0&0&-1&0\\0&0&0&0\end{pmatrix} \quad D.\begin{pmatrix}-1&0&0&0\\0&-1&0&0\\0&0&-1&0\\0&0&0&0\end{pmatrix} \quad E. \quad F. \quad G. \quad H.
$$
$$
设\;λ\;是A\mathrm{的特征值},则\;λ^2+λ=0,\mathrm{所以}A\mathrm{的特征值为}0,-1
$$



$$
设A为4\mathrm{阶是对称矩阵},\mathrm{且满足}A^2-A=0,若A\mathrm{的秩为}3,则A\mathrm{相似于}()
$$
$$
A.
\begin{pmatrix}1&0&0&0\\0&1&0&0\\0&0&1&0\\0&0&0&0\end{pmatrix} \quad B.\begin{pmatrix}1&0&0&0\\0&1&0&0\\0&0&-1&0\\0&0&0&0\end{pmatrix} \quad C.\begin{pmatrix}1&0&0&0\\0&-1&0&0\\0&0&-1&0\\0&0&0&0\end{pmatrix} \quad D.\begin{pmatrix}-1&0&0&0\\0&-1&0&0\\0&0&-1&0\\0&0&0&0\end{pmatrix} \quad E. \quad F. \quad G. \quad H.
$$
$$
设\;λ\;是A\mathrm{的特征值},则\;λ^2-λ=0,\mathrm{所以}A\mathrm{的特征值为}0,1
$$



$$
设A为4\mathrm{阶是对称矩阵},\mathrm{且满足}A^4-A^2=0,若A\mathrm{的秩为}2,则A\mathrm{相似于}()
$$
$$
A.
\begin{pmatrix}1&0&0&0\\0&1&0&0\\0&0&0&0\\0&0&0&0\end{pmatrix} \quad B.\begin{pmatrix}1&0&0&0\\0&-1&0&0\\0&0&0&0\\0&0&0&0\end{pmatrix} \quad C.\begin{pmatrix}-1&0&0&0\\0&-1&0&0\\0&0&0&0\\0&0&0&0\end{pmatrix} \quad D.\begin{pmatrix}1&0&0&0\\0&1&0&0\\0&0&1&0\\0&0&0&0\end{pmatrix} \quad E. \quad F. \quad G. \quad H.
$$
$$
设\;λ\;是A\mathrm{的特征值},则\;λ^4+λ^2=0,\mathrm{所以}A\mathrm{的特征值为}0,1,\mathrm{又因为}A\mathrm{的秩为}2,\mathrm{所以}A\mathrm{的特征值为}0,0,1,1
$$



$$
\mathrm{已知}5\mathrm{阶矩阵}A\mathrm{与对角矩阵相似},且2是A\mathrm{的三重特征值},则R\;(A\;-2E)=()
$$
$$
A.
1 \quad B.2 \quad C.3 \quad D.4 \quad E. \quad F. \quad G. \quad H.
$$
$$
R(A-2E)=5-3=2
$$



$$
\mathrm{已知矩阵}A=\begin{pmatrix}1&0&2\\0&2&0\\2&0&-2\end{pmatrix},B=\begin{pmatrix}2&0&0\\0&2&0\\0&0&-3\end{pmatrix},\mathrm{存在正交矩阵}P,\mathrm{使得}P^TAP=B,\;则P=(\;)
$$
$$
A.
\begin{pmatrix}\frac2{\sqrt5}&0&\frac1{\sqrt5}\\0&1&0\\\frac1{\sqrt5}&0&\frac{-2}{\sqrt5}\end{pmatrix} \quad B.\begin{pmatrix}\frac1{\sqrt2}&0&\frac{-1}{\sqrt2}\\0&1&0\\\frac1{\sqrt2}&0&\frac1{\sqrt2}\end{pmatrix} \quad C.\begin{pmatrix}\frac{\sqrt2}{\sqrt5}&0&\frac{\sqrt3}{\sqrt5}\\0&1&0\\\frac{\sqrt3}{\sqrt5}&0&\frac{-\sqrt2}{\sqrt5}\end{pmatrix} \quad D.\begin{pmatrix}\frac{\sqrt2}{\sqrt3}&0&\frac1{\sqrt3}\\0&1&0\\\frac1{\sqrt3}&0&\frac{-\sqrt2}{\sqrt3}\end{pmatrix} \quad E. \quad F. \quad G. \quad H.
$$
$$
\mathrm{因为}A∼ B,\mathrm{所以}A\mathrm{的特征值为}2,2,-3\\\mathrm{求解}(A-2E)x=0得ζ_1=\begin{pmatrix}2\\0\\1\end{pmatrix},ζ_2=\begin{pmatrix}0\\1\\0\end{pmatrix},\mathrm{单位化得}p_1=\begin{pmatrix}\frac2{\sqrt5}\\0\\\frac1{\sqrt5}\end{pmatrix},p_2=\begin{pmatrix}0\\1\\0\end{pmatrix}\\\mathrm{求解}(A+3E)x=0得ζ_3=\begin{pmatrix}1\\0\\-2\end{pmatrix},\mathrm{单位化得}p_3=\begin{pmatrix}\frac1{\sqrt5}\\0\\\frac{-2}{\sqrt5}\end{pmatrix}\\P=(p_1,p_2,p_3)
$$



$$
\mathrm{已知矩阵}A=\begin{pmatrix}1&0&-1\\0&2&0\\-1&0&1\end{pmatrix},B=\begin{pmatrix}0&&\\&2&\\&&2\end{pmatrix},\mathrm{存在正交矩阵}P,\mathrm{使得}P^TAP=B,则P=(\;\;)
$$
$$
A.
\begin{pmatrix}\frac1{\sqrt2}&\frac{-1}{\sqrt2}&0\\0&0&1\\\frac1{\sqrt2}&\frac1{\sqrt2}&0\end{pmatrix} \quad B.\begin{pmatrix}\frac1{\sqrt2}&0&\frac{-1}{\sqrt3}\\0&1&0\\\frac1{\sqrt2}&0&\frac1{\sqrt3}\end{pmatrix} \quad C.\begin{pmatrix}0&\frac{-1}{\sqrt5}&\frac2{\sqrt5}\\1&0&0\\0&\frac2{\sqrt5}&\frac1{\sqrt5}\end{pmatrix} \quad D.\begin{pmatrix}0&\frac1{\sqrt5}&\frac2{\sqrt5}\\1&0&0\\0&\frac2{\sqrt5}&\frac{-1}{\sqrt5}\end{pmatrix} \quad E. \quad F. \quad G. \quad H.
$$
$$
\begin{array}{l}\begin{array}{l}\mathrm{矩阵}A=\begin{pmatrix}1&0&-1\\0&2&0\\-1&0&1\end{pmatrix},由\left|λ E-A\right|=0\mathrm{解出}A\mathrm{的特征值为}λ_1=0,λ_2=λ_3=2\\当λ_1=0时,由(0E-A)x=0,\mathrm{解除特征向量}ξ_1=\begin{pmatrix}1\\0\\1\end{pmatrix},\mathrm{单位化}γ_1=\frac1{\sqrt2}\begin{pmatrix}1\\0\\1\end{pmatrix};\end{array}\\当λ_2=λ_3=2时,由(2E-A)x=0,\mathrm{解除特征向量}ξ_2=\begin{pmatrix}-1\\0\\1\end{pmatrix},\mathrm{单位化}ξ_3=\begin{pmatrix}-1\\1\\1\end{pmatrix},\\将ξ_2,\;ξ_3\mathrm{正交化},\mathrm{单位化得}γ_2=\frac{η_2}{\left|\left|η_2\right|\right|}\frac1{\sqrt2}\begin{pmatrix}-1\\0\\1\end{pmatrix},γ_3=η_3=\begin{pmatrix}0\\1\\0\end{pmatrix},令P=(γ_1,γ_2γ_3),则P^TAP=\begin{pmatrix}0&0&0\\0&2&0\\0&0&2\end{pmatrix}\\\end{array}
$$



$$
\mathrm{已知}A=\begin{pmatrix}\frac12&0&-\frac12\\0&1&0\\-\frac12&0&\frac12\end{pmatrix},B=\begin{pmatrix}1&&\\&1&\\&&0\end{pmatrix},\mathrm{存在正交矩阵}P\mathrm{使得}P^TAP=B,则P=()
$$
$$
A.
\begin{pmatrix}\frac{\sqrt2}2&0&\frac{-\sqrt2}2\\0&1&0\\\frac{\sqrt2}2&0&\frac{\sqrt2}2\end{pmatrix} \quad B.\begin{pmatrix}\frac2{\sqrt5}&0&\frac1{\sqrt5}\\0&1&0\\\frac{-1}{\sqrt5}&0&\frac2{\sqrt5}\end{pmatrix} \quad C.\begin{pmatrix}0&\frac{-\sqrt2}2&\frac{\sqrt2}2\\1&0&0\\0&\frac{\sqrt2}2&\frac{\sqrt2}2\end{pmatrix} \quad D.\begin{pmatrix}\frac2{\sqrt5}&0&\frac1{\sqrt5}\\0&1&0\\\frac1{\sqrt5}&0&\frac{-2}{\sqrt5}\end{pmatrix} \quad E. \quad F. \quad G. \quad H.
$$
$$
\begin{array}{l}\begin{array}{l}\mathrm{因为}A∼ B,\mathrm{所以}A\mathrm{的特征值为}λ_1=λ_2=1,λ_3=0,\\\mathrm{求解}(A-E)x=0得ζ_1=\begin{pmatrix}0\\1\\0\end{pmatrix},ζ_2=\begin{pmatrix}-1\\0\\1\end{pmatrix},\mathrm{单位化得}p_1=\begin{pmatrix}0\\1\\0\end{pmatrix},p_2=\begin{pmatrix}\frac{-1}{\sqrt2}\\0\\\frac1{\sqrt2}\end{pmatrix}\\\mathrm{求解}(A+0E)x=0得ζ_3=\begin{pmatrix}1\\0\\1\end{pmatrix},\mathrm{单位化得}p_3=\begin{pmatrix}\frac1{\sqrt2}\\0\\\frac1{\sqrt2}\end{pmatrix}\\则P=(p_1,p_2,p_3)\end{array}\\\end{array}
$$



$$
\mathrm{已知}A=\begin{pmatrix}0&0&1\\0&0&0\\1&0&0\end{pmatrix},B=\begin{pmatrix}1&&\\&-1&\\&&0\end{pmatrix},\mathrm{存在正交矩阵}P\mathrm{使得}P^TAP=B,则P=()
$$
$$
A.
\begin{pmatrix}\frac{-\sqrt2}2&0&\frac{\sqrt2}2\\0&1&0\\\frac{\sqrt2}2&0&\frac{\sqrt2}2\end{pmatrix} \quad B.\begin{pmatrix}\frac2{\sqrt5}&0&\frac1{\sqrt5}\\0&1&0\\\frac{-1}{\sqrt5}&0&\frac2{\sqrt5}\end{pmatrix} \quad C.\begin{pmatrix}\frac{\sqrt2}2&\frac{\sqrt2}2&0\\0&0&1\\\frac{\sqrt2}2&\frac{-\sqrt2}2&0\end{pmatrix} \quad D.\begin{pmatrix}\frac2{\sqrt5}&0&\frac1{\sqrt5}\\0&1&0\\\frac1{\sqrt5}&0&\frac{-2}{\sqrt5}\end{pmatrix} \quad E. \quad F. \quad G. \quad H.
$$
$$
\begin{array}{l}\begin{array}{l}\mathrm{因为}A∼ B,\mathrm{所以}A\mathrm{的特征值为}λ_1=1,λ_2=-1,λ_3=0,\\\mathrm{求解}(A-E)x=0得ζ_1=\begin{pmatrix}1\\0\\1\end{pmatrix},\mathrm{单位化得}p_1=\begin{pmatrix}\frac1{\sqrt2}\\0\\\frac1{\sqrt2}\end{pmatrix}\\\mathrm{求解}(A+E)x=0得ζ_2=\begin{pmatrix}1\\0\\-1\end{pmatrix},\mathrm{单位化得}p_2=\begin{pmatrix}\frac1{\sqrt2}\\0\\\frac{-1}{\sqrt2}\end{pmatrix}\\\mathrm{求解}(A+0E)x=0得ζ_3=\begin{pmatrix}0\\1\\0\end{pmatrix},\mathrm{单位化得}p_3=\begin{pmatrix}0\\1\\0\end{pmatrix}\\则P=(p_1,p_2,p_3)\end{array}\\\end{array}
$$



$$
\mathrm{已知}A=\frac13\begin{pmatrix}5&1&-1\\1&5&1\\-1&1&5\end{pmatrix},B=\begin{pmatrix}2&&\\&2&\\&&1\end{pmatrix},\mathrm{存在正交矩阵}P\mathrm{使得}P^TAP=B,则P=()
$$
$$
A.
\begin{pmatrix}\frac{\sqrt2}2&\frac{-1}{\sqrt6}&\frac1{\sqrt3}\\\frac{\sqrt2}2&\frac1{\sqrt6}&\frac{-1}{\sqrt3}\\0&\frac2{\sqrt6}&\frac1{\sqrt3}\end{pmatrix} \quad B.\begin{pmatrix}1&0&0\\0&\frac{\sqrt2}2&\frac{-\sqrt2}2\\0&\frac{\sqrt2}2&\frac{\sqrt2}2\end{pmatrix} \quad C.\begin{pmatrix}\frac{\sqrt2}2&\frac{\sqrt2}2&0\\0&0&1\\\frac{\sqrt2}2&\frac{-\sqrt2}2&0\end{pmatrix} \quad D.\begin{pmatrix}\frac2{\sqrt5}&0&\frac1{\sqrt5}\\0&1&0\\\frac1{\sqrt5}&0&\frac{-2}{\sqrt5}\end{pmatrix} \quad E. \quad F. \quad G. \quad H.
$$
$$
\begin{array}{l}\begin{array}{l}\mathrm{因为}A∼ B,\mathrm{所以}A\mathrm{的特征值为}λ_1=λ_2=2,λ_3=1,\\\mathrm{求解}(A-2E)x=0得ζ_1=\begin{pmatrix}1\\1\\0\end{pmatrix},ζ_1=\begin{pmatrix}0\\1\\1\end{pmatrix},\mathrm{正交单位化得}p_1=\begin{pmatrix}\frac{\sqrt2}2\\\frac{\sqrt2}2\\0\end{pmatrix},p_2=\begin{pmatrix}\frac{-1}{\sqrt6}\\\frac1{\sqrt6}\\\frac2{\sqrt6}\end{pmatrix}\\\mathrm{求解}(A-E)x=0得ζ_1=\begin{pmatrix}1\\-1\\1\end{pmatrix},\mathrm{单位化得}p_3=\begin{pmatrix}\frac1{\sqrt3}\\\frac{-1}{\sqrt3}\\\frac1{\sqrt3}\end{pmatrix}\\\\则P=(p_1,p_3,p_2)\end{array}\\\end{array}
$$



$$
\mathrm{已知}A=\begin{pmatrix}3&0&0\\0&1&2\\0&2&1\end{pmatrix},B=\begin{pmatrix}3&&\\&3&\\&&-1\end{pmatrix},\mathrm{存在正交矩阵}P\mathrm{使得}P^TAP=B,则P=()
$$
$$
A.
\begin{pmatrix}1&0&0\\0&\frac{\sqrt2}2&\frac{-\sqrt2}2\\0&\frac{\sqrt2}2&\frac{\sqrt2}2\end{pmatrix} \quad B.\begin{pmatrix}0&0&1\\\frac{-\sqrt2}2&\frac{\sqrt2}2&0\\\frac{\sqrt2}2&\frac{\sqrt2}2&0\end{pmatrix} \quad C.\begin{pmatrix}\frac{\sqrt2}2&\frac{\sqrt2}2&0\\0&0&1\\\frac{\sqrt2}2&\frac{-\sqrt2}2&0\end{pmatrix} \quad D.\begin{pmatrix}\frac2{\sqrt5}&0&\frac1{\sqrt5}\\0&1&0\\\frac1{\sqrt5}&0&\frac{-2}{\sqrt5}\end{pmatrix} \quad E. \quad F. \quad G. \quad H.
$$
$$
\begin{array}{l}\begin{array}{l}\mathrm{因为}A∼ B,\mathrm{所以}A\mathrm{的特征值为}λ_1=λ_2=3,λ_3=1,\\\mathrm{求解}(A-3E)x=0得ζ_1=\begin{pmatrix}1\\0\\0\end{pmatrix},ζ_1=\begin{pmatrix}0\\1\\1\end{pmatrix},\mathrm{正交单位化得}p_1=\begin{pmatrix}1\\0\\0\end{pmatrix},p_2=\begin{pmatrix}0\\\frac{\sqrt2}2\\\frac{\sqrt2}2\end{pmatrix}\\\mathrm{求解}(A-E)x=0得ζ_1=\begin{pmatrix}0\\-1\\1\end{pmatrix},\mathrm{单位化得}p_3=\begin{pmatrix}0\\\frac{-\sqrt2}2\\\frac{\sqrt2}2\end{pmatrix}\\\\则P=(p_1,p_3,p_2)\end{array}\\\end{array}
$$



$$
设A为2\mathrm{阶是对称矩阵},\mathrm{且满足}A^2+2A-3E=0,则A\mathrm{相似于}()
$$
$$
A.
\begin{pmatrix}-3&0\\0&1\end{pmatrix} \quad B.\begin{pmatrix}3&0\\0&-1\end{pmatrix} \quad C.\begin{pmatrix}-3&0\\0&-1\end{pmatrix} \quad D.\begin{pmatrix}3&0\\0&1\end{pmatrix} \quad E. \quad F. \quad G. \quad H.
$$
$$
设\;λ\;是A\mathrm{的特征值},则\;λ^2+2λ-3=0,\mathrm{所以}A\mathrm{的特征值为}-3,1
$$



$$
设A为2\mathrm{阶是对称矩阵},\mathrm{且满足}A^2+A-6E=0,则A\mathrm{相似于}()
$$
$$
A.
\begin{pmatrix}-3&0\\0&2\end{pmatrix} \quad B.\begin{pmatrix}3&0\\0&-2\end{pmatrix} \quad C.\begin{pmatrix}-3&0\\0&-2\end{pmatrix} \quad D.\begin{pmatrix}3&0\\0&2\end{pmatrix} \quad E. \quad F. \quad G. \quad H.
$$
$$
设\;λ\;是A\mathrm{的特征值},则\;λ^2+λ-6=0,\mathrm{所以}A\mathrm{的特征值为}-3,2
$$



$$
设A为2\mathrm{阶是对称矩阵},\mathrm{且满足}A^2+5A-6E=0,则A\mathrm{相似于}()
$$
$$
A.
\begin{pmatrix}-6&0\\0&1\end{pmatrix} \quad B.\begin{pmatrix}6&0\\0&-1\end{pmatrix} \quad C.\begin{pmatrix}-6&0\\0&-1\end{pmatrix} \quad D.\begin{pmatrix}6&0\\0&1\end{pmatrix} \quad E. \quad F. \quad G. \quad H.
$$
$$
设\;λ\;是A\mathrm{的特征值},则\;λ^2+5λ-6=0,\mathrm{所以}A\mathrm{的特征值为}-6,1
$$



$$
\begin{array}{l}设4\mathrm{阶实对称矩阵}A\mathrm{的特征值为}k_1=k_2=k_3=1,k_4=3\mathrm{且向量}β_1={(1,1,0,0)}^T,β_2={(1,0,1,0)}^T,β_3={(1,0,0,1)}^T\\\mathrm{都是对应于特征值}1\mathrm{的特征向量},则A\mathrm{的属于特征值}3\mathrm{的特征向量}β_4为(\;).\end{array}
$$
$$
A.
β_1,β_2,β_3\mathrm{中的某一个} \quad B.{(3,1,1,1)}^T \quad C.{(-1,1,1,1)}^T \quad D.\mathrm{从已知条件尚无法确定} \quad E. \quad F. \quad G. \quad H.
$$
$$
\mathrm{由于实对称矩阵不同特征值的特征向量正交可知},β_4与β_1,β_2,β_3\mathrm{正交},\mathrm{选项中的向量}{(-1,1,1,1)}^T\mathrm{符合条件}.
$$



$$
\begin{array}{l}\mathrm{设三阶实对称矩阵的特征值为}λ_1=λ_2=4,λ_3=2,\mathrm{向量}x_1=\begin{pmatrix}1\\1\\1\end{pmatrix},x_2=\begin{pmatrix}0\\2\\2\end{pmatrix},\mathrm{都是}A\mathrm{的对应于}4\mathrm{的特征向量},\\则A\mathrm{的对应于}λ_3=2\mathrm{的特征向量}\;x_3是(\;).\end{array}
$$
$$
A.
x_1,x_2\mathrm{中的某一个} \quad B.(2,1,-1)^T \quad C.(0,1,-1)^T \quad D.\mathrm{从已知条件尚无法确定} \quad E. \quad F. \quad G. \quad H.
$$
$$
\begin{array}{l}\mathrm{实对称矩阵不同特征值对应的特征向量正交},则x_3x_1^T=0,x_3x_2^T=0,\mathrm{可建立方程组直接求解};\;\\\mathrm{也可将选项中的向量代入},\mathrm{则只有向量}{(0,1,-1)}^T\mathrm{符合}.\end{array}
$$



$$
\begin{array}{l}\mathrm{设三阶实对称矩阵的特征值为}λ_1=λ_2=2,λ_3=8,\mathrm{对于}λ_1=2\mathrm{的特征值是}x_1=\begin{pmatrix}-1\\1\\0\end{pmatrix},x_2=\begin{pmatrix}-1\\0\\1\end{pmatrix},\\则A\mathrm{的对应于}λ_3=8\mathrm{的特征向量是}(\;).\end{array}
$$
$$
A.
x_1,x_2\mathrm{中的一个} \quad B.{(1,2,3)}^T \quad C.{(1,1,1)}^T \quad D.\mathrm{从已知条件尚无法确定} \quad E. \quad F. \quad G. \quad H.
$$
$$
\begin{array}{l}\mathrm{实对称矩阵不同特征值对应的特征向量正交},则x_3x_1^T=0,x_3x_2^T=0,\mathrm{可建立方程组直接求解};\;\\\mathrm{也可将选项中的向量代入},\mathrm{则只有向量}{(1,1,1)}^T\mathrm{符合}.\end{array}
$$



$$
设A为n\mathrm{阶实对称矩阵},B为n\mathrm{阶可逆矩阵},Q为n\mathrm{阶正交矩阵},\mathrm{则矩阵}()\mathrm{必与矩阵}A\mathrm{有相同的特征值}.
$$
$$
A.
B^{-1}Q^TAQB \quad B.(B^{-1})^TQ^TAQB^{-1} \quad C.B^TQ^TAQB \quad D.BQ^TAQ(B^T)^{-1} \quad E. \quad F. \quad G. \quad H.
$$
$$
\mathrm{由于}Q为n\mathrm{阶正交矩阵},则Q^T=Q^{-1},\mathrm{故矩阵}B^{-1}Q^TAQB=(QB)^{-1}A(QB)\mathrm{与矩阵}A\mathrm{相似},\mathrm{即有相同的特征值}.(\mathrm{相似矩阵有相同的特征值})
$$



$$
\begin{array}{l}\mathrm{设三阶实对称矩阵的特征值为}λ_1=1,λ_2=4,λ_3=2,\mathrm{向量}x_1=\begin{pmatrix}1\\-1\\1\end{pmatrix},x_2=\begin{pmatrix}1\\2\\1\end{pmatrix},\mathrm{分别为}A\mathrm{的对应于}λ_1=1,λ_2=4\mathrm{的特征向量},\\则A\mathrm{的对应于}λ_3=2\mathrm{的特征向量}\;x_3是(\;).\end{array}
$$
$$
A.
x_1,x_2\mathrm{中的某一个} \quad B.{(1,1,-1)}^T \quad C.{(1,0,-1)}^T \quad D.{(1,1,1)}^T \quad E. \quad F. \quad G. \quad H.
$$
$$
\begin{array}{l}\mathrm{实对称矩阵不同特征值对应的特征向量正交},则x_3x_1^T=0,x_3x_2^T=0,\mathrm{可建立方程组直接求解};\;\\\mathrm{也可将选项中的向量代入},\mathrm{则只有向量}{(1,0,-1)}^T\mathrm{符合}.\end{array}
$$



$$
\begin{array}{l}\mathrm{设三阶实对称矩阵的特征值为}λ_1=-2,λ_2=9,λ_3=-18,\mathrm{向量}x_1=\begin{pmatrix}1\\-1\\0\end{pmatrix},x_2=\begin{pmatrix}2\\2\\-1\end{pmatrix},\mathrm{分别为}A\mathrm{的对应于}λ_1=-2,λ_2=9\mathrm{的特征向量},\\则A\mathrm{的对应于}λ_3=-18\mathrm{的特征向量}\;x_3是(\;).\end{array}
$$
$$
A.
x_1,x_2\mathrm{中的某一个} \quad B.{(1,1,-4)}^T \quad C.{(1,1,4)}^T \quad D.{(1,0,4)}^T \quad E. \quad F. \quad G. \quad H.
$$
$$
\begin{array}{l}\mathrm{实对称矩阵不同特征值对应的特征向量正交},则x_3x_1^T=0,x_3x_2^T=0,\mathrm{可建立方程组直接求解};\;\\\mathrm{也可将选项中的向量代入},\mathrm{则只有向量}{(1,1,4)}^T\mathrm{符合}.\end{array}
$$



$$
\begin{array}{l}\mathrm{设三阶实对称矩阵的}A\mathrm{特征值为}λ_1=1,λ_2=-1,\lambda_3=0,\mathrm{向量}x_1=\begin{pmatrix}a\\2a-1\\1\end{pmatrix},x_2=\begin{pmatrix}a\\1\\1-3a\end{pmatrix},\mathrm{分别为}A\mathrm{的对应于}λ_1=1,λ_2=-1\mathrm{的特征向量},\\则\;a\mathrm{等于}().\end{array}
$$
$$
A.
0或1 \quad B.1或-1 \quad C.1或2 \quad D.-1或2 \quad E. \quad F. \quad G. \quad H.
$$
$$
\mathrm{实对称矩阵不同特征值对应的特征向量正交},则x_2x_1^T=0,\mathrm{解得}a=0或1
$$



$$
\begin{array}{l}\mathrm{设三阶实对称矩阵的}A\mathrm{特征值为}λ_1=1,λ_2=-1,λ_3=0,\mathrm{向量}x_1=\begin{pmatrix}1\\a\\1\end{pmatrix},x_2=\begin{pmatrix}a\\a+1\\1\end{pmatrix},\mathrm{分别为}A\mathrm{的对应于}λ_1=1,λ_2=-1\mathrm{的特征向量},\\则\;a\mathrm{等于}().\end{array}
$$
$$
A.
-1 \quad B.1 \quad C.0 \quad D.0或1 \quad E. \quad F. \quad G. \quad H.
$$
$$
\mathrm{实对称矩阵不同特征值对应的特征向量正交},则x_2x_1^T=0,\mathrm{解得}a=-1
$$



$$
\mathrm{三阶实对称矩阵}A\mathrm{的特征值为}1,2,3;\mathrm{矩阵}A\mathrm{的属于特征值}1,2\mathrm{的特征向量分别是}α=\begin{pmatrix}-1&-1&1\end{pmatrix}^T,β=\begin{pmatrix}1&-2&-1\end{pmatrix}^T\;,则\;A\mathrm{的属于}\\\mathrm{特征值}3\mathrm{的全部特征向量是}()
$$
$$
A.
γ=k\begin{pmatrix}1&0&1\end{pmatrix}^T,k\neq0 \quad B.γ=k\begin{pmatrix}0&1&1\end{pmatrix}^T,k\neq0 \quad C.γ=k\begin{pmatrix}1&0&1\end{pmatrix}^T \quad D.γ=k\begin{pmatrix}0&1&1\end{pmatrix}^T \quad E. \quad F. \quad G. \quad H.
$$
$$
\begin{array}{l}\mathrm{实对称矩阵不同特征值对应的特征向量是正交的},\mathrm{所以设}γ=\begin{pmatrix}x&y&z\end{pmatrix}^T,则\left\{\begin{array}{l}αγ^T=0\\βγ^T=0\end{array}\right.,\mathrm{求解得其基础解系}\begin{pmatrix}1&0&1\end{pmatrix}^T,\\\mathrm{又因为特征向量为非零向量},\mathrm{所以求得特征向量为}γ=k\begin{pmatrix}1&0&1\end{pmatrix}^T,k\neq0\end{array}
$$



$$
设A为n\mathrm{阶方阵},\mathrm{以下结论中},()\mathrm{成立}.
$$
$$
A.
若A\mathrm{可逆},\mathrm{则矩阵}A\mathrm{的属于特征值}λ\mathrm{的特征向量也是矩阵}A^{-1}\mathrm{的属于特征值}\frac1λ\mathrm{的特征向量}. \quad B.A\mathrm{的特征向量即为方程组}(A-λ E)x=0\mathrm{的全部解}. \quad C.A\mathrm{的特征向量的线性组合仍为特征向量}. \quad D.A与A^T\mathrm{有相同的特征向量}. \quad E. \quad F. \quad G. \quad H.
$$
$$
\begin{array}{l}若A\mathrm{可逆},\mathrm{且对应特征值}λ\mathrm{的特征向量}x为,则Ax=λ x,\mathrm{两边乘以}A^{-1}有A^{-1}x=\frac1λ x;\\A\mathrm{的属于特征值}λ\mathrm{的特征向量为方程组}(A-λ E)x=0\mathrm{的非零解};\\\mathrm{属于不同特征值的特征向量的线性组合不为特征向量};\\因\left|A-λ E\right|=\left|(A-λ E)^T\right|=\left|A^T-λ E\right|,\mathrm{故矩阵}A与A^T\mathrm{有相同的特征值};\mathrm{用反证法说明矩阵}A与A^T\mathrm{有不同的特征向量}.\end{array}
$$



$$
\mathrm{已知矩阵}\begin{pmatrix}22&30\\-12&x\end{pmatrix}\mathrm{有一个特征向量}\begin{pmatrix}-5\\3\end{pmatrix},则x=()
$$
$$
A.
-18 \quad B.-16 \quad C.-14 \quad D.-12 \quad E. \quad F. \quad G. \quad H.
$$
$$
\mathrm{设题设中的特征向量所对应的特征值为}λ,\mathrm{则有}\begin{pmatrix}22-λ&30\\-12&x-λ\end{pmatrix}\begin{pmatrix}-5\\3\end{pmatrix}=\begin{pmatrix}0\\0\end{pmatrix},即\left\{\begin{array}{l}5λ-110+90=0\\60+3x-3λ=0\end{array}\right.\mathrm{解得}λ=4,x=-16.
$$



$$
\mathrm{已知}ξ_1,ξ_2\mathrm{是方程}(A-λ E)x=0\mathrm{两个不同的解向量},\mathrm{则下列向量中},\mathrm{必是}A\mathrm{的对应于特征值}λ\mathrm{的特征向量的是}(\;).
$$
$$
A.
ξ_1 \quad B.ξ_2 \quad C.ξ_1+ξ_2 \quad D.ξ_1-ξ_2 \quad E. \quad F. \quad G. \quad H.
$$
$$
\begin{array}{l}\mathrm{由特征向量的定义可知},A\mathrm{的对应于特征值}λ\mathrm{的特征向量是齐次线性方程组}(λ E-A)x=0\mathrm{的非零解},\mathrm{由于}ξ_1\neqξ_2,则ξ_1-ξ_2\neq0,\\且ξ_1-ξ_2是(λ E-A)x=0\mathrm{的解},\mathrm{因此是}λ\mathrm{的特征向量};\mathrm{其它选项中的向量不一定非零}.\end{array}
$$



$$
设A为n\mathrm{阶方阵},且A^k=O(k\mathrm{为正整数}),则()
$$
$$
A.
A=O \quad B.A\mathrm{有一个不为零的特征值} \quad C.A\mathrm{的特征值全为零} \quad D.A有n\mathrm{个线性无关的特征向量} \quad E. \quad F. \quad G. \quad H.
$$
$$
设A\mathrm{的特征值为}λ,\mathrm{则存在非零}a,有A^ka=λ^ka=0,则λ^k=0⇒λ=0,故A\mathrm{的特征值都为零}.
$$



$$
\begin{array}{l}\mathrm{设三阶实矩阵}A\mathrm{的三个特征值为}:λ_1=λ_2=1,λ_3=2,\mathrm{向量}β_1=(1,2,2)^T,β_2=(2,1,-2)^T及β_3=β_1+β_2=(3,3,0)^T\\\mathrm{都是}A\mathrm{的特征向量},\mathrm{则下述结论正确的是}(\;).\end{array}
$$
$$
A.
β_1,β_2,β_3\mathrm{都属于特征值}λ_1=1\mathrm{的特征向量} \quad B.β_1,β_2\mathrm{都属于特征值}λ_1=1\mathrm{的特征向量},而β_3\mathrm{是属于特征值}λ_3=2\mathrm{的特征向量} \quad C.\mathrm{由题设条件不能得出肯定判断} \quad D.\mathrm{题设诸条件互不相容} \quad E. \quad F. \quad G. \quad H.
$$
$$
\begin{array}{l}\mathrm{由条件可知}β_1,β_2\mathrm{线性无关},\mathrm{若是分别属于特征值}1,2\mathrm{的特征向量},则β_3=β_1+β_2\mathrm{不是}A\mathrm{的特征向量},\mathrm{与题意矛盾},\\\mathrm{因此}β_1,β_2\mathrm{是属于同一个特征值的特征向量};\\故β_1,β_2\mathrm{是属于}λ_1=1\mathrm{的特征向量},\\\mathrm{又属于同一特征值的特征向量的线性组合仍是此特征值的特征向量},故β_3=β_1+β_2\mathrm{也是属于}λ_1=1\mathrm{的特征向量}.\\\mathrm{因此}β_1,β_2,β_3\mathrm{都属于特征值}λ_1=1\mathrm{的特征向量}.\end{array}
$$



$$
\begin{array}{l}\mathrm{若三阶方阵}A\mathrm{的三个特征值为}1,2,-3,\mathrm{属于特征值}1\mathrm{的特征向量为}:β_1={(-1,1,1)}^T;\mathrm{属于特征值}2\mathrm{的特征向量为}β_2={(1,1,0)}^T,\\\mathrm{则向量}β=β_1-β_2={(-2,0,1)}^T()\end{array}
$$
$$
A.
是A\mathrm{的属于特征值}1\mathrm{的特征向量} \quad B.是A\mathrm{的属于特征值}2\mathrm{的特征向量} \quad C.是A\mathrm{的属于特征值}-3\mathrm{的特征向量} \quad D.\mathrm{不是}A\mathrm{的特征向量} \quad E. \quad F. \quad G. \quad H.
$$
$$
\begin{array}{l}\mathrm{矩阵}A\mathrm{属于不同特征值的特征向量的线性组合不是}A\mathrm{的特征向量},\mathrm{可用反证法证明}:设β\mathrm{是属于特征值}λ\mathrm{的特征向量},即\\Aβ=λβ⇒ A(β_1-β_2)=λ(β_1-β_2)\\Aβ_1-Aβ_2=λ_1β_1-λ_2β_2⇒β_1-2β_2=λβ_1-λβ_2,由β_1,β_2\mathrm{线性无关},则λ=1,λ=2,\mathrm{矛盾}.故β\mathrm{不是}A\mathrm{的特征向量}.\end{array}
$$



$$
\begin{array}{l}\mathrm{设三阶方阵}A\mathrm{的特征值为}λ_1=0,λ_2=3,λ_3=-6,\mathrm{对应于}λ_1=0\mathrm{的特征向量为}x_1={(1,0,1)}^T,\mathrm{对应于}λ_2=3\mathrm{的特征向量为}x_2={(2,1,1)}^T\\\mathrm{则向量}x_3=x_1+x_2={(3,1,2)}^T()\end{array}
$$
$$
A.
\mathrm{是对应于特征值}λ_1=0\mathrm{的特征向量} \quad B.\mathrm{是对应于特征值}λ_2=3\mathrm{的特征向量} \quad C.\mathrm{是对应于特征值}λ_3=-6\mathrm{的特征向量} \quad D.\mathrm{不是}A\mathrm{的特征向量} \quad E. \quad F. \quad G. \quad H.
$$
$$
\mathrm{矩阵}A\mathrm{属于不同特征值的特征向量的线性组合不是}A\mathrm{的特征向量},\mathrm{可用反证法证明}
$$



$$
\begin{array}{l}\mathrm{设三阶实对称矩阵}A\mathrm{的特征值为}λ_1=1,λ_2=λ_3=2,\mathrm{向量}x_1={(1,0,1)}^T,x_2={(2,0,1)}^T,x_3=2x_1-x_2={(0,0,1)}^T\mathrm{都是}A\mathrm{的特征向量}\\则\;\mathrm{下列结论正确的是}()\end{array}
$$
$$
A.
x_1,x_2,x_3\mathrm{都是对应于特征值}λ_2=λ_3=2\mathrm{的特征向量} \quad B.x_1,x_2\mathrm{是对应于特征值}λ_2=2\mathrm{的特征向量},而x_3\mathrm{是对应于}λ_1=1\mathrm{的特征向量} \quad C.\mathrm{由题设条件不能得出肯定判断} \quad D.\mathrm{题设诸条件互不相容} \quad E. \quad F. \quad G. \quad H.
$$
$$
\begin{array}{l}\mathrm{由于}x_1,x_2\mathrm{线性无关},\mathrm{但不正交},故x_1,x_2\mathrm{是实对称矩阵属于同一特征值的特征向量},\mathrm{又实对称矩阵特征值的重数等于属于此特征值}\\\mathrm{的特征向量的个数},故x_1,x_2\mathrm{是属于}λ_2=λ_3=2\mathrm{的特征向量},而x_3是x_1,x_2\mathrm{的线性组合},\mathrm{也属于}λ_2=λ_3=2\mathrm{的特征向量}.\\\mathrm{因此}x_1,x_2,x_3\mathrm{都是对应于特征值}λ_2=λ_3=2\mathrm{的特征向量}.\end{array}
$$



$$
\begin{array}{l}\mathrm{设实三阶方阵}A\mathrm{的特征值为}λ_1=-1,λ_2=1,λ_3=2,\mathrm{对应于}λ_1=-1\mathrm{的特征向量为}x_1={(1,1,0)}^T,\mathrm{对应于}λ_2=1\mathrm{的特征向量为}x_2={(-2,0,5)}^T\\\mathrm{则向量}x_3=3x_1-x_2={(5,3,-5)}^T()\end{array}
$$
$$
A.
\mathrm{是对应于特征值}λ_1=-1\mathrm{的特征向量} \quad B.\mathrm{是对应于特征值}λ_2=1\mathrm{的特征向量} \quad C.\mathrm{是对应于特征值}λ_3=2\mathrm{的特征向量} \quad D.\mathrm{不是}A\mathrm{的特征向量} \quad E. \quad F. \quad G. \quad H.
$$
$$
\mathrm{属于不同特征值的特征向量的线性组合不是原矩阵的特征向量},\mathrm{可用反证法证明}
$$



$$
λ_1,λ_2是n\mathrm{阶矩阵}A\mathrm{的特征值},λ_1\neqλ_2,且α_1,α_2\mathrm{分别是对应于}λ_1,λ_2\mathrm{的特征向量},当()时,α=k_1α_1+k_2α_2\mathrm{必是}A\mathrm{的特征向量}.
$$
$$
A.
k_1=0或k_2=0 \quad B.k_1\neq0且k_2\neq0 \quad C.k_1k_2=0 \quad D.k_{1、}k_2\mathrm{中有且只有一个为零} \quad E. \quad F. \quad G. \quad H.
$$
$$
\begin{array}{l}\mathrm{因为}k_1α_1+k_2α_2是A\mathrm{的特征向量},\mathrm{设其对应的特征值为}λ,则\\A(k_1α_1+k_2α_2)=λ(k_1α_1+k_2α_2),\\k_1λ_1α_1+k_2λ_2α_2=k_1λα_1+k_2λα_2,即k_1(λ_1-λ)α_1+k_2(λ_2-λ)α_2=0\\\mathrm{因为}α_1,α_2\mathrm{线性无关},\mathrm{所以}k_1(λ_1-λ)=0,k_2(λ_2-λ)=0.\mathrm{又因}λ_1\neqλ_2,\mathrm{所以}λ_1-λ\neqλ_2-λ,故k_1· k_2=0\\\mathrm{但特征向量不能为零向量},\mathrm{所以}k_1· k_2=0且k_1+k_2\neq0,即k_1,k_2\mathrm{中有且只有一个零}.\end{array}
$$



$$
\begin{array}{l}\mathrm{设实三阶方阵}A\mathrm{的特征值为}λ_1=1,λ_2=2,λ_3=3,\mathrm{对应于}λ_1=1\mathrm{的特征向量为}x_1={(1,0,1)}^T,\mathrm{对应于}λ_2=2\mathrm{的特征向量为}x_2={(0,1,1)}^T\\\mathrm{则向量}x_3=x_1-x_2={(1,-1,0)}^T()\end{array}
$$
$$
A.
\mathrm{是对应于特征值}λ_1=1\mathrm{的特征向量} \quad B.\mathrm{是对应于特征值}λ_2=2\mathrm{的特征向量} \quad C.\mathrm{是对应于特征值}λ_3=3\mathrm{的特征向量} \quad D.\mathrm{不是}A\mathrm{的特征向量} \quad E. \quad F. \quad G. \quad H.
$$
$$
\mathrm{属于不同特征值的特征向量的线性组合不是原矩阵的特征向量},\mathrm{可用反证法证明}
$$



$$
\mathrm{设向量}α=\begin{pmatrix}1&k&1\end{pmatrix}^T\mathrm{是矩阵}A=\begin{pmatrix}2&1&1\\1&2&1\\1&1&2\end{pmatrix}\;\mathrm{的逆矩阵}A^{-1}\mathrm{的特征向量},\mathrm{则常数}k为\;()
$$
$$
A.
1 \quad B.-2 \quad C.1或-2 \quad D.1或0 \quad E. \quad F. \quad G. \quad H.
$$
$$
\mathrm{可逆矩阵}A\mathrm{与其逆矩阵}A^{-1},\mathrm{具有相同的特征向量},\mathrm{所以}Aα=λα,\mathrm{可解得}k=-2或1
$$



$$
设A为2\mathrm{阶矩阵},α_1,α_2\mathrm{为线性无关得}2\mathrm{维向量},且Aα_1=0,Aα_2=2α_1+α_2,则A\mathrm{的非零特征值为}()
$$
$$
A.
1 \quad B.2 \quad C.3 \quad D.4 \quad E. \quad F. \quad G. \quad H.
$$
$$
\begin{array}{l}设\;λ\;为A\mathrm{对应于}α_2\mathrm{的特征值},则Aα_2=λα_2,A^2α_2=λ^2α_2,\\\mathrm{所以由}Aα_2=2α_1+α_2\mathrm{两边同时左乘以}A得λ^2α_2=λα_2,\\即(λ^2-λ)α_2=0,\\由α_1,α_2\mathrm{为线性无关可知}α_2\neq0,\mathrm{所以}λ^2-λ=0,\mathrm{即非零特征值为}1\end{array}
$$



$$
设λ_1,λ_2\mathrm{是矩阵}A\mathrm{的两个不同的特征值},\mathrm{对应的特征向量分别是}α_1,α_2,则α_1,A(α_1+\;α_2)\mathrm{线性无关的充要条件是}()
$$
$$
A.
λ_1\neq0 \quad B.λ_2\neq0 \quad C.λ_1=0 \quad D.λ_2=0 \quad E. \quad F. \quad G. \quad H.
$$
$$
(α_1,A(α_1+\;α_2))=(α_1,\;α_2)\begin{pmatrix}1&λ_1\\0&λ_2\end{pmatrix},\;α_1,A(α_1\;+α_2)\mathrm{线性无关等价于}\left|\begin{pmatrix}1&λ_1\\0&λ_2\end{pmatrix}\right|\neq0,即λ_2\neq0
$$



$$
\mathrm{已知}ξ=\begin{pmatrix}1\\1\\-1\end{pmatrix}\mathrm{是矩阵}A=\begin{pmatrix}2&-1&2\\5&a&3\\-1&b&-2\end{pmatrix}\mathrm{的一个特征向量},\mathrm{则其对应的特征值是}()
$$
$$
A.
-1 \quad B.0 \quad C.1 \quad D.2 \quad E. \quad F. \quad G. \quad H.
$$
$$
\begin{pmatrix}2&-1&2\\5&a&3\\-1&b&-2\end{pmatrix}\begin{pmatrix}1\\1\\-1\end{pmatrix}=λ\begin{pmatrix}1\\1\\-1\end{pmatrix}得λ=-1
$$



$$
\mathrm{矩阵}A=\begin{pmatrix}-1&0\\5&6\end{pmatrix}\mathrm{属于特征值}-1\mathrm{的特征向量为}()
$$
$$
A.
k\begin{pmatrix}7\\-5\end{pmatrix},k\neq0 \quad B.k\begin{pmatrix}7\\-5\end{pmatrix} \quad C.k\begin{pmatrix}0\\1\end{pmatrix},k\neq0 \quad D.k\begin{pmatrix}0\\1\end{pmatrix} \quad E. \quad F. \quad G. \quad H.
$$
$$
\mathrm{求解}(A+E)x=0\mathrm{的基础解系得}\begin{pmatrix}7\\-5\end{pmatrix},\mathrm{所以}A\mathrm{的属于}-1\mathrm{的特征向量为}k\begin{pmatrix}7\\-5\end{pmatrix},k\neq0
$$



$$
\mathrm{矩阵}A=\begin{pmatrix}1&2\\2&1\end{pmatrix}\mathrm{属于特征值}-1\mathrm{的特征向量为}()
$$
$$
A.
k\begin{pmatrix}1\\-1\end{pmatrix},k\neq0 \quad B.k\begin{pmatrix}1\\-1\end{pmatrix} \quad C.k\begin{pmatrix}1\\1\end{pmatrix},k\neq0 \quad D.k\begin{pmatrix}1\\1\end{pmatrix} \quad E. \quad F. \quad G. \quad H.
$$
$$
\mathrm{求解}(A+E)x=0\mathrm{的基础解系得}\begin{pmatrix}1\\-1\end{pmatrix},\mathrm{所以}A\mathrm{的属于}-1\mathrm{的特征向量为}k\begin{pmatrix}1\\-1\end{pmatrix},k\neq0
$$



$$
\mathrm{矩阵}A=\begin{pmatrix}1&2\\-1&4\end{pmatrix}\mathrm{属于特征值}2\mathrm{的特征向量为}()
$$
$$
A.
k\begin{pmatrix}2\\1\end{pmatrix},k\neq0 \quad B.k\begin{pmatrix}2\\1\end{pmatrix} \quad C.k\begin{pmatrix}2\\-1\end{pmatrix},k\neq0 \quad D.k\begin{pmatrix}2\\-1\end{pmatrix} \quad E. \quad F. \quad G. \quad H.
$$
$$
\mathrm{求解}(A-2E)x=0\mathrm{的基础解系得}\begin{pmatrix}2\\1\end{pmatrix},\mathrm{所以}A\mathrm{的属于}2\mathrm{的特征向量为}k\begin{pmatrix}2\\1\end{pmatrix},k\neq0
$$



$$
\mathrm{矩阵}A=\begin{pmatrix}1&-1\\-1&4\end{pmatrix}\mathrm{属于特征值}-1\mathrm{的特征向量为}()
$$
$$
A.
k\begin{pmatrix}1\\2\end{pmatrix},k\neq0 \quad B.k\begin{pmatrix}1\\2\end{pmatrix} \quad C.k\begin{pmatrix}2\\1\end{pmatrix},k\neq0 \quad D.k\begin{pmatrix}2\\1\end{pmatrix} \quad E. \quad F. \quad G. \quad H.
$$
$$
\mathrm{求解}(A+E)x=0\mathrm{的基础解系得}\begin{pmatrix}1\\2\end{pmatrix},\mathrm{所以}A\mathrm{的属于}-1\mathrm{的特征向量为}k\begin{pmatrix}1\\2\end{pmatrix},k\neq0
$$



$$
\mathrm{矩阵}A=\begin{pmatrix}3&3\\2&4\end{pmatrix}\mathrm{属于特征值}1\mathrm{的特征向量为}()
$$
$$
A.
k\begin{pmatrix}3\\-2\end{pmatrix},k\neq0 \quad B.k\begin{pmatrix}3\\2\end{pmatrix} \quad C.k\begin{pmatrix}-2\\3\end{pmatrix},k\neq0 \quad D.k\begin{pmatrix}-2\\3\end{pmatrix} \quad E. \quad F. \quad G. \quad H.
$$
$$
\mathrm{求解}(A-E)x=0\mathrm{的基础解系得}\begin{pmatrix}3\\-2\end{pmatrix},\mathrm{所以}A\mathrm{的属于}1\mathrm{的特征向量为}k\begin{pmatrix}3\\-2\end{pmatrix},k\neq0
$$



$$
设λ=2\mathrm{是非奇异矩阵}A\mathrm{的一个特征值},\mathrm{则矩阵}\begin{pmatrix}\frac13A^2\end{pmatrix}^{-1}\mathrm{有一个特征值为}()
$$
$$
A.
\frac43 \quad B.\frac34 \quad C.\frac12 \quad D.\frac14 \quad E. \quad F. \quad G. \quad H.
$$
$$
\begin{array}{l}\mathrm{设与}λ=2\mathrm{对应的特征向量为}x,则Ax=2x,\mathrm{两边同乘}A,得:A^2x=A(2x)=2Ax=2·2x=4x,\frac13A^2x=\frac43x\\由A^{-1}\mathrm{的特征值为}\frac1λ\mathrm{的性质可知},\begin{pmatrix}\frac13A^2\end{pmatrix}^{-1}\mathrm{的有特征值}\frac34\end{array}
$$



$$
\mathrm{已知可逆矩阵}A\mathrm{的一个特征值为}λ,则(2A)^{-1}\mathrm{的特征值为}()
$$
$$
A.
\frac1{2λ} \quad B.2λ \quad C.\frac2λ \quad D.\fracλ2 \quad E. \quad F. \quad G. \quad H.
$$
$$
\mathrm{由矩阵多项式的特征值性质可知}:(2A)^{-1}\mathrm{的特征值为}(2λ)^{-1}=\frac1{2λ}
$$



$$
若n\mathrm{阶矩阵}A\mathrm{任意一行的}n\mathrm{个元素之和都是}a,则A^3\mathrm{的一个特征值为}()
$$
$$
A.
a^3 \quad B.-a^3 \quad C.0 \quad D.a^{-3} \quad E. \quad F. \quad G. \quad H.
$$
$$
\mathrm{由题意可知}:A\begin{pmatrix}1\\1\\\vdots\\1\end{pmatrix}=\begin{pmatrix}a\\a\\\vdots\\a\end{pmatrix}=a\begin{pmatrix}1\\1\\\vdots\\1\end{pmatrix},\;\;\mathrm{由特征值与特征向量的定义可知},\mathrm{矩阵}A\mathrm{有一特征值}a.\mathrm{所以}A^3\mathrm{有一特征值}a^3
$$



$$
\mathrm{可逆矩阵}A\mathrm{与矩阵}()\mathrm{有相同的特征值}.
$$
$$
A.
A^T \quad B.A^{-1} \quad C.A^2 \quad D.A+E \quad E. \quad F. \quad G. \quad H.
$$
$$
\begin{array}{l}因\left|A-λ E\right|=\left|(A-λ E)^T\right|=\left|A^T-λ E\right|,故A与A^T\mathrm{有相同的特征值}.\\若λ\mathrm{为矩阵}A\mathrm{的特征值},\mathrm{则矩阵}A^{-1}\mathrm{的特征值为}\frac1λ,\mathrm{矩阵}A^2\mathrm{的特征值为}λ^2,\mathrm{矩阵}A+E\mathrm{的特征值为}λ+1\end{array}
$$



$$
\mathrm{已知}f(x)=x^2-2x+1,\mathrm{方阵}A\mathrm{的特征值为}1,0,-1,则f(A)\mathrm{的特征值为}()
$$
$$
A.
0,1,4 \quad B.1,0,-1 \quad C.1,0,1 \quad D.0,-1,-4 \quad E. \quad F. \quad G. \quad H.
$$
$$
\mathrm{由矩阵特征值的性质可知},若λ 是A\mathrm{的特征值},则f(λ)是f(A)\mathrm{的特征值},则f(A)\mathrm{的特征值分别为}f(1)=0,f(0)=1,f(-1)=4
$$



$$
\mathrm{已知}f(x)=x^2-2x-1,\mathrm{方阵}A\mathrm{的特征值为}1,0,-1,则f(A)\mathrm{的特征值为}()
$$
$$
A.
-2,-1,2 \quad B.-2,-1,-2 \quad C.2,1,2 \quad D.2,0,-2 \quad E. \quad F. \quad G. \quad H.
$$
$$
\begin{array}{l}\mathrm{由矩阵多项式的特征值性质可知},若λ\mathrm{是矩阵}A\mathrm{的特征值},则f(λ)\mathrm{为矩阵}f(A)\mathrm{多项式的特征值},则f(A)\mathrm{的特征值为}:\\f(1)=-2,f(0)=-1,f(-1)=2\end{array}
$$



$$
\mathrm{设三阶矩阵}A\mathrm{的特征值为}-1,3,4,则A\mathrm{的伴随矩阵}A^*\mathrm{的特征值为}()
$$
$$
A.
12,-4,-3 \quad B.1,\frac13,\frac14 \quad C.2,5,6 \quad D.-1,6,9 \quad E. \quad F. \quad G. \quad H.
$$
$$
AA^*=\left|A\right|E⇒ A^*=\left|A\right|A^{-1},而A=-1×3×4=-12,故A^*\mathrm{的特征值为}\frac{\left|A\right|}λ,即\frac{-12}{-1}=12,\frac{-12}3=-4,\frac{-12}4=-3
$$



$$
\mathrm{已知三阶矩阵}A\mathrm{的特征值为}-1,1,2,\mathrm{则矩阵}B=(3A^*)^{-1},则B\;\mathrm{的特征值为}()
$$
$$
A.
1,-1,-2 \quad B.\frac16,-\frac16,-\frac13 \quad C.-\frac16,\frac16,\frac13 \quad D.\frac12,-\frac12,-1 \quad E. \quad F. \quad G. \quad H.
$$
$$
\mathrm{由于}\left|A\right|=-2,\mathrm{则由伴随矩阵的性质可得}B=(3A^*)^{-1}=\frac1{3\left|A\right|}A=-\frac16A,则B\mathrm{的特征值为}\frac16,-\frac16,-\frac13
$$



$$
设A是3\mathrm{阶方阵},\mathrm{特征值分别为}0,1,2,则()
$$
$$
A.
\mathrm{矩阵}A\mathrm{的秩}r(A)=3 \quad B.\mathrm{行列式}\left|A^TA\right|=1 \quad C.\mathrm{行列式}\left|A+E\right|=0 \quad D.\mathrm{矩阵}(A+E)^{-1}\mathrm{的特征值分别为}1,\frac12,\frac13 \quad E. \quad F. \quad G. \quad H.
$$
$$
\begin{array}{l}3\mathrm{阶矩阵}A有3\mathrm{个不同的特征值},\mathrm{因此与对角矩阵相似},\mathrm{相似矩阵的秩相等},\mathrm{由于非零特征值为}2个,故r(A)=2;\\\left|A^TA\right|=\left|A^T\right|\left|A\right|=\left|A\right|^2,\mathrm{由于}\left|A\right|=0×1×2=0,故\left|A^TA\right|=0;\\\mathrm{矩阵}\left|A+E\right|\mathrm{的特征值为}1,2,3,故\left|A+E\right|=1×2×3=6,(A+E)^{-1}\mathrm{的特征值为}1,\frac12,\frac13\end{array}
$$



$$
设A是n\mathrm{阶方阵},\mathrm{如果}\left|A\right|=0,则A\mathrm{的特征值}()
$$
$$
A.
\mathrm{全为零} \quad B.\mathrm{全不为零} \quad C.\mathrm{至少有一个是零} \quad D.\mathrm{可以是任意数} \quad E. \quad F. \quad G. \quad H.
$$
$$
\left|A\right|=0⇒\left|A\right|=λ_1⋯λ_n=0,\mathrm{则矩阵}A\mathrm{至少有一个特征值为}0
$$



$$
设λ=2\mathrm{是可逆矩阵}A\mathrm{的一个特征值},\mathrm{则矩阵}E+2(A^{-1})^3\mathrm{有一个特征值等于}()
$$
$$
A.
\frac14 \quad B.\frac54 \quad C.5 \quad D.\frac45 \quad E. \quad F. \quad G. \quad H.
$$
$$
若λ\mathrm{是可逆矩阵}A\mathrm{的一个特征值},\mathrm{则矩阵}E+2(A^{-1})^3\mathrm{的特征值为}1+\frac2{λ^3},\mathrm{则所求矩阵有一个特征值}1+\frac28=\frac54
$$



$$
\mathrm{设三阶方阵}A\mathrm{的特征值为}1,2,3,则6A^*\mathrm{的三个特征值为}()
$$
$$
A.
6,12,36 \quad B.6,3,2 \quad C.1,\frac12,\frac13 \quad D.36,18,12 \quad E. \quad F. \quad G. \quad H.
$$
$$
\mathrm{由于}AA^*=\left|A\right|E,则A^*=\left|A\right|A^{-1},又\left|A\right|=1×2×3=6,故6A^*=6\left|A\right|A^{-1}=36A^{-1},\mathrm{故有特征值}\frac{36}λ,即36,18,12
$$



$$
A为n\mathrm{阶方阵},Ax=0\mathrm{有非零解},则A\mathrm{必有一个特征值是}()
$$
$$
A.
0 \quad B.1 \quad C.n \quad D.2 \quad E. \quad F. \quad G. \quad H.
$$
$$
Ax=0\mathrm{有非零解}\Leftrightarrow\left|A\right|=0;\mathrm{因为}\left|A\right|=λ_1·λ_2⋯⋯λ_n=0,\mathrm{所以}A\mathrm{必有一个特征值等于}0
$$



$$
\mathrm{设矩阵}A=\begin{pmatrix}1&1&0\\1&0&1\\0&1&1\end{pmatrix},则A\mathrm{的特征值为}()
$$
$$
A.
1,0,1 \quad B.1,1,2 \quad C.-1,1,2 \quad D.-1,1,1 \quad E. \quad F. \quad G. \quad H.
$$
$$
\left|A-λ E\right|=\begin{vmatrix}1-λ&1&0\\1&-λ&1\\0&1&1-λ\end{vmatrix}=-(λ-1)(λ-2)(λ+1)=0,故λ_1=1,λ_2=-1,λ_3=2
$$



$$
n\mathrm{阶矩阵}A以0\mathrm{为其特征值是}A\mathrm{为奇异矩阵的}()
$$
$$
A.
\mathrm{充分非必要条件} \quad B.\mathrm{必要非充分条件} \quad C.\mathrm{既非充分也非必要条件} \quad D.\mathrm{充分必要条件} \quad E. \quad F. \quad G. \quad H.
$$
$$
\begin{array}{l}\mathrm{是充分必要条件},若λ=0\mathrm{为矩阵}A\mathrm{的特征值},\mathrm{则存在非零向量}α,\mathrm{使得}Aα=0α=0,\mathrm{即齐次线性方程组}Ax=0\mathrm{有非零解},故\left|A\right|=0\\\mathrm{反之},若A\mathrm{是奇异矩阵},即\left|A\right|=0,则\left|A-0E\right|=0,即λ=0\mathrm{为矩阵}A\mathrm{的特征值}.\end{array}
$$



$$
\mathrm{设矩阵}A=\begin{pmatrix}1&-3&3\\3&a&3\\6&-6&b\end{pmatrix}\mathrm{有特征值}λ_1=-2,λ_2=4,\mathrm{则参数}a,b\mathrm{的值为}()
$$
$$
A.
-5,4 \quad B.-5,-4 \quad C.5,4 \quad D.5,-4 \quad E. \quad F. \quad G. \quad H.
$$
$$
\begin{array}{l}由\left|λ E-A\right|=0,\mathrm{可求参数}a,b\mathrm{的值};\mathrm{因为}λ_1=-2,λ_2=4\mathrm{均为}A\mathrm{的特征值}.\\\mathrm{所以}\left|A-λ_1E\right|=\begin{vmatrix}1-λ_1&-3&3\\3&a-λ_1&3\\6&-6&b-λ_1\end{vmatrix}=\begin{vmatrix}3&-3&3\\3&2+a&3\\6&-6&2+b\end{vmatrix}=-3(5+a)(4-b)=0\\\left|A-λ_2E\right|=\begin{vmatrix}-3&-3&3\\3&a-4&3\\6&-6&b-4\end{vmatrix}=-3\lbrack-(7-a)(2+b)+72\rbrack=0,\mathrm{解得}a=-5,b=4.\end{array}
$$



$$
\mathrm{设矩阵}A=\begin{pmatrix}1&2&3\\1&0&1\\0&0&1\end{pmatrix},则A\mathrm{的特征值为}()
$$
$$
A.
1,-1,1 \quad B.1,-1,2 \quad C.2,-1,1 \quad D.2,-1,2 \quad E. \quad F. \quad G. \quad H.
$$
$$
\left|A-λ E\right|=\begin{vmatrix}1-λ&2&3\\1&-\lambda&1\\0&0&1-λ\end{vmatrix}=-(λ-1)(λ-2)(λ+1)=0,故λ_1=1,λ_2=-1,λ_3=2
$$



$$
\mathrm{设矩阵}A=\begin{pmatrix}4&-1&0\\6&-1&-2\\0&0&3\end{pmatrix},则A\mathrm{的特征值为}()
$$
$$
A.
1,2,1 \quad B.1,2,2 \quad C.1,2,3 \quad D.1,2,4 \quad E. \quad F. \quad G. \quad H.
$$
$$
\left|A-λ E\right|=\begin{vmatrix}4-λ&-1&0\\6&-λ-1&-2\\0&0&3-λ\end{vmatrix}=-(λ-1)(λ-2)(λ-3)=0,故\lambda_1=1,λ_2=2,λ_3=3
$$



$$
设3\mathrm{阶矩阵满足}A^3+2A^2-A-2E=0,则A\mathrm{的特征值为}()
$$
$$
A.
-1,1,2 \quad B.-1,1,-2 \quad C.-1,-1,-2 \quad D.-1,-1,2 \quad E. \quad F. \quad G. \quad H.
$$
$$
\begin{array}{l}设λ 为A\mathrm{的特征值},则λ^3+2λ^2-λ-2=(λ+2)(λ+1)(λ-1)=0\\\mathrm{所以}A\mathrm{的特征值为}-1,1,-2\end{array}
$$



$$
设3\mathrm{阶矩阵满足}A^3+4A^2+5A+2E=0,则A\mathrm{的特征值为}()
$$
$$
A.
-1,1,2 \quad B.-1,1,-2 \quad C.-1,-1,-2 \quad D.-1,-1,2 \quad E. \quad F. \quad G. \quad H.
$$
$$
\begin{array}{l}设λ 为A\mathrm{的特征值},则λ^3+4λ^2+5λ+2=(λ+2)(λ+1)(λ+1)=0\\\mathrm{所以}A\mathrm{的特征值为}-1,-1,-2\end{array}
$$



$$
设3\mathrm{阶矩阵满足}A^3-3A+2E=0,则A\mathrm{的特征值为}()
$$
$$
A.
-1,1,2 \quad B.-1,1,-2 \quad C.-1,-1,-2 \quad D.1,1,-2 \quad E. \quad F. \quad G. \quad H.
$$
$$
\begin{array}{l}设λ 为A\mathrm{的特征值},则λ^3-3λ+2=(λ+2)(λ-1)(λ-1)=0\\\mathrm{所以}A\mathrm{的特征值为}1,1,-2\end{array}
$$



$$
设3\mathrm{阶矩阵满足}A^3-3A^2+4E=0,则A\mathrm{的特征值为}()
$$
$$
A.
-1,2,2 \quad B.-1,-2,2 \quad C.-1,-2,-2 \quad D.1,2,2 \quad E. \quad F. \quad G. \quad H.
$$
$$
\begin{array}{l}设λ 为A\mathrm{的特征值},则λ^3-3λ^2+4=(λ+1)(λ-2)(λ-2)=0\\\mathrm{所以}A\mathrm{的特征值为}-1,2,2\end{array}
$$



$$
设3\mathrm{阶矩阵满足}A^3+3A^2-4E=0,则A\mathrm{的特征值为}()
$$
$$
A.
-1,-2,-2 \quad B.1,-2,-2 \quad C.1,2,-2 \quad D.1,2,2 \quad E. \quad F. \quad G. \quad H.
$$
$$
\begin{array}{l}设λ 为A\mathrm{的特征值},则λ^3+3λ^2-4=(λ+2)(λ+2)(λ-1)=0\\\mathrm{所以}A\mathrm{的特征值为}1,-2,-2\end{array}
$$



$$
\mathrm{已知}f(x)=x^2+2x-1,\mathrm{方阵}A\mathrm{的特征值为}1,0,-1,则f(A)\mathrm{的特征值为}()
$$
$$
A.
2,-1,2 \quad B.2,-1,-2 \quad C.2,1,2 \quad D.2,0,-2 \quad E. \quad F. \quad G. \quad H.
$$
$$
\begin{array}{l}\mathrm{由矩阵多项式的特征值性质可知},若λ\mathrm{是矩阵}A\mathrm{的特征值},则f(λ)\mathrm{为矩阵}f(A)\mathrm{多项式的特征值},则f(A)\mathrm{的特征值为}:\\f(1)=2,f(0)=-1,f(-1)=-2\end{array}
$$



$$
\mathrm{已知}f(x)=x^2+2x+1,\mathrm{方阵}A\mathrm{的特征值为}1,0,-1,则f(A)\mathrm{的特征值为}()
$$
$$
A.
4,-1,0 \quad B.4,1,0 \quad C.-4,1,0 \quad D.-4,-1,1 \quad E. \quad F. \quad G. \quad H.
$$
$$
\begin{array}{l}\mathrm{由矩阵多项式的特征值性质可知},若λ\mathrm{是矩阵}A\mathrm{的特征值},则f(λ)\mathrm{为矩阵}f(A)\mathrm{多项式的特征值},则f(A)\mathrm{的特征值为}:\\f(1)=4,f(0)=1,f(-1)=0\end{array}
$$



$$
\mathrm{已知}f(x)=x^2+x+2,\mathrm{方阵}A\mathrm{的特征值为}1,0,-1,则f(A)\mathrm{的特征值为}()
$$
$$
A.
4,2,0 \quad B.4,2,2 \quad C.-4,2,0 \quad D.-4,-2,2 \quad E. \quad F. \quad G. \quad H.
$$
$$
\begin{array}{l}\mathrm{由矩阵多项式的特征值性质可知},若λ\mathrm{是矩阵}A\mathrm{的特征值},则f(λ)\mathrm{为矩阵}f(A)\mathrm{多项式的特征值},则f(A)\mathrm{的特征值为}:\\f(1)=4,f(0)=2,f(-1)=2\end{array}
$$



$$
\mathrm{设三阶矩阵}A\mathrm{的特征值为}-1,3,4,则A\mathrm{的伴随矩阵}(A^*)^2\mathrm{的特征值为}()
$$
$$
A.
144,16,9 \quad B.144,\frac19,\frac1{16} \quad C.12,16,9 \quad D.-12,16,9 \quad E. \quad F. \quad G. \quad H.
$$
$$
AA^*=\left|A\right|E⇒ A^*=\left|A\right|A^{-1},而\left|A\right|=-1×3×4=-12,故(A^*)^2\mathrm{的特征值为}(\frac{\left|A\right|}λ)^2,即(\frac{-12}{-1})^2=144,(\frac{-12}3)^2=16,(\frac{-12}4)^2=9
$$



$$
\mathrm{设三阶矩阵}A\mathrm{的特征值为}-1,3,4,\mathrm{则矩阵}\left|A\right|A^*\mathrm{的特征值为}()
$$
$$
A.
-144,-48,9 \quad B.-144,-48,-36 \quad C.144,-48,36 \quad D.-144,48,36 \quad E. \quad F. \quad G. \quad H.
$$
$$
AA^*=\left|A\right|E⇒ A^*=\left|A\right|A^{-1},而\left|A\right|=-1×3×4=-12,故\left|A\right|A^*\mathrm{的特征值为}\frac{\left|A\right|^2}λ,即\frac{(-12)^2}{-1}=-144,\frac{(-12)^2}3=48,\frac{(-12)^2}4=36
$$



$$
\mathrm{设三阶矩阵}A\mathrm{的特征值为}-1,3,4,则A\mathrm{的伴随矩阵}A^*+E\mathrm{的特征值为}()
$$
$$
A.
12,-4,-3 \quad B.13,-3,-2 \quad C.13,-3,2 \quad D.12,-4,3 \quad E. \quad F. \quad G. \quad H.
$$
$$
AA^*=\left|A\right|E⇒ A^*=\left|A\right|A^{-1},而\left|A\right|=-1×3×4=-12,故A^*+E\mathrm{的特征值为}\frac{\left|A\right|}λ+1,即\frac{-12}{-1}+1=13,\frac{-12}3+1=-3,\frac{-12}4+1=-2
$$



$$
\mathrm{设矩阵}A=\begin{pmatrix}1&0&0\\0&1&8\\0&1&3\end{pmatrix},则A\mathrm{的特征值为}()
$$
$$
A.
1,1,-1 \quad B.1,-1,-1 \quad C.1,5,1 \quad D.1,5,-1 \quad E. \quad F. \quad G. \quad H.
$$
$$
\left|A-λ E\right|=\begin{vmatrix}1-λ&0&0\\0&1-λ&8\\0&1&3-λ\end{vmatrix}=-(λ-1)(λ-5)(λ+1)=0,故λ_1=1,λ_2=5,λ_3=-1
$$



$$
\mathrm 设n\;\mathrm{阶矩阵}A\;\mathrm{的元素全为}1,\mathrm 则A\mathrm{的一个}n-1\mathrm{的重特征值是}\;()
$$
$$
A.
0 \quad B.1 \quad C.2 \quad D.n \quad E. \quad F. \quad G. \quad H.
$$
$$
\left|A-λ E\right|=\begin{pmatrix}1-λ&1&…&1\\1&1-λ&⋯&1\\\vdots&\vdots&\vdots&\vdots\\1&1&1&1-λ\end{pmatrix}=(n-λ){(-λ)}^{n-1}=0,\mathrm{所以特征值为}n,0(n-1重)
$$



$$
\mathrm{设矩阵}A=\begin{pmatrix}0&1&0&0\\1&0&0&0\\0&0&y&1\\0&0&1&2\end{pmatrix},\mathrm{已知}\;\mathrm{的一个特征值为}3,则\;y\mathrm{的值为}(\;\;)
$$
$$
A.
1 \quad B.2 \quad C.3 \quad D.4 \quad E. \quad F. \quad G. \quad H.
$$
$$
\left|A-3E\right|=0\mathrm{可得}y=2
$$



$$
\mathrm{设四阶矩阵}A\mathrm{满足}\left|3E+A\right|=0,\;AA^T=2E,\left|A\right|<0,\;\mathrm{其中}E\mathrm{为四阶单位矩阵},\mathrm{则伴随矩阵}A^*\mathrm{的一个特征值为}\;()
$$
$$
A.
3 \quad B.\frac43 \quad C.4 \quad D.\frac34 \quad E. \quad F. \quad G. \quad H.
$$
$$
由\left|3E+A\right|=0\mathrm{可得}-3是A\mathrm{的一个特征值},\mathrm{又由}AA^T=2E及\left|A\right|<0,\mathrm{可知}\left|A\right|=-4,\mathrm{所以}A^*\mathrm{的特征值为}\frac{\left|A\right|}λ=\frac43
$$



$$
\mathrm{设向量}α=\begin{pmatrix}1,&1&,-1\end{pmatrix}^T\mathrm{是矩阵}A=\begin{pmatrix}2&-1&2\\5&a&3\\-1&b&-2\end{pmatrix}\;\mathrm{的一个特征向量},λ 为α\mathrm{对应的特征值},则a,b,λ\mathrm{分别为}\;()
$$
$$
A.
a=-3,b=0,λ=-1 \quad B.a=0,b=-3,λ=-1 \quad C.a=-3,b=-1,λ=0 \quad D.a=-1,b=0,λ=-3 \quad E. \quad F. \quad G. \quad H.
$$
$$
Aα=λα,\mathrm{可解得}a=-3,b=0,λ=-1
$$



$$
\mathrm{设三阶方阵}A\mathrm{的特征值为}:λ_1=1,λ_2=-1,λ_3=2,B=2A+E,则B\mathrm{特征值为}()
$$
$$
A.
3,-1,5 \quad B.1,-1,2 \quad C.2,1,-3 \quad D.5,1,-2 \quad E. \quad F. \quad G. \quad H.
$$
$$
\begin{array}{l}设A\mathrm{的特征值为}λ,\mathrm{则矩阵}B\mathrm{的特征值为}2λ+1,\mathrm{故可求出}B\mathrm{的特征值分别为}3,-1,5\\\end{array}
$$



$$
\mathrm{设三阶方阵}A\mathrm{的特征值为}:λ_1=1,λ_2=-1,λ_3=2,B=A^2-2A+E,则B\mathrm{特征值为}()
$$
$$
A.
0,1,4 \quad B.-1,0,1 \quad C.0,2,4 \quad D.1,2,4 \quad E. \quad F. \quad G. \quad H.
$$
$$
\begin{array}{l}设A\mathrm{的特征值为}λ,\mathrm{则矩阵}B\mathrm{的特征值为}λ^2-2λ+1,\mathrm{故可求出}B\mathrm{的特征值分别为}0,4,1\\\end{array}
$$



$$
\mathrm{设三阶方阵}A\mathrm{的特征值为}:λ_1=-1,λ_2=0,λ_3=1,B=A^3-E,则B\mathrm{特征值为}()
$$
$$
A.
-2,-1,0 \quad B.-1,0,1 \quad C.-3,-2,-1 \quad D.0,1,2 \quad E. \quad F. \quad G. \quad H.
$$
$$
\begin{array}{l}设A\mathrm{的特征值为}λ,\mathrm{则矩阵}B\mathrm{的特征值为}λ^3-1,\mathrm{故可求出}B\mathrm{的特征值分别为}-2,-1,0\\\end{array}
$$



$$
\mathrm{已知}3\mathrm{阶矩阵}\;\mathrm{的特征值为}1,3,5,\mathrm{则行列式}\left|A-2E\right|=()
$$
$$
A.
0 \quad B.-1 \quad C.-2 \quad D.-3 \quad E. \quad F. \quad G. \quad H.
$$
$$
A-2E\mathrm{的特征值为}-1,1,3,\mathrm{所以}\left|A-2E\right|=-3
$$



$$
\mathrm{已知}3\mathrm{阶矩阵}A\mathrm{的特征值为}1,2,3,则A^2+E\mathrm{的迹}\;tr(A^2+E)=()
$$
$$
A.
15 \quad B.17 \quad C.19 \quad D.21 \quad E. \quad F. \quad G. \quad H.
$$
$$
A^2+E\mathrm{的特征值为}2,5,10,\mathrm{所以}tr(A^2+E)=2+5+10=17
$$



$$
\mathrm{已知}3\mathrm{阶矩阵}A\mathrm{的行列式}\left|A\right|\;<\;0,\;\mathrm{且满足关系式}A^2-A-2E=0,则A\mathrm{的所有特征值为}()
$$
$$
A.
2,2,-1 \quad B.-2,1,1 \quad C.-2,1 \quad D.2,-1 \quad E. \quad F. \quad G. \quad H.
$$
$$
\begin{array}{l}设λ 是A\mathrm{的特征值},则λ^2-λ-2=0,即λ=-1,2\\\mathrm{由因为}\left|A\right|\;<\;0,\mathrm{所以}A\mathrm{的特征值为}\;2,2,-1\end{array}
$$



$$
\mathrm{矩阵}A=\begin{pmatrix}0&-2&-2\\2&2&-2\\-2&-2&2\end{pmatrix}\mathrm{的非零特征值是}()
$$
$$
A.
1 \quad B.2 \quad C.3 \quad D.4 \quad E. \quad F. \quad G. \quad H.
$$
$$
解\left|A-λ E\right|=0得λ=4,0,0
$$



$$
\mathrm{已知矩阵}A=\begin{pmatrix}1&0&2\\0&2&0\\2&0&-2\end{pmatrix},则A\mathrm{所有特征值为}(\;)
$$
$$
A.
1,2,-2 \quad B.2,2,-3 \quad C.1,4,-4 \quad D.2,3,-4 \quad E. \quad F. \quad G. \quad H.
$$
$$
\mathrm{求解}\left|A-λ E\right|=0\mathrm{可得特征值}2,2,-3
$$



$$
\mathrm{已知矩阵}A=\begin{pmatrix}-1&1&0\\-4&3&0\\1&0&2\end{pmatrix},则A\mathrm{所有特征值为}(\;)
$$
$$
A.
1,1,2 \quad B.1,2,-2 \quad C.1,2,-1 \quad D.-1,-1,2 \quad E. \quad F. \quad G. \quad H.
$$
$$
\mathrm{求解}\left|A-λ E\right|=0\mathrm{可得特征值}1,1,2
$$



$$
n\mathrm{阶矩阵}A\mathrm{与对角矩阵相似的充分必要条件是}()
$$
$$
A.
A有n\mathrm{个特征值} \quad B.A有n\mathrm{个线性无关的特征向量} \quad C.A\mathrm{的行列式}\left|A\right|\neq0 \quad D.A\mathrm{特征多项式没有重根} \quad E. \quad F. \quad G. \quad H.
$$
$$
n\mathrm{阶矩阵}A\mathrm{与对角矩阵相似的充分必要条件是}A有n\mathrm{个线性无关的特征向量}
$$



$$
设A=\begin{pmatrix}1&-1&1\\2&4&a\\-3&-3&5\end{pmatrix},且A\mathrm{的特征值为}λ_1=6,λ_2=λ_3=2,若A有3\mathrm{个线性无关的特征向量},则a=(\;)
$$
$$
A.
2 \quad B.-2 \quad C.4 \quad D.-4 \quad E. \quad F. \quad G. \quad H.
$$
$$
\begin{array}{l}\mathrm{由题设可知},\mathrm{矩阵可对角化},\mathrm{因此对应于特征值}2\mathrm{的线性无关的特征向量有}2个,则r(A-2E)=n-2=1\\A-2E=\begin{pmatrix}-1&-1&1\\2&2&a\\-3&-3&3\end{pmatrix}\rightarrow\begin{pmatrix}-1&-1&1\\0&0&a+2\\0&0&0\end{pmatrix}故a+2=0⇒ a=-2\end{array}
$$



$$
\mathrm{设三阶矩阵}A\mathrm{与对角阵}\begin{pmatrix}1&&\\&-2&\\&&3\end{pmatrix}\mathrm{相似},则()
$$
$$
A.
1,-2,3\mathrm{都是}A\mathrm{的特征值} \quad B.1,-2,3\mathrm{都不是}A\mathrm{的特征值} \quad C.1,-2,3\mathrm{三数中只有一个是}A\mathrm{的特征值},\mathrm{另二个不是}A\mathrm{的特征值} \quad D.1,-2,3\mathrm{三数中只有二个是}A\mathrm{的特征值},\mathrm{另一个不是}A\mathrm{的特征值} \quad E. \quad F. \quad G. \quad H.
$$
$$
\mathrm{由于相似矩阵有相同的特征值},\mathrm{且对角矩阵的对角元素为其特征值},故1,-2,3\mathrm{都是}A\mathrm{的特征值}
$$



$$
\mathrm{设三阶方阵}A\mathrm{的特征值为}1,2,-1,B=A^3-2A^2-A+2E,则B是()
$$
$$
A.
\mathrm{满秩阵} \quad B.r(B)=2 \quad C.r(B)=1 \quad D.B=0 \quad E. \quad F. \quad G. \quad H.
$$
$$
\begin{array}{l}\mathrm{由题设可知},\mathrm{存在可逆矩阵}P,\mathrm{使得}P^{-1}AP=\begin{pmatrix}1&&\\&2&\\&&-1\end{pmatrix},则\\B=P\left[\begin{array}{c}\begin{pmatrix}1&&\\&2&\\&&-1\end{pmatrix}^3-2\end{array}\begin{pmatrix}1&&\\&2&\\&&-1\end{pmatrix}^2-\begin{pmatrix}1&&\\&2&\\&&-1\end{pmatrix}+2\begin{pmatrix}1&&\\&1&\\&&1\end{pmatrix}\right]P^{-1}=0\end{array}
$$



$$
n\mathrm{阶矩阵}A\mathrm{具有个}n\mathrm{不同特征值是}A\mathrm{与对角矩阵相似的}()
$$
$$
A.
\mathrm{充分必要条件} \quad B.\mathrm{充分而非必要条件} \quad C.\mathrm{必要而非充分条件} \quad D.\mathrm{既非充分也非必要条件} \quad E. \quad F. \quad G. \quad H.
$$
$$
若n\mathrm{阶矩阵}A\mathrm{具有个}n\mathrm{不同特征值},\mathrm{则一定有}n\mathrm{个线性无关的特征向量},\mathrm{从而必相似于对角矩阵},\mathrm{但反之不成立},\mathrm{因此是充分而非必要条件}.
$$



$$
n\mathrm{阶方阵}A\mathrm{与某对角矩阵相似},则()
$$
$$
A.
\mathrm{方阵}A\mathrm{的秩等于}n \quad B.\mathrm{方阵}A有n\mathrm{个不同的特征值} \quad C.\mathrm{方阵}A\mathrm{一定是对称矩阵} \quad D.\mathrm{方阵}A有n\mathrm{个线性无关的特征向量} \quad E. \quad F. \quad G. \quad H.
$$
$$
\begin{array}{l}\mathrm{选项中只有}“\mathrm{方阵}A有n\mathrm{个线性无关的特征向量}”\mathrm{是矩阵}A\mathrm{与对角矩阵相似的充要条件},\\\mathrm{其它选项中的结论都不能由}“\mathrm{矩阵}A\mathrm{与对角矩阵相似}”\mathrm{而得到}.\end{array}
$$



$$
\mathrm{已知}A=\begin{pmatrix}0&0&1\\x&1&0\\1&0&0\end{pmatrix}\mathrm{有三个线性无关的特征向量},则x=(\;)
$$
$$
A.
0 \quad B.1 \quad C.-1 \quad D.3 \quad E. \quad F. \quad G. \quad H.
$$
$$
\begin{array}{l}由A\mathrm{的特征方程}\left|A-λ E\right|=\begin{vmatrix}-λ&0&1\\x&1-λ&0\\1&0&-λ\end{vmatrix}=-(λ-1)(λ^2-1)=0\mathrm{得到特征值}\;λ=1(\mathrm{二重}),λ=-1\\\mathrm{因为}A有3\mathrm{个线性无关的特征向量},故A\mathrm{可对角化},即λ=1\mathrm{必须有两个线性无关的特征向量}.\mathrm{那么},\mathrm{必有}r(A-E)=3-2=1.\\\mathrm{于是}A-E=\begin{pmatrix}1&0&-1\\-x&0&0\\-1&0&1\end{pmatrix}⇒\begin{pmatrix}1&0&-1\\-x&0&0\\0&0&0\end{pmatrix},得x=0.\end{array}
$$



$$
\mathrm{已知}A=\begin{pmatrix}-1&1&0\\-2&2&0\\4&x&1\end{pmatrix}\mathrm{能对角化},则x=(\;)
$$
$$
A.
-2 \quad B.-1 \quad C.0 \quad D.1 \quad E. \quad F. \quad G. \quad H.
$$
$$
\begin{array}{l}\mathrm{因为}A\mathrm{能对角化},A\mathrm{必有三个线性无关的特征向量},\mathrm{由于}\left|A-λ E\right|=\begin{vmatrix}-λ-1&1&0\\-2&2-λ&0\\4&x&1-λ\end{vmatrix}=-(λ-1)(λ^2-λ)\\λ=1\mathrm{是二重特征值},\mathrm{必有两个线性无关的特征向量},\mathrm{因此}r(A-E)=1,得x=-2.\end{array}
$$



$$
设A\mathrm{是三阶矩阵},A\mathrm{有特征值}λ_1=0,λ_2=-1,λ_3=1,\mathrm{其对应的特征向量分别为}ξ_1,ξ_2,ξ_3,设P=(ξ_1,ξ_2,ξ_3),则P^{-1}AP=\;()
$$
$$
A.
\begin{pmatrix}-1&0&0\\0&1&0\\0&0&0\end{pmatrix} \quad B.\begin{pmatrix}1&0&0\\0&-1&0\\0&0&0\end{pmatrix} \quad C.\begin{pmatrix}0&0&0\\0&-1&0\\0&0&1\end{pmatrix} \quad D.\begin{pmatrix}0&0&0\\0&1&0\\0&0&-1\end{pmatrix} \quad E. \quad F. \quad G. \quad H.
$$
$$
\begin{array}{l}AP=A(ξ_1,ξ_2,ξ_3)=(Aξ_1,Aξ_2,Aξ_3)=(0×ξ_1,-1×ξ_2,1×ξ_3)=(ξ_1,ξ_2,ξ_3)\begin{pmatrix}0&0&0\\0&-1&0\\0&0&1\end{pmatrix}=P\begin{pmatrix}0&0&0\\0&-1&0\\0&0&1\end{pmatrix}\\\mathrm{所以}P^{-1}AP=\begin{pmatrix}0&0&0\\0&-1&0\\0&0&1\end{pmatrix}\end{array}
$$



$$
设A∼\begin{pmatrix}\frac13&2&0\\0&\frac14&-1\\0&0&\frac15\end{pmatrix},则\underset{n\rightarrow∞}{lim}A^n=()
$$
$$
A.
\begin{pmatrix}\frac13&2&0\\0&\frac14&-1\\0&0&\frac15\end{pmatrix} \quad B.\begin{pmatrix}\frac13&0&0\\0&\frac14&0\\0&0&\frac15\end{pmatrix} \quad C.\begin{pmatrix}1&0&0\\0&1&0\\0&0&1\end{pmatrix} \quad D.\begin{pmatrix}0&0&0\\0&0&0\\0&0&0\end{pmatrix} \quad E. \quad F. \quad G. \quad H.
$$
$$
\begin{array}{l}\mathrm{易求}A\mathrm{的特征值为}\frac13,\frac14,\frac15,故A∼∧=diag(\frac13,\frac14,\frac15),\\\mathrm{即存在可逆矩阵}P,\mathrm{使得}∧=P^{-1}AP,\mathrm{进而}∧^n=P^{-1}A^nP\\故\underset{n\rightarrow∞}{lim}A^n=\underset{n\rightarrow∞}{lim}P∧^nP^{-1}=\underset{n\rightarrow∞}{lim}Pdiag(\frac1{3^n},\frac1{4^n},\frac1{5^n})P^{-1}=\begin{pmatrix}0&0&0\\0&0&0\\0&0&0\end{pmatrix}\\\end{array}
$$



$$
\mathrm{已知}A=\begin{pmatrix}2&0&1\\3&1&x\\4&0&5\end{pmatrix}\mathrm{能对角化},则x=()
$$
$$
A.
0 \quad B.1 \quad C.2 \quad D.3 \quad E. \quad F. \quad G. \quad H.
$$
$$
\begin{array}{l}\mathrm{因为}A\mathrm{能对角化},A\mathrm{必有三个线性无关的特征向量},\mathrm{由于}\left|A-λ E\right|=\begin{vmatrix}2-λ&0&1\\3&1-λ&x\\4&0&5-λ\end{vmatrix}=-(λ-1)^2(λ-6)\\λ=1\mathrm{是二重特征值},\mathrm{必有两个线性无关的特征向量},\mathrm{因此}r(A-E)=1,得x=3.\end{array}
$$



$$
设A=\begin{pmatrix}1&-1&1\\2&4&-2\\-3&-3&a\end{pmatrix},且A\mathrm{的特征值为}λ_1=6,λ_2=λ_3=2,若A有3\mathrm{个线性无关的特征向量},则a=(\;)
$$
$$
A.
2 \quad B.3 \quad C.4 \quad D.5 \quad E. \quad F. \quad G. \quad H.
$$
$$
\begin{array}{l}\mathrm{由题设可知},\mathrm{矩阵可对角化},\mathrm{因此对应于特征值}2\mathrm{的线性无关的特征向量有}2个,则r(A-2E)=n-2=1\\A-2E=\begin{pmatrix}-1&-1&1\\2&2&-2\\-3&-3&a-2\end{pmatrix}\rightarrow\begin{pmatrix}-1&-1&1\\0&0&0\\0&0&a-5\end{pmatrix}故a-5=0⇒ a=5\end{array}
$$



$$
\mathrm{已知}A=\begin{pmatrix}0&0&1\\0&1&a\\1&0&0\end{pmatrix}\mathrm{有三个线性无关的特征向量},则a=(\;)
$$
$$
A.
0 \quad B.1 \quad C.-1 \quad D.3 \quad E. \quad F. \quad G. \quad H.
$$
$$
\begin{array}{l}由A\mathrm{的特征方程}\left|A-λ E\right|=\begin{vmatrix}-λ&0&1\\0&1-λ&a\\1&0&-λ\end{vmatrix}=-(λ-1)(λ^2-1)=0\mathrm{得到特征值}\;λ=1(\mathrm{二重}),λ=-1\\\mathrm{因为}A有3\mathrm{个线性无关的特征向量},故A\mathrm{可对角化},即λ=1\mathrm{必须有两个线性无关的特征向量}.\mathrm{那么},\mathrm{必有}r(A-E)=3-2=1.\\\mathrm{于是}A-E=\begin{pmatrix}1&0&-1\\0&0&-a\\-1&0&1\end{pmatrix}⇒\begin{pmatrix}1&0&-1\\0&0&-a\\0&0&0\end{pmatrix},得a=0.\end{array}
$$



$$
\mathrm{已知}A=\begin{pmatrix}-1&1&0\\-2&2&0\\x&-2&1\end{pmatrix}\mathrm{能对角化},则x=(\;)
$$
$$
A.
1 \quad B.2 \quad C.3 \quad D.4 \quad E. \quad F. \quad G. \quad H.
$$
$$
\begin{array}{l}\mathrm{因为}A\mathrm{能对角化},A\mathrm{必有三个线性无关的特征向量},\mathrm{由于}\left|A-λ E\right|=\begin{vmatrix}-λ-1&1&0\\-2&2-λ&0\\x&-2&1-λ\end{vmatrix}=-(λ-1)(λ^2-λ)\\λ=1\mathrm{是二重特征值},\mathrm{必有两个线性无关的特征向量},\mathrm{因此}r(A-E)=1,得x=4.\end{array}
$$



$$
\mathrm{已知}A=\begin{pmatrix}-1&-2&0\\1&2&0\\x&-2&1\end{pmatrix}\mathrm{能对角化},则x=(\;)
$$
$$
A.
-2 \quad B.-1 \quad C.1 \quad D.2 \quad E. \quad F. \quad G. \quad H.
$$
$$
\begin{array}{l}\mathrm{因为}A\mathrm{能对角化},A\mathrm{必有三个线性无关的特征向量},\mathrm{由于}\left|A-λ E\right|=\begin{vmatrix}-λ-1&-2&0\\1&2-λ&0\\x&-2&1-λ\end{vmatrix}=-(λ-1)(λ^2-λ)\\λ=1\mathrm{是二重特征值},\mathrm{必有两个线性无关的特征向量},\mathrm{因此}r(A-E)=1,得x=-2.\end{array}
$$



$$
设A=\begin{pmatrix}1&-1&1\\2&4&-2\\a&-3&5\end{pmatrix},且A\mathrm{的特征值为}λ_1=6,λ_2=λ_3=2,若A有3\mathrm{个线性无关的特征向量},则a=(\;)
$$
$$
A.
-1 \quad B.-2 \quad C.-3 \quad D.-4 \quad E. \quad F. \quad G. \quad H.
$$
$$
\begin{array}{l}\mathrm{由题设可知},\mathrm{矩阵可对角化},\mathrm{因此对应于特征值}2\mathrm{的线性无关的特征向量有}2个,则r(A-2E)=n-2=1\\A-2E=\begin{pmatrix}-1&-1&1\\2&2&-2\\a&-3&3\end{pmatrix}\rightarrow\begin{pmatrix}-1&-1&1\\0&0&0\\a+3&0&0\end{pmatrix}故a+3=0⇒ a=-3\end{array}
$$



$$
\mathrm{设三阶方阵}A\mathrm{的特征值为}1,2,-3,B=A^3-7A+5E,则B=()
$$
$$
A.
E \quad B.-E \quad C.0 \quad D.diag(1,2,-3) \quad E. \quad F. \quad G. \quad H.
$$
$$
\begin{array}{l}\mathrm{由题设可知},\mathrm{存在可逆矩阵}P,\mathrm{使得}P^{-1}AP=\begin{pmatrix}1&&\\&2&\\&&-3\end{pmatrix},则\\B=P\left[\begin{array}{c}\begin{pmatrix}1&&\\&2&\\&&-3\end{pmatrix}^3\end{array}-7\begin{pmatrix}1&&\\&2&\\&&-3\end{pmatrix}+5\begin{pmatrix}1&&\\&1&\\&&1\end{pmatrix}\right]P^{-1}=-E\end{array}
$$



$$
\mathrm{已知}A=\begin{pmatrix}0&0&1\\a&1&b\\1&0&0\end{pmatrix}\mathrm{有三个线性无关的特征向量},则a,b\mathrm{应满足}(\;)
$$
$$
A.
a+b=0 \quad B.a-b=0 \quad C.a+b=1 \quad D.a-b=1 \quad E. \quad F. \quad G. \quad H.
$$
$$
\begin{array}{l}由A\mathrm{的特征方程}\left|A-λ E\right|=\begin{vmatrix}-λ&0&1\\a&1-λ&b\\1&0&-λ\end{vmatrix}=-(λ-1)(λ^2-1)=0\mathrm{得到特征值}\;λ=1(\mathrm{二重}),λ=-1\\\mathrm{因为}A有3\mathrm{个线性无关的特征向量},故A\mathrm{可对角化},即λ=1\mathrm{必须有两个线性无关的特征向量}.\mathrm{那么},\mathrm{必有}r(A-E)=3-2=1.\\\mathrm{于是}A-E=\begin{pmatrix}-1&0&1\\a&0&b\\1&0&-1\end{pmatrix}⇒\begin{pmatrix}1&0&-1\\0&0&a+b\\0&0&0\end{pmatrix},得a+b=0.\end{array}
$$



$$
\mathrm{对应于}n\;\mathrm{阶矩阵}A\;\mathrm{的每个}\;k\mathrm{重特征值}\;λ,\;有m\;\mathrm{个线性无关的特征向量},\;\;则\;(\;\;)
$$
$$
A.
当\;k=m\;时,A\;\mathrm{与对角阵相似} \quad B.当\;k\;<\;m\;时,A\;\mathrm{与对角阵相似} \quad C.当\;k>m\;时,A\;\mathrm{与对角阵相似} \quad D.A\;\mathrm{是否与对角阵相似},与k,m\mathrm{没关系} \quad E. \quad F. \quad G. \quad H.
$$
$$
\mathrm{方阵}A\mathrm{能够相似对角化的一个充要条件就是}A的\;k\mathrm{重特征根对应}k\mathrm{个线性无关的特征向量}
$$



$$
\mathrm{若矩阵}A=\begin{pmatrix}4&2\\x&5\end{pmatrix}与B=\begin{pmatrix}6&2\\-1&3\end{pmatrix}\mathrm{相似},则x=()
$$
$$
A.
0 \quad B.1 \quad C.2 \quad D.3 \quad E. \quad F. \quad G. \quad H.
$$
$$
\mathrm{相似矩阵具有相同的行列式},\mathrm{所以}\begin{vmatrix}4&2\\x&5\end{vmatrix}=\begin{vmatrix}6&2\\-1&3\end{vmatrix},\mathrm{解之得}x=0
$$



$$
\mathrm{若矩阵}A=\begin{pmatrix}1&0\\0&4\end{pmatrix}与B=\begin{pmatrix}3&b\\a&x\end{pmatrix}\mathrm{相似},则x=()
$$
$$
A.
0 \quad B.1 \quad C.2 \quad D.3 \quad E. \quad F. \quad G. \quad H.
$$
$$
\mathrm{相似矩阵具有相同的迹},\mathrm{所以}1+4=3+x,\mathrm{解之得}x=2
$$



$$
\mathrm{已知矩阵}A=\begin{pmatrix}1&2&0\\2&1&0\\-2&a&3\end{pmatrix}\mathrm{与对角阵相似},则a=()
$$
$$
A.
0 \quad B.1 \quad C.2 \quad D.3 \quad E. \quad F. \quad G. \quad H.
$$
$$
由\left|A-λ E\right|=0\mathrm{可得}λ=-2,λ=3(\mathrm{二重}),\mathrm{又由于}A\mathrm{与对角阵相似},则R(A-3E)=1,\mathrm{可得}a=2
$$



$$
\mathrm{已知}3\mathrm{阶矩阵}A和3\mathrm{维向量}x,\mathrm{使得}x,Ax,A^2x\mathrm{线性无关},\mathrm{且满足}A^3x=3Ax-2A^2x,则\left|A+E\right|=()
$$
$$
A.
-4 \quad B.4 \quad C.-2 \quad D.2 \quad E. \quad F. \quad G. \quad H.
$$
$$
\begin{array}{l}A(x\;Ax\;A^2x)=(x\;Ax\;A^2x)\begin{pmatrix}0&0&0\\1&0&3\\0&1&-2\end{pmatrix}=(x\;Ax\;A^2x)B,则A∼ B,故AE∼ BE,\\\left|A+E\right|=\left|B+E\right|=\left|\begin{pmatrix}1&0&0\\1&1&3\\0&1&-1\end{pmatrix}\right|=-4\end{array}
$$



$$
\mathrm{已知矩阵}A=\begin{pmatrix}2&2&0\\8&2&a\\0&0&6\end{pmatrix}\mathrm{与对角阵相似},则a=()
$$
$$
A.
0 \quad B.1 \quad C.2 \quad D.3 \quad E. \quad F. \quad G. \quad H.
$$
$$
由\left|A-λ E\right|=0\mathrm{可得}λ=-2,λ=6(\mathrm{二重}),\mathrm{又由于}A\mathrm{与对角阵相似},则R(A-6E)=1,\mathrm{可得}a=0
$$



$$
\mathrm{已知矩阵}A\mathrm{与矩阵}B\mathrm{相似},A=\begin{pmatrix}-2&0&0\\2&x&2\\3&1&1\end{pmatrix},B=\begin{pmatrix}-1&0&0\\0&2&0\\0&0&y\end{pmatrix},\mathrm 则x,y=()
$$
$$
A.
x=0,y=-2 \quad B.x=1,y=-2 \quad C.x=0,y=2 \quad D.x=1,y=2 \quad E. \quad F. \quad G. \quad H.
$$
$$
\begin{array}{l}\mathrm{因为}A∼ B,\mathrm{所以}A\mathrm{的特征值为}-1,2,y,\mathrm{所以}\left|A+E\right|=0,\mathrm{解之得}x=0,\\\mathrm{又因为相似矩阵具有相同的迹},\mathrm{所以}-2+0+1=-1+2+y,\mathrm{可得}y=-2\end{array}
$$



$$
\begin{array}{l}n\mathrm{阶矩阵}A\mathrm{与对角矩阵相似的充分必要条件是}()\\\end{array}
$$
$$
A.
A有n\mathrm{个特征值} \quad B.A有n\mathrm{个线性无关的特征向量} \quad C.\left|A\right|=0 \quad D.A\mathrm{的特征多项式没有重根} \quad E. \quad F. \quad G. \quad H.
$$
$$
\mathrm{考察方阵与对角矩阵相似的充要条件}
$$



$$
与n\mathrm{阶单位矩阵}E\mathrm{相似的矩阵是}()
$$
$$
A.
\mathrm{数量矩阵}kE(k\neq1) \quad B.\mathrm{对角矩阵}A(\mathrm{主对角线上元素不为}1) \quad C.E \quad D.\mathrm{任意}n\mathrm{阶可逆矩阵} \quad E. \quad F. \quad G. \quad H.
$$
$$
\mathrm{由相似矩阵的定义},\mathrm{若存在可逆矩阵}P,\mathrm{使得}P^{-1}EP=A,\mathrm{则称矩阵}A与E\mathrm{相似},\mathrm{显然}A=E.
$$



$$
\mathrm{设两个}n\mathrm{阶方阵}A与B\mathrm{相似},则()
$$
$$
A.
A与B\mathrm{合同} \quad B.A与B\mathrm{不合同} \quad C.A与B\mathrm{等价} \quad D.A与B\mathrm{不等价} \quad E. \quad F. \quad G. \quad H.
$$
$$
\begin{array}{l}n阶A与B\mathrm{相似},\mathrm{则存在}n\mathrm{阶可逆矩阵}P,\mathrm{使得}P^{-1}AP=B,\mathrm{由于可逆矩阵可表示为一系列初等矩阵的乘积},即A\mathrm{可经过一系列初等}\\\mathrm{变换得到}B,故A与B\mathrm{等价},\mathrm{故相似矩阵等价};\\\mathrm{若存在}n\mathrm{阶可逆矩阵}C,\mathrm{使得}C^TAC=B,\mathrm{则称矩阵}A\mathrm{合同于矩阵}B,\mathrm{由于}C^T与C^{-1}\mathrm{不一定相等},\mathrm{故相似矩阵不一定合同}.\end{array}
$$



$$
\mathrm{设两个}n\mathrm{阶方阵}A与B\mathrm{有相同的特征多项式},则()
$$
$$
A.
A与B\mathrm{相似} \quad B.A与B\mathrm{合同} \quad C.A与B\mathrm{等价} \quad D.\mathrm{以上三条均不成立} \quad E. \quad F. \quad G. \quad H.
$$
$$
\mathrm{两个}n\mathrm{阶方阵}A与B\mathrm{有相同的特征多项式},\mathrm{只能说明}A与B\mathrm{有相同的特征值},\mathrm{并不能说明}A与B\mathrm{相似}、\mathrm{等价或合同}.
$$



$$
\mathrm{设矩阵}A\mathrm{与矩阵}B\mathrm{相似},\mathrm{则必有}()
$$
$$
A.
A,B\mathrm{有相同的特征向量} \quad B.A,B\mathrm{有相同的行列式} \quad C.A,B\mathrm{相似于同一对角矩阵} \quad D.\mathrm{矩阵}A-λ E与B-λ E\mathrm{相等} \quad E. \quad F. \quad G. \quad H.
$$
$$
\begin{array}{l}\mathrm{矩阵}A\mathrm{与矩阵}B\mathrm{相似},则B=P^{-1}AP⇒\left|B\right|=\left|P^{-1}\right|\left|A\right|\left|P\right|=\left|A\right|,且\left|A-λ E\right|=\left|B-\lambda E\right|\mathrm{故有相同的特征值},\\\mathrm{但矩阵}A-λ E与B-λ E\mathrm{不一定相等},\mathrm{且不一定有相同的特征向量};\mathrm{矩阵}A与B\mathrm{矩阵不一定能对角化},\mathrm{故不相似于对角矩阵}.\end{array}
$$



$$
若n\mathrm{阶方阵}A与B\mathrm{相似},\;\mathrm{则下列说法不正确的是}()
$$
$$
A.
r(A)=r(B) \quad B.\left|A\right|=\left|B\right| \quad C.(λ E-A)^k与(λ E-B)^k\mathrm{相似},\mathrm{其中}K∈ N \quad D.A^{-1}=B^{-1} \quad E. \quad F. \quad G. \quad H.
$$
$$
\begin{array}{l}(1)若n\mathrm{阶方阵}A与B\mathrm{相似},\mathrm{则有可逆方阵}U,使U^{-1}AU=B,\mathrm{可逆方阵}U和U^{-1}\mathrm{都可以表示为有限个初等矩阵的乘积},\\即A\mathrm{经过有限次初等变换成为}B,\mathrm{所以}r(A)=r(B);\\(2)\mathrm{因为}U^{-1}AU=B,\mathrm{所以}\left|U^{-1}AU\right|=\left|B\right|,\mathrm{则有}\left|A\right|=\left|B\right|;\\(3)由U^{-1}AU=B得(λ E-B)^k=(λ E-U^{-1}AU)^k=(U^{-1}λ EU-U^{-1}AU)^k=(U^{-1}(λ E-A)U)^k\\=U^{-1}(λ E-A)UU^{-1}(λ E-A)U⋯ U^{-1}(λ E-A)U=U^{-1}(λ E-A)^kU,即(λ E-B)^k与(λ E-A)^k\mathrm{相似}.\\(4)\mathrm{逆矩阵只是相似},\mathrm{不相等},即A^{-1}\mathrm{相似于}B^{-1}\end{array}
$$



$$
设n\mathrm{阶方阵}A与B\mathrm{相似},\mathrm{则下列必成立的是}()
$$
$$
A.
λ E-A=λ E-B,\mathrm{其中}λ 为A与B\mathrm{的特征值} \quad B.\mathrm{对于任意常数}t,\left|A-tE\right|=\left|B-tE\right| \quad C.\mathrm{存在对角形矩阵}∧,\mathrm{使得}A与B\mathrm{都相似于}∧ \quad D.当λ_0是A与B\mathrm{的特征值时},n\mathrm{元齐次线性方程组}(A-λ_0E)X=0与(B-λ_0E)X=0\mathrm{同解} \quad E. \quad F. \quad G. \quad H.
$$
$$
\begin{array}{l}\mathrm{相似矩阵有相同的特征多项式},即\left|A-λ E\right|=\left|B-λ E\right|,\mathrm{而不是}A-λ E=B-λ E;\\\mathrm{由于}A与B\mathrm{相似},\mathrm{则存在可逆矩阵}P,\mathrm{使得}P^{-1}AP=B,则\\\left|B-tE\right|=\left|P^{-1}AP-P^{-1}tEP\right|=\left|P^{-1}(A-tE)P\right|=\left|P^{-1}\right|\left|A-tE\right|\left|P\right|=\left|A-tE\right|\\A与B\mathrm{不一定与对角矩阵相似},\mathrm{且相似矩阵有相同的特征值},\mathrm{但不一定有相同的特征向量}.\end{array}
$$



$$
设A为n× m\mathrm{矩阵},r(A)=n,则()
$$
$$
A.
AA^T\mathrm{为可逆矩阵} \quad B.A^TA\mathrm{为可逆矩阵} \quad C.AA^T\mathrm{必与}E\mathrm{相似} \quad D.A^TA\mathrm{必与}E\mathrm{相似} \quad E. \quad F. \quad G. \quad H.
$$
$$
\begin{array}{l}\mathrm{由条件可知},n\leq m,\mathrm{可将}A\mathrm{分块为}(A_1,A_2),\mathrm{其中}A_1为n\mathrm{阶矩阵},A_2为n×(m-n)\mathrm{阶矩阵},则A_1\mathrm{可逆}.AA^T=(A_1A^T,A_2A^T)\\\mathrm{因此}r(A^T)=r(A_1A^T)\leq r(AA^T)\leq r(A^T),即r(AA^T)=r(A^T),\mathrm{同理可证}r(A^TA)=r(A^T),\mathrm{结论对于时}n=m\mathrm{显然成立}.\\AA^T为n× n\mathrm{的矩阵},且r(AA^T)=r(A^T)=r(A)=n,故AA^T\mathrm{为可逆矩阵};\\A^TA为m× m\mathrm{的矩阵},且r(A^TA)=r(A^T)=r(A)=n\leq m,故A^TA\mathrm{不一定可逆};\\\mathrm{单位矩阵}E\mathrm{只与其自身相似}.\end{array}
$$



$$
设A∼\begin{pmatrix}1&0&0\\0&1&0\\0&0&-2\end{pmatrix},则R(A-E)+R(2E+A)=()
$$
$$
A.
0 \quad B.1 \quad C.2 \quad D.3 \quad E. \quad F. \quad G. \quad H.
$$
$$
\begin{array}{l}\mathrm{由题意可知},\mathrm{存在可逆矩阵}P\mathrm{使得}\\P^{-1}AP=\begin{pmatrix}1&0&0\\0&1&0\\0&0&-2\end{pmatrix}\\则A-E=P\begin{pmatrix}0&0&0\\0&0&0\\0&0&-3\end{pmatrix}P^{-1},2E+A=P\begin{pmatrix}3&0&0\\0&3&0\\0&0&0\end{pmatrix}P^{-1}\\\mathrm{由于相似矩阵具有相同的秩},\mathrm{所以}R(A-E)+R(2E+A)=3\end{array}
$$



$$
\mathrm{已知矩阵}A=\begin{pmatrix}1&2&2\\2&1&2\\2&2&1\end{pmatrix},B=\begin{pmatrix}-1&&\\&5&\\&&a\end{pmatrix},且A∼ B,则a=(\;)
$$
$$
A.
-1 \quad B.0 \quad C.1 \quad D.2 \quad E. \quad F. \quad G. \quad H.
$$
$$
\mathrm{因为}A∼ B,\mathrm{所以}tr(A)=tr(B),即1+1+1=-1+5+a,则a=-1
$$



$$
\mathrm{已知矩阵}A=\begin{pmatrix}1&1&0\\1&0&1\\0&1&1\end{pmatrix},B=\begin{pmatrix}-1&&\\&1&\\&&a\end{pmatrix},且A∼ B,则a=(\;)
$$
$$
A.
-1 \quad B.0 \quad C.1 \quad D.2 \quad E. \quad F. \quad G. \quad H.
$$
$$
\mathrm{因为}A∼ B,\mathrm{所以}tr(A)=tr(B),即1+0+1=-1+1+a,则a=2
$$



$$
\mathrm{已知矩阵}A=\begin{pmatrix}3&1&b\\2&1&2\\1&2&1\end{pmatrix},B=\begin{pmatrix}1&&\\&2&\\&&a\end{pmatrix},且A∼ B,则a=(\;)
$$
$$
A.
-1 \quad B.0 \quad C.1 \quad D.2 \quad E. \quad F. \quad G. \quad H.
$$
$$
\mathrm{因为}A∼ B,\mathrm{所以}tr(A)=tr(B),即3+1+1=1+2+a,则a=2
$$



$$
\mathrm{已知矩阵}A=\begin{pmatrix}3&1&b\\2&1&2\\b&2&-1\end{pmatrix},B=\begin{pmatrix}0&&\\&2&\\&&a\end{pmatrix},且A∼ B,则a=(\;)
$$
$$
A.
-1 \quad B.0 \quad C.1 \quad D.2 \quad E. \quad F. \quad G. \quad H.
$$
$$
\mathrm{因为}A∼ B,\mathrm{所以}tr(A)=tr(B),即3+1-1=0+2+a,则a=1
$$



$$
\mathrm{三阶方阵}A\mathrm{的特征值是}5,3,2;\mathrm{则矩阵}A^2+E\;\mathrm{的特征值是}(\;)
$$
$$
A.
5,3,2 \quad B.6,4,3 \quad C.25,9,4 \quad D.26,10,5 \quad E. \quad F. \quad G. \quad H.
$$
$$
若λ 是A\mathrm{的特征值},则f(λ)是f(A)\mathrm{的特征值}
$$



$$
\mathrm{三阶方阵}A\mathrm{的特征值是}5,3,2;\mathrm{则矩阵}A^2-E\;\mathrm{的特征值是}(\;)
$$
$$
A.
5,3,2 \quad B.4,2,1 \quad C.24,8,3 \quad D.26,10,5 \quad E. \quad F. \quad G. \quad H.
$$
$$
若λ 是A\mathrm{的特征值},则f(λ)是f(A)\mathrm{的特征值}
$$



$$
\mathrm{三阶方阵}A\mathrm{的特征值是}a(a>0),3,2,\mathrm{矩阵}A^2-E\;\mathrm{的特征值为}24,8,3,则a=(\;)
$$
$$
A.
1 \quad B.3 \quad C.5 \quad D.7 \quad E. \quad F. \quad G. \quad H.
$$
$$
若λ 是A\mathrm{的特征值},则f(λ)是f(A)\mathrm{的特征值}
$$



$$
设n\mathrm{阶方阵}A与B\mathrm{相似},E为n\mathrm{阶单位矩阵},\mathrm{则下列必成立的是}()
$$
$$
A.
λ E-A=λ E-B,\mathrm{其中}λ 为A与B\mathrm{的特征值} \quad B.\;A与\;B\mathrm{有相同的特征值和特征向量} \quad C.\;A与B\;\mathrm{都相似于一个对角矩阵} \quad D.\mathrm{对任意常数}t\;,\;A-tE与\;B-tE\mathrm{相似} \quad E. \quad F. \quad G. \quad H.
$$
$$
\begin{array}{l}\mathrm{相似矩阵有相同的特征多项式},即\left|A-λ E\right|=\left|B-λ E\right|,\mathrm{而不是}A-λ E=B-λ E;\\\mathrm{由于}A与B\mathrm{相似},\mathrm{则存在可逆矩阵}P,\mathrm{使得}P^{-1}AP=B,则\\\\A与B\mathrm{不一定与对角矩阵相似},\mathrm{且相似矩阵有相同的特征值},\mathrm{但不一定有相同的特征向量}.\\B-tE=P^{-1}AP-P^{-1}tEP=P^{-1}(A-tE)P,\mathrm{所以}A-tE与B-tE\mathrm{相似}\end{array}
$$



$$
\mathrm{若四阶矩阵}\;A与\;B\mathrm{相似},A\;\mathrm{的特征值为}\frac12,\frac13,\frac14,\frac15.\mathrm{则行列式}\left|B^{-1}-E\right|\;\mathrm{的值为}\;(\;\;\;)
$$
$$
A.
120 \quad B.\frac1{120} \quad C.24 \quad D.\frac1{24} \quad E. \quad F. \quad G. \quad H.
$$
$$
A∼ B,\mathrm{所以}A,B\mathrm{具有相同的特征值},\mathrm{那么}B^{-1}-E\mathrm{的特征值为}1,2,3,4,\mathrm{所以}\left|B^{-1}-E\right|=1×2×3×4=24
$$



$$
\mathrm{已知矩阵}A=\begin{pmatrix}-1&0&0\\2&x&2\\3&1&1\end{pmatrix},B=\begin{pmatrix}-1&0&0\\0&2&0\\0&0&y\end{pmatrix},且A∼ B,则x,y\mathrm{分别等于}()
$$
$$
A.
x=0,y=-1 \quad B.x=0,y=1 \quad C.x=-1,y=0 \quad D.x=-1,y=1 \quad E. \quad F. \quad G. \quad H.
$$
$$
\mathrm{因为}A∼ B,\mathrm{所以}\left\{\begin{array}{l}tr(A)=tr(B)\\\left|A\right|=\left|B\right|\end{array}\right.,即\left\{\begin{array}{l}x-y=1\\x-2y=2\end{array}\right.,则\left\{\begin{array}{l}x=0\\y=-1\end{array}\right.
$$



$$
\mathrm{已知矩阵}A=\begin{pmatrix}1&-1&1\\2&4&a\\-3&-3&5\end{pmatrix}\;\mathrm{的特征值为}6,\;2,\;λ,则a,\;λ\mathrm{的值分别是}()
$$
$$
A.
-1,1 \quad B.-2,2 \quad C.-1,2 \quad D.-2,1 \quad E. \quad F. \quad G. \quad H.
$$
$$
由\left\{\begin{array}{l}tr(A)=6 2 λ\\\left|A\right|=6×2×λ\end{array}\right.得a=-2,λ=2
$$



$$
\mathrm{已知矩阵}A=\begin{pmatrix}a&0&b\\0&2&0\\b&0&-2\end{pmatrix},B=\begin{pmatrix}2&&\\&2&\\&&-3\end{pmatrix},且A∼ B,则a,b=(\;)
$$
$$
A.
1,2 \quad B.0,1 \quad C.2,3 \quad D.-1,1 \quad E. \quad F. \quad G. \quad H.
$$
$$
\mathrm{因为}A∼ B,\mathrm{所以}tr(A)=tr(B)且\left|A\right|=\left|B\right|,\mathrm{计算可得}a=1,b=2
$$



$$
\mathrm{已知矩阵}A=\begin{pmatrix}1&2&b\\2&1&2\\2&2&1\end{pmatrix},B=\begin{pmatrix}-1&&\\&5&\\&&a\end{pmatrix},且A∼ B,则a,b=(\;)
$$
$$
A.
-1,2 \quad B.-1,0 \quad C.1,2 \quad D.1,0 \quad E. \quad F. \quad G. \quad H.
$$
$$
\mathrm{因为}A∼ B,\mathrm{所以}tr(A)=tr(B),\left|A\right|=\left|B\right|,\mathrm{计算可得}a=-1,b=2
$$



$$
y\mathrm{已知}A=\begin{pmatrix}1&b&1\\b&a&1\\1&1&1\end{pmatrix}与\begin{pmatrix}0&&\\&1&\\&&4\end{pmatrix}则a,b\mathrm{的值分别为}()
$$
$$
A.
a=3,b=1 \quad B.a=0,b=1 \quad C.a=1,b=4 \quad D.a=3,b=0 \quad E. \quad F. \quad G. \quad H.
$$
$$
\begin{array}{l}由A=\begin{pmatrix}1&b&1\\b&a&1\\1&1&1\end{pmatrix}与\begin{pmatrix}0&&\\&1&\\&&4\end{pmatrix}\mathrm{相似得},A\mathrm{的特征值是}0,1,4,\mathrm{那么}\\\left\{\begin{array}{l}1+a+1=0+1+4\\\left|A\right|=\begin{vmatrix}1&b&1\\b&a&1\\1&1&1\end{vmatrix}=\begin{vmatrix}0&b&1\\b-1&a&1\\0&1&1\end{vmatrix}=-(b-1)^2=0\end{array}\right.,\mathrm{所以}a=3,b=1.\end{array}
$$



$$
\mathrm{已知矩阵}A=\begin{pmatrix}-2&0&0\\2&x&2\\3&1&1\end{pmatrix},B=\begin{pmatrix}-1&0&0\\0&2&0\\0&0&y\end{pmatrix},且A∼ B,则x,y\mathrm{分别等于}()
$$
$$
A.
x=0,y=-2 \quad B.x=0,y=2 \quad C.x=-2,y=0 \quad D.x=-2,y=1 \quad E. \quad F. \quad G. \quad H.
$$
$$
\mathrm{因为}A∼ B,\mathrm{所以}\left\{\begin{array}{l}tr(A)=tr(B)\\\left|A+E\right|=0\end{array}\right.,即\left\{\begin{array}{l}x-y=2\\x=0\end{array}\right.,则\left\{\begin{array}{l}x=0\\y=-2\end{array}\right.
$$



$$
\mathrm{矩阵}(\;)\mathrm{是二次型}x_1^2+6x_1x_2+3x_2^2\mathrm{的矩阵}.
$$
$$
A.
\begin{pmatrix}1&-1\\-1&3\end{pmatrix} \quad B.\begin{pmatrix}1&2\\4&3\end{pmatrix} \quad C.\begin{pmatrix}1&3\\3&3\end{pmatrix} \quad D.\begin{pmatrix}1&5\\1&3\end{pmatrix} \quad E. \quad F. \quad G. \quad H.
$$
$$
\mathrm{由于二次型矩阵是对称矩阵},\mathrm{所以}a_{ij}=a_{ji},\mathrm{即二次型可表示成矩阵形式如下}:(x_1,x_2)\begin{pmatrix}1&3\\3&3\end{pmatrix}\begin{pmatrix}x_1\\x_2\end{pmatrix}
$$



$$
\mathrm{二次型}f=x^2+4xy+4y^2+2xz+z^2+4yz\mathrm{的矩阵表示为}()
$$
$$
A.
(x,y,z)\begin{pmatrix}1&2&1\\2&4&2\\1&2&1\end{pmatrix}\begin{pmatrix}x\\y\\z\end{pmatrix} \quad B.(x,y,z)\begin{pmatrix}1&4&1\\4&1&2\\1&2&1\end{pmatrix}\begin{pmatrix}x\\y\\z\end{pmatrix} \quad C.(x,y,z)\begin{pmatrix}1&4&2\\4&2&1\\2&1&2\end{pmatrix}\begin{pmatrix}x\\y\\z\end{pmatrix} \quad D.(x,y,z)\begin{pmatrix}1&2&1\\2&2&4\\1&4&1\end{pmatrix}\begin{pmatrix}x\\y\\z\end{pmatrix} \quad E. \quad F. \quad G. \quad H.
$$
$$
f=(x,y,z)\begin{pmatrix}1&2&1\\2&4&2\\1&2&1\end{pmatrix}\begin{pmatrix}x\\y\\z\end{pmatrix}
$$



$$
\mathrm{矩阵}\begin{pmatrix}0&\frac{\sqrt2}2&1\\\frac{\sqrt2}2&3&-\frac32\\1&-\frac32&0\end{pmatrix}\mathrm{对应的实二次型为}()
$$
$$
A.
x_1^2+\frac12x_1x_2+2\;x_1x_3-3x_2x_3 \quad B.2\sqrt2x_1x_2-x_2^2+\;x_1x_3-\frac32x_2x_3 \quad C.\sqrt2x_1x_2+3x_2^2+2\;x_1x_3-3x_2x_3 \quad D.x_1x_2-3x_2^2+\;x_1x_3-3x_2x_3 \quad E. \quad F. \quad G. \quad H.
$$
$$
\mathrm{矩阵}a_{ij}=a_{ji},且f=x^TAx=a_{11}x_1^2+a_{22}x_2^2+a_{33}x_3^2+2a_{12}x_1x_2+2a_{13}x_1x_3+2a_{23}x_2x_3=\sqrt2x_1x_2+3x_2^2+2\;x_1x_3-3x_2x_3
$$



$$
\mathrm{矩阵}\begin{pmatrix}0&\frac12&0&0\\\frac12&1&0&-3\\0&0&0&0\\0&-3&0&-3\end{pmatrix}\mathrm{对应的实二次型}f(x_1,x_2,x_3,x_4)=(\;)
$$
$$
A.
x_1x_2+x_2^2-6x_2x_4-3x_4^2 \quad B.\frac12x_1x_2+x_2^2-3x_2x_4-x_4^2 \quad C.x_1x_2+2x_2^2-3x_2x_4-x_4^2 \quad D.x_1x_2-x_2^2+6x_2x_4-3x_4^2 \quad E. \quad F. \quad G. \quad H.
$$
$$
f=x^TAx=x_1x_2+x_2^2-6x_2x_4-3x_4^2
$$



$$
\mathrm{二次型}f=x^2+4xy+4y^2+2xz+z^2+2yz\mathrm{的矩阵表示为}()
$$
$$
A.
(x,y,z)\begin{pmatrix}1&2&1\\2&4&1\\1&1&1\end{pmatrix}\begin{pmatrix}x\\y\\z\end{pmatrix} \quad B.(x,y,z)\begin{pmatrix}1&4&1\\4&1&2\\1&2&1\end{pmatrix}\begin{pmatrix}x\\y\\z\end{pmatrix} \quad C.(x,y,z)\begin{pmatrix}1&4&2\\4&2&1\\2&1&2\end{pmatrix}\begin{pmatrix}x\\y\\z\end{pmatrix} \quad D.(x,y,z)\begin{pmatrix}1&2&1\\2&2&4\\1&4&1\end{pmatrix}\begin{pmatrix}x\\y\\z\end{pmatrix} \quad E. \quad F. \quad G. \quad H.
$$
$$
f=(x,y,z)\begin{pmatrix}1&2&1\\2&4&1\\1&1&1\end{pmatrix}\begin{pmatrix}x\\y\\z\end{pmatrix}
$$



$$
\mathrm{二次型}f(x_1,x_2,x_3)=2x_1^2-x_1x_2+x_2^2\mathrm{的矩阵为}()
$$
$$
A.
\begin{pmatrix}2&-1\\-1&1\end{pmatrix} \quad B.\begin{pmatrix}2&-\frac12\\-\frac12&1\end{pmatrix} \quad C.\begin{pmatrix}2&-1&0\\-1&1&0\\0&0&0\end{pmatrix} \quad D.\begin{pmatrix}2&-\frac12&0\\-\frac12&1&0\\0&0&0\end{pmatrix} \quad E. \quad F. \quad G. \quad H.
$$
$$
\mathrm{根据二次型及其矩阵的对应关系可知},\mathrm{三元二次型}f(x_1,x_2,x_3)=2x_1^2-x_1x_2+x_2^2\mathrm{的矩阵为}\begin{pmatrix}2&-\frac12&0\\-\frac12&1&0\\0&0&0\end{pmatrix}
$$



$$
\mathrm{矩阵}\begin{pmatrix}1&\frac{\sqrt2}2&1\\\frac{\sqrt2}2&3&-\frac32\\1&-\frac32&0\end{pmatrix}\mathrm{对应的实二次型为}()
$$
$$
A.
x_1^2+\frac12x_1x_2+2\;x_1x_3-3x_2x_3 \quad B.2\sqrt2x_1x_2-x_2^2+\;x_1x_3-\frac32x_2x_3 \quad C.x_1^2+\sqrt2x_1x_2+3x_2^2+2\;x_1x_3-3x_2x_3 \quad D.x_1x_2-3x_2^2+\;x_1x_3-3x_2x_3 \quad E. \quad F. \quad G. \quad H.
$$
$$
\mathrm{矩阵}a_{ij}=a_{ji},且f=x^TAx=a_{11}x_1^2+a_{22}x_2^2+a_{33}x_3^2+2a_{12}x_1x_2+2a_{13}x_1x_3+2a_{23}x_2x_3=x_1^2+\sqrt2x_1x_2+3x_2^2+2\;x_1x_3-3x_2x_3
$$



$$
\mathrm{矩阵}\begin{pmatrix}0&\frac12&0&0\\\frac12&-1&0&3\\0&0&0&0\\0&3&0&-3\end{pmatrix}\mathrm{对应的实二次型}f(x_1,x_2,x_3,x_4)=(\;)
$$
$$
A.
x_1x_2-x_2^2+6x_2x_4-3x_4^2 \quad B.\frac12x_1x_2+x_2^2-3x_2x_4-x_4^2 \quad C.x_1x_2+2x_2^2-3x_2x_4-x_4^2 \quad D.x_1x_2+x_2^2-3x_2x_4-x_4^2 \quad E. \quad F. \quad G. \quad H.
$$
$$
f=x^TAx=x_1x_2-x_2^2+6x_2x_4-3x_4^2
$$



$$
\mathrm{矩阵}(\;)\mathrm{是二次型}x_1^2-2x_1x_2+3x_2^2\mathrm{的矩阵}.
$$
$$
A.
\begin{pmatrix}1&-1\\-1&3\end{pmatrix} \quad B.\begin{pmatrix}1&2\\4&3\end{pmatrix} \quad C.\begin{pmatrix}1&3\\3&3\end{pmatrix} \quad D.\begin{pmatrix}1&5\\1&3\end{pmatrix} \quad E. \quad F. \quad G. \quad H.
$$
$$
\mathrm{由于二次型矩阵是对称矩阵},\mathrm{所以}a_{ij}=a_{ji},\mathrm{即二次型可表示成矩阵形式如下}:(x_1,x_2)\begin{pmatrix}1&-1\\-1&3\end{pmatrix}\begin{pmatrix}x_1\\x_2\end{pmatrix}
$$



$$
\mathrm{二次型}f(x_1,x_2,x_3)=5x_1^2+x_2^2+2x_3^2+4x_1x_2-2x_1x_3-2x_2x_3\mathrm{的矩阵为}()
$$
$$
A.
\begin{pmatrix}5&2&-2\\2&1&-2\\-2&-2&2\end{pmatrix} \quad B.\begin{pmatrix}5&2&-1\\2&1&-1\\-1&-1&2\end{pmatrix} \quad C.\begin{pmatrix}5&4&-2\\4&1&-2\\-2&-2&2\end{pmatrix} \quad D.\begin{pmatrix}5&-2&1\\-2&1&1\\1&1&2\end{pmatrix} \quad E. \quad F. \quad G. \quad H.
$$
$$
5x_1^2+x_2^2+2x_3^2+4x_1x_2-2x_1x_3-2x_2x_3=(x_1,x_2,x_3)\begin{pmatrix}5&2&-1\\2&1&-1\\-1&-1&2\end{pmatrix}\begin{pmatrix}x_1\\x_2\\x_3\end{pmatrix}
$$



$$
\mathrm{实二次型}f(x,y,z)=xy+yz\mathrm{的矩阵表达式是}(x,y,z)A\begin{pmatrix}x\\y\\z\end{pmatrix},\mathrm{其中}A=(\;)
$$
$$
A.
\begin{pmatrix}0&0&0\\1&0&0\\0&1&0\end{pmatrix} \quad B.\begin{pmatrix}0&1&0\\0&0&1\\0&0&0\end{pmatrix} \quad C.\begin{pmatrix}0&1&0\\1&0&1\\0&1&0\end{pmatrix} \quad D.\begin{pmatrix}0&\frac12&0\\\frac12&0&\frac12\\0&\frac12&0\end{pmatrix} \quad E. \quad F. \quad G. \quad H.
$$
$$
xy+yz=(x,y,z)\begin{pmatrix}0&\frac12&0\\\frac12&0&\frac12\\0&\frac12&0\end{pmatrix}\begin{pmatrix}x\\y\\z\end{pmatrix}
$$



$$
\mathrm{矩阵}A=\begin{pmatrix}1&-1&-3&1\\-1&0&-2&\frac12\\-3&-2&\frac13&-\frac32\\1&\frac12&-\frac32&0\end{pmatrix}\mathrm{所对应的二次型为}()
$$
$$
A.
x_1^2-2x_1x_2-6x_1x_3+2x_1x_4-4x_2x_3+x_2x_4+\frac13x_3^2-3x_3x_4 \quad B.x_1^2-x_1x_2-3x_1x_3+x_1x_4-2x_2x_3+\frac12x_2x_4+\frac13x_3^2-\frac32x_3x_4 \quad C.x_1^2-2x_1x_2-3x_1x_3+2x_1x_4-4x_2x_3+\frac12x_2x_4+\frac13x_3^2-x_3x_4 \quad D.x_1^2-x_1x_2-6x_1x_3+x_1x_4-2x_2x_3+x_2x_4+\frac13x_3^2-3x_3x_4 \quad E. \quad F. \quad G. \quad H.
$$
$$
\mathrm{矩阵}A\mathrm{的二次型为}(x_1,x_2,x_3,x_4)\begin{pmatrix}1&-1&-3&1\\-1&0&-2&\frac12\\-3&-2&\frac13&-\frac32\\1&\frac12&-\frac32&0\end{pmatrix}\begin{pmatrix}x_1\\x_2\\x_3\\x_4\end{pmatrix}=x_1^2-2x_1x_2-6x_1x_3+2x_1x_4-4x_2x_3+x_2x_4+\frac13x_3^2-3x_3x_4
$$



$$
\mathrm{二次型}f(x)=x^T\begin{pmatrix}1&2&3\\4&5&6\\7&8&9\end{pmatrix}x\mathrm{的对称矩阵为}()
$$
$$
A.
\begin{pmatrix}1&3&5\\3&5&7\\5&7&9\end{pmatrix} \quad B.\begin{pmatrix}1&2&5\\2&5&6\\5&6&9\end{pmatrix} \quad C.\begin{pmatrix}1&4&5\\4&5&7\\5&7&8\end{pmatrix} \quad D.\begin{pmatrix}1&3&6\\3&6&7\\6&7&8\end{pmatrix} \quad E. \quad F. \quad G. \quad H.
$$
$$
\begin{array}{l}f(x)=(x_1,x_2,x_3)\begin{pmatrix}1&2&3\\4&5&6\\7&8&9\end{pmatrix}\begin{pmatrix}x_1\\x_2\\x_3\end{pmatrix}=x_1^2+5x_2^2+9x_3^2+6x_1x_2+10x_1x_3+14x_2x_3=(x_1,x_2,x_3)\begin{pmatrix}1&3&5\\3&5&7\\5&7&9\end{pmatrix}\begin{pmatrix}x_1\\x_2\\x_3\end{pmatrix}\\\mathrm{于是}f\mathrm{的矩阵为}\begin{pmatrix}1&3&5\\3&5&7\\5&7&9\end{pmatrix}\end{array}
$$



$$
\mathrm{二次型}f(x_1,x_2,x_3)=(ax_1+bx_2+cx_3)^2\mathrm{的矩阵为}()
$$
$$
A.
\begin{pmatrix}a^2&ab&ac\\ab&b^2&bc\\ac&bc&c^2\end{pmatrix} \quad B.\begin{pmatrix}a^2&a&a\\b&b^2&b\\c&c&c^2\end{pmatrix} \quad C.\begin{pmatrix}a^2&2ab&2ac\\2ab&b^2&2bc\\2ac&2bc&c^2\end{pmatrix} \quad D.\frac12\begin{pmatrix}2a^2&ab&ac\\ab&2b^2&bc\\ac&bc&2c^2\end{pmatrix} \quad E. \quad F. \quad G. \quad H.
$$
$$
\begin{array}{l}f(x_1,x_2,x_3)=(ax_1+bx_2+cx_3)^2=(ax_1+bx_2+cx_3)(ax_1+bx_2+cx_3)=(x_1,x_2,x_3)\begin{pmatrix}a\\b\\c\end{pmatrix}(a,b,c)\begin{pmatrix}x_1\\x_2\\x_3\end{pmatrix}\\=(x_1,x_2,x_3)\begin{pmatrix}a^2&ab&ac\\ab&b^2&bc\\ac&bc&c^2\end{pmatrix}\begin{pmatrix}x_1\\x_2\\x_3\end{pmatrix}\end{array}
$$



$$
\mathrm{二次型}f(x_1,x_2,x_3)=(a_1x_1+a_2x_2+a_3x_3)^2\mathrm{的矩阵为}()
$$
$$
A.
\begin{pmatrix}a_1^2&a_1a_2&a_1a_3\\a_1a_2&a_2^2&a_2a_3\\a_1a_3&a_2a_3&a_3^2\end{pmatrix} \quad B.\begin{pmatrix}a_1^2&a_2&a_3\\a_2&a_2^2&a_2\\a_3&a_2&a_3^2\end{pmatrix} \quad C.\begin{pmatrix}a_1^2&2a_1a_2&2a_1a_3\\2a_1a_2&a_2^2&2a_2a_3\\2a_1a_3&2a_2a_3&a_3^2\end{pmatrix} \quad D.\frac12\begin{pmatrix}2a_1^2&a_1a_2&a_1a_3\\a_1a_2&2a_2^2&a_2a_3\\a_1a_3&a_2a_3&2a_3^2\end{pmatrix} \quad E. \quad F. \quad G. \quad H.
$$
$$
\begin{array}{l}f(x_1,x_2,x_3)=a_1^2x_1^2+a_2^2x_2^2+a_3^2x_3^2+2a_1a_2x_1x_2+2a_1a_3x_1x_3+2a_2a_3x_2x_3,\mathrm{故其二次型矩阵为}\begin{pmatrix}a_1^2&a_1a_2&a_1a_3\\a_1a_2&a_2^2&a_2a_3\\a_1a_3&a_2a_3&a_3^2\end{pmatrix}\\\mathrm{或另解为}f(x_1,x_2,x_3)=(x_1,x_2,x_3)\begin{pmatrix}a_1\\a_2\\a_3\end{pmatrix}(a_1,a_2,a_3)\begin{pmatrix}x_1\\x_2\\x_3\end{pmatrix}=(x_1,x_2,x_3)\begin{pmatrix}a_1^2&a_1a_2&a_1a_3\\a_1a_2&a_2^2&a_2a_3\\a_1a_3&a_2a_3&a_3^2\end{pmatrix}\begin{pmatrix}x_1\\x_2\\x_3\end{pmatrix}\mathrm{故亦求得表示矩阵如上}\end{array}
$$



$$
\mathrm{二次型}f(x_1,x_2,⋯,x_n)=\begin{vmatrix}0&x_1&x_2&⋯&x_n\\-x_1&a_{11}&a_{12}&⋯&a_{1n}\\-x_2&a_{21}&a_{22}&⋯&a_{2n}\\⋯&⋯&⋯&⋯&⋯\\-x_n&a_{n1}&a_{n2}&⋯&a_{nn}\end{vmatrix}\mathrm{的矩阵为}()
$$
$$
A.
B=(b_{ij})_{n× n},b_{ij}=\frac{A_{ij}+A_{ji}}2(i,j=1,2,⋯,n) \quad B.B=(b_{ij})_{n× n},b_{ij}=\frac{A_{ij}-A_{ji}}2(i,j=1,2,⋯,n) \quad C.B=(b_{ij})_{n× n},b_{ij}=A_{ij}+A_{ji}(i,j=1,2,⋯,n) \quad D.B=(b_{ij})_{n× n},b_{ij}=A_{ij}-A_{ji}(i,j=1,2,⋯,n) \quad E. \quad F. \quad G. \quad H.
$$
$$
\begin{array}{l}\mathrm{将行列式按第一行展开得}f=-x_1\begin{vmatrix}-x_1&a_{12}&⋯&a_{1n}\\-x_2&a_{22}&⋯&a_{2n}\\⋯&⋯&⋯&⋯\\-x_n&a_{n2}&⋯&a_{nn}\end{vmatrix}+x_2\begin{vmatrix}-x_1&a_{11}&⋯&a_{1n}\\-x_2&a_{21}&⋯&a_{2n}\\⋯&⋯&⋯&⋯\\-x_n&a_{n1}&⋯&a_{nn}\end{vmatrix}+⋯+(-1)^{n+1}x_n\begin{vmatrix}-x_1&a_{11}&⋯&a_{1n-1}\\-x_2&a_{12}&⋯&a_{2n-1}\\⋯&⋯&⋯&⋯\\-x_n&a_{n1}&⋯&a_{nn-1}\end{vmatrix}\\=(A_{11}x_1^2+A_{12}x_1x_2+⋯+A_{1n}x_1x_n)+(A_{21}x_1x_2+A_{22}x_2^2+⋯+A_{2n}x_2x_n)+⋯+(A_{n1}x_1x_n+⋯+A_{n\;n-1}x_{n-1}x_n+A_{nn}x_n^2)\\\mathrm{故二次型的矩阵为}B=(b_{ij})_{n× n},b_{ij}=\frac{A_{ij}+A_{ji}}2(i,j=1,2,⋯,n)\end{array}
$$



$$
\mathrm{二次型}f=x^2+2xy+2y^2+4xz+4z^2\mathrm{的矩阵表示为}()
$$
$$
A.
(x,y,z)\begin{pmatrix}1&1&2\\1&2&0\\2&0&4\end{pmatrix}\begin{pmatrix}x\\y\\z\end{pmatrix} \quad B.(x,y,z)\begin{pmatrix}1&2&4\\2&4&0\\4&0&2\end{pmatrix}\begin{pmatrix}x\\y\\z\end{pmatrix} \quad C.(x,y,z)\begin{pmatrix}1&4&2\\4&2&0\\2&0&4\end{pmatrix}\begin{pmatrix}x\\y\\z\end{pmatrix} \quad D.(x,y,z)\begin{pmatrix}1&2&2\\2&2&0\\2&0&4\end{pmatrix}\begin{pmatrix}x\\y\\z\end{pmatrix} \quad E. \quad F. \quad G. \quad H.
$$
$$
f=(x,y,z)\begin{pmatrix}1&1&2\\1&2&0\\2&0&4\end{pmatrix}\begin{pmatrix}x\\y\\z\end{pmatrix}
$$



$$
\mathrm{二次型}f(x)=x^T\begin{bmatrix}2&1\\3&1\end{bmatrix}x\mathrm{的矩阵为}()
$$
$$
A.
\begin{bmatrix}2&1\\1&1\end{bmatrix} \quad B.\begin{bmatrix}2&2\\2&1\end{bmatrix} \quad C.\begin{bmatrix}2&1\\1&2\end{bmatrix} \quad D.\begin{bmatrix}1&1\\1&2\end{bmatrix} \quad E. \quad F. \quad G. \quad H.
$$
$$
\begin{array}{l}f(x)=(x_1,x_2)\begin{bmatrix}2&1\\3&1\end{bmatrix}\begin{pmatrix}x_1\\x_2\end{pmatrix}=2x_1^2+4x_1x_2+x_2^2=(x_1,x_2)\begin{bmatrix}2&2\\2&1\end{bmatrix}\begin{pmatrix}x_1\\x_2\end{pmatrix}\\\mathrm{于是}f\mathrm{的矩阵为}\begin{bmatrix}2&2\\2&1\end{bmatrix}\end{array}
$$



$$
\mathrm{二次型}f(x)=x^T\begin{pmatrix}1&1&2\\1&1&1\\0&1&1\end{pmatrix}x\mathrm{的矩阵为}()
$$
$$
A.
\begin{pmatrix}1&1&2\\1&1&1\\2&1&1\end{pmatrix} \quad B.\begin{pmatrix}1&1&0\\1&1&1\\0&1&1\end{pmatrix} \quad C.\begin{pmatrix}1&1&1\\1&1&1\\1&1&1\end{pmatrix} \quad D.\begin{pmatrix}1&1&1\\1&1&1\\1&1&2\end{pmatrix} \quad E. \quad F. \quad G. \quad H.
$$
$$
\begin{array}{l}f(x)=(x_1,x_2,x_3)\begin{pmatrix}1&1&2\\1&1&1\\0&1&1\end{pmatrix}\begin{pmatrix}x_1\\x_2\\x_3\end{pmatrix}=x_1^2+x_2^2+x_3^2+2x_1x_2+2x_1x_3+2x_2x_3=(x_1,x_2,x_3)\begin{pmatrix}1&1&1\\1&1&1\\1&1&1\end{pmatrix}\begin{pmatrix}x_1\\x_2\\x_3\end{pmatrix}\\\mathrm{于是}f\mathrm{的矩阵为}\begin{pmatrix}1&1&1\\1&1&1\\1&1&1\end{pmatrix}\end{array}
$$



$$
\mathrm{设矩阵}A=\begin{pmatrix}1&3&5\\3&-2&-4\\5&-4&-1\end{pmatrix},A\mathrm{相应的二次型的表达式为}()
$$
$$
A.
f=x_1^2-2x_2^2-x_3^2+3x_1x_2+5x_1x_3-4x_2x_3 \quad B.f=x_1^2-2x_2^2-x_3^2+6x_1x_2+10x_1x_3-8x_2x_3 \quad C.f=x_1^2-2x_2^2-x_3^2+6x_1x_2+5x_1x_3-8x_2x_3 \quad D.f=x_1^2-2x_2^2-x_3^2+6x_1x_2+10x_1x_3-4x_2x_3 \quad E. \quad F. \quad G. \quad H.
$$
$$
\begin{array}{l}\mathrm{设二次型的变量为}x=(x_1,x_2,x_3)\begin{pmatrix}1&2&3\\4&5&6\\7&8&9\end{pmatrix}\begin{pmatrix}x_1\\x_2\\x_3\end{pmatrix}=x_1^2+5x_2^2+9x_3^2+6x_1x_2+10x_1x_3+14x_2x_3=(x_1,x_2,x_3)\begin{pmatrix}1&3&5\\3&5&7\\5&7&9\end{pmatrix}\begin{pmatrix}x_1\\x_2\\x_3\end{pmatrix}\\\mathrm{于是}f\mathrm{的矩阵为}\begin{pmatrix}1&3&5\\3&5&7\\5&7&9\end{pmatrix}\end{array}
$$



$$
设A、B\mathrm{均为}n\mathrm{阶矩阵},x=(x_1,x_2,⋯,x_n)^T且x^TAx=x^TBx,\mathrm{则当}(\;)时,A=B.
$$
$$
A.
r(A)=r(B) \quad B.A^T=A \quad C.B^T=B \quad D.A^T=A且B^T=B \quad E. \quad F. \quad G. \quad H.
$$
$$
\mathrm{二次型的矩阵必须是对称矩阵},\mathrm{且二次型与矩阵是一一对应的},\mathrm{若二次型相等},\mathrm{则对应的矩阵相等}.
$$



$$
\mathrm{二次型}\;f(x_1,x_2,x_3)=x_1^2+4x_2^2+3x_3^2-4x_1x_2+2x_1x_3-4x_2x_3\mathrm{的秩等于}()
$$
$$
A.
1 \quad B.2 \quad C.3 \quad D.0 \quad E. \quad F. \quad G. \quad H.
$$
$$
\mathrm{二次型的矩阵为}A=\begin{pmatrix}1&-2&1\\-2&4&-2\\1&-2&3\end{pmatrix},\mathrm{由于}A\rightarrow\begin{pmatrix}1&-2&1\\0&0&2\\0&0&0\end{pmatrix},即r(A)=2,\mathrm{又二次型的秩为其矩阵的秩},\mathrm{故二次型的秩为}2
$$



$$
\mathrm{二次型}\;f(x_1,x_2,x_3)=2x_1^2+x_2^2-4x_3^2-4x_1x_2-4x_2x_3\mathrm{的秩等于}()
$$
$$
A.
0 \quad B.1 \quad C.2 \quad D.3 \quad E. \quad F. \quad G. \quad H.
$$
$$
\begin{array}{l}\mathrm{二次型的矩阵为}A=\begin{pmatrix}2&-2&0\\-2&1&-2\\0&-2&-4\end{pmatrix},\mathrm{对矩阵}A\mathrm{施以初等行变换得}\begin{pmatrix}2&-2&0\\-2&1&-2\\0&-2&-4\end{pmatrix}\rightarrow\begin{pmatrix}1&-1&0\\0&-1&-2\\0&-2&-4\end{pmatrix}\rightarrow\begin{pmatrix}1&-1&0\\0&1&2\\0&0&0\end{pmatrix}\\\mathrm{由于}r(A)=2,\mathrm{故二次型的秩也为}2.\end{array}
$$



$$
\mathrm{二次型}\;f(x_1,x_2,x_3)=2x_1^2-3x_2^2-4x_3^2-4x_1x_2+10x_1x_3-12x_2x_3\mathrm{的秩等于}()
$$
$$
A.
1 \quad B.2 \quad C.3 \quad D.0 \quad E. \quad F. \quad G. \quad H.
$$
$$
\begin{array}{l}\mathrm{二次型的矩阵为}A=\begin{pmatrix}2&-2&5\\-2&-3&-6\\5&-6&-4\end{pmatrix},则A=\begin{pmatrix}2&-2&5\\-2&-3&-6\\5&-6&-4\end{pmatrix}\rightarrow\begin{pmatrix}2&-2&5\\0&-5&-1\\5&-6&-4\end{pmatrix}\rightarrow\begin{pmatrix}2&-2&5\\0&-5&-1\\0&-1&-\frac{33}2\end{pmatrix}\rightarrow\begin{pmatrix}2&-2&5\\0&1&\frac{33}2\\0&0&\frac{163}2\end{pmatrix}\\\mathrm{由于}r(A)=3,\mathrm{故二次型的秩等于其矩阵的秩},\mathrm{也为}3.\end{array}
$$



$$
\mathrm{二次型}\;f(x_1,x_2,x_3)=x_1^2-4x_1x_2+2x_1x_3-2x_2^2+6x_3^2\mathrm{的秩等于}()
$$
$$
A.
1 \quad B.2 \quad C.3 \quad D.4 \quad E. \quad F. \quad G. \quad H.
$$
$$
\begin{array}{l}\mathrm{先求二次型的矩阵}f(x_1,x_2,x_3)=x_1^2-2x_1x_2+x_1x_3-2x_2x_1-2x_2^2+0x_2x_3+x_3x_1+0x_3x_2+6x_3^2\mathrm{所以}\\A=\begin{pmatrix}1&-2&1\\-2&-2&0\\1&0&6\end{pmatrix},对A\mathrm{初等变换}A\rightarrow\begin{pmatrix}1&-2&1\\0&-6&2\\0&2&5\end{pmatrix}\rightarrow\begin{pmatrix}1&-2&1\\0&2&5\\0&0&17\end{pmatrix}即r(A)=3,\mathrm{所以二次型的秩为}3.\end{array}
$$



$$
若n\mathrm{阶矩阵}A与B\mathrm{合同},则()
$$
$$
A.
A=B \quad B.A∼ B \quad C.\left|A\right|=\left|B\right| \quad D.r(A)=r(B) \quad E. \quad F. \quad G. \quad H.
$$
$$
A与B\mathrm{合同},\mathrm{即存在}n\mathrm{阶可逆矩阵}C\mathrm{使得}C^TAC=B,C^TAC\mathrm{相当于对}A\mathrm{初等变换}r(A)=r(B)
$$



$$
\mathrm{二次型}f(x_1,x_2,x_3)=2x_1^2-3x_2^2-4x_1x_2+10x_1x_3-12x_2x_3\mathrm{的秩是}()
$$
$$
A.
1 \quad B.2 \quad C.3 \quad D.4 \quad E. \quad F. \quad G. \quad H.
$$
$$
\mathrm{二次型的矩阵为}\begin{pmatrix}2&-2&5\\-2&-3&-6\\5&-6&0\end{pmatrix},且r(f)=r(A)=3
$$



$$
设A\mathrm{为实对称矩阵},且\left|A\right|\neq0\mathrm{把二次型}f=x^TAx\mathrm{化为}f=y^TA^{-1}y\mathrm{的线性变换是}x=(\;)y.
$$
$$
A.
A \quad B.A^T \quad C.A^{-1} \quad D.A^2 \quad E. \quad F. \quad G. \quad H.
$$
$$
令x=A^{-1}y,则(A^{-1})^TA(A^{-1})=(A^{-1})^T=(A^T)^{-1}=A^{-1},\mathrm{因此线性变换为}x=A^{-1}y.
$$



$$
\mathrm{二次型}\;f(x_1,x_2,x_3)=x_1^2+4x_2^2+x_3^2-4x_1x_2+2x_1x_3-4x_2x_3\mathrm{的秩等于}()
$$
$$
A.
1 \quad B.2 \quad C.3 \quad D.0 \quad E. \quad F. \quad G. \quad H.
$$
$$
\mathrm{二次型的矩阵为}A=\begin{pmatrix}1&-2&1\\-2&4&-2\\1&-2&1\end{pmatrix},\mathrm{由于}A\rightarrow\begin{pmatrix}1&-2&1\\0&0&0\\0&0&0\end{pmatrix},即r(A)=1,\mathrm{又二次型的秩为其矩阵的秩},\mathrm{故二次型的秩为}1.
$$



$$
\mathrm{二次型}\;f=2x_1^2+x_2^2-4x_1x_2-4x_2x_3\mathrm{的秩等于}()
$$
$$
A.
0 \quad B.1 \quad C.2 \quad D.3 \quad E. \quad F. \quad G. \quad H.
$$
$$
\begin{array}{l}\mathrm{二次型的矩阵为}A=\begin{pmatrix}2&-2&0\\-2&1&-2\\0&-2&0\end{pmatrix},\mathrm{对矩阵}A\mathrm{施以初等变换}\;\begin{pmatrix}2&-2&0\\-2&1&-2\\0&-2&0\end{pmatrix}\rightarrow\begin{pmatrix}1&-1&0\\0&-1&-2\\0&-2&0\end{pmatrix}\rightarrow\begin{pmatrix}1&-1&0\\0&1&2\\0&0&4\end{pmatrix}\\\mathrm{由于}r(A)=3,\mathrm{二次型的秩也为}3.\end{array}
$$



$$
\mathrm{二次型}\;f(x_1,x_2,x_3)=2x_1x_2-4x_1x_3+6x_2x_3\mathrm{的秩等于}()
$$
$$
A.
0 \quad B.2 \quad C.1 \quad D.3 \quad E. \quad F. \quad G. \quad H.
$$
$$
\begin{array}{l}\mathrm{根据条件可直接写出二次型的实对称矩阵为}A=\begin{pmatrix}0&1&-2\\1&0&3\\-2&3&0\end{pmatrix};\;\mathrm{二次型的秩即为二次型矩阵的秩},\\\mathrm{由于}\left|A\right|=-12\neq0⇒ r(A)=3,\mathrm{故二次型的秩为}3.\end{array}
$$



$$
\mathrm{二次型}f(x,y,z)=3x^2+7z^2+2xy-3xz+4yz\mathrm{的秩为}()
$$
$$
A.
0 \quad B.1 \quad C.2 \quad D.3 \quad E. \quad F. \quad G. \quad H.
$$
$$
\mathrm{根据条件可得二次型的矩阵为}A=\begin{pmatrix}3&1&-\frac32\\1&0&2\\-\frac32&2&7\end{pmatrix}\mathrm{由于}\left|A\right|\neq0⇒ r(A)=3,即f\mathrm{的秩为}3.
$$



$$
\mathrm{二次型}f=xy+xz+yz\mathrm{的秩等于}()
$$
$$
A.
0 \quad B.1 \quad C.3 \quad D.2 \quad E. \quad F. \quad G. \quad H.
$$
$$
\mathrm{根据条件可得二次型的矩阵}A=\begin{pmatrix}0&\frac12&\frac12\\\frac12&0&\frac12\\\frac12&\frac12&0\end{pmatrix},f\mathrm{的秩即为矩阵}A\mathrm{的秩为}3.
$$



$$
\mathrm{二次型}f(x,y,z)=2x^2+2y^2+5z^2-4xy-2xz+2yz\mathrm{的秩为}()
$$
$$
A.
0 \quad B.1 \quad C.2 \quad D.3 \quad E. \quad F. \quad G. \quad H.
$$
$$
\mathrm{根据条件可得二次型的矩阵为}A=\begin{pmatrix}2&-2&-1\\-2&2&1\\-1&1&5\end{pmatrix},f\mathrm{的秩即为}A\mathrm{的秩等于}2.
$$



$$
\mathrm{二次型}f(x,y,z)=4x^2+4y^2+4z^2+4xy+4xz-4yz\mathrm{的秩为}()
$$
$$
A.
1 \quad B.3 \quad C.2 \quad D.4 \quad E. \quad F. \quad G. \quad H.
$$
$$
\mathrm{根据条件易得二次型的秩矩阵为}A=\begin{pmatrix}4&2&2\\2&4&-2\\2&-2&4\end{pmatrix},f\mathrm{的秩即为}A\mathrm{的秩为}2.
$$



$$
设α=\begin{pmatrix}1\\2\\3\\4\end{pmatrix},β=\begin{pmatrix}-1\\2\\-2\\0\end{pmatrix},A=(αα^T)^2,若f(x)=x^TAx,则f(β)=(\;)
$$
$$
A.
270 \quad B.30 \quad C.-270 \quad D.-30 \quad E. \quad F. \quad G. \quad H.
$$
$$
\begin{array}{l}(αα^T)^2=αα^Tαα^T=α(αα^T)α^T=30αα^T\\f(β)=30β^Tαα^Tβ=30(β^Tα)(α^Tβ)=270\end{array}
$$



$$
\mathrm{二次型}\;f(x_1,x_2,x_3)=x_1^2+x_2^2+ax_3^2+4x_1x_2+6x_2x_3\mathrm{的秩为}2,则a\mathrm{的值为}()
$$
$$
A.
-3 \quad B.3 \quad C.-2 \quad D.2 \quad E. \quad F. \quad G. \quad H.
$$
$$
\mathrm{对二次型的矩阵作初等变换}\begin{pmatrix}1&2&0\\2&1&3\\0&3&a\end{pmatrix}\overset{r_3-2r_1}↔\begin{pmatrix}1&2&0\\0&-3&3\\0&3&a\end{pmatrix}\overset{r_3+r_2}↔\begin{pmatrix}1&2&0\\0&-3&3\\0&0&a+3\end{pmatrix},\mathrm{秩为}2,\mathrm{所以}a+3=0,a=-3.
$$



$$
\mathrm{二次型}\;f(x_1,x_2,x_3,x_4)=x_1^2-2x_2^2+x_3^2-5x_4^2+2x_2x_3\mathrm{的秩为}()
$$
$$
A.
1 \quad B.2 \quad C.3 \quad D.4 \quad E. \quad F. \quad G. \quad H.
$$
$$
f=x_1^2-2x_2^2+x_3^2-5x_4^2+2x_2x_3=(x_1,x_2,x_3,x_4)\begin{pmatrix}1&0&0&0\\0&-2&1&0\\0&1&1&0\\0&0&0&-5\end{pmatrix}\begin{pmatrix}x_1\\x_2\\x_3\end{pmatrix}=x^TAx,\mathrm{二次型的秩即为二次型矩阵的秩}r(A)=4
$$



$$
\mathrm{已知二次型}5x_1^2+5x_2^2+cx_3^2-2x_1x_2+6x_1x_3-6x_2x_3\mathrm{的秩为}2,则c=(\;)
$$
$$
A.
3 \quad B.1 \quad C.5 \quad D.2 \quad E. \quad F. \quad G. \quad H.
$$
$$
\mathrm{二次型矩阵为}\begin{pmatrix}5&-1&3\\-1&5&-3\\3&-3&c\end{pmatrix},\mathrm{秩为}2,\mathrm{解得}c=3.
$$



$$
\mathrm{二次型}\;f(x_1,x_2,x_3,x_4)=x_1^2+2x_1x_2-x_2^2+4x_2x_3+2x_3x_4+3x_3^2+2x_4^2\mathrm{的秩为}()
$$
$$
A.
1 \quad B.2 \quad C.3 \quad D.4 \quad E. \quad F. \quad G. \quad H.
$$
$$
\mathrm{二次型的矩阵为}A=\begin{pmatrix}1&1&0&0\\1&-1&2&0\\0&2&3&1\\0&0&1&2\end{pmatrix},\mathrm{二次型的秩即为矩阵}A\mathrm{的秩},\mathrm{由于}r(A)=4,\mathrm{因此二次型的秩为}4.
$$



$$
设A=\begin{pmatrix}1&1&1&1\\1&1&1&1\\1&1&1&1\\1&1&1&1\end{pmatrix},B=\begin{pmatrix}4&0&0&0\\0&0&0&0\\0&0&0&0\\0&0&0&0\end{pmatrix},则A与B()
$$
$$
A.
\mathrm{合同且相似} \quad B.\mathrm{合同但不相似} \quad C.\mathrm{不合同且相似} \quad D.\mathrm{不合同不相似} \quad E. \quad F. \quad G. \quad H.
$$
$$
\begin{array}{l}A\mathrm{的特征多项式为}\left|A-λ E\right|=-λ^3(λ-4),\mathrm{因此}A\mathrm{的特征值为}λ_1=4,λ_2=λ_3=λ_4=0.\\又A\mathrm{为实对称矩阵},\mathrm{因此存在正交矩阵}P,\mathrm{使得}P^{-1}AP=B,即A与B\mathrm{相似}.\\\mathrm{又由于正交矩阵}P^{-1}=P^T,\mathrm{因此}P^TAP=B,即A与B\mathrm{合同}.\end{array}
$$



$$
\mathrm{二次型}f(x_1,x_2,x_3)=x^T\begin{pmatrix}1&2&1\\0&1&0\\1&2&1\end{pmatrix}x\mathrm{的秩为}()
$$
$$
A.
1 \quad B.2 \quad C.3 \quad D.0 \quad E. \quad F. \quad G. \quad H.
$$
$$
\begin{array}{l}\mathrm{二次型可表示为}x_1^2+x_2^2+x_3^2+2x_1x_2+2x_1x_3+2x_2x_3,\mathrm{因此二次型的矩阵为对称矩阵}A=\begin{pmatrix}1&1&1\\1&1&1\\1&1&1\end{pmatrix},\\\mathrm{由于}r(A)=1,\mathrm{所以二次型的秩为}1.\end{array}
$$



$$
\mathrm{二次型}f(x_1,x_2,x_3)=x^T\begin{pmatrix}1&2&1\\3&1&4\\2&2&1\end{pmatrix}x\mathrm{的秩为}()
$$
$$
A.
0 \quad B.1 \quad C.2 \quad D.3 \quad E. \quad F. \quad G. \quad H.
$$
$$
\begin{array}{l}\mathrm{由于二次型的矩阵为对称矩阵},故\begin{pmatrix}1&2&1\\3&1&4\\2&2&1\end{pmatrix}\mathrm{不是二次型的矩阵}.\\f(x_1,x_2,x_3)=x^T\begin{pmatrix}1&2&1\\3&1&4\\2&2&1\end{pmatrix}x=x_1^2+x_2^2+x_3^2+5x_1x_2+3x_1x_3+6x_2x_3\\\mathrm{因此二次型的矩阵为}\begin{pmatrix}1&\frac52&\frac32\\\frac52&1&3\\\frac32&3&1\end{pmatrix},\mathrm{二次型的秩即为二次型矩阵的秩},\mathrm{等于}3.\end{array}
$$



$$
\mathrm{对于对称矩阵}A=\begin{pmatrix}0&1&1\\1&2&1\\1&1&0\end{pmatrix},B=\begin{pmatrix}2&1&1\\1&0&1\\1&1&0\end{pmatrix},\mathrm{若存在非奇异矩阵}C,使C^TAC=B,则C=(\;).
$$
$$
A.
\begin{pmatrix}0&1&0\\1&0&0\\0&0&1\end{pmatrix} \quad B.\begin{pmatrix}0&0&1\\1&0&0\\0&0&1\end{pmatrix} \quad C.\begin{pmatrix}0&1&0\\0&0&1\\1&0&0\end{pmatrix} \quad D.\begin{pmatrix}1&0&0\\0&0&1\\0&1&0\end{pmatrix} \quad E. \quad F. \quad G. \quad H.
$$
$$
\begin{array}{l}设f=(x_1,x_2,x_3)\begin{pmatrix}0&1&1\\1&2&1\\1&1&0\end{pmatrix}\begin{pmatrix}x_1\\x_2\\x_3\end{pmatrix}=2x_2^2+2x_1x_2+2x_1x_3+2x_2x_3,g=(y_1,y_2,y_3)\begin{pmatrix}2&1&1\\1&0&1\\1&1&0\end{pmatrix}\begin{pmatrix}y_1\\y_2\\y_3\end{pmatrix}=2y_1^2+2y_1y_2+2y_1y_3+2y_2y_3\\令\left\{\begin{array}{c}y_1=x_2\\y_2=x_1\\y_3=x_3\end{array}\right.则,g=2x_2^2+2x_1x_2+2x_1x_3+2x_2x_3=f,\mathrm{因此}A和B\mathrm{分别是二次型}f\mathrm{在变元}x_1,x_2,x_3\mathrm{下的二次型及}y_1,y_2,y_3\mathrm{下的二次型},\\\mathrm{由二次型的性质},令\left\{\begin{array}{c}x_1=y_2\\x_2=y_1\\x_3=y_3\end{array}\right.,\mathrm{即当}C=\begin{pmatrix}0&1&0\\1&0&0\\0&0&1\end{pmatrix}时,\left|C\right|=-1\neq0,\mathrm{非奇异}\begin{pmatrix}x_1\\x_2\\x_3\end{pmatrix}=\begin{pmatrix}0&1&0\\1&0&0\\0&0&1\end{pmatrix}\begin{pmatrix}y_1\\y_2\\y_3\end{pmatrix},\\f=(x_1,x_2,x_3)A\begin{pmatrix}x_1\\x_2\\x_3\end{pmatrix}=\begin{pmatrix}C\begin{pmatrix}y_1\\y_2\\y_3\end{pmatrix}\end{pmatrix}^TAC\begin{pmatrix}y_1\\y_2\\y_3\end{pmatrix}=(y_1,y_2,y_3)C^TAC\begin{pmatrix}y_1\\y_2\\y_3\end{pmatrix}=(y_1,y_2,y_3)B\begin{pmatrix}y_1\\y_2\\y_3\end{pmatrix},B=C^TAC,即C=\begin{pmatrix}0&1&0\\1&0&0\\0&0&1\end{pmatrix}\mathrm{为所求}.\end{array}
$$



$$
\begin{array}{l}\mathrm{设二次型}f=2x_1^2+x_2^2-4x_1x_2-4x_2x_3\mathrm{分别作下列两种可逆矩阵变换},(1)x=\begin{pmatrix}1&1&-2\\0&1&-2\\0&0&1\end{pmatrix}y;(2)x=\begin{pmatrix}\frac1{\sqrt2}&1&-1\\0&1&-1\\0&0&\frac12\end{pmatrix}y\\\mathrm{则新的二次型分别为}()\end{array}
$$
$$
A.
f=2y_1^2-y_2^2+4y_3^2;f=y_1^2-y_2^2+4y_3^2 \quad B.f=2y_1^2-y_2^2+3y_3^2;f=y_1^2-y_2^2+4y_3^2 \quad C.f=2y_1^2-y_2^2+3y_3^2;f=y_1^2-y_2^2+y_3^2 \quad D.f=2y_1^2-y_2^2+4y_3^2;f=y_1^2-y_2^2+y_3^2 \quad E. \quad F. \quad G. \quad H.
$$
$$
\begin{array}{l}(1)\;f=x^T\begin{pmatrix}2&-2&0\\-2&1&-2\\0&-2&0\end{pmatrix}x=y^T\begin{pmatrix}1&1&-2\\0&1&-2\\0&0&1\end{pmatrix}^T\begin{pmatrix}2&-2&0\\-2&1&-2\\0&-2&0\end{pmatrix}\begin{pmatrix}1&1&-2\\0&1&-2\\0&0&1\end{pmatrix}y=y^T\begin{pmatrix}2&0&0\\0&-1&0\\0&0&4\end{pmatrix}y\\f=2y_1^2-y_2^2+4y_3^2\\(2)\;f=x^T\begin{pmatrix}2&-2&0\\-2&1&-2\\0&-2&0\end{pmatrix}x=x^T\begin{pmatrix}\frac1{\sqrt2}&1&-1\\0&1&-1\\0&0&\frac12\end{pmatrix}^T\begin{pmatrix}2&-2&0\\-2&1&-2\\0&-2&0\end{pmatrix}\begin{pmatrix}\frac1{\sqrt2}&1&-1\\0&1&-1\\0&0&\frac12\end{pmatrix}y=y^T\begin{pmatrix}1&0&0\\0&-1&0\\0&0&1\end{pmatrix}y;\\f=y_1^2-y_2^2+y_3^2\end{array}
$$



$$
\mathrm{二次型}\;f(x_1,x_2,x_3)=x_1^2-x_2^2+3x_3^2\mathrm{的秩等于}()
$$
$$
A.
0 \quad B.1 \quad C.2 \quad D.3 \quad E. \quad F. \quad G. \quad H.
$$
$$
\begin{array}{l}\mathrm{二次型}\;f(x_1,x_2,x_3)=x_1^2-x_2^2+3x_3^2\mathrm{为标准形},\mathrm{所以}f\mathrm{的秩为}3\\\end{array}
$$



$$
\mathrm{二次型}x_1^2+2x_2^2+2x_1x_2-2x_1x_3\mathrm{的秩为}()
$$
$$
A.
0 \quad B.1 \quad C.2 \quad D.3 \quad E. \quad F. \quad G. \quad H.
$$
$$
f=(x_1+x_2-x_3)^2+x_2^2+2x_2x_3-x_3^2=(x_1+x_2-x_3)^2+(x_2+x_3)-2x_3^2=y_1^2+y_2^3-y_3^2\;\mathrm{秩为}3
$$



$$
\mathrm{二次型}x_1^2+x_2^2-x_4^2-2x_1x_4\mathrm{的秩为}()
$$
$$
A.
1 \quad B.2 \quad C.3 \quad D.4 \quad E. \quad F. \quad G. \quad H.
$$
$$
\begin{array}{l}\mathrm{该二次型的矩阵为}A=\begin{pmatrix}1&0&0&-1\\0&1&0&0\\0&0&0&0\\-1&0&0&-1\end{pmatrix},\mathrm{于是}\left|λ E-A\right|=λ(λ-1)(λ^2-2),\mathrm{从而}A\mathrm{的特征值为}:\\λ_1=0,λ_2=1,λ_3=\sqrt2,λ_4=-\sqrt2,\mathrm{故二次型的秩为}3\end{array}
$$



$$
\mathrm{二次型}2x_1x_2+2x_2x_3+2x_3x_4+2x_1x_4\mathrm{的秩为}()
$$
$$
A.
1 \quad B.2 \quad C.3 \quad D.4 \quad E. \quad F. \quad G. \quad H.
$$
$$
\begin{array}{l}\mathrm{二次型矩阵为}A=\begin{pmatrix}0&1&0&1\\1&0&1&0\\0&1&0&1\\1&0&1&0\end{pmatrix},\mathrm{于是}\left|λ E-A\right|=λ^2(λ-2)(λ+2)\mathrm{从而}\;\mathrm{的特征值为}λ_1=λ_2=0,λ_3=2,λ_4=-2\\\mathrm{故二次型的秩为}2\end{array}
$$



$$
\mathrm{二次型}\;f(x_1,x_2,⋯,x_n)=(nx_1)^2+(nx_2)^2+⋯+(nx_n)^2-(x_1+x_2+⋯+x_n)^2(n>1),则f\mathrm{的秩为}()
$$
$$
A.
1 \quad B.n-1 \quad C.0 \quad D.n \quad E. \quad F. \quad G. \quad H.
$$
$$
\begin{array}{l}\mathrm{二次型的矩阵为}A=\begin{bmatrix}n^2-1&-1&-1&⋯&-1\\-1&n^2-1&-1&⋯&-1\\\vdots&\vdots&\vdots&&\vdots\\-1&-1&-1&⋯&n^2-1\end{bmatrix}\rightarrow\begin{bmatrix}n^2-n&n^2-n&n^2-n&⋯&n^2-n\\-1&n^2-1&-1&⋯&-1\\\vdots&\vdots&\vdots&&\vdots\\-1&-1&-1&⋯&n^2-1\end{bmatrix}\\\xrightarrow[{n>1}]{}\begin{bmatrix}1&1&1&⋯&1\\-1&n^2-1&-1&⋯&-1\\\vdots&\vdots&\vdots&&\vdots\\-1&-1&-1&⋯&n^2-1\end{bmatrix}\rightarrow\begin{bmatrix}1&1&1&⋯&1\\0&n^2&0&⋯&0\\\vdots&\vdots&\vdots&&\vdots\\0&0&0&⋯&n^2\end{bmatrix}\\\mathrm{可见}f\mathrm{的秩为}n.\end{array}
$$



$$
\mathrm{二次型}f(x_1,x_2,x_3)=x^T\begin{pmatrix}1&1&1\\1&1&1\\1&1&1\end{pmatrix}x\mathrm{的秩为}()
$$
$$
A.
0 \quad B.1 \quad C.2 \quad D.3 \quad E. \quad F. \quad G. \quad H.
$$
$$
\begin{array}{l}\mathrm{二次型}f(x_1,x_2,x_3)\mathrm{的矩阵为}A=\begin{pmatrix}1&1&1\\1&1&1\\1&1&1\end{pmatrix}\\\\\mathrm{由于}r(A)=1,\mathrm{所以二次型矩阵的秩为}1.\end{array}
$$



$$
\mathrm{二次型}\;f(x_1,x_2,x_3)=x_1^2+6x_1x_2+4x_1x_3+x_2^2+2x_2x_3+tx_3^2,\mathrm{若其秩等于}2,\;则\;t\mathrm{应为}()
$$
$$
A.
0 \quad B.2 \quad C.\frac78 \quad D.1 \quad E. \quad F. \quad G. \quad H.
$$
$$
\begin{array}{l}\mathrm{二次型的矩阵为}A=\begin{pmatrix}1&3&2\\3&1&1\\2&1&t\end{pmatrix}\rightarrow;\;\mathrm{二次型的秩即为二次型矩阵的秩},\\\mathrm{由于}\left|A\right|=-12\neq0⇒ r(A)=3,\mathrm{故二次型的秩为}3.\end{array}
$$



$$
\mathrm{二次型}f(x_1,x_2,x_3)=-4x_1x_2+2x_2x_3+2x_1x_3\mathrm{经非奇异变换}\left\{\begin{array}{c}x_1=y_1+y_2+\frac12y_3\\x_2=y_1-y_2+\frac12y_3\\x_3=y_3\end{array}\right.\mathrm{所得到的标准形为}(\;)
$$
$$
A.
-4y_1^2-4y_2^2+y_3^2 \quad B.-4y_1^2+4y_2^2+y_3^2 \quad C.-4y_1^2+4y_2^2-y_3^2 \quad D.4y_1^2-4y_2^2+y_3^2 \quad E. \quad F. \quad G. \quad H.
$$
$$
\begin{array}{l}\mathrm{由题设可知二次型的矩阵为}A=\begin{pmatrix}0&-2&1\\-2&0&1\\1&1&0\end{pmatrix}\mathrm{经过的线性变换为}\begin{pmatrix}x_1\\x_2\\x_3\end{pmatrix}=\begin{pmatrix}1&1&\frac12\\1&-1&\frac12\\0&0&1\end{pmatrix}\begin{pmatrix}y_1\\y_2\\y_3\end{pmatrix}\mathrm{则变换后的二次型的矩阵为}\\C=B^TAB=\begin{pmatrix}1&1&0\\1&-1&0\\\frac12&\frac12&1\end{pmatrix}\begin{pmatrix}0&-2&1\\-2&0&1\\1&1&0\end{pmatrix}\begin{pmatrix}1&1&\frac12\\1&-1&\frac12\\0&0&1\end{pmatrix}=\begin{pmatrix}-4&0&0\\0&4&0\\0&0&1\end{pmatrix},\mathrm{则变换后的二次型为}f(y_1,y_2,y_3)=-4y_1^2+4y_2^2+y_3^2\end{array}
$$



$$
\mathrm{用初等变换法化二次型}2x_1x_2+2x_1x_3-4x_2x_3\mathrm{的标准型为}()
$$
$$
A.
2z_1^2-\frac12z_2^2+4z_3^2 \quad B.2z_1^2-z_2^2-2z_3^2 \quad C.2z_1^2-2z_2^2-4z_3^2 \quad D.2z_1^2-z_2^2+2z_3^2 \quad E. \quad F. \quad G. \quad H.
$$
$$
\begin{array}{l}\mathrm{此二次型对应的矩阵为}A=\begin{pmatrix}0&1&1\\1&0&-2\\1&-2&0\end{pmatrix}\mathrm{用初等变换法}\begin{pmatrix}A\\E\end{pmatrix}=\begin{pmatrix}0&1&1\\1&0&-2\\1&-2&0\\1&0&0\\0&1&0\\0&0&1\end{pmatrix}\rightarrow\begin{pmatrix}2&0&0\\0&-\frac12&0\\0&0&4\\1&-\frac12&2\\1&\frac12&-1\\0&0&1\end{pmatrix}\\\mathrm{所以}C=\begin{pmatrix}1&-\frac12&2\\1&\frac12&-1\\0&0&1\end{pmatrix},\left|C\right|=1\neq0,令\left\{\begin{array}{c}x_1=z_1-\frac12z_2+2z_3\\x_2=z_1+\frac12z_2-z_3\\x_3=z_3\end{array}\right.\mathrm{代入原二次型可得标准形}2z_1^2-\frac12z_2^2+4z_3^2\end{array}
$$



$$
\mathrm{用初等变换法化二次型}x_1^2+2x_2^2+x_3^2+2x_1x_2+2x_1x_3+4x_2x_3\mathrm{的标准型为}()
$$
$$
A.
z_1^2+z_2^2-z_3^2 \quad B.z_1^2+2z_2^2-3z_3^2 \quad C.z_1^2+z_2^2+z_3^2 \quad D.z_1^2+2z_2^2+3z_3^2 \quad E. \quad F. \quad G. \quad H.
$$
$$
\begin{array}{l}\mathrm{二次型矩阵为}A=\begin{pmatrix}1&1&1\\1&2&2\\1&2&1\end{pmatrix}\mathrm{利用初等变换有}\\\begin{pmatrix}A\\E\end{pmatrix}=\begin{pmatrix}1&1&1\\1&2&2\\1&2&1\\1&0&0\\0&1&0\\0&0&1\end{pmatrix}\xrightarrow[{c_3-c_1}]{c_2-c_1}\begin{pmatrix}1&0&0\\1&1&1\\1&1&0\\1&-1&-1\\0&1&0\\0&0&1\end{pmatrix}\xrightarrow[{r_3-r_1}]{r_2-r_1}\begin{pmatrix}1&0&0\\0&1&1\\0&1&0\\1&-1&-1\\0&1&0\\0&0&1\end{pmatrix}\xrightarrow[{r_3-r_2}]{c_3-c_2}\begin{pmatrix}1&0&0\\0&1&0\\0&0&-1\\1&-1&0\\0&1&-1\\0&0&1\end{pmatrix}\\\mathrm{因此},C=\begin{pmatrix}1&-1&0\\0&1&-1\\0&0&1\end{pmatrix},\left|C\right|=1\neq0,令\left\{\begin{array}{c}x_1=z_1-z_2\\x_2=z_2-z_3\\x_3=z_3\end{array}\right.,\mathrm{代入原二次型可得标准形}z_1^2+z_2^2-z_3^3\end{array}
$$



$$
\mathrm{用初等变换法化二次型}f(x_1,x_2,x_3)=x_1^2-x_3^2+2x_1x_2+2x_2x_3\mathrm{的标准型为}()
$$
$$
A.
y_1^2-y_2^2 \quad B.y_1^2+y_2^2 \quad C.y_1^2-y_2^2+y_3^2 \quad D.y_1^2+y_2^2-y_3^2 \quad E. \quad F. \quad G. \quad H.
$$
$$
\begin{array}{l}\mathrm{此二次型对应的矩阵为}A=\begin{pmatrix}1&1&0\\1&0&1\\0&1&-1\end{pmatrix},\begin{pmatrix}A\\E\end{pmatrix}=\begin{pmatrix}1&1&0\\1&0&1\\0&1&-1\\1&0&0\\0&1&0\\0&0&1\end{pmatrix}\xrightarrow{c_2-c_1}\begin{pmatrix}1&0&0\\1&-1&1\\0&1&-1\\1&-1&0\\0&1&0\\0&0&1\end{pmatrix}\xrightarrow{r_2-r_1}\begin{pmatrix}1&0&0\\0&-1&1\\0&1&-1\\1&-1&0\\0&1&0\\0&0&1\end{pmatrix}\\\xrightarrow{c_3+c_2}\begin{pmatrix}1&0&0\\0&-1&0\\0&1&0\\1&-1&-1\\0&1&1\\0&0&1\end{pmatrix}\xrightarrow{r_3+r_2}\begin{pmatrix}1&0&0\\0&-1&0\\0&0&0\\1&-1&-1\\0&1&1\\0&0&1\end{pmatrix},\mathrm{所以}C=\begin{pmatrix}1&-1&-1\\0&1&1\\0&0&1\end{pmatrix},\left|C\right|=1\neq0,令\begin{pmatrix}x_1\\x_2\\x_3\end{pmatrix}=\begin{pmatrix}1&-1&-1\\0&1&1\\0&0&1\end{pmatrix}\begin{pmatrix}y_1\\y_2\\y_3\end{pmatrix}\\\mathrm{代入原二次型可得标准形}f=y_1^2-y_2^2\end{array}
$$



$$
\mathrm{对于二次型}f(x)=x^TAx,\mathrm{其中}A为n\mathrm{阶实对称矩阵},\mathrm{下列结论中},\mathrm{正确的是}()
$$
$$
A.
化f(x)\mathrm{为标准形的非退化线性替换是惟一的} \quad B.化f(x)\mathrm{为规范形的非退化线性替换是惟一的} \quad C.f(x)\mathrm{的标准形是惟一的} \quad D.f(x)\mathrm{的规范形是惟一的} \quad E. \quad F. \quad G. \quad H.
$$
$$
\mathrm{二次型的规范形是由二次型本身决定的唯一形式},\mathrm{其它选项中的描述都不正确}
$$



$$
\mathrm{二次型}f(x_1,x_2,x_3,x_4)=-x_1^2+x_2^2+x_3^2-x_4^2,则f\mathrm{的负惯性指数是}()
$$
$$
A.
1 \quad B.2 \quad C.3 \quad D.4 \quad E. \quad F. \quad G. \quad H.
$$
$$
\mathrm{条件中的二次型为规范形},\mathrm{可直接根据定义求出正惯性指数即规范形中的正项个数为}2
$$



$$
\mathrm{二次型}f(x_1,x_2)=5x_1^2-x_2^2在x_1^2+x_2^2=1\mathrm{下的最大值是}()
$$
$$
A.
0 \quad B.1 \quad C.2 \quad D.5 \quad E. \quad F. \quad G. \quad H.
$$
$$
\mathrm{条件中的二次型为标准形式},\mathrm{可直接得出在}x_1^2+x_2^2=1时,取x_1=1,x_2=0,则f_\mathrm{最大}=5
$$



$$
\mathrm{二次型}f(x_1,x_2,x_3)=x_1^2+3x_3^2+2x_1x_2+4x_1x_3+2x_2x_3\mathrm{的符号差为}()
$$
$$
A.
0 \quad B.1 \quad C.2 \quad D.3 \quad E. \quad F. \quad G. \quad H.
$$
$$
\begin{array}{l}\mathrm{二次型的矩阵为}A=\begin{pmatrix}1&1&2\\1&0&1\\2&1&3\end{pmatrix},由\left|λ E-A\right|=0\mathrm{易求得矩阵的特征值为}λ_1=0,λ_2=2-\sqrt7,λ_3=2+\sqrt7\\\mathrm{则用正交变换法可将}f\mathrm{化为标准形}f=(2+\sqrt7)y_1^2+(2-\sqrt7)y_2^2,\mathrm{于是规范形为}f=y_1^2-y_2^2,\\\mathrm{因此}f\mathrm{的正负惯性指数均为}1,\mathrm{符号差为}0\end{array}
$$



$$
\mathrm{二次型}x_1^2+2x_2^2+2x_1x_2-2x_1x_3\mathrm{的正惯性指数为}()
$$
$$
A.
0 \quad B.1 \quad C.2 \quad D.3 \quad E. \quad F. \quad G. \quad H.
$$
$$
f=(x_1+x_2-x_3)^2+x_2^2+2x_2x_3-x_3^2=(x_1+x_2-x_3)^2+(x_2+x_3)-2x_3^2=y_1^2+y_2^3-y_3^2\;\mathrm{正惯性指数为}2
$$



$$
\mathrm{二次型}x_1^2+x_2^2-x_4^2-2x_1x_4\mathrm{的正惯性指数为}()
$$
$$
A.
1 \quad B.2 \quad C.3 \quad D.0 \quad E. \quad F. \quad G. \quad H.
$$
$$
\begin{array}{l}\mathrm{该二次型的矩阵为}A=\begin{pmatrix}1&0&0&-1\\0&1&0&0\\0&0&0&0\\-1&0&0&-1\end{pmatrix},\mathrm{于是}\left|λ E-A\right|=λ(λ-1)(λ^2-2),\mathrm{从而}A\mathrm{的特征值为}:\\λ_1=0,λ_2=1,λ_3=\sqrt2,λ_4=-\sqrt2,\mathrm{故二次型的规范形为}y_1^2+y_2^2-y_3^2,\mathrm{于是正惯性指数为}2\;\end{array}
$$



$$
\mathrm{二次型}2x_1x_2+2x_2x_3+2x_3x_4+2x_1x_4\mathrm{的正惯性指数为}()
$$
$$
A.
1 \quad B.2 \quad C.3 \quad D.4 \quad E. \quad F. \quad G. \quad H.
$$
$$
\begin{array}{l}\mathrm{二次型矩阵为}A=\begin{pmatrix}0&1&0&1\\1&0&1&0\\0&1&0&1\\1&0&1&0\end{pmatrix},\mathrm{于是}\left|λ E-A\right|=λ^2(λ-2)(λ+2)\mathrm{从而}\;\mathrm{的特征值为}λ_1=λ_2=0,λ_3=2,λ_4=-2\\\mathrm{故二次型的规范形为}y_1^2-y_2^2\;.\;\mathrm{于是正惯性指数为}1\end{array}
$$



$$
\mathrm{二次型}f=2x_1x_2+2x_1x_3-6x_2x_2\mathrm{的正惯性指数为}()
$$
$$
A.
0 \quad B.1 \quad C.2 \quad D.3 \quad E. \quad F. \quad G. \quad H.
$$
$$
\begin{array}{l}f\mathrm{经线性变换}\left\{\begin{array}{c}x_1=z_1+z_2+3z_3\\x_2=z_1-z_2-z_3\\x_3=z_3\end{array}\right.\mathrm{化为标准型}f=2z_1^2-2z_2^2+6z_3^2\\令\left\{\begin{array}{c}w_1=\sqrt2z_1\\w_3=\sqrt2z_2\\w_3=\sqrt6z_3\end{array}\right.⇒\left\{\begin{array}{c}z_1=\frac1{\sqrt2}w_1\\z_2=\frac1{\sqrt2}w_3\\z_3=\frac1{\sqrt6}w_2\end{array}\right.\mathrm{就把}f\mathrm{化成规范形}f=w_1^2+w_2^2-w_3^2,且f\mathrm{的正惯性指数为}2\\附:\mathrm{也可以根据二次型矩阵的特征值的正负来判定}\end{array}
$$



$$
\mathrm{二次型}f(x_1,x_2,x_3)=x_1x_2+x_1x_3+x_2x_3\mathrm{的符号差为}()
$$
$$
A.
0 \quad B.-1 \quad C.1 \quad D.-2 \quad E. \quad F. \quad G. \quad H.
$$
$$
\begin{array}{l}\mathrm{二次型的矩阵为}A=\begin{pmatrix}0&\frac12&\frac12\\\frac12&0&\frac12\\\frac12&\frac12&0\end{pmatrix}\mathrm{易求得}A\mathrm{的特征值为}λ_1=1,λ_2=λ_3=-\frac12,故f\mathrm{的标准型为}f=z_1^2-\frac12z_2^2-\frac12z_3^2\\故f\mathrm{的正惯性指数为}1,\mathrm{负惯性指数为}2,\mathrm{符号差为}-1\end{array}
$$



$$
\mathrm{二次型}f(x_1,x_2,x_3)=(x_1+x_2)^2+(x_2-x_3)^2+(x_1+x_3)^2\mathrm{的正}、\mathrm{负惯性指数分别为}()
$$
$$
A.
2,0 \quad B.2,1 \quad C.3,0 \quad D.1,1 \quad E. \quad F. \quad G. \quad H.
$$
$$
\begin{array}{l}f=(2x_1^2+2x_1x_2+2x_1x_3)+2x_2^2-2x_2x_3+2x_3^2=2(x_1+\frac12x_2+\frac12x_3)^2+\frac32(x_2-x_3)^2\mathrm{由于二次型的标准形是}2y_1^2+\frac32y_2^2\\\mathrm{所以正惯性指数}p=2,\mathrm{负惯性指数}q=0\end{array}
$$



$$
\mathrm{二次型}\;f(x_1,x_2,x_3)=x_1^2-x_2^2+3x_3^2\mathrm{的正惯性指数等于}()
$$
$$
A.
0 \quad B.1 \quad C.2 \quad D.3 \quad E. \quad F. \quad G. \quad H.
$$
$$
\begin{array}{l}\mathrm{二次型}\;f(x_1,x_2,x_3)=x_1^2-x_2^2+3x_3^2\mathrm{为标准形},\mathrm{所以}f\mathrm{的正惯性指数为}2\\\end{array}
$$



$$
\mathrm{二次型}f(x_1,x_2,x_3)=x_1^2+3x_2^2+x_3^2+2x_1x_2+2x_1x_3+2x_2x_3,\mathrm 则f\mathrm{的正惯性指数为}()
$$
$$
A.
0 \quad B.1 \quad C.2 \quad D.3 \quad E. \quad F. \quad G. \quad H.
$$
$$
\begin{array}{l}\mathrm{该二次型的矩阵为}A=\begin{bmatrix}1&1&1\\1&3&1\\1&1&1\end{bmatrix},\mathrm{于是}\left|λ E-A\right|=λ(λ-1)(λ-4),\mathrm{从而}A\mathrm{的特征值为}:\\λ_1=0,λ_2=1,λ_3=4,\mathrm{故二次型的正惯性指数为}2\end{array}
$$



$$
\mathrm{已知二次型}f=x_1^2-2x_2^2+ax_3^2+2x_1x_2-4x_1x_3+2x_2x_3\mathrm{的秩为}2,\mathrm 则f\mathrm{的规范形为}()
$$
$$
A.
y_1^2+y_2^2+y_3^2 \quad B.y_1^2+y_2^2-y_3^2 \quad C.y_1^2-y_2^2 \quad D.y_1^2+y_2^2 \quad E. \quad F. \quad G. \quad H.
$$
$$
\begin{array}{l}\begin{array}{l}\begin{array}{l}\mathrm{二次型}f\mathrm{的矩阵为}A=\begin{bmatrix}1&1&-2\\1&-2&1\\-2&1&a\end{bmatrix},\mathrm 由r(A)=2,\mathrm 知\vert A\vert=0,\mathrm 即3(1-a)=0.\mathrm{解之得}a=1.\mathrm{由矩阵}A\mathrm{的特征多项式}\\\end{array}\\\vertλ E-A\vert=\begin{vmatrix}λ-1&-1&2\\-1&λ+2&-1\\2&-1&λ-1\end{vmatrix}=λ(λ-3)(λ+3),\mathrm 得A\mathrm{的特征值}λ_1=0,λ_2=-3,λ_3=3.\end{array}\\\mathrm{由于}A\mathrm{的正负特征值各有一个},\mathrm{因此}f\mathrm{的规范形为}y_1^2-y_2^2.\\\end{array}
$$



$$
\mathrm{二次型}f(x_1,x_2)=2x_1^2+2x_2^2-2x_1x_2\mathrm{的标准型是}()
$$
$$
A.
-y_1^2-3y_2^2 \quad B.2y_1^2 \quad C.y_1^2+3y_2^2 \quad D.3y_2^2 \quad E. \quad F. \quad G. \quad H.
$$
$$
\begin{array}{l}f(x_1,x_2)=2x_1^2+2x_2^2-2x_1x_2=2(x_1^2+\frac14x_2^2-x_1x_2)+\frac32x_2^2=2z_1^2+\frac32z_2^2\mathrm{虽然二次型的标准形不唯一},\\\mathrm{但惯性指数和秩是由二次型本身唯一决定的},\mathrm{二次型的秩和正惯性指数都为}2,\mathrm{符合条件的只有}y_1^2+3y_2^2\end{array}
$$



$$
\mathrm{二次型}f(x_1,x_2)=5x_1^2+2x_2^2+4x_1x_2\mathrm{的标准型是}()
$$
$$
A.
y_1^2+6y_2^2 \quad B.y_1^2 \quad C.-6y_2^2 \quad D.-y_1^2-6y_2^2 \quad E. \quad F. \quad G. \quad H.
$$
$$
\begin{array}{l}\mathrm{用配方法可求得二次型的标准形}:f(x_1,x_2)=5x_1^2+2x_2^2+4x_1x_2=5(x_1^2+\frac45x_1x_2+\frac4{25}x_2^2)+\frac65x_2^2=5z_1^2+\frac65z_2^2\\\mathrm{故二次型的秩为}2,\mathrm{且正惯性指数为}2,\mathrm{符合条件的只有}y_1^2+6y_2^2\end{array}
$$



$$
\mathrm{二次型}f(x_1,x_2)=2x_1x_2\mathrm{的标准型是}()
$$
$$
A.
y_1^2+y_2^2 \quad B.-y_1^2-y_2^2 \quad C.3y_2^2 \quad D.y_1^2-y_2^2 \quad E. \quad F. \quad G. \quad H.
$$
$$
令x_1=z_1-z_2,x_2=z_1+z_2,则f(x_1,x_2)=2x_1x_2=2z_1^2-2z_2^2,\mathrm{故二次型的秩为}2,\mathrm{正负惯性指数都为}1,\mathrm{符合条件的为}y_1^2-y_2^2
$$



$$
\mathrm{二次型}f(x_1,x_2)=2x_1^2+2x_2^2+3x_1x_2\mathrm{的标准型是}()
$$
$$
A.
-y_1^2-2y_2^2 \quad B.-y_1^2 \quad C.y_1^2 \quad D.\frac72y_1^2+\frac12y_2^2 \quad E. \quad F. \quad G. \quad H.
$$
$$
\begin{array}{l}f(x_1,x_2)=2x_1^2+2x_2^2+3x_1x_2=2(x_1^2+\frac32x_1x_2+\frac9{16}x_2^2)+\frac78x_2^2=2z_1^2+\frac78z_2^2\mathrm{故二次型的秩和正惯性指数都为}2,\\\mathrm{符合条件的只有}\frac72y_1^2+\frac12y_2^2\end{array}
$$



$$
\mathrm{利用配方法化二次型}f(x_1,x_2,x_3)=2x_1^2+x_2^2-4x_3^2-4x_1x_2-2x_2x_3\mathrm{的标准型是}()
$$
$$
A.
2y_1^2-y_2^2-3y_3^2 \quad B.-2y_1^2-y_2^2-3y_3^2 \quad C.-2y_1^2 \quad D.2y_1^2+y_2^2+3y_3^2 \quad E. \quad F. \quad G. \quad H.
$$
$$
\begin{array}{l}f(x_1,x_2,x_3)=2x_1^2+x_2^2-4x_3^2-4x_1x_2-2x_2x_3=2(x_1^2-2x_1x_2+x_2^2)-x_2^2-2x_2x_3-4x_3^2\\\;\;\;\;\;\;\;\;\;\;\;\;\;\;\;\;\;=2(x_1-x_2)^2-(x_2+x_3)^2-3x_3^2=2y_1^2-y_2^2-3y_3^2\end{array}
$$



$$
\mathrm{利用配方法化二次型}f(x_1,x_2,x_3)=x_1^2+2x_2^2+3x_3^2-2x_1x_2+2x_2x_3\mathrm{的标准型是}()
$$
$$
A.
y_1^2+y_2^2+2y_3^2 \quad B.-y_1^2-y_2^2+2y_3^2 \quad C.y_1^2+y_2^2 \quad D.y_1^2-y_2^2 \quad E. \quad F. \quad G. \quad H.
$$
$$
f(x_1,x_2,x_3)=x_1^2+2x_2^2+3x_3^2-2x_1x_2+2x_2x_3=(x_1-x_2)^2+(x_2+x_3)^2+2x_3^2=y_1^2+y_2^2+2y_3^2
$$



$$
\mathrm{利用配方法化二次型}f(x_1,x_2)=2x_1^2+2x_2^2-4x_1x_2\mathrm{的标准型是}()
$$
$$
A.
-y_1^2-3y_2^2 \quad B.2y_1^2 \quad C.y_1^2+3y_2^2 \quad D.-3y_2^2 \quad E. \quad F. \quad G. \quad H.
$$
$$
f(x_1,x_2)=2x_1^2+2x_2^2-4x_1x_2=2(x_1^2-2x_1x_2+x_2^2)=2y_1^2
$$



$$
\mathrm{二次型}f(x_1,x_2)=5x_1x_2\mathrm{的标准型是}()
$$
$$
A.
y_1^2+y_2^2 \quad B.-y_1^2-y_2^2 \quad C.3y_2^2 \quad D.y_1^2-y_2^2 \quad E. \quad F. \quad G. \quad H.
$$
$$
令x_2=z_1-z_2,x_2=z_1+z_2,则f(x_1,x_2)=5x_1x_2=5z_1^2-5z_2^2,\mathrm{故二次型的秩为}2,\mathrm{正负惯性指数都为}1,\mathrm{符合条件的为}y_1^2-y_2^2
$$



$$
\mathrm{二次型}f(x_1,x_2)=x_1^2+2x_2^2+6x_1x_2\mathrm{的标准型是}()
$$
$$
A.
-y_1^2-2y_2^2 \quad B.-y_1^2 \quad C.y_1^2-y_2^2 \quad D.\frac72y_1^2+\frac12y_2^2 \quad E. \quad F. \quad G. \quad H.
$$
$$
f(x_1,x_2)=x_1^2+2x_2^2+6x_1x_2=(x_1^2+6x_1x_2+9x_2^2)-7x_2^2=z_1^2-7z_2^2,\mathrm{故二次型的秩和正惯性指数都为}1,\mathrm{符合条件的只有}y_1^2-y_2^2
$$



$$
\mathrm{二次型}f(x_1,x_2)=5x_1^2+2x_2^2-4x_1x_2\mathrm{的标准型是}()
$$
$$
A.
y_1^2+6y_2^2 \quad B.y_1^2 \quad C.-6y_2^2 \quad D.-y_1^2-6y_2^2 \quad E. \quad F. \quad G. \quad H.
$$
$$
\begin{array}{l}\mathrm{用配方法可求得二次型的标准形}:f(x_1,x_2)=5x_1^2+2x_2^2-4x_1x_2=5(x_1^2-\frac45x_1x+\frac4{25}x_2^2)+\frac65x_2^2=5z_1^2+\frac65z_2^2\\\mathrm{故二次型的秩为}2,\mathrm{且正惯性指数为}2,\mathrm{符合条件的只有}y_1^2+6y_2^2\end{array}
$$



$$
\mathrm{二次型}f(x_1,x_2,x_3)=2x_1^2+x_2^2-x_3^2-4x_1x_2-2x_2x_3\mathrm{的标准型是}()
$$
$$
A.
2y_1^2-y_2^2-3y_3^2 \quad B.-2y_1^2-y_2^2-3y_3^2 \quad C.2y_1^2-y_2^2 \quad D.2y_1^2+y_2^2+3y_3^2 \quad E. \quad F. \quad G. \quad H.
$$
$$
f(x_1,x_2,x_3)=2x_1^2+x_2^2-x_3^2-4x_1x_2-2x_2x_3=2(x_1^2-2x_1x_2+x_2^2)-x_2^2-2x_2x_3-x_3^2=2(x_1-x_2)^2-(x_2+x_3)^2=2y_1^2-y_2^2
$$



$$
\mathrm{二次型}f(x_1,x_2,x_3)=x_1^2+2x_2^2+3x_3^2-2x_1x_2+4x_2x_3\mathrm{的标准型是}()
$$
$$
A.
y_1^2+y_2^2+2y_3^2 \quad B.-y_1^2-y_2^2+2y_3^2 \quad C.y_1^2+y_2^2-y_3^2 \quad D.y_1^2-y_2^2 \quad E. \quad F. \quad G. \quad H.
$$
$$
f(x_1,x_2,x_3)=x_1^2+2x_2^2+3x_3^2-2x_1x_2+4x_2x_3=(x_1-x_2)^2+(x_2+2x_3)^2-x_3^2=y_1^2+y_2^2-y_3^2
$$



$$
\mathrm{利用配方法化二次型}x_1^2+2x_1x_2+2x_1x_3+2x_2^2+4x_2x_3+x_3^2\mathrm{的标准型是}()
$$
$$
A.
y_1^2+y_2^2-y_3^2 \quad B.y_1^2+y_2^2+y_3^2 \quad C.y_1^2-2y_2^2-y_3^2 \quad D.-y_1^2-2y_2^2-y_3^2 \quad E. \quad F. \quad G. \quad H.
$$
$$
\begin{array}{l}\mathrm{因标准形是平方项的代数和},\mathrm{可利用配方法解之}\\x_1^2+2x_1x_2+2x_1x_3+2x_2^2+4x_2x_3+x_3^2=x_1^2+2x_1(x_2+x_3)+(x_2+x_3)^2-(x_2+x_3)^2+2x_2^2+4x_2x_3+x_3^2\\=(x_1+x_2+x_3)^2+x_2^2+2x_2x_3=(x_1+x_2+x_3)^2+(x_2+x_3)^2-x_3^2\;\;\;\;\;\;\;\;\;(1)\\令\left\{\begin{array}{c}y_1=x_1+x_2+x_3\\y_2=x_2+x_3\\y_3=x_3\end{array}\right.⇒ 即\left\{\begin{array}{c}x_1=y_1-y_2\\x_2=y_2-y_3\\x_3=y_3\end{array}\right.,\mathrm{其线性变换矩阵的行列式}\left|C\right|=\begin{vmatrix}1&-1&0\\0&1&-1\\0&0&1\end{vmatrix}=1\neq0\\\mathrm{代入}(1)\mathrm{式得二次型的标准形}y_1^2+y_2^2-y_3^2\end{array}
$$



$$
\mathrm{利用配方法化二次型}f=2x_1x_2+2x_1x_3-6x_2x_3\mathrm{的标准型为}()
$$
$$
A.
2z_1^2-2z_2^2+6z_3^2 \quad B.2z_1^2-2z_2^2-z_3^2 \quad C.2z_1^2+2z_2^2+6z_3^2 \quad D.-z_1^2-2z_2^2-z_3^2 \quad E. \quad F. \quad G. \quad H.
$$
$$
\begin{array}{l}\mathrm{由于所给二次型中无平方项},\mathrm{所以令}\left\{\begin{array}{c}x_1=y_1+y_2\\x_2=y_1-y_2\\x_3=y_3\end{array}\right.,即\begin{pmatrix}x_1\\x_2\\x_3\end{pmatrix}=\begin{pmatrix}1&1&0\\1&-1&0\\0&0&1\end{pmatrix}\begin{pmatrix}y_1\\y_2\\y_3\end{pmatrix}\mathrm{代入原二次型得}f=2y_1^2-2y_2^2-4y_1y_3+8y_2y_3\\\mathrm{再配方}f=2(y_1-y_3)^2-2(y_2+2y_3)^2+6y_3^2,令\left\{\begin{array}{c}z_1=y_1-y_3\\z_2=y_2+2y_3\\z_3=y_3\end{array}\right.⇒\left\{\begin{array}{c}y_1=z_1+z_3\\y_2=z_2-2z_3\\y_3=z_3\end{array}\right.,即\begin{pmatrix}y_1\\y_2\\y_3\end{pmatrix}=\begin{pmatrix}1&0&1\\0&1&-2\\0&0&1\end{pmatrix}\begin{pmatrix}z_1\\z_2\\z_3\end{pmatrix}\\\mathrm{代入原二次型得标准形}f=2z_1^2-2z_2^2+6z_3^2\end{array}
$$



$$
\mathrm{用正交变换法化二次曲面方程}x_1^2+2x_2^2+x_3^2-2x_1x_3=1\mathrm{的标准方程为}()
$$
$$
A.
2y_2^2+2y_3^2=1 \quad B.y_2^2+y_3^2=1 \quad C.2y_1^2+2y_2^2+2y_3^2=1 \quad D.y_1^2+y_2^2+y_3^2=1 \quad E. \quad F. \quad G. \quad H.
$$
$$
\begin{array}{l}\begin{array}{l}\mathrm{对应的矩阵}A=\begin{pmatrix}1&0&-1\\0&2&0\\-1&0&1\end{pmatrix},由\left|λ E-A\right|=0\mathrm{解出}A\mathrm{的特征值为}λ_1=0,λ_2=λ_3=2\\当λ_1=0时,由(0E-A)x=0,\mathrm{解除特征向量}ξ_1=\begin{pmatrix}1\\0\\1\end{pmatrix},\mathrm{单位化}γ_1=\frac1{\sqrt2}\begin{pmatrix}1\\0\\1\end{pmatrix};\end{array}\\当λ_2=λ_3=2时,由(2E-A)x=0,\mathrm{解除特征向量}ξ_2=\begin{pmatrix}-1\\0\\1\end{pmatrix},\mathrm{单位化}ξ_3=\begin{pmatrix}-1\\1\\1\end{pmatrix},\\将ξ_2、ξ_3\mathrm{正交化},\mathrm{单位化得}γ_2=\frac{η_2}{\left|\left|η_2\right|\right|}\frac1{\sqrt2}\begin{pmatrix}-1\\0\\1\end{pmatrix},γ_3=η_3=\begin{pmatrix}0\\1\\0\end{pmatrix},令Q=(γ_1,γ_2γ_3),则Q^TAQ=\begin{pmatrix}0&0&0\\0&2&0\\0&0&2\end{pmatrix}\\\mathrm{经过正交变换}x=Qy\mathrm{二次曲面方程化为}2y_2^2+2y_3^2=1(\mathrm{椭圆柱面})\end{array}
$$



$$
\mathrm{二次型}f(x_1,x_2,x_3)=x_1^2+x_2^2+x_3^2+2x_1x_2+2x_1x_3+2x_2x_3\mathrm{的标准型是}()
$$
$$
A.
-y_1^2 \quad B.-y_1^2-y_2^2-y_3^2 \quad C.y_1^2+y_2^2 \quad D.3y_1^2 \quad E. \quad F. \quad G. \quad H.
$$
$$
\begin{array}{l}\mathrm{二次型的矩阵为}A=\begin{pmatrix}1&1&1\\1&1&1\\1&1&1\end{pmatrix}\mathrm{由于}r(A)=1,\mathrm{故二次型的秩为}1,\mathrm{即二次型的标准形只有一项平方项的系数非零}.\mathrm{由于二次型用正交变换化}\\\mathrm{为标准形的系数为矩阵}A\mathrm{的特征值},\mathrm{故只有一个非零特征值},\mathrm{又矩阵的特征值之和等于对角元素之和},\mathrm{则非零特征值为}3,\mathrm{标准形为}3y_1^2\end{array}
$$



$$
\mathrm{二次型}f(x_1,x_2,x_3)=2x_1^2+5x_2^2+5x_3^2+4x_1x_2-4x_1x_3-8x_2x_3\mathrm{的标准形是}()
$$
$$
A.
10y_3^2 \quad B.y_1^2+y_2^2+10y_3^2 \quad C.-y_1^2-y_2^2-10y_3^2 \quad D.y_1^2-y_2^2-10y_3^2 \quad E. \quad F. \quad G. \quad H.
$$
$$
\begin{array}{l}\mathrm{二次型矩阵为}A=\begin{pmatrix}2&2&-2\\2&5&-4\\-2&-4&5\end{pmatrix}则\left|λ E-A\right|=\begin{vmatrix}λ-2&-2&2\\-2&λ-5&4\\2&4&λ-5\end{vmatrix}=\begin{vmatrix}λ-2&-2&2\\-2&λ-5&4\\0&λ-1&λ-1\end{vmatrix}=\begin{vmatrix}λ-2&-4&2\\-2&λ-9&4\\0&0&λ-1\end{vmatrix}\\=(λ-1)(λ^2-11λ+10)=(λ-1)^2(λ-10),即λ_1=λ_2=1,λ_3=10,\mathrm{故二次型可通过正交变换化为标准形}:y_1^2+y_2^2+10y_3^2\end{array}
$$



$$
\mathrm{二次型}f(x_1,x_2)=2x_1^2+2x_2^2+3x_1x_2在x_1^2+x_2^2=1\mathrm{的条件下的最大值是}()
$$
$$
A.
5 \quad B.2 \quad C.\frac12 \quad D.\frac72 \quad E. \quad F. \quad G. \quad H.
$$
$$
\begin{array}{l}\mathrm{二次型的矩阵为}A=\begin{pmatrix}2&\frac32\\\frac32&2\end{pmatrix},\mathrm{易求得矩阵}A\mathrm{的特征值为}λ_1=\frac72,\lambda_2=\frac12,\mathrm{故二次型}f\mathrm{由正交变换可化为标准形}f=\frac72y_1^2+\frac12y_2^2\\\mathrm{正交变换不改变向量的长度},即\left|\left|x\right|\right|=\left|\left|y\right|\right|=1,当x_1^2+x_2^2=1时,y_1^2+y_2^2=1,取y_1=1,y_2=0,\mathrm{则可得到}f\mathrm{的最大值为}\frac72\end{array}
$$



$$
\mathrm{二次型}f(x_1,x_2)=2x_1^2+2x_2^2+3x_1x_2在x_1^2+x_2^2=1\mathrm{的条件下的最小值是}()
$$
$$
A.
\frac72 \quad B.\frac12 \quad C.1 \quad D.0 \quad E. \quad F. \quad G. \quad H.
$$
$$
\begin{array}{l}\mathrm{二次型的矩阵为}A=\begin{pmatrix}2&\frac32\\\frac32&2\end{pmatrix},\mathrm{易求得矩阵}A\mathrm{的特征值为}λ_1=\frac72,λ_2=\frac12,\mathrm{故二次型}f\mathrm{由正交变换可化为标准形}f=\frac72y_1^2+\frac12y_2^2\\\mathrm{正交变换不改变向量的长度},即\left|\left|x\right|\right|=\left|\left|y\right|\right|=1,当x_1^2+x_2^2=1时,y_1^2+y_2^2=1,取y_1=0,y_2=-1,\mathrm{则可得到}f\mathrm{的最小值为}\frac12\end{array}
$$



$$
\mathrm{二次型}f(x_1,x_2,x_3)=x_1^2+x_2^2+x_3^2+2x_1x_2+2x_1x_3+2x_2x_3在x_1^2+x_2^2+x_3^2=1\mathrm{的条件下的最大值是}()
$$
$$
A.
3 \quad B.1 \quad C.0 \quad D.\frac43 \quad E. \quad F. \quad G. \quad H.
$$
$$
\begin{array}{l}\mathrm{二次型的矩阵为}A=\begin{pmatrix}1&1&1\\1&1&1\\1&1&1\end{pmatrix},\mathrm{易求得矩阵}A\mathrm{的特征值为}λ_1=λ_2=0,λ_3=3,\mathrm{故二次型}f\mathrm{由正交变换可化为标准形}f=3y_1^2\\\mathrm{正交变换不改变向量的长度},即\left|\left|x\right|\right|=\left|\left|y\right|\right|=1,当x_1^2+x_2^2=1时,y_1^2+y_2^2+y_3^2=1,取y_1=1,y_2=y_3=0,\mathrm{则可得到}f\mathrm{的最大值为}3\end{array}
$$



$$
\mathrm{二次型}f(x_1,x_2,x_3)=x_1^2+x_2^2+x_3^2+2x_1x_2+2x_1x_3+2x_2x_3在x_1^2+x_2^2+x_3^2=1\mathrm{的条件下的最小值是}()
$$
$$
A.
3 \quad B.\frac43 \quad C.1 \quad D.0 \quad E. \quad F. \quad G. \quad H.
$$
$$
\begin{array}{l}\mathrm{二次型的矩阵为}A=\begin{pmatrix}1&1&1\\1&1&1\\1&1&1\end{pmatrix},\mathrm{易求得矩阵}A\mathrm{的特征值为}λ_1=λ_2=0,λ_3=3,\mathrm{故二次型}f\mathrm{由正交变换可化为标准形}f=3y_1^2\\\mathrm{正交变换不改变向量的长度},即\left|\left|x\right|\right|=\left|\left|y\right|\right|=1,当x_1^2+x_2^2+x_3^2=1时,y_1^2+y_2^2+y_3^2=1,取y_1=0,y_2=0,,y_3=1,\mathrm{则可得到}f\mathrm{的最小值为}0\end{array}
$$



$$
\mathrm{二次型}f(x_1,x_2)=2x_1^2+2x_2^2-2x_1x_2在x_1^2+x_2^2=1\mathrm{的条件下的最大值等于}()
$$
$$
A.
4 \quad B.1 \quad C.2 \quad D.3 \quad E. \quad F. \quad G. \quad H.
$$
$$
\begin{array}{l}\mathrm{二次型的矩阵为}A=\begin{pmatrix}2&-1\\-1&2\end{pmatrix},\mathrm{易求得矩阵}A\mathrm{的特征值为}λ_1=1,λ_2=3,\mathrm{故二次型}f\mathrm{由正交变换可化为标准形}f=y_1^2+3y_2^2\\\mathrm{正交变换不改变向量的长度},即\left|\left|x\right|\right|=\left|\left|y\right|\right|=1,当x_1^2+x_2^2=1时,y_1^2+y_2^2=1,取y_1=0,y_2=1,\mathrm{则可得到}f\mathrm{的最大值为}3\end{array}
$$



$$
\mathrm{二次型}f(x_1,x_2)=5x_1^2+2x_2^2+4x_1x_2在x_1^2+x_2^2=1\mathrm{的条件下的最大值等于}()
$$
$$
A.
6 \quad B.1 \quad C.7 \quad D.9 \quad E. \quad F. \quad G. \quad H.
$$
$$
\begin{array}{l}\mathrm{二次型的矩阵为}A=\begin{pmatrix}5&2\\2&2\end{pmatrix},\mathrm{易求得矩阵}A\mathrm{的特征值为}λ_1=6,λ_2=1,\mathrm{故二次型}f\mathrm{由正交变换可化为标准形}f=6y_1^2+y_2^2\\\mathrm{正交变换不改变向量的长度},即\left|\left|x\right|\right|=\left|\left|y\right|\right|=1,当x_1^2+x_2^2=1时,y_1^2+y_2^2=1,取y_1=1,y_2=0,\mathrm{则可得到}f\mathrm{的最大值为}6\end{array}
$$



$$
\mathrm{二次型}f(x_1,x_2)=5x_1^2+2x_2^2+4x_1x_2在x_1^2+x_2^2=1\mathrm{的条件下的最小值等于}()
$$
$$
A.
1 \quad B.0 \quad C.5 \quad D.2 \quad E. \quad F. \quad G. \quad H.
$$
$$
\begin{array}{l}\mathrm{二次型的矩阵为}A=\begin{pmatrix}5&2\\2&2\end{pmatrix},\mathrm{易求得矩阵}A\mathrm{的特征值为}λ_1=6,λ_2=1,\mathrm{故二次型}f\mathrm{由正交变换可化为标准形}f=6y_1^2+y_2^2\\\mathrm{正交变换不改变向量的长度},即\left|\left|x\right|\right|=\left|\left|y\right|\right|=1,当x_1^2+x_2^2=1时,y_1^2+y_2^2=1,取y_1=0,y_2=1,\mathrm{则可得到}f\mathrm{的最小值为}1\end{array}
$$



$$
\mathrm{二次型}f(x_1,x_2)=2x_1x_2在x_1^2+x_2^2=1\mathrm{的条件下的最大值等于}()
$$
$$
A.
1 \quad B.2 \quad C.\frac12 \quad D.\frac{\sqrt2}2 \quad E. \quad F. \quad G. \quad H.
$$
$$
\begin{array}{l}\mathrm{二次型的矩阵为}A=\begin{pmatrix}0&1\\1&0\end{pmatrix},\mathrm{易求得矩阵}A\mathrm{的特征值为}λ_1=1,λ_2=-1,\mathrm{故二次型}f\mathrm{由正交变换可化为标准形}f=y_1^2-y_2^2\\\mathrm{正交变换不改变向量的长度},即\left|\left|x\right|\right|=\left|\left|y\right|\right|=1,当x_1^2+x_2^2=1时,y_1^2+y_2^2=1,取y_1=1,y_2=0,\mathrm{则可得到}f\mathrm{的最大值为}1\end{array}
$$



$$
\mathrm{二次型}f(x_1,x_2)=2x_1x_2在x_1^2+x_2^2=1\mathrm{的条件下的最小值等于}()
$$
$$
A.
1 \quad B.0 \quad C.\frac12 \quad D.-1 \quad E. \quad F. \quad G. \quad H.
$$
$$
\begin{array}{l}\mathrm{二次型的矩阵为}A=\begin{pmatrix}0&1\\1&0\end{pmatrix},\mathrm{易求得矩阵}A\mathrm{的特征值为}λ_1=1,λ_2=-1,\mathrm{故二次型}f\mathrm{由正交变换可化为标准形}f=y_1^2-y_2^2\\\mathrm{正交变换不改变向量的长度},即\left|\left|x\right|\right|=\left|\left|y\right|\right|=1,当x_1^2+x_2^2=1时,y_1^2+y_2^2=1,取y_1=0,y_2=-1,\mathrm{则可得到}f\mathrm{的最小值为}-1\end{array}
$$



$$
\mathrm{二次型}f(x_1,x_2)=2x_1^2+2x_2^2-2x_1x_2在x_1^2+x_2^2=1\mathrm{的条件下的最小值等于}()
$$
$$
A.
1 \quad B.2 \quad C.3 \quad D.0 \quad E. \quad F. \quad G. \quad H.
$$
$$
\begin{array}{l}\mathrm{二次型的矩阵为}A=\begin{pmatrix}2&-1\\-1&2\end{pmatrix},\mathrm{易求得矩阵}A\mathrm{的特征值为}λ_1=1,λ_2=3,\mathrm{故二次型}f\mathrm{由正交变换可化为标准形}f=y_1^2+3y_2^2\\\mathrm{正交变换不改变向量的长度},即\left|\left|x\right|\right|=\left|\left|y\right|\right|=1,当x_1^2+x_2^2=1时,y_1^2+y_2^2=1,取y_1=1,y_2=0,\mathrm{则可得到}f\mathrm{的最小值为}1\end{array}
$$



$$
\mathrm{已知二次型}f(x_1,x_2,x_3)=2x_1^2+3x_2^2+3x_3^2+2ax_2x_3(a>0)\mathrm{通过正交变换化成标准形}f(y_1,y_2,y_3)=y_1^2+2y_2^2+5y_3^2,则a=(\;)
$$
$$
A.
1 \quad B.4 \quad C.-2 \quad D.2 \quad E. \quad F. \quad G. \quad H.
$$
$$
\begin{array}{l}\mathrm{二次型}f\mathrm{对应的实对称阵}A=\begin{pmatrix}2&0&0\\0&3&a\\0&a&3\end{pmatrix}\mathrm{由标准形可知的}A\mathrm{特征值为}λ_1=1,λ_2=2,λ_3=5⇒\left|A\right|=λ_1λ_2λ_3=1·2·5=10\\故\left|A\right|=2(9-a^2)=10⇒ a=2(舍-2)\end{array}
$$



$$
\mathrm{二次型}f(x_1,x_2,x_3)=2x_1^2+x_2^2+x_3^2+2x_1x_2+2x_1x_3在x_1^2+x_2^2+x_3^2=1\mathrm{的条件下的最大值是}()
$$
$$
A.
4 \quad B.6 \quad C.2 \quad D.3 \quad E. \quad F. \quad G. \quad H.
$$
$$
\begin{array}{l}\begin{array}{l}\mathrm{二次型的矩阵为}A=\begin{pmatrix}2&1&1\\1&1&0\\1&0&1\end{pmatrix},\mathrm{易求得矩阵}A\mathrm{的特征值为}λ_1=3,,λ_2=1,λ_3=0,\mathrm{故二次型}f\mathrm{由正交变换}x=Py\mathrm{可化为标准形}f=3y_1^2+y_2^2\\\mathrm{由于正交变换不改变向量的长度},即\left|\left|x\right|\right|=\left|\left|Py\right|\right|=\left|\left|y\right|\right|=1,当x_1^2+x_2^2+x_3^2=1时,y_1^2+y_2^2+y_3^2=1,取y_1=1,y_2=y_3=0,\mathrm{则可得到}f\mathrm{的最大值为}3\end{array}\\\\\end{array}
$$



$$
\mathrm{二次型}f(x_1,x_2,x_3)=2x_1^2+x_2^2+x_3^2+2x_1x_2+2x_1x_3在x_1^2+x_2^2+x_3^2=1\mathrm{的条件下的最小值是}()
$$
$$
A.
0 \quad B.1 \quad C.2 \quad D.3 \quad E. \quad F. \quad G. \quad H.
$$
$$
\begin{array}{l}\begin{array}{l}\mathrm{二次型的矩阵为}A=\begin{pmatrix}2&1&1\\1&1&0\\1&0&1\end{pmatrix},\mathrm{易求得矩阵}A\mathrm{的特征值为}λ_1=3,,λ_2=1,λ_3=0,\mathrm{故二次型}f\mathrm{由正交变换}x=Py\mathrm{可化为标准形}f=3y_1^2+y_2^2\\\mathrm{由于正交变换不改变向量的长度},即\left|\left|x\right|\right|=\left|\left|Py\right|\right|=\left|\left|y\right|\right|=1,当x_1^2+x_2^2+x_3^2=1时,y_1^2+y_2^2+y_3^2=1,取y_1=y_2=0,y_3=1,\mathrm{则可得到}f\mathrm{的最小值为}0\end{array}\end{array}
$$



$$
\begin{array}{l}\mathrm{用正交变换}X=PY\mathrm{化二次型}f(x_1,x_2,x_3)=ax_1^2+ax_2^2+6x_3^2+8x_1x_2+4x_2x_3-4x_1x_3\mathrm{为标准型}\\f=7y_1^2+7y_2^2-2y_3^3,\mathrm{则参数}a为()\end{array}
$$
$$
A.
-1 \quad B.1 \quad C.-3 \quad D.3 \quad E. \quad F. \quad G. \quad H.
$$
$$
\mathrm{二次型矩阵}A=\begin{pmatrix}a&4&-2\\4&a&2\\-2&2&6\end{pmatrix},A\mathrm{的特征值为}7,7,-2.由\left|A\right|=7·7·(-2)或A\mathrm{的迹即可得}a=3
$$



$$
\mathrm{二次型}f(x_1,x_2,x_3)=x_1^2+x_2^2+x_3^2+2x_1x_2+2x_2x_3+2x_1x_3\mathrm{的标准型是}()
$$
$$
A.
-y_1^2 \quad B.-y_1^2-y_2^2-y_3^2 \quad C.y_1^2+y_2^2 \quad D.3y_1^2 \quad E. \quad F. \quad G. \quad H.
$$
$$
\begin{array}{l}\mathrm{二次型的矩阵为}A=\begin{pmatrix}1&1&1\\1&1&1\\1&1&1\end{pmatrix},\mathrm{由于}r(A)=1\mathrm{故二次型的秩为}1,\mathrm{即二次型的标准形只有一项平方项的系数非零}.\mathrm{由于二次型用}\\\mathrm{正交变换化为标准形的系数为矩阵}A\mathrm{的特征值},故A\mathrm{只有一个非零特征值},\mathrm{又矩阵的特征值之和等于对角元素之和},\\\mathrm{则非零特征值为}3,\mathrm{标准形为}3y_1^2\end{array}
$$



$$
\mathrm{用正交变换法化二次型}f(x_1,x_2,x_3)=17x_1^2+14x_2^2+14x_3^2-4x_1x_2-4x_1x_3-8x_2x_3\mathrm{的标准型是}()
$$
$$
A.
9y_1^2+18y_2^2+9y_3^2 \quad B.9y_1^2+18y_2^2+18y_3^2 \quad C.y_1^2+y_2^2+18y_3^2 \quad D.18y_1^2-9y_2^2+18y_3^2 \quad E. \quad F. \quad G. \quad H.
$$
$$
\begin{array}{l}\mathrm{二次型矩阵}A=\begin{pmatrix}17&-2&-2\\-2&14&-4\\-2&-4&14\end{pmatrix},由\left|λ E-A\right|=\begin{vmatrix}λ-17&2&2\\2&λ-14&4\\2&4&λ-14\end{vmatrix}=(λ-18)^2(λ-9)=0⇒λ_1=9,λ_2=λ_3=18\\\mathrm{故原二次型可通过正交变换化为标准形}f=9y_1^2+18y_2^2+18y_3^2\end{array}
$$



$$
设α=\begin{pmatrix}1\\2\end{pmatrix},A为2\mathrm{阶正交矩阵},β=Aα,\mathrm{则向量}β\mathrm{的长度为}()
$$
$$
A.
\sqrt5 \quad B.\sqrt{10} \quad C.2\sqrt5 \quad D.10 \quad E. \quad F. \quad G. \quad H.
$$
$$
A\mathrm{为正交矩阵},故β=Aα\mathrm{为正交变换},\mathrm{由于正交变换保持向量的长度不变},故\left|\left|β\right|\right|=\left|\left|α\right|\right|=\sqrt5
$$



$$
设α=\begin{pmatrix}-1&0&2\end{pmatrix}^T,\mathrm{矩阵}A=\begin{pmatrix}\frac45&0&-\frac35\\\frac35&0&\frac45\\0&1&0\end{pmatrix},β=Aα,\mathrm{则向量}β\mathrm{的长度为}()
$$
$$
A.
\sqrt5 \quad B.\sqrt3 \quad C.1 \quad D.\sqrt2 \quad E. \quad F. \quad G. \quad H.
$$
$$
A\mathrm{为正交矩阵},故β=Aα\mathrm{为正交变换},\mathrm{由于正交变换保持向量的长度不变},故\left|\left|β\right|\right|=\left|\left|α\right|\right|=\sqrt5
$$



$$
\mathrm{二次型}x^2+ay^2+z^2+2bxy+2xz+2yz\mathrm{可经过正交变换化为标准形}η^2+4ζ^2,则a,b\mathrm{的值分别为}()
$$
$$
A.
a=3,b=1 \quad B.a=0,b=1 \quad C.a=1,b=4 \quad D.a=3,b=0 \quad E. \quad F. \quad G. \quad H.
$$
$$
\begin{array}{l}由A=\begin{pmatrix}1&b&1\\b&a&1\\1&1&1\end{pmatrix}与\begin{pmatrix}0&&\\&1&\\&&4\end{pmatrix}\mathrm{相似得},A\mathrm{的特征值是}0,1,4,\mathrm{那么}\\\left\{\begin{array}{l}1+a+1=0+1+4\\\left|A\right|=\begin{vmatrix}1&b&1\\b&a&1\\1&1&1\end{vmatrix}=\begin{vmatrix}0&b&1\\b-1&a&1\\0&1&1\end{vmatrix}=-(b-1)^2=0\end{array}\right.,\mathrm{所以}a=3,b=1.\end{array}
$$



$$
A\mathrm{为三阶实对称矩阵},\mathrm{且满足}A^3-A^2-A=2E,\mathrm{二次型}x^TAx\mathrm{正交变换可化为标准形},\mathrm{则此标准形的表达式为}()
$$
$$
A.
2y_1^2+2y_2^2+2y_3^2 \quad B.2y_1^2+2y_2^2+y_3^2 \quad C.y_1^2+y_2^2+y_3^2 \quad D.2y_1^2+2y_2^2 \quad E. \quad F. \quad G. \quad H.
$$
$$
\begin{array}{l}由A\mathrm{为三阶实对称矩阵},\mathrm{设经正交变换}x=Qy\mathrm{化为标准型}Q^TAQ=\begin{pmatrix}λ_1&0&0\\0&λ_2&0\\0&0&λ_3\end{pmatrix},\mathrm{其中}Q\mathrm{为正交矩阵},\mathrm{所以}\\A^K=Q\begin{pmatrix}λ_1&0&0\\0&λ_2&0\\0&0&λ_3\end{pmatrix}^KQ^T=Q\begin{pmatrix}λ_1^K&0&0\\0&λ_2^K&0\\0&0&λ_3^K\end{pmatrix}Q^T,由A^3-A^2-A=2E\mathrm{代入},得λ_i^3-λ_i^2-λ_i=2,\mathrm{解得}λ_i=2(i=1,2,3)\\\mathrm{故此标准形的表达式为}2y_1^2+2y_2^2+2y_3^2\end{array}
$$



$$
\mathrm{已知二次型}f(x_1,x_2,x_3)=5x_1^2+5x_2^2+cx_3^2-2x_1x_2+6x_1x_3-6x_2x_3\mathrm{的秩为}2,\mathrm{则方程}f(x_1,x_2,x_3)=1\mathrm{的二次曲面为}()
$$
$$
A.
\mathrm{椭圆柱面} \quad B.\mathrm{圆柱面} \quad C.\mathrm{双曲柱面} \quad D.\mathrm{抛物柱面} \quad E. \quad F. \quad G. \quad H.
$$
$$
\begin{array}{l}\mathrm{二次型矩阵为}A=\begin{pmatrix}5&-1&3\\-1&5&-3\\3&-3&c\end{pmatrix},由r(A)=2\mathrm{可知}\left|A\right|=0⇒ c=3;由\left|A-λ E\right|=0\mathrm{可求得矩阵}A\mathrm{的特征值为}λ_1=0,λ_2=4,λ_3=9\\故f\mathrm{可通过正交变换}X=PY\mathrm{化为}f=4y_2^2+9y_3^2,\mathrm{即方程}4y_2^2+9y_3^2=1,\mathrm{故方程表示的图形为椭圆柱面}.\end{array}
$$



$$
\mathrm{已知二次型}f(x_1,x_2,x_3)=-x_1^2+3x_2^2+3x_3^2+ax_2x_3(a>1)\mathrm{通过正交变换化成标准形}ky_1^2+2y_2^2+4y_3^2,则k及a\mathrm{的值分别为}()
$$
$$
A.
k=1,a=2 \quad B.k=-1,a=2 \quad C.k=1,a=-2 \quad D.k=-1,a=-2 \quad E. \quad F. \quad G. \quad H.
$$
$$
\begin{array}{l}\mathrm{依题意},\mathrm{通过正交变换化成的标准形的系数即为}f\mathrm{的矩阵}A\mathrm{的特征值},\mathrm{二次型的矩阵为}A=\begin{pmatrix}-1&0&0\\0&3&\frac a2\\0&\frac a2&3\end{pmatrix}A\mathrm{的特征值为}k,2,4,\\即A\mathrm{相似与对角矩阵}\begin{pmatrix}k&&\\&2&\\&&4\end{pmatrix}=B,\mathrm{由相似矩阵的性质},\mathrm{知有}-1+3+3=k+2+4,\mathrm{解得}k=-1,又\left|A\right|=\left|B\right|,\mathrm{解得}a=2\end{array}
$$



$$
\begin{array}{l}设α_1、α_2为n\mathrm{维列向量},A为n\mathrm{阶正交矩阵},\mathrm{下列两个命题正确的是}()\\(1)<\;Aα_1,Aα_2>=-<α_1,α_2>\;\;\;\;\;\;(2)\left|\left|Aα_1\right|\right|=\left|\left|α_1\right|\right|\end{array}
$$
$$
A.
(1) \quad B.(2) \quad C.(1)\;(2) \quad D.\mathrm{都不正确} \quad E. \quad F. \quad G. \quad H.
$$
$$
\begin{array}{l}(1)<\;Aα_1,Aα_2>=(Aα_1)^TAα_2=α_1^T(A^TA)α_2=α_1^TE_nα_2=α_1^Tα_2=<α_1,α_2>\;\\(2)由(1),\left|\left|Aα_1\right|\right|^2=<\;Aα_1,Aα_1>=<\;α_1,α_1>=\left|\left|α_1\right|\right|^2⇒\left|\left|Aα_1\right|\right|=\left|\left|α_1\right|\right|\end{array}
$$



$$
\mathrm{设矩阵}P=\begin{pmatrix}\cosθ&-\sinθ&0\\\sinθ&\cosθ&0\\0&0&1\end{pmatrix},α=\begin{pmatrix}-\frac13\\\frac23\\\frac23\end{pmatrix},β=Pα,\mathrm{则向量}β\mathrm{的长度等于}()
$$
$$
A.
\frac13 \quad B.3 \quad C.1 \quad D.0 \quad E. \quad F. \quad G. \quad H.
$$
$$
β=Pα=\begin{pmatrix}\frac13\cosθ-\frac23\sinθ\\-\frac13\sinθ+\frac23\cosθ\\\frac23\end{pmatrix},则\left|\left|β^2\right|\right|=\lbrackβ,β\rbrack=\frac19\lbrack(\cosθ+2\sinθ)^2+(2\cosθ-\sinθ)^2\rbrack+\frac49=\frac19(5\cos^2θ+5\sin^2θ)+\frac49=1
$$



$$
\mathrm{设矩阵}P=\begin{pmatrix}\frac1{\sqrt2}&-\frac1{\sqrt2}&0\\\frac1{\sqrt2}&\frac1{\sqrt2}&0\\0&0&1\end{pmatrix},α=\begin{pmatrix}1\\2\\3\end{pmatrix},β=Pα,\mathrm{则向量}β\mathrm{的长度等于}()
$$
$$
A.
3 \quad B.\sqrt2 \quad C.14 \quad D.\sqrt{14} \quad E. \quad F. \quad G. \quad H.
$$
$$
β=Pα=\begin{pmatrix}-\frac1{\sqrt2}\\\frac3{\sqrt2}\\3\end{pmatrix},则\left|\left|β^2\right|\right|=\lbrackβ,β\rbrack=\frac12+\frac92+9=14,故\left|\left|β\right|\right|=\sqrt{14}
$$



$$
\mathrm{已知}(1,-1,0)^T\mathrm{是二次型}x^TAx=ax_1^2+x_3^2-2x_1x_2+2x_1x_3+2bx_2x_3\mathrm{的矩阵}A\mathrm{的特征向量},\mathrm{则二次型的标准形为}()
$$
$$
A.
y_1^2+\sqrt3y_2^2-\sqrt3y_3^2 \quad B.\sqrt3y_1^2+y_2^2-y_3^2 \quad C.y_1^2+y_2^2-3y_3^2 \quad D.y_1^2+3y_2^2-y_3^2 \quad E. \quad F. \quad G. \quad H.
$$
$$
\begin{array}{l}\mathrm{由于}A=\begin{pmatrix}a&-1&1\\-1&0&b\\1&b&1\end{pmatrix}设(1,-1,0)^T是λ\mathrm{所对应的特征向量},\mathrm{那么}\begin{pmatrix}a&-1&1\\-1&0&b\\1&b&1\end{pmatrix}\begin{pmatrix}1\\-1\\0\end{pmatrix}=λ\begin{pmatrix}1\\-1\\0\end{pmatrix},\\\mathrm{于是}a+1=λ,-1=-λ,1-b=0,\mathrm{从而}A=\begin{pmatrix}0&-1&1\\-1&0&1\\1&1&1\end{pmatrix},\mathrm{由于}\left|λ E-A\right|=\begin{vmatrix}λ&1&-1\\1&λ&-1\\-1&-1&λ-1\end{vmatrix}=\begin{vmatrix}λ-1&1-λ&0\\1&λ&-1\\-1&-1&λ-1\end{vmatrix}=(λ-1)(λ^2-3)\\\mathrm{得到}A\mathrm{的特征值是}1,\sqrt3,-\sqrt3,\mathrm{则二次型可以通过正交变换化为}:x^TAx=y_1^2+\sqrt3y_2^2-\sqrt3y_3^2\end{array}
$$



$$
\mathrm{设二次型}f=x_1^2+x_2^2+x_3^2+2ax_1x_2+2bx_2x_3+2x_1x_3\mathrm{经正交变换}x=Qy\mathrm{化成}f=y_2^2+2y_3^2,\mathrm{则常数}a,b\mathrm{分别为}()
$$
$$
A.
a=b=0 \quad B.a=b=1 \quad C.a=0,b=1 \quad D.a=1,b=0 \quad E. \quad F. \quad G. \quad H.
$$
$$
\begin{array}{l}\mathrm{设变换前后二次型的矩阵分别为}A=\begin{pmatrix}1&a&1\\a&1&b\\1&b&1\end{pmatrix},B=\begin{pmatrix}0&0&0\\0&1&0\\0&0&2\end{pmatrix},则A与B\mathrm{是相似矩阵},故\left|λ E-A\right|=\left|λ E-B\right|,即\\\begin{vmatrix}λ-1&-a&-1\\-a&λ-1&-b\\-1&-b&λ-1\end{vmatrix}=\begin{vmatrix}λ&0&0\\0&λ-1&0\\0&0&λ-2\end{vmatrix},λ^3-3λ^2+(2-a^2-b^2)λ+(a-b)^2=λ^3-3λ^2+2λ,\mathrm{必有}a=b=0,\mathrm{即为所求的常数}.\end{array}
$$



$$
\begin{array}{l}\mathrm{已知二次型}f(x_1,x_2,x_3)=x^TAx\mathrm{在正交变换}x=Qy\mathrm{下的标准型为}y_1^2+y_2^2,则Q\mathrm{的第三列为}(\frac{\sqrt2}2,0,\frac{\sqrt2}2)^T,\mathrm{则矩阵}A=()\\且A+E为()\mathrm{矩阵}(\mathrm{其中}E为3\mathrm{阶单位矩阵}).\end{array}
$$
$$
A.
A=\begin{pmatrix}\frac12&0&-\frac12\\0&1&0\\-\frac12&0&\frac12\end{pmatrix},\mathrm{负定} \quad B.A=\begin{pmatrix}\frac12&0&\frac12\\0&1&0\\\frac12&0&\frac12\end{pmatrix},\mathrm{负定} \quad C.A=\begin{pmatrix}\frac12&0&-\frac12\\0&1&0\\-\frac12&0&\frac12\end{pmatrix},\mathrm{正定} \quad D.A=\begin{pmatrix}\frac12&0&\frac12\\0&1&0\\\frac12&0&\frac12\end{pmatrix},\mathrm{正定} \quad E. \quad F. \quad G. \quad H.
$$
$$
\begin{array}{l}\begin{array}{l}\mathrm{因为二次型}f(x_1,x_2,x_3)=x^TAx\mathrm{在正交变换}x=Qy\mathrm{下的标准型为}y_1^2+y_2^2,\mathrm{所以}A\mathrm{的特征值为}λ_1=λ_2=1,λ_3=0,\\Q\mathrm{的第三列为}(\frac{\sqrt2}2,0,\frac{\sqrt2}2)^T,\mathrm{所以}λ_3=0\mathrm{对应的线性无关的特征向量为}ξ_3=\begin{pmatrix}1\\0\\1\end{pmatrix}\mathrm{因为}A\mathrm{为实对称矩阵},\\\mathrm{所以}A\mathrm{的不同特征值对应的特征向量正交},令λ_1=λ_2=1\mathrm{对应的特征向量为}ξ=\begin{pmatrix}x_1\\x_2\\x_3\end{pmatrix},由x_1+x_3=0的λ_1=λ_2=1\mathrm{对应的线性无关的特征向量为}\\ξ_1=\begin{pmatrix}0\\1\\0\end{pmatrix},ξ_2=\begin{pmatrix}-1\\0\\1\end{pmatrix},令γ_1=\begin{pmatrix}0\\1\\0\end{pmatrix},γ_2=\frac1{\sqrt2}\begin{pmatrix}-1\\0\\1\end{pmatrix},γ_3=\frac1{\sqrt2}\begin{pmatrix}1\\0\\1\end{pmatrix},则Q=(γ_1,γ_2,γ_3)\end{array}\\由Q^TAQ=\begin{pmatrix}1&&\\&1&\\&&0\end{pmatrix},得A=\begin{pmatrix}\frac12&0&-\frac12\\0&1&0\\-\frac12&0&\frac12\end{pmatrix}\\(2)\mathrm{因为}A+E=\begin{pmatrix}\frac32&0&-\frac12\\0&2&0\\-\frac12&0&\frac32\end{pmatrix}\mathrm{是实对称矩阵},且A\mathrm{的特征值为}λ_1=λ_2=1,λ_3=0,\mathrm{所以}A+E\mathrm{的特征值为}λ_1=λ_2=2,λ_3=1\\\mathrm{因为其特征值都大于零},\mathrm{所以}A+E\mathrm{为正定矩阵}.\end{array}
$$



$$
\mathrm{已知二次型}f(x_1,x_2,x_3)=a(x_1^2+x_2^2+x_3^2)+4x_1x_2+4x_1x_3+4x_2x_3\mathrm{经过正交变换}x=Py\mathrm{化为标准形}f=6y_1^2,\mathrm 则a\mathrm{的值为}()
$$
$$
A.
a=3 \quad B.a=0 \quad C.a=1 \quad D.a=2 \quad E. \quad F. \quad G. \quad H.
$$
$$
\begin{array}{l}\begin{array}{l}令A=\begin{pmatrix}a&2&2\\2&a&2\\2&2&a\end{pmatrix},令Λ=\begin{pmatrix}6&&\\&0&\\&&0\end{pmatrix},\mathrm{由题设},P^TAP=Λ.\mathrm 又P\mathrm{是正交矩阵},\mathrm 故P^{-1}AP=Λ,\mathrm 即A\mathrm 与Λ\mathrm{相似}\;,\\\mathrm{有相同的特征值},\mathrm{所以解得}a=2.\end{array}\\\end{array}
$$



$$
\mathrm{若二次曲面的方程为}x^2+3y^2+z^2+2axy+2xz+2yz=4,\mathrm{经过正交变换化为}y_1^2+4z_1^2=4,\mathrm 则a\mathrm{的值为}()
$$
$$
A.
a=3 \quad B.a=-1 \quad C.a=1 \quad D.a=2 \quad E. \quad F. \quad G. \quad H.
$$
$$
\begin{array}{l}\begin{array}{l}\begin{array}{l}\mathrm{本题等价于将二次型}f(x,y,z)=x^2+3y^2+z^2+2axy+2xz+2yz\mathrm{经过正交变换化为}y_1^2+4z_1^2.\mathrm{由正交变换的特点可知},\\\mathrm{该二次型对应矩阵的特征值为}1,4,0.\end{array}\\\mathrm{该二次型的矩阵为}A=\begin{pmatrix}1&a&1\\a&3&1\\1&1&1\end{pmatrix},\mathrm{可知}\vert A\vert=-(a-1)^2=0,\mathrm{因此}a=1.\end{array}\\\end{array}
$$



$$
\mathrm{二次型}\;f(x_1,x_2,x_3)=(λ-1)x_1^2+λ x_2^2+(λ+1)x_3^2,\mathrm{当满足}()时,\mathrm{是正定二次型}.
$$
$$
A.
λ>-1 \quad B.λ>0 \quad C.λ>1 \quad D.λ\geq1 \quad E. \quad F. \quad G. \quad H.
$$
$$
\begin{array}{l}\begin{array}{l}\mathrm{二次型矩阵为}A=\begin{bmatrix}λ-1&&\\&λ&\\&&λ+1\end{bmatrix},\mathrm{用顺序主子式判别法有},\lambda-1>0,\begin{vmatrix}λ-1&\\&λ\end{vmatrix}=(λ-1)λ>0,\begin{vmatrix}λ-1&&\\&λ&\\&&λ+1\end{vmatrix}=(λ-1)\lambda(λ+1)>0\end{array}\\故\left\{\begin{array}{c}λ>1\\λ>0\\λ>-1\end{array}\right.⇒λ>1\end{array}
$$



$$
\mathrm{实二次型}\;kx_1^2+kx_2^2+\;kx_3^2+2kx_1x_2+2x_1x_3\mathrm{正定的}k()
$$
$$
A.
<0 \quad B.>0 \quad C.\mathrm{不存在} \quad D.\mathrm{为任意实数} \quad E. \quad F. \quad G. \quad H.
$$
$$
\mathrm{二次型的矩阵为}\begin{pmatrix}k&k&1\\k&k&0\\1&0&k\end{pmatrix},\mathrm{矩阵正定即各阶顺序主子式大于零}:k>0,\begin{vmatrix}k&k\\k&k\end{vmatrix}=0>0\mathrm{矛盾},故k\mathrm{值不存在}.
$$



$$
\mathrm{关于二次型}\;f(x,y,z)=x^2+100y^2-z^2+2xy-xz+yz\mathrm{的正定性的正确判断是}()
$$
$$
A.
\mathrm{正定的} \quad B.\mathrm{负定的} \quad C.\mathrm{半正定的} \quad D.\mathrm{不定的} \quad E. \quad F. \quad G. \quad H.
$$
$$
\begin{array}{l}\mathrm{二次型的矩阵为}\begin{pmatrix}1&1&-\frac12\\1&100&\frac12\\-\frac12&\frac12&-1\end{pmatrix},\mathrm{由于}1>0,\begin{vmatrix}1&1\\1&100\end{vmatrix}=99>0,\begin{vmatrix}1&1&-\frac12\\1&100&\frac12\\-\frac12&\frac12&-1\end{vmatrix}=99×(-\frac54)-1<0,\\\mathrm{二次型正定的充要条件是矩阵的顺子主子式全大于零},\mathrm{半正定的充要条件是所有主子式大于或等于零},\mathrm{负定的充要条件是}(-1)^k\left|A_k\right|>0(k=1,2,3)\\\mathrm{而由上可知都不满足},\mathrm{故二次型是不定二次型}.\end{array}
$$



$$
\mathrm{设二次型}\;f(x_1,x_2,x_3)=x_1^2-x_2^2+x_3^2+4x_1x_2+4x_1x_3+4x_2x_3,则()
$$
$$
A.
f\mathrm{为正定的} \quad B.f\mathrm{为负定的} \quad C.f\mathrm{既不正定},\mathrm{也不负定} \quad D.f\mathrm{的秩为}1 \quad E. \quad F. \quad G. \quad H.
$$
$$
\begin{array}{l}\mathrm{二次型的矩阵为}A=\begin{pmatrix}1&2&2\\2&-1&2\\2&2&1\end{pmatrix},\mathrm{计算矩阵}A\mathrm{的所有顺序主子式得},\left|A_1\right|=1>0,\left|A_2\right|=\begin{vmatrix}1&2\\2&-1\end{vmatrix}=-5<0,\left|A_3\right|=\left|A\right|=11>0\\\mathrm{根据此结果可知二次型不正定},\mathrm{也不负定},\mathrm{由于}\left|A\right|=11\neq0,故r(A)=3,即f\mathrm{既不正定},\mathrm{也不负定}.\end{array}
$$



$$
\mathrm{设二次型}\;f(x_1,x_2,x_3)=3x_1^2+2x_2^2+3x_3^2-2x_1x_2-2x_2x_3,则()
$$
$$
A.
f\mathrm{为正定的} \quad B.f\mathrm{为负定的} \quad C.f\mathrm{既不正定},\mathrm{也不负定} \quad D.f\mathrm{的秩为}2 \quad E. \quad F. \quad G. \quad H.
$$
$$
\mathrm{二次型的矩阵为}A=\begin{pmatrix}3&-1&0\\-1&2&-1\\0&-1&3\end{pmatrix},则\left|A_1\right|=3>0,\left|A_2\right|=\begin{vmatrix}3&-1\\-1&2\end{vmatrix}=5>0,\left|A_3\right|=\left|A\right|=12>0,故f\mathrm{是正定二次型}.
$$



$$
\mathrm{设二次型}\;f(x_1,x_2,x_3)=2x_1^2+3x_2^2+3x_3^2+4x_2x_3,则()
$$
$$
A.
f\mathrm{为正定的} \quad B.f\mathrm{为负定的} \quad C.f\mathrm{的秩为}1 \quad D.f\mathrm{的秩为}2 \quad E. \quad F. \quad G. \quad H.
$$
$$
\mathrm{二次型矩阵为}A=\begin{pmatrix}2&0&0\\0&3&2\\0&2&3\end{pmatrix},则\left|A_1\right|=2>0,\left|A_2\right|=\begin{vmatrix}2&0\\0&3\end{vmatrix}=6>0,\left|A_3\right|=\left|A\right|=10>0,故f\mathrm{为正定二次型},\mathrm{秩为}3.
$$



$$
\mathrm{设二次型}\;f(x_1,x_2,x_3)=x_1^2+2x_2^2+2x_1x_2+4x_1x_3+6x_2x_3+λ x_3^{2\;}\;\mathrm{正定},则()
$$
$$
A.
λ>5 \quad B.λ<5 \quad C.λ\geq5 \quad D.λ\leq5 \quad E. \quad F. \quad G. \quad H.
$$
$$
\mathrm{二次型的矩阵}A=\begin{pmatrix}1&1&2\\1&2&3\\2&3&λ\end{pmatrix},\mathrm{因为}\left|A_1\right|=1>0,\left|A_2\right|=\begin{vmatrix}1&1\\1&2\end{vmatrix}=1>0,\left|A_3\right|=\left|A\right|=λ-5>0,\mathrm{所以}λ>5时,f(x_1,x_2,x_3)\mathrm{为正定的}.
$$



$$
\mathrm{实二次型}\;kx_1^2+kx_2^2+\;kx_3^2+2kx_1x_2+2x_1x_3\mathrm{负定的}k()
$$
$$
A.
k<0 \quad B.k>0 \quad C.k\mathrm{不存在} \quad D.k\mathrm{为任意实数} \quad E. \quad F. \quad G. \quad H.
$$
$$
\begin{array}{l}\mathrm{二次型的矩阵为}\begin{pmatrix}k&k&1\\k&k&0\\1&0&k\end{pmatrix},\mathrm{矩阵负定即各阶顺序主子式}(-1)^k\left|A_k\right|>0,k=1,2,⋯3:\\k<0,\begin{vmatrix}k&k\\k&k\end{vmatrix}=0>0\mathrm{矛盾},故k\mathrm{值不存在}.\end{array}
$$



$$
\mathrm{设二次型}f(x_1,x_2,x_3)=3x_1^2+3x_3^2-2x_1x_2-2x_2x_3,则()
$$
$$
A.
f\mathrm{为正定的} \quad B.f\mathrm{为负定的} \quad C.f\mathrm{既不正定},\mathrm{也不负定} \quad D.f\mathrm{的秩为}2 \quad E. \quad F. \quad G. \quad H.
$$
$$
\begin{array}{l}\mathrm{二次型的矩阵为}A=\begin{pmatrix}3&-1&0\\-1&0&-1\\0&-1&3\end{pmatrix},则\left|A_1\right|=3>0,\left|A_2\right|=\begin{vmatrix}3&-1\\-1&0\end{vmatrix}=-1<0,\left|A_3\right|=\left|A\right|=-6<0\\故f\mathrm{是不定二次型},则f\mathrm{的秩为}3.\end{array}
$$



$$
\mathrm{已知}A=\begin{pmatrix}2-a&1&0\\1&1&0\\0&0&a+3\end{pmatrix}\mathrm{是正定矩阵},则a\mathrm{的值为}()
$$
$$
A.
-3<\;a<1 \quad B.-3<\;a<-1 \quad C.1<\;a<3 \quad D.-1<\;a<3 \quad E. \quad F. \quad G. \quad H.
$$
$$
\mathrm{因为}\left|A\right|>0,\mathrm{所以}\begin{vmatrix}2-a&1&0\\1&1&0\\0&0&a+3\end{vmatrix}=(a+3)(1-a)>0,得-3<\;a<1
$$



$$
\mathrm{三元二次型}f=x_1^2+5x_2^2+x_3^2+4x_1x_2-4x_2x_3\mathrm{的正定性为}()
$$
$$
A.
\mathrm{正定的} \quad B.\mathrm{负定的} \quad C.\mathrm{半正定的} \quad D.\mathrm{不定的} \quad E. \quad F. \quad G. \quad H.
$$
$$
\mathrm{由于二次型矩阵}A=\begin{pmatrix}1&2&0\\2&5&-2\\0&-2&1\end{pmatrix},\mathrm{其顺序主子式}\bigtriangleup_1=1,\bigtriangleup_2=\begin{vmatrix}1&2\\2&5\end{vmatrix}=1,\bigtriangleup_3=\left|A\right|=-3<0,\mathrm{所以}f\mathrm{是不定二次型}
$$



$$
当\left|t\right|<(\;),\mathrm{实二次型}6x_1^2+2tx_1x_2+2x_2^2+x_3^2\mathrm{是正定二次型}.
$$
$$
A.
2\sqrt3 \quad B.3\sqrt2 \quad C.2 \quad D.3 \quad E. \quad F. \quad G. \quad H.
$$
$$
\begin{array}{l}\mathrm{根据条件可知},\mathrm{二次型的矩阵为}A=\begin{pmatrix}6&t&0\\t&2&0\\0&0&1\end{pmatrix},\mathrm{二次型正定的充要条件是矩阵}A\mathrm{的顺序主子式都为零},即\\6>0,\begin{vmatrix}6&t\\t&2\end{vmatrix}=12-t^2>0,\left|A\right|=12-t^2>0,故t^2<12⇒ t<2\sqrt3\end{array}
$$



$$
\mathrm{已知二次型的矩阵}A=\begin{pmatrix}1&1&0\\1&a&0\\0&0&a^2\end{pmatrix}\mathrm{正定},则a\mathrm{的值为}()
$$
$$
A.
a>1 \quad B.a\geq1 \quad C.0<\;a<1 \quad D.a<1 \quad E. \quad F. \quad G. \quad H.
$$
$$
\mathrm{矩阵的顺序主子式}\begin{vmatrix}1&1\\1&a\end{vmatrix}>0⇒ a>1,\begin{vmatrix}1&1&0\\1&a&0\\0&0&a^2\end{vmatrix}>0⇒ a^2(a-1)>0,\mathrm{所以}a>1时,\mathrm{二次型正定}.
$$



$$
\mathrm{设矩阵}A=\begin{pmatrix}-2&0&0\\0&t&-1\\0&1&t^2\end{pmatrix}\mathrm{负定时},t\mathrm{应满足的条件是}(\;).\;
$$
$$
A.
-1<\;t<0\; \quad B.-1<\;t<1 \quad C.0<\;t<1 \quad D.t\geq1 \quad E. \quad F. \quad G. \quad H.
$$
$$
\begin{array}{l}A\mathrm{负定的充要条件是}(-1)^k\left|A_k\right|>0(k=1,2,⋯,n),\mathrm{其中}A_k是A的k\mathrm{阶顺序主子式},则\\-2t>0,2(t^3+1)>0⇒-1<\;t<0\end{array}
$$



$$
\mathrm{二次型}f=λ(x_1^2+x_2^2+x_3^2)+2x_1x_2+2x_1x_3-2x_2x_3+x_4^2\mathrm{为正定的},则λ\mathrm{的值为}()
$$
$$
A.
λ>2 \quad B.λ>1或λ<-1 \quad C.1<λ<2 \quad D.-1<λ<2 \quad E. \quad F. \quad G. \quad H.
$$
$$
\begin{array}{l}\mathrm{二次型的矩阵为}A=\begin{pmatrix}λ&1&1&0\\1&λ&-1&0\\1&-1&λ&0\\0&0&0&1\end{pmatrix},\mathrm{二次型正定的充要条件是}:\\λ>0,\begin{vmatrix}λ&1\\1&λ\end{vmatrix}=λ^2-1>0,\begin{vmatrix}λ&1&1\\1&λ&-1\\1&-1&λ\end{vmatrix}=(λ+1)^2(λ-2)>0,故\left\{\begin{array}{l}λ^2>1\\λ-2>0\end{array}\right.⇒λ>2\end{array}
$$



$$
\mathrm{二次型}x_1^2+x_2^2+5x_3^2+2ax_1x_2-2x_1x_3+4x_2x_3\mathrm{正定},则()
$$
$$
A.
0.8<\;a<1 \quad B.-0.8<\;a<1 \quad C.0<\;a<0.8 \quad D.-0.8<\;a<0 \quad E. \quad F. \quad G. \quad H.
$$
$$
\begin{array}{l}A=\begin{pmatrix}1&a&-1\\a&1&2\\-1&2&5\end{pmatrix},\left|A_1\right|=1,\left|A_2\right|=\begin{vmatrix}1&a\\a&1\end{vmatrix}=1-a^2,\left|A_3\right|=\begin{vmatrix}1&a&-1\\a&1&2\\-1&2&5\end{vmatrix}=-5a^2-4a\\要A\mathrm{正定},\mathrm{必须}\left\{\begin{array}{l}1-a^2>0\\-5a^2-4a>0\end{array}\right.,\mathrm{解不等式组得}-0.8<\;a<0\end{array}
$$



$$
\mathrm{二次型}f(x_1,x_2,x_3)=ax_1^2+bx_2^2+ax_3^2+2cx_1x_3\mathrm{为正定的},则a,b,c\mathrm{应满足的条件为}()
$$
$$
A.
a>b>0,a>c \quad B.a>b>0,a>\left|c\right| \quad C.a>0,b>0,a>c \quad D.a>0,b>0,a>\left|c\right| \quad E. \quad F. \quad G. \quad H.
$$
$$
\begin{array}{l}\mathrm{二次型}f\mathrm{的矩阵为}\begin{pmatrix}a&0&c\\0&b&0\\c&0&a\end{pmatrix},f\mathrm{为正定的},\mathrm{故其顺序主子式都大于零},得a>0,ab>0,b(a^2-c^2)>0,\\a,b,c\mathrm{应满足条件}a>0,b>0,a>\left|c\right|\end{array}
$$



$$
设f=λ(x_1^2+x_2^2+x_3^2)+2x_1x_2+2x_1x_3-2x_2x_3,若f\mathrm{是负定的},则λ\mathrm{的值为}()
$$
$$
A.
λ>2 \quad B.λ<2 \quad C.λ>-1 \quad D.λ<-1 \quad E. \quad F. \quad G. \quad H.
$$
$$
\begin{array}{l}\mathrm{二次型}f\mathrm{的矩阵为}A=\begin{pmatrix}λ&1&1\\1&λ&-1\\1&-1&λ\end{pmatrix},\mathrm{其各阶顺序主子式为}λ,\begin{vmatrix}λ&1\\1&λ\end{vmatrix}=λ^2-1,\left|A\right|=(λ-2)(λ+1)^2\\\mathrm{若为}f\mathrm{负定的},则λ\mathrm{应满足}\begin{array}{c}λ<0\\λ^2-1>0\\(λ-2)(λ+1)^2<0\end{array}⇒λ<-1\end{array}
$$



$$
\mathrm{二次曲面}x^2+(λ+2)y^2+λ z^2+2xy=5\mathrm{是一个椭球面},则λ\mathrm{的取值为}()
$$
$$
A.
λ>0 \quad B.λ<0 \quad C.λ>-1 \quad D.λ<-1 \quad E. \quad F. \quad G. \quad H.
$$
$$
\begin{array}{l}\mathrm{设方程左边为}f=x^TAx,x=(x,y,z)^T,\mathrm{其中}A=\begin{pmatrix}1&1&0\\1&λ+2&0\\0&0&λ\end{pmatrix},\mathrm{因为原二次曲面是一个椭球面},\mathrm{因此}f\mathrm{化为标准形时为正定}\\\mathrm{二次型},\mathrm{因此是正定的},\mathrm{故的顺序主子式}\left|1\right|=1>0,\begin{vmatrix}1&1\\1&λ+2\end{vmatrix}=λ+1>0,\left|A\right|=λ(λ+1)>0,\mathrm{所以}λ>0\end{array}
$$



$$
\mathrm{若二次型}f=x_1^2+4x_2^2+2x_3^2+2tx_1x_2+2x_1x_3\mathrm{正定},则t\mathrm{满足}()
$$
$$
A.
0<\;t<\sqrt2 \quad B.-\sqrt2<\;t<0 \quad C.-\sqrt2<\;t<\sqrt2 \quad D.0<\;t<\sqrt2 \quad E. \quad F. \quad G. \quad H.
$$
$$
\begin{array}{l}\mathrm{二次型}f\mathrm{对应的实对称阵}A=\begin{pmatrix}1&t&1\\t&4&0\\1&0&2\end{pmatrix}\mathrm{正定时},\mathrm{顺序主子式全大于零},则1>0,\begin{vmatrix}1&t\\t&4\end{vmatrix}=4-t^2>0,\left|A\right|=-2t^2+4>0,\\\mathrm{所以}-\sqrt2<\;t<\sqrt2\end{array}
$$



$$
\mathrm{二次型}f(x_1,x_2,x_3)=2x_1^2+x_2^2+3x_3^2+2tx_1x_2+2x_1x_3\mathrm{正定},\mathrm{则实数}t\mathrm{的值为}()
$$
$$
A.
t<\sqrt2 \quad B.t<\sqrt{\frac53} \quad C.\left|t\right|<\sqrt2 \quad D.\left|t\right|<\sqrt{\frac53} \quad E. \quad F. \quad G. \quad H.
$$
$$
A=\begin{pmatrix}2&t&1\\t&1&0\\1&0&3\end{pmatrix},由\triangle_1=2>0,\triangle_2=2-t^2>0⇒\left|t\right|<\sqrt2,\triangle_3=\left|A\right|=-3t^2+5>0⇒\left|t\right|<\sqrt{\frac53},\;\mathrm{故当}\left|t\right|<\sqrt{\frac53},f\mathrm{正定}
$$



$$
\mathrm{二次型}f(x_1,x_2,x_3)=2x_1^2+2x_2^2+x_3^2+2tx_1x_2+6x_1x_3+2x_2x_3\mathrm{正定},\mathrm{则实数}t\mathrm{的值为}()
$$
$$
A.
-2<\;t<2 \quad B.t>3 \quad C.t>2 \quad D.\mathrm{无解} \quad E. \quad F. \quad G. \quad H.
$$
$$
A=\begin{pmatrix}2&t&3\\t&2&1\\3&1&1\end{pmatrix},\triangle_1=2>0,\triangle_2=4-t^2>0⇒\left|t\right|<2;\triangle_3=-7-(t-3)^2<0,\mathrm{故对于}t\mathrm{取任何值},f\mathrm{都不正定}.
$$



$$
\mathrm{若二次型}f(x_1,x_2,x_3)=2x_1^2+x_2^2+x_3^2+2x_1x_2+tx_2x_3\mathrm{是正定的},则t\mathrm{的取值范围是}()
$$
$$
A.
-\sqrt2<\;t<2 \quad B.-\sqrt2<\;t<\sqrt2 \quad C.-2<\;t<2 \quad D.-2<\;t<\sqrt2 \quad E. \quad F. \quad G. \quad H.
$$
$$
\begin{array}{l}\mathrm{二次型的矩阵为}A=\begin{pmatrix}2&1&0\\1&1&\frac t2\\0&\frac t2&1\end{pmatrix},\mathrm{因为}f\mathrm{为正定二次型},\mathrm{因此}A\mathrm{的各阶主子式都为正},即\\2>0,\begin{vmatrix}2&1\\1&1\end{vmatrix}=1>0,\begin{vmatrix}2&1&0\\1&1&\frac t2\\0&\frac t2&1\end{vmatrix}=1-\frac{t^2}2>0,\mathrm{所以}-\sqrt2<\;t<\sqrt2\end{array}
$$



$$
\mathrm{二次型}\;f(x_1,x_2,x_3)=x_1^2+2x_2^2+(1-k)x_3^2+2kx_1x_2+2x_1x_3\mathrm{正定},则k\mathrm{的值为}()
$$
$$
A.
-1<\;k<0 \quad B.-1<\;k<1 \quad C.0<\;k<1 \quad D.k>1或k<-1 \quad E. \quad F. \quad G. \quad H.
$$
$$
\begin{array}{l}\mathrm{可以用顺序主子式法}:\left|A_2\right|=\begin{vmatrix}1&k\\k&2\end{vmatrix}=2-k^2>0,\left|A_3\right|=\begin{vmatrix}1&k&1\\k&2&0\\1&0&1-k\end{vmatrix}=k(k^2-k-2)>0\\由\left\{\begin{array}{l}2-k^2>0\\k(k^2-k-2)>0\end{array}\right.,得\left\{\begin{array}{l}-\sqrt2<\;k<\sqrt2\\k<0,-1<\;k<0\end{array}\right.,\mathrm{其公共部分为}-1<\;k<0,\mathrm{故使正定的取值范围是}-1<\;k<0\end{array}
$$



$$
\;\mathrm{设二次型}f(x_1,x_2,x_3)=x_1^2+x_2^2+6x_3^2+4x_1x_2+6x_1x_3+6x_2x_3,则()
$$
$$
A.
f\mathrm{为正定的} \quad B.f\mathrm{为负定的} \quad C.f\mathrm{的秩为}1 \quad D.f\mathrm{既不正定},\mathrm{也不负定} \quad E. \quad F. \quad G. \quad H.
$$
$$
\begin{array}{l}\mathrm{二次型的矩阵为}A=\begin{pmatrix}1&2&3\\2&1&3\\3&3&6\end{pmatrix},则\left|A_1\right|=1>0,\left|A_2\right|=\begin{vmatrix}1&2\\2&1\end{vmatrix}=-3<0,\left|A_3\right|=\left|A\right|=0\\故f\mathrm{不正定},\mathrm{且不负定},\mathrm{由于}A\mathrm{存在二阶子式不为零},故r(A)=2,即f\mathrm{的秩也为}2.\end{array}
$$



$$
\;\mathrm{设二次型}f(x_1,x_2,x_3)=4x_1^2+4x_2^2+4x_3^2+2x_1x_2+2x_1x_3+2x_2x_3,则()
$$
$$
A.
f\mathrm{的秩为}1 \quad B.f\mathrm{的秩为}1 \quad C.f\mathrm{为负定的} \quad D.f\;\mathrm{为负定的} \quad E. \quad F. \quad G. \quad H.
$$
$$
\begin{array}{l}\mathrm{二次型的矩阵为}A=\begin{pmatrix}4&1&1\\1&4&1\\1&1&4\end{pmatrix},则\left|A_1\right|=4>0,\left|A_2\right|=\begin{vmatrix}4&1\\1&4\end{vmatrix}=15>0,\left|A_3\right|=\left|A\right|54>0\\故f\mathrm{为正定二次型},\mathrm{且秩为}3\end{array}
$$



$$
\mathrm{二次型}f(x_1,x_2,x_3)=2x_1^2+x_2^2+x_3^2-2tx_1x_2+2x_1x_3\mathrm{正定时},t\mathrm{应满足的条件是}(\;).\;
$$
$$
A.
-1<\;t<1 \quad B.-\sqrt2<\;t<\sqrt2 \quad C.-\sqrt2<\;t<1 \quad D.-1<\;t<\sqrt2 \quad E. \quad F. \quad G. \quad H.
$$
$$
\begin{array}{l}\mathrm{二次型的矩阵为}A=\begin{pmatrix}2&-t&1\\-t&1&0\\1&0&1\end{pmatrix},\mathrm{故二次型正定的充要条件是}2>0,\begin{vmatrix}2&-t\\-t&1\end{vmatrix}>0,\left|A\right|>0\\故\left\{\begin{array}{l}2-t^2>0\\1-t^2>0\end{array}\right.⇒ t^2<1⇒-1<\;t<1\end{array}
$$



$$
\mathrm{二次型}5x_1^2+x_2^2+ax_3^2+4x_1x_2-2x_1x_3-2x_2x_3\mathrm{正定},则()
$$
$$
A.
a>2 \quad B.a<2 \quad C.0<\;a<2 \quad D.2<\;a<5 \quad E. \quad F. \quad G. \quad H.
$$
$$
A=\begin{pmatrix}5&2&-1\\2&1&-1\\-1&-1&a\end{pmatrix},\left|A_1\right|=5>0,\left|A_2\right|=\begin{vmatrix}5&2\\2&1\end{vmatrix}=1>0,\left|A_3\right|=\begin{vmatrix}5&2&-1\\2&1&-1\\-1&-1&a\end{vmatrix}=a-2>0,\mathrm{从而}A\mathrm{要正定},\mathrm{必须}a>2
$$



$$
\begin{array}{l}\mathrm{下列二次型的正定性分别为}()\\(1)\;f=-2x_1^2-6x_2^2-4x_3^2+2x_1x_2+2x_1x_3\\(2)\;f=x_1^2+3x_2^2+9x_3^2+19x_4^2-2x_1x_2+4x_1x_3+2x_1x_4-6x_2x_4-12x_3x_4\end{array}
$$
$$
A.
(1)\mathrm{正定};(2)\mathrm{负定} \quad B.(1)\mathrm{正定};(2)\mathrm{正定} \quad C.(1)\mathrm{负定};(2)\mathrm{负定} \quad D.(1)\mathrm{负定};(2)\mathrm{正定} \quad E. \quad F. \quad G. \quad H.
$$
$$
\begin{array}{l}(1)A=\begin{pmatrix}-2&1&1\\1&-6&0\\1&0&-4\end{pmatrix},a_{11}=-2<0,\begin{vmatrix}-2&1\\1&-6\end{vmatrix}=11>0,\begin{vmatrix}-2&1&1\\1&-6&0\\1&0&-4\end{vmatrix}=-38<0,\mathrm{所以}f\mathrm{负定}.\\(2)A=\begin{pmatrix}1&-1&2&1\\-1&3&0&-3\\2&0&9&-6\\1&-3&-6&19\end{pmatrix},a_{11}=1>0,\begin{vmatrix}1&-1\\-1&3\end{vmatrix}=2>0,\begin{vmatrix}1&-1&2\\-1&3&0\\2&0&9\end{vmatrix}=6>0,\left|A\right|=24>0,故f\mathrm{正定}\end{array}
$$



$$
n\mathrm{阶矩阵}A\mathrm{为正定的充要条件}.
$$
$$
A.
\left|A\right|>0 \quad B.\mathrm{存在}n\mathrm{阶矩阵}C,使A=C^TC \quad C.A\mathrm{的特征值全大于零} \quad D.\mathrm{存在}n\mathrm{维列向量}α\mathrm{不等于}0,使α^TAα>0 \quad E. \quad F. \quad G. \quad H.
$$
$$
\begin{array}{l}n\mathrm{阶矩阵}A\mathrm{为正定的充要条件}:\\(1)A\mathrm{的特征值全大于零}\\(2)\mathrm{存在}n\mathrm{阶可逆矩阵}C,使A=C^TC\\(3)\mathrm{对任何}n\mathrm{维列向量}α\mathrm{不等于}0,使α^TAα>0\end{array}
$$



$$
设\;f(x_1,x_{2,...,}x_n)=x^tAx,\mathrm{下列说法中},\mathrm{不正确的是}()
$$
$$
A.
若\;f\;\mathrm{正定},则\left|A\right|>0 \quad B.若\;f\;\mathrm{正定},\mathrm{则负惯性指数为零} \quad C.\mathrm{若负惯性指数为零},则\;f\;\mathrm{正定} \quad D.若\;f\;\mathrm{正定},\mathrm{则各阶顺序主子式为正数} \quad E. \quad F. \quad G. \quad H.
$$
$$
\mathrm{负惯性指数为零},若\;f的\;\mathrm{秩不等于}n,则\;f\;\mathrm{也不是正定的}
$$



$$
n\;\mathrm{元实二次型}f(x_1,x_{2,...,}x_n)=x^tAx\mathrm{正定},\mathrm{它的正惯性指数}p,秩\;r\;与\;n\;\mathrm{的关系是}()
$$
$$
A.
p=r=n \quad B.p=r\leq n \quad C.p\leq r\leq n \quad D.p\leq r=n \quad E. \quad F. \quad G. \quad H.
$$
$$
\mathrm{由正定矩阵的判别法即可得到}p=r=n
$$



$$
A为\;n\;\mathrm{阶实对称方阵},\mathrm{且正定},则()
$$
$$
A.
A=E \quad B.A与E\mathrm{相似} \quad C.A^2=E \quad D.A\mathrm{合同于}E \quad E. \quad F. \quad G. \quad H.
$$
$$
\mathrm{矩阵}A为\;\;n\;\mathrm{阶正定矩阵的充要条件是}:\mathrm{存在可逆矩阵}C,使A=C^TC,即A与E\mathrm{合同}.
$$



$$
若A、B\mathrm{为同阶正定矩阵},则()
$$
$$
A.
AB,A+B\mathrm{都正定} \quad B.AB\mathrm{正定},A+B\mathrm{非正定} \quad C.AB\mathrm{非正定},A+B\mathrm{正定} \quad D.AB\mathrm{不一定正定},A+B\mathrm{正定} \quad E. \quad F. \quad G. \quad H.
$$
$$
\begin{array}{l}\mathrm{根据正定矩阵的定义}:\mathrm{对任何非零向量}x,有\\x^TAx>0,x^TBx>0⇒ x^T(A+B)x=x^TAx+x^TBx>0\\故A+B\mathrm{正定},\mathrm{但无法判断}x^T(AB)x\mathrm{是否一定大于零},\mathrm{因此}AB\mathrm{不一定正定}.\end{array}
$$



$$
n\mathrm{阶实方阵}A\mathrm{不可逆的充要条件}()
$$
$$
A.
\left|A\right|\neq0 \quad B.r(A)=n \quad C.A'A\mathrm{正定} \quad D.0是A\mathrm{的一个特征值} \quad E. \quad F. \quad G. \quad H.
$$
$$
\begin{array}{l}n\mathrm{阶实方阵}A\mathrm{不可逆的充要条件是}\left|A\right|=0,又\left|A\right|=λ_1...λ_n=0,故A\mathrm{至少有一个特征值为}0\\\mathrm{选项中}r(A)=n⇒ r(A)=n;\mathrm{且又由于正定矩阵的行列式大于零},故A'A\mathrm{正定可得}\left|AA'\right|=\left|A\right|\left|A'\right|>0,\\即\left|A\right|\neq0,\mathrm{因此都不正确}.\end{array}
$$



$$
\mathrm{正定二次型的矩阵必有}()
$$
$$
A.
\mathrm{对称且所有元素都大于零} \quad B.\mathrm{对称且主对角线上的元素都大于零} \quad C.\mathrm{对称且各阶顺序主子式都大于零} \quad D.\mathrm{对称且各阶子式都大于零} \quad E. \quad F. \quad G. \quad H.
$$
$$
\begin{array}{l}\mathrm{正定二次型的矩阵必为实对称矩阵},\mathrm{且各阶顺序主子式大于零}.\;\\\;\mathrm{若将矩阵对角化后},\mathrm{对角矩阵中对角线的元素全大于零},\mathrm{即矩阵的特征值全大于零}.\end{array}
$$



$$
\mathrm{设矩阵}A=\begin{pmatrix}-a&b\\b&a\end{pmatrix},\mathrm{其中}a>b>0,a^2+b^2=1,则A为()
$$
$$
A.
\mathrm{正定矩阵} \quad B.\mathrm{初等矩阵} \quad C.\mathrm{正交矩阵} \quad D.\mathrm{以上都不对} \quad E. \quad F. \quad G. \quad H.
$$
$$
AA^T=\begin{pmatrix}-a&b\\b&a\end{pmatrix}\begin{pmatrix}-a&b\\b&a\end{pmatrix}=\begin{pmatrix}a^2+b^2&0\\0&a^2+b^2\end{pmatrix}=\begin{pmatrix}1&0\\0&1\end{pmatrix}=E,由\;正\;交\;矩\;阵\;的\;定\;义\;可\;知\;,\;矩\;阵\;A\;为\;正\;交\;矩\;阵\;,\;其\;它\;选\;项\;无\;法\;判\;断\;.
$$



$$
\mathrm{如果}n\mathrm{阶实对称矩阵}A\mathrm{的特征值为}λ_1,λ_2,⋯,λ_n,当\;t\;\mathrm{满足条件}()时,A-tE\mathrm{为正定矩阵}.
$$
$$
A.
t<\;min\left\{λ_1,λ_2,⋯,λ_n\right\} \quad B.t<\;max\left\{λ_1,λ_2,⋯,λ_n\right\} \quad C.t>\;min\left\{λ_1,λ_2,⋯,λ_n\right\} \quad D.t>\;max\left\{λ_1,λ_2,⋯,λ_n\right\} \quad E. \quad F. \quad G. \quad H.
$$
$$
\begin{array}{l}\mathrm{易知}A-tE\mathrm{的特征值为}λ_1-t,λ_2-t,⋯,λ_n-t,\mathrm{为正定矩阵的充分必要条件是}λ_1-t>0,t<λ_i(i=1,2,⋯,n),\\故\;t<\;min\left\{λ_1,λ_2,⋯,λ_n\right\}\end{array}
$$



$$
设A是n\mathrm{阶正定矩阵},E是n\mathrm{阶单位矩阵},\mathrm{则下列正确的是}()
$$
$$
A.
\left|A+E\right|>1 \quad B.\left|A+E\right|<1 \quad C.\left|A+E\right|\geq1 \quad D.\left|A+E\right|\leq1 \quad E. \quad F. \quad G. \quad H.
$$
$$
\begin{array}{l}设λ 是A\mathrm{的特征值},\mathrm{既存在}x\neq0\;,使Ax=λ x,由(A+E)x=(λ+1)x,知A+E\mathrm{有特征值}λ+1\\设A的n\mathrm{个特征值为}λ_1,λ_2,⋯,λ_n(λ_i=0,i=1,⋯,n),则A+E的n\mathrm{个特征值为}λ_1+1,λ_2+1,⋯,λ_n+1(λ_i+1>1,i=1,2,⋯,n)\\故\;\left|A+E\right|=(λ_1+1)(λ_2+1)⋯(λ_n+1)>1\end{array}
$$



$$
\mathrm{若二次型}f(x_1,x_2,x_3)=(k+1)x_1^2+(k-1)x_2^2+(k-2)x_3^2\mathrm{为正定二次型},则k\mathrm{的取值范围}()
$$
$$
A.
k>2 \quad B.k>1 \quad C.1<\;k<2 \quad D.k>-1 \quad E. \quad F. \quad G. \quad H.
$$
$$
\mathrm{二次型已经是标准形},\mathrm{根据正定二次型的性质},有\begin{array}{c}k+1>0\\k-1>0\\k-2>0\end{array},\mathrm{解得}k>2
$$



$$
\mathrm{二次型}f(x_1,x_2)=x_1^2+kx_2^2-4x_1x_2\mathrm{为正定的},则k\mathrm{的值为}()
$$
$$
A.
k>4 \quad B.k<4 \quad C.k>2 \quad D.k<2 \quad E. \quad F. \quad G. \quad H.
$$
$$
f(x_1,x_2)=x_1^2+kx_2^2-4x_1x_2=(x_1-2x_2)^2+(k-4)x_2^2=y_1^2+(k-4)y_2^2,故f\mathrm{正定的充要条件是}k-4>0⇒ k>4
$$



$$
\mathrm{设二次型}f(x_1,x_2,x_3)=(λ+1)x_1^2+λ x_2^2+(4-λ^2)x_3^2,\mathrm{若二次型是正定二次型},则λ\mathrm{满足}()
$$
$$
A.
2>λ>1 \quad B.2>λ>0 \quad C.2>λ>-2 \quad D.1>λ>0 \quad E. \quad F. \quad G. \quad H.
$$
$$
\begin{array}{l}\mathrm{二次型的矩阵为对角矩阵}A=\begin{pmatrix}λ+1&&\\&λ&\\&&4-λ^2\end{pmatrix}\\\mathrm{根据对角矩阵正定的判定定理可知},\mathrm{二次型}f\mathrm{正定}\Leftrightarrow A\mathrm{正定}\Leftrightarrow\left\{\begin{array}{c}λ+1>0\\λ>0\\4-λ^2>0\end{array}\right.,故2>λ>0\end{array}
$$



$$
设A\mathrm{是三阶实对称矩阵},\mathrm{且满足条件}A^2+2A=0,\mathrm{已知}r(A)=2,A+kE\mathrm{为正定矩阵},则k\mathrm{的值为}()
$$
$$
A.
k>2 \quad B.k<2 \quad C.k>-2 \quad D.k<-2 \quad E. \quad F. \quad G. \quad H.
$$
$$
\begin{array}{l}\mathrm{因为}A^2+2A=0,\mathrm{所以}A\mathrm{的特征值}λ\mathrm{满足}λ^2+2λ=0,\mathrm{因此}A\mathrm{的三个特征值可能是}-2或0,\mathrm{由于}A\mathrm{是实对称矩阵},\mathrm{故存在可逆矩阵}P,\;\\\mathrm{使得}P^{-1}AP=ω,\mathrm{其中}ω=diag(λ_1,λ_2,λ_3);\\\mathrm{因为}r(ω)=r(A)=2,\mathrm{所以}A\mathrm{的三个特征值中有两个非零值},\mathrm{一个零值},即λ_1=λ_2=-2,λ_3=0;\\\mathrm{从而可得出}A+kE\mathrm{的特征值为}λ_1=λ_2=-2+k,λ_3=k,\mathrm{根据正定矩阵的特征值全大于零可知}-2+k>0,k>0,则k>2\end{array}
$$



$$
\mathrm{设矩阵}A=\begin{pmatrix}1&0&1\\0&2&0\\1&0&1\end{pmatrix},\mathrm{矩阵}B=(kE+A)^2,\mathrm{其中}k\mathrm{为实数},E\mathrm{为单位矩阵},B\mathrm{为正定矩阵},则k\mathrm{的值为}()
$$
$$
A.
k\neq-2 \quad B.k\neq0 \quad C.k\neq-2且k\neq0 \quad D.k\neq-2或k\neq0 \quad E. \quad F. \quad G. \quad H.
$$
$$
\begin{array}{l}设A\mathrm{的特征值为}λ,则\left|λ E-A\right|=\begin{vmatrix}λ-1&0&-1\\0&λ-2&0\\-1&0&λ-1\end{vmatrix}=λ(λ-2)^2,A\mathrm{的特征值为}λ_1=0,λ_2=λ_3=2,\\设λ 为A\mathrm{的特征值},x\neq0\mathrm{为相应的特征向量},即Ax=λ x,\mathrm{则有}Bx=(kE+A)^2x=(k+λ)^2x,即B\mathrm{有特征值}(k+λ)^2\\故B\mathrm{的特征值为}(k+2)^2,(k+2)^2,k^2,\mathrm{由此可得}k\neq-2且k\neq0时,B\mathrm{的所有特征值都大于零}.\mathrm{这时}B\mathrm{是正定阵}.\end{array}
$$



$$
\mathrm{设二次型}f(x_1,x_2,x_3)=2x_1x_2+2x_1x_3+2x_2x_3,则()
$$
$$
A.
f\mathrm{是正定的} \quad B.f\mathrm{是负定的} \quad C.f\mathrm{既不正定},\mathrm{也不负定} \quad D.f\mathrm{的秩为}2 \quad E. \quad F. \quad G. \quad H.
$$
$$
\begin{array}{l}令x_1=y_1+y_2,x_2=y_1-y_2,\mathrm{则原二次型可化为}2y_1^2-2y_2^3+4y_1y_3,\mathrm{再用配方法可化为标准形}\\2(y_1+y_3)^2-2y_2^2-2y_3^2,即z_1^2-z_2^2-z_3^2\\\mathrm{由于二次型的正惯性指数为}1\neq3\mathrm{即不正定},\mathrm{又负惯性指数为}2\neq3,\mathrm{故也不负定},\mathrm{且秩为}3.\end{array}
$$



$$
\mathrm{设二次型}f(x_1,x_2,x_3)=\frac13x_1^2+\frac13x_2^2+\frac43x_1x_3+\frac43x_2x_3,则()
$$
$$
A.
f\mathrm{是正定的} \quad B.f\mathrm{是负定的} \quad C.f\mathrm{的秩为}1 \quad D.f\mathrm{既不正定},\mathrm{也不负定} \quad E. \quad F. \quad G. \quad H.
$$
$$
\begin{array}{l}f(x_1,x_2,x_3)=\frac13x_1^2+\frac13x_2^2+\frac43x_1x_3+\frac43x_2x_3=\frac13(x_1+2x_3)^2+\frac13(x_2+2x_3)^2-\frac83x_3^2=\frac13y_1^2+\frac13y_22-\frac83y_3^2\\\mathrm{由于}f\mathrm{的正负惯性指数分别为}2和1,\mathrm{则既不正定},\mathrm{也不负定},\mathrm{且秩为}3\end{array}
$$



$$
设A为n\mathrm{阶方阵},则()
$$
$$
A.
A\mathrm{必与一对角阵合同} \quad B.若A\mathrm{与正定矩阵合同},则A\mathrm{为正定矩阵} \quad C.若A\mathrm{的所有主子式都大于}0,则A\mathrm{为正定矩阵} \quad D.若A\mathrm{与对角矩阵相似},则A\mathrm{也必与一对角矩阵合同} \quad E. \quad F. \quad G. \quad H.
$$
$$
\mathrm{与正定矩阵合同的矩阵也正定}.\;\;\mathrm{此题需要理解矩阵相似和合同概念},\mathrm{正定矩阵的性质等等}
$$



$$
\mathrm{如果}A,B\mathrm{都是}n\mathrm{阶正定实矩阵},则AB\mathrm{一定是}()
$$
$$
A.
\mathrm{实对称矩阵} \quad B.\mathrm{正交矩阵} \quad C.\mathrm{正定矩阵} \quad D.\mathrm{可逆矩阵} \quad E. \quad F. \quad G. \quad H.
$$
$$
\begin{array}{l}由A,B\mathrm{都是}n\mathrm{阶正定实矩阵可知},A^T=A,B^T=B,(AB)^T=B^TA^T=BA,故AB\mathrm{不是实对称矩阵};\\\mathrm{从而}AB\mathrm{肯定不是正定矩阵};\mathrm{且无法判断}AB\mathrm{是否为正交矩阵};\\由A,B\mathrm{正定可知},\left|A\right|\neq0,\left|B\right|\neq0,\mathrm{因此}\left|AB\right|=\left|A\right|·\left|B\right|\neq0,\mathrm{所以}AB\mathrm{可逆}.\end{array}
$$



$$
设A为n\mathrm{阶实对称矩阵},且A\mathrm{正定},\mathrm{如果矩阵}A与B\mathrm{相似},则B\mathrm{必为}()
$$
$$
A.
\mathrm{实对称矩阵} \quad B.\mathrm{正交矩阵} \quad C.\mathrm{可逆矩阵} \quad D.\mathrm{正定矩阵} \quad E. \quad F. \quad G. \quad H.
$$
$$
\begin{array}{l}A\mathrm{正定},则\left|A\right|\neq0,\mathrm{如果矩阵}B与A\mathrm{相似},则B=P^{-1}AP,\left|B\right|=\left|P^{-1}\right|\left|A\right|\left|P\right|=\left|A\right|\neq0,\mathrm{因此}B\mathrm{可逆};\\\mathrm{相似变换一般不保持对称性},\mathrm{例如矩阵}A=\begin{pmatrix}2&&\\&1&\\&&3\end{pmatrix}\mathrm{为对称矩阵},\mathrm{矩阵}B=\begin{pmatrix}1&2&3\\-1&4&2\\0&0&1\end{pmatrix}\mathrm{相似于}A,\mathrm{显然}B\mathrm{不是对称矩阵},\\\mathrm{更不是正交矩阵},\mathrm{且不是正定矩阵}.\end{array}
$$



$$
设A^2=E_n,\mathrm{则必有}()
$$
$$
A.
A\mathrm{是正交阵} \quad B.A\mathrm{是正定阵} \quad C.A\mathrm{是对称阵} \quad D.r(A+E_n)+r(A-E_n)=n \quad E. \quad F. \quad G. \quad H.
$$
$$
\begin{array}{l}A^2=E_n⇒ A^2-E_n=(A+E_n)(A-E_n)=0,\mathrm{则由矩阵秩的性质可知}r(A+E_n)+r(A-E_n)\leq n;\\又r(A+E_n)+r(A-E_n)=r(A+E_n)+r(E_n-A)\geq r(A+E_n+E_n-A)=r(2E_n)=n;故r(A+E_n)+r(A-E_n)=n\end{array}
$$



$$
n\mathrm{阶实对称矩阵}A\mathrm{正定的充分必要条件是}()
$$
$$
A.
\left|A\right|>0 \quad B.A\mathrm{的负惯性指数为零} \quad C.A^2\mathrm{正定} \quad D.\mathrm{存在}n\mathrm{阶可逆矩阵}C,使A=C^TC \quad E. \quad F. \quad G. \quad H.
$$
$$
n\mathrm{阶实对称矩阵}A\mathrm{正定的充要条件是存在}n\mathrm{阶可逆矩阵}C,使A=C^TC,\mathrm{或正惯性指数等于}n,\mathrm{若秩不为}n,\mathrm{则负惯性指数为零的矩阵也不正定}
$$



$$
若A\mathrm{既是正交矩阵又是正定矩阵},则A=()
$$
$$
A.
A^2 \quad B.2A \quad C.E \quad D.2E \quad E. \quad F. \quad G. \quad H.
$$
$$
\begin{array}{l}\mathrm{因为}A\mathrm{正交矩阵},\mathrm{又显然}A\mathrm{为对称矩阵},则A^TA=E,A^T=A,故A^2=E,即(A+E)(A-E)=0\\又A\mathrm{是正定矩阵},\mathrm{其特征值全为正},\mathrm{因此}-1\mathrm{不是}A\mathrm{的特征值},即\left|A+E\right|\neq0,即A+E\mathrm{可逆},则\\(A+E)^{-1}(A+E)(A-E)=0,故A-E=0⇒ A=E\end{array}
$$



$$
n\mathrm{元二次型}∑_{i=1}^nx_i^2+∑_{1\leq i  <  j\leq n}^{}x_ix_j\mathrm 的\mathrm 正\mathrm 定\mathrm 性\mathrm 为\;\;()
$$
$$
A.
\mathrm{正定的} \quad B.\mathrm{负定的} \quad C.\mathrm{半正定的} \quad D.\mathrm{不定的} \quad E. \quad F. \quad G. \quad H.
$$
$$
\begin{array}{l}\mathrm{由于}A=\frac12\begin{pmatrix}2&1&⋯&1\\1&2&⋯&1\\\vdots&\vdots&&\vdots\\1&1&⋯&2\end{pmatrix}=\frac12\left[E+\begin{pmatrix}1\\1\\\vdots\\1\end{pmatrix}(11⋯1)\right],记B=\begin{pmatrix}1\\1\\\vdots\\1\end{pmatrix}(11⋯1),则B^2=nB,\mathrm{那么}B\mathrm{的特征值是}n与0,\\\;\mathrm{于是}A\mathrm{的特征值是}\frac12(n+1),\frac12.\mathrm{由于}A\mathrm{的特征值全大于}0,\;故A\mathrm{正定},\mathrm{即二次型是正定的}\end{array}
$$



$$
设A\mathrm{是三阶实对称矩阵},\mathrm{且满足条件}A^2+2A=O,\mathrm{已知}r(A)=2,kA+E\mathrm{为正定矩阵},则k\mathrm{的值为}()
$$
$$
A.
k<\frac12 \quad B.k>\frac12 \quad C.k>-2 \quad D.k<-2 \quad E. \quad F. \quad G. \quad H.
$$
$$
\begin{array}{l}\mathrm{因为}A^2+2A=0,\mathrm{所以}A\mathrm{的特征值}λ\mathrm{满足}λ^2+2λ=0,\mathrm{因此}A\mathrm{的三个特征值可能是}-2或0,\mathrm{由于}A\mathrm{是实对称矩阵},\mathrm{故存在可逆矩阵}P,\;\\\mathrm{使得}P^{-1}AP=ω,\mathrm{其中}ω=diag(λ_1,λ_2,λ_3);\\\mathrm{因为}r(ω)=r(A)=2,\mathrm{所以}A\mathrm{的三个特征值中有两个非零值},\mathrm{一个零值},即λ_1=λ_2=-2,λ_3=0;\\\mathrm{从而可得出}kA+E\mathrm{的特征值为}λ_1=λ_2=-2k+1,λ_3=1,\mathrm{根据正定矩阵的特征值全大于零可知}-2k+1>0,则k<\frac12\end{array}
$$



$$
设A\mathrm{是三阶实对称矩阵},\mathrm{且满足条件}A^2+A=O,\mathrm{已知}r(A)=2,A+kE\mathrm{为正定矩阵},则k\mathrm{的值为}()
$$
$$
A.
\;0\;<\;k\;<\;1 \quad B.k>-1 \quad C.k>0 \quad D.k>1 \quad E. \quad F. \quad G. \quad H.
$$
$$
\begin{array}{l}\mathrm{因为}A^2+A=0,\mathrm{所以}A\mathrm{的特征值}λ\mathrm{满足}λ^2+λ=0,\mathrm{因此}A\mathrm{的三个特征值可能是}-1或0,\mathrm{由于}A\mathrm{是实对称矩阵},\mathrm{故存在可逆矩阵}P,\;\\\mathrm{使得}P^{-1}AP=ω,\mathrm{其中}ω=diag(λ_1,λ_2,λ_3);\\\mathrm{因为}r(ω)=r(A)=2,\mathrm{所以}A\mathrm{的三个特征值中有两个非零值},\mathrm{一个零值},即λ_1=λ_2=-1,λ_3=0;\\\mathrm{从而可得出}A+kE\mathrm{的特征值为}λ_1=λ_2=-1+k,λ_3=k,\mathrm{根据正定矩阵的特征值全大于零可知}-1+k>0,k>0\;,则k>1\end{array}
$$



$$
设A\mathrm{是三阶实对称矩阵},\mathrm{且满足条件}A^2+3A=O,\mathrm{已知}r(A)=2,kA+E\mathrm{为正定矩阵},则k\mathrm{的值为}()
$$
$$
A.
k>0 \quad B.k>-3 \quad C.k<\frac13 \quad D.k>\frac13 \quad E. \quad F. \quad G. \quad H.
$$
$$
\begin{array}{l}\mathrm{因为}A^2+3A=0,\mathrm{所以}A\mathrm{的特征值}λ\mathrm{满足}λ^2+3λ=0,\mathrm{因此}A\mathrm{的三个特征值可能是}-3或0,\mathrm{由于}A\mathrm{是实对称矩阵},\mathrm{故存在可逆矩阵}P,\;\\\mathrm{使得}P^{-1}AP=ω,\mathrm{其中}ω=diag(λ_1,λ_2,λ_3);\\\mathrm{因为}r(ω)=r(A)=2,\mathrm{所以}A\mathrm{的三个特征值中有两个非零值},\mathrm{一个零值},即λ_1=λ_2=-3,λ_3=0;\\\mathrm{从而可得出}kA+E\mathrm{的特征值为}λ_1=λ_2=-3k+1,λ_3=1,\mathrm{根据正定矩阵的特征值全大于零可知}-3k+1>0\;,则k<\frac13\end{array}
$$



$$
\begin{array}{l}\mathrm{二次型}\;f=(x_1,x_2,x_3)=x_1^2-2x_1x_2+x_2^2+x_3^2是()\\\end{array}
$$
$$
A.
\mathrm{正定的} \quad B.\mathrm{半正定的} \quad C.\mathrm{不定的} \quad D.\mathrm{半负定的} \quad E. \quad F. \quad G. \quad H.
$$
$$
f=(x_1,x_2,x_3)=x_1^2-2x_1x_2+x_2^2+x_3^2=(x_1-x_2)^2+x_3^2=y_1^2+y_2^2,\mathrm{则由定义可知},\mathrm{二次型为正定的}
$$



$$
\mathrm{二次型}\;f=(x_1,x_2,x_3)=x_1^2-2x_1x_2+2x_2^2+x_3^2是()
$$
$$
A.
\mathrm{正定的} \quad B.\mathrm{半正定的} \quad C.\mathrm{不定的} \quad D.\mathrm{半负定的} \quad E. \quad F. \quad G. \quad H.
$$
$$
f=(x_1,x_2,x_3)=x_1^2-2x_1x_2+2x_2^2+x_3^2=(x_1-x_2)^2+x_2^2+x_3^2=y_1^2+y_2^2+y_3^2,\mathrm{则由定义可知},\mathrm{二次型为正定的}
$$



$$
\mathrm{下列二次型为正定的是}()
$$
$$
A.
f(x,y,z)=-5x^2-6y^2-4z^2+4xy+4xz \quad B.f(x_1,x_2)=x_1^2-2x_2^2 \quad C.f(x_1,x_2,x_3)=-x_1^2+2x_1x_2+4x_1x_3-x_2^2+4x_2x_3-4x_3^2 \quad D.f(x_1,x_2,...,x_n)=x_1^2+x_2^2+...+x_n^2 \quad E. \quad F. \quad G. \quad H.
$$
$$
\begin{array}{l}\begin{array}{l}1.\mathrm{二次型}\;f(x_1,x_2,...,x_n)=x_1^2+x_2^2+...+x_n^2,当\;x=(x_1,x_2,...,x_n)^T\neq0时,\mathrm{显然有}f(x_1,x_2,...,x_n)>0,\mathrm{所以这个二次型是正定的},\\\mathrm{其矩阵}E_n\mathrm{是正定矩阵}.\\2.\mathrm{二次型}\;f(x_1,x_2,x_3)=-x_1^2+2x_1x_2+4x_1x_3-x_2^2+4x_2x_3-4x_3^2,\mathrm{将其改写成}f(x_1,x_2,x_3)=-(x_1+x_2-2x_3)^2\leq0,当x_1+x_2-2x_3=0时,\\f(x_1,x_2,x_3)=0,故f(x_1,x_2,x_3)\mathrm{是半负定的}.\\3.f(x_1,x_2)=x_1^2-2x_2^2\mathrm{是不定二次型},\mathrm{因其符号有时正有时负},如\;f(1,1)=-1<0,f(2,1)=2>0\end{array}\\4.\mathrm{根据顺序主子式判别法可判断二次型}f(x,y,z)=-5x^2-6y^2-4z^2+4xy+4xz\mathrm{为负定的}.\end{array}
$$



$$
\begin{array}{l}有n\mathrm{元实二次型}f(x_1,x_2,⋯,x_n)=(x_1+a_1x_2)^2+(x_2+a_2x_3)^2+⋯+(x_{n-1}+a_{n-1}x_n)^2+(x_n+a_nx_1)^2,\mathrm{其中}a_i(i=1,2,⋯,n)为\\\mathrm{实数}.\mathrm{二次型}f(x_1,x_2,⋯,x_n)\mathrm{为正定二次型},则a_1,a_2,⋯ a_n\mathrm{满足条件}(\;).\end{array}
$$
$$
A.
a_1a_2⋯ a_n\neq(-1)^n \quad B.a_1a_2⋯ a_n\neq(-1)^{n-1} \quad C.a_1a_2⋯ a_n\neq-1 \quad D.a_1a_2⋯ a_n=-1 \quad E. \quad F. \quad G. \quad H.
$$
$$
\begin{array}{l}\begin{array}{l}\begin{array}{l}\mathrm{由题设条件知},\mathrm{对于任意的}x_1,x_2,⋯,x_n,有f(x_1,x_2,⋯,x_n)\geq0,\mathrm{其中等号成立当且仅当}\left\{\begin{array}{c}x_1+a_1x_2=0\\x_2+a_2x_3=0\\⋯⋯⋯⋯,\\\begin{array}{c}x_n+a_nx_1=0\end{array}\end{array}\right.\\\mathrm{上述方程组仅有零解的充分必要条件是其系数行列式不为}0,即\\\;\;\;\;\;\;\;\;\;\;\;\;\;\;\;\;\begin{vmatrix}1&a_1&0&0&0\\0&1&a_2&0&0\\⋯&⋯&⋯&⋯&⋯\\0&0&0&1&a_{n-1}\\a_n&0&0&0&1\end{vmatrix}=1+(-1)^{n+1}a_1a_2⋯ a_n\neq0\end{array}\\\mathrm{所以},当1+(-1)^{n+1}a_1a_2⋯ a_n\neq0时,\mathrm{对于任意不全为}0的x_1,x_2,⋯,x_n,有f(x_1,x_2,⋯,x_n)>0,\mathrm{即当}a_1a_2⋯ a_n\neq(-1)^n\end{array}\\时,\mathrm{二次型}f(x_1,x_2,⋯,x_n)\mathrm{为正定二次型}.\end{array}
$$



$$
\begin{array}{l}\mathrm{二次型}\;f=(x_1,x_2,x_3)=x_1^2-x_1x_2+x_2^2是()\\\end{array}
$$
$$
A.
\mathrm{正定的} \quad B.\mathrm{半正定的} \quad C.\mathrm{不定的} \quad D.\mathrm{半负定的} \quad E. \quad F. \quad G. \quad H.
$$
$$
f=(x_1,x_2,x_3)=x_1^2-x_1x_2+x_2^2=(x_1-\frac{x_2}2)^2+\frac34x_2^2=y_1^2+y_2^2,\mathrm{则由定义可知},\mathrm{二次型为正定的}
$$



$$
\begin{array}{l}\mathrm{二次型}\;f=(x_1,x_2,x_3)=2x_1^2+3x_2^2+3x_3^2+4x_2x_3是()\\\end{array}
$$
$$
A.
\mathrm{正定的} \quad B.\mathrm{半正定的} \quad C.\mathrm{不定的} \quad D.\mathrm{半负定的} \quad E. \quad F. \quad G. \quad H.
$$
$$
f=(x_1,x_2,x_3)=2x_1^2+3x_2^2+3x_3^2+4x_2x_3=y_1^2+2y_2^2+5y_3^2,\mathrm{则由定义可知},\mathrm{二次型为正定的}
$$



$$
\begin{array}{l}\mathrm{二次型}\;f=(x_1,x_2,x_3)=-2x_1^2-6x_2^2-4x_3^2+2x_1x_2+2x_1x_3是()\\\end{array}
$$
$$
A.
\mathrm{正定的} \quad B.\mathrm{半正定的} \quad C.\mathrm{负定的} \quad D.\mathrm{半负定的} \quad E. \quad F. \quad G. \quad H.
$$
$$
f=(x_1,x_2,x_3)=-2x_1^2-6x_2^2-4x_3^2+2x_1x_2+2x_1x_3=-y_1^2-y_2^2-y_3^2,\mathrm{则由定义可知},\mathrm{二次型为负定的}
$$



$$
设α_1=(1,1,-1)^T,α_2=(-2,-1,2)^T,\mathrm{向量}α=(2,λ,μ)^T与α_1,α_2\mathrm{都正交},则\;λ=\left(\;\;\;\right).
$$
$$
A.
1 \quad B.2 \quad C.0 \quad D.3 \quad E. \quad F. \quad G. \quad H.
$$
$$
\begin{array}{l}\mathrm{根据条件可得}α_1^Tα=0,α_2^Tα=0,即\;\;\;\;\\\;\;\;\;\;\;\;\;\;\;\;\;\;\;\;\;\;\;\;\;\;\;\;\;\;\;\;\;\;\;\;\;\;\left\{\begin{array}{l}2+λ-μ=0\\-4-λ+2μ=0\end{array}\right.⇒λ=0,μ=2.\;\;\;\;\;\;\;\;\;\;\;\;\;\;\;\;\;\;\;\;\;\;\end{array}
$$



$$
设α=(x,0,{\textstyle\frac12})^T,β=(0,2y,0)^T\mathrm{均为单位向量},则x,y\mathrm{分别为}\left(\right)
$$
$$
A.
\textstyle\frac{±\sqrt3}2,±\frac12 \quad B.\textstyle\frac{±\sqrt3}2,\frac12 \quad C.\textstyle\frac{\sqrt3}2,±\frac12 \quad D.\textstyle\frac{\sqrt3}2,\frac12 \quad E. \quad F. \quad G. \quad H.
$$
$$
\begin{array}{l}\mathrm{由于}α,β\mathrm{都为单位向量},\mathrm{则其长度都为}1,即\;\;;\\\;\;\;\;\;\;\;\;\;\;\;\;\;\;\;\;\;\;\;\;\;\;\;\;\;\;\;\;\;\;\;\;\;\;\;\;\;\;\;\;\;\;\;\;\;\;\;\;\;\;\;\;\;\;\;\;\;\;\;\;\;\;\;\;\;x^2+{\textstyle\frac14}=1⇒ x={\textstyle\frac{±\sqrt3}2};\left(2y\right)^2=1⇒ y={\textstyle±}{\textstyle\frac12}\end{array}
$$



$$
\mathrm{设向量}α=\left(-1,a,3,1\right)^T与β=\left(2,-1,1,b\right)^T\mathrm{正交},则a与b\mathrm{的关系式是}\left(\;\right)
$$
$$
A.
a-b=1 \quad B.a+b=1 \quad C.a+b=-1 \quad D.a-b=-1 \quad E. \quad F. \quad G. \quad H.
$$
$$
α,β\mathrm{正交}⇒α^Tβ=0,即-1×2+a×\left(-1\right)+3×1+1× b=0,,则b-a=-1
$$



$$
\mathrm{向量}\begin{pmatrix}2\\1\\0\\3\end{pmatrix}\mathrm{与向量}\begin{pmatrix}1\\-2\\1\\k\end{pmatrix}\mathrm{的内积为}2,则\;k=\left(\;\;\right)
$$
$$
A.
\textstyle\frac23 \quad B.\textstyle\frac13 \quad C.\textstyle-\frac13 \quad D.\textstyle-\frac23 \quad E. \quad F. \quad G. \quad H.
$$
$$
\begin{pmatrix}2&1&0&3\end{pmatrix}\begin{pmatrix}1\\-2\\1\\k\end{pmatrix}=2⇒2-2+3k=2⇒ k={\textstyle\frac23}
$$



$$
\begin{array}{l}\\\begin{array}{l}\mathrm{设向量}α_1=(-1,1,2,-1)^T,{\boldsymbolα}_\mathbf2\boldsymbol=\boldsymbol(\mathbf0\boldsymbol,\mathbf3\boldsymbol,\mathbf8\boldsymbol,\boldsymbol-\mathbf2\boldsymbol)^\mathbf T\boldsymbol,{\boldsymbolα}_\mathbf3\boldsymbol=\boldsymbol(\mathbf3\boldsymbol,\mathbf1\boldsymbol,\mathbf2\boldsymbol,\mathbf2\boldsymbol)^\mathbf T\boldsymbol,则\\\left\|3α_1-α_2+α_3\right\|=\left(\;\;\right)\end{array}\end{array}
$$
$$
A.
\textstyle\sqrt2 \quad B.\textstyle2 \quad C.\textstyle\sqrt{10} \quad D.1 \quad E. \quad F. \quad G. \quad H.
$$
$$
\begin{array}{l}3α_1-α_2+α_3=3(-1,1,2,-1)^T-(0,3,8,-2)^T+(3,1,2,2)^T=(0,1,0,1)^T\\\;则\left\|3α_1-α_2+α_3\right\|=\sqrt{1^2+1^2}=\sqrt2.\end{array}
$$



$$
设α_1,α_2∈ R^n,\left\|α_1\right\|=\left\|α_2\right\|=1,\left\langleα_1,α_2\right\rangle={\textstyle\frac14},则\left\|α_1+α_2\right\|=\left(\;\;\right)
$$
$$
A.
\frac{\sqrt{10}}2 \quad B.\frac{\sqrt5}2 \quad C.\frac52 \quad D.5 \quad E. \quad F. \quad G. \quad H.
$$
$$
\begin{array}{l}\left\|α_1+α_2\right\|^2=\left\langleα_1+α_2,α_1+α_2\right\rangle=\left\langleα_1,α_1\right\rangle+2\left\langleα_1,α_2\right\rangle+\left\langleα_2,α_2\right\rangle\\=\left\|α_1\right\|^2+2\left\langleα_1,α_2\right\rangle+\left\|α_2\right\|^2=1+2×{\textstyle\frac14}+1={\textstyle\frac52}\\故\left\|α_1+α_2\right\|={\textstyle\frac{\sqrt{10}}2}\end{array}
$$



$$
\mathrm{与向量}α_1=(1,1,-1,1)^T,α_2=(1,-1,1,1)^T,α_3=(1,1,1,1)^T\mathrm{都正交的单位向量为}().\;
$$
$$
A.
\textstyle±\frac1{\sqrt2}(1,0,0,-1)^T\; \quad B.\textstyle\;\frac1{\sqrt2}(1,0,0,-1)^T\; \quad C.\textstyle±\frac1{\sqrt5}(2,0,0,1)^T \quad D.\textstyle\frac1{\sqrt5}(2,0,0,1)^T \quad E. \quad F. \quad G. \quad H.
$$
$$
\begin{array}{l}\mathrm{设向量}α=\left(x{}_1,x_2,x_3,x_4\right)与α_1,α_2,α_3\mathrm{都正交},则α\mathrm{应满足方程}\\\;\;\;\;\;\;\;\;\;\;\;\;\;\;\;\;\;\;\;\;\;\;\;\;\;\;\;\;\;\;\;\;\;\;\;\;\;\;\;\;\;\;\;\;\;\;\;\;\;\;\;\;\;α\;_i^Tα=0\;\;\;\;\left(i=1,2,3\right)\;,\;\;\;\;\;\;\;\;\;\;\;\;\;\;\;\;\;\;\;\;\;\;\;\\\mathrm{即满足方程组}\;\;\left\{\begin{array}{l}\begin{array}{c}x_1+x_2-x_3+x_4=0\\x_1-x_2+x_3+x_4=0\end{array}\\\begin{array}{c}x_1+x_2+x_3+x_4\end{array}=0\end{array}\right.\\\mathrm{它的基础解系为}\;ζ=±1·(1,0,0,-1)^T\;,\\\mathrm{单位化},\mathrm{得向量}±{\textstyle\frac1{\sqrt2}}(1,0,0,-1)^T,\;\mathrm{即为所求}.\end{array}
$$



$$
在R^3\mathrm{中与向量}α=(1,1,1)^T\mathrm{正交的全体向量可表示为}()
$$
$$
A.
V=\left\{(-k_1-k_2,k_1,k_2)^T│k_1,k_2∈ R\right\} \quad B.V=\left\{(k_1+k_2,k_1,k_2)^T│k_1,k_2∈ R\right\} \quad C.V=\left\{(k_1-k_2,k_1,k_2)^T│k_1,k_2∈ R\right\} \quad D.V=\left\{(k_2-k_1,k_1,k_2)^T│k_1,k_2∈ R\right\} \quad E. \quad F. \quad G. \quad H.
$$
$$
\begin{array}{l}设β=(b_1,b_2,b_3)^T与α\mathrm{正交}⇒\left\langleα,β\right\rangle=b_1+b_2+b_3.=0\;\;\\令\;\;b_2=k_1,b_3=k_2⇒ b_1=-k_1-k_2\\\mathrm{于是与}α=(1,1,1)^T\mathrm{正交的全体向量为}\;\;\\\;\;\;\;\;\;\;\;\;\;\;\;\;\;\;\;\;\;\;\;\;V=\left\{(-k_1-k_2,k_1,k_2)^T│k_1,k_2∈ R\right\}\end{array}
$$



$$
在R^n\mathrm{空间中},\mathrm{向量}a\mathrm{与任意向量}β\mathrm{的内积都等于零的充分必要条件是}\left\|a\right\|=\left(\;\;\right)
$$
$$
A.
0 \quad B.1 \quad C.-1 \quad D.2 \quad E. \quad F. \quad G. \quad H.
$$
$$
\mathrm{由题设可知},\mathrm{向量}a\mathrm{与其本身的内积也等于零},\mathrm{即向量}a\mathrm{的长度等于零},则a=0;\;\;\mathrm{反之},若a=0,\mathrm{则内积}\left\langle a,a\right\rangle=0,\mathrm{因此构成充要条件}.
$$



$$
\;\mathrm{与向量}α_1=(2,2,2)^T,α_2=(3,1,3)^T\mathrm{都正交的一个向量}α=(1,λ,μ)^T,则μ=(\;\;).
$$
$$
A.
0 \quad B.1 \quad C.-1 \quad D.2 \quad E. \quad F. \quad G. \quad H.
$$
$$
\begin{array}{l}\mathrm{由题设可知}\\\;\;\;\;\;\;\;\;\;\;\;\;\;\;\;\;\;\;\;\;\;\;\;\;\left\{\begin{array}{l}2+2λ+2μ=0\\3+λ+3μ=0\end{array}\right.⇒\left\{\begin{array}{l}λ=0\\μ=-1\end{array}\right.\end{array}
$$



$$
\mathrm{设向量}α=(-3,4,-2,4)^T,则\left\|α\right\|=\left(\;\;\;\right)
$$
$$
A.
3\sqrt5 \quad B.5\sqrt3 \quad C.5 \quad D.7 \quad E. \quad F. \quad G. \quad H.
$$
$$
∥α∥=\sqrt{(-3)^2+4^2+(-2)^2+4^2}=\sqrt{45}=3\sqrt5
$$



$$
\boldsymbol\;\boldsymbol 设{\boldsymbolα}_\mathbf1\boldsymbol=\boldsymbol(\mathbf2\boldsymbol,\boldsymbol-\mathbf2\boldsymbol,\mathbf1\boldsymbol)^\mathbf T\boldsymbol,{\boldsymbolα}_\mathbf2\boldsymbol=\boldsymbol(\mathbf0\boldsymbol,\mathbf1\boldsymbol,\boldsymbol-\mathbf1\boldsymbol)^\mathbf T\boldsymbol,\boldsymbol 且\boldsymbolβ\boldsymbol={\boldsymbol k}_\mathbf1{\boldsymbolα}_\mathbf1\boldsymbol+{\boldsymbol k}_\mathbf2{\boldsymbolα}_\mathbf2\mathbf{是与}{\boldsymbolα}_\mathbf1\mathbf{正交的单位向量}\boldsymbol,\boldsymbol 则\boldsymbolβ\boldsymbol 为\boldsymbol(\boldsymbol\;\boldsymbol\;\boldsymbol\;\boldsymbol)\boldsymbol.
$$
$$
A.
\textstyle\frac{\mathbf1}{\mathbf3}\begin{pmatrix}\mathbf2\\\mathbf1\\\boldsymbol-\mathbf2\end{pmatrix}\boldsymbol\;\boldsymbol\; \quad B.\textstyle\boldsymbol±\frac{\mathbf1}{\mathbf3}\begin{pmatrix}\mathbf2\\\mathbf1\\\boldsymbol-\mathbf2\end{pmatrix}\boldsymbol\; \quad C.\textstyle\mathbf3\begin{pmatrix}\mathbf2\\\mathbf1\\\boldsymbol-\mathbf2\end{pmatrix} \quad D.\textstyle\boldsymbol\;\boldsymbol±\mathbf3\begin{pmatrix}\mathbf2\\\mathbf1\\\boldsymbol-\mathbf2\end{pmatrix} \quad E. \quad F. \quad G. \quad H.
$$
$$
\begin{array}{l}由\;\;\\α_1^Tβ=k_1α_1^Tα_1+k_2α_1^Tα_2=0,\;\;\\得9k_1-3k_2=0,k_2-3k_1=0.令k_1=t故\;\;\\β=t\begin{pmatrix}2\\-2\\1\end{pmatrix}+3t\begin{pmatrix}0\\1\\-1\end{pmatrix}=t\begin{pmatrix}2\\1\\-2\end{pmatrix}.\;\;\\且\left\|β\right\|^2=t^2\lbrack2^2+1^2+(-2)^2\rbrack=9t^2=1,\;\;\\故t=±\frac13.\mathrm{故得}k_1=\frac13,k_2=1或k_1=-\frac13,k_2=-1\mathrm{为所求}.故\;\;\;\\β=\frac13\begin{pmatrix}2\\-2\\1\end{pmatrix}+\begin{pmatrix}0\\1\\-1\end{pmatrix}=\frac13\begin{pmatrix}2\\1\\-2\end{pmatrix}\\或\\β=-\frac13\begin{pmatrix}2\\-2\\1\end{pmatrix}-\begin{pmatrix}0\\1\\-1\end{pmatrix}=-\frac13\begin{pmatrix}2\\1\\-2\end{pmatrix}.\\\end{array}
$$



$$
\mathrm{如果向量}α=(1,-2,2,-1)^T\mathrm{与向量}β=(1,1,k,3)^T\mathrm{正交},则\;k=()
$$
$$
A.
2 \quad B.0 \quad C.1 \quad D.3 \quad E. \quad F. \quad G. \quad H.
$$
$$
\begin{array}{l}\mathrm{由条件可知}α^Tβ=0,即\\\;\;\;\;\;\;\;\;\;\;\;\;\;\;\;\;\;\;\;\;\;\;\;\;\;\;\;\;\;\;\;\;\;\;\;\;\;\;\;\;\;\;\;\;\;\;\;\;\;\;\;1×1+(-2)×1+2× k+(-1)×3=0⇒ k=2\end{array}
$$



$$
\mathrm{设向量}α=\left(1,a,b\right)^{\;T}\mathrm{与向量}α_1=\left(2,2,2\right)^{\;\;T},α_2=\left(3,1,3\right)^{\;T}\mathrm{都正交},则a,b\mathrm{分别为}\left(\;\;\;\right)
$$
$$
A.
a=0,b=-1 \quad B.a=0,b=1 \quad C.a=1,b=0 \quad D.a=-1,b=0 \quad E. \quad F. \quad G. \quad H.
$$
$$
\begin{array}{l}\\\begin{array}{l}\mathrm{由条件可知}\\\;\;\;\;\;\;\;\;\;\;\;\;\;\;\;\;\;\;\;\;\;\;\;\;\;\;\;\;\;\;\;\;\;\;\;\;\;\;\;\;\;\;\;\;\;\left\{\begin{array}{l}\left\langleα,α_1\right\rangle=2+2a+2b=0\\\left\langleα,α_2\right\rangle=3+a+3b=0\end{array}\right.⇒\left\{\begin{array}{l}a=0\\b=-1\end{array}\right.\end{array}\end{array}
$$



$$
\mathrm{向量}α_1=(1,5,k,-1)^T,α_2=(2k,3,-2,k)^T\mathrm{正交},则k为(\;).\;
$$
$$
A.
15 \quad B.-15 \quad C.5 \quad D.-5 \quad E. \quad F. \quad G. \quad H.
$$
$$
由\left\langleα_1,α_2\right\rangle=0,得2k+15-2k-k=0,k=15
$$



$$
设α_1=(2k,k-1,0,3)^T,α_2=(5,-3,k,k+1)^T\mathrm{正交},则k为(\;).\;
$$
$$
A.
k=\frac35 \quad B.k=-\frac35 \quad C.k=-\frac15 \quad D.k=\frac15 \quad E. \quad F. \quad G. \quad H.
$$
$$
由\left\langleα_1,α_2\right\rangle=0,得\;10k-3k+3+3k+3=0,k=-\frac35
$$



$$
\mathrm{若向量}α=(-1,0,3,1)^T与β=(2,-1,1,b)^T\mathrm{正交},则b=\left(\;\;\;\right)\;
$$
$$
A.
3 \quad B.-3 \quad C.1 \quad D.-1 \quad E. \quad F. \quad G. \quad H.
$$
$$
\mathrm{根据定义有}:\;-1×2+0×(-1)+3×1+1× b=0⇒ b=-1
$$



$$
设α_1=(1,0,-1)^T,α_2=(1,2,-2)^T,\;\;则α_1,α_2\mathrm{的夹角为}(\;\;\;\;)\;
$$
$$
A.
\textstyleθ=-{\displaystyle\fracπ6} \quad B.\textstyleθ={\displaystyle\fracπ6} \quad C.\textstyleθ=-{\displaystyle\fracπ4} \quad D.\textstyleθ={\displaystyle\fracπ4} \quad E. \quad F. \quad G. \quad H.
$$
$$
\textstyle\cosθ=\frac3{\sqrt2×3}=\frac{\sqrt2}2,\mathrm{所以}θ={\displaystyle\fracπ4}
$$



$$
设α_1=(1,1,-1)^T,α_2=(-2,-1,2)^{T,}则α_1,α_2\mathrm{夹角的余弦值为}(\;\;\;\;)
$$
$$
A.
\textstyle\frac5{3\sqrt3} \quad B.\textstyle\frac{-5}{3\sqrt3} \quad C.\textstyle\frac2{3\sqrt3} \quad D.\textstyle\frac{-2}{3\sqrt3}\;\; \quad E. \quad F. \quad G. \quad H.
$$
$$
\boldsymbol\;\boldsymbol\;\mathbf{\textstyle\cos}\mathbf{\textstyleθ}\mathbf{\textstyle=}\mathbf{\textstyle\frac{-2-1-2}{3\sqrt3}}\mathbf{\textstyle=}\mathbf{\textstyle\frac{-5}{3\sqrt3}}
$$



$$
\boldsymbol 设\boldsymbolα\boldsymbol=\boldsymbol(\mathbf0\boldsymbol,\boldsymbol y\boldsymbol,\boldsymbol-{\textstyle\frac{\mathbf1}{\sqrt{\mathbf2}}}\boldsymbol)^\mathbf T\boldsymbol,\boldsymbolβ\boldsymbol=\boldsymbol(\boldsymbol x\boldsymbol,\mathbf0\boldsymbol,\mathbf0\boldsymbol)^\mathbf T\boldsymbol,\boldsymbol 若\boldsymbolα\boldsymbol,\boldsymbolβ\mathbf{为标准正交向量组}\boldsymbol,\boldsymbol 则\boldsymbol x\boldsymbol 和\boldsymbol y\mathbf{分别为}\boldsymbol(\boldsymbol\;\boldsymbol\;\boldsymbol)\boldsymbol.
$$
$$
A.
\boldsymbol x\boldsymbol=\mathbf1\boldsymbol,\boldsymbol y\boldsymbol={\textstyle\frac{\sqrt{\mathbf2}}{\mathbf2}}\boldsymbol\; \quad B.\boldsymbol\;\boldsymbol x\boldsymbol=\boldsymbol±\mathbf1\boldsymbol,\boldsymbol y\boldsymbol=\boldsymbol±{\textstyle\frac{\sqrt{\mathbf2}}{\mathbf2}} \quad C.\boldsymbol\;\boldsymbol x\boldsymbol=\boldsymbol-\mathbf1\boldsymbol,\boldsymbol y\boldsymbol=\boldsymbol-{\textstyle\frac{\sqrt{\mathbf2}}{\mathbf2}} \quad D.\boldsymbol x\boldsymbol=\mathbf1\boldsymbol,\boldsymbol y\boldsymbol=\boldsymbol-{\textstyle\frac{\sqrt{\mathbf2}}{\mathbf2}} \quad E. \quad F. \quad G. \quad H.
$$
$$
\mathbf{由已知}\boldsymbol y^\mathbf2\boldsymbol+\frac{\mathbf1}{\mathbf2}\boldsymbol=\mathbf1\boldsymbol,\boldsymbol 则\boldsymbol y\boldsymbol=\boldsymbol±\frac{\sqrt{\mathbf2}}{\mathbf2}\boldsymbol,\boldsymbol x^\mathbf2\boldsymbol=\mathbf1\boldsymbol,\boldsymbol x\boldsymbol=\boldsymbol±\mathbf1\boldsymbol,\boldsymbol 由\boldsymbolα^\mathbf T\boldsymbolβ\boldsymbol=\mathbf0\boldsymbol 知\boldsymbol x\boldsymbol,\boldsymbol y\mathbf{为任意值}\boldsymbol,\mathbf{综上得}\boldsymbol x\boldsymbol=\boldsymbol±\mathbf1\boldsymbol,\boldsymbol y\boldsymbol=\boldsymbol±\frac{\sqrt{\mathbf2}}{\mathbf2}
$$



$$
\boldsymbol 设\boldsymbolα\boldsymbol,\boldsymbolβ\mathbf{都是}\boldsymbol n\mathbf{维的单位列向量}\boldsymbol,\boldsymbol 则\boldsymbolα\boldsymbol-\boldsymbolβ\boldsymbol 与\boldsymbolα\boldsymbol+\boldsymbolβ\mathbf{的内积为}\left(\boldsymbol\;\boldsymbol\;\right)
$$
$$
A.
1 \quad B.0 \quad C.-1 \quad D.2 \quad E. \quad F. \quad G. \quad H.
$$
$$
\begin{array}{l}\left\langle\left(\mathbfα\boldsymbol-\mathbfβ\right)\boldsymbol,\left(\mathbfα\boldsymbol+\mathbfβ\right)\right\rangle\boldsymbol=\left\langle\mathbfα\boldsymbol,\mathbfα\right\rangle\boldsymbol-\left\langle\mathbfβ\boldsymbol,\mathbfα\right\rangle\boldsymbol+\left\langle\mathbfα\boldsymbol,\mathbfβ\right\rangle\boldsymbol-\left\langle\mathbfβ\boldsymbol,\mathbfβ\right\rangle\\\boldsymbol\;\mathbf 由\left\langle\mathbfα\boldsymbol,\mathbfβ\right\rangle\boldsymbol=\left\langle\mathbfβ\boldsymbol,\mathbfα\right\rangle\mathbf 知\boldsymbol,\mathbf{故所求的内积为}\boldsymbol:\\\boldsymbol\;\boldsymbol\;\boldsymbol\;\boldsymbol\;\boldsymbol\;\boldsymbol\;\boldsymbol\;\boldsymbol\;\boldsymbol\;\boldsymbol\;\boldsymbol\;\boldsymbol\;\boldsymbol\;\boldsymbol\;\boldsymbol\;\boldsymbol\;\boldsymbol\;\boldsymbol\;\boldsymbol\;\boldsymbol\;\boldsymbol\;\boldsymbol\;\boldsymbol\;\boldsymbol\;\boldsymbol\;\boldsymbol\;\boldsymbol\;\boldsymbol\;\boldsymbol\;\boldsymbol\;\boldsymbol\;\boldsymbol\;\boldsymbol\;\boldsymbol\;\boldsymbol\;\boldsymbol\;\boldsymbol\;\boldsymbol\;\boldsymbol\;\boldsymbol\;\boldsymbol\;\boldsymbol\;\boldsymbol\;\boldsymbol\;\boldsymbol\;\boldsymbol\;\boldsymbol\;\boldsymbol\;\boldsymbol\;\boldsymbol\;\boldsymbol\;\boldsymbol\;\boldsymbol\;\boldsymbol\;\boldsymbol\;\boldsymbol\;\boldsymbol\;\boldsymbol\;\boldsymbol\;\boldsymbol\;\boldsymbol\;\boldsymbol\;\boldsymbol\;\boldsymbol\;\left\langle\mathbfα\boldsymbol,\mathbfα\right\rangle\boldsymbol-\left\langle\mathbfβ\boldsymbol,\mathbfβ\right\rangle\boldsymbol=\left\|\mathbfα\right\|^\mathbf2\boldsymbol-\left\|\mathbfβ\right\|^\mathbf2\boldsymbol=\mathbf1\boldsymbol-\mathbf1\boldsymbol=\mathbf0\end{array}
$$



$$
\mathbf{向量}\boldsymbolα\boldsymbol=\begin{pmatrix}\mathbf1\\\mathbf0\\\boldsymbol-\mathbf1\\\mathbf0\end{pmatrix}\boldsymbol,\boldsymbolβ\boldsymbol=\begin{pmatrix}\mathbf0\\\mathbf1\\\mathbf0\\\mathbf2\end{pmatrix}\mathbf{之间的夹角为}\left(\boldsymbol\;\boldsymbol\;\boldsymbol\;\right)
$$
$$
A.
\mathbf{90}^\boldsymbol. \quad B.\boldsymbol\;\boldsymbol\;\mathbf0^\boldsymbol. \quad C.\boldsymbol\;\mathbf{60}^\boldsymbol. \quad D.\mathbf{180}^\boldsymbol. \quad E. \quad F. \quad G. \quad H.
$$
$$
\begin{array}{l}\boldsymbol\;\boldsymbol\;\boldsymbol\;\boldsymbol\;\boldsymbol\;\boldsymbol\;\boldsymbol\;\boldsymbol\;\boldsymbol\;\boldsymbol\;\boldsymbol\;\boldsymbol\;\boldsymbol\;\boldsymbol\;\boldsymbol\;\boldsymbol\;\boldsymbol\;\boldsymbol\;\boldsymbol\;\boldsymbol\;\boldsymbol\;\boldsymbol\;\boldsymbol\;\boldsymbol\;\boldsymbol\;\boldsymbol\;\boldsymbol\;\boldsymbol\;\left\|\mathbfα\right\|\boldsymbol=\sqrt{\mathbf1^\mathbf2\boldsymbol+\mathbf0^\mathbf2\boldsymbol+\left(\boldsymbol-\mathbf1\right)^\mathbf2\boldsymbol+\mathbf0^\mathbf2}\boldsymbol=\sqrt{\mathbf2}\boldsymbol,\\\boldsymbol\;\mathbf{同理}\boldsymbol\;\left\|\mathbfβ\right\|\boldsymbol=\sqrt{\mathbf5}\boldsymbol,\boldsymbol\;\boldsymbol\;\\\mathbf 又\left[\mathbfα\boldsymbol,\mathbfβ\right]\boldsymbol\;\boldsymbol=\mathbf1\boldsymbol·\mathbf0\boldsymbol+\mathbf0\boldsymbol·\mathbf1\boldsymbol+\left(\boldsymbol-\mathbf1\right)\boldsymbol·\mathbf0\boldsymbol+\mathbf0\boldsymbol·\mathbf2\boldsymbol=\mathbf0\boldsymbol,\boldsymbol\;\\\boldsymbol\;\mathbf{所以}\boldsymbol\;\mathbf{cosθ}\boldsymbol={\textstyle\frac{\left\langle\mathbfα\boldsymbol,\mathbfβ\right\rangle}{\left\|\mathbfα\right\|\left\|\mathbfβ\right\|}}\boldsymbol=\mathbf0\boldsymbol⇒\mathbfθ\boldsymbol=\mathbf{90}^\boldsymbol.\end{array}
$$



$$
\mathrm{若向量}α=(-1,0,-1,1)^T与β=(2,-1,1,b)^T\mathrm{内积为}3,则b=\left(\;\;\;\right)\;
$$
$$
A.
6 \quad B.-6 \quad C.3 \quad D.-3 \quad E. \quad F. \quad G. \quad H.
$$
$$
(-1)×2+0×(-1)+(-1)×1+1× b=3⇒ b=6
$$



$$
\mathrm{已知}α_1=\left[1,1,1\right]^T,\mathrm{若存在向量}α_2,α_3,使α_1,α_2,α_3\mathrm{正交},则(\;\;).
$$
$$
A.
α_2=\left[-1,1,0\right]^T,α_3=\left[-1,0,1\right]^T \quad B.α_2=\left[-1,1,0\right]^T,α_3=\left[-\frac12,-\frac12,1\right]^T \quad C.α_2=\left[1,1,0\right]^T,α_3=\left[-1,0,1\right]^T \quad D.α_2=\left[-1,1,0\right]^T,α_3=\left[\frac12,\frac12,1\right]^T \quad E. \quad F. \quad G. \quad H.
$$
$$
\begin{array}{l}\mathrm{由题设可知}α_2,α_3\mathrm{应满足方程}α_1^Tx=0,设x=\left[x_1,x_2,x_3\right],即x_1+x_2+x_3=0,\mathrm{它的基础解系含向量的个数为}\\3-1=2,\mathrm{其解为}\;\\\;\;\;\;\;\;\;\;\;\;\;\;\;\;\;\;\;\;\;\;\;\;\;\;\;\;\;\;\;\;\;\;\;\;\;\;\;\;\;\;\;\begin{pmatrix}x_1\\x_2\\x_3\end{pmatrix}=c_1\begin{pmatrix}-1\\1\\0\end{pmatrix}+c_2\begin{pmatrix}-1\\0\\1\end{pmatrix},\;\;\\\mathrm{其中均为}c_1,c_2\mathrm{常数},\mathrm{基础解系为}\;\\\;\;\;\;\;\;\;\;\;\;\;\;\;\;\;\;\;\;\;\;\;\;\;\;\;\;\;\;\;\;\;\;\;\;\;\;\;\;\;\;\;\;\;\;\;\;\;\;\;\;\;\;\;\;\;\;\;\;\;\;\;\;\;\;\;\;\;ξ_1=\left[-1,1,0\right]^T\;,\;ξ_2=\left[-1,0,1\right]^T,\;\;\\\mathrm{可验证}ξ_1,ξ_2与α_1\mathrm{正交},\mathrm{只要将}ξ_1,ξ_2\mathrm{正交化就可以了}.令α_2=ξ_1=\left[-1,1,0\right]^T,则\;\;\\\;\;\;\;\;\;\;\;\;\;\;\;\;\;\;\;\;\;\;\;\;\;\;\;\;\;\;\;\;\;\;\;\;\;\;\;\;\;\;\;\;\;\;\;\;\;\;\;\;\;\;\;\;\;\;\;\;\;\;\;\;\;\;\;\;\;\;\;α_3\;=ξ_2-\frac{\left\langleξ_1,ξ_2\right\rangle}{\left\langleξ_1,ξ_1\right\rangle}ξ_1=\begin{pmatrix}-1\\0\\1\end{pmatrix}-{\textstyle\frac12}\begin{pmatrix}-1\\1\\0\end{pmatrix}=\begin{pmatrix}-{\textstyle\frac12}\\-{\textstyle\frac12}\\1\end{pmatrix},\;\;\\\mathrm{求得}α_2=\begin{pmatrix}-1\\1\\0\end{pmatrix},α_3=\begin{pmatrix}-{\textstyle\frac12}\\-{\textstyle\frac12}\\1\end{pmatrix},\mathrm{使得}α_1,α_2,α_3\mathrm{两两正交}.\end{array}
$$



$$
\textstyle\boldsymbol 与{\boldsymbolα}_\mathbf1\boldsymbol=\boldsymbol(\mathbf1\boldsymbol,\mathbf2\boldsymbol,\boldsymbol-\mathbf1\boldsymbol)^\mathbf T\boldsymbol,{\boldsymbolα}_\mathbf2\boldsymbol=\boldsymbol(\mathbf4\boldsymbol,\mathbf0\boldsymbol,\mathbf2\boldsymbol)^\mathbf T\mathbf{都正交的所有向量}\boldsymbolβ\boldsymbol=\left(\boldsymbol\;\boldsymbol\;\right)^{}
$$
$$
A.
\textstyle\boldsymbol(\mathbf1\boldsymbol,\boldsymbol-\mathbf3\boldsymbol,\boldsymbol-\mathbf5\boldsymbol)^\mathbf T\boldsymbol\; \quad B.\textstyle\boldsymbol\;\boldsymbol k\boldsymbol(\mathbf1\boldsymbol,\boldsymbol-\mathbf3\boldsymbol,\boldsymbol-\mathbf5\boldsymbol)^\mathbf T\boldsymbol k \quad C.\textstyle\boldsymbol(\mathbf2\boldsymbol,\boldsymbol-\mathbf3\boldsymbol,\boldsymbol-\mathbf4\boldsymbol)^\mathbf T \quad D.\textstyle\boldsymbol k\boldsymbol(\mathbf2\boldsymbol,\boldsymbol-\mathbf3\boldsymbol,\boldsymbol-\mathbf4\boldsymbol)^\mathbf T \quad E. \quad F. \quad G. \quad H.
$$
$$
\begin{array}{l}设β=(x_1,x_2,x_3),\mathrm{则由}α_1^Tβ=0,α_2^Tβ=0\mathrm{可得}\\\left\{\begin{array}{l}x_1+2x_2-x_3=0\\4x_1+2x_3=0\end{array}\right.⇒β=k(2,-3,-4)^T\end{array}
$$



$$
\textstyle\boldsymbol 设{\boldsymbolα}_\mathbf1\boldsymbol=\boldsymbol(\mathbf1\boldsymbol,\mathbf1\boldsymbol,\boldsymbol-\mathbf1\boldsymbol)^\mathbf T\boldsymbol,{\boldsymbolα}_\mathbf2\boldsymbol=\boldsymbol(\mathbf0\boldsymbol,\boldsymbol-\mathbf1\boldsymbol,\mathbf2\boldsymbol)^\mathbf T\boldsymbol,\mathbf{向量}\boldsymbolα\boldsymbol=\boldsymbol(\mathbf2\boldsymbol,\boldsymbolλ\boldsymbol,\boldsymbolμ\boldsymbol)^\mathbf T\boldsymbol 与{\boldsymbolα}_\mathbf1\boldsymbol,{\boldsymbolα}_\mathbf2\mathbf{都正交}\boldsymbol,\mathbf{则有}\boldsymbol\;\left(\boldsymbol\;\right)
$$
$$
A.
\textstyle\boldsymbolλ\boldsymbol=\mathbf4\boldsymbol,\boldsymbolμ\boldsymbol=\boldsymbol-\mathbf2\boldsymbol\; \quad B.\textstyle\boldsymbolλ\boldsymbol=\boldsymbol-\mathbf4\boldsymbol,\boldsymbolμ\boldsymbol=\mathbf2 \quad C.\textstyle\boldsymbol\;\boldsymbolλ\boldsymbol=\mathbf4\boldsymbol,\boldsymbolμ\boldsymbol=\mathbf2 \quad D.\textstyle\boldsymbolλ\boldsymbol=\boldsymbol-\mathbf4\boldsymbol,\boldsymbolμ\boldsymbol=\boldsymbol-\mathbf2 \quad E. \quad F. \quad G. \quad H.
$$
$$
\begin{array}{l}\\\textstyle\begin{array}{l}\mathbf{根据题意知}\boldsymbol:\left\{\begin{array}{l}\mathbf2\boldsymbol+\mathbfλ\boldsymbol-\mathbfμ\boldsymbol=\mathbf0\\\boldsymbol-\mathbfλ\boldsymbol+\mathbf2\mathbfμ\boldsymbol=\mathbf0\end{array}\right.\boldsymbol,\mathbf{所以λ}\boldsymbol=\boldsymbol-\mathbf4\boldsymbol,\mathbfμ\boldsymbol=\boldsymbol-\mathbf2\\\end{array}\end{array}
$$



$$
设α=(1,0,1)^T,β=(1,-1,0)^T,则α×β=\left(\;\;\;\;\right)
$$
$$
A.
(1,0,-1)^T \quad B.(1,1,-1)^T \quad C.(1,-1,-1)^T \quad D.(0,1,-1)^T \quad E. \quad F. \quad G. \quad H.
$$
$$
α×β=\begin{vmatrix}i&j&κ\\1&0&1\\1&-1&0\end{vmatrix}=i+j-k=(1,1,-1)^T
$$



$$
设α=(2,1,-1)^T,β=(1,-1,2)^T,则α×β=\left(\;\right)
$$
$$
A.
(-1,5,3)^T \quad B.(1,5,-3)^T \quad C.(1,-5,-3)^T \quad D.(1,-5,3)^T \quad E. \quad F. \quad G. \quad H.
$$
$$
α×β=\begin{vmatrix}i&j&k\\2&1&-1\\1&-1&2\end{vmatrix}=i-5j-3k=(1,-5,-3)^T
$$



$$
\;设α=(1,0,1)^T,β=(1,-1,0)^T,γ=(-2,\;\;0\;,\;1)^T,则α×\left(β+γ\right)=\left(\;\;\right)
$$
$$
A.
(1,2,-1)^T \quad B.(1,-2,1)^T \quad C.(1,-2,-1)^T \quad D.(-1,2,1)^T \quad E. \quad F. \quad G. \quad H.
$$
$$
\begin{array}{l}α×\left(β+γ\right)=\begin{vmatrix}i&j&k\\1&0&1\\-1&-1&1\end{vmatrix}=i-2j-k=(1,-2,-1)^T\\\end{array}
$$



$$
设α=\left(1,0,1\right)^T,β=\left(1,-1,0\right)^T,则\left(-2α\right)×β=\left(\;\;\;\right)
$$
$$
A.
\left(2,-2,2\right)^T \quad B.\left(2,2,-2\right)^T \quad C.\left(-4,-4,4\right)^T \quad D.\left(-2,-2,2\right)^T \quad E. \quad F. \quad G. \quad H.
$$
$$
\left(-2α\right)×β=(-2)×\begin{vmatrix}i&j&k\\1&0&1\\1&-1&0\end{vmatrix}=\left(-2,-2,2\right)^T
$$



$$
设α=\left(1,0,1\right)^T,β=\left(1,-1,0\right)^T,则\;β×\left(-2α\right)=\left(\;\;\;\right)
$$
$$
A.
\left(2,2,-2\right)^T \quad B.\left(-2,-2,2\right)^T \quad C.\left(2,-2,-2\right)^T \quad D.\left(-2,2,-2\right)^T \quad E. \quad F. \quad G. \quad H.
$$
$$
β×\left(-2α\right)=-2\begin{vmatrix}i&j&k\\1&-1&0\\1&0&1\end{vmatrix}=\left(2,2,-2\right)^T
$$



$$
设α=\left(1,0,1\right)^T,β=\left(1,-1,0\right)^T,\mathrm{则与}α,β\mathrm{都垂直的向量是}(\;\;).
$$
$$
A.
\left(3,-3,-3\right)^T \quad B.\left(1,-1,-1\right)^T \quad C.\left(3,3,-3\right)^T \quad D.\left(-1,3,-1\right)^T \quad E. \quad F. \quad G. \quad H.
$$
$$
α×β=\begin{vmatrix}i&j&k\\1&0&1\\1&-1&0\end{vmatrix}=\left(1,1,-1\right)^T,与α,β\mathrm{都垂直的向量与}\left(1,1,-1\right)^T\mathrm{对应成比例}.
$$



$$
设α=\left(1,0,1\right)^T,β=\left(1,-1,0\right)^T,\mathrm{则与}α,β\;\mathrm{都垂直的向量为}(\;\;).
$$
$$
A.
\left(-1,1,1\right)^T \quad B.\left(-1,1,-1\right)^T \quad C.\left(1,1,-1\right)^T \quad D.\left(1,-1,-1\right)^T \quad E. \quad F. \quad G. \quad H.
$$
$$
α×β=\begin{vmatrix}i&j&k\\1&0&1\\1&-1&0\end{vmatrix}=\left(1,1,-1\right)^T
$$



$$
设α=\left(2,1,-1\right)^T,β=\left(1,-1,2\right)^T,则\;α×β=\left(\;\;\right)
$$
$$
A.
\left(1,5,-3\right)^T \quad B.\left(1,-5,-3\right)^T \quad C.\left(1,-5,3\right)^T \quad D.\left(-1,-5,-3\right)^T \quad E. \quad F. \quad G. \quad H.
$$
$$
α×β=\begin{vmatrix}i&j&k\\2&1&-1\\1&-1&2\end{vmatrix}=\left(1,-5,-3\right)^T
$$



$$
设α=(a,b,-1)^T,β=(-1,a,b)^T,且α×β//(1,-1,3)^T,\mathrm{则有}(\;\;).
$$
$$
A.
a=-5,b=2 \quad B.a=5,b=-2 \quad C.a=5,b=2 \quad D.a=-5,b=-2 \quad E. \quad F. \quad G. \quad H.
$$
$$
\begin{array}{l}解1:设γ=(1,-1,3)^T,\mathrm{因为}α×β⁄⁄γ,\mathrm{所以}α·γ=0,β·γ=0,即\;\;\;\\\;\;\;\;\;\;\;\;\;\;\;\;\;\;\mathrm{所以}\left\{\begin{array}{l}a-b-3=0\\-1-a+3b=0\end{array}\right.\\\;\;\;\;\;\;\;\;\;\;\;\;\;\;\;\;\;\mathrm{解得}:a=5,b=2\\解2:α×β=\begin{vmatrix}i&j&k\\a&b&-1\\-1&a&b\end{vmatrix}=(b^2+a,-ab+1,a^2+b)^T,又α×β⁄⁄(1,-1,3)^T,\mathrm{故有}\\{\textstyle\frac{b^2+a}1}={\textstyle\frac{-ab+1}{-1}}={\textstyle\frac{a^2+b}3}\mathrm{所以}a=5,b=2\end{array}
$$



$$
设α=\left(1,1,0\right)^T,β=\left(1,0,1\right)^T,\mathrm{则以}α,β\mathrm{为邻边的平行四边形的面积为}(\;).
$$
$$
A.
\sqrt2 \quad B.2 \quad C.\sqrt5 \quad D.\sqrt3 \quad E. \quad F. \quad G. \quad H.
$$
$$
α×β=\begin{vmatrix}i&j&k\\1&1&0\\1&0&1\end{vmatrix}=\left(1,-1,-1\right)^T,\mathrm{所以围得面积为}\sqrt{1^2+\left(-1\right)^2+\left(-1\right)^2}=\sqrt3
$$



$$
设α=(1,b,-1)^T,β=(-1,a,2)^T,且α×β//(1,-1,3)^T,\mathrm{则有}(\;\;).\;
$$
$$
A.
a=-5,b=-2 \quad B.a=5,b=2 \quad C.a=6,b=-2 \quad D.a=5,b=-2 \quad E. \quad F. \quad G. \quad H.
$$
$$
\mathrm{因为}α×β\boldsymbol 与\boldsymbolα\boldsymbol,\boldsymbolβ\mathbf{都垂直}\boldsymbol,\mathbf{所以}\left\{\begin{array}{l}\mathbf1\boldsymbol-\mathbf b\boldsymbol-\mathbf3\boldsymbol=\mathbf0\\\boldsymbol-\mathbf1\boldsymbol-\mathbf a\boldsymbol+\mathbf6\boldsymbol=\mathbf0\end{array}\right.\mathbf{所以a}\boldsymbol=\mathbf5\boldsymbol,\mathbf b\boldsymbol=\boldsymbol-\mathbf2
$$



$$
设α=(2,1,-1)^T,β=(1,-1,2)^T,则\;-2α×2β=\left(\;\;\right)
$$
$$
A.
(-4,-20,12)^T \quad B.(-4,20,-12)^T \quad C.(-4,20,12)^T \quad D.(-4,16,12)^T \quad E. \quad F. \quad G. \quad H.
$$
$$
\begin{array}{l}\;-2α×2β=-4\begin{vmatrix}i&j&k\\2&1&-1\\1&-1&2\end{vmatrix}=(-4,20,12)^T\\\end{array}
$$



$$
\;设\left\|α\right\|=\sqrt2,\left\|β\right\|=\sqrt2,且α·β=1,则\left\|α×β\right\|=\left(\;\;\;\;\right)
$$
$$
A.
\sqrt3 \quad B.\sqrt2 \quad C.1 \quad D.\textstyle\frac{\sqrt3}2 \quad E. \quad F. \quad G. \quad H.
$$
$$
\begin{array}{l}\;\mathrm{因为}\;α·β=\sqrt2×\sqrt2\cosθ,\mathrm{所以cos}θ={\textstyle\frac12}\\\left\|α×β\right\|=\sqrt2×\sqrt2\sinθ=\sqrt3\end{array}
$$



$$
设α,β,γ\mathrm{为非零向量},且α·β=0,α×γ=0,则
$$
$$
A.
α⁄⁄β 且β⟂γ, \quad B.α⟂β 且β⁄⁄γ, \quad C.α⁄⁄γ 且β⟂α, \quad D.α⟂γ 且β⁄⁄γ, \quad E. \quad F. \quad G. \quad H.
$$
$$
α⁄⁄γ 且β⟂α,
$$



$$
设α=\left(1,0,1\right)^T,β=\left(1,-1,0\right)^T,γ=\left(-2,0,1\right)^T,则α×\left(β-γ\right)=\left(\;\;\right)
$$
$$
A.
\left(1,-4,1\right)^T \quad B.\left(-1,4,1\right)^T \quad C.\left(1,4,-1\right)^T \quad D.\left(-1,4,-1\right)^T \quad E. \quad F. \quad G. \quad H.
$$
$$
α×\left(β-γ\right)=\begin{vmatrix}i&j&k\\1&0&1\\3&-1&-1\end{vmatrix}=\left(1,4,-1\right)^T
$$



$$
设α=(1,0,1)^T,β=(1,-1,0)^T,γ=(-2,0,1)^T,则\left(α+γ\right)×β=\left(\;\;\right)
$$
$$
A.
(2,-2,1)^T \quad B.(-2,-2,1)^T \quad C.(-2,2,1)^T \quad D.(2,2,1)^T \quad E. \quad F. \quad G. \quad H.
$$
$$
\left(α+γ\right)×β=\begin{vmatrix}i&j&k\\-1&0&2\\1&-1&0\end{vmatrix}=(2,2,1)^T
$$



$$
设\left\|α\right\|=2,\left\|β\right\|=\sqrt2,且α·β=2,则\left\|α×β\right\|=\left(\;\;\;\right)
$$
$$
A.
4 \quad B.\sqrt2 \quad C.1 \quad D.2 \quad E. \quad F. \quad G. \quad H.
$$
$$
\mathrm{因为}α·β=2,2=2×\sqrt2\cosθ,\mathrm{所以}θ={\textstyle\fracπ4},\mathrm{所以}\left\|α×β\right\|=2×\sqrt2\sinθ=2
$$



$$
设α=(3,-5,8)^T,β=(-1,1,z)^T,\mathrm{满足}\left|\vertα+β\right|\vert=\left|\vertα-β\vert\right|,则\;z=\left(\;\;\right)
$$
$$
A.
1 \quad B.2 \quad C.-1 \quad D.-2 \quad E. \quad F. \quad G. \quad H.
$$
$$
\begin{array}{l}\mathrm{因为}\left|\vertα+β\vert\right|=\left|\vertα-β\right|\vert,α+β=(2,-4,8+z)^T,α-β=(4,-6,8-z)^T\\\mathrm{所以}2^2+\left(-4\right)^2+\left(8+z\right)^2=4^2+\left(-6\right)^2+\left(8-z\right)^2,\mathrm{所以}z=1\end{array}
$$



$$
设α=(a,-1,2)^T,β=(-2,0,b)^T,且α×β\;//(1,2,-1)^T,\mathrm{则有}(\;\;).
$$
$$
A.
a=-4,b=-2 \quad B.a=-4,b=2 \quad C.a=4,b=-2 \quad D.a=4,b=2 \quad E. \quad F. \quad G. \quad H.
$$
$$
α×β=\begin{vmatrix}i&j&k\\a&-1&2\\-2&0&b\end{vmatrix}=(-b,-ab-4,-2)^T,又α×β⁄⁄(1,2,-1)^{T,},\mathrm{故有}{\textstyle\frac{-b}1}={\textstyle\frac{-ab-4}2}={\textstyle\frac{-2}{-1}}⇒ a=4,b=-2
$$



$$
\;设\left\|α\right\|=4,\left\|β\right\|=3,且α·β=6\sqrt3,则\left\|α×β\right\|=\left(\;\;\;\right)
$$
$$
A.
2\sqrt3 \quad B.3 \quad C.4 \quad D.6 \quad E. \quad F. \quad G. \quad H.
$$
$$
\left\|α×β\right\|=4×3×\sinθ,而\cosθ={\textstyle\frac{6\sqrt3}{4×3}}={\textstyle\frac{\sqrt3}2},\mathrm{所以}\left\|α×β\right\|=4×3×{\textstyle\frac12}=6
$$



$$
设α=(2,-1,2)^T,β=(3,0,1)^T,\mathrm{则以}α,β\mathrm{为边的平行四边形的面积为}(\;\;).\;
$$
$$
A.
,3\sqrt3 \quad B.\sqrt5 \quad C.5 \quad D.\sqrt{26} \quad E. \quad F. \quad G. \quad H.
$$
$$
α×β=\begin{vmatrix}i&j&k\\2&-1&2\\3&0&1\end{vmatrix}=(-1,4,3)^T,\mathrm{所以面积为}\left\|α×β\right\|=\sqrt{\left(-1\right)^2+4^2+3^3}=\sqrt{26}
$$



$$
设α=(3,b,-1)^T,β=(-1,a,2)^T,且α×β//(1,-1,1)^T,\mathrm{则有}(\;\;).\;
$$
$$
A.
a=-1,b=-2 \quad B.a=-1,b=2 \quad C.a=1,b=-2 \quad D.a=1,b=2 \quad E. \quad F. \quad G. \quad H.
$$
$$
α×β 与α,β\mathrm{都垂直},\mathrm{因为}\left\{\begin{array}{l}3-b-1=0\\-1-a+2=0\end{array}\right.\mathrm{所以}a=1,b=2
$$



$$
\;设α=\left(1,0,1\right)^T,β=\left(1,-1,0\right)^T,\mathrm{则两向量}α,β\mathrm{的夹角的正弦值为}(\;\;).
$$
$$
A.
\textstyle\frac{\sqrt3}6 \quad B.\textstyle\frac32 \quad C.\textstyle\frac{\sqrt3}4 \quad D.\textstyle\frac{\sqrt3}2 \quad E. \quad F. \quad G. \quad H.
$$
$$
\mathrm{因为}\left\|α×β\right\|=\left\|α\right\|·\left\|β\right\|\sinθ,\mathrm{因为}\begin{vmatrix}i&j&k\\1&0&1\\1&-1&0\end{vmatrix}=\left(1,1,-1\right)^T,\mathrm{所以sin}θ={\textstyle\frac{\sqrt3}2}
$$



$$
设α=(1,0,-1)^T,β=(2,2,1)^T,γ=(1,1,-1)^T,则\;2α×\left(β-γ\right)=\left(\;\;\;\;\right)
$$
$$
A.
(2,6,2)^T \quad B.(-2,-6,-2)^T \quad C.(-2,-6,2)^T \quad D.(2,-6,2)^T \quad E. \quad F. \quad G. \quad H.
$$
$$
\;2α×\left(β-γ\right)=2\begin{vmatrix}i&j&k\\1&0&-1\\1&1&2\end{vmatrix}=(2,-6,2)^T
$$



$$
设α=(1,a,-1)^T,β=(0,1,2)^T,\mathrm{若以}α,β\mathrm{为边的平行四边形的面积为}\sqrt{54},则(\;\;).\;
$$
$$
A.
a=3 \quad B.a=-4或a=3 \quad C.a=-4 \quad D.a=-3 \quad E. \quad F. \quad G. \quad H.
$$
$$
\begin{array}{l}\mathrm{因为}α×β=\begin{vmatrix}i&j&k\\1&a&-1\\0&1&2\end{vmatrix}=(2a+1,-2,1)^T\\\left(2a+1\right)^2+\left(-2\right)^2+1^2=54\\\mathrm{所以}a=-4或a=3\end{array}
$$



$$
\mathrm{设向量}α=\left(1,1,-1\right)^T,β=\left(-2,-1,2\right)^T,\mathrm{有一向量}γ=\left(2,a,b\right)^T,与α 和β\mathrm{都正交},则a=\left(\;\;\;\right)
$$
$$
A.
1 \quad B.2 \quad C.0 \quad D.3 \quad E. \quad F. \quad G. \quad H.
$$
$$
\begin{array}{l}γ=\left(2,a,b\right)^T与α 和β\mathrm{都正交},\mathrm{故有}\left\langleγ,α\right\rangle=\left\langleγ,β\right\rangle=0,即\\\left\{\begin{array}{l}2+a-b=0\\-4-a+2b=0\end{array}\right.⇒ a=0,b=2\end{array}
$$



$$
\mathrm{设向量}α=(1,1,-1)^T,β=(-2,-1,2)^T,\mathrm{有一向量}γ=(2,a,b)^T,与α 和β\mathrm{都正交},则\;b=\left(\;\;\;\right)
$$
$$
A.
1 \quad B.2 \quad C.3 \quad D.0 \quad E. \quad F. \quad G. \quad H.
$$
$$
\begin{array}{l}γ=(2,a,b)^T与α 和β\mathrm{都正交},\mathrm{故有}\left\langleγ,α\right\rangle=\left\langleγ,β\right\rangle=0,即\\\left\{\begin{array}{l}2+a-b=0\\-4-a+2b=0\end{array}\right.⇒ a=0,b=2\end{array}
$$



$$
\mathrm{设矩阵}A=\begin{pmatrix}\textstyle\frac23&\textstyle\frac1{\sqrt2}&\textstyle\frac1{\sqrt{18}}\\a&b&\textstyle\frac{-4}{\sqrt{18}}\\\textstyle\frac23&\textstyle\frac{-1}{\sqrt2}&\textstyle\frac1{\sqrt{18}}\end{pmatrix}\mathrm{为正交矩阵},则a,b\mathrm{分别为}\left(\;\;\;\;\right)
$$
$$
A.
a={\textstyle\frac13},b=0 \quad B.a={\textstyle±\frac13},b=0 \quad C.a={\textstyle0},b={\textstyle\frac1{\sqrt2}} \quad D.a={\textstyle0},b=±{\textstyle\frac1{\sqrt2}} \quad E. \quad F. \quad G. \quad H.
$$
$$
\mathrm{正交矩阵的列向量是正交的单位向量组},\mathrm{因此}\left({\textstyle\frac23}\right)^2+a^2+\left({\textstyle\frac23}\right)^2=1,\left({\textstyle\frac1{\sqrt2}}\right)^2+b^2+\left({\textstyle-\frac1{\sqrt2}}\right)^2=1,\mathrm{且第一列向量与第三列向量正交},则a={\textstyle\frac13},b=0.
$$



$$
\mathrm{与向量}α_1=\left(1,0,1\right)^T,α_2=\left(0,1,1\right)^T\mathrm{等价的标准正交向量组是}(\;\;\;\;)
$$
$$
A.
\textstyle\frac1{\sqrt2}\left(1,0,1\right)^T,\frac1{\sqrt6}\left(-1,2,1\right)^T \quad B.\textstyle\frac1{\sqrt2}\left(-1,0,1\right)^T,\frac1{\sqrt6}\left(-1,2,1\right)^T \quad C.\textstyle\frac1{\sqrt2}\left(1,0,-1\right)^T,\frac1{\sqrt6}\left(-1,2,1\right)^T \quad D.\textstyle\frac1{\sqrt2}\left(1,0,1\right)^T,\frac1{\sqrt6}\left(1,-2,1\right)^T \quad E. \quad F. \quad G. \quad H.
$$
$$
\begin{array}{l}\mathrm{由施密特正交化方法},\mathrm{先正交化},\mathrm{即令}β_1=α_1=(1,0,1)^T\;,\\β_2=α_2-{\textstyle\frac{\left\langleα_2,β_1\right\rangle}{\left\langleβ_1,β_1\right\rangle}}β_1=(0,1,1)^T-{\textstyle\frac12}(1,0,1)^T={\textstyle\frac12}{\textstyle(}{\textstyle-}{\textstyle1}{\textstyle,}{\textstyle2}{\textstyle,}{\textstyle1}{\textstyle)}{\textstyle{}^T}{\textstyle,}\\\mathrm{再将其单位化得},η_1={\textstyle\frac{β_1}{\left\|β_1\right\|}}={\textstyle\frac1{\sqrt2}}(1,0,1)^T,η_2={\textstyle\frac{β_2}{\left\|β_2\right\|}}={\textstyle\frac1{\sqrt6}}(-1,2,1)^T\end{array}
$$



$$
\mathrm{已知}A=\begin{pmatrix}x&\textstyle\frac12\\\textstyle\frac12&y\end{pmatrix}\mathrm{是正交矩阵},且x>0,则\begin{pmatrix}x\\y\end{pmatrix}=\left(\;\;\right)
$$
$$
A.
x={\textstyle\frac{\sqrt3}2},y=-{\textstyle\frac{\sqrt3}2} \quad B.x=-{\textstyle\frac{\sqrt3}2},y={\textstyle\frac{\sqrt3}2} \quad C.x={\textstyle\frac12},y={\textstyle\frac{\sqrt3}2} \quad D.x={\textstyle\frac12},y=-{\textstyle\frac{\sqrt3}2} \quad E. \quad F. \quad G. \quad H.
$$
$$
\begin{array}{l}\mathrm{由于}A\mathrm{是正交矩阵},AA^T=A^TA=E,即\;\;\\\;\;\;\;\;\;\;\;\;\;\;\begin{pmatrix}x&\textstyle\frac12\\\textstyle\frac12&y\end{pmatrix}\begin{pmatrix}x&\textstyle\frac12\\\textstyle\frac12&y\end{pmatrix}=\begin{pmatrix}1&\textstyle0\\0&\textstyle1\end{pmatrix}⇒\left\{\begin{array}{l}\begin{array}{c}x^2+{\textstyle\frac14}=1\\{\textstyle\frac12}x+{\textstyle\frac12}y=0\end{array}\\{\textstyle\frac14}\begin{array}{c}+y^2=1\end{array}\end{array}\right.,\\\;\;\mathrm{所以}\left\{\begin{array}{l}χ={\textstyle\frac{\sqrt3}2}\\y=-{\textstyle\frac{\sqrt3}2}\end{array}\right.;或\begin{array}{l}χ={\textstyle\frac{-\sqrt3}2}\\y={\textstyle\frac{\sqrt3}2}\end{array}(\mathrm{舍去}).\end{array}
$$



$$
设A=\begin{pmatrix}1&4&8\\4&7&-4\\8&-4&1\end{pmatrix},若B=kA\mathrm{为正交矩阵},则\;k\mathrm{的值为}(\;\;\;\;\;).
$$
$$
A.
±9 \quad B.9 \quad C.±{\textstyle\frac19} \quad D.\textstyle\frac19 \quad E. \quad F. \quad G. \quad H.
$$
$$
\begin{array}{l}\mathrm{由于矩阵}A\mathrm{的列向量相互正交},且\sqrt{1^2+4^2+8^2}=9,\sqrt{4^2+7^2+\left(-4\right)^2}=9,取k=±{\textstyle\frac19},则\\B=kA=±{\textstyle\frac19}\begin{pmatrix}1&4&8\\4&7&-4\\8&-4&1\end{pmatrix}\mathrm{为正交}\end{array}
$$



$$
\mathrm{与向量}α_1=\left(1,1,1\right)^T,α_2=\left(0,1,1\right)^T,\mathrm{等价的标准正交向量组是}(\;\;\;\;)\;
$$
$$
A.
{\textstyle\frac1{\sqrt2}}\left(1,1,1\right)^T,\;\;\frac1{\sqrt6}\left(-1,2,1\right)^T \quad B.\frac1{\sqrt3}\left(1,1,1\right)^T,\;\;\frac1{\sqrt6}\left(-2,1,1\right)^T \quad C.\frac1{\sqrt3}\left(1,1,1\right)^T,\;\;\frac1{\sqrt2}\left(0,1,1\right)^T \quad D.\frac1{\sqrt3}\left(1,1,1\right)^T,\;\;\frac1{\sqrt3}\left(0,-1,1\right)^T \quad E. \quad F. \quad G. \quad H.
$$
$$
\begin{array}{l}\mathrm{由施密特正交化方法},\mathrm{先正交化},\mathrm{即令}\;β_1=α_1=\left(1,1,1\right)^T,\;β_2=α_2-{\textstyle\frac{\left\langle\;α_2,\;β_1\right\rangle}{\left\langleβ_1,\;β_1\right\rangle}}β_1={\textstyle\frac13}\left(-2,1,1\right)^T,\\\mathrm{再将其单位化得},η_1={\textstyle\frac{β_1}{\left\|β_1\right\|}}={\textstyle\frac1{\sqrt3}}\left(1,1,1\right)^T,η_2={\textstyle\frac{β_2}{\left\|β_2\right\|}}={\textstyle\frac1{\sqrt6}}\left(-2,1,1\right)^T\end{array}
$$



$$
\mathrm{与向量组}α_1=\left(1,1,1\right)^T,α_2=\left(1,2,3\right)^T,α_3=\left(1,4,9\right)^T,\mathrm{都等价的标准正交向量组是}(\;\;\;)
$$
$$
A.
\textstyle\frac1{\sqrt3}\left(1,1,1\right)^T,\frac1{\sqrt2}\left(1,0,-1\right)^T,\frac1{\sqrt6}\left(1,-2,1\right)^T \quad B.\textstyle\frac1{\sqrt3}\left(1,1,1\right)^T,\frac1{\sqrt2}\left(-1,0,1\right)^T,\frac1{\sqrt6}\left(1,-2,-1\right)^T \quad C.\textstyle\frac1{\sqrt3}\left(1,1,1\right)^T,\frac1{\sqrt2}\left(-1,0,1\right)^T,\frac1{\sqrt6}\left(1,-2,1\right)^T \quad D.\textstyle\frac1{\sqrt3}\left(1,1,1\right)^T,\frac1{\sqrt2}\left(-1,0,1\right)^T,\frac1{\sqrt6}\left(-1,2,1\right)^T \quad E. \quad F. \quad G. \quad H.
$$
$$
\begin{array}{l}\mathrm{由施密特正交化方法},\mathrm{先正交化},\mathrm{即令}β_1=α_1=\left(1,1,1\right)^T,\\β_2=α_2-{\textstyle\frac{\left\langleα_2,β_1\right\rangle}{\left\langleβ_1,β_1\right\rangle}}β_1=\left(1,2,3\right)^T-{\textstyle\frac63}\left(1,1,1\right)^T=\left(-1,0,1\right)^T,\\β_3=α_3-{\textstyle\frac{\left\langleα_3,β_1\right\rangle}{\left\langleβ_1,β_1\right\rangle}}β_1-\frac{\left\langleα_3,β_2\right\rangle}{\left\langleβ_2,β_2\right\rangle}β_2=\left(1,4,9\right)^T-{\textstyle\frac{14}3}\left(1,1,1\right)^T-{\textstyle4}\left(-1,0,1\right)^T={\textstyle\frac13}\left(1,-2,1\right)^T,\\\mathrm{再将其单位化得},η_1={\textstyle\frac{β_1}{\left\|β_1\right\|}}={\textstyle\frac1{\sqrt3}}\left(1,1,1\right)^T,η_2={\textstyle\frac{β_2}{\left\|β_2\right\|}}={\textstyle\frac1{\sqrt2}}\left(-1,0,1\right)^T,\\η_3={\textstyle\frac{β_3}{\left\|β_3\right\|}}={\textstyle\frac1{\sqrt6}}\left(1,-2,1\right)^T\end{array}
$$



$$
设A=\begin{pmatrix}2&3&6\\3&-6&2\\6&2&-3\end{pmatrix},若B=kA\mathrm{为一正交矩阵},则k\mathrm{的值为}(\;\;\;\;).
$$
$$
A.
7 \quad B.±{\textstyle\frac17} \quad C.±7 \quad D.\textstyle\frac17 \quad E. \quad F. \quad G. \quad H.
$$
$$
取k=±{\textstyle\frac1{\sqrt{2^2+3^2+6^2}}}=±{\textstyle\frac17},记B=±{\textstyle\frac17}\begin{pmatrix}2&3&6\\3&-6&2\\6&2&-3\end{pmatrix},则B\mathrm{为正交阵}.
$$



$$
\mathrm{设矩阵}A={\textstyle\frac12}\begin{pmatrix}1&2a&1\\-1&\sqrt2&2b\\\sqrt2&2c&-\sqrt2\end{pmatrix}\mathrm{为正交矩阵},则a,b,c\mathrm{的值分别为}(\;\;\;\;).
$$
$$
A.
a={\textstyle\frac1{\sqrt2}},b=-{\textstyle\frac12},c=0 \quad B.a={\textstyle0},b=-{\textstyle\frac12},c={\textstyle\frac12} \quad C.a={\textstyle\frac1{\sqrt2}},b={\textstyle0},c={\textstyle\frac12} \quad D.a=-{\textstyle\frac1{\sqrt2}},b=-{\textstyle\frac12},c=1 \quad E. \quad F. \quad G. \quad H.
$$
$$
\begin{array}{l}A\mathrm{为正交矩阵},A\mathrm{的列向量组相互正交},由\;\;\;\\\;\;\;\;\;\;\;\;\;\;\;\;\;\;\;\;\;\;\;\;\;\;\;\;\;\;\;\;\;\;\;\;\;\;\;\;\;\;\;\;\;\;\;\;\;\;\;\;\;\;\;\;\;\;\;\;\;\;\;\;\;\;\;\;\;\;\;\;\;\;\;\;\;\;\;\;\;\left\{\begin{array}{l}\begin{array}{c}2a-\sqrt2+2\sqrt2c=0\\1-2b-2=0\end{array}\\\begin{array}{c}\mathbf2\boldsymbol a\boldsymbol+\mathbf2\sqrt{\mathbf2}\boldsymbol b\boldsymbol-\mathbf2\sqrt{\mathbf2}\boldsymbol c\boldsymbol=\mathbf0\end{array}\end{array}\right.\;\;\;\;\;\;\;\;\;\;\;\;\;\;\;\;\;\;\;\;\;\;\;\;\;\;\;\;\;\;\;\;\;\;\;\;\;\;\;\;\;\;\;\;\;\;\;\;\\\\\mathrm{可得}a={\textstyle\frac1{\sqrt2}},b=-{\textstyle\frac12},c=0.\mathrm{此时}A\mathrm{的各列为单位向量},故A\mathrm{为正交矩阵}.\end{array}
$$



$$
\mathrm{已知}A=\begin{pmatrix}\textstyle\frac{-2}{\sqrt5}&\textstyle\frac2{3\sqrt5}&χ\\\textstyle\frac1{\sqrt5}&\textstyle\frac4{3\sqrt5}&\textstyle\frac23\\0&\textstyle\frac5{3\sqrt5}&-{\textstyle\frac23}\end{pmatrix}\mathrm{是正交矩阵},则χ=\left(\;\;\;\right)
$$
$$
A.
\textstyle\frac13 \quad B.\textstyle-\frac13 \quad C.\textstyle\frac23 \quad D.\textstyle-\frac23 \quad E. \quad F. \quad G. \quad H.
$$
$$
\begin{array}{l}\mathrm{正交矩阵的列向量组是单位正交向量组},\mathrm{考察矩阵的第一列和第三列}:\;\\\;\;\;\;\;-{\textstyle\frac2{\sqrt5}}χ+{\textstyle\frac1{\sqrt5}}×{\textstyle\frac23}=0⇒χ={\textstyle\frac13}\;.\end{array}
$$



$$
\mathrm{下列矩阵中不是正交矩阵的是}(\;\;\;\;).
$$
$$
A.
\begin{pmatrix}0&1&0\\\textstyle\frac1{\sqrt2}&0&\textstyle\frac1{\sqrt2}\\\textstyle\frac1{\sqrt2}&0&\textstyle-\frac1{\sqrt2}\end{pmatrix} \quad B.\begin{pmatrix}\textstyle\frac1{\sqrt3}&\textstyle\frac1{\sqrt3}&\textstyle\frac1{\sqrt3}\\-{\textstyle\frac2{\sqrt6}}&\textstyle\frac1{\sqrt6}&\textstyle\frac1{\sqrt6}\\0&-\frac1{\sqrt2}&\textstyle\frac1{\sqrt2}\end{pmatrix} \quad C.\begin{pmatrix}1&-{\textstyle\frac12}&\textstyle\frac13\\-{\textstyle\frac12}&1&\textstyle\frac12\\\textstyle\frac13&\textstyle\frac12&-1\end{pmatrix} \quad D.\begin{pmatrix}\textstyle\frac19&-{\textstyle\frac89}&-{\textstyle\frac49}\\-{\textstyle\frac89}&\textstyle\frac19&-{\textstyle\frac49}\\-{\textstyle\frac49}&-{\textstyle\frac49}&\textstyle\frac79\end{pmatrix} \quad E. \quad F. \quad G. \quad H.
$$
$$
\mathrm{矩阵}\begin{pmatrix}1&-{\textstyle\frac12}&\textstyle\frac13\\-{\textstyle\frac12}&1&\textstyle\frac12\\\textstyle\frac13&\textstyle\frac12&-1\end{pmatrix}\mathrm{中第一个列向量非单位向量},\mathrm{故不是正交阵}.
$$



$$
设A是n\mathrm{阶正交矩阵},且\left|A\right|<0,则\left|A+E\right|=\left(\;\;\right).
$$
$$
A.
0 \quad B.1 \quad C.-1 \quad D.±1 \quad E. \quad F. \quad G. \quad H.
$$
$$
\begin{array}{l}\;\left|A+E\right|=\left|A+AA^T\right|=\left|A\left(E+A^T\right)\right|=\left|A\right|\left|\left(E+A^T\right)\right|=\left|A\right|\left|E+A\right|\\⇒\left(1-\left|A\right|\right)\left|A+E\right|=0⇒\left|A+E\right|=0\left(因1-\left|A\right|>0\right).\end{array}
$$



$$
设A=\begin{pmatrix}1&2&2&0\\2&-1&0&2\\2&0&-1&-2\\0&2&-2&1\end{pmatrix},若B=kA\mathrm{为正交阵},则k=\left(\;\;\;\right)
$$
$$
A.
\textstyle\frac13 \quad B.-{\textstyle\frac13} \quad C.±{\textstyle\frac13} \quad D.±3 \quad E. \quad F. \quad G. \quad H.
$$
$$
\mathrm{矩阵}A\mathrm{的列向量相互正交},\mathrm{当取}k=±{\textstyle\frac1{\sqrt{1+4+4}}}=±{\textstyle\frac13},即B=±{\textstyle\frac13}A时B\mathrm{是正交阵}
$$



$$
\begin{array}{l}\mathrm{下列矩阵为正交矩阵的是}\left(\;\;\;\;\right).\;\\\left(1\right)\;\begin{pmatrix}1&-{\textstyle\frac12}&\textstyle\frac13\\-{\textstyle\frac12}&1&\textstyle\frac12\\\textstyle\frac13&\textstyle\frac12&-1\end{pmatrix};\;\;\left(2\right)\;\begin{pmatrix}\textstyle\frac19&-{\textstyle\frac89}&\textstyle-\frac49\\-{\textstyle\frac89}&\textstyle\frac19&\textstyle-\frac49\\\textstyle-\frac49&\textstyle-\frac49&\textstyle\frac79\end{pmatrix}\;\;\;\;\;\;\;.\end{array}
$$
$$
A.
\left(1\right)\; \quad B.\left(2\right)\; \quad C.\left(1\right)\;\left(2\right)\; \quad D.\mathrm{都不是} \quad E. \quad F. \quad G. \quad H.
$$
$$
\begin{array}{l}(1)\mathrm{考察矩阵的第一列和第二列},\\\;\mathrm{因为}\;\;\;1×\left(-{\textstyle\frac12}\right)+\left(-{\textstyle\frac12}\right)×1+{\textstyle\frac13}×{\textstyle\frac12}\neq0,\;\;\;\;\;\;\\\mathrm{所以它不是正交矩阵};\;\\(2)\mathrm{由正交矩阵的定义},\\\mathrm{因为}\;\;\begin{pmatrix}\textstyle\frac19&-{\textstyle\frac89}&\textstyle-\frac49\\-{\textstyle\frac89}&\textstyle\frac19&\textstyle-\frac49\\\textstyle-\frac49&\textstyle-\frac49&\textstyle\frac79\end{pmatrix}\;\begin{pmatrix}\textstyle\frac19&-{\textstyle\frac89}&\textstyle-\frac49\\-{\textstyle\frac89}&\textstyle\frac19&\textstyle-\frac49\\\textstyle-\frac49&\textstyle-\frac49&\textstyle\frac79\end{pmatrix}^T=\begin{pmatrix}1&0&0\\0&1&0\\0&0&1\end{pmatrix}\;,\;\;\;\;\;\;\;\\\mathrm{所以它是正交矩阵}.\end{array}
$$



$$
设A\mathrm{为正交矩阵},α_i,α_j\mathrm{分别是}A\mathrm{的第}i,j列\left(i\neq j\right),则\left\langleα_i,α_j\right\rangle 和\left\langleα_j,α_j\right\rangle\mathrm{分别为}\left(\;\;\;\right)
$$
$$
A.
0;1 \quad B.0;0 \quad C.1;1 \quad D.1;0 \quad E. \quad F. \quad G. \quad H.
$$
$$
\begin{array}{l}\mathrm{正交矩阵的列向量都是单位正交向量组},\mathrm{则不同列的向量正交},即\left\langleα_i,α_j\right\rangle=0;\;\;\\\mathrm{又由于列向量为单位向量},则\left\langleα_j,α_j\right\rangle=\left\|α_j\right\|^2=1\end{array}
$$



$$
\begin{array}{l}\mathrm{已知}α_1=\left({\textstyle\frac1{\sqrt3}},{\textstyle\frac1{\sqrt3}},{\textstyle\frac1{\sqrt3}}\right),α_2=\left({\textstyle-\frac1{\sqrt2},\frac1{\sqrt2},0}\right),α_3=\left({\textstyle-\frac1{\sqrt6},-\frac1{\sqrt6},\frac2{\sqrt6}}\right)是R^3\mathrm{的一个标准正交基},\\\mathrm{若用这个基来线性表示}R^3\mathrm{中的向量}α=\left({\textstyle1,-1,-1}\right),则α=\left(\;\;\;\right)\end{array}
$$
$$
A.
-{\textstyle\frac1{\sqrt3}}α_1-\sqrt2α_2-{\textstyle\frac2{\sqrt6}}α_3 \quad B.{\textstyle\frac1{\sqrt3}}α_1-\sqrt2α_2-{\textstyle\frac2{\sqrt6}}α_3 \quad C.-{\textstyle\frac1{\sqrt3}}α_1+\sqrt2α_2+{\textstyle\frac2{\sqrt6}}α_3 \quad D.{\textstyle\frac1{\sqrt3}}α_1+\sqrt2α_2-{\textstyle\frac2{\sqrt6}}α_3 \quad E. \quad F. \quad G. \quad H.
$$
$$
\begin{array}{l}设κ_1α_1+κ_2α_2+κ_3α_3=α,\mathrm{则对矩阵}\left(α_1,α_2,α_3,α\right)\mathrm{进行初等行变换},得\;\\\;\;\begin{pmatrix}\textstyle\frac1{\sqrt3}&-{\textstyle\frac1{\sqrt2}}&-{\textstyle\frac1{\sqrt6}}&1\\\textstyle\frac1{\sqrt3}&\textstyle\frac1{\sqrt2}&-{\textstyle\frac1{\sqrt6}}&-1\\\textstyle\frac1{\sqrt3}&0&\textstyle\frac2{\sqrt6}&-1\end{pmatrix}\rightarrow\begin{pmatrix}\textstyle\frac1{\sqrt3}&-{\textstyle\frac1{\sqrt2}}&-{\textstyle\frac1{\sqrt6}}&1\\\textstyle0&\textstyle\frac2{\sqrt2}&0&-2\\\textstyle0&\textstyle\frac1{\sqrt2}&\textstyle\frac3{\sqrt6}&-2\end{pmatrix}\rightarrow\begin{pmatrix}\textstyle\frac1{\sqrt3}&0&0&\textstyle-\frac13\\\textstyle0&\textstyle\frac1{\sqrt2}&0&-1\\\textstyle0&\textstyle0&\textstyle\frac3{\sqrt6}&-1\end{pmatrix}\rightarrow\begin{pmatrix}\textstyle1&0&0&\textstyle-\frac1{\sqrt3}\\\textstyle0&\textstyle1&0&-\sqrt2\\\textstyle0&\textstyle0&\textstyle1&-{\textstyle\frac2{\sqrt6}}\end{pmatrix},\;\\\;\mathrm{则由最后的矩阵可知}α=-{\textstyle\frac1{\sqrt3}}α_1-\sqrt2α_2-{\textstyle\frac2{\sqrt6}}α_3.\end{array}
$$



$$
若α_1=(\frac23,{\textstyle\frac13}{\textstyle,}{\textstyle\frac23})^T,α_2=({\textstyle\frac13}{\textstyle,}{\textstyle\frac23}{\textstyle,}{\textstyle-}{\textstyle\frac23})^T,及α_3\mathrm{是两两正交的单位向量组},则α_3=\left(\;\;\;\right)
$$
$$
A.
±(-{\textstyle\frac23},{\textstyle\frac23},{\textstyle\frac13})^T \quad B.(\boldsymbol-{\textstyle\frac{\mathbf2}{\mathbf3}}\boldsymbol,{\textstyle\frac{\mathbf2}{\mathbf3}}\boldsymbol,{\textstyle\frac{\mathbf1}{\mathbf3}})^\mathbf T \quad C.(\frac23,-{\textstyle\frac23},-{\textstyle\frac13})^T \quad D.±(-{\textstyle\frac23},{\textstyle\frac13},{\textstyle\frac23})^T \quad E. \quad F. \quad G. \quad H.
$$
$$
\begin{array}{l}设α_3=(x_1,x_2,x_3)^T,\mathrm{由条件可知}α_1^Tα_3=0,α_2^Tα_3=0即\;\;\\\;\;\;\;\;\;\;\;\;\;\;\;\;\;\;\;\;\;\;\;\;\;\;\;\;\;\;\;\;\;\;\;\;\;\;\;\;\;\;\;\;\;\;\;\;\;\;\;\;\;\;\;\;\;\;\;\;\;\;\;\;\;\;\;\;\;\;\;\;\;\;\;\;\;\;\;\;\;\;\;\;\;\;\;\;\;\;\;\;\;\;\;\;\;\;\left\{\begin{array}{l}{\textstyle\frac23}x_1+{\textstyle\frac13}{\textstyle x}{\textstyle{}_2}{\textstyle+}{\textstyle\frac23}{\textstyle x}{\textstyle{}_3}{\textstyle=}{\textstyle0}\\\textstyle\frac13x_1+\frac23x_2-\frac23x_3=0\end{array}\right.,\;\;\\令x{\textstyle{}_3}{\textstyle=}{\textstyle k},\mathrm{解得}α_3=k\left(-2,2,1\right)^T,又\left\|α_3\right\|=1⇒ k^2\left(4+4+1\right)=1⇒ k=±{\textstyle\frac13},则\\α_3=±(-{\textstyle\frac23},{\textstyle\frac23},{\textstyle\frac13})^T\end{array}
$$



$$
\mathrm{已知两个单位正交向量}ζ_1=(\frac19,-{\textstyle\frac89},-{\textstyle\frac49})^T,ζ_2=({\textstyle-}{\textstyle\frac89}{\textstyle,}{\textstyle\frac19}{\textstyle,}{\textstyle-}{\textstyle\frac49})^T,以ζ_1,ζ_2,ζ_3\mathrm{为列向量构成的矩阵}\mathbb{Q}\mathrm{是正交矩阵},\mathrm{则列向量}ζ_3为(\;\;\;).
$$
$$
A.
±(\frac49,{\textstyle\frac49},{\textstyle\frac79})^T \quad B.({\textstyle\frac49}{\textstyle,}{\textstyle\frac49}{\textstyle,}{\textstyle\frac79})^T \quad C.({\textstyle\frac49}{\textstyle,}{\textstyle\frac49}{\textstyle,}{\textstyle-}{\textstyle\frac79})^T \quad D.±({\textstyle\frac49}{\textstyle,}{\textstyle\frac49}{\textstyle,}{\textstyle-}{\textstyle\frac79})^T \quad E. \quad F. \quad G. \quad H.
$$
$$
\begin{array}{l}设ζ_3=\left(x_1,x_2,x_3\right),\mathrm{则由}ζ_1·ζ_3=0,ζ_2·ζ_3=0,得\;\;\\\left\{\begin{array}{l}\textstyle\frac19x_1-\frac89x_2-\frac49x_3=0\\-{\textstyle\frac89}{\textstyle x}{\textstyle{}_1}{\textstyle+}{\textstyle\frac19}{\textstyle x}{\textstyle{}_2}{\textstyle-}{\textstyle\frac49}{\textstyle x}{\textstyle{}_3}{\textstyle=}{\textstyle0}\end{array}\right.,\;\;\\\mathrm{解得}x{\textstyle{}_1}\;=-{\textstyle\frac47}x_3,x{\textstyle{}_2}\;=-{\textstyle\frac47}x_3\;.\\\mathrm{代入}\left|ζ_3\right|^2=x_1^2+x_2^2+x_3^2=1,得x_3=±{\textstyle\frac79},\mathrm{所以}ζ_3=±({\textstyle\frac49}{\textstyle,}{\textstyle\frac49}{\textstyle,}{\textstyle-}{\textstyle\frac79})^T.\end{array}
$$



$$
\mathrm{已知}A=\begin{pmatrix}\textstyle\frac1{\sqrt3}&0&χ\\\textstyle\frac1{\sqrt3}&\textstyle\frac{-1}{\sqrt2}&\textstyle\frac1{\sqrt6}\\\textstyle\frac1{\sqrt3}&\textstyle\frac1{\sqrt2}&\textstyle\frac1{\sqrt6}\end{pmatrix}^T\mathrm{是正交矩阵},则χ=\left(\;\;\;\right)
$$
$$
A.
-{\textstyle\frac2{\sqrt6}} \quad B.\textstyle\frac2{\sqrt6} \quad C.\textstyle±\frac2{\sqrt6} \quad D.0 \quad E. \quad F. \quad G. \quad H.
$$
$$
\begin{array}{l}\mathrm{正交矩阵的列向量相互正交},\mathrm{考察矩阵的第一列和第三列},因\;\;\\\;\;\;\;\;\;\;\;\;\;{\textstyle\frac1{\sqrt3}}χ+\;{\textstyle\frac1{\sqrt3}}×\;{\textstyle\frac1{\sqrt6}}+\;{\textstyle\frac1{\sqrt3}}×\;{\textstyle\frac1{\sqrt6}}=0⇒χ=-{\textstyle\frac2{\sqrt6}}.\end{array}
$$



$$
\mathrm{如果}α_1=\left(1,0,0\right)^T,α_2=\left(0,{\textstyle\frac1{\sqrt2}},{\textstyle\frac1{\sqrt2}}\right)^T及α_3是R^3\mathrm{中两两正交的单位向量组},则\;α_3=\left(\;\;\;\;\right)
$$
$$
A.
±(0,-{\textstyle\frac1{\sqrt2}},{\textstyle\frac1{\sqrt2}})^T \quad B.±(\frac1{\sqrt2},-{\textstyle\frac1{\sqrt2}},{\textstyle\frac1{\sqrt2}})^T \quad C.±(-\frac1{\sqrt2},0,{\textstyle\frac1{\sqrt2}})^T \quad D.±(\frac1{\sqrt2},-{\textstyle\frac1{\sqrt2}},{\textstyle0})^T \quad E. \quad F. \quad G. \quad H.
$$
$$
\begin{array}{l}设α_3=(x_1,x_2,x_3)^T,\mathrm{则由条件可知}α_1^Tα_3=0,α_2^Tα_3=0,即\;\;\\\;\;\;\;\;\;\;\;\;\;\;\;\;\;\;\;\;\;\;\;\;\;\;\;\;\;\;\;\;\;\;\;\;\;\;\;\;\;\;\;\;\;\;\;\;\;\;\;\;\;\;\;\;\;\left\{\begin{array}{l}x_1=0\\{\textstyle\frac1{\sqrt2}}x_2+{\textstyle\frac1{\sqrt2}}x{\textstyle{}_3}{\textstyle=}{\textstyle0}\end{array}\right.,\mathrm{解之得}α_3=k(0,-1,1)^T,\mathrm{其中}k∈ R,\;\;\\由α_3\mathrm{为单位向量},则\left\|α_3\right\|=1⇒ k=±{\textstyle\frac1{\sqrt2}},则α_3=±(0,-{\textstyle\frac1{\sqrt2}},{\textstyle\frac1{\sqrt2}})^T.\end{array}
$$



$$
\mathrm{设方阵}A\mathrm{为正交矩阵},且A^T=A^*,\mathrm{其中}A^* 是A\mathrm{的伴随矩阵},则A\mathrm{的行列式等于}\left(\;\;\;\;\right)
$$
$$
A.
1 \quad B.-1 \quad C.0 \quad D.±1 \quad E. \quad F. \quad G. \quad H.
$$
$$
\begin{array}{l}A\mathrm{为正交矩阵},则AA^T=E,又A^T=A^*,则AA^*=E;\;\;\\\mathrm{由伴随矩阵的性质可知}AA^*=\left|A\right|E,即\left|\mathbf A\right|\boldsymbol E\boldsymbol=\boldsymbol E\boldsymbol⇒\left|\mathbf A\right|\boldsymbol=\mathbf1.\end{array}
$$



$$
设α=(0,y,-{\textstyle\frac1{\sqrt2}})^T,β=(x,0,0)^T,若α,β\mathrm{是标准正交向量组},则\left(\;\;\;\;\right)
$$
$$
A.
x\mathrm{任意},y=-{\textstyle\frac1{\sqrt2}} \quad B.x\mathrm{任意},y={\textstyle\frac12} \quad C.x=±1,y=±{\textstyle\frac1{\sqrt2}} \quad D.x=1,y=-{\textstyle\frac1{\sqrt2}} \quad E. \quad F. \quad G. \quad H.
$$
$$
\mathrm{标准正交向量组中的向量必须是单位向量及向量长度为}1,\mathrm{且两两正交},\mathrm{则可得}x=±1,y=±{\textstyle\frac1{\sqrt2}},
$$



$$
\begin{array}{l}\mathrm{下列两个矩阵中是正交矩阵的为}\left(\;\;\;\right)\\\left(1\right)\;\;\begin{pmatrix}3&-3&1\\-3&1&3\\1&3&-3\end{pmatrix};\;\;\;\;\left(2\right)\;\;\begin{pmatrix}\textstyle\frac23&\textstyle\frac23&\textstyle\frac13\\\textstyle\frac23&-{\textstyle\frac13}&-{\textstyle\frac23}\\\textstyle\frac13&-{\textstyle\frac23}&\textstyle\frac{\mathbf2}{\mathbf3}\end{pmatrix}\;\;\;\;\;\;\;\;\;.\end{array}
$$
$$
A.
\left(1\right) \quad B.\left(2\right) \quad C.\left(1\right)\left(2\right) \quad D.\mathrm{两个都不是} \quad E. \quad F. \quad G. \quad H.
$$
$$
\begin{array}{l}\left(1\right)\mathrm{第一个行向量非单位向量},\mathrm{故不是正交矩阵};\;\\\left(2\right)\mathrm{该方阵每一个行向量均是单位向量},\mathrm{且两两正交},\mathrm{故为正交矩阵}.\end{array}
$$



$$
α_1,α_2,α_3\mathrm{是一个标准正交组},则\left\|4α_1-7α_2+4α_3\right\|=\left(\;\;\right).
$$
$$
A.
9 \quad B.81 \quad C.36 \quad D.18 \quad E. \quad F. \quad G. \quad H.
$$
$$
\begin{array}{l}\left\|4α_1-7α_2+4α_3\right\|^2=\left\langle4α_1-7α_2+4α_3,4α_1-7α_2+4α_3\right\rangle\;\;\;\;\;\;\;\;\;\;\\\;\;\;\;\;\;\;\;\;\;\;\;\;\;\;\;\;\;\;\;\;\;\;\;\;\;\;\;\;=4^2+\left(-7\right)^2+4^2\;=81\;\;\;\;\;\;\;\;\;\;\;\;\;\;\;\;\;\;\;,\;\;\\故\;\left\|4α_1-7α_2+4α_3\right\|=\sqrt{81}\;\;=9\;\end{array}
$$



$$
设A\mathrm{为正交矩阵},且A^T=-A^*,\mathrm{其中}A^* 是A\mathrm{的伴随矩阵},则A\mathrm{的行列式等于}(\;\;\;\;).
$$
$$
A.
1 \quad B.-1 \quad C.0 \quad D.±1 \quad E. \quad F. \quad G. \quad H.
$$
$$
\begin{array}{l}A\mathrm{为正交矩阵},则AA^T=E,又A^T=-A^*,则AA^*=-E;\;\;\\\mathrm{由伴随矩阵的性质可知}AA^*=\left|A\right|E,即\left|A\right|E=-E⇒\left|A\right|=-1.\end{array}
$$



$$
设n\mathrm{元向量组}α_1,α_2,⋯,α_m\mathrm{是正交向量组},则m与n\mathrm{的大小关系为}(\;\;\;\;).
$$
$$
A.
m\leq n \quad B.m\geq n \quad C.m=n \quad D.m>n \quad E. \quad F. \quad G. \quad H.
$$
$$
\mathrm{因为}n\mathrm{元向量组}α_1,α_2,⋯,α_m\mathrm{是正交向量组},\mathrm{所以向量组}α_1,α_2,⋯,α_m\mathrm{是线性无关的向量组},\mathrm{因此}m\leq n,\mathrm{因为}n+1个n\mathrm{维的向量必线性相关}.
$$



$$
设A,B\mathrm{均是}m\mathrm{阶正交矩阵},且\left|A\right|=1,\left|B\right|=-1,则\left|A+B\right|\mathrm{的值为}(\;\;\;\;).
$$
$$
A.
1 \quad B.-1 \quad C.±1 \quad D.0 \quad E. \quad F. \quad G. \quad H.
$$
$$
\begin{array}{l}\left|A+B\right|=\left|AB^TB+AA^TB\right|=\left|A\right|\left|B^T+A^T\right|\left|B\right|=\left|\mathbf A\right|\left|\mathbf B\boldsymbol+\mathbf A\right|\left|\mathbf B\right|=-\left|A+B\right|,\\故\left|A+B\right|=0\end{array}
$$



$$
\mathrm{下列矩阵中不是正交矩阵的为}(\;\;\;\;).
$$
$$
A.
\begin{pmatrix}\textstyle\frac{\sqrt3}3&\textstyle\frac{\sqrt3}3&\textstyle\frac{\sqrt3}3\\0&-{\textstyle\frac1{\sqrt2}}&\textstyle\frac1{\sqrt2}\\-{\textstyle\frac2{\sqrt6}}&\textstyle\frac1{\sqrt6}&\textstyle\frac1{\sqrt6}\end{pmatrix} \quad B.\begin{pmatrix}1&0&0&0\\0&\textstyle\frac1{\sqrt3}&\textstyle\frac1{\sqrt2}&0\\0&\textstyle\frac1{\sqrt3}&0&1\\0&\textstyle\frac1{\sqrt3}&-{\textstyle\frac1{\sqrt2}}&0\end{pmatrix} \quad C.\begin{pmatrix}\textstyle\frac12&\textstyle\frac12&\textstyle\frac12&\textstyle\frac12\\\textstyle\frac12&-{\textstyle\frac56}&\textstyle\frac16&\textstyle\frac16\\\textstyle\frac12&\textstyle\frac16&\textstyle\frac16&-{\textstyle\frac56}\\\textstyle\frac12&\textstyle\frac16&-{\textstyle\frac56}&\textstyle\frac16\end{pmatrix} \quad D.\begin{pmatrix}\textstyle\frac67&\textstyle\frac27&\textstyle\frac37\\\textstyle\frac27&\textstyle\frac37&-{\textstyle\frac67}\\-{\textstyle\frac37}&\textstyle\frac67&\textstyle\frac27\end{pmatrix} \quad E. \quad F. \quad G. \quad H.
$$
$$
\mathrm{因为}\begin{pmatrix}1&0&0&0\\0&\textstyle\frac1{\sqrt3}&\textstyle\frac1{\sqrt2}&0\\0&\textstyle\frac1{\sqrt3}&0&1\\0&\textstyle\frac1{\sqrt3}&-{\textstyle\frac1{\sqrt2}}&0\end{pmatrix}\mathrm{的第}2\mathrm{列与第}4\mathrm{列不正交},\mathrm{故不是正交矩阵}.
$$



$$
\mathrm{已知三维向量空间中两个向量}α_1=\begin{pmatrix}1\\1\\1\end{pmatrix},α_2=\begin{pmatrix}1\\-2\\1\end{pmatrix}\mathrm{正交},若α_1,α_2,α_3\mathrm{构成三维空间的一个正交基},则α_3=\left(\right).
$$
$$
A.
\left(-1,0,1\right)^T \quad B.\left(-1,1,0\right)^T \quad C.\left(0,-1,1\right)^T \quad D.\left(-1,0,0\right)^T \quad E. \quad F. \quad G. \quad H.
$$
$$
\begin{array}{l}设α_3=\left(χ_1,χ_2,χ_3\right)^{\;T}\neq0,\mathrm{且分别与}α_1,α_2\mathrm{正交},则\;\\\;\;\;\;\;\;\;\;\;\;\;\;\;\;\;\;\;\;\;\;\left\langleα_1,α_3\right\rangle=\left\langleα_2,α_3\right\rangle=0\;\;\;\;\;\;\;\;\;\;\;\;\;\;\;\;\;\\即\;\;\;\;\;\;\;\;\;\;\;\;\;\;\;\;\;\left\{\begin{array}{l}\;\left\langleα_1,α_3\right\rangle=χ_1+χ_2+χ_3=0\\\left\langleα_2,α_3\right\rangle=χ_1-2χ_2+χ_3=0\end{array}\right.\;\;\;\\\;\mathrm{解之得}\;\;\;\;\;\;\;\;\;χ_1=-χ_3,χ_2=0\;\;\;\;\;\;\;\;\;\;\\令χ_3=1⇒α_3=\begin{pmatrix}χ_1\\χ_2\\{\boldsymbolχ}_\mathbf3\end{pmatrix}\;=\begin{pmatrix}-1\\0\\1\end{pmatrix}\;\;\;\\\mathrm{由上可知}α_1,α_2,α_3\mathrm{构成三维空间的一个正交基}.\end{array}
$$



$$
设A=\begin{pmatrix}a&b&c&x_1\\b&-a&d&x_2\\c&-d&-a&x_3\\d&c&-b&x_4\end{pmatrix},a,b,c,d\mathrm{是不全为}0\mathrm{的实数},若B=kA\mathrm{为正交矩阵},则x_i\left(i=1,2,3,4\right)及k\mathrm{分别为}(\;\;\;).
$$
$$
A.
x_1=d,x_2=-c,x_3=b,x_4=-a,k=±{\textstyle\frac1{\sqrt{a^2+b^2+c^2+d^2}}} \quad B.x_1=-d,x_2=-c,x_3=b,x_4=-a,k=±{\textstyle\frac1{\sqrt{a^2+b^2+c^2+d^2}}} \quad C.x_1=d,x_2=-c,x_3=b,x_4=-a,k={\textstyle\frac1{\sqrt{a^2+b^2+c^2+d^2}}} \quad D.x_1=-d,x_2=-c,x_3=b,x_4=-a,k={\textstyle\frac1{\sqrt{a^2+b^2+c^2+d^2}}} \quad E. \quad F. \quad G. \quad H.
$$
$$
\begin{array}{l}若kA\mathrm{为正交矩阵},\mathrm{则它的行向量成正交向量组},\mathrm{由此有}\;\\ab+b\left(-a\right)+cd+x_1x_2=0\;,cd+x_1x_2=0,\;\;\\\mathrm{同理有}-bd+x_1x_3=0,ad+x_1x_4=0,\;\;\\取x_1=d,x_2=-c,x_3=b,x_4=-a.\;\;\\\mathrm{经验证}:A的4\mathrm{个列向量两两正交},\mathrm{且它们的模都是}\sqrt{a^2+b^2+c^2+d^2},取k=±{\textstyle\frac1{\sqrt{a^2+b^2+c^2+d^2}}},则\\是B=±{\textstyle\frac1{\sqrt{a^2+b^2+c^2+d^2}}}\begin{pmatrix}a&b&c&d\\b&-a&d&-c\\c&-d&-a&b\\d&c&-b&-a\end{pmatrix}\mathrm{正交阵}.\end{array}
$$



$$
\text{方程}y^2=9z\text{在空间表示()}
$$
$$
A.
\text{母线平行}y\text{轴的抛物柱面} \quad B.\text{母线平行}z\text{轴的抛物柱面} \quad C.\text{母线平行}yoz\text{面的抛物柱面} \quad D.\text{母线平行}x\text{轴的抛物柱面} \quad E. \quad F. \quad G. \quad H.
$$
$$
\text{母线平行}x\text{轴的抛物柱面}
$$



$$
\text{方程}x^2+y^2+z^2-2x-4y-4z-7=0\text{表示()曲面.}
$$
$$
A.
\text{以}\left(1,2,2\right)\text{为球心,半径为}4\text{的球面} \quad B.\text{以}\left(1,-2,2\right)\text{为球心,半径为}4\text{的球面} \quad C.\text{以}\left(1,-2,-2\right)\text{为球心,半径为}4\text{的球面} \quad D.\text{以}\left(-1,-2,2\right)\text{为球心,半径为}4\text{的球面} \quad E. \quad F. \quad G. \quad H.
$$
$$
\begin{array}{l}\text{这类缺}xy,yz,zx\text{项的二次方程总可以经配方处理化为球面的标准方程.对本题,将方程写成}\\\left(x-1\right)^2+\left(y-2\right)^2+\left(z-2\right)^2=16\text{, 可见题设方程表示以}\left(1,2,2\right)\text{为球心,半径为4的球面.}\end{array}
$$



$$
\text{方程}\frac{x^2}4+\frac{y^2}3-\frac{z^2}5=-1\text{在空间中表示(  ). }
$$
$$
A.
\text{单叶双曲面} \quad B.\text{双曲柱面} \quad C.\text{二次锥面} \quad D.\text{双叶双曲面} \quad E. \quad F. \quad G. \quad H.
$$
$$
\text{平方项有2个负号,所以是双叶双曲面}
$$



$$
\text{方程}\frac{x^2}5+\frac{y^2}3-\frac{z^2}4=0\text{在空间中表示(  ). }
$$
$$
A.
\text{二次锥面} \quad B.\text{单叶双曲面} \quad C.\text{双叶双曲面} \quad D.\text{2条直线} \quad E. \quad F. \quad G. \quad H.
$$
$$
\text{二次锥面}
$$



$$
\text{方程}x^2+\frac{y^2}9-\frac{z^2}{25}=-1\text{是().}
$$
$$
A.
\text{单叶双曲面} \quad B.\text{双叶双曲面} \quad C.\text{椭球面} \quad D.\text{双曲抛物面} \quad E. \quad F. \quad G. \quad H.
$$
$$
\text{由双叶双曲面的定义可知,形如}\frac{x^2}{a^2}+\frac{y^2}{b^2}-\frac{z^2}{c^2}=-1\text{的曲面称为双叶双曲面.}
$$



$$
\text{方程}x^2+\frac{y^2}9-\frac{z^2}{25}=1\text{是().}
$$
$$
A.
\text{单叶双曲面} \quad B.\text{双叶双曲面} \quad C.\text{椭球面} \quad D.\text{双曲抛物面} \quad E. \quad F. \quad G. \quad H.
$$
$$
\text{由双叶双曲面的定义可知,形如}\frac{x^2}{a^2}+\frac{y^2}{b^2}-\frac{z^2}{c^2}=1\text{的曲面称为单叶双曲面.}
$$



$$
\text{方程}\frac{x^2}5+\frac{y^2}3-\frac{z^2}4=1\text{在空间中表示(  ). }
$$
$$
A.
\text{双叶双曲面} \quad B.\text{单叶双曲面} \quad C.\text{二次锥面} \quad D.\text{双曲柱面} \quad E. \quad F. \quad G. \quad H.
$$
$$
\text{单叶双曲面}
$$



$$
\text{方程}\frac{x^2}4-\frac{y^2}9=z\text{在空间中表示(  ). }
$$
$$
A.
\text{双曲柱面} \quad B.\text{双曲抛物面(马鞍面)} \quad C.\text{二次锥面} \quad D.\text{单叶双曲面} \quad E. \quad F. \quad G. \quad H.
$$
$$
\text{双曲抛物面(马鞍面)}
$$



$$
\text{方程}\frac{x^2}4+y^{{}^2}=z^2\text{表示的是().}
$$
$$
A.
\text{二次锥面} \quad B.\text{椭球面} \quad C.\text{双曲面} \quad D.\text{双曲线} \quad E. \quad F. \quad G. \quad H.
$$
$$
\text{由二次锥面的定义可得形如}\frac{x^2}{a^2}+\frac{y^2}{b^2}-\frac{z^2}{c^2}=0\text{的曲面叫二次锥面.}
$$



$$
x^2-\frac{y^2}4+z^2=-1\text{表示的曲面是().}
$$
$$
A.
\text{单叶双曲面} \quad B.\text{双叶双曲面} \quad C.\text{双曲柱面} \quad D.\text{锥面} \quad E. \quad F. \quad G. \quad H.
$$
$$
\text{根据双叶双曲面的定义可知}x^2-\frac{y^2}4+z^2=-1\text{为双叶双曲面.}
$$



$$
\text{方程}x^2+y^2+z^2=0\text{在空间表示()}
$$
$$
A.
\text{球面} \quad B.\text{一点} \quad C.\text{圆锥面} \quad D.\mathrm{圆柱面} \quad E. \quad F. \quad G. \quad H.
$$
$$
\text{由题目可知}x=y=z=0\text{,所以表示的为一点.}
$$



$$
\text{方程}z^2-x^2-y^2=0\text{在空间表示().}
$$
$$
A.
\text{柱面} \quad B.\text{圆锥面} \quad C.\text{旋转双曲面} \quad D.\text{平面} \quad E. \quad F. \quad G. \quad H.
$$
$$
\text{根据圆锥面的定义可知.}
$$



$$
\text{方程}z^2-x^2=1\text{在空间表示}
$$
$$
A.
\text{双曲线} \quad B.\text{母线平行}z\text{轴的双曲柱面} \quad C.\text{母线平行}y\text{轴的双曲柱面} \quad D.\text{母线平行}x\text{轴的双曲柱面} \quad E. \quad F. \quad G. \quad H.
$$
$$
\text{根据双曲柱面的定义可知.}
$$



$$
\text{方程}x^2=2y\text{在空间表示的是().}
$$
$$
A.
\text{抛物线} \quad B.\text{抛物柱面} \quad C.\text{母线平行}x\text{轴的柱面} \quad D.\text{旋转抛物面} \quad E. \quad F. \quad G. \quad H.
$$
$$
\text{由抛物柱面的定义可知.}
$$



$$
\text{方程}\frac{y^2}4+\frac{z^2}9=1\text{在空间表示()}
$$
$$
A.
\text{椭圆柱面} \quad B.\text{双曲柱面} \quad C.\text{抛物柱面} \quad D.\text{椭圆} \quad E. \quad F. \quad G. \quad H.
$$
$$
\text{椭圆柱面}
$$



$$
\text{方程}\frac{y^2}4\text{-}\frac{z^2}9=1\text{在空间表示()}
$$
$$
A.
\text{双曲线} \quad B.\text{双曲柱面} \quad C.\text{抛物柱面} \quad D.\text{二次锥面} \quad E. \quad F. \quad G. \quad H.
$$
$$
\text{双曲柱面}
$$



$$
\text{方程}\frac{x^2}2+\frac{y^2}4+\frac{z^2}9=1\text{在空间表示().}
$$
$$
A.
\text{二次锥面} \quad B.\text{椭球面} \quad C.\text{单叶双曲面} \quad D.\text{椭圆柱面} \quad E. \quad F. \quad G. \quad H.
$$
$$
\text{是椭球面}
$$



$$
\text{方程}x^2=9z\text{在空间表示().}
$$
$$
A.
\text{二次锥面} \quad B.\text{抛物柱面} \quad C.\text{抛物线} \quad D.\text{双曲柱面} \quad E. \quad F. \quad G. \quad H.
$$
$$
\text{是抛物柱面}
$$



$$
\text{方程}x^2=1\text{在空间表示().}
$$
$$
A.
\text{两点} \quad B.\text{母线平行}x\text{轴的柱面} \quad C.\text{平行}yOz\text{平面的两个平面} \quad D.\text{旋转曲面} \quad E. \quad F. \quad G. \quad H.
$$
$$
x^2=1\text{,解得}x=±1\text{,此表示平行}yOz\text{平面的两个平面.}
$$



$$
\text{方程}y^2=4x\text{在空间表示()}
$$
$$
A.
\text{母线平行}z\text{的抛物柱面} \quad B.\text{母线平行}x\text{的抛物柱面} \quad C.\text{母线平行}y\text{的抛物柱面} \quad D.\text{母线平行}xoy\text{面的抛物柱面} \quad E. \quad F. \quad G. \quad H.
$$
$$
\text{含两个变量,所以是母线平行}z\text{的抛物柱面}
$$



$$
\text{方程}8x^2+4y^2-z^2=-64\text{在空间中表示().}
$$
$$
A.
\text{单叶双曲面} \quad B.\text{椭球面} \quad C.\text{双叶双曲面} \quad D.\text{二次锥面} \quad E. \quad F. \quad G. \quad H.
$$
$$
\text{化为}-\frac{x^2}8-\frac{y^2}{16}+\frac{z^2}{64}=1\text{,是双叶双曲面}
$$



$$
\text{方程}8x^2+4y^2-3z^2=0\text{在空间中表示().}
$$
$$
A.
\text{单叶双曲面} \quad B.\text{双叶双曲面} \quad C.\text{椭球面} \quad D.\text{二次锥面} \quad E. \quad F. \quad G. \quad H.
$$
$$
\frac{x^2}{\displaystyle\frac38}+\frac{y^2}{\displaystyle\frac34}=z^2\text{,是二次锥面}
$$



$$
\text{方程}\frac{x^2}4-\frac{y^2}9=3\text{在空间表示().}
$$
$$
A.
\text{母线平行}y\text{轴的双曲柱面} \quad B.\text{母线平行}x\text{轴的双曲柱面} \quad C.\text{母线平行}xoy\mathrm 面\text{的双曲柱面} \quad D.\text{母线平行}z\text{轴的双曲柱面} \quad E. \quad F. \quad G. \quad H.
$$
$$
\text{母线平行}z\text{轴的双曲柱面}
$$



$$
\text{方程}\frac{x^2}3-\frac{y^2}5=z\text{在空间表示().}
$$
$$
A.
\text{二次锥面} \quad B.\text{双曲抛物面} \quad C.\text{双曲线} \quad D.\text{单叶双曲面} \quad E. \quad F. \quad G. \quad H.
$$
$$
\text{双曲抛物面}
$$



$$
\text{下面方程在空间中表示柱面的是(  ).}
$$
$$
A.
2x^2+3y^2=z^2 \quad B.2x^2+3y^2=5 \quad C.2x^2+3y^2+z^2=1 \quad D.x^2+\frac{y^2}2-z^2=1 \quad E. \quad F. \quad G. \quad H.
$$
$$
2x^2+3y^2=5\mathrm{是柱面}
$$



$$
\text{方程}\frac{x^2}4-\frac{y^2}9+\frac{z^2}5=0\text{在空间表示().}
$$
$$
A.
\text{单叶双曲面} \quad B.\text{双叶双曲面} \quad C.\text{双曲抛物面} \quad D.\text{二次锥面} \quad E. \quad F. \quad G. \quad H.
$$
$$
\text{二次锥面}
$$



$$
\begin{array}{l}\text{下列曲面方程在空间中表示柱面的是(  ).}\\\text{(1)}z^2=2x,\text{(2)}z=\sqrt{x^2+y^2}\\\text{(3)}\frac{x^2}4-\frac{y^2}3=2,\text{(4)}x^2+y^2=2\end{array}
$$
$$
A.
\text{(1)(2)} \quad B.\text{(2)(3)} \quad C.(2)(4) \quad D.(1)(3)(4) \quad E. \quad F. \quad G. \quad H.
$$
$$
\mathrm{根据柱面方程的定义},(1)(3)(4)\mathrm{是柱面}.
$$



$$
\text{方程}16x^2+4y^2+z^2=64\text{表示().}
$$
$$
A.
\text{椭球面} \quad B.\text{二次锥面} \quad C.\text{单叶双曲面} \quad D.\text{抛物柱面} \quad E. \quad F. \quad G. \quad H.
$$
$$
\text{椭球面}
$$



$$
\begin{array}{l}\text{下列曲面方程在空间中表示二次锥面的是(  ).}\\\text{(1)}z^2=2x\text{,(2)}z=\sqrt{x^2+2y^2}\text{,}\\\text{(3)}4x^2+3y^2-36z^2=0\text{,(4)}\frac{x^2}2+y^2=3z\\\end{array}
$$
$$
A.
\text{(1)(3)} \quad B.\text{(2)(3)} \quad C.\text{(3)(4)} \quad D.\text{(2)(4)} \quad E. \quad F. \quad G. \quad H.
$$
$$
\text{(2)(3)是二次锥面}
$$



$$
\text{方程}8x^2-4y^2+3z^2=-5\text{在空间中表示().}
$$
$$
A.
\text{单叶双曲面} \quad B.\text{双叶双曲面} \quad C.\text{二次锥面} \quad D.\text{椭球面} \quad E. \quad F. \quad G. \quad H.
$$
$$
\text{可化为}-\frac{8x^2}5+\frac{4y^2}5-\frac{3z^2}5=1\text{,所以是双叶双曲面}
$$



$$
\begin{array}{l}\text{下列曲面方程在空间中表示柱面的是(  ).}\\\text{(1)}z^2=2y\text{,(2)}z=\sqrt{x^2+y^2}\text{,}\\\text{,(3)}\frac{x^2}4-\frac{y^2}3=2\text{,(4)}x^2-y^2=1\\\end{array}
$$
$$
A.
\text{(1)(2)} \quad B.\text{(2)(3)} \quad C.\text{(2)(4)} \quad D.\text{(1)(3)(4)} \quad E. \quad F. \quad G. \quad H.
$$
$$
\text{根据柱面方程的定义(1)(3)(4)是柱面}
$$



$$
\text{方程}\frac{x^2}3+\frac{y^2}5=9z\text{在空间表示().}
$$
$$
A.
\text{椭球面} \quad B.\text{二次锥面} \quad C.\text{椭圆抛物面} \quad D.\text{双叶双曲面} \quad E. \quad F. \quad G. \quad H.
$$
$$
\text{是椭圆抛物面}
$$



$$
\begin{array}{l}\text{下列曲面方程在空间中表示柱面的是(  ).}\\\text{(1)}z^2=2x\text{,(2)}z=\sqrt{x^2+y^2}\text{,}\\\text{,(3)}4x^2-4y^2+36z^2=144\text{,(4)}x^2+y^2=2\\\end{array}
$$
$$
A.
\text{(1)(2)} \quad B.\text{(2)(3)} \quad C.\text{(1)(4)} \quad D.\text{(2)(4)} \quad E. \quad F. \quad G. \quad H.
$$
$$
\text{(1)(4)是柱面}
$$



$$
\text{方程}3z=y^2+\frac{x^2}4\text{的图形是(  ). }
$$
$$
A.
\text{椭球面} \quad B.\text{双曲面} \quad C.\text{旋转抛物面} \quad D.\text{椭圆抛物面} \quad E. \quad F. \quad G. \quad H.
$$
$$
\text{是椭圆抛物面}
$$



$$
\text{方程}8x^2+4y^2-z^2=64\text{在空间中表示().}
$$
$$
A.
\text{双叶双曲面} \quad B.\text{单叶双曲面} \quad C.\text{二次锥面} \quad D.\text{椭球面} \quad E. \quad F. \quad G. \quad H.
$$
$$
\text{正确答案是:单叶双曲面}
$$



$$
\text{方程}8x^2-4y^2+3z^2=0\text{在空间中表示().}
$$
$$
A.
\text{单叶双曲面} \quad B.\text{双叶双曲面} \quad C.\text{二次锥面} \quad D.\text{两条直线} \quad E. \quad F. \quad G. \quad H.
$$
$$
\text{二次锥面}
$$



$$
\begin{array}{l}\text{下面方程的图形是柱面的是(  ).}\\\text{(1)}\frac{x^2}9+\frac{z^2}4=1\text{,(2)}\frac{x^2}9+\frac{y^2}4+z^2=1,\\\text{(3)}-\frac{x^2}9+\frac{z^2}4=1\text{,(4)},y^2-z=0\end{array}
$$
$$
A.
\text{(2)(3)} \quad B.\text{(1)(2)} \quad C.\text{(1)(3)(4)} \quad D.\text{(2)(4)} \quad E. \quad F. \quad G. \quad H.
$$
$$
\text{(1)(3)(4)}
$$



$$
\text{方程}16x^2+4y^2-z^2=64\text{表示().}
$$
$$
A.
\text{锥面} \quad B.\text{单叶双曲面} \quad C.\text{双叶双曲面} \quad D.\text{椭圆抛物面} \quad E. \quad F. \quad G. \quad H.
$$
$$
\text{方程变形为}\frac{x^2}4+\frac{y^2}{16}-\frac{z^2}{64}=1\text{,为单叶双曲面}
$$



$$
\text{方程}\frac{x^2}2\text{+}\frac{y^2}2\text{-}\frac{z^2}3\text{=0表示旋转曲面,它的旋转轴是 ()}
$$
$$
A.
\text{x轴} \quad B.\text{y轴} \quad C.\text{z轴} \quad D.\text{直线x=y=z} \quad E. \quad F. \quad G. \quad H.
$$
$$
\text{根据旋转曲面的特征来判断,因为}x^2,y^2\text{的系数相同,与}z^2\text{的系数不同}
$$



$$
\text{关于旋转曲面}x^2-\frac{y^2}4+z^2=1\text{形成叙述正确的是()}
$$
$$
A.
\text{双曲线}x^2-\frac{y^2}4=1\text{绕}y\text{轴旋转一周而成的旋转单叶双曲面} \quad B.\text{双曲线}x^2+\frac{y^2}4=1\text{绕}y\text{轴旋转一周而成的旋转单叶双曲面} \quad C.\text{双曲线}x^2-\frac{y^2}4=1\text{绕}x\text{轴旋转一周而成的旋转单叶双曲面} \quad D.\text{双曲线}x^2-\frac{y^2}4=1\text{绕}z\text{轴旋转一周而成的旋转单叶双曲面} \quad E. \quad F. \quad G. \quad H.
$$
$$
\text{原方程改写为}\left(x^2+z^2\right)-\frac{y^2}4=1\text{,可看作}xOy\text{平面上的双曲线}x^2-\frac{y^2}4=1\text{绕}y\text{轴旋转一周而成的旋转}\mathrm{单叶双曲面}
$$



$$
\text{将}xOz\text{坐标面上的圆}x^2+z^2=9\text{绕}z\text{轴旋转一周,则生成的旋转曲面方程为().}
$$
$$
A.
x^2-y^2+z^2=9 \quad B.-x^2-y^2+z^2=9 \quad C.x^2+y^2+z^2=-9 \quad D.x^2+y^2+z^2=9 \quad E. \quad F. \quad G. \quad H.
$$
$$
\text{根据旋转曲面的定义得:}x^2+y^2+z^2=9
$$



$$
\text{将}yoz\text{坐标面上的曲线}y^2-3z^2=8\text{绕}y\text{轴旋转一周,则生成的旋转曲面方程为(  ). }
$$
$$
A.
3x^2+y^2-3z^2=8 \quad B.3x^2-y^2+3z^2=-8 \quad C.3x^2+y^2+3z^2=8 \quad D.-3x^2-y^2-3z^2=8 \quad E. \quad F. \quad G. \quad H.
$$
$$
\text{根据旋转曲面的定义知旋转曲面的方程为:}3x^2-y^2+3z^2=-8
$$



$$
\text{将x}oz\text{坐标面上的直线}z=3x\text{绕}z\text{轴旋转一周,则生成的旋转曲面方程为(  ). }
$$
$$
A.
9x^2+9y^2=z^2 \quad B.3x^2+3y^2=z^2 \quad C.9z^2+9y^2=x^2 \quad D.9x^2+9y^2=-z^2 \quad E. \quad F. \quad G. \quad H.
$$
$$
\text{根据旋转曲面的定义知旋转曲面的方程为:}9x^2+9y^2=z^2
$$



$$
\text{旋转曲面}x^2-y^2-z^2=1\text{是().}
$$
$$
A.
xOy\text{平面上的双曲线}x^2-y^2=1\text{绕}y\text{轴旋转所得} \quad B.xOz\text{平面上的双曲线}x^2-z^2=1\text{绕}z\text{轴旋转所得} \quad C.xOy\text{平面上的双曲线}x^2-y^2=1\text{绕}x\text{轴旋转所得} \quad D.xOy\text{平面上的椭圆}x^2+y^2=1\text{绕}x\text{轴旋转所得} \quad E. \quad F. \quad G. \quad H.
$$
$$
x^2-y^2-z^2=1\text{确定的曲面为双叶双曲面,可看作是}xOy\text{平面上的双曲线}x^2-y^2=1\text{绕}x\text{轴旋转所得.}
$$



$$
\text{将}xOz\text{坐标面上的抛物线}z^2=5x\text{绕}x\text{轴旋转一周所生成的旋转曲面的方程为().}
$$
$$
A.
y^2+z^2=5x \quad B.y^2+z^2=-5x \quad C.y^2-z^2=5x \quad D.y^2-z^2=-5x \quad E. \quad F. \quad G. \quad H.
$$
$$
\begin{array}{l}\text{对方程}z^2=5x,x\text{不变,将}z\text{改成}\left(±\sqrt{y^2+z^2}\right)\text{,于是得所求旋转曲面的方程为:}\left(±\sqrt{y^2+z^2}\right)^2=5x\text{,即}y^2+z^2=5x\text{.}\\\text{此为旋转抛物面的方程.}\end{array}
$$



$$
\text{将}xOz\text{坐标面上的圆}x^2+z^2=9\text{绕}z\text{轴旋转一周,则所生成的旋转曲面的方程为().}
$$
$$
A.
x^2+y^2+z^2=9 \quad B.x^2+y^2+z^2=8 \quad C.x^2+y^2+z^2=16 \quad D.-x^2-y^2+z^2=9 \quad E. \quad F. \quad G. \quad H.
$$
$$
\begin{array}{l}\text{在方程}x^2+z^2=9\text{中,}z\text{不变,将}x\text{改成}\left(±\sqrt{x^2+y^2}\right)\text{,得所求旋转曲面的方程为:}\\\left(±\sqrt{x^2+y^2}\right)^2+z^2=9\text{,即}x^2+y^2+z^2=9\text{,}\\\text{显见这是以原点为球心,半径为3的球面方程.}\end{array}
$$



$$
\text{以曲线}Γ\text{:}\left\{\begin{array}{l}f\left(y,z\right)=0\\x=0\end{array}\right.\text{为母线,以}Oz\text{轴为旋转轴的旋转曲面的方程是().}
$$
$$
A.
f\left(±\sqrt{x^2+y^2},z\right)=0 \quad B.f\left(\sqrt{x^2+y^2},z\right)=0 \quad C.f\left(\sqrt{x^2+y^2},x\right)=0 \quad D.f\left(±\sqrt{x^2+y^2},y\right)=0 \quad E. \quad F. \quad G. \quad H.
$$
$$
\text{由题意可得:曲线以}Oz\text{轴为旋转轴的旋转曲面的方程为:}f\left(±\sqrt{x^2+y^2},z\right)=0
$$



$$
zOx\text{平面上曲线}z^{2=}5x\text{绕}z\text{轴旋转而成的旋转曲面方程为().}
$$
$$
A.
z^4=25\left(x^2+y^2\right) \quad B.z^4=25\left(x^2-y^2\right) \quad C.z^4=5\left(x^2+y^2\right) \quad D.z=25\left(x^2+y^2\right) \quad E. \quad F. \quad G. \quad H.
$$
$$
\text{原方程}z\text{不变,}x\text{变成}z^2=±5\sqrt{x^2+y^2}\text{,即}z^4=25\left(x^2+y^2\right)
$$



$$
\text{曲面}z=\sqrt{x^2+y^2}\text{是().}
$$
$$
A.
zOx\text{平面上曲线}z=x\text{绕}z\text{轴旋转而成的旋转曲面} \quad B.yOz\text{平面上曲线}z=\left|y\right|\text{绕}z\text{轴旋转而成的旋转曲面} \quad C.zOx\text{平面上曲线}z=x\text{绕}x\text{轴旋转而成的旋转曲面} \quad D.yOz\text{平面上曲线}z=\left|y\right|\text{绕}y\text{轴旋转而成的旋转曲面} \quad E. \quad F. \quad G. \quad H.
$$
$$
\text{图面为二次锥面,为}zOy\text{平面上曲线}z=\left|y\right|\text{绕}z\text{轴旋转而成的旋转曲面}
$$



$$
\text{将}xOz\text{坐标面上的曲线}x^2-z^2=9\text{绕}x\text{轴旋转一周,则生成的旋转曲面方程为().}
$$
$$
A.
x^2+y^2-z^2=9 \quad B.x^2-y^2-z^2=9 \quad C.-x^2-y^2+z^2=9 \quad D.x^2-y^2+z^2=-9 \quad E. \quad F. \quad G. \quad H.
$$
$$
\text{根据旋转曲面的方程的定义知所求曲面为}x^2-y^2-z^2=9
$$



$$
\text{将坐标面}yoz\text{上的曲线}f\left(y,z\right)=0\text{绕}y\text{轴旋转一周得到的旋转曲面方程是().}
$$
$$
A.
f\left(y,\sqrt{x^2+z^2}\right)=0 \quad B.f\left(y,±\sqrt{y^2+z^2}\right)=0 \quad C.f\left(y,±\sqrt{x^2+z^2}\right)=0 \quad D.f\left(±\sqrt{x^2+z^2},z\right)=0 \quad E. \quad F. \quad G. \quad H.
$$
$$
\text{根据旋转曲面的定义得: 旋转一周得到的旋转曲面方程}f\left(y,±\sqrt{x^2+z^2}\right)=0
$$



$$
\text{将}xOy\text{坐标面上的双曲线}4x^2-9y^2=36\text{绕}y\text{轴旋转一周,则所生成的旋转曲面的方程为().}
$$
$$
A.
4x^2-9y^2+4z^2=36 \quad B.4x^2-9y^2-4z^2=36 \quad C.-4x^2+9y^2+4z^2=36 \quad D.-4x^2+9y^2-4z^2=36 \quad E. \quad F. \quad G. \quad H.
$$
$$
\text{绕}y\text{轴旋转一周生成的曲面方程为}4x^2-9y^2+4z^2=36
$$



$$
\text{将}xoz\text{坐标面上的曲线}z^2=6x\text{绕}x\mathrm{轴旋转一周},\mathrm{则生成的旋转曲面的方程为}\;(\;\;)\;
$$
$$
A.
6y^2+z^2-6x=0 \quad B.-y^2-z^2-6x=0 \quad C.y^2+z^2-6x=0 \quad D.y^2-z^2-6x=0 \quad E. \quad F. \quad G. \quad H.
$$
$$
\text{根据旋转曲面的定义知所求旋转曲面的方程为:}y^2+z^2-6x=0
$$



$$
\text{旋转曲面}x^2-y^2-z^2=1\text{是().}
$$
$$
A.
xOy\text{平面上的双曲线绕}x\mathrm{轴旋转所得} \quad B.xOz\text{平面上的双曲线绕}z\text{轴旋转所得} \quad C.xOy\text{平面上的椭圆绕}x\text{轴旋转所得} \quad D.xOz\text{平面上的椭圆绕}x\text{轴旋转所得} \quad E. \quad F. \quad G. \quad H.
$$
$$
\begin{array}{l}x^2-y^2-z^2=1\text{变形为}y^2+z^2-x^2=-1\text{,}\\\text{此为双叶双曲面,可看作是}xOy\text{平面上的双曲线绕}x\text{轴旋转所得.}\end{array}
$$



$$
\text{曲面}\frac{x^2}4+\frac{y^2}9+\frac{z^2}9=1\text{是().}
$$
$$
A.
\text{球面} \quad B.xOy\text{平面上的曲线}\frac{x^2}4+\frac{y^2}9=1\text{绕}y\text{轴旋转而成旋转椭球面} \quad C.xOz\text{平面上的曲线}\frac{x^2}4+\frac{z^2}9=1\text{绕x轴旋转而成旋转椭球面} \quad D.\text{柱面} \quad E. \quad F. \quad G. \quad H.
$$
$$
\text{曲面方程为}\frac{x^2}4+\frac{y^2+z^2}9=1\text{,是}xOz\text{平面上的曲线}\frac{x^2}4+\frac{z^2}9=1\text{绕}x\text{轴旋转而成的椭球面.}
$$



$$
\text{曲面}2x^2+4y^2+4z^2=1\text{是().}
$$
$$
A.
\text{球面} \quad B.xOy\text{平面上曲线}2x^2+4y^2=1\text{绕}y\text{轴旋转而成} \quad C.\mathrm{柱面} \quad D.xOz\text{平面上曲线}2x^2+4z^2=1\text{绕}x\text{轴旋转而成} \quad E. \quad F. \quad G. \quad H.
$$
$$
\text{曲面是二次锥面,是}xOz\text{平面上曲线}2x^2+4z^2=1\text{绕}x\text{轴旋转而成.}
$$



$$
\text{将}xOy\text{坐标面上的双曲线}4x^2-9y^2=36\text{绕}x\text{轴旋转一周,则所生成的旋转曲面的方程为().}
$$
$$
A.
4x^2-9y^2-9z^2=36 \quad B.4x^2+9y^2+9z^2=36 \quad C.4x^2+9y^2-4z^2=36 \quad D.x^2+9y^2+4z^2=36 \quad E. \quad F. \quad G. \quad H.
$$
$$
\text{绕}x\text{轴旋转一周生成的曲面方程为 }4x^2-9y^2-9z^2=36
$$



$$
\text{xOy平面上曲线}x^2-4y^2=9\text{绕y轴旋转一周所得旋转曲面方程为().}
$$
$$
A.
-x^2+z^2-4y^2=9 \quad B.-x^2+z^2+4y^2=9 \quad C.x^2-z^2-4y^2=9 \quad D.x^2+z^2-4y^2=9 \quad E. \quad F. \quad G. \quad H.
$$
$$
\text{由曲面的定义可知旋转曲面的方程为}x^2+z^2-4y^2=9.
$$



$$
\text{求}xOy\text{平面上曲线}\frac{x^2}9+\frac{y^2}4=1\text{绕}x\text{轴旋转而成的旋转曲面方程为(  ). }
$$
$$
A.
\frac{x^2}9+\frac{y^2}4+\frac{z^2}9=1 \quad B.\frac{x^2}9+\frac{y^2}4-\frac{z^2}4=1 \quad C.\frac{x^2}9+\frac{y^2}4+\frac{z^2}4=1 \quad D.\frac{x^2}4+\frac{y^2}9+\frac{z^2}4=1 \quad E. \quad F. \quad G. \quad H.
$$
$$
\text{根据旋转曲面的定义得:}\frac{x^2}9+\frac{y^2}4+\frac{z^2}4=1
$$



$$
\text{下列各曲线中,绕}y\text{轴旋转而成椭球面}3x^2+2y^2+3z^2=1\text{的曲线是().}
$$
$$
A.
\left\{\begin{array}{l}2x^2+3y^2=1\\y=0\end{array}\right. \quad B.\left\{\begin{array}{l}3y^2+2z^2=1\\x=0\end{array}\right. \quad C.\left\{\begin{array}{l}3x^2+2y^2=1\\z=0\end{array}\right. \quad D.\left\{\begin{array}{l}3x^2+3z^2=1\\y=0\end{array}\right. \quad E. \quad F. \quad G. \quad H.
$$
$$
3x^2+2y^2+3z^2=1\text{在}yOx\text{上的投影为}\left\{\begin{array}{l}3x^2+2y^2=1\\z=0\end{array}\right.\text{,此线绕}y\text{轴旋转即成椭球面}3x^2+2y^2+3z^2=1
$$



$$
\text{旋转曲面}\frac{x^2}9-\frac{y^2}9-\frac{z^2}9=1\text{的旋转轴是(  ). }
$$
$$
A.
x\text{轴} \quad B.y\text{轴} \quad C.\boldsymbol z\text{轴} \quad D.\text{直线}x=y=z \quad E. \quad F. \quad G. \quad H.
$$
$$
x\text{轴}
$$



$$
\text{坐标面}yoz\text{上的直线}z=y+a\text{绕}z\text{轴旋转一周得到的旋转曲面方程为(  ). }
$$
$$
A.
\left(z-a\right)^2=x^2-y^2 \quad B.z^2=x^2+\left(y+a\right)^{{}^2} \quad C.\left(z+a\right)^2=x^2+y^2 \quad D.\left(z-a\right)^2=x^2+y^2 \quad E. \quad F. \quad G. \quad H.
$$
$$
\text{根据旋转曲面的定义知:}\left(z-a\right)^2=x^2+y^2
$$



$$
\text{旋转曲面}x^2-y^2-z^2=1\text{是().}
$$
$$
A.
xOy\text{平面上的双曲线绕}x\text{轴旋转所得} \quad B.xOy\text{平面上的双曲线绕}z\text{轴旋转所得} \quad C.xOy\text{平面上的椭圆绕}x\text{轴旋转所得} \quad D.xOy\text{平面上的椭圆绕y轴旋转所得} \quad E. \quad F. \quad G. \quad H.
$$
$$
x^2-y^2-z^2=1\text{变形为}y^2+z^2-x^2=-1\text{,  此为双叶双曲面,可看作是}xOy\text{平面上的双曲线绕}x\text{轴旋转所得.}
$$



$$
\text{曲面}x^2+y^2=9z^2\text{是().}
$$
$$
A.
\text{球面} \quad B.xOy\text{平面上曲线}x^2=9y^2\text{绕}x\text{轴旋转而成的} \quad C.xOz\text{平面上曲线}x^2=9z^2\text{绕}y\text{轴旋转而成的} \quad D.yOz\text{平面上曲线y}=3z\text{绕}z\text{轴旋转而成的} \quad E. \quad F. \quad G. \quad H.
$$
$$
\text{该曲面是圆锥面,是由}yOz\text{平面上曲线}y=3z\text{绕}z\text{轴旋转而成的.}
$$



$$
xOy\text{平面上曲线}x^2-4y^2=9\text{绕}x\text{轴旋转一周所得旋转曲面方程为()}
$$
$$
A.
x^2-4y^2+4z^2=9 \quad B.x^2-4y^2-z^2=9 \quad C.x^2-4y^2-4z^2=9 \quad D.x^2-4y^2+z^2=9 \quad E. \quad F. \quad G. \quad H.
$$
$$
xOy\text{平面上曲线}x^2-4y^2=9\text{绕}x\text{轴旋转,}x\text{不变,把}y^2\text{变为}y^2+z^2\text{,即得旋转曲面方程}x^2-4y^2-4z^2=9
$$



$$
\text{下列关于旋转曲面}\frac{x^2}4+\frac{y^2}9+\frac{z^2}9=1\text{形成叙述正确的是().}
$$
$$
A.
\frac{x^2}4+\frac{z^2}9=1\text{绕}x\text{轴旋转一周而成的旋转椭球面} \quad B.\frac{x^2}4+\frac{z^2}9=1\text{绕}y\text{轴旋转一周而成的旋转椭球面} \quad C.\frac{x^2}4-\frac{z^2}9=1\text{绕}x\text{轴旋转一周而成的旋转椭球面} \quad D.\frac{x^2}4+\frac{z^2}9=1\text{绕z轴旋转一周而成的旋转椭球面} \quad E. \quad F. \quad G. \quad H.
$$
$$
\begin{array}{l}\text{将方程改写为}\frac{x^2}4+\frac{y^2+z^2}9=1\text{,可看作}\frac{x^2}4+\frac{y^2}9=1\text{绕}x\text{轴旋转一周而成的旋转椭球面;或}\\\frac{x^2}4+\frac{z^2}9=1\text{绕}x\text{轴旋转一周而成的旋转椭球面.}\end{array}
$$



$$
\mathrm{双曲面}x^2-y^2/4-z^2/9=1\mathrm{与平面}y=4\mathrm{交线为}(\;).
$$
$$
A.
\mathrm{双曲线} \quad B.\mathrm{椭圆} \quad C.\mathrm{抛物线} \quad D.\mathrm{一对相交直线} \quad E. \quad F. \quad G. \quad H.
$$
$$
将y=4\mathrm{代入到双曲面方程得}x^2-z^2/9=5,\mathrm{所以相交的曲线为双曲线}.
$$



$$
\mathrm{双曲面}x^2-y^2/4-z^2/9=1与yOz\mathrm{平面}(\;).
$$
$$
A.
\mathrm{交于一双曲线} \quad B.\mathrm{交于一对相交直线} \quad C.\mathrm{不交} \quad D.\mathrm{交于一椭圆} \quad E. \quad F. \quad G. \quad H.
$$
$$
\mathrm{平面}yOz是x=0,\mathrm{代入到双曲面}x^2-y^2/4-z^2/9=1\mathrm{中等式不成立},\mathrm{所以双曲面与平面}yOz\mathrm{不相交}.
$$



$$
\mathrm{曲线}\left\{\begin{array}{l}x^2+y^2+z^2=4\\y=z\end{array}\right.在yOz\mathrm{平面上的投影曲线是}(\;)\;
$$
$$
A.
\left\{\begin{array}{l}y=z\\x=0\end{array}\right.(\vert y\vert\leq\sqrt2) \quad B.\left\{\begin{array}{l}y=z\\x=0\end{array}\right.(\vert y\vert\leq\sqrt3) \quad C.\left\{\begin{array}{l}y=z\\x=1\end{array}\right.(\vert y\vert\leq\sqrt2) \quad D.\left\{\begin{array}{l}y=z\\x=1\end{array}\right.(\vert y\vert\leq\sqrt3) \quad E. \quad F. \quad G. \quad H.
$$
$$
\mathrm{由曲线方程把}x\mathrm{消去},\mathrm{直接可得}\;\left\{\begin{array}{l}y=z\\x=0\end{array}\right.(\vert y\vert\leq\sqrt2).
$$



$$
\mathrm{球面}x^2+y^2+(a-x)^2=R^2与x+z=a(0<\;a\;<\;R)\mathrm{交线在}xOy\mathrm{平面上投影曲线的方程是}\;(\;).
$$
$$
A.
\left\{\begin{array}{l}x^2+y^2+(a-x)^2=R^2\\z=0\end{array}\right. \quad B.\left\{\begin{array}{l}x^2+y^2+2ax=R^2\\z=0\end{array}\right. \quad C.\left\{\begin{array}{l}x^2+y^2+(a-y)^2=R^2\\z=0\end{array}\right. \quad D.\left\{\begin{array}{l}x^2+y^2+2ay=R^2\\z=0\end{array}\right. \quad E. \quad F. \quad G. \quad H.
$$
$$
\mathrm{由题意可得空间曲线方程为}:\left\{\begin{array}{l}x^2+y^2+(a-x)^2=R^2\\x+z=a\end{array}\right.,\mathrm{消去}z\mathrm{得到}:\left\{\begin{array}{l}x^2+y^2+(a-x)^2=R^2\\z=0\end{array}\right.,\mathrm{即为所求}.
$$



$$
\mathrm{圆锥面}z=\sqrt{x^2+y^2}\mathrm{与旋转抛物面}z=2-x^2-y^2\mathrm{所围立体在}xOy\mathrm{上的投影区域为}(\;).\;
$$
$$
A.
\left\{\begin{array}{l}x^2+y^2=1\\z=0\end{array}\right. \quad B.\left\{\begin{array}{l}x^2+y^2\leq1\\z=1\end{array}\right. \quad C.\left\{\begin{array}{l}x^2+y^2\leq1\\z=0\end{array}\right. \quad D.\left\{\begin{array}{l}x^2+y^2\leq1\\z=-2\end{array}\right. \quad E. \quad F. \quad G. \quad H.
$$
$$
\mathrm{两方程联立},消z得x^2+y^2=1,\mathrm{所以投影区域为}\left\{\begin{array}{l}x^2+y^2\leq1\\z=0\end{array}\right.
$$



$$
\mathrm{椭圆抛物面}z=x^2+2y^2\mathrm{与柱面}z=2-x^2\mathrm{所围立体在}xOy\mathrm{上的投影区域为}(\;).\;
$$
$$
A.
x^2+y^2\leq1 \quad B.\left\{\begin{array}{l}x^2+y^2=1\\z=0\end{array}\right. \quad C.x^2+y^2=1 \quad D.\left\{\begin{array}{l}x^2+y^2\leq1\\z=0\end{array}\right. \quad E. \quad F. \quad G. \quad H.
$$
$$
由\left\{\begin{array}{l}z=x^2+2y^2\\z=2-x^2\end{array}\right.得:x^2+y^2=1,\mathrm{所以在}xOy\mathrm{面上投影区域为}\left\{\begin{array}{l}x^2+y^2\leq1\\z=0\end{array}\right.
$$



$$
\mathrm{曲线}\left\{\begin{array}{l}y^2+z^2-2x=0\\z=3\end{array}\right.\mathrm{关于}xoy\mathrm{面的投影柱面方程是}(\;).
$$
$$
A.
\left\{\begin{array}{l}y^2=2x\\z=0\end{array}\right. \quad B.y^2=2x-9 \quad C.\left\{\begin{array}{l}y^2=2x-9\\z=0\end{array}\right. \quad D.y^2=2x+9 \quad E. \quad F. \quad G. \quad H.
$$
$$
\mathrm{由题设方程组中消去变量}z\mathrm{后得}y^2+9-2x=0,\mathrm{所以}y^2=2x-9\mathrm{即为在}xOy\mathrm{面上的投影方程}.
$$



$$
\mathrm{方程}\left\{\begin{array}{l}x^2+4y^2+9z^2=36\\y=1\end{array}\right.\mathrm{所表示的曲线为}(\;).\;
$$
$$
A.
\mathrm{曲线是平面}y=1\mathrm{截椭球面}x^2+4y^2+9z^2=36\mathrm{而得的椭圆}\left\{\begin{array}{l}x^2+9z^2=32\\y=1\end{array}\right. \quad B.\mathrm{曲线是平面}y=1\mathrm{截椭球面}x^2+4y^2+9z^2=36\mathrm{而得的椭圆}\left\{\begin{array}{l}x^2+9z^2=16\\y=-1\end{array}\right. \quad C.\mathrm{曲线是平面}y=1\mathrm{截椭球面}x^2+4y^2+9z^2=36\mathrm{而得的椭圆}\left\{\begin{array}{l}x^2+9z^2=16\\y=1\end{array}\right. \quad D.\mathrm{曲线是平面}y=1\mathrm{截椭球面}x^2+4y^2+9z^2=36\mathrm{而得的椭圆}\left\{\begin{array}{l}x^2+3z^2=32\\y=1\end{array}\right. \quad E. \quad F. \quad G. \quad H.
$$
$$
\begin{array}{l}\mathrm{曲线是平面}y=1\mathrm{截椭球面}x^2+4y^2+9z^2=36\mathrm{而得的椭圆}\left\{\begin{array}{l}x^2+9z^2=32\\y=1\end{array}\right..\\\mathrm{它在平面上}y=1,\mathrm{椭圆中心在点}(0,1,0)处,长,\mathrm{短半轴分别与}x\mathrm{轴和}y\mathrm{轴平行},\mathrm{其长分别为}4\sqrt[{}]2与4/3\sqrt2.\end{array}
$$



$$
\mathrm{方程}\left\{\begin{array}{l}x^2-4y^2+z^2=25\\x=-3\end{array}\right.\mathrm{所表示的曲线是}(\;).\;
$$
$$
A.
\mathrm{曲线是平面}x=-3\mathrm{截单叶双曲面}x^2-4y^2+z^2=25\mathrm{而得的双曲线}\left\{\begin{array}{l}z^2-4y^2=16\\x=-3\end{array}\right. \quad B.\mathrm{曲线是平面}x=3\mathrm{截单叶双曲面}x^2-4y^2+z^2=25\mathrm{而得的双曲线}\left\{\begin{array}{l}z^2-4y^2=16\\x=3\end{array}\right. \quad C.\mathrm{曲线是平面}x=-3\mathrm{截单叶双曲面}x^2-4y^2+z^2=25\mathrm{而得的双曲线}\left\{\begin{array}{l}z^2-4y^2=25\\x=-3\end{array}\right. \quad D.\mathrm{曲线是平面}x=-3\mathrm{截单叶双曲面}x^2-4y^2+z^2=25\mathrm{而得的双曲线}\left\{\begin{array}{l}z^2-4y^2=5\\x=-3\end{array}\right. \quad E. \quad F. \quad G. \quad H.
$$
$$
\begin{array}{l}\mathrm{曲线是平面}x=-3\mathrm{截单叶双曲面}x^2-4y^2+z^2=25\mathrm{而得的双曲线}\left\{\begin{array}{l}z^2-4y^2=16\\x=-3\end{array}\right..\;\;\\\mathrm{它在平面}x=-3上,\mathrm{双曲线中心在点}(-3,0,0)处,\mathrm{实轴与虚轴分别平行于}x\mathrm{轴和}y轴,\mathrm{半实轴长为}4,\mathrm{半虚轴长为}2.\end{array}
$$



$$
\mathrm{曲线}\left\{\begin{array}{l}z=2-x^2-y^2\\z=(x-1)^2+(y-1)^2\end{array}\right.\mathrm{在坐标面}xOy\mathrm{上的投影曲线的方程为}(\;).\;
$$
$$
A.
\left\{\begin{array}{l}z=0\\x^2+y^2=x+y\end{array}\right. \quad B.\left\{\begin{array}{l}z=0\\x^2-y^2=x+y\end{array}\right. \quad C.\left\{\begin{array}{l}z=0\\x^2+y^2=x-y\end{array}\right. \quad D.\left\{\begin{array}{l}z=0\\x^2-2y^2=x+y\end{array}\right. \quad E. \quad F. \quad G. \quad H.
$$
$$
\begin{array}{l}\mathrm{由曲线方程消去}z,\mathrm{得此曲线在}xOy\mathrm{坐标面上的投影}\\\mathrm{柱面方程}:\;\;\;2-x^2-y^2=(x-1)^2+(y-1)^2,\\\mathrm{整理化简得}\;\;\;x^2+y^2=x+y,\\\mathrm{故曲线在}xOy\mathrm{面上的投影曲线方程为}\left\{\begin{array}{l}z=0\\x^2+y^2=x+y\end{array}\right.\end{array}
$$



$$
\mathrm{球面}x^2+y^2+z^2=4\mathrm{与椭圆抛物面}x^2+y^2=3z\mathrm{所围立体在上}xoy\mathrm{面上的投影区域为}(\;).\;
$$
$$
A.
x^2+y^2\leq3 \quad B.x^2+y^2=3 \quad C.\left\{\begin{array}{l}x^2+y^2\leq3\\z=0\end{array}\right. \quad D.\left\{\begin{array}{l}x^2+y^2\leq3\\z=1\end{array}\right. \quad E. \quad F. \quad G. \quad H.
$$
$$
由\left\{\begin{array}{l}x^2+y^2+z^2=4\\x^2+y^2=3z\end{array}\right.得:z=-4\mathrm{舍去},z=1,即x^2+y^2=3,\mathrm{所以在}xOy\mathrm{面上的投影区域为}\left\{\begin{array}{l}x^2+y^2\leq3\\z=0\end{array}\right.
$$



$$
\mathrm{椭圆抛物面}z=x^2+y^2\mathrm{与椭球面}2x^2+2y^2+z^2=8\mathrm{所围立体在}xOy\mathrm{上的投影区域为}(\;).\;
$$
$$
A.
\left\{\begin{array}{l}x^2+y^2=2\\z=2\end{array}\right. \quad B.\left\{\begin{array}{l}x^2+y^2=2\\z=0\end{array}\right. \quad C.\left\{\begin{array}{l}x^2+y^2\leq2\\z=2\end{array}\right. \quad D.\left\{\begin{array}{l}x^2+y^2\leq2\\z=0\end{array}\right. \quad E. \quad F. \quad G. \quad H.
$$
$$
由\left\{\begin{array}{l}2x^2+2y^2+z^2=8\\z=x^2+y^2\end{array}\right.得:z=2,z=-4(\mathrm{舍去}),即x^2+y^2\leq2,\mathrm{所以投影区域为}\left\{\begin{array}{l}x^2+y^2\leq2\\z=0\end{array}\right.
$$



$$
\mathrm{曲线}\left\{\begin{array}{l}z=x^2+2y^2\\z=2-x^2\end{array}\right.\mathrm{关于}xOy\mathrm{平面的投影柱面为}(\;).\;
$$
$$
A.
x^2+y^2=1 \quad B.\left\{\begin{array}{l}x^2+y^2=2\\z=0\end{array}\right. \quad C.\left\{\begin{array}{l}x^2+y^2=1\\z=0\end{array}\right. \quad D.\left\{\begin{array}{l}x^2+y^2=2\\z=1\end{array}\right. \quad E. \quad F. \quad G. \quad H.
$$
$$
\mathrm{由曲线的表达式消去}z,\mathrm{得到投影柱面}:x^2+y^2=1
$$



$$
\mathrm{直线}\frac{x-1}1=\frac{y+3}{-2}=\frac{z-1}{-1}在xoy\mathrm{平面上的投影直线方程为}\;(\;).
$$
$$
A.
\left\{\begin{array}{l}2x+y-1=0\\z=0\end{array}\right. \quad B.\left\{\begin{array}{l}2x+y+1=0\\z=1\end{array}\right. \quad C.\left\{\begin{array}{l}2x+y-1=0\\z=1\end{array}\right. \quad D.\left\{\begin{array}{l}2x+y+1=0\\z=0\end{array}\right. \quad E. \quad F. \quad G. \quad H.
$$
$$
\mathrm{由直线方程可得}:z-1=-(x-1),z-1=\frac{y+3}2,\mathrm{因此消掉}z得:x-1=\frac{y+3}{-2},\mathrm{则直线在}xOy\mathrm{平面上的投影直线方程为}:\left\{\begin{array}{l}x-1=\frac{y+3}{-2}\\z=0\end{array}\right.或\left\{\begin{array}{l}2x+y+1=0\\z=0\end{array}\right.
$$



$$
\mathrm{方程组}\left\{\begin{array}{l}x^2+y^2=4\\x+y=1\end{array}\right.\mathrm{在空间表示}\;(\;).
$$
$$
A.
圆 \quad B.\mathrm{椭圆} \quad C.\mathrm{圆柱面} \quad D.\mathrm{两条直线} \quad E. \quad F. \quad G. \quad H.
$$
$$
\mathrm{方程组}\left\{\begin{array}{l}x^2+y^2=4\\x+y=1\end{array}\right.\mathrm{解得}\left\{\begin{array}{l}x=1/2-\sqrt7/2\\y=1/2+\sqrt7/2\end{array}\right.或\left\{\begin{array}{l}y=1/2-\sqrt7/2\\x=1/2+\sqrt7/2\end{array}\right.,\mathrm{所以该方程表示两条直线}.
$$



$$
\mathrm{双曲面}x^2/9+y^2/16-z^2/49=1与y=4\mathrm{交线为}(\;).\;
$$
$$
A.
\mathrm{双曲线} \quad B.\mathrm{椭圆} \quad C.\mathrm{抛物线} \quad D.\mathrm{一对相交直线} \quad E. \quad F. \quad G. \quad H.
$$
$$
将y=4\mathrm{代入到}x^2/9+y^2/16-z^2/49=1\mathrm{中得}x=±\frac37z,\mathrm{所以相交的曲线为一对相交直线}.
$$



$$
\mathrm{锥面}x^2+y^2/16=z^2\mathrm{与平面}yOz\mathrm{的交线为}(\;).
$$
$$
A.
\mathrm{椭圆} \quad B.\mathrm{双曲线} \quad C.\mathrm{一对相交直线} \quad D.\mathrm{一点} \quad E. \quad F. \quad G. \quad H.
$$
$$
yOz\mathrm{平面为}x=0,\mathrm{代入到锥面方程中得}y=±4z\mathrm{为一对相交直线}
$$



$$
\mathrm{曲面}x^2+y^2+z^2=25与z=3\mathrm{的交线为}(\;).
$$
$$
A.
圆 \quad B.\mathrm{椭圆} \quad C.点 \quad D.\mathrm{两条直线} \quad E. \quad F. \quad G. \quad H.
$$
$$
\mathrm{是圆}
$$



$$
\mathrm{曲面}x^2+4y^2+9z^2=36与y=1\mathrm{的交线为}(\;).
$$
$$
A.
圆 \quad B.\mathrm{椭圆} \quad C.点 \quad D.\mathrm{两条直线} \quad E. \quad F. \quad G. \quad H.
$$
$$
\mathrm{椭圆}
$$



$$
\mathrm{曲面}x^2-4y^2+z^2=25与z=-3\mathrm{的交线为}(\;).
$$
$$
A.
圆 \quad B.\mathrm{椭圆} \quad C.\mathrm{双曲线} \quad D.\mathrm{两条直线} \quad E. \quad F. \quad G. \quad H.
$$
$$
\mathrm{双曲线}
$$



$$
\mathrm{曲面}y^2+z^2-4x+8=0与y=4\mathrm{的交线为}(\;).
$$
$$
A.
\mathrm{椭圆} \quad B.\mathrm{双曲线} \quad C.\mathrm{抛物线} \quad D.\mathrm{两条直线} \quad E. \quad F. \quad G. \quad H.
$$
$$
\mathrm{抛物线}
$$



$$
\mathrm{假定直线}L在yoz\mathrm{平面上的投影方程为}\left\{\begin{array}{l}2y-3z=1\\x=0\end{array}\right.,\mathrm{而在}zOx\mathrm{平面上的投影方程为}\left\{\begin{array}{l}x+z=2\\y=0\end{array}\right.,\mathrm{则直线}L在xOy\mathrm{面上的投影方程为}(\;).
$$
$$
A.
\left\{\begin{array}{l}3x+2y=7\\z=0\end{array}\right. \quad B.\left\{\begin{array}{l}3x+2y=6\\z=0\end{array}\right. \quad C.\left\{\begin{array}{l}3x-2y=7\\z=0\end{array}\right. \quad D.\left\{\begin{array}{l}3x+5y=7\\z=0\end{array}\right. \quad E. \quad F. \quad G. \quad H.
$$
$$
\begin{array}{l}\mathrm{依题设},\mathrm{所求直线}L\mathrm{方程为}\left\{\begin{array}{l}2y-3z=1\\x+z=2\end{array}\right.,\\\mathrm{消去}z,\mathrm{得到直线在}xOy\mathrm{面上的投影直线方程为}\left\{\begin{array}{l}3x+2y=7\\z=0\end{array}\right..\end{array}
$$



$$
\mathrm{锥面}x^2+y^2/25=z^2/16与xoy\mathrm{平面的交线为}(\;).
$$
$$
A.
\mathrm{一对相交直线} \quad B.\mathrm{一点} \quad C.\mathrm{椭圆} \quad D.\mathrm{双曲线} \quad E. \quad F. \quad G. \quad H.
$$
$$
xOy\mathrm{平面为}z=0\mathrm{代入}x^2+y^2/25=z^2/16\mathrm{锥面得}x=y=0,\mathrm{所以相交于原点}.
$$



$$
\mathrm{曲面}x^2-y^2=z在xOz\mathrm{平面上的投影曲线是}\;(\;).
$$
$$
A.
x^2=z \quad B.\left\{\begin{array}{l}y^2=-z\\x=0\end{array}\right. \quad C.\left\{\begin{array}{l}x^2-y^2=0\\z=0\end{array}\right. \quad D.\left\{\begin{array}{l}x^2=z\\y=0\end{array}\right. \quad E. \quad F. \quad G. \quad H.
$$
$$
xOz\mathrm{平面为}y=0,\mathrm{代入曲面方程得}x^2=z,\mathrm{所以题目所求方程为}\left\{\begin{array}{l}x^2=z\\y=0\end{array}\right..
$$



$$
\mathrm{曲面}x^2+y^2+z^2=a^2与x^2+y^2=2az\mathrm{的交线为}(\;).\;
$$
$$
A.
\mathrm{圆周} \quad B.\mathrm{椭圆} \quad C.\mathrm{抛物线} \quad D.\mathrm{双曲线} \quad E. \quad F. \quad G. \quad H.
$$
$$
\mathrm{曲面}x^2+y^2+z^2=a^2\mathrm{为一个球面},x^2+y^2=2az\mathrm{为椭圆抛物面},\mathrm{因此交线为圆周}.
$$



$$
\mathrm{曲面}9x^2+9y^2=10z与yOz\mathrm{平面的交线为}(\;).\;
$$
$$
A.
\left\{\begin{array}{l}y^2=10/9z\\x=0\end{array}\right. \quad B.\left\{\begin{array}{l}y^2=z\\x=0\end{array}\right. \quad C.\left\{\begin{array}{l}y^2=-10/9z\\x=0\end{array}\right. \quad D.\left\{\begin{array}{l}y^2=10/9z\\x=10\end{array}\right. \quad E. \quad F. \quad G. \quad H.
$$
$$
\begin{array}{l}\mathrm{根据旋转曲面的定义},\mathrm{可见方程}9x^2+9y^2=10z\mathrm{表示旋转抛物面}.\\yOz\mathrm{平面的方程为}x=0,\mathrm{代入得到}y^2=10/9z,\mathrm{从而所求交线的方程为}\left\{\begin{array}{l}y^2=10/9z\\x=0\end{array}\right..\end{array}
$$



$$
\mathrm{曲面}x^2+y^2+z^2=a^2与z=1/2a\mathrm{的交线为}(\;).\;
$$
$$
A.
\mathrm{椭圆线} \quad B.\mathrm{圆线} \quad C.\mathrm{双曲线} \quad D.\mathrm{抛物线} \quad E. \quad F. \quad G. \quad H.
$$
$$
x^2+y^2+1/4a^2=a^2,x^2+y^2=3/4a^2,\mathrm{是圆线}
$$



$$
\mathrm{曲线}\left\{\begin{array}{l}z=2-x^2-y^2\\z=(x-1)^2+(y-1)^2\end{array}\right.在xOy\mathrm{面上的投影曲线的方程为}(\;).\;
$$
$$
A.
\left\{\begin{array}{l}z=1\\x^2+y^2=2x+y\end{array}\right. \quad B.\left\{\begin{array}{l}z=0\\x^2+y^2=2x+y\end{array}\right. \quad C.\left\{\begin{array}{l}z=1\\x^2+y^2=x+y\end{array}\right. \quad D.\left\{\begin{array}{l}z=0\\x^2+y^2=x+y\end{array}\right. \quad E. \quad F. \quad G. \quad H.
$$
$$
\begin{array}{l}\mathrm{由曲线的表达式我们可以消掉}z,\\\mathrm{得到}:(x-1)^2+(y-1)^2=2-x^2-y^2,\\\mathrm{整理得}:x^2+y^2=x+y,\\\mathrm{则此时曲线在}xOy\mathrm{面上的投影曲线的方程为}:\left\{\begin{array}{l}z=0\\x^2+y^2=x+y\end{array}\right.\end{array}
$$



$$
\mathrm{曲面}x^2+4y^2+z^2=4\mathrm{与平面}x+z=a\mathrm{的交线在}yOz\mathrm{平面上投影方程是}\;(\;).
$$
$$
A.
\left\{\begin{array}{l}4y^2+z^2+2ay=4\\x=0\end{array}\right. \quad B.\left\{\begin{array}{l}4y^2+z^2+2az=4\\x=0\end{array}\right. \quad C.\left\{\begin{array}{l}(a-z)^2+4y^2+z^2=4\\x=0\end{array}\right. \quad D.\left\{\begin{array}{l}(a-y)^2+4y^2+z^2=4\\x=0\end{array}\right. \quad E. \quad F. \quad G. \quad H.
$$
$$
\mathrm{由题意可得曲线方程为}:\left\{\begin{array}{l}x^2+4y^2+z^2=4\\x+z=a\end{array}\right.,\mathrm{消去}x得:\left\{\begin{array}{l}(a-z)^2+4y^2+z^2=4\\x=0\end{array}\right.\;,\mathrm{即为所求}
$$



$$
\mathrm{直线}\left\{\begin{array}{l}x+y+z=a\\x+cy=b\end{array}\right.在yOz\mathrm{平面上投影是}\;(\;).
$$
$$
A.
\left\{\begin{array}{l}(1-c)y+z=a-b\\x=0\end{array}\right. \quad B.\left\{\begin{array}{l}(1-c)z+y=a-b\\x=0\end{array}\right. \quad C.\left\{\begin{array}{l}(1-c)y+z=a+b\\x=0\end{array}\right. \quad D.\left\{\begin{array}{l}(1+c)y+z=a-b\\x=0\end{array}\right. \quad E. \quad F. \quad G. \quad H.
$$
$$
\mathrm{由直线的方程},\mathrm{消去}x,\mathrm{可得}:\left\{\begin{array}{l}(1-c)y+z=a-b\\x=0\end{array}\right.,\mathrm{即为所求}.
$$



$$
\mathrm{方程}\left\{\begin{array}{l}x^2+y^2+z^2=25\\x=3\end{array}\right.\mathrm{所表示的曲线是}(\;).\;
$$
$$
A.
\mathrm{曲线是平面}x=3\mathrm{上的一个圆}y^2+z^2=16 \quad B.\mathrm{曲线是平面}x=3\mathrm{上的一个圆}y^2+z^2=8 \quad C.\mathrm{曲线是平面}x=-3\mathrm{上的一个圆}y^2+z^2=16 \quad D.\mathrm{曲线是平面}x=3\mathrm{上的一个圆}y^2+z^2=9 \quad E. \quad F. \quad G. \quad H.
$$
$$
\begin{array}{l}\mathrm{曲线是平面}x=3\mathrm{上的一个圆}y^2+z^2=16,\\\mathrm{它由平面}x=3\mathrm{截球面}x^2+y^2+z^2=25\mathrm{而得},\mathrm{圆心在点}(3,0,0),\mathrm{半径为}4.\end{array}
$$



$$
\mathrm{由曲面}z=\sqrt{2-x^2-y^2}及z=x^2+y^2\mathrm{所围的立体在}xOy\mathrm{面上的投影为}(\;).\;
$$
$$
A.
\left\{\begin{array}{l}x^2+y^2=1\\z=0\end{array}\right. \quad B.\left\{\begin{array}{l}x^2+y^2=\sqrt2\\z=0\end{array}\right. \quad C.\left\{\begin{array}{l}x^2+y^2\leq\sqrt2\\z=0\end{array}\right. \quad D.\left\{\begin{array}{l}x^2+y^2\leq1\\z=0\end{array}\right. \quad E. \quad F. \quad G. \quad H.
$$
$$
\mathrm{消去}z,得x^2+y^2=1,\;\mathrm{所以投影为}\left\{\begin{array}{l}x^2+y^2\leq1\\z=0\end{array}\right.
$$



$$
\mathrm{旋转抛物面}z=x^2+y^2(0\leq z\leq4)\mathrm{在坐标面}xOy\mathrm{上的投影为}(\;).
$$
$$
A.
\left\{\begin{array}{l}x^2+y^2\leq4\\z=0\end{array}\right. \quad B.\left\{\begin{array}{l}x^2+y^2\leq16\\z=0\end{array}\right. \quad C.\left\{\begin{array}{l}x^2-y^2\leq4\\z=0\end{array}\right. \quad D.\left\{\begin{array}{l}x-y^2\leq16\\z=0\end{array}\right. \quad E. \quad F. \quad G. \quad H.
$$
$$
\begin{array}{l}\mathrm{从方程组}\left\{\begin{array}{l}z=x^2+y^2\\z=4\end{array}\right.\mathrm{消去}z\mathrm{得到向}xOy\mathrm{面的投影柱面方程为}:x^2+y^2=4,\mathrm{故该立体在面上的投影为}\left\{\begin{array}{l}x^2+y^2\leq4\\z=0\end{array}\right.;\\z=x^2+y^2与yOz\mathrm{面的交线为}\left\{\begin{array}{l}z=y^2(0\leq z\leq4)\\x=0\end{array}\right.,\\\mathrm{故该立体在}yOz\mathrm{面上的投影为}\left\{\begin{array}{l}y^2\leq z\leq4\\x=0\end{array}\right.;\\\mathrm{而该立体在}xOz\mathrm{面上的投影为}\left\{\begin{array}{l}x^2\leq z\leq4\\y=0\end{array}\right..\end{array}
$$



$$
\mathrm{曲线}\left\{\begin{array}{l}z=f(x,y)\\z=g(x,y)\end{array}\right.\mathrm{关于}xOy\mathrm{面的投影柱面方程是}(\;).
$$
$$
A.
f(x,y)=g(x,y) \quad B.f(x,y)=-g(x,y) \quad C.\left\{\begin{array}{l}f(x,y)=g(x,y)\\z=0\end{array}\right. \quad D.\left\{\begin{array}{l}f(x,y)=g(x,y)\\z=1\end{array}\right. \quad E. \quad F. \quad G. \quad H.
$$
$$
f(x,y)=g(x.y)
$$



$$
\mathrm{曲面}x^2+y^2+z^2=a^2与x^2+y^2=2az(a>\;0)\mathrm{的交线是}\;(\;).
$$
$$
A.
\mathrm{抛物线} \quad B.\mathrm{双曲线} \quad C.\mathrm{圆周} \quad D.\mathrm{椭圆} \quad E. \quad F. \quad G. \quad H.
$$
$$
\mathrm{联立}\left\{\begin{array}{l}x^2+y^2+z^2=a^2\\x^2+y^2=2az\end{array}\right.\mathrm{解得}z=(-1+\sqrt2)a,x^2+y^2=\lbrack1-(1-\sqrt2)^2\rbrack a^2,\mathrm{所以交线为一个圆周}.
$$



$$
\mathrm{锥面}z=\sqrt{x^2+y^2}\mathrm{与柱面}z^2=2x\mathrm{所围立体在}xOy\mathrm{面上的投影为}\;(\;).
$$
$$
A.
\left\{\begin{array}{l}(x-1)^2+y^2\leq1\\z=0\end{array}\right. \quad B.\left\{\begin{array}{l}(x-1)^2+y^2\leq1\\z=3\end{array}\right. \quad C.\left\{\begin{array}{l}(x-1)^2+y^2\geq1\\z=0\end{array}\right. \quad D.\left\{\begin{array}{l}(x-1)^2+y^2=1\\z=0\end{array}\right. \quad E. \quad F. \quad G. \quad H.
$$
$$
\begin{array}{l}\mathrm{锥面与柱面的交线为}:\left\{\begin{array}{l}z=\sqrt{x^2+y^2}\\z^2=2x\end{array}\right.,\\\mathrm{则由曲线的表达式可消去}z,\mathrm{得到}:x^2+y^2=2x,\\\mathrm{整理得}:(x-1)^2+y^2=1,\\\mathrm{得曲线在面上的投影曲线的方程为}:\left\{\begin{array}{l}(x-1)^2+y^2=1\\z=0\end{array}\right.\;,\;\\\mathrm{故所围立体在}xOy\mathrm{面上的投影应为圆在}xOy\mathrm{面上所围的部分}:\left\{\begin{array}{l}(x-1)^2+y^2\leq1\\z=0\end{array}\right.\end{array}
$$



$$
\mathrm{双曲面}x^2-y^2/4-z^2/9=1\mathrm{与平面}y=4\mathrm{交线为}(\;).
$$
$$
A.
\mathrm{双曲线} \quad B.\mathrm{椭圆} \quad C.\mathrm{抛物线} \quad D.\mathrm{一对相交直线} \quad E. \quad F. \quad G. \quad H.
$$
$$
将y=4\mathrm{代入到双曲面方程得}x^2-z^2/9=5,\mathrm{所以相交的曲线为双曲线}.
$$



$$
\begin{array}{l}\mathrm{直线}\left\{\begin{array}{l}2x-y+3z-4=0\\x+y+z+1=0\end{array}\right.\;\mathrm{的标准方程为}(\;)\\\end{array}
$$
$$
A.
\frac{x-1}{-4}=\frac{\displaystyle y+2}{\displaystyle-1}=\frac{\displaystyle z}{\displaystyle3} \quad B.\frac{x-1}{-4}=\frac{\displaystyle y+2}{\displaystyle1}=\frac{\displaystyle z}{\displaystyle3} \quad C.\frac{x-1}{-4}=\frac{\displaystyle y+2}{\displaystyle1}=\frac{\displaystyle z}{\displaystyle-3} \quad D.\frac{x-1}4=\frac{\displaystyle y+2}{\displaystyle0}=\frac{\displaystyle z}{\displaystyle-3} \quad E. \quad F. \quad G. \quad H.
$$
$$
\begin{array}{l}\mathrm{直线的方向向量}s=\begin{vmatrix}i&j&k\\2&-1&3\\1&1&1\end{vmatrix}=\{-4,1,3\},\mathrm{再在直线上取定}\;\;x_0=1,⇒ y_0\;=-2,z_0=0.\\\mathrm{所得直线的标准方程为}\frac{x-1}{-4}=\frac{\displaystyle y+2}{\displaystyle1}=\frac{\displaystyle z}{\displaystyle3}\;.\end{array}
$$



$$
\mathrm{过点}(-1,2,0)\mathrm{且与平面}2x+y-z=0\mathrm{垂直的直线方程为}()
$$
$$
A.
\frac{x+1}2=\frac{\displaystyle y-2}{\displaystyle1}=\frac{\displaystyle z}{\displaystyle-1} \quad B.\frac{x-1}2=\frac{\displaystyle y-2}{\displaystyle1}=\frac{\displaystyle z}{\displaystyle-1} \quad C.\frac{x+1}2=\frac{\displaystyle y+2}{\displaystyle1}=\frac{\displaystyle z}{\displaystyle-1} \quad D.\frac{x+1}2=\frac{\displaystyle y-2}{\displaystyle-1}=\frac{\displaystyle z}{\displaystyle1} \quad E. \quad F. \quad G. \quad H.
$$
$$
\mathrm{所求直线的方向向量}\;\;s=\{2,1,-1\}\;,\mathrm{由直线的标准式方程可得}\frac{x+1}2=\frac{\displaystyle y-2}{\displaystyle1}=\frac{\displaystyle z}{\displaystyle-1}.
$$



$$
\mathrm{下列方程中表示的是空间的一条直线的是}()
$$
$$
A.
x+y+z+1=0 \quad B.x+y+1=0 \quad C.\left\{\begin{array}{l}x=1\\y=1\end{array}\right. \quad D.z=a \quad E. \quad F. \quad G. \quad H.
$$
$$
\mathrm{答案}\;\;\;A,B,D\;\mathrm{均表示平面}
$$



$$
\;\mathrm{过点}(1,0,1)\mathrm{且垂直于平面}x+3y-2z+2=0\mathrm{的直线方程为}
$$
$$
A.
\frac{x+1}1=\frac{\displaystyle y}{\displaystyle3}=\frac{\displaystyle z+1}{\displaystyle-2} \quad B.\frac{x-1}1=\frac{\displaystyle y}{\displaystyle3}=\frac{\displaystyle z-1}{\displaystyle2} \quad C.\frac{x+1}1=\frac{\displaystyle y}{\displaystyle-3}=\frac{\displaystyle z-1}{\displaystyle-2} \quad D.\frac{x-1}1=\frac{\displaystyle y}{\displaystyle3}=\frac{\displaystyle z-1}{\displaystyle-2} \quad E. \quad F. \quad G. \quad H.
$$
$$
\begin{array}{l}\mathrm{所求直线的方向向量可取作已知平面的法向量}\overrightarrow n=\{1,3,-2\},\mathrm{又过点}(1,0,1),\\\mathrm{故直线方程为}\frac{x-1}1=\frac{\displaystyle y}{\displaystyle3}=\frac{\displaystyle z-1}{\displaystyle-2}\end{array}
$$



$$
\mathrm{过点}(1,-3,0)\mathrm{与平面}2x+3y-z+7=0\mathrm{垂直的直线方程为}()
$$
$$
A.
\frac{x-1}1=\frac{\displaystyle y}{\displaystyle3}=\frac{\displaystyle z-1}{\displaystyle-2} \quad B.\frac{x-1}1=\frac{\displaystyle y+3}{\displaystyle-3}=\frac{\displaystyle z}{\displaystyle1} \quad C.\frac{x-1}2=\frac{\displaystyle y+3}{\displaystyle3}=\frac{\displaystyle z}{\displaystyle-1} \quad D.\frac{x-1}2=\frac{\displaystyle y-3}{\displaystyle3}=\frac{\displaystyle z}{\displaystyle-1} \quad E. \quad F. \quad G. \quad H.
$$
$$
\begin{array}{l}\mathrm{所求直线的方向向量可取作已知平面的法向量}\overrightarrow n=\{2,3,-1\},\mathrm{又过点}(1,-3,0),\\\mathrm{股直线方程为}\frac{x-1}2=\frac{\displaystyle y+3}{\displaystyle3}=\frac{\displaystyle z}{\displaystyle-1}.\end{array}
$$



$$
\mathrm{过点}(0,1,-3)\mathrm{与平面}3x-y+4z-8=0\mathrm{垂直的直线方程为}(\;)
$$
$$
A.
\frac{x-1}1=\frac{\displaystyle y+3}{\displaystyle3}=\frac{\displaystyle z}{\displaystyle-1} \quad B.\frac x3=\frac{\displaystyle y-1}{\displaystyle-1}=\frac{\displaystyle z+3}{\displaystyle4} \quad C.\frac x3=\frac{\displaystyle y+1}{\displaystyle-1}=\frac{\displaystyle z+3}{\displaystyle4} \quad D.\frac x3=\frac{\displaystyle y-1}{\displaystyle-1}=\frac{\displaystyle z-3}{\displaystyle4} \quad E. \quad F. \quad G. \quad H.
$$
$$
\begin{array}{l}\mathrm{所求直线的方向向量可取作已知平面的法向量}\overrightarrow n=\{3,-1,4\},\mathrm{又过点}(0,1,-3),\\\mathrm{故直线方程为}\frac x3=\frac{\displaystyle y-1}{\displaystyle-1}=\frac{\displaystyle z+3}{\displaystyle4}.\end{array}
$$



$$
\mathrm{过两点}M_1(2,-1,5)和M_2(-1,0,6)\mathrm{的直线方程为}(\;).
$$
$$
A.
\frac{x-2}{-3}=\frac{\displaystyle y+1}{\displaystyle1}=\frac{\displaystyle z-5}{\displaystyle1} \quad B.\frac{x-2}4=\frac{\displaystyle y+1}{\displaystyle1}=\frac{\displaystyle z-5}{\displaystyle1} \quad C.\frac{x-2}{-3}=\frac{\displaystyle y-1}{\displaystyle1}=\frac{\displaystyle z+5}{\displaystyle1} \quad D.\frac{x-2}{-3}=\frac{\displaystyle y+1}{\displaystyle0}=\frac{\displaystyle z-5}{\displaystyle1} \quad E. \quad F. \quad G. \quad H.
$$
$$
\begin{array}{l}\mathrm{可取方向向量}\overrightarrow s\;=\overrightarrow{M_1M_2}\;=\{-3,1,1\},\\\mathrm{于是所求直线方程为}\frac{x-2}{-3}=\frac{\displaystyle y+1}{\displaystyle1}=\frac{\displaystyle z-5}{\displaystyle1}.\end{array}
$$



$$
\mathrm{直线}\left\{\begin{array}{l}5x+y-3z-7=0\\2x+y-3z-7=0\end{array}\right.(\;).
$$
$$
A.
\mathrm{垂直}yOz\mathrm{平面} \quad B.在yOz\mathrm{平面内} \quad C.\mathrm{平行}x轴 \quad D.在xOy\mathrm{平面内} \quad E. \quad F. \quad G. \quad H.
$$
$$
\mathrm{由直线方程可知},x=0,\mathrm{所以直线是在}yOz\mathrm{平面内}
$$



$$
\begin{array}{l}\mathrm{设空间直线的标准方程是}\frac x0=\frac y1=\frac z2,\mathrm{该直线过原点},且().\\\end{array}
$$
$$
A.
\mathrm{垂直于}x轴 \quad B.\mathrm{垂直于}z轴,\mathrm{但不平行于}x轴 \quad C.\mathrm{垂直于}y轴,\mathrm{但不平行于}x轴 \quad D.\mathrm{平行于}x轴 \quad E. \quad F. \quad G. \quad H.
$$
$$
\mathrm{该直线的方向向量为}\{0,1,2\},x\mathrm{轴的方向向量为}\{1,0,0\},\mathrm{向量的数量积为零},\mathrm{所以该直线于}x\mathrm{轴垂直}
$$



$$
\mathrm{一直线过点}A(2,-3,4),\mathrm{且和}y\mathrm{轴垂直相交},\mathrm{则其方程为}(\;).
$$
$$
A.
\frac{x-2}2=\frac{\displaystyle y+3}{\displaystyle0}=\frac{\displaystyle z-4}{\displaystyle4} \quad B.\frac{x-2}2=\frac{\displaystyle y+3}{\displaystyle0}=\frac{\displaystyle z+4}{\displaystyle4} \quad C.\frac{x+2}2=\frac{\displaystyle y+3}{\displaystyle0}=\frac{\displaystyle z-4}{\displaystyle4} \quad D.\frac{x-2}2=\frac{\displaystyle y+3}{\displaystyle1}=\frac{\displaystyle z-4}{\displaystyle3} \quad E. \quad F. \quad G. \quad H.
$$
$$
\begin{array}{l}\mathrm{因为直线和}y\mathrm{轴垂直相交},\\\mathrm{所以交点为}B(0,-3,0)\\取\overrightarrow s=\overrightarrow{BA}=\{2,0,4\}\\\mathrm{所求直线方程}\\\frac{x-2}2=\frac{\displaystyle y+3}{\displaystyle0}=\frac{\displaystyle z-4}{\displaystyle4}\end{array}
$$



$$
\mathrm{过点}(0,-3,2)\mathrm{且与两点}P_1(3,4,-7),P_2(2,7,-6)\mathrm{的连线平行的直线的对称式方程为}(\;).
$$
$$
A.
\frac x{-1}=\frac{y+3}3=\frac{z-2}1 \quad B.\frac x1=\frac{y+3}3=\frac{z-2}1 \quad C.\frac x{-1}=\frac{y+3}{-3}=\frac{z-2}1 \quad D.\frac x1=\frac{y+3}3=\frac{z-2}{-1} \quad E. \quad F. \quad G. \quad H.
$$
$$
\begin{array}{l}\overrightarrow{P_1P_2}=\{-1,3,1\},\mathrm{由题意可知所求直线的方向向量也为}:\{-1,3,1\},\\\mathrm{又点}(0,-3,2)\mathrm{在直线上},\mathrm{则对称式方程为}:\frac x{-1}=\frac{y+3}3=\frac{z-2}1\end{array}
$$



$$
\mathrm{已知两直线的方程分别为}l_1:\frac x{-7}=\frac{y-{\displaystyle\textstyle\frac{27}7}}{-2}=\frac{z+{\displaystyle\textstyle\frac97}}{10};l_2:x=y-1=\frac{z+3}{-1}\;则(\;).\;\;
$$
$$
A.
l_1与l_2\mathrm{平行},\mathrm{但不重合} \quad B.l_1与l_2\mathrm{仅有一个交点} \quad C.l_1与l_2\mathrm{重合} \quad D.l_1与l_2\mathrm{是异面直线} \quad E. \quad F. \quad G. \quad H.
$$
$$
\begin{array}{l}\;l_1\mathrm{方向向量为}\{-7,-2,10\},l_2\mathrm{的方向向量为}\{1,1,-1\},\mathrm{所以两直线不平行},\\-7·1+(-2)·1+10·(-1)\neq0,\mathrm{所以也不垂直},\mathrm{可以解得有一交点为}(4,5,-7),\mathrm{所以}l_1与l_2\mathrm{仅有一个交点}.\end{array}
$$



$$
\mathrm{过点}(0,2,4)\mathrm{且与平面}x+2z=1及y-3z=2\mathrm{都平行的直线方程为}(\;).
$$
$$
A.
\frac x{-2}=\frac{y-2}3=\frac{\displaystyle z-4}1 \quad B.\frac x2=\frac{y-2}3=\frac{\displaystyle z-4}1 \quad C.\frac x{-2}=\frac{y-2}3=\frac{\displaystyle z-4}{-1} \quad D.\frac x2=\frac{y-2}{-3}=\frac{\displaystyle z-4}1 \quad E. \quad F. \quad G. \quad H.
$$
$$
\begin{array}{l}\mathrm{该直线方向}n\mathrm{与两平面法向量都垂直},则:\\n=n_1× n_2=\begin{vmatrix}i&j&k\\1&0&2\\0&1&-3\end{vmatrix}=\{-2,3,1\},\\\mathrm{故所求直线方程为}:\frac x{-2}=\frac{y-2}3=\frac{\displaystyle z-4}1(或\left\{\begin{array}{l}x+2z-8=0\\y-3z+10=0\end{array}\right.).\end{array}
$$



$$
\mathrm{用对称方程及参数方程表示直线}\left\{\begin{array}{l}x+y+z+1=0\\2x-y+3z+4=0\end{array}\right.,\mathrm{下面叙述正确的是}(\;).
$$
$$
A.
\mathrm{对称式方程}\frac{x-1}4=\frac{y-0}{-1}=\frac{z+2}{-3},\mathrm{参数方程}\left\{\begin{array}{c}x=1+4t\\y=-t\\z=-2-3t\end{array}\right. \quad B.\mathrm{对称式方程}\frac{x-1}4=\frac{y-0}1=\frac{z+2}3,\mathrm{参数方程}\left\{\begin{array}{c}x=1+4t\\y=-t\\z=-2-3t\end{array}\right. \quad C.\mathrm{对称式方程}\frac{x-1}4=\frac{y-0}{-1}=\frac{z+2}{-3},\mathrm{参数方程}\left\{\begin{array}{c}x=1-4t\\y=-t\\z=-2-3t\end{array}\right. \quad D.\mathrm{对称式方程}\frac{x-1}3=\frac{y-0}{-1}=\frac{z+2}{-3},\mathrm{参数方程}\left\{\begin{array}{c}x=1+3t\\y=-t\\z=-2-3t\end{array}\right. \quad E. \quad F. \quad G. \quad H.
$$
$$
\begin{array}{l}\mathrm{在直线上任取一点}(x_0,\;y_0,\;z_0),\\\mathrm{例如},取x_0=1\rightarrow\left\{\begin{array}{l}y_0+z_0+2=0\\y_0-3z_0-6=0\end{array}\right.\rightarrow y_0=0,z_0=-2,\\\mathrm{得点坐标}(1,0,-2),\mathrm{因所求直线与两平面的法向量者垂直},\mathrm{可取}\\\;\;\;\;\;\;\;\;\;\;\;\;\;\;\;\;\;\;\;\;\;\;\;\;\;\;\;\;\;\;\;\;\;\;\;\;\;\;\;\;\;\;\;\;\;\;\;\;\;\;\;\;\;\;\;\;\;\;\overrightarrow s=\overrightarrow{n_1}×\overrightarrow{n_2}=\begin{vmatrix}\overrightarrow i&\overrightarrow j&\overrightarrow k\\1&1&1\\2&-1&3\end{vmatrix}=\{4,-1,-3\},\\\mathrm{对称式方程}\frac{x-1}4=\frac{y-0}{-1}=\frac{z+2}{-3},\mathrm{参数方程}\left\{\begin{array}{c}x=1+4t\\y=-1\\z=-2-3t\end{array}\right..\end{array}
$$



$$
\mathrm{过点}(3,4,-4)\mathrm{且方向角为}\fracπ3,\fracπ4,\fracπ3\mathrm{的直线的对称式方程为}(\;).
$$
$$
A.
\frac{x-3}1=\frac{y-4}{\sqrt2}=\frac{z+4}1 \quad B.\frac{x-3}1=\frac{y-4}{-\sqrt2}=\frac{z+4}1 \quad C.\frac{x-3}1=\frac{y-4}{\sqrt2}=\frac{z+4}{-1} \quad D.\frac{x-3}{-1}=\frac{y-4}{\sqrt2}=\frac{z+4}1 \quad E. \quad F. \quad G. \quad H.
$$
$$
\begin{array}{l}\mathrm{设所求直线的方向向量为}:\{A,B,C\},\mathrm{则由题意得到}:\\\cos\fracπ3=\frac A{\sqrt{A^2+B^2+C^2}}=\frac12,\cos\fracπ4=\frac B{\sqrt{A^2+B^2+C^2}}=\frac{\sqrt2}2\\\cos\fracπ3=\frac C{\sqrt{A^2+B^2+C^2}}=\frac12,\\\mathrm{则方向向量为}:\left\{\frac12\right.,\begin{array}{c}\frac{\sqrt2}2,\end{array}\left.\frac12\right\},\mathrm{对称式方程为}:\frac{x-3}{\displaystyle\frac12}=\frac{y-4}{\displaystyle\frac{\sqrt2}2}=\frac{z+4}{\displaystyle\frac12},\\\mathrm{整理得}:\frac{x-3}1=\frac{y-4}{\sqrt2}=\frac{z+4}1.\end{array}
$$



$$
\mathrm{过点}(2,3,-5)\mathrm{且与直线}\frac{x-2}{-1}=\frac y3=\frac{z+1}4\mathrm{平行的直线的标准方程是}(\;)
$$
$$
A.
\frac{x+2}{-1}=\frac{y-3}3=\frac{z+5}4 \quad B.\frac{x-2}{-1}=\frac{y-3}3=\frac{z+5}{-4} \quad C.\frac{x-2}{-1}=\frac{y+3}3=\frac{z+5}4 \quad D.\frac{x-2}{-1}=\frac{y-3}3=\frac{z+5}4 \quad E. \quad F. \quad G. \quad H.
$$
$$
\begin{array}{l}\mathrm{两条直线平行},\mathrm{方向向量平行},\mathrm{故所求直线的方向向量可取}\overrightarrow n=\{-1,3,4\},\mathrm{又过点}(2,3,-5),\\\mathrm{所以直线的标准方程为}\frac{x-2}{-1}=\frac{y-3}3=\frac{z+5}4\end{array}
$$



$$
\mathrm{过点}M_1(3,-2,1),M_2(-1,0,2)\mathrm{的直线方程为}(\;).
$$
$$
A.
\frac{x+1}4=\frac y{-2}=\frac{z-2}{-1} \quad B.\frac{x+1}4=\frac y2=\frac{z-2}{-1} \quad C.\frac{x+1}4=\frac y{-2}=\frac{z-2}1 \quad D.\frac{x+1}{-4}=\frac y2=\frac{z-2}{-1} \quad E. \quad F. \quad G. \quad H.
$$
$$
\begin{array}{l}n=\{3-(-1),-2-0,1-2\}=\{4,-2,-1\},点M_2\mathrm{在直线上},\mathrm{则直线方程式为}\frac{x+1}4=\frac y{-2}=\frac{z-2}{-1}.\\\end{array}
$$



$$
若\left\{\begin{array}{l}2x+3y-z+D=0\\2x-2y+2z-6=0\end{array}\right.与x\mathrm{轴有交点},则D=(\;).
$$
$$
A.
-6 \quad B.6 \quad C.-3 \quad D.3 \quad E. \quad F. \quad G. \quad H.
$$
$$
\mathrm{根据题意当}y=z=0时,\mathrm{方程有解}\;,则:\;2x+D=0,2x-6=0\;,\mathrm{得到}D=-6.
$$



$$
\mathrm{设空间直线方程为}\left\{\begin{array}{l}x=0\\2y-z=0\end{array}\right.\mathrm{则该直线必}(\;).
$$
$$
A.
\mathrm{过原点且垂直于}x轴 \quad B.\mathrm{过原点且垂直于}y轴 \quad C.\mathrm{过原点且垂直于}z轴 \quad D.\mathrm{过原点且平行于}x轴 \quad E. \quad F. \quad G. \quad H.
$$
$$
\mathrm{由题意得},\mathrm{该直线肯定过原点},又x=0,\mathrm{所以该直线在平面}yOz内,\mathrm{所以垂直于}x轴
$$



$$
\mathrm{过点}(0,2,4)\mathrm{且与两平面}x+2z=1和y-3z=2\mathrm{平行的直线方程为}().
$$
$$
A.
\frac x{-2}=\frac{y-2}3=\frac{z-4}1 \quad B.\frac x{-2}=\frac{y-2}3=\frac{z+4}1 \quad C.\frac x{-2}=\frac{y-2}{-3}=\frac{z-4}1 \quad D.\frac x{-2}=\frac{y+2}3=\frac{z+4}1 \quad E. \quad F. \quad G. \quad H.
$$
$$
\begin{array}{l}\mathrm{该直线的方向向量}\overrightarrow s\mathrm{与两平面的法向量}\overrightarrow{n_1}与\overrightarrow{n_2}\mathrm{都垂直}.\mathrm{因为}\\\overrightarrow s=\overrightarrow{n_1}×\overrightarrow{n_2}=\begin{vmatrix}\overrightarrow i&\overrightarrow j&\overset{\boldsymbol\rightarrow}{\mathbf k}\\1&0&2\\0&1&-3\end{vmatrix}=\{-2,3,1\},\\\mathrm{故所求直线的方程为}\frac x{-2}=\frac{y-2}3=\frac{z-4}1\end{array}
$$



$$
\mathrm{过点}(-1,2,1)\mathrm{且平行直线}\left\{\begin{array}{l}x+y-2z-1=0\\x+2y-z+1=0\end{array}\right.\mathrm{的直线方程为}(\;).
$$
$$
A.
\frac{x+1}3=\frac{y-2}1=\frac{z-1}1 \quad B.\frac{x+1}3=\frac{y-2}{-1}=\frac{z-1}1 \quad C.\frac{x+1}{-3}=\frac{y-2}{-1}=\frac{z-1}1 \quad D.\frac{x+1}3=\frac{y-2}{-1}=\frac{z-1}{-1} \quad E. \quad F. \quad G. \quad H.
$$
$$
\begin{array}{l}\mathrm{直线}\left\{\begin{array}{l}x+y-2z-1=0\\x+2y-z+1=0\end{array}\right.\mathrm{的方向向量为}:\\n_1=\begin{vmatrix}i&j&k\\1&1&-2\\1&2&-1\end{vmatrix}=3i-j+k,\mathrm{由题意可得}n=n_1=\{3,-1,1\},\mathrm{则直线}\\\mathrm{方程为}\frac{x+1}3=\frac{y-2}{-1}=\frac{z-1}1\end{array}
$$



$$
\mathrm{两平面}x-2y-z=3,2x-4y-2z=5\mathrm{各自与平面}x+y-3z=0\mathrm{的交线是}(\;).
$$
$$
A.
\mathrm{相交的} \quad B.\mathrm{平行的} \quad C.\mathrm{异面的} \quad D.\mathrm{重合的} \quad E. \quad F. \quad G. \quad H.
$$
$$
\begin{array}{l}\mathrm{两平面}x-2y-z=3\mathrm{与平面}x+y-3z=0\mathrm{的交线为}\\\left\{\begin{array}{l}x-2y-z=3\\x+y-3z=0\end{array}\right.,\mathrm{其对称式方程为}\frac{x-1}1=\frac{y+1}{\displaystyle\frac27}=\frac z{\displaystyle\frac37}\\\mathrm{同理可得}2x-4y-2z=5\mathrm{与平面}x+y-3z=0\mathrm{的交线方程为}\\\frac{x-1}1=\frac{y+{\displaystyle\frac{11}{14}}}{\displaystyle\frac27}=\frac{z-{\displaystyle\frac1{14}}}{\displaystyle\frac37},\mathrm{所以两线是平行的}.\end{array}
$$



$$
\mathrm{用参数方程表示直线}\left\{\begin{array}{l}2x-y-3z+2=0\\x+2y-z-6=0\end{array}\right.\mathrm{下列正确的是}(\;).
$$
$$
A.
\left\{\begin{array}{c}x=7t+2/5\\y=-t+14/5\\z=5t\end{array}\right. \quad B.\left\{\begin{array}{c}x=7t+5\\y=-t+14/5\\z=5t\end{array}\right. \quad C.\left\{\begin{array}{c}x=7t+2/5\\y=t+14/5\\z=t\end{array}\right. \quad D.\left\{\begin{array}{c}x=7t+2/5\\y=t+14/5\\z=5t\end{array}\right. \quad E. \quad F. \quad G. \quad H.
$$
$$
\begin{array}{l}\mathrm{设直线的方向向量为}n,\mathrm{则可取}\\n=n_1× n_2=\begin{vmatrix}i&j&k\\2&-1&-3\\1&2&-1\end{vmatrix}=7i-j+5k.\\\mathrm{再在直线上取一点},\mathrm{例如},\mathrm{可令}z=0,得\\\left\{\begin{array}{l}2x-y+2=0\\x+2y-6=0\end{array}\right.⇒ x=\frac25,y=\frac{14}5,\\\mathrm{于是},\mathrm{直线的对称式方程}\frac{x-2/5}7=\frac{y-14/5}{-1}=\frac{z-0}5,\\\mathrm{参数方程为}\left\{\begin{array}{c}x=7t+2/5\\y=-t+14/5\\z=5t\end{array}\right.\end{array}
$$



$$
\mathrm{设有直线}l_1:\frac{x-1}1=\frac{y-5}{-2}=\frac{z+8}1与l_2:\left\{\begin{array}{l}x-y=6\\2y+z=3\end{array}\right.,则l_1与l_2\mathrm{的夹角为}(\;).
$$
$$
A.
\fracπ6 \quad B.\fracπ4 \quad C.\fracπ3 \quad D.\fracπ2 \quad E. \quad F. \quad G. \quad H.
$$
$$
\begin{array}{l}\mathrm{两直线的方向向量分别为}\{1,-2,1\}和\{1,1,-2\},\\\mathrm{所以可得cos}θ=\frac{\begin{vmatrix}1-2-2\end{vmatrix}}{\sqrt{1+1+4}\sqrt{1+1+4}}=\frac12,\mathrm{夹角为}θ=\fracπ3.\end{array}
$$



$$
\mathrm{过点}(-3,2,5)\mathrm{且与两平面}x-4z=3和2x-y-5z=1\mathrm{的交线平行的直线方程为}().
$$
$$
A.
\frac{x+3}4=\frac{y-2}3=\frac{z-5}1 \quad B.\frac{x+3}4=\frac{y-2}{-3}=\frac{z-5}1 \quad C.\frac{x+3}4=\frac y3=\frac{z-5}1 \quad D.\frac{x+3}{-4}=\frac{y-2}3=\frac{z+5}{-1} \quad E. \quad F. \quad G. \quad H.
$$
$$
\begin{array}{l}\mathrm{设所求直线的方向向量为}\overrightarrow s=\{m,n,p\},\\\mathrm{根所题意知}\overrightarrow s⟂\overrightarrow{n_1},\overrightarrow s⟂\overrightarrow{n_2},\\\;\;\;\;\;\;\;\;\;\;\;\;\;\;\;\;\;\;\;\;\;\;\;\;\;\;\;\;\;\;\;\;\;\;\;\;\;\;\;\;\;\;\;\;取\overrightarrow s=\overrightarrow{n_1}×\overrightarrow{n_2}=\begin{vmatrix}\overrightarrow i&\overrightarrow j&\overrightarrow k\\1&0&-4\\2&-1&-5\end{vmatrix}=\{-4,-3,-1\},\\\mathrm{所求直线方程}\;\frac{x+3}4=\frac{y-2}3=\frac{z-5}1\end{array}
$$



$$
\mathrm{直线}\left\{\begin{array}{l}x-2y+z-9=0\\3x-6y+z-27=0\end{array}\right.\mathrm{的对称式方程为}(\;).
$$
$$
A.
\frac{x-9}4=\frac{y-1}2=\frac z1 \quad B.\frac{x-9}4=\frac{y-1}2=\frac z0 \quad C.\frac{x-9}4=\frac y2=\frac z1 \quad D.\frac{x-9}4=\frac y2=\frac z0 \quad E. \quad F. \quad G. \quad H.
$$
$$
\begin{array}{l}\mathrm{设直线方向向量为}n,\mathrm{则可取}:\\n=n_1× n_2=\begin{vmatrix}i&j&k\\1&-2&1\\3&-6&1\end{vmatrix}=4i+2j,\\\mathrm{在直线上取一点},令z=0,\mathrm{则带入直线}\left\{\begin{array}{l}x-2y+z-9=0\\3x-6y+z-27=0\end{array}\right.,\mathrm{得到}:x=9,y=0,\mathrm{则对称式方程为}\\\frac{x-9}4=\frac y2=\frac z0\end{array}
$$



$$
\mathrm{过点}(0,2,4)\mathrm{且与两平面}x+2z=1和y-3z=2\mathrm{平行的直线方程是}(\;).
$$
$$
A.
\frac x{-2}=\frac{y-2}3=\frac{z-4}1 \quad B.\frac x2=\frac{y-2}3=\frac{z-4}1 \quad C.\frac x2=\frac{y+2}3=\frac{z+4}1 \quad D.\frac x{-2}=\frac{y-2}{-3}=\frac{z-4}1 \quad E. \quad F. \quad G. \quad H.
$$
$$
\begin{array}{l}\mathrm{该直线的方向向量}n\mathrm{与两平面的法向量}n_1,n_2\mathrm{都垂直},\mathrm{因为}\\n=n_1× n_2=\begin{vmatrix}i&j&k\\1&0&2\\0&1&-3\end{vmatrix}=\{-2,3,1\},\\\mathrm{故所求直线方程为}:\frac x{-2}=\frac{y-2}3=\frac{z-4}1\end{array}
$$



$$
\mathrm{直线}\frac{x+1}3=\frac{y+3}2=\frac{\displaystyle z}1与\frac x1=\frac{y+5}2=\frac{\displaystyle z-2}7\mathrm{的关系是}(\;).
$$
$$
A.
\mathrm{平行} \quad B.\mathrm{相交} \quad C.\mathrm{垂直不相交} \quad D.\mathrm{不垂直不相交} \quad E. \quad F. \quad G. \quad H.
$$
$$
\begin{array}{l}s_1=\{3,2,1\},s_2=\{1,2,7\},\\s_1· s_2=14\neq0,\mathrm{所以不垂直},\mathrm{且他们也不平行}\\又\begin{vmatrix}3&2&1\\1&2&7\\-1-0&-3-(-5)&0-2\end{vmatrix}=-60\neq0,\mathrm{所以不相交}\end{array}
$$



$$
点A(-1,2,0)\mathrm{在平面}x+2y-z+3=0\mathrm{上的投影为}().
$$
$$
A.
(2,0,-1) \quad B.(2,0,1) \quad C.(-2,0,-1) \quad D.(-2,0,1) \quad E. \quad F. \quad G. \quad H.
$$
$$
\begin{array}{l}\mathrm{平面的}x+2y-z+3=0\mathrm{的法向量为}\{1,2,-1\},\mathrm{则过点}A\;\mathrm{且垂直于该平面的直线方程为}\\\frac{x+1}1=\frac{y-2}2=\frac z{-1},\mathrm{该直线于平面的交点即为点}A\mathrm{在平面上的投影}.\mathrm{联立两方程解的交点为}(-2,0,1)\end{array}
$$



$$
\mathrm{直线}\left\{\begin{array}{l}x=3z-5\\y=2z-8\end{array}\right.\mathrm{的对称式方程为}(\;).\;\;
$$
$$
A.
\frac{x+5}3=\frac{y+8}2=\frac z1 \quad B.\frac{x-5}3=\frac{y-8}2=\frac z1 \quad C.\frac{x+5}3=\frac{y+8}{-2}=\frac z1 \quad D.\frac{x-5}{-3}=\frac{y-8}{-2}=\frac z1 \quad E. \quad F. \quad G. \quad H.
$$
$$
\begin{array}{l}\mathrm{设直线的方向向量为}n,则:\\n=n_1× n_2=\begin{vmatrix}i&j&k\\1&0&-3\\0&1&-2\end{vmatrix}=3i+2j+k,\mathrm{在直线上任取一点},令z=0,\mathrm{由直线方程}\left\{\begin{array}{l}x=3z-5\\y=2z-8\end{array}\right.得\\x=-5,y=-8,\mathrm{对称式方程为}:\frac{x+5}3=\frac{y+8}2=\frac z1.\end{array}
$$



$$
\mathrm{过点}(2,2,4)\mathrm{且与平面}x+z=1及x+y-3z=2\mathrm{都平行的直线方程为}(\;).
$$
$$
A.
\frac{x-2}{-2}=\frac{y-2}4=\frac{\displaystyle z-4}1 \quad B.\frac{x-2}{-1}=\frac{y-2}{-4}=\frac{\displaystyle z-4}1 \quad C.\frac{x-2}{-1}=\frac{y-2}4=\frac{\displaystyle z-4}1 \quad D.\frac{x-2}{-1}=\frac{y-2}4=\frac{\displaystyle z-4}{-2} \quad E. \quad F. \quad G. \quad H.
$$
$$
\begin{array}{l}\mathrm{该直线方向向量}n\mathrm{与两平面的法向量垂直},则\\n=n_1× n_2=\begin{vmatrix}i&j&k\\1&0&1\\1&1&-3\end{vmatrix}=-i+4j+k,\mathrm{则直线方程为}:\frac{x-2}{-1}=\frac{y-2}4=\frac{\displaystyle z-4}1(或\left\{\begin{array}{l}x+z=6\\x+y-3z=-8\end{array}\right.).\\\end{array}
$$



$$
\mathrm{已知点}P(1,3,-4),\mathrm{平面}{\textstyle\prod_{}}\mathrm{的方程为}3x+y-2z=0,\mathrm{则与点}P\mathrm{关于平面}{\textstyle\prod_{}}\mathrm{对称的}Q\mathrm{点的坐标是}(\;).
$$
$$
A.
(5,-1,0) \quad B.(5,1,0) \quad C.(-5,-1,0) \quad D.(-5,1,0) \quad E. \quad F. \quad G. \quad H.
$$
$$
\begin{array}{l}过P\mathrm{做垂直与平面}π\mathrm{的直线}L:\left\{\begin{array}{c}x=1+3λ\\y=3+λ\\z=-4-2λ\end{array}\right.·\\\mathrm{求直线}L\mathrm{与平面}π\mathrm{的交点}M\mathrm{之坐标}(x_0,y_0,z_0),由\\3x+y-2z=3(1+3λ)+(3+λ)-2(-4-2λ)=14λ+14=0\\\;\;\;\;\;\;\;\;\;\;\;\;\;\;\;\;\;\;⇒λ=-1⇒ x_0=-2,y_0=2,z_0=-2.\\\mathrm{令与点}P\mathrm{关于平面}π\mathrm{对称点}Q\mathrm{的坐标为}(x_1,y_1,z_1),\mathrm{利用中点公式有}\\\frac{1+x_1}2=-2,\frac{3+y_1}2=2,\frac{-4+z_1}2=-2,\\\;\;\;\;\;\;\;\;\;\;\;⇒ x_1=-5,y_1=1,z_1=0,\\即Q\mathrm{点坐标为}(-5,1,0).\end{array}
$$



$$
\mathrm{若两直线}\frac{x-1}1=\frac{y+1}2=\frac{z-1}λ,\frac{x+1}1=\frac{y-1}1=\frac z1\mathrm{相交},则λ=(\;).
$$
$$
A.
1 \quad B.\frac23 \quad C.-\frac54 \quad D.\frac54 \quad E. \quad F. \quad G. \quad H.
$$
$$
s_1=\{1,2,λ\},s_2=\{1,1,1\},\mathrm{由两线相交得}\begin{vmatrix}-1-1&1+1&0-1\\1&2&λ\\1&1&1\end{vmatrix}=\begin{vmatrix}-2&2&-1\\1&2&λ\\1&1&1\end{vmatrix}=0,λ=\frac54.
$$



$$
\mathrm{直线}L:\left\{\begin{array}{l}x+y-z-1=0\\x-y+z+1=0\end{array}\right.,\mathrm{在平面}{\textstyle\prod_{}}:x+y+z=0\mathrm{上的投影直线的方程为}(\;).
$$
$$
A.
\left\{\begin{array}{l}y-z=1\\x+y+z=0\end{array}\right. \quad B.\left\{\begin{array}{l}y+z=1\\x+y+z=0\end{array}\right. \quad C.\left\{\begin{array}{l}y-z=1\\x+2y-z=0\end{array}\right. \quad D.\left\{\begin{array}{l}y-z=2\\x+y+z=0\end{array}\right. \quad E. \quad F. \quad G. \quad H.
$$
$$
\begin{array}{l}设{\textstyle\prod_1}\mathrm{是过直线}L\mathrm{且垂直于平面}\;{\textstyle\prod_\;}\mathrm{的平面},则{\textstyle\prod_1}与{\textstyle\prod_\;}\mathrm{的交线即为}L在{\textstyle\prod_\;}\mathrm{上的投影直线},\mathrm{下面求}{\textstyle\prod_1}\mathrm{的方程}.\\\;设{\textstyle\prod_\;的}\mathrm{法向量为}\overrightarrow n,\mathrm{直线}L\mathrm{的方向向量为}\overrightarrow v,{\textstyle\prod_1}\mathrm{的法向量为}\overrightarrow{n_1},\mathrm{则有}\overrightarrow{n_1}⟂\overrightarrow n,\\\mathrm{故可取}\overrightarrow{n_1}=\overrightarrow n×\overrightarrow v=\begin{vmatrix}\overrightarrow i&\overrightarrow j&\overrightarrow k\\1&1&1\\0&-2&-2\end{vmatrix}=\overrightarrow{2j}-\overrightarrow{2k},\\\mathrm{在直线}L\mathrm{上取一点}P(0,1,0),\mathrm{得平面}{\textstyle\prod_1}\mathrm{的方程为}2(y-1)-2z=0,\\\mathrm{再把它与平面}x+y+z=0\mathrm{联立},得\left\{\begin{array}{l}y-z=1\\x+y+z=0\end{array}\right.\\\mathrm{此即为所求投影直线方程}.\end{array}
$$



$$
\begin{array}{l}\mathrm{设有两条直线}L_1:\left\{\begin{array}{l}x+y-1=0\\x-y+z+1=0\end{array}\right.,L_2:\left\{\begin{array}{l}2x-y+z-1=0\\x+y-z+1=0\end{array}\right.\mathrm{及平面}\;π:x+y+z=0,\mathrm{则在}π\mathrm{上且与直线}L_1和L_2\mathrm{相交的}\\\mathrm{直线方程为}(\;).\end{array}
$$
$$
A.
x=y=z \quad B.\frac x1=\frac{y-{\displaystyle\frac12}}2=\frac{z+{\displaystyle\frac12}}{-3} \quad C.\frac x1=\frac{y+{\displaystyle\frac12}}2=\frac{z-{\displaystyle\frac12}}{-3} \quad D.\frac x2=\frac{y-{\displaystyle1}}3=\frac{z-2}1 \quad E. \quad F. \quad G. \quad H.
$$
$$
\begin{array}{l}L_1:\left\{\begin{array}{l}x+y-1=0\\x-y+z+1=0\end{array}\;,\;L_2:\left\{\begin{array}{l}2x-y+z-1=0\\x+y-z+1=0\end{array}\right.\right.,\mathrm{与平面交点分别为}(\frac12,\frac12,-1)和(0,-\frac12,\frac12),\\\mathrm{两点所确定的直线}\frac x1=\frac{y+{\displaystyle\frac12}}2=\frac{z-{\displaystyle\frac12}}{-3}\mathrm{即为所求}.\end{array}
$$



$$
\mathrm{设平面}2x+y+kz+1=0\mathrm{平行于平面}4x+ky+4z-3=0,则k=().
$$
$$
A.
0 \quad B.1 \quad C.2 \quad D.3 \quad E. \quad F. \quad G. \quad H.
$$
$$
\mathrm{由题意得}:\frac24=\frac1k=\frac k4,\mathrm{解得}:k=2.
$$



$$
\mathrm{设平面}2x+y+kz+1=0\mathrm{到原点}O\mathrm{的距离为}\frac13,则k=().
$$
$$
A.
±2 \quad B.2 \quad C.±1 \quad D.1 \quad E. \quad F. \quad G. \quad H.
$$
$$
\mathrm{原点}O=(0,0,0),则:d=\frac{\left|0+0+0+1\right|}{\sqrt{2^2+1+k^2}}=\frac13,\mathrm{解得}:k=±2.
$$



$$
\mathrm{两平面}-x+2y-z+1=0,y+3z-1=0\mathrm{的位置关系为}().
$$
$$
A.
\mathrm{两平面相交},\mathrm{夹角为}θ=arc\cos\frac1{\sqrt{60}} \quad B.\mathrm{两平面相交},\mathrm{夹角为}θ=\fracπ6 \quad C.\mathrm{两平面相交},\mathrm{夹角为}θ=\fracπ3 \quad D.\mathrm{两平面平行} \quad E. \quad F. \quad G. \quad H.
$$
$$
\begin{array}{l}\overrightarrow{n_1}=\{-1,2,-1\},\overrightarrow{n_2}=\{0,1,3\}且\cosθ=\frac{\left|-1×0+2×1-1×3\right|}{\sqrt{(-1)^2+2^2+(-1)^2·\sqrt{1^2+3^2}}}=\frac1{\sqrt{60}},\\\mathrm{故两平面相交},\mathrm{夹角为}θ=arc\cos\frac1{\sqrt{60}}.\end{array}
$$



$$
\mathrm{过点}(1,2,-1)\mathrm{且与直线}\frac{x-1}2=y+1=\frac{z-1}{-1}\mathrm{垂直的平面方程为}(\;\;\;)
$$
$$
A.
2x+y-z-5=0 \quad B.2x+y-z+5=0 \quad C.2x-y-z-5=0 \quad D.2x-y+z-5=0 \quad E. \quad F. \quad G. \quad H.
$$
$$
\begin{array}{l}\mathrm{平面的法向量}n=\{2,1,-1\},\\\mathrm{由点法式方程得}:2·(x-1)+1·(y-2)-1·(z+1)=0\;\;\;\\\mathrm{整理得},2x+y-z-5=0.\end{array}
$$



$$
\mathrm{已知三平面的方程分别为}:π_1:x-5y+2z+1=0;π_2:3x-2y+5z+8=0;π_3:2x-10y+4z-9=0\mathrm{则必有}().
$$
$$
A.
π_1与π_2\mathrm{平行} \quad B.π_1与π_3\mathrm{平行} \quad C.π_2与π_3\mathrm{垂直} \quad D.π_2与π_3\mathrm{平行} \quad E. \quad F. \quad G. \quad H.
$$
$$
\mathrm{显然}π_1与π_3\mathrm{的平行},\mathrm{其它选项均不正确}.
$$



$$
点(1,2,1)\mathrm{到平面}x+y+z+2=0\mathrm{的距离为}().
$$
$$
A.
1 \quad B.2 \quad C.2\sqrt3 \quad D.\sqrt3 \quad E. \quad F. \quad G. \quad H.
$$
$$
\begin{array}{l}\mathrm{由距离公式},得\\d=\frac{\left|1×1+2×1+1×1+2\right|}{\sqrt{1^2+1^2+1^2}}=2\sqrt3.\end{array}
$$



$$
\mathrm{平行于}x轴,\mathrm{且过点}P(3,-1,2)及Q(0,1,0)\mathrm{的平面方程是}().
$$
$$
A.
y+z=1 \quad B.y-z=1 \quad C.-y+z=1 \quad D.y+z=0 \quad E. \quad F. \quad G. \quad H.
$$
$$
\begin{array}{l}\mathrm{平行于}x轴,\mathrm{则可设平面为}By+Cz+D=0,\mathrm{代入两点得}\left\{\begin{array}{l}-B+2C+D=0\\B+D=0\end{array}\right.,\\\mathrm{解得}B=C=-D,\mathrm{所以该平面方程为}y+z=1.\end{array}
$$



$$
\mathrm{平面}Ax+By+Cz+D=0过z轴,则\;
$$
$$
A.
A=D=0 \quad B.B=0,C\neq0 \quad C.C=D=0 \quad D.B=C=0 \quad E. \quad F. \quad G. \quad H.
$$
$$
\mathrm{平面}Ax+By+Cz+D=0过z轴,\mathrm{可得平面平行于}z\mathrm{轴且过原点}.\mathrm{所以可得}C=D=0;
$$



$$
\mathrm{已知原点到平面}2x-y+kz=6\mathrm{的距离等于}2,则k\mathrm{的值为}().
$$
$$
A.
±2 \quad B.±1 \quad C.2 \quad D.1 \quad E. \quad F. \quad G. \quad H.
$$
$$
d=\frac{\left|0-0+0-6\right|}{\sqrt{2^2+(-1)^2+k^2}}=2,\mathrm{解得}:k=±2.
$$



$$
\mathrm{平面}3x-3y=0是\;().
$$
$$
A.
\mathrm{平行于}xOy\mathrm{平面} \quad B.\mathrm{平行于}z轴,\mathrm{但不通过}z轴 \quad C.\mathrm{垂直于}y轴 \quad D.\mathrm{通过}z轴 \quad E. \quad F. \quad G. \quad H.
$$
$$
\mathrm{平面的法线为}\{1,-1,0\},\mathrm{所以平面是平行于}z轴,\mathrm{又因为经过原点},\mathrm{所以通过}z轴.
$$



$$
\mathrm{平面}Ax+By+Cz+D=0过x轴,则().
$$
$$
A.
A=D=0 \quad B.B=0,C\neq0 \quad C.B\neq0,C=0 \quad D.A=0,C=0 \quad E. \quad F. \quad G. \quad H.
$$
$$
\mathrm{平面}Ax+By+Cz+D=0\mathrm{过轴},\mathrm{可得平面平行于}x\mathrm{轴且过原点}.\mathrm{所以可得}A=D=0;
$$



$$
\mathrm{平面}3x-5y+1=0().
$$
$$
A.
\mathrm{平行于}zOx\mathrm{平面} \quad B.\mathrm{平行于}z轴 \quad C.\mathrm{垂直于}y轴 \quad D.\mathrm{垂直于}x轴 \quad E. \quad F. \quad G. \quad H.
$$
$$
\mathrm{平面法向量为}\{3,-5,0\}\mathrm{垂直于}z轴,\mathrm{所以平面平行于}z轴.
$$



$$
\mathrm{过点}M_0(2,-1,3)\mathrm{且与连接坐标原点及点}M_0\mathrm{的线段}OM_0\mathrm{垂直的平面方程是}()
$$
$$
A.
2x-y+3z-12=0 \quad B.2x-y+3z-14=0 \quad C.2x-y+3z+10=0 \quad D.2x-y+3z+11=0 \quad E. \quad F. \quad G. \quad H.
$$
$$
\begin{array}{l}\mathrm{注意到}\overrightarrow{OM_0}\mathrm{即为所求平面的法向量},由\;\overrightarrow n=\overrightarrow{OM_0}=±\{2,-1,3\},\;\mathrm{根据点法式平面方程},\mathrm{所求平面方程为}\;,\\±\lbrack2(x-2)-(y+1)+3(z-3)\rbrack=0,即2x-y+3z-14=0.\end{array}
$$



$$
点M(1,-2,1)\mathrm{到平面}x-2y+2z-10=0\mathrm{的距离为}(),
$$
$$
A.
1 \quad B.2 \quad C.3 \quad D.\frac13 \quad E. \quad F. \quad G. \quad H.
$$
$$
\mathrm{根据点到平面的距离公式可得}d=\frac{\left|1+4+2-10\right|}{\sqrt{1+4+4}}=1.
$$



$$
\mathrm{设平面}2x+y+kz+1=0\mathrm{垂直于平面}-2x+y+z+2=0,则k=().
$$
$$
A.
1 \quad B.2 \quad C.3 \quad D.0 \quad E. \quad F. \quad G. \quad H.
$$
$$
\mathrm{由题意可得}:2·(-2)+1·1+k=0,\mathrm{解得}:k=3.
$$



$$
\mathrm{已知平面}Ax+By+Cz+D=0\mathrm{过点}(k,k,0)与(2k,2k,0),k\neq0\mathrm{且垂直于}xOy\mathrm{平面},\mathrm{则其系数满足}()
$$
$$
A.
A=-B,C=D=0 \quad B.B=-C,A=D=0 \quad C.C=-A,B=D=0 \quad D.C=A,B=D=0 \quad E. \quad F. \quad G. \quad H.
$$
$$
\begin{array}{l}\mathrm{两点代入平面方程得}\left\{\begin{array}{l}Ak+Bk+D=0\\2Ak+2Bk+D=0\end{array}\right.\mathrm{可得}\left\{\begin{array}{l}D=0\\A=-B\end{array}\right.,\mathrm{平面又垂直于}xOy\mathrm{平面},\mathrm{所以}C=0,\mathrm{即系数满足}\\A=-B,C=D=0.\end{array}
$$



$$
\mathrm{过点}(1,5,-2)\mathrm{与平面}2x-7y+z-8=0\mathrm{平行的平面方程为}(\;\;\;)\;
$$
$$
A.
2x-7y+z-35=0 \quad B.2x-7y-z+35=0 \quad C.2x-7y+z+5=0 \quad D.2x-7y+z+35=0 \quad E. \quad F. \quad G. \quad H.
$$
$$
\mathrm{由点法式方程得}:2·(x-1)-7·(y-5)+1·(z+2)=0,\;\mathrm{整理得},2x-7y+z+35=0
$$



$$
\mathrm{过点}(1,0,-1)\mathrm{且垂直于直线}\left\{\begin{array}{l}x+2y-z-1=0\\2x-y+z+1=0\end{array}\right.\mathrm{的平面方程为}(\;\;\;)
$$
$$
A.
x+3y-5z-6=0 \quad B.x-3y+5z-6=0 \quad C.x-3y-5z+6=0 \quad D.x-3y-5z-6=0 \quad E. \quad F. \quad G. \quad H.
$$
$$
\begin{array}{l}\mathrm{直线的方向向量}s=\begin{vmatrix}i&j&k\\1&2&-1\\2&-1&1\end{vmatrix}=\left\{1,-3,-5\right\},\\\mathrm{也就是所求平面的法向量},\;\;\;\\\mathrm{由点法式方程得}:\;1·(x-1)-3·(y-0)-5·(z+1)=0,\;\\\mathrm{整理得},x-3y-5z-6=0.\end{array}
$$



$$
\mathrm{若平面}x+2y-kz=1\mathrm{与平面}y-z=3成\fracπ4角,则k=().
$$
$$
A.
\frac14 \quad B.\frac34 \quad C.-\frac14 \quad D.-\frac34 \quad E. \quad F. \quad G. \quad H.
$$
$$
\begin{array}{l}n_1=\{1,2,-k\},n_2=\{0,1,-1\},则:\cos\fracπ4=\frac{\left|2+k\right|}{\sqrt{1+4+k}^2\sqrt{1+1}}=\frac{\sqrt2}2,\;\\\mathrm{解得}:k=\frac14.\end{array}
$$



$$
\mathrm{通过}x\mathrm{轴且与平面}2x+y-z=0\mathrm{垂直的平面方程为}\;(\;\;\;\;\;)
$$
$$
A.
y-z=0 \quad B.y+z=0 \quad C.x-z=0 \quad D.x-y=0 \quad E. \quad F. \quad G. \quad H.
$$
$$
\begin{array}{l}\mathrm{平面经过}x轴,\mathrm{故可设平面的方程为}By+Cz=0,\\\;\mathrm{已知平面的法向量}n=\left\{2,1,-1\right\},\;\;\mathrm{由条件知},2·0+1· B+(-1)· C=0,⇒ B=C,\\\mathrm{故所求平面为}y+z=0.\end{array}
$$



$$
\mathrm{两平面}2x-y+z-1=0,-4x+2y-2z-1=0\;\mathrm{的位置关系为}().
$$
$$
A.
\mathrm{两平面平行但不重合} \quad B.\mathrm{两平面重合} \quad C.\mathrm{两平面相交},\mathrm{夹角为}θ=arc\cos\frac1{\sqrt{60}} \quad D.\mathrm{两平面相交},\mathrm{夹角为}θ=arc\cos\frac2{\sqrt{60}} \quad E. \quad F. \quad G. \quad H.
$$
$$
\overrightarrow{n_1}=\{2,-1,1\},\overrightarrow{n_2}=\{-4,2,-2\}且\;\;\;\frac2{-4}\;=\frac{-1}2\;=\frac1{-2}\;,\;\;又M(1,1,0)∈{\textstyle\prod_1},M(1,1,0)\not∈{\textstyle\prod_2},\mathrm{故两平面平行但不重合}.
$$



$$
\mathrm{过点}(2,-3,1)\mathrm{和平面}y+3z+4=0\mathrm{平行的平面方程为}(\;\;\;\;)
$$
$$
A.
y-3z=0. \quad B.y+3z=0. \quad C.3y+z=0. \quad D.3y-z=0. \quad E. \quad F. \quad G. \quad H.
$$
$$
\mathrm{由点法式方程得}:0·(x-2)+1·(y+3)+3·(z-1)=0\;\mathrm{整理得},y+3z=0
$$



$$
点(3,1,-1)\mathrm{到平面}22x+4y-20z-45=0\mathrm{的距离等于}().\;
$$
$$
A.
\frac32 \quad B.\frac12 \quad C.1 \quad D.2 \quad E. \quad F. \quad G. \quad H.
$$
$$
d=\frac{\left|22·3+4·1-20·(-1)-45\right|}{\sqrt{22^2+4^2+(-20)^2}}=\frac32
$$



$$
\mathrm{过点}(4,0,-2)\mathrm{和点}(5,1,7)\mathrm{且平行于}z\mathrm{轴的平面方程为}(\;\;\;\;\;)
$$
$$
A.
x+y-4=0 \quad B.x-y+4=0 \quad C.x-y-4=0 \quad D.x+y+4=0 \quad E. \quad F. \quad G. \quad H.
$$
$$
\begin{array}{l}\mathrm{可设平面的方程为}Ax+By+D=0,\\\mathrm{又经过点}(4,0,-2)\mathrm{和点}(5,1,7),得\;\left\{\begin{array}{l}4A+D=0\\5A+B+D=0\end{array}\right.,⇒ B=-A,D=-4A,\\\mathrm{故所求平面为}x-y-4=0\end{array}
$$



$$
\mathrm{平面}19x-4y+8z+21=0和19x-4y+8z+42=0\mathrm{之间的距离等于}().
$$
$$
A.
1 \quad B.2 \quad C.3 \quad D.4 \quad E. \quad F. \quad G. \quad H.
$$
$$
\mathrm{由于}:\frac{19}{19}=\frac{-4}{-4}=\frac88=1,\mathrm{这两平面平行},又\frac{21}{42}=\frac12,\mathrm{因此相差一个单位},\mathrm{得两平面间距离为}1.
$$



$$
\mathrm{过点}(1,0,1)\mathrm{且垂直于直线}\left\{\begin{array}{l}x+2y-z-1=0\\2x-y+z+1=0\end{array}\right.\mathrm{的平面方程为}(\;\;\;)\;
$$
$$
A.
x-3y-5z+4=0 \quad B.x+3y-5z-6=0 \quad C.x-3y+5z-6=0 \quad D.x-3y-5z+6=0 \quad E. \quad F. \quad G. \quad H.
$$
$$
\begin{array}{l}\mathrm{直线的方向向量}s=\begin{vmatrix}i&j&k\\1&2&-1\\2&-1&1\end{vmatrix}=\left\{1,-3,-5\right\},\mathrm{也就是所求平面的法向量},\;\;\;\\\mathrm{由点法式方程得}:1·(x-1)-3·(y-0)-5·(z-1)=0\;\;\mathrm{整理得},x-3y-5z+4=0.\end{array}
$$



$$
\mathrm{过点}(2,-3,1)\mathrm{与平面}x+4y-3z+7=0\mathrm{平行的平面方程为}(\;\;\;)
$$
$$
A.
x+4y-3z+13=0 \quad B.x+4y-3z-13=0 \quad C.x+4y+3z+13=0 \quad D.x-4y+3z-13=0 \quad E. \quad F. \quad G. \quad H.
$$
$$
\mathrm{由点法式方程得}:1·(x-2)+4·(y+3)-3·(z-1)=0\;\mathrm{整理得},x+4y-3z+13=0
$$



$$
\mathrm{过点}M(2,3,5),\mathrm{且平行于平面}5x-3y+2z-10=0\mathrm{的平面是}(),
$$
$$
A.
5x+3y+2z-11=0 \quad B.5x-3y+2z+11=0 \quad C.5x-3y+2z-11=0 \quad D.5x+3y+2z+11=0 \quad E. \quad F. \quad G. \quad H.
$$
$$
\begin{array}{l}\mathrm{根据两平面平行的性质},\mathrm{可得所求平面的方程为}5x-3y+2z+D=0\;\;,\\\mathrm{且过点}M(2,3,5),得D=-11,\;\;\mathrm{所以所求方程为}5x-3y+2z-11=0.\end{array}
$$



$$
\mathrm{平面}3x-5z+1=0(),
$$
$$
A.
\mathrm{平行于}zOx\mathrm{平面} \quad B.\mathrm{平行于}y轴 \quad C.\mathrm{垂直于}y轴 \quad D.\mathrm{垂直于}x轴 \quad E. \quad F. \quad G. \quad H.
$$
$$
\mathrm{平面法向量为}\{3,0,-5\}\mathrm{垂直于}y轴,\mathrm{所以平面平行于}y轴.
$$



$$
\mathrm{过点}(1,2,1)\mathrm{与向量}S_1=i-2j-3k,S_2=-j-k\mathrm{平行的平面方程为}().
$$
$$
A.
x-y+z=0 \quad B.x-y-z=0 \quad C.x+y+z=0 \quad D.x+y-z=0 \quad E. \quad F. \quad G. \quad H.
$$
$$
\begin{array}{l}\mathrm{设所求平面方程的法向量为}\overrightarrow n=\{A,B,C\},\mathrm{则由题意得}:A-2B-3C=0\;,-B-C=0,\\\mathrm{可得}\overrightarrow n=\{1,-1,1\},\mathrm{又点}(1,2,1)\mathrm{在平面上},\mathrm{得平面方程为}:(x-1)-(y-2)+(z-1)=0,\mathrm{整理得}:x-y+z=0.\end{array}
$$



$$
\mathrm{平面}2x-3y-z+12=0在x轴,y轴,z\mathrm{轴上的截距分别为}().
$$
$$
A.
-6,4,12 \quad B.6,4,12 \quad C.-6,-4,12 \quad D.6,-4,12 \quad E. \quad F. \quad G. \quad H.
$$
$$
\begin{array}{l}当y=z=0时,\mathrm{那么}:x=-6,\mathrm{则平面在轴上截距为}:a=-6;\\当x=z=0时,\mathrm{那么}:y=4,\mathrm{则平面在轴上截距为}:b=4;\;\\当x=y=0时,\mathrm{那么}:z=12,\mathrm{则平面在轴上截距为}:c=12\end{array}
$$



$$
\mathrm{求两平行平面}{\textstyle\prod_1}:10x+2y-2z-5=0和{\textstyle\prod_2}{\textstyle:}{\textstyle5}{\textstyle x}{\textstyle+}{\textstyle y}{\textstyle-}{\textstyle z}{\textstyle-}{\textstyle1}{\textstyle=}{\textstyle0}\mathrm{之间的距离}d=().
$$
$$
A.
\frac{\sqrt3}6 \quad B.\frac{5\sqrt3}6 \quad C.\frac{\sqrt3}5 \quad D.\frac{\sqrt3}{16} \quad E. \quad F. \quad G. \quad H.
$$
$$
\begin{array}{l}\mathrm{可在平面}\prod\nolimits_2\mathrm{上任取一点},\mathrm{该点到平面}\prod\nolimits_1\mathrm{的距离即为这两平面间的距离}.\;\\\mathrm{为此},\mathrm{在平面上取点}(0,1,0),则\;d=\frac{\left|10×0+2×1+)-(-2)×0-5\right|}{\sqrt{10^2+2^2+(-2)^2}}=\frac3{\sqrt{108}}=\frac{\sqrt3}6\;.\end{array}
$$



$$
\mathrm{经过}z\mathrm{轴和点}(1,1,1)\mathrm{的平面方程的为}(\;\;\;)
$$
$$
A.
x+y=0 \quad B.x-z=0 \quad C.x-y=0 \quad D.z-y=0 \quad E. \quad F. \quad G. \quad H.
$$
$$
\begin{array}{l}\mathrm{平面经过}z轴,\mathrm{故可设平面的方程为}Ax+By=0\;\mathrm{又过点}(1,1,1),\\得A=-B,\mathrm{整理得}x-y=0,\end{array}
$$



$$
\mathrm{平行于}x轴,\mathrm{且过点}P(3,-1,2)及Q(0,1,0)\mathrm{的平面方程是}().
$$
$$
A.
y+z=1 \quad B.y-z=1 \quad C.-y+z=1 \quad D.y+z=0 \quad E. \quad F. \quad G. \quad H.
$$
$$
\begin{array}{l}\mathrm{平行于}x轴,\mathrm{则可设平面为}By+Cz+D=0,\\\mathrm{代入两点得}\left\{\begin{array}{l}-B+2C+D=0\\B+D=0\end{array}\right.,\mathrm{解得}B=C=-D,\mathrm{所以该平面方程为}y+z=1.\end{array}
$$



$$
\mathrm{平面是}3x-3y-6=0是(),
$$
$$
A.
\mathrm{平行于}xOy\mathrm{平面} \quad B.\mathrm{平行于}z轴,\mathrm{但不通过}z轴 \quad C.\mathrm{垂直于}y轴 \quad D.\mathrm{通过}z轴 \quad E. \quad F. \quad G. \quad H.
$$
$$
\mathrm{平面的法向量为}\{3,-3,0\},\mathrm{所以平面是平行于}z轴,\mathrm{又因为不过原点},\mathrm{所以不通过}z轴.
$$



$$
\mathrm{过点}M(2,4,-3)\mathrm{且与平面}2x+3y-5z=5\mathrm{平行的平面方程为}().
$$
$$
A.
2x+3y-5z=31 \quad B.2x+3y+5z=31 \quad C.x+3y-5z=17 \quad D.x-3y-5z=17 \quad E. \quad F. \quad G. \quad H.
$$
$$
\begin{array}{l}\mathrm{因为所求平面和已知平面平行},\mathrm{而已知平面的法向量为}\overrightarrow n=\{2,3,-5\}.\\\;\mathrm{设所求平面的法向量为}\overrightarrow n,则\overrightarrow n∥\overrightarrow{n_1},\mathrm{故可取}\overrightarrow n=\overrightarrow{n_1},\\\mathrm{于是},\mathrm{所求平面方程为}\;2(x-2)+3(y-4)-5(z+3)=0\;,\;\;\\即\;2x+3y-5z=31\;.\end{array}
$$



$$
\mathrm{过点}M_0(1,0,2)\mathrm{且与连接坐标原点及点}M_0\mathrm{的线段}OM_0\mathrm{垂直平面方程是}().
$$
$$
A.
x+2z-5=0 \quad B.x+2z+5=0 \quad C.x+y+2z-5=0 \quad D.x+2z-3=0 \quad E. \quad F. \quad G. \quad H.
$$
$$
\begin{array}{l}\mathrm{注意到}\overrightarrow{OM_0}\mathrm{即为所求平面的法向量},由\overrightarrow n=\;\overrightarrow{OM_0}=±\{1,0,2\},\\\;\mathrm{根据点法式平面方程},\mathrm{所求平面方程为}\;±\lbrack(x-1)+0(y-0)+2(z-2)\rbrack=0,\\即x+2z-5=0.\end{array}
$$



$$
\mathrm{经过两平面}4x-y+3z-1=0,x+5y-z+2=0\mathrm{的交线作平面π},\mathrm{并使π与}y\mathrm{轴平行},\mathrm{则平面π的方程为}().
$$
$$
A.
14x-21z-3=0 \quad B.21x-14z+3=0 \quad C.21x+14z-3=0 \quad D.21x+14z+3=0 \quad E. \quad F. \quad G. \quad H.
$$
$$
\begin{array}{l}\mathrm{经过两平面}4x-y+3z-1=0,x+5y-z+2=0\mathrm{的交线方程}\left\{\begin{array}{l}4x-y+3z-1=0\\x+5y-z+2=0\end{array}\right.\\\mathrm{通过此直线的平面束方程为}4x-y+3z-1+λ(x+5y-z+2)=0,\\即(4+λ)x+(5λ-1)y+(3-λ)z-1+2λ=0\;\mathrm{该平面与}y\mathrm{轴平行},\mathrm{则有}5λ-1=0,λ=\frac15,\\\mathrm{所以平面方程为}21x+14z-3=0.\end{array}
$$



$$
\mathrm{过点}A(2,-1,4),B(-1,3,-2)和C(0,2,3)\mathrm{的平面方程为}().
$$
$$
A.
14x+9y-z-15=0 \quad B.14x-9y-z-33=0 \quad C.14x-9y+z-15=0 \quad D.x+9y-z-15=0 \quad E. \quad F. \quad G. \quad H.
$$
$$
\begin{array}{l}解\;\;\overrightarrow{AB}=\{-3,4,-6\},\;\;\overrightarrow{AC}=\{-2,3,-1\}\;,\;\;\\取\;\overrightarrow n\;=\;\overrightarrow{AB}×\overrightarrow{AC}\;=\begin{vmatrix}\overrightarrow i&\overrightarrow j&\overrightarrow k\\-3&4&-6\\-2&3&-1\end{vmatrix}\;=\overrightarrow{14}+\overrightarrow{9j}-\overrightarrow k,\;\;\\\mathrm{所以平面方程为}\;\;14(x-2)+9(y+1)-(z-4)=0\;\;.\;\;\mathrm{化简得}\;14x+9y-z-15=0\;.\end{array}
$$



$$
\mathrm{过点}M_1(1,1,2),M_2(3,2,3),M_3(2,0,3)\mathrm{三点的平面方程为}().
$$
$$
A.
2x-y-3z+5=0 \quad B.2x-y-z+5=0 \quad C.2x-y-3z+1=0 \quad D.2x+y-3z+3=0 \quad E. \quad F. \quad G. \quad H.
$$
$$
\begin{array}{l}\mathrm{设所求平面方程向量为}n,\mathrm{因为}\overrightarrow{M_1M_2}=2i+j+k,\overrightarrow{M_1M_3}=i-j+k\\\mathrm{故可取}:n=\overrightarrow{M_1M_2}×\overrightarrow{M_1M_3}=\begin{vmatrix}i&j&k\\2&1&1\\1&-1&1\end{vmatrix}=2i-j-3k,\mathrm{又平面过点}M_1(1,1,2),\\\mathrm{从而所求平面方程为}:2(x-1)-(y-1)-3(z-2)=0,即:2x-y-3z+5=0.\end{array}
$$



$$
\begin{array}{l}\mathrm{已知三角形的顶点为点}A(2,1,5),B(0,4,-1)和C(3,4,-7),\mathrm{通过点}M(2,-6,3)\mathrm{作一平面}π,使π\mathrm{平行于}\\\triangle ABC\mathrm{所在平面},\mathrm{则方程为}\;().\end{array}
$$
$$
A.
6x+10y+3z+39=0 \quad B.2x+3y+4z-7=0 \quad C.3x+2y-z+18=0 \quad D.4y-x+z-10=0 \quad E. \quad F. \quad G. \quad H.
$$
$$
\begin{array}{l}\mathrm{考察三角形所在平面}\overrightarrow{AB}=\{-2,3,-6\},\overrightarrow{AC}=\{1,3,-12\},\\\mathrm{所以}\overrightarrow n=\overrightarrow{AB}×\overrightarrow{AC}=\{-18,-30,-9\},\mathrm{且过点}A(2,1,5)\mathrm{所以该平面为}\;6(x-2)+10(y-1)+3(z-5)=0,\\\mathrm{整理得}6x+10y+3z-37=0,\mathrm{所以平面}π\mathrm{的方程设为}6x+10y+3z+D=0,\\\mathrm{代入点}M(2,-6,3)\mathrm{解得}D=39,\mathrm{所以}π\mathrm{的方程为}6x+10y+3z+39=0.\end{array}
$$



$$
\mathrm{适合哪一组条件的平面是存在且唯一的}().
$$
$$
A.
\mathrm{过一已知点},\mathrm{且与两条已知异面直线平行} \quad B.\mathrm{垂直于已知平面},\mathrm{且经过一已知直线} \quad C.\mathrm{与两条已知直线垂直},\mathrm{又经过一已知点} \quad D.\mathrm{过一已知点},\mathrm{并与另一已知直线平行} \quad E. \quad F. \quad G. \quad H.
$$
$$
\mathrm{过一已知点},\mathrm{且与两条已知异面直线平行};\mathrm{过此点分别作两异面直线的平行线},\mathrm{所得到两条直线所在的平面即为所求},\mathrm{是唯一的}.
$$



$$
\mathrm{过点}M_0(2,9,-6)\mathrm{且与连接坐标原点及点}M_0\mathrm{的线段}OM_0\mathrm{垂直平面方程是}().
$$
$$
A.
2x+9y-6z-121=0 \quad B.2x+9y-6z+121=0 \quad C.2x+9y-6z-124=0 \quad D.x+9y-6z-121=0 \quad E. \quad F. \quad G. \quad H.
$$
$$
\begin{array}{l}\mathrm{注意到}\overrightarrow{OM_0}\mathrm{即为所求平面的法向量},由\;\overrightarrow n=\overrightarrow{OM_0}=±\{2,9,-6\},\;\\\mathrm{根据点法式平面方程},\mathrm{所求平面方程为}\;±\lbrack2(x-2)+9(y-9)-6(z+6)\rbrack=0,\\即2x+9y-6z-121=0.\end{array}
$$



$$
\mathrm{过点}(1,2,1)\mathrm{且与两直线}\left\{\begin{array}{l}x+2y-z+1=0\\x-y+z-1=0\end{array}\right.\;和\left\{\begin{array}{l}2x-y+z=0\\x-y+z=0\end{array}\right.\mathrm{都平行的平面方程是}().
$$
$$
A.
x-y+z=0 \quad B.x-y-z=0 \quad C.x+y+z-4=0 \quad D.2x-y+z=0 \quad E. \quad F. \quad G. \quad H.
$$
$$
\begin{array}{l}\mathrm{注意到所求平面的法向量}\overrightarrow n\mathrm{与两直线的方向向量}\overrightarrow{s_1}\overrightarrow{s_2}\mathrm{都垂直},\mathrm{故可设}\overrightarrow n=\overrightarrow{s_1}×\overrightarrow{s_2}.\mathrm{因为}\\\overrightarrow{s_1}\;=\begin{vmatrix}\overrightarrow i&\overrightarrow j&\overrightarrow k\\1&2&-1\\1&-1&1\end{vmatrix}\;=\{1,-2,-3\},\overrightarrow{s_2}\;=\begin{vmatrix}\overrightarrow i&\overrightarrow j&\overrightarrow k\\2&-1&1\\1&-1&1\end{vmatrix}\;=\{0,-1,-1\},\;∴\overrightarrow n=\overrightarrow{s_1}×\overrightarrow{s_2}=\begin{vmatrix}\overrightarrow i&\overrightarrow j&\overrightarrow k\\1&-2&-3\\0&-1&-1\end{vmatrix}\;=\{-1,1,-1\}\;.\;\\\;\mathrm{故所求平面方程为}-(x-1)+(y-2)-(z-1)=0,\;\;即x-y+z=0.\end{array}
$$



$$
\mathrm{平行于平面}x+y+z=100\mathrm{且与球面}x^2+y^2+z^2=4\mathrm{相切的平面方程为}().
$$
$$
A.
x+y+z+2\sqrt3=0或x+y+z-2\sqrt3=0 \quad B.x+y+2z+2\sqrt3=0或x+y+2z-2\sqrt3=0 \quad C.x+y+z+2\sqrt3=0或x+2y+z-2\sqrt3=0 \quad D.x+y+z+2=0或x+y+z-2\sqrt3=0 \quad E. \quad F. \quad G. \quad H.
$$
$$
\begin{array}{l}\mathrm{平行于}x+y+z=100\mathrm{的平面方程可设为}\;\;:{\textstyle\prod_{}}:x+y+z+D=0\\\;\mathrm{因为}{\textstyle\prod_{}}与x^2+y^2+z^2=4\mathrm{相切},\mathrm{所以}\;\;{\frac{\left|x+y+z+D\right|}{\sqrt{1^2+1^2+1^2}}}_{(x,y,z)=(0,0,0)}=2,\;\;\\即\left|D\right|=2\sqrt3.\;\;\\\mathrm{所以要求的平面方程为}x+y+z+2\sqrt3=0或x+y+z-2\sqrt3=0.\end{array}
$$



$$
\mathrm{通过}x\mathrm{轴和点}(4,-3,-1)\mathrm{的平面方程为}().
$$
$$
A.
y-3z=0 \quad B.y+3z=0 \quad C.2x+y-3z=0 \quad D.x+2y-3z=0 \quad E. \quad F. \quad G. \quad H.
$$
$$
\begin{array}{l}\mathrm{设所求平面的一般方程为}\;Ax+By+Cz+D=0\;,\;\;\mathrm{因为所求平面通过}x轴,\\\mathrm{且法向量垂直于}x轴,\mathrm{于是法向量在}x\mathrm{轴上的投影为零},\\即A=0,\mathrm{又平面通过原点},\mathrm{所以}D=0,\mathrm{从而方程成为}\;By+Cz=0\;\;\\\mathrm{又因平面过点}(4,-3,-1),\mathrm{因此有}\;-3B-C=0\;,即C=-3B.\;\;\mathrm{以此代入方程}(1),\\\mathrm{再除以}B(B\neq0),\mathrm{便得到所求方程为}\;y-3z=0\;.\end{array}
$$



$$
\mathrm{设平面过原点及点}(6,-3,2),\mathrm{且与平面}4x-y+2z=8\mathrm{垂直},\mathrm{则此平面方程为}().
$$
$$
A.
2x+2y-3z=0 \quad B.2x-2y+3z=0 \quad C.2x+2y-z=0 \quad D.2x-y-3z=0 \quad E. \quad F. \quad G. \quad H.
$$
$$
\begin{array}{l}\mathrm{设平面为}Ax+By+Cz+D=0\;\;\;\;,\;\;\mathrm{由平面过原点知}D=0,\;\;\\\mathrm{由平面过点}(6,-3,2)知.\;6A-3B+2C=0\;,∵\{A,B,C\}⟂\{4,-1,2\},∵4A-B+2C=0\rightarrow A=B=-\frac23C,\;\;\\\mathrm{所求平面方程为}\;2x+2y-3z=0.\;\;\;\;\;\;\;\;\;\;\;\;\;\;\;\;\end{array}
$$



$$
\mathrm{过点}M_0(2,9,-6)\mathrm{且与连接坐标原点及点}M_0\mathrm{的线段}OM_0\mathrm{垂直的平面方程是}().
$$
$$
A.
2x+9y-6z-121=0 \quad B.2x+9y-6z+121=0 \quad C.2x+9y-6z-124=0 \quad D.x+9y-6z-121=0 \quad E. \quad F. \quad G. \quad H.
$$
$$
\begin{array}{l}\mathrm{注意到}\overrightarrow{OM_0}\mathrm{即为所求平面的法向量},由\;\overrightarrow n=\overrightarrow{OM_0}=±\{2,9,-6\},\\\;\mathrm{根据点法式平面方程},\mathrm{所求平面方程为}\;±\lbrack2(x-2)+9(y-9)-6(z+6)\rbrack=0,\\即2x+9y-6z-121=0.\end{array}
$$



$$
\mathrm{通过直线}x=2t-1,y=3t+2,z=2t-3\mathrm{和直线}x=2t+3,y=3t-3,z=2t+1\mathrm{的平面方程为}().
$$
$$
A.
x-z-2=0 \quad B.x+z=0 \quad C.x-2y+z=0 \quad D.2x+3y+2z=-2 \quad E. \quad F. \quad G. \quad H.
$$
$$
\begin{array}{l}\mathrm{两直线的方向向量都为}s_1=\{2,3,2\},\mathrm{此为两平行直线},\\\mathrm{分别取两直线上的点}(-1,2,-3)和(3,-3,1),\mathrm{两点所在直线的方向向量为}s_2=\{4,-5,4\},\\\mathrm{平面的法向量为}s_1× s_2=\{1,0,-1\},\mathrm{可解得平面方程为}x-z-2=0\end{array}
$$



$$
\mathrm{过直线}L:\left\{\begin{array}{l}x+2y-z-6=0\\x-2y+z=0\end{array}\right.\mathrm{作平面}{\textstyle\prod_{}},\mathrm{使它垂直于平面}{\textstyle\prod_1}:x+2y+z=0,\mathrm{则平面}{\textstyle\prod_{}}\mathrm{的方程为}().
$$
$$
A.
3x-2y+z+6=0 \quad B.3x-2y+z-6=0 \quad C.3x-2y-z-6=0 \quad D.x-2y-z-6=0 \quad E. \quad F. \quad G. \quad H.
$$
$$
\begin{array}{l}\mathrm{设过直线的平面束的方程为}(x+2y-z-6)+λ(x-2y+z)=0\;\;,\;\\\;即(1+λ)x+2(1-λ)y+(λ-1)z-6=0.\;\;\\\mathrm{现要在上述平面束中找出一个平面}{\textstyle\prod_{}},\mathrm{使它垂直于题设平面}{\textstyle\prod_1},\\\mathrm{因平面}{\textstyle\prod_{}}\mathrm{垂直于平面}{\textstyle\prod_1},\mathrm{故平面}{\textstyle\prod_{}}\mathrm{的法向量}\overrightarrow n(λ)\mathrm{垂直于平面}{\textstyle\prod_1}\mathrm{的法向量}\overrightarrow{n_1}=\{1,2,1\},\\\mathrm{于是}\overset{}{\overrightarrow n(λ)}·\overrightarrow n=0,即\;\;1·(1+λ)+4(1-λ)+(λ-1)=0\;,\;\;\\\mathrm{解得}λ=2,\mathrm{故所求平面方程为}\;\;π:3x-2y+z-6=0\;\;\;\;.\;\;\\\end{array}
$$



$$
\mathrm{过点}M_1(1,1,2),M_2(3,2,3),M_3(2,0,3)\mathrm{三点的平面方程为}().
$$
$$
A.
2x-y-3z+5=0 \quad B.2x-y-z-5=0 \quad C.2x-3y-13z+5=0 \quad D.2x+y-3z+5=0 \quad E. \quad F. \quad G. \quad H.
$$
$$
\begin{array}{l}\mathrm{设所求平面的法向量}\overrightarrow n,\mathrm{因为}\;\overrightarrow{M_1M_2}=\overrightarrow{2i}\;+\overrightarrow j+\overrightarrow k,\;\overrightarrow{M_1M_3}=\overrightarrow i\;-\overrightarrow j+\overrightarrow k,\;\\\;\mathrm{故可取}\overrightarrow n=\;\overrightarrow{M_1M_2}×\overrightarrow{M_1M_3}=\begin{vmatrix}\overrightarrow i&\overrightarrow j&\overrightarrow k\\2&1&1\\1&-1&1\end{vmatrix}=\overrightarrow{2i}-\overrightarrow j-\overrightarrow{3k},\;\;\mathrm{又平面过点}M_1(1,1,2),\\\mathrm{从而所求平面方程为}2(x-1)-(y-1)-3(z-2)=0,\;\;即2x-y-3z+5=0.\end{array}
$$



$$
\mathrm{平面过原点}O,\mathrm{且垂直于平面}{\textstyle\prod_1}\;:x+2y+3z-2=0,{\textstyle\prod_2}:6x-y+5z+2=0\;\;\mathrm{则此平面方程为}().
$$
$$
A.
x+y-z=0 \quad B.x-y-z=0 \quad C.x+2y-z=0 \quad D.x+y-2z=0 \quad E. \quad F. \quad G. \quad H.
$$
$$
\begin{array}{l}\mathrm{设所求的平面的法向量}\overrightarrow n,\mathrm{依题设有}\overrightarrow n⟂\overrightarrow{n_1},\overrightarrow n⟂\overrightarrow{n_2},\;\;\mathrm{从而}\overrightarrow n=\overrightarrow{n_1}×\overrightarrow{n_2},\\\mathrm{其中}\overrightarrow{n_1}=\overrightarrow i+\overrightarrow{2j}+\overrightarrow{3k},\overrightarrow{n_2}=\overrightarrow{6i}-\overrightarrow j+\overrightarrow{5k},\;∴\overrightarrow n=\overrightarrow{n_1}×\overrightarrow{n_2}=\;\begin{vmatrix}\overrightarrow i&\overrightarrow j&\overrightarrow k\\1&2&3\\6&-1&5\end{vmatrix}=\overrightarrow{13i}+\overrightarrow{13j}-\overrightarrow{13k},\;\;\\\mathrm{由点法式得所求平面方程}13x+13y-13z=0.\;\;即x+y-z=0.\end{array}
$$



$$
\mathrm{平行于平面}6x+y+6z+5=0\mathrm{而与三个坐标面所围成的四面体体积}v\mathrm{为一个单位的平面方程为}().
$$
$$
A.
6x+y+6z=6 \quad B.6x+y+6z=1 \quad C.x-y+6z=6 \quad D.6x-3y+z=5 \quad E. \quad F. \quad G. \quad H.
$$
$$
\begin{array}{l}\frac x1+\frac y6+\frac z1=1解\;\;\mathrm{设平面方程为}\frac xa+\frac yb+\frac zc=1,∵ V=1,∴\frac13·\frac12abc=1.\;\;\\\mathrm{由所求平面与已知平面平行得}\;\frac{\displaystyle\frac1a}6\;=\frac{\displaystyle\frac1b}1=\frac{\displaystyle\frac1c}6,\\\;\mathrm{向量平行的充要条件}\;\;\\令\frac1{6a}=\frac1b=\frac1{6c}=t\rightarrow a=\frac1{6t},b=\frac1t,c=\frac1{6t}.\;\;\\由1=\frac16·\frac1{6t}·\frac1t·\frac1{6t}\rightarrow t=\frac16.\;∴ a=1,b=6,c=1\;.\;\;\mathrm{所求平面方程为}\frac x1+\frac y6+\frac z1=1,\\即6x+y+6z=6.\;\\\end{array}
$$



$$
\mathrm{经过两平面}4x-y+3z-1=0,x+5y-z+2=0\mathrm{的交线作平面π},\mathrm{并使π与}y\mathrm{轴平行},\mathrm{则平面π的方程为}().
$$
$$
A.
14x-21z-3=0 \quad B.21x-14z+3=0 \quad C.21x+14z-3=0 \quad D.21x+14z+3=0 \quad E. \quad F. \quad G. \quad H.
$$
$$
\begin{array}{l}\mathrm{经过两平面}4x-y+3z-1=0,x+5y-z+2=0\mathrm{的交线方程}\left\{\begin{array}{l}4x-y+3z-1=0\\x+5y-z+2=0\end{array}\right.\\\mathrm{通过此直线的平面束方程为}4x-y+3z-1+λ(x+5y-z+2)=0,\\即(4+λ)x+(5λ-1)y+(3-λ)z-1+2λ=0\mathrm{该平面与}y\mathrm{轴平行},\\\mathrm{则有}5λ-1=0,λ=1/5,\mathrm{所以平面方程为}21x+14z-3=0.\end{array}
$$



$$
\mathrm{经过两点}M_1(3,-2,9)和M_2(-6,0,-4)\mathrm{且与平面}2x-y+4z-8=0\mathrm{垂直的平面的方程为}().
$$
$$
A.
x-2y-z+2=0 \quad B.x-2y-z-2=0 \quad C.x+2y-z+21=0 \quad D.3x-2y-5z+2=0 \quad E. \quad F. \quad G. \quad H.
$$
$$
\begin{array}{l}\mathrm{设所求的平面方程为}Ax+By+Cz+D=0\;\;\\\;\mathrm{由于点}M_1和M_2\mathrm{在平面上},故\;\;3A-2B+9C+D=0,-6A-4C+D=0\;.\;\;\\\mathrm{又由于所求平面与平面}2x-y+4z-8=0\mathrm{垂直},\\\mathrm{由两平面垂直条件有}\;2A-B+4C=0\;.\;\;\\\mathrm{从上面三个方程中解出}A,B,C,得A=\frac D{2,}\;\;B=-D,C=-\frac D2,\;\\\;\mathrm{代入所设方程},\mathrm{并约去因子}\frac D2,\mathrm{得所求的平面方程}\;x-2y-z+2=0\;.\end{array}
$$



$$
\mathrm{已知动点与平面}yOz\mathrm{的距离为}4\mathrm{个单位},\mathrm{且与定点}A(5,2,-1)\mathrm{的距离为}3\mathrm{个单位},\mathrm{则动点的轨迹是}().
$$
$$
A.
\mathrm{圆柱面} \quad B.\mathrm{平面}x=4\mathrm{上的圆} \quad C.\mathrm{平面}x=4\mathrm{上的椭圆} \quad D.\mathrm{椭圆柱面} \quad E. \quad F. \quad G. \quad H.
$$
$$
\begin{array}{l}\mathrm{动点与}yOz\mathrm{平面的距离为}4\mathrm{个单位},\\\mathrm{则可得出该点是平面}x=±4\mathrm{上的点},\\\mathrm{与定点}A(5,2,-1)\mathrm{的距离为}3\mathrm{个单位},\mathrm{设该点为}(±4,y,z),\\当x=-4时,\mathrm{此点与}A(5,2,-1)\mathrm{距离不可能为}3,\mathrm{所以该点为}(4,y,z),\\\mathrm{有等式}1+(y-2)^2+(z+1)^2=9,\mathrm{其为平面}x=4\mathrm{上的圆}\end{array}
$$



$$
\mathrm{两平面}x-2y-z=3,2x-4y-2z=5\mathrm{各自与平面}x+y-3z=0\mathrm{的交线是}().
$$
$$
A.
\mathrm{相交的} \quad B.\mathrm{平行的} \quad C.\mathrm{异面的} \quad D.\mathrm{重合的} \quad E. \quad F. \quad G. \quad H.
$$
$$
\begin{array}{l}\mathrm{两平面}x-2y-z=3\mathrm{与平面}x+y-3z=0\mathrm{的交线为}\;\left\{\begin{array}{l}x-2y-z=3\\x+y-3z=0\end{array}\right.,\\\mathrm{其对称式方程为}\frac{x-1}1=\frac{y+1}{\displaystyle\frac27}=\frac z{\displaystyle\frac37},\;\\\mathrm{同理可得}2x-4y-2z=5\mathrm{与平面}x+y-3z=0\mathrm{的的交线方程为}\;\;\frac{x-1}1=\frac{y+{\displaystyle\frac{11}{14}}}{\displaystyle\frac27}=\frac{z-{\displaystyle\frac1{14}}}{\displaystyle\frac37},\\\mathrm{所以两线是平行的}.\end{array}
$$



$$
\mathrm{直线}\left\{\begin{array}{l}3x+2z≡0\\5x-1=0\end{array}\right.\left(\;\right)
$$
$$
A.
\mathrm{平行}y轴 \quad B.\mathrm{垂直}y轴 \quad C.\mathrm{平行}x轴 \quad D.\mathrm{平行}zOx\mathrm{平面} \quad E. \quad F. \quad G. \quad H.
$$
$$
\mathrm{由题意得}\left\{\begin{array}{l}x={\textstyle\frac15}\\z=-{\textstyle\frac3{10}}\end{array}\right.,\mathrm{所以是平行于}y轴;
$$



$$
\mathrm{直线}l_1:\left\{\begin{array}{l}\begin{array}{c}x=-4+t\\y=3-2t\end{array}\\\begin{array}{c}z=2+3t\end{array}\end{array}\right.\mathrm{与直线}l_2:\left\{\begin{array}{l}\begin{array}{c}x=-3+2t\\y=-1-4t\end{array}\\\begin{array}{c}z=5+6t\end{array}\end{array}\right.\mathrm{之间的位置关系是}\left(\;\;\;\right)
$$
$$
A.
\mathrm{平行} \quad B.\mathrm{垂直} \quad C.\mathrm{相交} \quad D.\mathrm{重合} \quad E. \quad F. \quad G. \quad H.
$$
$$
\mathrm{由两直线的参数表达式可得}:s_1=\left\{1,-2,3\right\},s_2=\left\{2,-4,6\right\},得:{\textstyle\frac12}={\textstyle\frac{-2}{-4}}={\textstyle\frac36},\mathrm{显然不重合},\mathrm{因此两直线平行}.
$$



$$
\mathrm{直线}l:{\textstyle\frac{x+2}3}={\textstyle\frac{y-2}{-1}}={\textstyle\frac{z+3}2}\mathrm{和平面}n:2x+3y+3z-8=0\mathrm{的交点是}\left(\;\;\;\right)
$$
$$
A.
\left(3,{\textstyle\frac13},{\textstyle\frac13}\right) \quad B.\left(-1,{\textstyle1},1\right) \quad C.\left(1,-1,1\right) \quad D.\left(1,1,-1\right) \quad E. \quad F. \quad G. \quad H.
$$
$$
\mathrm{联立方程}\left\{\begin{array}{l}{\textstyle\frac{x+2}3}={\textstyle\frac{y-2}{-1}}={\textstyle\frac{z+3}2}\\2x+3y+3z=8\end{array}\right.\mathrm{解得}\left\{\begin{array}{l}\begin{array}{c}x=3\\y={\textstyle\frac13}\end{array}\\z={\textstyle\frac13}\end{array}\right.,\mathrm{所以交点为}\left(3,{\textstyle\frac13},{\textstyle\frac13}\right).
$$



$$
\mathrm{直线}l_1:x-1=y=-\left(z+1\right),l_2:x=-\left(y-1\right)={\textstyle\frac{z+1}0}\mathrm{相对关系是}\left(\;\;\right)
$$
$$
A.
\mathrm{平行} \quad B.\mathrm{重合} \quad C.\mathrm{垂直} \quad D.\mathrm{异面} \quad E. \quad F. \quad G. \quad H.
$$
$$
\mathrm{直线}l_1\mathrm{的方向向量为}\left\{1,1,-1\right\},\mathrm{直线}l_2\mathrm{的方向向量为}\left\{1,-1,0\right\},\mathrm{两向量的数量积为零},\mathrm{所以两直线是垂直的}.
$$



$$
\mathrm{设直线}L_1:{\textstyle\frac{x-1}1}={\textstyle\frac{y-5}{-2}}={\textstyle\frac{z+8}1}与L_2:\left\{\begin{array}{l}x-y=6\\2y+z=3\end{array}\right.,则L_1与L_2\mathrm{的夹角为}(\;\;)
$$
$$
A.
\textstyle\fracπ6 \quad B.\textstyle\fracπ4 \quad C.\textstyle\fracπ3 \quad D.\textstyle\fracπ2 \quad E. \quad F. \quad G. \quad H.
$$
$$
\begin{array}{l}\mathrm{因为}L_1\mathrm{的方向是}\overset\rightharpoonup{n_1}=\left(1,-2,1\right),L_2\mathrm{的方向是}\overset\rightharpoonup{n_2}=\left(-1,-1,2\right)\\\mathrm{所以}L_1与L_2\mathrm{的夹角}θ\mathrm{的余弦cos}θ={\textstyle\frac{\overset\rightharpoonup{n_1}·\overset\rightharpoonup{n_2}}{\left|\overset\rightharpoonup{n_1}\right|\left|\overset\rightharpoonup{n_2}\right|}}={\textstyle\frac12},θ={\textstyle\fracπ3}.\end{array}
$$



$$
\mathrm{直线}{\textstyle\frac{x+3}{-2}}={\textstyle\frac{y+4}{-7}}={\textstyle\frac z3}\mathrm{与平面}4x-2y-2z=3\mathrm{的关系是}\left(\;\;\right)
$$
$$
A.
\mathrm{平行},\mathrm{但直线不在平面上} \quad B.\mathrm{直线在平面上} \quad C.\mathrm{垂直相交} \quad D.\mathrm{相交但不垂直} \quad E. \quad F. \quad G. \quad H.
$$
$$
\begin{array}{l}\mathrm{因为}Am+Bn+Cp=\left(-2\right)·4+\left(-2\right)·\left(-7\right)+\left(-2\right)·3=0,\;\\\;\mathrm{所以直线与平面平行},\mathrm{直线上的一点}\left(-3,-4,0\right)\mathrm{不在平面上},\mathrm{所以直线不在平面上}.\end{array}
$$



$$
\mathrm{直线}l:{\textstyle\frac{x+3}{-2}}={\textstyle\frac{y+4}{-7}}={\textstyle\frac z3}\mathrm{与平面}π:4x-2y-z-3=0\mathrm{的位置关系是}\left(\;\;\;\right)
$$
$$
A.
l与π\mathrm{平行} \quad B.l在π 上 \quad C.l与π\mathrm{相交} \quad D.l与π\mathrm{垂直} \quad E. \quad F. \quad G. \quad H.
$$
$$
\begin{array}{l}{\textstyle\frac{-2}4}{\textstyle\neq}{\textstyle\frac{-7}{-2}},\mathrm{所以}l与π\mathrm{不垂直},\left(-2\right)·4+\left(-7\right)·\left(-2\right)+3·\left(-1\right){\textstyle\neq}{\textstyle0},\;\;\\\mathrm{所以}l与π\mathrm{不平行},\mathrm{直线上的点}\left(-3,-4,0\right)\mathrm{不在平面上},\mathrm{所以位置关系是相交}.\end{array}
$$



$$
\mathrm{空间直线}{\textstyle\frac{x+2}3}={\textstyle\frac{y-2}1}={\textstyle\frac{z+1}{-5}}\mathrm{与平面}4x+3y+3z+1=0\mathrm{的位置关系是}\left(\;\;\;\right)
$$
$$
A.
\mathrm{互相垂直} \quad B.\mathrm{互相平行} \quad C.\mathrm{不平行也不垂直} \quad D.\mathrm{直线在平面上} \quad E. \quad F. \quad G. \quad H.
$$
$$
3·4+1·3-5·3=0,\mathrm{又因为点}\left(-2,2,-1\right)\mathrm{不在平面内},\mathrm{所以是相互平行的}.
$$



$$
\mathrm{设直线}L:\left\{\begin{array}{l}x+3y+2y+1=0\\2x-y-10z+3=0\end{array}\right.,\mathrm{设平面}π:4x-2y+z-2=0,\mathrm{则直线}L(\;\;)
$$
$$
A.
\mathrm{平行于π} \quad B.在\mathrmπ 上 \quad C.\mathrm{垂直于π} \quad D.与π\mathrm{相交},\mathrm{但不垂直}. \quad E. \quad F. \quad G. \quad H.
$$
$$
\begin{array}{l}L\mathrm{由平面}A:x+3y+2z+1=0\mathrm{与平面}B:2x-y-10z+3=0\mathrm{相交而成}\;\;\\\left(1,3,2\right)\left(4,-2,1\right)=4-6+2=0\\\mathrm{所以}A与π\mathrm{垂直}\;\\\left(2,-1,-10\right)\left(4,-2,1\right)=8+2-10=0\\\mathrm{所以}B与\mathrm{π垂直}\;\\\mathrm{所以π}与A,B\mathrm{的交线}L\mathrm{垂直}.\end{array}
$$



$$
在yOz\mathrm{平面上},\mathrm{且垂直于向量}a=\left\{5,4,3\right\}\mathrm{的单位向量}b=\left(\;\;\;\;\right)
$$
$$
A.
\left\{0,-{\textstyle\frac34},1\right\} \quad B.\left\{0,{\textstyle\frac35},{\textstyle\frac45}\right\} \quad C.\left\{0,±{\textstyle\frac35},∓{\textstyle\frac45}\right\} \quad D.\left\{0,{\textstyle-3},{\textstyle4}\right\} \quad E. \quad F. \quad G. \quad H.
$$
$$
\begin{array}{l}设yOz\mathrm{平面的向量为}c=\left\{0,m,n\right\},a· c=4m+3n=0,m:n=-3:4,\;\;\\\mathrm{所以}c=\left\{0,-3,4\right\}或\boldsymbol c\boldsymbol=\left\{\mathbf0\boldsymbol,\mathbf3\boldsymbol,\boldsymbol-\mathbf4\right\},\mathrm{其单位向量为}\left\{0,±{\textstyle\frac35},∓{\textstyle\frac45}\right\}.\end{array}
$$



$$
点M\left(1,2,1\right)\mathrm{到平面}x+2y+2z-10=0\mathrm{的距离为}\left(\;\;\;\right)
$$
$$
A.
1 \quad B.±1 \quad C.-1 \quad D.\textstyle\frac13 \quad E. \quad F. \quad G. \quad H.
$$
$$
\mathrm{根据点到平面的距离公式可得}d={\textstyle\frac{\left|1+4+2-10\right|}{\sqrt{1+4+4}}}=1.
$$



$$
\mathrm{直线}\left\{\begin{array}{l}5x+y-3z-7=0\\2x+y-3z-7=0\end{array}\right.\left(\;\;\;\right)
$$
$$
A.
\mathrm{垂直}yOz\mathrm{平面} \quad B.在yOz\mathrm{平面内} \quad C.\mathrm{平行}x轴 \quad D.在xOy\mathrm{平面内} \quad E. \quad F. \quad G. \quad H.
$$
$$
\mathrm{由直线方程可知},x=0,\mathrm{所以直线是在}yOz\mathrm{平面内}
$$



$$
\mathrm{已知点}A\left(6,6,-1\right),点B\left(-2,-6,3\right),则AB与xOy\mathrm{面交点的坐标是}\left(\;\;\;\right)
$$
$$
A.
\left(4,-3,1\right) \quad B.\left(4,-3,0\right) \quad C.\left(4,3,1\right) \quad D.\left(4,3,0\right) \quad E. \quad F. \quad G. \quad H.
$$
$$
\begin{array}{l}AB\mathrm{的对称式方程为}{\textstyle\frac{x-6}8}={\textstyle\frac{y-6}{12}}={\textstyle\frac{z+1}{-4}},xOy\mathrm{面为}z=0,\;\\\;\mathrm{代入直线方程得}\left\{\begin{array}{l}x≡4\\y=3\end{array}\right.\mathrm{所以交点的坐标为}\left(4,3,0\right).\end{array}
$$



$$
\mathrm{两条平行直线}l_1:x=t+1,y=2t-1,z=t\;\;\;l_2:x=t+2,y=2t-1,z=t+1\mathrm{的距离是}\left(\;\;\;\right)
$$
$$
A.
\sqrt2 \quad B.\textstyle\frac23\sqrt3 \quad C.\textstyle\frac23 \quad D.2 \quad E. \quad F. \quad G. \quad H.
$$
$$
L_1\mathrm{的一点为}M\left(1,-1,0\right),L_2\mathrm{的方向向量为}s=\left\{1,2,1\right\},L_2\mathrm{上的一点}M_0\left(2,-1,1\right),\mathrm{所以两直线的距离为}d={\textstyle\frac{\left|\overset\rightharpoonup{MM_0}× s\right|}{\left|s\right|}}={\textstyle\frac23}{\textstyle\sqrt3}.
$$



$$
\mathrm{若平面}kx+y-2z=1\mathrm{与直线}{\textstyle\frac x2}={\textstyle\frac{y-1}4}={\textstyle\frac{z+2}3}\mathrm{平行},则k=\left(\;\;\;\;\right)
$$
$$
A.
1 \quad B.2 \quad C.3 \quad D.0 \quad E. \quad F. \quad G. \quad H.
$$
$$
\mathrm{由题意可得}:2k+4-6=0,\mathrm{解得}:k=1.
$$



$$
\mathrm{两直线}\left\{\begin{array}{l}x+2y-z=7\\-2x+y+z=7\end{array}\right.与\left\{\begin{array}{l}3x+6y-3z=8\\2x-y-z=0\end{array}\right.\mathrm{的位置关系为}\left(\;\;\;\right).
$$
$$
A.
\mathrm{平行} \quad B.\mathrm{垂直} \quad C.\mathrm{相交不不垂直} \quad D.\mathrm{异面} \quad E. \quad F. \quad G. \quad H.
$$
$$
\begin{array}{l}\mathrm{设两直线的方向向量为}s_1与s_2,\;\;\\则s_1=\begin{vmatrix}\xrightarrow[i]{}&\xrightarrow[j]{}&\xrightarrow[k]{}\\1&2&-1\\-2&1&1\end{vmatrix},\;\;\;\;\;\;\;\;s_2=\begin{vmatrix}\xrightarrow[i]{}&\xrightarrow[j]{}&\xrightarrow[k]{}\\1&2&-1\\2&-1&-1\end{vmatrix},\\\;\;\;\;\;\;\;\;=\left\{3,1,5\right\}\;\;\;\;\;\;\;\;\;\;\;\;\;\;\;\;\;\;\;\;\;\;=\left\{-3,-1,-5\right\}\;\;\;\;\\\;\;\;\;\;\;\;\;\;\;\;\;\;\;\;\;\;\;\;\;\;\;\;\;\;\;\;\;\;\;\;\;\;\;\;\;\;\;\;\;\;\;\;\;\;\;\;=-\left\{3,1,5\right\}\;\;\;\;\;\\\mathrm{所以}\;\;\xrightarrow[s_1]{}⁄⁄\;\xrightarrow[s_2]{}\;\;\;\;\;\;\;\;\;\;\;\;\;\;\;\;\;\;\;\;\;\;\;\;\;\;\;\;\;\;\;\;\;\;\;\;\;\;\;\;\;\;\;\;\;\;\;\;\;\;\;\;\;\;\;\;\;\;\;\;\;\;\;\;\;\;\;\;\;\;\;\;\;\;\;\;\;\;\;\;\;\;\;\;\;\;\;\;\;\;\;\;\;\;\;\;\;\;\;\;\;\;\;\;\;\;\;\;\;\;\;\;\;\;\;\;\;\;\;\;\end{array}
$$



$$
\mathrm{两条平行直线}L_1:x=t+1,y=2t-1,z=t\;\;L_2:x=t+2,y=2t-1,z=t+1\mathrm{的距离是}\left(\;\;\;\;\right)
$$
$$
A.
\sqrt2 \quad B.\textstyle\frac23\sqrt3 \quad C.\textstyle\frac23 \quad D.2 \quad E. \quad F. \quad G. \quad H.
$$
$$
\begin{array}{l}L_1\mathrm{的一点为}M\left(1,-1,0\right),L_2\mathrm{的方向向量为}s=\left\{1,2,1\right\},\mathrm{上的一点}M_0\left(2,-1,1\right),\mathrm{所以两直线的距离为}\\d={\textstyle\frac{\left|\overset\rightharpoonup{MM_0}× s\right|}{\left|s\right|}}≡{\textstyle\frac23}{\textstyle\sqrt3}.\end{array}
$$



$$
\mathrm{直线}{\textstyle\frac x1}={\textstyle\frac{y+2}3}={\textstyle\frac{z+7}5}\mathrm{与平面}3x+y-9z+17=0\mathrm{的交点为}\left(\;\;\;\;\right)
$$
$$
A.
\left(0,-2,-7\right) \quad B.\left(3,7,8\right) \quad C.\left(1,1,-2\right) \quad D.\left(2,4,3\right) \quad E. \quad F. \quad G. \quad H.
$$
$$
\mathrm{解方程组}:\left\{\begin{array}{l}\begin{array}{c}x={\textstyle\frac{y+2}3}\\x={\textstyle\frac{z+7}5}\end{array}\\3x+y-9z+17=0\end{array}\right.\mathrm{解得}:x=2,y=4,z=3.
$$



$$
\mathrm{直线}{\textstyle\frac{x-12}4}={\textstyle\frac{y-9}3}={\textstyle\frac{z-1}1}\mathrm{与平面}3x+5y-z-2=0\mathrm{的交点为}\left(\;\;\;\;\;\right)
$$
$$
A.
\left(0,0,-2\right) \quad B.\left(0,0,2\right) \quad C.\left(12,9,1\right) \quad D.\left(4,3,-1\right) \quad E. \quad F. \quad G. \quad H.
$$
$$
\mathrm{解方程组}:\left\{\begin{array}{l}\begin{array}{c}{\textstyle\frac{x-12}4}={\textstyle\frac{z-1}1}\\{\textstyle\frac{y-9}3}={\textstyle\frac{z-1}1}\end{array}\\\begin{array}{c}3x+5y-z-2=0\end{array}\end{array}\right.\mathrm{解得}:x=0,y=0,z=-2.
$$



$$
\mathrm{要使直线}{\textstyle\frac{x-a}3}={\textstyle\frac y{-2}}={\textstyle\frac{z+1}a}\mathrm{在平面}3x+4y-az=3a-1内,则a=\left(\;\;\;\right)
$$
$$
A.
-1 \quad B.-2 \quad C.1 \quad D.2 \quad E. \quad F. \quad G. \quad H.
$$
$$
\begin{array}{l}\mathrm{由题意可得直线的方向向量为}:\left\{3,-2,a\right\},\mathrm{平面的法向量为}:\left\{3,4,-a\right\},,则:\\3·3+\left(-2\right)·4-a· a=0,\mathrm{解得}a=±1.\;\\当a=1时,\mathrm{直线方程为}:{\textstyle\frac{x-1}3}={\textstyle\frac y{-2}}={\textstyle\frac{z+1}1},\mathrm{任意直线上一点}\left(4,-2,0\right),\mathrm{但是不满足平面方程}\\3x+4y-z=2,\mathrm{因此不符合条件};\;\\当a=-1时,\mathrm{直线方程为}:{\textstyle\frac{x+1}3}={\textstyle\frac y{-2}}={\textstyle\frac{z+1}{-1}},\mathrm{任意直线上一点}\left(2,-2,-2\right),\mathrm{满足平面方程}\\3x+4y+z=-4,\mathrm{因此}a=-1\end{array}
$$



$$
\mathrm{与两直线}l_1:\left\{\begin{array}{l}\begin{array}{c}x=1\\y=t-1\end{array}\\\begin{array}{c}z=2+t\end{array}\end{array}\right.及l_2:{\textstyle\frac{x+1}1}={\textstyle\frac{y+2}2}={\textstyle\frac{z+1}1}\mathrm{都平行且过原点的平面方程为}\left(\;\;\;\right)
$$
$$
A.
x-y+z=0 \quad B.x-y-z=0 \quad C.x-2y+z=0 \quad D.x+y-z=0 \quad E. \quad F. \quad G. \quad H.
$$
$$
\begin{array}{l}\mathrm{设平面方程为}:Ax+By+Cz+D=0,\mathrm{由题意得}\;\;\\\mathrm{直线的方向向量为}:\left\{0,1,1\right\},\\\;\mathrm{直线的方向向量为}:\left\{1,2,1\right\},\\\;\mathrm{然后由平行得到}:B+C=0,A+2B+C=0,\;\\\mathrm{平面过原点得}:D=0,\mathrm{得到}A=C,B=-C,\\\;\mathrm{因此平面方程为}:x-y+z=0\end{array}
$$



