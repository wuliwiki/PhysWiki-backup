% 素数证明书(综合)
% license CCBYSA3
% type Wiki

本文根据 CC-BY-SA 协议转载翻译自维基百科\href{https://en.wikipedia.org/wiki/Primality_certificate}{相关文章}。

在数学和计算机科学中,素数证明书或素性证明是指一种简洁、形式化的证明,用来表明一个数是质数。素数证明书的作用是:无需运行代价高昂或不够可靠的素数判定算法,就能快速验证一个数的素性。这里的“简洁”通常意味着,该证明的规模至多是这个数位数的多项式倍(例如,如果该数有 $b$ 位二进制数位,证明可能大约包含 $b^2$ 位)。

素数证明书直接表明了以下结论:素数判定问题,以及其补问题——整数分解的补集问题——属于 NP 类(即给出解后可在多项式时间内验证的问题类)。这些问题显然也属于 co-NP 类。这是首次有力的证据表明这些问题并不是 NP 完全的,因为如果它们是 NP 完全的,就意味着 NP 是 co-NP 的子集,而这一结论被广泛认为是错误的;事实上,这也是当时首次展示了一个已知属于 NP ∩ co-NP 但不知是否属于 P 类 的问题。

对于其补问题(证明一个数是合数)的证明书,生成方法非常直接:只需给出一个非平凡因子即可。标准的概率型素数判定算法(如 Baillie–PSW 素数判定法、费马素数检验、米勒–拉宾素数检验)在输入是合数的情况下,也会生成“合数证明书”,但对于质数输入则不会生成证明书。
\subsection{普拉特证明书}
素数证明书这一概念最早是由 普拉特证明书引入的,该方法由 Vaughan Pratt 于 1975 年提出【1】。他描述了这种证明书的结构,并证明了它的规模是多项式级的,而且可以在多项式时间内完成验证。

普拉特证明书基于**卢卡斯素性检验**(Lucas primality test),该检验本质上是**费马小定理的逆命题**,并附加了一个条件使其成立:

卢卡斯定理:
假设存在一个整数 $a$,使得:
\begin{itemize}
\item $a^{n - 1} \equiv 1 \pmod{n}$,
\item 对于 $n - 1$ 的每个质因数 $q$,都有 $a^{(n - 1) / q} \not\equiv 1 \pmod{n}$,
\end{itemize}
那么 $n$ 是质数。

给定这样的 $a$(称为见证,witness)以及 $n - 1$ 的质因数分解,我们可以很快验证上述条件:只需做线性数量的模指数运算(因为任何整数的质因子数不会超过它的比特数)。每次模指数运算可以通过平方乘法在 $O(\log n)$ 次乘法中完成(见大 O 符号)。

即便采用最基础的“学校式”整数乘法,总耗时也只有 $O((\log n)^{4})$;
如果使用 **David Harvey** 和 **Joris van der Hoeven** 提出的目前已知渐进时间最优的乘法算法,则可以将其降低到 $O((\log n)^{3} \log\log n)$,
使用软 O 符号则记作 $\tilde{O}((\log n)^{3})$。

---

你需要我继续帮你翻译**普拉特证明书的递归结构**部分吗?那是它的关键特点。
