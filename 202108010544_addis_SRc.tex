% 约化光速
\pentry{斜坐标系表示洛伦兹变换\upref{SROb}, 无单位的物理公式\upref{NoUnit}}

\subsection{约去量纲}

物理量运算的规则有一个约束,合法的带单位多项式中,相加减的必然是相同的物理量.在无单位的物理公式\upref{NoUnit}中我们知道,当单位统一时,可以将物理量的单位和数值分开计算.

\begin{example}{约去量纲的例子}
不考虑相对论效应.一辆火车以速度 $3.6\opn{km/h}$ 向东行驶,车上有一个小朋友以速度 $4\opn{m/s}$ 向东奔跑,那么小朋友相对铁轨的速度就是 $3.6\times\frac{\opn{km}}{\opn{h}}+4\times\frac{\opn{m}}{\opn{s}}=3.6\times\frac{1000\cdot \opn{m}}{3600\opn{s}}+4\times\frac{\opn{m}}{\opn{s}}=(3.6\times\frac{1000}{3600}+4)\frac{\opn{m}}{\opn{s}}=5\frac{\opn{m}}{\opn{s}}=5\opn{m/s}$.
\end{example}

\subsection{约去光速}

如果取长度单位为 $\opn{m}$,时间单位为 $\opn{\tau}=299792458\opn{s}$,那么光速就可以写为 $1\opn{m/\tau}$.用 $\opn{\tau}$ 取代 $\opn{s}$ 作为时间单位,那么一切涉及光速的等式中,我们都可以把光速的\textbf{数值}写为 $1$,大大简化计算,而光速的量纲 $[\opn{m/s}]=[\opn{m/\tau}]$ 是独立于数值进行计算的.在这种写法中,$0.1$ 倍光速就可以写为 $0.1\opn{m/\tau}$.

所以,在具体数值的计算中,使用上述单位制,则任何地方的光速的\textbf{数值}都可以视为 $1$.当然,任何物理量都可以通过选择适当的单位制,来让其数值上等于 $1$,但在狭义相对论中只有令光速的数值为 $1$ 是最方便的,因为任何惯性系中光速都不变.

如果\textbf{任何}数值计算中都可以把光速设为 $1$,那么我们可以干脆把这个规律一般化,把光速本身就当成 $1$,没有量纲;或者说,这种情况下,我们把长度和时间看成同一个物理量,其换算关系是 $1\opn{m}=299792458\opn{s}$,正如长度的换算关系 $1\opn{km}=1000\opn{m}$ 一样.使用这种技巧,那么狭义相对论中的一切公式都会简洁得多.

当然,实际应用中我们还是要把长度和时间分开来看的,这就要求我们在适当的地方要能够把光速的具体数值代回去,使得\textbf{量纲和物理量相符,并且相加减的项具有相同的量纲},方式是进行量纲分析.我将用例子解释如下:

\begin{example}{}
\begin{itemize}
\item 如果约去光速后,某\textbf{长度}的表达式为 $xv$,那么在国际单位制下,这个长度应该是 $xv/c$.
\item 如果 $tv^2+x$ 表示某个\textbf{长度},那么在国际单位制下,这个长度应该是 $tv^2/c+x$.
\item 如果 $tv^2+x$ 表示某个\textbf{加速度},那么在国际单位制下,无论如何都没法把 $c$ 添加回去以得到合适的量纲,因此这个表达式不可能表示加速度.

\end{itemize}
\end{example}

\subsection{约去光速后洛伦兹变换的表达}

约去光速后,洛伦兹变换变得高度对称.

\begin{equation}\label{SRc_eq3}
\leftgroup{
&x' = \frac{x - vt}{\sqrt{1 - v^2}}\\
&y'= y\\
&z' = z\\
&t' = \frac{t - vx}{\sqrt{1 - v^2}}
}
\qquad
\leftgroup{
&x = \frac{x' + vt'}{\sqrt{1 - v^2}}\\
&y = y'\\
&z = z'\\
&t = \frac{t' + vx'}{\sqrt{1 - v^2}}
}
\end{equation}

同时洛伦兹矩阵也变为:

\begin{equation}
L=
\left[\begin{matrix}
\frac{1}{\sqrt{1-v^2}}& -\frac{v}{\sqrt{1-v^2}}& 0& 0\\
-\frac{v}{\sqrt{1-v^2}}& \frac{1}{\sqrt{1-v^2}}& 0& 0\\
0&0&1&0\\
0&0&0&1
\end{matrix}\right]
\end{equation}

回归国际单位制也很简单.考虑到 $x'$ 是长度,结合“相加减项必然有相同的量纲”,可以在适当位置添加光速以满足这两个条件,得到 $x' = \frac{x - vt}{\sqrt{1 - v^2/c^2}}$.同样,考虑到 $t'$ 是时间,可以得到 $t' = \frac{t - vx/c^2}{\sqrt{1 - v^2/c^2}}$.

由此可见,约去光速的表达可以很自然地转化为国际单位制的表达,但是前者更加简洁优美,计算也不那么繁琐了.事实上,理论物理学家几乎不会真的用上 $c=299792458\opn{m/s}$,而是都默认 $c=1$.今后,我们将沿用 $c=1$ 的表达方法,约去 $c$;如果你不习惯的话,不妨在见到每一个表达式的时候,练习一下该如何适当添加 $c$ 以回归国际单位制.

注意,在约去光速后,速度 $v$ 的取值范围是 $[0,1]$,应理解为光速的倍数.

\begin{exercise}{}
将事件与尺缩效应\upref{SRsmt}、时间的变换与钟慢效应\upref{SRtime}、洛伦兹变换\upref{SRLrtz}和斜坐标系表示洛伦兹变换\upref{SROb}中的所有推导,在约去光速的条件下重新进行一遍,体会约去光速的便捷性\footnote{事实上,笔者在推导这些词条中的公式时,就是使用了约去光速的技巧,再在结果中把光速添加回去而得到的.}.
\end{exercise}


