% 天津大学 2011 年考研量子力学答案
% keys 考研|天津大学|量子力学|2011|答案
% license Copy
% type Tutor

\begin{issues}
\issueDraft
\end{issues}

\subsection{ }
\begin{enumerate}
\item 根据题意有:
\begin{equation}
\begin{aligned}
Ax\psi_n(x)&=\frac{A}{\alpha}\qty[\sqrt{\frac{n}{2}}\psi_{n-1}(x)+\sqrt{\frac{n+1}{2}}\psi_{n+1}(x)] \\
&=\frac{A}{\alpha}\sqrt{\frac{n}{2}}\psi_{n-1}(x)+\frac{A}{\alpha}\sqrt{\frac{n+1}{2}}\psi_{n+1}(x)~.
\end{aligned}
\end{equation}
则归一化系数$A$有:
\begin{equation}
\qty(\frac{A}{\alpha}\sqrt{\frac{n}{2}}+\frac{A}{\alpha}\sqrt{\frac{n+1}{2}}=1)\Rightarrow A=\sqrt{\frac{2\alpha^2}{2n+1}}~.
\end{equation}
\end{enumerate}
\subsection{ }
\begin{enumerate}
\item $\hat A=(\hat F+\hat F^\dagger),\hat B=\mathrm i(\hat F-\hat F^\dagger)$
\item 
由于$[\hat L_i,\hat L_j]=\mathcal i\epsilon_{ijk}\hat L_k$,设$\ket{x}$为角动量分量算符的本征态,则我们有:
\begin{equation}
\bra{x}\hat L_i\hat L_j\ket{x}-\bra{x}\hat L_j\hat L_i\ket{x}=\mathcal i\epsilon_{ijk}\bra{x}\hat L_k\ket{x}~.
\end{equation}
设$\hat L_i\ket{x}=\lambda\ket{x}$,则可得$\overline {\hat L_k}=0$。$j$与$k$对换后可得$\overline {\hat L_j}=0$。$i$为任意分量,证毕。
\end{enumerate}
\subsection{ }
\subsection{ }
\subsection{ }
