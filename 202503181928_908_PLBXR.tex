% 泡利不相容原理(综述)
% license CCBYSA3
% type Wiki

本文根据 CC-BY-SA 协议转载翻译自维基百科\href{https://en.wikipedia.org/wiki/Pauli_exclusion_principle}{相关文章}。

\begin{figure}[ht]
\centering
\includegraphics[width=6cm]{./figures/5713325387bd5c08.png}
\caption{沃尔夫冈·泡利在1929年哥本哈根的一次讲座中。[1] 沃尔夫冈·泡利提出了泡利不相容原理。} \label{fig_PLBXR_1}
\end{figure}

在量子力学中,泡利不相容原理(德语:Pauli-Ausschlussprinzip)指出,在遵循量子力学定律的系统中,两个或多个具有半整数自旋(即费米子)的相同粒子不能同时占据相同的量子态。奥地利物理学家沃尔夫冈·泡利于1925年首先为电子提出了这一原理,随后在1940年通过自旋-统计定理将其推广至所有费米子。

对于原子中的电子,不可违背原理可以表述如下:在一个多电子原子中,不可能有两个电子的所有四个量子数取值都相同。这四个量子数分别是:主量子数 \( n \),角量子数 \( \ell \),磁量子数 \( m_\ell \),以及自旋量子数 \( m_s \)。例如,如果两个电子处于同一个轨道中,那么它们的 \( n \)、\( \ell \) 和 \( m_\ell \) 值相等。在这种情况下,它们的自旋量子数 \( m_s \) 值必须不同。由于自旋量子数 \( m_s \) 仅能取 \( +1/2 \) 或 \( -1/2\),因此其中一个电子必须具有 \( m_s = +1/2 \),另一个必须具有 \( m_s = -1/2 \)。

具有整数自旋的粒子(玻色子)不受泡利不相容原理的约束。任意数量的相同玻色子可以占据同一量子态,例如激光产生的光子或玻色-爱因斯坦凝聚态中的原子。  

更严格的表述是:在交换两个相同粒子的情况下,总(多粒子)波函数对于费米子是反对称的,而对于玻色子是对称的。这意味着,如果交换两个相同粒子的空间和自旋坐标,则总波函数对费米子会改变符号,而对玻色子不会改变符号。

因此,假设两个费米子处于相同的状态——例如,在同一原子的同一轨道上且具有相同的自旋——那么交换它们后系统不会发生任何变化,总波函数也应保持不变。然而,对于费米子来说,总波函数必须在交换两个粒子时改变符号,而同时又保持不变的唯一可能性是该波函数在所有地方都为零,这意味着这种状态不可能存在。这种推理对玻色子不适用,因为对于玻色子而言,交换粒子时波函数的符号不会改变。
\subsection{概述}  
泡利不相容原理描述了所有费米子(具有半整数自旋的粒子)的行为,而玻色子(具有整数自旋的粒子)遵循其他原则。费米子包括夸克、电子和中微子等基本粒子。此外,重子(如质子和中子,它们由三个夸克组成)以及某些原子(如氦-3)也是费米子,因此同样受泡利不相容原理的约束。  

原子的整体自旋可以不同,这决定了它们是费米子还是玻色子。例如,氦-3的自旋为\(1/2 \),因此是费米子,而氦-4的自旋为 0,因此是玻色子。泡利不相容原理支撑了日常物质的许多特性,从其大尺度稳定性到原子的化学行为。

半整数自旋意味着费米子的固有角动量值是\( \hbar = h / 2\pi \)(约化普朗克常数)的半整数倍(1/2、3/2、5/2 等)。在量子力学理论中,费米子由反对称态描述。相比之下,具有整数自旋的粒子(玻色子)具有对称的波函数,并且可以占据相同的量子态。玻色子包括光子、负责超导现象的库珀对,以及\(W\)和\(Z\)玻色子。费米子得名于服从费米–狄拉克统计分布,而玻色子则得名于玻色–爱因斯坦统计分布。
\subsection{历史}  
在20世纪初,人们发现具有偶数个电子的原子和分子在化学上比具有奇数个电子的原子和分子更稳定。例如,在1916年吉尔伯特·N·路易斯发表的文章《原子与分子》中,他提出了六条化学行为的基本假设,其中第三条指出,原子倾向于在任何给定的电子层中保持偶数个电子,特别是八个电子。他假设这些电子通常对称地排列在一个立方体的八个角上。1919年,化学家欧文·朗缪尔提出,如果原子中的电子以某种方式连接或聚集在一起,那么元素周期表就可以得到解释。当时,人们认为电子以一定的电子层(壳层)围绕原子核分布。1922年,尼尔斯·玻尔更新了他的原子模型,假设某些特定数量的电子(例如2、8和18)对应于稳定的“闭合壳层”。

泡利试图寻找这些数字的解释,这些数字最初只是基于实验观察的经验规律。与此同时,他也在尝试解释原子光谱学中的**塞曼效应**(Zeeman effect)以及**铁磁性**(ferromagnetism)的实验结果。他在1924年埃德蒙·C·斯托纳(Edmund C. Stoner)的一篇论文中发现了一个关键线索。斯托纳指出,在碱金属光谱的外磁场中,对于给定的**主量子数** \( n \),所有简并能级分裂后,单个电子的能级数目恰好等于相同 \( n \)值下**稀有气体(惰性气体)闭合电子层**中的电子数目。这一发现使泡利意识到,若以四个量子数来定义电子态,那么复杂的闭合电子层电子数目可以归结为每个状态只能容纳一个电子的简单规则。为此,他引入了一个新的二值量子数,后来由萨缪尔·古兹密特和乔治·乌伦贝克确定为\textbf{电子自旋}。