% 开普勒定律(综述)
% license CCBYSA3
% type Wiki

本文根据 CC-BY-SA 协议转载翻译自维基百科\href{https://en.wikipedia.org/wiki/Kepler\%27s_laws_of_planetary_motion}{相关文章}。

\begin{figure}[ht]
\centering
\includegraphics[width=8cm]{./figures/e5bb3dbe32521ea0.png}
\caption{用两个行星轨道说明开普勒定律。这些轨道是椭圆形的,行星1的焦点为F1和F2,行星2的焦点为F1和F3。太阳位于F1。阴影区域A1和A2是相等的,并且是由行星1的轨道在相等的时间内扫过的。行星1的轨道周期与行星2的轨道周期的比值为 \(\left(\frac{a_1}{a_2}\right)^{3/2}\)。} \label{fig_KPL_1}
\end{figure}
在天文学中,开普勒的行星运动定律由约翰内斯·开普勒于1609年发布(除了第三定律,后者于1619年完全发布),描述了行星围绕太阳的轨道。这些定律用椭圆轨道代替了哥白尼日心说中的圆形轨道和本轮,并解释了行星速度的变化。这三条定律如下:
\begin{enumerate}
\item 行星的轨道是椭圆,太阳位于其中一个焦点上。
\item 连接行星和太阳的线段在相等的时间间隔内扫过相等的面积。
\item 行星轨道周期的平方与其轨道半长轴的立方成正比。
\end{enumerate}
行星的椭圆轨道通过火星轨道的计算得到了证明。从这些计算中,开普勒推断出太阳系中其他天体(包括距离太阳较远的天体)也具有椭圆轨道。第二定律确定了当行星靠近太阳时,其运动速度较快。第三定律表达了行星距离太阳越远,其轨道周期越长。

艾萨克·牛顿在1687年证明,像开普勒定律这样的关系,作为他自己运动定律和万有引力定律的结果,也适用于太阳系。

更精确的历史方法可以在《新天文学》和《哥白尼天文学概要》中找到。
\subsection{与哥白尼的比较 } 
约翰内斯·开普勒的定律改进了哥白尼的模型。根据哥白尼的观点:
\begin{enumerate}
\item 行星轨道是一个带有本轮的圆形。
\item 太阳大致位于轨道的中心。
\item 行星在主轨道中的速度是恒定的。
\end{enumerate}
尽管哥白尼正确地指出了行星绕太阳转,但他在定义行星轨道时是错误的。开普勒通过引入超越几何的物理解释,正确地定义了行星轨道,如下所示:
\begin{enumerate}
\item 行星轨道不是带有本轮的圆形,而是椭圆形。
\item 太阳不在轨道的中心,而是在椭圆轨道的一个焦点上。
\item 行星在轨道中的线速度和角速度都不是恒定的,但面积速度(与角动量的概念密切相关)是恒定的。
\end{enumerate}
地球轨道的离心率使得从3月春分到9月秋分的时间约为186天,不等于从9月秋分到3月春分的时间约为179天。如果用一条直径将轨道分成两部分,轨道将被平分,但通过太阳的平面与地球赤道平行的平面将轨道分成两部分,面积比约为186比179。因此,地球轨道的离心率约为:
\[
e \approx \frac{\pi}{4} \cdot \frac{186-179}{186+179} \approx 0.015~
\]
这个值接近正确值(0.016710218)。这个计算的准确性要求选择的两个日期位于椭圆轨道的短轴上,并且每一半的中点位于长轴上。由于这里选择的日期是春分和秋分,这个计算在近日点(地球最接近太阳的日期)恰好落在冬至时是正确的。目前的近日点,接近1月4日,比较接近12月21日或22日的冬至。
\subsection{命名法}  
开普勒工作的当前表述形式花了近两个世纪才定型。伏尔泰的《牛顿哲学要义》(*Eléments de la philosophie de Newton*,1738年)是第一部使用“定律”术语的出版物。[6][7] 《天文学家传记百科全书》在关于开普勒的条目中(第620页)指出,至少从约瑟夫·德·拉兰德(Joseph de Lalande)时代起,科学定律的术语就已被使用。[8] 是罗伯特·斯莫尔(Robert Small)在其1814年出版的《开普勒的天文发现概述》(*An Account of the Astronomical Discoveries of Kepler*)中,将这三条定律定型,加入了第三条。[9] 斯莫尔还声称(与历史记载不符),这些定律是基于归纳推理的经验定律。[7][10]

此外,目前使用的“开普勒第二定律”其实是一个误称。开普勒有两个版本的定律,在定性上有所关联:“距离定律”和“面积定律”。“面积定律”最终成为三条定律中的第二条;但开普勒自己并没有以这种方式优先考虑它。[11]
\subsection{历史}  
开普勒于1609年发布了他的前两条行星运动定律,[12] 这些定律是通过分析第谷·布拉赫的天文观测数据发现的。[13][14][15][5]: 53 开普勒的第三定律则在1619年发布。[16][14] 开普勒信仰哥白尼的太阳系模型,该模型要求行星沿圆形轨道运行,但他无法将布拉赫非常精确的观测数据与火星轨道的圆形拟合起来——火星的离心率恰好是除水星之外所有行星中最高的。[17] 他的第一定律反映了这一发现。

在1621年,开普勒注意到他的第三定律也适用于木星的四颗最亮的卫星。[Nb 1] 戈德弗罗伊·温德林(Godefroy Wendelin)也在1643年做出了这一观察。[Nb 2] 第二定律(以“面积定律”形式)在1664年由尼古拉斯·梅尔卡托尔(Nicolaus Mercator)在一本书中提出异议,但到了1670年,他的《哲学汇刊》已支持这一理论。[18][19] 随着世纪的推进,这一理论变得更加广泛接受。[20] 德国的接受情况在1688年(牛顿的《原理》出版并被视为基本的哥白尼主义作品)和1690年(戈特弗里德·莱布尼茨关于开普勒工作的成果出版)之间发生了显著变化。[21]

牛顿被认为理解了第二定律并非仅适用于万有引力的平方反比定律,而是仅作为该定律径向性质的结果,而其他定律则依赖于吸引力的平方反比形式。卡尔·龙格(Carl Runge)和威廉·伦茨(Wilhelm Lenz)在很晚时候在行星运动的相空间中识别出一个对称性原理(正交群 O(4) 的作用),该原理解释了牛顿引力下的第一和第三定律,类似于角动量的守恒通过旋转对称性解释第二定律。[22]
\subsection{公式} 
行星在开普勒定律作用下的运动学数学模型允许进行大量进一步的计算。  
\subsubsection{第一定律}  
开普勒的第一定律指出:

每颗行星的轨道是椭圆形的,太阳位于椭圆的两个焦点之一。  
\begin{figure}[ht]
\centering
\includegraphics[width=8cm]{./figures/16318e9d2bc7251f.png}
\caption{开普勒的第一定律将太阳置于椭圆轨道的一个焦点上。} \label{fig_KPL_2}
\end{figure}

