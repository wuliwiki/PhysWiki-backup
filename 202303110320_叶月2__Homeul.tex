% 齐次函数的欧拉定理
% 欧拉定理|齐次函数

\begin{issues}
\issueOther{简明微积分中不应该出现矢量空间,应当从预备知识中删去}
\end{issues}


\pentry{矢量空间\upref{LSpace}, 复合函数求导 链式法则\upref{ChainR}}

首先介绍一下什么是齐次函数。

\begin{definition}{齐次函数}
假设 $f: V \to W $ 是域 $ F $ 内的两个向量空间之间的函数。

我们说 $f$ 是 $k$ \textbf{次齐次函数},如果对于所有非零的 $\alpha \in F$ 和 $\mathbf{v} \in V$ ,都有:
\begin{equation}
f(\alpha \mathbf{v}) = \alpha^k f(\mathbf{v}) 
\end{equation}
即是,在欧几里得空间,$f(\alpha \mathbf{v}) = f(k) \ f(\mathbf{v})$ , 其中 $f(k)$ 为指数函数。
\end{definition}

\begin{example}{}
$f(x,y,z)=x^5y^2z^3$ 是 $10$ 次齐次函数,因为 $(\alpha x)^5(\alpha y)^2(\alpha z)^3=\alpha^{10}x^5y^2z^3$。

$f(x,y)=x^5 + 2 x^3 y^2 + 9 x y^4$ 是 $5$ 次齐次函数。
\end{example}

齐次函数的欧拉定理表述如下:

\begin{theorem}{齐次函数的欧拉定理}
若 $k$ 齐次函数 $ f:\mathbb{R}^n \to \mathbb{R}$ 是可导的,那么
\begin{equation}
{\displaystyle \mathbf {x} \cdot \nabla f(\mathbf {x} )=kf(\mathbf {x} )\qquad }
\end{equation}
\end{theorem}
\textbf{证明}: 记 $f=f(x_{1},\ldots ,x_{n})=f(\mathbf {x} )$,把以下等式两端对 $\alpha$ 求导:
\begin{equation}
{\displaystyle f(\alpha \mathbf {x} )=\alpha ^{k}f(\mathbf {x} )}
\end{equation}
利用复合函数求导法则,可得:
\begin{equation}
{\frac {\partial }{\partial {\alpha x}_{1}}}f(\alpha \mathbf {x} ){\frac {\mathrm {d} }{\mathrm {d} \alpha }}(\alpha x_{1})+\cdots +{\frac {\partial }{\partial {\alpha x}_{n}}}f(\alpha \mathbf {x} ){\frac {\mathrm {d} }{\mathrm {d} \alpha }}(\alpha x_{n})=k\alpha ^{k-1}f(\mathbf {x} )
\end{equation}
因此:
\begin{equation}
x_{1}{\frac {\partial }{\partial {\alpha x}_{1}}}f(\alpha \mathbf {x} )+\cdots +x_{n}{\frac {\partial }{\partial {\alpha x}_{n}}}f(\alpha \mathbf {x} )=k\alpha ^{k-1}f(\mathbf {x} )
\end{equation}
即
\begin{equation}
\mathbf {x} \cdot \nabla f(\alpha \mathbf {x} )=k\alpha ^{k-1}f(\mathbf {x} )
\end{equation}
令 $\alpha=1$ 得证。 证毕。

类似上面的推导过程,我们还可以得到如下推论:
\begin{corollary}{}
若 $f:\mathbb{R}^n \to \mathbb{R}$ 是可导的,且是 $ k $ 阶齐次函数。则它的一阶偏导数 $\partial f/\partial x_i$ 是 $k-1$ 阶齐次函数。
\end{corollary}
\textbf{证明}: 记 $ f=f(x_{1},\ldots ,x_{n})=f(\mathbf {x} )$,并把以下等式两端对 $x_{i}$ 求导:
\begin{equation}
f(\alpha \mathbf {x} )=\alpha ^{k}f(\mathbf {x} )
\end{equation}
利用复合函数求导法则,可得:
\begin{equation}
\frac {\partial }{\partial x_{i}}f(\alpha \mathbf {x} ){\frac {\mathrm {d} }{\mathrm {d} x_{i}}}(\alpha x_{i})=\alpha ^{k}{\frac {\partial }{\partial x_{i}}}f(\mathbf {x} ){\frac {\mathrm {d} }{\mathrm {d} x_{i}}}(x_{i})
\end{equation}
因此:
\begin{equation}
\alpha {\frac {\partial }{\partial x_{i}}}f(\alpha \mathbf {x} )=\alpha ^{k}{\frac {\partial }{\partial x_{i}}}f(\mathbf {x} )
\end{equation}
所以
\begin{equation}
{\frac {\partial }{\partial x_{i}}}f(\alpha \mathbf {x} )=\alpha ^{k-1}{\frac {\partial }{\partial x_{i}}}f(\mathbf {x} )
\end{equation}
证毕。
