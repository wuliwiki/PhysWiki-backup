% 中国传媒大学 2014 年数据结构与计算机网络
% 中国传媒大学 2014  数据结构 计算机网络

答题说明:答案- -律写在答题纸上,不需抄题,标明题号即可,答在试题上无效.

\subsection{一、单项选择题}
(每小题2分,共50分)

1.下列关于栈和队列说法中,正确的是(    ). \\
A.消除递归不一定需要使用栈 \\
B.对同一输入序列进行两组不同的合法入栈和出栈组合操作,所得的输出序列也一定相同 \\
C.通常使用队列来处理函数或过程处理 \\
D.队列和栈是运算受限的线性表,只允许在表的两端进行运算

2.一个栈的入栈序列是1,2,3,4,5,则栈的不可能的输出序列是( ). \\
A. 5,4,3,2,1,  $\qquad$ B.4,5,3,2,1 $\qquad$ C.4,3,5,1,2, $\qquad$ D.1,2,3,4,5

3.已知栈的输入序列为$1$,$2$,$3$,...$n$,输出序列为$p_1$, $p_2$, $p_3$, ..., $p_n$,若$p_1=3$,则$p_2$的值为(    ). \\
A.一定是2 $\qquad$ B.-定是1 $\qquad$ C.可能是1 $\qquad$ D.可能是2

4.循环队列用数组A[0..m-1]存放其元素值,已知其头尾指针分别为front和rear,则当前元素个数为( ). \\
A. (rear-front+m) MOD m $\qquad$ B. rear-front+1 \\
C. rear-front-1 $\qquad$ D. rear-front

5.已知有一维数组$A[0..m*n-1]$,若要对应为$m$行、$n$列的矩阵,将元素$A[k](0\leqslant k<m*n)$表示成矩阵的第$i$行、第$j$列的元素$(0\leqslant i<m, 0\leqslant j<n)$,则下面的对应关系是(    ). \\
A. i=k/n, j=k\%m $\qquad$ B. i=k/m, j=k\%m \\
C. i=k/n, j=k\%n $\qquad$ D. i=k/m, j=k\%n

6.设有一个$10$阶的对称矩阵$A$,采用压缩存储方式,以行序为主存储,$a_{1,1}$. 为第一元素,其存储地址为$1$,每个元素占一个地址空间,则$a_{8,5}$的地址是(). \\
A.13 $\qquad$ B.33 $\qquad$ C.18 $\qquad$ D.40

7.含有$n$个结点的三叉树的最小高度是(    ). \\
A.$n$ $\qquad$ B.$\lfloor n/3 \rfloor$ $\qquad$ C.$\lfloor log_3n\rfloor+1$ $\qquad$ D.$\lceil log_3(2n+1)\rceil$

8.在一棵具有n个结点的二叉树中,所有结点的空子树个数等于(    ). \\
A. n $\qquad$ B. n-1 $\qquad$ C. n十1 $\qquad$ D.2*n

9.在常用的描述二叉排序树的存储结构中,关键字值最大的结点是( ).
A.左指针一定为空
B.右指针- -定为空
C.左右指针均为空
D.左右指针均不为空
10. 由权值为9、2、5、7的四个叶子构造一 棵哈夫曼树,该树的带权路径长度为
(1).
A.23
B.37
C.44
D.46
11.若一个具有n个结点、k条边的非连通无向图是- -个森林(n>k), 则该森林中
必有树的数目是( ).
A. k
B. n
C. n-k
D. n+k
12.采用邻接表存储的图的广度优先遍历算法类似于树的( ).
A.中根遍历
B.先根遍历
C.后根遍历
D.按层次遍历
13.在有向图G的拓扑序列中,若顶点V;在顶点V;之前,则下列情形不可能出现
的是().
A. G中有弧<V,V>
B. G中有一条从V;到V;的路径
C. G中没有弧<V,V>
D. G中有一条从V;到V;的路径
14.有一个长度为12的有序表,按折半查找法对该表进行查找,在表内各元素等
概率情况下,查找成功所需的平均比较次数是( ).
A.37/12
B.35/12
C.39/12
D.43/12
15.假设有k个关键字互为同义词,若用线性探查法把这k个关键字存入,至少要
进行的探查次数是( ).
A. k-I
B. k
C. k+1
D. k(k+1)/2
16.下列序列中,满足堆定义的是( ).
A. (100, 86, 48,73,35, 39,42,57, 66, 21)
B. (12, 70, 33, 65,24, 56, 48, 92, 86,33)
C. (103, 97,56,38,66, 23,42,12, 30,52, 6, 26)
D. (5, 56, 20, 23, 40, 38, 29, 61, 36,76, 28, 100)
17.对于一个长度为n的任意表进行排序,至少需要进行的比较次数是( ).
A. O(m)
B. O(n)
C.0(logn)
D. O(nlogn)
18.关于网络分层结构,下列说法正确的是( ).
A.某一层可以使用其上一层提供的服务而不需知道服务是如何实现的
B.层次划分越多,灵活性越好,协议效率也越高
C.由于结构彼此分离,实现和维护更加困难
D.当某一层发生变化时,只要接口关系不变,以上或以下的各层均不受影响
19.不受电磁干扰或噪声影响的介质是( ).
A.双绞线
B.光纤
C.同轴电缆
D.微波
20.要在带宽为4kHz的信道上用2秒钟发送80kb的数据块,按照香农定理,信
