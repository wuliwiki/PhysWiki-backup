% 广度优先搜索(BFS)
% keys 算法|广度优先搜索|BFS|C++

上文提到了 DFS,本文将介绍与它类似的一种搜索方式:广度优先搜索或者叫宽度优先搜索(Breadth First Search)简称 BFS.

BFS 可以用队列来实现,BFS 的搜索顺序为:当前的状态面临的所有状态优先搜索.

对于一棵搜索树,BFS 不是一个节点一个节点的搜索,而是一层一层的搜索,如果当一个状态第一次入队的时候,那么就是从起始状态到该状态的最小步数,因此 BFS 可以求最小值.对于一张权值都为 $1$ 的图,可以求得从 $1$ 号点到 $n$ 号点的最短距离.

\begin{figure}[ht]
\centering
\includegraphics[width=14.25cm]{./figures/BFS_1.png}
\caption{BFS 的搜索树} \label{BFS_fig1}
\end{figure}


BFS 的伪代码:\begin{lstlisting}[language=cpp]
while (队列不空) 
{
    t <--队头
    从队头更新其他节点
}
\end{lstlisting}

我们还是看一道例题来学习 BFS:

题意:有一张地图,$0$ 表示可以走,$1$ 表示不能走,你可以从上下左右四个方向任意一个方向移动一个位置,问你从 $1,1$ 走到 $n,n$ 最少需要花费多少步?

样例:\begin{lstlisting}[language=cpp]
输入:
5 5
0 1 0 0 0
0 1 0 1 0
0 0 0 0 0
0 1 1 1 0
0 0 0 1 0

输出:
8
\end{lstlisting}
\begin{figure}[ht]
\centering
\includegraphics[width=14.25cm]{./figures/BFS_2.png}
\caption{样例的搜索过程}} \label{BFS_fig2}
\end{figure}

\textbf{思路:}

上图就为 BFS 的搜索顺序,每次搜索到达的点,然后次数加 $1$,可以保证答案一定是最小值.先把 $0$ 号点入队,然后出队,然后把 $0$ 号点出队,然后访问队头的可以到达的节点,然后在把队头可以到达的节点入队,可以到达的节点的答案为从起点到队头的答案再加从队头到可以到达的节点的步数也就是直接加 $1$.

\textbf{代码:}
\begin{lstlisting}[language=cpp]
const int N = 110;

typedef pair<int, int> PII;

int n, m, d[N][N], g[N][N]; // d 为每个点的距离,g 为地图
PII q[N * N];   // q 数组为数组模拟队列,因为是二元组,所以要开二倍空间

bool check(int x, int y)
{
    // 如果越界了或者当前遇到了不可通过的墙壁或者访问到了更新过的节点,就直接返回
    if (x < 0 || x >= n || y < 0 || y >= m || d[x][y] != -1 || g[x][y] == 1) 
        return false;
    return true;
}

int bfs()
{
    // 数组模拟队列
    int hh = 0, tt = -1;
    q[ ++ tt] = {0, 0};     // 最开始起点入队
    memset(d, -1, sizeof d);    // 初始化所有的点都还没被更新过
    d[0][0] = 0;    // 起点的距离为 1
    
    int dx[4] = {-1, 0, 1, 0};  // x 轴的偏移量
    int dy[4] = {0, 1, 0, -1};  // y 轴的偏移量
    
    while (hh <= tt)    // 队列不为空
    {
        auto t = q[hh ++ ];     // 取出队头并把队头出队
        for (int i = 0; i < 4; i ++ )   // 枚举上下左右 4 个方向
        {
            // a、b 为从队头往上下左右四个方向之一走下去点
            int a = t.first + dx[i], b = t.second + dy[i];   
            if (!check(a, b)) continue;     // 如果下一个点不满足条件的话就往下一个方向走
            q[ ++ tt] = {a, b}; // 把走过的点加入队头
            d[a][b] = d[t.first][t.second] + 1;  // 更新距离
        }
    }
    
    return d[n - 1][m - 1]; // 返回从 1 号点走到 (n, m) 的距离
}  
\end{lstlisting}

\begin{figure}[ht]
\centering
\includegraphics[width=14.25cm]{./figures/BFS_3.png}
\caption{偏移量技巧} \label{BFS_fig3}
\end{figure}

可以看到,对于涉及到上下左右方向类的问题,可以不需要写 $4$ 个 \verb|if| 判断,我们直接使用两个偏移量,这样可以帮助我们省很多代码.

\begin{enumerate}
\item 往上走就是横坐标减 $1$,纵坐标不变.对应:$\mathtt{-1, 0}$
\item 往右走就是横坐标不变,纵坐标加 $1$.对应:$\mathtt{0, 1}$
\item 往下走就是横坐标加 $1$,纵坐标不变.对应:$\mathtt{1, 0}$
\item 往左走就是横坐标不变,纵坐标减 $1$.对应:$\mathtt{0, -1}$
\end{enumerate}

让我们来看一道使用偏移量的好题:\href{http://poj.org/problem?id=3322}{例题}

题目大意:控制一个 $1 \times 1 \times 2$ 小木块,问从起点走到终点最少需要操作多少次,“.”表示硬地,可以躺着也可以站立着,