% 库仑规范
% keys 麦克斯韦方程组|规范变换|库仑规范|失势|播送方程|调和场
% license Xiao
% type Tutor

\pentry{规范变换\nref{nod_Gauge}, 泊松方程\nref{nod_PoiEqu}}{nod_31e3}

当我们规范变换\upref{Gauge}中选择 $\lambda$ 使得 $\div \bvec A \equiv 0$ 时,就得到了\textbf{库仑规范}。 这种选择在很多情况下可以简化计算,因为这样标势和矢势的麦克斯韦方程组(\autoref{eq_EMPot_4}~\upref{EMPot} \autoref{eq_EMPot_5}~\upref{EMPot})化简为 (真空中的光速为 $c = 1/\sqrt{\mu_0\epsilon_0}$)%链接未完成
\begin{equation}\label{eq_Cgauge_4}
\laplacian \varphi = -\frac{\rho}{\epsilon_0}~,
\end{equation}
\begin{equation}\label{eq_Cgauge_3}
\laplacian \bvec A - \frac{1}{c^2} \pdv[2]{\bvec A}{t} = -\mu_0\bvec j + \frac{1}{c^2} \grad\qty(\pdv{\varphi}{t})~.
\end{equation}
其中\autoref{eq_Cgauge_4} 的形式与静止电荷分布的泊松方程\upref{PoiEqu}形式一样,但同样适用于随时间变化的 $\rho$。这看起来似乎电荷对 $\varphi$ 存在瞬时作用,但由于标势和矢势都只是数学上的量而不是物理上的可观测量, 所以是不违背相对论中 “信息不能超光速的” 的。

在没有净电荷和电流的区域, 以上两式进一步化简为
\begin{equation}
\laplacian \varphi = 0~,
\end{equation}
\begin{equation}
\laplacian \bvec A - \frac{1}{c^2}\pdv[2]{\bvec A}{t} = \frac{1}{c^2} \grad\qty(\pdv{\varphi}{t})~.
\end{equation}
% (这段似乎没什么意义)
% 若假设无穷远处没有电荷电流也没有电场磁场, 那么无穷远处的标势矢势需满足
% \begin{equation}
% \laplacian \varphi = 0 \qquad
% \grad \varphi + \pdv{\bvec A}{t} = 0
% \end{equation}
% 以及 $\bvec A$ 为调和场(无散无旋场)\upref{HarmF}。 

任何满足 $\laplacian \lambda = 0$ 的规范变换都能保持 $\div \bvec A = 0$ 不变。 可见只有 $\div \bvec A = 0$ 并不能唯一确定标势矢势,还需要一定的边界条件。 库仑规范的另一个条件是: 令标势的边界条件为无穷远处 $\varphi = 0$, 于是标量势可以唯一确定为
\begin{equation}\label{eq_Cgauge_1}
\varphi(\bvec r, t) = \frac{1}{4\pi\epsilon_0} \int \frac{\rho(\bvec r', t)}{\abs{\bvec r - \bvec r'}} \dd[3]{r'}~.
\end{equation}

\subsection{无源的情况}

如果空间中没有电荷(称为\textbf{无源}),那么 $\varphi$ 处处为零。由此得到一个常见的结论是
\begin{equation}\label{eq_Cgauge_2}
\bvec {E}(t) = -\pdv{\bvec A}{t}~.
\end{equation}
矢势的波动方程\autoref{eq_Cgauge_3} 也变得非常简单
\begin{equation}
\laplacian \bvec A - \frac{1}{c^2} \pdv[2]{\bvec A}{t} = 0~.
\end{equation}
该方程的通解就是由任意极化方向 $\bvec A_0$ 和延传播方向 $\bvec k$ 光速传播的平面简谐波\upref{PWave}
\begin{equation}
\bvec A(\bvec r, t) = \bvec A_0 \cos(\bvec k \vdot \bvec r - \omega t + \phi) \qquad (\omega = ck)~.
\end{equation}
叠加而成。 最后施加 $\div \bvec A = 0$ 条件后得 $\bvec A_0 \vdot \bvec k = 0$, 即振动方向与传播方向垂直。
