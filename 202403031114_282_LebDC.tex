% 勒贝格控制收敛定理
% license Usr
% type Tutor

\begin{issues}
\issueDraft
\end{issues}

\pentry{Lebesgue 积分\nref{nod_Lebes1}}{nod_3b68}

(DCT)是测度论和实分析中的一个强大结果,它提供了在一序列函数的积分的极限等于极限函数的积分的条件。这里是该定理的正式陈述:

支配收敛定理:
设 $(X,M,\mu)$ 是一个测度空间,让 $\{f_n\}$ 是一序列可测函数 $f_n: X\to\mathbb R$ 或 $f_n:X\to\mathbb C$。 它们几乎处处逐点收敛于函数 $f: X\to\mathbb R$ 或 $f:X\to\mathbb C$。 如果存在一个非负的可积函数 $g:X\to[0,\infty)$, 对所有的 $n$ 和几乎所有的 $x\in X$ 有,
∣
�
�
(
�
)
∣
≤
�
(
�
)
,
∣f 
n
​
 (x)∣≤g(x),
那么 
�
f 是可积的,并且
lim
⁡
�
→
∞
∫
�
�
�
 
�
�
=
∫
�
�
 
�
�
.
lim 
n→∞
​
 ∫ 
X
​
 f 
n
​
 dμ=∫ 
X
​
 fdμ.

定理的关键组成部分:
