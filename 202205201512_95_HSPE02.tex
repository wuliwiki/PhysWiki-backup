% 电势能与电势(高中)
% keys 电势能|电势|电势差

\pentry{静电场\upref{HSPE01}, 功和机械能\upref{HSPM07}}

\subsection{静电力做功}

与重力做功类似,电荷在电场中运动时,静电力对电荷做的功只与电荷的初末位置有关,而与电荷的运动路径无关.

在场强为$\bvec E$的匀强电场中,若电荷沿电场线方向运动的距离为$d$,则静电力做功为
\begin{equation}\label{HSPE02_eq4}
W=qEd
\end{equation}

对于非匀强电场的静电力做功,可考虑利用动能定理(\autoref{HSPM07_eq6}~\upref{HSPM07})、电势能(\autoref{HSPE02_eq1} )或电势差等进行求解.

\subsection{电势能}

电荷在电场中所具有的势能叫\textbf{电势能},用符号$E_p$表示.

与重力势能一样,电荷的电势能与零势能点的选取有关.电荷在电场中某一点电势能大小,等于电荷从这一点移动到零电势能点时静电力对它做的功.

电荷在电场中的$A$点运动到$B$点,类似重力做功(\autoref{HSPM07_sub1}~\upref{HSPM07}),静电力做功与电势能变化的关系为
\begin{equation}\label{HSPE02_eq1}
W_{AB}=E_{pA}-E_{pB}
\end{equation}

静电力做正功时,电荷的电势能减小;静电力做负功时,电荷的电势能增加.

\subsection{电势}

电荷在电场中某一点的电势能与其电荷量的比值,叫做\textbf{电势},用$\varphi$表示.电势的单位为\textbf{伏特}(简称\textbf{伏}),符号为$V$,$1\mathrm{V}=1\mathrm{J/C}$.

\begin{equation}\label{HSPE02_eq2}
\varphi = \frac{E_p}{q}
\end{equation}

同一点的电势与零电势能位置的选取有关,定义了零电势能点后,该点的电势为零.

电场线的方向指向电势降落的方向.选取无穷远处的电势为零,在正电荷产生的电场中电势都为正值,越靠近场源电荷电势越高,负电荷则反之.

\subsection{电势差}

电场中两点之间电势的差值叫做\textbf{电势差}(或\textbf{电压}),用$U$表示.电势差是一个标量,表示两点间电势的高低关系.如$A$和$B$两点间的电压可表示为
\begin{equation}\label{HSPE02_eq3}
U_{AB}=\varphi_A - \varphi_B
\end{equation}

若$W_{AB} < 0$,则表示$A$点的电势比$B$点的电势低.

结合\autoref{HSPE02_eq1} 、\autoref{HSPE02_eq2} 和\autoref{HSPE02_eq3} 可知,电荷在电场中从$A$点移动至$B$点时有
\begin{equation}\label{HSPE02_eq5}
U_{AB} = \frac{W_{AB}}{q}
\end{equation}

由上式可以看出,$U_{AB}$的数值等于单位正电荷从$A$点移动至$B$点是静电力所做的功.

再结合\autoref{HSPE02_eq4} 和\autoref{HSPE02_eq5} 可得,在匀强电场中,两点之间的电势差等于电场强度与这两点沿电场方向的距离的乘积,即
\begin{equation}
U=Ed
\end{equation}

\subsection{等势面}

由电场中电势相同的点所组成的面叫做\textbf{等势面},任何两个等势面都不会相交.

在等势面上任意两点的电势差为零,则电荷在等势面上运动时,静电力不做功.

在匀强电场中,各等势面是相互平行的平面.
