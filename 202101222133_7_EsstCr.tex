% 爱森斯坦判别式
% 多项式|Eisenstein|不可约多项式
\pentry{素理想与极大理想\upref{Ideals}}

爱森斯坦判别式可以用来判断多项式是否可约.满足爱森斯坦判别式条件的多项式必然是不可约的,即无法表示为两个多项式的乘积.

最简单的多项式是整系数多项式,也被称为\textbf{整数环上的}多项式,因此我们先介绍此类多项式上的爱森斯坦判别式.这只是更一般的爱森斯坦判别式的特例.

\begin{theorem}{整数环上的爱森斯坦判别式}\label{EsstCr_the1}
设有整系数多项式$f(x)=f_0+f_1x+f_2x^2+\cdots+f_nx^n$,若存在素数$p$使得:
\begin{enumerate}
\item $p\nmid f_n$;
\item 对于$i\in\{0, 1, 2, \cdots, n-1\}$,都有$p|f_i$;
\item $p^2|f_0$.
\end{enumerate}
那么$f(x)$就是不可约多项式.
\end{theorem}

\textbf{证明}:

反设$f(x)$可约,则可以写成$f(x)=(h_0+h_1x+h_2x^2+\cdots+h_rx^r)(g_0+g_1x+g_2x^2+\cdots+g_sx^s)$.

由于$f(x)=f_0+f_1x+f_2x^2+\cdots+f_nx^n$,故\footnote{这种运算也被称为卷积.}\begin{equation}f_i=\sum\limits_{j+k=i}h_jg_k\end{equation}

\textbf{利用条件3}:注意到$f_0=h_0g_0$,而$p^2\nmid f_0=h_0g_0$,故$h_0$和$g_0$中有一个是不能被$p$整除的.不妨设$p\nmid h_0$.

\textbf{利用条件2}:由于$p\mid f_0=h_0g_0$,故$p\mid g_0$.又由于$p\mid f_1=h_0g_1+h_1g_0$且$p\mid g_0$,故$p\mid h_0g_1$;再由于$p\nmid h_0$,故$p\mid g_1$.

\textbf{归纳}:类似地,$p\mid f_2=h_0g_2+h_1g_1+h_2g_0$,而$p\mid g_1, p\mid g_0$和$p\nmid h_0$又能推出$p\mid g_2$.以此类推,可以得到$p\mid g_s$.

\textbf{利用条件1}:由于$p\nmid f_n=h_rg_s$,故应有$p\nmid g_s$,与上一步矛盾!

由此可知反设不成立.

\textbf{证毕}.

虽然判别式的表达看起来很复杂,但是从证明过程可以看出来,三个条件的设置都自有其考虑.

爱森斯坦判别式不止是可以用于整系数多项式,也可用于任意整环上的多项式.整系数多项式的爱森斯坦判别式只需要整除知识即可理解,而任意整环上的形式则需要素理想的概念.

\begin{theorem}{整环上的爱森斯坦判别式}
设整环$R$上有多项式$f(x)=f_0+f_1x+f_2x^2+\cdots+f_nx^n$,若存在$R$的\textbf{素理想}$P$,使得:
\begin{enumerate}
\item $f_n\not\in P$;
\item 对于$i\in\{0, 1, 2, \cdots, n-1\}$,都有$f_i\in P$;
\item $f_0\not\in P^2$,其中$P^2=\{p_1p_2|p_i\in P\}$.
\end{enumerate}
那么$f(x)$就是不可约多项式.
\end{theorem}

其证明可以通过改写\autoref{EsstCr_the1} 的过程得到,把每一个形如$p|a$的式子都写为$a\in P$即可,其中$a$只是个符号,可以表示任意$f_i, g_i, h_i$





