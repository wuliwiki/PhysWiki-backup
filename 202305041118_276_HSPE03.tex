% 恒定电流(高中)
% 电流|电压|电阻|欧姆定律|伏安特性曲线

\pentry{静电场\upref{HSPE01}}

\subsection{电流}

导体两端存在电势差时,导体中的自由电荷就会在电场力的作用下发生定向移动,电荷的定向移动形成了\textbf{电流}。为了表示电流强弱,定义流过导体某一个横截面的电荷量$q$和所用时间$t$的比值叫做电流,用$I$表示:
\begin{equation}\label{eq_HSPE03_1}
I = \frac{q}{t}~.
\end{equation}

国际单位制中,电流的单位是\textbf{安培}(简称\textbf{安}),符号为$\mathrm{A}$,由式知 $1\mathrm{A} = 1\mathrm{C}/\mathrm{s}$。常用的电流单位还有\textbf{毫安}($\mathrm{mA}$)和\textbf{微安}($\mathrm{\mu A}$),$1\mathrm{A}=10^3\mathrm{mA}=10^6\mathrm{\mu A}$。

电流由电荷的定向移动产生,是有方向的,规定正电荷定向移动的方向为电流的方向,负电荷定向移动方向与电流的方向相反,\autoref{eq_HSPE03_1} 的$q$表示的是某段时间$t$内按规定正方向通过截面的净电荷量。要注意的是,电流虽有方向,但电流的合成不遵循平行四边形定则,因此电流是个标量。

\subsubsection{分类}
\begin{itemize}
\item 电流方向不随时间改变的电流叫做\textbf{直流电流}。
\item 方向和大小都不随时间改变的电流叫\textbf{恒定电流}。
\item 方向随时间发生周期性改变的电流叫做\textbf{交变电流}。
\end{itemize}

\subsubsection{微观表达式}

设一段均匀导体的横截面积为$S$,导体内的自由电荷密度为$n$,每个自由电荷的电荷量为$q$,自由电荷定向移动的平均速率为$\bar v$。在单位时间内通过导体横截面的自由电荷所占体积为$S\bar v $,所含自由电荷数为$nS\bar v $,则单位时间内通过导体横截面的电荷量为$nqS\bar v$,由电流的定义可知
\begin{equation}
I=nqS\bar v~.
\end{equation}

\subsection{电阻}

导体两端的电压与通过导体的电流之比叫做电阻,用$R$表示:
\begin{equation}\label{eq_HSPE03_2}
R=\frac{U}{I}~.
\end{equation}

电阻反映了导体对电流阻碍作用的大小。

在国际单位制中,电阻的单位是\textbf{欧姆}(简称\textbf{欧}),符号为$\mathrm{\Omega}$。常用的电阻单位还有千欧($\mathrm{k\Omega}$)和兆欧($\mathrm{M\Omega}$),$1\mathrm{M\Omega}=10^3\mathrm{k\Omega}=10^6\mathrm{\Omega}$。

根据\autoref{eq_HSPE03_2} ,用电压表测出导体两端电压、用电压表测出通过导体电流来计算导体电阻的方法叫做\textbf{伏安法}。

\subsubsection{欧姆定律}

通过导体的电流$I$跟导体两端的电压$U$成正比,跟导体的电阻$R$成反比:
\begin{equation}
I=\frac{U}{R}~.
\end{equation}

\subsubsection{电阻定律}

在温度一定的条件下,对于同种材料的均匀导体,其电阻$R$与它的长度$l$成正比,与它的横截面积$S$成反比:
\begin{equation}\label{eq_HSPE03_3}
R=\rho \frac{l}{S}~.
\end{equation}

\autoref{eq_HSPE03_3} 中的$\rho$是\textbf{电阻率},是反映导体导电性能的物理量。在长度、横截面积一定的条件下,电阻率越小的导体,其电阻越小,导电性能越好。国际单位中,电阻率的单位是\textbf{欧姆米}(简称\textbf{欧米}),符号为$\mathrm{\Omega\cdot m}$。

电阻率的倒数叫做\textbf{电导率},用$\sigma$表示,即:
\begin{equation}
\sigma = \frac{1}{\rho}~.
\end{equation}

\subsection{伏安特性曲线}

为了研究导体的电阻变化规律,常通过测量多组电压和电流数据,并绘制图像。以横坐标表示导体两端电压,纵坐标表示通过导体电流的$I-U$图像叫做导体的\textbf{伏安特性曲线}。

对于金属导体,在温度无明显变化的时候,其伏安特性曲线是一条通过原点的直线,即通过电阻的电流与电阻两端的电压成正比,导体的电阻为定值,与电压和电流无关。具有这样伏安特性的电学元件叫做\textbf{线性元件}。

对于如二极管等伏安特性曲线不是直线的电学元件,叫做\textbf{非线性元件}。欧姆定律不适用于非线性元件,但仍可以用电阻的定义式(\autoref{eq_HSPE03_2} )来计算元件在不同情况下的导电性能。

\subsection{电阻的串联和并联}

\subsubsection{电阻的串联}

串联电路中,电流处处相等,电路两端的总电压等于各部分电路两端的电压之和。

$n$个电阻串联时,有

\begin{equation}\label{eq_HSPE03_4}
U=U_1+U_2+\dots +U_n=IR_1+IR_2+\dots +IR_n
\end{equation}

对于串联后的总电阻$R$有

\begin{equation}\label{eq_HSPE03_5}
U=IR
\end{equation}

由\autoref{eq_HSPE03_4} 和 \autoref{eq_HSPE03_5} 可得

\begin{equation}
R=\sum_{i=1}^{n}R_i
\end{equation}

即在串联电路的总电阻等于各部分电路的电阻之和。

\subsubsection{电阻的并联}

并联电路中,干路电流等于各支路电流之和,电路的总电压与各支路电压相等。

$n$个电阻并联时,有

\begin{equation}\label{eq_HSPE03_6}
I=I_1+I_2+\dots +I_n=\frac{U}{R_1}+\frac{U}{R_2}+\dots +\frac{U}{R_n}
\end{equation}

对于并联后的总电阻$R$有

\begin{equation}\label{eq_HSPE03_7}
I=\frac{U}{R}
\end{equation}

由\autoref{eq_HSPE03_6} 和\autoref{eq_HSPE03_7} 可得

\begin{equation}
\frac{1}{R}=\sum_{i=1}^{n}\frac{1}{R_i}
\end{equation}

即在并联电路中总电阻的倒数等于各支路电阻的倒数之和。
