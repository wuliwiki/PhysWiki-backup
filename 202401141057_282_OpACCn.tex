% OpenACC 笔记
% license Usr
% type Note

\begin{issues}
\issueDraft
\end{issues}

\pentry{C++ 基础\upref{Cpp0}}

\href{https://www.openacc.org/}{OpenACC} 可以像 OpenMP 那样,通过添加少量的 \verb`#pragma` (预编译指令)给已有的程序进行 GPU 加速。本文以 C++ 为例。 它由于易用性,它常用于高性能科学计算中。

和 OpenMP 一样, OpenACC 需要编译器支持。一般使用英伟达显卡专用的 \href{https://developer.nvidia.com/hpc-sdk}{NVIDIA HPC SDK} 或者 \href{https://www.openacc.org/tools/gcc-for-openacc}{GCC}。

下面给出一个入门例程。 在 GCC 中,直接加上 \verb`-fopenacc` 选项即可! 例如 \verb`g++ -fopenacc main.cpp -o main.x`。

如果运行中显示 \verb`Not executing on GPU.` 那么说明可能环境配的不对。 在编译时可以加上 \verb`-fopt-info` 选项输出更多信息。

\begin{lstlisting}[language=none,caption=test1.cpp]
#include <iostream>
#include <vector>
#include <openacc.h>

// Function to initialize the vectors with values
void initialize(std::vector<double>& a, std::vector<double>& b, int n) {
	for(int i = 0; i < n; ++i) {
		a[i] = static_cast<double>(i);
		b[i] = static_cast<double>(2 * i);
	}
}

// detect if GPU is actually running
void detect_gpu()
{
	#pragma acc parallel loop
	for (int i = 0; i < 10; ++i) {
		if (i == 5) {
			if (acc_on_device(acc_device_none))
				printf("Not executing on GPU.\n");
			else
				printf("Executing on GPU.\n");
		}
	}
}

// Main function
int main() {
	const int n = 1000000; // Size of the vectors
	std::vector<double> a(n), b(n), c(n);
	double *pa = a.data(), *pb = b.data(), *pc = c.data();

	// Initialize vectors a and b
	initialize(a, b, n);

	// detect GPU
	detect_gpu();

	// Using OpenACC to offload the following computation to an accelerator
	// and explicitly handle data movement
	#pragma acc data copyin(pa[0:n], pb[0:n]) copyout(pc[0:n])
	{
		while (true) {
			#pragma acc parallel loop
				for(int i = 0; i < n; ++i) {
				pc[i] = pa[i] + pb[i];
			}
		}
	}

	// Display the first 10 results
	for(int i = 0; i < 10; ++i) {
		std::cout << "c[" << i << "] = " << c[i] << std::endl;
	}

	return 0;
}
\end{lstlisting}

\subsubsection{循环}
\begin{itemize}
\item \verb`#pragma acc parallel`: GPU 并行运算
\item \verb`#pragma acc kernels`: Identifies a code block for parallelization, allowing the compiler to automatically manage parallelism.
\item \verb`#pragma acc loop`: Used within parallel or kernels regions to indicate loops that should be parallelized.
\end{itemize}

\subsubsection{函数和变量}
\begin{itemize}
\item \verb`#pragma acc routine`: 让一个函数可以在 GPU 代码中被调用(也可以在 CPU 代码调用)。
\item \verb`#pragma acc declare`: Used for declaring variables or creating a data region.
\end{itemize}

\subsubsection{数据传输}
\begin{itemize}
\item \verb`#pragma acc data`: Manages data movement to and from the GPU.
\item \verb`#pragma acc enter data`: Specifies data that should be moved to the GPU.
\item \verb`#pragma acc exit data`: Specifies data to be moved back from the GPU.
\item \verb`#pragma acc update`: Synchronizes data between the host and the GPU.
\item \verb`copy, copyin, copyout`, create, present: Clauses for data construct to define how data is handled (e.g., whether it's copied to/from the GPU or just created there).
\end{itemize}

\subsubsection{线程精细控制}
\begin{itemize}
\item \verb`gang, worker, vector`: Used with loop directive to control how loop iterations are distributed over parallel execution units.
\item \verb`collapse(n)`: Collapses nested loops to enhance parallelism.
\item \verb`reduction(operator:list)`: Performs a reduction operation (like sum, max) across parallel elements.
\end{itemize}

\subsection{编译器选项}
\begin{itemize}
\item \verb`-ta=tesla`: Compiler option to target NVIDIA Tesla GPUs.
\item \verb`-Minfo=accel`: Provides feedback about the code generated by the compiler.
\end{itemize}
