% 爱因斯坦宇宙场方程
% keys 广义相对论|宇宙常数|爱因斯坦场方程
% license Usr
% type Tutor

\pentry{爱因斯坦场方程\nref{nod_EinEqn},第二 Bianchi 恒等式(微分)\nref{nod_RicciC}}{nod_88fb}

在爱因斯坦推出场方程的年代(约 $1916$ 年),人们一般认为宇宙是静止不动的。但根据当时的爱因斯坦场方程 $ G_{\mu \nu} = 8 \pi G  T_{\mu \nu}$,人们意识到这使得宇宙是正在膨胀或收缩的。(实际上银河系以外存在其他星系都是在 $1924$ 年被埃德温·哈勃首次发现的。)

为此,由第二 Bianchi 恒等式(微分)不难想到,为了仍保证能量守恒,只能对原本的场方程修正。原本的场方程只剩下yi'ge