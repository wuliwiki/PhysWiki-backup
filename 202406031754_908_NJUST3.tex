% 南京理工大学 2007 量子真题
% license Usr
% type Note

\textbf{声明}:“该内容来源于网络公开资料,不保证真实性,如有侵权请联系管理员”

\subsection{回答下列问题(考生在下列12题中选作10题,每题6分,共60分}

1.(1)德布罗意关系式是仅适用于基本粒子如电子、中子,还是同样适用于具有内部结构的复合体系?(2)粒子的德布意波长是否可以比其本身线度长或短?二者之间是否有必然联系?

2.一维空间两粒子体系的波函数为$\psi (x_1, x_2, )$,写出下列概率:
\begin{enumerate}
    \item $x \leq x_1 \leq x + dx, \alpha \leq x_2 \leq \beta$
    \item 发现粒子1的位置介于$x$和$x + dx$之间(不对粒子2进行观察)
\end{enumerate}

3.玻恩近似法的基本思想是什么?

4.什么是束缚态?什么是定态?

5.写出量子力学五个基本假设中的任意三个。

6厄米算符有哪些特性?

7.判断下列论断的正误:

(1)一体系的哈密顿量是守恒量,则体系必处于定态。

(2)若$\comm{\Q F}{\Q G}$ 都是体系的守恒量,则它们都具有确定的测量值。

(3)量子力学适用于微观体系,而经典力学适用于宏观体系。

8.试述电子具有自旋的实验证据。

9.十九世纪末期人们发现了哪些不能被经典物理学所解释的新的物理现象?

10.具体分立本征值谱的力学量在其自身表象中如何表示?其本征矢量是什么?

11.如果有心力场不是库伦场(即V(r)不与 $\frac{1}{r}$ ),则角分布函数将取什么形式?

12.量子力学中角动量是如何定义的?地球自转是否与量子力学中的自转概念相对应?

\subsection{计算题(请考生在下列7题中选6题,每题15分,共90分}

1.
