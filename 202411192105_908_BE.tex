% 马克斯·玻恩(综述)
% license CCBYSA3
% type Wiki

本文根据 CC-BY-SA 协议转载翻译自维基百科\href{https://en.wikipedia.org/wiki/Max_Born}{相关文章}。

\textbf{马克斯·玻恩}(Max Born,FRS FRSE)(德语发音:[ˈmaks ˈbɔʁn];1882年12月11日-1970年1月5日)是一位德裔英国物理学家和数学家,对量子力学的发展起到了关键作用。他还在固态物理学和光学领域做出了重要贡献,并在20世纪20年代和30年代指导了许多著名物理学家的研究工作。玻恩因其“对量子力学的基础研究,特别是对波函数统计解释的贡献”于1954年获得了诺贝尔物理学奖。[1]

玻恩于1904年进入哥廷根大学,在那里他结识了三位著名的数学家:费利克斯·克莱因(Felix Klein)、大卫·希尔伯特(David Hilbert)和赫尔曼·闵可夫斯基(Hermann Minkowski)。他以弹性丝和带的稳定性为主题撰写了博士论文,并因此获得了该校哲学系奖项。1905年,他开始与闵可夫斯基研究狭义相对论,随后完成了以汤姆森原子模型为主题的资格论文(habilitationsschrift)。1918年,他在柏林偶然遇到弗里茨·哈伯(Fritz Haber),讨论了一种金属与卤素反应生成离子化合物的过程,这一过程今天被称为\textbf{玻恩–哈伯循环}(Born–Haber cycle)。

在第一次世界大战期间,玻恩最初被安置为无线电操作员,但由于他的专业知识,他被调往从事声测位研究工作。1921年,玻恩返回哥廷根大学,为他长期的朋友和同事詹姆斯·弗兰克(James Franck)安排了另一张教授职位。在玻恩的领导下,哥廷根大学成为当时世界上物理学的主要中心之一。

1925年,玻恩与维尔纳·海森堡(Werner Heisenberg)共同提出了量子力学的矩阵力学表述。次年,他提出了薛定谔方程中 \textbf{ψ\ψ} 概率密度函数的标准解释,这一贡献为他赢得了1954年的诺贝尔物理学奖。

玻恩的影响远远超出了他的个人研究领域。包括马克斯·德尔布吕克(Max Delbrück)、齐格弗里德·弗吕格(Siegfried Flügge)、弗里德里希·洪德(Friedrich Hund)、帕斯夸尔·约尔丹(Pascual Jordan)、玛丽亚·哥佩特-梅耶(Maria Goeppert-Mayer)、洛塔尔·沃尔夫冈·诺德海姆(Lothar Wolfgang Nordheim)、罗伯特·奥本海默(Robert Oppenheimer)和维克多·魏斯科普夫(Victor Weisskopf)等人都在哥廷根大学师从玻恩取得了博士学位。

此外,玻恩的助手中也包括许多著名的物理学家,例如恩里科·费米(Enrico Fermi)、维尔纳·海森堡(Werner Heisenberg)、格哈德·赫兹伯格(Gerhard Herzberg)、弗里德里希·洪德(Friedrich Hund)、沃尔夫冈·泡利(Wolfgang Pauli)、莱昂·罗森菲尔德(Léon Rosenfeld)、爱德华·泰勒(Edward Teller)以及尤金·维格纳(Eugene Wigner)。

1933年1月,纳粹党在德国上台,作为犹太人的玻恩被暂停了他在哥廷根大学的教授职位。他移居到英国,并在剑桥大学圣约翰学院担任职务,同时撰写了一本科普书《不安定的宇宙》(*The Restless Universe*)以及《原子物理学》(*Atomic Physics*),后者很快成为一本标准教材。

1936年10月,他成为爱丁堡大学自然哲学的泰特教授。在那里,他与两位德国出生的助手E.沃尔特·凯勒曼(E. Walter Kellermann)和克劳斯·富克斯(Klaus Fuchs)合作,继续他的物理学研究。

1939年8月31日,也就是第二次世界大战在欧洲爆发的前一天,玻恩成为了英国公民。他一直在爱丁堡工作到1952年,随后退休回到西德的巴特皮尔蒙特(Bad Pyrmont)。1970年1月5日,玻恩在哥廷根的一家医院去世。[2]

