% 正则系宗法
% keys 统计力学|亥姆霍兹自由能|正则系综

\begin{issues}
\issueDraft
\end{issues}

正则系综的分布函数可以通过让系统与一个大热源接触来导出.考虑将待研究的系统和一个温度恒定为 $T$ 的大热源\footnote{即热容为无穷大的系统.}接触.我们想要研究,当两者达到平衡时,系统处于某一量子态 $S$ 的概率 $\rho_S$.我们很快将发现,这个几率只和系统处于该量子态时的能量有关:即
\begin{equation}
\rho_S=\frac{1}{Z}\exp\qty(-\beta E_S)
\end{equation}
其中 $Z$ 为归一化常数,使得 $\sum_S \rho_S=1$.我们称 $Z$ 为正则系综的配分函数,它由下式给出($S$ 遍历所有量子态):
\begin{equation}
Z=\sum_S \exp\qty(-\beta E_S)
\end{equation}
一旦得到了系统的配分函数,系统的一切平衡态热力学性质都可以由配分函数导出\upref{TheSta}.注意这里的配分函数是系统的,而不是词条\upref{MBsta}中采用的单粒子配分函数.而本文中采用的系统配分函数,可以推广到更普遍的情况,如气体分子间有相互作用势的情况.

对于\textbf{近独立子系},系统的总能量 $E$ 可以表示成所有粒子的能量之和.于是
\begin{equation}
\begin{aligned}
\rho_S=\exp(-\beta E_S)=\prod_{i=1}^n\exp(-\beta\epsilon_i)
Z=\sum_S\exp(-\beta E_S)=
\end{aligned}
\end{equation}
