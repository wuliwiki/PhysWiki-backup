% jQuery 笔记

\pentry{JavaScript 入门笔记\upref{JS}}

\begin{itemize}
\item jQuery 命令通常在 document ready 内部使用 \verb|$(document).ready(function(){/*jQuery 命令*/});|, 另一种等效的格式为 \verb|$(function(){/*jQuery 命令*/})|
\item Selector: \verb|$("p")| 选择所有 \verb|<p>...</p>| 元素. 例程: \verb|$(document).ready(function(){$("button").click(function(){$("p").hide();});});| 其中 \verb|click(回调函数)| 设置按钮的回调函数, 即 \verb|$("p").hide()|, 隐藏所有段落. 相反, \verb|*.show()| 重新显示.
\item \verb|.hide(speed, callback)| 其中 \verb|speed| 是毫秒, \verb|callback| 是完成后的回调
\item \verb|.toggle()| 或者 \verb|.toggle(speed, callback)| 让隐藏的元素显示, 显示的元素隐藏
\item \verb|$("#id")| 选中所有指定 id 的元素
\item \verb|$(".class")| 选中所有指定 class 的元素
\end{itemize}

事件
\begin{itemize}
\item \verb|$(document).ready()| 网页渲染完成
\item \verb|*.click()| 按下, \verb|*.dbclick()| 双击
\item \verb|*.mouseenter()| 鼠标进入元素. 例子: \verb|$(document).ready(function(){$("#p1").mouseenter(function(){做一些事});});|
\item \verb|*.mouseleave()| 离开元素, \verb|*.mousedown()| 按下鼠标左键, \verb|*.mouseup()| 松开鼠标左键.
\item \verb|*.hover(fun1,fun2)| 相当于 \verb|*.mouseenter(fun1); *.mouseleave(fun2);|
\item \verb|<input>| 元素有 \verb|$("input").focus()| 和 \verb|*.blur()|, 分别是光标选中和离开
\item 
\end{itemize}
