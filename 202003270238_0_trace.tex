% 矩阵的迹
\pentry{矩阵的本征方程\upref{MatEig}, 矩阵的相似变换}% 未完成

\begin{definition}{矩阵的迹}
令 $N$ 维方阵 $\mat A$ 的矩阵元为 $a_{ij}$, 它的\textbf{迹(trace)}定义为它对角线上矩阵元之和
\begin{equation}
\opn{tr}(A) = \sum_{i=0}^N a_{ii}
\end{equation}
\end{definition}

\begin{theorem}{}
相似变换不会改变矩阵的迹.
\end{theorem}

\begin{corollary}{}
矩阵的迹等于它的所有 $N$ 个本征值 $\lambda_i$ 相加. 如果某个本征值有 $n$ 重简并, 就视为 $n$ 个本征值.
\begin{equation}
\opn{tr}(A) = \sum_{i=0}^N \lambda_i
\end{equation}
\end{corollary}

\subsection{性质}
\begin{itemize}
\item 迹运算是\textbf{线性}的, 即
\begin{equation}
\opn{tr}(c_1A+c_2B) = c_1\opn{tr}(A) + c_2\opn{tr}(B)
\end{equation}

\item 矩阵乘法的迹满足
\begin{equation}\label{trace_eq1}
\opn{tr}(AB) = \opn{tr}(BA)
\end{equation}
\end{itemize}

\subsection{证明相似变换不变性}
根据性质\autoref{trace_eq1}
\begin{equation}
\opn{tr}(\mat P^{-1}\mat A\mat P) = \opn{tr}(\mat A\mat P\mat P^{-1}) = \opn{tr}(\mat A)
\end{equation}
