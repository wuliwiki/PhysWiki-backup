% 图
% 图

\begin{issues}
\issueTODO
\end{issues}


\begin{definition}{图;点;边}\pentry{二元关系\upref{Sample}}
图G(V,E)是集合V上的一种二元关系E.
集合V的元素称为图的点,若两个点之间有这种确定的二元关系,则称有一条边连这两个点.
一个图的点的数目称为这个图的阶,记作|G|,图的边的数目称为它的度,记作||G||.
\end{definition}

\begin{definition}{关联;相邻}
\begin{itemize}
\item 若有一条边连一个图的某两个点,则称这两个点相邻,并称这两个点为这条边的端点.
\item 若某一点是某一条边的端点,则称这个点和这条边关联.
\item 若两条边和同一点关联,则称这两条边相邻.
\end{itemize}
\end{definition}
\begin{definition}{特殊点;特殊边}
\begin{itemize}
\item 两个端点是同一个点的边称为环.
\item 若某条边的两个端点不是同一个点,且只有一条边连这两个点,则称这条边为杆.
\item 以某两节点为端点的边可能不止一条,这时称连这两个节点的边为重边.
\end{itemize}
\end{definition}
\begin{definition}{特殊图}
\begin{itemize}
\item 只有一个点而没有边的图称为平凡图,没有边的图称为孤立图.
\item 既可以有环,也可以有重边的图称为准图.
\item 没有环而可能有重边的图称为带重图.
\item 没有重边而可能有环的图称为带环图.
\item 既没有重边也没有环的图称为简单图,每两个点都相邻的简单图称为完全图.
\item 若一个图的阶是有限的,则称这个图为有限图,否则称这个图为无限图.
\item 若一个”阶图的点用 1 , 2 , … , n 来代表,则称它为标定图,若在图的每一条边上赋以一个实数或者对于每个节点赋以一个实数,则称它为赋权图.
\end{itemize}
\end{definition}

注:以上内容参考了《数学辞海》卷二