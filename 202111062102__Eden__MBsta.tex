% 玻尔兹曼分布(统计力学)
% 玻尔兹曼分布|等概率原理|经典统计

\pentry{麦克斯韦—玻尔兹曼分布\upref{MxwBzm},理想气体的状态密度\upref{IdSDp},拉格朗日乘数法\upref{LagMul}}

根据等概率原理,对于平衡状态的孤立系统,每个微观状态出现的概率是相等的.

根据统计力学中的量子力学假设,系统中单粒子态能级为分立的:$\epsilon_1,\epsilon_2,\cdots$.其中设第 $l$ 个能级的简并数为 $\omega_l$(意味着这个能级一共有 $\omega_l$ 种线性无关的态).设第 $l$ 个能级上共有 $a_l$ 个粒子,序列 ${a_l}$ 构成粒子系统的一种分布.微观状态数最多的分布出现的概率最大,称为\textbf{最概然分布}.

我们下面要谈的是玻尔兹曼分布,它\textbf{不涉及全同粒子假设},即粒子之间是可区分的.我们将得到同经典统计中一样的结果\upref{MxwBzm}.

\subsection{推导}
\begin{equation}
\Omega=\frac{N!}{\Pi_l a_l!}
\end{equation}