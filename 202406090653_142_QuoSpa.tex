% 商空间(线性代数)
% keys 商集|商空间|向量空间
% license Xiao
% type Tutor

\begin{issues}
\issueDraft
\end{issues}

\pentry{二元关系\nref{nod_Relat},仿射集\nref{nod_AffSet}}{nod_f4b0}

原作者:FFjet

\addTODO{用仿射集的语言重写}


\footnote{参考《物理学中的几何方法》} % Giacomo:改成cite

设 $W$ 是向量空间 $V$ 的一个子空间。我们想通过 $W$ 来定义 $V$ 中元素的一个等价关系,并由此得到 $V$ 的一个划分(商集)。对于 $v \in V$,$v + W$ 是 $V$ 的一个仿射集,全体(关于 $W$ 的)仿射集的集合构成里一个划分(商集),定义了一个等价关系——对于 $v_1, v_2\in V$,如果 $v_1 - v_2\in W$,则称它们是关于 $W$ 等价的,记为 $v_1 \sim v_2$。我们把这个划分(商集)记为$V / W: = V / \sim$,包含 $v$ 的等价类记为 $[v]: = v + W$.
% Giacomo: 从来没见过用`( )`表示等价类的,感觉不是很标准的记号,改了。

% 不难证明这是一个等价关系。因此,它就确定了 $V$ 的一个分类。元素 $v$ 的等价类,即 $V$ 中所有与 $v$ 的等价的元的全体,用符号 $[v] $ 标记。这样,有等价类为元素的商集
% \begin{equation}
% V / W=\{[v] | v \in V\}~.
% \end{equation}

现在我们在商集 $V / W$ 中引入线性运算,使它也成为一个线性空间。当然这种线性运算应与 $V$ 中原有的线性运算要有联系。为此,对于等价类的加法和数乘,我们自然采用下列定义
\begin{equation}
\begin{array}{l}
[v_1] + [v_2] = [v_1 + v_2]~, \\
a \cdot [v] = [a v]\qquad (a \in K)~.
\end{array}
\end{equation}
这里的定义与等价类的代表的选取无关,故是有意义的,容易证明 $V/W $ 在这些运算下构成体 $K $ 上的向量空间。$V/W $ 称为 $V$ 关于子空间 $W$ 的\textbf{商空间},$[0] = W$ 是它的零元。

\subsection{商空间的维度}

\pentry{基底(线性代数)\nref{nod_VecSpn}}{nod_5d4b}

下面我们再来分析一下 $V, W$ 和 $V/W $ 三者的基底之间的关系。设 $\{u_1, u_2, \cdots, u_m\}$ 是 $W $ 的一个基底。我们再补充 $n- m$ 个元素 ${u}_{m+1}, \cdots {u}_{n}$ 使 $\{u_1, u_2, \cdots, u_{n}\}$ 成为 $V$ 的一个基底。



% by FFjet
% 
% 于是对 $V$ 中任一元 $v=a^iu_i$ 所确定的等价类,有
% \begin{equation}
% [v]=[a^{i} u_{i}]=a^{i}[u_{i}]~.
% \end{equation}
% (未完成)
%因此任意等价类可用(U1) , (Uz) , …, (u.) 的线性组合表示。但是这并不是说,
%(u心(u心…, Cun) 构成V/W 的一个基底因为它们是线性相关的。由于u,EW
%(i=l, …,m), 故, u,-OEW, 即u, (i= l, …, m) 与0 是关于mod W 等价,所以
%最后有
%(u,) = (0) , i = l , 2 ,\dots, m
%类似地分析可知
%(u,)\#-(0) ,j=m+l, \dots,n
%且容易证明(作为练习), (u1) (j = m + 1, \dots, n) 是线性无关的,它们构成V/W
%的一个基底。作为推论,我们有dim V / W= n-m.
%这些概念和结构,我们将在讨论同调群和上同调群时用到。
