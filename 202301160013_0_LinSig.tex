% Linux 进程的信号(笔记)

\begin{issues}
\issueDraft
\end{issues}

\begin{itemize}
\item 参考\href{https://unix.stackexchange.com/questions/6332/what-causes-various-signals-to-be-sent}{这里} 以及 Wikipedia \href{https://en.wikipedia.org/wiki/Signal_(IPC)}{相关页面}。
\item Signal 用于 interrupt, suspend, terminate 一个进程
\item 进程可以向系统注册 signal handler, 用户处理不同的信号
\item \verb|SIGSEGV| (segmentation fault) 是最常见的, 一般是读写数组长度外的内存导致的
\item \verb|SIGBUS| (bus error) 现在已经不那么常见
\item \verb|SIGHUP| 当关闭命令行时产生
\end{itemize}

\subsubsection{键盘错误}
\begin{itemize}
\item \verb|SIGINT| 但键盘按下 Ctrl+C 产生
\item \verb|SIGTSTP| 暂停程序, 按下 Ctrl+Z 产生
\end{itemize}

\subsubsection{CPU 错误}
\begin{itemize}
\item \verb|SIGILL| (Illegal instruction) 是非法的 CPU 指令
\item \verb|SIGFPE| 最可能除数为零时发生
\end{itemize}
