% 哈尔滨工业大学 2004 年 考研 量子力学
% license Usr
% type Note

\textbf{声明}:“该内容来源于网络公开资料,不保证真实性,如有侵权请联系管理员”

\subsection{一、(40 分)回答下列问题}
\begin{enumerate}
\item 简述运动粒子的波动一粒子两象性。
\item 波函数$\psi(\vec{r},t)$是用来描述什么的?它应该满足什么样的自然条件?$|\psi(\vec{r},t|^2$,的物理含义是什么?
\item 分别说明什么样的状态是束缚态、简并态与负宇称态?
\item 物理上可观测量应该对应什么样的算符?为什么?
\item 直接写出对易关系$[x,\hat{p}]$,并给出两者满足的不确定(测不准)关系式。
\item 一个粒子处于长、宽、高分别为$a,b,c$的方形盒子中,设其所处的状态用波函数$\psi(x,y,z)$来描述,问该粒子处于盒子的上三分之一空间中的几率是多少?
\item 已知一个哈密顿算符$\hat{H}_0$的解,即$\hat{H}_0\psi_n(x)=E^0_n\psi_0(x)$,试给出$\hat{H}=\hat{H}_0\pm V_0$的解,式中$V_0$已为实常数。
\item 设有两个自由电子,已知甲电子的运动速度是乙电子的 3 倍,求出两者德布罗意波长之比。
\end{enumerate}
\subsection{二、(25 分)}
自旋为$\frac{1}{2}$的粒子,处于宽度为$a$的一维非对称无限深方势阱中,设$t=0$时粒子处于状态
$$\psi(x,0) = \frac{1}{3} \varphi_0(x) 
\begin{pmatrix}
1 \\\\
0
\end{pmatrix}
-\frac{2}{3} \varphi_1(x) 
\begin{pmatrix}
0 \\\\
1
\end{pmatrix}
+ \frac{\sqrt{2}}{3} \varphi_1(x)
\begin{pmatrix}
1 \\\\
0
\end{pmatrix}~$$
\subsection{三、(25分)}
设粒子处于一维势阱之中
$$V(x,y)=\begin{cases}
\infty,&x < 0  \\\\
-V_0 ,& 0≤x≤a, \\\\
0,&x>a
\end{cases}~
$$
\subsection{四、(20分)}

\subsection{五、(20分)}

\subsection{六、(20分)}