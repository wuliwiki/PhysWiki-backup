% 南京理工大学 普通物理 B(845)模拟五套卷 第四套
% license Usr
% type Note

\textbf{声明}:“该内容来源于网络公开资料,不保证真实性,如有侵权请联系管理员”

\subsection{一、 填空题 I(24 分,每空 2 分)}
1.一质点作圆周运动,设半径为 $R$,运动方程$s=v_0t-\frac{1}{2}bt^2$ ,其中 $s$ 为弧长,$v_0$为初速度,$b$ 为常数。则任一时刻 $t$ 质点的法向加速度为_____,切向加速度为__________。

2. 质量为 $4.25Kg$ 的质点,在合力$F=5i-3j(N)$的作用下由静止从原点运动到$r=5i-3j(m)$时,合力所做的功为_________;此时质点的运动速度大小为_______________。

3. 花样滑冰运动员绕通过自身的竖直轴转动,开始时两臂伸开,转动惯量为 $J$,角速度为$\omega$ ,然后她将两臂收回,使转动惯量减少为 $J/2$,这时她转动的角速度变为____________。

4. 质量为 2kg 的质点,按方程 $x=0.2\sin[5t-(\pi/6)]$沿着 $x$ 轴振动,则 $t=0$ 时,作用于质点的力的大小为__________;作用于质点的力的最大值为________,此时质点的位置________。

5. 设平面简谐波沿 $x$ 轴传播时在 $x=0$ 处发生反射 ,反射波的表达式为$y_2=A\cos [2\pi(vt-x/\lambda)+\pi/2]$,已知反射点为一自由端,则由入射波和反射波形成驻波波节的位置坐标为__________。

6. 如图,真空中一长为$L$ 的均匀带电细直杆,总电量为 $q$,则在直杆延长线上
距杆一端距离为 $d$ 的 $P$ 点的电场强度为___________。
\begin{figure}[ht]
\centering
\includegraphics[width=6cm]{./figures/f20e00a1183b6526.png}
\caption{} \label{fig_NJUD4_1}
\end{figure}
7. 一气缸内储有 $10mol$ 单原子分子理想气体,在压缩过程中,外力做功 $209J$,
气体温度升高 $1K$,则气体内能的增量$\Delta E$ 为________$J$,吸收的热量 $Q $____________$J$。
\subsection{二、 填空题 I(24 分,每空 2 分)}
1. 迈克耳孙干涉仪可用来测量单色光的波长,当 M2 反射镜移动距离d=0.3220mm 时,测得某单色光的干涉条纹移过 N=1204 条,则该单色光的波长为_______。

2. 一个 50 匝的半径 R=5.0cm 的通电线圈处于 B=1.5(T)的均匀磁场中,线圈中电流 I=0.2A,则该线圈中的磁矩大小为________,当线圈的磁矩与外磁场方向的夹角从 0 转到 π 时,外磁场对线圈所作的功为___________。

3. 波长为 680nm 的平行光垂直地照射在 12cm 长的两块玻璃片上,两玻璃片一边相互接触,另一边被厚 0.048mm 的纸片隔开,则在这 12cm 内呈现的明条纹数为____________。

4. 一电子在 B=2×10-3T 的磁场中沿半径为 R=2×10-2m、螺距为 h=5×10-2m 的螺旋运动,如图所示,则磁场方向为__________,电子速度大小为________。