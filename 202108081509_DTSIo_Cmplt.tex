% 完备公理

\pentry{实数\upref{ReNum}}

通过戴德金分割, 我们得以填补有理数集中的"空隙", 从而得到了实数集. 从这里开始, 我们将不再区分有理数和由有理数确定的戴德金分割. 纯粹从四则运算的角度, 还不能看出实数与有理数之间有何区别. 然而如果引入一些超出四则运算范围的操作, 即可看出实数与有理数的决定性区别: 实数集是\textbf{完备 (complete)} 的, 或者模糊地说, 实数集里"没有空隙".

\subsection{完备公理}

怎么刻画"不存在缝隙"这样的直观特性呢?

仍旧以数$\sqrt{2}$为例. 我们记得这个数的下类是$L=\{l\leq0\}\cup\{l>0:l^2<2\}$, 上类是$R=\{r>0:r^2>2\}$. $L$和$R$的并集就是有理数集$\mathbb{Q}$, 而由于$\sqrt{2}$并非有理数, 因此$L$和$R$之间"空无一物", 分割$L|R$并不是由一个实际存在的分点确定的. 但实数集却与此不同: 在区间$(-\infty,\sqrt{2})$和$(\sqrt{2},+\infty)$之间的确存在着一个分点$\sqrt{2}$. 

一般来说, 如果像定义戴德金分割那样对实数集$\mathbb{R}$进行操作, 那么得不到任何新的对象: \textbf{每一个这样的分割都必定是由一个分点确定的.} 实际上, 如果将实数集$\mathbb{R}$分成不相交的两部分$A\cup B$, 使得$L=A\cap\mathbb{Q}$满足分割下类的定义, 那么$R=B\cap\mathbb{Q}$自动满足分割上类的定义, 于是$L|R$自动成为一个戴德金分割, 它自然就确定了一个实数$x$. 