% 基尔霍夫电路定律

\pentry{电势\upref{QEng},电流\upref{I}, 电压\upref{Voltag},麦克斯韦方程组\upref{MWEq}}
\subsection{基尔霍夫电流定律}

\begin{theorem}{基尔霍夫电流定律}
\textbf{基尔霍夫电流定律}又称为\textbf{基尔霍夫第一定律},规定所有进入某节点的电流的总和等于所有离开这节点的电流的总和. 或者说,假设进入某节点的电流为正值,离开这节点的电流为负值,则所有涉及这节点的电流的代数和等于零.以方程式表达,对于电路的任意节点,有
\begin{equation}
\sum_{k=1}^n i_k =0
\end{equation}
其中,$i_k$是第$k$个进入或离开这节点的电流,是流过与这节点相连接的第$k$个支路的电流,可以是实数或复数.
\end{theorem}

由于累积的电荷是电流与时间的乘积,从电荷守恒定律可以推导出这条定律.其实质是稳恒电流的连续性方程,即根据电荷守恒定律,流向节点的电流之和等于流出节点的电流之和.

\subsubsection{定理证明}
考虑电路的某节点,跟这节点相连接有$n$个支路.假设进入这节点的电流为正值,离开这节点的电流为负值,则这节点的总电流$i$等于流过支路$k$的电流$i_k$的代数和:
\begin{equation}
i=\sum_{k=1}^n i_k
\end{equation}
将这方程式对某段时间积分,可以得到这段时间该节点电荷的增加
\begin{equation}
q=\sum_{k=1}^n q_k
\end{equation}
其中 $q = \int_{t_1} i \dd{t}$ $q_k=\int_0^t i_k(t') \mathrm{d}t'$是流过支路$k$的电荷,$t$是末时刻,$t'$是积分时间变量.

假设$q>0$,则正电荷会累积于节点;否则,负电荷会累积于节点.根据电荷守恒定律,$q$是个常数,不能够随着时间推移而改变.由于这节点是个导体,不能储存任何电荷.所以,$q=0$、$i=0$,基尔霍夫电流定律成立:
\begin{equation}
\sum_{k=1}^n i_k =0
\end{equation}

从上述推导可以看到,\textbf{只有当电荷量为常数时,基尔霍夫电流定律才会成立}.通常,这不是个问题,因为静电力相斥作用,会阻止任何正电荷或负电荷随时间演进而累积于节点,大多时候,节点的静电荷是零.

不过,电容器的两块导板可能会允许正电荷或负电荷的累积.这是因为电容器的两块导板之间的空隙,会阻止分别累积于两块导板的异性电荷相遇,从而互相抵消.对于这种状况,流向其中任何一块导板的电流总和等于电荷累积的速率,而不是零.但是,若将位移电流$\mathbf{J}_D$纳入考虑,则基尔霍夫电流定律依然有效.

从电动力学的角度怎么来考虑推导呢?其实对含电介质的安培定律取散度,然后与高斯定律相结合,即可得到基尔霍夫电流定律:
\begin{equation}
\nabla \vdot \mathbf {J} =-\epsilon _{0}\nabla \vdot {\frac {\partial \mathbf {E} }{\partial t}}=-{\frac {\partial \rho }{\partial t}}
\end{equation}
 
其中,$\mathbf{J}$是电流密度,$\epsilon_0$是电常数,$\mathbf{E}$是电场,$\rho$是电荷密度.

这是电荷守恒的微分方程式.以积分的形式表述,从封闭表面流出的电流等于在这封闭表面内部的电荷$Q$的流失率:
\begin{equation}
\oint _{S}\mathbf {J} \cdot \mathrm {d} \mathbf {a} =-{\frac {\mathrm {d} Q}{\mathrm {d} t}}
\end{equation}

基尔霍夫电流定律等价于电流的散度是零.所以我们可以看出,对于不含时电荷密度$\rho$,该定律成立.对于含时电荷密度,则必需将位移电流纳入考虑.

\subsection{基尔霍夫电压定律}
\subsubsection{定理内容}
\begin{theorem}{基尔霍夫电压定律}
\textbf{基尔霍夫电压定律}又称为\textbf{基尔霍夫第二定律},表明沿着闭合回路所有元件两端的电势差(电压)的代数和等于零.或者,换句话说,沿着闭合回路的所有电动势的代数和等于所有电压降的代数和.以方程式表达,对于电路的任意闭合回路,
\begin{equation}
\sum_{k=1}^m V_k = 0
\end{equation}
其中,$m$是此闭合回路的元件数目,$V_k$是元件两端的电压,可以是实数或复数.
\end{theorem}

基尔霍夫电压定律不仅应用于闭合回路,也可以把它推广应用于回路的部分电路.

\subsubsection{定理推导}
我们回顾一下电势的定义:
\begin{equation}
\phi(\mathbf{r})\stackrel{def}{=} - \int_{L} \mathbf{E} \cdot \mathrm{d} \mathbf{l}
\end{equation}

容易发现基尔霍夫电压定律可以等价表达为:
\begin{equation}
\oint _{{C} }\mathbf {E} \cdot d\mathbf {l} =0
\end{equation}
其中,${C}$是积分的闭合回路.

这是什么?这正是环路定理!而我们也知道,此方程式是法拉第电磁感应定律对于恒定磁场的的简化版本.假设通过闭合回路${C}$的磁通量为常数,则此方程式成立.此方程式指明,电场沿着闭合回路${C}$的线积分为零.将这线积分切割为几段支路,就可以分别计算每一段支路的电压.

由于要求闭合回路${C}$的磁通量为常数,基尔霍夫第二定律也存在着理论限制.由于含时电流会产生含时磁场,通过闭合回路${C}$的磁通量是时间的函数,根据法拉第电磁感应定律,会有电动势$\mathcal{E}$出现于闭合回路${C}$.所以,电场沿着闭合回路${C}$的线积分不等于零.这是因为电流会将能量传递给磁场;反之亦然,磁场亦会将能量传递给电流.

对于含有电感器的电路,必须将基尔霍夫电压定律加以修正.由于含时电流的作用,电路的每一个电感器都会产生对应的电动势$\mathcal{E}_k$,需要将这电动势纳入基尔霍夫电压定律,才能求得正确答案.
