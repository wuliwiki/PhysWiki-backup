% 汉斯·贝特(综述)
% license CCBYSA3
% type Wiki

本文根据 CC-BY-SA 协议转载翻译自维基百科\href{https://en.wikipedia.org/wiki/Hans_Bethe#Honors_and_awards}{相关文章}。


汉斯·阿尔布雷希特·贝特(Hans Albrecht Bethe,1906年7月2日–2005年3月6日)是一位德裔美籍物理学家,他在核物理学、天体物理学、量子电动力学和固态物理学方面做出了重要贡献,并因其在恒星核合成理论方面的工作获得了1967年诺贝尔物理学奖。[1][2][3][4] 贝特大部分职业生涯在康奈尔大学担任教授。[5]

1939年,贝特发表了一篇论文,确立了CNO循环作为更大质量恒星在主序星阶段的主要能量来源,这一贡献为他赢得了1967年的诺贝尔奖。[6] 在第二次世界大战期间,贝特担任洛斯阿拉莫斯国家实验室的理论部主任,该实验室研发了第一颗原子弹。他在计算武器的临界质量并开发用于“胖子”原子弹(在1945年8月投放到长崎)的内爆方法方面发挥了关键作用。

战后,贝特在氢弹的研发中发挥了重要作用,他还担任该项目的理论部门负责人,尽管他最初加入该项目的目的是证明氢弹无法制造。[7] 后来,他与阿尔伯特·爱因斯坦以及原子科学家紧急委员会一起,反对核试验和核军备竞赛。他帮助说服肯尼迪和尼克松政府分别签署了1963年的部分核试验禁令条约和1972年的反弹道导弹条约(SALT I)。1947年,他写了一篇重要论文,提供了对兰姆位移的计算,这一工作被认为彻底改变了量子电动力学,并进一步“为现代粒子物理学时代开辟了道路”。[8][9][10] 他对中微子的理解做出了贡献,[11] 并在解决太阳中微子问题中发挥了关键作用。[12] 他还为理解超新星及其过程做出了贡献。[13]

他的科学研究从未停止,直到九十多岁时他仍在发表论文,这使他成为少数几位在其职业生涯的每个十年中至少发表过一篇重要论文的科学家之一,而贝特的职业生涯几乎跨越了七十年。物理学家弗里曼·戴森(曾是他的博士生)称他为“20世纪的最高级问题解决者”,[14] 而宇宙学家爱德华·科尔布则称他为“物理学界最后的老大师”。[15]
\subsection{早年生活}  
贝特于1906年7月2日出生在斯特拉斯堡,当时该地区是德国的莱茵省-阿尔萨斯-洛林的一部分。他是安娜(娘家姓库恩)和阿尔布雷希特·贝特(斯特拉斯堡大学生理学 Privatdozent)的独生子。[16] 尽管他的母亲是阿布拉罕·库恩(斯特拉斯堡大学教授)的女儿,具有犹太背景,[17] 贝特在成长过程中像父亲一样被抚养成新教徒;[18][19] 他后来成为无神论者。[20]
\begin{figure}[ht]
\centering
\includegraphics[width=6cm]{./figures/f8a1eabf841b3f3b.png}
\caption{12岁的汉斯·贝特与父母一起。} \label{fig_Hans_1}
\end{figure}
1912年,他的父亲接受了基尔大学生理学研究所教授兼所长的职位,家人搬进了研究所的所长公寓。最初,他由一位专业教师私下授课,和其他七个男孩女孩一起学习。[21] 1915年,当他的父亲成为新成立的法兰克福歌德大学生理学研究所的所长时,家人再次搬迁。[18]

贝特曾就读于德国法兰克福的歌德中学。1916年,他因患上结核病而中断学业,被送往巴德克罗伊茨纳赫休养。到1917年,他恢复得足够好,能够就读当地的实科学校。次年,他被送往欧登瓦尔德学校,这是一所私立的男女同校寄宿学校。[22] 1922年至1924年,他重新回到歌德中学,完成了中学最后三年的学业。[23]

贝特在通过了高中毕业考试(Abitur)后,于1924年进入法兰克福大学。他决定主修化学。由于物理教学质量较差,尽管法兰克福有像卡尔·路德维希·齐格尔和奥托·萨茨等杰出的数学家,但贝特并不喜欢他们的方法,这些方法将数学呈现得脱离其他学科。[24] 贝特发现自己是一个不太擅长实验的人,他曾因将硫酸洒在实验服上而把实验服弄坏,但他发现由副教授沃尔特·格拉赫教授的高阶物理更有趣。[24][25] 1925年,格拉赫离开了,接替他的卡尔·迈斯纳建议贝特应该去一所理论物理学更强的大学,尤其是慕尼黑大学,在那里他可以师从阿诺德·索末菲尔德。[26][27]

贝特于1926年4月进入慕尼黑大学,索末菲尔德在迈斯纳的推荐下接纳了他成为学生。[28] 索末菲尔德开设了一门关于物理中微分方程的高级课程,贝特非常喜欢这门课。由于索末菲尔德是如此著名的学者,他常常提前收到科学论文的稿件,并将这些论文作为讨论材料,在每周的晚间研讨会上进行讨论。当贝特到达时,索末菲尔德刚刚收到埃尔文·薛定谔关于波动力学的论文。[29]

对于贝特的博士论文,索末菲尔德建议他研究晶体中的电子衍射。作为起点,索末菲尔德建议他参考保罗·埃瓦尔德1914年关于晶体中X射线衍射的论文。贝特后来回忆说,他变得过于雄心勃勃,为了追求更高的准确性,他的计算变得不必要地复杂。[30] 当他第一次见到沃尔夫冈·泡利时,泡利对他说:“听了索末菲尔德讲的关于你的事,我原本期待你能做出比你论文更好的成果。”[31] 贝特后来回忆道:“我想那应该是泡利的夸奖。”[31]
\subsection{早期工作}  
在贝特获得博士学位后,厄尔温·马德隆为他提供了法兰克福的助教职位,1928年9月,贝特搬到父亲家中,父亲在此时刚刚与母亲离婚。贝特的父亲在1929年与维拉·孔盖尔结婚,并且他们有了两个孩子,分别是1933年出生的多丽丝和1934年出生的克劳斯。[32]

贝特并未在法兰克福的工作中找到太多刺激性,1929年,他接受了厄尔温·埃瓦尔德的邀请,前往斯图加特的技术大学。在那里,他写下了他认为自己最伟大的论文《快的粒子射线穿过物质的理论》(Zur Theorie des Durchgangs schneller Korpuskularstrahlen durch Materie)。[33] 从马克斯·博恩对薛定谔方程的解释出发,贝特使用傅里叶变换为碰撞问题导出了一个简化的公式,这个公式今天被称为贝特公式。他在1930年提交了这篇论文以完成他的高等教职资格论文(Habilitation)。[33][35][36]

1929年,索末菲尔德推荐贝特获得洛克菲勒基金会的旅行奖学金。这项奖学金每月提供150美元(约合2024年3000美元[A]),用于资助他出国学习。1930年,贝特选择在英国剑桥大学的卡文迪什实验室做博士后工作,在那里他在拉尔夫·福勒的指导下工作。[37] 应帕特里克·布莱克特的要求,布莱克特当时正在研究云室,贝特为贝特公式创建了一个相对论版本。[38]

贝特以幽默感著称,他与吉多·贝克和沃尔夫冈·里茨勒(两位博士后研究员)共同创作了一篇恶搞论文《绝对零度温度的量子理论》,在论文中他根据摄氏温标的绝对零度温度计算了精细结构常数。[39] 这篇论文讽刺了当时理论物理学中某些类的文章,这些文章纯粹是推测性的,且基于虚假的数值论证,例如阿瑟·爱丁顿在早期论文中试图从基本量中解释精细结构常数的值。最终,他们被迫为此道歉。[40]

在奖学金的后半段,贝特选择于1931年2月前往恩里科·费米的实验室,在罗马工作。他对费米印象深刻,并且感到遗憾没有先去罗马。[41] 贝特发展了贝特假设(Bethe ansatz),这是一种用于求解某些一维量子多体模型的特征值和特征向量的精确解的方法。[42] 他受到了费米简洁性和索末菲尔德严谨性解决问题方法的影响,这些品质也影响了他后来的研究。[43]

洛克菲勒基金会延长了贝特的奖学金,使他得以在1932年返回意大利。[44] 与此同时,贝特在慕尼黑为索末菲尔德工作,担任私人讲师。由于贝特流利的英语,索末菲尔德让他监督所有讲英语的

1932年,贝特接受了图宾根大学的助理教授职位,当时汉斯·盖格是实验物理学的教授。[47][48] 纳粹新政府通过的第一部法律是《恢复职业公务员法》。由于贝特的犹太背景,他被解除了在大学的职位,因为这是一个政府职位。盖格拒绝提供帮助,但索末菲尔德立即将贝特的奖学金恢复到慕尼黑。索末菲尔德花了1933年夏季学期的大部分时间为犹太学生和同事找到新的职位。[49]

贝特于1933年离开德国,之后通过索末菲尔德与威廉·劳伦斯·布拉格的关系,接受了曼彻斯特大学为期一年的讲师职位。[49] 他搬进了他的朋友鲁道夫·佩尔斯和佩尔斯的妻子杰尼娅家。佩尔斯是另一位德国物理学家,也因犹太身份被禁止在德国担任学术职务。这意味着贝特有一个可以用德语交流的人,而且他也不用吃英国的食物。[50] 他们的关系既是职业上的也是个人上的。佩尔斯激起了贝特对核物理的兴趣。[51] 在詹姆斯·查德威克和莫里斯·戈德哈伯发现氘的光解离现象后,[52] 查德威克挑战贝特和佩尔斯为这一现象提供理论解释。他们在从剑桥到曼彻斯特的四小时火车旅程中解决了这个问题。[53] 贝特在接下来的几年里将进一步探讨这一问题。[51]

在1933年,纽约康奈尔大学的物理系正在寻找一位新的理论物理学家,劳埃德·史密斯强烈推荐了贝特。布拉格也支持了这一推荐,当时他正在康奈尔大学访问。1934年8月,康奈尔大学为贝特提供了一个代理助理教授的职位。贝特已经接受了布里斯托大学的一个奖学金,计划在1935年春季与内维尔·莫特合作一个学期,但康奈尔同意让他在1935年春季开始工作。在前往美国之前,贝特于1934年9月访问了哥本哈根的尼尔斯·玻尔研究所,并向希尔达·莱维求婚,后者答应了。然而,贝特的母亲反对这段婚姻,尽管她自己有犹太背景,却不希望他娶一个犹太女子。就在他们定于12月结婚的前几天,贝特取消了他们的订婚。尼尔斯·玻尔和詹姆斯·弗兰克对贝特这一举动感到震惊,以至于直到第二次世界大战后,贝特才再次被邀请到玻尔研究所。
\subsection{美国}  
贝特于1935年2月抵达美国,并以3000美元的年薪加入了康奈尔大学的教职。贝特的任命是康奈尔物理系新任系主任罗斯威尔·克利夫顿·吉布斯刻意推动核物理学发展的努力的一部分。吉布斯聘请了曾与欧内斯特·劳伦斯合作的斯坦利·利文斯顿来康奈尔建造回旋加速器。为了完成团队的组建,康奈尔需要一位实验物理学家,经过贝特和利文斯顿的建议,聘请了罗伯特·巴赫尔。贝特收到了来自多所大学的邀请,其中包括以西多尔·艾萨克·拉比邀请他访问哥伦比亚大学,爱德华·康东邀请他访问普林斯顿大学,李·杜布里奇邀请他访问罗切斯特大学,卡尔·拉克-霍罗维茨邀请他访问普渡大学,弗朗西斯·惠勒·卢米斯邀请他访问伊利诺伊大学厄本那—香槟分校,约翰·哈斯布鲁克·范·弗雷克邀请他访问哈佛大学。吉布斯为了防止贝特被其他大学挖走,于是将他于1936年任命为正式的助理教授,并保证不久后会晋升为教授。

与巴赫尔和利文斯顿一起,贝特发表了一系列三篇文章,[60][61][62] 这些文章总结了直到那时为止核物理学领域的大部分已知内容,这些文章后来被非正式地称为“贝特圣经”。它在许多年里一直是这一领域的标准著作。在这篇综述中,他也继续了其他人的工作,填补了旧文献中的空白。[63] 卢米斯曾向贝特提供了伊利诺伊大学厄本那—香槟分校的正教授职位,但康奈尔大学为此提供了相同的职位,并提供了6000美元的年薪。[64] 贝特写信给母亲:

“我现在是美国最重要的理论物理学家之一。这并不意味着我是最好的。维格纳肯定更好,奥本海默和泰勒可能也同样优秀。但我做得更多,讲得更多,这也算数。”[65]
\begin{figure}[ht]
\centering
\includegraphics[width=6cm]{./figures/f1e9ca8bd91015fb.png}
\caption{质子-质子链反应序列示意图} \label{fig_Hans_2}
\end{figure}
\begin{figure}[ht]
\centering
\includegraphics[width=6cm]{./figures/6bff20b366a868e8.png}
\caption{CNO-I循环概述 —— 氦核在左上角的步骤释放出来} \label{fig_Hans_3}
\end{figure}
1938年3月17日,贝特参加了卡内基学会和乔治·华盛顿大学举办的第四届年度华盛顿理论物理学会议。会议仅邀请了34名与会者,但包括了格雷戈里·布雷特、苏布拉马尼扬·钱德拉塞卡、乔治·伽莫夫、唐纳德·门泽尔、约翰·冯·诺伊曼、本特·斯特罗姆格伦、爱德华·泰勒和梅尔·图夫等人。贝特最初拒绝了邀请,因为会议的主题——恒星能量生成——对他不感兴趣,但泰勒说服他参加。会议上,斯特罗姆格伦详细介绍了已知的太阳温度、密度和化学成分,并挑战物理学家们提出解释。伽莫夫和卡尔·弗里德里希·冯·魏茨泽克在1937年的一篇论文中提出,太阳的能量是质子-质子链反应的结果:[66][67]
\[ p + p \rightarrow \ ^2_1D + e^+ + \nu_e ~\]
但这并未解释比氦更重的元素的观测结果。会议结束时,贝特与查尔斯·克里奇菲尔德合作,提出了一系列后续的核反应,解释了太阳如何发光:[68]
\[ ^2_1D + p \rightarrow ^3_2He + \gamma~\]
\[ ^3_2He + ^4_2He \rightarrow ^7_4Be + \gamma~\]
\[ ^7_4Be + e^- \rightarrow ^7_3Li + \nu_e~\]
\[ ^7_3Li + p \rightarrow 2 \ ^4_2He~\]
这并没有解释较重恒星中的过程,这一点并没有被忽视。当时,人们对质子-质子循环是否描述了太阳内部的过程持怀疑态度,但最近对太阳核心温度和亮度的测量显示,质子-质子循环确实解释了太阳的能量来源。[66] 当贝特回到康奈尔大学后,他研究了相关的核反应和反应截面,最终发现了碳-氮-氧循环(CNO循环):[69][70]
\[ ^{12}_6C + p \rightarrow ^{13}_7N + \gamma~\]
\[ ^{13}_7N \rightarrow ^{13}_6C + e^+ + \nu_e~\]
\[ ^{13}_6C + p \rightarrow ^{14}_7N + \gamma~\]
\[ ^{14}_7N + p \rightarrow ^{15}_8O + \gamma~\]
\[ ^{15}_8O \rightarrow ^{15}_7N + e^+ + \nu_e~\]
\[ ^{15}_7N + p \rightarrow ^{12}_6C + ^4_2He~\]
贝特与克里奇菲尔德合著的关于质子-质子循环的论文以及关于碳-氮-氧(CNO)循环的另一篇论文,已被提交至《物理评论》期刊以供发表。[71]

在水晶之夜后,贝特的母亲开始害怕留在德国。借助她的斯特拉斯堡籍背景,她得以在1939年6月通过法国配额,而非德国配额(已经满额),移民到美国。[72] 贝特的研究生罗伯特·马沙克注意到纽约科学院为关于太阳和恒星能量的最佳未出版论文提供了500美元的奖金。由于贝特需要250美元来支付母亲家具的释放费用,他决定撤回CNO循环的论文,并将其提交给纽约科学院。论文获得了奖金,贝特给了马沙克50美元的介绍费,并用250美元解除了母亲家具的扣押。该论文随后于3月在《物理评论》期刊上发表。这是对恒星理解的一次突破,贝特因此获得了1967年诺贝尔物理学奖。[73][71] 2002年,96岁的贝特给约翰·N·巴哈尔写了一封手写信,祝贺他利用太阳中微子观测证明CNO循环约占太阳能量的7\%;这些中微子观测始于雷蒙德·戴维斯 Jr.,他的实验基于巴哈尔的计算和鼓励,这封信最终促使戴维斯分享了2002年诺贝尔奖。[74]

贝特于1939年9月13日与保罗·埃瓦尔德的女儿罗斯·埃瓦尔德结婚,婚礼是简单的民事仪式。[75] 罗斯已移民到美国,并在杜克大学就读,他们在1937年贝特在那里讲学时相识。他们有两个孩子,亨利和莫妮卡。[76](亨利是一位合约桥专家,也是基蒂·蒙森·库珀的前夫。)[77]

贝特于1941年3月成为美国的自然化公民。[78] 1947年,他写信给索默费尔德时透露:“我在美国比在德国时更感到如鱼得水。仿佛我在德国只是因为一个错误才出生,直到28岁才真正来到我的故乡。”[79]
\subsection{曼哈顿计划}\begin{figure}[ht]
\centering
\includegraphics[width=6cm]{./figures/fb3b1a759ab01ec6.png}
\caption{} \label{fig_Hans_4}
\end{figure}

