% 范数

\pentry{矢量空间\upref{LSpace}}

一些矢量空间中, 我们可以给每个矢量都定义一个\textbf{范数}, 例如常见的 $N$ 维 “几何矢量\upref{GVec}” 空间的模长就是一种范数. 同一个矢量空间可以存在多种不同的范数. 如果一个矢量空间中定义了范数, 我们就把它称为\textbf{赋范空间(normed space)}\footnote{满足一定收敛条件的赋范空间也叫做 \textbf{巴拿赫空间(Banach space)}, 有限维的赋范空间必定是巴拿赫空间.%链接未完成
}. 范数通常用双竖线表示, 如 $\norm{v}$. 范数必须满足以下条件

\begin{enumerate}
\item $\norm{v} \geqslant 0$ (正定)
\item $\norm{v} = 0$ 当且仅当 $v = 0$
\item $\norm{\lambda v} = \norm{\lambda} v$
\item $\norm{v_1 + v_2} \leqslant \norm{v_1} + \norm{v_2}$ (三角不等式)

\end{enumerate}

下面是一些常见的范数定义.

\subsection{列矢量的 $n$-范数}
定义 $\mathbb R^N$ 或 $\mathbb C^N$ 空间(即 $N$ 维实数或复数列矢量空间) 的 \textbf{$n$-范数}为
\begin{equation}
\norm{\bvec x}_n = \qty(\sum_i \abs{x_i}^n)^{1/n}
\end{equation}
物理中常见的是 \textbf{2-范数}, 即
\begin{equation}
\norm{\bvec x}_2 = \sqrt{\abs{x_1}^2 + \abs{x_2}^2 + \dots}
\end{equation}

可以证明极限情况 $n \to \infty$ 时, 绝对值最大的 $x_i$ 对求和的贡献将远大于其他分量, 所以定义\textbf{无穷范数}为
\begin{equation}
\norm{\bvec x}_\infty = \max \qty{\abs{x_i}}
\end{equation}

\subsection{函数的范数}
多元函数 $f(x_1, \dots x_N)$ 的范数在物理中常定义为
\begin{equation}
\norm{f} = \int \abs{f(x_1, \dots, x_N)}^2 \dd{x_1}\dots \dd{x_N}
\end{equation}
另一种简单定义是使用函数的最大值
\begin{equation}
\norm{f} = \max\qty{\abs{f(x_1, \dots, x_N)}}
\end{equation}
