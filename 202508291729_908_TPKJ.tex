% 拓扑空间(综述)
% license CCBYSA3
% type Wiki

本文根据 CC-BY-SA 协议转载翻译自维基百科\href{https://en.wikipedia.org/wiki/Topological_space}{相关文章}

在数学中,拓扑空间可以粗略地理解为一种定义了“接近性”但不一定能用数值距离来度量的几何空间。更具体地说,拓扑空间是一个集合,其元素称为点,并配备一种称为拓扑的附加结构。这个拓扑可以定义为:每个点都对应一组满足特定公理的邻域集合,这些公理用来形式化“接近性”的概念。拓扑有多种等价的定义方式,其中最常用的是通过开集来定义。

拓扑空间是最一般的数学空间形式,能够支持极限、连续性以及连通性等概念的定义\(^\text{[1][2]}\)。常见的拓扑空间类型包括欧几里得空间、度量空间以及流形。

尽管拓扑空间的概念非常抽象和宽泛,但它是数学中的一个基础性概念,几乎存在于现代数学的各个分支中。专门研究拓扑空间本身的学科被称为一般拓扑学,也叫点集拓扑学。
\subsection{历史}
大约在1735 年,莱昂哈德·欧拉发现了公式:$V - E + F = 2$其中 $V$ 表示**顶点数**,$E$ 表示**边数**,$F$ 表示**面数**,该公式刻画了凸多面体(以及平面图)的数量关系。这一公式的研究与推广,尤其是由 **柯西(Augustin-Louis Cauchy, 1789–1857)** 和 **吕利耶(Simon Antoine Jean L'Huilier, 1750–1840)** 的深入研究,极大地推动了拓扑学的发展。

在1827 年,卡尔·弗里德里希·高斯发表了《曲面的一般研究》。在第三节中,他以类似现代拓扑的方式定义了曲面:

“如果从曲面上一点 A 向与 A 无穷接近的曲面点引出的所有直线的方向,相对于通过 A 的同一平面发生无穷小的偏转,那么该曲面在 A 点就被称为具有连续曲率。”

然而,“直到黎曼在19 世纪 50 年代初的工作之前,曲面一直是从局部角度(作为参数曲面)来研究的,拓扑问题从未被考虑过。”莫比乌斯和乔尔当似乎是最早意识到拓扑学的核心问题是寻找不变量(最好是数值不变量),以此来判断两个曲面是否等价,即判断它们是否同胚。

费利克斯·克莱因在其“埃尔兰根纲领”(Erlangen Program, 1872中明确提出:研究任意连续变换下的不变量是一种新的几何学,即拓扑学的雏形。“拓扑学”一词由约翰·本尼迪克特·李斯廷于 **1847 年**首次提出,尽管他在几年前的通信中已经使用过该术语来代替当时常用的 **“位置分析(Analysis situs)”**。
**昂利·庞加莱(Henri Poincaré)** 将这一学科扩展到**任意维度的空间**,并奠定了现代拓扑学的基础。他关于该主题的第一篇论文发表于 **1894 年**。

在 **1930 年代**,**詹姆斯·沃德尔·亚历山大二世(James Waddell Alexander II)** 和 **哈斯勒·惠特尼(Hassler Whitney)** 首次提出了这样的观点:**曲面是一种局部类似欧几里得平面的拓扑空间**。

**拓扑空间(topological space)** 的现代定义最早由 **费利克斯·豪斯多夫(Felix Hausdorff)** 于 **1914 年**在其奠基性著作 *《集合论的原理》(Principles of Set Theory)* 中提出。
**度量空间(metric space)** 则更早在 **1906 年**由 **莫里斯·弗雷歇(Maurice Fréchet)** 定义,但“度量空间”(德语:*metrischer Raum*)这一术语的普及归功于豪斯多夫【6】【7】。
