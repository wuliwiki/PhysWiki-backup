% 厦门大学 2012 年 考研 量子力学
% license Usr
% type Note

\textbf{声明}:“该内容来源于网络公开资料,不保证真实性,如有侵权请联系管理员”

\subsection{一 、}

(1)波函数归一化条件的物理意义是什么?物理上对波函数有哪些要求?

(2)什么是幺正算符?若$ A,B,C$ 为幺正算符则它们的积 $ABC$ 是不是幺正算符?为什么?

(3)光的辐射分成几种过程?若粒子由能级 $E2$ 跃迁到能级 $E1$,写出辐射光子的频率。

(4)对处于某量子态的电子,如沿 $Z$ 轴方向测量其自旋,总是得到$+\frac{h}{2}$
 的结果,那么沿 $X$ 轴方向测量其自旋会得到什么结果?

(5)若算符 $A$ 不显含时间且与体系的哈密顿量 $H$ 对易,即$[A,H]=0$,那么$A$是体系的守恒量吗?说出你的判断理论。
\subsection{二、}
设粒子处在下列一维无限深势阱中,
\[V(x) = \begin{cases} 0, & 0 \leq x \leq a \\\\\infty, & \text{其余区域}\end{cases}~\]

(1) 写出粒子波函数所满足的边界条件;

(2) 写出粒子的能级 $E_n$ 以及相应的波函数 $\Psi_n(x)$;

(3) 若粒子的初始波函数为 $\Psi(x,0) = A\sin^3 \left(\frac{\pi x}{a}\right)$,$0 \leq x \leq a$,式中 $A$ 为归一化常数。

[提示:$\sin 3\theta = 3 \sin \theta - 4 \sin^3 \theta$,不知道这个式子对不对]

①求出 $A$ 和 $t$ 时刻的波函数$\Psi(x,t)$;

②计算粒子的坐标平均值$x$.
\subsection{三、}
设算符 $A$ 和 $B$ 不对易,$[A,B]=C$,但 $A$ 和 $B$ 都与 $C$ 对易,即
$[A,C]=0,[B,C]=0 试证明:
(1)[A, B  ] =n CB  ,m 为整数;
(2) [A, e  ] =Ce  转子绕一固定点转动;
(3) e  =e  +e  +e 

