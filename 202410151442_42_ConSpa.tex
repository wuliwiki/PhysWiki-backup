% 共轭空间与代数共轭空间
% keys 共轭空间|代数共轭空间
% license Usr
% type Tutor

\pentry{泛函与线性泛函\nref{nod_Funal}}{nod_c37f}
就像线性泛函和线性空间上的线性函数是同样的概念,共轭空间和\enref{对偶空间}{DualSp}是也是同样的概念。区别仅仅在于在有限维线性空间中,人们较多的使用“线性函数”和“对偶空间”的术语,而在一般的(包括无限维)线性空间中较多的使用 “泛函”和“共轭空间”的术语。

\begin{lemma}
设 $L$ 是\enref{线性空间}{LSpace},则其上的线性泛函的全体在下面的加法和数乘之下构成一个线性空间:
\begin{equation}
\begin{aligned}
ji
\end{aligned}
\end{equation}
\end{lemma}

\begin{definition}{}

\end{definition}
