% 克拉默法则
% 克拉默|线性方程组|线性代数|行列式|代数余子式

\begin{issues}
\issueTODO
\end{issues}

\pentry{行列式\upref{Deter}}

\footnote{参考 Wikipedia \href{https://en.wikipedia.org/wiki/Cramer's_rule}{相关页面}.}\textbf{克拉默法则(Kramer's rule)}是一种直接用行列式解线性方程组的方法. 把线性方程组记为矩阵乘法\upref{Mat}的形式
\begin{equation}\label{kramer_eq1}
\mat A \bvec x = \bvec b
\end{equation}
其中 $\mat A$ 为系数矩阵. 当 $\mat A$ 为 $N\times N$ 的方阵且行列式 $\abs{\mat A} \ne 0$ 时(即满秩矩阵\upref{MatRnk}), 方程有唯一解(见 “线性方程组解的结构\upref{LinEq}”). 该解可以用克拉默法则直接写出:
\begin{equation}\label{kramer_eq2}
x_i = \frac{\abs{\mat A_i}}{\abs{\mat A}} \qquad (i = 1, \dots, N)
\end{equation}
其中 $\mat A_i$ 是把 $\mat A$ 的第 $i$ 列替换为 $\bvec b$ 而来.

\begin{example}{解方程组}
令\autoref{kramer_eq1} 中 $\mat A = \pmat{2 & 1\\ -1 & 3}$, $\bvec b = \pmat{4\\5}$, 求解方程组.

解: $\abs{\mat A} = 7$, $\abs{\mat A_1} = \vmat{4 & 1\\ 5 & 3} = 7$, $\abs{\mat A_2} = \vmat{2 & 4\\ -1 & 5} = 14$. 代入\autoref{kramer_eq2} 得 $\bvec x = \pmat{1\\2}$.
\end{example}

在数值计算时, 克拉默法则解方程组效率较低, 直接用高斯消元法求逆矩阵高斯消元法求逆矩阵\upref{InvMGs}会更快.

\subsection{证明}
证明需要代数余子式 $\mat A^*$(未完成)
\begin{equation}
\bvec A^{-1} = \frac{\mat A^*}{\abs{\mat A}}
\end{equation}
\begin{equation}
\bvec x = \frac{\mat A^* \bvec b}{\abs{\mat A}}
\end{equation}
