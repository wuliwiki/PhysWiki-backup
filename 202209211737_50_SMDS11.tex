% 上海海事大学 2011 年数据结构
% keys 上海海事大学 2011 年数据结构

\subsection{一.判断题(本题20分,每小题2分)}
1.为了很方便地插入和删除数据,可以使用双向链表存放数据.

2.两个栈共享一片连续内存空间时,为了提高内存利用率,减少溢出机会,应把两个栈的栈底分别设在这片内存空间的两端.

3.数组是同类型值的集合.

4.在查找树(二叉排序树)中插入一个新结点,总是插入到叶子结点的下面.

5.用邻接矩阵存储一个图时,在不考虑压缩存储的情况下,所占用的存储空间大小与图中顶点的个数有关,而与图的边数无关.

6.顺序存储方式只能用于存储线性结构,不能用于存储二叉树.

7.在执行某个排序算法过程中,出现了排序码朝着最终排序序列位置相反方向移动,则该算法是不稳定的.

8.数据的逻辑结构被分为集合结构、线性结构、树型结构、图结构四种,

9.将一棵树转换成二叉树后,根结点没有左子树.

10.哈夫曼树是带权路径长度最短的树,路径上权值较大的结点离根较近.


\subsection{二.填空题(本题30分,每空2分)}
1.分析下列程序段,其时间复杂度分别为:( (1) ),( (2) ).
\begin{lstlisting}[language=cpp]
i=1;
while(i<=n)
    i=i*3;

void test(int m) {
    int i=0, s=0;
    while (s<n) {
        i++;
        s=s+i;
    }
}
\end{lstlisting}

2.堆栈的插入和删除操作都是在栈顶位置进行,而队列的( (3) )操作在队尾进行,( (4) )操作在队头进行.

3.对具有n个结点的二叉树采用二叉链表存储结构,则该链表中有( (5) )个指针域,其中有( (6) )个指针域用于链接孩子结点,( (7) )个 指针域空闲存放着NULL.

4.对线性表采用折半查找方法,该线性表必须采用( (8) )存储结构, 并且数据元素按值( (9) ).

5.除了顺序存储结构与链式存储结构之外,数据的存储结构通常还有( (10) )结构和( (11) )结构.

6.已知具有4行6列的矩阵A采用行序为主序方式存储,每个元素占用4个存储单元,并且a[3][4]的存储地址为1234,元素a[1][1]的存储地址是( (12) ).

7.对于长度为n的线性表,采用顺序存储结构存储,插入或删除一个元素的时间复杂度为( (13) ).

8.若对线性表进行的操作主要不是插入和删除,则该线性表宜采用( (14) )存储结构,若频繁地对线性表进行插入和删除操作,则该线性表宜采用( (09) )存储结构.

