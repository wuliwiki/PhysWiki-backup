% 人工神经网络(综述)
% license CCBYSA3
% type Wiki

本文根据 CC-BY-SA 协议转载翻译自维基百科\href{https://en.wikipedia.org/wiki/Neural_network_(machine_learning)}{相关文章}。

\begin{figure}[ht]
\centering
\includegraphics[width=6cm]{./figures/9fddb6b4ea5a9fd3.png}
\caption{人工神经网络是一个互相连接的节点群,灵感来源于大脑中神经元的简化模型。在这里,每个圆形节点代表一个人工神经元,箭头代表从一个人工神经元的输出到另一个人工神经元输入的连接。} \label{fig_RGSJ_1}
\end{figure}
在机器学习中,神经网络(也称为人工神经网络或神经网,缩写为ANN或NN)是一种受到动物大脑生物神经网络结构和功能启发的模型。[1][2]

一个人工神经网络由连接的单元或节点组成,这些节点被称为人工神经元,粗略地模拟大脑中的神经元。这些神经元通过边(连接)相互连接,模拟大脑中的突触。每个人工神经元接收来自连接神经元的信号,然后处理这些信号并将结果传送给其他连接的神经元。这个“信号”是一个实数,每个神经元的输出通过某个非线性函数计算,该函数作用于输入的和,这个过程被称为激活函数。每个连接的信号强度由一个权重决定,权重会在学习过程中调整。

通常,神经元会被聚合成层。不同的层可能对其输入执行不同的变换。信号从第一层(输入层)传输到最后一层(输出层),可能会经过多个中间层(隐藏层)。如果一个网络至少有两个隐藏层,通常称其为深度神经网络。[3]

人工神经网络被用于各种任务,包括预测建模、自适应控制和解决人工智能中的问题。它们能够从经验中学习,并能够从复杂且看似不相关的信息集中推导结论。
\subsection{训练}
神经网络通常通过经验风险最小化进行训练。这种方法基于优化网络参数,以最小化预测输出与给定数据集中的实际目标值之间的差异或经验风险的理念。[4] 基于梯度的方法,如反向传播,通常用于估计网络的参数。[4] 在训练阶段,人工神经网络通过标记的训练数据进行学习,通过迭代更新参数以最小化定义的损失函数。[5] 这种方法使得网络能够对未见过的数据进行泛化。