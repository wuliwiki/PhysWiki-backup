% 比较定理
% 比较定理
\pentry{微分方程}
比较定理可以形象的描述为:若甲乙两人在一直线上跑步,在直线上每一点,甲的速度都比乙的速度慢, 且在某一时刻 $t_0$,甲乙两人相遇,那么在相遇前,甲始终跑在乙前面,而在相遇后,甲则始终跑在乙的后面。

这一定理几乎是显然的,但是应该注意在某一时刻,甲的速度可能比乙快。
\begin{theorem}{比较定理}
若 $v_1,v_2$ 是实轴区间 $U$ 上定义的两实连续函数,且 $v_1<v_2$。又设 $\varphi_1,\varphi_2$ 分别是微分方程
\begin{equation}
\dv{\varphi_1}{t}|_{t=\tau}=v_1(\varphi_1(\tau)), \quad \dv{\varphi_2}{t}|_{t=\tau}=v_2(\varphi_2(\tau))
\end{equation}
的解。其中 $\varphi_1,\varphi_2$ 是将区间 $(a,b)\;(-\infty\leq a<b\leq+\infty)$ 映射到 $U$ 上的函数。那么
\begin{equation}\label{CompT_eq1}
\left\{\begin{aligned}
    &\varphi_1(t)\geq\varphi_2(t),\quad t\leq t_0\\
    &\varphi_1(t)\leq\varphi_2(t),\quad t\geq t_0~.
\end{aligned}\right.
\end{equation}

\end{theorem} 
\textbf{证明:}

我们来证\autoref{CompT_eq1} 中的第一式,而第二式只需用同样的方法证明即可。设 $T$ 是对于 $\tau<t\leq t_0$ 中的一切 $t$ 使
\begin{equation}
\varphi_1(t)\geq\varphi_2(t),\quad t\leq t_0
\end{equation}
成立的数 $\tau$ 的集合的下确界(上确界与下确界\upref{SupInf})。由假设知 $a\leq T\leq t_0$。
由连续性知 $\varphi_1(T)=\varphi_2(T)$,且由条件
\begin{equation}
\dv{\varphi_1}{t}|_{t=T}<\dv{\varphi_2}{t}|_{t=T},
\end{equation}
因此对充分接近 $T$ 的一切 $t<T$ 的点,有 $\varphi_1>\varphi_2$。故若 $a<T$ ,则 $T$ 不能是下确界。这一矛盾表明,$T=a$。

\textbf{证毕!}
