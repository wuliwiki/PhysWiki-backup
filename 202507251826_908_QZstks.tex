% 乔治·斯托克斯(综述)
% license CCBYSA3
% type Wiki

本文根据 CC-BY-SA 协议转载翻译自维基百科\href{https://en.wikipedia.org/wiki/Sir_George_Stokes,_1st_Baronet}{相关文章}。


乔治·加布里埃尔·斯托克斯爵士,第一代从男爵(/stoʊks/;1819年8月13日-1903年2月1日),是爱尔兰数学家和物理学家。斯托克斯出生于爱尔兰斯莱戈郡,在剑桥大学度过了整个职业生涯,并在1849年至1903年去世期间担任卢卡斯数学教授长达54年,是该职位任期最长的持有者。

作为物理学家,斯托克斯在流体力学领域作出了开创性的贡献,包括纳维-斯托克斯方程;在物理光学方面,他的研究涵盖偏振和荧光等现象。作为数学家,他普及了矢量微积分中的“斯托克斯定理”,并对渐近展开理论作出了贡献。斯托克斯与菲利克斯·霍普-塞勒一道,首次揭示了血红蛋白的携氧功能,并展示了血红蛋白溶液通气后所产生的颜色变化。

1889年,斯托克斯被英国君主封为从男爵。1893年,他因“在物理科学领域的研究与发现”获得当时全球最负盛名的科学奖项——皇家学会的科普利奖章。他曾于1887年至1892年在英国下议院担任剑桥大学选区的议员,隶属保守党。斯托克斯还于1885年至1890年担任皇家学会会长,并曾短暂出任剑桥大学彭布罗克学院院长。由于他的大量通信往来以及担任皇家学会秘书期间的工作,他被称为维多利亚时代科学的大门守卫者,其贡献远远超越了他发表的论文本身\(^\text{[1]}\)。
\subsection{传记}
乔治·斯托克斯是加布里埃尔·斯托克斯牧师(1834年去世)的幼子。加布里埃尔是爱尔兰圣公会的牧师,担任斯莱戈郡斯克林的教区牧师;其母为伊丽莎白·霍顿,是约翰·霍顿牧师的女儿。斯托克斯的家庭生活深受父亲福音派新教信仰的影响,他的三个兄弟也都进入教会,其中最杰出的是阿马郡副主教约翰·惠特利·斯托克斯\(^\text{[2]}\)。斯托克斯自幼对新教信仰虔诚,他在斯克林的童年经历也对他后来的研究方向产生了重大影响,尤其是他选择流体力学作为研究领域\(^\text{[3]}\)。他的女儿伊莎贝拉·汉弗莱斯曾写道,她父亲“告诉我,他小时候在斯莱戈海岸游泳时差点被一股大浪卷走,这第一次引起了他对波浪的兴趣”\(^\text{[4]}\)。

约翰与乔治兄弟情深,乔治在都柏林上学期间就住在约翰家中。在所有家庭成员中,他与妹妹伊丽莎白关系最为亲近。家族中回忆其母亲是“美丽但非常严厉”的。斯托克斯在斯克林、都柏林和布里斯托尔的学校接受教育,1837年,他进入剑桥大学彭布罗克学院。四年后,他以“高级数学状元”和史密斯一等奖毕业,凭此成就他被选为该学院研究员\(^\text{[5]}\)。

根据当时学院章程,斯托克斯于1857年结婚时必须辞去研究员职务。十二年后,在新章程下,他重新当选为研究员,并一直担任此职直到1902年。在他83岁生日的前一天,他被选为彭布罗克学院院长。然而他担任院长时间不长,于次年(1903年)2月1日在剑桥去世,并被安葬于米尔路公墓\(^\text{[6]}\)。他在西敏寺北侧走廊也有一块纪念碑\(^\text{[7]}\)。
\subsubsection{职业生涯}
1849年,乔治·斯托克斯被任命为剑桥大学卢卡斯数学教授(Lucasian Professor of Mathematics),他一直担任这一职位直到1903年去世。1899年6月1日,剑桥大学为庆祝斯托克斯就任卢卡斯教授五十周年举行了纪念典礼,众多来自欧洲和美国的大学代表参加了此次活动。剑桥大学校长向斯托克斯颁发了纪念金质奖章;哈莫·索尼克罗夫特(Hamo Thornycroft)为斯托克斯雕刻的大理石半身像,被开尔文勋爵正式赠送给彭布罗克学院和剑桥大学收藏。斯托克斯担任卢卡斯数学教授的时间长达54年,为历史上最长。

斯托克斯于1889年获封从男爵(Baronet),并在1887年至1892年期间代表剑桥大学选区担任英国议会议员,为母校继续服务。此外,他还在1885年至1890年间担任英国皇家学会会长,在此之前自1854年起便担任该学会秘书。由于他同时还是卢卡斯数学教授,斯托克斯也成为第一个同时担任这三个职位的人;牛顿虽然也曾担任过这三个职务,但并非同时\(^\text{[6]}\)。

斯托克斯是与詹姆斯·克拉克·麦克斯韦和开尔文勋爵齐名的三位自然哲学家中最年长的一位,他们三人是剑桥数学物理学派在19世纪中叶声名显赫的主要奠基人。

斯托克斯的原创性研究始于1840年左右,其成果在数量和质量上都堪称卓越。根据英国皇家学会的科学论文目录,他在1883年之前已发表逾百篇论文。这些作品中有些只是简短的笔记,有些是简要的争论性或纠正性陈述,但也有大量是长篇且详尽的论文\(^\text{[8]}\)。
\subsection{对科学的贡献}
\begin{figure}[ht]
\centering
\includegraphics[width=6cm]{./figures/a9c1766ec957cae4.png}
\caption{} \label{fig_QZstks_2}
\end{figure}
在研究范围上,斯托克斯的工作涵盖了广泛的物理问题,但正如玛丽·阿尔弗雷德·科尔努在1899年的瑞德讲座中指出的那样,\(^\text{[9]}\)他的大部分研究都集中在波动及其在穿越不同介质时所经历的变化上。\(^\text{[10]}\)
\subsubsection{流体动力学}
斯托克斯最早发表的论文出现在1842年和1843年,内容是关于不可压缩流体的稳态运动以及某些流体运动情形的分析。\(^\text{[11][12]}\)随后,他于1845年发表了一篇关于流体在运动中所产生的摩擦力以及弹性固体的平衡与运动的论文,\(^\text{[13]}\)1850年又发表了一篇探讨流体内摩擦对钟摆运动影响的文章。\(^\text{[14]}\)他还对声学理论作出了一些贡献,包括讨论风对声音强度的影响,\(^\text{[15]}\)以及气体性质如何影响声波强度的解释。\(^\text{[16]}\)这些研究共同为流体动力学奠定了新的基础,不仅解释了许多自然现象(如云悬浮于空中、水面涟漪与波浪的消退),也为诸多实际问题的解决提供了关键理论支持,例如河流与渠道中水流的行为、以及船体的表面阻力问题。\(^\text{[10]}\)

\textbf{缓慢流动}
\begin{figure}[ht]
\centering
\includegraphics[width=6cm]{./figures/747120e82639f11a.png}
\caption{} \label{fig_QZstks_1}
\end{figure}
斯托克斯在流体运动和黏性方面的研究使他计算出了球体在黏性介质中下落时的终端速度,这一结果被称为斯托克斯定律。他推导出了一个表达式,用于计算在雷诺数非常小的情况下,作用在球体上的摩擦力(也称阻力)。

他的工作成为落球式粘度计的基础:该装置中,液体静置于一根垂直的玻璃管中,一个已知大小和密度的球体被释放,让其在液体中下落。若参数选择得当,球体会达到终端速度,并可以通过它通过管上两个标记的时间来测量这个速度。在处理不透明流体时,可以使用电子感应装置。已知终端速度、球体的尺寸和密度以及液体的密度,斯托克斯定律就可以用来计算液体的粘度。经典实验中通常使用不同直径的钢珠以提高计算精度。学校实验常用甘油作为流体;在工业上,这一技术被用于检测工艺流程中使用流体的粘度。

同样的理论还解释了为什么小的水滴(或冰晶)可以在空气中悬浮形成云,直到它们长大到一定尺寸后才会作为雨(或雪、冰雹)落下。类似地,该公式也可以用于解释细小颗粒在水或其他液体中的沉降过程。

为表彰斯托克斯的工作,CGS单位制中运动粘度的单位“斯托克斯”就是以他命名的。
\subsubsection{光学研究}
他最著名的研究之一或许是关于光的波动理论的工作。他的光学研究在其科研生涯的早期就已开始。他在1845年和1846年发表了关于光的像差的首批论文\(^\text{[18][19]}\),并在1848年发表了一篇关于光谱中某些条纹的理论的论文\(^\text{[20][10]}\)。

1849年,他发表了一篇关于衍射的动力学理论的长文,在其中他证明了偏振光的振动平面必须垂直于传播方向\(^\text{[21]}\)。两年后,他又讨论了厚玻璃片产生的颜色\(^\text{[22][10]}\)。

斯托克斯还研究了乔治·艾里对彩虹的数学描述\(^\text{[23]}\)。艾里的研究涉及一个难以求解的积分,而斯托克斯将该积分表达为一个发散级数(在当时还鲜有了解)。但他通过巧妙地截断级数(只保留前几项),得到了一个对积分值非常精确的近似表达,比直接求积分更容易\(^\text{[24]}\)。他对渐近级数的研究也为这一领域带来了根本性的洞见\(^\text{[25]}\)。
\subsubsection{荧光}
\begin{figure}[ht]
\centering
\includegraphics[width=6cm]{./figures/1e069d1a77846397.png}
\caption{萤石} \label{fig_QZstks_3}
\end{figure}
1852年,斯托克斯在他著名的关于光波长变化的论文中,描述了荧光现象,这种现象可以在萤石和铀玻璃等材料中观察到。他认为这些材料具有将不可见的紫外辐射转化为较长波长的可见辐射的能力。为纪念他在这一方面的贡献,这一转化现象被称为斯托克斯位移。他还展示了一个机械模型,用以说明斯托克斯所作解释的动力学原理。由此衍生出的“斯托克斯线”概念,后来成为拉曼散射理论的基础。

1883年,开尔文勋爵在皇家研究院的一次讲座中提到,他多年前曾听斯托克斯讲解过这一原理,并多次恳求他发表这一研究成果,但都未能如愿。
\subsubsection{偏振}
\begin{figure}[ht]
\centering
\includegraphics[width=6cm]{./figures/e78fa4800cb14d6e.png}
\caption{一块方解石晶体放在印有字母的纸上,显示出双折射现象。} \label{fig_QZstks_4}
\end{figure}
1852年同年,斯托克斯发表了题为《关于不同光源所发偏振光的叠加与分解》的论文\(^\text{[28]}\),次年(1853年)他又研究了某些非金属物质呈现的金属反射现象\(^\text{[29]}\),这些研究凸显了光的偏振现象的重要性。大约在1860年,他致力于研究由一叠平板反射或透射的光的强度\(^\text{[30]}\);1862年,他为英国科学促进会撰写了一份极具价值的报告,内容涉及双折射现象\(^\text{[10]}\)——某些晶体沿不同轴线表现出不同的折射率。最著名的例子可能是冰洲石,即透明的方解石晶体\(^\text{[31]}\)。

同年,他还发表了一篇关于电光长波谱的论文\(^\text{[32]}\),并紧接着展开了对血液吸收光谱的研究\(^\text{[10][33]}\)。
\subsubsection{化学分析}
1864年,他探讨了通过光学性质来识别有机化合物的方法\(^\text{[34]}\);之后,他与威廉·弗农·哈考特合作研究了玻璃的化学成分与光学性质之间的关系,其目标是提高玻璃的透明度以及改进消色差望远镜的性能\(^\text{[35]}\)。在之后的论文中,他还探讨了与光学仪器构造有关的问题,尤其是显微镜物镜孔径的理论极限\(^\text{[36][10]}\)。
\subsubsection{眼科学}
1849年,斯托克斯发明了用于检测散光的“斯托克斯透镜”\(^\text{[37]}\)。这是一种由两个等功率但方向相反的柱面透镜组成的透镜组合,其结构允许两个透镜相对旋转,从而检测眼睛的屈光不正问题\(^\text{[38]}\)。
\subsubsection{其他工作}
\begin{figure}[ht]
\centering
\includegraphics[width=6cm]{./figures/4a1d6d6707b10966.png}
\caption{克鲁克斯辐射计} \label{fig_QZstks_5}
\end{figure}
在物理学的其他领域中,值得一提的有他关于晶体中热传导的论文(1851年)\(^\text{[39]}\),以及他与克鲁克斯辐射计相关的研究\(^\text{[40]}\);他对常出现在照片中、在暗色物体轮廓外、靠近天空背景处观察到的亮边现象的解释(1882年)\(^\text{[41]}\);还有他对X射线的理论,其提出X射线可能是以无数个孤立波而非规则波列形式传播的横波(较晚时期的工作)\(^\text{[42]}\)。

1849年他发表了两篇长篇论文——一篇关于引力与克莱罗定理\(^\text{[43]}\),另一篇关于地球表面重力变化(即“斯托克斯重力公式”)\(^\text{[44]}\)——这两篇都值得注意。除此之外,他在周期级数和的临界值方面的数学论文(1847年)\(^\text{[45]}\)、关于某类定积分与无穷级数数值计算的研究(1850年)\(^\text{[46]}\),以及他对涉及铁路桥断裂问题的微分方程的讨论(1849年)\(^\text{[47][10]}\),也都具有重要意义。这些研究与他在1847年迪河桥灾难之后,向“铁路结构中铁材使用皇家委员会”提供的证词密切相关。
\subsubsection{未发表的研究}
斯托克斯的许多发现并未发表,或者仅在他口头讲授的课程中稍作提及。其中一个例子是他在光谱学理论方面的研究\(^\text{[10]}\)。
\begin{figure}[ht]
\centering
\includegraphics[width=6cm]{./figures/6435af8071eefcbe.png}
\caption{开尔文勋爵} \label{fig_QZstks_6}
\end{figure}
1871年,开尔文勋爵在英国科学促进会年会的主席致辞中表示,他相信在斯托克斯于1852年夏季之前某个时间在剑桥大学教他光的棱镜分析应用于太阳和恒星化学之前,从未有任何人直接或间接提出过这一想法。他在致辞中阐述了自己当时从斯托克斯那里学到的理论与实践结论,并表示这些内容后来一直被他作为格拉斯哥大学公开讲座的常规授课内容\(^\text{[48]}\)。
\begin{figure}[ht]
\centering
\includegraphics[width=6cm]{./figures/a394fad846f5d979.png}
\caption{基尔霍夫} \label{fig_QZstks_7}
\end{figure}
这些陈述阐明了光谱学赖以建立的物理基础,以及它在识别太阳和恒星中存在物质方面的应用方式,因此看起来斯托克斯至少比古斯塔夫·基尔霍夫早七到八年提出了这一理论。然而,斯托克斯在这次演讲多年后发表的一封信中表示,他在论证中未能迈出一个关键步骤——他没有意识到,某一特定波长的光的发射不仅是允许的,而且必然要求吸收同一波长的光。他谦逊地否认自己“对基尔霍夫杰出发现有任何贡献”,并补充说他觉得有些朋友在支持自己时过于热忱\(^\text{[49]}\)。然而必须指出的是,英国的科学界并未完全接受他的这一否认,仍然将首次阐述光谱学基本原理的功劳归于斯托克斯\(^\text{[10]}\)。

此外,斯托克斯在数学物理的发展方面也做出了巨大贡献。在被选为卢卡斯教授不久后,他宣布将帮助剑桥大学任何在数学学习中遇到困难的成员视为自己的职责之一,他所提供的帮助极为实际,以至于即使一些人后来成为他的同事,也仍乐于就他们在数学与物理问题上遇到的难题向他请教。在担任皇家学会秘书的三十年中,他对数学和物理科学的发展产生了巨大而低调的影响,这不仅体现在他自身的研究中,也体现在他提出值得研究的问题、鼓励他人攻克这些问题、并始终乐于提供鼓励与支持上\(^\text{[10]}\)。

\textbf{对工程学的贡献}

斯托克斯曾参与多起铁路事故的调查,尤其是1847年5月在切斯特发生的迪河桥灾难。他还是随后成立的关于铁路结构中铸铁使用情况的皇家委员会成员。他参与了移动机车对桥梁施加的力的计算。这座桥之所以坍塌,是因为使用了一根铸铁梁来承载列车通过时的负载。铸铁在受拉或弯曲时较为脆弱,因此许多类似的桥梁不得不被拆除或加固。
\begin{figure}[ht]
\centering
\includegraphics[width=8cm]{./figures/25fb3846b432d895.png}
\caption{从北侧拍摄的倒塌后的泰桥} \label{fig_QZstks_8}
\end{figure}
他曾作为专家证人在泰桥惨案调查中出庭作证,讲述风载对桥梁结构的影响。1879年12月28日,一场风暴摧毁了该桥的中心段(被称为“高桁段”),当时一列特快列车正行驶其上,车上所有人全部遇难,死亡人数超过75人。调查委员会听取了多位专家证人的证词,最终得出结论:该桥“设计糟糕、施工粗劣、维护不到位”。\(^\text{[50]}\)

由于他在此次调查中的证词,他被任命为后续皇家委员会的成员,负责研究风压对结构物的影响。当时对强风对大型结构的影响尚未引起足够重视,该委员会在英国各地开展了一系列测量,以了解暴风中风速的实际情况及其对暴露表面的压力作用。

\textbf{宗教方面的工作}

\begin{figure}[ht]
\centering
\includegraphics[width=6cm]{./figures/4c78320d3e0dd86e.png}
\caption{斯克林,斯莱戈郡的爱尔兰圣公会教堂} \label{fig_QZstks_9}
\end{figure}
斯托克斯普遍持有保守的宗教价值观和信仰。1886年,他成为维多利亚研究院的主席,该机构成立的目的是为了捍卫福音派基督教原则,抵御来自新兴科学,尤其是达尔文生物进化论的挑战。他于1891年发表了吉福德自然神学讲座\(^\text{[51][52]}\)。他还担任英国与外国圣经协会的副会长,并积极参与有关传教工作的教义辩论\(^\text{[53]}\)。不过,尽管他的宗教观点在大多数方面是正统的,他在维多利亚时代的福音派中却显得与众不同——他拒绝相信地狱中的永恒惩罚,而是主张基督教条件主义\(^\text{[54]}\)。

作为维多利亚研究院主席,斯托克斯曾写道:“我们都承认,自然之书与启示之书同样出自上帝之手,因此,如果解释得当,两者之间不可能存在真正的冲突。科学与启示大多各行其道,几无冲突之虞。但若出现表面上的矛盾,我们原则上无权偏袒其一。不论我们多么坚信启示的真实性,我们都必须承认自己可能会在理解启示的范围或诠释上犯错;同样,不论某个科学理论的证据多么强大,我们也必须记住,这类证据本质上仅具有概率性,随着科学知识的拓展,我们完全可能改变原有看法”\(^\text{[55]}\)。
\subsection{个人生活}
斯托克斯于1857年7月4日,在阿马郡圣帕特里克大教堂与玛丽·苏珊娜·罗宾逊结婚。玛丽是爱尔兰天文学家托马斯·罗姆尼·罗宾逊牧师的独生女。他们育有五个孩子:阿瑟·罗姆尼,继承了从男爵爵位;苏珊娜·伊丽莎白,婴儿期夭折;伊莎贝拉·露西(Isabella Lucy,劳伦斯·汉弗莱夫人),为《已故乔治·加布里埃尔·斯托克斯从男爵的回忆与科学通信录》撰写了父亲的个人回忆录;威廉·乔治·加布里埃尔医生,一位内心痛苦的人,30岁时因短暂精神失常而自尽;以及多拉·苏珊娜,同样婴儿期夭折。他的男性后裔线断绝,因此斯托克斯的从男爵爵位已不再延续。

剑桥大学卢卡斯数学教授
他于1851年成为英国皇家学会会士,并于1852年因其对光波长的研究获得皇家学会的朗福德奖章,1893年又获得科普利奖章。
1869年,他主持了英国科学促进会在埃克塞特的会议。
1874年,他被选为美国艺术与科学学院国际荣誉会员【56】。
1883年,他被选为美国国家科学院国际会员【57】。
1883年至1885年间,他在阿伯丁大学担任伯内特讲座教授,他关于光的讲座于1884至1887年出版,讲授光的本质、作为研究工具的作用及其有益效应【10】。
1888年4月18日,他被授予伦敦城市自由人称号【58】。
1889年7月6日,维多利亚女王授予他“兰斯菲尔德小屋的乔治·加布里埃尔·斯托克斯爵士”头衔,封为英国从男爵,该爵位于1916年绝嗣而灭【59】。
同年,他被选为美国哲学会国际会员【60】。
1891年,作为吉福德讲座教授,他出版了《自然神学》一书。
他还被授予普鲁士“功绩勋章”。

他的学术荣誉包括来自多所大学的荣誉学位,其中包括:
1902年9月6日,在挪威皇家腓特烈大学(现奥斯陆大学)庆祝数学家尼尔斯·亨利克·阿贝尔诞辰百年之际,他被授予名誉数学博士学位【61】【62】。
为表彰他的贡献,以他命名的运动粘度单位“斯托克斯”(stokes)成为CGS单位制的一部分。
1909年,彭布罗克学院成立了“斯托克斯学会”,作为全校本科科学家们的学术交流平台,截至2023年仍活跃【63】。
2017年7月,都柏林城市大学以斯托克斯的名字命名了一座大楼,以表彰他在物理和数学方面的贡献【64】。
