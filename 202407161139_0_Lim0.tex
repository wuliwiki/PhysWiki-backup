% 数列的极限(极简微积分)
% keys 微积分|极限|数列极限|函数极限|无穷小
% license Xiao
% type Tutor

微积分的核心概念是\textbf{极限(limit)},常见的极限有数列的极限以及\enref{函数的极限}{FunLim}。数列是离散的,比较容易理解。事实上,数列就是一种特殊的函数(定义域为$N^{*}$)。

我们通过一些例子引入。

\begin{example}{圆周率}\label{ex_Lim0_1}
我们知道 $\pi$ 是一个无理数,所以 $\pi$ 的小数部分是无限多的。目前用计算机已经可以将 $\pi$ 精确地计算到小数点后数亿位。然而在实际应用中,往往只用取前几位小数的近似即可。下面给出一个数列,定义第 $n$ 项是 $\pi$ 的前 $n$ 位小数近似(去尾),即
\begin{equation}\label{eq_Lim0_1}
a_0 = 3,\,\, a_1 = 3.1,\,\, a_2 = 3.14,\,\, a_3 = 3.141,\,\dots~
\end{equation}

显而易见,当 $n$ \textbf{趋于无穷}时,$a_n$ \textbf{趋于} $\pi$。 \textbf{无穷(infinity)}用符号 $\infty$ 来表示。 我们说该数列的\textbf{极限值}是 $\pi$, 也可以简称\textbf{极限(limit)}。 以上这种情况,用极限符号表示,就是
\begin{equation}\label{eq_Lim0_3}
\lim_{n \to \infty } {a_n} = \pi ~.
\end{equation}
这里 $\lim$ 是极限的数学符号,下方用箭头表示某个量 “无止境” 变化的过程。 对于数列而言, 唯一的 “无止境” 变化就是项标 $n$ 不断增加。% 严格来说即使是数列也是可以定义“子序列极限”的,所以说只有一种极限不是很合理。
\end{example}

% \addTODO{请先申请成为作者}
% \begin{example}{收敛的有界无穷数列}\label{ex_Lim0_2}
% 对于数列这种东西,相信读者早在高中就对其司空见惯。在本例中,我们从另一个不同的角度来看待这种业已熟悉的数学对象。
% 考虑数列 $ a_n= \frac {1}{n} $($n \in N^{*}$)。由于正整数集集合元素个数无穷多,故这个数列的项数也无穷多。

% 也许读者已经接触过数列的单调性这个概念,例如,对于数列$ b_n= n$,自变量$n$越大,$b_n$就越大。要想$b_n$足够大(大于一个预先指定的正数,比如一千万),只需n足够大。

% 例如,要让$b_n$比一千万亿大,只需n大于一千万亿。要让$b_n$大于$10^{100}$,只需$n$大于$10^{100}$。从这个意义上讲,我们可以说数列$\begin{Bmatrix} b_n \end{Bmatrix}$是一个“无穷大量”。需要注意的是,这并不意味着$b_n = \infty $,事实上,$\infty$不是一个实数,不能被变量等于。无论$n$有多大,$a_n$都只能任意地大,只要$n$够大,$a_n$就可以超过任何正数,但不会等于$\infty$,因为$\infty$并不存在于实数系,它不是一个数。

% 相对应的,数列$ a_n= \frac {1}{n} $随$n$增大而不断减小。我们可以注意到,$n$本身越大,$a_n$随$n$增大减小得也越少。可以想象到,只要$n$足够大,$a_n$就可以任意地小,或者说与$0$任意地接近。要让$ a_n $小于$10^{-100}$,只需$n$大于$10^{100}$。

% 此时,我们便可以说,数列$\begin{Bmatrix} a_n \end{Bmatrix}$的“极限”是0,也即
% \begin{equation}\label{eq_Lim0_8}
% \lim_{n \to \infty } {a_n} = 0 ~.
% \end{equation}

% 数列值$  a_n  $不会小于$0$,但可以任意接近$0$,换言之,$0$是数列$\begin{Bmatrix} a_n \end{Bmatrix}$不可逾越的“极限”。相对应的,我们也可以称数列$\begin{Bmatrix} a_n \end{Bmatrix}$是一个“无穷小量”,但这并不意味着存在一个$m \in N^{*}$,使得$a_m = 0$,对于任意$n \in N^{*}$,$a_n $都不等于0,但它可以“无穷地小”,可以任意地接近$0$。
% \end{example}

$\lim\limits_{n \to \infty }$ 在这里相当于一个“操作”,叫做\textbf{算符(operator)}。它作用在整个数列上,并输出一个数, 也就是数列的极限值。 这有些类似于函数输入一个自变量并输出一个函数值, 只不过 $\lim$ 算符的自变量从数换成了数列。 所以不要误以为\autoref{eq_Lim0_3} 是说当 $n = \infty$ 时,有\footnote{有两个理由可以说明这种理解不正确:首先,按定义,每个 $a_n$ 都是有理数,而 $\pi$ 是无理数,所以不应该有任何一个 $a_n=\pi$;其次,$\infty$ 不是一个实数,不存在 $n=\infty$ 的说法。这里的 $n\to\infty$ 只是表示 $n$ 无限增加的过程。} $a_n=\pi$, 而要理解成数列 $a_n$ 经过算符 $\lim\limits_{n \to \infty }$ 的作用以后,得出的结果是 $\pi$。 类比函数 $\sin x = y$,是说 $x$ 经过正弦函数作用后等于 $y$。 所以从概念上来说,极限中的 “趋于” 和“等于” 是不同的。趋于是数列整体的性质,而不是某一个项性质。

我们可以总结出以上数列的一个性质, 并把它作为数列极限的一般定义。 具有极限的数列最显著的特征是, 随着 $n$ 增加, 后面的所有项都越来越接近极限值。 可是一个难点在于如何定义 “越来越接近”。 先看一个错误的理解: 考虑数列
\begin{equation}\label{eq_Lim0_2}
a_1 = 3.21,\ a_2 = 3.201,\ a_3 = 3.2001,\ a_4 = 3.20001, \dots~
\end{equation}
这个数列也同样越来越接近 $\pi$, 但直觉告诉我们, 它的极限是 $3.2$ 而不是 $\pi$。

既然要讨论有多接近, 那就要定义距离。 我们可以把第 $a_n$ 和极限值 $\pi$ 的\textbf{距离}用绝对值定义为 $\abs{a_n - \pi}$。 对\autoref{eq_Lim0_1} 的数列, 可以发现这个距离不光是\textbf{越来越小},而且\textbf{想要多小就有多小}。 例如要求 $\abs{a_n - \pi} < 10^{-2}$, 容易发现 $n > 1$ 时就总能满足; 又例如要求 $\abs{a_n - \pi} < 10^{-10}$, 容易发现 $n > 9$ 时就总能满足;一般地如果要求 $\abs{a_n - \pi} < 10^{-q}$ ($q$ 为整数), 只要 $n > q-1$ 就总能满足,这就意味着 $\lim\limits_{n \to \infty } a_n = \pi$。

有了这个定义, 我们就可以轻易地判断\autoref{eq_Lim0_2} 的极限不是 $\pi$。 因为无论 $n$ 为多大, 总是有 $\abs{a_n - \pi} > 3.2 - \pi > 0.05$,所以如果要求一个比这更小的距离, 那么就没有任何 $n$ 可以满足。

\begin{definition}{数列的极限}\label{def_Lim0_2}
考虑无穷项的实数数列 $a_1, a_2, \dots$, 若存在一个实数 $A$,使得:无论要求多么小的正数 $\epsilon$,总存在正整数 $N$,对任意 $n > N$ 都满足 $\abs{a_n - A} < \epsilon$, 则该数列的\textbf{极限}就是 $A$。
\end{definition}

\begin{figure}[ht]
\centering
\includegraphics[width=14cm]{./figures/7e37c5b14cfe1ef1.png}
\caption{数列的极限(\href{https://wuli.wiki/apps/Lim0.html}{查看动画}): 图中每点代表数列的一项, 数列的极限值为 $1$, 红色区域表示对距离 $\abs{a_n - A}$ 的要求, 满足要求的 $a_n$ 为红色, 不满足的为蓝色。 无论红色区域有多窄(只要不为零), 总能找到一个不等式 $n > N$ 使之后所有的点都是红色。} \label{fig_Lim0_1}
\end{figure}

我们来看几个简单的例题,加深一下印象。

\begin{example}{}
考虑数列 $a_n= {(-1)^n}/{2^n}$,根据定义可证明 $\lim\limits_{n\to\infty}a_n=0$。
\end{example}

一些数列不存在极限:
\begin{exercise}{}\label{exe_Lim0_10}
考虑数列 $a_n = n$ 以及 $a_n=(-1)^n$。 它们存在极限吗?
\end{exercise}

\begin{definition}{数列的敛散性}\label{def_Lim0_4}
如果一个数列不存在极限, 就称它是\textbf{发散(divergent)}的。 如果存在极限, 则称它是\textbf{收敛(convergent)}的。
\end{definition}

\begin{example}{}
容易发现数列的极限和前面有限项的值都无关, 例如把\autoref{eq_Lim0_1} 中的前 10 项都改成 $0$, 那么该数列的极限仍然是 $\pi$。 把前一万项改成 $0$ 也同理。
\end{example}

\begin{example}{}
根据定义, 数列极限也并不要求 $n\to \infty$ 时数列的项不能等于极限值, 例如数列
\begin{equation}
b_0 = 3.3~,\,\, b_1 = 3.2~, \,\, b_2 = \pi~, \,\, b_3 = \pi~, \,\, b_n = \pi \;\; (n \ge 2)~.
\end{equation}
当 $n \ge 2$ 时所有的项都等于 $\pi$, 那么根据\autoref{def_Lim0_2} 他的极限显然也是 $\pi$。 因为令 $n > 1$ 即可满足定义中对距离的任何限制。
\end{example}

\begin{example}{}
注意存在极限的数列未必要求距离 $\abs{a_n - A}$ 是严格递减的, 例如数列
\begin{equation}
\frac{1}{2},\;\; \frac{1}{4},\;\; \frac{1}{3},\;\; \frac{1}{5},\;\; \frac{1}{4},\;\; \frac{1}{6},\;\; \frac{1}{5},\;\; \frac{1}{7},\;\; \dots~
\end{equation}
的极限是 $0$。
\end{example}
