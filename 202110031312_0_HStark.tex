% 类氢原子斯塔克效应(微扰)

\begin{issues}
\issueDraft
\end{issues}

\pentry{不含时微扰理论\upref{TIPT}}

微扰理论:
\begin{equation}
H' = E z
\end{equation}
假设 $[H, H'] = 0$, 存在共同本征态, 那么
\begin{equation}\label{HStark_eq1}
H'_{l',l} = \mel{\psi_{n,l',m}}{z}{\psi_{n,l,m}}
\end{equation}


\begin{example}{氢原子 $n=2$ 的 Stark 效应}
根据\autoref{HDipM_tab1}~\upref{HDipM}, \autoref{HStark_eq1} 为
\begin{equation}
\mat H' = -3\pmat{0 & 1\\ 1 & 0}
\end{equation}
本征值为 $E_{\pm}^1 = \mp 3E$, 本征矢 $\ket{2\pm} = (\ket{20} \pm \ket{21})/{\sqrt 2}$.

\begin{figure}[ht]
\centering
\includegraphics[width=8cm]{./figures/HStark_1.png}
\caption{$\ket{2+}$ 的概率密度函数的 $x$-$z$ 切面, 可见电子向下偏移, 电场向上为正, 所以本征能量变小 $-3E$.} \label{HStark_fig1}
\end{figure}
\end{example}
