% Floyd 判圈算法
% keys Floyd|判环|判圈
% license Usr
% type Tutor

Floyd 判圈算法的思想类似于快慢指针。原理是“龟兔赛跑”,慢指针每次向前移动 $1$ 步、快指针每次向前移动两步。如果两者在遍历链表的\textbf{过程中}相遇,则说明链表存在一个圈;如果快指针达到了链表的结尾(有尾则一定无环),说明链表无环。

下面给出一个常见的实现方法:
\begin{lstlisting}[language=cpp]
bool hasCyc(ListNod* head) {
    if(head == nullptr)  return false;
    ListNod *slow = head, *fast = head;
    fast = fast -> next;

    while(fast != nullptr && fast -> next != nullptr) {
        if(slow == fast)  return true;
        slow = slow -> next;
        fast = fast -> next -> next;
    }
    
    return false;
}
\end{lstlisting}

在pan