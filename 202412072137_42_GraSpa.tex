% 图空间与赋权图
% keys 图空间|赋权图
% license Usr
% type Tutor

\pentry{图\nref{nod_Graph},向量空间\nref{nod_LSpace}}{nod_7029}
图空间是指定义在图的点集和边集上的函数全体,它们刚好满足\enref{向量空间}{LSpace}的定义。定义在点集上的函数全体叫做点空间,而定义在边集上的函数全体叫做边空间。赋权图则是在边空间中选择一个函数和图组成的二元组,所选的函数叫做权函数,权函数在边的值叫做改边的权。

\begin{definition}{点空间,边空间}
设 $D$ 是一个图\upref{Graph},其点集和边集为 $V(D)=\{v_1,\cdots,v_n\},E(D)=\{a_1,\cdots,e_m\}$。则定义在 $V(D)$ 上,值域为 $\mathbb R$ 的函数全体称为 $G$ 的\textbf{点空间}(vertex space),记作 $\mathcal V(G)$。而定义在 $E(D)$ 上,值域为 $\mathbb R$ 的函数全体称为 $G$ 的\textbf{边空间}(edge space),记作 $\mathcal E(G)$。
\end{definition}


\begin{exercise}{}
 $\mathcal V(G),\mathcal E(G)$ 上的加法和数乘定义如下:设
\begin{equation}
f,g\in\mathcal V(G),x\in V(D)\text{或} f,g\in\mathcal E(G),x\in E(D),\lambda\in \mathbb R,~
\end{equation}
那么
\begin{equation}
\begin{aligned}
&(f+g)(x):=f(x)+g(x),\\
&(\lambda f)(x):=\lambda f(x).
\end{aligned}~
\end{equation}
试证明:点空间 $\mathcal V(G)$ 和边空间 $\mathcal E(G)$ 满足\aref{向量空间}{def_LSpace_2}的定义。
\end{exercise}

注意到 $f\in\mathcal V(D)$ 可由在 $V(D)$ 上 $n$ 个点处的取值 $f(v_1),\cdots f_(v_n)$ 唯一确定,且每个 $f(v_i)=f(v_i)\cdots$ 








