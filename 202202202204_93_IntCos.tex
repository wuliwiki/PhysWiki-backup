% 干涉,光强的余弦平方分布
% 光的矢量叠加;远场光强

\addTODO{巴俾涅原理,线偏振光,辐照度,远场}
\pentry{电场波动方程\upref{EWEq},平面简谐波\upref{PWave}}
\subsection{干涉}

光是一种电磁波,满足由麦克斯韦方程组导出的电场波动方程\upref{EWEq}.它服从重要的\textbf{叠加原理}.简单地说,干涉就是两束或多束光波的相互作用,这种相互作用产生的总辐照度不等于各束光波的辐照度之和.今后我们将在衍射中了解到巴俾涅原理.

考虑多个光源产生的场$\bvec E_1$ , $\bvec E_2$ , ...空间一点的总电场强度 $\bvec E$ 由下式给出:

$$ \bvec E = \bvec E_1 + \bvec E_2 + \cdots$$

光扰动以大约$\Si{10}^{14}$ Hz 的频率随时间飞快地变化,使得每时每刻的实际光场无法探测. 另一方面, 辐照度 $I$ 则可以使用各种探头直接测量. 因此, 研究干涉最好从辐照度的角度研究.

考虑由两个点光源 $S_1$ 和 $S_2$ 在均匀介质(即各向同性介质)中发射同一频率的平面波. 令 $S_1$ 与 $S_2$ 的间隔为 $a$, 且 $ a \gg \lambda $( $\lambda$ 是光的波长 ),考虑简单的线偏振波:

$$ \bvec E_1 ( \bvec r , t) = \bvec E_{01} \cos (\bvec k_1 \vdot \bvec r_1 - \omega t + \varepsilon_1 )$$
$$ \bvec E_2 ( \bvec r , t) = \bvec E_{02} \cos (\bvec k_2 \vdot \bvec r_2 - \omega t + \varepsilon_2 )$$

则 $ $ P $ $ 点的辐照度由下式给出:
$$ I = \epsilon v \langle \bvec E^2 \rangle _T $$
$\langle \bvec E^2 \rangle _T$ 的含义为电场强度的平方在一段时间内的平均值. 我们只关注 $ I $ 与 $ \bvec E $ 的关系,因此忽略常系数 $ \epsilon v$, 便于讨论. 

 $ P $ 点的合场强为$ \bvec E = \bvec E_1 + \bvec E_2 $ , 则辐照度  
\begin{aligned}
I = \langle( \bvec E_1 + \bvec E_2)\vdot( \bvec E_1 + \bvec E_2 )\rangle _T \\
 = \langle \bvec E_1^2 \rangle _T + \langle \bvec E_2^2 \rangle _T + 2\langle \bvec E_1 \vdot \bvec E_2 \rangle _T = I_1 + I_2 + I_{12}
\end{aligned}

其中 $ I _{12} =2 \langle \bvec E_1 \vdot \bvec E_2 \rangle _T$.

将 $\bvec E_1$ 和 $ \bvec E_2$ 的表达式代入 $I_{12}$可得:
$$I_{12} = \bvec E_{01}\bvec E_{02} \cos\delta$$
$$\delta = (\bvec k_1 \vdot \bvec r_1 - \bvec k_2 \vdot \bvec r_2 ) + \varepsilon_1 - \varepsilon_2 $$

称 $ I_{12} $ 为\textbf{干涉项}, $ \delta $ 为\textbf{相差}.
\begin{exercise}{}
请读者验证上述结论. 提示:$\langle \cos ^2 \omega t \rangle _T = \langle \sin ^2 \omega t \rangle _T = 1/2 $ , $ \langle \cos \omega t \sin \omega t \rangle_T =0 $. 或者,将电场用复数表示再代入验证.
\end{exercise}

\subsection{完全相长(相消)干涉}
我们通常遇到的情况为 $\bvec E_1 \parallel \bvec E_2$,此时
$$ I_{12} = E_{01} E_{02}\cos\delta$$

利用$ I = \langle \bvec E^2 \rangle _T = \dfrac{1}{2}E^2$,将 $ E_{ 01} = \sqrt{2I_1}$, $ E_{ 02} = \sqrt{2I_2}$  代入上式,可以得到 $ I $更简洁的表达式:
$$  I = I_1 + I_2 + 2\sqrt{I_1 I_2}\cos\delta$$

(1)当 $\delta = 0, 2\pi, 4\pi, \cdots$ 时,$\cos \delta = 1$,得到 $ I_{max} = I_1 + I_2 +2\sqrt{I_1 I_2}$,这种情况称为\textbf{完全相长干涉};

(2)当 $\delta = -\pi, \pi, 3\pi, \cdots$ 时,$\cos \delta = 0$,得到 $ I_{min} = I_1 + I_2 $,这种情况称为\textbf{完全相消干涉}.

\subsection{场强的余弦平方分布}

若 $ \bvec E_{01} \parallel \bvec E_{02} $ 且 $S_1$ 和 $S_2$ 发出的光场到达 $ P $ 点的振幅相同, 即 $ I_1 = I_2 \equiv I_0 $, 则
$$ I = 2 I_0 ( 1 + \cos\delta ) = 4 I_0 \cos ^2\dfrac { \delta } { 2 } $$

这就是远场近似时,场强的\textbf{余弦平方分布}. 你将在杨氏双缝实验、菲涅尔双面镜以及薄膜的干涉中多次遇见这个式子.

上面我们讨论了两个平面波在空间某一点的叠加产生的总场强,它在满足 $ \bvec E_{01} = \bvec E_{02}$时,服从余弦平方分布. 对于更一般的球面波, 上式也同样成立. 

从 $ S_1 $ 和 $ S_2 $ 发出的球面波分别表示为:
$$ \bvec E_1 = \bvec E_{ 01 } ( r_1 ) \exp[ i( k r_1 - \omega t+ \varepsilon_1 )]$$
$$ \bvec E_2 = \bvec E_{ 02 } ( r_2 ) \exp[ i( k r_2 - \omega t+ \varepsilon_2 )]$$

其$ r $ 是在 $ P $ 点重叠的球形波阵面的半径. 电场的振幅与 $ r $ 成反比. 利用叠加原理, 同样可得 $I_{12} = \bvec E_{01}\bvec E_{02} \cos\delta$. 这时, $ \delta = k ( r_1 - r_2 ) + ( \varepsilon_1 - \varepsilon_2 )$.  请读者自行验算.

我们依然关注特殊的情况, 比如,当$ a \ll r $ 时, 可视 $ \bvec E_1$ 、 $ \bvec E_2$ 在小区域内为常量, 即 $ E_{ 01} \approx E_{02}$; 若 $ \bvec E_{01} = \bvec E_{02} $ , $ I_1 = I_2 = I_0 $, 同样可以得到:
$$ I = 4 I_0 \cos ^2\dfrac { \delta } { 2 }$$

\begin{exercise}{}
仿照对 $ \delta$ 不同取值的讨论,请读者计算 $ I $ 取极大值和极小值时,$ r_1 - r_2 $ 需满足什么条件.
\end{exercise}

\begin{itemize}
\item 当 $ \delta = \pi m'$, $ m' = \pm 1, \pm 3, \cdots $ 时, $ I $ 取极小值, 
$$ r_1 - r_2 = \dfrac{ m' \pi - ( \varepsilon _1 - \varepsilon _2)} { k } $$
\item 当 $ \delta = 2 \pi m$, $ m = 0, \pm 1, \pm 2, \cdots $ 时, $ I $ 取极大值, 
$$ r_1 - r_2 = \dfrac{ 2m \pi - ( \varepsilon _1 - \varepsilon _2)} { k } $$
\end{itemize}





