% 列维—奇维塔符号
% Levi|Civita

\begin{issues}
\issueDraft
\end{issues}

\pentry{逆序数\upref{InvNum}}

\footnote{本文参考 Wikipedia \href{https://en.wikipedia.org/wiki/Levi-Civita_symbol}{相关页面}.}\textbf{列维—奇维塔符号(Levi-Civita symbol, 简称 LC 符号)}是一个函数, 记为 $\epsilon_{i_1, i_2, \dots, i_N}$. $N$ 叫做它的\textbf{维数(dimension)}. 它的自变量是 $N$ 个正整数 $i_1, \dots, i_N$, 可以称为角标. 每个角标从 $1, 2, \dots, N$ 中取值. 函数值只能取 $0, 1, -1$ 中的一个. 当 $i_1, \dots, i_N$ 中有任意两个重复时函数值为 0; 若没有重复, 则函数值为 $(-1)^{N_p}$, $N_p$ 为排列 $i_1, \dots, i_N$ 的逆序数\upref{InvNum}

综上, $N$ 维 LC 符号的角标共有 $N^N$ 种不同可能, 而使函数值不为零的有 $N!$ 种.

\subsection{三阶定义}
三阶 LC 符号是矢量分析中常用的, 可以直接用穷举法来定义.
\begin{equation}
\epsilon_{123} = \epsilon_{231} = \epsilon_{312} = 1
\end{equation}
\begin{equation}
\epsilon_{321} = \epsilon_{213} = \epsilon_{132} = -1
\end{equation}
当 $i,j,k$ (只能取 1,2,3)中任意两个重复时, $\epsilon_{ijk} = 0$.

\subsection{N 阶}
\begin{equation}
\epsilon_{i_1,i_2,\dots, i_n}
\end{equation}
