% 相互作用表象
% 相互作用表象|相互作用场论
\pentry{海森堡绘景\upref{HsbPic}}

一个相互作用系统中,哈密顿量一般可以写为自由哈密顿量与相互作用哈密顿量之和 $H=H_0(m_0)+H_{int}$.我们已经知道如何在海森堡表象中讨论自由场论,但当相互作用引入时,原先的能量本征态不再是完全哈密顿量 $H$ 的本征态,系统的真空态(基态)也不再是原来的真空态,这造成了很大的困难.为此我们引入相互作用表象,试图在微扰论的框架下解决相应的问题.我们约定上下标中的 $S$ 代表薛定谔表象,$I$ 代表相互作用表象.
\subsection{相互作用场}
我们知道在海森堡表象下态矢量是不变的,场算符随时间的演化方程为
\[
\phi(t,\bvec x)=e^{H(t-t_0)}\phi_S(\bvec x)e^{-H(t-t_0)}
\]

