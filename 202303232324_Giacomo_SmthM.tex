% 光滑映射(简明微积分)
% 光滑函数|解析函数|smooth|smooth function|smooth map|analytical function|微分|欧几里得空间

\pentry{线性映射\upref{LinMap},矢量的导数\upref{DerV},偏导数(简明微积分)\upref{ParDer}}
%预备知识需要创作+整合
%注意,缺失了欧几里得空间的相关词条。思考该如何撰写和整理相关概念。

% 欧几里得空间作为线性空间,彼此之间可以进行线性映射;同时,欧几里得空间中可以进行微分操作,因此我们还可以研究这些线性映射的微积分性质。

本节讨论欧几里得空间间一类可以无穷次求(偏)导的线性映射,称为光滑映射。

\subsection{光滑函数}

为了讨论光滑映射,我们首先引入光滑函数的概念。

\begin{definition}{实数集上的光滑函数}
给定函数 $f:\mathbb{R}\rightarrow\mathbb{R}$,如果在某一点 $x_0\in\mathbb{R}$,对于任意的 $k\in\mathbb{Z}^+$,存在 $f$ 的 $k$ 阶导数 $f^{(k)}(x_0)$,则称 $f$ 是在点 $x_0$ 处\textbf{任意阶可导的},或 $C^\infty$ 的,或\textbf{光滑的(smooth)}。今后,我们将术语 $C^{\infty}$ 和\textbf{光滑}看成完全等价的。如果 $f$ 处处光滑,则称其本身是一个\textbf{光滑函数}。
\end{definition}

光滑函数的概念可以通过偏导数来推广到多自变量的情况。

\begin{definition}{多元光滑函数}
给定函数 $f:\mathbb{R}^n\rightarrow\mathbb{R}$,如果在某一点 $(x_0^1, x_0^2, \cdots x_0^n) \in\mathbb{R}^n$,对于任意的 $k\in\mathbb{Z}^+$,存在 $f$ 的 $k$ 阶偏导数$$\frac{\partial^k}{\partial x^{i_1}\partial x^{i_2}\cdots\partial x^{i_k}}f(x_0^1, x_0^2, \cdots x_0^n)$$其中各 $i_j$ 都是 $1$ 到 $n$ 的任意整数,可以有重复;则称 $f$ 在 $(x_0^1, x_0^2, \cdots x_0^n)$ 处\textbf{光滑}。如果 $f$ 处处光滑,则简称 $f$ \textbf{光滑}。
\end{definition}

\subsection{光滑映射}

两个欧几里得空间之间的线性映射 $f:\mathbb{R}^m\rightarrow \mathbb{R}^n$,可以看成是 $m$ 个自变量的 $n$ 维向量值函数;进一步,也可以把向量看成 $n$ 个标量,也就是说把这个线性映射看成 $n$ 个线性函数组合在一起。

\begin{definition}{光滑映射}
设 $f:\mathbb{R}^m\rightarrow \mathbb{R}^n$ 是欧几里得空间之间的线性映射。对于 $\forall \bvec{x_0} \in \mathbb{R}^m$,记 $f(\bvec{x_0})=(f_1(\bvec{x_0}), f_2(\bvec{x_0}),\cdots, f_n(\bvec{x_0}))$。如果每一个 $f_i$ 都是\textbf{光滑函数},则称 $f$ 是一个\textbf{光滑映射(smooth map)}。
\end{definition}

\subsection{光滑函数和解析函数的比较}

解析函数是一种性质良好的函数,它可以表示成幂级数的形式。如果一个函数解析,那么我们总可以在任何点处对它进行泰勒展开,展开结果就是其幂级数表达。能进行泰勒展开固然需要可以无穷求导,也就是说解析函数必然是光滑的;但反过来却不成立,有的光滑函数就不解析。

\begin{example}{}\label{SmthM_ex1}
考虑函数 $f:\mathbb{R}\rightarrow\mathbb{R}$,其中当 $x>0$ 时 $f(x)=\exp{(-1/x)}$,当 $x\leq 0$ 时 $f(x)=0$。可以验证,$f$ 处处光滑,但是它在 $x=0$ 处的泰勒展开是常数 $0$,而它显然不是一个常函数,因此 $f$ 在 $x=0$ 处不解析。
\end{example}



