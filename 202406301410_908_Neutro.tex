% 中子
% license CCBYSA3
% type Wiki

(本文根据 CC-BY-SA 协议转载自原搜狗科学百科对英文维基百科的翻译)

\textbf{中子}是亚原子粒子,符号为$n$或者n⁰,净电荷为零且质量略大于质子。质子和中子构成了原子的核子。因为质子和中子在原子核内的行为相似,每个质子和中子的质量大约为1原子质量单位,它们统称为核子。[1]它们的性质和相互作用由核物理描述。

原子核的化学和核性质由质子数以及中子数决定,分别称为原子序数和中子数。原子质量数是核子的总数。比如,碳原子序数为6,并且常见的碳-12有6个中子,而罕见的碳-13有7个中子。有些元素在自然界中只有一种稳定同位素,例如氟。有些元素则有许多稳定的同位素,比如锡有十种稳定同位素。

在原子核内,质子和中子通过核力结合在一起。除了单质子氢原子之外,原子核的稳定需要中子。中子在核裂变和聚变中大量产生。它们通过裂变、聚变和中子俘获过程参与在恒星内化学元素的核合成。

中子对核能生产至关重要。在詹姆斯·查德威克1932年发现中子后的十年里,中子被用来诱导许多不同类型的核嬗变。随着1938年核子裂变的发现,[2]人们很快认识到,裂变事件产生的每一个中子都可能引起进一步的裂变事件,这种级联反应被称为核链式反应。这些发现导致了第一个自持的核反应堆 ( 芝加哥一号堆,1942年)和第一个核武器 ( 三一,1945年)的产生。

自由中子虽然不是电离原子,但会导致致电离辐射。因此,当剂量较大时,它们具备生物性危害。地球上存在有小的自然的自由中子,主要来源于宇宙线射线浴以及地壳内自发裂变元素产生的辐射。用于辐射和中子散射实验的自由中子来自于专用的中子源,如中子发生器、研究堆和散裂源。

\subsection{描述}
原子核由许多质子形成,$Z$(原子序数)和中子数,$N$(中子数),由核力结合在一起。原子序数定义原子的化学性质,中子数决定同位素或核素。[3]同位素和核素这两个术语经常被当作同义词使用,但是它们分别指化学性质和核性质。严格来说,同位素是两种或多种质子数相同的核素;具有相同中子数的核素被称为同中子异荷素。原子质量数,符号$A$,等于$Z+N$。具有相同原子质量数的核素称为同量异位素。氢原子最常见的同位素(带有化学符号${}^{1}h)$),其核子是一个单独的质子。重氢同位素的原子核$(D\text{或}{}^{2}H)$和氚$(T\text{或}{}^{3}H)$分别包含一个和两个中子。其他所有类型的原子核都由两个或多个质子和不同数量的中子组成。化学元素铅中最常见的核素,${}^{208}Pb)$有82个质子和126个中子。核素表包括所有已知的核素。尽管中子不是化学元素,但它包含在这个表中。[3]

自由中子的质量为939,565,$413.3 eV /c^{2}$,或$1.674927471\times10^{-27}$ kg,或1.00866491588 u。中子的均方半径约为$0.8\times10^{-15}$ m,或0.8 fm ,[4]这是自旋为1的费米子。[5]质子带正电荷,直接受电场影响,而中子不带电荷,故不受电场影响。然而,中子具有磁矩,因此受磁场影响。中子的磁矩是负的,其取向与其自旋方向相反。[6]

自由中子是不稳定的,其衰变产物为质子,电子和反中微子,平均寿命不到15分钟(881.5±1.5 s)。这种被称为β衰变的放射性衰变是可能的,因为中子的质量略大于质子且自由质子是稳定的。然而,结合在原子核中的中子或质子可以是稳定的,也可以是不稳定的,这取决于核素。在$\beta$衰变,中子衰变为质子,反之亦然,受弱力支配,它需要发射或吸收电子和中微子,或它们的反粒子。
\begin{figure}[ht]
\centering
\includegraphics[width=6cm]{./figures/a27c6c16b4c60c50.png}
\caption{铀-235吸收中子引起的核裂变。重核素分裂成更轻的成分和额外的中子。} \label{fig_Neutro_1}
\end{figure}
质子和中子在核子核力的影响下表现几乎相同。同位旋的概念视质子和中子为同一粒子的两个量子态,用于模拟核子在强力和弱子下的相互作用。由于核力在短距离内的强度,核子的结合能比原子中束缚电子的电磁能大7个数量级。因此核反应(例如核子裂变)产生的能量密度是化学反应的1000多万倍。根据质能公式,核结合能降低了原子核的质量。因此,核子内部电磁排斥所产生的能量是核能的主要来源,这也使得核反应堆和核子炸弹成为可能。在核裂变中,中子被重核素吸收(例如,铀-235)导致核素变得不稳定,并分裂成轻核素和额外的中子。带正电荷的轻核素然后排斥,释放电磁势能。

中子被归类为强子,因为它是由夸克组成的复合粒子。同时它也被分类为重子,因为它由三个价夸克组成[7]。中子的大小及其磁矩表明中子是一种复合粒子,而不是基本粒子。一个中子包含两个带电荷-1/3e的下夸克和一个带电荷+2/3e的上夸克。

像质子一样,中子的夸克通过由胶子做介质的强力结合在一起。[8]核力源于更基本的强力的次要影响。

\subsection{发现}
中子及其性质的发现是20世纪上半页原子物理学非凡发展的核心,最终导致了1945年的原子弹。在1911年的卢瑟福模型中,原子由一个小的带正电的原子核及其周围大得多的带负电荷的电子云组成。1920年,卢瑟福提出,原子核由带正电的质子和不带电的粒子组成,被认为是以某种方式束缚的质子和电子。[9]电子被认为在原子核内,因为已知的β辐射可以从原子核发射出电子。[9]卢瑟福称这些不带电荷的粒子为中子,由拉丁语词根表示中立(中性)和希腊语后缀-打开(亚原子粒子名称中使用的后缀,即电子和质子)组成。[10][11]然而,在1899年的文献中就发现了对这个词中子的引用。[12]

在整个20世纪20年代,物理学家假设原子核是由质子和“核电子”组成[13][14]。但是这个假设有个明显的问题,那就是很难调和原子核的质子-电子模型与量子力学的海森堡不确定关系。[15]由奥斯卡·克莱因于1928年发现的[16][17]克莱因悖论,从量子力学层面上进一步反对电子限制在原子核内的假设。[16]观察到的原子和分子的自旋跟质子-电子假设预期的合自旋不一致。质子和电子的自旋均为 ½ ħ。相同元素的同位素(具有相同质子数的元素)具有整数自旋或分数自旋,例如中子自旋必须是分数。然而,没有可能让电子和质子的合自旋(应该结合形成中子)来获得中子的分数自旋。

1931年,瓦尔特·博特和赫伯特·贝克尔发现,来自钋的α粒子辐射到铍、硼或锂后,就会产生具备异常穿透性的辐射。该辐射不受电场的影响,所以两人都认为是伽马辐射。[18][19]次年,巴黎的伊雷娜·约里奥-居里和弗雷德里克·约里奥-居里表明,这种“伽马”辐射落在石蜡或任何其他含氢的化合物上,它会射出能量非常高的质子。[20]卢瑟福和詹姆斯·查德威克在卡文迪许实验室在剑桥被伽马射线的解释所说服。[21]查德威克很快进行了一系列实验,表明新的辐射由不带电荷的粒子组成,这些粒子的质量大约与质子相同。[22][22][23]这些粒子是中子。查德威克因这一发现获得了1935年的诺贝尔物理学奖。

维尔纳·海森堡和其他人很快发展了由质子和中子组成的原子核模型[24][25][26]。[27][28]质子-中子模型解释了核自旋的难题。$\beta$辐射的起源可以解释为恩利克·费密1934年由$\beta$衰变过程中子通过衰变变成质子,一个电子和一个(尚未发现的)中微子。[29]1935年,查德威克和他的博士生莫里斯·戈德哈伯报告了中子质量的第一次精确测量。[30][31]

1934年,费米用中子轰击了较重的元素,以在高原子序数元素中诱导放射性。1938年,费米获得了诺贝尔物理学奖“因为他证明了中子辐照产生的新放射性元素的存在,也因为他发现了慢中子引起的核反应“。[32]1938年奥托·哈恩,莉泽·迈特纳和弗里茨·施特拉斯曼发现核子裂变,或中子轰击导致铀核分馏成轻元素。[33][34][35]1945年,哈恩获得了1944年的诺贝尔化学奖 "因为他发现了重原子核的裂变。"[36][37][38]核裂变的发现导致了第二次世界大战结束时核能和原子弹的发展。
\begin{figure}[ht]
\centering
\includegraphics[width=10cm]{./figures/72376c3d96f99be8.png}
\caption{氢、氦、锂和氖原子中原子核和电子能级的模型。实际上,原子的直径大约是原子核直径的10万倍。} \label{fig_Neutro_2}
\end{figure}

\subsection{β衰变与原子核的稳定性}
\begin{figure}[ht]
\centering
\includegraphics[width=6cm]{./figures/3b2f9e698de6f000.png}
\caption{中子经由中间重 W玻色子β衰变为质子、电子和电子反中微子的费曼图} \label{fig_Neutro_3}
\end{figure}
在粒子物理学的标准模型下,中子保持重子数守恒的的唯一可能衰变模式是中子的一个夸克通过弱相互作用变性。中子的一个下夸克通过发射一个 W玻色子衰变为更轻的上夸克。通过这个过程,中子衰变为质子(包含一个下夸克和两个上夸克),一个电子和一个电子反中微子。此即为标准的β衰变的过程。

因为质子之间的电磁排斥远大于使得它们吸引的核相互作用,中子是任何包含一个以上质子的原子核的必要组成部分(参见双质子和中子质子比)。[39]中子通过核力与质子在原子核中相互结合,有效地调节质子之间的排斥力并使得原子核稳定。

\subsubsection{3.1 自由中子衰变}
在原子核外,自由中子是不稳定的,其平均寿命为$881.5\pm1.5 s$(大约14分42秒)。因此,该过程的半衰期(与平均寿命的差异为ln(2) = 0.693)是$611.0\pm1.0 s$(大约10分11秒)。[40]如上所述,中子的Β衰变可以用放射性衰变表示:[40]
\begin{align}
n^0 &\rightarrow p^+ + e^- + \bar{\nu}_e~
\end{align}
其中 $p^+$、$e^-$ 和 $\bar{\nu}_e$ 分别表示质子、电子和电子反中微子。自由中子的衰变能量(基于中子、质子和电子的质量)为 0.782343 MeV。β 衰变中电子能获得的最大能量(在中微子动能几乎为零的情况下)为 $0.782 \pm 0.013 MeV$。在其他方法有所局限的情况下,该最大的电子能量还不足够好的来确定中微子的静止质量(理论上中微子的静止质量必须从电子的最大动能中减掉)。

一小部分(大约千分之一)的自由中子衰变产物相同,但是增加了一个伽马射线的粒子:

\begin{align}
n^0 &\rightarrow p^+ + e^- + \bar{\nu}_e + \gamma~
\end{align}

这种伽马射线被认为是由 β 粒子与质子的电磁相互作用产生的“内部跃迁辐射”的结果。伽马射线的产生也是束缚中子$\beta$衰变的一个必要特征(如下所述)。
\begin{figure}[ht]
\centering
\includegraphics[width=6cm]{./figures/3bca3d6ca3381cd0.png}
\caption{原子核的β−辐射的示意图,即从原子核发射出的快速电子(省略了伴随的反中微子)。在卢瑟福模型核中,红色球体是带正电荷的质子,蓝色球体是与的电子紧密结合的净电荷为零的质子。 在当前理论模型下,右下角的小图显示了的自由中子的β衰变,在这个过程中产生了电子和反中微子。} \label{fig_Neutro_4}
\end{figure}
极少数中子衰变(约百万分之四)是所谓的“双体(中子)衰变”,其中照常产生质子、电子和反中微子,但电子无法获得逃离质子所需的13.6 eV能量(氢的电离能),因此依附于质子,成为中性氢原子(“两个体”之一)。在这种类型的自由中子衰变中,几乎所有的中子衰变能量都被反中微子(另一个“体”)带走。(氢原子以大约(衰变能量)/(氢静止能量)乘以光速的速度反冲,即250千米/秒。)

自由质子转化为中子(加上正电子和中微子)在能量上是不可能的,因为自由中子的质量比自由质子大。但是质子和电子或中微子的高能碰撞会产生中子。

\subsubsection{3.2 束缚中子衰变}
虽然自由中子的半衰期约为10.2分钟,但原子核内的大多数中子是稳定的。根据核壳模型,核素的质子和中子同属一个量子力学系统, 它们分属不同的能级且量子数唯一。如果中子要衰变,产生的质子需要占据比初始中子更低的能态。在稳定原子核中,可能的低能态都被填满了,这意味着每个能态都被两个自旋相反的质子所占据。因此,泡利不相容原理不允许中子在稳定的原子核内衰变为质子。这种情况类似于原子中的电子,每个电子占据不同的原子轨道,根据不相容原理,它们不可以通过发射出一个光子而衰变到更低的能态。

如上所述,不稳定原子核中的中子可以进行β衰变。在这种情况下,衰变产生的质子必须有一个可用的量子态。其中一个例子是碳-14 (6个质子,8个中子),它衰变为氮-14 (7个质子,7个中子),半衰期约为5730年。

在原子核内,如果中子有可用的量子态,质子可以通过逆β衰变转化为中子。这种转变是通过发射一个正电子和一个电子中微子来实现的:
$$p^+ \rightarrow n^0 + e^+ + \nu_e~$$
质子在原子核内转化为中子也可以通过电子俘获实现:
$$p^+ + e^- \rightarrow n^0 + \nu_e~$$
在含有过量中子的原子核中,中子捕获正电子也是可能的,但容易被阻碍,这是因为正电子被原子核排斥,遇到电子会迅速湮灭。

\subsubsection{β衰变类型的竞争}
竞争中的三种β衰变由单同位素铜-64 (29个质子,35个中子)说明,其半衰期约为12.7小时。这种同位素有一个不成对的质子和一个不成对的中子,所以质子或中子都可以衰变。这种特定核素发生质子衰变(通过正电子发射,18\%或通过电子俘获,43\%)或中子衰变(通过电子发射,39\%)的概率相同。

\subsection{内禀性质}
\subsubsection{4.1 质量}
由于电荷为零,中子的质量不能通过质谱直接测定。然而,由于质子和氘核的质量可以用质谱仪测量,中子的质量可以通过从氘核质量中减去质子质量来推断,其差值是中子的质量加上氘的结合能(表示为正发射能量)。后者可以直接通过测量中子被质子俘获时发射的伽马光子能量$B_d$ (放热过程,且中子能量为零),加上氘核的小反冲动能$E_{rd}$(约占总能量的0.06\%)。
\begin{equation}
m_n = m_d - m_p + B_d - E_{rd}~
\end{equation}
伽马射线的能量可以通过X射线衍射技术高精度地测量,正如贝尔和埃利奥特在1948年首次做的那样。1986年格林等人提供了用这种技术测量中子质量的最佳值。测量的中子质量为:
\begin{equation}
m_{\text{中子}} = 1.008644904(14) \\ \text{u}~
\end{equation}
由于从$u$到MeV的转换精度较低,以MeV表示的中子质量值不太准确:
\begin{equation}
m_{\text{中子}} = 939.56563(28) \\, \text{MeV}/c^2~
\end{equation}
中子质量也可通过测量中子β衰变产生的质子和电子的动量来获得。

\subsubsection{4.2 电荷}
中子的电荷是 $0  e$. 这个零值已经通过实验测试, 中子电荷的当前实验极限是 $-2(8) \times 10^{-22}  e$, 或者 $-3(13) \times 10^{-41}  C$. 考虑到实验的不确定性(用括号表示), 该值与零一致。相比之下, 质子的电荷是 $+1  e$.

\subsubsection{4.3 磁矩}
即使中子是中性粒子,其磁矩也不为零。中子不受电场影响,但受磁场影响。中子的磁矩表明了它的夸克子结构和内部电荷分布。[45]1940年,路易斯·阿尔瓦雷斯和费利克斯·布洛赫在伯克利(加利福尼亚州) 首次直接测量中子磁矩值。[46]阿尔雷斯和布洛赫确定中子的磁矩为 $\mu_n = -1.93(2) \mu_N$,其中 $\mu_N$ 是核磁子。

在强子的夸克模型中,中子由一个上夸克(电荷+2/3)和两个下夸克(电荷-1/3 e)组成。[45]中子的磁矩可以建模为组夸克磁矩的总和。[47]假设夸克的行为像点状狄拉克粒子,每个粒子都有自己的磁矩,因此中子的磁矩可以被视为三个夸克磁矩的矢量和以及再加上三个带电夸克在中子内运动引起的轨道磁矩。

标准模型( SU(6)理论,现在用夸克行为来理解)的早期成功之处在于,1964年米尔扎A.B贝格、本杰明·w·李和亚伯拉罕·派斯从理论上计算出质子和中子磁矩的比率为-3/2,[48][49][50]该比率的测量值为−1.45989805(34),两者结果的误差在3\%以内。[51]该计算结果基于量子力学,但其与泡利不相容原理的矛盾导致奥斯卡·格林伯格在1964年发现了夸克的色荷。[48]

上述模型将中子和质子进行了比较,并在模型中减去夸克的复杂行为,仅探索不同夸克电荷(或夸克类型)的影响。这种计算足以表明中子的内部非常类似于质子的内部,除了中子和质子中的夸克组成不同之外。

中子磁矩可以通过一个研究重子的简单非相对论、量子力学波函数数粗略计算。该计算给出了中子、质子和其他重子磁矩相当精确的估计。[47] 计算的结果表明中子的磁矩由下式给出$\mu_n = 4/3 \mu_d - 1/3 \mu_u~$
其中$\mu_d$和$\mu_u$分别是上下夸克的磁矩。这一结果考虑了夸克的固有磁矩和轨道磁矩,并假设三个夸克处于特定的主导量子态。
\begin{figure}[ht]
\centering
\includegraphics[width=14.25cm]{./figures/fdbaeebf9d55acf5.png}
\caption \label{fig_Neutro_5}
\end{figure}
这个计算的结果令人鼓舞,但是上夸克或下夸克的质量被假设为1/3核子的质量。[47]夸克的质量实际上只有核子的大约1\%。[51]这种差异源于核子标准模型的复杂性,其中它们的大部分质量来源于胶子场、虚粒子及其相关能量,这些能量是强力的重要方面。[51][52]此外,构成中子的夸克和胶子的复杂系统需要相对论处理。[53]然而,核子磁矩已经用第一性原理成功地计算出来,包括所有提到的效应,并使用更现实的夸克质量值。该计算给出的结果与测量结果相当一致,但它需要大量的计算资源。[54][55]

\subsubsection{4.4 旋转}
中子是自旋1/2粒子,也就是说,它是固有角动量为1/2ħ的费米子,其中ħ是约化普朗克常量。在中子被发现后的许多年里,它的确切自旋是不清楚的。虽然被认为是自旋为1/2的狄拉克粒子,但是其自旋为3/2的可能性仍然存在。中子磁矩与外部磁场的相互作用被用来最终确定中子的自旋。[56]1949年,休斯和伯奇测量了铁磁反射镜反射的中子,发现反射的角度分布与自旋1/2一致。[57]1954年,舍伍德、斯蒂芬森和伯恩斯坦在施特恩-格拉赫实验使用磁场来分离中子自旋态。他们记录了两种这样的自旋态,且与自旋1/2粒子一致。[56][58]

作为费米子,中子服从泡利不相容原理;两个中子不能有相同的量子数。这是简并压力的来源并使得中子星成为可能。

\subsubsection{4.5 电荷分布的结构和几何形状}
2007年发表的一篇文章进行了不依赖于模型的分析后得出结论,中子具有带负电的外部、带正电的中间和带负电的核心。[59]简单的说,中子的电负性外壳同质子相互吸引。但是,在质子核中,质子和中子之间最主要的作用力为核力。这种力跟粒子是否带电荷无关。

中子电荷分布的简化观点也“解释”了为什么中子磁偶极子指向与其自旋角动量矢量相反的方向(与质子相比)。中子的电负性外壳相当给了中子一个磁矩,由于其负电荷在中子表面,因此具有较大的分布半径,从而对粒子的磁偶极矩的贡献大于更靠近其中心的正电荷部分。

\subsubsection{4.6 电偶极矩}
粒子物理标准模型预测中子内具有微小但非零的电偶极矩。[60]然而,测量其数值所需要的精度远远超过目前的实验条件。标准模型不可能是对物理现实的最终和最完整描述。超越标准模型的新理论得到的数值一般要比标准模型大得多。目前,至少由四组直言力图测量中子的电偶极矩包括:
\begin{itemize}
\item 在劳厄-朗之万研究所进行的低温中子电偶极矩实验 (CryoEDM)[61]
\item 在瑞士的保罗谢尔研究所进行的中子电偶极矩(nEDM)实验[62]
\item 在橡树岭国家实验室散裂中子源(Spallation Neutron Source)进行的中子电偶极矩(nEDM)实验[63][64]
\item 在劳厄-朗之万研究所进行的中子电偶极矩(nEDM)实验[65]
\end{itemize}

\subsubsection{4.7 反中子}
反中子是中子的反粒子。它是由布鲁斯·科克在1956年发现的,也就是反质子被发现的一年后。 CPT对称对粒子和反粒子的相对性质提出了强约束,因此研究反中子为CPT对称提供了严格的实验证明。中子和反中子质量的分数差是$(9\pm6)\times10{-5}$。因为差异离零只有大约两个标准差,所以没有给出令人信服的违反CPT的证据。[66]

\subsection{中子化合物}
\subsubsection{5.1 双中子和四中子}
基于对铍 -14原子核衰变的观察,CNRS核物理实验室的弗朗西斯科-米格尔·马尔奎斯领导的一个小组假设存在稳定的4中子团簇,即四中子。有趣的是当前的理论表明这些簇应该是不稳定的。

2016年2月,东京大学的日本物理学家Susumu Shimoura和他的同事报告说,他们首次实验观察到了所谓的四正电子。[66]世界各地的核物理学家表示,这一发现如果得到证实,将是核物理领域的一个里程碑,肯定会加深我们对核力的理解。[67][68]

双中子是另一种假设粒子。2012年,密歇根州立大学的阿尔特弥斯·斯皮鲁和他的同事报告说,他们首次观察到了在16Be的双中子辐射。两个中子之间的小发射角证明了双中子的特性。作者测量了双中子分离能量为1.35(10) MeV,与使用该质量区域的壳模型计算结果非常一致。[69]

\subsubsection{5.2 中子物质和中子星}
在极高的压力和温度下,人们认为核子和电子坍缩成块状中子物质,称为中子物质。中子星中也会发生类似的反应,其内部的极端压力可能会使中子变形为立方对称,从而中子可以更紧密地堆积在一起。[70]

\subsection{检测}
常见的通过寻找电离轨道(例如威尔逊云雾室)而检测带电粒子的方法对中子不起作用。中子弹性散射出原子后可产生可检测的电离轨道,但是实验层面上不容易实现。其他较为常用的办法,主要利用中子和原子核的相互作用。因此,中子的检测方法可以根据其核过程来分类,主要分为中子俘获或弹性散射。[71]

\subsubsection{6.1 中子俘获方法}
该测量方法主要通过将中子俘获后释放的能量转换成电信号而进行的。某些核素具有较高的中子俘获截面,也就是较高的中子俘获概率。中子俘获后,复合核辐射出更容易被探测的粒子,例如α粒子。核素$^3He,^6Li,^{10}B,^{233}U,^{235}U,^{237}Np$ 和$^{239}Pu$经常被用于此种方法。

\subsubsection{6.2 弹性散射方法}
中子可以和原子核发生弹性碰撞,导致原子核反冲。在运动学上,中子向轻核(如氢或氦)转移的能量比向重核转移的更多。依赖弹性散射的探测器被称为快中子探测器。反冲原子核可以通过碰撞电离和激发更多的原子。以这种方式产生的电荷和/或闪烁光可以被收集以产生检测信号。快速中子探测的一个主要挑战是从同一探测器中伽马辐射产生的错误信号中识别出这种信号。

快中子探测器的优点是不需要减速剂,因此能够测量中子的能量、到达时间,以及在某些情况下的入射方向。

\subsection{来源和生产}
尽管中子的半衰期比任何其他亚原子粒子都长,然而它们依然是不稳定的,其半衰期大约只有10分钟。因此中子只能现制现用。

\textbf{自然中子。}地球上到处存在自由中子束。在大气和深海中,自由中子是由高能宇宙线轰击大气层的上层不停产生的。这些高能中子能够较深地穿透水和土壤。在撞击原子核的过程中,除了其他反应之外,它们还诱导使中子从原子核中释放出来的散裂反应。在地壳中,中子的第二个来源来自于地壳矿物中存在的铀和钍的自发裂变。中子的强度不具备生物性危害,但是对高分辨粒子探测器特别重要。[72]最近的研究表明,即使雷暴也可以产生能量高达几十兆电子伏的中子。[72]这些中子的浓度大约在$10^{9}-10^{13}$每毫秒和每平方米,主要取决于探测高度。大多数中子的能量,即使初始能量为20MeV,也在1ms内下降到keV范围。[73]在火星表面大气浓厚到一定程度的地方,由宇宙射线产生的中子更多。这些中子不但在火星表面直接造成自上而下的辐射危害,还能够经地表反射后形成自下而上的辐射。这是火星载人航天计划不得不考虑得一个问题。[74]

\textbf{研究用中子源。}某些放射性衰变 (比如自发裂变和中子发射)以及一些核反应堆也可以产生中子。某些核反应,比如用自然产生的$\alpha$粒子或$\gamma$粒子轰击一些核素(主要是轻元素,比如铍和氘)以及诱发的核裂变也可产生中子。此外,一些高能核反应,比如高能宇宙射线爆发和粒子加速器中用高等粒子轰击靶子使其原子核发生裂变,也能产生中子流。一些小型粒子加速器经过优化后专门用于产生中子,被称为中子发生器。

在实验室,最常用的中子源是某些衰变时释放中子的核素,比如锎-252衰变(半衰期2.65年)的自发裂变,100个原子中有3个锎原子核裂变时会释放中子,每次裂变会平均产生3.7个中子。用阿尔法粒子轰击核铍靶也可制造中子。一个较为流行的系统是由锑-124和金属铍构成。将金属锑置于反应堆中以中子活化,锑-123(天然丰度为42.8\%)便会转化为锑-124,半衰期为60.9天的。其优点是便于储存和运输。[75]

核裂变反应堆产生自由中子;它们的作用是维持产生能量的链式反应。强烈的中子辐射也可以通过中子活化过程产生各种放射性同位素,中子活化是一种中子俘获。

在核聚变反应堆中,自由中子是反应的副产品,但是这些中子携带了巨大的动能。如何把这些动能转化为人类可用的能源,这是一个重大的挑战。这些自由中子还会制造出大量的中子激活产物,最后必须当作核废料处理。[76]
\begin{figure}[ht]
\centering
\includegraphics[width=8cm]{./figures/602a22b96792871c.png}
\caption{位于法国格勒诺布尔的劳厄-朗之万研究所是世界上最重要的中子研究机构之一。} \label{fig_Neutro_6}
\end{figure}

\subsubsection{7.1 中子束和生产后束的调制}
自由中子束可以通过中子源产生。研究者们可以去特殊研究机构使用其研究反应堆或三裂中子源。

因为其电中性,中子很难加速,减速,聚焦或偏转。带电粒子可以通过电场或磁场实现上述操作,但是这些方法对中子几乎没有影响。然而,因为中子拥有微笑但非零的磁矩,非均匀磁场可以起到一些控制作用。中子可以通过减速、反射和速度选择来控制。如同光子的法拉第效应,热中子通过磁性材料后可以被偏振化。通过使用磁镜和磁性干涉滤镜,可以制成较高偏振度、波长为6-7埃的冷中子束。[77]

\subsection{应用}
中子在许多核反应中起着重要作用。例如,中子俘获通常导致中子活化,诱导放射性。中子及其行为的知识在核反应堆和核武器的开发中非常重要。像铀-235 和钚-239 等元素的裂变是由它们吸收中子引起的。

冷的,温的,和热的 中子辐射通常用于中子散射设施,其中辐射的使用方式类似用X射线分析凝聚态物质。中子通过不同的散射在原子对比度方面与后者互补横截面;对磁性的敏感性;非弹性中子光谱学的能量范围:和对物质的深度渗透。

基于中空玻璃毛细管内全内反射或波纹铝板反射的“中子透镜”的发展推动了对中子显微镜和中子/伽马射线断层成像的持续研究。[78][79][80]

中子的一个主要用途是从材料中激发延迟以及促进γ射线,这个是中子活化分析 (NAA)和即时伽马中子活化分析 (PGNAA)的基础。NAA常用于分析核反应堆中的小样本材料,而PGNAA最常用于分析钻孔周围的地下岩石和传送带上的工业块状材料。

中子发射器的另一个用途是检测轻核,特别是水分子中的氢。当快中子与轻核碰撞时,它损失了很大一部分能量,变成慢中子,通过测量慢中子在被氢核反射后的速率,中子探针可以确定土壤中的含水量。

\subsection{医学疗法}
中子辐射既有穿透性又有电离性,所以可以用于医疗。然而,中子辐射可能会产生副作用,使受影响的区域具有放射性。因此,中子层析成像不是一种可行的医学应用。

快中子疗法利用通常大于20兆电子伏的高能中子来治疗癌症。癌症的放射疗法基于细胞对电离辐射的生物反应。如果辐射是在小范围内进行的,以损伤癌变区域,正常组织将有时间自我修复,而肿瘤细胞通常不能。[81]中子辐射以比γ辐射大一个数量级的速率将能量传递给癌症区域。[82]

低能中子束用于硼俘获疗法,以治疗癌症。在硼捕获疗法中,患者服用一种含有硼的药物,该药物优先在靶向肿瘤中积累,然后用非常低能的中子(尽管通常高于热能)轰击肿瘤,中子被硼中的硼-10 同位素俘获,产生硼-11的激发态,然后衰变产生锂-7 和α粒子,它们有足够的能量杀死恶性细胞,同时不会损伤附近的细胞。这种疗法使用的中子源强度为十亿(109)中子每秒每厘米2。这种通量需要一个核反应堆。

\subsection{保护}
暴露于自由中子是危险的,因为中子与体内分子的相互作用会导致分子和原子的分裂,还会产生辐射反应(例如质子)。辐射防护的常规预防措施:避免暴露,尽可能远离辐射源,并尽量减少暴露时间。然而,必须特别考虑如何保护免受中子照射。高原子序数和高密度的材料能够有效地屏蔽其他类型的辐射,例如,$\alpha$粒子、$\beta$粒子或伽马射线,常见的防辐射材料为铅。然而,这种方法不适用于中子,因为中子的吸收不像$\alpha$、$\beta$和$\gamma$辐射那样随着原子序数的增加而直接增加。相反,我们需要研究中子与物质之间的特定相互作用(参见上面关于检测的一节)。例如,富氢材料通常用于屏蔽中子,因为普通氢既散射中子,又减慢中子。这意味着简单的混凝土甚至石蜡填充的塑料块比高密度的材料能够提供更好的中子防护。减速后,中子可以被对慢中子具有高亲和力的同位素吸收,而不会引起二次俘获辐射,如锂-6。

富氢的普通水在核裂变反应堆中能够影响中子的吸收:通常,中子被普通水强烈吸收,因此需要用可裂变同位素富集燃料。[需要解释]重水中的氘对中子的吸收亲和力比原生(正常的轻氢)低得多。因此,氘用于CANDU型反应堆,以减慢(中等)中子速度,从而比中子俘获增加核裂变的概率。

\subsection{中子温度}
\subsubsection{11.1 热中子}
热中子是自由中子,在室温下其能量符合$kT=0.0253 eV(4.0\times10^{-21} J)$的麦克斯韦-玻尔兹曼分布。热中子的特征速度为2.2千米/秒(不是平均速度或中间速度)。名称中的“热”来自于它们所穿过的室温气体或物质的能量(参见动力学理论分子的能量和速度)。在与原子核发生多次碰撞(通常在10-20次)后,在没有被吸收的前提下,中子达到热中子的能级。

许多物质的热中子反应比超快中子反应具有更大的有效反应界面,因此热中子可以更容易(以更高的概率)被它们碰撞的任何原子核吸收,从而使该原子核产生更重的并且通常不稳定的同位素。

大多数裂变反应堆使用中子减速剂来减速,或者热化由核裂变发射的中子,使得中子更容易被俘获,从而导致进一步裂变。其他的快中子增殖反应堆直接使用裂变能中子。

\subsubsection{11.2 冷中子}
冷中子是热中子是在非常冷的物质如液体氘中平衡的。这样一个冷源放置在研究反应器或散裂源的减速剂中。冷中子对中子散射实验特别有用。
\begin{figure}[ht]
\centering
\includegraphics[width=8cm]{./figures/bbd2f183c9aba600.png}
\caption{冷中子源在液态氢的温度下提供中子} \label{fig_Neutro_7}
\end{figure}

\subsubsection{11.3 超冷中子}
超冷中子是在温度为几开尔文的物质中弹性散射冷中子而产生的,例如固体氘或超流体氦。另一种生产方法是通过冷中子的机械减速来实现。

\subsubsection{11.4 裂变中子}
中子是一个自由中子,其动能接近1 MeV$(1.6\times10^{-13} J)$,因此速度约为14000 km/s(约为光速的5\%)。他们被命名为裂变或者快中子,以区别于低能热中子和宇宙射线浴或加速器中产生的高能中子。快中子是由核裂变等核过程产生的。因此,裂变产生的中子具有符合麦克斯韦-玻尔兹曼分布的从0到约14兆电子伏的动能,其平均能量为2兆电子伏(对于U-235裂变中子),并且能量众数仅为0.75兆电子伏,这意味着其中一半以上不具备快速裂变的资格(因此几乎没有机会在增殖性燃料中启动裂变,例如U-238和Th-232)。

快中子通过慢化过程被制成热中子,一般通过中子慢化堆完成,且通常使用重水、轻水或石墨来缓和中子

\subsubsection{11.5 聚变中子}
D–T(氘–氚)聚变是核聚变反应产生能量最高的中子,在速度为17\%的光速时,其动能为14.1兆电子伏。D-T聚变也是最容易点燃的聚变反应,即使氘和氚核的动能只有14.1兆电子伏的千分之一,也达到接近峰值的速率。

14.1兆电子伏中子的能量大约是裂变中子的10倍,即使是非裂变也非常有效。 重核这些高能裂变平均产生的中子比低能中子产生的裂变多。这使得D-T聚变中子源如所提出的托卡马克动力反应堆可用于变化超铀元素垃圾。14.1兆电子伏中子也可以通过以下方式产生中子将它们从原子核中释放出来。

另一方面,这些高能中子不太可能简单被捕获而不引起裂变或散裂。出于这些原因,核武器设计广泛利用D-T聚变14.1兆电子伏中子来导致更多裂变。聚变中子能够在通常不裂变的材料中引起裂变,例如贫化铀(铀-238),这些材料已经用于热核武器s.聚变中子也可以在不适合或难以制成初级裂变炸弹的物质中导致裂变,例如反应堆级钚。因此,这一物理事实使得普通的非武器级材料在某些情况下变得令人担忧核扩散讨论和条约。

其他聚变反应产生的中子能量要低得多。D-D聚变一半时间产生2.45MeV中子和氦-3 ,其余时间产生氚和质子,但没有中子。$d-^{3}$核聚变不产生中子。
\begin{figure}[ht]
\centering
\includegraphics[width=8cm]{./figures/a0ef5d289d9df2f4.png}
\caption{DT反应速率随熔化温度迅速增加,直到达到最大值,然后逐渐下降。DT速率在较低温度下达到峰值(约70 keV,或8亿开尔文),并且其值高于其他的聚变反应。} \label{fig_Neutro_8}
\end{figure}

\subsubsection{11.6 中能中子}
已经减速但尚未达到热能的裂变能中子被称为超热中子。

捕获和分裂反应的横截面通常在超热能量范围内有多个共振能量峰值。在一个快中子反应堆内,大多数中子在减速到超热能量范围之前被吸收。在一个慢化热反应堆,其中超热中子主要与慢化剂核相互作用,而不是与易裂变的或丰富的锕系核素相互作用 。然而,在超热中子与重金属核相互作用更多的部分慢化反应堆中,反应性的瞬态变化的可能性更大,这使反应堆控制更困难。

大多数原子核燃料的中子俘获裂变比更差,例如钚-239,这导致使用这些燃料的超热中子反应堆不太理想。因为俘获过程不仅是浪费了一个中子,而且导致本应该能够用快中子裂变的核素无法裂变。例外的有铀-233的钍循环,它在任何中子能下具有良好的俘获裂变比。
\begin{figure}[ht]
\centering
\includegraphics[width=8cm]{./figures/11197fc225ce172e.png}
\caption{轻水堆中的嬗变流} \label{fig_Neutro_9}
\end{figure}

\subsubsection{11.7 高能中子}
高能中子比裂变能中子具有更多的能量,并在粒子加速器或在大气宇宙射线中作为次级粒子产生。这些高能中子具有极高的电离效率,比X射线或质子更容易导致细胞死亡。[83][84]

\subsubsection{11.8 中子源}
\begin{itemize}
\item 中子发生器
\item 中子源
\end{itemize}

\subsubsection{11.9 涉及中子的过程}
\begin{itemize}
\item 中子弹
\item 中子衍射
\item 中子通量
\item Neutron transport
\item 宇宙成因放射性核素年代测定
\end{itemize}

\subsection{参考文献}
[1]
^Thomas, A.W.; Weise, W. (2001), The Structure of the Nucleon, Wiley-WCH, Berlin, ISBN 978-3-527-40297-7.

[2]
^Hahn, O. & Strassmann, F. (1939). "Über den Nachweis und das Verhalten der bei der Bestrahlung des Urans mittels Neutronen entstehenden Erdalkalimetalle" [On the detection and characteristics of the alkaline earth metals formed by irradiation of uranium with neutrons]. Die Naturwissenschaften. 27 (1): 11–15. Bibcode:1939NW.....27...11H. doi:10.1007/BF01488241..

[3]
^Glasstone, Samuel; Dolan, Philip J., eds. (1977), The Effects of Nuclear Weapons (3rd ed.), U.S. Dept. of Defense and Energy Research and Development Administration, U.S. Government Printing Office, ISBN 978-1-60322-016-3.

[4]
^Povh, B.; Rith, K.; Scholz, C.; Zetsche, F. (2002). Particles and Nuclei: An Introduction to the Physical Concepts. Berlin: Springer-Verlag. p. 73. ISBN 978-3-540-43823-6..

[5]
^Basdevant, J.-L.; Rich, J.; Spiro, M. (2005). Fundamentals in Nuclear Physics. Springer. p. 155. ISBN 978-0-387-01672-6..

[6]
^Tipler, Paul Allen; Llewellyn, Ralph A. (2002). Modern Physics (4 ed.). Macmillan. p. 310. ISBN 978-0-7167-4345-3..

[7]
^Adair, R.K. (1989). The Great Design: Particles, Fields, and Creation. Oxford University Press. p. 214. Bibcode:1988gdpf.book.....A..

[8]
^Cottingham, W.N.; Greenwood, D.A. (1986). An Introduction to Nuclear Physics. Cambridge University Press. ISBN 9780521657334..

[9]
^Rutherford, E. (1920). "Nuclear Constitution of Atoms". Proceedings of the Royal Society A. 97 (686): 374–400. Bibcode:1920RSPSA..97..374R. doi:10.1098/rspa.1920.0040..

[10]
^Pauli, Wolfgang; Hermann, A.; Meyenn, K.v; Weisskopff, V.F (1985). "Das Jahr 1932 Die Entdeckung des Neutrons". Wolfgang Pauli. Sources in the History of Mathematics and Physical Sciences. 6. pp. 105–144. doi:10.1007/978-3-540-78801-0_3. ISBN 978-3-540-13609-5..

[11]
^Hendry, John, ed. (1984). Cambridge Physics in the Thirties. Bristol: Adam Hilger. ISBN 978-0852747612..

[12]
^Feather, N. (1960). "A history of neutrons and nuclei. Part 1". Contemporary Physics. 1 (3): 191–203. Bibcode:1960ConPh...1..191F. doi:10.1080/00107516008202611..

[13]
^Brown, Laurie M. (1978). "The idea of the neutrino". Physics Today. 31 (9): 23–28. Bibcode:1978PhT....31i..23B. doi:10.1063/1.2995181..

[14]
^弗里兰德、肯尼迪和米勒·j . m .(1964)核能和放射化学(第二版),威利,第22-23页和38-39页.

[15]
^Klein, O. (1929). "Die Reflexion von Elektronen an einem Potentialsprung nach der relativistischen Dynamik von Dirac". Zeitschrift für Physik. 53 (3–4): 157–165. Bibcode:1929ZPhy...53..157K. doi:10.1007/BF01339716..

[16]
^Stuewer, Roger H. (1985). "Niels Bohr and Nuclear Physics". In French, A.P.; Kennedy, P.J. Niels Bohr: A Centenary Volume. Harvard University Press. pp. 197–220. ISBN 978-0674624160..

[17]
^Pais, Abraham (1986). Inward Bound. Oxford: Oxford University Press. p. 299. ISBN 978-0198519973..

[18]
^Bothe, W.; Becker, H. (1930). "Künstliche Erregung von Kern-γ-Strahlen" [Artificial excitation of nuclear γ-radiation]. Zeitschrift für Physik. 66 (5–6): 289–306. Bibcode:1930ZPhy...66..289B. doi:10.1007/BF01390908..

[19]
^Becker, H.; Bothe, W. (1932). "Die in Bor und Beryllium erregten γ-Strahlen" [Γ-rays excited in boron and beryllium]. Zeitschrift für Physik. 76 (7–8): 421–438. Bibcode:1932ZPhy...76..421B. doi:10.1007/BF01336726..

[20]
^Joliot-Curie, Irène & Joliot, Frédéric (1932). "Émission de protons de grande vitesse par les substances hydrogénées sous l'influence des rayons γ très pénétrants" [Emission of high-speed protons by hydrogenated substances under the influence of very penetrating γ-rays]. Comptes Rendus. 194: 273..

[21]
^Brown, Andrew (1997). The Neutron and the Bomb: A Biography of Sir James Chadwick. Oxford University Press. ISBN 978-0-19-853992-6..

[22]
^Chadwick, James (1932). "Possible Existence of a Neutron". Nature. 129 (3252): 312. Bibcode:1932Natur.129Q.312C. doi:10.1038/129312a0..

[23]
^Chadwick, J. (1933). "Bakerian Lecture. The Neutron". Proceedings of the Royal Society A. 142 (846): 1–25. Bibcode:1933RSPSA.142....1C. doi:10.1098/rspa.1933.0152..

[24]
^Heisenberg, W. (1932). "Über den Bau der Atomkerne. I". Zeitschrift für Physik. 77 (1–2): 1–11. Bibcode:1932ZPhy...77....1H. doi:10.1007/BF01342433..

[25]
^Heisenberg, W. (1932). "Über den Bau der Atomkerne. II". Zeitschrift für Physik. 78 (3–4): 156–164. Bibcode:1932ZPhy...78..156H. doi:10.1007/BF01337585..

[26]
^Heisenberg, W. (1933). "Über den Bau der Atomkerne. III". Zeitschrift für Physik. 80 (9–10): 587–596. Bibcode:1933ZPhy...80..587H. doi:10.1007/BF01335696..

[27]
^Iwanenko, D. (1932). "The Neutron Hypothesis". Nature. 129 (3265): 798. Bibcode:1932Natur.129..798I. doi:10.1038/129798d0..

[28]
^Miller A.I. (1995)早期量子电动力学:原始资料剑桥大学出版社,剑桥,ISBN 0521568919,第84-88页。.

[29]
^Wilson, Fred L. (1968). "Fermi's Theory of Beta Decay". American Journal of Physics. 36 (12): 1150–1160. Bibcode:1968AmJPh..36.1150W. doi:10.1119/1.1974382..

[30]
^Chadwick, J.; Goldhaber, M. (1934). "A nuclear photo-effect: disintegration of the diplon by gamma rays". Nature. 134 (3381): 237–238. Bibcode:1934Natur.134..237C. doi:10.1038/134237a0..

[31]
^Chadwick, J.; Goldhaber, M. (1935). "A nuclear photoelectric effect". Proceedings of the Royal Society of London A. 151 (873): 479–493. Bibcode:1935RSPSA.151..479C. doi:10.1098/rspa.1935.0162..

[32]
^Cooper, Dan (1999). Enrico Fermi: And the Revolutions in Modern physics. New York: Oxford University Press. ISBN 978-0-19-511762-2. OCLC 39508200..

[33]
^Hahn, O. (1958). "The Discovery of Fission". Scientific American. 198 (2): 76–84. Bibcode:1958SciAm.198b..76H. doi:10.1038/scientificamerican0258-76..

[34]
^Rife, Patricia (1999). Lise Meitner and the dawn of the nuclear age. Basel, Switzerland: Birkhäuser. ISBN 978-0-8176-3732-3..

[35]
^Hahn, O.; Strassmann, F. (10 February 1939). "Proof of the Formation of Active Isotopes of Barium from Uranium and Thorium Irradiated with Neutrons; Proof of the Existence of More Active Fragments Produced by Uranium Fission". Die Naturwissenschaften. 27 (6): 89–95. Bibcode:1939NW.....27...89H. doi:10.1007/BF01488988..

[36]
^"The Nobel Prize in Chemistry 1944". Nobel Foundation. Retrieved 2007-12-17..

[37]
^Bernstein, Jeremy (2001). Hitler's uranium club: the secret recordings at Farm Hall. New York: Copernicus. p. 281. ISBN 978-0-387-95089-1..

[38]
^"The Nobel Prize in Chemistry 1944: Presentation Speech". Nobel Foundation. Retrieved 2008-01-03..

[39]
^詹姆斯·查德威克爵士发现中子。ANS核能咖啡馆。检索于2012-08-16。.

[40]
^重子粒子数据组汇总数据表。lbl.gov(2007)。检索于2012-08-16。.

[41]
^核物理的基本思想和概念:导论,第三版;海德·Taylor & Francis,2004年。打印ISBN 978-0-7503-0980-6。doi:10.1201/9781420054941。全文.

[42]
^Greene, GL; et al. (1986). "New determination of the deuteron binding energy and the neutron mass". Physical Review Letters. 56 (8): 819–822. Bibcode:1986PhRvL..56..819G. doi:10.1103/PhysRevLett.56.819. PMID 10033294..

[43]
^伯恩,j .中子、原子核和物质米尼奥拉(纽约州) Dover Publications,2011年,ISBN 0486482383,第18-19页.

[44]
^Olive, K.A.; (Particle Data Group); et al. (2014). "Review of Particle Physics" (PDF). Chinese Physics C. 38 (9): 090001. Bibcode:2014ChPhC..38i0001O. doi:10.1088/1674-1137/38/9/090001..

[45]
^Gell, Y.; Lichtenberg, D.B. (1969). "Quark model and the magnetic moments of proton and neutron". Il Nuovo Cimento A. Series 10. 61 (1): 27–40. Bibcode:1969NCimA..61...27G. doi:10.1007/BF02760010..

[46]
^Alvarez, L.W; Bloch, F. (1940). "A quantitative determination of the neutron magnetic moment in absolute nuclear magnetons". Physical Review. 57 (2): 111–122. Bibcode:1940PhRv...57..111A. doi:10.1103/physrev.57.111..

[47]
^Perkins, Donald H. (1982). Introduction to High Energy Physics. Addison Wesley, Reading, Massachusetts. pp. 201–202. ISBN 978-0-201-05757-7..

[48]
^Greenberg, O.W. (2009), "Color charge degree of freedom in particle physics", Compendium of Quantum Physics, Springer Berlin Heidelberg, pp. 109–111, arXiv:0805.0289, doi:10.1007/978-3-540-70626-7_32, ISBN 978-3-540-70622-9.

[49]
^Beg, M.A.B.; Lee, B.W.; Pais, A. (1964). "SU(6) and electromagnetic interactions". Physical Review Letters. 13 (16): 514–517, erratum 650. Bibcode:1964PhRvL..13..514B. doi:10.1103/physrevlett.13.514..

[50]
^Sakita, B. (1964). "Electromagnetic properties of baryons in the supermultiplet scheme of elementary particles". Physical Review Letters. 13 (21): 643–646. Bibcode:1964PhRvL..13..643S. doi:10.1103/physrevlett.13.643..

[51]
^Mohr, P.J.; Taylor, B.N. and Newell, D.B. (2014), "The 2014 CODATA Recommended Values of the Fundamental Physical Constants" (Web Version 7.0). The database was developed by J. Baker, M. Douma, and S. Kotochigova. (2014). National Institute of Standards and Technology, Gaithersburg, Maryland 20899..

[52]
^Wilczek, F. (2003). "The Origin of Mass" (PDF). MIT Physics Annual: 24–35. Retrieved May 8, 2015..

[53]
^Ji, Xiangdong (1995). "A QCD Analysis of the Mass Structure of the Nucleon". Physical Review Letters. 74 (7): 1071–1074. arXiv:hep-ph/9410274. Bibcode:1995PhRvL..74.1071J. doi:10.1103/PhysRevLett.74.1071. PMID 10058927..

[54]
^Martinelli, G.; Parisi, G.; Petronzio, R.; Rapuano, F. (1982). "The proton and neutron magnetic moments in lattice QCD". Physics Letters B. 116 (6): 434–436. Bibcode:1982PhLB..116..434M. doi:10.1016/0370-2693(82)90162-9. Retrieved May 8, 2015..

[55]
^Kincade, Kathy (2 February 2015). "Pinpointing the magnetic moments of nuclear matter". Phys.org. Retrieved May 8, 2015..

[56]
^J. Byrne (2011). Neutrons, Nuclei and Matter: An Exploration of the Physics of Slow Neutrons. Mineola, New York: Dover Publications. pp. 28–31. ISBN 978-0486482385..

[57]
^Hughes, D.J.; Burgy, M.T. (1949). "Reflection and polarization of neutrons by magnetized mirrors" (PDF). Physical Review. 76 (9): 1413–1414. Bibcode:1949PhRv...76.1413H. doi:10.1103/PhysRev.76.1413..

[58]
^Sherwood, J.E.; Stephenson, T.E.; Bernstein, S. (1954). "Stern-Gerlach experiment on polarized neutrons". Physical Review. 96 (6): 1546–1548. Bibcode:1954PhRv...96.1546S. doi:10.1103/PhysRev.96.1546..

[59]
^Miller, G.A. (2007). "Charge Densities of the Neutron and Proton". Physical Review Letters. 99 (11): 112001. arXiv:0705.2409. Bibcode:2007PhRvL..99k2001M. doi:10.1103/PhysRevLett.99.112001. PMID 17930428..

[60]
^"Pear-shaped particles probe big-bang mystery" (Press release). University of Sussex. 20 February 2006. Retrieved 2009-12-14..

[61]
^寻找中子电火花加工的低温实验。hewww . rl . AC . uk .检索于2012-08-16。.

[62]
^寻找中子电偶极矩:nEDM 。Nedm.web.psi.ch (2001-09-12)。检索于2012-08-16。.

[63]
^美国内德姆·ORNL实验公共版。检索于2017-02-08。.

[64]
^SNS中子电火花加工实验 Archived 2011-02-10 at the Wayback Machine。P25ext.lanl.gov。检索于2012-08-16。.

[65]
^中子电偶极矩测量。Nrd.pnpi.spb.ru .检索于2012-08-16。.

[66]
^Nakamura, K (2010). "Review of Particle Physics". Journal of Physics G. 37 (7A): 075021. Bibcode:2010JPhG...37g5021N. doi:10.1088/0954-3899/37/7A/075021. PDF,2012年版的2011年部分更新 由于实验结果相互矛盾,平均寿命的确切值仍然不确定。 粒子数据组报告的值间隔高达6秒(超过4个标准偏差),并评论道“我们2006、2008和2010年的评论保持在885.7±0.8秒;但是我们注意到根据SEREBROV 05,我们的价值应该被认为是可疑的,直到进一步的实验澄清了问题。自2010年审查以来,PICHLMAIER 10的平均寿命为880.7±1.8秒,比我们的平均寿命更接近SEREBROV 05的值。和塞里布罗夫10B[...]声称它们的值应该降低大约6秒,这将使它们与两个较低的值一致。然而,这些重新评估没有得到相关实验者的热情回应;在任何情况下,粒子数据组将不得不等待(由那些实验者)公布的值的变化。 在这一点上,我们认为没有什么比平均七个最好但不一致的测量值更好的了,得到881.5±1.5s。请注意,误差包括2.7的比例因子。这是4.2个旧(和2.8个新)标准差的跳跃。这种情况尤其令人不快,因为价值如此重要。我们再次呼吁实验者澄清这一点。".

[67]
^"Physicists find signs of four-neutron nucleus". 2016-02-24..

[68]
^Orr, Nigel (2016-02-03). "Can Four Neutrons Tango?". Physics. 9. Retrieved 2016-04-11..

[69]
^Spyrou, A.; et al. (2012). "First Observation of Ground State Dineutron Decay: 16Be". Physical Review Letters. 108 (10): 102501. Bibcode:2012PhRvL.108j2501S. doi:10.1103/PhysRevLett.108.102501. PMID 22463404..

[70]
^Llanes-Estrada, Felipe J.; Moreno Navarro, Gaspar (2012). "Cubic neutrons". Modern Physics Letters A. 27 (6): 1250033–1–1250033–7. arXiv:1108.1859. Bibcode:2012MPLA...2750033L. doi:10.1142/S0217732312500332..

[71]
^Knoll, Glenn F. (1979). "Ch. 14". Radiation Detection and Measurement. John Wiley & Sons. ISBN 978-0471495451..

[72]
^Carson, M.J.; et al. (2004). "Neutron background in large-scale xenon detectors for dark matter searches". Astroparticle Physics. 21 (6): 667–687. arXiv:hep-ex/0404042. Bibcode:2004APh....21..667C. doi:10.1016/j.astropartphys.2004.05.001..

[73]
^Köhn, C.; Diniz, G.; Harakeh, Muhsin (2017). "Production mechanisms of leptons, photons, and hadrons and their possible feedback close to lightning leaders". Journal of Geophysical Research: Atmospheres. 122 (2): 1365–1383. Bibcode:2017JGRD..122.1365K. doi:10.1002/2016JD025445. PMC 5349290. PMID 28357174..

[74]
^Clowdsley, MS; Wilson, JW; Kim, MH; Singleterry, RC; Tripathi, RK; Heinbockel, JH; Badavi, FF; Shinn, JL (2001). "Neutron Environments on the Martian Surface" (PDF). Physica Medica. 17 (Suppl 1): 94–96. PMID 11770546. Archived from the original (PDF) on 2005-02-25..

[75]
^伯恩,j .中子、原子核和物质米尼奥拉(纽约州) Dover Publications,2011年,ISBN 0486482383,第32-33页。.

[76]
^科学/自然|问答;答:核聚变反应堆。英国广播公司新闻(2006-02-06)。检索于2010年12月4日。.

[77]
^伯恩,j .中子、原子核和物质米尼奥拉(纽约州) Dover Publications,2011年,ISBN 0486482383,第453页。.

[78]
^Kumakhov, M.A.; Sharov, V.A. (1992). "A neutron lens". Nature. 357 (6377): 390–391. Bibcode:1992Natur.357..390K. doi:10.1038/357390a0..

[79]
^Physorg.com,“新的‘观察方式’:中子显微镜”。Physorg.com(2004-07-30)。检索于2012-08-16。.

[80]
^美国宇航局开发了一个掘金来寻找太空生命“”。NASA.gov(2007-11-30)。检索于2012-08-16。.

[81]
^霍尔·EJ(2000)。放射科医师的放射生物学。利平科特·威廉姆斯和威尔金斯;第五版.

[82]
^约翰·何和小坎宁安(1978)。放射学物理学。查尔斯·托马斯第三版.

[83]
^Freeman, Tami (May 23, 2008). "Facing up to secondary neutrons". Medical Physics Web. Archived from the original on 2010-12-20. Retrieved 2011-02-08..

[84]
^Heilbronn, L.; Nakamura, T; Iwata, Y; Kurosawa, T; Iwase, H; Townsend, LW (2005). "Expand+Overview of secondary neutron production relevant to shielding in space". Radiation Protection Dosimetry. 116 (1–4): 140–143. doi:10.1093/rpd/nci033. PMID 16604615..