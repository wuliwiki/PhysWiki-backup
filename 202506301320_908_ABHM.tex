% 罗伯特·奥本海默(综述)
% license CCBYSA3
% type Wiki

本文根据 CC-BY-SA 协议转载翻译自维基百科\href{https://en.wikipedia.org/wiki/J._Robert_Oppenheimer}{相关文章}。

\begin{figure}[ht]
\centering
\includegraphics[width=6cm]{./figures/0ac14d0320e1b5df.png}
\caption{} \label{fig_ABHM_1}
\end{figure}
J·罗伯特·奥本海默(出生名朱利叶斯·罗伯特·奥本海默,/ˈɒpənhaɪmər/,1904年4月22日-1967年2月18日),是一位美国理论物理学家,曾在第二次世界大战期间担任曼哈顿计划洛斯阿拉莫斯实验室的主任。他因在监督第一批核武器的研发中所扮演的角色而常被称为“原子弹之父”。

奥本海默出生于纽约市,1925年在哈佛大学获得化学学位,1927年在德国哥廷根大学师从马克斯·玻恩获得物理学博士学位。在其他机构从事研究后,他加入加利福尼亚大学伯克利分校物理系,并于1936年成为正教授。奥本海默在量子力学和核物理领域作出了重要贡献,包括提出用于分子波函数的玻恩–奥本海默近似;在正电子理论、量子电动力学和量子场论方面的工作;以及核聚变中的奥本海默–菲利普斯过程。他与学生们还在天体物理领域做出了重要贡献,包括宇宙射线簇射理论,以及中子星和黑洞理论。

1942年,奥本海默被招募参与曼哈顿计划,并于1943年被任命为该计划位于新墨西哥州的洛斯阿拉莫斯实验室主任,负责研发第一批核武器。他的领导能力和科学专长对计划的成功起到了关键作用。1945年7月16日,他出席了代号“三位一体”的首次原子弹试爆。1945年8月,这些核武器在广岛和长崎的原子弹轰炸中被用于对日本作战,这也是迄今为止核武器在战争中唯一一次被使用。

1947年,奥本海默被任命为新泽西州普林斯顿高等研究院院长,并出任新成立的美国原子能委员会(AEC)总顾问委员会主席。他主张对核能和核武器实行国际管控,以避免与苏联陷入军备竞赛,后来又出于部分道德原因反对氢弹的研发。在第二次红色恐慌期间,他的这些立场,加上他过去与美国共产党有过的联系,导致1954年美国原子能委员会对其进行安全听证,并最终撤销了他的安全许可。此后,他继续从事物理学领域的讲学、写作和研究工作,并于1963年因其对理论物理的贡献获得恩里科·费米奖。2022年12月16日,美国能源部长詹妮弗·格兰霍姆撤销了1954年的决定,称当时的决定是“有缺陷的程序”的结果,并确认奥本海默一直是忠诚的。
\subsection{早年生活}
\subsubsection{童年与教育}
奥本海默于1904年4月22日出生在纽约市的一个不虔诚犹太家庭,出生名朱利叶斯·罗伯特·奥本海默\(^\text{[note 1]}\)。母亲埃拉(娘家姓弗里德曼)是一位画家,父亲朱利叶斯·塞利格曼·奥本海默是一位成功的纺织品进口商。\(^\text{[5][6]}\)罗伯特有一个弟弟弗兰克,后来也成为物理学家。\(^\text{[7]}\)他的父亲出生在普鲁士王国黑森-拿骚省仍属于哈瑙时,1888年青少年时期只身前往美国,身无分文,没有高等教育,甚至不会英语。他被一家纺织公司雇用,并在十年内成为公司高管,最终积累了财富。\(^\text{[8]}\)1912年,家族搬到纽约哈德逊高地西88街附近的河滨大道的一套公寓,那一带以豪华的府邸和联排别墅闻名。\(^\text{[6]}\)他们的艺术收藏包括巴勃罗·毕加索、爱德华·维亚尔和文森特·梵高的作品。\(^\text{[9]}\)

奥本海默最初在阿尔奎因预备学校接受教育。1911年,他进入由费利克斯·阿德勒创办的伦理文化学会学校,\(^\text{[10]}\)该校以伦理运动为基础进行教育,其校训是“行为重于信仰”。奥本海默的父亲多年是该学会成员,并担任董事会成员。\(^\text{[11]}\)奥本海默是一名兴趣广泛的学生,热衷于英语和法语文学,特别喜欢矿物学。\(^\text{[12]}\)他在一年内完成了三、四年级课程,并跳过了八年级的一半。\(^\text{[10]}\)他还向著名法国长笛演奏家乔治·巴雷尔私下学习音乐。在学业最后一年,奥本海默开始对化学产生兴趣。\(^\text{[13]}\)1921年毕业,但在捷克斯洛伐克的家族度假期间,他在雅希莫夫探矿时感染了结肠炎,因此耽误了一年继续深造。他在新墨西哥州康复,并在那里爱上了骑马和美国西南部地区。\(^\text{[14]}\)

1922年,18岁的奥本海默进入哈佛学院。他主修化学;哈佛还要求学习历史、文学、哲学或数学。为了弥补因疾病耽误的时间,他每学期修六门课程,而非通常的四门。他被接纳为本科荣誉学会Phi Beta Kappa成员,并因独立学习被授予物理学研究生资格,使他能够跳过基础课程直接修读高级课程。他因珀西·布里奇曼教授的热力学课程而被实验物理学吸引。奥本海默仅用三年时间,于1925年以优等生身份从哈佛获得文学士学位毕业。\(^\text{[15]}\)
\subsubsection{在欧洲的求学经历}
\begin{figure}[ht]
\centering
\includegraphics[width=6cm]{./figures/bfcfea0693f0bc9c.png}
\caption{1927年7月,荷兰莱顿,海克·卡末林·昂内斯实验室。奥本海默位于中排,从左数第二位} \label{fig_ABHM_2}
\end{figure}
1924年,奥本海默被剑桥大学基督学院录取后,曾致信欧内斯特·卢瑟福,请求允许他在卡文迪许实验室工作,尽管布里奇曼在推荐信中指出,奥本海默在实验室的笨拙表明,相较于实验物理,他更适合从事理论物理研究。卢瑟福对此并不感兴趣,但奥本海默仍然前往剑桥求学;\(^\text{[16]}\)最终,J.J. 汤姆森同意接收他,但条件是他必须完成基础实验课程。\(^\text{[17]}\)

奥本海默在剑桥期间非常不开心,他曾写信给一位朋友说:“我现在过得相当糟糕。实验室的工作无聊透顶,而我做得又极差,以至于完全感觉不到自己学到了任何东西。”\(^\text{[18]}\)他与导师帕特里克·布莱基特(未来的诺贝尔奖得主)关系紧张。根据奥本海默的朋友弗朗西斯·弗格森的说法,奥本海默曾坦白说他曾在布莱基特的办公桌上放过一个涂了毒药的苹果,后来奥本海默的父母说服校方不将他开除。虽然并没有关于投毒事件或留校察看的官方记录,但奥本海默曾定期在伦敦哈雷街接受精神科医生的治疗。\(^\text{[19][20][21][22]}\)

奥本海默身材高瘦,是个抽烟成瘾的人,\(^\text{[23]}\)在专注时常常忘记进食。许多朋友都说他有自我毁灭的倾向。弗格森曾试图通过讲述自己向女友弗朗西丝·基利求婚的故事,分散奥本海默明显的抑郁情绪,但奥本海默突然跳向弗格森,试图掐死他。奥本海默一生都受到抑郁症的困扰,\(^\text{[24][25]}\)曾对弟弟说:“我需要物理,甚于朋友。”\(^\text{[26]}\)

1926年,奥本海默离开剑桥,前往哥廷根大学师从马克斯·玻恩学习;当时哥廷根是全球理论物理学的中心之一。奥本海默在此结识了后来取得巨大成就的朋友,包括维尔纳·海森堡、帕斯卡尔·约尔当、沃尔夫冈·泡利、保罗·狄拉克、恩里科·费米和爱德华·泰勒。他在讨论中非常热情,以至于有时会主导整个讨论。\(^\text{[27]}\)玛丽亚·格佩特曾向玻恩递交了一份由她和其他人签署的请愿书,威胁如果玻恩不让奥本海默安静下来,他们将抵制这门课。玻恩把请愿书放在桌上,让奥本海默看见,却一句话都没说,这一举动收到了预期效果。\(^\text{[28]}\)

1927年3月,年仅23岁的奥本海默在玻恩的指导下获得了哲学博士学位。\(^\text{[29][30]}\)据说在口试结束后,主持口试的詹姆斯·弗兰克教授说道:“我很高兴终于结束了,他差点开始反过来考我。”\(^\text{[31]}\)奥本海默在欧洲期间发表了十余篇论文,包括在量子力学这一新兴领域中的许多重要贡献。他与玻恩共同发表了一篇著名论文,提出了玻恩–奥本海默近似,将分子数学处理中核的运动与电子运动分离,使得在计算时可以忽略核的运动以简化计算。这篇论文至今仍是他引用次数最多的学术成果。\(^\text{[32]}\)
\subsection{早期职业生涯}
\subsubsection{教学工作}
\begin{figure}[ht]
\centering
\includegraphics[width=8cm]{./figures/71ea0980a3ef4c69.png}
\caption{} \label{fig_ABHM_3}
\end{figure}
1927年9月,奥本海默获得了美国国家研究委员会颁发的奖学金,前往加州理工学院从事研究。布里奇曼也希望他能留在哈佛,因此达成了一个折中方案,即在1927–1928学年中,他将奖学金时间分为两段,1927年在哈佛,1928年前往加州理工学院。\(^\text{[33]}\)

在加州理工期间,奥本海默与莱纳斯·鲍林建立了密切的友谊;他们计划联合攻克化学键本质这一领域,鲍林是该领域的先驱,计划由奥本海默提供数学支持,鲍林负责解释结果。然而,这一合作以及两人的友谊在奥本海默邀请鲍林的妻子艾娃·海伦·鲍林一起前往墨西哥幽会后宣告结束。\(^\text{[34]}\)

后来,奥本海默邀请鲍林出任曼哈顿计划化学部主任,但鲍林以自己是和平主义者为由拒绝了邀请。\(^\text{[35]}\)

1928年秋,奥本海默访问了荷兰莱顿大学保罗·厄恩费斯特的研究所,他在几乎没有语言经验的情况下,用荷兰语讲课给大家留下了深刻印象。在那里,他获得了“Opje”的绰号,\(^\text{[36]}\)后来他的学生将其英文化为“Oppie”。\(^\text{[37]}\)从莱顿,他继续前往瑞士联邦理工学院(ETH Zürich),与沃尔夫冈·泡利一起从事量子力学和连续谱的研究。奥本海默尊敬并喜爱泡利,可能还模仿了泡利的个人风格以及他对问题的批判性处理方式。\(^\text{[38]}\)

回到美国后,奥本海默接受了加利福尼亚大学伯克利分校的副教授职位,雷蒙德·塞耶·伯吉极力邀请他,甚至表示愿意与加州理工学院共享他。\(^\text{[35]}\)

在开始伯克利的教授职务之前,奥本海默被诊断出患有轻度肺结核,并在新墨西哥州的一处牧场上与弟弟弗兰克一起度过了几周时间,他租下了这片牧场,并最终购买了它。当他得知牧场可供租赁时,他兴奋地喊道:“热狗!”后来,他将其命名为Perro Caliente(西班牙语中的“热狗”)。\(^\text{[39]}\)后来他常说,“物理学和沙漠”是他“两个伟大的爱”。\(^\text{[40]}\)他从肺结核中恢复后返回伯克利,在那里作为顾问和合作者,与一代物理学家共同发展,这些物理学家钦佩他卓越的才智和广泛的兴趣。他的学生和同事认为他具有迷人的魅力:在私人互动中具有催眠般的吸引力,但在更公开的场合中则常显得冷淡。他的同事们对他有两种看法:一种认为他是一个冷漠且令人敬畏的天才和美学家,另一种认为他是一个自命不凡且不安的装腔作势者。\(^\text{[41]}\)他的学生几乎总是认为他是前者,模仿他的步伐、言语以及其他举止,甚至他倾向于以原始语言阅读完整的文本。\(^\text{[42]}\)汉斯·贝特曾这样评价他:

他带到教学中最重要的因素可能就是他那高雅的品味。他总是知道什么是重要的问题,这从他选择的课题中可以看出来。他真正与这些问题同在,为解决这些问题而努力,并把他的关注传达给了学生群体。在他最鼎盛时期,他的研究小组大约有八到十个研究生和六名博士后。他每天都会在办公室与这些人见面,逐一讨论学生研究问题的进展。他对一切都感兴趣,某个下午他们可能会讨论量子电动力学、宇宙射线、电子对产生和核物理等问题。\(^\text{[43]}\)

奥本海默与诺贝尔奖得主、实验物理学家欧内斯特·劳伦斯以及他的一些回旋加速器开创者紧密合作,帮助他们理解他们的机器在伯克利辐射实验室产生的数据,这些数据最终发展成了今天的劳伦斯伯克利国家实验室。\(^\text{[44]}\)1936年,伯克利大学将奥本海默晋升为正教授,年薪为3300美元(相当于2024年的75,000美元)。作为交换,伯克利要求他减少在加州理工学院的教学时间,因此达成了一个折中的方案,即伯克利每年允许他有六周时间外出,这足以让他在加州理工学院教一个学期的课程。\(^\text{[45]}\)

奥本海默曾多次试图为罗伯特·瑟伯争取伯克利的职位,但遭到了伯吉的阻挠,伯吉认为“系里一个犹太人已经够了”。\(^\text{[46]}\)
\subsubsection{科学研究}
奥本海默在理论天文学(尤其是与广义相对论和核理论相关的领域)、核物理学、光谱学和量子场论方面做出了重要研究,包括其在量子电动力学中的扩展。他也对相对论性量子力学的形式数学产生了兴趣,尽管他对其有效性表示怀疑。他的工作预测了许多后来的发现,包括中子、介子和中子星。[47]

最初,他的主要兴趣是连续谱的理论。他的第一篇公开发表的论文发表于1926年,内容是分子带状光谱的量子理论。他开发了一种方法来计算其跃迁几率。他还计算了氢和X射线的光电效应,得出了K边的吸收系数。他的计算结果与对太阳的X射线吸收观测结果一致,但与氦的观测结果不符。多年后,人们意识到太阳主要由氢组成,他的计算是正确的。[48][49]

奥本海默在宇宙射线簇射理论方面做出了重要贡献。他还研究了场电子发射问题。[50][51] 这项工作为量子隧穿效应的概念发展做出了贡献。\(^\text{[52]}\)1931年,他与学生哈维·霍尔共同撰写了一篇论文《光电效应的相对论理论》,\(^\text{[53]}\)在该文中,基于实验证据,他正确地反驳了保罗·狄拉克关于氢原子两能级具有相同能量的说法。随后,他的一位博士生威利斯·兰姆确定这一现象是后来被称为兰姆位移的结果,兰姆因此获得了1955年诺贝尔物理学奖。\(^\text{[47]}\)

与梅尔巴·菲利普斯(奥本海默的第一位博士生)合作,奥本海默研究了氘离子轰击下的人工放射性计算。欧内斯特·劳伦斯和埃德温·麦克米兰用氘离子轰击原子核,发现结果与乔治·伽莫夫的预测非常吻合,但当涉及到更高的能量和更重的原子核时,结果则不符合预期。1935年,奥本海默与菲利普斯共同提出了一个理论——后来称为奥本海默–菲利普斯过程——以解释这些结果。这个理论至今仍在使用。[55][note 3]

早在1930年,奥本海默就写了一篇论文,基本上预测了正电子的存在。这是在狄拉克提出电子既可能具有正电荷也可能具有负能量的论文之后。狄拉克的论文引入了后来被称为狄拉克方程的方程式,统一了量子力学、特殊相对论和当时的新概念——电子自旋,以解释泽曼效应。\(^\text{[57]}\)在大量实验数据的支持下,奥本海默拒绝了将预测中的正电荷电子解释为质子的观点。他认为这些粒子必须与电子具有相同的质量,而实验表明质子要比电子重得多。两年后,卡尔·大卫·安德森发现了正电子,因此获得了1936年诺贝尔物理学奖。\(^\text{[58]}\)

在1930年代末,奥本海默开始对天体物理学产生兴趣,这很可能是通过与理查德·托尔曼的友谊促成的,之后他发表了一系列相关论文。在其中的第一篇论文《恒星中子核的稳定性》(1938年),\(^\text{[59]}\)他与瑟伯共同探讨了白矮星的性质。接着,他与其中一位学生乔治·沃尔科夫共同撰写了《大质量中子核的研究》,\(^\text{[60]}\)该论文证明了恒星质量有一个上限,称为托尔曼–奥本海默–沃尔科夫极限,超出该极限的恒星将无法保持中子星的稳定性,最终会发生引力崩塌。1939年,奥本海默与他的另一位学生哈特兰·斯奈德共同发表了论文《持续引力收缩》,\(^\text{[61]}\)该论文预测了后来被称为黑洞的天体的存在。继玻恩–奥本海默近似论文之后,这些论文仍是他被引用最多的作品,并且在1950年代美国天体物理学研究复兴中起到了关键作用,主要是由约翰·A·惠勒推动的。\(^\text{[62]}\)

奥本海默的论文即使按照他擅长的抽象主题的标准,也被认为很难理解。他喜欢使用优雅但极为复杂的数学技巧来展示物理原理,尽管有时因急于求成而犯数学错误,受到批评。“他的物理学很好,”他的学生斯奈德说,“但他的算术糟糕透了。”\(^\text{[47]}\)

第二次世界大战后,奥本海默仅发表了五篇科学论文,其中一篇是生物物理学方面的,且在1950年之后再也没有发表过论文。后来的诺贝尔奖得主、曾作为访问学者在1951年与他一起在高等研究院工作的穆雷·盖尔曼对此表达了这样的看法:

他没有“坐得住”,即“坐在椅子上的肉”,我知道他从未写过长篇论文或做过长时间的计算之类的事。他没有耐心去做这些;他自己的工作由一些简短的见解构成,但都是相当精彩的。不过,他激励了其他人去做事情,他的影响力是巨大的。\(^\text{[63]}\)
\subsection{私人生活与政治生涯}
\begin{figure}[ht]
\centering
\includegraphics[width=6cm]{./figures/bbe07c2cafdc3619.png}
\caption{1946年的奥本海默} \label{fig_ABHM_4}
\end{figure}
奥本海默的母亲于1931年去世,此后他与父亲的关系更加亲密。尽管父亲仍住在纽约,但他经常前往加利福尼亚探访。\(^\text{[64]}\)1937年,父亲去世,遗留下392,602美元(相当于2024年的860万美元),这笔遗产将在奥本海默和他的弟弟弗兰克之间分配。奥本海默立即立下遗嘱,将自己的遗产捐赠给加利福尼亚大学,用于研究生奖学金。\(^\text{[65]}\)
\subsubsection{政治}

在1920年代,奥本海默对世界事务保持不关注。他声称自己不看报纸或流行杂志,直到1929年华尔街股灾发生六个月后,他和欧内斯特·劳伦斯散步时才得知此事。[66][67] 他曾说过,在1936年总统选举之前,他从未投过票。从1934年起,他对政治和国际事务变得越来越关注。1934年,他将年薪的3\%——大约100美元(相当于2024年的2400美元)——作为两年的资助,用于支持逃离纳粹德国的德国物理学家。\(^\text{[68]}\)在1934年西海岸码头工人罢工期间,他与一些学生,包括梅尔巴·菲利普斯和瑟伯,参加了码头工人的集会。\(^\text{[46]}\)

1936年西班牙内战爆发后,奥本海默为西班牙共和党事业主持募捐活动。1939年,他加入了美国民主与知识自由委员会,积极参与反对纳粹德国对犹太科学家的迫害。像那个时代的大多数自由派团体一样,该委员会后来被贴上了共产主义阵线的标签。\(^\text{[68]}\)

奥本海默许多最亲近的同事在1930年代或1940年代都曾活跃于共产党,包括他的弟弟弗兰克、弗兰克的妻子杰基、基蒂、简·塔特洛克、他的房东玛丽·埃伦·沃什本以及伯克利的几位研究生。[69][70][71][72] 奥本海默是否是共产党成员一直存在争议。卡西迪指出,他从未公开加入美国共产党(CPUSA),\(^\text{[68]}\)但海恩斯、克莱尔和瓦西里耶夫则认为,他“实际上在1930年代末是美国共产党一个隐秘成员”。\(^\text{[73]}\)从1937年到1942年,奥本海默在伯克利参与了一个他称之为“讨论小组”的组织,后来小组成员哈孔·谢瓦利耶和戈登·格里菲斯表示,这个小组是伯克利教职员工中的一个“封闭” (秘密)单位,是美国共产党的一部分。\(^\text{[76]}\)

1941年3月,联邦调查局(FBI)为奥本海默建立了档案。档案记录了他参加了1940年12月在谢瓦利耶家中举行的会议,会议上还在场的有加州共产党州秘书威廉·施奈德曼和财务主管艾萨克·福尔科夫。FBI注意到奥本海默是美国公民自由联盟的执行委员会成员,而该组织被认为是一个共产主义阵线组织。不久后,FBI将奥本海默列入其“拘留名单”,以便在国家紧急状态时进行逮捕。\(^\text{[77]}\)

1942年,奥本海默加入曼哈顿计划时,他在个人安全调查问卷中写道,他曾是“几乎所有西海岸共产主义阵线组织的成员。”\(^\text{[78]}\)许多年后,他声称不记得写过这句话,称其不真实,如果他写过类似的内容,那只是“半开玩笑的夸张”。\(^\text{[79]}\)他曾是《人民世界》的订阅者,\(^\text{[80]}\)这是共产党的一份机关报,他在1954年作证时表示:“我曾与共产主义运动有过联系。”\(^\text{[81]}\)

1953年,奥本海默是由反共文化组织文化自由大会主办的“科学与自由”会议的赞助委员会成员。\(^\text{[82]}\)

在1954年的安全许可听证会上,奥本海默否认自己是共产党员,但他认同自己是“共同行动者”,并将这一身份定义为一个同意共产主义许多目标,但不愿意盲目听从任何共产党组织命令的人。\(^\text{[83]}\)根据传记作家雷·蒙克的说法:“从一个非常实际和真实的角度来看,他是共产党的支持者。此外,考虑到他在党活动中投入的时间、精力和金钱,他是一个非常坚定的支持者。”\(^\text{[84]}\)
\subsubsection{关系与子女}
1936年,奥本海默与简·塔特洛克建立了关系,简是伯克利大学一位文学教授的女儿,也是斯坦福大学医学院的学生。两人有着相似的政治观点;她为《西方工人报》写稿,这是美国共产党的一份报纸。\(^\text{[85]}\)1939年,在一段风波不断的关系后,塔特洛克与奥本海默分手。同年8月,他遇到了凯瑟琳(“基蒂”)普宁,一位曾经的共产党成员。基蒂的第一次婚姻仅持续了几个月。她与第二任同居丈夫乔·达利特的关系从1934年持续到1937年,乔·达利特是共产党的活跃成员,并在1937年西班牙内战中丧生。\(^\text{[86]}\)

基蒂从欧洲返回美国,在宾夕法尼亚大学获得了植物学学士学位。1938年,她与医生兼医学研究员理查德·哈里森结婚,并于1939年6月随他搬到加利福尼亚州帕萨迪纳,在那里理查德成为一家当地医院的放射科主任,而她则在加利福尼亚大学洛杉矶分校注册为研究生。她和奥本海默在托尔曼的一次聚会后发生了短暂的丑闻,他们在一起过夜。1940年夏天,她在奥本海默的新墨西哥州牧场与他同住。当她怀孕时,基蒂向理查德要求离婚,理查德同意了。1940年11月1日,她在内华达州雷诺市迅速办理了离婚手续,并与奥本海默结婚。[87][88]

他们的第一个孩子彼得于1941年5月出生,第二个孩子凯瑟琳(“托尼”)于1944年12月7日在新墨西哥州洛斯阿拉莫斯出生。\(^\text{[89]}\)在婚姻期间,奥本海默重新点燃了与塔特洛克的恋情。\(^\text{[90]}\)后来,由于塔特洛克与共产主义的关联,这段持续的联系成为了奥本海默安全审查听证会中的一个问题。\(^\text{[91]}\)

在原子弹的开发过程中,奥本海默因其过去的左翼关系而受到联邦调查局和曼哈顿计划内部安全部门的调查。1943年6月,他前往加利福尼亚探望正在遭受抑郁症困扰的塔特洛克时,受到军方安全人员的跟踪。奥本海默在她的公寓里过夜。\(^\text{[92]}\)1944年1月4日,塔特洛克自杀,奥本海默为此深感悲痛。\(^\text{[93]}\)

在洛斯阿拉莫斯,奥本海默与心理学家、朋友理查德·托尔曼的妻子鲁思·托尔曼发生了感情纠葛。此段关系在奥本海默回到东部担任高等研究院院长后结束,但在理查德于1948年8月去世后,他们重新建立了联系,直到鲁思于1957年去世,两人偶尔见面。虽然他们的信件大部分已丢失,但现存的信件反映了他们之间亲密而充满感情的关系,奥本海默称她为“我的爱”。[94][95]
\subsubsection{神秘主义}
奥本海默的广泛兴趣有时会干扰他对科学的专注。他喜欢那些困难的事情,因为许多科学工作对他来说似乎很简单,他开始对神秘学和隐秘的事物产生兴趣。\(^\text{[97]}\)在哈佛大学学习后,他通过英文翻译开始接触经典的印度教经典。\(^\text{[98]}\)他还对学习语言感兴趣,并在1933年在伯克利大学向亚瑟·W·赖德学习梵语。[100][101] 他最终阅读了《博伽梵歌》和《梅格杜塔》等文学作品,并深入思考它们。他后来提到《博伽梵歌》是最深刻影响他人生哲学的书籍之一。[102][103] 他写信给弟弟说,《博伽梵歌》“非常简单,且十分神奇”。他后来称它为“任何已知语言中最美丽的哲学歌曲”,并将其复印本作为礼物送给朋友,还将一本个人使用的、破旧的复印本放在他桌子旁的书架上。他在指导洛斯阿拉莫斯实验室时不断引用它,并在弗兰克林·罗斯福总统的纪念服务上引用了《博伽梵歌》中的一段话。[104][101] 他给自己的车起名为“迦楼罗”,这是印度教神祇毗湿奴的坐骑鸟。\(^\text{[105]}\)

奥本海默从未在传统意义上成为印度教徒;他没有加入任何寺庙,也没有向任何神祈祷。[106][107] 他的弟弟说:“他真的被《博伽梵歌》的魅力和其中的一般智慧吸引了。”[106] 有人推测,奥本海默对印度教思想的兴趣始于他早期与尼尔斯·玻尔的交往。玻尔和奥本海默都对古代印度神话故事及其中蕴含的形而上学进行了深入的分析和批判。在一次与大卫·霍金斯的对话中,战争前,奥本海默谈论古希腊文学时说道:“我读过希腊人;我发现印度人更深刻。”[108] 奥本海默曾是《世界视野》书系的编辑委员会成员,该书系出版了多种关于哲学的书籍。[109] 在1930年代,当他在伯克利教书时,奥本海默成为了湾区一群人的一部分,这个小组由心理学家齐格弗里德·伯恩费尔德召集,讨论精神分析。[110]

他的亲密朋友和同事伊西多尔·艾萨克·拉比曾看到奥本海默度过伯克利、洛斯阿拉莫斯和普林斯顿的岁月,他曾困惑地想:“为什么像奥本海默这样的天才人物不会发现所有值得发现的东西?”[111] 拉比反思道:

奥本海默在那些科学传统之外的领域受到了过多的教育,比如他对宗教,尤其是印度教的兴趣,这使得他对宇宙的神秘感有了深刻的体悟,这种感觉几乎像雾霾一样笼罩着他。他看待物理学非常清晰,关注已做的工作,但在边界处,他倾向于觉得存在比实际更多神秘和新奇的东西……[他转向] 远离理论物理学的艰难、粗糙的方法,进入了一个广泛直觉的神秘领域……在奥本海默身上,现实感的元素是薄弱的。然而,正是这种精神品质,正如他在言语和举止中的精致表现,是他个人魅力的基础。他从未完全表达自己,总是给人一种感觉,似乎有尚未揭示的深度敏感性和洞察力。这些可能是天生领导者的品质,似乎拥有未被利用的力量储备。[112]

尽管如此,物理学家路易斯·阿尔瓦雷兹和杰里米·伯恩斯坦等观察者曾建议,如果奥本海默能活到足够长的时间,以便看到他的预测通过实验得到了验证,他可能会因他在引力塌缩方面的工作(涉及中子星和黑洞)而获得诺贝尔奖。[113][114][115] 回顾过去,一些物理学家和历史学家认为这是他最重要的贡献,尽管在他生前,其他科学家并未深入研究这一领域。[116] 物理学家和历史学家亚伯拉罕·佩斯曾问奥本海默他认为自己最重要的科学贡献是什么——奥本海默提到了他在电子和正电子方面的工作,而不是他在引力收缩方面的研究。[117] 奥本海默四次被提名诺贝尔物理学奖,分别是在1946年、1951年、1955年和1967年,但他从未获奖。[118][119]