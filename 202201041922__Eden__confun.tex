% 函数的连续与间断
% 连续函数|间断点

\pentry{函数极限的性质\upref{limff}}

函数的连续与间断

\begin{definition}{左连续,右连续,连续}
  若函数 $f(x)$ 在 $U(x_0,\delta_0)$ 内有定义,且 $\lim\limits_{x\rightarrow x_0}f(x)=f(x_0)$,则称 $f(x)$ 在 $x_0$ \textbf{连续},$x_0$ 为 $f(x)$ 的一个\textbf{连续点}.

  若函数 $f(x)$ 在 $U^+(x_0,\delta_0)$ 内有定义,且 $f(x_0^+)=f(x_0)$,则称 $f(x)$ 在 $x_0$ \textbf{右连续}.

  若函数 $f(x)$ 在 $U^-(x_0,\delta_0)$ 内有定义,且 $f(x_0^-)=f(x_0)$,则称 $f(x)$ 在 $x_0$ \textbf{左连续}.
\end{definition}

  \textbf{连续}的另一种等价定义:若 $f(x)$ 在 $x_0$ 处左连续且右连续($i.e. $ 在 $U(x_0,\delta_0)$ 上有定义,且 $f(x_0^+)=f(x_0^-)=f(x_0)$),那么称 $f(x)$ 在 $x_0$ 处\textbf{连续}.

  设函数 $f(x)$ 在 $(a,b)$ 内有定义.若 $\forall x_0\in(a,b)$,$f(x)$ 在 $x_0$ 连续,那么称 $f(x)$ \textbf{在 $(a,b)$ 内连续},记为 $f(x)\in C(a,b)$.

  设函数 $f(x)$ 在 $[a,b]$ 内有定义.若 $f(x)\in C(a,b)$,且 $f(x)$ 在 $a$ 处右连续,在 $b$ 处左连续,那么称 $f(x)$ **在 $[a,b]$ 内连续**,记为 $f(x)\in C[a,b]$.

\begin{exercise}{}
\begin{enumerate}
\item $f(x)=1/x-[1/x]$,求 $f(x)$ 的所有连续点.
\item $f(x)=1/x\ (x\in(0,1))$,证明 $f(x)\in C(0,1)$.
\item $f(x)=\left\{\begin{aligned}&x\cdot \sin(1/x),&x\in(0,1]\\&0,&x=0 \end{aligned}\right.$,证明 $f(x)\in C[0,1]$.
\item 证明:若 $f(x)\in C(a,b)$,且极限 $\lim\limits_{x\rightarrow a^+}f(x)=A,\lim\limits_{x\rightarrow b^-}f(x)=B$ 存在,则可以连续延拓到 $(-\infty,+\infty)$.
\end{enumerate}
\end{exercise}

\subsection{定义  间断,几类间断点}

  设函数 $f(x)$ 在 $U(x_0,\delta_0)$ 内有定义,若 $x_0$ 不是 $f(x)$ 的连续点,则称 $f(x)$ 在 $x_0$ \textbf{间断}(或不连续),并称 $x_0$ 为 $f(x)$ 的一个\textbf{间断点}(或不连续点).

  用左右极限来刻画间断点,则有以下几种情况:

  1. 若 $f(x_0^+)$ 和 $f(x_0^-)$ 都存在,则称 $x_0$ 为 $f(x)$ 的**第一类间断点**.此时若 $f(x_0^+)=f(x_0^-)\neq f(x_0)$,则称它为**可去间断点**,否则称它为**跳跃间断点**.

  2. 若 $f(x_0^+)$ 和 $f(x_0^-)$ 至少有一个不存在,则称 $x_0$ 为 $f(x)$ 的**第二类间断点**.

  ##### Question:

  1. $f(x)=1/x-[1/x]$,求 $f(x)$ 的所有间断点.
  2. $f(x)=sin(1/x)$,考察 $f(x)$ 在 $x=0$ 处的连续性.
  3. 若 $f(x)$ 在 $(0,1)$ 上单调,证明 $f(x)$ 的间断点都是跳跃间断点.
  4. 构造定义域为 $\mathbb{R}$ 的函数 $f(x)$,使得处处都是间断点. 

  考察黎曼函数的连续性:
\begin{equation}
f(x)=\left\{
\begin{aligned}
&1/q, &x=\frac{p}{q}\ (p,q\in \mathbb{N}, \frac{p}{q}\text{为既约真分数})\\
&0,&x=0\text{或}x=1\text{或} x\notin \mathbb{Q}
\end{aligned} \right.
\end{equation}
  容易证明,对于 $x_0\in(0,1)$,若 $x_0$ 是有理数,则 $x_0$ 是 $f(x)$ 的可去间断点;若 $x_0$ 是无理数,则 $f(x)$ 在 $x_0$ 连续.

  

- #### 连续函数的性质

  **性质1**:**局部有界性**.若 $f(x)$ 在 $x_0$ 连续,则必存在 $x_0$ 的一个邻域 $U(x_0,\delta)$,使得 $f(x)$ 在该邻域内有界.

  **性质2**:**局部保号性**.若 $f(x)$ 在 $x_0$ 连续,而 $f(x_0)>0$,则必存在 $x_0$ 的一个邻域 $U(x_0,\delta)$,使得 $f(x)$ 在该邻域内 $>0$.

  更强的命题:若 $f(x)$ 在 $x_0$ 连续,而 $f(x_0)>A$,则必存在 $x_0$ 的一个邻域 $U(x_0,\delta)$,使得 $f(x)$ 在该邻域内 $>A$.

  **性质3**:连续函数经四则运算后仍然连续,即若 $f(x),g(x)$ 在 $x_0$ 处连续,则函数 $f(x)+g(x),f(x)-g(x),f(x)g(x),f(x)/g(x)(g(x_0)\neq 0)$ 仍然在 $x_0$ 处连续.

  **定理1**:**复合函数连续性**:设 $u=g(x)$ 在点 $x_0$ 连续,$y=f(u)$ 在点 $u_0=g(x_0)$ 连续,复合函数 $f(g(x))$ 在点 $x_0$ 连续.

  由连续性的定义,我们还可以推出复合连续函数极限的性质:

  设 $u=g(x)$ 在点 $x_0$ 连续,$y=f(u)$ 在点 $u_0=g(x_0)$ 连续,那么$\lim\limits_{x\rightarrow x_0}f(g(x))=f(\lim\limits_{x\rightarrow x_0}g(x))$ 一定成立.

  **定理2**:**反函数连续性**:设 $f(x)$ 是区间 $I$ 上严格单调的连续函数,则其反函数 $x=f^{-1}(y)$ 在 $f(I)$ 上连续.

  ##### Question:

  1. 对于给定的常数 $a>0$,应如何定义指数函数 $f(x)=a^x$ ($x$ 是无理数的情况)?
  2. 判断由 $1$ 定义的指数函数 $f(x)=a^x(a>0)$ 的连续性.
  3. $u(x)$ 在 $x_0$ 处连续,$u(x_0)>0$;$v(x)$ 在 $x_0$ 处连续.证明: $f(x)=u(x)^{v(x)}$ 在 $x_0$ 处连续,且 $f(x_0)=u(x_0)^{v(x_0)}$.

  由以上的 Question,可以推出一个重要结论:**初等函数在其定义域内是连续的**.