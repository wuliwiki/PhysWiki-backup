% 高阶导数(简明微积分)
% keys 微积分|求导|导数|高阶导数|求导法则

\begin{issues}
\issueOther{未完成:说明越高阶可导越光滑, 无穷阶可导简称为 “光滑”, 或 “解析”}
\end{issues}

\pentry{求导法则\upref{DerRul}}

一个一元函数 $y=f(x)$ 可导, 如果导函数 $y'=f'(x)$ 仍可导, 则称 $y'=f'(x)$ 的导数为函数 $y=f(x)$ 的二阶导数,记为
\begin{equation}
y'',\quad f''(x),\quad \frac{\dd{^2y}}{\dd{x^2}},\quad \frac{\dd{^2f}}{\dd{x^2}},\quad y''(x)
\end{equation}
即
\begin{equation}
y''=(y')',\quad\frac{\dd{^2y}}{\dd{x^2}}=\frac{\dd{ }}{\dd{x}}\qty(\frac{\dd{x}}{\dd{y}})
\end{equation}

由导数中的定义\autoref{Der_eq2}~\upref{Der},可得二阶导数的计算公式
\begin{equation}
f''(x)=\lim_{\Delta x \to 0} \frac{f'(x+ \Delta x)-f'(x)}{\Delta x}
\end{equation}

若把 $y=f(x)$ 的导数 $y'=f'(x)$ 称为 $y=f(x)$ 的\textbf{一阶导数}, 那么, 一阶导数的导数就称为\textbf{二阶导数}。 若二阶导数 $y''=f''(x)$ 仍然可导,我们就把二阶导数的数 $y''=f''(x)$ 的导数称为\textbf{三阶导数}, 记为
\begin{equation}
y''',\quad f'''(x),\quad \frac{\dd{^3y}}{\dd{x^3}},\quad \frac{\dd{^3f}}{\dd{x^3}},\quad y'''(x)
\end{equation}

一般地,如果 $y=f(x)$ 的 $n-1$ 阶导数是可导的,我们就把 $n-1$ 阶导数的导数称为果 $y=f(x)$ 的\textbf{n阶导数},记为
\begin{equation}
y^{(n)},\quad f^{(n)}(x),\quad \frac{\dd{^ny}}{\dd{x^n}},\quad \frac{\dd{^nf}}{\dd{x^n}},\quad y^{(n)}(x)
\end{equation}

二阶和二阶以上的导数统称为\textbf{高阶导数}。 利用求导公式和求导法则就可以求出高阶导数。

%未完成 % 可以举例 sin,说明其高阶导数的周期性,还有 exp,高阶导数都一样
