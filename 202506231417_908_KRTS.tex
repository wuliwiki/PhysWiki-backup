% 卡尔·魏尔施特拉斯(综述)
% license CCBYSA3
% type Wiki

本文根据 CC-BY-SA 协议转载翻译自维基百科 \href{https://en.wikipedia.org/wiki/Karl_Weierstrass}{相关文章}。

卡尔·西奥多·威廉·魏尔斯特拉斯(Karl Theodor Wilhelm Weierstrass,/ˈvaɪərˌstrɑːs, -ˌʃtrɑːs/,[德语发音:Weierstraß [ˈvaɪɐʃtʁaːs];1815年10月31日-1897年2月19日)是德国数学家,常被誉为“现代分析学之父”。尽管他大学未取得学位便退学,但他自学数学并接受了师范培训,最终在学校教授数学、物理、植物学和体操。后来他获得了荣誉博士学位,并成为柏林大学的数学教授。

在众多贡献中,魏尔斯特拉斯形式化了函数连续性的定义和复分析理论,证明了中值定理和博尔查诺–魏尔斯特拉斯定理,并利用后者研究了闭有界区间上连续函数的性质。
\subsection{生平}
魏尔斯特拉斯出生于普鲁士威斯特法伦省恩尼格尔洛附近的奥斯滕费尔德村的一个罗马天主教家庭。\(^\text{[4]}\)

卡尔·魏尔斯特拉斯是威廉·魏尔斯特拉斯与特奥多拉·冯德福斯特的儿子,父亲是一名政府官员,父母皆为信奉天主教的莱茵兰人。他在帕德博恩的特奥多里安文理中学(Theodorianum)求学期间便对数学产生了浓厚兴趣。中学毕业后,他被送往波恩大学,目的是为将来从政做准备,因此被安排学习法律、经济和财政等科目——这与他一心想学习数学的志向发生了直接冲突。他通过对既定学业置之不理、私下自学数学的方式来解决这种矛盾,这也最终导致他未能获得学位便中途退学。

魏尔斯特拉斯随后在明斯特学院继续数学学习(该学院当时已以数学著称),他的父亲还为他争取到了明斯特师范学院的一个名额,他在那里努力学习,最终获得了教师资格。在此期间,他听了克里斯托夫·古德曼的课程,并由此对椭圆函数产生了浓厚兴趣。

1843年,魏尔斯特拉斯在西普鲁士的德意志克罗讷任教;自1848年起,他在布劳恩斯贝格的霍西安文理学院任教。\(^\text{[5]}\)除了教授数学,他还讲授物理、植物学和体操。\(^\text{[4]}\)有传言称,魏尔斯特拉斯可能与他朋友卡尔·威廉·博尔哈特的遗孀育有一名私生子“弗朗茨”。\(^\text{[6]}\)

1850年后,魏尔斯特拉斯长期饱受疾病困扰,但他仍然发表出质量和独创性俱佳的数学论文,由此声名鹊起。1854年3月31日,哥尼斯堡大学授予他名誉博士学位。1856年,他在柏林工艺学院获得教职,该学院致力于培养技术工人,后来与建筑学院合并,形成位于夏洛滕堡的柏林工业大学(今柏林工业大学)。1864年,他成为柏林弗里德里希-威廉大学的教授,该校后来更名为柏林洪堡大学。

1870年,55岁的魏尔斯特拉斯结识了索菲娅·柯瓦列夫斯卡娅,因她无法正式被大学录取,他便私下为她授课。他们建立了既富有学术成效又充满温情的关系,“远远超越了通常意义上的师生关系”。他指导她四年,视她为自己最优秀的学生,并帮助她绕过口试程序,从海德堡大学获得博士学位。

从1870年到柯瓦列夫斯卡娅于1891年去世之间,他们始终保持通信。得知她去世的消息后,魏尔斯特拉斯烧毁了她写给他的信件;而他写给她的信件中仍有大约150封保存至今。德国教授莱因哈德·贝林发现了柯瓦列夫斯卡娅于1883年抵达斯德哥尔摩、被任命为斯德哥尔摩大学私人讲师时写给魏尔斯特拉斯的一封信的草稿。\(^\text{[7]}\)

魏尔斯特拉斯生命的最后三年几乎完全丧失行动能力,最终于1897年2月19日在柏林因肺炎去世。\(^\text{[8]}\)
\subsection{数学贡献}
\subsubsection{微积分的严谨性}
魏尔斯特拉斯关注微积分理论的严谨性。在他所处的时代,微积分的基础概念定义尚不明确,导致许多重要定理难以用足够的数学 rigor(严谨性)加以证明。虽然波尔查诺(Bolzano)早在1817年(甚至可能更早)就已经提出了一个相当严谨的极限定义,但他的工作在当时鲜为人知,许多数学家对极限和函数连续性的理解仍然模糊不清。

Δ-ε 证明(即“极限的ε-δ定义”)的基本思想可以说最早出现在柯西19世纪20年代的著作中。[9][10] 然而,柯西并未清晰地区分“连续性”与“一致连续性”。特别是在他1821年出版的《分析教程》中,柯西曾主张:逐点连续函数的逐点极限仍为连续函数。但这一说法在一般情形下是错误的。正确的命题应是:连续函数的一致极限是连续的(更进一步,一致连续函数的一致极限也是一致连续的)。要说明这个命题,就必须引入“一致收敛”的概念。

一致收敛这一现象最早是魏尔斯特拉斯的导师克里斯托夫·古德尔曼(Christoph Gudermann)在1838年一篇论文中观察到的。尽管他指出了这一现象,但并未给出正式定义,也没有详细展开。魏尔斯特拉斯看到了该概念的重要性,不仅正式定义了一致收敛,还在微积分基础理论中广泛应用。

魏尔斯特拉斯对“函数连续性”的形式化定义如下:

函数 $f(x)$ 在点 $x = x_0$ 处连续,当且仅当:对于任意 $\varepsilon > 0$,存在一个 $\delta > 0$,使得对定义域中任意的 $x$,只要满足$|x - x_0| < \delta$
就有$|f(x) - f(x_0)| < \varepsilon$。用简单的语言来说:如果 $x$ 足够接近 $x_0$,那么 $f(x)$ 的值就会非常接近 $f(x_0)$ 的值;其中,“足够接近”的标准取决于你希望 $f(x_0)$ 与 $f(x)$ 之间有多接近。魏尔斯特拉斯使用这个定义,证明了介值定理。他还证明了波尔查诺–魏尔斯特拉斯定理,并利用它研究了闭区间上连续函数的性质。
\subsubsection{变分法}
魏尔斯特拉斯在变分法领域也做出了重要贡献。他利用自己帮助建立的分析工具体系,对整个变分法理论进行了彻底重构,为现代变分法的研究奠定了基础。在提出的若干公理中,魏尔斯特拉斯给出了变分问题中强极值存在的必要条件。此外,他还参与提出了魏尔斯特拉斯–埃尔德曼条件,该条件为极值路径在某点出现“拐角”提供了充分条件,并可用于寻找某一泛函的极小曲线。
\subsubsection{其他分析相关定理}
\begin{itemize}
\item 波尔查诺–魏尔斯特拉斯定理
\item 斯通–魏尔斯特拉斯定理
\item 卡索拉蒂–魏尔斯特拉斯定理
\item 魏尔斯特拉斯椭圆函数
\item 魏尔斯特拉斯函数
\item 魏尔斯特拉斯 M-判别法
\item 魏尔斯特拉斯预备定理
\item 林德曼–魏尔斯特拉斯定理
\item 魏尔斯特拉斯因式分解定理
\item 魏尔斯特拉斯–恩内珀参数化
\end{itemize}
\subsection{荣誉与奖项}
月球上的魏尔斯特拉斯环形山和小行星14100 Weierstrass皆以他命名。此外,德国柏林还有以他名字命名的魏尔斯特拉斯应用分析与随机研究所。
\subsection{精选著作}
\begin{itemize}
\item 《论阿贝尔函数理论》(1854)
\item 《阿贝尔函数理论》(1856)
\item 《论文集 第一卷》,《数学著作》第1卷,柏林,1894
\item 《论文集 第二卷》,《数学著作》第2卷,柏林,1895
\item 《论文集 第三卷》,《数学著作》第3卷,柏林,1903
\item 《关于阿贝尔超越函数理论的讲义》,《数学著作》第4卷,柏林,1902
\item 《关于变分法的讲义》,《数学著作》第7卷,莱比锡,1927
\end{itemize}
\subsection{另见}
\begin{itemize}
\item 以卡尔·魏尔斯特拉斯命名的事物列表
\end{itemize}
\subsection{参考文献}

* “Weierstrass”,《兰登书屋韦氏未删节词典》。
* 《杜登发音词典》第7版,柏林:书目研究出版社,2015年,ISBN 978-3-411-04067-4。
* Weierstrass, Karl Theodor Wilhelm. (2018). 载于 Helicon 主编《哈钦森未删节百科全书(含地图和气象指南)》。\[在线] 阿宾顿:Helicon。可访问地址:[http://libezproxy.open.ac.uk/login?url=(链接访问时间:2018年7月8日)。](http://libezproxy.open.ac.uk/login?url=(链接访问时间:2018年7月8日)。)
* O'Connor, J. J.; Robertson, E. F.(1998年10月),“Karl Theodor Wilhelm Weierstrass”,苏格兰圣安德鲁斯大学数学与统计学院。访问时间:2014年9月7日。
* Elstrodt, Jürgen (2016),载于König, Wolfgang;Sprekels, Jürgen(编)《Karl Weierstraß (1815–1897)》(德文),威斯巴登:Springer Fachmedien Wiesbaden,第11–51页,doi:10.1007/978-3-658-10619-5\_2,ISBN 978-3-658-10618-8,访问时间:2023年8月12日。
* Biermann, Kurt-R.; Schubring, Gert(1996),“Karl Weierstraß传记若干补记”(德文)\[Some postscripts to the biography of Karl Weierstrass],《数学史》。圣地亚哥,加州:Academic Press,第65–91页。
* Kuznetsov, Vadim B., 编(2002),“Roger L. Cooke 撰《索菲娅·柯瓦列夫斯卡娅生平》”,载于《Kowalevski 性质》(2000年利兹会议),CRM 讲义与会议记录,第32卷。美国数学会,第1–19页,ISBN 978-0-8218-7330-4;参见该书第7页。在线文本。
* 《科学传记词典》,主编 Charles Coulston Gillispie,美国学术团体协会。纽约,1970年,第223页,ISBN 978-0-684-12926-6,OCLC 89822。
* Grabiner, Judith V.(1983年3月),“你是从谁那儿学会ε的?——Cauchy 与严密微积分的起源”(PDF),《美国数学月刊》,第90卷第3期,第185–194页,doi:10.2307/2975545,JSTOR 2975545,原文PDF存档于2014年11月29日。
* Cauchy, A.-L.(1823),“第七讲——关于表达式在不定形式(如 ∞/∞, ∞^0, …)下的值,与差分商与导函数之间关系”,载于《巴黎皇家理工学院微积分课程讲义摘要》,巴黎,第44页,原文存档于2009年5月4日,访问时间:2009年5月1日。
