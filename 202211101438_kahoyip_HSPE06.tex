% 磁场(高中)
% 磁场|安培力|洛伦兹力|磁感应强度|磁通量

\subsection{磁场}

具有磁性的物质叫做\textbf{磁体},能吸引铁、钴、镍等物质.磁体上磁性最强的部分叫做磁极,分为\textbf{北极}($N$)和\textbf{南极}($S$),磁极之间的作用规律为:同名磁极相互排斥,异名磁极相互吸引.

\textbf{磁场}是磁体或电流周围存在的一种看不见、摸不着的特殊物质.磁体与磁体、磁体与电流、电流与电流之间都存在相互作用,统称为磁相互作用,这种相互作用是通过磁场发生的.

磁场中某点磁场方向的表述:

\begin{enumerate}
\item 小磁针北极受磁场力的方向;
小磁针静止时北极所指的方向;
磁感线某点的切线方向;
磁感应强度的方向.
\end{enumerate}

\subsubsection{磁感线}
与电场线类似,为了形象地描述磁场,在磁场中画出一系列有方向的假想曲线,曲线上每一点的切线方向都跟该点的磁场方向相同,这样的曲线叫做\textbf{磁感线}.在磁体的外部,磁感线从北极到南极,在磁体内部则从南极到北极,由此可见每一条磁感线都是闭合曲线.

磁感线的疏密程度反映了磁场的强弱,磁感线密的位置磁场强,磁感线疏的位置磁场强.
