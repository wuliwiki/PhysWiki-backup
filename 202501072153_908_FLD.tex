% 迈克尔·法拉第(综述)
% license CCBYSA3
% type Wiki

本文根据 CC-BY-SA 协议转载翻译自维基百科\href{https://en.wikipedia.org/wiki/Stokes\%27_theorem}{相关文章}。

迈克尔·法拉第(Michael Faraday,/ˈfærədeɪ, -di/;1791年9月22日-1867年8月25日)是一位英国物理学家和化学家,对电磁学和电化学的研究作出了重要贡献。他的主要发现包括电磁感应、抗磁性和电解等基本原理。尽管法拉第接受的正式教育很少,他作为一个自学成才的人,成为历史上最有影响力的科学家之一。[1] 通过研究直流电导体周围的磁场,法拉第在物理学中确立了电磁场的概念。他还发现了磁性可以影响光线,并揭示了两者之间的内在联系。[2][3] 他同样发现了电磁感应、抗磁性和电解定律的基本原理。他发明的电磁旋转装置奠定了电动机技术的基础,并且主要由于他的努力,使电的应用在技术中成为可能。[4]

作为一名化学家,法拉第发现了苯,研究了氯的包合水合物,发明了早期版本的本生灯以及氧化数系统,并推广了“阳极”、“阴极”、“电极”和“离子”等术语。最终,法拉第成为英国皇家学会的首任兼最重要的富勒讲座化学教授(Fullerian Professor of Chemistry),这一职位是终身制的。

法拉第是一位实验科学家,他以清晰简洁的语言表达自己的想法。他的数学能力并不高深,仅限于基础代数,甚至未涉猎三角学。詹姆斯·克拉克·麦克斯韦(James Clerk Maxwell)基于法拉第及其他人的工作,总结出一组方程,这些方程被认为是所有现代电磁现象理论的基础。关于法拉第使用“力线”的方法,麦克斯韦写道,这表明法拉第“实际上是一位非常高水平的数学家——未来的数学家可以从他那里获得宝贵而富有成果的方法。”[5] 国际单位制中电容的单位“法拉”(farad)就是以他的名字命名的。

阿尔伯特·爱因斯坦在书房的墙上挂着法拉第的画像,与艾萨克·牛顿和詹姆斯·克拉克·麦克斯韦的画像并列。[6] 物理学家欧内斯特·卢瑟福(Ernest Rutherford)曾说:“当我们考虑他的发现的广度与深度,以及它们对科学与工业发展的影响时,再大的荣誉也不足以表达对法拉第——这一有史以来最伟大的科学发现者之一——的敬意。”[1]
\subsection{传记}
\subsubsection{早年生活}
迈克尔·法拉第于1791年9月22日出生在萨里的纽因顿巴茨(Newington Butts),该地区现为伦敦南华克区的一部分。[7][8] 他的家境贫困。他的父亲詹姆斯·法拉第(James Faraday)是一个基督教格拉塞派(Glasite sect)的成员。1790年冬天,詹姆斯带着妻子玛格丽特(原姓哈斯特威尔,Margaret Hastwell)[9]和两个孩子从西摩兰郡的奥思吉尔(Outhgill)搬到伦敦。在奥思吉尔,詹姆斯曾是村庄铁匠的学徒。[10] 迈克尔是在次年秋天出生的,是家中四个孩子中的第三个。年幼的法拉第仅接受过最基本的学校教育,因此他不得不自学成才。[11]

14岁时,法拉第成为布兰德福德街(Blandford Street)当地书籍装订匠兼书商乔治·里博(George Riebau)的学徒。[12] 在七年的学徒期内,法拉第阅读了许多书籍,包括艾萨克·沃茨(Isaac Watts)的《提升心智》(*The Improvement of the Mind*)。他充满热情地实践了书中提出的原则和建议。[13] 在此期间,法拉第与同伴们在“城市哲学会”(City Philosophical Society)中进行讨论,并参加了关于各种科学主题的讲座。[14] 他对科学,特别是对电学产生了浓厚的兴趣。法拉第尤其受到简·马西特(Jane Marcet)所著的《化学对话》(*Conversations on Chemistry*)一书的启发。[15][16]
\subsubsection{成年生活}
\begin{figure}[ht]
\centering
\includegraphics[width=6cm]{./figures/80600dcd950a25f1.png}
\caption{1842年托马斯·菲利普斯绘制的法拉第肖像} \label{fig_FLD_1}
\end{figure}
1812年,20岁的法拉第在学徒期结束时,参加了英国皇家学会和皇家学院著名化学家汉弗里·戴维(Humphry Davy)以及“城市哲学会”创始人约翰·塔图姆(John Tatum)的讲座。这些讲座的许多门票由皇家爱乐学会的创始人之一威廉·丹斯(William Dance)赠送给了法拉第。随后,法拉第将他根据这些讲座笔记整理成的一本300页的书寄给了戴维。戴维很快回信,态度友好且积极。1813年,戴维在一次与三氯化氮的实验中发生事故,导致视力受损,他决定雇用法拉第作为助手。巧合的是,皇家学院的一位助理约翰·佩恩(John Payne)被解雇了,而戴维正被要求寻找替代者,因此他于1813年3月1日任命法拉第为皇家学院的化学助理。[2] 不久之后,戴维委托法拉第准备三氯化氮样品,他们两人在处理这种极为敏感的物质时都在一次爆炸中受了伤。[17]

1821年6月12日,法拉第与萨拉·巴纳德(Sarah Barnard,1800–1879)结婚。[18] 他们通过各自的家庭在桑德曼教会(Sandemanian church)中相识。结婚一个月后,法拉第向桑德曼教会的会众表白了他的信仰。他们没有子女。[7] 法拉第是一位虔诚的基督徒,其所属的桑德曼教派是苏格兰教会的一个分支。婚后多年,他曾在少年时参加的教会中担任执事,并两次担任长老。他所在的教会位于巴比肯(Barbican)的保罗巷(Paul's Alley)。1862年,这座教堂搬迁到了北伦敦伊斯灵顿的巴恩斯伯里格罗夫(Barnsbury Grove)。法拉第在担任第二任长老的最后两年时间里一直服务于这个新地点,直到辞去职务。[19][20] 传记作者指出,“对上帝和自然统一性的强烈信念贯穿了法拉第的生活和工作。”[21]
\subsubsection{晚年生活}
1832年6月,牛津大学授予法拉第名誉民法博士学位。在他的一生中,他因对科学的贡献被提议授予爵士头衔,但由于宗教原因,他拒绝了这一荣誉。他认为根据《圣经》的教义,积累财富和追求世俗奖励是错误的,并表示自己更愿意一直保持“普通的法拉第先生”身份。[22] 法拉第于1824年当选为皇家学会会士,但他两次拒绝担任会长职务。[23] 1833年,他成为皇家学院首任富勒讲座化学教授(Fullerian Professor of Chemistry)。[24]  

1832年,法拉第被选为美国艺术与科学学院的外籍名誉会员。[25] 1838年,他被选为瑞典皇家科学院外籍院士。1840年,他当选为美国哲学会会员。[26] 1844年,他成为法国科学院八位外籍院士之一。[27] 1849年,他被选为荷兰皇家研究院的准会员,该研究院两年后更名为荷兰皇家艺术与科学学院,随后法拉第被授予外籍院士身份。[28]

1839年,法拉第经历了一次神经崩溃,但最终他重新投入到对电磁学的研究中。[29] 1848年,在阿尔伯特亲王的建议下,法拉第获赠了一栋位于米德尔塞克斯汉普顿宫的恩典与厚爱之屋(grace and favour house),完全免除了所有费用和维护支出。这栋房屋是原来的石匠总监之家(Master Mason's House),后来被称为法拉第之家(Faraday House),现为汉普顿宫路37号(No. 37 Hampton Court Road)。1858年,法拉第退休后便搬到这里居住。[30]

法拉第为英国政府提供了许多服务项目,但当政府要求他就克里米亚战争(1853–1856)中化学武器的生产提供建议时,法拉第以伦理原因拒绝参与。[31] 他还拒绝了出版其讲座的提议,认为如果没有现场实验的配合,讲座内容的影响力会减弱。他在给出版商的一封信中写道:“我一直热爱科学胜过金钱,因为我的职业几乎完全是个人的,我无法承担变得富有的代价。”[32]

1867年8月25日,法拉第在汉普顿宫的住所中去世,享年75岁。[33] 在数年前,他拒绝了死后安葬于威斯敏斯特教堂的提议,但在那里靠近艾萨克·牛顿墓旁有一块纪念牌以纪念他。[34] 法拉第被安葬在海格特公墓的异议者(非英国国教徒)区。[35]