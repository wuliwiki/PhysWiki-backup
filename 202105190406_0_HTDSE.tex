% 氢原子球坐标数值解 TDSE 2
% keys 氢原子|球坐标|薛定谔方程|TDSE|分波|径向方程

\begin{issues}
\issueDraft
\end{issues}

\pentry{电磁场中氢原子薛定谔方程数值解\upref{HyTDSE}, 张量积空间\upref{DirPro}}

相比于 “电磁场中氢原子薛定谔方程数值解\upref{HyTDSE}”, 这里的描述更数学.

总波函数在球谐基底上展开(我们把 $(l,m)$ 按一定顺序排序后的序号记为 $\lambda$)
\begin{equation}\label{HTDSE_eq1}
\Psi(\bvec r, t) = \sum_\lambda \ket{\psi_\lambda(t)} \otimes \ket{Y_\lambda(\uvec r)}
\end{equation}
哈密顿算符为
\begin{equation}\label{HTDSE_eq2}
H = H_0 + V_F(t) = K_r\otimes I + \frac{1}{2r^2}\otimes L^2 + V(r) \otimes I + V_F(t)
\end{equation}
耦合方程组为
\begin{equation}\label{HTDSE_eq3}
\sum_{\lambda'} \mel{\lambda}{H}{\lambda'} \ket{\psi_{\lambda'}(t)} = -\I\pdv{t} \ket{\psi_\lambda(t)}
\end{equation}
其中每个矩阵元 $ \mel{\lambda}{H}{\lambda'}$ 都是一个关于 $r$ 空间的算符(张量空间的算符如果关于一个小空间的基底求矩阵元, 那么每个矩阵元都会是另一个小空间中的算符).

无外场的哈密顿算符的 $\mel{\lambda}{H_0}{\lambda'}$ 将会是对角的, 即每个 $\psi_{\lambda}(r, t)$ 都会独立传播, 传播子为 $\exp(-\I \mel{\lambda}{H}{\lambda} \Delta t)$.

\subsection{算符拆分(split operators)}
无论用什么传播算法, 传播子总可以记为 $\exp(-\I H \Delta t)$, 这里近似 $\Delta t$ 内 $H$ 不随时间变化. 如果 $H$ 可以拆分为几个算符之和(\autoref{HTDSE_eq2}), $\exp(-\I H \Delta t)$ 不一定能拆分成几个传播子的乘积(因为这些项不互相对易), 但 $\Delta t$ 很小时会近似成立.

一种精度比较高的拆分方法是将不含外场(field free)的 $H_0$ 和含外场的 $V_F$ 分开传播
\begin{equation}
\exp(-\I H\Delta t) = \exp(-\I H_0\frac{\Delta t}{2})\exp(-\I V_F\Delta t) \exp(-\I H_0\frac{\Delta t}{2}) + \order{\Delta t^3}
\end{equation}
由于 $\mel{\lambda}{H_0}{\lambda'}$ 都是对角的, 根据\autoref{HTDSE_eq3}, $\exp(-\I H_0 \Delta t/2)$ 作用在总波函数上其实就相当于 $\exp(-\I\mel{\lambda}{H_0}{\lambda} \Delta t)$ 分别作用在每个 $\psi_{\lambda}(r)$ 上.

\subsection{线偏振外场}
在 length gauge 下, 如果有只延 $z$ 方向的外场, 那么
\begin{equation}
V_F(t) = \bvec E(t) \vdot \bvec r =  E(t) z = E(t) r \otimes Y_{10}(\uvec r)
\end{equation}
所以矩阵元为
\begin{equation}
\mel{\lambda}{V_F(t)}{\lambda'} =  E(t) r \mel{\lambda}{Y_{10}}{\lambda'}
\end{equation}

如果我们在 $r$ 空间中取类似 $\delta(r - r_i)$ 的基底 $\ket{r_i}$(例如等间距基底或 FEDVR 基底), 那么 $V_F$ 也会有一个很好的性质就是它们可以表示为 $r$ 空间的对角矩阵(矩阵元为 $r_i$)和角向空间中的一个算符(矩阵)的张量积. 

与\autoref{HTDSE_eq1} 相反, 将总波函以不同径向基底拆分成若干个角向波函数
\begin{equation}
\Psi(\bvec r, t) = \sum_i \ket{r_i} \otimes f_i(\uvec r)
\end{equation}
将 $V_F$ 作用在总波函数上得
\begin{equation}
(r \otimes Y_{10})\sum_i \ket{r_i} \otimes f_i(\uvec r)
= \sum_i r_i\ket{r_i} \otimes [Y_{10} f_i(\uvec r)]
\end{equation}
所以我们只需要对每个 $f_i(\uvec r)$ 使用传播子 $\exp[-\I (r_i Y_{10}) \Delta t]$ 进行传播即可. 容易证明% 未完成: 该证明不应该出现在这里吧
\begin{equation}
\exp(\sum_i \ket{i}\bra{i} \otimes \Q B_i) = \prod_i \ket{i}\bra{i} \otimes  \E^{\Q B_i}
\end{equation}
\begin{equation}
\exp(\sum_i  \Q A_i \otimes \ket{i}\bra{i}) = \prod_i  \E^{\Q B_i} \otimes \ket{i}\bra{i}
\end{equation}

\subsection{选择定则起到的作用}
在对每个 $f_i(\uvec r)$ 使用传播子 $\exp[-\I (r_i Y_{10}) \Delta t]$ 的时候, 由于 expokit 只需要用户提供矩阵 $\mel{\lambda}{Y_{10}}{\lambda'}$ 与列矢量相乘的 implementation. 如果知道选择定则, 即那些矩阵元为 0, 我们就可以使用 sparse matrix 与列矢量的乘法从而提高计算效率.

例如对于线偏振光, 只选取 $m = 0$ 的基底, 选择定则要求 $\Delta l = \pm 1$, $\mel{\lambda}{Y_{10}}{\lambda'}$ 只是一个三对角矩阵且对角元为 0.

选择定则对于氢原子其实并没有太大的性能提升, 因为实践表明两个径向传播子 $\exp(-\I H_0 \Delta t/2)$ 的传播才是最耗时的. 而对于氦原子, 使用选择定则可能有更明显的优势.


\subsection{平均能量}

