% 仿射变换在解析几何中的应用
% keys 仿射变换 圆锥曲线
\begin{definition}{仿射变换}
设椭圆\,\(\frac{x^2}{a^2}+\frac{y^2}{b^2}=1\),其中\,\(a>b>0\),置变换:
$$x'=\frac{x}{a},y'=\frac{y}{b}$$
则椭圆化为单位圆\,\(C:x'^2+y'^2=1\)
\end{definition}
届时,我们可以就可以抛开繁琐的代数计算,运用几何性质解决问题.此前,我们先介绍仿射变换的几个性质.
\begin{lemma}{}
变换后,平面内任意一条直线的斜率变为原来的\,\(\frac{a}{b}\)
\end{lemma}
\begin{lemma}{}
变换后,平面上任意区域的面积变为原来的\,\(\frac1{ab}\)
\end{lemma}
\begin{lemma}{}
变换后,线段中点依然是线段中点;关于坐标轴对称的元素依然关于坐标轴对称;平面区域的重心保持不变
\end{lemma}
\begin{lemma}{}
变换前后,平行关系保持不变
\end{lemma}
\begin{lemma}{}
变换前后,平行线段的长度比保持不变
\end{lemma}
\begin{corollary}{1}
设椭圆 \(\frac{x^2}{a^2}+\frac{y^2}{b^2}\),直线 \(l\) 交椭圆于点 \(A\) 和\,\(B\),点 \(P\) 为线段  的中点,求直线斜率


解:作变换  ,则椭圆化为单位圆 ,

故 

由性质3,  为  的中点

在圆中,由垂径定理, 

而  ,得 

由上述性质 1, 


\end{corollary}