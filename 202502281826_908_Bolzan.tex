% 伯恩哈德·波尔查诺(综述)
% license CCBYSA3
% type Wiki

本文根据 CC-BY-SA 协议转载翻译自维基百科\href{https://en.wikipedia.org/wiki/Bernard_Bolzano}{相关文章}。

\begin{figure}[ht]
\centering
\includegraphics[width=6cm]{./figures/f44771c523b92573.png}
\caption{} \label{fig_Bolzan_1}
\end{figure}
伯纳德·博尔扎诺(UK: /bɒlˈtsɑːnoʊ/, US: /boʊltˈsɑː-, boʊlˈzɑː-/;德语:[bɔlˈtsaːno];意大利语:[bolˈtsaːno];原名伯纳尔杜斯·普拉西杜斯·约翰·内波穆克·博尔扎诺;1781年10月5日–1848年12月18日)是捷克数学家、逻辑学家、哲学家、神学家和天主教神父,具有意大利血统,以其自由主义观点而著称。

博尔扎诺使用德语写作,这是他的母语。[6] 大部分他的工作是在他去世后才获得广泛关注。
\subsection{家庭}  
博尔扎诺是两位虔诚天主教徒的儿子。他的父亲,伯纳德·庞培乌斯·博尔扎诺,是一位意大利人,曾移居布拉格,在那里娶了来自布拉格讲德语的毛雷尔家族的玛丽亚·凯瑟莉亚·毛雷尔。他们的十二个孩子中只有两个活到成年。

