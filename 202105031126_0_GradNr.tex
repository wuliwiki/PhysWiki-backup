% 用梯度求曲线和曲面的法向量

预备知识: 多元函数的微分; 梯度定理
结论

$F(x, y) = 0$ 表示的曲线和 $F(x, y, z)$ 表示的曲面在某点的法向量就是他们在该点的梯度.

推导

平面曲线可以表示为 $F(x, y) = 0$ . 即 $x, y$ 在变化的过程中始终满足这一条件. 根据微分定理, 一点 $(x, y)$ 在曲线上移动的过程中, 显然有
\begin{equation}
\dd{F} = \grad F \vdot \dd{\bvec r} = \pdv{F}{x} \dd{x} + \pdv{F}{y} \dd{y} = 0
\end{equation}
其中 $\dd{\bvec r}$ 表示曲线上的一段微小位移, 延曲线的切向.

上式表示, 这两个矢量的点乘为零, 即 $\grad F$ 就是就是曲线在 $(x,y)$ 点的法向量.

空间直角坐标系中的曲面同样也可以用 $F(x, y, z)$ 来表示, 从曲面上 $P_0 = (x_0, y_0)$ 点出发, 延曲面的任意微小位移 $\dd{\bvec r}$ 都满足微分关系
\begin{equation}
\dd{F} = \grad F \vdot \dd{\bvec r} = \pdv{F}{x} \dd{x} + \pdv{F}{y} \dd{y} + \pdv{F}{z} \dd{z} = 0
\end{equation}
