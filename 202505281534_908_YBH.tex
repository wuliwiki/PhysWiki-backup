% 伊本·海什木(综述)
% license CCBYSA3
% type Wiki

本文根据 CC-BY-SA 协议转载翻译自维基百科 \href{https://en.wikipedia.org/wiki/Ibn_al-Haytham}{相关文章}。

哈桑·伊本·海赛姆(Ḥasan Ibn al-Haytham,拉丁化名为 Alhazen,/ælˈhæzən/;全名:阿布·阿里·哈桑·伊本·哈桑·伊本·海赛姆,阿拉伯语:أبو علي، الحسن بن الحسن بن الهيثم;约965年—约1040年),是伊斯兰黄金时代的一位中世纪数学家、天文学家和物理学家,来自今天的伊拉克地区。[6][7][8][9]他被誉为“现代光学之父”,[10][11][12] 尤其在光学原理与视觉感知领域做出了重要贡献。他最具影响力的著作是《光学书》(阿拉伯语:كتاب المناظر,*Kitāb al-Manāẓir),写于1011年至1021年之间,现存有其拉丁文译本。[13]在科学革命时期,艾萨克·牛顿、约翰内斯·开普勒、克里斯蒂安·惠更斯和伽利略·伽利莱等人经常引用海赛姆的著作。

伊本·海赛姆是第一个正确解释视觉理论的人,[14] 他提出视觉是在大脑中形成的,并指出视觉具有主观性,会受到个体经验的影响。[15] 他还首次提出了光在折射时走最短时间路径的原理,这一原理后来被称为费马原理。[16]在镜学和透镜学领域,他通过对反射、折射以及光线成像性质的研究做出了重大贡献。[17][18]伊本·海赛姆还是最早倡导假设必须通过可验证程序或数学推理支持的实验来检验的人之一——在文艺复兴科学家出现前五个世纪,他就已是科学方法的早期奠基者,[19][20][21][22] 有时他被称为世界上“第一位真正的科学家”。[12]
此外,他还是一位博学多才的学者,著述涵盖哲学、神学和医学等多个领域。[23]

伊本·海赛姆是第一个正确解释视觉理论的人,[14] 他提出视觉是在大脑中形成的,并指出视觉具有主观性,会受到个体经验的影响。[15] 他还首次提出了光在折射时走最短时间路径的原理,这一原理后来被称为费马原理。[16] 在镜学和透镜学领域,他通过对反射、折射以及光线成像性质的研究做出了重大贡献。[17][18]伊本·海赛姆还是最早倡导假设必须通过可验证程序或数学推理支持的实验来检验的人之一——在文艺复兴科学家出现前五个世纪,他就已是科学方法的早期奠基者,[19][20][21][22] 有时他被称为世界上“第一位真正的科学家”。[12]此外,他还是一位博学多才的学者,著述涵盖哲学、神学和医学等多个领域。[23]
\subsection{生平}
伊本·海赛姆(拉丁化名 Alhazen)出生于约公元965年,其家族具有阿拉伯[9][31][32][33][34]或波斯[35][36][37][38][39]血统,出生地为伊拉克的巴士拉,当时属于布韦王朝的统治范围。他最初受宗教研究和公共服务的影响较深。当时社会中存在许多相互冲突的宗教观点,他最终决定淡出宗教领域,转而投身于数学与科学的研究。[40]他在家乡巴士拉曾担任“宰相”的职位,并因其应用数学方面的才能而闻名,其中一个例证是他曾尝试设计方案以调控尼罗河的泛滥。[41]

他回到开罗后,被任命为一个行政职务。然而,他最终也未能胜任该职,引起了哈里发哈基姆的愤怒,[42] 据说他因此被迫躲藏,直到哈里发于公元1021年去世,他才得以恢复自由,并领回被没收的财产。[43] 传说称,海赛姆假装疯癫,并在这段时间中被软禁在家中。[44] 正是在这段时期,他撰写了其最有影响力的著作《光学书》。此后,海赛姆继续留居开罗,住在著名的爱资哈尔大学附近,靠其著作收入为生,直到约公元1040年去世。[45][41]现存有一部伊本·海赛姆亲笔誊写的阿波罗尼奥斯《圆锥曲线论》手稿,藏于圣索菲亚图书馆,编号为 MS Aya Sofya 2762,第307叶,落款日期为伊斯兰历415年萨法尔月(公元1024年)。[46]:注2

他的学生中包括一位来自塞姆南的波斯人苏尔哈布(Sorkhab,或作 Sohrab),以及一位埃及王子阿布·瓦法·穆巴希尔·伊本·法泰克。[47]
\subsection{《光学书》}
伊本·海赛姆最著名的著作是其七卷本的光学论文集——《光学书》,写于公元1011年至1021年间。[48] 在该书中,他是第一个解释视觉是由于光线从物体反射后进入眼睛而产生的人,[14] 同时他还首次提出视觉是在大脑中形成的,并指出视觉具有主观性,会受到个人经验的影响。[15]

这部著作在12世纪末或13世纪初由一位不知名的学者翻译成拉丁文。[49][a]

该书在中世纪享有极高声誉。其拉丁文版本 De aspectibus 于14世纪末被译成意大利通俗语,题为 《De li aspecti》。[50]

1572年,弗里德里希·里斯纳将其印刷出版,书名为:Opticae thesaurus: Alhazeni Arabis libri septem, nunc primum editi; Eiusdem liber De Crepusculis et nubium ascensionibus(中文译名:《光学宝藏:阿拉伯人阿尔哈曾七卷本著作,首版;另附其关于黄昏与云层高度的著作》)[51]“Alhazen”这一名字变体即由里斯纳所创,在此之前他在西方被称为“Alhacen”。[52]1834年,E. A. 塞迪约在巴黎国家图书馆发现了海赛姆关于几何的若干著作。根据A. Mark Smith 的研究,目前共发现18部完整或近完整的手稿和5部残卷,分布于14个地点,包括牛津大学的博德利图书馆和布鲁日图书馆等地。[53]
\subsubsection{光学理论}
\begin{figure}[ht]
\centering
\includegraphics[width=6cm]{./figures/0669b50632580a1e.png}
\caption{} \label{fig_YBH_1}
\end{figure}
在古典时代,关于视觉的主要理论有两种:第一种是发射理论,由欧几里得和托勒密等思想家支持,他们认为视觉是由于眼睛发出光线与物体接触而产生的。第二种是摄入理论,由亚里士多德及其追随者支持,他们认为物体以某种物理形式将图像传入眼中。早期伊斯兰世界的学者(如金迪 al-Kindi)基本上沿用了欧几里得、盖伦或亚里士多德的理论体系。《光学书》最强烈的影响来源于托勒密的《光学》,而其中关于眼睛的解剖与生理结构的描述,则是基于盖伦的医学论述。[54]海赛姆的成就在于,他创造性地提出了一个理论,成功结合了:欧几里得的数学光线理论;盖伦的医学传统;以及亚里士多德的摄入理论中的部分要素。在他的摄入理论中,海赛姆继承了金迪的观点(并不同于亚里士多德),提出:“在任何被光照亮的彩色物体上,从其每一个点都会沿着所有可以从该点画出的直线,向外发出光与颜色。”[55]这就为他留下了一个重要问题:如何从如此多独立来源的辐射中形成一个连贯的图像?——特别是,当物体的每一个点都向眼睛的每一个点发送光线时,图像如何保持清晰与一致?

海赛姆所需要解决的问题是:物体上的每一个点如何只对应眼睛上的一个点。[55]为了解决这个问题,他提出了一个观点:眼睛只感知来自物体的垂直光线——也就是说,对于眼睛上的任意一点,只有那条直接进入该点、且未被眼睛其他部分折射的光线才会被感知。他使用了一个物理类比来说明垂直光线比斜射光线更“有力”:就像一个球如果垂直击中木板,可能会将其击碎;但如果是斜着打过去,就会被弹开。同样的道理,垂直入射的光线比折射偏转的光线更“强”,因此只有垂直光线才会被眼睛感知。由于从物体某一点发出的光线中,只有一条垂直光线能准确地进入眼睛的某一点,且所有这些光线在眼睛中形成一个朝向中心的光锥结构,这一观点便使他成功解决了“物体每个点发出大量光线会造成视觉混乱”的问题。换句话说,如果只有垂直光线会被感知,那么就可以实现“物体点”和“眼睛点”之间的一对一对应关系,避免了图像混乱。[56]后来,在《光学书》第七卷中,海赛姆又进一步提出:其他(非垂直)光线在进入眼睛后,会被折射,并最终“仿佛”以垂直方式被感知。[57]然而,他关于垂直光线的论证仍存在解释不足之处:[58]为什么只有垂直光线被感知?为什么较“弱”的斜射光线不会被较弱地感知?
他后来提出的“折射光线被看作垂直”的观点,[59] 从逻辑上也不具备足够的说服力。尽管如此,这一理论在当时仍是最为全面的光学体系,并具有极其深远的影响,尤其是在西欧中世纪至近代早期。海赛姆的《De Aspectibus》(即《光学书》的拉丁译本)直接或间接地激发了13至17世纪间大量光学研究活动。开普勒后来关于视网膜成像的理论,正是在海赛姆的概念框架基础上发展而来,最终彻底解决了“物体点与眼睛点一一对应”的问题。[60]

海赛姆通过实验证明了光沿直线传播,并进行了大量关于透镜、镜子、折射和反射的实验研究。[61] 他在分析反射与折射现象时,会将光线的垂直分量与水平分量分开考虑。[62]

海赛姆研究了视觉的形成过程、眼睛的结构、图像在眼内的成像机制以及视觉系统的整体工作原理。在1996年发表于《感知》期刊的一篇文章中,伊恩·P·霍华德指出,许多原本归功于几个世纪后西欧学者的发现与理论,其实应该归功于海赛姆。例如:他描述了后来在19世纪被称为“赫林等神经支配定律”的原理;他比阿奎洛纽斯早600年就对垂直视对应线进行了描述,而且其定义比阿奎洛纽斯的更接近现代观点;他对双眼视差的研究,在1858年被帕努姆重复过一次。[63]尽管如此,学者克雷格·阿恩-斯托克戴尔在肯定海赛姆诸多贡献的同时,也表示需要审慎对待,尤其是在将海赛姆的成就与托勒密脱离来看的时候要特别小心。海赛姆虽然纠正了托勒密在双眼视觉方面的一个重大错误,但除此之外,他的整体理论与托勒密非常相似——托勒密其实也尝试解释过类似于赫林定律的内容。[64]

总体而言,海赛姆是在继承并扩展托勒密光学体系的基础上,进一步推进光学发展的。[65]

在对伊本·海赛姆关于双眼视觉研究贡献的更为详细的分析中,雷诺基于勒琼[66]和萨布拉[67]的研究指出,[68] 在海赛姆的光学理论中,关于视点对应、同名复视和交叉复视的概念已经确立。但与霍华德的观点不同,雷诺解释了为何海赛姆没有提出以圆形形式描述视等线的原因,并指出,通过实验推理的方式,海赛姆实际上更接近于发现“帕努姆融合区”,而非后来的“维特-缪勒圆”。在这一方面,伊本·海赛姆的双眼视觉理论面临两大局限:未能认识到视网膜在视觉中的关键作用;显而易见地,缺乏对眼球通路进行系统的实验性研究。
\begin{figure}[ht]
\centering
\includegraphics[width=6cm]{./figures/062a3ea76c606192.png}
\caption{} \label{fig_YBH_2}
\end{figure}
海赛姆最具原创性的贡献在于:在描述完他对眼睛解剖结构的理解之后,他进一步思考了这种结构作为一个光学系统在功能上的表现。[69]他通过实验掌握了针孔成像的原理,这种理解显然影响了他对眼内图像倒置问题的思考,[70] 并试图设法避免图像倒置的发生。[71]他认为,垂直射入晶状体(他称之为“玻璃状液”,glacial humor)的光线,在穿出玻璃状液时还会被向外进一步折射,从而使图像在进入眼后部的视神经时依然保持正立。[72]他继承了盖伦的观点,认为晶状体是视觉感受器官,但他的一些论述也暗示他可能认为视网膜也在视觉中起到作用。[73]

海赛姆对光与视觉的综合理论严格遵循亚里士多德的体系,以逻辑、完整的方式详尽描述了视觉过程。[74]

他在镜学方面的研究主要集中于球面镜、抛物面镜以及球差问题。他还观察到:入射角与折射角之间的比率并不恒定,并研究了透镜的放大能力。[61]
\subsubsection{反射定律}
海赛姆是第一个完整陈述反射定律的物理学家。[75][76][77]他首次明确指出:入射光线、反射光线与入射点处的法线三者共面,且都位于垂直于反射面的同一平面内。[17][78]
\subsubsection{海赛姆问题}
\begin{figure}[ht]
\centering
\includegraphics[width=8cm]{./figures/ea3add2dbd5ba4cf.png}
\caption{} \label{fig_YBH_3}
\end{figure}
他在《光学书》第五卷中关于镜学的研究,涉及了一个后来被称为“海赛姆问题”的著名几何问题,该问题最早由托勒密于公元150年提出。该问题的基本形式是:从平面内的两个点作线,经过圆周上的某一点,并使这两条线与该点处的法线所成角相等。这个问题的几何意义相当于:
在一张圆形台球桌上,已知两个点,求一个圆周上的反弹点,使得一颗球从第一点击向圆边,在该点反弹后正好击中第二个点上的球。在光学中的应用则是解决这样一个问题:“已知一个光源和一个球面镜,求光线应从镜面上的哪个点反射,才能正好进入观察者的眼睛。”这个问题最终可以归结为一个四次方程。[79]为了解决它,海赛姆推导出了四次幂求和公式,而在此之前,人们只知道平方和与立方和的公式。他的方法实际上可以推广到任意次幂求和的情形,尽管他本人并没有进行这一推广(可能是因为他只需要四次幂来计算其关注的抛物旋转体的体积)。他利用所得的幂和公式,完成了我们今天所说的“积分”运算。具体而言,他用平方和与四次幂和的公式来求抛物体(抛物面旋转体)体积。[80]海赛姆最终用圆锥曲线和几何证明解决了这个问题。不过他的解法极其冗长复杂,可能在拉丁文译本中难以被后来的数学家完全理解。此后,数学家们开始使用笛卡尔的解析几何方法来分析这一问题。[81]一直到1965年,精算师杰克·M·埃尔金才首次给出该问题的代数解法。[82]1989年,哈拉尔德·里德(Harald Riede)也提出了解法,[83]1997年,牛津大学数学家彼得·M·诺伊曼给出了另一种解法。[84][85]近年来,三菱电机研究实验室的研究人员更进一步,解决了海赛姆问题在更广义的情形下的推广版本——即适用于任意旋转对称二次曲面反射镜(包括双曲面、抛物面与椭圆面)的解法。[86]
\subsubsection{暗箱}
暗箱的原理早在中国古代就已为人所知,北宋博学家沈括在其科学著作《梦溪笔谈》(1088年出版)中曾加以描述。亚里士多德也在其著作《问题集》中讨论过其基本原理,但海赛姆的研究提供了关于暗箱的第一个清晰描述[87],并进行了该装置的早期分析[88]。

伊本·海赛姆主要使用暗箱来观察日偏食。[89] 在一篇论文中,他写道,自己观察到日食时太阳呈弯月形。论文的引言写道: “日食之际,若非全食,则太阳之像,当其光穿过一小而圆的孔洞,投射到对面平面上时,便呈现出新月状的形状。”

学界普遍认为,海赛姆的发现巩固了暗箱在科学史中的重要地位,[90] 而这篇论文的重要性远不止于此。

在古代与中世纪,光学被分为两类领域:视觉光学和灼光镜学。前者聚焦于视觉机制的研究,后者则探讨光线与发光射线的物理特性。而这篇关于日食形状的论文,可能是海赛姆首次尝试将这两门学科整合为一体的尝试。

伊本·海赛姆的许多发现都源于数学与实验的交汇,这在《论日食形状》中体现尤为明显。这篇论文不仅让更多人可以研究太阳的日偏食,更重要的是促进了对暗箱成像原理的深入理解。这是一篇关于暗箱内成像的物理-数学研究。海赛姆采用实验方法,通过改变孔径的大小和形状、暗箱的焦距、光源的形状与强度,得出了具体成像结果。[91]

在这项研究中,他解释了暗箱成像的倒置现象,[92] 说明了当孔洞很小时,图像类似于原始光源;但也指出,当孔洞较大时,图像会与原始物体产生差异。这一切结果都是通过图像的点分析法得出的。[93]
\subsubsection{折射仪}
在《光学书》第七卷中,海赛姆描述了一种用于研究折射现象的实验装置,以探究入射角、折射角与偏转角之间的关系。这个装置是他对托勒密所使用仪器的改进版本,用于相似目的的实验。[94][95][96]
\subsubsection{无意识推理}
海赛姆在讨论颜色的过程中,基本上提出了“无意识推理”这一概念。他指出,从感知颜色到辨别颜色之间的推理过程,要比感知并识别其他可见特性(除光之外)所需的时间更短,“其时间之短,以致观察者无法明显察觉。” 这显然意味着,颜色与形状的感知并非发生在眼睛本身,而是另有处理之处。

海赛姆进一步说明,信息必须传递到“中央神经腔”进行加工,并写道:“感官器官在被来自可见物体的形状所作用之前,不会感知这些形状;因此,它不会将颜色当作颜色、将光当作光加以感知,除非它先受到颜色或光的‘形状’的影响。而感官器官从颜色或光的形状中接收到的影响,是一种特定的变化;而变化必定在时间中发生;……正是在从感官表面到‘共同神经腔’的过程中,以及随后的那段时间内,分布于整个感知身体中的感知能力,才会将颜色感知为颜色本身……所以,感知者对“颜色作为颜色”以及“光作为光”的真正感知,其实是在颜色的‘形状’自感官表面到达“共同神经腔”之后才发生的。”[97]
\subsubsection{颜色恒常性}
海赛姆在研究中对颜色恒常性现象进行了说明。他观察到,从物体反射的光线会受到物体颜色的调制影响。他解释说:光的性质与物体颜色会混合在一起传入眼中,而视觉系统再将光与颜色区分开来。

在《光学书》第二卷第三章中,他写道:“从有色物体发出的光,并不会脱离颜色而单独到达眼睛;颜色的形状也不会脱离光而单独进入眼睛。存在于有色物体中的光的形状与颜色的形状,只能混合在一起传播,而感知器官只能以混合状态来感知它们。然而,感知者仍能意识到:可见物体是发光的,且物体中所见的光不同于颜色,这二者是两个不同的性质。”[98]
\subsubsection{其他贡献}
《光学书》中描述了海赛姆进行的多项实验观察,以及他如何利用实验结果通过机械类比来解释某些光学现象。他曾进行抛射体实验,得出结论:只有垂直撞击表面的抛射物才有足够的力量穿透它,而倾斜撞击则倾向于被表面反弹开。例如,为了解释从稀介质到密介质的折射现象,他借用了一个机械类比:将一个铁球投向覆盖在金属板大孔上的薄石板。如果*垂直投掷,铁球会击破石板并穿过孔洞;而等距、等力的斜向投掷则无法击穿石板。[99]他还利用这一类比来解释为何强烈的直射光会伤害眼睛:海赛姆将“强光”类比为垂直射入的光线,而“弱光”则对应斜射光。对于“一个物体表面多个光线入眼”的问题,他的解决方式就是:选择那条垂直射入的光线,因为从物体表面每个点出发,只有一条垂直光线能穿透眼球。[100]

苏丹心理学家奥马尔·哈利法认为,海赛姆应被视为实验心理学的奠基人,因为他在视觉感知心理学与视觉错觉研究方面有开创性的工作。[101]
哈利法还进一步提出,海赛姆应被视为“心理物理学”的奠基者,这是一门现代心理学的先驱学科。[101]然而,尽管海赛姆在视觉方面写下了许多主观描述,目前并无证据表明他采用了定量的心理物理学方法,因此这一主张已受到反驳。[102]

海赛姆曾对月亮错觉提出了解释,这一错觉在中世纪欧洲科学传统中占据了重要地位。[103]
许多学者都试图解释:为什么月亮在靠近地平线时看起来比在高空中更大。海赛姆反对托勒密关于大气折射的解释,并将这一现象定义为感知上的放大,而非真实的放大。他指出,判断物体距离的过程依赖于观察者与物体之间是否存在连续的中介物体。当月亮位于高空时,其与观察者之间没有中介物体,因此人们会感知它距离较近。由于一个具有恒定视角大小的物体,其感知大小会随着感知距离的不同而变化——因此,月亮在高空时看起来更近也更小,而在地平线附近时则看起来更远也更大。海赛姆的解释通过罗杰·培根、约翰·佩克汉和维特罗等人的著作被传播开来,使得“月亮错觉”逐渐被接受为一种心理现象,而托勒密的折射理论则在17世纪被逐步放弃。[104]尽管“感知距离解释”常被归功于海赛姆,他却并非第一个提出这一观点的学者。克勒俄美得斯(约公元2世纪)就提出了这种解释(并同时提到折射理论),而他将这一理论归功于波西多尼乌斯(约公元前135–50年)。[105]托勒密在其《光学》一书中可能也提到过这种解释,但原文较为晦涩难解。[106]由于海赛姆的著作在中世纪时期比上述早期学者的著作更为广泛传播,这或许就是他最终获得该理论广泛认同和归功的原因。
\subsection{科学方法}
“因此,真正追求真理的人,并不是那种仅仅研读古人著作、凭借自己的天性便盲目相信其中内容的人,而应是对自己所信持怀疑态度、对所获取的知识进行质疑的人,是那种只服从论证与证据,而不服从某个凡人言语的人,因为人性充满各种不完美与缺陷。一个以求真为目标、研究科学家著作的人,责任在于成为所读之物的敌人,……从各个角度予以攻击。他在批判性地审查这些著作时,也应对自己保持怀疑,以避免落入偏见或过度宽容之中。”

—— 海赛姆(Alhazen)[67]

与海赛姆光学研究相关的一个重要方面,是他在科学探索中系统性地依赖实验(阿拉伯语:iʿtibār,اختبار)与可控测试的方法论原则。此外,他的实验指导思想基于将古典物理学(ʿilm ṭabīʿī)与数学(taʿālīm,特别是几何学)相结合。这种数学-物理结合的实验科学方法支撑了他在《光学书》(Kitāb al-Manāẓir,拉丁译名 De aspectibus 或 Perspectivae)中的大部分命题,并构成了他关于视觉、光与颜色理论的基础,也支撑了他在镜学与透镜学方面的研究,即对光的反射与折射现象的系统探索。[107][108]

据马蒂亚斯·施拉姆所言,[109] 海赛姆是第一个系统性地使用“在实验中以恒定且一致方式改变实验条件”的方法的人。他曾设计一个实验:在屏幕上投影通过两个小孔的月光斑点,并观察当逐渐遮挡其中一个小孔时,光斑强度随之持续减弱,从而展示光强与孔径的直接关系。[110]然而,学者G. J. 图默对施拉姆的观点持保留态度,[111] 部分原因是1964年时《光学书》的阿拉伯文原文尚未完整翻译,图默担心在缺乏上下文的情况下,特定段落可能会被历史错置地解读。图默在承认海赛姆在发展实验技术方面的重要性同时,指出不应将海赛姆与其他伊斯兰与古代思想家割裂开来单独看待。[111]他在评论结尾中表示:在更多海赛姆著作被翻译、其对后期中世纪作者的影响被充分研究之前,尚无法断言他是否应被称为“现代物理学的真正奠基人”。[112]
\subsection{关于物理学的其他著作}
\subsubsection{光学论文}
除了《光学书》之外,海赛姆还撰写了多篇关于相同主题的论文,其中包括《论光》。他研究了光亮度的性质、彩虹、日食、黄昏和月光等现象。他通过在空气、水和玻璃的界面上进行的实验,使用立方体、半球和四分之一球体镜面,为其镜学理论奠定了基础。[113]
\subsubsection{天体物理学}
海赛姆在其《天文学概要》中探讨了天体区域的物理机制,他主张托勒密的模型必须以物理实体的方式加以理解,而非抽象的数学假设。换句话说,他认为应该可以构建一种不会发生天体相撞的物理模型(例如地心模型中的行星系统)。这种为地心托勒密模型提出机械结构的建议,“极大地推动了托勒密体系在西方基督教世界的最终胜利”。更重要的是,海赛姆坚持将天文学建立在物理实在的基础上,意味着天文学的假设必须遵循物理法则,并据此受到批评、改进和验证。[114]

他还撰写了《论月光》。
\subsubsection{力学}
在其著作中,海赛姆还探讨了物体运动的理论。[113]
\subsection{天文学著作}
\subsubsection{《论世界的构造》}
在其著作《论世界的构造》中,海赛姆对地球的物理结构进行了详细描述:

“整个地球是一个圆形的球体,其中心即为世界的中心。它静止地位于世界的中心,固定不动,既不向任何方向移动,也不参与任何形式的运动,而是始终处于静止状态。”[115]

这部著作是对托勒密《天文学大成》的非技术性解说。该书最终于13至14世纪被翻译为希伯来文和拉丁文,并在欧洲中世纪与文艺复兴时期对诸如格奥尔格·冯·皮尔巴赫等天文学家产生了重要影响。[116][117]
\subsubsection{《质疑托勒密》}
在其著作《质疑托勒密》(Al-Shukūk ʿalā Baṭlamyūs,亦译作《反驳托勒密》或《对托勒密的疑问》)中,海赛姆对托勒密的《天文学大成》、《行星假说》以及《光学》提出了批评。该书成书于1025年至1028年间。在书中,海赛姆指出了这些著作中,尤其是在天文学方面所存在的多处自相矛盾之处。托勒密的《天文学大成》主要是关于行星运动的数学理论,而《行星假说》则试图描述托勒密所认为的行星实际构造。托勒密本人也承认,他的理论与构造图并不总是一致,并辩称只要不会导致明显误差,这种不一致并无大碍。但海赛姆对这些内在矛盾提出了尖锐批评。[118]他认为托勒密在天文学中引入的一些数学工具——尤其是均差点,违背了天体运动必须为匀速圆周运动的物理要求。他指出,把真实的物理运动建立在想象中的数学点、线和圆之上是荒谬的:[119]

“托勒密所设想的构型(hay’a)根本不可能存在,即使他在想象中用这个构型成功地再现了行星的运动,也无法掩盖他在假设构型中所犯下的错误。因为现实中行星的运动不可能是某种不可能存在的构型的产物……
一个人想象天空中有一个圆圈,并设想行星在其中运动,并不能因此就产生行星的真实运动。”[120]

在指出这些问题之后,海赛姆似乎打算在后续著作中尝试解决托勒密体系中的这些矛盾。他相信,行星运动应有一个“真正的结构”,而托勒密并未完全掌握。他的目标是完善和修补托勒密的体系,而并非彻底替代它。[118]在《质疑托勒密》中,海赛姆还表达了他对科学知识难以获得以及质疑权威与现有理论的必要性的深刻看法:

真理应为自身而被追寻,但他提醒说:真理被不确定性所包围,而科学权威(如他十分尊敬的托勒密)也不免犯错……”[67]

他认为,对现有理论的批判——正如这本书所主导的那样——在科学知识的发展中占据特殊地位。
\subsubsection{《七大行星各自运动的模型》}
海赛姆的《七大行星各自运动的模型》写于约公元1038年。目前仅存一份残缺的手稿,仅保存了引言和第一部分,即关于行星运动理论的部分。(据推测,此书还应包含第二部分:天文计算,以及第三部分:天文仪器。)作为《质疑托勒密》的延续,海赛姆在此书中提出了一个新的、基于几何学的行星运动模型,他使用球面几何、微几何和三角学来描述行星的运动。他保留了地心宇宙模型,并假定天体运动为匀速圆周运动,这就需要引入本轮和均轮来解释观测到的行星逆行等现象,但他设法消除了托勒密模型中的“均差点”。总体而言,海赛姆的模型不试图提供行星运动的因果解释,而是专注于构建一个完整且几何自洽的描述系统,可以解释观测结果,同时避免托勒密体系内在的矛盾。[121]
\subsubsection{其他天文学著作}
海赛姆共撰写了25部天文学著作,内容大致可分为四类:技术性著作,如《子午线的精确测定》;精确观测类著作,关注天文观测技术;探讨具体天文学问题与现象的著作,如银河的位置等。他对银河视差进行了首次系统性的分析尝试,结合托勒密的数据与自身观测,得出结论:银河的视差远小于月球的视差,因此银河应属于天体,而非大气层的一部分。虽然早在他之前已有学者提出“银河非属大气”的观点,但海赛姆是第一个对此进行定量分析的人。[122]天文理论类著作,共十部,其中包括前述的《质疑托勒密》和《七大行星的运动模型》这两部。[123]
\subsection{数学著作}
在数学方面,海赛姆在欧几里得与撒比特·伊本·库拉的基础上继续发展,并致力于探索代数学与几何学之间联系的起点。他在圆锥曲线和数论方面也有所建树。[124]他还推导出前100个自然数求和的公式,并利用几何证明对这一公式进行了论证。[125]
\subsubsection{几何学}
\begin{figure}[ht]
\centering
\includegraphics[width=6cm]{./figures/dedeb152585a56a0.png}
\caption{海赛姆的弯月形:这两个蓝色弯月形的总面积与绿色的直角三角形面积相等。} \label{fig_YBH_4}
\end{figure}
海赛姆研究了现今被称为欧几里得平行公设的问题,即欧几里得《几何原本》第五公设。他采用反证法加以探讨,[126] 并在实践中引入了“运动”这一概念到几何中,[127] 可谓开创性尝试。他构造了兰伯特四边形,该形被鲍里斯·罗岑菲尔德命名为“伊本·海赛姆–兰伯特四边形”。[128]
不过,他的做法也遭到奥马尔·海亚姆的批评,后者指出亚里士多德曾反对在几何学中使用运动的概念。[129]

在初等几何中,海赛姆尝试用弯月形面积来解决“化圆为方”的问题,但最终放弃了这一不可能完成的任务。[130]他构造了两个著名的“海赛姆弯月”:以直角三角形为基础,在三边上分别作半圆——斜边朝内,两直角边朝外。所得的两个弯月形的总面积等于该三角形面积。[131]
\subsubsection{数论}
在数论方面,海赛姆的贡献包括他对完全数的研究。在其《分析与综合》中,他可能是第一个指出每一个偶完全数都可以写成形式:$2^{n-1}(2^n - 1)$其中 $2^n - 1$ 为素数。但他并未能证明这一命题;后来由欧拉在18世纪完成证明,这一定理如今称为欧几里得–欧拉定理。[130]

海赛姆还研究了同余问题,并使用类似威尔逊定理的方法解决问题。在他的《小论文集》中,他探讨了一个同余方程组的求解,并给出了两种通用解法:第一种方法是“规范方法”,涉及到威尔逊定理;第二种方法采用了中国剩余定理的某种形式。[130]
\subsubsection{微积分}
海赛姆发现了四次幂求和公式,其方法实际上可以推广至任意整数幂求和。他利用这一结果来求解抛物旋转体(抛物面体)的体积。[132]
尽管他没有建立出一般性公式,但他能够为任意多项式函数找到积分表达式,这在当时是极为先进的成果。
\subsection{其他著作}
\subsubsection{《旋律对动物灵魂的影响》}
海赛姆还写过一篇题为《旋律对动物灵魂的影响》的论文,但该书现已无存抄本。据记载,该论文探讨了动物是否能对音乐作出反应的问题,例如:一只骆驼听到音乐时会加快或放慢步伐。
\subsubsection{工程学}
在工程方面,有一则关于他担任土木工程师的记载称,海赛姆曾被法蒂玛王朝的哈里发哈基姆召至埃及,以研究如何调控尼罗河泛滥。他对尼罗河的年度洪水进行了详尽的科学研究,并提出了在今日阿斯旺大坝所在地建坝的规划。
但在进行实地勘察之后,他意识到这一工程在技术上不可行,于是他假装疯癫以避免因未能完成任务而遭到哈里发惩罚。[133]
\subsubsection{哲学}
在其《论空间》一书中,海赛姆反对亚里士多德“自然厌恶真空”的观点,他尝试用几何学论证:空间(是容器内部表面之间所构成的三维虚空。[134]持亚里士多德“空间哲学”观点的阿卜杜勒·拉蒂夫后来在其著作《反驳伊本·海赛姆关于空间的观点》中批评海赛姆“将空间几何化”的做法。[134]

在《光学书》中,海赛姆也探讨了空间感知及其认识论意义。他认为,视觉对空间的感知必须依赖于先前身体经验的参与,因而明确否定了空间感知的直观性,也否定了视觉的自主性:“如果缺乏可触知的距离与大小概念作为参照,那么视觉几乎无法为我们提供关于这些事物的任何信息。”[135]
\subsubsection{神学}
海什木是一位穆斯林,大多数资料表明他是逊尼派穆斯林,并属于阿什阿里学派。[136][137][138][139] 齐亚乌丁·萨达尔指出,一些最杰出的穆斯林科学家,如伊本·海什木和比鲁尼,作为科学方法的先驱者,本身就是阿什阿里学派的追随者。[138] 和其他阿什阿里派信徒一样,海什木认为信仰应当仅限于伊斯兰教,而不应盲从古希腊—希腊化时期的权威。[140] 他主张信仰的对象只能是伊斯兰的先知,而不是其他任何权威人物,这一思想构成了他在《对托勒密的质疑》和《光学书》中对托勒密及其他古代权威进行科学怀疑与批判的理论基础。[141]

海什木还撰写过一部关于伊斯兰神学的著作,讨论了先知制度,并在书中提出了一套哲学标准,用以识别他那个时代伪先知的虚假主张。[142] 他还写了一篇题为《用计算确定朝向克尔白的方向》的小论文,探讨了如何用数学方法确定穆斯林礼拜时所面向的克尔白方向。[143]

他的技术著作中偶尔也会提及神学或宗教情感,例如在《对托勒密的质疑》中写道:

真理是为真理本身而被追寻的……寻求真理是困难的,其道路是崎岖的,因为真理深藏于晦暗之中……然而,上帝并未使科学家免于错误,也未使科学摆脱缺陷与不足。若果真如此,科学家们便不会在任何科学问题上存在分歧……[144]

在《曲折运动》中,他批评道:

从这位尊贵谢赫的言论中可以明显看出,他对托勒密的言辞言听计从,毫不求证,也不诉诸论证,而是单纯地盲从;这种信仰方式就如同传述圣训的专家们对先知的信仰,愿真主赐福于他们。然而这并不是数学家们对证明性科学专家所抱持的信仰方式。[145]

关于客观真理与上帝之间的关系,他写道:

我不断追求知识与真理,并坚信:若要通达神圣的光辉与亲近真主,没有比寻求真理与知识更优越的道路了。[146]
\subsection{遗产}
\begin{figure}[ht]
\centering
\includegraphics[width=6cm]{./figures/963f7c4f039190cc.png}
\caption{} \label{fig_YBH_5}
\end{figure}
海什木在光学、数论、几何、天文学以及自然哲学领域都有重要贡献。他在光学方面的研究被认为为实验方法带来了新的重视。

他的代表作《光学书》在穆斯林世界广为流传,尤其是通过13世纪学者卡马勒·丁·法里西所作的注释本《光学的修订与洞察者之辨析》。[147] 在安达卢斯,该书被萨拉戈萨的胡德王朝十一世纪的王子兼重要数学著作作者穆塔曼·伊本·胡德所采用。该书的大致于12世纪末或13世纪初被译为拉丁文。[148]这部拉丁译本对基督教欧洲的多位学者产生了深远影响,包括:罗杰·培根、[149] 罗伯特·格罗塞泰斯特、[150] 维特鲁、贾姆巴蒂斯塔·德拉·波尔塔(、[151] 列奥纳多·达·芬奇、[152] 伽利略·伽利莱、[153] 克里斯蒂安·惠更斯、[154] 勒内·笛卡尔、[155] 和约翰内斯·开普勒。[156]在伊斯兰世界,海什木的光学著作亦对阿维罗伊(Averroes)的光学理论产生影响(引文待补),其思想又通过波斯科学家卡马勒·丁·法里西(卒于约1320年)在《光学的修订》一书中的“重构”得以发扬光大。[108]海什木据说著述多达200部,但目前仅存55部。他关于光学的部分论文仅通过拉丁译本流传下来。在中世纪,他关于宇宙论的著作被译成拉丁文、希伯来文以及其他语言。

英国科学史学家 H. J. J. 温特在总结伊本·海什木在物理学史上的重要地位时写道:

自阿基米德去世之后,直至伊本·海什木之前,再未出现过真正伟大的物理学家。因此,如果我们仅将视野局限在物理学史上,那么从古希腊的黄金时代到穆斯林经院哲学的时代之间,存在着一段长达一千二百多年的空白。而古代最崇高的物理学家所秉持的实验精神,终于在这位来自巴士拉的阿拉伯学者身上得以复兴。[157]

尽管在伊斯兰中世纪时期,仅有一部关于海什木光学著作的注释本得以保存,但杰弗里·乔叟(Geoffrey Chaucer)在《坎特伯雷故事集》中提及了他的著作:[158]

“他们谈论着海森与维特鲁,
还有亚里士多德——这些人一生都在书写,
关于奇异的镜子与光学器具。”

为了纪念他,月球上的“海什木陨石坑”(Alhazen)以其名字命名,[159] 小行星“59239 Alhazen”亦同样以他命名。[160] 巴基斯坦的阿迦汗大学为表彰海什木的贡献,将其眼科学教席命名为“伊本·海什木副教授暨眼科主任”。[161]

2015年国际光年纪念了伊本·海什木关于光学研究的一千周年。[162]
\begin{figure}[ht]
\centering
\includegraphics[width=6cm]{./figures/153ebfe54888487e.png}
\caption{} \label{fig_YBH_6}
\end{figure}
2014年,尼尔·德格拉斯·泰森主持的纪录片《宇宙:时空漫游》中,播出了一集名为《光中隐匿》的节目,聚焦于伊本·海什木的成就。在该集中,阿尔弗雷德·莫里纳为海什木配音。

四十多年前,雅各布·布罗诺夫斯基也在类似的电视纪录片(及其同名图书)《人类的崛起》中介绍了海什木的研究。在第五集《天球的音乐》中,布罗诺夫斯基评论说,在他看来,海什木是“阿拉伯文化所孕育出的唯一真正原创的科学头脑”,其光学理论直到牛顿与莱布尼茨时代才得以超越。

联合国教科文组织将2015年定为“国际光年”,其总干事伊琳娜·博科娃称伊本·海什木为“光学之父”。[163] 这一纪念活动旨在庆祝伊本·海什木在光学、数学与天文学领域的成就。由“1001项发明”组织发起的国际宣传活动《1001项发明与伊本·海什木的世界》通过互动展览、工作坊、现场演出等形式向公众介绍其研究成果,该项目与科学中心、科学节、博物馆及教育机构合作,同时也通过数字媒体与社交平台进行推广。[164] 该项目还制作并发布了教育短片《1001项发明与伊本·海什木的世界》。

伊本·海什木的肖像也出现在伊拉克第2003年版的10,000第纳尔纸币上。[165]
\subsection{著作目录}
据中世纪传记作者记载,海什木共撰写了200余部著作,涵盖广泛主题,其中至少有96部属于科学著作。目前大多数著作已佚失,但仍有50多部部分或全部保存下来。现存作品中,约一半属于数学领域,23部为天文学著作,14部为光学著作,另有少量涉及其他主题。[166] 并非所有现存作品都已被研究过,以下是其中部分已被研究的作品:[167]
\begin{enumerate}
\item 《光学书》
\item 《分析与综合》
\item 《智慧之衡》
\item 《〈天文学大成〉的校正》
\item 《关于“位置”的论述》
\item 《极点的精确测定》
\item 《子午线的精确定义》
\item 《通过计算确定朝向克尔白》
\item 《水平日晷》
\item 《时线研究》
\item 《对托勒密的质疑》
\item 《关于克拉斯通的论文》
\item 《圆锥曲线的补全》
\item 《观星记》
\item 《圆的求积》
\item 《燃烧球面论》
\item 《世界的构造》
\item 《日食的形式》
\item 《星光之本性》[168]
\item 《月亮的光》
\item 《银河探论》
\item 《影之本质》
\item 《彩虹与晕圈》
\item 《小品集》
\item 《对〈天文学大成〉质疑的分析》
\item 《对“曲折运动”质疑的分析》
\item 《天文计算的修正》
\item 《行星的高度差异》
\item 《麦加方向的测定》
\item 《七大行星各自运动模型》
\item 《宇宙模型》
\item 《月亮的运动》
\item 《各时弧与其高度之比》
\item 《曲折运动论》
\item 《光学论文》[169]
\item 《位置论文》
\item 《旋律对动物灵魂的影响》
\item 《几何问题分析书》
\item 《算术原理大全》
\item 《球体测度论》
\item 《交易计算中的奇异算法》
\item 《关于垂线边三角形的性质》
\item 《等容体体积论文》
\item 《欧几里得原理注解》
\item 《抛物反射镜研究》
\item 《重心论》
\end{enumerate}
\subsubsection{佚失著作}
\begin{enumerate}
\item 《我所编写的一本书,汇总了欧几里得与托勒密两部著作中的光学知识,并补充了托勒密著作中缺失的第一篇论述的相关概念》[170]
\item 《燃烧镜论文》
\item 《关于视器本质及其如何实现视觉的论文》
\end{enumerate}
\subsection{参见}
\begin{itemize}
\item 伊本·苏菲
\item 《光中隐匿》
\item 数学史
\item 理论物理学
\item 光学史
\item 物理学史
\item 科学史
\item 科学方法史
\item 霍克尼–法尔科假说
\item 伊斯兰教中世纪数学
\item 伊斯兰教中世纪物理
\item 伊斯兰世界的中世纪科学
\item 法蒂玛·菲赫里
\item 伊斯兰黄金时代
\end{itemize}
\subsection{注释}\\
a. 马克·史密斯通过对译文中译者对阿拉伯语掌握程度的分析,判断出至少有两位译者参与了该翻译工作:第一位较为资深的学者从第一卷开始翻译,翻译至第三卷第三章中途时将工作交给了他人。参见 Smith 2001 年第1卷《注释与拉丁文本》第 xx–xxi 页。此外可参阅他在2006年、2008年和2010年的译本。
\subsection{参考文献}
\begin{enumerate}
\item Lorch, Richard(2017年2月1日)。《伊本·海什木:阿拉伯天文学家与数学家》,《大英百科全书》。原文存档于2018年8月12日。检索于2022年1月14日。
\item O'Connor 与 Robertson, 1999。
\item El-Bizri, 2010,第11页:“伊本·海什木在光学方面的开创性研究,包括他在反射光学(catoptrics)与折射光学(dioptrics)中的探索(分别研究光的反射和折射原理及相关仪器的学科),主要汇集于他那部具有纪念意义的巨著:《光学书》(Kitāb al-Manāẓir,拉丁文译名为 De Aspectibus 或 Perspectivae,成书时间为公元1028年至1038年之间)。”
\item Rooney, 2012,第39页:“作为一位严谨的实验物理学家,他有时被认为是科学方法的发明者。”
\item Baker, 2012,第449页:“如前所述,伊本·海什木是最早开展动物心理学实验的学者之一。”
\item “Alhazen”也被称作Alhacen、Avennathan、Avenetan等;“Alhazen”与巴士拉的伊本·海什木是同一人,这一身份确认大约是在19世纪末期。(Vernet, 1996,第788页)
\item “Ibn al-Haytham”,《美语传统词典》(第五版),HarperCollins出版社。检索于2019年6月23日。
\item Esposito, John L.(2000年),《牛津伊斯兰教史》,牛津大学出版社,第192页:“伊本·海什木(卒于1039年),西方称其为Alhazan,是阿拉伯著名的数学家、天文学家与物理学家。他的光学巨著《光学书》是中世纪最杰出的光学著作。”
\item 关于他的主要研究领域,参见如 Vernet 1996,第788页:“他是最重要的阿拉伯数学家之一,毫无疑问,是最杰出的物理学家。”;Sabra 2008,Kalin, Ayduz 与 Dagli 2009:“伊本·海什木是11世纪著名的阿拉伯光学家、几何学家、算术家、代数学家、天文学家与工程师。”;Dallal 1999:“伊本·海什木(卒于1039年),西方称为Alhazan,是阿拉伯著名的数学家、天文学家与物理学家。他的光学巨著《光学书》是中世纪最杰出的光学著作。”
\item Masic, Izet(2008年),“伊本·海什木——光学之父与视觉理论的阐述者”,《医学档案》,第62卷第3期:183–188页,PMID: 18822953。
\item “国际光年:伊本·海什木,现代光学的先驱,在教科文组织获表彰”,联合国教科文组织。原文存档于2015年9月18日。检索于2018年6月2日。
\item Al-Khalili, Jim(2009年1月4日)。“‘第一位真正的科学家’”,BBC 新闻。原文存档于2015年4月26日。检索于2018年6月2日。
\item Selin, 2008:“在气象学领域最具代表性的三位伊斯兰学者是:亚历山大城的数学家与天文学家伊本·海什木(Alhazen,965–1039)、讲阿拉伯语的波斯医生伊本·西那(Avicenna,980–1037),以及西班牙穆斯林医生兼法学家伊本·鲁世德(Averroes,1126–1198)。”
联合国教科文组织称他为“现代光学之父”。引自《科学对社会的影响》,UNESCO,第26–27期,第140页,1976年。原文存档于2023年2月5日。检索于2019年9月12日。
“国际光年——伊本·海什木与阿拉伯光学的遗产”,[www.light2015.org。原文存档于2014年10月1日。检索于2017年10月9日。](http://www.light2015.org。原文存档于2014年10月1日。检索于2017年10月9日。)
“国际光年:现代光学先驱伊本·海什木在联合国教科文组织获表彰”,UNESCO。原文存档于2015年9月18日。检索于2017年10月9日。
他是第一位明确解释“视觉是由于光线从物体反射后进入眼睛而产生”的人。
\item Adamson, Peter(2016)。《伊斯兰世界的哲学:无缝隙的哲学史》,牛津大学出版社,第77页,ISBN 978-0-19-957749-1。原文存档于2023年2月5日。检索于2016年10月3日。
\item Baker, 2012,第445页。
\item Rashed, Roshdi(2019年4月1日)。“费马与最短时间原理”,《法兰西科学院力学纪要》,第347卷第4期:357–364页。Bibcode:2019CRMec.347..357R,doi:10.1016/j.crme.2019.03.010,ISSN 1631-0721,S2CID 145904123。
\item vSelin, 2008,第1817页。
\item Boudrioua, Azzedine;Rashed, Roshdi;Lakshminarayanan, Vasudevan(2017)。《基于光的科学:技术与可持续发展,伊本·海什木的遗产》,CRC出版社,ISBN 978-1-351-65112-7。原文存档于2023年3月6日。检索于2023年2月22日。
\item Haq, Syed(2009),“伊斯兰中的科学”,《牛津中世纪词典》,ISSN 1703-7603。检索于2014年10月22日。
\item G. J. Toomer,在JSTOR编号228328,第464页。Toomer于1964年对马提亚斯·施拉姆(Matthias Schramm)1963年所著《伊本·海什木通往物理学之路》的评论中写道:[施拉姆总结了伊本·海什木在科学方法发展方面的成就]。原文存档于2017年3月26日,Wayback Machine。
\item “国际光年——伊本·海什木与阿拉伯光学的遗产”,原文存档于2014年10月1日。检索于2015年1月4日。
\item Gorini, Rosanna(2003年10月),“经验之人海什木:视觉科学的初步探索”,《国际伊斯兰医学史学会期刊》,第2卷第4期:53–55页。原文PDF于2022年10月9日存档。检索于2008年9月25日。
\item Roshdi Rashed,《伊本·海什木的几何方法与数学哲学:阿拉伯科学与数学史》第5卷,Routledge出版社(2017年),第635页。
\item 根据阿尔·基夫提的记载。参见 O'Connor 与 Robertson,1999年。
\item O'Connor 与 Robertson, 1999。
\item O'Connor 与 Robertson, 1999。
\item 存疑记载:Corbin, 1993,第149页。
\item 被阿布·哈桑·拜哈基(Abu'l-Hasan Bayhaqi,约1097–1169年)以及
萨布拉(Sabra, 1994年)记录,第197页。原文存档于2023年2月5日,Wayback Machine。
卡尔·博耶(Carl Boyer, 1959年),第80页。
\item 林德伯格(Lindberg, 1967年),第331页:“佩克汉持续地尊崇海什木的权威,称他为‘作者’或‘物理学家’。”
\item A. Mark Smith(1996年),《托勒密的视觉感知理论:<光学>的英文译本》,美国哲学学会,第57页,ISBN 978-0-87169-862-9。原文存档于2023年2月5日。检索于2019年8月16日。
\item Simon, 2006年。
\item 格雷戈里(Gregory, Richard Langton,2004年),《牛津心智指南》,牛津大学出版社,第24页,ISBN 978-0-19-866224-2。原文存档于2023年12月4日。检索于2023年6月28日。
\item “海森:阿拉伯数学家与物理学家,约生于965年,今伊拉克境内。”引自《乔叟批判指南:关于其生平与作品的文学参考书》
\item 埃斯波西托(Esposito, 2000),《牛津伊斯兰教史》,牛津大学出版社,第192页:“伊本·海什木(卒于1039年),西方称为Alhazan,是杰出的阿拉伯数学家、天文学家与物理学家。他的光学总集《光学书》是中世纪最伟大的光学著作。”
\item 哈里·瓦尔沃格利斯(Varvoglis, Harry,2014年1月29日),《物理学概念的历史与演进》,Springer出版社,第24页,ISBN 978-3-319-04292-3。原文存档于2023年6月20日。检索于2023年3月13日。
\item 《化学新闻与工业科学期刊》,1876年1月6日,第59页。原文存档于2023年3月26日。检索于2023年3月13日。
\item Hendrix, John Shannon;Carman, Charles H.(2016年12月5日)。《文艺复兴时期的视觉理论》,由John Shannon Hendrix与Charles Carman编辑,Routledge出版社,第77页,ISBN 978-1-317-06640-8。原文存档于2023年6月20日。检索于2023年3月13日。
\item Suhail Zubairy, M.(2024年1月6日)。《量子力学入门:兼论量子通信应用》,牛津大学出版社,第81页,ISBN 978-0-19-885422-7。原文存档于2023年6月20日。检索于2023年3月13日。
\item (Child, Shuter & Taylor, 1992年,第70页)、(Dessel, Nehrich & Voran, 1973年,第164页),引自《理解历史》,作者为John Child、Paul Shuter 和 David Taylor,第70页:“波斯科学家海森表明,眼睛是通过接收其他物体发出的光而看见的。这开创了光学——即研究光的科学。阿拉伯人还研究了天文学,即研究星体的学科。”
\item Tbakhi, Abdelghani;Amr, Samir S.(2007年)。《伊本·海什木:现代光学之父》,《沙特医学年鉴》,第27卷第6期:464–467页。doi:10.5144/0256-4947.2007.464,ISSN 0256-4947,PMC: 6074172,PMID: 18059131。
\item Corbin, 1993年,第149页。
\item 《哈基姆的囚徒》,克利夫顿,新泽西:Blue Dome出版社,2017年,ISBN 1682060160。
\item 卡尔·布罗克尔曼,《阿拉伯文学史》,第一卷(1898年),第469页。
\item 《伊斯兰大百科全书》,Cgie.org.ir。原文存档于2011年9月30日。检索于2012年5月27日。
\item 关于伊本·海什木的生平与著作,Smith(2001年)第cxix页推荐阅读 Sabra(1989年),第二卷第xix–lxxiii页。
\item “A. I. Sabra,encyclopedia.com,《伊本·海什木,阿布》”,原文存档于2023年3月26日。检索于2018年11月4日。
\item Sajjadi, Sadegh,“海森”,《伊斯兰大百科全书》,第1卷,第1917号条目。
\item Al-Khalili, 2015年。
\item Crombie, 1971年,第147页,注释2。
\item 恩里科·纳尔杜奇(Enrico Narducci,1871年):“关于十四世纪对海森《光学论》所作意大利语翻译的注记”,刊于《数学与物理科学文献与历史公报》,第4卷,第1–40页。关于此版本,参见 Raynaud 2020年,第139–153页。
\item “海森(965–1040)”,美国国会图书馆引文,马拉斯皮纳《伟大著作集》,原文存档于2007年9月27日,检索于2008年1月23日。
\item Smith, 2001年,第xxi页。
\item Smith, 2001年,第xxii页。
\item Smith, 2001年,第lxxix页。
\item Lindberg, 1976年,第73页。
\item Lindberg, 1976年,第74页。
\item Lindberg, 1976年,第76页。
\item Lindberg, 1976年,第75页。
\item Lindberg, 1976年,第76–78页。
\item Lindberg, 1976年,第86页。
\item Al Deek, 2004年。
\item Heeffer, 2003年。
\item Howard, 1996年。
\item Aaen-Stockdale, 2008年。
\item Wade, 1998年,第240、316、334、367页;Howard & Wade, 1996年,第1195、1197、1200页。
\item Lejeune, 1958年。
\item Sabra, 1989年。
\item Raynaud, 2003年。
\item Russell, 1996年,第691页。
\item Russell, 1996年,第689页。
\item Lindberg, 1976年,第80–85页。
\item Smith, 2004年,第186、192页。
\item Wade, 1998年,第14页。
\item Smith, A. Mark(2001年),《海森视觉感知理论:〈视觉论〉前三卷的校勘本、英文译文与评注,拉丁文版〈De Aspectibus〉,即伊本·海什木〈光学书〉的中世纪拉丁译本:第二卷》,刊于《美国哲学学会汇刊》(,第91卷第5期:339–819页,doi:10.2307/3657357,JSTOR: 3657357。原文存档于2015年6月30日。检索于2015年1月12日。
\item Stamnes, J. J.(2017年),《聚焦区域中的波:光、声与水波的传播、衍射与聚焦》,Routledge出版社,ISBN 978-1-351-40468-6。原文存档于2023年3月31日。检索于2023年2月22日。
\item 恩斯特·马赫(Ernst Mach,2013年),《物理光学原理:历史与哲学的探讨》,Courier出版社,ISBN 978-0-486-17347-4。原文存档于2023年3月31日。检索于2023年2月22日。
\item Iizuka, Keigo(2013年),《工程光学》,Springer Science & Business Media,ISBN 978-3-662-07032-1。原文存档于2023年3月31日。检索于2023年2月22日。
\item Mach, Ernst(2013年),《物理光学原理:历史与哲学的探讨》,Courier Corporation,ISBN 978-0-486-17347-4。原文存档于2023年3月31日。检索于2023年2月22日。
\item O'Connor & Robertson,1999年,Weisstein,2008年。
\item Katz, 1995年,第165–169页,第173–174页。
\item Smith, 1992年。
\item Elkin, Jack M.(1965年),"一个看似简单的问题",《数学教师》(*Mathematics Teacher*),第58卷第3期:194–199页,doi:10.5951/MT.58.3.0194,JSTOR: 27968003。
\item Riede, Harald(1989年),"球面镜的反射,或:海森的问题",《数学实践》,第31卷第2期:65–70页。
\item Neumann, Peter M.(1998年),"关于球面镜反射的思考",《美国数学月刊》,第105卷第6期:523–528页,doi:10.1080/00029890.1998.12004920,JSTOR: 2589403,MR 1626185。
\item Highfield, Roger(1997年4月1日),"唐恩解决了古希腊人遗留的最后一个谜题",《电子电报》,676期,原文存档于2004年11月23日。
\item Agrawal, Taguchi & Ramalingam,2011年。
\item Kelley, Milone & Aveni(2005年),第83页:“海森的《光学书》首次清晰描述了这一设备。”
\item Wade & Finger(2001年):“照相暗箱的原理首次在11世纪被正确分析,当时伊本·海什木对其进行了概述。”
\item 德国物理学家艾尔哈德·维德曼首次提供了《日食的形状》的简略德文翻译:Eilhard Wiedemann(1914年)。“关于伊本·海什木的照相暗箱”,刊于《物理医学会会议记录》,第46期:155–169页。该作品现已完全出版:Raynaud 2016年。
\item Eder, Josef(1945年),《摄影史》,纽约:哥伦比亚大学出版社,第37页。
\item Raynaud, 2016年,第130–160页。
\item Raynaud, 2016年,第114–116页。
\item Raynaud, 2016年,第91–94页。
\item 《伊斯兰科学与技术史》,Fuat Sezgin,2011年。
\item Gaukroger, Stephen(1995年),《笛卡尔:一部知识分子传记》,Clarendon Press,ISBN 978-0-19-151954-3。
\item 牛顿,艾萨克(1984年),《艾萨克·牛顿的光学文献》,第1卷:1670–1672年的光学讲座,剑桥大学出版社,ISBN 978-0-521-25248-5。
\item Boudrioua, Azzedine;Rashed, Roshdi;Lakshminarayanan, Vasudevan(2017年),《基于光的科学:技术与可持续发展,伊本·海什木的遗产》,CRC Press,ISBN 978-1-4987-7940-1。
\item Boudrioua, Azzedine;Rashed, Roshdi;Lakshminarayanan, Vasudevan(2017年),《基于光的科学:技术与可持续发展,伊本·海什木的遗产》,CRC Press,ISBN 978-1-4987-7940-1。
\item Russell, 1996年,第695页。
\item Russell, 1996年。
\item Khaleefa, 1999年。
\item Aaen-Stockdale, 2008年。
\item Ross & Plug, 2002年。
\item Hershenson, 1989年,第9–10页。
\item Ross, 2000年。
\item Ross & Ross, 1976年。
\item 参见例如《视觉论》第7卷,原文存档于2018年8月18日,Wayback Machine,涉及其在折射中的实验。
\item El-Bizri, 2005a, 2005b。
\item "参见Schramm的Habilitationsschrift,《伊本·海什木通向物理学之路》,Steiner出版社,1963年,如Rüdiger Thiele(2005)《历史数学》32卷,第271–274页所引,“怀念:马蒂亚斯·施拉姆,1928–2005”,PDF,原文存档于2017年10月25日。检索于2017年10月25日。
\item Toomer, 1964年,第463–464页。
\item Toomer, 1964年,第465页
\item G. J. Toomer,《马蒂亚斯·施拉姆(Matthias Schramm,1963)〈伊本·海什木通向物理学之路〉》书评,刊于 Toomer 1964 年评论,原文存档于 2017 年 3 月 26 日,Wayback Machine:第 464 页:“施拉姆总结了伊本·海什木在科学方法发展中的成就。”第 465 页:“施拉姆无可争议地证明,伊本·海什木是伊斯兰科学传统中的重要人物,尤其是在实验技术的创造方面。”第 465 页:“只有在认真研究伊本·海什木及其他人对中世纪后期主流物理著作的影响之后,我们才能真正评估施拉姆所声称的‘伊本·海什木是现代物理学真正奠基人’的论断。”
\item El-Bizri, 2006年。
\item 杜恒,1969年,第28页。
\item 兰格尔曼,1990年,第2章第22节,第61页。
\item Lorch, 2008年。
\item Langermann, 1990年,第34–41页;Gondhalekar, 2001年,第21页。
\item Sabra, 1998年。
\item Langermann, 1990年,第8–10页。
\item Sabra, 1978b年,第121页,注释13
\item Rashed, 2007年。
\item Eckart, 2018年。
\item Rashed, 2007年,第8–9页。
\item Faruqi, 2006年,第395–396页:“在17世纪的欧洲,伊本·海什木(965–1041)提出的问题被称为‘海森问题’(Alhazen's problem)……海什木在几何和数论方面的贡献远超阿基米德传统。他还研究了解析几何,以及代数与几何之间联系的初步建立。这项工作最终推动了纯数学领域中代数与几何的和谐融合,这一融合在笛卡尔的几何分析中以及在牛顿的微积分中达到巅峰。伊本·海什木是一位在10世纪下半叶对数学、物理和天文学作出重大贡献的科学家。”
\item Rottman, 2000年,第1章。
\item Eder, 2000年。
\item Katz, 1998年,第269页:“事实上,这种方法将平行线定义为始终保持等距的直线,并引入了‘运动’这一概念到几何学中。”
\item Rozenfeld, 1988年,第65页。
\item Boyer, Carl B.;Merzbach, Uta C.(2011年),《数学史》,约翰·威利父子公司,ISBN 978-0-470-63056-3。原文存档于2023年9月7日。检索于2023年3月19日。
\item O'Connor 与 Robertson, 1999年。
\item Alsina 与 Nelsen, 2010年。
\item Katz, Victor J.(1995年),《伊斯兰与印度的微积分思想》,《数学杂志》,第68卷第3期:163–174页[尤其为165–169,173–174页],doi:10.2307/2691411,JSTOR: 2691411。
\item Plott, 2000年,第II部,第459页。
\item El-Bizri, 2007年。
\item Smith, 2005年,第219–240页。
\item Ishaq, Usep Mohamad 和 Wan Mohd Nor Wan Daud:“伊本·海什木的传记与书目综述”,《历史学:历史教育专业期刊》,第5卷第2期(2017年):第107–124页。
\item Kaminski, Joseph J.:“伊斯兰思想发展的轨迹——早期与后期两位学者的比较”,载于《当代伊斯兰治国国家》,Palgrave Macmillan,Cham,2017年,第31–70页:“例如,伊本·海什木和比鲁尼是中世纪最重要的学者之一,他们在自然科学研究中使用科学方法,且二人均为阿什阿里学派信徒。”
\item Sardar, 1998年。
\item Bettany, 1995年,第251页。
\item Anwar, Sabieh(2008年10月),“加扎利真的是伊斯兰科学的毁灭者吗?”,载于《月刊复兴》,第18卷第10期,检索于2008年10月14日。
\item Rashed, Roshdi(2007年),“伊本·海什木的天体运动学”,刊于《阿拉伯科学与哲学》,剑桥大学出版社,第17卷第1期:7–55页,第11页,doi:10.1017/S0957423907000355。
\item Plott, 2000年,第II部,第464页。
\item Topdemir, 2007年,第8–9页。
\item 引用自S. Pines的翻译,见Sambursky, 1974年,第139页。
\item Rashed, 2007年,第11页。
\item Plott, 2000年,第II部,第465页。
\item Sabra, 2007年。
\item Sabra, 2007年,第122、128–129页;Grant, 1974年,第392页指出《光学书》亦被称作《阿拉伯人海森的光学宝库》(、《视觉论》或《透视学》。
\item Lindberg, 1996年,第11页,及其他相关内容。
\item Authier, 2013年,第23页:“海森的著作启发了许多中世纪科学家,例如英国主教罗伯特·格罗塞泰斯特(Robert Grosseteste,约1175–1253)、英国方济各会士罗杰·培根(Roger Bacon,约1214–1294)、西里西亚出生的波兰修士、哲学家和学者维特鲁(Erazmus Ciolek Witelo 或 Witelon,约1230–1280),后者在1270年左右出版了关于光学的论文《透视学》,该书在很大程度上基于海森的研究。”
\item Magill & Aves, 1998年,第66页:“罗杰·培根、约翰·佩克汉和贾姆巴蒂斯塔·德拉·波尔塔只是众多受海森作品影响的思想家中的一部分。”
\item Zewail & Thomas, 2010年,第5页:“海森著作的拉丁译本影响了像(罗杰)培根和达·芬奇这样的科学家与哲学家,并成为开普勒、笛卡尔和惠更斯等数学家研究的基础……”
\item El-Bizri, 2010年,第12页:“伊本·海什木的《光学书》的拉丁版本在印刷后,被开普勒、伽利略、笛卡尔与惠更斯等科学家和哲学家阅读和引用,如纳德尔·比兹里(Nader El-Bizri)所述。”
\item Magill & Aves, 1998年,第66页:“Sabra详细讨论了海森思想对笛卡尔和惠更斯等人的光学发现所产生的影响;参见 El-Bizri, 2005a。”
\item El-Bizri, 2010年,第12页。
\item Magill & Aves, 1998年,第66页:“即使是开普勒,也使用了海森的一些思想,例如物体上的点与眼睛内的点之间的一一对应关系。可以毫不夸张地说,海森的光学理论从他的时代起,就定义了该领域的范围与目标,延续至今。”
\item Winter, H. J. J.(1953年9月),“伊本·海什木的光学研究”,《半人马座》,第3卷第1期:190–210页,Bibcode:1953Cent....3..190W,doi:10.1111/j.1600-0498.1953.tb00529.x,ISSN 0008-8994,PMID: 13209613。
\item “伊本·海什木的科学方法”,联合国教科文组织(UNESCO),2018年5月14日。原文存档于2021年10月25日,检索于2021年10月25日。
\item Chong、Lim 与 Ang(2002年),附录3,第129页。
\item NASA, 2006年。
\item “阿迦汗大学研究出版物 1995–98”,PDF原文存档于2015年1月4日。
\item “伊本·海什木与阿拉伯光学的遗产”,2015年国际光年,2015年。原文存档于2014年10月1日,检索于2015年1月4日。
\item “2015年国际光年”(PDF),PDF原文存档于2017年4月15日,检索于2017年10月10日。
\item “阿拉伯光学一千年成为2015年国际光年的重点”,联合国,原文存档于2014年11月21日,检索于2014年11月27日。
\item “伊拉克十第纳尔纸币”,en.numista.com,检索于2024年5月28日。
\item Rashed, 2002a年,第773页。
\item Rashed, 2007年,第8–9页;Topdemir, 2007年。
\item 伊本·海什木、W. Arafat 和 H. J. J. Winter(1971年),“星光论:伊本·海什木的一篇短论文”,发表于《英国科学史期刊》,第5卷第3期(1971年6月):第282–288页,JSTOR 4025317,原文存档于2022年9月21日,Wayback Machine。
\item Alhacen(约1035年)《光论》(Treatise on Light,阿拉伯语:رسالة في الضوء),引自 Shmuel Sambursky 编(1975年)《从前苏格拉底到量子物理学家的物理思想选集》,第137页。
\item 引自伊本·阿比·乌萨比亚的著作目录,见 Smith, 2001年,第91卷,第1册,第xv页。
\end{enumerate}
\subsection{资料来源翻译}
\begin{itemize}
\item Simon, G(2006年),“伊本·海什木的凝视”,《中世纪历史期刊》,第9卷第1期:89–98页,doi:10.1177/097194580500900105,S2CID: 170628785。
\item Child, John;Shuter, Paul;Taylor, David(1992年),《理解历史》,牛津:海尼曼教育出版社,ISBN 0435312111,OCLC 27338645。
\item Daneshfard, Babak(2016年),“伊本·海什木(公元965–1039年):现代视觉理论的最初阐述者”(Ibn al-Haytham (965–1039 AD), the original portrayal of the modern theory of vision),《医学传记杂志》,第24卷第2期:227–231页,Sage出版社,doi:10.1177/0967772014529050,PMID: 24737194,S2CID: 39332483。
\item Dessel, Norman F.;Nehrich, Richard B.;Voran, Glenn I.(1973年),《科学与人类命运》,纽约:麦格劳-希尔出版社,ISBN 9780070165809。
\item Masoud, Mohammad T;Masoud, Faiza(2006年),“伊斯兰如何改变医学:伊本·海什木与光学”,《英国医学杂志》,第332卷第7533期:第120页,英国医学协会,doi:10.1136/bmj.332.7533.120-a,PMC: 1326979,PMID: 16410601。
\item Masic, I(2008年),“伊本·海什木——光学之父与视觉理论的阐述者”,《医学档案》,第62卷第3期:183–188页,波黑医学科学院出版,PMID: 18822953。
\item Sweileh, Waleed M;Al-Jabi, Samah W;Shanti, Yousef I;Sawalha, Ansam F;Zyoud, Sa'ed H(2015年),《阿拉伯研究者对眼科学的贡献:文献计量与比较分析》,发表于 *SpringerPlus*,第4卷,第42篇文章,Springer出版社,doi:10.1186/s40064-015-0806-0,PMC: 4318829,PMID: 25674499。
\item Aaen-Stockdale, C. R.(2008年),《伊本·海什木与心理物理学》,发表于 Perception,第37卷第4期:636–638页,doi:10.1068/p5940,PMID: 18546671,S2CID: 43532965。
\item Agrawal, Amit;Taguchi, Yuichi;Ramalingam, Srikumar(2010年),《轴向非中心折射与折反射相机的解析前向投影模型》,发表于 欧洲计算机视觉会议,原文存档于2012年3月7日。
\item Agrawal, Amit;Taguchi, Yuichi;Ramalingam, Srikumar(2011年),《超越海森问题:带二次曲面镜的非中心折反射相机的解析投影模型》,发表于 IEEE计算机视觉与模式识别会议,CiteSeerX: 10.1.1.433.9727,原文存档于2012年3月7日。
\item Alsina, Claudi;Nelsen, Roger B.(2010年),“9.1 可平方的新月形区域”(,收录于《迷人的证明:优雅数学之旅》,Dolciani 数学阐释丛书第42卷,美国数学协会,第137–144页,ISBN 978-0-88385-348-1。
\item Arjomand, Kamran(1997年),《科学现代性的兴起:19世纪中期伊朗占星术与现代天文学的争议》,发表于 Iranian Studies,第30卷第1期:5–24页,doi:10.1080/00210869708701857。
\item Authier, André(2013年),《第三章:光的双重性》,载于《X射线晶体学的早期历史》,牛津大学出版社,ISBN 978-0-19-965984-5。
\item Baker, David B.(编)(2012年),《牛津心理学史手册:全球视角》,牛津大学出版社,ISBN 978-0-19-536655-6。
\item Bettany, Laurence(1995年),《伊本·海什木:多元文化科学教学的一个答案?》,《物理教育》,第30卷第4期:247–252页,Bibcode:1995PhyEd..30..247B,doi:10.1088/0031-9120/30/4/011,S2CID: 250826188。
\item Eckart, Andreas(2018年3月),《早期伟大论战:关于伊本·海什木关于银河位置相对于地球的研究的评论》,《阿拉伯科学与哲学》,第28卷第1期:1–30页,doi:10.1017/S0957423917000078,S2CID: 171746839。
\item El-Bizri, Nader(2005a年),《关于海森光学的哲学视角》,《阿拉伯科学与哲学》,第15卷第2期,剑桥大学出版社:189–218页,doi:10.1017/S0957423905000172,S2CID: 123057532。
\item El-Bizri, Nader(2005b年),“伊本·海什木”,收录于 Faith Wallis 编,《中世纪的科学、技术与医学:百科全书》,纽约与伦敦:劳特利奇出版社,第237–240页,ISBN 0-415-96930-1,OCLC: 218847614。
\item El-Bizri, Nader(2006年),“伊本·海什木或海森”,收录于 Josef W. Meri 编,《中世纪伊斯兰文明:百科全书》,第二卷,纽约与伦敦:劳特利奇出版社,第343–345页,ISBN 0-415-96692-2,OCLC: 224371638。
\item El-Bizri, Nader(2007年),《为哲学主权辩护:巴格达迪对伊本·海什木几何化“位置”概念的批判》,《阿拉伯科学与哲学》,第17卷,剑桥大学出版社:第57–80页,doi:10.1017/S0957423907000367,S2CID: 170960993。
\item El-Bizri, Nader(2009a年),《深度知觉:海森、贝克莱与梅洛-庞蒂》,《东方与西方》(,第5卷第1期,巴黎:法国国家科学研究中心(CNRS):第171–184页。
\item El-Bizri, Nader(2009b年),《伊本·海什木与颜色问题》,《东方与西方》,第7卷第1期,巴黎:法国国家科学研究中心(CNRS):第201–226页。
\item El-Bizri, Nader(2010年),《古典光学与通往文艺复兴的视觉传统》,收录于 John Shannon Hendrix 与 Charles H. Carman 编,《文艺复兴视觉理论》(Renaissance Theories of Vision,视觉文化与早期现代性系列),萨里郡法纳姆:Ashgate出版社,第11–30页,ISBN 978-1-4094-0024-0。
\item Burns, Robert(1999年8月8日),《一些人担心伊拉克可能正在重建大规模杀伤性武器》,《托皮卡首都日报》,原文存档于2009年3月15日,检索于2008年9月21日。
\item  Chong, S. M.;Lim, A. C. H.;Ang, P. S.(2002年),《月球摄影图谱》,剑桥大学出版社,ISBN 978-0-521-81392-1。
\item Corbin, Henry(1993年)[原法文版1964年],《伊斯兰哲学史》,由Liadain Sherrard与Philip Sherrard翻译,伦敦:凯根·保罗国际出版社,与伊斯兰出版物协会、伊斯玛仪派研究院联合出版,ISBN 0-7103-0416-1,OCLC 22109949。
\item Crombie, A. C.(1971年),《罗伯特·格罗塞泰斯特与实验科学的起源:1100–1700》,牛津大学克拉伦登出版社。
\item Dallal, Ahmad S.(1999年),《科学、医学与技术》,收录于 John L. Esposito 编,《牛津伊斯兰教史》,牛津大学出版社。
\item Al Deek, Mahmoud(2004年),“伊本·海什木:光学、数学、物理与医学大师”,发表于《Al Shindagah》杂志(2004年11月–12月),原文存档于2008年6月17日,检索于2008年9月21日。
\item Duhem, Pierre(1969年)[初版1908年],《拯救现象:从柏拉图到伽利略的物理理论观念研究》,芝加哥大学出版社,ISBN 0-226-16921-9,OCLC 12429405。
\item Eder, Michelle(2000年),《古希腊与伊斯兰中世纪对欧几里得平行公设的观点》,罗格斯大学,原文存档于2016年8月19日,检索于2008年1月23日。
\item Faruqi, Yasmeen M.(2006年),《伊斯兰学者对科学事业的贡献》,《国际教育期刊》,第7卷第4期:391–396页。
\item Gondhalekar, Prabhakar M.(2001年),《重力的控制:探索运动与引力定律》,剑桥大学出版社,ISBN 0-521-80316-0,OCLC 224074913。
\item Grant, Edward(1974年),《中世纪科学文献选集》,第一卷,剑桥,马萨诸塞州:哈佛大学出版社。
\item Grant, Edward(2008年),《海森》,《Encarta 在线百科全书》,微软公司,原文存档于2008年5月26日,检索于2008年9月16日。
\item Heeffer, Albrecht(2003年9月14–15日),《开普勒对正弦定律的近似发现:一个定性计算模型》,发表于第三届国际研讨会《科学推理与应用的计算模型》,阿根廷国家图书馆,布宜诺斯艾利斯,PDF原文存档于2022年10月9日,检索于2008年1月23日。
\item Hershenson, Maurice(1989年),《月亮错觉》,劳伦斯·厄尔鲍姆联合出版社,ISBN 0-8058-0121-9,OCLC 20091171,原文存档于2016年3月22日,检索于2008年9月22日。
\item Hess, David J.(1995年),《多元文化世界中的科学与技术:事实与人工物的文化政治》,哥伦比亚大学出版社,ISBN 0-231-10196-1。
\item Highfield, Roger(1997年4月1日),《唐破解古希腊人留下的最后谜题》,《每日电讯报》第676期,原文存档于2015年4月10日,检索于2008年9月24日。
\item Hodgson, Peter Edward(2006年),《神学与现代物理》,阿什盖特出版社,ISBN 978-0-7546-3622-9,OCLC 56876894。
\item Howard, Ian P.(1996年),《海什木被忽视的视觉现象发现》,《感知》,第25卷第10期:1203–1217,doi:10.1068/p251203。
\item Howard, Ian P.; Wade, Nicholas J.(1996年),《托勒密在双眼视觉几何中的贡献》,《感知》,第25卷第10期:1189–1201,doi:10.1068/p251189。
\item Kalin, Ibrahim;Ayduz, Salim;Dagli, Caner(编)(2009年),词条“伊本·海什木”,《牛津伊斯兰哲学、科学与技术百科全书》,牛津大学出版社。
\item Katz, Victor J.(1995年),《伊斯兰与印度的微积分思想》,《数学杂志》,第68卷第3期:163–174,doi:10.2307/2691411。
\item Katz, Victor J.(1998年),《数学史导论》,艾迪生-韦斯利出版社,ISBN 0-321-01618-1。
\item Kelley, David H.; Milone, E. F.; Aveni, A. F.(2005年),《探索古代星空:考古天文学百科全书式概览》,Birkhäuser,ISBN 0-387-95310-8,原文存档于2023年2月5日,检索于2014年4月7日。
\item Khaleefa, Omar(1999年),《谁是心理物理学与实验心理学的奠基人?》,《美国伊斯兰社会科学期刊》,第16卷第2期。
\item Al-Khalili, Jim(2015年2月12日),《回顾:<光学之书>》,《自然》,第518卷第7538期:164–165,doi:10.1038/518164a。
\item Langermann, Y. Tzvi(1990年),《伊本·海什木论世界结构》。
\item Lejeune, Albert(1958年),《托勒密关于双眼视觉的研究》,《雅努斯》,第47卷:79–86页。
\item Lindberg, David C.(1967年),《海什木的视觉理论及其在西方的接受》,《科学史杂志》,第58卷第3期:321–341,doi:10.1086/350266。
\item Lindberg, David C.(1976年),《从肯迪到开普勒的视觉理论》,芝加哥大学出版社,ISBN 0-226-48234-0。
\item Lindberg, David C.(1996年),《罗杰·培根与中世纪透视学的起源》,克拉伦登出版社。
\item Lorch, Richard(2008年),词条“伊本·海什木”,《大英百科全书》,原文存档于2008年6月12日,检索于2008年8月6日。
\item Magill, Frank Northen;Aves, Alison(1998年),词条“中世纪:海什木”,《世界人物传记辞典》,第二卷,劳特利奇出版社,ISBN 978-1-57958-041-4。
\item Mohamed, Mohaini(2000年),《伟大的穆斯林数学家》=,马来西亚科技大学出版社,ISBN 983-52-0157-9,原文存档于2017年8月30日,检索于同日。
\item Murphy, Dan(2003年10月17日),《伊拉克人启用新货币,“萨达姆”不再》=,《基督教科学箴言报》,原文存档于2021年4月17日,检索于2008年9月21日。




* NASA(2006年3月22日),《59239 Alhazen (1999 CR2)》,NASA喷气推进实验室,小天体数据库浏览器,原文存档于2011年8月7日,检索于2008年9月20日。

* O'Connor, J. J.; Robertson, E. F.(编)(1999年11月),《阿布·阿里·哈桑·伊本·海什木》(*Abu Ali al-Hasan ibn al-Haytham*),《MacTutor 数学史档案》,苏格兰圣安德鲁斯大学数学与统计学院,原文存档于2009年4月19日,检索于2008年9月20日。

* Omar, Saleh Beshara(1977年),《伊本·海什木的光学:实验科学起源研究》(*Ibn al-Haytham's Optics: A Study of the Origins of Experimental Science*),明尼阿波利斯:伊斯兰图书馆,ISBN 0-88297-015-1,OCLC 3328963。

* Plott, C.(2000年),《全球哲学史:经院哲学时期》(*Global History of Philosophy: The Period of Scholasticism*),莫提拉尔·巴纳西达斯出版社,ISBN 8120805518。

* Rashed, Roshdi(2002年8月a),《10世纪的一位博学家》(*A Polymath in the 10th Century*),《科学》(*Science*),第297卷第5582期:773,doi:10.1126/science.1074591,ISSN 0036-8075,PMID 12161634。

* Rashed, Roshdi(2002年b),《科学人物画像:10世纪的一位博学家》(*Portraits of Science: A Polymath in the 10th Century*),《科学》杂志,第297卷第5582期:773,doi:10.1126/science.1074591,ISSN 0036-8075,PMID 12161634。

* Rashed, Roshdi(2007年),《伊本·海什木的天体运动学》(*The Celestial Kinematics of Ibn al-Haytham*),《阿拉伯科学与哲学》(*Arabic Sciences and Philosophy*),第17卷,剑桥大学出版社:7–55,doi:10.1017/S0957423907000355,S2CID 170934544。

* Raynaud, Dominique(2003年),《伊本·海什木关于双眼视觉的研究:生理光学的先驱者》(*Ibn al-Haytham sur la vision binoculaire: un précurseur de l'optique physiologique*),《阿拉伯科学与哲学》,第13卷第1期,剑桥大学出版社:79–99,doi:10.1017/S0957423903003047,S2CID 231735113。

\end{itemize}