% 汇编语言笔记(GAS, x86-64)

\begin{issues}
\issueDraft
\end{issues}

参考\href{https://cs.lmu.edu/~ray/notes/gasexamples/}{一些 x86-64 的例子}, \href{https://www.tutorialspoint.com/assembly_programming/index.htm}{Tutorials Point}(提供一些 cpu 原理, 但用的是 NASM 不是 GAS)。

\subsection{基础}
一个例子:
\begin{lstlisting}[language=none]
        .global _start

        .text
_start:
        # write(1, message, 13)
        mov     $1, %rax            # system call 1 is write
        mov     $1, %rdi            # file handle 1 is stdout
        mov     $message, %rsi      # address of string to output
        mov     $13, %rdx           # number of bytes
        syscall                     # invoke operating system to do the write

        # exit(0)
        mov     $60, %rax           # system call 60 is exit
        xor     %rdi, %rdi          # we want return code 0
        syscall                     # invoke operating system to exit
message:
        .ascii  "Hello, world\n"
\end{lstlisting}

编译: \verb|gcc -c hello.s && ld hello.o && ./a.out|, 运行: \verb|./hello.x|
\begin{itemize}
\item assembler 是汇编语言的编译器
\item 汇编语言的语法依赖于 CPU 架构和具体的 assembler, 完全不 portable。
\item \verb|gcc| 使用的 assembler 是 \textbf{GNU Assembler (GAS)}
\item 注释用 \verb|#|
\item 缩进没有意义
\item 寄存器相当于临时变量。
\item \verb|mov $4, %eax| 把 register \verb|eax| 赋值为 4。 \verb|%| 表示 register, \verb|$| 表示 immediate value, 可以用十进制 \verb|4| 也可以 16 进制 \verb|0x4|。
\item \verb|mov (%rdi), %rax| 可以把 \verb|rdi| 作为指针 dereference, 然后赋值给 \verb|rax|。
\item \verb|xor %寄存器, %寄存器| 可以把 \verb|寄存器| 的值设为 0。 \verb|xor| 是逐 bit 异或, 结果存到第一个变量。 也可以用 \verb|mov $0, %寄存器|, 但是这样会读取内存, 减慢速度。
\item \verb|$| 和 \verb|%| 往往是可省略的
\item 在 32 位系统上, \verb|mov| 等效于 \verb|movl|, 64 位系统上等效于 \verb|movq|。
\item \verb|.section .text| 用于声明一个 section, text 说明这是执行程序。 \verb|.section| 可以省略。
\item \verb|syscall| 一般代替 \verb|int 0x80| 进行 systemcall, 后者已经过时了。
\item \verb|标签:| 是一个 label, 用于指定代码的某个位置, 用于跳转。 也可以定义一个函数。
\end{itemize}

\subsubsection{64 位寄存器}
\begin{itemize}
\item 提供 16 个\textbf{通用寄存器(general-purpose registers)}, 每个都是 16 bit。
\item \verb|rax, rbx, rcx, rdx|: arithmetic and data manipulation
\item \verb|rsi, rdi, rbp, rsp|: storing data and addressing memory
\item \verb|r8|-\verb|r15|: additional arithmetic and data manipulation
\item \verb|r| (register), \verb|x| (extended)
\item \verb|rsi|(source index register), \verb|rdi|(destination index register), \verb|rbp|(base pointer register), \verb|rsp|(stack pointer register)
\end{itemize}

\subsubsection{32 位寄存器}
\begin{itemize}
\item 开头把 \verb|r| 换成 \verb|e| (extended)表示 32 位
\item \verb|eax|: Accumulator
\item \verb|ebx|: Base
\item \verb|ecx|: Counter
\item \verb|edx|: Data
\item \verb|esi|: Source index
\item \verb|edi|: Destination index
\item \verb|ebp|: Base pointer
\item \verb|esp|: Stack pointer
\end{itemize}

\subsubsection{函数调用}
\begin{itemize}
\item 函数调用时, 寄存器用于储存自变量。
\item 整数或指针参数顺序: \verb|rdi, rsi, rdx, rcx, r8, r9|
\item 浮点参数顺序: \verb|xmm0, xmm1, xmm2, xmm3, xmm4, xmm5, xmm6, xmm7|
\end{itemize}

\subsubsection{system call 调用}
\begin{itemize}
\item 方法和函数类似。
\item 每个 system call 有一个编号, 把编号赋给 \verb|rax|, 再按普通函数的变量顺序给参数赋值。
\item 最后用 \verb|syscall| 命令。
\item 一些常用的 system call 编号
\item 1: \verb|sys_write| - write to a file descriptor
\item 2: \verb|sys_open| - open a file
\item 3: \verb|sys_close| - close a file descriptor
\item 60: \verb|sys_exit| - exit the process
\item 231: \verb|sys_exit_group| - exit all threads in a process
\end{itemize}

\subsubsection{算数命令}
\begin{itemize}
\item \verb|inc %xxx| 把寄存器的值加 1。 \verb|dec %xxx| 寄存器值减 1。
\item \verb|push %xxx| 是把寄存器的值 push 到 stack 顶部。 \verb|pop %xxx| 把 stack 顶部的值取回到寄存器。
\item \verb|jnz 标签| 是条件跳转。 当 \verb|ecx| 为 0 时就不跳。
\item \verb|add %xxx, %yyy| 把两个值相加, 存到第一个。
\item \verb|sub %xxx, %yyy| 同理,相减。
\item \verb|and %xxx, %yyy| 逐 bit 相 \&
\end{itemize}

\subsubsection{gdb 调试}
\begin{itemize}
\item 编译时加上选项 \verb|-g|, 就可以用 \verb|gdb ./hello.x| 来调试程序。
\item \verb|b main| 可以在 main 的第一行设置 break point, \verb|r| 跑程序。
\item \verb|p $xxx| 可以查看 register 的值。 注意不是 \verb|%|
\item 在用 gdb 调试 c/c++ 语言时, \verb|disassemble| 命令可以看到当前的汇编码。
\item 要直接把 c 编译成汇编码, 用 \verb|gcc -S main.c|, 但是生成汇编码的可读性比较差(比手写的复杂)。
\end{itemize}

\subsection{Fibonacci 数列}
\begin{lstlisting}[language=none, caption=fib.s]
# # A 64-bit Linux application that writes the first 90 Fibonacci numbers.
# gcc -no-pie -g fib.s -o fib.x && ./fib.x

.global main
.text

main:
        push    %rbx                    # we have to save this since we use it
        mov     $10, %ecx               # ecx will countdown to 0
        xor     %rax, %rax              # rax will hold the current number
        xor     %rbx, %rbx              # rbx will hold the next number
        inc     %rbx                    # rbx is originally 1

print:
        # We need to call printf, but we are using  printf
        # may destroy eax and ecx so we will save these before the call and
        # restore them afterwards.
        push    %rax                    # caller-save register
        push    %rcx                    # caller-save register

        mov     $format, %rdi           # set 1st parameter (format)
        mov     %rax, %rsi              # set 2nd parameter (current_number)
        xor     %rax, %rax              # because printf is varargs

        # Stack is already aligned because we pushed three 8 byte registers
        call    printf                  # printf(format, current_number)

        pop     %rcx                    # restore caller-save register
        pop     %rax                    # restore caller-save register

        mov     %rax, %rdx              # save the current number
        mov     %rbx, %rax              # next number is now current
        add     %rdx, %rbx              # get the new next number
        dec     %ecx                    # count down
        jnz     print                   # if not done counting, do some more

        pop     %rbx                    # restore rbx before returning
        ret

format:
        .asciz  "%20ld\n" # null terminated string
\end{lstlisting}

\begin{itemize}
\item gdb 中, \verb|x/2 $rsp| 可以查看 stack 顶部的值。 在程序一开始记录下该值可以知道 stack 第一个元素之前的位置。 \verb|2| 代表打印 2 个 hex, 每个 hex 4 字节(32bit)。 \verb|x| 是 examine。
\item call-save registers: 如果函数内部使用了这些寄存器, 必须要把它们还原回原先的值: \verb|rbx, rbp, r12-r15|
\item \verb|call printf| 调用标准库中的 \verb|printf| 函数, 函数的参数由以下 register 按照顺序携带
\end{itemize}
