% 圆锥曲线的统一定义(高中)
% keys 准线|第二定义|焦点|圆锥曲线|焦点-准线定义|射影几何|对偶原理
% license Xiao
% type Tutor

\begin{issues}
\issueDraft
\end{issues}

\pentry{圆锥曲线与圆锥\nref{nod_ConSec}}{nod_55cd}

在\enref{圆锥曲线与圆锥}{ConSec} 中曾提到,古希腊时期,人们通过截取圆锥面,得到了圆、椭圆、抛物线和双曲线这四类曲线,它们后来被统称为“圆锥曲线”。这种从几何构造出发的方法,直观地揭示了它们的共同起源。然而,尽管阿波罗尼乌将它们放入了同一个圆锥面,在后来一千多年的研究中却仍然被当作彼此独立的对象来看待。无论是图像形状还是代数表达式,它们看起来都完全不同,彼此之间似乎没有直接联系。这种割裂也延续到了现代教学中。

以椭圆为例,高中教材通常在介绍完长轴、短轴、焦点等要素之后,会突然引入一个叫“离心率”的量\footnote{本站的前置内容,为免突兀,并未立即引入。},并简单解释为“衡量椭圆扁平程度的参数”。它的定义是一个比值,看起来更像三角函数的形式,而不像之前那些能直接对应到具体图形的长度。更令人疑惑的是,不只是椭圆,抛物线和双曲线也各有自己的离心率。这不禁让人想问:为什么所有圆锥曲线都有离心率?这个量到底意味着什么?不同曲线之间的离心率又有什么关系?

这类困惑其实很正常。毕竟,在长轴、焦点等概念被提出的年代,离心率这个概念还没有出现,那时的研究更关注图形本身,而不是参数之间的抽象联系。也正因为如此,当它和其他几何要素一同出现在课本中时,常常会让人感到有些突兀。

随着解析几何的发展,数学家们发现,这些看似不同的圆锥曲线,在引入一个定点和一条定直线后,竟然可以通过一个简洁而优雅的定义统一起来。从这个统一的角度出发,离心率不再是某种“空降”的数值,而是整个几何结构的内在参数。更重要的是,这一定义不仅回答了之前的问题,更在射影几何等更深层的理论中展现出非凡的结构美感。遗憾的是,这部分内容在现行高中课程中已被完全删去。为了带给读者获得更全面的视角,本文将从统一定义出发,系统梳理圆锥曲线的几何构造及其背后隐藏的深层联系。

% 先以“焦点-准线”的统一定义为出发点,引出为什么引入射影几何的视角可以进一步理解这种统一性。

\subsection{关于统一定义的思考}

初中阶段学过,矩形和菱形都是平行四边形的特殊情况。而前文也提到了,圆、椭圆、抛物线和双曲线也有着相同的几何起源——它们都可以看作是平面截取圆锥面后所形成的轨迹。它们之间的区别,仅仅在于截面与圆锥的母线或对称轴之间的夹角满足不同的条件。既然这些曲线本质上出自同一个构造,便很自然地会想到,它们应当可以用一种统一的方式加以描述。现有的不同代数表达式,可能只是这种统一形式在参数取值不同情况下的具体表现。

既然目标是寻找一种统一的表达形式,参照初中对平行四边形的处理思路,首先要做的就是提取三种曲线定义中的共同要素。回顾定义可发现,唯一在椭圆、抛物线和双曲线中都明确出现的,是“焦点”。这表明,焦点应该会出现在统一定义之中。

与焦点直接相关的参数主要有两类:椭圆和双曲线中使用的是两个焦点之间的距离(焦距 $2c$),而抛物线则使用焦点到准线的距离(焦准距 $p$)。很显然,若要建立统一的定义,要么将椭圆和双曲线纳入抛物线的定义框架中,要么反过来,将抛物线纳入椭圆与双曲线的定义框架中。这就引出了一个关键选择:是为抛物线补上第二个焦点,还是为椭圆和双曲线补上一条与焦点对应的准线,然后忽略一个焦点?

\subsection{圆锥曲线的“焦点-准线”定义}

\textbf{开普勒(Kepler)}关于行星运动的研究为前面的问题提供了一个重要的启发。他起初与同时代的许多人一样,认为行星的轨道是圆形的。但在利用\textbf{第谷·布拉赫(Tycho Brahe)}留下的高精度观测数据反复检验后发现,只有当轨道是一个椭圆,且太阳位于其中一个焦点上时,才能与观测结果严格吻合——这就是著名的“\enref{开普勒第一定律}{Keple}”。

值得注意的是,轨道所在的椭圆中的另一个焦点只是几何上的构造点,并不对应任何实际存在的天体。因此,在这个物理模型中,一个焦点具有实际意义,而第二个焦点则并非必需。天文学家据此总结出,行星的轨道可以看作是一个点与某种约束条件下的轨迹,即使圆轨道也能纳入这种描述方式。而\textbf{牛顿(Newton)}给出了“约束条件”的具体形式——万有引力定律——不过这是后来的故事了。

不过,上面的路径也带来了一个启示:既然椭圆中的一个焦点只是辅助构造,那么看上去好像可以在椭圆和双曲线中忽略掉一个焦点,转而人为引入一条“准线”作为约束,基于“焦点–准线”的结构,实现圆锥曲线的统一定义。

\subsubsection{推导与验证}

既然希望将椭圆和双曲线纳入抛物线的定义框架,不妨先回顾抛物线的定义:设点 $P$ 到定点 $F$ 的距离为 $|PF|$,到定直线 $L$ 的距离为 $|PL|$,那么满足
\begin{equation}\label{eq_HsCsFD_4}
|PF| = |PL|~.
\end{equation}
的点的轨迹,就是抛物线。

参考之前从圆的定义出发推广其他圆锥曲线的思路,若要得到统一定义,只需保留原有的表达形式,并将这个唯一的约束条件 \autoref{eq_HsCsFD_4} 转写为一个包含参数的等价形式,使不同的参数值对应不同的曲线。一个自然的做法是,仿照阿波罗尼斯圆的方式,将其推广为:
\begin{equation}\label{eq_HsCsFD_1}
\frac{|PF|}{|PL|} = e,\quad(e>0)~.
\end{equation}

这样当参数 $e = 1$ 时,自然就得到抛物线。接下来,将探究在 $e \ne 1$ 的情形下,这个轨迹将对应怎样的曲线。

\begin{example}{设点 $F(m,0)$,直线 $L : x = m - p,(p>0)$。\footnote{由于定义中的两个距离不再相等,直觉上不宜像抛物线那样,将点与准线对称地放在原点两侧。因此,这里先引入一个待定参数 $m$,后续再根据条件进行求解。抛物线的标准定义中相当于令$\displaystyle m={p\over2}$。这是一种常见的处理方式。}若点 $P(x,y)$ 满足 \autoref{eq_HsCsFD_1},且$e\neq 1$,求点 $P$ 的轨迹方程。}\label{ex_HsCsFD_1}
解:

根据点到直线和两点之间的距离公式,有:
\begin{equation}\label{eq_HsCsFD_7}
\frac{\sqrt{(x - m)^2 + y^2}}{|x - m + p|} = e~.
\end{equation}

由于 $e > 0$,可直接两边平方并整理,得到:
\begin{equation}\label{eq_HsCsFD_2}
\begin{split}
&(x - m)^2 + y^2 = e^2(x - m + p)^2\\
\implies &(1 - e^2)x^2 + y^2 = 2[m - e^2(m - p)]x + [e^2(m - p)^2 - m^2]~.
\end{split}
\end{equation}

为使结果中右侧仅出现常数项\footnote{这是为了让结果恰巧关于$y$轴对称。如果不理解,可以通过配方,将\autoref{eq_HsCsFD_2} 变形为
\begin{equation}\label{eq_HsCsFD_9}
(1 - e^2)\left(x+\frac{m-e^2(m-p)}{1-e^2}\right)^2 + y^2 = \frac{e^2 p^2}{1 - e^2}~.
\end{equation}
,根据二次函数的知识,$\displaystyle\left(x+\frac{m-e^2(m-p)}{1-e^2}\right)^2$的对称轴为$\displaystyle x=-\frac{m-e^2(m-p)}{1-e^2}$。因此,通过调整$m$的值来调整对称轴。
},可令一次项系数为零,即:
\begin{equation}
m - e^2(m - p) = 0\implies e^2(m - p)^2 = \frac{m^2}{e^2}~.
\end{equation}

此时有:
\begin{equation}\label{eq_HsCsFD_5}
m = -\frac{pe^2}{1 - e^2}, \qquad m-p= -\frac{p}{1 - e^2}~.
\end{equation}

将上述结果代入 \autoref{eq_HsCsFD_2},可得:
\begin{equation}\label{eq_HsCsFD_3}
\begin{split}
&(1 - e^2)x^2 + y^2 = \frac{m^2}{e^2}(1 - e^2) = \frac{p^2 e^2}{1 - e^2}\\
\implies&\frac{x^2}{\displaystyle\frac{p^2 e^2}{(1 - e^2)^2}} + \frac{y^2}{\displaystyle\frac{p^2 e^2}{1 - e^2}} = 1~.
\end{split}
\end{equation}

讨论:
\begin{enumerate}
\item 当 $0 < e < 1$ 时,\autoref{eq_HsCsFD_3} 形式上符合椭圆的标准方程,可以得到如下参数对应关系:
\begin{equation}
a = \frac{p e}{1 - e^2}, \qquad b = \frac{p e}{\sqrt{1 - e^2}}~.
\end{equation}
显然 $a > b$,因此点 $P$ 的轨迹是一个长轴沿 $x$ 轴方向的椭圆。

又根据 \autoref{eq_HsCsFD_5} 可知点 $F$ 在 $x$ 轴负半轴,准线 $L$ 在 $F$ 左侧,也在 $x$ 轴负半轴。结合椭圆的参数关系有:
\begin{equation}\label{eq_HsCsFD_6}
c = \sqrt{a^2 - b^2} = \frac{p e^2}{1 - e^2}~,
\end{equation}
与 \autoref{eq_HsCsFD_5} 对比可知,$F$ 正是椭圆的一个焦点。另外,由于 $a = e |m - p| < |m - p|$,说明准线 $L$ 在椭圆之外\footnote{这也可以通过在\autoref{eq_HsCsFD_9} 中讨论得到,由于$y^2\geq0$,因此}。

\item 当 $e > 1$ 时,\autoref{eq_HsCsFD_3} 可整理为:
\begin{equation}
\frac{x^2}{\displaystyle\frac{p^2 e^2}{(e^2 - 1)^2}} - \frac{y^2}{\displaystyle\frac{p^2 e^2}{e^2 - 1}} = 1~,
\end{equation}
这符合双曲线的标准方程。此时的参数对应关系为:
\begin{equation}
a = \frac{p e}{e^2 - 1}, \qquad b = \frac{p e}{\sqrt{e^2 - 1}}~.
\end{equation}
因此,点 $P$ 的轨迹是一个实轴沿 $x$ 轴方向的双曲线。由 \autoref{eq_HsCsFD_5} 可知,点 $F$ 位于 $x$ 轴正半轴,准线 $L$ 位于 $F$ 左侧,但仍在 $x$ 轴正半轴内。根据双曲线的参数关系,有:
\begin{equation}
c = \sqrt{a^2 + b^2} = \frac{p e^2}{e^2 - 1}~.
\end{equation}
与 \autoref{eq_HsCsFD_5} 对比可知,$F$ 恰好是双曲线的一个焦点。

此外,由于 $a = e |m - p| > |m - p|$,可以判断准线 $L$ 位于双曲线之外。
\end{enumerate}
\end{example}

尽管在\autoref{ex_HsCsFD_1} 的推导时,要求$e\neq1$。但对于 \autoref{eq_HsCsFD_2},当 $e = 1$ 时,有:
\begin{equation}
y^2 = 2px + (2m - p)p~.
\end{equation}
若令 $\displaystyle m = \frac{p}{2}$,即可得到抛物线的标准方程,正好与最初的要求相符。

由此可以看出,在“焦点–准线”定义的推导过程中,完全未涉及“另一个焦点”。这一点恰好呼应了前文关于行星轨道的讨论:只需给定一个点和约束条件——焦点到准线的距离 $p$ 和距离比值 $e$,就能唯一确定一条圆锥曲线。其中,当 $e < 1$ 时对应椭圆,$e > 1$ 对应双曲线,而 $e = 1$ 则是抛物线。

\begin{figure}[ht]
\centering
\includegraphics[width=14.25cm]{./figures/d6efb4f2895bcd01.png}
\caption{$p = 1$时,不同 $e$ 对应的结果} \label{fig_HsCsFD_1}
\end{figure}

\addTODO{这里要描述一下不同e对应的焦点和准线的位置。}

在之前的学习中,经常使用三个参数 $a, b, c$ 来描述椭圆或双曲线的性质。但无论是代数表达式中的 $a, b$,还是几何结构中的 $a, c$,本质上都只包含两个自由变量。事实上,这三个参数始终满足一个固定的约束关系,所以椭圆和双曲线只有两个自由度。这也从另一个角度说明,使用 $p$ 与 $e$ 作为定义参数是合理且充分的。

至此,终于验证了前面的猜想,得到了\textbf{圆锥曲线的焦点–准线定义(Focus–Directrix Definition of Conic Sections)}。

\begin{definition}{圆锥曲线的焦点–准线定义}\label{def_HsCsFD_1}
在平面上,给定一个定点 $F$ 和一条定直线 $L$,若 $F$ 不在 $L$ 上,则所有满足点到 $F$ 的距离与到 $L$ 的距离之比为某个固定常数 $e$ 的点 $P$ 的轨迹,称为\textbf{圆锥曲线(conic section)},即点 $P$ 满足:
\begin{equation}
\frac{|PF|}{|PL|} = e,\quad(e>0)~.
\end{equation}
其中,$F$称为圆锥曲线的\textbf{焦点(focus)},$L$称为圆锥曲线的\textbf{准线(directrix)},二者相互对应。焦点到准线的距离称为\textbf{焦准距(focal parameter)},通常记作 $p$;距离比值称为\textbf{离心率(eccentricity)},通常记作 $e$。

特别的:
\begin{itemize}
\item 当 $e = 0$ 时,轨迹为\textbf{圆(circle)};
\item 当 $0 < e < 1$ 时,轨迹为\textbf{椭圆(ellipse)};
\item 当 $e = 1$ 时,轨迹为\textbf{抛物线(parabola)};
\item 当 $e > 1$ 时,轨迹为\textbf{双曲线(hyperbola)}。
\end{itemize}
\end{definition}

其他三种情况在前文中已有说明,唯一尚未解释的是 $e = 0$ 所对应的情形。在圆的定义中,只涉及一个圆心和一个半径。将统一定义中的焦点看作圆心并不难理解,毕竟在前面的推导中,其他圆锥曲线的焦点也是从圆心发展而来。而现在又给出了定义中的 $e = 0$ ,那么问题来了:准线去哪里了?而圆的半径,又该如何与焦准距建立联系?

解答这些问题,需要回到 \autoref{eq_HsCsFD_7}。当 $e = 0$ 时,有两种可能的情形:
\begin{itemize}
\item 第一种情况是分子为零,此时无论分母为何,等式恒成立,轨迹退化为定点 $F$ 自身。这与 \autoref{fig_HsCsFD_1} 中的红色轨迹所示的情况近似。
\item 第二种情况是当 $p \to +\infty$,即分母趋于无穷大,若分子为有限常数 $r$,则根据极限计算,有 $\displaystyle\frac{r}{\infty} \to 0$,同样满足 $e = 0$。此时分子就对应圆的标准方程 $(x - m)^2 + y^2 = r^2$,定点 $F$ 为圆心,$p \to \infty$ 表示准线远离至无穷远处。由于\autoref{fig_HsCsFD_1} 中 $p$ 为有限值,不能直接显示这种情况,图中的绿色曲线近似反映了这种情况。
\end{itemize}

同理,当分母趋近于 $0$ 时,有 $e \to \infty$,此时轨迹趋近于准线本身,几乎退化为一条直线,而焦点看上去仿佛与 $(p, 0)$ 重合,这正对应 \autoref{fig_HsCsFD_1} 中的青色情况。至于“准线趋于无穷远”或“焦点趋于无穷远”在几何上究竟意味着什么,则需要借助 \aref{射影几何}{sub_HsCsFD_1} 的视角才能更深入理解。不过,根据 \autoref{def_HsCsFD_1},现在有充分的理由这样说——所有圆锥曲线本质上是一类几何对象。

\addTODO{焦点和准线固定的图像}

当然,\autoref{ex_HsCsFD_1} 的推导是基于给出定义之后的代数推导,直接根据定义也可以进行几何推导。

设平面上有两个定点 $F_1, F_2$ 和两条定直线 $L_1, L_2$,其中每个定点与一条定直线一一对应。若 $L_1$ 与 $L_2$ 相互平行,且关于$F_1F_2$ 的垂直平分线对称。若点 $P$ 满足:
\begin{equation}\label{eq_HsCsFD_8}
\frac{|PF_1|}{|PL_1|} = \frac{|PF_2|}{|PL_2|} = e~,
\end{equation}
则点 $P$ 的轨迹是一条圆锥曲线。特别地,当$|L_1L_2|>|F_1F_2|$时,轨迹为椭圆;当$|L_1L_2|<|F_1F_2|$时,轨迹为双曲线。\footnote{当两者相等时,轨迹退化为一对关于两点连线对称的点,读者可自行探究。}

推导过程也不困难,根据\autoref{eq_HsCsFD_8} ,当点$P$在$L_1,L_2$之间时,有:
\begin{equation}
|PF_1|+|PF_2|=e(|PL_1|+|PL_2|)=e|L_1L_2|~.
\end{equation}
由于$e,|L_1L_2|$均为定值,此时$P$点轨迹为椭圆。此时,根据$\displaystyle\frac{|PF_1|}{|PL_1|} = \frac{|PF_2|}{|PL_2|} $的约束关系,必有$|L_1L_2|>|F_1F_2|$。

当点$P$在$L_1,L_2$之外时,有:
\begin{equation}
||PF_1|-|PF_2||=e||PL_1|-|PL_2||=e|L_1L_2|~.
\end{equation}
由于$e,|L_1L_2|$均为定值,此时$P$点轨迹为双曲线。

由直角坐标方程可知对称性,可在椭圆的两边做两条准线,令椭圆上任意一点到两焦点的距离分别为 $r_1$ 和 $r_2$,到两准线的距离分别为 $d_1$ 和 $d_2$,则有
\begin{equation}
e = \frac{r_1}{d_1} = \frac{r_2}{d_2} = \frac{r_1 + r_2}{d_1 + d_2}~,
\end{equation}
所以
\begin{equation}
r_1 + r_2 = e(d_1+d_2) = 2e(c + h) = 2\frac{c}{a} \qty( c + \frac{b^2}{c} ) = 2a~,
\end{equation}
证毕。

双曲线的另一种定义是, 曲线上任意一点到两个焦点距离之差等于 $2a$。 这里证明前两种定义满足该性质。 由对称性, 不妨只考虑右支上的某点, 令其到右焦点和右准线的距离分别为 $r_1$ 和 $d_1$, 到左焦点和左准线的距离分别为 $r_2$ 和 $d_2$。 由离心率的定义, 有
\begin{equation}
e = \frac{r_1}{d_1} = \frac{r_2}{d_2} = \frac{r_2 - r_1}{d_2 - d_1}~,
\end{equation}
由于两准线之间的距离恒为 $2a^2/c$, 上式变为
\begin{equation}
r_2 - r_1 = e(d_2 - d_1) = 2a~,
\end{equation}
证毕。


\subsection{圆锥曲线的性质}

在研究椭圆等圆锥曲线时,通常在建立定义之后,会进一步分析它们各自的几何性质。而在前面已经建立了圆锥曲线的焦点–准线统一定义后,现在可以从一个统一的视角出发,探究它们所共有的一些几何特征。

首先,通过 \autoref{ex_HsCsFD_1},除了得到统一定义本身,还可以进一步推导出以下结论:
\begin{enumerate}
\item 对于任意一条圆锥曲线,定点到定直线的垂线方向就是该曲线的对称轴。而对于椭圆和双曲线,由于它们具有两条对称轴,还可以通过另一条对称轴构造出第二组对应的焦点与准线。至于圆,由于圆具有无数条对称轴,可以视作所有焦点都与圆心重合,且存在无数条对应的准线。不过,正如前文所提,这些准线都位于无穷远处,而事实上,这些准线也是彼此重合的。这一点一样会在后文介绍。
\item 从 \autoref{ex_HsCsFD_1} 出发,还可以得到基于椭圆和双曲线的几何参数计算焦准距与离心率的表达式:
\begin{equation}
\begin{split}
p &= \left| \frac{a^2}{c} - c \right| = \frac{b^2}{c}\
e &= \frac{c}{a}~.
\end{split}
\end{equation}
特别地,标准椭圆与双曲线的准线方程可根据焦点位于 $x$ 轴或 $y$ 轴的位置分别表示为:
\begin{equation}
x = \pm\frac{a^2}{c} \qquad \text{或} \qquad y = \pm\frac{a^2}{c}~.
\end{equation}
\end{enumerate}

\subsubsection{$p$和$e$对圆锥曲线的影响}

离心率表示“扁平程度”:
$$ e = \frac{c}{a} = \sqrt{1 - \frac{b^2}{a^2}} \in [0, 1) ~.$$
椭圆越接近 1 越扁。
离心率的几何意义再探
	•	离心率 $e$ 是控制形状的核心参数。
	•	举例说明:当 $e$ 增大,曲线形状如何改变?尤其是焦点与准线在图中如何移动。
	•	动态几何软件(几何画板、Desmos等)可用来展示 $e$ 从 $0.1 \to 5$ 的过程,鼓励学生观察准线、焦点、轨迹三者的动态关系。

点与准线位置等信息;

离心率越小,曲线越“闭合”,越趋近于圆;越大,则越“张开”,焦点越偏离中心。


参数统一与变换视角

1. 坐标变换下的不变量
	•	探讨:在平移、旋转等坐标变换下,$e$ 和 $p$ 的行为如何?哪一个是几何不变量?
	•	引出:离心率是射影不变量,这也是它在统一定义中如此重要的原因。

2. 以 $e$ 为参数的族(Pencil)与变形动画
	•	同一个焦点、同一条准线,$e$ 不同的圆锥曲线形成一个“变形序列”。
	•	可展示一个动态图,让学生看到椭圆如何“张开”成抛物线、再“展开”成双曲线。



第四部分:几何意义与物理直觉

目的:构建直观理解
	•	利用“光的反射”解释焦点定义的实际意义:
	•	抛物线:来自远处平行光线汇聚于焦点。
	•	椭圆:一个焦点发出的光线经过反射经过另一个焦点。
	•	双曲线:延长反射路径的特性。



\subsubsection{圆锥曲线族}

	•	如果把“离心率相同”的所有组成一个“”,这个族是怎样的?
	•	如果我们改变焦点的位置但保持离心率不变,会发生什么?

\subsection{*射影几何视角下的圆锥曲线}\label{sub_HsCsFD_1}

这一章在高中阶段是完全超纲的内容,大部分的老师甚至都不会提及。但在这个视角下能够观察到非常美妙的统一性质,对于理解圆锥曲线的整体为何会跟随之前给出的准线交点定义有非常重要的作用。因此,此处承接定义进行一些简单的介绍供读者感受数学之美。

\subsubsection{平行线可以相交吗?}

在熟悉的欧式几何中,有一些“默认”的限制,比如:
	•	平行线不会相交;
	•	直线是无限延伸的;
	•	点只能表示有限的位置。

这些看起来都很自然。可是,在现实中,类似的情形却经常出现:平行的铁路轨道,在远方看起来会相交;沿着很高的楼的一面看,墙的两侧最终好像汇聚到了一个点,在绘画中称为“消失点”

\begin{figure}[ht]
\centering
\includegraphics[width=11cm]{./figures/aa409536797b866f.png}
\caption{平行轨道的消失点} \label{fig_HsCsFD_2}
\end{figure}

\begin{figure}[ht]
\centering
\includegraphics[width=11cm]{./figures/040cd6e71df42276.png}
\caption{仰望高楼时墙体延伸至消失点} \label{fig_HsCsFD_3}
\end{figure}
射影几何的创立,就是为了用数学方法来描述前面提到的这些现象。

观察前面的现象,几个与之前欧氏几何中区别的点是:
平行线也有交点
无穷远处的点在平面上也是不可忽略,或者说存在的。

在射影几何中,我们引入了“无穷远点”和“无穷远线”(例如所有平行线在“无穷远点”相交),这是因为我们不关心距离和角度,只关心交点关系(即拓扑结构/位置结构)。

所有直线在射影几何中都相交——平行线也会在“无穷远点”相交。

于是扩展了平面,引入了一个“无穷远直线”,把所有方向的“无穷远点”放在这条线上。

射影几何给展示了:
	•	同一个对象可以从不同的角度理解;
	•	表面看起来不同的东西,背后可能有统一的结构;
	•	有时,必须打破一些“习惯的规则”,才能看到更完整的图景。

有趣的是,在这样定义之后,在射影几何中,点和直线可以互换、对称对待;也就是说,直线也可以看成是“由点组成”的,点也可以像直线一样进行变换。


\subsubsection{对偶原理}

在十九世纪,法国数学家腾塞叶(Jean-Victor Poncelet)在俄国战俘营中写下了他的重要著作《论图形的摄影性质》。在这部作品中,他首次系统提出了“对偶原理”与“投影不变性”这两个深刻的几何思想。所谓对偶原理,是指在射影几何中,点与直线可以互换,互换后许多几何命题依然成立。例如,“两点决定一条直线”的命题,对偶后变成了“两条直线决定一个交点”,这两者在射影几何中都同样成立。这种点与线之间的对称关系,揭示了几何结构中隐藏的深层对称性,使人们重新思考“几何事实”背后的逻辑构造。

与此同时,腾塞叶还指出,一些几何性质在投影变换下是保持不变的。换句话说,即使我们改变观察角度或从不同平面进行投影,某些关系仍旧成立,这被称为“投影下的不变性”。比如,共线的点经过中心投影后仍然共线;一个圆在透视下可能变成椭圆、抛物线或双曲线,但这些曲线本质上都是圆锥曲线,因此在射影几何中是等价的。这一思想打破了古典欧几里得几何中对“形状”的执着,把几何研究的焦点从“看上去的样子”转向了“结构中的本质”。

到了二十世纪,随着公理化几何的发展,数学家们进一步发现:在许多几何定理中,把“点共线”换成“线共点”、把“点”换成“直线”后,新的表述仍然成立。这些互换后的命题不仅不是偶然巧合,而是源自整个射影几何体系中点与线的对等地位。对偶原理的提出不仅丰富了几何的思维方式,也为代数几何、拓扑学、以及更现代的数学分支奠定了基础。在这种视角下,几何的研究不再只是对现实图形的模仿,而是一种对空间逻辑结构的深刻把握。


\subsubsection{圆锥曲线}


尤其是当在统一定义中使用了“准线”这个概念时,会发现一个问题:

准线是直线,而焦点是点,它们的地位并不对等。

比如,在抛物线中,焦点和准线之间的距离决定了曲线的开口程度;但在椭圆和双曲线中,焦点之间的关系常常比准线更显眼。这种“不对等”让难以一眼看出统一性。

要解决这个问题,需要让“点”和“直线”变得对等、互换,这正是射影几何擅长的。




射影几何中的视角使能够用一种统一且优雅的方式看待圆锥曲线。但在射影几何中,这些差异被看作是坐标选择与观察角度所导致的表象变化,它们在更本质的层面上是一类对象的不同表现:它们都是圆锥曲线。圆锥曲线不是三类不同的曲线,而是一个统一的几何实体的三种视角。它让跳出了直观图形的束缚,从结构上理解几何对象之间的联系,也为代数几何、复几何乃至更高维的几何打下了坚实的基础。

从射影几何的角度看,圆锥曲线定义为一个圆锥面与一个平面相交的轨迹。这个定义在欧几里得空间中也成立,但射影几何更进一步地指出:在射影平面中,所有非退化的圆锥曲线都是射影等价的。这意味着可以通过一个合适的射影变换(即坐标的线性变换加上归一化),将任意一个圆锥曲线变为另一个圆锥曲线——比如将一个椭圆变为一个双曲线或抛物线。

换句话说:
\begin{itemize}
\item 椭圆是在射影平面中与无穷远直线没有实交点的圆锥曲线;
\item 双曲线是在射影平面中与无穷远直线有两个实交点的圆锥曲线;
\item 抛物线是恰好与无穷远直线有一个交点的极限情形。
\end{itemize}

这种分类在射影几何中失去了意义,因为无穷远直线被作为与其他直线同等地位来处理,不再是“例外的部分”。因此,抛物线、椭圆和双曲线不再是本质不同的几何对象,而只是一个对象的不同投影或表示。


为什么 $e=1$ 是一个“分界线”?

离心率为什么只分成这三类,而不是连续变化出更多种曲线?

四、重新看焦点和准线:变换下的对等性

回到的统一定义:

到焦点距离与到准线距离的比值等于 $e$

焦点是一个点,准线是一个直线,它们是不一样的。但在射影几何中,可以把直线看成是“一个方向上的点的集合”,特别是在加入了“无穷远点”之后,直线也可以被看作是特殊的“点”。

这就让焦点和准线,在某种意义上变得“对等”。

更重要的是:

在射影几何中,通过变换,可以把一个圆锥曲线变换成另一种类型的圆锥曲线,只要它们满足相同的基本结构。

举个例子:
	•	一个椭圆,通过一个适当的“射影变换”,可以变成一个抛物线;
	•	抛物线也可以变成双曲线;
	•	这些变换不会改变圆锥曲线的“本质”,只改变它在眼中的“样子”。

这就说明,射影几何的世界中,圆锥曲线是一个统一的整体,而不是三种各自孤立的图形。


在射影几何中,圆锥曲线的统一定义并不依赖焦点–准线,而是:

所有与一条圆锥面相交的平面交线,在射影平面中都是圆锥曲线。其本质是一个二次齐次方程在 $\mathbb{P}^2$ 中的零点集合。

但——

焦点–准线结构仍然可以嵌入射影几何中,你可以这样理解:
	•	准线可以是一个射影直线;
	•	焦点是一个射影点;
	•	离心率可以通过某种射影不变量(例如交比)来表达。

在射影几何中,“一个点到一条线的比值”不再有意义,但你可以通过共轭二次曲线、交比等射影结构,重新定义出类似“焦点–准线”的行为。

1. “双焦点”与“单焦点”的统一
	•	问题:我们习惯使用两个焦点描述椭圆/双曲线,而这里却只用一个焦点和一条准线,为什么可以?
	•	解读:“另一个焦点”其实可以看作是准线的对偶,它不是必须的,只是在对称性下自然出现。

2. 对偶结构初探
	•	可引入投影几何中“点–直线对偶”的思想,让学生意识到准线和焦点在某种意义下可以互换角色。

此外,射影几何还强调了极点与极线的对偶性,并引入了极线极点变换的工具来研究圆锥曲线的性质,使得很多命题具有了对称且优美的形式。例如:对于一个给定的圆锥曲线,任意一点都有与之对应的一条极线,反之亦然。这种对偶关系在欧氏几何中并不自然存在。对于任意一条圆锥曲线,定点到定直线的垂线方向就是该曲线的对称轴。而对于椭圆和双曲线,由于它们具有两条对称轴,还可以通过另一条对称轴构造出第二组对应的焦点与准线。当然,对于圆,所有的焦点都与圆心重合,由于有无数条对称轴,那些对应的准线都在无穷远处,这一点会在后面说明。