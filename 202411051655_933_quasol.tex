% 二次方程求根公式
% license Xiao
% type Tutor


通常二次方程
\begin{equation}
ax^2+bx+c=0 \qquad (a\neq 0)~.
\end{equation}
可以变形得到如下形式:
\begin{equation}
\begin{split}
 & ax^2+bx+c = 0 \\ 
\iff&x^2+2{b\over 2a} x+\left({b\over 2a}\right)^2 = \left({b\over 2a}\right)^2-{c\over a} \\ 
\iff&\left(x+{b\over 2a}\right)^2 = {b^2-4ac\over 4a^2} \\ 
\end{split}~.
\end{equation}

由于左侧及右侧分母是一个平方形式,因此$x$的值只与右侧分子的符号相关。定义判别式
\begin{equation}
\Delta = b^2-4ac~.
\end{equation}

则:
\begin{itemize}
\item $\Delta > 0$时,方程有实数解,为两个不同的实根。
\item $\Delta = 0$时,方程有实数解,为两个相同的实根。
\item $\Delta < 0$时,方程无实数解。
\end{itemize}

且方程有解时,两个根分别为:
\begin{equation}
x_1=\frac{-b+\sqrt{\Delta}}{2a}\qquad x_2=\frac{-b-\sqrt{\Delta}}{2a}~.
\end{equation}
这被称为\textbf{二次方程的求根公式(Quadratic Formula)}。



\subsubsection{几何含义}
假设$f(x)=ax^2+bx+c$,那么求解方程$ax^2+bx+c=0$即化为寻找函数$f(x)$的所有零点$f(x)=0$。
\begin{figure}[ht]
\centering
\includegraphics[width=14cm]{./figures/652b2f4efe76656f.pdf}
\caption{$f(x)$示意图。从左到右为$\Delta > 0, \Delta = 0, \Delta < 0$} \label{fig_quasol_1}
\end{figure}
可见,$x=-\frac{b}{2a}$ 为函数$f(x)$的对称轴,两个零点(如果存在)关于该轴对称。

\subsection{配方法}
对于一些特定的问题,可以将方程配方并求解,有时这比直接使用求根公式更为简便。

例如,可以将方程配方为如下形式:
$$(x-a)(x-b)=0\Rightarrow x_1=a, x_2=b~.$$
\begin{figure}[ht]
\centering
\includegraphics[width=5cm]{./figures/fe527c666ebfd775.pdf}
\caption{$f(x)=(x-a)(x-b)$示意图} \label{fig_quasol_2}
\end{figure}
或者
$$(x-a)^2=b\Rightarrow x_1=\sqrt{b}+a, x_2=-\sqrt{b}+a~.$$

\subsection{韦达定理}

\textbf{韦达定理(Vieta's formulas)}是一组描述代数方程的根和方程系数的关系的公式。

\begin{theorem}{韦达定理(二次情况)}
考虑上一元二次方程 $a x^2 + b x + c = 0\quad(a \neq 0)$,如果它有有两个不相等根 $x_1$ 和 $x_2$,则有:
$$\begin{aligned}
x_1 + x_2 &= -\frac{b}{a} ~,\\
x_1 x_2 &= \frac{c}{a}~.
\end{aligned}$$
如果是只有一个根(重根)$x_0$,那么
$$
x_0 = - \frac{b}{2 a}~.
$$
\end{theorem}

证明:
$$\begin{aligned}
a x^2 + b x + c &= a (x - x_1) (x - x_2) ~,\\
a x^2 + b x + c &= a (x^2 - (x_1 + x_2) x + x_1 x_2)~, \\
x^2 + \frac{b}{a} x + \frac{c}{a} &=  x^2 - (x_1 + x_2) x + x_1 x_2~.
\end{aligned}$$
证明完毕。

同理可证,对于任意$n$次代数方程的情况。
\begin{theorem}{韦达定理}
对于$n$次代数方程$a_0x^n+a_1x^{n-1}+\cdots+a_n=0\quad(a_0\neq0)$,若存在$n$个根$x_1,x_2,\cdots,x_n$,则有:
\begin{equation}
\begin{split}
x_1+x_2+\cdots +x_n&=-{a_1\over a_0}\\
x_1x_2\cdots x_n&=(-1)^n{a_n\over a_0}
\end{split}~.
\end{equation}
\end{theorem}

