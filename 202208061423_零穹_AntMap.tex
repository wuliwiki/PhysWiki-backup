% 斜对称映射
% 斜对称映射

\pentry{映射\upref{map}}

\begin{definition}{斜对称映射}\label{AntMap_def1}
设 $X,Y$ 是任一集合,其中 $Y$ 上的任一元素 $y$ 都有一元 $-y=-1\cdot y$ 与之对应,且 $y$ 上的元素与 $-1$ 的作用满足 $\underbrace{(-1)\cdot((-1)\cdots((-1)}_{n\text{个}}\cdot y)\cdots)=(-1)^n y$.

 $n$ 元映射 $f:X^n\rightarrow Y$ 叫作\textbf{斜对称的},若对任意 $k=1,\cdots,n-1$,成立
\begin{equation}
f(\cdots,x_k,x_{k+1},\cdots)=-f(\cdots,x_{k+1},x_k,\cdots)
\end{equation}
即当交换任意两个相邻变量的位置时,映射值变号.
\end{definition}
\begin{theorem}{}
交换任意两个变量的位置,斜对称映射都变号.
\end{theorem}
\textbf{证明:}设交换第 $i,j$ 个变量的位置,且 $i<j$,那么位于 $i,j$ 之间的自变量的个数 $l=j-i-1$.利用数学归纳法证明如下: $l=0$ 时正好满足斜对称的定义,故成立.设 $l\leq k$ 时定理成立,那么 $l=k$ 时有
\begin{equation}
\begin{aligned}
f(\cdots,x_i,x_{i+1},\cdots,x_{j-1},x_j,\cdots)&=-f(\cdots,x_{i+1},x_{i},\cdots,x_{j-1},x_j,\cdots)\\
&=(-1)^2 f(\cdots,x_{i+1},x_{j},\cdots,x_{j-1},x_i,\cdots)\\
&=(-1)^3 f(\cdots,x_{j},x_{i+1},\cdots,x_{j-1},x_i,\cdots)\\
&=-f(\cdots,x_{j},x_{i+1},\cdots,x_{j-1},x_i,\cdots)
\end{aligned}
\end{equation}
  
\textbf{证毕!}
\begin{example}{外积}
在斜对称映射的帮助下,可以直接讨论外积(wedge)的一些性质,而不关心具体的集合.外积映射
\begin{equation}
\wedge:V^k\rightarrow \Lambda^k(V^*)
\end{equation}
是个斜对称映射,其中 $k$ 是任一自然数,$\Lambda^k(V^*)$ 是具有\autoref{AntMap_def1} 中 $Y$ 的一切性质.在 $k=2$ 时,斜对称性意味着 $\wedge(x,y)=-\wedge(y,x)$. 通常记 $\wedge(x,y)=x\wedge y$,于是这相当于
\begin{equation}
x\wedge y=-y\wedge x
\end{equation}
在一般情形,也记 
\begin{equation}
\wedge(x_1,x_2,\cdots,x_k)=x_1\wedge x_2\wedge\cdots\wedge x_k
\end{equation}
\end{example}
此时斜对称映射 $\wedge$ 意味着:
\begin{equation}
x_1\wedge\cdots\wedge (x_i\wedge x_{i+1})\wedge\cdots\wedge x_k=x_1\wedge\cdots\wedge (x_{i+1}\wedge x_{i})\wedge\cdots\wedge x_k
\end{equation}
