% 量子信息守恒
% keys 量子不可克隆定理|量子不可删除定理
% license Xiao
% type Tutor

在这一章中,我们将会讨论量子信息中的两个基本原理:不可克隆定理和不可删除定理。它们有时也被冠名为量子信息守恒原理,因为它们告诉了我们量子信息相比于经典信息的不同之处:它不可被复制,也不可被删除。这样的特性在很大程度上成为了量子通信、量子密码学中的无条件安全性的来源。

\subsection{量子不可克隆定理}

在日常生活中,我们经常需要对信息进行拷贝、复制或者说克隆。这一点可以通过手动抄写、复印机、文件拷贝等办法进行处理。不管怎么样,在信息的复制过程中,我们需要将想要复制的信息(这里用$\ket{\psi}\rangle$来标记)和一份空白的信息(这里用$\ket{0}\rangle$来标记)放进一个克隆机器中,这台机器便会输出两份信息$\ket{\psi}\rangle\ket{\psi}\rangle$。注意我们要求$\ket{\psi}\rangle$是任意的。不然这也称不上是一个克隆机器,只是一个把$\ket{\psi}\rangle$写入到$\ket{0}\rangle$的机器罢了。我们同样要求克隆机不会破坏初始输入的信息。

对于经典信息来说,$\ket{\psi}\rangle=0,1$,在这种情况下显然是存在克隆机的。只需要让它执行这样的程序:
\begin{itemize}
\item 如果$\ket{\psi}\rangle=0$,那么在$\ket{0}\rangle$写入0。
\item 如果$\ket{\psi}\rangle=1$,那么在$\ket{1}\rangle$写入1。
\end{itemize}
这也是拷贝文件时的处理逻辑。

下面的问题是:对于量子信息(也就是$)是否也存在这种克隆机呢?具体来说,是否有这样一个幺正演化$U$,使得对于任意的输入