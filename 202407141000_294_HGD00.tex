% 哈尔滨工业大学 2000 年硕士物理考试试题
% keys 哈尔滨工业大学|考研|物理|2000年
% license Copy
% type Tutor

\textbf{声明}:“该内容来源于网络公开资料,不保证真实性,如有侵权请联系管理员”

\begin{enumerate}
\item 设光纤芯线与外套的折射率$n_g>n_c$,垂直端面外介质的折射率为$n_a$(如图1),试求能使光在光纤内发生全反射的入射光束的最大孔径角$i_0$。若将光纤制成圆锥腔,试问有否“聚光”作用,并说明理由。
\begin{figure}[ht]
\centering
\includegraphics[width=8cm]{./figures/5885197456b5d97b.png}
\caption{} \label{fig_HGD00_1}
\end{figure}
\item 如图二所示望远镜,物镜$L_1$的焦距$f_1$,镜框内径$D_1$;目镜$L_2$的焦距$f_2$,镜框内径$D_2$。在重合焦平面上的光阑$A$的孔径为$d$。试求该望远镜的孔径光阑、入射瞳、出射瞳和视场光阑的位置和大小。又若当你用此望远镜观测时上述量有何不同。
\begin{figure}[ht]
\centering
\includegraphics[width=8cm]{./figures/80fcbd27be813432.png}
\caption{} \label{fig_HGD00_2}
\end{figure}
\item 将焦距为的柱透镜沿轴线对半剖开,分成两片半透镜$L$和$LB$,按图三位置安放。P点为波长入的单色线光源。在两束光交叠区域放置观察屏Q1、写出明暗条纹的条件;
\end{enumerate}