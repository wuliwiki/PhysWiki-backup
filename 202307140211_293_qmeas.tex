% 量子测量
% 测量|投影测量|POVM测量

\pentry{量子力学基本原理\upref{QMPrcp}}

在量子力学的学习中,我们了解了测量公设。粗略地来讲,在量子态上进行对某个可观测量的测量,就会让这个量子态坍缩到这个可观测量的某个本征态上,同时返回这个本征态对应的本征值作为测量结果。得到这个本征态的概率由Born定则给出。在本节中,我们将会从量子信息的角度来研究量子测量,首先我们会回顾投影测量的基本结构,然后我们将会介绍von Neumann的测量理论,最后我们将会给出投影测量的推广——POVM测量。

\subsection{投影测量}

在数学上我们知道,可观测量$A$对应的是一个厄米算符,谱分解定理表明$A$的本征向量构成一组Hilbert空间的完备正交基,也即(先假设没有发生本征值的简并)
\begin{equation}
A=\sum_i\lambda_i\ket{i}\bra{i}.~
\end{equation}
对于待测量的物理系统,我们可以用这一组完备正交基来对量子态进行展开,具体来说,对于一个量子态$\ket{\psi}$,在上面测量$A$得到结果$\lambda_i$的概率为$p(\lambda_i)=|\braket{\psi}{i}|^2$。

对于含有简并的情况也可以类似地处理。虽然在这种情况下本征态分解并不唯一,但是可以通过

