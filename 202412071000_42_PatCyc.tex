% 链、路和圈
% keys 链|路|圈
% license Usr
% type Tutor

\pentry{图\nref{nod_Graph}}{nod_9c36}
本节介绍一类特殊的图——链,而路是一类特殊的链,圈是一类特殊的路。

\subsection{链}
想象我们对链的映像,比如一个铁链,其是有两个端点的,把一端拎起,另一端就自然下降了;还有手链,项链,它们的是端点可以重合的链。对于链而言(想象铁链)它的端点可以通过一个个类似“环”的结构相扣而成。这些环可以看着点和线构成的一个封闭圈。我们的链的映像似乎可以抽象成点和线相互交叉排列称的一个结构,并且我们总是能认为它有两个(可以重合的)端点。对于那种不是点线连接的现状相似的均匀结构,我们往往称作带了。

\begin{definition}{链}
设 $G=(V,E,\varphi)$ 是一个\enref{图}{Graph}。若其点和边可以构成以下长为 $k$ 的序列
\begin{equation}
v_0 e_1v_1e_2\cdots v_{k-1} e_kv_k,~
\end{equation}

\end{definition}














