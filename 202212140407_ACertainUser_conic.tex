% 圆锥曲线

\begin{issues}
\issueDraft
\end{issues}

\subsection{圆锥曲线的定义}
圆锥曲线的一般性定义:曲线上任意一点至一定点(焦点)的距离,与它至一直线(准线)的垂直距离始终成固定比例(离心率):
\begin{equation}
\left \{P \vert \frac{\abs{PF}}{\abs{PL}}=e \right \}
\end{equation}
其中$P$为曲线上一点,$F$为该定点,$L$为定直线,$\abs{PL}$表示点到直线的垂直距离.

根据$e$的取值,可将圆锥曲线分为三类.
\begin{table}[ht]
\centering
\caption{圆锥曲线的分类}\label{conic_tab2}
\begin{tabular}{|c|c|}
\hline
离心率 & 名称\\
\hline
$0<e<1$ & 椭圆\\
\hline
$e=1$ & 抛物线\\
\hline
$e>1$ & 双曲线\\
\hline
\end{tabular}
\end{table}

\subsection{圆锥曲线方程}
本节使用的术语请参考下一章.
\subsubsection{极坐标方程}
\begin{equation}
r(\theta)  = \frac{l}{1 - e\cos \theta }
\end{equation}

该方程使用的原点为圆锥曲线的(一个)焦点.其中 $e$ 是离心率, $l$ 是半通径,极角 $\theta$ 的取值范围是所有使 $r>0$ 的值. 
具体的推导与说明请参考圆锥曲线的极坐标方程\upref{Cone}.
\subsubsection{直角坐标方程}
\begin{table}[ht]
\centering
\caption{圆锥曲线直角坐标方程}\label{conic_tab3}
\begin{tabular}{|c|c|}
\hline
* & * \\
\hline
* & * \\
\hline
* & * \\
\hline
* & * \\
\hline
\end{tabular}
\end{table}

一些术语:
\begin{itemize}
\item 焦距:两焦点的距离
\item 焦准距:焦点至与之对应的准线的距离
\item 通径:过焦点做垂线与曲线相交于两点,这两点所确定的直线段
\end{itemize}

\subsection{常见圆锥曲线}

\begin{figure}[ht]
\centering
\includegraphics[width=10cm]{./figures/conic_2.pdf}
\caption{椭圆} \label{conic_fig2}
\end{figure}

\begin{figure}[ht]
\centering
\includegraphics[width=10cm]{./figures/conic_3.pdf}
\caption{抛物线} \label{conic_fig3}
\end{figure}

\begin{figure}[ht]
\centering
\includegraphics[width=10m]{./figures/conic_4.pdf}
\caption{双曲线} \label{conic_fig4}
\end{figure}

\begin{table}[ht]
\centering
\caption{圆锥曲线}\label{conic_tab1}
\begin{tabular}{|c|c|c|c|c|c|}
\hline
名称 & 直角坐标方程 & 半焦距 Linear Eccentricity $c$ & 离心率 Eccentricity $e = \frac{c}{a}$ & 半通径 Semi Latus Rectum $l=\frac{b^2}{a}$ & 焦准距 Focal Parameter$p=\frac{b^2}{c}$ & 备注\\
\hline
% (圆)Circle & $x^2+y^2=a^2$ & 0 & 0 & $a$ & \ & 一般不认为是圆锥曲线;在思考“离心率、半焦距时”可以不严谨地认为圆是两焦点重合于圆心、准线在无穷远处的椭圆\\
% \hline
椭圆 Ellipse & $\frac{x^2}{a^2} + \frac{y^2}{b^2} = 1$ & $\sqrt{a^2-b^2}$, $b^2+c^2=a^2$ & $\sqrt{1-\frac{b^2}{a^2}} < 1$ & $\frac{b^2}{a}$ & $\frac{b^2}{\sqrt{a^2-b^2}}$ &  \\
\hline
抛物线 Parabola & $y^2=4ax$ & \ & 1 & $2a$ & $2a$ & 只有一条准线和一个焦点\\
\hline
双曲线 Hyperbola & $\frac{x^2}{a^2} - \frac{y^2}{b^2} = 1$ & $\sqrt{a^2+b^2}$,$a^2+b^2=c^2$ & $\sqrt{1+\frac{b^2}{a^2}}$ > 1 & $\frac{b^2}{a}$ & $\frac{b^2}{\sqrt{a^2+b^2}}$ & 分为互不相连的两支 \\
\hline
\end{tabular}
\end{table}

\footnote{本文参考自Wikipedia的Conic Section、圆锥曲线词条.本文适用于CC-BY-SA.}
