% 第四代反应堆
% license CCBYSA3
% type Wiki

(本文根据 CC-BY-SA 协议转载自原搜狗科学百科对英文维基百科的翻译)

\textbf{第四代反应堆}是由第四代核能系统国际论坛目前正在研究的一系列用于商业应用的核反应堆设计方案。器技术准备水平从需要示范的水平到具有经济竞争力的实施水平之间存在差异。[1]它们的设计理念是为了实现各种各样的目标,其中包括提高安全性、提高可持续性、提高反应堆运行效率以及降低安装成本、运行成本等。

目前最先进的第四代反应堆设计为钠冷快堆,多年来它获得了最大份额的资金,并且运行了许多示范设施。 该设计的第四代主要内容是为反应堆开发出一种可持续的闭式燃料循环。 熔盐堆是一种较不成熟的技术,但在六种第四代反应堆设计方案中被认为可能具有最大的固有安全性。[2][3]超高温反应堆被设计在更高的温度下运行。这允许在反应堆运行期间可进行高温电解,从而有效地生产氢气以及合成碳中性燃料[1]

这六种第四代反应堆设计中的大多数需要到2020-2030年期间才能用于商业建设。[4]目前,世界上运行的大多数反应堆被认为是第二代反应堆系统,因为绝大多数第一代系统在一段时间前已经退役,截至2014年,只有少数第三代反应堆在运行。第五代反应堆是指纯理论的反应堆,在短期时间内尚不可行,因此相应的研发资金相当有限。

\subsection{历史}
第四代核能系统国际论坛(GIF)是“一项国际性的合作努力,旨在开展必要的研究和开发,以实现下一代核能系统的可行性和性能。”[5]它成立于2001年。目前,第四代核能系统国际论坛的活跃成员包括:澳大利亚,加拿大、中国、欧洲原子能共同体(Euratom)、法国、日本、俄罗斯、南非、韩国、瑞士和美国。非活跃成员包括:阿根廷、巴西和英国。[6]瑞士于2002年加入,欧洲原子能共同体(Euratom)于2003年加入,中国和俄罗斯于2006年加入,澳大利亚[7]2016年加入。其余国家是创始成员国。[6]

第36届GIF会议于2013年11月在布鲁塞尔举行。[8][9]《第四代核能系统技术路线图更新》于2014年1月发表,详细介绍了未来十年的研发目标。[10]每个论坛成员正在研究的反应堆设计的细目已经公布。[11]

据报道,2018年1月,“世界上第一个第四代反应堆压力容器盖的首次安装”已在HTR项目中完成。[12]

\subsection{反应堆类型}
最初考虑了许多反应器类型;但是为了将重点放在最有前景的技术和那些最有可能实现第四代计划目标的技术上,删减了许多名单上的方案。[4]三个设计方案名义上是热中子反应堆,四个是快中子反应堆。超高温反应堆(VHTR)也正在研究中,它可为氢生产的过程中提供高质量的热能。快中子反应堆提供了燃烧锕系元素以进一步减少废物的可能性,并且能够“产生比它们消耗的更多的燃料”。这些系统在可持续性、安全性和可靠性、经济性、防扩散能力(视情况而定)和物理保护方面取得了重大进展。

\subsubsection{2.1 热中子反应堆}
热中子反应堆是使用慢中子或者热中子的核反应堆。中子慢化剂用于减缓裂变所释放的中子,使它们更有可能被燃料俘获。

\textbf{超高温堆}

\begin{figure}[ht]
\centering
\includegraphics[width=8cm]{./figures/bb468d9b88c246d0.png}
\caption{高温堆} \label{fig_FYD_1}
\end{figure}

\textbf{超高温反应堆的}概念是使用了石墨作为中子慢化剂以、铀作为燃料,使用氦或者熔盐作为冷却剂。该反应堆的设计出口温度为1000℃。反应堆堆芯可以是棱柱状块体或者球床反应堆设计。高温使得诸如工艺热或通过热化学碘-硫工艺生产氢气的应用成为可能。超高温反应堆也具有固有安全性。

第一座超高温反应堆的计划建设是南非的PBMR ( 床模块化反应堆)。但在2010年2月,政府中断了对该项目的资助。[13]该项目成本的显著增加以及担忧可能出现的意外技术问题令潜在投资者和客户望而却步。

2012年,中国开始建造一座200兆瓦的高温球床反应堆,作为HTR-10反应堆的延续。[14]

同样在2012年,作为下一代核电厂竞争的一部分,Idaho National Laboratory 批准了一项类似于阿海珐(Areva)棱柱状块体安塔雷斯(Antares)反应堆的设计,作为选定的HTGR,将于2021年作为原型堆部署。它与通用原子(General Atomics) 的燃气轮机模块化氦反应堆和西屋的球床模块高温气冷堆竞争。[15]

X-energy 获得了美国能源部(Department of Energy)一份为期五年、价值5300万美元的先进反应堆概念合作协议(Advanced Reactor Concept cooperation Agreement),用于推进其反应堆开发的各项内容。[16] Xe-100 是球床模块化反应堆,将产生200MWt和大约76MWe。标准Xe-100“四包式工厂可生产约300 MWe,占地仅13英亩。Xe-100的所有组件都可通过公路进行运输,并且在项目现场进行安装,而不是建造,从而可大大简化施工。[17]

\textbf{熔盐反应堆(MSR)}

\begin{figure}[ht]
\centering
\includegraphics[width=8cm]{./figures/0a53dbafe1d63342.png}
\caption{熔盐堆(MSR)} \label{fig_FYD_2}
\end{figure}

\textbf{熔盐反应堆}[18]是一种核反应堆,其主冷却剂,甚至是燃料本身都是熔盐混合物。对于这种类型的反应堆,科研人员已经提出了许多设计方案,并建造了一些原型堆。早期的概念以及当前的许多概念都在于核燃料方案的选择,四氟化铀(UF4)或四氟化钍(ThF4)溶解在熔融的氟化物盐中。该流体将流入一个石墨充当慢化剂的核心,从而达到临界状态。当前的许多概念依赖于分散在石墨基体中的燃料,其中熔融盐提供了一个低压、高温冷却的环境。

第四代反应堆中的熔盐反应堆应该称为超热中子反应堆,而不是热中子反应堆,因为堆芯中导致其燃料发生裂变事件的中子平均速度比热中子更快[19]

MSR的工作原理可用于热中子反应堆、超热中子反应堆和快中子反应堆。自2005年以来,焦点已转向快速频谱MSR (MSFR)。[20]

虽然大多数MSR的设计主要源自20世纪60年代的熔盐反应堆实验(MSRE),熔盐堆相关技术的发展包括概念上的双流反应堆设计以及使用铅作为冷却介质,但熔盐燃料通常为金属氯化物,例如三氯化钚,从而提高“核废料”封闭式燃料循环能力。其他显著不同于MSRE的方法包括MOLTEX提出的稳定盐反应堆(SSR)概念,该概念将熔融盐包裹在数百个已经在核工业中得到广泛应用的普通固体燃料棒中。2015年,总部位于英国的咨询公司Energy Process development发现,后一种英国设计被认为是最具竞争力的小型模块化反应堆。[21][22]

MSR的另一个显著特点是有可能使用热光谱核废料燃烧器。传统上认为,只有快速反应堆被认为是可行的利用或减少核废料。Seaborg Technologies在2015年春季发布的白皮书中首次展示了热废热燃烧器的概念可行性。[23]热废物的燃烧是通过用钍代替乏燃料中的一小部分铀来实现的。超铀元素(例如钚和镅)的净生产量率降低到消耗率以下,从而减少了核储存问题的规模,且不涉及核扩散以及与快堆相关的其他技术问题。

\textbf{超临界水冷反应堆(SCWR)}

\begin{figure}[ht]
\centering
\includegraphics[width=8cm]{./figures/cbed4f48e66bc5db.png}
\caption{v} \label{fig_FYD_3}
\end{figure}

\textbf{超临界水堆(SCWR)}[18]是一个减温水反应堆概念,由于导致燃料内裂变事件的中子平均速度比热中子快,因此它被更准确地称为超热中子反应堆而不是热中子反应堆。它使用超临界水作为工作流体。超临界水冷反应堆基本上是在较高的压力和温度下运行的轻水反应堆(LWR),具有直接的单程热交换循环。正如最常见的设想,它将在直接循环中运行,与沸水堆(BWR )很相似,但是由于它使用超临界水(不要与临界质量混淆)作为工作流体,因此它将只存在一个水相,这使得超临界热交换方法更类似于压水堆( PWR )。它可以在比当前压水堆和沸水堆高得多的温度下运行。

超临界水冷反应堆(SCWRs)是有前途的先进核系统,因为它们具有高的热效率(其效率约为45\%,而目前轻水反应堆的效率约为33\%)和显著的设备简化。

超临界水反应堆的主要任务是生产低成本的电力。它基于两种成熟的技术,一种是轻水堆,这是世界上最常用的发电反应堆,另一项是过热的化石燃料锅炉,世界各地也在大量使用这种锅炉。13个国家的32个组织正在进行超临界水反应堆的相关研究。

由于超临界水反应堆是水反应堆,它们与沸水堆和轻水堆一样,都具有蒸汽爆炸和放射性蒸汽释放的危险,并且需要极其昂贵的重型压力容器、管道、阀门和泵。由于超临界水反应堆在较高的温度下运行,这些共同的问题对超临界水反应堆来说,本质上更加严重。

正在开发中的超临界水反应堆设计是VVER -1700/393(VVER-SCWR或VVER-SKD),这是俄罗斯的一个超临界水冷反应堆,具有双入口堆芯,其增值比为0.95。[24]

\subsubsection{2.2 快中子反应堆}
快中子反应堆直接利用裂变释放出的快中子,不进行慢化。与热中子反应堆不同,快中子反应堆可以被配置为”\textbf{烧伤}“,或裂变,所有的锕系元素,并给予足够的时间,因此大大降低了当前世界热中子轻水反应堆产生的乏核燃料中锕系元素的分数,从而结束了核燃料循环。或者,如果配置不同,它们也可以\textbf{类型}锕系元素燃料比它们消耗的要多。

\textbf{气冷快中子反应堆}

\begin{figure}[ht]
\centering
\includegraphics[width=8cm]{./figures/103cb2902b09e106.png}
\caption{气冷快中子反应堆} \label{fig_FYD_4}
\end{figure}

\textbf{气冷快堆(GFR)}[18]系统具有快中子能谱和封闭的燃料循环,可有效地转换富铀和处理锕系元素。该反应堆采用氦气冷却,出口温度为850℃。 是超高温反应堆(VHTR)向更可持续的燃料循环的发展。它将直接使用一个布雷顿循环燃气轮机,热效率很高。目前正在考虑几种燃料形式,因为它们有可能在非常高的温度下工作,并确保裂变产物的良好保留:复合陶瓷燃料、先进的燃料颗粒或锕系化合物的陶瓷包覆元素。并且正在考虑基于销或板基燃料组件或棱柱块的堆芯配置。

欧洲可持续核工业倡议(European Sustainable Nuclear Industrial Initiative)正在为第三代(Generation IV)反应堆系统提供资金,其中一个是名为Allegro的气冷快堆(100 MW(t)),将在中欧或东欧国家建造,预计2018年开始建设。[25]]欧洲中部Visegrad集团致力于这项技术的研发。[26]德国、英国和法国的研究机构完成了一项为期3年的工业规模设计合作研究,称为GoFastR。[27]它们是由欧盟第七届FWP框架方案资助的,其目标是制造一个可持续的VHTR。[28]