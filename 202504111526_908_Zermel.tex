% 策梅洛-弗兰克尔集合论(综述)
% license CCBYSA3
% type Wiki

本文根据 CC-BY-SA 协议转载翻译自维基百科\href{https://en.wikipedia.org/wiki/Zermelo\%E2\%80\%93Fraenkel_set_theory}{相关文章}。

在集合论中,泽梅洛-弗兰克尔集合论,以数学家恩斯特·泽梅洛和亚伯拉罕·弗兰克尔命名,是一个公理化系统,旨在20世纪初提出,以构建一个没有类似罗素悖论之类悖论的集合理论。如今,泽梅洛-弗兰克尔集合论,包含历史上具有争议的选择公理(AC),是标准的公理化集合论形式,因此也是数学的最常见基础。包含选择公理的泽梅洛-弗兰克尔集合论简称为ZFC,其中C代表“选择”\(^\text{[1]}\),而ZF指的是没有包含选择公理的泽梅洛-弗兰克尔集合论的公理。

非正式地说\(^\text{[2]}\),泽梅洛-弗兰克尔集合论旨在形式化一个单一的基本概念,即遗传的良基集合,使得所有在讨论宇宙中的实体都是这样的集合。因此,泽梅洛-弗兰克尔集合论的公理仅涉及纯集合,并防止其模型中包含尿元素(不是集合本身的元素)。此外,适当类(由其成员共享的属性定义的数学对象集合,这些集合过大无法作为集合处理)只能间接处理。具体来说,泽梅洛-弗兰克尔集合论不允许存在一个普遍集合(包含所有集合的集合),也不允许无限制的理解,从而避免了罗素悖论。冯·诺依曼-伯奈斯-哥德尔集合论(NBG)是泽梅洛-弗兰克尔集合论的一个常用保守扩展,它确实允许对适当类进行显式处理。

