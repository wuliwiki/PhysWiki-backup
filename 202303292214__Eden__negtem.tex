% 负温度
% 负温度|核自旋系统|熵

1951年珀塞耳(Purcell)和庞德(Pound)发现氟化锂(LiF) 晶体中的核自旋系统可以处于负绝对温度状态。
1956年喇姆塞(Ramsey)给予了理论解释。\cite{热统}

这里的温度在统计力学中的严格意义是\autoref{tmp_eq1}~\upref{tmp}
\begin{equation}
\frac{1}{T}=\left(\frac{\dd S}{\dd U}\right)_{V,N}^{-1}
\end{equation}
在一般系统中,熵随内能的增大而增大(直观的理解就是,系统越热,就越“乱”)。但也存在一些特殊的系统,它们的熵在特定温度范围内随内能的升高而减小,此时由上式得出的温度就是负的。

\subsection{理论简介}
设核自旋量子数为 $j=\frac{1}{2}$,在外磁场中,核磁矩 $\mu$ 相对于外磁场 $H$ 只有平行和反平行两种取向。相应的能量也只有两种取值:$\epsilon_1=-\mu H,\quad \epsilon_2=\mu H$。假设每个能级只有一个量子态(简并度 $g_1=g_2=1$)。由于核自旋彼此之间相互作用很弱,它们组成了近独立的\textbf{定域子系}