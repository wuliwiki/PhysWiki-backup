% 引力的弱场近似
% keys 广义相对论|relativity|引力|gravity|弱场|weak field|测地线|geodesic|牛顿力学|闵可夫斯基时空|Minkowski spacetime|时空|spacetime|流形|manifold|闵可夫斯基度规|Minkowski metric

\pentry{测地线\upref{geodes}}

广义相对论的革命性创见在于将引力解释为时空的几何效应.广义相对论将引力视为非力作用,并假设不受力的物质的运动轨迹是测地线,其参数取该物质的本征时间.从测地线一节中的\autoref{geodes_eq1}~\upref{geodes}可知,非平坦的度量在给定图中可能有非零的Christoffel符号,使得测地线方程的解不再是这个图中的一条匀速直线,这就启发了我们,如果把时空看成是流形,那么引力可能通过影响联络来造成表面上的“扭转”物质轨迹.当然,由于曲率是由联络来定义的,也可以说是引力改变了时空的曲率.

以上只是定性说法,那么我们是否真的可以用这种方法来描述引力作用呢?由于牛顿的引力论在低速、弱场的情况下已经被实验反复证实,我们可以从此下手,在低速弱场近似下尝试解一个质点的测地线方程,看能不能回归到牛顿的引力方程上.

\subsubsection{近似假设}

低速近似意味着,在所讨论的参考系里,质点的四速度非常接近$\pmat{1}$
























