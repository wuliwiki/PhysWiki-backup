% 卷绕数(综述)
% license CCBYSA3
% type Wiki

本文根据 CC-BY-SA 协议转载翻译自维基百科\href{https://en.wikipedia.org/wiki/Winding_number}{相关文章}。

\begin{figure}[ht]
\centering
\includegraphics[width=6cm]{./figures/801932f041ea8a31.png}
\caption{这条曲线相对于点 $p$ 的绕数是2。} \label{fig_JRS_1}
\end{figure}
在数学中,闭合曲线相对于平面中某一点的绕数或绕线指数是一个整数,表示该曲线绕该点逆时针方向环绕的总次数,也就是曲线的“转数”。对于某些非闭合的平面曲线,其绕数可能是非整数。绕数依赖于曲线的方向:如果曲线沿顺时针方向绕点运动,则绕数为负数。

绕数是代数拓扑中的基础研究对象,并且在向量分析、复分析、几何拓扑、微分几何以及物理学(例如弦理论)中都扮演着重要角色。
\subsection{直观描述}
沿着红色曲线运动的一个物体,会绕位于原点的人逆时针转两圈。
假设我们有一条位于 $xy$ 平面上的闭合、有方向的曲线。我们可以把这条曲线想象成某个物体的运动轨迹,而曲线的方向表示物体运动的方向。这样,这条曲线的绕数就等于该物体绕原点逆时针转的总圈数。

在计算总圈数时,逆时针的运动计为正数,而顺时针的运动计为负数。例如,如果一个物体先绕原点逆时针转了四圈,然后又绕原点顺时针转了一圈,那么这条曲线的总绕数就是3。

按照这个规则,一条完全没有绕过原点的曲线,其绕数为0;而一条绕原点顺时针运动的曲线,其绕数为负数。因此,曲线的绕数可以是任意整数。下面的图示展示了绕数从 −2到3的不同曲线。
\begin{figure}[ht]
\centering
\includegraphics[width=10cm]{./figures/f02fc89a7a934eb5.png}
\caption{} \label{fig_JRS_2}
\end{figure}
\subsection{形式化定义}
设$\gamma: [0,1] \to \mathbb{C} \setminus \{a\}$是一条平面上去掉点 $a$ 后的连续闭合路径。曲线$\gamma$绕点$a$的绕数定义为整数:
$$
\mathrm{wind}(\gamma, a) = s(1) - s(0),~
$$
其中 $(\rho, s)$ 是该路径用极坐标表示的形式,即通过如下覆盖映射 $p$ 提升得到的路径:
$$
p: \mathbb{R}_{>0} \times \mathbb{R} \to \mathbb{C} \setminus \{a\} : (\rho_0, s_0) \mapsto a + \rho_0 e^{i 2\pi s_0}.~
$$
绕数的良好定义性源于提升路径的存在性和唯一性(给定覆盖空间的起点),并且因为映射 $p$ 的所有纤维都具有以下形式:$\rho_0 \times (s_0 + \mathbb{Z})$,所以上述表达式不依赖于起点的具体选择。由于路径是闭合的,最终得到的绕数是一个整数。
\subsection{替代表述}
在数学的不同分支中,绕数常常有不同的定义方式。下面列出的所有定义都与前面给出的定义是等价的:
\subsubsection{亚历山大编号法}
1865 年,奥古斯特·费迪南德·莫比乌斯,首次提出了一种用简单组合规则来定义绕数的方法\(^\text{[1]}\);1928 年,小詹姆斯·沃德尔·亚历山大又独立地提出了同样的规则。\(^\text{[2]}\)根据该方法,任意一条曲线都会将平面分割成若干个连通区域,其中有一个区域是无界的。这一定义具有以下性质:在同一个区域内的任意两点,其对应的绕数是相同的。无界区域内(即平面外延处任意点)的绕数为0。相邻两个区域的绕数之差正好是1;其中绕数较大的区域出现在曲线运动方向的左侧。
\subsubsection{微分几何}
在微分几何中,参数方程通常假定是可微的(或至少是分段可微的)。在这种情况下,极坐标角 $\theta$ 与直角坐标 $x$ 和 $y$ 的关系由以下公式给出:
$$
d\theta = \frac{1}{r^{2}}\left(x\,dy - y\,dx\right)
\quad\text{其中 } r^{2} = x^{2} + y^{2}.~
$$
这个公式可以通过对 $\theta$ 的定义进行求导得到:
$$
\theta(t) = \arctan\!\left(\frac{y(t)}{x(t)}\right).~
$$
根据微积分基本定理,$\theta$ 的总变化量等于 $d\theta$ 的积分。因此,一条可微曲线的绕数可以表示为一条曲线积分:
$$
\text{wind}(\gamma, 0) = 
\frac{1}{2\pi} 
\oint_{\gamma} 
\left(\frac{x}{r^{2}}\,dy - \frac{y}{r^{2}}\,dx\right).~
$$
在去掉原点的平面上,微分形式 $d\theta$ 是一个闭合但非全微分的 1-形式,它生成了穿孔平面的第一 de Rham 上同调群。特别地,如果 $\omega$ 是定义在去掉原点的平面上的任意一个闭合的、可微的 1-形式,那么沿任意闭合路径积分 $\omega$ 的结果都会是绕数的某个倍数。
\subsubsection{复分析}
在复分析中,绕数在许多理论中都起着极其重要的作用(例如留数定理的表述)。在复分析的语境下,复平面上一条闭合曲线 $\gamma$ 的绕数可以用复坐标 $z = x + iy$ 来表示。若写作 $z = r e^{i\theta}$,则有:
$$
dz = e^{i\theta}dr + i r e^{i\theta} d\theta~
$$
因此:
$$
\frac{dz}{z} = \frac{dr}{r} + i\,d\theta = d[\ln r] + i\,d\theta.~
$$
由于 $\gamma$ 是闭合曲线,$\ln(r)$ 的总变化量为零,因此积分 $\frac{dz}{z}$ 等于 $i$ 乘以 $\theta$ 的总变化量。由此,曲线 $\gamma$ 绕原点的绕数为\(^\text{[3]}\):
$$
\frac{1}{2\pi i} \oint_{\gamma} \frac{dz}{z}.~
$$
更一般地,如果曲线 $\gamma$ 由参数 $t \in [\alpha, \beta]$ 表示,且对于复平面上的某点 $z_0$,有 $z_0 \notin \gamma([\alpha, \beta])$,那么曲线 $\gamma$ 绕点 $z_0$ 的绕数(也称为曲线 $\gamma$ 关于 $z_0$ 的指标)定义为\(^\text{[4]}\):
$$
\mathrm{Ind}_{\gamma}(z_0)
= \frac{1}{2\pi i} \oint_{\gamma} \frac{d\zeta}{\zeta - z_0}
= \frac{1}{2\pi i} \int_{\alpha}^{\beta} 
\frac{\gamma'(t)}{\gamma(t) - z_0}\, dt.~
$$
这个公式正是著名的柯西积分公式的一个特例。

在复平面中,绕数的一些基本性质可以通过以下定理给出:\(^\text{[5]}\)

\textbf{定理}:设$\gamma:[\alpha,\beta]\to\mathbb{C}$是一条闭合路径,并令$\Omega := \mathbb{C} \setminus \gamma([\alpha,\beta])$表示曲线 $\gamma$ 的像的补集。那么,曲线 $\gamma$ 关于点 $z$ 的指标(即绕数)定义为:
$$
\mathrm{Ind}_{\gamma}:\Omega\to\mathbb{C}, \quad 
z \mapsto 
\frac{1}{2\pi i}
\oint_{\gamma} 
\frac{d\zeta}{\zeta - z},~
$$
并且它满足以下性质:1.$\mathrm{Ind}_{\gamma}(z) \in \mathbb{Z}, \quad \forall z\in\Omega$;2. 分区常数性:在 $\Omega$ 的每一个连通分支(即最大连通子集)上,$\mathrm{Ind}_{\gamma}(z)$ 是常数;3. 无界区域的零绕数:若点 $z$ 位于 $\Omega$ 的无界连通区域,则$\mathrm{Ind}_{\gamma}(z) = 0$.

\textbf{推论}:该定理直接给出了圆形路径绕某点的绕数。如预期一样,绕数等于曲线 $\gamma$ 绕该点逆时针环绕的次数。若路径 $\gamma$ 定义为:$\gamma(t) = a + r e^{int}, \quad 0 \leq t \leq 2\pi, \quad n \in \mathbb{Z},$则有:
$$
\mathrm{Ind}_{\gamma}(z) =
\begin{cases}
n, & |z-a|<r;\\[6pt]
0, & |z-a|>r.
\end{cases}~
$$
\subsubsection{拓扑学}
在拓扑学中,绕数是连续映射的度的另一种称呼;在物理学中,绕数则常被称为拓扑量子数。两种情况下,本质上描述的都是同一个概念。

上述曲线绕某点的例子有一个简单的拓扑解释:平面去掉一点后的空间与一个圆 $S^1$ 是同伦等价的,因此从圆到自身的映射是需要重点考虑的对象。可以证明,每一个这样的映射都可以连续变形(同伦)为某个标准映射:$S^1 \;\to\; S^1: \; s \;\mapsto\; s^n$,其中,圆上的乘法是通过将其识别为复平面上的单位圆来定义的。从一个圆映射到某个拓扑空间的所有同伦类组成一个群,这个群被称为该空间的**第一同伦群(fundamental group 或 $\pi_1$)。圆的第一同伦群是整数群 $\mathbb{Z}$,而复平面上一条曲线的绕数,实际上就是它对应的同伦类。

类似地,从三维球面 $S^3$ 到自身的映射也可以用一个整数进行分类,这个整数同样被称为绕数,有时也称为庞特里亚金指数。
\subsection{转数}
\begin{figure}[ht]
\centering
\includegraphics[width=6cm]{./figures/801932f041ea8a31.png}
\caption{} \label{fig_JRS_3}
\end{figure}
人们也可以考虑一条路径相对于**自身切向方向**的绕数。把路径看作随时间变化的运动轨迹时,这相当于求速度向量相对于原点的绕数。在这种情况下,本篇文章开头的示例中,由于包含了小环路,所以绕数为3*。

这种定义仅适用于浸入路径(即处处可微且导数不为零的路径),其值等于**切向高斯映射的度。

这种数值被称为**转数、**旋转数[6]、旋转指数[7] 或 曲线指数,并且可以通过**总曲率除以 $2\pi$** 来计算。
\subsubsection{多边形}
在多边形中,**转数**也被称为**多边形密度(polygon density)**。对于**凸多边形**,以及更一般的**简单多边形**(即不自交的多边形),根据**约旦曲线定理**,其密度为 **1**。相比之下,对于一个正星形多边形 $\{p/q\}$,其密度为 **q**。
\subsubsection{空间曲线}
对于**空间曲线**,无法直接定义转数,因为“度”的概念要求维度匹配。然而,对于**局部凸的闭合空间曲线**,可以定义一个**切向转向符号**:$(-1)^d$,其中 $d$ 是其切向指示曲线的**立体投影的转数**。该符号的两种取值对应于**局部凸曲线的两类非退化同伦类**。[8][9]
