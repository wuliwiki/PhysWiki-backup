% 抛物线(高中)
% keys 极坐标系|直角坐标系|圆锥曲线|抛物线
% license Xiao
% type Tutor

\begin{issues}
\issueDraft
\end{issues}

\pentry{解析几何\nref{nod_JXJH},圆\nref{nod_HsCirc},双曲线\nref{nod_Hypb3}}{nod_7c17}

不知道读者在初次接触双曲线时,是否产生了一种似曾相识的感觉:它的一支看起来与初中阶段学习过的二次函数图像——抛物线,非常相似。二者都不封闭、有一个开口、略微弯曲、向无限延伸,甚至也拥有一条对称轴。相信一些读者可能早已不禁在心中将双曲线的一支等同于抛物线,认为双曲线不过是“两个抛物线”的组合而已。

这种误解并不罕见,难以否认,抛物线与双曲线在形状上确实有些相像。但从本质上看,二者在几何定义、解析式结构以及性质上都有明显区别。之所以容易混淆,很大程度上与初中学习的重点有关。当时的教学更侧重于二次函数的代数表达与图像之间的关系,例如判断开口方向、对称轴位置、顶点坐标等。这些内容有助于建立对抛物线的基本印象,但主要仍停留在函数视角,对抛物线作为几何图形本身的理解较为有限。

大多数人对抛物线的印象,往往停留在现实生活中物体被抛出后所形成的轨迹。在理想状态下,这类轨迹正是一条抛物线,这也正是“抛物线”名称的来源。然而,随着人们的进一步研究发现,抛物线并不仅仅出现在物理运动中,它还具有独特的几何性质,在许多实际工程中发挥着重要作用。例如,雷达天线的反射面通常设计成抛物面结构,原因就在于抛物线具备一种精确的聚焦特性:来自远处的平行电磁波在抛物面上反射后,会准确地汇聚到焦点;而从焦点出发的信号,也能被反射成方向一致的平行波。这一聚焦能力,使抛物面非常适合实现能量的集中与传输,使抛物线广泛出现于雷达、卫星通信设备、汽车大灯以及太阳能灶等场景中。

\subsection{抛物线的几何定义}

由于在初中阶段已经接触过抛物线,并对它的图像有一定了解,因此在这里不再从函数视角出发,而是直接进入几何定义的探讨。

回顾研究椭圆和双曲线的定义时,都曾尝试从圆的定义进行推广。圆可以看作是满足“到两个顶点$O_1,O_2$满足距离$d=0$的点的距离相等”的点的轨迹,也就是满足 $|O_1P| = |O_2P| = r$ 的点 $P$ 所组成的图形。椭圆和双曲线就是通过调整$d$之后,再改变第一个等号而得到的。

那么,如果不去动第一个等号,而是打开第二个等号,也就是只要求 $|O_1P| = |O_2P|$,又会得到什么?之前提到过,这就是所有关于两个定点的垂直平分线上的点的集合。

此时可以尝试进一步变换思路:如果不再比较“点到点”的距离,而是比较“点到点”与“点到线”的距离呢?换句话说,如果一个点到某个定点的距离,等于它到一条固定直线的距离,会发生什么?

当这条直线恰好经过定点时,满足条件的只有一个点,结果非常平凡。但若这条直线和定点之间保持一定距离,那么情况就变得有趣起来。所有满足“到定点与到直线距离相等”的点所形成的轨迹,正是一条新的曲线——这便是\textbf{抛物线(parabola)}的几何定义来源。

通过这一小小的元素替换,原本的“两个点”变成了“一个点和一条直线”,最终产生了又一种常见且极具应用价值的曲线。抛物线的几何构造,不仅补充了对二次函数图像的理解,也为后续学习提供了全新的角度。



在学习椭圆和双曲线的过程中,常常尝试从圆的定义出发,通过调整其表达形式来进行推广。圆的定义可以写成:对于平面上任意一点 $P$,满足 $|O_1P| = |O_2P| = r$,其中 $O_1$ 与 $O_2$ 实际上是同一个圆心,只是从定义表达角度拆分出来。通过放松这个等式中的约束条件,可以构造出不同的新曲线。

例如,椭圆与双曲线都可以看作是“打开”第一个等号的结果。若仅要求 $|O_1P| + |O_2P| = 2a$,就得到了\textbf{椭圆(ellipse)};而如果改为 $||O_1P| - |O_2P|| = 2a$,则得到\textbf{双曲线(hyperbola)}。这两类曲线的本质,都来源于对圆中“点到定点距离相等”这一条件的调整。

现在再来看第二个等号,也就是只保留 $|O_1P| = |O_2P|$ 的形式,会得到什么呢?此时,点 $P$ 的轨迹变成了所有点关于两个定点的垂直平分线的集合,即平面上一条过这两个定点中点、垂直于它们连线的直线。这个结果虽然数学上成立,但图形过于简单,难以构成有趣的新曲线。

不过,换个思路,如果不再是两个定点之间作比较,而是保留其中一个点不变,把另一个“点”换成一条直线,会发生什么?换句话说,考虑满足“点到某一固定点与到某一固定直线距离相等”的所有点的轨迹,这样是否能构成新的图形?

需要说明的是,如果这条直线正好通过该点,那么除了这个点本身外,不存在其他任何点能同时满足两个距离相等的条件,此时轨迹非常简单,结论也缺乏研究价值。但如果这条直线与这个点有一定的分离呢?这便构成了一个全新的情形,也就是\textbf{抛物线(parabola)}的几何定义所刻画的图像。

这一看似微小的调整,实际上开启了又一种典型曲线的研究大门。抛物线正是通过将“两个定点之间的比较”转换为“点与直线之间的比较”而得到的,背后蕴含着极为简洁且深刻的几何构造思想。


在研究椭圆和双曲线的定义时,都尝试从圆的定义出发推广,对定义的表达方式进行调整。核心就是调整 $|O_1P| = |O_2P| = r$ 这个限定条件中的一些定价的表达。不论是椭圆还是双曲线,都是打开了第一个等号。下面考虑打开第二个等号,之前提过,只保证$|O_1P| = |O_2P|$可以得到所有点关于两个定点的垂直平分线的集合。看上去似乎没有什么其他可改动的空间了。不过,如果改变一下元素呢?如果原本只是改成了两个点,现在把其中的一个点改成一条直线会得到什么呢?当然,这条直线如果和点重合,那么除了这个点,没有任何其他点能满足条件,结论很平凡,没什么意思。如果让直线和点分开一些呢?



标准定义:平面上到定点(焦点)和定直线(准线)距离相等的点的轨迹
这就是 “\enref{圆锥曲线的极坐标方程}{Cone}” 中对抛物线的定义。
\begin{figure}[ht]
\centering
\includegraphics[width=4.2cm]{./figures/c89771dd2fef516e.pdf}
\caption{抛物线的定义} \label{fig_Para3_1}
\end{figure}

在 $x$ 轴正半轴作一条与准线平行的直线 $L$, 则抛物线上一点 $P$ 到其焦点的距离 $r$ 与 $P$ 到 $L$ 的距离之和不变。

如\autoref{fig_Para3_1}, 要证明由焦点和准线定义的抛物线满足该性质, 只需过点 $P$ 作从准线到直线 $L$ 的垂直线段 $AB$, 由于 $r$ 等于线段 $PA$ 的长度, 所以 $r$ 加上 $PB$ 的长度等于 $AB$ 的长度, 与 $P$ 的位置无关。 证毕。


\subsection{抛物线的方程}

\begin{theorem}{抛物线的标准方程}

\end{theorem}
	•	顶点在原点,轴为 $y$ 轴的标准式:$x^2=2py$
	•	讨论参数 $p$ 的意义(焦点到顶点的距离)
\begin{theorem}{抛物线的参数方程}
	•	用参数表示抛物线上的点,如 $x=pt^2,,y=2pt$ 等(视教学安排可选讲)
\end{theorem}

\subsection{抛物线的几何性质}
	•	对称性(关于轴对称)
	•	顶点、焦点、准线的定义和关系
	•	开口方向与参数正负有关
	•	通用式推导(顶点在 $(h,k)$ 时的方程)
    反射性质(光线从焦点发出反射后平行于轴)
\subsubsection{切线}
	•	给定抛物线方程和点,求切线方程
	•	切线的几何意义(过点,与焦点、准线的关系)
