% 如何自学物理
% license CCBYSA3
% type Tutor

\begin{issues}
\issueDraft
\end{issues}

要谈论这个话题,就不得不提荷兰诺贝尔物理学奖得主 Gerard 't Hooft 的博文 \href{https://webspace.science.uu.nl/~hooft101/theorist.html}{How to become a GOOD theoretical physicist} (\href{https://xialab.pku.edu.cn/kytdyw1/zdylm.m.jsp?wbtreeid=1011&tstreeid=11956&_t_uid=2945&language=en&homepageuuid=BF649325C5584FC683CE0B601D21AC65&templateuuid=4CC182410BA14FF8B55ED726FB2087FB&producttype=0&_tmode_=99&tsitesapptype=zdylm}{一个翻译}), 我们不妨就这篇文章的核心内容展开讨论。

\subsection{物理学的主要研究方向}
首先,虽然文章标题说的是 “理论物理学家”, 但大致来说也\textbf{同样适用于物理学其他研究方向}(除了推荐书目中的基础部分无论对什么方向都是适用的)。 对于公众来说,可能一谈到物理学马上就会想到那些喜闻乐见的理论物理话题例如宇宙、黑洞、弦论等。 这些话题的确非常引人入胜,但远非物理学的全部。 事实上只有一少部分从事物理学研究的人会研究这些领域。

\begin{figure}[ht]
\centering
\includegraphics[width=14.25cm]{./figures/b898880f835f999d.png}
\caption{美国一项两年调查中 1900 名物理博士的研究领域占比(\href{https://ww2.aip.org/statistics/trends-in-physics-phds}{来源})。从上到下分别是凝聚态物理、粒子和场、天体物理/宇宙学、原子分子光学、生物物理、核物理、材料/纳米/表面、光学/光子学、计算物理、等离子/聚变物理、应用/工程/能源研究、量子基础/信息理论、复杂系统/统计/非线性/热物理、相对论/引力、软物质/聚合物物理、其他} \label{fig_SdyPhy_1}
\end{figure}

\subsection{关于民间科学家}
\textbf{民间科学家}简称\textbf{民科},原本是指不在相关大学或研究机构学习或工作但却仍自行从事科学研究的人。 但许多情况下,这个词具有严重贬义,指的是自行发明一些错误理论的科学爱好者。 数学和物理学都是民科的高发地,许多喜欢独立思考的人如果被某个门槛(例如英语、数学、或正规本科教育)卡住而无法学习从事物理学习研究的基本方法,则很可能因为对科普内容或生活经验等过度思考而误入歧途,成为所谓的民科。

“思而不学则殆”, “吾尝终日而思矣,不如须臾之所学也”,以及王阳明的 “格竹” 都是古人对过度思考而不学习的警示。 作为诺奖获得者, Hooft 自然也会经常被各种民科用他们 “一无是处” 的理论骚扰,于是在上面的文章之后,他又写了一片讽刺民科的博文:

\href{https://webspace.science.uu.nl/~hooft101/theoristbad.html#:~:text=On\%20your\%20way\%20towards\%20becoming,have\%20your\%20work\%20published\%20anyway.}{How to become a BAD theoretical physicist}(\href{https://zhuanlan.zhihu.com/p/38680467}{一个翻译})

完全的灌输式教育固然不可取,但现代物理学俨然已经是一座摩天大楼了, 只有快速地熟练掌握相当多的基础知识才有可能为其添砖加瓦,所以若为了 “独立思考” 而走向另一个极端,企图一个人去重建这个大厦则是更危险的。 不仅如此,由于这座大楼如此之高,以至于如果你在最下面几层东张西望,细嚼慢咽,将很难在有限的人生中爬到你想去的位置。

另一方面,即使你的天赋惊人,可以从头建立一套更好的物理学,那么为了让现有的物理学家愿意了解你的成果而不把你误会成其他民科而习惯性无视,你也要先熟悉他们已经掌握了了什么样的理论,他们已经解决了哪些问题,还有哪些缺陷,才能论证你的理论是否比现有的更好。 注意这里的了解并\textbf{不是从文字的角度了解},而是了解物理理论本来的面貌——它们往往\textbf{使用数学}来表达,而这些内容往往出现在\textbf{正规教材和文献}中。

\subsection{英语}
在详细讲解需要学什么以前, Hooft 博士首先提到的就是英语,可见英语的重要性。 注意这篇文章是面向全世界读者的,所以对中国人也同样适用。 英语是事实上的国际学术交流语言。 在物理领域(以及其他大部分自然科学领域),绝大多数论文都是用英语在欧美期刊上发表的,包括国内的研究者。 哪怕是为数较少的国产 SCI 期刊,也多数是英语或双语的。 也就是说,即使是国内的物理研究者之间想要互相了解研究进展,也一般需要读英语论文。

即使抛开学术期刊只谈物理科普和教学方面的\textbf{网络资源}(其他自然科学也基本适用),英语内容无论是数量还是质量都要\textbf{远超中文}。 所以如果你用中文在百度搜索感兴趣的话题但没发现什么有价值的资源,你要意识到这远非互联网上的全部内容。 正如 Hooft 博士所说,你成为理论物理学家所需的一切都可以在互联网上找到, 但关键是你需要分辨哪些是真正对你有用的。 具体到语言,他所说的这些资源几乎都是以英语书写的。

所以无论是学习还是科研,一定程度的英语水平是进入物理领域的门槛, 多早开始学英语(尤其是数学物理专业英语)都不为过。 但从实际一些的角度出发,如果你的英语水平还不足以学习文中推荐书单中的基础内容,国内也可以找到中文代替品,但随着学习的课程更深入,你会发现中文资源变得越来越匮乏。

\begin{figure}[ht]
\centering
\includegraphics[width=8cm]{./figures/636ff07a6af97580.png}
\caption{附:世界网站数量的语言分布,中文仅略高于印度尼西亚语(来源:\href{https://w3techs.com/technologies/overview/content_language}{W3Tech})} \label{fig_SdyPhy_2}
\end{figure}

\subsection{数学}
介绍完语言的重要性后,Hooft 老先生并没有急着讲述物理,而是花了另外两节讲解学习他们所需的数学,可见数学是进入物理殿堂的又一大门槛。没错,物理殿堂的门槛远比许多人想象的高。

现代物理学的两大支柱是相对论和量子力学。 你可能听说过广义相对论的 “引力来自于时空扭曲”, 但如果你不知道黎曼几何的度量张量,或者爱因斯坦场方程,那这句话就不知所云。 你可能听说过量子力学的 “电子云”,“概率波”,但如果不懂薛定谔方程和测量理论,那你永远将停留在大众科普阶段。而令人称奇的是,许多民科在仅了解文字科普内容后,就开始急着要推翻物理学大厦,又或者在最顶端添砖加瓦,指点江山。 殊不知抛开数学描述,这些科普性质的文字描述只是在管中窥豹,看完后甚至不会知道这些理论具体在讲什么,能计算出哪些与实验相符的结果,具体如何计算等,而只是学了一些粗浅的文学比喻。

\addTODO{待续}
