% Slurm 笔记
% license Xiao
% type Note

\begin{issues}
\issueDraft
\end{issues}

\subsection{使用 cpu}
\begin{itemize}
\item \verb` sbatch` 命令用于提交任务
\item \verb` --mem-per-cpu=512M` 指定内存
\item \verb` --time=hours:minutes:seconds` 指定时间
\item \verb` --cpus-per-task=1`
\item \verb` --ntasks=1`
\item \verb` --nodes=1` 机器的数量
\item 一个 batch script 的例子如
\begin{lstlisting}[language=bash]
#!/bin/sh
srun hello.x < ./param.inp
\end{lstlisting}
\item 文件 IO 目录是相对于提交任务时的 \verb` pwd` 的, stdout 会自动生成 \verb` slurm-任务编号.out`
\item \verb` scancel 任务编号` 可以取消某个任务
\item home 文件夹有 1T 的空间
\item 要运行, 例如
\verb|sbatch --time=24:0:0 --mem-per-cpu=300M --cpus-per-task=10 --ntasks=1 --nodes=1 ./job.sh|
\item 要取消, 用 \verb` scancel JOB_ID`, 要取消个人的所有任务, 用 \verb` scancel -u 用户名`
\end{itemize}

\subsection{使用 GPU 节点}
\begin{itemize}
\item 创建 gpu interactive session \verb|srun -J srun -N 1 -n 32 -t 24:00:00 --mem=120G --partition=ksu-gen-gpu.q --gres=gpu:1 --pty bash|
\item \verb`--exclusive` 独占整个节点
\item \verb`--exclude=节点名` 不在指定节点上运行
\item 使用 \verb` module load` 才能加载中 cuda 的 nvcc 和 nvprof。 详见 Environment Modules 笔记\upref{EnvMod}。
\end{itemize}

\subsection{beocat 命令}
\begin{itemize}
\item \verb` kstat` 会列出所有人的所有任务
\item \verb` kstat --me` 可以查看我正在运行的所有任务以及所在节点的其他任务。 或者用 \verb` kstat | grep 用户名` 也可以
\item \verb` kstat | grep 用户名` 也可以查看某个其他用户的所有任务
\item \verb` kstat -c` 可以看到所有用户的 cpu 使用情况
\end{itemize}
