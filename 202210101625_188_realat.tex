% 实数中的开集和闭集
% keys 数学分析



\pentry{度量空间\upref{Metric}}

从实数理论中我们可以知道,有理数集和实数集都是域,同时他们都是良序的.以三个集合的描绘角度来说,它们的代数结构,序结构都是一致的,而第三个拓扑结构才是这两个集合的真正不同之处.我们要在一个足够普遍的空间上进行,但是我的能力有限,不能揭示出其中朴素的思想.总而言之,我们只需要一个有距离的空间——度量空间上去讨论.

利用度量空间上的距离函数,我们可以定义有界集.

\begin{definition}{有界集}
设$E$是度量空间$X$的子集,如果$\exists{p}\in{X}$使得$\exists{M}\in{\mathbb{R}}$,$\forall{q}\in{E}$,总是满足
\[d(p,q)<M\]
则称$E$是$X$上的有界集.
\end{definition}

\textbf{注意}:这里的有界的概念是度量空间的子集中的概念,和有序域中的的上下界以及上下确界是不同的.很快我们会发现它们之间的联系.

\begin{definition}{邻域}
在度量空间$X$中,$p$是$X$上的一个点,围绕$p$的满足$d(p,q)<r$的点的集合,称为$p$的半径为$r$的邻域记作$N_r(p)$
\begin{equation}
N_r(p):=\{q|q\in{X},d(p,q)<r\}
\end{equation}
\end{definition}


\begin{definition}{极限点}
设$E$是度量空间$X$的一个子集,$p\in{X}$是$X$上一个点.当只要$r$大于零时,总有{\heiti 去心邻域}$\hat{N_r}(p)\cap{E}\not=\varnothing$,则称$p$点为$E$上的一个{\heiti 极限点}.
\begin{equation}
\forall{r>0},\hat{N_r}(p)\cap{E}\not=\varnothing
\end{equation}
\end{definition}

%\begin{definition}{稠密集}
%\end{definition}


\begin{definition}{内点}
设$E$是度量空间$X$的一个子集,$p\in{E}$是$E$上一个点.总是存在大于零的$r$,使得点$p$的邻域$N_r(p)\subset{E}$,则称$p$点为$E$上的一个{\heiti 内点}.
\begin{equation}
\exists{r>0},N_r(p)\subset{E}
\end{equation}
\end{definition}

\begin{definition}{闭集}
设$E$是度量空间$X$的一个子集,$E$中所有的极限点都是$E$的元素,那么这样的集合称为闭集.
\begin{equation}
\forall{p\in{E}},\forall{r>0},\text{使得}\hat{N}_r(p)\cap{E}\not=\varnothing
\end{equation}
\end{definition}

\begin{definition}{开集}
设$E$是度量空间$X$的一个子集,$E$中所有的点都是$E$的内点,那么这样的集合称为开集.
\begin{equation}
\forall{p\in{E}},\exists{r>0},\text{使得}N_r(p)\subset{E}
\end{equation}
\end{definition}

\begin{theorem}{开集的补集是闭集}

\end{theorem}



\begin{definition}{完备集}

\end{definition}

\begin{definition}{闭包}
\end{definition}

\begin{definition}{相对开集}

\end{definition}