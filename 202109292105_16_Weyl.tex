% Weyl旋量
% keys Weyl|旋量|费米子

洛仑兹群的狄拉克表示是可约的.我们可以构造两个二维表示
\begin{equation}
\psi = \begin{pmatrix}
\psi_L \\
\psi_R
\end{pmatrix}
\end{equation}
$\psi_L$被称为左手的Weyl旋量,$\psi_R$被称为右手的Weyl旋量.在无穷小转动$\mathbf \theta$和boost $\mathbf \beta$下, 它们的变换规则为
\begin{align}
\psi_L \rightarrow (1-i \boldsymbol \theta \cdot \frac{\boldsymbol \sigma}{2} - \boldsymbol \beta \cdot \frac{\boldsymbol \sigma}{2})\psi_L ~, \\
\psi_R \rightarrow (1-i\boldsymbol \theta \cdot \frac{\boldsymbol \sigma}{2}+ \boldsymbol \beta \cdot \frac{\boldsymbol \sigma}{2})\psi_R ~.
\end{align}

