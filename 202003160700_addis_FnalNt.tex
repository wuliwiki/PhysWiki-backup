% 泛函分析笔记
% 泛函分析|数学分析|空间|Banach 空间|希尔伯特空间

参考书: Applied Functional Analysis application to mathematical physics (Zeidler)
% 另参考 OneNote 和 iMessage 中的笔记

\subsection{Banach 空间}

\subsection{1.1 Linear Spaces and Dimension}
\begin{itemize}
\item $\mathbb R$ 和 $\mathbb C$ 分别表示实数域和复数域, $\mathbb K$ 表示二者中的一个

\item $\mathbb K^N$ 表示 $N$ 元 \textbf{tuple} $(\xi_1, \xi_2, \dots, \xi_N)$

\item 某区间上的连续函数 $u:[a, b] \to \mathbb{R}$ 可以表示为 $C[a, b]$

\item $\mathbb{K}$ \textbf{上(over $\mathbb K$)}的矢量空间(linear space 就是 vector space) 表示只能以 $\mathbb K$ 的元乘以某个矢量

\item $C[a, b]$ 是无穷维矢量空间
\end{itemize}

\subsection{1.2 Normed Spaces and Convergence}
\begin{itemize}
\item 用 $\norm{u} > 0$ 表示\textbf{范数(norm)}

\item 定义了范数的空间就叫\textbf{赋范空间(normed space)}, 满足 (1) $\norm{u} \geqslant 0$, (2) $\norm{u} = 0$ iff $u = 0$, (3) $\norm{\alpha u} = \abs{\alpha} \norm{u}$, (4) $\norm{u + v} \leqslant \norm{u} + \norm{v}$

\item 定义两个矢量之间的\textbf{距离(distance)} 为 $\norm{u - v}$

\item 模可以用于定义极限 $\lim_{n\to\infty} u_n = u$ 为 $\lim_{n\to\infty} \norm{u_n - u} = 0$, 即 $u_n$ 收敛到 $u$

\item 上一条中, (1) $u$ 是唯一的, (2) $u_n$ 是有界的(bounded), (3) $\norm{u_n} \to \norm{u}$, (4) $u_n + v_n \to u + v$, (5) $\alpha_n u_n \to \alpha u$

\item 柯西序列(Cauchy sequence): 对任意 $\varepsilon > 0$, 存在 $N$, 当 $n, m \geqslant N$ 就有 $\norm{u_n - u_m} < \varepsilon$

\item 在赋范空间中, 每个收敛序列都是柯西序列
\end{itemize}

\subsection{1.3 Banach Spaces and the Cauchy Convergence Criterion}
\begin{itemize}
\item 赋范空间 $X$ 是 Banach 空间当且仅当每个柯西数列都收敛

\item 在 Banach 空间中, 收敛序列都是柯西序列

\item 空间 $X := C[a, b]$ 是实数 Banach 空间, 定义模长为 $\norm{u} := \max_{a \leqslant x \leqslant b} \abs{u(x)}$. $u_n \to u$ 意味着 $\norm{u_n - u} = \max_{a \leqslant x \leqslant b}  \abs{u_n(x) - u(x)} \to 0$. 也就是 $u_n(x)$ 一致收敛到 $u$

\item 如果柯西序列 $u_n$ 的子序列 $u_{n'} \to u$, 那么 $u_n \to u$

\item 若 $\sum_{j=1}^\infty \norm{u_{j+1} - u_j} < \infty$, 那么 $u_n$ 是柯西序列

\item 集合 $U_\varepsilon (u_0) := \{u \in X: \norm{u - u_0} < \varepsilon\}$ 叫做 $u_0$ 的 $\varepsilon$-\textbf{邻域 (neighborhood)}
\end{itemize}

\subsection{1.4 Open and Closed Sets}
\begin{itemize}
\item $X$ 的子集 $M$ 是\textbf{开集} 当且仅当对任意 $u \in M$ 都存在属于 $M$ 的邻域

\item $X$ 的子集 $M$ 是\textbf{闭集} 当且仅当每个序列的极限都属于 $M$

\item $X$ 的子集 $M$ 是 closed 当且仅当 $X - M$ 是开的
\end{itemize}

\subsection{1.5 Operators}

\begin{itemize}
\item $M$ 和 $Y$ 是集合, $u \in M, v \in Y$ 算符 $A: M \to Y$ 代表映射 $v = Au$, 其中 $M$ 是\textbf{定义域(domain of definition)}, 也记为 $D(A)$. 值域(range) 是 $A(M) := \{v \in Y: v = Au, u \in M\}$, 也记为 $R(A)$

\item $A$ 叫\textbf{满射(surjective)} 当且仅当 $A(M) = Y$, 叫做\textbf{单射(injective)} 当且仅当 $Au = Av$ 意味着 $u = v$, \textbf{双射(bijective)} 如果前两者都符合

\item 如果 $A$ 是 bijective, 存在逆算符 $A^{-1}: Y \to M$, 定义为 $A^{-1} v = u$ 当且仅当 $Au = v$


\item 算符也叫函数

\item $A: M \subseteq X \to Y$ 表示 $A: M \to Y$ 且 $M \subseteq X$, 当 $Y = \mathbb K$, 就把 $A$ 叫做\textbf{泛函(functional)}
\end{itemize}

\subsection{1.9 Continuity}
\begin{itemize}
\item \textbf{Sequentially continuous}: $\lim_{n\to\infty} u_n = u$ ($u\in M$) 意味着 $\lim_{n\to\inf} Au_n = Au$

\item \textbf{continuous operator}: 对任意 $u\in M$, 以及 $\varepsilon>0$, 存在 $\delta >0$ 使得 $\norm{v-u}<\delta$($v\in M$) 意味着 $\norm{Av-Au}<\varepsilon$

\item \textbf{uniformly continuous operator}: 上面的 $\delta$ 不取决于 $u$.

\item \textbf{Lipschitz continuous}: 存在 $L > 0$ 使得 $\norm{Av-Au} \le L\norm{v-u}$ 对所有 $u,v \in M$ 都成立.

\item continuous 和 sequentially continuous 是充要条件

\item 两个连续算符的复合算符 $C =  A \circ B$ 也是连续的

\item 两个几何叫 \textbf{homeomorphic} (注意不是 homomorphic)当且仅当存在一个算符, 满足连续,双射,且逆算符也连续.
\end{itemize}

\subsection{1.10 Convexity}
\begin{itemize}
\item \textbf{凸集(convex set)}: 线性空间中的子集 $M$ 是凸的当且仅当 $u, v\in M$ 且 $0\le\alpha\le1$ 意味着 $\alpha u+(1-\alpha)v\in M$ (两点之间的连线也属于集合)

\item \textbf{凸函数}: 凸集合上的函数 $f:M\to\mathbb R$ 满足 $f(\alpha u + (1-\alpha)v) \le \alpha f(u)+(1-\alpha)f(v)$ 对所有 $u, v\in M$ 和 $0\le\alpha\le1$ 都成立.

\item 赋范空间中的函数 $f(u) := \norm{u}$ 是凸函数.

\item \textbf{linear subspace}, \textbf{closed linear subspace}

\item 显然所有线性子空间都是凸的

\item \textbf{linear hull}: $\opn{span} M$, \textbf{closure}: $\bar M$, \textbf{convex hull}: $\opn{co} M$

\item $u \in \opn{co} M$: 对固定的 $n = 1, 2,\dots$, $u = \alpha_1u_1 + \dots + \alpha_n u_n$, 其中 $u_1,\dots,u_n\in M$, $0\le\alpha_1,\dots,\alpha_n\le1$, $\alpha_1+\dots+\alpha_n = 1$
\end{itemize}


\subsection{1.11 Compactness}
\begin{itemize}
\item 如果赋范空间的集合 $M$ 满足每个序列都有收敛的子序列, 那么 $M$ 就是\textbf{相对紧的(relatively compact)}

\item 如果赋范空间的集合 $M$ 满足每个序列都有收敛的子序列且收敛到 $M$ 中, 那么 $M$ 就是\textbf{紧的(compact)}

\item 如果存在 $r \geqslant 0$ 使任意 $u \in M$ 都有 $\norm{u} \leqslant r$, 那么 $M$ 就是\textbf{有界的(bounded)}

\item $M$ 是紧的当且仅当它是相对紧的且闭合

\item 每个紧集都是有界的

\item $\mathbb K^N$ 上的子集若使用 $\norm{u} := \abs{u}_\infty$, 那么它是相对紧的当且仅当它是有界的

\item \textbf{Arzela-Ascoli theorem}: 令 $X := C[a, b]$, 且 $\norm{u} := \max_{a\leqslant x\leqslant b}\abs{u(x)}$. 那么若 $M \subseteq X$ 有界且一致连续, 那么 $M$ 就是相对紧的

\item Weierstrass 定理: 令 $f: M\to \mathbb R$ 为赋范空间中非空紧子集 $M$ 上的连续函数, 那么 $f$ 在 $M$ 上存在一个最大值和最小值

\item 令 $X, Y$ 为 $\mathbb K$ 上的赋范空间, 令 $A: M \subseteq X \to Y$ 中 $M$ 为非空紧集, $A$ 为连续算符. 那么 $A$ 是一致连续的.

\item \textbf{finite $\epsilon$ net} 对任意 $\epsilon > 0$, 存在有限多个点 $v_1, \dots, v_J \in M$ 

\item $A: M \subseteq X \to Y$ 称为\textbf{紧的(compact)} 当且仅当 $A$ 是连续的且 $A$ 将有界集变换到相对紧集.

\item 算符 $(Au)(x) := \int_a^b F(x, y, u(y)) \dd{y}$ 是紧的, 其中 $-\infty < a < b < \infty$, $u(x)$ 有界.
\end{itemize}

\subsection{1.13 The Minkowski Functional and Homeomorphisms}
\begin{itemize}
\item 赋范空间 $X$ 中的两个范数叫做 \textbf{equivalent} 当且仅当存在正数 $\alpha$ 和 $\beta$ 使得 $\alpha \norm{u} \leqslant \norm{u}_1 \leqslant \beta \norm{u}$ 对所有的 $u \in X$ 成立

\item 有限维空间的任意两个范数都是 equivalent 的

\item 每个有限维赋范空间都是 Banach space.

\item 如果 $u_0, \dots, u_N$  满足 $u_1 - u_0, u_2 - u_0,\dots, u_N - u_0$ 线性无关, 那么就说它们在\textbf{一般位置(general position)}. 这个性质与 $u_0, \dots, u_N$ 的顺序无关.

\item $N$-单纯形($N$-simplex) 是 $\mathcal S := \opn{co}\qty{u_0, \dots, u_N}$ ($\opn{co}$ is convex hull), 其中 $u_0, \dots, u_N$ 在一般位置. $0$-单纯形是一个点.

\item 单纯形的 \textbf{barycenter} 是 $b := \sum_{j=0}^N u_j / (N+1)$

\item $\mathcal S$ 的 $k$-face 是单纯形中 $k+1$ 个不同 vertices 的 convex hull

\item $\opn{diam} M := \sup_{u, v\in M} \norm{u - v}$ is the \textbf{diameter} of $M$, and $\opn{dist}(u, M) := \inf_{w \in M} \norm{u - w}$ is the \textbf{distance} of the point $u$ from the point $M$.
\end{itemize}

\subsection{1.14 The Brouwer Fixed-Point Theorem}

\begin{itemize}
\item 当 $M$ 是有限维赋范空间中紧的, 凸的, 非空的子集, 连续算符 $A: M \to N$ 拥有 \textbf{fixed point}
\end{itemize}

\subsection{1.15 The Schauder Fixed-Point Theorem}
\begin{itemize}
\item 紧算符 $A: M \to M$ 有一个 fixed point 如果 $M$ 是 Banach 空间的一个有界的, 闭得, 突的, 非空的子集.
\end{itemize}

\subsection{1.20 Linear Operators}
\begin{itemize}
\item \textbf{线性算符} $A: L \subseteq X\to Y$ 是线性的当且仅当 $A (\alpha u + \beta v) = \alpha Au + \beta Av$

\item 用 $L(X, Y)$ 表示线性连续算符 $A: X \to Y$, $X$ 是 $\mathbb K$ 上的赋范空间, $Y$ 是 $\mathbb K$ 上的 Banach 空间. $L(X, Y)$ 是 $\mathbb K$ 上的 Banach 空间, 范数就是算符的范数

\item 有限维矢量空间中的线性算符可以表示为矩阵, 所有这些算符都是连续的

\item 算符的\textbf{零空间(null space)} 为 $N(A) := \qty{u \in X: Au = 0}$

\item 线性算符 $A$ 是连续的, 当且仅当存在 $c > 0$ 使 $\norm{Au} \leqslant c\norm{u}$ 对所有 $u$ 都成立

\item 线性算符是单射的当且仅当 $N(A) = {0}$

\item 定义线性连续算符 $A: X \to Y$ 的\textbf{算符范数(operator norm)} 为 $\norm{A} := \sup_{\norm{v} \leqslant 1} \norm{A v}$

\item 当 $X \ne {0}$, 有 $\norm{A} := \sup_{\norm{v} = 1} \norm{A v}$
\end{itemize}

\subsection{1.21 The Dual Space}

\begin{itemize}
\item 令 $X$ 为 $\mathbb K$ 上的一个赋范空间, 一个线性的连续算符 $f: X \to \mathbb K$ 称为\textbf{线性连续泛函(linear continuous functional)}

\item 所有 $X$ 上的线性连续泛函叫做 $X$ 的\textbf{对偶空间(dual space)} $X^*$, $X^* = L(X, \mathbb K)$

\item $f \in X^*$ 作用在 $u \in X$ 上可以记为 $\ev{f, u} := f(u)$

\item $f\in X^*$ 的范数为 $\norm{f} := \sup_{\norm{v} \leqslant 1} \abs{f(v)}$, 所以 $\abs{f(u)} \leqslant \norm{f}\norm{u}$

\item 令 $X$ 为 $\mathbb K$ 上的赋范空间, 那么对偶空间 $X^*$ 使用上述范数就是 $\mathbb K$ 上的 Banach 空间.
\end{itemize}

\subsection{1.23 Banach Algebras and Operator Functions}
\begin{itemize}
\item By a \textbf{Banach algebra} $\mathcal B$ over $\mathbb K$ we understand a Banach space over $\mathbb K$, where an additional multiplication $AB$ is defined such that $AB \in \mathcal B$ for all $A, B \in \mathcal B$. More over, for $A, B, C \in \mathcal B$ and $\alpha \in \mathbb K$, $(AB)C = A(BC)$, $A(B+C) = AB + AC$, $(B+C)A = BA + CA$, $\alpha(AB) = (\alpha A)B = A(\alpha B)$, $\norm{AB} \leqslant \norm{A}\norm{B}$. Exist $E \in \mathcal B$ such that $AE = EA$ for all $A \in \mathcal B$ and $\norm{E} = 1$

\item define \textbf{operator function} through $F(A) := \sum_{j=0}^\infty a_j A^j$, and $F(z) = \sum_{j=0}^\infty a_j z^j, z\in \mathbb K$

\item Let $X$ be a Banach space over $\mathbb K$. For each $A \in L(X, X)$ with $\norm{A} < r$, $F(A) \in L(X, X)$
\end{itemize}

\subsection{1.25 Application to the Spectrum}
\begin{itemize}
\item 考虑 $Au = \lambda u, u \in X, \lambda \in \mathbb C$

\item 令 $A\in L(X,X)$, $X$ 是非空的复 Banach 空间.  $\lambda$ 是\textbf{本征值(eigen value)} 当 $u\ne 0$.

\item \textbf{预解集(resolvent set)} $\rho(A)$: 使得 $(A-\lambda I)^{-1}: X \to X$ 存在且 $\in L(X,X)$. $(A-\lambda I)$ 叫做 resolvent.

\item 预解集中的 $\lambda$ 带入本征方程只能解得 0 矢量.

\item $\sigma(A) := \mathbb C - \rho(A)$ 叫做\textbf{谱 (spectrum)}

\item 谱是 $\mathbb C$ 中的紧子集且 $\abs{\lambda} \leqslant \norm{A}$ 对所有 $\lambda\in\sigma(A)$ 成立

\item 每个本征值都属于谱

\item 预解集 $\rho(A)$ 是一个开集

\item Banach 空间 $X$ 上的一个算符 $B: X\to X$ 如果值域 $R(B)$ 是闭的且零空间是有限维的, 那么他就是 \textbf{semi-Fredholm} 的

\item \textbf{本质谱(essential spectrum)} $\sigma_e(A)$ 包括所有使得 $(A - \lambda I)$ 不是 semi-Fredholm 的 $\lambda$
\item $\sigma_e(A) \subseteq \sigma(A)$
\item $\sigma_e(A)$ 就是所有具有无穷\textbf{简并(degeneracy)}的本征值的集合
\item 如果 $X$ 是有限维的, 那么 $A$ 的本质谱是空的 
\end{itemize}

\subsection{1.26 Density and Approximation}
\begin{itemize}
\item $M \subseteq X$ 被称为在 $X$ 中\textbf{稠密的(dense)} 当且仅当 $\bar M = X$, 其中 $\bar M$ 是 $M$ 的闭包.

\item \textbf{可数的(countable)}, \textbf{至多可数(at most countable)}

\item $X$ 叫\textbf{可分的(separable)} 的当且仅当它存在至多可数的稠密子集 $M \subseteq X$

\item \textbf{Weierstrass 近似理论}: $X := C[a, b]$, $-\infty < a < b < \infty$. 所有实系数多项式的集合在 $X$ 内稠密

\item $C[a, b]$ 是可分的.

\item 任何有限维赋范空间都是可分的.

\item 令 $X$ 为可分的赋范空间. 存在一个序列 $\qty{X_n}$ ($X_n$ 是 $X$ 的有限维线性子空间), 使得 $X_1 \subseteq X_2 \subseteq \dots \subseteq X$ 以及 $\bigcup_{n=1}^\infty X_n = X$

\end{itemize}

\subsection{2.1 Hilbert Spaces}
\begin{itemize}
\item 内积\upref{InerPd} 记为 $(u|v)$, $(u|v) \in K$

\item \textbf{pre-Hilbert 空间}  就是定义了内积的线性空间

\item \textbf{柯西—施瓦兹不等式}\upref{CSNeq} 是 (pre-) Hilbert 空间中最重要的性质.

\item pre-Hilbert 空间都是赋范空间, 范数为 $\norm{u} := \sqrt{(u|u)}$

\item 希尔伯特空间定义: 1. 是一个内积空间, 2. 是一个 Banach 空间(或者任意柯西序列的极限都属于它本身)

\end{itemize}

\subsection{2.2 Standard Examples}

\begin{itemize}
\item 空间 $X := \mathbb K^N$ 是一个希尔伯特空间, 内积为 $(x|y) := \sum_j \bar \xi_j \eta_j$, 范数为 $\norm{x} = (x|x)^{1/2}$

\item 令 $-\infty \leqslant a < b \leqslant \infty$, 令 $L_2(a, b)$ 为所有 measurable 函数 $u :]a, b[ \to \mathbb R$(其中 $]a, b[$ 表示开区间) $\qty{x \in R : a < x < b}$, 满足 $\int_a^b \abs{u}^2 \dd{x} < \infty$. 那么 $L_2(a, b)$ 是无穷维的实希尔伯特空间, 内积为 $(u, v) := \int_a^b uv \dd{x}$.

\item $L_2(a, b)$ 的 identification principle: 两个函数 $u$ 和 $v$ 是同一个元素当且仅当 $u(x) = v(x)$ 对几乎所有 $x \in ]a, b[$ 成立.

\item 令 $G$ 表示 $\mathbb R^N$ ($N \geqslant 1$)中的可测度非空子集, $L_2^{\mathbb K}(G)$ 表示可测度函数 $u: G \to \mathbb K$ 的集合, 满足 $\int_G \abs{u}^2 \dd{x} < \infty$. 那么 $L_2^{\mathbb K}(G)$ 是一个希尔伯特空间, 内积定义为 $(u|v) := \int_G \bar u v \dd{x}$

\item 请写出 $L_2^{\mathbb K}(G)$ 中的\textbf{施瓦兹(Schwarz)不等式}

\item 令 $G$ 为 $\mathbb R^N$ 中的非空开子集($N > 1$). 那么 $C^k(G)$ 表示 $k$ 阶连续可偏导的函数 $u: G \to R$ 的集合.

\item $C^k(\bar G)$ 包含 $C^k$ 中所有满足各阶偏导数能拓展到 $G$ 的闭包 $\bar G$ 上的函数.

\item 如果 $u \in C^k(G)$ 对所有的 $k = 0, 1, \dots$ 都成立, 那么我们记 $u \in C^\infty(G)$. 同理可以定义 $C^\infty(\bar G)$

\item $C_0^\infty (G)$ 是所有 $C^\infty(G)$ 中的函数, 满足在 $G$ 的紧子集 $C$ 恒外为零.

\item 令 $G$ 为 $\mathbb R^N$ 中的一个非空开集, $N \geqslant 1$. 那么 (i) $C_0^\infty(G)$ 和 $C(\bar G)$ 在 $L_2(G)$ 中稠密.

\item $C_0^\infty(G)_{\mathbb C}$ 和 $C(\bar G)_{\mathbb C}$  在 $L_2^{\mathbb C}(G)$ 中稠密

\item 分部积分的高阶拓展: $\int_G (\partial_j u) v\dd{x} = \int_{\partial G} uvn_j \dd{O} - \int_G u \partial_j v\dd{x}$. 其中 $x = (\xi_1, \dots, \xi_N)$, $\partial_j u := \partial u/\partial\xi_j$, $n_j$ 是边界 $\partial G$ 的法向量 $n = (n_1, \dots, n_N)$. $\int\dd{O}$ 代表边界上的积分, 二维情况下 $\int\dd{O}$ 是逆时针的环积分. (推导可以参考 “浮力\upref{Buoy}”). 当 $u$ 或 $v$ 在 $\partial G$ 上为零时, 边界积分为零.

\item 上一条的分部积分公式对所有 $u, v \in C^1(\bar G)$ 成立, $G$ 是 $\mathbb R^N$ 中的有界非空开集, 边界足够光滑. 没有边界积分的分部积分对所有 $u \in C^1(G)$ 和 $v \in C_0^\infty(G)$ 成立. $G$ 是 $\mathbb R^N$ 中的一个非空开集.

\item 微积分基本定理 $\int_a^b w' \dd{x} = \eval{w}_a^b$ 在高维中拓展为\textbf{高斯定理(Gauss theorem)} $\int_G \partial_j w \dd{x} = \int_{\partial G} wn_j\dd{O}$. 令 $w = uv$, 可得分部积分.
\end{itemize}

\subsection{2.3 Bilinear Forms}
\begin{itemize}
\item 赋范空间 $X$ 上的 \textbf{bounded bilinear form} 是一个函数 $a: X\times X\to\mathbb K$ 且具有性质 (1) 双线性(bilinear): 对所有 $u, v, w\in X$ 以及 $\alpha, \beta \in \mathbb K$, $a(\alpha u + \beta v, w) = \alpha a(u, w) + \beta a(v, w)$ 以及 $a(w, \alpha u + \beta v) = \alpha a(w, u) + \beta a(w, v)$, (2) \textbf{有界性(Boundedness)} 存在常数 $d > 0$ 使得 $\abs{a(u, v)} \le d\norm{u}\norm{v}$ 对所有 $u, v\in X$ 成立

\item 双线性函数 $a(\cdot,\cdot)$ 叫做对称的当且仅当 $a(u, v) = a(v, u)$ 对任意 $u, v\in X$ 成立, 叫做 \textbf{positive} 当且仅当 $0\le a(u,u)$ 对所有 $u\in X$ 成立. 叫做 \textbf{strongly positive} 当且仅当存在常数 $c > 0$ 使 $c\norm{u}^2 \le a(u, u)$ 对所有 $u\in X$ 成立.
\end{itemize}

\subsection{2.4 The Main Theorem on Quadratic Variational Problems}
\begin{itemize}
\item 令 $a: X\times X\to\mathbb R$ 是实希尔伯特空间 $X$ 上的 symmetric, bounded, strongly positive, bilinear form, $b: X\to\mathbb R$ 是一个 $X$ 上的线性连续函数. 那么 variational problem $a(u, u)/2 - b(u) = \min!$ ($u\in X$) 有唯一解, 且该式等效于 variational equation $a(u, v) = b(v)$ 对所有 $v\in X$ 成立. (想一想哈密顿原理\upref{HamPrn}如何推出欧拉—拉格朗日方程\upref{Lagrng})
\end{itemize}

\subsection{2.5 The Functional Analytic Justification of the Dirichlet Principle}
\begin{itemize}
\item \textbf{广义导数}: 分部积分 $\int_G u\partial_j v\dd{x} = -\int_G (\partial_j u)v\dd{x}$ 对所有 $v \in C_0^\infty(G)$ 和 $u \in C^1(G)$ 成立. 若 $w, u\in L_2(G)$ 能使所有 $\int_G u\partial_j v\dd{x} = -\int_G wv\dd{x}$ 对所有 $v \in C_0^\infty(G)$ 成立, 那么 $w$ 就是 $u$ 的广义导数($G$ 是 $\mathbb R^N$ 中的非空开集). 同样记 $w = \partial_j u$

\item 广义导数 $w = \partial_j u$ 能在一个 $N$ 维零测度集外被唯一确定.

\item \textbf{Sobolev 空间 $W_2^1(G)$}: $G$ 是 $\mathbb R^N$ 中的非空开集, $u, \partial_ju \in L_2(G)$. 定义内积为 $(u|v)_{1,2} := \int_G(uv + \sum_j \partial_j u\partial_j v) \dd{x}$

\item $W_2^1(G)$ 是一个希尔伯特空间, 如果我们认为两个在大多数地方相等的函数是同一个函数.

\item 定义 $\mathring {W_2^1}(G)$ 为 $C_0^\infty(G)$ 在 $W_2^1(G)$ 上的闭包

\item $\mathring {W_2^1}(G)$ 是 $W_2^1(G)$ 的(实的)子希尔伯特空间.

\item 从广义的角度理解, $\mathring {W_2^1}(G)$ 中函数的边界值为 0.
\end{itemize}

\subsection{2.8 Generalized Functions and Linear Functionals}
\begin{itemize}
\item \textbf{多重指标(multiindex)}: $\alpha = (\alpha_1, \dots, \alpha_N)$, 令 $\abs{\alpha} := \alpha_1 + \dots + \alpha_N$, 以及 $\partial^\alpha u := \partial_1^{\alpha_1} \dots \partial_N^{\alpha N} u = \partial^{\abs{\alpha}} u/ (\partial\xi_1^{\alpha_1} \dots \partial\xi_N^{\alpha N})$

\item 分部积分: $G$ 为 $\mathbb R^N \ge 1$ 上的所有非空开区间. 对所有 $u, v \in C_0^\infty(G)$ 以及所有多重指标 $\alpha$, $\int_G u\partial^\alpha v \dd{x} = (-1)^{\abs{\alpha}} \int_G (\partial^\alpha u)v \dd{x}$ (重复使用分部积分即可证明)

\item 令 $\mathcal D(G) := C_0^\infty(G)$. 令 $\phi_n, \phi \in \mathcal D(G)$. $\phi_n \to \phi$ 的定义是: 对所有的多重指标 $\alpha$, $K$ 上有一致收敛\upref{UniCnv} $\partial^\alpha \phi_n \to \partial^\alpha \phi(x)$.

\item 如果 $G = ]\alpha,\beta[$, 那么 $\mathcal D(\alpha,\beta) := \mathcal D(G)$

\item \textbf{广义函数(generalized function)} $U \in \mathcal D'(G)$ 定义: 线性, sequentially continuous 泛函 $U: \mathcal D(G) \to \mathbb R$. 广义函数也叫\textbf{分布(distribution)}

\item $L_2(G) \subseteq \mathcal D'(G)$ 的意思是每个 $u\in L_2(G)$ 都对应(identified with)一个广义函数 $U(\phi) := \int_G u(x) \phi(x) \dd{x}$ 对所有 $\phi \in \mathcal D(G)$ 成立. $U \in \mathcal D'(G)$. 如果 $u = v$, 那么 $U = V$.

\item \textbf{狄拉克 $\delta_y$ 分部}: $\delta_y(\phi) := \phi(y)$ 对所有 $\phi\in\mathcal D(G)$ 成立.

\item 定义广义函数 $U \in \mathcal D'(G)$ 的导数 $\partial^\alpha U$ 为 $(\partial^\alpha U)(\phi) := (-1)^{\abs{\alpha}} U(\partial^\alpha \phi)$ 对所有 $\phi\in\mathcal D(G)$ 成立.

\item 如果 $u \in\mathcal D(G)$ 对应的广义函数为 $U$, $\partial^\alpha u$ 对应的广义函数为 $V$, 那么 $V = \partial^\alpha U$

\item 如果 $U \in\mathcal D'(G)$, 那么 $\partial^\alpha U \in\mathcal D'(G)$ 对所有的 $\alpha$ 成立.

\item 广义函数存在任意阶导数.

\item 广义函数的极限 $U_n \to U$ 的定义: $U_n(\phi)\to U(\phi)$ 对所有 $\phi \in\mathcal D(G)$ 都成立.

\item 在 $L_2(G)$ 中 $u_n\to u$ 意味着对应的 $U_n \to U$.

\item 在 $\mathcal D(\alpha,\beta)$ 中, $f_{y,\epsilon}$ 对应的泛函 $F_{y, \epsilon}\to \delta_y$ 当 $\epsilon \to +0$. $f_{y,\epsilon}$ 是区间 $[y-\epsilon,y+\epsilon]$ 外为零, 积分为 1 的函数.

\item 在 $\mathcal D(\alpha,\beta)$ 中, 方程 $-U'' = \delta_y$ 的解 $U$ 对应的就是格林函数, $U(\phi) = \int \mathcal G(x, y) \phi(x) \dd{x}$

\item 令 $u, w \in L_2(G)$,广义导数 $w = \partial^\alpha u$ 的定义为: 对应的广义函数满足 $W = \partial^\alpha U$. 这比之前定义的广义导数更一般化.

\item 广义导数(除了零测度集)是唯一确定的.
\end{itemize}

\subsection{2.9 Orthogonal Projection}
\begin{itemize}
\item \textbf{orthogonal complement}: $M^\bot := {w \in X: (w|v) = 0\ \ \forall\ \ v\in M}$

\item \textbf{perpendicular principle}: 令 $M$ 为希尔伯特空间 $X$ 上闭合的线性子空间. 对于给定的 $u\in X$, $\norm{u - v} = \min!$ 存在唯一的解 $v$, 且 $u - v\in M^\bot$

\item \textbf{orthogonality decomposition}:如果给定 $u \in M$, 要求 $w \in M^\bot$, 那么 $u = v + w$ 是唯一的.

\item 
\end{itemize}

\subsection{2.10 Linear Functionals and the Riesz Theorem}
\begin{itemize}
\item \textbf{Riesz teorem}: 令 $X$ 为 $\mathbb K$ 上的希尔伯特空间, 令 $X^*$ 为 $X$ 的对偶空间. 那么 $f\in X^*$ 当且仅当存在 $v\in X$ 使得 $f(u) = (v|u)$ 对所有 $u\in X$. $v$ 可以由 $f$ 唯一确定, 且 $\norm{f} = \norm{v}$

\item 如果 $f$ 是 Hilbert 空间中的非零线性连续函数, 那么它的零空间 $N(f)$ 是一个闭合平面且 orthogonal complement $N(f)^\bot$ 是一维的.
\end{itemize}

\subsection{2.11 The Duality Map}
\begin{itemize}
\item \textbf{duality map} $J: X\to X^*$ 把 $v\in X$ 映射到 $f(u) = (v|u)$ (对所有 $u\in X$).

\item 定义 $\ev{f, u} = f(u)$($f\in X^*, u\in X$), 那么 $\ev{J(v), u} := (v|u)$ (对所有 $u, v\in X$)

\item 对偶映射 $J$ 是双射的, 连续的以及 norm preserving. 即 $\norm{J(u)} = \norm{u}$ 对所有 $u\in X$

\item 如果 $X$ 是实 Hilbert 空间, 那么 $J$ 是线性的. 如果 $X$ 是复 Hilbert 空间, 那么 $J$ 是反线性的, 即 $J(\alpha v + \beta w) = \bar \alpha Ju + \bar \beta Jw$ (对所有 $\alpha,\beta\in\mathbb C, u, w\in X$)
\end{itemize}

\subsection{2.13 The Linear Orthogonality Princple}
\begin{itemize}
\item 以下三个条件互相等价: (1) 二次最小值问题的 existence principle (2) 垂直定理 (3) Riesz 定理
\end{itemize}

\subsection{2.14 Nonlinear Monotone Operators}
\begin{itemize}
\item 实 Hilbert 空间 $X$ 上的\textbf{强单调(strongly monotone)} 算符(注意不一定是线性的!) $A:X\to X$ 定义为: 存在常数 $c > 0$ 使得 $(u-v|Au - Av) \ge c\norm{u-v}^2$ 对所有 $u, v\in X$ 成立.

\item 对任意 $z \in X$ 以及强单调, Lipschitz 连续的算符 $A$, $Au = z$ 存在唯一解 $u\in X$.
\end{itemize}

\subsection{Problems}
\begin{itemize}
\item \textbf{special tensor product}: 经典分析中, 两个函数的张量积定义为 $(\phi\otimes\psi)(x,y) := \phi(x)\psi(y)$. 令广义函数 $U, \delta \in \mathcal D'(\mathbb R)$. 定义 $(U\otimes V)(\chi) = U(V(\chi))$
\end{itemize}

\subsection{3.1 Orthonormal Series}
\begin{itemize}
\item \textbf{正交归一系(orthonormal system)}: $(u_k|u_m) = \delta_{k,m}$ 对所有 $k, m$ 成立.

\item 令 $X$ 为实 Hilbert 空间, $\{u_0, u_1,\dots\}$ 为 $X$ 中至多可数的正交归一系. 对有限正交归一系, 如果 $u = \sum_{n=0}^N(u_n|u)u_n$ 对所有 $u\in X$ 成立, 它就叫做\textbf{完备的(complete)}. 对可数的正交归一系, 令 $s_m := \sum_{n=0}^m(u_n|u)u_n$, 如果 $u = \lim_{m\to\infty} s_m$ 对所有 $u\in X$ 成立, 它就叫做\textbf{完备的(complete)}.

\item 有限的正交归一系在 Hilbert 空间 $X$ 中是完备的当且仅当它是 $X$ 的一组基底.

\item 令 $\{u_n\}$ 为 Hilbert 空间中可数的正交归一系. 如果无穷级数 $u = \sum_{n=0}^\infty c_n u_n$ 对某个 $u\in X$ 收敛, 那么 $c_n = (u_n|u)$ 对所有 $n$ 成立.

\item \textbf{Bessel inequality}: $\sum_{n=0}^m \abs{(u_n|u)}^2 \le \abs{u}^2$ 对所有 $u \in X$ 和 $m$ 成立.

\item \textbf{Convergence criterion}: 令 $\{u_n\}$ 为 Hilbert 空间 $X$ 中的可数正交归一系. 那么 $\sum_{n=0}^\infty c_nu_n$ 收敛当且仅当 $\sum_{n=0}^\infty \abs{c_n}^2$ 收敛.

\item 令 $\qty{u_n}$ 为可数正交归一系, 那么以下两个条件等效: (1) 它在 $X$ 中是完备的. (2) 它的 linear hull (span) 在 $X$ 中是稠密的.

\item 对 Hilbert 空间 $X$ 中可数完备的正交归一系: (1) \textbf{Parseval equation}: $\qty{u_n}$ (1) $(u|v) = \sum_{n=0}^\infty \bar c_n(u) c_n(v)$, (2) $\norm{u}^2 = \sum_{n=0}^\infty \abs{(u_n|u)}^2$, (3) 如果 $(u_n|u) = 0$ 对所有 $n$ 和固定的 $u\in X$ 成立, 那么 $u=0$.

\item 对每个 $u\in L_2(-\pi,\pi)$, 傅里叶级数收敛. 即 $\lim_{m\to\infty}\int_{-\pi}^\pi [u(x)-a_0/2-\sum_k a_k\cos kx + b_k \sin kx]^2 \dd{x} = 0$
\end{itemize}

\subsection{3.5 Unitary Operators}
\begin{itemize}
\item 令 $X$ 和 $Y$ 为 $\mathbb K$ 上的希尔伯特空间. 算符 $U: X\to Y$ 叫做 \textbf{unitary} 当且仅当 $U$ 是线性的, 满射的, 且 $(Uv|Uw) = (v|w)$ 对所有 $v, w \in X$ 成立.

\item 如果算符 $U$ 是 unitary 的, 那么它就是双射的, 连续的, 且 $\norm{Uv} = \norm{v}$ 对所有 $v\in X$ 成立. 而且, 存在逆算符 $U^{-1}: Y\to X$, 同样是 unitary 的.
\end{itemize}

\subsection{3.6 The Extension Principle}
\begin{itemize}
\item 对 Banach 空间 $X$ 和 $Y$, 令线性算符 $A: D\subseteq X\to Y$ 的范数为有限值, $D \subseteq X$ 是稠密线性的. 那么 (1) $A$ 可以唯一地拓展(extended)到 $A:X\to Y$ 上, 范数仍然为有限. (2)如果 $A$ 在 $D$ 上的进算符, 那么拓展后也是.
\end{itemize}

\subsection{3.7 Applications to the Fourier Transformation}
\begin{itemize}
\item $\mathcal S$ 空间: 包含所有 $C^\infty$ 函数 $u: \mathbb R \to \mathbb C$, 满足 $\norm{u}_{p,q} < \infty$ 对所有 $p, q=0,1,\dots$ 成立. 其中 $\norm{u}_{p,q} := \sup_{x\in\mathbb R} (1+\abs{x}^p) \sum_{n=0}^q\abs{u^{(n)}(x)}$. 极限 $u_n \to u$ 以 $\norm{u}_{p,q}$ 为准.

\item $\mathcal S$ 空间中的函数叫做\textbf{在无穷远处 rapidly decreasing}. 且满足 $\abs{\int_{-\infty}^\infty u^{(n)}(x) \dd{x}} < \infty$ 对所有 $n = 0, 1,\dots$ 成立. 且分部积分的边界项为零.

\item 傅里叶变换 $F:\mathcal S\to\mathcal S$ 是线性的,双射的,sequentially continuous. 反傅里叶变换同样是 sequentially continuous.

\item 傅里叶变换 $F:\mathcal S\to\mathcal S$ 可以唯一地拓展到酋算符 $F: L_2^{\mathbb C}(\mathbb R) \to L_2^{\mathbb C}(\mathbb R)$. 注意 $C_0^\infty(\mathbb R)_{\mathbb C} \subseteq \mathcal S \subseteq L_2^{\mathbb C}(\mathbb R)$
\end{itemize}

\subsection{3.8 The Fourier Transform of Tempered General Functions}
\begin{itemize}
\item $\mathcal S'$ 定义为所有线性, sequentially continuous 映射 $T: \mathcal S\to \mathbb C$. $T$ 叫做 \textbf{tempered generalized functions} 或者 \textbf{tempered distributions}

\item 定义 $(FT)(u) := T(Fu)$ ($u \in \mathcal S$).

\item 算符 $F : \mathcal S' \to \mathcal S'$ 是线性且双射的.

\item 令 $v : \mathbb R\to\mathbb C$ 为可测度的有界函数. 算符 $T(u) = \int_{-\infty}^\infty v(x)u(x) \dd{x}$ 对所有 $u \in \mathcal S$. 那么 $T \in \mathcal S'$

\item 令 $y \in \mathbb R$, 定义 \textbf{tempered delta distribution} $\delta_y$: $\delta_y(u) := u(y)$ (对所有 $u \in \mathcal S$). 那么 (1) $\delta_y \in \mathcal S'$, (2) $F\delta_y = (2\pi)^{-1/2} “e^{-iky}”$, (3) $(2\pi)^{-1/2} F^{-1} (“e^{-iky}”) = \delta_y$. 其中定义 $“e^{-iky}”(u) := \int_{-\infty}^\infty e^{-iky} u(k) \dd{k}$ 对所有 $u\in\mathcal S$
\end{itemize}

\subsection{Problems}
\begin{itemize}
\item \textbf{Tensor Product} $X\otimes Y$.
\end{itemize}

\subsection{4.1 Symmetric Operators}
\begin{itemize}
\item Hilbert 空间 $X$ 上的线性算符 $A:D(A)\subseteq X\to X$ 是对称的当且仅当 $D(A)$ 在 $X$ 中稠密且 $(Au|u) = (u|Av)$ 对所有 $u,v\in D(A)$ 成立.

\item Hilbert 空间 $X$ 上的线性对称算符 $A:D(A)\subseteq X\to X$ 满足: (1) $(Au|u)$ 是实数, (2) 所有本征值为实数, (3) 不同本征值对应的本征矢正交, (4) 本征矢构成的至多可数的完备正交归一系对应的本征值包含 $A$ 的所有本征值.
\end{itemize}


\subsection{4.2 The Hilbert-Schmidt Theory}
\begin{itemize}
\item 令 $A:X\to X$ 为可分 Hilbert 空间中的一个非空线性对称紧算符, 那么 (1) $A$ 存在本正矢构成的完备正交归一系

\item 所有本征值 $\lambda$ 为实数, 且每个 $\lambda\ne0$ 存在有限简并

\item 不同本征值对应的本征矢正交

\item 如果 $A$ 有可数个本征值(注意 $\lambda=0$ 不是本征值), 那么本征值序列 $\lambda_n \to 0$
\end{itemize}


\subsection{5.1 Extensions and Embeddings}
\begin{itemize}
\item 线性空间 $X$ 到 $Y$ 的算符 $A$ 和 $B$ 记为 $B \subseteq A$ 当且仅当定义域 $D(B) \subseteq D(A)$ 且它们在 $D(B)$ 上是同一算符. 这时 $A$ 是 $B$ 的 \textbf{extension}

\item $A = B$ 当且仅当 $A\subseteq B$ 且 $B\subseteq A$

\item \textbf{embedding} $X \subseteq Y$ 是\textbf{连续的}当且仅当存他们之间存在线性,单射,连续的算符

\item \textbf{embedding} $X \subseteq Y$ 是\textbf{紧的}当且仅当存他们之间存在线性,单射, 紧的算符
\end{itemize}


\subsection{5.2 Self-Adjoint Operators}
\begin{itemize}
\item 令线性算符 $A: D(A) \subseteq X \to X$ 的定义域 $D(A)$ 在希尔伯特空间 $X$ 上稠密. 定义 $v \in D(A^*)$ 当且仅当存在 $w\in X$ 使 $(v|Au) = (w|u)$ 对任意 $u\in D(A)$ 都成立. 令 $A^*v := w$, 就得到了\textbf{伴随(adjoint)}算符 $A^*: D(A^*) \subseteq X \to X$. 所以 $D(A^*)$ 是满足定义的最大集合.

\item 伴随算符: (1) 是线性的 (2) $(\alpha A)^* = \bar \alpha A^*$ (3) $A \subseteq B$ 意味着 $B^* \subseteq A^*$

\item 如果 $D(A^*)$ 在 $X$ 上稠密, 那么存在 $(A^*)^*$, 记为 $A^{**}$.

\item 希尔伯特空间 $X$ 上的线性算符 $A$ 是\textbf{对称的(symmetric)} 当且仅当 $A \subseteq A^*$, 即 $(Au|v)=(u|Av)$ 对所有 $u, v\in D(A)$ 成立

\item 希尔伯特空间 $X$ 上的线性算符 $A$ 是\textbf{自伴的(self-adjoint)} 当且仅当 $A = A^*$, 注意 $D(A) = D(A^*)$. 自伴算符都是对称算符

\item 希尔伯特空间 $X$ 上的线性算符 $A$ 是 \textbf{skew-symmetric} 当且仅当 $A \subseteq -A^*$

\item 希尔伯特空间 $X$ 上的线性算符 $A$ 是 \textbf{skew-adjoint} 当且仅当 $A = -A^*$

\item 希尔伯特空间 $X$ 上的线性连续算符 $A$ 的伴随算符也是线性连续的, 且 $\norm{A} = \norm{A^*}$ 以及 $A^{**} = A$

\item 令 $f:[a, b]\times[a,b]\to\mathbb R$ 为连续函数且 $-\infty< a < b < \infty$. 定义算符 $A$ 为 $(Au)(x) := \int_a^b f(x, y)u(y)\dd{y}$, 令 $X := L_2(a, b)$. 那么 (1) $A: X\to X$ 是线性紧算符, (2) $(A^*u)(x) = \int_a^b f(y, x) u(y)\dd{y}$, $A^*$ 也是线性紧算符. (3) 如果 $f(x, y) = f(y, x)$ 对任意 $x, y\in[a, b]$ 成立, 那么算符 $A$ 是自伴算符.

\item 希尔伯特空间 $X$ 上的任意的线性自伴算符 $A$ 都是 \textbf{maximally symmetric}. 也就是说, 如果 $S$ 是对称算符且 $A \subseteq S$, 那么 $A = S$

\item 令 $X := L_2^{\mathbb C}(\mathbb R)$, 且 $(Au)(x) := u'(x)$ ($x \in \mathbb R$), $D(A) := {u\in X: u'\in X}$, $u'$ 为广义导数, 那么 (1) 算符 $A$ 是 skew-adjoint 的, (2) $\I A$ 是自伴算符.

\item 令 $B$ 为上一条中的 $A$ 改成 “非广义” 的求导算符, 且定义域为 $X \bigcap C^1(\mathbb R)$, 那么 $B$ 是 skew-symmetric 算符且 $\I B$ 是对称算符

\item 令 $X := L_2^{\mathbb C}(\mathbb R)$, 定义 $(Mu)(x) := xu(x)$ ($\forall x\in R$), $D(M) := {u\in X: Mu\in X}$. 那么算符 $M$ 是自伴算符.

\item 回忆:令 $M$ 为 Hilbert 空间 $X$ 的闭线性子空间, 任何 $u\in X$ 存在唯一的分解 $u = v+w$($v\in M$, $w \in M^\bot$). \textbf{正交投影算符} $P:X\to M$ 定义为 $Pu := v$.

\item 正交投影算符 $P$ 是线性的, 连续的, 自伴的. 且 $P^2 = P$. 如果 $M$ 为非空, 那么 $\norm{P} = 1$

\item 如果 $P:X\to X$ 是一个线性连续的自伴算符且 $P^2 = P$, 那么 $P$ 就是正交投影算符.

\item 令 $A:D(A)\subseteq X\to X$ 为 Hilbert 空间 $X$ 上的线性对称算符, 且 $R(A)$ 在 $X$ 上稠密, 那么 $(A^{-1})^* = (A^*)^{-1}$. 如果这个 $A$ 是自伴算符, 那么 $A^{-1}$ 也是.

\item 对 Hilbert 空间 $X$ 上的线性算符 $U:X\to X$, 以下条件等价. (1) $U$ 是 unitary 算符, (2) $UU^* = U^*U = I$, (3) $U$ 是双射且 $U^{-1} = U^*$, (4) $U$ 是满射且 $\norm{Uv} = \norm{v}$ 对所有 $v\in X$ 成立.
\end{itemize}

\subsection{5.9 Semigroups, One-Parameter Groups, and Teir Physical Relevance}
\begin{itemize}
\item 令 $X$ 为 Banach 空间. $X$ 上的 \textbf{semigroup} $\qty{S(t)}_{t\ge0}$ 包含一系列算符 $S(t):X\to X$ ($\forall t\ge0$) 使得: $S(t+s) = S(t)S(s)$ ($\forall t, s\ge 0$) 以及 $S(0) = I$

\item Semigroup $\qty{S(t)}_{t\ge0}$ 的 \textbf{generator} $A:D(A) \subseteq X\to X$ 定义为 $Au: = \lim_{t\to+0} [S(t)-I]t^{-1} u$, $u\in D(A)$ 当且仅当这个极限存在.

\item Banach 空间 $X$ 上的 \textbf{one-parameter group} $\qty{S(t)}_{t\in\mathbb R}$ 包含一系列算符 $S(t):X\to X$ ($\forall t\in\mathbb R$), 使得: $S(t+s) = S(t)S(s)$ ($\forall t, s\in \mathbb R$) 以及 $S(0) = I$

\item One-parameter group 的 \textbf{generator} $A:D(A) \subseteq X\to X$ 定义为 $Au: = \lim_{t\to0} [S(t)-I]t^{-1} u$, $u\in D(A)$ 当且仅当这个极限存在.

\item 令 $A:X\to X$ 为 Banach 空间 $X$ 上的线性连续算符, 令 $S(t) := e^{tA}$(\forall t\in\mathbb R). 那么 (1) $\mathbb S = \qty{S(t)}$ 是 $X$ 上的线性 one-parameter group, generator 为 $A$

\item $\mathcal S$ 是一直连续的, 所以也是强连续的.

\item 给出 $u_0 \in X$, 令 $u(t) := e^{tA} u_0$ ($\forall t\in\mathbb R$), 那么 $u = u(t)$ 是微分方程 $u'(t) = Au(t)$ ($-\infty<t<\infty$), $u(0) = u_0$ 的唯一解.

\item 如果上面的 $A$ 是线性连续的, 那么上面的微分方程无法用于描述自然中的不可逆过程.

\item \textbf{one-parameter unitary group}: 强连续, one-parameter group, 每个 $S(t)$ 都是 unitary 的.

\item 令 $A$ 为复 Hilbert 空间 $X$ 中的线性连续的自伴算符, 令 $S(t) := e^{iAt}$($\forall t\in\mathbb R$), 那么 $\qty{S(t)}$ 是 one-parameter unitary group, generator 是 $iA$
\end{itemize}

\subsection{5.13 Applications to the Schrodinger Equation}
\begin{itemize}
\item $u'(t) = -iAu(t)$ ($-\infty<t<\infty$), $u(0) = u_0$

\item 令 $A:D(A) \subseteq X\to X$ 为复 Hilbert 空间 $X$ 的自伴算符, 
\end{itemize}


\subsection{5.14 Applications to Quantum Mechanics}
\begin{itemize}
\item Hilbert 空间中单位矢量 $\psi\in X$ ($(\psi|\psi)=1$)叫做\textbf{态(state)}

\item Hilbert 空间中的自伴算符 $A$ 叫做可观测量

\item 测量

\item 不确定原理

\item \textbf{动量算符} $A:D(A)\subseteq X\to X$ 定义为 $(A\phi)(x) := -i\hbar \dv*{\phi(x)}{x}$ ($\forall x\in\mathbb R$). 其中 $D(A) := \qty{\phi\in X: \phi'\in X}$, $\phi'$ 是广义导数.

\item $(B\phi)(x) := x\phi(x)$ ($x\in\mathbb R$) 叫做\textbf{位置算符}, 其中 $D(B) := {\phi\in X: B\phi\in X}$

\item 位置算符和动量算符都是自伴算符(见上文).
\end{itemize}

\subsection{5.15 Generalized Eigenfunctions}
\begin{itemize}
\item 令 $X := L_2^{\mathbb C}(\mathbb R)$, $A:D(A) \subseteq X\to X$ 为对称算符, $\mathcal S\subseteq D(A)$ 那么非零的 tempered distribution $T \in\mathcal S'$ 叫做 $A$ 的 \textbf{广义本征函数(generalized eigenfunction)} (具有实数本征值) 当且仅当 $T(A\phi) = \lambda T(\phi)$ 对所有 $\phi\in\mathcal S$ 成立.

\item 广义本征函数系 $\qty{T_\alpha}_{\alpha\in\mathcal A}$ 成为\textbf{完备的(complete)} 当且仅当 $T_\alpha(\phi) = 0$ 对所有 $\alpha\in\mathcal A$ 和固定的 $\phi\in\mathcal S$ 成立.

\item 令 $T(\phi) := (\psi|\phi) = \int_{-\infty}^\infty \overline{\psi(x)} \phi(x) \dd{x}$ ($\forall \phi\in\mathcal S$), 那么 (1) 对每个 $\psi\in X$, $T\in\mathcal S'$, (2) 映射 $\psi\mapsto T$ 是从 $X$ 到 $\mathcal S'$ 的线性双射算符.

\item 上面 $A$ 中的每个本征函数 $\psi\in D(A)$ 可以映射到一个广义本征函数.

\item 令 $p\in\mathbb R$, $T_p(\phi) := \int_{-\infty}^\infty \overline{\phi_p(x)}\phi(x) \dd{x}$ ($\forall \phi\in\mathcal S$) 其中 $\phi_p(x) := \exp(ipx/\hbar)$. $T_p \in \mathcal S'$.

\item $\qty{\phi_p}_{p\in\mathbb R}$ (对应的广义函数 $T_p(\phi)$)是动量算符的完备广义本征函数.

\item $\qty{\delta_y}_{y\in\mathbb R}$ 是位置算符 $B$ 的完备广义本征函数. 即 $\delta_y(B\phi) = y\delta_y(\phi)$ ($\forall \phi\in\mathcal S$)
\end{itemize}

\subsection{5.20 A Look at Scattering Theory}
\begin{itemize}
\item 实际运动: $\psi(t) := \exp(-itH/\hbar)\psi(0)$, 自由运动: $\psi_0(t) := \exp(-itH_0/\hbar)\psi_0(0)$
\item 运动 $\psi = \psi(t)$ 被称为 \textbf{asymptotically free} 当 $t\to+\infty$

\item $\psi(0)$ 是 asymptotically free motion 的初态当且仅当 $\psi(0)$ 与 $H$ 的所有本征矢(bound states)正交.
\end{itemize}


\subsection{5.21 The Language of Physicists in Quantum Physics and the Justification of the Dirac Calculus}
\begin{itemize}
\item 令 $\phi_k(x) := (2\pi)^{-1/2} e^{ikx}$ $x, k\in\mathbb R$
\end{itemize}
