% 厦门大学 2014 年 考研 量子力学
% license Usr
% type Note

\textbf{声明}:“该内容来源于网络公开资料,不保证真实性,如有侵权请联系管理员”

\subsection{一、}
(1)下列波函数所描述的状态是否为定态?并说明。

①$\phi(x,t)=\varphi(x)e^{\frac{i}{\hbar}Et}+\varphi(x)e^{\frac{-i}{\hbar}Et}$

②$\phi(x,t)=\varphi(x)_1e^{\frac{i}{\hbar}(px-Et)}+\varphi(x)_2e^{\frac{-i}{\hbar}(px+Et)}$

2)已知算符 $A,B,C$ 中,$A$ 和 $B$ 对易,且 $A$ 和 $C$ 也对易,问 $B$ 和 $C$ 是否一定对易?举例说明你的结论。

(3)对于全同粒子体系,什么是全同性原理?描写分别由电子和光子组成的全同粒子体系的波函数的特点。

(4)什么是电子的自旋?电子自旋与轨道角动量有什么不同之处?

(5)写出电磁场中带电粒子的薛定谔方程:它是否具有规范不变性?若有,指出在规范变换下波函数应作何变换?
\subsection{二、}
设质量为 m 的粒子处于下列一维无限深势阱中,
$$V(x,y)=\begin{cases}
0,&0 <  < \\\\
\infty ,& x < 0,  > 
\end{cases}~
$$
已知初始时刻粒子的波函数为$\phi(x,0)=AX(a-X)$