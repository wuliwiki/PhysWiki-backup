% 皮埃尔·德·费马(综述)
% license CCBYSA3
% type Wiki

本文根据 CC-BY-SA 协议转载翻译自维基百科\href{https://en.wikipedia.org/wiki/Pierre_de_Fermat}{相关文章}。

\begin{figure}[ht]
\centering
\includegraphics[width=6cm]{./figures/645e6146de629a4b.png}
\caption{《皮埃尔·德·费马》,17世纪未知作者画作} \label{fig_Pierre_1}
\end{figure}
皮埃尔·德·费马(法语:[pjɛʁ də fɛʁma];1601年8月17日 – 1665年1月12日)是法国数学家,他因早期的研究成果而被认为是微积分学的奠基人之一,其中包括他使用的“等值法”(adequality)。特别地,他因发现了一种原始的方法,用于求解曲线的最大和最小纵坐标,这一方法与当时未知的微分学相类似。此外,他还对数论进行了深入研究,并在解析几何、概率论和光学方面作出了重要贡献。他最著名的贡献包括光传播的费马原理和费马大定理,后者是他在一份《丢番图算术》副本的空白页上所写的一条注释中提出的。他还是法国图卢兹议会的律师。
\subsection{传记}
《皮埃尔·德·费马》,17世纪由罗兰·勒费弗(Rolland Lefebvre)绘画

费马于1601年出生在法国博蒙德·德·洛曼热(Beaumont-de-Lomagne),他出生的那座15世纪末的宅邸现在已成为博物馆。他来自加斯科涅地区,父亲多米尼克·费马(Dominique Fermat)是一位富有的皮革商人,并曾担任博蒙德·德·洛曼热的四位市政官之一,连任三年。母亲是克莱尔·德·隆(Claire de Long)。费马有一个兄弟和两个姐妹,他几乎可以确定是在出生地成长的。[注:此处需引用来源]

费马于1623年进入奥尔良大学,1626年获得民法学士学位后,前往波尔多。在波尔多,他开始了第一次严肃的数学研究,并在1629年将他恢复的阿波罗尼乌斯的《平面位置论》副本送给了当地的一位数学家。显然,在波尔多期间,他与博格朗(Beaugrand)有过接触,并在此期间进行了一些重要的关于极大值和极小值的工作,并将其交给了与他有共同数学兴趣的埃蒂安·德·埃斯帕涅(Étienne d'Espagnet)。在波尔多,费马受到了弗朗索瓦·维埃特(François Viète)学说的深刻影响。[4]

1630年,费马购买了法国图卢兹议会的议员职位,这是法国最高法院之一,并于1631年5月在大法庭宣誓就职。他终身担任此职务。通过这一职位,费马得以将自己的名字从皮埃尔·费马改为皮埃尔·德·费马。1631年6月1日,费马与路易丝·德·隆(Louise de Long)结婚,路易丝是费马母亲克莱尔·德·费马(娘家姓德·隆)的四代堂亲。费马夫妇共有八个孩子,其中五个活到了成年:克莱芒-萨缪尔、让、克莱尔、凯瑟琳和路易丝。[5][6][7]
\begin{figure}[ht]
\centering
\includegraphics[width=6cm]{./figures/4e2f7d3869d3585a.png}
\caption{《皮埃尔·德·费马》,17世纪由罗兰·勒费弗(Rolland Lefebvre)绘画} \label{fig_Pierre_2}
\end{figure}
费马精通六种语言(法语、拉丁语、奥克语、古典希腊语、意大利语和西班牙语),因其在多种语言中创作的诗篇而受到赞誉,他的建议在希腊文献修订方面也广受求助。他通过信件与朋友们沟通大部分工作,通常没有或者几乎没有证明他的定理。在这些给朋友们的信件中,他探讨了许多微积分的基本思想,早于牛顿和莱布尼茨。费马是经过训练的律师,使得数学更多是一种爱好而非职业。尽管如此,他在解析几何、概率论、数论和微积分等领域作出了重要贡献。[8] 当时,欧洲数学界普遍保密,这自然导致了与同时代人物如笛卡尔和沃利斯的优先权争议。[9]

安德斯·哈尔德(Anders Hald)写道:“费马数学的基础是经典希腊学术著作与维埃特(Vieta)新型代数方法的结合。”[10]
\subsubsection{工作}
费马在解析几何方面的开创性工作《Methodus ad disquirendam maximam et minimam et de tangentibus linearum curvarum》于1636年以手稿形式传播(基于1629年取得的结果),[11] 早于笛卡尔的《La géométrie》(1637年)出版,而后者借鉴了费马的工作。[12] 该手稿于1679年在《Varia opera mathematica》一书中以《Ad Locos Planos et Solidos Isagoge》(平面与立体位置引论)的形式出版。[13]

在《Methodus ad disquirendam maximam et minimam et de tangentibus linearum curvarum》中,费马发展了一种方法(等值法),用于确定各种曲线的极大值、极小值和切线,这一方法与微分学等效。[14][15] 在这些工作中,费马还提出了一种寻找各种平面和立体图形重心的技术,这为他在求积方面的进一步研究奠定了基础。

费马是已知的第一位对一般幂函数的积分进行了求解的人。通过他的这一方法,他能够将积分计算简化为几何级数的和。[16] 这一公式对牛顿和莱布尼茨在独立发展微积分基本定理时提供了帮助。

在数论方面,费马研究了佩尔方程、完美数、友好数以及后来的费马数。正是在研究完美数时,他发现了费马小定理。他发明了一种因式分解方法——费马因式分解法,并推广了无限下降法的证明方法,费马使用这一方法证明了费马直角三角形定理,其中包括作为推论的费马大定理在n = 4时的情况。费马发展了两平方定理和多边形数定理,后者指出每个数都可以表示为三个三角形数、四个平方数、五个五边形数,依此类推。

尽管费马声称自己已经证明了所有的算术定理,但他的证明记录几乎没有保存下来。许多数学家,包括高斯,对他的几项主张表示怀疑,尤其是考虑到一些问题的难度以及费马当时可用的有限数学方法。他的大定理首次由他的儿子在费马父亲的一本《丢番图算术》版本的空白页上发现,并包括了空白页太小,无法容纳证明的声明。似乎他并没有写信给马林·梅尔森(Marin Mersenne)讨论这个问题。费马大定理最终在1994年由安德鲁·怀尔斯爵士证明,使用了当时费马无法获得的技术。

通过1654年的信件往来,费马与布莱兹·帕斯卡(Blaise Pascal)帮助奠定了概率论的基础。在关于“点的问题”的简短但富有成效的合作中,他们被认为是概率论的共同创始人。[17] 费马被认为是首次进行严格概率计算的人。在这个计算中,他被一位职业赌徒问到,为什么在四次掷骰子中至少掷出一个六点会使他长期获胜,而在24次掷两颗骰子中至少掷出一对双六却会导致他输钱。费马通过数学证明了这一现象的原因。[18]

物理学中第一个变分原理由欧几里得在《镜像学》(Catoptrica)中提出。该原理指出,对于反射光的路径,入射角等于反射角。亚历山大港的赫罗(Hero of Alexandria)后来证明,这条路径具有最短的长度和最短的时间。[19] 费马对这一原理进行了完善和推广,提出“光在两个给定点之间沿着最短时间的路径传播”,这现在被称为最短时间原理。[20] 正因为如此,费马被认为是物理学中最小作用原理历史发展的关键人物。费马原理和费马泛函的名称正是为了表彰他在这一方面的贡献。[21]
\subsubsection{去世}  
皮埃尔·德·费马于1665年1月12日去世,地点在卡斯特尔(Castres),现属于塔恩省(Tarn)。[22] 图卢兹最古老且最有声望的高中以他命名:皮埃尔·德·费马中学(Lycée Pierre-de-Fermat)。法国雕塑家特奥菲尔·巴罗(Théophile Barrau)创作了一座名为《致敬皮埃尔·费马》(Hommage à Pierre Fermat)的雕像,以此向费马致敬,这座雕像现存于图卢兹市的市政厅(Capitole de Toulouse)。
\begin{figure}[ht]
\centering
\includegraphics[width=6cm]{./figures/a5d4db98060b5beb.png}
\caption{皮埃尔·德·费马的埋葬地点位于卡斯特尔的让·乔雷广场(Place Jean Jaurés)。铭牌的翻译:“皮埃尔·德·费马,法兰西国王法庭议员(根据《南特敕令》设立的法庭)和享有盛誉的数学家,于1665年1月13日埋葬于此,以其定理而闻名:an + bn ≠ cn,适用于n>2。”} \label{fig_Pierre_3}
\end{figure}
\begin{figure}[ht]
\centering
\includegraphics[width=6cm]{./figures/0aafb701390973f4.png}
\caption{位于法国南部塔恩-加龙省博蒙德·德·洛曼热的费马纪念碑} \label{fig_Pierre_4}
\end{figure}
\begin{figure}[ht]
\centering
\includegraphics[width=6cm]{./figures/6ab2138c9ee31b26.png}
\caption{位于图卢兹市政厅亨利·马丁厅(Salle Henri-Martin)的费马半身像} \label{fig_Pierre_5}
\end{figure}
\begin{figure}[ht]
\centering
\includegraphics[width=6cm]{./figures/ba608ed031723c5f.png}
\caption{费马于1660年3月4日亲笔手写的 holographic遗嘱,目前保存在图卢兹的上加龙省档案馆(Departmental Archives of Haute-Garonne)中。} \label{fig_Pierre_6}
\end{figure}
\subsection{对其工作的评估}  
与勒内·笛卡尔(René Descartes)一起,费马是17世纪上半叶两位杰出的数学家之一。根据彼得·L·伯恩斯坦(Peter L. Bernstein)在1996年出版的《Against the Gods》一书中的说法,费马“是一位具有非凡才华的数学家。他是解析几何的独立发明者,为微积分的早期发展作出了贡献,他研究了地球的重量,且在光的折射和光学方面也有工作。在与布莱兹·帕斯卡尔(Blaise Pascal)的长期通信过程中,他对概率论作出了重要贡献。但费马的巅峰成就是在数论领域。”[23]

关于费马在分析学方面的工作,艾萨克·牛顿(Isaac Newton)写道,他自己对微积分的早期想法直接来源于“费马的切线法。”[24]

20世纪数学家安德烈·韦伊(André Weil)评价费马在数论方面的工作时写道:“我们所拥有的费马处理一类曲线(即一代曲线)的方法是非常一致的;这仍然是现代一代曲线理论的基础。它自然地分为两部分;第一部分...可以方便地称为上升法,与其正确地被视为费马所独有的下降法相对。”[25] 关于费马使用上升法,韦伊继续说道:“新颖之处在于费马极大地扩展了它的应用,这使他至少部分获得了我们通过系统使用有理点在标准三次曲面上的群论性质所得到的结果。”[26] 凭借他在数值关系方面的天赋,以及找到许多定理证明的能力,费马实际上创造了现代数论的理论。
\subsection{另见}  
\begin{itemize}
\item 对角形式  
\item 欧拉定理  
\item 以皮埃尔·德·费马命名的事物列表  
\item 加斯帕尔·德·菲布(Gaspard de Fieubet)
\end{itemize}
\subsection{注释}  
a.大多数来源将费马的出生年份定为1601年;然而,一些来源将费马的出生年份定为1607年,但最近的研究表明,这一年是费马同母异父的哥哥皮埃尔(Piere)出生的年份。[2] 皮埃尔在费马出生后去世。
\subsection{参考文献}  
\begin{enumerate}
\item Benson, Donald C. (2003). *A Smoother Pebble: Mathematical E
xplorations*, Oxford University Press, p. 176.  
\item "When Was Pierre de Fermat Born? | Mathematical Association of America". www.maa.org. Retrieved 2017-07-09.  
\item W.E. Burns, *The Scientific Revolution: An Encyclopedia*, ABC-CLIO, 2001, p. 101.  
\item Chad (2013-12-26). "Pierre de Fermat Biography - Life of French Mathematician". *Totally History*. Retrieved 2023-02-22.  
\item "Fermat, Pierre De". www.encyclopedia.com. Retrieved 2020-01-25.  
\item Davidson, Michael W. "Pioneers in Optics: Pierre de Fermat". micro.magnet.fsu.edu. Retrieved 2020-01-25.  
\item "Pierre de Fermat's Biography". www.famousscientists.org. Retrieved 2020-01-25.  
\item Larson, Ron; Hostetler, Robert P.; Edwards, Bruce H. (2008). \item *Essential Calculus: Early Transcendental Functions*. Boston: Houghton Mifflin. p. 159. ISBN 978-0-618-87918-2.  
\item Ball, Walter William Rouse (1888). *A short account of the history of mathematics*. General Books LLC. ISBN 978-1-4432-9487-4.  
\item Faltings, Gerd (1995). "The proof of Fermat's last theorem by R. \item Taylor and A. Wiles" (PDF). *Notices of the American Mathematical Society*. 42 (7): 743–746. MR 1335426.  
\item Daniel Garber, Michael Ayers (eds.), *The Cambridge History of Seventeenth-century Philosophy*, Volume 2, Cambridge University Press, 2003, p. 754 n. 56.  
\item "Pierre de Fermat | Biography & Facts". *Encyclopedia Britannica*. Retrieved 2017-11-14.  
\item Gullberg, Jan. *Mathematics from the birth of numbers*, W. W. \item Norton & Company; p. 548. ISBN 0-393-04002-X ISBN 978-0393040029.  
\item Pellegrino, Dana. "Pierre de Fermat". Retrieved 2008-02-24.  
\item Florian Cajori, "Who was the First Inventor of Calculus" *The American Mathematical Monthly* (1919) Vol.26.  
\item Paradís, Jaume; Pla, Josep; Viader, Pelegrí (2008). "Fermat's method of quadrature". *Revue d'Histoire des Mathématiques*. 14 (1): 5–51. MR 2493381. Zbl 1162.01004. Archived from the original on 2019-08-08.  
\item O'Connor, J. J.; Robertson, E. F. "The MacTutor History of Mathematics archive: Pierre de Fermat". Retrieved 2008-02-24.  
\item Eves, Howard. *An Introduction to the History of Mathematics*, Saunders College Publishing, Fort Worth, Texas, 1990.  
\item Kline, Morris (1972). "The Greek Rationalization of Nature". \item *Mathematical Thought from Ancient to Modern Times*. New York: Oxford University Press. pp. 167–168. ISBN 978-0-19-501496-9. Retrieved 2024-10-09 – via Internet Archive text collection.  
\item "Fermat's principle for light rays". Archived from the original on March 3, 2016. Retrieved 2008-02-24.  
\item Červený, V. (July 2002). "Fermat's Variational Principle for \item Anisotropic Inhomogeneous Media". *Studia Geophysica et Geodaetica*. 46 (3): 567. Bibcode:2002StGG...46..567C. doi:10.1023/A:1019599204028. S2CID 115984858.  
\item Klaus Barner (2001): *How old did Fermat become?* Internationale Zeitschrift für Geschichte und Ethik der Naturwissenschaften, Technik und Medizin. ISSN 0036-6978. Vol 9, No 4, pp. 209-228.  
\item Bernstein, Peter L. (1996). *Against the Gods: The Remarkable Story of Risk*. John Wiley & Sons. pp. 61–62. ISBN 978-0-471-12104-6.  
\item Simmons, George F. (2007). *Calculus Gems: Brief Lives and Memorable Mathematics*. Mathematical Association of America. p. 98. ISBN 978-0-88385-561-4.  
\item Weil 1984, p.104  
\item Weil 1984, p.105
\end{enumerate}