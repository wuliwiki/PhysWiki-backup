% 热动平衡判据
% 熵判据
\pentry{热平衡 热力学第零定律\upref{TherEq}热力学第一定律\upref{Th1Law}热力学第二定律\upref{Td2Law}}

对于一个单元单相的孤立热力学系统(不考虑外场),它的平衡态意味着系统的\textbf{各个宏观性质在长时间内不发生变化},热平衡的判据为热学平衡、力学平衡、化学平衡\upref{TherEq}.为了能更好地将方法推广到更一般的系统(例如等温等压系统,例如单元复相系),来讨论相变和化学变化问题,我们需要具体地给出判据并在数学上进行分析.

\subsection{单元单相孤立系统的熵判据}

对于单元单相的孤立系统(体积 $V$,内能 $U$ 都不变),由熵增加定理\upref{Td2Law},我们可以利用\textbf{熵判据}判定孤立系统的某一状态是否为平衡态.也就是说,系统在任意的虚位移下,$\delta S=0$;熵还应当具有极大值,所以 $\delta^2S<0$,这是熵判据的稳定性条件.

虽然系统的内部可能宏大复杂的,但可以将系统“划分”成许多小部分,每一个小部分的内部 $P,V,T$ 近似处处相等,又仍有大量的微观粒子,这称为\textbf{子系统};对这样的子系统,熵是容易计算的.熵是广延量,将所有子系统的熵相加,可以得到整个孤立系统的熵.类似地,$U$ 是每个子系统内能之和,$V$ 是每个子系统体积之和.

现在设想系统发生一个“虚位移”,某两个子系统发生了变化:系统 $1$ 的内能变化为 $\delta U_1$,体积变化为 $\delta V_1$,系统 $2$ 的内能变化为 $\delta U_2$,体积变化为 $\delta V_2$.有 $\delta U_1+\delta U_2=0$,$\delta V_1+\delta V_2=0$.现在来看 $\delta S=\delta S_1+\delta S_2$:

\begin{align}
\delta S&=\delta S_1+\delta S_2=\frac{\delta U_1+P_1\delta V_1}{T_1}+\frac{\delta U_2+P_2\delta V_2}{T_2}
\\
&=\delta U_1\left(\frac{1}{T_1}-\frac{1}{T_2}\right)
+\delta V_1\left(\frac{P_1}{T_1}-\frac{P_2}{T_2}\right)=0
\end{align}

由于虚位移的 $\delta U_1$ 和 $\delta V_1$ 可以独立地改变,所以可以得到平衡条件:

\begin{equation}
T_1=T_2,P_1=P_2
\end{equation}

这恰好对应着热学平衡和力学平衡条件.下面我们对稳定性条件 $\delta^2 S<0$ 进行分析.经过一系列计算,可以得到新的