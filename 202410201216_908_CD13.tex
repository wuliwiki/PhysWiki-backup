% 四川大学 2013 年硕士物理考试试题(933)
% license Usr
% type Note

\textbf{声明}:“该内容来源于网络公开资料,不保证真实性,如有侵权请联系管理员”

\subsubsection{1.(本题8分)}
如一,3题图所示,一理想气体分别经①、②、③过程出$A$状态到达 $B,C,D$状态,其中②)是绝热过程。试分析这三个过程中哪些是吸热过程,哪些是放热过程。
\begin{figure}[ht]
\centering
\includegraphics[width=6cm]{./figures/800d56d6ce15aa0d.png}
\caption{} \label{fig_CD13_1}
\end{figure}
\subsubsection{2.(本题6分)}
有直径为$16cm$及$10cm$ 的非常薄的两个铜制球壳,同心放置时,内球的电势为 $2700V$,外球带有电荷量为$8.0\times10^{-9}C$。现把内球和外球接触,两球的电势各变化多少$(\varepsilon_0=8.85\times10^{-12} C^2/(N\cdot m^2))$
\subsubsection{3.(本题6分)}
所示的三个闭合回路1、2、3,分别写出磁感应强度$B$沿它们的环流值。设$I_1=I_2=I$。并讨论:
\begin{enumerate}
\item 在每个闭合回路上各点的$B$值是否相等?
\item 在回路3上各点的$B$是否均等于零?
\end{enumerate}
\subsubsection{4.(本题8分)}
判断下列各一.6 题图中的导线段 AC.或者导线框内的感应电动势的方向。
\begin{figure}[ht]
\centering
\includegraphics[width=6cm]{./figures/2e68364f6c75a0b1.png}
\caption{} \label{fig_CD13_2}
\end{figure}
\subsubsection{5.(本题6分)}
钠黄光中包含着两条相近的谱线,其波长分别为$\lambda_1=589.0nm$和$\lambda_2=589.6nm$。用钠黄光照射迈克耳孙干涉仪,当干涉仪的可动反射镜连续地移动时,视场中的干涉条纹将周期性地由清晰逐渐变模糊,再逐渐变清晰,再变模糊,….求视场中的干涉条纹某一次由最清晰变为最模糊的过程中可动反射镜移动的距离$d$.
\subsubsection{6.(本题6分)}
一台望远镜物镜的直径为$3.5m$,对子平均波长入$\lambda=550nm$ 的可见光,其最小分辨角为多少?其正常放大率为多少?(人眼的分辨极限角$\theta=1'-2.9\times10^{-4}rad$)
\subsubsection{7.(本题6分)}
对于自然光,圆偏振光,线偏振光,你如何通过实验做出判别?
\subsection{计算题(共90分)。}
\subsubsection{1、(本题10分)}
一体积为$10\times10\times3m^3$的密封房间,室温为27℃,已知空气的密度。$\rho=1.29kg/m$摩尔质量$29x10^{-3}kg/mol$,且空气分子可认为是刚性双原子分子。$(R=8.31J/(mol\cdot K))$。
求
\begin{enumerate}
\item 室内空气分子热运动的平均平动动能的总和是多少?
\item 如果气体的温度升高$1.0K$,而体积不变,则气体的内能变化多少?
\item 气体分子的方均根速率增加多少?
\end{enumerate}
\subsubsection{2、(本题 12 分)}
$3mol$ 氧气在压强为2个标准大气压时的体积为 $40L$,先将它绝热压缩到体积的一半,再令它等温膨胀回原来的体积。求在这一过程中氧气吸收的热量、对外做的功以及内能的变化。
\subsubsection{3、(本题8分)}
一定量的理想气体经历如二3题图所示的循环过程,$A\to B$和$C\to D$ 是等压过程,$B\to C$和 $D\to A$是绝热过程。已知:7=300K,7=400K,试求:此环的效率。
\begin{figure}[ht]
\centering
\includegraphics[width=6cm]{./figures/d99e7d36ca8f88d1.png}
\caption{} \label{fig_CD13_3}
\end{figure}
\subsubsection{4、(本题10分)}
电荷面密度分别为 $+\sigma$ 和 $-\sigma$ 的两块无限大均匀带电平行平面,分别与 x 轴垂直相交于 $x_1 = b, x_2 = -b$ 两点,如图 2.4 图所示。设坐标原点 $O$ 处电势为零,试求空间的电势分布表示式并画出其曲线。
\begin{figure}[ht]
\centering
\includegraphics[width=6cm]{./figures/8b2a211455d484aa.png}
\caption{} \label{fig_CD13_4}
\end{figure}
\subsubsection{5、(12 分)}
一无限长圆柱形铜导体(磁导率$\mu_0$),半径为$R$,通有均匀分布的电流$I$。今取一矩形平面$S$(长为$L$,宽为$2R$),位置如二5题图中阴影部分所示,求通过该矩形平面的磁量。
\begin{figure}[ht]
\centering
\includegraphics[width=6cm]{./figures/1ed58dca7d8213b7.png}
\caption{} \label{fig_CD13_5}
\end{figure}
\subsubsection{6、(8 分)}