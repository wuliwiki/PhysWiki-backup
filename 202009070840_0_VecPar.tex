% 平行性与和乐群 (向量丛)
\pentry{曲率 (向量丛)\upref{VecCur}, 费罗贝尼乌斯定理\upref{FrobTh}, 基本群\upref{HomT3}}

本节采用爱因斯坦求和约定.

设$M$是$n$维微分流形, $E$是其上秩为$k$的光滑向量丛. 设给定了$E$上的联络$D$.

\subsection{平行截面}

向量丛$E$的截面$\xi\in\Gamma(E)$称为在联络$D$之下\textbf{平行的 (parallel)}, 如果
$$D\xi=0.$$
在局部标架$\{s_\alpha\}_{\alpha=1}^k$之下, 如果$\xi=\xi^\alpha s_\alpha$, $\omega$是此标架下的联络1-形式矩阵, 则$D\xi=0$等价于
$$
d\xi^\alpha+\xi^\beta\omega^\alpha_\beta=0,\,1\leq \alpha\leq k.
$$
进一步给定切丛和余切丛的局部标架$\{e_i\},\{\theta_j\}$ (二者为对偶) 之后, 就有了克氏符$\Gamma_{\beta i}^\alpha$, 因此上式进一步等价于方程组
$$
e_i(\xi^\alpha)+\xi^\beta\Gamma^\alpha_{\beta i}=0,\,1\leq \alpha\leq k,\,1\leq i\leq n.
$$
这是普法夫系. 在局部上, 根据费罗贝尼乌斯定理\upref{FrobTh}, 这方程组可解 (局部上相当于有$k$个函数$\{\xi^\alpha\}_{\alpha=1}^k$ 满足上面的偏微分方程组) 当且仅当
$$
0=d(\omega^\alpha_\beta\xi^\beta)=d\xi^\beta\wedge\omega^\alpha_\beta+\xi^\beta d\omega^\alpha_\beta=\xi^\beta\wedge\Omega_\beta^\alpha.
$$
这也就表示这个方程组可解当且仅当
$$
\xi^\beta\wedge\Omega_\beta^\alpha=0.
$$
因此如果$D$的曲率在某个开集上等于零, 则上述普法夫系是局部上可积的, 于是此开集上存在$E$的平行截面. 这样一来, 曲率是平行截面存在的障碍.

$\mathbb{R}^n$上平凡丛$\mathbb{R}^n\times\mathbb{R}^k$上的平凡联络就是通常的微分运算, 因此当然是可交换的, 曲率算子为零. 设$\partial_\alpha$是$\mathbb{R}^k$上的一个仿射坐标向量, 则它在此联络下就是平行的. 这很符合"平行"的直观意义. 

\subsection{平行移动}
More refined arguments are as follows. Let $\gamma:[0,1]\to M$ be a Lipschitz curve, and $\xi_0\in E_{\gamma(0)}$. The \emph{parallel transport of $\xi_0$ along $\gamma$} is a section $s(t)$ of $E$ along $c$ that satisfies $D_{\gamma'}s=0$, or in local coordinates (with $s=\xi^\alpha s_\alpha$),
$$\frac{d\xi^\alpha}{dt}+\Gamma_{\beta i}^\alpha\xi^\beta\frac{d\gamma^i}{dt}=0.$$
This is a linear ODE in $(\xi^\alpha(t))$ with bounded coefficients, so it has a unique solution with any given initial value. This induces a linear operator $P_{\gamma}:E_{\gamma(0)}\to E_{\gamma(1)}$, i.e., sending $\xi_0\in E_{\gamma(0)}$ to $s(1)\in E_{\gamma(1)}$. Note that the differential equations imply that this operator does not depend on the parametrization of the path $\gamma$. Easy calculation gives that $P_\gamma^{-1}$ is in fact the parallel transform along the inverse path $\gamma^{-1}$. Furthermore, let $P_\gamma^{\tau,t}$ be the parallel transport from $E_{\gamma(s)}$ to $E_{\gamma(t)}$. Then an easy calculation gives that for any section $s\in\Gamma(E)$ and almost all $t$ (if $\gamma$ is $C^1$ then every $t$),
$$D_{\gamma'(t)}\xi=\lim_{\tau\to t}\frac{P_\gamma^{\tau,t}s(\gamma(\tau))-s(\gamma(t))}{\tau-t}.$$
Thus up to first order derivative, parallel transports along curves completely determine a connection.

\subsection{和乐群}
Now fix a point $p\in M$. Consider all parallel transports $P_\gamma:E_p\to E_p$, where $\gamma$ is a $C^1$ loop based at $p$. It is easily verified that $P_\gamma\cdot P_\eta=P_{\gamma\cdot\eta}$ (where $\gamma\cdot\eta$ is the usual articulation of paths), $P_{\text{const.}}=\text{id}$, and $P_\gamma^{-1}=P_{\gamma^{-1}}$. So the totality of such transforms forms a subgroup of $GL(E_x)$, called the \emph{holonomy group} of $D$ with base point $p$, which is denoted as $\text{Hol}_p(D)$. It is worth noting that if $q\in M$ is another point, then $\text{Hol}_p(D)$ is isomorphic to $\text{Hol}_q(D)$ via any path connecting $p$ to $q$ (just as the fundamental group). If one restricts to consider only null-homotopic paths insdead, then the resulting group is called the restricted holonomy group, which is denoted as $\text{Hol}^0_p(D)$. The quotient $\text{Hol}_p(D)/\text{Hol}_p^0(D)$ becomes a quotient group of $\pi_1(M)$ in a nearly obvious way.

One can topologize $\text{Hol}_p(D)$ and $\text{Hol}_p^0(D)$ with the induced topology. Suppose $\gamma:S^1\to M$ is a null-homotopic loop based at $p$. Let $\gamma_u(t):[0,1]\times S^1\to M$ be a homotopy to $\gamma_1(t)\equiv p$. Then by standard theory of ODE's, the map $u\to P_{\gamma_u}$ is a continuous path connecting $P_\gamma$ to $\text{id}$, which is also contained in $\text{Hol}_p^0(D)$. Hence $\text{Hol}_p^0(D)$ is a path-connected subgroup of $GL(E_p)$. By a theorem of Yamabe, $\text{Hol}_p^0(D)$ is thus a \emph{Lie subgroup}, and so is $\text{Hol}_p(D)$; $\text{Hol}_p^0(D)$ is in fact the identity component of $\text{Hol}_p^0(D)$.

\subsection{和乐定理}
The Lie algebra of $\text{Hol}_p^0(D)$ and $\text{Hol}_p(D)$ coincide since the quotient is countable; there Lie algebra $\mathfrak{h}_p(D)$ is called the \emph{holonomy algebra} of $D$ at $p\in M$, which is a subalgebra of $\text{End}(E_x)$. To characterize the holonomy algebra, one should establish Cartan's characterization of curvature via parallel transports. To achieve this, consider a smooth mapping $f:\Delta_2\to M$, where $\Delta_2$ is the $2$-simplex (left down corner of the $1\times1$ square), such that $f(0,0)=p$, and $\partial_xf(0,0)=X$, $\partial_yf(0,0)=Y$, where $X,Y\in T_pM$. Let $f_u(x,y):=f(ux,uy)$, which is a contracting homotopy from the loop $f_{\partial\Delta_2}:\partial\Delta_2\to M$ to the constant loop $p$. Now let $t$ be the length parameter of this loop, and denote by $\gamma_u$ the loop corresponding to $u$. Then $\gamma_u(t)=f(ux(t),uy(t))$, and an easy calculation gives
$$\left.\frac{d}{du}P_u\right|_{u=0}=0,\,\left.\frac{d^2}{du^2}P_u\right|_{u=0}=-2R(X,Y).$$
This argument is amplified to show that the infinitesimal generators of $\text{Hol}_p^0(D)$ (i.e. generators of the holonomy algebra) should all take the form indicated above. Thus follows the theorem of holonomy:

\begin{theorem}{Ambrose-Singer 和乐定理, 向量丛版本}
设$R$是联络$D$的曲率算子. 则对于任何一点$p\in M$, 和乐代数$\mathfrak{h}_p(D)$是由变换$\{R(X,Y):X,Y\in T_pM\}$的集合生成的.
\end{theorem}