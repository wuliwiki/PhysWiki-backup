% FLINT 库笔记

\begin{issues}
\issueDraft
\end{issues}

\pentry{GNU Multiple Precision(GMP)库笔记\upref{GMP}}

\subsubsection{flint.h}
\begin{itemize}
\item \href{http://flintlib.org/sphinx/fmpz.html}{文档}
\item \verb|fmpz_t| 相比于 GMP 的大整数类型 \verb|mpz_t| (即 \verb|__mpz_struct[1]|)对较小的整数进行了优化, 使其无需额外的 alloc. 当大小超过一定范围时才变为 \verb|__mpz_struct| 的指针, 使用 \verb|mpz_t| 的功能.
\item \verb|fmpz| 底层就是 \verb|slong|. 当第二最重要 bit 为 0 就是一个普通的 slong, 绝对值最多有 \verb|FLINT_BITS - 2| bits.
\item 当第二重要 bit 为 1 时, 它代表一个指针 (把 \verb|slong| 左移 \verb|<< 2| 就是 \verb|__mpz_struct *|)只想 GMP 的大整数.
\end{itemize}
\begin{lstlisting}[language=cpp]
// COEFF_MIN, COEFF_MAX 是 fmpz 不额外 alloc 时能表示的最小和最大整数
#define COEFF_MAX ((WORD(1) << (FLINT_BITS - 2)) - WORD(1))
#define COEFF_MIN (-((WORD(1) << (FLINT_BITS - 2)) - WORD(1)))
/* fmpz x 是否是指针,存在 alloc */
#define COEFF_IS_MPZ(x) (((x) >> (FLINT_BITS - 2)) == WORD(1))

// from flint.h
#define ulong mp_limb_t
#define slong mp_limb_signed_t

#if GMP_LIMB_BITS == 64
    #define FLINT_BITS 64 // 一个 limb 或者 slong 的 bit 数
    #define FLINT_D_BITS 53
    #define FLINT64 1
#else 
    #define FLINT_BITS 32
    #define FLINT_D_BITS 31
#endif

typedef slong fmpz;
typedef fmpz fmpz_t[1];

// 当 fmpz 存在额外 alloc 时, 在 fmpz (即 slong)和 mpz 指针之间转换
// fmpz-conversions.h
#define PTR_TO_COEFF(x) (((ulong) (x) >> 2) | (WORD(1) << (FLINT_BITS - 2)))
#define COEFF_TO_PTR(x) ((__mpz_struct *) ((x) << 2))

// alloc 并 init 一个 GMP 大整数结构 __mpz_struct
// fmpz_gc.c (gc 是 garbage collection)

ulong mpz_alloc = 0; // 总的 alloc 的 __mpz_struct 的数量

// mpz_free_arr[i] 是第 i 个未使用的 __mpz_struct 的地址
__mpz_struct ** mpz_free_arr = NULL;
__mpz_struct ** mpz_arr = NULL;
ulong mpz_num = 0;
ulong mpz_alloc = 0;
ulong mpz_free_num = 0; // alloc 未使用的 __mpz_struct 的数量
ulong mpz_free_alloc = 0;

__mpz_struct * _fmpz_new_mpz(void)
{
    __mpz_struct * z = NULL;

#if FLINT_USES_PTHREAD
    pthread_once(&fmpz_initialised, fmpz_lock_init);
    pthread_mutex_lock(&fmpz_lock);
#endif

    if (mpz_free_num != 0) // 存在未使用的 __mpz_struct
        z = mpz_free_arr[--mpz_free_num];
    else
    { // 需要新 alloc 一个 __mpz_struct
        z = flint_malloc(sizeof(__mpz_struct));

        if (mpz_num == mpz_alloc) /* store pointer to prevent gc cleanup */
        {
            mpz_alloc = FLINT_MAX(64, mpz_alloc * 2);
            mpz_arr = flint_realloc(mpz_arr, mpz_alloc
                * sizeof(__mpz_struct *));
        }
        mpz_arr[mpz_num++] = z;

        mpz_init(z);
    }

#if FLINT_USES_PTHREAD
    pthread_mutex_unlock(&fmpz_lock);
#endif

    return z;
}

// 把 fmpz 从非额外 alloc 变为额外 alloc
// fmpz_gc.c
__mpz_struct * _fmpz_promote(fmpz_t f)
{
    if (!COEFF_IS_MPZ(*f)) /* f is small so promote it first */
    {
        __mpz_struct * mf = _fmpz_new_mpz();
        (*f) = PTR_TO_COEFF(mf);
        return mf;
    }
    else /* f is large already, just return the pointer */
        return COEFF_TO_PTR(*f);
}

FMPZ_INLINE void
fmpz_set_si(fmpz_t f, slong val)
{
    if (val < COEFF_MIN || val > COEFF_MAX) /* val is large */
    {
        // __mpz_struct 是 GMP 的大整数数据结构
        __mpz_struct *mpz_coeff = _fmpz_promote(f);
        flint_mpz_set_si(mpz_coeff, val);
    }
    else
    {
        _fmpz_demote(f);
        *f = val;               /* val is small */
    }
}
\end{lstlisting}
