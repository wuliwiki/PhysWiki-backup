% 萨尔瓦多·达利(综述)
% license CCBYSA3
% type Wiki

本文根据 CC-BY-SA 协议转载翻译自维基百科\href{https://en.wikipedia.org/wiki/Salvador_Dal\%C3\%AD}{相关文章}。
\begin{figure}[ht]
\centering
\includegraphics[width=6cm]{./figures/642fd2d2a2e5e3b7.png}
\caption{1939年的达利} \label{fig_SRWD_1}
\end{figure}
萨尔瓦多·多明戈·费利佩·哈辛托·达利·多梅内克,布波尔侯爵\textbf{(Salvador Domingo Felipe Jacinto Dalí i Domènech, Marquess of Dalí of Púbol),gcYC}(1904年5月11日-1989年1月23日),通常被称为\textbf{萨尔瓦多·达利}(\textbf{Salvador Dalí},发音为/ˈdɑːli, dɑːˈliː/,[2]加泰罗尼亚语:[səlβəˈðo ðəˈli];西班牙语:[salβaˈðoɾ ðaˈli]),是一位西班牙超现实主义艺术家,以其卓越的技术技巧、精准的素描功底以及作品中引人注目且怪诞的画面而闻名。

达利出生于加泰罗尼亚的菲格雷斯,曾在马德里接受正式的美术教育。从年轻时起,他便受到了印象派和文艺复兴大师的影响,后来逐渐被立体派和先锋派艺术运动所吸引。[3]20世纪20年代末,他逐渐向超现实主义靠拢,并于1929年加入超现实主义团体,很快成为其中的核心人物之一。他最著名的作品《记忆的永恒》完成于1931年8月。在西班牙内战期间(1936年至1939年),达利生活在法国,并于1940年移居美国,在那里获得了商业上的成功。1948年,他回到西班牙,宣布重新信仰天主教,并发展出基于对古典主义、神秘主义以及当代科学发展兴趣的“核神秘主义”风格。[4]