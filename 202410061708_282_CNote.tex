% C 语言笔记
% license Xiao
% type Note

\begin{itemize}
\item \verb`gcc` 的 C 语言标准有 \verb`-std=c90`, \verb`c99`, \verb`c11`, \verb`c17`, \verb`c2x` 若要使用 GNU 拓展, 也可以用 \verb`gnu99` 等。
\item 宏 \verb`__STDC_VERSION__` 可以在程序中判断编译器的 C 标准。例如 \verb`199901L` 就是 \verb`c99` 标准, \verb`201112L` 就是 \verb`c11` 标准。 \verb`L` 表示 \verb`long int` 的 literal。 打印如 \verb`printf("C version: %ld\n", __STDC_VERSION__);`
\end{itemize}


\subsection{常用的字符串处理}
\begin{lstlisting}[language=cpp]
#include "string.h"
char s1[20] = "some string1";
char s2[20] = "some string2";
strlen(s1) // 字符串长度
strcpy(s1, s2) // 复制(s2 可以是 literal)
strcmp(s1, s2) // 比较
strcat(s1, s2) // 连接
strrev(s1) // 反转
\end{lstlisting}

\subsection{printf}
\begin{lstlisting}[language=cpp]
#include<stdio.h>

void main()
{
char s[20];
scanf("%s",&s);
printf("%s\n",s);
}
\end{lstlisting}
具体的格式代码见\href{https://www.tutorialspoint.com/c_standard_library/c_function_printf.htm}{这里}。

\begin{lstlisting}[language=cpp]
int i = 1234;
printf("=== int ===\n");
printf("%6.5d\n", i); // " 01234"

printf("=== string ===\n");
string s = "abcdABCD";
printf("%s\n", s.c_str()); // "abcdABCD"

double x = 12.345;
printf("=== double 1 ===\n");
printf("%8.3f\n", x); // "  12.345"
printf("%10.3e\n", x); // " 1.235e+01"
printf("%g\n", x); // "12.345"
printf("%5.3g\n", x); // " 12.3"

double y = 0.000012345678;
printf("=== double 2 ===\n");
printf("%g\n", y); // "1.23457e-05"
printf("%10.3g\n", y); // "  1.23e-05"
\end{lstlisting}

如果不想输出到命令行而是输出到字符串, 就用 \verb`int sprintf(char *str, const char *format, ...)`。 返回写入的字符数。 该函数返回写入到 \verb`str` 的总字符数。

\subsection{scanf}
\begin{lstlisting}[language=cpp]
#include <stdio.h>
int main()
{
  int x,y,z;
  scanf("%d+\n,\n=%d",&x,&y);
  z=x*y;
  printf("x=%d,y=%d\n",x,y);
  printf("xy=%d\n",z);
}
// 输入:
// 123+
// ,
// =234
// 输出:
// x=123,y=234
// xy=28782
\end{lstlisting}
引号内除了特殊字符,其它都需要输入一摸一样的,否则会出错。但是,1.变量前面可以多打任意多个空格和回车(后面不可以),2.任意多个空格或回车相连等效。

\subsection{getchar}
\begin{lstlisting}[language=cpp]
#include <stdio.h>
#include <string.h>
void main()
{
 int i=1;
 char str[5]={0};
    while(i<=5)
   {str[i]=getchar();i++;}

 i=1;
 while(i<=5)
 {printf("%d  ",str[i]);i++;}
 printf("\n");

 i=1;
 while(i<=5)
 {printf("%c",str[i]);i++;}
 printf("\n");
}
\end{lstlisting}

\subsection{enum、 struct 和 union}
\verb`union` 和 \verb`struct` 类似, 但变量共用内存
\begin{lstlisting}[language=cpp]
struct 名称 {} // 声明变量: struct 名称 变量
enum 名称 {}   // 声明变量: enum   名称 变量
union 名称 {}  // 声明变量: union  名称 变量
typedef struct {} 名称 // 声明变量: 名称 变量
typedef enum {} 名称 // 同理
typedef union {} 名称 // 同理
\end{lstlisting}

两种定义方式的使用方法并不完全等效(该代码已使用 \verb`g++` 和 \verb`gcc` 的不同标准测试)。最简单的记忆方法就是第一种声明在使用时需要加 \verb`enum`, 第二种声明不用。
\begin{lstlisting}[language=cpp]
enum B {B1, B2};

enum B b1; // GOOD in C++ and c
// B b2; // good in C++, error in c: unknown type name ‘B’

typedef enum B BB; // GOOD in C++ and c
// typedef B BBB; // good in C++, error in c

// ==========================

typedef enum {A1, A2} A;

A a1; // GOOD in C++ and c
// enum A a2; // error in C++ and c
//     error: using typedef-name ‘A’ after ‘enum’

typedef A AA; // GOOD in C++ and c
// typedef enum A C; // error in C++, good in c
\end{lstlisting}

匿名 structure:
\begin{lstlisting}[language=cpp]
void myfun(struct {int a; double b;} param) {
    /*...*/
}
int main() {
  struct {
    int i;
    double d;
  } var1{1,1.1}, var2{2,2.2};
  myfun(var1);
  myfun(var2);
}
\end{lstlisting}
注意匿名 structure 只能传给接受同样匿名 structure 的函数,不能给接受实名 structure 的函数,即使函数结构一样。

\subsection{函数指针}
\begin{lstlisting}[language=cpp]
// 函数指针(每个括号左右都可以有空格)
double(*p1)(double) = sin; // 也可以用 \verb`&sin`,下同
// 定义函数指针类
typedef double(*Tpf)(double);
Tpf p3 = sin;
// 定义函数类
typedef double(Tf)(double);
Tf *p2 = sin; // 注意 `Tf my_fun;` 是非法的
// 作为类型参数
vector<double(*)(double)> arr{&sin, cos};
// 调用函数(dereference 可选)
cout << "sin(1) = " << arr[0](1) << endl;
cout << "cos(1) = " << (*arr[1])(1) << endl;
\end{lstlisting}

要获取函数指针,既可以用 \verb`函数名` 也可以用 \verb`&函数名`。

\subsection{文件处理}
\begin{itemize}
\item \verb`FILE *fopen(const char *filename, const char *mode);` 创建新文件,读取失败返回 \verb`NULL`。
\item \verb`fscanf, fgets` 读取文件
\item \verb`fprintf` 写文件
\item \verb`fseek, rewind` 移动到文件的某个位置。
\item \verb`fclose` 关闭文件
\item "r": 只读 (文件必须存在).
\item "w": 只写 (新建或覆盖文件).
\item "a": 最后写 (不存在则新建).
\item "r+": 读写 (文件必须存在).
\item "w+": 读写 (新建或覆盖文件).
\item "a+": 最后写和读 (不存在则新建).
\item \verb`r` 模式的例子:
\begin{lstlisting}[language=cpp]
#include <stdio.h>
#include <stdlib.h>

int main() {
    FILE *file;
    char *buffer;
    int start_ind = 5; // Example start index
    int length = 10; // Length of the string to read

    // Open the file in read mode
    file = fopen("example.txt", "r");
    if (file == NULL) {
        perror("Error opening file");
        return 1;
    }

    // Allocate memory for the string
    buffer = (char *)malloc((length + 1) * sizeof(char));
    if (buffer == NULL) {
        perror("Memory allocation error");
        fclose(file);
        return 1;
    }

    // Move the file pointer to the start index
    fseek(file, start_ind, SEEK_SET);

    // Read the bytes from start_ind
    fread(buffer, sizeof(char), length, file);
    buffer[length] = '\\0';  // Null-terminate the string

    // Print the string
    printf("Read string: %s\\n", buffer);

    // Free the allocated memory and close the file
    free(buffer);
    fclose(file);

    return 0;
}
\end{lstlisting}
\item \verb`w+` 模式的例子:
\begin{lstlisting}[language=cpp]
#include <stdio.h>
#include <stdlib.h>

int main() {
    FILE *file;
    
    // Open the file in "w+" mode
    // (write and read, truncate if exists)
    file = fopen("example.txt", "wb+");
    if (file == NULL) {
        printf("Error opening file!\n");
        return 1;
    }

    // Step 1: Write to the file
    fputs("Hello, World!\n", file);
    fputs("This is a test file.\n", file);
    fputs("End of writing.\n", file);

    // Step 2: Jump to the beginning of the file using fseek()
    fseek(file, 0, SEEK_SET);

    // Step 3: Read from the file and print its
    // contents to the console
    char buffer[100];
    printf("Reading the file contents:\n");
    while (fgets(buffer, sizeof(buffer), file) != NULL) {
        printf("%s", buffer);
    }

    // Step 4: Move the file pointer to a specific location, modify contents
    fseek(file, 7, SEEK_SET); // Jump to the 8th character
    fputs("Universe", file);  // Overwrite "World" with "Universe"

    // Step 5: Jump to the beginning again to read the updated content
    fseek(file, 0, SEEK_SET);
    printf("\nUpdated file contents:\n");
    while (fgets(buffer, sizeof(buffer), file) != NULL) {
        printf("%s", buffer);
    }

    // Close the file
    fclose(file);
    return 0;
}
\end{lstlisting}
\item \verb`int fseek(FILE *stream, long offset, int whence);` 可以跳到文件的某个位置。 例如 \verb`fseek(file, 7, SEEK_SET);` 相对于文件开头, \verb`fseek(file, -7, SEEK_CUR);` 相对当前位置, \verb`SEEK_END` 相对于文件 EOF。 如果成功,\verb`fseek()` 返回 0。如果返回非 0 则失败(例如超出范围)。
\item \verb`void perror(const char *str)` 打印错误信息到 \verb`stderr` 并根据全局变量 \verb`errno` 在后面附上 \verb`: 具体错误原因`。
\item \verb`int fgetc(file)` 获取一个字节。 如果因为超出文件结尾没有获取到,则返回 EOF。
\item \verb`EOF` 是 \verb`stdio.h` 中的一个宏,通常是 -1 但不能保证。
\item 例如要获取文件最后的三个字节,用 \verb`fseek(file, -3, SEEK_END)` 然后 \verb`fgetc(file)` 三次即可。
\item ASCII 表不存在 EOF,EOF 是一个概念不是一个字符。
\end{itemize}



\subsection{内存分配}
\begin{itemize}
\item \verb`void *malloc(size_t size);` 分配内存,如果失败返回 \verb`NULL`。
\item \verb`void free(void *ptr);` 释放内存。
\item \verb`void *realloc(void *ptr, size_t new_size);` 会保留之前的数据。 如果可以延长内存就延长, 如果不能就重新分配并复制。 新分配的数据不会初始化。 如果失败返回 \verb`NULL`。
\end{itemize}
