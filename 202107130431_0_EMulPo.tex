% 电多极展开
% 偶极子|多极子|球谐函数|电势

\pentry{球谐函数\upref{SphHar}, 电偶极子\upref{eleDpl}}

若空间中的一个球内($r < a$) 存在静止的电荷分布 $\rho(\bvec r)$, 那么球外的电势 $V(\bvec r)$($r > a$)可以展开为径向函数和球谐函数之积的形式
\begin{equation}\label{EMulPo_eq2}
V(\bvec r) = \sum_{l = 0}^\infty \frac{1}{r^{l+1}}\sum_{m = -l}^l C_{l,m} Y_{l,m}(\uvec r) \qquad (r \geqslant a)
\end{equation}
其中常数 $C_{l,m}$ 为(证明见下文)
\begin{equation}
C_{l,m} = \frac{1}{(2l+1)\epsilon_0} \int_{r\leqslant a} r^l \rho(\bvec r) Y_{l,m}^*(\uvec r) \dd[3]{r}
\end{equation}

$l = 0$ 的项就是\textbf{电单极子(electric monopole)} 具有球对称的电荷分布和势能分布, $l = 1$ 的项就是\textbf{电偶极子(dipole)}\upref{eleDP2}($\propto\cos \theta$), $l = 2$ 的项是\textbf{电四极子(quadrupole)}……  $l = N$ 的项叫做电 $2^N$ 极子.

为什么这么叫呢? 因为电 $2^N$ 极子产生的势能项可以用 $2^{N-1}$ 个电荷为 $q$ 的点电荷以及 $2^{N-1}$ 个电荷为 $-q$ 的点电荷按照一定空间位置摆放后, 取 $r \to \infty$ 得到. 用数学归纳法的思想, 我们可以认为若电 $2^{N-1}$ 极子是一个点, 渐进势能为 $\sim 1/r^N$, 那么把两个电荷相反的 $2^{N-1}$ 极子放在一根短棍的两端组成电 $2^N$ 极子, 那么渐进势能就必须是更高阶无穷小, 即 $\sim 1/r^{N+1}$.

电多极展开的优势在于只需要预先算出系数 $C_{l,m}$ 就可以无需积分得到球外任意位置的电势分布. 而如果用常规的方法, 则计算每个位置的电势都需要重新做一次积分(\autoref{EMulPo_eq5}). 而它的缺点在于实际计算中必须取\autoref{EMulPo_eq2} 中的有限项, 会产生截断误差. 另一点是无法计算球内的电势分布.

\subsubsection{内展开}
相反, 若电荷分布 $\rho(\bvec r)$ 只存在于 $r \geqslant a$ 的球外空间, 那么可以把球内部任意点的电势 $V(\bvec r)$ 展开为
\begin{equation}\label{EMulPo_eq4}
V(\bvec r) = \sum_{l = 0}^\infty r^l \sum_{m = -l}^l D_{l,m} Y_{l,m}(\uvec r) \qquad (r \leqslant a)
\end{equation}
其中(证明见下文)
\begin{equation}\label{EMulPo_eq3}
D_{l,m} = \frac{1}{(2l+1)\epsilon_0} \int_{r\geqslant a} \frac{1}{r^{l+1}} \rho(\bvec r) Y_{l,m}^*(\uvec r) \dd[3]{r}
\end{equation}

\subsubsection{一般展开}
要求任意给定点 $\bvec r$ 的电势, 可取过该点的球面 $a = r$, 并把\autoref{EMulPo_eq2} 和\autoref{EMulPo_eq4} 合并就可以得到 $r$ 取任意值的展开
\begin{equation}
V(\bvec r) = \sum_{l = 0}^\infty\sum_{m = -l}^l \qty[\frac{1}{r^{l+1}} C_{l,m}(r) + r^l D_{l,m}(r)] Y_{l,m}(\uvec r)
\end{equation}
但这计算起来比较麻烦, 因为现在 $C_{l,m}, D_{l,m}$ 成了 $r$ 的函数, 对每个在 $\rho(r)\ne 0$ 范围的 $r$ 都需要计算一次积分, 当 $r$ 大于 $\rho(r)$ 的最大半径(如果存在), $C_{l,m}$ 重新变为常数, $D_{l,m}= 0$, 就回到\autoref{EMulPo_eq2}. 当 $r$ 小于 $\rho(r)$ 的最小半径(如果不为零), $D_{l,m}$ 重新变为常数, $C_{l,m} = 0$, 就回到\autoref{EMulPo_eq4}.

\subsection{证明}
首先我们给出单个点电荷势能的球谐展开公式(链接未完成)
\begin{equation}\label{EMulPo_eq1}
\frac{1}{\abs{\bvec r - \bvec r'}} = 4\pi \sum_{l=0}^{\infty} \frac{1}{2l+1} \frac{r_<^l}{r_>^{l+1}} \sum_{m = -l}^l Y_{l,m}^*(\uvec r') Y_{l,m}(\uvec r)
\end{equation}
其中 $r_< = \min\qty{r, r'}$, $r_> = \max\qty{r, r'}$. 另外连续电荷分布的库仑势能公式为(\autoref{QEng_eq14}~\upref{QEng})
\begin{equation}\label{EMulPo_eq5}
V(\bvec r) = \frac{1}{4\pi\epsilon_0}\int \frac{\rho(\bvec r')}{\abs{\bvec r - \bvec r'}} \dd[3]{r'}
\end{equation}

若我们要求电荷必须在球内($r' \leqslant a$)而 $V(\bvec r)$ 中的 $\bvec r$ 必须在球外($r \geqslant a$), 那么始终有 $r_> = r, r_< = r'$, 代入\autoref{EMulPo_eq1} 再代入\autoref{EMulPo_eq5} 可得\autoref{EMulPo_eq2}.

反之, 若电荷分布 $r' \geqslant a$, 场点 $r \leqslant a$, 那么始终有 $r_> = r', r_< = r$. 代入\autoref{EMulPo_eq1} 再代入\autoref{EMulPo_eq5} 得\autoref{EMulPo_eq4}.
