% 斯里尼瓦瑟·拉马努金(综述)
% license CCBYSA3
% type Wiki

本文根据 CC-BY-SA 协议转载翻译自维基百科 \href{https://en.wikipedia.org/wiki/Srinivasa_Ramanujan}{相关文章}。

斯里尼瓦瑟·拉马努金·艾扬加尔(Srinivasa Ramanujan Aiyangar)FRS(1887年12月22日-1920年4月26日)是一位印度数学家。他常被视为史上最伟大的数学家之一。尽管几乎没有接受过纯数学的正规训练,他仍在数学分析、数论、无穷级数和连分数等领域做出了重要贡献,并提出了当时被认为无法解决的数学问题的解法。

拉马努金最初是在孤立的环境中自行开展数学研究的。正如汉斯·艾森克所说:“他曾试图让当时最顶尖的职业数学家对他的研究产生兴趣,但大多数时候都失败了。他所展示的成果太新颖、太陌生,而且呈现方式也很不寻常;那些人懒得去理会。”\(^\text{[4]}\)为了寻找能真正理解他工作的数学家,1913年,他开始与英国剑桥大学的数学家G.H.哈代通信。哈代意识到拉马努金的研究非同寻常,便为他安排了赴剑桥的行程。在笔记中,哈代评论道,拉马努金提出了具有突破性的全新定理,其中一些“令我完全败下阵来;我从未见过任何类似的东西”,\(^\text{[5]}\)还有一些则是刚刚被证明、极为高深的成果。

在他短暂的一生中,拉马努金独立整理出了近 3900 条数学成果(主要是恒等式和方程)。\(^\text{[6]}\)其中许多都是前所未见的原创成果;他那些独特而极不寻常的发现,如“拉马努金素数”、“拉马努金θ函数”、“整数划分公式”以及“拟θ函数”等,不仅开辟了全新的研究领域,也激发了大量后续研究。\(^\text{[7]}\)在他成千上万的研究成果中,大多数后来都被证明是正确的。\(^\text{[8]}\)以他的名字命名的《拉马努金期刊》应运而生,专门发表受他研究影响的各类数学成果。\(^\text{[9]}\)他留下的笔记本——记录了他已发表和未发表成果的摘要——至今仍被数学家们分析和研究,成为不断涌现新数学思想的重要来源。直到 2012 年,研究者们仍不断发现,他笔记中那些仅以“简单性质”或“相似结果”带过的评论,其实暗藏着深奥而精妙的数论定理,且这些定理直到他去世近百年后才被真正识别出来。\(^\text{[]}\)$10$$11$ 拉马努金是最年轻的英国皇家学会会士之一,是第二位印度籍成员,也是首位当选剑桥大学三一学院会士的印度人。
