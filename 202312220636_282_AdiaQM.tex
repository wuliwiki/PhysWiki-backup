% 绝热近似(量子力学)
% license Usr
% type Tutor

\begin{issues}
\issueTODO
\end{issues}

\pentry{薛定谔方程(单粒子一维)\upref{TDSE11},量子简谐振子(升降算符法)\upref{QSHOop}}

\footnote{参考 Griffiths\cite{GriffE} 的章节: The Adiabatic Approximation、 Shankar\cite{Shankar} 的 Chap18-P478、 Wikipedia \href{https://en.wikipedia.org/wiki/Adiabatic_theorem}{相关页面}。}量子力学中,\textbf{绝热近似(adiabatic approximation)}说的大概是: 若系统初始时处于某个离散非简并的本征态,那么当哈密顿量随时间缓慢改变时(改变的特征时间远大于本征态的), 那改变过程中波函数将仍然处于同一个本征态,但整体相位会发生某种改变。下面先给出定量结论,证明留到文末。

令含时薛定谔方程为(\autoref{eq_TDSE11_6}~\upref{TDSE11})
\begin{equation}
H(t)\Psi(t) = \I\hbar\dot\Psi(t)~.
\end{equation}
当系统不存在简并时, 绝热近似下含时薛定谔方程的通解可以表示为($C_n$ 为常数,由初始波函数决定)
\begin{equation}\label{eq_AdiaQM_2}
\Psi(t) \approx \sum_n C_n \psi_n(t) \E^{\I\theta_n(t)}~.
\end{equation}
其中 $\psi_n(t)$ 是 $H(t)$ 一组正交归一本征态,任意时刻都满足不含时薛定谔方程(时间看作数学参数)
\begin{equation}\label{eq_AdiaQM_3}
H(t)\psi_n(t) = E_n\psi_n(t)~.
\end{equation}
和正交归一化
\begin{equation}\label{eq_AdiaQM_6}
\braket*{\psi_m(t)}{\psi_n(t)} = \delta_{m,n}~.
\end{equation}
为了方便且不失一般性本文规定 $\psi_n(t)$ 始终是实值函数(否则有可能出现一个随时间变化的整体相位让事情更复杂)。 另外相位函数为
\begin{equation}
\theta_n(t) = -\frac{1}{\hbar} \int_0^t E_n(t')\dd{t'}~.
\end{equation}

\begin{example}{}
\begin{enumerate}
\item 当无限深势阱\upref{ISW}缓慢变长。
\item 量子简谐振子(升降算符法)\upref{QSHOop}的劲度系数 $k$ 缓慢变化。
\end{enumerate}
\end{example}

容易看出若 $H(t)$ 不随时间变化时,通解就回到了熟悉的通解(\autoref{eq_TDSE11_5}~\upref{TDSE11})
\begin{equation}
\Psi(t) = \sum_n C_n \psi_n \E^{-\I E_n t/\hbar}~.
\end{equation}

\autoref{eq_AdiaQM_2} 中 $C_n$ 为常数是一个很有力的结论。它告诉我们若开始时波函数处于某个(非简并)本征态,那么它将始终(近似)处于该本征态。

该理论在对分子的计算中有广泛的应用,且有一个响亮的名字,叫\textbf{波恩—奥本海默近似(Born–Oppenheimer approximation)}。 这是因为在分子运动中,原子核的运动速度通常要比电子慢得多,使绝热近似效果较好。

同为含时近似理论,绝热近似和含时微扰理论\upref{TDPTc}有什么区别呢? 前者不要求 $H(t)$ 缓慢变化,例如用激光波包对原子光电离时,电场随时间的周期变化往往并不算慢。 那可以使用绝热近似的情况是否可以使用含时微扰理论呢? 理论上可以,但计算比较麻烦,因为含时微扰使用初始的本征态展开任意时刻的波函数。

\subsection{能级分裂}
\pentry{一阶不含时微扰理论(量子力学)\upref{TIPT}}
若考虑的时间段内,只有初始的一瞬间存在简并, 那么可以认为这个瞬间波函数几乎不发生变化(毕竟 $H(t)$ 是缓慢变化),令 $\psi_n(0)$ 取好量子态\upref{TIPT},并假设系统始终是非简并的即可。
\begin{example}{}
给氢原子的任意束缚态 $\psi_{n,l,m}$ 缓慢施加外电场或磁场(参考 “类氢原子斯塔克效应(微扰)\upref{HStark}”,以及“塞曼效应\upref{ZemEff}”)。注意 $\psi_{n,l,m}$ 并不是好本征态,需要先做投影。
\addTODO{推导}
\end{example}

\subsection{推导:含时薛定谔方程的一种矩阵形式}
我们把含时波函数用瞬时本征态 $\psi_n(t)$ 展开(注意这里的系数是含时的)
\begin{equation}
\Psi(t) \equiv \sum_n c_n(t) \psi_n(t) \E^{\I \theta_n(t)}~,
\end{equation}
为了下面化简方便,不失一般性,令
\begin{equation}
\theta_n(t) \equiv -\frac{1}{\hbar} \int_0^t E_n(t')\dd{t'}~.
\end{equation}
代入含时薛定谔方程
\begin{equation}\label{eq_AdiaQM_1}
H(t)\Psi(t) = \I\hbar \dot \Psi(t)~,
\end{equation}
得
\begin{equation}\label{eq_AdiaQM_5}
\dot c_m(t) = -\sum_n c_n \braket*{\psi_m}{\dot\psi_n}\E^{\I(\theta_n-\theta_m)}~.
\end{equation}
这可以表示为矩阵乘法
\begin{equation}\label{eq_AdiaQM_9}
\dot{\bvec c}(t) = \frac{1}{\I\hbar}\mat A(t) \bvec c(t)~.
\end{equation}
其中矩阵 $\mat A$ 定义为
\begin{equation}\label{eq_AdiaQM_10}
A_{ij}(t) = -\I\hbar \braket*{\psi_m}{\dot\psi_n}\E^{\I(\theta_n-\theta_m)}~.
\end{equation}
注意\autoref{eq_AdiaQM_9} 可以看作含时薛定谔方程的一种矩阵形式,和\autoref{eq_AdiaQM_1} 完全等效。 这类似于 “含时微扰理论\upref{TDPT}” 中的\autoref{eq_TDPT_3}。

另外对\autoref{eq_AdiaQM_3} 求时间偏导得
\begin{equation}\label{eq_AdiaQM_4}
\mel*{\psi_m}{\dot H}{\psi_n} = (E_n-E_m)\braket*{\psi_m}{\dot\psi_n} + \delta_{m,n}\dot E_n~.
\end{equation}
对\autoref{eq_AdiaQM_6} 求导可以证明矩阵 $\braket*{\psi_m}{\dot\psi_n}$ 是一个反对称矩阵,即满足
\begin{equation}\label{eq_AdiaQM_7}
\braket*{\psi_m}{\dot\psi_n} = -\braket*{\psi_n}{\dot\psi_m}~.
\end{equation}
注意 $n=m$ 时该式两边恒为零,\autoref{eq_AdiaQM_10} 中 \textbf{$\mat A$ 的对角元也恒为零}。 且通过该式容易证明 \textbf{$\mat A$ 是厄米矩阵}。

要精确计算矩阵 $\mat A$,一般直接根据定义直接求解 $\psi_n(t)$(\autoref{eq_AdiaQM_3})再代入 \autoref{eq_AdiaQM_10} 即可。 但为了估计 $\mat A$ 矩阵元的大小,我们可以由\autoref{eq_AdiaQM_4} 得
\begin{equation}\label{eq_AdiaQM_11}
\braket*{\psi_m}{\dot\psi_n} = \frac{\mel*{\psi_m}{\dot H}{\psi_n}}{E_n-E_m} \qquad (E_m\ne E_n)~.
\end{equation}

到现在为止,所有推导都是精确的。 绝热近似的关键就在于假设 $H$ 随时间变化缓慢,即 $\dot H$ 非常小,以至于如果两个能级 $E_n$ 和 $E_m$ 不是特别接近时,可以近似认为\autoref{eq_AdiaQM_11} 对应的矩阵元 $A_{m,n}$ 可以忽略不计。

\subsection{非简并情况}
若在考虑的时间区间内, $H(t)$ 始终没有发生简并,不同的能级之间也没有太接近, 那么可以假设\autoref{eq_AdiaQM_11} 对全部 $m\ne n$ 为零,也就是 $\mat A = \bvec 0$。 此时\autoref{eq_AdiaQM_9} 直接变为
\begin{equation}
\dot{\bvec c}(t) = \bvec 0~.
\end{equation}
这说明所有系数都不随时间变化,令常数 $C_n = c_n(0)$,得到\autoref{eq_AdiaQM_2}。

\subsection{简并但不分裂}

\addTODO{其实最想知道的就是,每个简并子空间中是否会存在一些好量子态使得它们可以独立绝热改变}

=========== 重写 =================

若\autoref{eq_AdiaQM_11} 右边不能全部忽略,


\begin{equation}
\dot c_m(t) = -\sum_{n}^{E_n\approx E_m} c_n \braket*{\psi_m}{\dot\psi_n}\E^{\I(\theta_n-\theta_m)}~.
\end{equation}
这是齐次亥姆霍兹方程组,若把矩阵 $\I\braket*{\psi_m}{\dot\psi_n}\E^{\I(\theta_n-\theta_m)}$ 记为矩阵 $\mat A$,。  $\mat A$ 就是一个块对角矩阵,每个对角块代表一个本征子空间(或者若干个可能在某时刻本征值相同的本征子空间张成的空间),不同子空间之间不存在耦合。把所有 $c_m(t)$ 按顺序排成列向量 $\bvec c(t)$,


其通解为(引用未完成)
\begin{equation}
\bvec c(t) = \mat U(t)\bvec c(0)~.
\end{equation}
其中
\begin{equation}
\mat U(t) = \hat{\mathcal{T}}\exp[-\I \int_0^t \mat A(t') \dd{t'}]~
\end{equation}
是一个和 $\mat A$ 结构一样的块对角的酉矩阵\upref{UniMat},也就是每个对角块都分别是一个酉矩阵。这类似于薛定谔方程的演化子(链接未完成),且每个子空间独立演化,概率保持不变。

\addTODO{有没有可能即使 $E_m=E_n$,$\braket*{\psi_m}{\dot\psi_n}$ 也恒为零呢?什么时候?例如缓慢增加氢原子的核电荷时 $\psi_{n,l,m}$ 之间会耦合嘛? stark 的好本征态之间呢?}

\subsubsection{能级分裂与合并}
?? 的两个求和中,至于哪些项放在第二个求和进而被忽略,可能取决于时刻 $t$。 这就是说矩阵 $\mat A(t)$ 的对角块可能有些时候会发生拆分或合并(多个对角块发生耦合后合并为一个)。 但若这种合并的持续时间只有很短乃至一瞬间,那我们可以认为合并前后波函数不发生变化,也就是假设合并不存在。

\addTODO{或许可以搞个矩阵来数值验证一下,例如氢原子的 stark 效应是否真的可以这么搞}

\addTODO{啥时候讲 avoided crossing 啊……}
