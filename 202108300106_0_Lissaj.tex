% 利萨茹曲线

\begin{issues}
\issueDraft
\end{issues}

\footnote{参考 Wikipedia \href{https://en.wikipedia.org/wiki/Lissajous_curve}{相关页面}.}\textbf{利萨茹曲线(Lissajous curve)}是平面上一点在两个垂直的方向分别做相简谐运动形成的轨迹. 把这两个方向作为 $x, y$ 坐标轴, 可以用以下参数方程表示\footnote{把\autoref{Lissaj_eq1} 中的 $\sin$ 都换成 $\cos$ 曲线也一样, 因为相当于给 $x, y$ 同时加上 $\pi/2$ 相位. 重要的是相位差 $\phi$.}
\begin{equation}\label{Lissaj_eq1}
\leftgroup{
x &= A\sin(a t + \phi)\\
y &= B\sin(b t)
}\end{equation}

当 $\phi = 0$ 时是一个椭圆.

\addTODO{相同相同振幅频率不同相位会形成什么曲线? }
