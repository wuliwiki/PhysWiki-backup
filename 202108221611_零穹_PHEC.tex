% 现象类
% 现象|现象类|数的等式

\pentry{量类和单位\upref{QCU}}
\subsection{数的等式}
物理规律是物理量之间关系的反映.既然选定单位后每个量可用一个数代表,物理规律也就可用数的等式表示.反应物理规律的数的等式称为物理规律的\textbf{数值表达式},简称\textbf{数的等式}.

到此为止,我们只定义了同类量之间的一些运算,尚未对量的运算进行定义.所以,一个物理规律,比如牛顿第二定律 $f=ma$ 当中的每个字母应理解成一个数,而反应物理规律的等式应理解成数的等式,而非量的等式,该字母所对应的数是用选定的单位去测该量所得的数值.只有当对量的运算进行了定义的时候,才能将反应物理规律的等式理解成量的等式.在此之前,我们强调:所有物理公式都应理解成数的等式(同类量的等式除外),式中的每一字母代表用某一选定单位测该量所得的数.
\subsection{现象类}
在几何学中,存在中矩形、三角形、柱体等几何对象(亦称\textbf{几何现象}).当几何现象涉及物理问题时,几何现象便转化为\textbf{物理现象},简称\textbf{现象}.例如,“一条导线”是一个几何现象,“一条导线载有电流”就是一种物理现象.所有物理现象构成的集合称为\textbf{现象集},为了更清晰的讨论各种现象之间的关系,我们可以对现象进行分类,即定义现象集的一个分划.

\begin{definition}{}\label{PHEC_def1}
现象 $X_1$ 和 $X_2$ 称为\textbf{同类现象},如果
\begin{enumerate}
\item $X_1$ 和 $X_2$ 所涉及的量类完全一样,记作 $\tilde{\boldsymbol{Q_1}},\cdots,\tilde{\boldsymbol{Q_n}}$,称\textbf{涉及量类}; 
\item 对涉及量类选定单位 $\hat{\boldsymbol{Q_1}},\cdots,\hat{\boldsymbol{Q_n}}$,用它们测 $X_1,X_2$ 的涉及量 $\boldsymbol{Q_1},\cdots ,\boldsymbol{Q_n}$ 所得的数 $Q_1,\cdots,Q_n$ 满足相同的物理方程
\begin{equation}
f(Q_1,\cdots,Q_n)=0
\end{equation}
\end{enumerate}
\end{definition}
显然,\autoref{PHEC_def1} 定义的同类现象是一种等价关系,即同类现象定义了现象集的一个分划.
\begin{definition}{}
所有同类现象的集合称为一个\textbf{现象类}.
\end{definition}
\begin{example}{}
\begin{figure}[ht]
\centering
\includegraphics[width=10cm]{./figures/PHEC_1.pdf}
\caption{“载流导线”现象类(左)和“并联电阻丝”现象类(右)} \label{PHEC_fig1}
\end{figure}
\autoref{PHEC_fig1} 中两图都涉及三个物理量,即导线电流$\boldsymbol{I}$、通电时间 $\boldsymbol{t}$ 和通过导线截面的电荷量 $\boldsymbol{q}$.无论选择何种单位 $\hat{\boldsymbol{I}},\hat{\boldsymbol{t}},\hat{\boldsymbol{q}}$,测得的数 $I,t,q$ 必定满足
\begin{equation}
q=\mu It
\end{equation}
式中,$\mu$ 只取决于单位 $\hat{\boldsymbol{I}},\hat{\boldsymbol{t}},\hat{\boldsymbol{q}}$ 而与该现象类中具体现象无关.对电阻丝有相同电阻 $R_1=R_2$ ,若只关心通过主干导线的电荷量 $\boldsymbol{q}$ 与电阻丝的电流 $\boldsymbol{I}$
\end{example}