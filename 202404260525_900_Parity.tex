% 宇称算符
% keys 宇称|奇宇称|偶宇称|厄米算符
% license Xiao
% type Tutor

\pentry{多元函数积分和宇称\nref{nod_IntPry}}{nod_df63}

对多元函数 $f(\bvec x) = f(x_1, \dots, x_N)$。  定义宇称算符 $\Pi$ 如下
\begin{equation}
\Pi f(\bvec x) = f(-\bvec x)~.
\end{equation}
若函数内积\upref{InerPd}定义为(星号表示复共轭)
\begin{equation}
\braket{f}{g} = \int f^*(\bvec x) g(\bvec x) \dd[N]{x}~,
\end{equation}
容易证明宇称算符是一个厄米算符。% 链接未完成
对本征方程
\begin{equation}
\Pi f(\bvec x) = \lambda f(\bvec x)~,
\end{equation}
容易证明\footnote{分别令 $\bvec x$ 为某对称的两点 $\bvec x_1$ 和 $-\bvec x_1$, 使函数值不为零。 本征方程要求 $f(-\bvec x_1) = \lambda f(\bvec x_1)$ 且 $f(\bvec x_1) = \lambda f(-\bvec x_1)$, 所以必有 $\lambda^2 = 1$, $\lambda = \pm 1$。}宇称算符的本征值 $\lambda$ 只可能等于 $1$ 或 $-1$。 我们说 $\lambda = 1$ 的函数具有\textbf{偶宇称(even parity)}, $\lambda = -1$ 的函数具有\textbf{奇宇称(odd parity)}。 它们分别满足
\begin{equation}
f(-\bvec x) = \pm f(\bvec x)~.
\end{equation}
对于一元函数, 具有奇宇称的就是\textbf{奇函数(odd function)}, 具有偶宇称的就是\textbf{偶函数(even function)}。
