% 刚体的内力
\pentry{简单刚体系统的静力学分析\upref{RGDFA}}
在材料力学中,我们关心一个刚体内部的受力情况,即刚体的内力.本文简要探讨平面杆件某截面处内力的分析方法.

\begin{example}{强度理论}
强度理论认为,材料失效的原因是由于材料内部某处的内力超过了材料所能承受的极限.因此,分析材料内力对于设计或选用可靠的材料起重要作用.
\end{example}

\subsection{截面法 Section Method}
分析材料内力的常用方法是“截面法”,该方法非常的通俗易懂.

0. 先确定作用在该杆件上的外力\upref{RGDFA}.

1. 在需要计算内力的截面处,假想切割杆件,将其“一分为二”.
\begin{figure}[ht]
\centering
\includegraphics[width=8cm]{./figures/INTFRC_1.png}
\caption{假想切割杆件} \label{INTFRC_fig1}
\end{figure}

2. 选取左半段(或右半段)杆件.此时杆件截面处的内力变成了“外力”.类似于钉子模型\upref{RGDFA},内力的效果可看作一个平行于杆件拉力$F_T$、一个垂直于杆件的剪切力$F_S$与一个力偶$M$.
\begin{figure}[ht]
\centering
\includegraphics[width=8cm]{./figures/INTFRC_2.png}
\caption{画出杆件截面处的内力} \label{INTFRC_fig2}
\end{figure}

3. 根据力的平衡条件\upref{RGDFA},即可解出内力.

\subsection{内力分布图}
我们可以做出沿杆件方向,各截面处内力的分布情况.这便于看出内力的分布规律、最大最小值等.
\begin{figure}[ht]
\centering
\includegraphics[width=8cm]{./figures/INTFRC_3.png}
\caption{杆件内剪切力分布图示意图.蓝色为作用于刚体上的外力} \label{INTFRC_fig3}
\end{figure}

可以通过截面法解各处的内力.在外力分布规则的情况下,有一些辅助技巧可以帮助快速解内力.
