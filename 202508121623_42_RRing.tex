% 剩余类环
% keys 同余式|剩余类环|模n同余
% license Usr
% type Tutor

\pentry{环\nref{nod_Ring}}{nod_87af}
在代数中,剩余类环是各类一般概念的出发点。
\subsection{同余}
\begin{definition}{模 $m$同余}\label{def_RRing_1}
设 $n, n^{\prime}$ 是两整数,若用正整数 $m$ 去除它们时余数相同,则称 $n, n^{\prime}$ \textbf{模 $m$ 同余},记作 $n \equiv n^{\prime}(\bmod m)$ 或 $n \equiv n^{\prime}(m)$ 。 
\end{definition} 
由于除数为 $m$ 时余数只能为 $\{0,1, \cdots, m-1\}$ 中的一个,我们把在除数 $m$ 时余数为 $r$ 的所有自然数构成的集合记作 $\{r\}_m$ ,称为模 $m$ 的\textbf{同余类 $\mathrm{Q}$ (或剩余类)},更数学一点说:同余关系是个等价关系(\autoref{def_Relat_1})。于是
\begin{equation}
\{r\}_m=r+m \mathbb{N}=\{r+m k \mid k \in \mathbb{N}\}~.
\end{equation}
这就相当于给整数集分了类:
\begin{equation}
\mathbb{Z}=\{0\}_m \cup\{1\}_m \cdots \cup\{m-1\}_m~.
\end{equation}

为了方便叙述,定义(或参见\autoref{def_ExDiv_1})
\begin{definition}{整除的记号}
若正整数 $m$ 整除整数 $n$ ,即用 $m$ 除 $n$ 余数为 0 ,则记 $m \mid n$.
\end{definition}
\textbf{同余性质1:} 若 $n \equiv n^{\prime}(m)$ 且 $n \geq n^{\prime}$ ,则 $m \mid\left(n-n^{\prime}\right)$.

这是因为模 $m$ 同余的两个自然数,它们余数相同,所以它们之差把余数给消掉了,剩下两个 $m$ 的倍 数之差。

\textbf{同余性质2:} 若 $k \equiv k^{\prime}(m)$ 且 $l \equiv l^{\prime}(m)$ ,则 $k+l=k^{\prime}+l^{\prime}(m), k l \equiv k^{\prime} l^{\prime}(m)$.

\textbf{证明:}因为 $k, k^{\prime}$ 余数相同, $l, l^{\prime}$ 余数也相同,那么 $k+l$ 和 $k^{\prime}+l^{\prime}$ 余数当然相同了。
更清晰点:
\begin{equation}
\begin{aligned}
& k \equiv k^{\prime}(m) \Rightarrow k=a_1 m+b, k^{\prime}=a_2 m+b~, \\
& l \equiv l^{\prime}(m) \Rightarrow l=n_1 m+r, l^{\prime}=n_2 m+r~.
\end{aligned}
\end{equation}
所以 $k+l=a_1 m+n_1 m+(b+r), \quad k^{\prime}+l^{\prime}=a_2 m+n_2 m+(b+r).$
所以 $k+l, k+l^{\prime}$ 被 $m$ 除的余数和 $b+r$ 被 $m$ 除的余数相同,所以 $k+l=k^{\prime}+l^{\prime}(m)$
,同样的 $k l \equiv k^{\prime} l^{\prime}(m)$ 。

\textbf{证毕!}

性质 $2$ 表明,同余类之间在加法和乘法之下的余数和集合内的元素没关系,随便取该类的任一元素和另一类的任一元素做加法和乘法,得到的自然数用 $m$ 除余数都一样。这就是如果我们只看自然数和某个正整数的余数 $m$ 的话,那只要把同余类都当作一个元素就好了,毕竟取类中哪个元素都没关系。既然同余类和其上元素没有关系,那么可把 $\{r\}_m$ 中的 $r$ 用任一数 $r+k m \quad(k \in \mathbb{N})$ 替换,即 $\{r\}_m=\{r+k m\}_m$ 。
现在把每个同余类 $\{r\}_m$ 都当成一个元素,或说一个单一的个体,为了强调这一点,直接记 $\bar{r}=\{r\}_m$ 那么新得到的集合:
\begin{equation}
\mathbb{Z}_m=\{\overline{0}, \overline{1}, \cdots, \overline{m-1}\}~.
\end{equation}
其上诱导了新的加运算 $\oplus$ 和积运算 $\otimes$ ,使得
\begin{equation}
\bar{k} \oplus \bar{l}=\overline{k+l}, \quad \bar{k} \otimes \bar{l}=\overline{k l}~.
\end{equation}
这一好处是,若注意力仅仅放在被 $m$ 除的余数上,那么考虑整数集和考虑 $\mathbb{Z}_m$ 是等价的,这样放弃上横线,取集合
$$
\{0,1 \cdots, m-1\}~.
$$
于是此时就只需考虑这个只含前 $m$ 个自然数的集合了,这集合称为\textbf{模 $m$ 的同余类的导出集}。
\subsection{剩余类环}
\begin{theorem}{剩余类环}\label{the_RRing_1}
$\{\mathbb Z_m,\oplus,\otimes\}$ 是个带有单位元 $\bar 1=1+m\mathbb Z$ 的交换环。
\end{theorem}
\begin{exercise}{}
证明\autoref{the_RRing_1}。
\end{exercise}