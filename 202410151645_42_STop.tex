% 共轭空间中的强拓扑
% keys 强拓扑|完备
% license Usr
% type Tutor
\pentry{共轭空间与代数共轭空间\nref{nod_ConSpa},线性算子的范数\nref{nod_ONorm}}{nod_0a2f}

\subsection{赋范空间的强拓扑}

由\autoref{ex_tvs_1} 可知,赋范空间是一个拓扑线性空间,因此其上自然由\enref{线性连续泛函}{LinCon}的定义。而赋范空间上根据
\begin{equation}\label{eq_STop_1}
\norm{f}:=\sup_{x\neq0}\frac{\abs{f(x)}}{\norm{x}}~
\end{equation}
可引入线性连续泛函的范数,证明\autoref{eq_STop_1} 满足\enref{范数}{norm}的定义和\autoref{ex_ONorm_1} 的证明完全一样。因此赋范空间的共轭空间可赋予赋范空间的自然结构。范数可以用来定义度量,度量有一个自然定义开集的方式,即在赋范空间上有一个自然定义的拓扑,在赋范空间的共轭空间中这样定义的拓扑就是强拓扑。
\begin{definition}{强拓扑}
设 $E$ 是赋范空间,则其\enref{共轭空间}{ConSpa} $E^*$ 上由\autoref{eq_STop_1} 定义的范数相应的拓扑称为 $E^*$ 的\textbf{强拓扑}。
\end{definition}

当希望把 $E^*$ 当作赋范空间时,我们将其写作 $(E^*,\norm{*})$。

\begin{theorem}{}
设 $E$ 是赋范空间,则 $(E^*,\norm{*})$ 是\enref{完备}{ComSpa}的。
\end{theorem}

\textbf{证明:}设 $\{f_n\}$ 是线性连续泛函的柯西序列(\autoref{def_ComSpa_1})。那么对任意 $\epsilon>0$,存在 $N$,使得对所有的 $n,m>N$ 有 $\norm{f_n-f_m}<\epsilon$。由此,对任意 $x\in E$ 得到
\begin{equation}\label{eq_STop_2}
\abs{f_n(x)-f_m(x)}\leq\norm{f_n-f_m}\cdot\norm{x}<\epsilon\norm{x},~
\end{equation}
即对任意 $x\in E$,数列 $\{f_n(x)\}$ 收敛($\norm{x}$ 是某一确定的实数,而 $\epsilon$ 任意)。

令 $f(x):=\lim\limits_{n\rightarrow\infty}f_n(x)$。则由
\begin{equation}
\begin{aligned}
f(\alpha x+\beta y)=&\lim_{n\rightarrow\infty}(\alpha x+\beta y)\\
=&\lim_{n\rightarrow\infty}\qty[\alpha f_n(x)+\beta f_n(y)]\\
=&\alpha f(x)+\beta f(y),
\end{aligned}~
\end{equation}
得 $f$ 是线性的。此外,取\autoref{eq_STop_2} 的 $m\rightarrow\infty$,则 
\begin{equation}\label{eq_STop_3}
\abs{f(x)-f_n(x)}\leq \epsilon\norm{x}.~
\end{equation}
由此得 $f-f_n$ 有界($\norm{x}$ 有限,$\epsilon>0$ 任意)。因而 $f=f_n+(f-f_n)$ 有界($\{f_n(x)\}$ 对任意 $x$ 收敛)。由\autoref{the_LinCon_2} , $f$ 连续。 另外,由\autoref{eq_STop_3} 得 $\norm{f-f_n}\leq\epsilon$,即 $\{f_n\}$ 收敛于 $f$。


\textbf{证毕!}

该定理表明,无论赋范空间是否完备,其共轭空间都完备。

\begin{theorem}{}
若赋范空间 $E$ 不完备,而 $\bar{E}$ 是 $E$ 的完备化空间,则 $E^*,\bar{E}^*$ 同构。
\end{theorem}



