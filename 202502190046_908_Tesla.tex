% 尼古拉·特斯拉(综述)
% license CCBYSA3
% type Wiki

本文根据 CC-BY-SA 协议转载翻译自维基百科\href{https://en.wikipedia.org/wiki/Nikola_Tesla}{相关文章}。


\begin{figure}[ht]
\centering
\includegraphics[width=6cm]{./figures/4da9af8e4f34bc2b.png}
\caption{} \label{fig_Tesla_1}
\end{figure}
尼古拉·特斯拉(/ˈnɪkələ ˈtɛslə/;塞尔维亚西里尔字母:Никола Тесла,[nǐkola têsla];1856年7月10日 – 1943年1月7日)是塞尔维亚裔美国工程师、未来学家和发明家。他以对现代交流电(AC)电力供应系统设计的贡献而闻名。

特斯拉出生并成长于奥斯曼帝国,在1870年代,他首先学习了工程学和物理学,但并未获得学位。随后,他在1880年代初期,在电话通信和大陆爱迪生公司(Continental Edison)新兴的电力行业中积累了实践经验。1884年,他移民到美国,并成为美国公民。他在纽约市的爱迪生机械厂工作了短暂时间后,便开始独立创业。在合作伙伴的帮助下,为了融资和推广自己的创意,特斯拉在纽约设立了实验室和公司,开发各种电气和机械设备。他的交流电感应电动机和相关的多相交流电专利,于1888年获得了西屋电气公司的许可,这使他赚得了可观的财富,并成为该公司最终推广的多相电系统的基石。

为了开发可以申请专利并商业化的发明,特斯拉进行了多种实验,包括机械振荡器/发电机、电气放电管和早期的X射线成像。他还制造了一艘无线控制的船,是最早展出的一批之一。特斯拉作为发明家广为人知,并在他的实验室向名人和富有的赞助人展示自己的成就,他的公共讲座也因其表演性质而备受关注。整个1890年代,特斯拉在纽约和科罗拉多斯普林斯进行高电压、高频率的电力实验,追求无线照明和全球无线电力传输的构想。1893年,他宣布了使用自己设备进行无线通信的可能性。特斯拉试图将这些想法付诸实践,通过未完成的沃登克利夫塔项目,这是一个跨洲的无线通信和电力传输塔,但在资金耗尽之前他未能完成该项目。

在沃登克利夫塔之后,特斯拉在1910年代和1920年代进行了一系列发明实验,取得了不同程度的成功。由于花费了大部分的钱,特斯拉在一系列纽约酒店中居住,并留下了未付的账单。他于1943年1月在纽约市去世。特斯拉的工作在他去世后逐渐被遗忘,直到1960年,国际计量大会将国际单位制(SI)中磁通密度的单位命名为“特斯拉”,以此向他致敬。自1990年代以来,特斯拉的公众兴趣重新兴起。
\subsection{早年时期}
\begin{figure}[ht]
\centering
\includegraphics[width=6cm]{./figures/4e958945d0c8af76.png}
\caption{特斯拉重建的出生地(教区大厅)和他父亲曾服务的教堂,位于克罗地亚的斯米扬。该地点已被改建为博物馆,以纪念他。} \label{fig_Tesla_2}
\end{figure}
尼古拉·特斯拉于1856年7月10日出生在奥斯曼帝国(今克罗地亚)军事边境的斯米扬村,来自一个塞尔维亚裔家庭。他的父亲米卢廷·特斯拉(1819–1879)是东正教的牧师。他父亲的兄弟约瑟夫是军事学院的讲师,撰写了几本数学教材。

特斯拉的母亲,乔治娜“杜卡”曼迪奇(1822–1892),她的父亲也是一位东正教牧师,具有制作家用工具和机械设备的天赋,并能背诵塞尔维亚史诗诗篇。杜卡从未接受过正式教育。特斯拉将自己的过目不忘的记忆力和创造力归功于母亲的遗传和影响。

特斯拉是五个孩子中的第四个。他有三个姐妹,分别是米尔卡、安杰丽娜和马里察,还有一个名叫丹尼的哥哥,他在特斯拉六七岁时因马术事故去世。1861年,特斯拉在斯米扬的初级学校上学,学习德语、算术和宗教。1862年,特斯拉一家搬到附近的戈斯皮奇镇,特斯拉的父亲在那儿担任教区牧师。尼古拉完成了小学学业后,继续上了中学。1870年,特斯拉搬到卡尔洛瓦茨[19],在那里他进入高等实科中学(Higher Real Gymnasium)上学,课堂上讲授的是德语,这是奥匈帝国军事边境地区学校的常规语言。[20][21] 后来,在申请专利时,在获得美国国籍之前,特斯拉会将自己标识为“来自奥匈帝国边界地区的斯米扬,利卡”。[22]
\begin{figure}[ht]
\centering
\includegraphics[width=6cm]{./figures/f94da1546174f1e0.png}
\caption{特斯拉的父亲米卢廷是斯米扬村的东正教牧师。} \label{fig_Tesla_3}
\end{figure}
特斯拉后来写道,他对物理教授展示的电学实验产生了兴趣。[a] 特斯拉指出,这些“神秘现象”的展示让他想“了解更多关于这种奇妙力量的知识”。[25] 特斯拉能够在脑海中进行积分计算,这使得他的老师们认为他在作弊。[26] 他在三年内完成了四年的学业,并于1873年毕业。[27]

毕业后,特斯拉返回了斯米扬,但很快感染了霍乱,卧床不起九个月,并多次濒临死亡。在绝望时刻,特斯拉的父亲(原本希望他进入神职)[28]承诺,如果他从病中恢复,将送他去最好的工程学院。[29] 特斯拉后来表示,在恢复期间,他读了马克·吐温的早期作品。[30][31]

次年,特斯拉通过逃避征召,避开了奥匈帝国军队的征兵,他从斯米扬逃到利卡东南部的托米纳吉,靠近格拉查茨。在那里,他穿着猎人的衣服探索山脉。特斯拉表示,这种与大自然的接触让他在身体和心理上都变得更强壮。1875年,特斯拉凭借军事边境奖学金入读格拉茨的帝国皇家技术学院。在那里,特斯拉通过了九门考试(几乎是所需考试数量的两倍[33]),并收到了技术系院长写给他父亲的表扬信,信中写道:“您的儿子是一级明星。”[33] 在格拉茨,特斯拉提到他对雅各布·佩什尔教授关于电学的详细讲座产生了浓厚的兴趣,并描述了他如何提出改进教授展示的电动机设计的建议。[30][需要更好的来源][34] 但到第三年时,他在学校成绩不佳,最终未能毕业,并于1878年12月离开了格拉茨。某位传记作家认为,特斯拉并没有认真学习,可能因为赌博和花心被开除。[35]
\begin{figure}[ht]
\centering
\includegraphics[width=6cm]{./figures/6c6ed982c821e12f.png}
\caption{大约在1879年,23岁的特斯拉} \label{fig_Tesla_4}
\end{figure}
特斯拉离开学校后,他的家人再也没有收到他的消息。[36] 他同学中有一个谣言,说他在附近的穆尔河里溺水身亡,但到了1月,其中一位同学在马里博尔镇遇到了特斯拉,并将这一遭遇报告给了特斯拉的家人。[37] 结果发现,特斯拉当时在那里做草图员,每月赚取60弗罗林。[35] 1879年3月,米卢廷终于找到了他的儿子,并试图说服他回家继续在布拉格接受教育。[37] 特斯拉在同月晚些时候回到了戈斯皮奇,但因没有居住许可被驱逐出境。[37] 特斯拉的父亲在次月去世,享年60岁,死于一种不明的疾病。[37] 在那一年剩余的时间里,特斯拉在戈斯皮奇的老学校教授了一个大班的学生。

1880年1月,特斯拉的两位叔叔凑够了足够的钱,帮助他离开戈斯皮奇前往布拉格,那里他计划继续学习。然而,他到达时已经错过了查理-费迪南大学的报名时间,而且他从未学习过希腊语,这是一门必修科目;同时,他也不懂捷克语,另一门必修科目。然而,特斯拉确实以听课生的身份参加了大学的哲学讲座,但他并未为这些课程获得成绩。[38][39]
\subsubsection{在布达佩斯电话交换公司工作}
特斯拉于1881年移居匈牙利布达佩斯,在蒂瓦达尔·普什卡什的领导下,加入了一家电报公司——布达佩斯电话交换公司。到达后,特斯拉发现该公司当时正在建设中,并未投入使用,因此他改为在中央电报局担任草图员。几个月后,布达佩斯电话交换公司开始投入使用,特斯拉被分配为首席电工。在任职期间,特斯拉对中央站的设备进行了许多改进,并声称自己完善了一种电话中继器或放大器,但这一发明从未获得专利,也未公开描述。[30]\subsection{在爱迪生公司工作}  
1882年,蒂瓦达尔·普什卡什为特斯拉在巴黎的大陆爱迪生公司(Continental Edison Company)找到了另一份工作。[40] 特斯拉开始从事当时全新的行业——在大规模电力公用事业中,安装室内白炽灯,覆盖整个城市。公司有多个子公司,特斯拉在位于巴黎伊夫里-苏尔-塞纳郊区的爱迪生电气公司(Société Electrique Edison)工作,负责安装照明系统。在那里,他获得了大量电气工程的实践经验。管理层注意到他在工程学和物理学方面的深厚知识,很快就让他设计和建造改进版的发电机和电动机。[41] 他们还派他去法国和德国其他爱迪生公用事业公司解决工程问题。

\begin{figure}[ht]
\centering
\includegraphics[width=6cm]{./figures/e2a69211c5447039.png}
\caption{} \label{fig_Tesla_5}
\end{figure}