% 卡迈克尔函数(综述)
% license CCBYSA3
% type Wiki

本文根据 CC-BY-SA 协议转载翻译自维基百科\href{https://en.wikipedia.org/wiki/Carmichael_function}{相关文章}。

在数论(数学的一个分支)中,正整数 $n$ 的卡迈克尔函数$\lambda(n)$ 定义为满足下列条件的最小正整数 $m$:
$$
a^{m} \equiv 1 \pmod{n}~
$$
其中 $a$ 为任意与 $n$ 互素的整数。从代数的角度来看,$\lambda(n)$ 是模 $n$ 的整数乘法群的指数。由于这是一个有限阿贝尔群,必然存在一个元素,其阶等于该指数 $\lambda(n)$。这样的元素被称为模 $n$ 的本原 $\lambda$-根(primitive $\lambda$-root modulo $n$)。
\begin{figure}[ht]
\centering
\includegraphics[width=14.25cm]{./figures/133eee57c8a8a0bb.png}
\caption{卡迈克尔 λ 函数:$1 \le n \le 1000$ 时的 $\lambda(n)$(与欧拉 φ 函数对比)} \label{fig_KMKRhs_1}
\end{figure}
卡迈克尔函数以美国数学家罗伯特·卡迈克尔的名字命名,他于 1910 年首次给出了这一函数的定义【1】。它也被称为卡迈克尔 $\lambda$ 函数、约化欧拉函数以及最小通用指数函数。

模 $n$ 的整数乘法群的阶是 $\varphi(n)$,其中 $\varphi$ 是**欧拉函数**。由于有限群中任一元素的阶都会整除群的阶,所以 $\lambda(n)$ 一定整除 $\varphi(n)$。

下表比较了 $\lambda(n)$(OEIS 数列 A002322)与 $\varphi(n)$ 的前 36 个值(当二者不同的时候,$\varphi(n)$ 用粗体表示;使它们不同的那些 $n$ 的值列在 OEIS 数列 A033949 中)。
\begin{figure}[ht]
\centering
\includegraphics[width=14.25cm]{./figures/309a6eadd8675298.png}
\caption{} \label{fig_KMKRhs_2}
\end{figure}
\subsection{数值例子}
\begin{itemize}
\item $ n = 5$小于且与 5 互素的数集合是 $\{1, 2, 3, 4\}$。因此,欧拉函数的值为 $\varphi(5) = 4$,而卡迈克尔函数 $\lambda(5)$ 的值必须是 4 的约数。约数 1 不满足卡迈克尔函数的定义,因为除了$a \equiv 1 \pmod{5}$之外,$a^1 \not\equiv 1 \pmod{5}$。约数 2 也不行,因为$2^2 \equiv 3^2 \equiv 4 \not\equiv 1 \pmod{5}$。因此,$\lambda(5) = 4$。实际上:$1^4 \equiv 2^4 \equiv 3^4 \equiv 4^4 \equiv 1 \pmod{5}$。其中 2 和 3 是模 5 的本原 $\lambda$-根,同时它们也是模 5 的本原根。
\item $ n = 8$小于且与 8 互素的数集合是 $\{1, 3, 5, 7\}$。因此 $\varphi(8) = 4$,而 $\lambda(8)$ 必须是 4 的约数。事实上,$\lambda(8) = 2$,因为:$
1^2 \equiv 3^2 \equiv 5^2 \equiv 7^2 \equiv 1 \pmod{8}$。模 8 的本原 $\lambda$-根是 3、5 和 7,但模 8 没有本原根。
\end{itemize}
\subsection{λ(n) 的递推公式}
质数幂的卡迈克尔$\lambda$函数可以用欧拉函数表示。任何不是 1 且不是质数幂的整数,都可以唯一地分解为不同质数幂的乘积,此时该数的$\lambda$值等于这些质数幂的 $\lambda$值的最小公倍数。具体来说,$\lambda(n)$ 由以下递推式给出:
$$
\lambda(n) =
\begin{cases}
\varphi(n), & \text{如果 } n \text{ 是 1、2、4,或奇质数幂}, \\[6pt]
\dfrac{1}{2} \varphi(n), & \text{如果 } n = 2^{r},\ r \geq 3, \\[8pt]
\operatorname{lcm}\bigl(\lambda(n_{1}),\lambda(n_{2}),\dots,\lambda(n_{k})\bigr), & \text{如果 } n = n_{1} n_{2} \dots n_{k}, \ \text{且 } n_{1},n_{2},\dots,n_{k} \text{ 是不同质数的幂}.
\end{cases}~
$$
其中,质数幂 $p^r$($p$ 为质数且 $r \geq 1$)的欧拉函数为:
$$
\varphi(p^{r}) = p^{r-1}(p - 1)~
$$
