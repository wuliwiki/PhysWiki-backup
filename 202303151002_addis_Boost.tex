% C++ Boost 库笔记

\pentry{C++ 基础\upref{Cpp0}}

Boost 是一些 C++ 库, 是 C++ 最常用的库之一。

\subsubsection{安装}
Ubuntu/Debian 可以直接 \verb|sudo apt install libboost-all-dev| 或者指定版本号如 \verb|libboost1.65-all-dev|。 注意这个安装非常大, 建议安装某个具体的 lib, 如 \verb|libboost-filesystem-dev| (输入 \verb|sudo apt install libboost-| 然后按 Tab 就可以显示所有子 lib)。

但如果想获取最新版本可以直接\href{https://www.boost.org/users/history/version_1_76_0.html}{下载}。 Boost 绝大部分库都可以是 header only 的。 用 \verb|tar -xzvf boost_1_76_0.tar.gz| 解压以后, 在解压文件夹中运行 \verb|sudo ./bootstrap.sh| (可以添加一些选项, 见\href{https://www.boost.org/doc/libs/1_66_0/more/getting_started/unix-variants.html#easy-build-and-install}{这里}。), 然后 \verb|sudo ./b2 install [--with-libraries=filesystem] [-j12]| 即可(默认需要 \verb|gcc| 编译器, 注意解压和编译时间可能较长)。 如果只需要头文件, 用 \verb|./b2 headers|。 然后把 \verb|boost| 文件夹复制到任何目录, 编译时 \verb|-I 目录/boost| 即可。

boost 的一些模块需要依赖第三方库, 如果没有这些库, \verb|b2| 就会跳过, 最后显示 \verb|..failed updating 54 targets... skipped 10 targets... updated 18065 targets...| 然后返回 exit code 1。 但这并不影响其他库, 详见\href{https://stackoverflow.com/questions/12906829/failed-updating-58-targets-when-trying-to-build-boost-what-happened}{这里}。

头文件默认安装路径是 \verb|/usr/local/include/boost/|, 二进制文件在子目录 \verb|/usr/local/include/lib/| 如果没有 sudo 权限也可以指定安装目录, 详情参考\href{https://www.boost.org/doc/libs/1_76_0/more/getting_started/unix-variants.html#easy-build-and-install}{安装说明}。

\subsubsection{数学物理相关}
\begin{itemize}
\item Safe Numerics (带有溢出检测的整数以及运算)
\item Rational (有理数)
\item Math Quaternion (四元数\upref{Quat})
\item Math/Special Functions
\item Geometry (多边形,平面几何,坐标旋转,非欧几何等)
\item uBLAS (线性代数库,定义容器类)
\item Units (量纲分析)
\item Random (随机数)
\item \href{https://www.boost.org/doc/libs/1_72_0/libs/multiprecision/doc/html/index.html}{Multiprecision} (任意精度计算)
\item Math/Statistical Distributions (统计)
\item Odeint (常微分方程)
\end{itemize}

\subsubsection{其他}
\begin{itemize}
\item Sort (排序)
\item Chrono (时间工具)
\item Date Time (时间工具)
\item Generic Image Library(读写图片,基本色彩,变形, 像素处理)
\item Python (C++ 和 Python 之间的接口)
\item Regex (正则表达式)
\item Serialization (把数据类型保存文件)
\item Stacktrace (记录函数调用顺序)
\item MPI (多机器并行计算)
\item JSON (读写 json 格式文件)
\item Filesystem (文件处理)
\item Compute (GPU 并行运算库)
\end{itemize}


\subsection{filesystem 3}
一个获取文件大小的程序(同时显示 boost 版本号)
\begin{lstlisting}[language=cpp, caption=test\_filesystem3.cpp]
#include <iostream>
#include <boost/filesystem.hpp>
using namespace std;
using namespace boost::filesystem;

int main(int argc, char* argv[])
{
	int major_ver = BOOST_VERSION / 100000,
        minor_ver = (BOOST_VERSION / 100) % 1000,
		sub_minor_ver = BOOST_VERSION % 100;
	cout << "boost version: " << major_ver << "." << minor_ver << "."
        << sub_minor_ver << endl;
	cout << "file size: " << file_size(argv[1]) << " bytes" << endl;
	return 0;
}
\end{lstlisting}

编译 \verb|g++ test_filesystem3.cpp -o filesize -lboost_system -lboost_filesystem|, 运行 \verb|./filesize filesize|。 运行结果:
\begin{lstlisting}[language=none]
boost version: 1.65.1
file size: 26800 bytes
\end{lstlisting}

\begin{itemize}
\item \verb|boost::filesystem::path|
\end{itemize}
