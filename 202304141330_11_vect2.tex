% 四维矢量
% keys 相对论
\subsection{概念}
设$K$与$K'$为两个坐标原点重合的惯性系。在开始计时后,$K'$相对$K$有沿着$x$轴的相对速度。从$0$时刻开始,光信号沿着x轴运动,设其在两个参考系的时空坐标分别为$(t,x,y,z)$与$(t',x',y',z')$,由光速不变,我们有:

$$ds^2=ct^2-x^2-y^2-z^2=ct'^2-x'^2-y'^2-z'^2$$

从直观上,这看起来是坐标矢量$(t,x,y,z)$“长度”的平方不随惯性系的改变而改变。准确来说,是其逆变矢量和协变矢量的对偶内积。形式化这个运算,该闵氏时空下的度规为$\eta_{\mu\nu}=diag(+1,-1,-1,-1) $ ,设光速为1,我们有
\begin{equation}\label{eq_vect2_1}
x^\mu x_\mu =\eta_{\mu\nu}x^\mu x^\nu=\eta_{\rho \sigma}x'^\rho x'^\sigma   
\end{equation}

这意味着改变惯性系相当于对原坐标矢量进行保距变换,即正交线性变换,我们把这个正交线性变换称之为洛伦兹变换。设洛伦兹变换矩阵为$\Lambda$,在矩阵下,这个线性变换表示为:$x^T x=x^T\Lambda^{-1}\Lambda x$,通常用$\Lambda^\mu_\nu $的形式表示矩阵,配合指标表示法进行运算。那么把该矩阵回代\autoref{eq_vect2_1} 我们有
\begin{equation}\label{eq_vect2_2}
\Lambda^\rho_\nu x^\nu=x^\rho
\end{equation}
\begin{equation}
\eta_{\mu\nu}=\eta_{\rho \sigma}\Lambda^\rho_\nu \Lambda^\sigma_\mu 
\end{equation}

利用坐标矢量的洛伦兹变换,我们可以把很多物理量扩展成四维形式。
\begin{exercise}{力学实例}
利用洛伦兹变换,构建内积不变的速度与动量。
提示:利用标量是洛伦兹不变的,矢量与标量相乘。
\end{exercise}
\begin{exercise}{梯度算符}
证明:梯度算符(分量为$\partial_\mu=\frac{\partial}{\partial x^\mu}$)在洛伦兹变换下恰似一协变矢量

如无特别说明,一般坐标矢量都是指逆变矢量,即主空间向量,指标在上侧(可视为列向量,则相应的,协变矢量是行向量)。由\autoref{eq_vect2_2} 我们得:
\begin{equation}
\Lambda^\mu_\nu=\frac{\partial x'^\mu}{\partial x^\nu}
\end{equation}
相应的,我们有
\begin{equation}
(\Lambda^{-1})^\mu_\nu=\frac{\partial x^\mu }{\partial x'_\nu}
\end{equation}
因此,通过链式法则可求证。
\end{exercise}


\subsection{电动力学实例}
\subsubsection{电流密度}
任一区域内的电荷总量不随参考系的改变而改变。那么,我们写下电荷守恒定律的微分形式
\begin{equation}
\frac{\partial\rho}{\partial t}+\nabla \cdot J=0
\end{equation}
通过习题2,我们可以知道,如果把电流密度扩展为四矢量(\rho,\vec J),那么上式可以写为
\begin{equation}
\partial_\mu J_\mu=0
\end{equation}
显然,这是一个洛伦兹协变方程。但这样的扩展到底是否可行呢?我们需要检验电流密度是否满足洛伦兹变换。设带电粒子速度为$v$,$\rho$为该区域的电荷密度(以粒子为参考系),则$\vec J=\rho \vec v$。转换惯性系时,
\subsubsection{矢势}
\subsection{拓展:洛伦兹张量}
我们知道,张量实际上是多重线性映射,而洛伦兹张量则默认了基变换的过渡矩阵为洛伦兹矩阵。以二阶张量$F^{\mu\nu} $为例,
\begin{equation}
F'^{\mu\nu}=\Lambda_\rho^\mu \Lambda_\sigma^\nu F^{\rho\sigma}  
\end{equation}
\begin{equation}
F'_{\mu\nu}=(\Lambda^{-1})_\mu^\rho (\Lambda^{-1})_\nu^\sigma F_{\rho\sigma}
\end{equation}


