% 南京理工大学 2009 量子真题
% license Usr
% type Note

\textbf{声明}:“该内容来源于网络公开资料,不保证真实性,如有侵权请联系管理员”

请考生在下列13题中选作10题,每题15分,满分150分
\subsection{简要回答下列问题}
\begin{enumerate}
\item 最子力学中角动量是如何定义的?地球自转是否与量子力学中的自概念相对应?
\item 玻恩近似法的基本思想是什么?
\item 如果有心力场不是库仑场(即$V(r)$不与$\frac{1}{r}$成比例),则角分布函数将取什么形式?
\item 如何理解波函数必须满足的标准条件?
\item 在什么情况下力学量的测量具有确定值?两个不对易的力学量是否一定不能同时其有确定的测量值?
\end{enumerate}
二、定义 $[\hat{A}, \hat{B}]_{+} = \hat{A}\hat{B} + \hat{B}\hat{A}$ (反对易式),已知 $\hat{a}, \hat{b}$ 均与 $\hat{A}, \hat{B}$ 对易,证明:

\begin{enumerate}
    \item 
    \[
    [\hat{A}, \hat{B}\hat{C}] = \hat{A}[\hat{B}, \hat{C}]_{+} - [\hat{B}, \hat{C}]_{+}\hat{A} + \hat{C}[\hat{A}, \hat{B}]_{+} - [\hat{A}, \hat{C}]_{+}\hat{B}~
    \]
    \item 
    \[
    [\hat{a}\hat{A}, \hat{b}\hat{B}]_{+} = \frac{1}{2}[\hat{a}, \hat{b}]_{+}[\hat{A}, \hat{B}] + \frac{1}{2}[\hat{a}, \hat{b}][\hat{A}, \hat{B}]_{+}~
    \]
\end{enumerate}
三、计算受力 $F = -kx + k_e (k = m\omega^2)$ 作用的一个粒子的波函数和能量允许值。\\
四、已知氢原子的电子波函数数为:
\[
\psi_{n l m_l m_s}(r,\theta,\phi,s_z) = \frac{1}{\sqrt{4}} R_{31}(r) Y_{11}(\theta, \phi)\chi_{1/2}(s_z) + \sqrt{\frac{3}{4}} R_{32}(r) Y_{20}(\theta, \phi)\chi_{-1/2}(s_z)~
\]
求在 $\psi$ 态中测量氢原子能量 $E$、$L^2$、$L_z$、$s^2$、$s_z$ 的可能值和这些力学量的平均值。\\
五、一维运动的粒子处于状态 $\psi(x) = 
\begin{cases} 
    Axe^{-\lambda x}, & x \geq 0 \\
    0, & x < 0 
\end{cases}
$ 其中 $\lambda > 0, A$ 为待求的归一化常数。

\begin{enumerate}
    \item 求归一化常数;
    \item 求粒子坐标的平均值和粒子坐标平方的平均值。
    \item 粒子动量的平均值和粒子动量平方的乎均值。
\end{enumerate}\\
六、设粒子处在宽度为a的非对称一维无限深势阱中(坐标原点取在势阱左侧阱壁处),求能量表象中粒子坐标和动量的矩阵表示。\\
七、质量为 $m$ 的粒子,在非对称一维无限深势阱中运动,若 $t=0$ 时,粒子处于
\[
\psi(x,0) = \sqrt{\frac{1}{2}} \varphi_1(x) - \sqrt{\frac{1}{3}} \varphi_2(x) + \frac{1}{2} \varphi_3(x)~
\]
状态上,其中,$\varphi_n(x) = \sqrt{\frac{2}{a}} \sin \frac{n\pi x}{a}$ 为粒子的第 $n$ 个能量本征态。

\begin{enumerate}
    \item 求 $t=0$ 时能量的可能值与相应的取值概率;
    \item 求 $t>0$ 时的波函数 $\psi(x,t)$ 及能量的可能值与相应的取值概率。
\end{enumerate}\\
八、厄密算符 $\hat{A}$ 与 $\hat{B}$,满足 $\hat{A}^2 = \hat{B}^2 = 1$ 和 $\hat{A}\hat{B} + \hat{B}\hat{A} = 0$,求:在 $\hat{A}$ 表象中,$\hat{A}$ 与 $\hat{B}$ 的矩阵表示形式。\\
九、已知电子自旋在空间任一方向上的投影只有两个可能取值:$\pm \frac{\hbar}{2}$,试求电子自旋在空间任意方向 $\vec{n}$ 上的投影 $\hat S_n = \hat{\vec{S}} \cdot \vec{n} =\hat S_x \cos \alpha + \hat S_y \cos \beta + \hat S_z \cos \gamma$ 的归一化本征矢量。设单位矢量 $\vec{n}$ 的方向余弦为 $(\cos \alpha, \cos \beta, \cos \gamma)$。\\
十、设粒子被限制在矩形匣子中运动,即 $V(x,y,z) = 
\begin{cases} 
0, & 0 < x < a, \, 0 < y < b, \, 0 < z < c \\
\infty, & \text{其他区域}
\end{cases}$,
求粒子的能量本征值和本征波函数,当$a=b=c$时,讨论能量的简并情况。\\
十一、有一定域电子(作为近似模型,可以不考虑轨道运动)受到均匀磁场 $\vec{B}$ 的作用,磁场 $\vec{B}$ 指向 $x$ 轴正方向,磁相互作用为 $\hat{H} = \frac{eB}{\mu c} \hat{s}_x = \frac{eB\hbar}{2\mu c} \hat{\sigma}_x$。设 $t=0$ 时,电子的自旋向上,即 $S_z = \frac{\hbar}{2}$。求 $t=0$ 时 $\hat S_x, \hat S_y, \hat S_z$ 的平均值。\\
十二、设 $\hat{K} = \hat{L}\hat{M},\hat{L}\hat{M} - \hat{M}\hat{L}=1$,$\phi$ 为 $\hat{K}$ 的本征矢,即 $\hat{K}\phi = \lambda \phi$,$\lambda$为本征值。试证明 $\mu = \hat{L}\phi, \nu = \hat{M}\phi$ 也是 $\hat{K}$ 的本征矢,相应的本征值分别为 $\lambda - 1, \lambda + 1$。\\
十三、设在 $H^0$ 表象中,$\hat{H}$ 的矩阵为:
\[
\hat{H} = \begin{pmatrix}
E_1^{(0)} & 0 & a \\
0 & E_2^{(0)} & 0 \\
a^\prime & 0^\prime & E_3^{(0)}
\end{pmatrix} \quad E_1^{(0)} < E_2^{(0)} < E_3^{(0)}~
\]
试用微扰论求能量的二级修正。
