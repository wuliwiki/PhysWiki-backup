% 向量与矩阵导航
% 线性代数|矢量|几何矢量|矢量函数|矢量空间|线性组合|数乘|内积|叉乘|正交|归一化|线性相关|线性无关|基底|映射|线性变换|矩阵

\begin{issues}
\issueDraft
\end{issues}

\subsection{几何矢量}
高中数学和物理中最熟悉的矢量就是\textbf{几何矢量},这里先回顾几何矢量(以下简称矢量)并引入一些新的概念.

矢量的存在与坐标系无关,可以将其想象成空间中的一些有长度有方向的箭头.我们对它的位置不感兴趣,所有长度和方向相同的矢量都视为同一矢量.对于讨论问题的不同,我们有时仅需要处于同一平面(\textbf{二维空间})的所有矢量,有时需要\textbf{三维空间}中的所有矢量,最简单的情况下只需要沿某条线(\textbf{一维空间})的所有矢量(这时我们可以规定一个正方向,且仅使用矢量的模长加正负号来表示矢量以简化书写).

矢量的一些基本运算\upref{GVec} 同样不需要有任何坐标系的概念,\textbf{矢量相加}按照三角形法则或平行四边形法则即可.
\textbf{矢量数乘}就是把矢量的模长乘以一个实数,若乘以正数,方向不变,若乘以负数,取相反方向. \textbf{矢量的线性组合}是把若干矢量分别乘以一个实数再相加得到新的矢量.

矢量的\textbf{内积}\upref{Dot}等于一个矢量在另一个矢量上的投影长度乘以另一个矢量的模长得到一个实数,矢量的\textbf{模长}等于矢量与自身内积再开方,把矢量除以自身模长使模长变为单位长度的过程叫做\textbf{归一化}.若两矢量内积为零,这两个矢量相互\textbf{正交}\footnote{对于几何矢量,正交就是方向垂直,不加区分.}.

三维欧几里得空间中,两矢量\textbf{叉乘}\upref{Cross}得到的矢量垂直于两矢量,模长为一个矢量在另一个矢量垂直方向的投影长度乘以另一个矢量的模长.

为了方便描述矢量之间的关系,我们选取一些\textbf{线性无关}的矢量作为所有几何矢量的\textbf{基底},使空间中的任何矢量可以用这些基底的唯一一种线性组合来表示,$N$ 维空间需要 $N$ 个基底.一般来说,基底不必互相正交.我们先把这些基底排序,任意矢量表示成它们的线性组合时,把式中的 $N$ 个系数按照顺序排列,就是该矢量的\textbf{坐标},通常用列矢量表示.由于线性组合的唯一性,每个矢量的坐标是唯一的.

为了方便计算任意矢量的坐标,往往取\textbf{正交归一}的基底\upref{OrNrB}(所有基底模长为1,任意两基底互相正交).这样,任意矢量的坐标都可以通过与基底的内积得到.

\subsection{线性变换}

我们可以设计一种规则把某个空间的任意矢量对应(\textbf{映射})到另一个矢量,叫做\textbf{变换}\footnote{更广义地,变换可以在不同的空间中进行,例如把一个三维空间中的矢量映射到一个二维空间中的矢量}.如果对于某个变换,任意矢量线性组合的变换等于这些矢量分别进行该变换再线性组合,这个变换就是\textbf{线性变换}\upref{LTrans}.

\subsection{矩阵}

\textbf{矩阵}\upref{Mat},是把标量排成矩形所得到的数学对象.在选定某组基底之后,矢量的线性变换可以用其坐标的线性变换表示,并且可以写成矩阵与坐标列矢量相乘的形式.

% 词条中要讲逆变换和逆矩阵,这里就算了.

\textbf{旋转矩阵}\upref{Rot2D}可以有两种理解,一是矢量绕某个轴相对于当前的正交归一基底转动,其坐标产生了变换,二是矢量本身没有变,只是其坐标在两个不同的正交归一基底中不同.这种矩阵的特点是所有列(行)矢量都正交归一,所以叫做\textbf{单位正交阵}.单位正交阵的特点是逆矩阵等于转置矩阵.

% Giacomo: 应当移动到其他地方,比如多元微积分导航
% \subsection{矢量微积分}
% $N$ 维矢量可以作为一个或多个标量的函数(\textbf{矢量函数}),可以看成是 $N$ 个普通函数与矢量基底的数乘.矢量函数同样可以对其自变量求导(或求偏导),也可以积分.不同的是,矢量函数还可以进行曲线积分和面积分 %未完成

%介绍几何矢量其实是一种抽象的存在,并不需要坐标的辅助就可以定义相加,数乘,内积,叉乘等运算

%然后引入基底的概念,尤其是正交基的概念.然后便是基底转换了.


% 经研究,力学分册绝对不需要用矢量空间的概念! 基底,线性组合,线性变换,坐标,等等所有概念都可以通过几何矢量讲解! 唯一可能用到其他矢量空间的可能性就是傅里叶级数而已!而且傅里叶级数在力学分册中也不会用到啊.
% 正交变换的最高目标估计就是讲明吧旋转矩阵了.
% 讲讲线性方程组的解还是有必要的.
% 要添加矩阵求导法则.

\subsection{线性方程组}

(抽象的)向量和矩阵可以方便地被用来解\textbf{线性方程组}\upref{LinEqu}.
