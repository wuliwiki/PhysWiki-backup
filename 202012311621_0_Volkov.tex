% Volkov 波函数
% 波函数|本征波函数|偶极子近似|速度规范|库仑势

\begin{issues}
\issueTODO
\end{issues}

\pentry{加速度规范\upref{AccGau}}

本文使用原子单位制\upref{AU}. 若初始时, 一个自由粒子的波函数是平面波, 波数为 $\bvec k$. 接下来空间中出现了随时间变化的电场 $\bvec {\mathcal E}(t)$ (我们采用偶极子近似\upref{DipApr}, 即假设电场不随位置变化, 感生磁场可以忽略), 那么我们把该电场中的含时波函数称为 \textbf{Volkov 波函数}.

以下我们分别在长度、速度、加速度规范中求解 Volkov 波函数, 初始时 $\bvec E(-\infty) = \bvec A(-\infty) = \bvec 0$. 三种规范下的初始波函数都是
\begin{equation}\label{Volkov_eq3}
\Psi_{\bvec k}(\bvec r, t) = (2\pi)^{-3/2} \exp\qty(\I\bvec k \vdot \bvec r - \I E_0 t)
\end{equation}
其中 $E_0 = \bvec k^2/(2m)$.

\subsection{加速度规范}
求解 Volkov 波函数最容易的方法就是使用加速度规范\upref{AccGau}, 由于空间中没有净电荷, 薛定谔方程(\autoref{AccGau_eq4}~\upref{AccGau})中 $V(\bvec r) = 0$
\begin{equation}
\frac{\bvec p^2}{2m}\Psi_{\bvec k}^A = \I \pdv{t} \Psi_{\bvec k}^A
\end{equation}
显然\autoref{Volkov_eq3} 就是该方程的解
\begin{equation}\label{Volkov_eq2}
\Psi_{\bvec k}^A(\bvec r, t) = (2\pi)^{-3/2} \exp\qty(\I\bvec k \vdot \bvec r - \I E_0 t)
\end{equation}
可见在 K-H 参考系中, 波函数始终保持平面波的形式.

\begin{example}{高斯波包与电磁场}
在 K-H 参考系中, 若使用偶极子近似且令势能函数 $V(\bvec r) = 0$, 我们会发现高斯始终是高斯波包. 电磁波消失以后, K-H 系变为原来的惯性系, 这是因为电磁波包不能含有直流分量. (未完成:讲详细点?)
\end{example}

\subsection{速度规范}
使用\autoref{AccGau_eq1}~\upref{AccGau}对\autoref{Volkov_eq2} 做规范变换, 得速度规范\upref{LVgaug}下的 Volkov 波函数为
\begin{equation}\label{Volkov_eq1}
\Psi_{\bvec k}^V(\bvec r, t) = (2\pi)^{-3/2} \exp\qty{\I \bvec k \vdot [\bvec r - \bvec \alpha(t)] - \I E_0 t}
\end{equation}
其中 $\bvec \alpha(t)$ 对应的是一个经典点电荷在电场中的位移(\autoref{AccGau_eq3}~\upref{AccGau})
\begin{equation}
\bvec \alpha(t) = -\frac{q}{m} \int_{-\infty}^t \bvec A(t') \dd{t'}
\end{equation}
这是以下薛定谔方程的解(见\autoref{LVgaug_eq7}~\upref{LVgaug})
\begin{equation}
\I \pdv{t} \Psi^V = \qty[\frac{\bvec p^2}{2m} - \frac{q}{m}\bvec A(t) \vdot \bvec p] \Psi^V
\end{equation}
注意 $E_0$ 只是初始时的动能, 电场中的粒子能量不守恒.
\addTODO{任意时刻的波函数是否是动能的本征矢? 动能是多少?}

\subsection{长度规范}
要求长度规范长度规范\upref{LenGau}下的 Volkov 波函数只需要对\autoref{Volkov_eq1} 再次做规范变换即可(\autoref{LVgaug_eq6}~\upref{LVgaug}), 得
\begin{equation}
\Psi_{\bvec k}^L(\bvec r, t) = (2\pi)^{-3/2} \exp\qty[\bvec p(t)\vdot \bvec r - \frac{1}{2}\int_{-\infty}^t \bvec p(t')^2 \dd{t'}]
\end{equation}
其中 $\bvec p(t)$ 是库仑规范或速度或中的广义动量(\autoref{QMEM_eq6}~\upref{QMEM})
\begin{equation}
\bvec p(t) = \bvec k + q\bvec A(t)
\end{equation}
注意这里的 $\bvec A(t)$ 是库仑规范或速度规范中的矢势(同上文)而不是长度规范中的. 长度规范中 $\bvec A(t)$ 恒为零(\autoref{LenGau_eq4}~\upref{LenGau}).
