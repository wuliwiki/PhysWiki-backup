% 小时百科符号与规范
% 符号|规范|单位制

这里列出本书中一些可能会产生歧义的符号以及规范, 并给出对应词条的链接, 如需补充, 请与管理员协商.

\subsection{数学通用}
\begin{itemize}
\item 定义、定理等环境中的内容必须是严谨的, 否则不使用. “证明” 也必须是严谨的, 否则用 “不严谨的证明”.
\item 较长的整数可以使用逗号对每三位进行分割, 如 $2,456,789$.
\item 集合\upref{Set} 中 $\subseteq$ 和 $\subset$ 表示子集, $\subsetneq$ 表示真子集, 但应该尽量避免使用 $\subset$.
\item 映射\upref{map} 的分类使用\autoref{map_def1}~\upref{map}.
\item 虚数单位\upref{CplxNo} 用正体的 $\I$.
\item 四元数中的三个虚数单位用 $\I, \mathrm{j}, \mathrm{k}$.
\item 自然对数底\upref{E} 用正体的 $\E$.
\end{itemize}

\subsection{线性代数}
\begin{itemize}
\item 同义词: “矢量”、 “向量”\upref{GVec}; “矢量空间”、 “线性空间”\upref{LSpace}.
\item 同义词: “线性映射”、 “线性变换”\upref{LinMap}.
\item 使用粗体与正体表示矩阵\upref{Mat}, 矩阵一般用大写(如 $\mat M$). 特殊地, 列矩阵可以用小写(如线性方程组记为 $\mat A \bvec x = \bvec b$). 一个例外是当矩阵符号是希腊字母时, 使用粗斜体而不是粗正体(如 $\bvec \beta$). 这与 Wikipedia 的规范一致(\href{https://en.wikipedia.org/wiki/Angular_acceleration}{例子}).
\item 由于习惯原因, 在不需要与列矩阵区分的情况下几何矢量\upref{GVec}也可以采用粗体加正体(大小写均可, 如 $\bvec a, \bvec R$).
\item 单位几何矢量在粗体与正体矢量上方加 hat 表示, 如 $\uvec x$.
\item 几何矢量的点成(内积)记为 $\bvec u \vdot \bvec v$, 叉乘记为 $\bvec u \cross \bvec v$.
\item 矢量的坐标表示为实数或复数的笛卡尔积\autoref{Set_eq1}~\upref{Set}(如 $\mathbb R^N, \mathbb C^N$ 或 $(c_1, c_2, c_3)$), 也可以表示为列矩阵(如 $\bvec c, \bvec x$)以便和矩阵相乘.
\item 当需要区分矢量和坐标的不同时, 矢量空间中的矢量不可以使用粗体正体, 而是直接用普通变量字母(如 $x, y, u, v$), 这时内积\upref{InerPd}可以记为 $\ev{u, v}$.
\item 也可以用狄拉克符号表示任意矢量空间的矢量, 如 $\ket{v}$, 对偶矢量如 $\bra{v}$, 内积如 $\braket{u}{v}$.
\end{itemize}

\subsection{数学分析}
\begin{itemize}
\item 定义度量空间中序列\textbf{收敛}当且仅当它是柯西序列(\autoref{cauchy_def1}~\upref{cauchy}).
\end{itemize}

\subsection{微分几何}
\begin{itemize}
\item 同义词: “逆变向量(contravariant vectors)”、“切向量(tangent vectors)”、 “$(1, 0)$型张量”
\item 同义词: “协变向量(covariant vectors)”、“余切向量(cotangent vectors)”、“切向量的对偶向量(dual vectors)”、 “线性映射”、 “线性1-形式(linear 1-forms)”、 “$(0, 1)$型张量”
\item 同义词: “张量(tensors)”、“多重线性映射(multilinear maps)”
\end{itemize}

\subsection{物理通用}
\begin{itemize}
\item 如无声明词条默认使用国际单位制\upref{Consts}, 若使用其他单位制, 要在每个词条开头用脚注声明 “本文使用 xx 单位制”. 例如原子单位\upref{AU}, 厘米—克—秒\upref{CGS}, 或高斯单位制\upref{GaussU}.
\item 若要区分能量和电场, 用 $E$ 表示能量, $\mathcal E$ 表示电场.
\end{itemize}

\subsection{电动力学}
\begin{itemize}
\item 使用 $\epsilon$ 而不是 $\varepsilon$ 表示电介质常量.
\end{itemize}

\subsection{相对论}
\begin{itemize}
\item 用 $\gamma$ 表示 $1/\sqrt{1 - v^2/c^2}$.
\item 用 $\beta$ 表示 $v/c$.
\end{itemize}

\subsection{统计力学}
\begin{itemize}
\item $\Xi$ 表示配分函数.
\item $k_B$ 表示玻尔兹曼常数.
\item 如果温度使用能量单位, 意思是该温度(开尔文温标)乘以玻尔兹曼常数 $k_B T$.
\end{itemize}
