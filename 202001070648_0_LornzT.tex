% 洛伦兹变换
% keys 光速不变|洛伦兹变换|狭义相对论|四维时空
% 参考 docx 版物理百科

\pentry{相对论简介, 伽利略变换} % 未完成

\subsection{相对论的基本假设}
\begin{enumerate}
\item 相对性原理——任何惯性系中,物理定律及相同实验的结果都相同
\item 光速不变原理——任何惯性系中,光速都不改变
\end{enumerate}

设 $t = 0$ 时空间直角坐标系 $S$ 与 $S'$ 系重合. $S'$ 系相对 $S$ 系沿 $x$ 轴方向以速度 $v$ 匀速运动.

洛伦兹变换的假设是,在 $S$ 系中的坐标 $(x, y, z, t)$ 一一对应到 $S'$ 系中的坐标 $(x', y', z', t')$. 变换一定是线性变换,因为时间和空间是均匀的, 这也叫物理定律的平移对称及时间对称.

\subsection{洛伦兹变换}
\begin{equation}
\leftgroup{
&x' = \frac{x - vt}{\sqrt{1 - v^2/c^2}}\\
&t' = \frac{t - vx/c^2}{\sqrt{1 - v^2/c^2}}
}
\qquad
\leftgroup{
&x = \frac{x' + vt'}{\sqrt{1 - v^2/c^2}}\\
&t = \frac{t' + vx'/c^2}{\sqrt{1 - v^2/c^2}}
}
\end{equation}

\subsection{推导}
先考虑 $x$ 轴的一维情况, 令线性变换为
\begin{equation}
\leftgroup{
x' &= ax + bt\\
t' &= mx + nt
}
\end{equation}
逆变换为
\begin{equation}
\leftgroup{
x &= \frac{nx' - bt'}{an - bm}\\
t &= \frac{mx' - at'}{bm - an}
}
\end{equation}
由相对性原理, 正变换和逆变换必须完全相同, 对比系数并化简得
\begin{equation}
\leftgroup{
&a = -n\\
&bm - an = 1
}
\end{equation}
由光速不变原理,(为方便起见,设长度单位为秒,1秒=299792458米,则光速为1,速度无单位)% 未完成, 这个表述太奇怪了
当速度 $v = \dv*{x}{t} = 1$ 时, 必须有
\begin{equation}
v' = \dv{x'}{t'} = \frac{a\dd{x} + b\dd{t}}{m\dd{x} + n\dd{t}} = \frac{a + b}{m + n} = -1
\end{equation}
结合(引用未完成) 得
\begin{equation}
\leftgroup{
&n = -a\\
&m = -b\\
&a^2 - b^2 = 1
}
\end{equation}
即
\begin{equation}
\leftgroup{
&b = \pm \sqrt{a^2 - 1}\\
&n = -a\\
&m = \mp \sqrt{a^2 - 1}
}
\end{equation}
由两系相对运动的方向,当 $x$ 不变, $t$ 增大, $x'$ 必然减小,所以 $b < 0$, 即
\begin{equation}
\leftgroup{
&b = -\sqrt{a^2 - 1}\\
&n = -a\\
&m = \sqrt{a^2 - 1}
}
\end{equation}
当 $x = 0$,
\begin{equation}
\dv{x'}{t'} = \frac{\sqrt{a^2 - 1}}{a} = -v \quad (v > 0) \Rightarrow a = -\frac{1}{1 - v^2}
\end{equation}
把(4)(5)(6)(7)代入变换得% 引用
正逆变换分别为
\begin{equation}
\leftgroup{
&x' = - \frac{x + vt}{\sqrt{1 - v^2}}\\
&t' = \frac{t + vx}{\sqrt{1 - v^2}}
}
\qquad
\leftgroup{
&x = -\frac{x' + vt'}{\sqrt{1 - v^2}}\\
&t = \frac{t' + vx'}{\sqrt{1 - v^2}}
}
\end{equation}
通常,为了方便起见,把 轴的方向改成与 轴一样,即,在上式的 前面加负号,得
\begin{equation}
\leftgroup{
&x' = \frac{x - vt}{\sqrt{1 - v^2}}\\
&t' = \frac{t - vx}{\sqrt{1 - v^2}}
}
\qquad
\leftgroup{
&x = \frac{x' + vt'}{\sqrt{1 - v^2}}\\
&t = \frac{t' + vx'}{\sqrt{1 - v^2}}
}
\end{equation}
考虑其他维度, 设
\begin{equation}
\leftgroup{
&x' = \frac{x - vt}{\sqrt{1 - v^2}} + ay + bz\\
&t' = \frac{t - vx}{\sqrt{1 - v^2}} + cy + ez
}
\end{equation}
求逆变换,对比系数,得出 $a, b, c, e$ 为零. 同理可得
\begin{equation}
\leftgroup{
y' &= y\\
z' &= z
}
\end{equation}
这就是常见的洛伦兹变换

由此可以看出,撇开人为的单位规定,洛伦兹变换对时间和空间的效果是相同的,这非常耐人寻味.但这也具有必然性,因为从数学的角度说,上面的推导过程中并没有体现出这两个物理量的区别.

若转换成普通单位,
\begin{equation}
\leftgroup{
&x \Rightarrow x/c\\
&v \Rightarrow v/c\\
&x' \Rightarrow x'/c\\
&v' \Rightarrow v'/c
}
\end{equation}
就得到了% 引用未完成
