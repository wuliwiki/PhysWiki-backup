% 可微映射的导数
% keys 导数|可微映射
% license Xiao
% type Tutor

\pentry{常微分方程的几何图像\upref{GofODE}}
可微映射(\autoref{def_GofODE_2}~\upref{GofODE})$f:U\rightarrow V$ 将 $\mathbb R^n$ 空间的区域 $U$ 映射到 $\mathbb R^m$ 空间区域 $V$,于是就将 $U$ 上的曲线(\autoref{sub_GofODE_1}~\upref{GofODE})$\varphi$ 映射到 $V$ 上的曲线 $\phi$,可微性意味着这一对应是一一的。而切向量是曲线的等价类(\autoref{def_GofODE_3}~\upref{GofODE}),于是曲线 $\varphi,\phi$ 各自对应一切向量 $\dv{}{t}(x\circ\varphi),\dv{}{t}(y\circ\varphi)$。这就是说在可微映射 $f$ 作用下,$U$中的切向量 $\dv{}{t}(x\circ\varphi)$ 和 $V$ 中的切向量 $\dv{}{t}(y\circ\varphi)$ 一一对应。 