% 中国科学技术大学 2015 年考研普通物理考试试题
% keys 中国科学技术大学|考研|物理|2015年
% license Copy
% type Tutor

\textbf{声明}:“该内容来源于网络公开资料,不保证真实性,如有侵权请联系管理员”
\begin{enumerate}
\item 电荷$Q$均匀分布在球面上,在球心有一个电荷量为$q$的点电荷。由对称性和高斯定理可知:球面上的$Q$在球心产生的电场强度为零,因此$Q$作用在$q$上的力为零;但$q$在球面上产生的电场强度不为零,因而有力作用在$Q$上。请问这违反牛顿第三定律吗?并说明为什么?
\item 一沿$y$轴运动的质点,加速度与位置的关系为$a=2+12y^2$。如质点在原点处的速度为$2m/s$,试写出质点在任意位置处的速度值的表达式。
\item 将不可压缩的空气吹进一个球对称的气球,当它的半径为$6.50cm$ 时,其半径增加速率为$0.900cm/s$。①给出此时气球体积的增加速率。②如果单位时间空气进入气球的体积是恒定的,试求当气球半径为$13.0cm$时,其半径增加速率。③如果②中的答案不同于$0.90cm/s$,请解释原因。
\item 在登月工程设计中,为安全起见要将登月舱与轨道舱分离的动作在朝向地球的一面完成,让登月舱在月球背向地球的一面着月。设轨道舱的质量为$M$,登月舱的质量为$m$,分离前轨道舱和登月舱对接在一起沿半径为$R$的圆轨道绕月飞行。分离时,使登月舱相对于轨道舱以速率$v$向后运动,两者脱离,登月舱开始下降,在月球背面着陆。已知月球的质量为$M_m$,半径为$R_m$,请确定:①登月舱和轨道舱分离时的相对速度v;②从分离到登月舱着陆的时间。
\item 什么是卡诺循环?在p-V图和S-T上表示之。推导可逆卡诺循环热机的效率?
5.1mol 理想气体经历了体积从V到2V,的可逆等温膨胀过程,问:①气体的熵变是多少?②整个体系的总熵变是多少?如假定同样的膨胀为自由膨胀,上述结果又如何。
6.电荷量都是q的三个点电荷,分别放在正三角形的三个顶点上。试问:在这三角形中心放一个什么样的点电荷,可以使每个电荷受到其他三个电荷的库仑力之和均为零?②这种平衡与正三角形的边长有无关系?③这样的平衡是
稳定平衡还是不稳定平衡?
7.如图所示的法拉第圆盘发电机,其圆盘半径为R,圆盘的轴线与均匀外磁场B方向一致,当它以角速度ω,绕轴线转动时,试求:①该发电机的电动势ε;②盘心与盘边哪一点的电势高;③设R=0.15cm,B=0.6T,圆盘的转速为30圈/秒,其电动势s=?
\end{enumerate}