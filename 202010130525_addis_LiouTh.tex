% 刘维尔定理

\begin{issues}
\issueDraft
\end{issues}

$(q, p)$ 空间称为相空间, 复杂系统的所有系综看成多维相空间中的流体, 每个具体系统的状态是相空间中的一点, 随时间变化. 跟随一点时, 周围密度不随时间变化.

$t$ 等于零时在相空间中取一块小区域, 具有边界 $\mathcal B$. 可以证明随着时间变化, 虽然边界开始变形, 但边界两边的点不会跨越边界. 也可以证明, 这个区域的体积始终保持不变.

刘维尔定理的一个最直接推论是, 如果开始时相空间中这种流体的密度处处相同, 那么接下来在任意时刻 $t$, 流体密度仍然处处相同.

在这样的流体里面随机抽取一个点, 那么这个点几乎肯定是平衡态. 热力学第二定律就是在这个 “几乎肯定” 上成立的. 对于混沌系统, 任意一个点都会遍历整个相空间, 然而比起

\subsection{难点}
对于足够复杂的系统, 即使开始时相空间中流体密度并不均匀, 在经过足够长的时间以后也会在相空间中变得均匀. 这看似违背了刘维尔定理, 但从严格的数学角度来说是没有的, 只是原来的体积被非常均匀地 “揉” 到了整个相空间中. 想象我们在一块白色的密度较小的橡皮泥中混入一小块黑色的密度较大的橡皮泥, 假设两种橡皮泥之间始终保持明确的边界, 当我们不停地以某种复杂的方式揉动橡皮泥, 经过足够长一段时间之后, 虽然从微观上两者的边界依然明确, 但是从更宏观的角度来看两种橡皮泥已经混合均匀了. 所以我们可以认为
