% 进制、二进制
% license Usr
% type Tutor

\begin{issues}
\issueDraft
\end{issues}

\subsection{$N$ 进制需要几个符号?}
我们日常熟悉使用十进制。 但我们可能由于对其习以为常,而忘记它一般的规则。 所以我们不妨总结一下。 用\textbf{阿拉伯数字}表示十进制数时,\textbf{每一位}共有十个符号: $0,1,\dots, 9$。要特别注意其中并没有 “十” 这个符号,必须要用\textbf{两位数} $10$ 才能表示出十。

许多国家的自然语言中存在比 $9$ 大的数字的单词, 如英语的 eleven, twelve 可以用一个单词表示 $11$ 和 $12$, 但是当我们讨论十进制的阿拉伯数字写法时, 一位数符号中只有 $0,1,\dots, 9$ 而没有其他符号。 

下面我们会看到这个规律对任何进制都是一样的: $N$ 进制中的一位数只需要包括 $0$ 在内的 $N$ 个不同的符号,可以表示 $0,1,\dots,N-1$。 而需要进位成两位数才可以表示 $N$ 本身。

\subsection{如何进位?}
如果从 $0$ 开始, 如何数数呢? 从十进制中我们可以总结出来, 当一位数的符号从第一个数到最后一个后,如果还需要下一个,就在左边一位使用下一个符号,并把当前位归零。 例如 $9$ 可以看成 $09$, 下一个数在第二位使用下一个符号 $10$。 又例如 $199$ 的下一个数是 $200$, 这是因为最右边两位同时达到了最后一个符号,所以要在第三位使用下一个符号并把前两位归零。 注意我们假设可以在一个数字左边写上任意多位的 $0$ 而不影响它表示的值。

所以十进制的(右边)第二位的 $1$ 代表 $10$ 倍,而第三位数代表 $100$ 倍…… 例如十进制的 $43576$ 表示
\begin{equation}\label{eq_Binary_1}
48576 = 4\e{4} + 3\e{3} + 5\e{2} + 7\times 10\e{1} + 6\e{0}~.
\end{equation}
这里的每一项使用了科学计数法,% 连接未完成
$10^4$ 表示 $10000$, $10^3$ 表示 $1000$ 等。 特殊地, $10^1=10$,$10^0=1$。

那么相似地, 对于任意的 $N$ 进制, 当某一位从 $0$ 变到 $N-1$ 后, 如果还需要下一个符号, 就在下一位加 $1$。 也就是下一位的一个数等于上一位的 $N$ 倍。

\subsection{八进制}
作为一个和十进制相近的例子,我们先来学八进制。 八进制下, 我们每位只能使用 $0,\dots,7$ 这 $8$ 个符号。 当我们需要 $7$ 的下一个符号时, 就进位到下一位, 即 $10$。 为了和十进制的 $10$ 区分,我们把它写成 $10_\text{8}$。 所以 $10_\text{8} = 8$。

我们可以继续数七个数,到 $17_\text{8}$, 那么下一个数又该进位并把第一个归零, 即成为 $20_\text{8}$。 再数七个到 $27_\text{8}$, 然进位成 $30_\text{8}$。 当你继续数到 $77_\text{8}$, 那么下一个数需要向第二位进位,但第二位也已经满了,所以需要向第三位进位,成为 $100_\text{8}$。

所以如果把\autoref{eq_Binary_1} 左边的数视为 $8$ 进制的 $48576_\text{8}$, 它在十进制中表示多少呢? 类比过来,就是
\begin{equation}
48576_\text{8} = 4\times 8^{4} + 3\times 8^{3} + 5\times 8^{2} + 7\times 8^1 + 6\times 8^0 = 18302~.
\end{equation}

\subsection{二进制}
根据上文, 二进制只有两个符号 $0$ 和 $1$。 二进制是最小的进制,因为 “一进制” 将不能表示 $0$ 以外的任何数。

我们用二进制数数: $0_\text{2}$,$1_\text{2}$ 这时已经到最后一个符号了,于是进位得到 $10_\text{2}$,再数 $11_\text{2}$,这时发现两位数都是最后一个符号,向第三位进位得 $100_\text{2}$,$101_\text{2}$,$110_\text{2}$,$111_\text{2}$。向第四位进位:$1000_\text{2}$,$1001_\text{2}$,$1010_\text{2}$,$1011_\text{2}$,$1100_\text{2}$,$1101_\text{2}$,$1110_\text{2}$,$1111_\text{2}$……

二进制数如何转换位十进制呢?一个例子:
\begin{equation}
11010010_\text{2} = 2^7 + 2^6 + 0\times 2^5 + 2^4 + 0\times 2^3 + 0 \times 2^2 + 1\times 2^1 + 2^0 = 210~.
\end{equation}

\addTODO{负数参考\enref{原码、反码、补码}{InvCom}}

\subsection{十六进制}
十六进制需要 16 个符号, 所以我们需要在十进制的 10 个符号基础上额外添加 6 个, 也就是 $a,b,c,d,e,f$,分别表示 $10$ 到 $15$, 也可以用大写。 例如
\begin{equation}
\mathrm{47a58d}_\text{16} = 4\times 16^5 + 7\times 16^4 + 10\times 16^3 + 5\times 16^2 + 8 \times 16^1 + 14 = 4695438~.
\end{equation}

十六进制在计算机领域经常被使用。 这是因为一位十六进制数与 $4$ 位二进制数具有一一对应的关系。 $4$ 位二进制数 

\subsection{小数}
我们来思考 $N$ 进制的小数点表示什么。 十进制中整数可以看作小数点在最右边, 每当小数点向左移动一位, 就代表原来的数的 $1/10$, 向右移动一位就是乘以 $10$。

那么同理, $N$ 进制中,每把小数点向左移动一位,就代表原来的数乘以 $1/N$, 向右移动一位就代表乘以 $N$。
\addTODO{...}

\subsection{进制的转换}
\addTODO{...}
