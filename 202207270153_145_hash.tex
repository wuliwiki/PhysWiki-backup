% 哈希表
% 哈希表|散列表|数据结构|C++

哈希表又称散列表,哈希通常是把一些值域较大的数映射成值域较小的数.比如把 $10^8$ 映射成 $1$,$10^8 + 100$ 映射成 $2$.来看一道具体问题:

维护一个集合,支持如下几种操作:

\begin{enumerate}
\item 插入一个数 $x$;

\item 询问数 $x$ 是否在集合中出现过;

\end{enumerate}
现在要进行 $N$ 次操作,对于每个询问操作输出对应的结果.

数据范围:$ 1 \le N \le 10^5$ ,$\ -10^9 \le x \le 10^9$

因为 $x$ 的值域非常大,所以我们需要设计一个\textbf{哈希函数},能把一个值域比较大的数映射到从 $0 \sim 10^5$ 之间的一个数.

一般哈希函数设计为 \verb|Hash(x) = (x mod P)|.

因为因为值域变小了,很有可能把原始值不同的两个数映射到同一个位置,所以我们需要处理这种\textbf{冲突情况}.处理冲突的两种方式:\textbf{拉链法},\textbf{开放寻址法}.

\textbf{拉链法:}

先开一个一维数组来保存每个数的哈希值,如果有数映射到了某个位置,我们就在那个位置开一个链表用于记录每个数.查询每个数是否存在的的时候,就在对应的哈希位置遍历一下整个链表,对其中的每个数比较其值与查询的值是否一致.

拉链法的时间复杂度为 $O(N)$
\begin{figure}[ht]
\centering
\includegraphics[width=14.25cm]{./figures/hash_1.png}
\caption{拉链法} \label{hash_fig1}
\end{figure}

一般来说,设计哈希算法取模的数要设置为一个质数,数学上可以证明这么取冲突的概率是最小的.
\begin{lstlisting}[language=cpp]
int h[N], e[N], ne[N], idx;
int n;

// 头插法
void add(int a, int b)
{
    e[idx] = b, ne[idx] = h[a], h[a] = idx ++ ;
}

void insert(int x)
{
    int k = (x % N + N) % N;   // 哈希位置
    add(k, x);
}

bool find(int x)
{
    int k = (x % N + N) % N;
    for (int i = h[k]; ~i; i = ne[i])  // 遍历整个链表
        if (e[i] == x)
            return true;
    return false;
}
\end{lstlisting}

\textbf{开放寻址法:}

