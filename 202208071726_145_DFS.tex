% 深度优先搜索(DFS)
% DFS|算法|C++|搜索

深度优先搜索(DFS,Depth First Search)简称深搜或者爆搜,DFS 的搜索顺序是按照深度优先搜索,简单来说就是“一条路走到黑”,搜索是把所有方案都试一遍,再判断是不是一个可行解.搜索与“递归”和“栈”有很大的联系,递归是实现搜索的一种方式,而栈则是计算机实现递归的方式.每个搜索过程都对应着一棵\textbf{递归搜索树},递归搜索树可以让我们更加容易的理解 DFS.
整个搜索过程就是基于该搜索树完成的,为了不重复遍历每个结点,会对每个节点进行标记,也可以对树中不可能是答案的分支进行删除,从而更高效的找到答案,这种方法被称为\textbf{剪枝}.如果搜索树在某个子树中搜索到了叶结点,想继续搜索只能返回上个或多个状态,返回的过程称为\textbf{回溯},回溯要记得\textbf{恢复状态},