% 自旋统计定理(综述)
% license CCBYSA3
% type Wiki

本文根据 CC-BY-SA 协议转载翻译自维基百科\href{https://en.wikipedia.org/wiki/Spin\%E2\%80\%93statistics_theorem}{相关文章}。

自旋-统计定理证明了粒子的内禀自旋(不源于轨道运动的角动量)与该类粒子集合的量子统计性质之间的关系是量子力学数学的必然结果。在以约化普朗克常数\( \hbar \)为单位的描述下,所有在三维空间中运动的粒子具有以下特性:整数自旋的粒子服从玻色-爱因斯坦统计;半整数自旋的粒子服从费米-狄拉克统计\(^\text{[1][2]}\)。
\subsection{自旋-统计关系} 
所有已知粒子都遵循费米-狄拉克统计或玻色-爱因斯坦统计。粒子的内禀自旋总是能够预测该类粒子集合的统计性质,反之亦然 \(^\text{[3]}\):  
\begin{itemize}
\item 整数自旋的粒子是玻色子,遵循玻色-爱因斯坦统计;  
\item 半整数自旋的粒子是费米子,遵循费米-狄拉克统计。  
\end{itemize}
自旋-统计定理证明了量子力学的数学逻辑预测或解释了这一物理结果\(^\text{[4]}\)。  

对于不可区分粒子的统计性质,其影响是最基本的物理效应之一。例如:泡利不相容原理 —— 规定每个占据的量子态中最多只能容纳一个费米子,决定了物质的形成。物质的基本组成部分,如质子、中子和电子,都是费米子。另一方面,介导物质粒子之间相互作用的粒子,如光子,都是玻色子\([\text{citation needed}]\)。自旋-统计定理试图解释这一基本的二分性的起源\(^\text{[5]: 4}\)。
\subsection{背景}
从直观上看,自旋作为粒子的内禀角动量属性,似乎与该类粒子集合的基本性质无关。然而,这些粒子是不可区分的,因此涉及多个不可区分粒子的任何物理预测在交换这些粒子时都不应发生变化。