% 南京理工大学 2005 年 研究生入学考试试题 普通物理(B)
% license Usr
% type Note

\textbf{声明}:“该内容来源于网络公开资料,不保证真实性,如有侵权请联系管理员”

\subsection{一、填空题(26 分,每题 2 分)}
1. 一质点作平面运动,运动方程为$r=6ti+5t^2j$ ,则 $t$ 时刻质点的速度为__________,加速度为______。

2. 一均质细棒,长为 $L$,质量为 $m$,一端固定于 $O$ 点,棒可绕 $O$ 点转动。开始时托住 $B$ 端让棒处在水平位置,然后松手,则棒处在水平位置时的角加速度大小为___________,棒到达竖直位置时的角加速度为_______,转动动能为___________。

3. 27°C的 $1mol$ 氧气分子的最概然速率为____________,其物理意义为__________,分子的平均平动动能为_________,$1mol$ 理想氧气的内能为________。

4. 以波速 $u$ 向 $X$ 正方向传播的平面简谐波,振幅为 $A$,圆频率为$\omega$ ,设位于坐标 $x_0$ 处的质点,$t=0$ 时,位移 $y=A/2$,且向 $y$ 负方向运动,则该质点的振动方程为__________,该平面简谐波的波动方程(波函数)为___________。
5.一平面简谐波方程(波函数)为$y_1=4\cos2\pi(\frac{t}{T}+\frac{x}{\lambda})$,在$x=\frac{\lambda}{2}$-一处有一反射壁若平面波从空气传到反射壁而反射,反射时振幅不变,已知空气为波疏介质,则反射波波动方程为__________,波节点的位置为___________。
\subsection{二、填空题(28 分,每空 2 分)}
1.一无限长均匀带电直线,电荷线密度为$\lambda$,则离直线距离为$r$处电场强度大小为__________,距直线距离为$r_a$和$r_b(r_b>r_a)$的两点间的电势差 $\varphi_a-\varphi_b$__________。

2.一个电矩为$p=q$的电偶极子轴线(即沿着的方向)距其中心为$r(r>>1)$处的一点的场强(用$r$和$p$表示)为__________。

3.若氢原子中电子(电量为$e$)在一半径为R的圆轨道上以速率v作匀速率圆周运动,则圆心处的磁感应强度大小为__________,圆心处的磁场能量密度为__________。

4.白然光入射到空气和某玻璃的分界面上,当入射角为60°时,反射光为完全偏振光,则该玻璃的折射率为__________,一束强度为$I0$的自然光垂直入射于两块平行放置且透光轴方向夹角为60。的偏振光上,则透射光的强度为__________。

5.用氦一氖激光器发出的波长为 $632.8nm$ 的单色光做牛顿环实验,测得第k级暗环的半径为 $5.625mm$,第$k+5$ 级暗环的半径为7.956nm,,则所用平凸透镜的曲率半径 $R=$__________,$k=$__________。

6. S 系中一质量为 $m0$、体积为 $V0$ 的立方体,若使此立方体沿平行于一边的方向以速度 $v=0.6c$ 运动,则在 $S$ 系中测得其体积为____________,动能为________。

7. 基态氢原子中的电子吸收一个能量为 $12eV$ 的光子后所能到达的最高能态为__
________,从该能态直接跃迁回基态辐射的光子频率为__________。

5.已知一维无限深势阱中粒子的波函数为:$\psi_n(x)=\sqrt{\frac{2}{a}}\sin\frac{n\pi}{a}x$,则$n=2$时,粒子在$x=\frac{a}{2}$处出现的概率密度为__________。
\subsection{三、(12 分)}
质量为 $m$ 的小球在光滑的水平面内沿半径为 $R$ 的固定圆环作圆周运动。已知小球与圆环间的滑动摩擦系数为 $\mu k$,小球的初速度大小为 $v0$,求小球运动三周回到原来位置时的速度大小。
\subsection{四、(12 分)}
已知轻弹簧的劲度系数为 $k$,定滑轮可看作质量为 $M$、半径为 $R$的均质圆盘,物体的质量为 $m$,试求:(1)系统的振动周期;(2)当将$m$ 托至弹簧原长并释放时,求 $m$ 的运动方程(以向下为正方向)。
\subsection{五、(12 分)}
1 摩尔单原子分子组成的理想气体系统,经如右图所示的循环过程,求:(1)一个循环中,系统对外做的功;(2)循环效率
\begin{figure}[ht]
\centering
\includegraphics[width=6cm]{./figures/29e25ff22af999bb.png}
\caption{} \label{fig_NJU05_1}
\end{figure}
\subsection{六、(12 分)}
内外半径分别为 $R1,R2$ 的金属球壳中心处放置一点电荷 $q$,金属球壳带电 $Q$,试求:(1)电场强度的分布;(2)金属球壳的电势(设无穷远处为零电势点)。
\subsection{七、(12 分)}
一个半径 $R=0.18m$ 的半圆形闭合线圈,载有电流 $I=10A$,放在$B=10T$ 的均匀外磁场中,磁场方向与线圈平面平行,如图所示,求:(1)线圈磁矩的大小与方向;(2)线圈所受磁力矩的大小与方向。(3)在磁力矩的作用下,线圈平面绕纸面内过 $O$ 点的竖直轴转过$\pi/2$ 时,磁力矩作的功(设 $I$ 在旋转过程中不变)
\begin{figure}[ht]
\centering
\includegraphics[width=6cm]{./figures/3c14b6ba382406f9.png}
\caption{} \label{fig_NJU05_2}
\end{figure}
\subsection{八、(12 分)}
磁感应强度为 $B$ 的均匀磁场垂直纸面向内,一长度为 $L$ 的金属杆$ab$ 以 $a$ 为中心、角速度为 $w$ 旋转,$b$ 在半径为 $L$ 的金属圆环上滑动,接触良好,旋转中心与金属环之间接一电阻 $R$,求:(1)$ab$ 杆上动生电动势的大小与方向;(2)$ab$ 杆受到的磁力矩。