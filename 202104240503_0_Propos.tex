% 命题及其表示法

\begin{definition}{}
能表达判断的语言是陈述句, 它称作\textbf{命题}. 一个命题,总是具有一个 “值”, 称为\textbf{真值}. 真值只有\textbf{真(True)}和\textbf{假(False)} 两种. 只有具有确定真值的陈述句才是命题, 一切没有判断内容的句子, 无所谓是非的句子, 如感叹句、疑问句、祈使句等都不能作为命题.
\end{definition}

命题有两种类型:
\begin{enumerate}
\item \textbf{原子命题}: 不能分解为更简单的陈述语句
\item \textbf{复合命题}: 由联结词、标点符号和原子命题复合构成的命题
\end{enumerate} 

\subsection{联结词}
\begin{definition}{否定 $\neg$} \end{definition} 
%\begin{definition}[合取] $\wedge$ \end{definition} 
%\begin{definition}[析取] $\vee$ \end{definition} 
%\begin{definition}[条件] $\rightarrow$ \end{definition} 
%\begin{definition}[双条件] $\leftrightarrow$ \end{definition} 
