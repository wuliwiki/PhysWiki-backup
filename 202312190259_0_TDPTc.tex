% 含时微扰理论
% license Xiao
% type Tutor

\pentry{含时微扰理论(束缚态)\upref{TDPT}, 一维散射态的正交归一化\upref{ScaNrm}}

在 “含时微扰理论(束缚态)\upref{TDPT}” 中, 我们只讨论了如何用含时微扰理论计算离散的束缚态之间的跃迁。 但如果我们要讨论光电离也就是熟知的光电效应, 那么我们除了离散本征值的束缚态还需要加入连续本征值的散射态, 二者共同作为展开波函数的完备正交归一基底。

\subsection{薛定谔方程的矩阵形式}
本文使用原子单位制\upref{AU}。 一般三维波函数表示为束缚态的求和以及连续态的积分
\begin{equation}\label{eq_TDPTc_2}
\ket{\Psi(t)} = \sum_n c_n(t)\ket{n}\E^{-\I\omega_n t} + \int c_{\bvec k}(t)\ket{\bvec k}\E^{-\I k^2 t/2} \dd[3]{k}~.
\end{equation}
其中 $\ket{n}$, $\ket{\bvec k}$ 一起构成一组正交归一基
\begin{equation}
\braket{m}{n} = \delta_{m,n}~, \qquad
\braket{\bvec k'}{\bvec k} = \delta(\bvec k' - \bvec k)~, \qquad
\braket{\bvec k}{n} = 0~.
\end{equation}
为了书写方便我们把\autoref{eq_TDPTc_2} 的求和与积分一起记为
\begin{equation}
\ket{\Psi(t)} = \sumint_\alpha c_\alpha(t) \ket{\alpha(t)} = \sumint_\alpha c_\alpha(t) \ket{\alpha}\E^{-\I\omega_\alpha t} ~.
\end{equation}
$\alpha$ 既包括离散指标 $n$ 又包括连续指标 $\bvec k$。 求和积分号 $\displaystyle\sumint$ 对离散指标求和, 对连续指标重积分。

无误差的薛定谔方程变为(类比\autoref{eq_TDPT_11}~\upref{TDPT})
\begin{equation}
\sumint_\alpha c_\alpha(t) H'(t) \ket{\alpha(t)} = \I \sumint_\alpha \dot c_\alpha(t) \ket{\alpha(t)}~.
\end{equation}
要得到矩阵形式, 投影到 $\bra{\alpha'}$ 后变为(类比\autoref{eq_TDPT_2}~\upref{TDPT})
\begin{equation}
\sumint_\alpha \mel{\alpha'}{H'(t)}{\alpha} \E^{\I \omega_{\alpha'\alpha} t} c_\alpha(t)
= \I \dot c_{\alpha'} (t)~.
\end{equation}
其中
\begin{equation}
\omega_{\alpha'\alpha} = E_{\alpha'}-E_\alpha~.
\end{equation}
注意这不是一个常规意义的矩阵, 因为 $\alpha$ 既有连续也有离散部分。

$\mat H'$  “矩阵” 可以想象成是这个样子的
\begin{figure}[ht]
\centering
\includegraphics[width=5cm]{./figures/16846e826f12f9e5.pdf}
\caption{$\mat H'$ 矩阵的结构} \label{fig_TDPTc_1}
\end{figure}
图中方格子代表 $c_{mn} = \bra{m} H' \ket{n}$, 横条代表 $H_{m,\bvec k'} = \mel{m}{H'}{\bvec k'}$,  纵条代表 $H_{\bvec k, n} = \mel{\bvec k}{H'}{n}$,右下角的连续方块代表 $\mel{\bvec k}{H'}{\bvec k'}$。

\subsection{含时微扰}
薛定谔方程各阶分离后变为(类比\autoref{eq_TDPT_6}~\upref{TDPT})
\begin{equation}
\dot c_{\alpha'}^{(n)}(t) = -\I \sumint_\alpha \mel{\alpha'}{H'(t)}{\alpha} \E^{\I \omega_{\alpha'\alpha} t} c_{\alpha}^{(n-1)}(t)~.
\end{equation}
两边对时间积分
\begin{equation}
c_{\alpha'}^{(n)}(t) = -\I \int^t \sumint_\alpha \mel{\alpha'}{H'(t)}{\alpha} \E^{\I \omega_{\alpha'\alpha} t} c_{\alpha}^{(n-1)}(t) \dd{t}~,
\end{equation}
具体写出来, 就是
\begin{equation}
c_i^{(n + 1)}(t) = \frac{1}{\I\hbar} \int \dd{t'} \qty(\sum_{j \ne i} H'_{i,j} c_j^{(n)} + \int H_{i,\bvec k'} \phi ^{(n)}(\bvec k') \dd[3]{k'} )~,
\end{equation}
\begin{equation}
c^{(n+1)}_{\bvec k}(t) = \frac{1}{\I\hbar} \int \dd{t'} \qty(\sum_j H'_{\bvec k,j} c_j^{(n)} + \int H_{\bvec k,\bvec k'} \phi ^{(n)}(\bvec k') \dd[3]{k'})~.
\end{equation}

\subsection{简单的一阶微扰}
假设初态为 $\ket{\alpha}$, 即 $c_\alpha = 1$, 其他系数都为零。 令 $H'(t) = W f(t)$, $f(t)$ 的傅里叶变换(\autoref{eq_FTExp_6}~\upref{FTExp})记为 $\tilde f(\omega)$, 那么简单的一阶微扰为
\begin{equation}\label{eq_TDPTc_1}
c_{\alpha'}^{(1)}(+\infty) = -\I\sqrt{2\pi} \mel{\alpha'}{W}{\alpha} \tilde f(-\omega_{\alpha'\alpha})~.
\end{equation}
具体到\autoref{eq_TDPTc_2}, 当 $\alpha'$ 取离散值 $n'$ 时, $|c_{n'}^{(1)}(+\infty)|^2$ 就是跃迁到该离散态的概率, 取连续值 $\bvec k$ 时, $|c_{\bvec k'}^{(1)}(+\infty)|^2$ 就是三维 $\bvec k'$ 空间的概率密度\upref{RandF}, 概率微元是 $|c_{\bvec k'}^{(1)}(+\infty)|^2\dd[3]{k}$。

\addTODO{举例: 引用氢原子的单电离,不要直接写在这里}

\subsection{光电子的能谱}
那光电子的什么谱正比于电磁波包的能量谱 $\abs*{\tilde f}^2$ 呢?如果在某个方向画出 $f(k) = \abs{c_{\bvec k'}^{(1)}(+\infty)}^2$ 谱, 如果波包带宽足够小可以认为 $\mel{\alpha'}{W}{\alpha}$ 几乎不变, 那么
\begin{equation}
f(k) = \abs{c_{\bvec k'}^{(1)}(+\infty)}^2 \propto \abs{\tilde f\qty(E_0-\frac{k^2}{2})}^2~.
\end{equation}
这是光电子的一个动量谱,那么对应的能量谱为(见 “随机变量的变换\upref{RandCV}”)
\begin{equation}
g(E) = \frac{1}{k}f(k) \propto \frac{1}{k}\abs{\tilde f\qty(E_0-E)}^2~
\end{equation}
但这同样不是我们想要的,其实我们想要的是
\begin{equation}
h(E) = f\qty(\sqrt{2E}) \propto \abs{\tilde f(E_0-E)}^2~.
\end{equation}
这样,例如 $f(t)$ 是一个高斯波包,那么 $h(E)$ 就近似是高斯波包。 若要更精确就乘以跃迁矩阵元 $\abs{\mel{\alpha'}{W}{\alpha}}^2$ 的调制。

$\displaystyle\int_0^\infty \int k^2 f(k) \dd{\Omega}\dd{k}$

$\displaystyle k f(k)$ $\Delta\Omega$

\begin{equation}
k f(k) \propto k \abs{\mel{\alpha'}{W}{\alpha}}^2 \abs{\tilde f\qty(E_0-\frac{k^2}{2})}^2~.~
\end{equation}
