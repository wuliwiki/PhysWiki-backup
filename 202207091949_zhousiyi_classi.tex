% 经典场论基础
% 经典场

\pentry{经典场论}


\subsection{拉格朗日场论}
这一节里面,我们复习一下经典场的知识,为后面的量子场论做铺垫.首先要复习的一个重要的量就是拉式量了,定义如下
\begin{equation}
S = \int L dt = \int \mathcal L(\phi,\partial_\mu \phi)d^4 x
\end{equation}
经典场论的重要原理是变分原理 $\delta S = 0$.
\begin{equation}
\begin{aligned}
0 &=\delta S \\
&=\int d^{4} x\left\{\frac{\partial \mathcal{L}}{\partial \phi} \delta \phi+\frac{\partial \mathcal{L}}{\partial\left(\partial_{\mu} \phi\right)} \delta\left(\partial_{\mu} \phi\right)\right\} \\
&=\int d^{4} x\left\{\frac{\partial \mathcal{L}}{\partial \phi} \delta \phi-\partial_{\mu}\left(\frac{\partial \mathcal{L}}{\partial\left(\partial_{\mu} \phi\right)}\right) \delta \phi+\partial_{\mu}\left(\frac{\partial \mathcal{L}}{\partial\left(\partial_{\mu} \phi\right)} \delta \phi\right)\right\}
\end{aligned}
\end{equation} 
最后一项是一个表面项,这里我们考虑边界条件是 $\delta \phi$ 为零的构型,这一项就可以忽略.现在我们看前两项.因为对于任意的 $\delta \phi$ 这个式子都为零,所以我们必须让 $\delta \phi$ 前面的系数为零,这样,我们就推出了著名的欧拉-拉格朗日方程
\begin{equation}
\partial_\mu \bigg( \frac{\partial \mathcal L}{\partial(\partial_\mu\phi)} \bigg) - \frac{\partial \mathcal L}{\partial \phi} = 0 
\end{equation}

\subsection{哈密顿场论}
拉式量的方法的优点是所有的量都是明显洛仑兹不变的.哈密顿场论的优点是更容易过度到量子力学.

对于一个分立系统,我们可以定义共轭动量
\begin{definition}{共轭动量}
对于每个动力学变量 $q$,我们可以定义它的相应的共轭动量
\begin{equation}
p \equiv \frac{\partial L}{\partial \dot q}
\end{equation}
\end{definition}
那么哈密顿量的定义如下
\begin{definition}{哈密顿量}
\begin{equation}\label{classi_eq1}
H \equiv \sum p \dot q - L
\end{equation}
\end{definition}
上面的定义也可以推广到连续系统.只要假设空间坐标 $\mathbf x$ 是分立的就可以了,这样对于连续系统,我们可以定义如下的共轭动量
\begin{definition}{连续系统的共轭动量}
\begin{equation}
\begin{aligned}
p(\mathbf x) & \equiv \frac{\partial L}{\partial \dot \phi(\mathbf x)} = \frac{\partial}{\partial \dot \phi(\mathbf x)} \int \mathcal L(\phi(\mathbf y),\dot \phi(\mathbf y)) d^3 y \\
& \sim \frac{\partial}{\partial \dot \phi(\mathbf x)} \sum_{\mathbf y} \mathcal L(\phi(\mathbf y,\dot \phi(\mathbf y))) d^3 y=\pi(\mathbf x) d^3 x
\end{aligned}
\end{equation}
其中
\begin{equation}
\pi(\mathbf x) \equiv \frac{\partial \mathcal L}{\partial \dot \phi(\mathbf x)}
\end{equation}
是与 $\phi(\mathbf x)$ 共轭的哈密顿量密度.
\end{definition}
因此哈密顿量为
\begin{equation}
H = \int d^3 x\,\, p(\mathbf x) \dot \phi(\mathbf x) - L
\end{equation}
现在我们来看一个简单的例子.
\begin{align}\nonumber
\mathcal L & = \frac{1}{2} \dot \phi^2 - \frac{1}{2} (\nabla \phi)^2 - \frac{1}{2} m^2 \phi^2 \\
& = \frac{1}{2} (\partial_\mu\phi)^2 - \frac{1}{2} m^2 \phi^2
\end{align}
根据这个拉式量可以写出运动方程
\begin{equation}
\bigg( \frac{\partial^2}{\partial t^2} - \nabla^2 +m^2 \bigg)\phi = 0~,\quad (\partial^\mu\partial_\mu+m^2)\phi = 0
\end{equation}
这就是克莱因戈登方程.这个标量场对应的哈密顿量为
\begin{equation}
H =  \int d^3x \mathcal H = \int d^3 x \bigg[ \frac{1}{2} \pi^2 + \frac{1}{2} (\nabla \phi)^2 + \frac{1}{2} m^2 \phi^2 \bigg] 
\end{equation} 

\subsection{诺特定理}
\begin{theorem}{诺特定理}
每个\textbf{连续对称性}都有着\textbf{相应的守恒定律}.
\begin{itemize}
\item 物理系统的\textbf{空间平移不变性}(物理定律不随着空间中的位置而变化)给出了\textbf{动量守恒}律;
\item \textbf{转动不变性}给出了\textbf{角动量守恒}律;
\item \textbf{时间平移不变性}给出了\textbf{能量守恒}定律.
\end{itemize}
\end{theorem}
现在考虑标量场 $\phi$ 的无穷小变换
\begin{equation}
\phi(x) \rightarrow \phi'(x) = \phi(x) +\alpha \Delta \phi (x)
\end{equation}
这里 $\alpha$ 是一个无穷小参数,$\Delta \phi$ 是场的变化.如果这个变换\textbf{令 $\phi$ 场的运动方程保持不变}的话,我们就把这个变换称为一个\textbf{对称性}.因为拉式量的不变性总是跟运动方程的不变性相联系的,所以我们也可以说,如果这个变换令拉式量保持不变的话,我们就说这个变换是一个对称性.

要注意的点是如果一个变换令作用量的改变是一个全导数,我们也可以称这个变换是一个对称性.因为一个作用量的改变是一个全导数的时候,对应的运动方程仍然是不变的.具体来说就是,如果一个变换令运动方程的改变为如下形式的时候
\begin{equation}
\mathcal L(x) \rightarrow \mathcal L (x) +\alpha \partial_\mu \mathcal J^\mu (x)
\end{equation}
我们就可以说这个变换是一个对称.

我们可以对拉式量 $\mathcal L$ 进行变分.
\begin{align}\nonumber
\alpha \Delta \mathcal L & = \frac{\partial \mathcal L}{\partial \phi} (\alpha \Delta \phi) + \bigg( \frac{\partial \mathcal L}{\partial(\partial_\mu \phi)} \partial_\mu(\alpha \Delta \phi)\bigg) \\
& = \alpha \partial_\mu \bigg( \frac{\partial \mathcal L}{\partial (\partial_\mu\phi)} \Delta \phi \bigg) + \alpha \bigg[ \frac{\partial \mathcal L}{\partial \phi} - \partial_\mu \bigg( \frac{\partial \mathcal L}{\partial(\partial_\mu \phi)} \bigg) \bigg]
\end{align}
由欧拉-拉格朗日方程可知,第二项为零.剩余的第一项我们记作 $\alpha \partial_\mu \mathcal J$,于是我们有
\begin{equation}
\partial_\mu j^\mu(x) = 0~, \quad {\rm for}\quad j^\mu(x) = \frac{\partial \mathcal L}{\partial(\partial_\mu \phi)} \Delta \phi - \mathcal J^\mu
\end{equation}
这里 $j^\mu(x)$ 是守恒流.对于 $\mathcal L$ 的连续对称性来说,我们得到了这样一个守恒律.

守恒律的另一种表述是:电荷
\begin{equation}
Q \equiv \int_{\rm all\,\, space} j^0 d^3 x
\end{equation}
是一个不随时间变化而变化的常数.
\subsubsection{例子1:只有动能项的实标量场}
现在我们来举个最简单的例子,考虑只有动能项的标量场,其拉式量为
\begin{equation}
\mathcal L = \frac{1}{2} (\partial_\mu \phi)^2
\end{equation}
我们来考虑这样一个变换 $\phi \rightarrow \phi + \alpha $,在这个变换下拉式量不变.那么对应的流
\begin{equation}
j^\mu = \partial^\mu \phi
\end{equation}
就是守恒流.
\subsubsection{例子2:有质量的复标量场}
现在我们来考虑一个更复杂一些的例子,也就是有质量的标量场.拉式量如下
\begin{equation}
\mathcal L = |\partial_\mu\phi|^2 - m^2 |\phi|^2
\end{equation}
这里 $\phi$ 是一个复标量场.这个拉式量在 $\phi\rightarrow e^{i\alpha}\phi$ 变换下保持不变.对于无穷小变换
\begin{equation}\label{classi_eq18}
\alpha \Delta \phi = i \alpha \phi~,\quad \alpha \Delta \phi^* = -i\alpha \phi^*
\end{equation}
来说,我们可以推出对应的诺特流
\begin{equation}
j^\mu = i[(\partial^\mu \phi^*)\phi-\phi^*(\partial^\mu \phi)]
\end{equation}
是守恒的.这个 $j^\mu$ 就是场带的电磁场的流密度.而 $j^0$ 就是对应的电荷.

\begin{exercise}{证明上面的$j^\mu$是复标量场的诺特流}
从\autoref{classi_eq18} 的变换规则可知,
\begin{align}\nonumber
\alpha \Delta \mathcal L = \frac{\partial \mathcal L}{\partial\phi} (\alpha \Delta \phi) +\bigg( \frac{\partial \mathcal L}{\partial(\partial_\mu\phi)} \bigg) \partial_\mu(\alpha\Delta\phi) \\\nonumber
+\frac{\partial \mathcal L}{\partial\phi^*} (\alpha \Delta \phi^*) +\bigg( \frac{\partial \mathcal L}{\partial(\partial_\mu\phi^*)} \bigg) \partial_\mu(\alpha\Delta\phi^*) \\\nonumber
= \frac{\partial \mathcal L}{\partial\phi} ( i \alpha \phi) +\bigg( \frac{\partial \mathcal L}{\partial(\partial_\mu\phi)} \bigg) \partial_\mu(i \alpha \phi) \\\nonumber
+\frac{\partial \mathcal L}{\partial\phi^*} (i \alpha \phi^*) +\bigg( \frac{\partial \mathcal L}{\partial(\partial_\mu\phi^*)} \bigg) \partial_\mu(i \alpha \phi^*) \\\nonumber
= \frac{\partial \mathcal L}{\partial\phi} ( i \alpha \phi) +\partial_\mu \bigg(\bigg( \frac{\partial \mathcal L}{\partial(\partial_\mu\phi)} \bigg) (i \alpha \phi)\bigg) -\partial_\mu\bigg( \frac{\partial \mathcal L}{\partial(\partial_\mu)} \bigg)\\\nonumber
+\frac{\partial \mathcal L}{\partial\phi^*} (i \alpha \phi^*) + \partial_\mu\bigg(\bigg( \frac{\partial \mathcal L}{\partial(\partial_\mu\phi^*)} \bigg) (i \alpha \phi^*) \bigg)
\end{align}




\end{exercise}


诺特定理也可以用到时空的变换中.比如说时空的平移和旋转.比如我们考虑这样的时空平移
\begin{equation}
x^\mu \rightarrow x^\mu - a^\mu 
\end{equation}
场的变换是
\begin{equation}
\phi(x) \rightarrow \phi (x+a) = \phi (x) + a^\mu \partial_\mu \phi(x)
\end{equation}
因为拉式量也是一个标量,它的变换是
\begin{equation}
\mathcal L \rightarrow \mathcal L + a^\mu \partial_\mu \mathcal L  = \mathcal L + a^\nu \partial_\mu (\delta^\mu_\nu \mathcal L)
\end{equation}
那么现在我们得到了四个守恒流
\begin{equation}\label{classi_eq2}
T^\mu{}_\nu \equiv \frac{\partial \mathcal L}{\partial (\partial_\mu \phi)} \partial_\nu \phi - \mathcal L \delta^\mu{}_\nu
\end{equation}
这个就是能量动量张量.那时间平移不变对应的守恒量就是哈密顿量
\begin{equation}
H = \int T^{00} d^3 x = \int \mathcal H d^3 x
\end{equation}
空间平移不变对应的守恒量就是
\begin{equation}
P^i = \int T^{0i} d^3x = - \int \pi \partial_i \phi d^3 x 
\end{equation}
















