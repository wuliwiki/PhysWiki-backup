% LR电路

\footnote{本文参考了哈里德的《物理学基础》.}
LR电路指由电阻-电感组成的电路.

\subsection{充电}
假设原先电路中电流为$0$. $t=0$时刻,我们将一电动势为$\mathscr{E}_0$的电源连入LR电路.
\begin{figure}[ht]
\centering
\includegraphics[width=6cm]{./figures/LRC_1.pdf}
\caption{电路图1} \label{LRC_fig1}
\end{figure}

对该电路运用基尔霍夫第二定律\upref{Kirch}.别忘了,电感的两端由于自感现象将会有一个反向电动势$U_{ind}$\upref{Induct}.
$$
\mathscr{E}_0 - U_{ind} - IR = 0
$$
即
\begin{equation}\label{LRC_eq2}
\mathscr{E}_0 - L\dv{I}{t} - IR = 0
\end{equation}
这是一个\textsl{简单的}微分方程,解得
\begin{equation}\label{LRC_eq1}
I = \frac{\mathscr{E}_0}{R} (1-e^{-\frac{R}{L}t})
\end{equation}

\begin{figure}[ht]
\centering
\includegraphics[width=10cm]{./figures/LRC_3.pdf}
\caption{充电时$I-t$关系示意图} \label{LRC_fig3}
\end{figure}
可见,当时间足够久后,$I\to\frac{\mathscr{E_0}}{R}$.

从能量角度看,$U_{ind}$该项意味着部分能量被电感转换为磁场能\upref{BEng},而$IR$则意味着其余能量被电阻\upref{Resist}转换为热能等.

\subsection{放电}
假设$t=0$时刻,LR电路中的电流为$I_0$.现在我们撤去电源并以导线代替,此时电路中只有电阻和电感.
\begin{figure}[ht]
\centering
\includegraphics[width=6cm]{./figures/LRC_2.pdf}
\caption{电路图2} \label{LRC_fig2}
\end{figure}
从能量角度看,这相当于电感中的磁场能被放出,并被电阻转换为热能等.某种意义上,此时的电感就相当于一个小“电池”,他以磁场能的形式存储了一些可被利用的能量.

同理我们列出方程.这相当\autoref{LRC_eq2}  中$\mathscr{E}_0=0$的情况
\begin{equation}
- L\dv{I}{t} - IR = 0
\end{equation}
解得
\begin{equation}
I = I_0 e^{-\frac{R}{L}t}
\end{equation}
\begin{figure}[ht]
\centering
\includegraphics[width=8cm]{./figures/LRC_4.pdf}
\caption{放电时$I-t$关系示意图} \label{LRC_fig4}
\end{figure}
