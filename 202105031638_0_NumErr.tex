% 数值计算的误差

\subsection{误差}
误差(Errors)的基本概念在高中物理里面应该有所涉及,这里就不仔细展开了. 对于科学计算而言,我们所关注的主要是\textbf{相对误差(relative error)}和\textbf{绝对误差(absolute error)}.

如果我们把一个数据的实际值记做 $x$ , 把它的近似值记为 $\hat x$. 在这个近似的过程,误差产生的原因主要有:
\begin{itemize}
\item 测量中的误差
\item 运算和计算机表达中的不精确所引入的机器误差或者舍入误差(round-off error)
\item 数值方法和离散化带来的误差(discretization error)
\end{itemize}
尤其是后面两种,是科学计算中要面临的两个主要的误差来源,下面我们会对它们进行逐一分析. 因此,在评估和使用数值方法时,我们需要系统的衡量这两项误差,并且要\textbf{掌握它们的来源},对它们的\textbf{大小有足够的控制}.

\subsection{范数}
在了解误差以前,我们先来回顾一下线性代数中的一个常规概念,范数.

科学计算中经常会涉及到向量(vector),矩阵(matrix)和张量(tensor). 计算与它们有关的误差,需要使用更为一般化的“绝对值”函数,也就是范数 $\norm{\cdot}$.

下面是几个常见的范数,其中向量由小写字母表示,矩阵由大写字母表示.

\begin{itemize}
\item 1-范数: $\norm{v}_1 = (\abs{v_1} + \abs{v_2} + \dots + \abs{v_N})$
\item 2-范数:
\end{itemize}
