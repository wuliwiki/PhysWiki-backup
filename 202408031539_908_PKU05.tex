% 北京大学 2005 年 考研 量子力学
% license Usr
% type Note

\textbf{声明}:“该内容来源于网络公开资料,不保证真实性,如有侵权请联系管理员”

1. (60分)简答题,可直接写出结果。

(a) 约化普朗克常量$\hbar = ?$

(b) 氢原子、二维谐振子、三维谐振子的简并度分别为?

(c) 一维谐振子、二维谐振子的第一激发态的节拍数分别为?

(d) 已知 $l_\pm = l_x \pm il_y $,求 $[l_+, l_-]$, $[l^2, l_+]$。

(e) 求 $[\hat p, \frac{1}{r}]$, $[\hat p, r^2]$。

(f) 在$x$表象中自旋本征值为 $x_0$ 的坐标波函数和动量波函数。

(g) 在$p$表象中自旋本征值为 $p_0$ 的坐标波函数和动量波函数。

(h) 求 $\hat l_x,\hat l_y$ 的共同本征态。

(i) 在相似表象中求 $e^i\frac{\pi}{4}\sigma_x\alpha$,其中 $\alpha$是$S_z =\frac{\hbar}{2} $ 的自旋态。

(j) 写出二维谐振子的两个自量子量完全集。

(k) 粒子处于势 $V(x) = \frac{1}{2}m\omega^2x^2$中,试在动量表象中写出其薛定谔方程。

2.(40分)简答题,可直接写出结果。

(a)在自然单位制下,已知相位差为$V(x) = \frac{1}{2} (x-a)^2$, 能量本征值为 $\frac{13}{2}$, 在此能量本征态下$x,\hat p,x^2,\hat p^2$的平均值。

(b)证明 $F-H$ 定理,即 
$\left( \frac{\partial E_{\lambda}}{\partial \lambda} \right) = \left( \frac{\partial H}{\partial \lambda} \right)_n$。

(c)$\alpha,\beta$是自旋向上、向下态,有归一化本征函数$\Psi=c_1\alpha+c_2\beta$,求算符