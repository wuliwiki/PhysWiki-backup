% 大卫·希尔伯特(综述)
% license CCBYSA3
% type Wiki

本文根据 CC-BY-SA 协议转载翻译自维基百科\href{https://en.wikipedia.org/wiki/David_Hilbert}{相关文章}。

\begin{figure}[ht]
\centering
\includegraphics[width=6cm]{./figures/9019106ae7482c98.png}
\caption{1912年的希尔伯特} \label{fig_David_1}
\end{figure}
大卫·希尔伯特(David Hilbert,发音:/ˈhɪlbərt/;德语:[ˈdaːvɪt ˈhɪlbɐt];1862年1月23日 – 1943年2月14日)是德国数学家和数学哲学家,是他那个时代最具影响力的数学家之一。

希尔伯特发现并发展了广泛的基础性思想,包括不变理论、变分法、交换代数、代数数论、几何学基础、算子谱理论及其在积分方程中的应用、数学物理学,以及数学基础(特别是证明理论)。他采纳并捍卫了乔治·康托尔的集合论和超限数理论。1900年,他提出了一系列问题,为20世纪的数学研究指明了方向。

希尔伯特及其学生为建立严格的数学理论做出了贡献,并发展了现代数学物理中重要的工具。他是证明理论和数学逻辑的共同创始人。
\subsection{生活}  
\subsubsection{早期生活与教育}  
希尔伯特是奥托(Otto),一位县法官,和玛丽亚·特蕾莎·希尔伯特(Maria Therese Hilbert,原姓Erdtmann,一位商人的女儿)的长子和唯一的儿子。他出生在普鲁士省(当时属于普鲁士王国),具体地点是哥尼斯堡(根据希尔伯特本人所说)或哥尼斯堡附近的维劳(自1946年起称为兹南门斯克),当时他的父亲在该地工作。希尔伯特的祖父是大卫·希尔伯特,一名法官和秘密顾问(Geheimrat)。母亲玛丽亚对哲学、天文学和质数有兴趣,而父亲奥托则教他普鲁士的美德。父亲成为市法官后,家庭迁至哥尼斯堡。大卫的妹妹伊丽丝(Elise)在他六岁时出生。他在八岁时开始上学,比通常的入学年龄晚了两年。

1872年底,希尔伯特进入了弗里德里希中学(Friedrichskolleg Gymnasium,亦称哥尼斯堡皇家学院,是伊曼努尔·康德140年前曾就读的学校);然而,在一段不愉快的时期后,他于1879年底转学并于1880年初从更注重科学的威廉中学(Wilhelm Gymnasium)毕业。毕业后,希尔伯特于1880年秋季入读哥尼斯堡大学(“阿尔贝尔蒂纳”大学)。1882年初,赫尔曼·闵可夫斯基(Hermann Minkowski,希尔伯特比他年长两岁,同为哥尼斯堡人,但曾到柏林学习了三个学期)回到哥尼斯堡并进入了这所大学。希尔伯特与这位害羞但才华横溢的闵可夫斯基建立了终生的友谊。
\subsubsection{职业生涯}
\begin{figure}[ht]
\centering
\includegraphics[width=6cm]{./figures/2e091f1db63eaea3.png}
\caption{1886年的希尔伯特} \label{fig_David_2}
\end{figure}
\begin{figure}[ht]
\centering
\includegraphics[width=6cm]{./figures/f6ef6825f2a61abb.png}
\caption{1907年的希尔伯特} \label{fig_David_3}
\end{figure}
1884年,阿道夫·赫尔维茨(Adolf Hurwitz)从哥廷根大学来到哥尼斯堡大学,担任外籍教授(即副教授)。三人之间开始了密切且富有成果的学术交流,尤其是闵可夫斯基和希尔伯特,他们在各自的科学事业中多次互相影响。希尔伯特于1885年获得博士学位,博士论文题为《Über invariante Eigenschaften spezieller binärer Formen, insbesondere der Kugelfunktionen》(《关于特殊二元形式的不变性质,特别是球面谐波函数》),该论文是在费尔迪南·冯·林德曼(Ferdinand von Lindemann)的指导下写的。

希尔伯特于1886年到1895年期间,担任哥尼斯堡大学的私人讲师(Privatdozent)。1895年,在费利克斯·克莱因(Felix Klein)的帮助下,他获得了哥廷根大学数学教授的职位。在克莱因和希尔伯特的领导下,哥廷根大学成为了数学界的顶尖学府。他在那里度过了余生。
\subsubsection{哥廷根学派}
\begin{figure}[ht]
\centering
\includegraphics[width=8cm]{./figures/e40cafaed8f9a186.png}
\caption{哥廷根数学研究所。其新建筑由洛克菲勒基金会资助,希尔伯特和库朗于1930年共同揭幕。} \label{fig_David_4}
\end{figure}
希尔伯特的学生包括赫尔曼·外尔(Hermann Weyl)、国际象棋冠军埃马努埃尔·拉斯克(Emanuel Lasker)、恩斯特·策梅洛(Ernst Zermelo)和卡尔·古斯塔夫·亨佩尔(Carl Gustav Hempel)。约翰·冯·诺伊曼(John von Neumann)曾是他的助手。在哥廷根大学,希尔伯特与20世纪一些最重要的数学家共同工作,他的社交圈中包括了艾米·诺瑟(Emmy Noether)和阿隆佐·丘奇(Alonzo Church)等人。

希尔伯特在哥廷根的69名博士生中,有许多人后来成为了著名的数学家,包括(及其论文答辩年份):奥托·布卢门塔尔(Otto Blumenthal,1898年)、费利克斯·伯恩斯坦(Felix Bernstein,1901年)、赫尔曼·外尔(Hermann Weyl,1908年)、理查德·库朗(Richard Courant,1910年)、埃里希·黑克(Erich Hecke,1910年)、雨果·施泰因豪斯(Hugo Steinhaus,1911年)和威廉·阿克曼(Wilhelm Ackermann,1925年)。  

1902年至1939年间,希尔伯特担任《数学年刊》(Mathematische Annalen)的编辑,这是当时最重要的数学期刊之一。1907年,他被选为美国国家科学院的国际会员。


\begin{figure}[ht]
\centering
\includegraphics[width=6cm]{./figures/734202e44a335bb0.png}
\caption{希尔伯特与他的妻子凯瑟·耶罗施(1892年)} \label{fig_David_5}
\end{figure}
\begin{figure}[ht]
\centering
\includegraphics[width=6cm]{./figures/70a233823a56b9b1.png}
\caption{弗朗茨·希尔伯特} \label{fig_David_6}
\end{figure}
1892年,希尔伯特与凯瑟·耶罗施(Käthe Jerosch,1864–1945)结婚,她是哥尼斯堡一位商人的女儿,“是一位直言不讳、思想独立的年轻女士,与希尔伯特的独立思想不谋而合。”在哥尼斯堡期间,他们有了唯一的孩子,弗朗茨·希尔伯特(Franz Hilbert,1893–1969)。弗朗茨一生饱受精神疾病困扰,在他被送入精神病诊所后,希尔伯特曾说:“从今以后,我必须认为自己没有儿子。”他对弗朗茨的态度给凯瑟带来了相当大的痛苦。

希尔伯特认为数学家赫尔曼·闵可夫斯基是他“最好的和最忠实的朋友”。

希尔伯特在普鲁士福音教会接受洗礼并成长为一名加尔文主义者。[a] 后来他离开了教会,成为了一名不可知论者。[b] 他还认为,数学真理独立于上帝的存在或其他先验假设。[c][d] 当伽利略·伽利莱因未能坚持他的日心说理论时,希尔伯特对此提出异议:“但[伽利略]并不是傻瓜。只有傻瓜才会认为科学真理需要殉道;这在宗教中或许是必要的,但科学结果终会自己证明。”[e]