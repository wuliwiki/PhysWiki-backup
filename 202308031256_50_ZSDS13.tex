% 浙江理工大学 2013 年数据结构
% keys 浙江理工大学 2013 年数据结构
% license Xiao
% type Tutor

\subsection{一、 单选题(在每小题的四个备选答案中选出一一个正确答案。每小题2分,共20分。)}

1. 链表不具备的特点是 \\
A.可随机访问任一结点 \\
B .插入删除不需要移动元素 \\
C .不必事先估计存储空间 \\
D .所需空间与其长度成正比

2. 设线性表有n个元素,以下算法中,()在顺序表 上实现比在链表上实现效率更高。 \\
A.交换第0个元素与第1个元素的值 \\
B .顺序输出这n个元素的值 \\
C .输出第i(Osisn-1)个元素值 \\
D .输出与给定值x相等的元素在线性表中的序号

3. 设输入序列为a、b、C、d ,则借助栈所得到的输出序列不可能是()。 \\
A.a、b、c、d \\
B.d、c、b、a \\
C.a、c、d、b \\
D.d、a、b、c

4. 为解决计算机主机与打印机之间的速度不匹配问题,通常设计一个打印数据缓冲区, 主机将要输出的数据依次写入到该缓冲区,而打印机则依次从该缓冲区中取出数据。该缓冲区的逻辑结构应该是 \\
A.栈 $\qquad$ B.队列 $\qquad$ C.树 $\qquad$ D.图

5. 设哈夫曼树中的叶子结点总数为m ,若用二叉链表作为存储结构,则该哈夫曼树中总共有()个空指针域。 \\
A.2m $\qquad$ B.4m $\qquad$ C . 2m+1 $\qquad$ D.2m-1

6. 二叉树若用顺序存储结构表示,则下列四种运算中()最容易实现。 \\
A.先序遍历二叉树 $\qquad$ B.层次遍历二叉树 \\
C.中序遍历二叉树 $\qquad$ D.后序遍历二又树

7. 以下关于有向图的说法正确的是() \\
A .强连通图是任何顶点到其他所有顶点都有边 \\
B .完全有向图一定是强连通图 \\
C .有向图中某顶点的入度等于出度 \\
D .有向图边集的子集和顶点集的子集可构成原有向图的子图

8. 若一个有向图中的顶点不能排成一一个拓扑结构序列,则可断定该有向图 \\
A.含有多个出度为0的顶点 \\
B.是个强连通图 \\
C.含有多个入度为0的顶点 \\
D.含有顶点数目大于1的强连通分量

9. 顺序查找法适合于存储结构为的线性表。 \\
A.哈希存储 \\
B.压缩存储 \\
C.顺序存储或链式存储 \\
D.索引存储

10. 在所有排序方法中,关键字比较的次数与记录地初始排列次序无关的是() \\
A.shell排序 \\
B.冒泡排序 \\
C.直接插入排序 \\
D.简单选择排序

\subsection{二、填空题(每空2分,共30分。)}

1. 下面程序段的时间复杂度是
\begin{lstlisting}[language=cpp]
for (i=0; i<n; i++)
  for (j=0; j<m; j++)
    A[i][j]=0;
\end{lstlisting}

2. 向一个不带头节点的栈指针为Ist的链式栈中插入一个*s所指节点时,则执行(   )和(    )。

3. 在二叉链表中判断某指针p所指结点为叶子结点的条件是按()遍历一棵二叉排序树所得到的结点访问序列是一一个有序序列。

5. 广义表A=((a,b,c,d),( ))的表尾是(    ).

6. 有一个10阶对称矩阵A ,采用压缩存储方式(以行序为主存储,且A[0][0]=1) ,则A[8][5]的地址是(    )

7. 高度为h(>=0)的二叉树,至少有(    )个结点,最多有(    )个结点。

8. 普里姆(PRIM)算法更适合于求边(    )的网的最小生成树。

9. 在无向图G的邻接矩阵A中,若A[i][j]等于1 ,则A[j][i]等于(    ).

10. 在对一组记录(54, 38, 96, 23, 15, 72, 60, 45 , 83)进行直接插入排序时,当把第7个记录60插入到有序表时,为寻找插入位置需比较(    )次。

11. 若一组记录的排序码为( 46, 79, 56, 38, 40, 84) ,则利用堆排序的方法建立的初始堆为()。

12. 有一个长度为10的有序表,按折半查找法对该表进行查找,在表内各元素等概率情况下查找成功所需的平均比较次数为().

13. 在一棵平衡的二叉树中,每个节点的平衡因子B的取值范围是()。

\subsection{三、判断题(每小题2分,共20分。)}

1. 对于数据结构,相同的逻辑结构,对应的存储结构也必相同。( )

2. 哈夫曼树中没有度数为1的结点。( )

3. 线性表中的所有元素都有一个前驱元素和后继元素。()

4. 除了删除和插入操作外,数组的主要操作还有存取、修改、检索和排序。()

5. 链表的每一个结点都恰好包含一个指针。 ( )

6. 无向图的邻接矩阵一定是对称矩阵,且有向图的邻接矩阵一定是非对称矩阵。 ( )

7. 若有一个结点是某二叉树子树的中序遍历序列中的最后一个结点,则它必是该子树的前序遍历序列中的最后一个结点。( )

8. 冒泡排序在初始关键字序列为逆序的情况下执行的交换次数最多。 ( )

9. 满二叉树一定是完全二叉树,完詮二叉树不一定是满二叉树。( )

10. 快速排序是排序算法中平均性能最好的一种排序。( )

\subsection{四、应用题(共50分。)}

1. (14分)已知一棵二叉树如右图所示: \\
(1)中序全线索化二叉树; \\
(2)写出对该二叉树进行先序遍历和后序遍历的结果; \\
(3)试画出其相应的树。

2. (12分)已知某有向图的邻接矩阵为:
\begin{table}[ht]
\centering
\caption{第四2题图:邻接矩阵}\label{tab_ZSDS13_1}
\begin{tabular}{|c|c|c|c|c|c|c|c|c|c|c|}
\hline
 & V1 & V2 & V3 & V4 & V5 & V6 & V7 & V8 & V9 & V10 \\
\hline
V1 & 0 & 1 & 1 & 1 & 0 & 0 & 0 & 0 & 0 & 0 \\
\hline
V2 & 0 & 0 & 0 & 1 & 1 & 0 & 0 & 0 & 0 & 0 \\
\hline
V3 & 0 & 0 & 0 & 1 & 0 & 1 & 0 & 0 & 0 & 0 \\
\hline
V4 & 0 & 0 & 0 & 0 & 0 & 1 & 1 & 0 & 1 & 0 \\
\hline
V5 & 0 & 0 & 0 & 0 & 0 & 0 & 1 & 0 & 0 & 0 \\
\hline
V6 & 0 & 0 & 0 & 0 & 0 & 0 & 0 & 1 & 1 & 0 \\
\hline
V7 & 0 & 0 & 0 & 0 & 0 & 0 & 0 & 0 & 1 & 0 \\
\hline
V8 & 0 & 0 & 0 & 0 & 0 & 0 & 0 & 0 & 0 & 1 \\
\hline
V9 & 0 & 0 & 0 & 0 & 0 & 0 & 0 & 0 & 0 & 1 \\
\hline
V10 & 0 & 0 & 0 & 0 & 0 & 0 & 0 & 0 & 0 & 0 \\
\hline
\end{tabular}
\end{table}
(1)画出此图的对应邻接表,要求边结点按照序号从大到小排序; \\
(2)写出以(1)为存储结构的、顶点V1为出发点的深度优先遍历次序; \\
(3)写出以(1)为存储结构的、顶点V1为出发点的广度优先遍历次序。

3. (12分)设散列表的长度m=13,散列函数为H(K)=K mod m ,给定的关键码序列为20 , 11, 14 , 68 , 19, 23, 10, 1,84, 55, 27 , 79。 \\
(1)使用线性探查再散列法来构造散列表; \\
(2)并求出在等概率的情况下,这种方法在搜索成功时的平均搜索长度。

4. (12分)已知序列{503 , 87 , 512 , 61, 908, 170 , 897 , 275 , 653 , 462} ,采用基数排序法对该序列作升序排序时的每一趟的结果。

\subsection{五、算法设计题(每小题15分,共30分)}

1.设有两个集合A和B,要求设计生成集合C=A∩B的算法,其中集合A、B和C分别用链式存储结构表示。

2.设有一个顺序表L,其元素为整型数据,试编写一算法将L中所有小于0的整数放在前半部分,大于等于0的整数放在后半部分。
