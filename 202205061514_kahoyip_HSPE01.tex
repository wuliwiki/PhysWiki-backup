% 静电场(高中)
% 电荷|电场|库仑定律

\subsection{电荷}

经摩擦的物体能吸引轻小的物体,我们就说它带有\textbf{电荷}.电荷分为\textbf{正电荷}和\textbf{负电荷},用毛皮摩擦过的橡胶棒带负电荷,用丝绸摩擦过的玻璃棒带正电荷.

电荷间的相互作用规律:同种电荷相互排斥,异种电荷相互吸引.

\subsubsection{电荷量}

电荷量表示电荷的多少.电荷量的单位是\textbf{库仑}(简称\textbf{库}),符号为$\mathrm{C}$.

正电荷的电荷量为正值($+$号通常会省略),负电荷的电荷量为负值,要注意这里的正、负表示电荷种类,比较电荷量的多少要看电荷量的绝对值.

\subsubsection{元电荷}

又称\textbf{基本电荷},是一个电子或一个质子所带电荷量的绝对值,用字母$e$表示,$e = 1.602176634 \times 10^{-19} \mathrm{C}$\footnote{详见“物理学常数\upref{Consts}”},近似计算时取$e \approx 1.60 \times 10^{-19} \mathrm{C}$,任何带电体所带的电荷量都是元电荷的整数倍.
