% 渐进估计与阶
% keys 渐进|渐进等价|阶
% license Usr
% type Tutor

这里介绍大 $\mathcal O$ 符号、小 $\mathcal o$ 符号、$\asymp$、$\prec$、$\succ $ 等符号的意义。

对于自变量 $x$ 与其的函数 $f(x)$ 和其总正的函数 $\varphi(x)$。
\begin{itemize}
\item 若 $f=\mathcal O(\varphi)$ 表示存在一个足够大的常数 $A$ 使得对于任意 $x$ 总有 $|f(x)| < A \varphi(x)$。
\item $f = \mathcal o(\varphi)$ 表示 $f/\varphi \rightarrow 0$。
\item $f \sim \varphi$ 表示 $f / \varphi \rightarrow 1$。
\end{itemize}

例如当 $x \rightarrow +\infty$ 时:
\begin{equation}
\begin{aligned}
10x &= \mathcal O(x),\sin x &= \mathcal O(x), ~ x &= \mathcal O(x^2), ~\\
x &= \mathcal o(x^2),\sin x &= \mathcal o(x), ~ x + 1 &\sim x ~.
\end{aligned}
\end{equation}
