% Python模块
\pentry{Python 入门\upref{Python}}

模块是包含一组\textbf{函数}的文件, 希望在应用程序中引用.
\subsection{创建模块}
如需创建模块,只需将所需函数代码保存在文件扩展名为 \verb|.py| 的文件中. 例如
在名为 \verb|mymodule.py| 的文件中保存代码:
\begin{lstlisting}[language=python]
def greeting(name):
    print("Hello, " + name)

def saying():
    print("Hello! ")
\end{lstlisting}
在该模块中写了两个简单的函数.

\subsection{使用模块}
现在,我们就可以用 \verb|import| 语句来使用我们刚刚创建的模块.

导入名为 \verb|mymodule| 的模块,并调用 \verb|greeting| 函数与\verb|saying|函数.
\begin{lstlisting}[language=python]
import mymodule
mymodule.greeting("Bill")
mymodule.saying()
\end{lstlisting}
注释:如果使用模块中的函数时,请使用以下语法:\verb|module_name.function_name|.

\subsection{为模块命名}
可以随意对模块文件命名,但是文件扩展名必须是 \verb|.py|.

\subsection{重命名模块}
有时候模块名比较长,为了方便使用,可以在导入模块时使用 \verb|as| 关键字创建别名.
\begin{lstlisting}[language=python]
import mymodule as mx
mx.saying()
\end{lstlisting}