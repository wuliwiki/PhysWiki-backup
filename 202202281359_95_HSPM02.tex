% 相互作用
% keys 高中物理|相互作用|力|牛顿

\begin{issues}
\issueDraft
\issueTODO
\end{issues}

\subsection{力}
\subsubsection{基本概念}
力是物体对物体的\textbf{相互作用}.力不能离开物体而独立存在,力的产生一定同时涉及施力物体和受力物体.

符号:$\bvec F$.

单位:牛顿(简称 牛),$\mathrm N$.

\textbf{相互性}:物体间力的作用是相互的.一个物体对另一个物体施力时,另一个物体也同时对它施加力的作用.力总是成对出现的,具有同时性,不存在先后关系.

\textbf{矢量性}:力既有大小,也有方向.

\textbf{作用效果}:改变物体的形状(\textbf{形变});改变物体的运动状态(包括\textbf{速度大小}和\textbf{方向}).

\textbf{三要素}:\textbf{大小}、\textbf{方向}和\textbf{作用点}.三要素决定一个力,当其中一个要素改变时,力也发生改变,同时力的作用效果也随之改变.两个完全相同的力必须是三要素完全相同.

\textbf{力的图示}:用有向线段把力的三要素表示出来,箭头表示力的方向,端点表示力的作用点,按选定标度的线段长表示力的大小.

\textbf{力的示意图}:与力的图示基本一样,用适当长度的有向线段表示力,但无需按选定标度来表示大小.

\subsubsection{力的分类}
(1)根据性质(产生的原因):重力、弹力、摩擦力、电磁力、库仑力、分子力、核力等.

(2)根据作用效果:拉力、压力、动力、阻力、支持力等.

(3)根据研究对象:内力、外力等.

(4)根据作用方式:接触力、非接触力等.

\subsubsection{四种基本相互作用}
\textbf{万有引力}:一切具有质量的物体之间存在着相互吸引的力,相互作用的强度随距离的增大而减小,是长程力\footnote{长程力:作用强度随距离的增加而减少,从理论上说,可以作用到无限远.}.

\textbf{电磁相互作用}:带电粒子与电磁场的相互作用,以及带电粒子之间通过电磁场传递的相互作用,是长程力.

\textbf{强相互作用}:克服原子核内核子之间的斥力并维持原子核稳定的作用力,是短程力\footnote{短程力:作用范围很小,在原子核尺度内.}.

\textbf{弱相互作用}:在某些放射现象中起作用的一种基本相互作用,是短程力.

\subsection{常见力}

\subsubsection{弹力}
形变分为\textbf{弹性形变}(去掉外力能恢复原状)和\textbf{塑性形变}(去掉外力不能恢复原状).对于弹性形变来说,存在一个形变的极限,若形变超过这个极限,物体就不能完全恢复原状,这个极限称为\textbf{弹性限度}.

发生弹性形变的物体,由于要恢复原状而对与其接触的物体会产生力的作用,这种力叫做\textbf{弹力}.

\textbf{弹力的产生条件}:两个物体直接接触;两个物体发生形变并有恢复原状的趋势.

\textbf{弹力的三要素}:大小与物体的弹性强弱和形变程度有关;方向与物体形变的方向相反,指向恢复原状的方向;作用点在两物体的接触面上,作用在使物体发生弹性形变的施力物体上.

\textbf{胡克定律}:在弹性限度内,弹簧发生弹性形变时,弹力的大小跟弹簧伸长或缩短的长度成正比.表达式为

\begin{equation}
F=kx
\end{equation}

$k$为劲度系数,由弹簧本身决定,单位是$N/m$;$x$是形变量,即弹簧相对于原长的改变量.

\subsubsection{重力}
由于地球的吸引而使物体受到的力,叫做重力,通常用字母$G$表示,施力物体是地球.需要留意的是,重力是地球吸引而产生的,但其大小不一定等于地球的吸引力,不能说“重力就是地球对物体的吸引力”.

\textbf{重力的三要素}:大小和物体的质量成正比;方向竖直向下;作用点是重心.

\textbf{计算公式}:
\begin{equation}
G=mg
\end{equation}

$m$为物体的质量;$g$为重力系数,常取$g=9.8\mathrm{N/kg}$或$g=\mathrm{10N/kg}$.

\textbf{重心}:重心是物体各部分所受重力的\textbf{等效作用点}.重心的位置与物体的质量分布和形状有关,一旦物体的质量分布发生变化,其重心的位置也发生变化.对于形状为中心对称的物体,其重心位于对称中心.重心的位置不一定在物体上,如质量均匀分布的圆环,其重心在圆心处,不在圆环上.对于一般的物体,可以用悬挂法测量其重心.


\subsubsection{摩擦力}
当两个相互接触挤压的物体做\textbf{相对运动}或\textbf{有相对运动趋势}时,在接触面上会产生一种\textbf{阻碍物体发生相对运动}的力,叫做\textbf{摩擦力}.两个相互接触挤压的物体相对静止(相对于参考系静止或都做速度相同的运动)且存在相对运动趋势而没有发生相对运动时,在它们的接触面上产生的摩擦力叫做\textbf{静摩擦力}.两个相互接触挤压的物体因相对滑动而产生的摩擦力叫做\textbf{滑动摩擦力}.此外,初中物理中还介绍过\textbf{滚动摩擦力}.

\textbf{摩擦力的方向}:与物体相对运动或相对运动趋势的方向相反.

\textbf{摩擦力的作用点}:摩擦力作用在整个接触面上,但为了研究方便,可以把摩擦力的作用等效到一个点上,可以取在接触面上,也可以取在物体的重心上.

\textbf{摩擦力的产生条件}:(1)两个物体间有弹力;(2)接触面不光滑;(3)两个物体发生了相对滑动或有相对运动趋势.只有同时满足这三个条件,才能确定摩擦力的存在.

\textbf{滑动摩擦力的大小}:影响因素有二,一是两个物体接触面间的压力大小,二是接触面的粗糙程度.计算公式为
\begin{equation}
F_f=\mu F_N
\end{equation}

$\mu$为动摩擦因素,没有单位,与接触面的粗糙程度有关;$F_N$是两个物体接触面之间的压力,在性质力上是弹力,等于物体受到的支持力.

\textbf{静摩擦力的大小}:随外力的变化而变化,必须结合物体的运动状态及其受力情况确定.物体刚要发生相对运动时所受的静摩擦力称为最大静摩擦力$F_{max}$,略大于滑动摩擦力.

\subsection{力的合成与分解}
