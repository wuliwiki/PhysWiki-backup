% 斜对称双线性型的规范型
% keys 斜对称|双线性型|辛平面|规范型

\begin{issues}
\issueTODO
\end{issues}

\pentry{二次型的规范型\upref{GuaOQu}}
在二次型\upref{QuaFor}一节中,已经知道每一对称双线性型 $f$ 对应一个二次型 $q$ ,并且每一对称双线性型 $f$ 都有一规范基底,在此基底下,二次型 $q$ 呈现一种简单的形式
\begin{equation}
q(\bvec x)=\sum_{i}f_{ii}x_i^2
\end{equation}
现在我们转向斜对称的双线性型\autoref{MulMap_def1}~\upref{MulMap},也就是满足 
\begin{equation}
f(\bvec x,\bvec y)=-f(\bvec y,\bvec x)\quad \forall \bvec x,\bvec y\in V
\end{equation}
的2-线性函数.

对于斜对称双线性型 $f$,设其对应矩阵为 $F=(f_{ij})$ ,那么
\begin{equation}
f(\bvec x,\bvec y)=X^{T}FY=\sum_{1\leq i<j\leq n}f_{ij}(x_iy_j-x_jy_i)
\end{equation}

若 $V_0$ 是斜对称双线性型 $f$ 的\textbf{核},也就是子空间\upref{SubSpc}
\begin{equation}
V_0=\mathrm{Ker} f=\{\bvec v\in V|f(\bvec v,\bvec x)=0,\forall\bvec x\in V\}
\end{equation}
那么 $f$ 在 $V_0$ 的补空间(\autoref{DirSum_def1}~\upref{DirSum}) $V_1$ 上的限制 $f|_{V_1}$ 必为非退化的斜对称型.这是因为,如果 $\bvec a\neq0\in V_1$ 且 $f(\bvec a,\bvec x_1)=0$ 对所有的 $\bvec x_1\in V$ 都成立(这句话意味着 $f$ 是退化的,因为它将非零向量 $\bvec a$ 映射到零向量),那么任意向量 $\bvec x=\bvec x_0+\bvec x_1\in V \;(\bvec x_0\in V_0)$ ,有
\begin{equation}
f(\bvec a,\bvec x)=f(\bvec a,\bvec x_0+\bvec x_1)=f(\bvec a,\bvec x_0)+f(\bvec a,\bvec x_1)=-f(\bvec x_0,\bvec a)=0
\end{equation}
 这就是说 $\bvec a\in V_0$,而这是不可能的.

 上面论断使得我们可将对 $f$ 的研究归结为非退化的情形.即认为 $f:V\times V\rightarrow\mathbb{F}$ 是个非退化的双线性型.
 \subsection{斜对称双线性型的规范型}
 \begin{definition}{辛平面}
 若斜对称双线性型 $f$ 在矢量空间 $V$ 的二维子空间 $W$ 上的限制 $f|_W\neq0$ ,则二维子空间 $W$ 称为 $V$ 中的\textbf{辛平面}.
 \end{definition}
 辛平面也可描述为:对任意向量 $\bvec e'_1\neq0$,必存在一向量 $\bvec e'_2$ 使得 $f(\bvec e'_1,\bvec e'_2)\neq0$ .
 \begin{theorem}{}
 设 $V$ 是个实空间(也就是 $\mathbb{F}=\mathbb{R}$),其上有一个非退化的斜对称的双线性型 $f$.那么,$\mathrm{dim}\;V=2m$ 且 $V$ 是 $m$ 个对于 $f$ 而言两两斜正交的辛平面的直和.也就是说\autoref{VecSpn_def1}~\upref{VecSpn}
 \begin{equation}
 \begin{aligned}
 &V=\langle\bvec e_1,\cdots,\bvec e_n\rangle=\langle\bvec e'_1,\bvec e'_2\rangle\oplus\cdots\oplus\langle\bvec e'_{2m-1},\bvec e'_{2m}\rangle\\
&f(\alpha\bvec e'_{2i-1}+\beta\bvec e'_{2i},\gamma\bvec e'_{2j-1}+\delta\bvec e'_{2j})=0,\quad i\neq j\\
 &f(\bvec e'_{2i-1},\bvec e'_{2i})=1
 \end{aligned}
 \end{equation}

 \end{theorem}
\textbf{证明:}对 $n=\mathrm{dim} V$ 用归纳法.当 $n=2$ 时,显然对辛平面 $W=\langle \bvec e'_1,\bvec e'_2\rangle$ 成立.当 $n>2$ ,$W$ 在 $V$ 中的斜正交空间 $W^{\perp}$ 为
\begin{equation}
W^{\perp}=\langle\bvec x\in V|f(\bvec e'_i,\bvec x)=0,i=1,2\rangle
\end{equation}
把 $\bvec e'_1,\bvec e'_2$ 扩充成空间 $V$ 的基底:
\begin{equation}
V=\langle \bvec e'_1,\bvec e'_2,\cdots,\bvec e'_n\rangle, \bvec x=x'_1\bvec e'_1+\cdots+x'_n\bvec e'_n
\end{equation}
那么,
\begin{equation}
\begin{aligned}
&f(\bvec e'_1,\bvec x)=\sum_{i\neq 1}f'_{1i}x'_i=0\\
&f(\bvec e'_2,\bvec x)=\sum_{i\neq 2}f'_{2i}x'_i=0
\end{aligned}
\end{equation}
是秩为2的方程组(因为 $f$ 是非退换的,与之对应的矩阵 $F$ 的行自然线性无关).于是该方程组解空间 $\langle\bvec e'_1,\bvec e'_2\rangle^{\perp}$ 的维度是 n-2(\autoref{MatLS2_cor1}~\upref{MatLS2}).因为
\begin{equation}
\langle\bvec e'_1,\bvec e'_2\rangle\cap\langle\bvec e'_1,\bvec e'_2\rangle^{\perp}\in \mathrm{Ker} f
\end{equation}
而 $f$ 非退化意味着对任意非零向量 $\bvec a\in V$,至少有一向量 $\bvec b$ ,使得 $f(\bvec a,\bvec b)\neq0$,这表明$\mathrm{Ker} f=\bvec 0$.因此
\begin{equation}
\langle\bvec e'_1,\bvec e'_2\rangle\cap\langle\bvec e'_1,\bvec e'_2\rangle^{\perp}=\bvec 0
\end{equation}
于是
\begin{equation}
V=\langle\bvec e'_1,\bvec e'_2\rangle\oplus\langle\bvec e'_1,\bvec e'_2\rangle^{\perp}
\end{equation}
