% 拉马努金和(数论)
% keys 拉马努金和|Ramanujan|Ramanujan Sum|数论
% license Usr
% type Tutor

\pentry{数论三角和与高斯和\nref{nod_ntrtre}}{nod_8219}

\textbf{这里探讨的是数论中的拉马努金和(Ramanujan's Sum),非级数等中的拉马努金求和。}

\begin{definition}{拉马努金和}
\textbf{拉马努金和(Ramanujan's Sum)} $c(m; n)$ 定义为:
\begin{equation}
c(m; n) = c_n(m) = \sum_{a=1, \gcd(a, n)=1}^{n} e\left(\frac{ma}{n}\right) ~.
\end{equation}
\end{definition}

\pentry{单位根与本原单位根(数论)\nref{nod_priru}}{nod_49e2}
\begin{theorem}{}
利用本原单位根,可以将 Ramanujan 和表示为
\begin{equation}
c_q(m) = \sum \rho^m~,
\end{equation}
其中 $\rho$ 取遍 $q$ 的所有本原 $q$ 次单位根。
\end{theorem}

% 由定义就不难有 $c(m; n) = \sum_{d | \gcd(m, n)} (\mu(n/d) d)$。

% 特别的,当 $n | m$ 时,$c(m; n) = \varphi(n)$,是欧拉函数。
