% 开普勒定律(综述)
% license CCBYSA3
% type Wiki

本文根据 CC-BY-SA 协议转载翻译自维基百科\href{https://en.wikipedia.org/wiki/Kepler\%27s_laws_of_planetary_motion}{相关文章}。

\begin{figure}[ht]
\centering
\includegraphics[width=8cm]{./figures/e5bb3dbe32521ea0.png}
\caption{用两个行星轨道说明开普勒定律。这些轨道是椭圆形的,行星1的焦点为F1和F2,行星2的焦点为F1和F3。太阳位于F1。阴影区域A1和A2是相等的,并且是由行星1的轨道在相等的时间内扫过的。行星1的轨道周期与行星2的轨道周期的比值为 \(\left(\frac{a_1}{a_2}\right)^{3/2}\)。} \label{fig_KPL_1}
\end{figure}
在天文学中,开普勒的行星运动定律由约翰内斯·开普勒于1609年发布(除了第三定律,后者于1619年完全发布),描述了行星围绕太阳的轨道。这些定律用椭圆轨道代替了哥白尼日心说中的圆形轨道和本轮,并解释了行星速度的变化。这三条定律如下:
\begin{enumerate}
\item 行星的轨道是椭圆,太阳位于其中一个焦点上。
\item 连接行星和太阳的线段在相等的时间间隔内扫过相等的面积。
\item 行星轨道周期的平方与其轨道半长轴的立方成正比。
\end{enumerate}
行星的椭圆轨道通过火星轨道的计算得到了证明。从这些计算中,开普勒推断出太阳系中其他天体(包括距离太阳较远的天体)也具有椭圆轨道。第二定律确定了当行星靠近太阳时,其运动速度较快。第三定律表达了行星距离太阳越远,其轨道周期越长。

艾萨克·牛顿在1687年证明,像开普勒定律这样的关系,作为他自己运动定律和万有引力定律的结果,也适用于太阳系。

更精确的历史方法可以在《新天文学》和《哥白尼天文学概要》中找到。

