% 牛顿运动定律(综述)
% license CCBYSA3
% type Wiki

(本文根据 CC-BY-SA 协议转载翻译自维基百科\href{https://en.wikipedia.org/wiki/Newton\%27s_laws_of_motion#}{相关文章})

牛顿运动定律是三条描述物体运动与作用于其上的力之间关系的物理定律。这些定律构成了牛顿力学的基础,可概括为以下内容:

\begin{enumerate}
\item 一个物体将保持静止状态或以恒定速度沿直线运动,除非有力作用在其上。
\item 在任一时刻,物体所受的合力等于物体的加速度乘以其质量,或等同于物体动量随时间变化的速率。
\item 如果两个物体相互施加力,这些力的大小相等但方向相反。[1][2]
\end{enumerate}
牛顿的三大运动定律最早由艾萨克·牛顿在他的著作《自然哲学的数学原理》(拉丁文:《Philosophiæ Naturalis Principia Mathematica》)中提出,该书首次出版于1687年。牛顿利用这些定律研究并解释了许多物理物体和系统的运动。自牛顿以来,新的见解,特别是关于能量的概念,基于他的基础构建了经典力学的领域。同时也发现了牛顿定律的局限性:当物体以极高的速度运动时(狭义相对论)、质量极大时(广义相对论)、或在极小尺度上(量子力学),需要新的理论来解释。
\subsection{前提条件 }
牛顿定律通常是以点或质点的形式表述的,即体积可以忽略不计的物体。这在实际物体中是一个合理的近似,当内部各部分的运动可以忽略不计,并且物体之间的距离远大于它们各自的大小时,牛顿定律是适用的。例如,当考虑地球围绕太阳的轨道时,可以将地球和太阳都近似为点状物体,但在考虑地球表面的活动时,地球显然不是点状物体。[注1]

运动的数学描述,即运动学,是基于使用数值坐标来指定位置的概念。物体的运动通过这些数值随时间变化来表示:物体的轨迹是一个函数,它将每个时间变量的值与所有位置坐标的值关联起来。最简单的情况是一维运动,也就是当一个物体只能沿直线运动时。此时,它的位置可以用一个数值表示,说明它相对于某个选定参考点的位置。例如,一个物体可能在一条从左到右的轨道上自由滑动,因此它的位置可以通过距离某个方便的零点(原点)的距离来指定,负数表示位于左侧的位置信息,正数表示位于右侧的位置信息。如果物体的位置是时间的函数 \( s(t) \),那么物体在从 \( t_0 \) 到 \( t_1 \) 的时间间隔内的平均速度为:
\[
\frac{\Delta s}{\Delta t} = \frac{s(t_1) - s(t_0)}{t_1 - t_0}~
\]
在这里,按照传统使用希腊字母 Δ(delta)表示“变化量”。正的平均速度意味着在所讨论的时间间隔内,位置坐标 \( s \) 增加;负的平均速度表示该间隔内的位置净减少;而零的平均速度则意味着物体在时间间隔结束时与开始时位于同一位置。微积分提供了定义瞬时速度的方法,它衡量物体在某一时刻的速度和运动方向,而不是在一个时间区间内。瞬时速度的一种表示法是用符号 \( d \) 替代 \( \Delta \),例如:
\[
v = \frac{ds}{dt}~
\]
这表示瞬时速度是位置随时间变化率的导数。

这表明瞬时速度是位置对时间的导数。大致可以理解为位置的微小变化 \( ds \) 与发生在微小时间间隔 \( dt \) 之间的比率【7】。更准确地说,速度和所有其他导数都可以通过极限的概念来定义【6】。一个函数 \( f(t) \) 在给定的输入值 \( t_0 \) 处有极限 \( L \),如果通过选择足够接近 \( t_0 \) 的输入,可以使 \( f(t) \) 与 \( L \) 之间的差异任意小。我们写作:
\[
\lim _{t\to t_{0}}f(t)=L~
\]
瞬时速度可以定义为当时间间隔趋于零时,平均速度的极限:
\[
\frac{ds}{dt}=\lim _{\Delta t\to 0}\frac{s(t+\Delta t)-s(t)}{\Delta t}~
\]
加速度与速度的关系,就像速度与位置的关系一样:它是速度对时间的导数【注2】。加速度同样可以通过极限定义:
\[
a=\frac{dv}{dt}=\lim _{\Delta t\to 0}\frac{v(t+\Delta t)-v(t)}{\Delta t}~
\]
因此,加速度是位置的二阶导数,通常写作:
\[
\frac{d^{2}s}{dt^{2}}~
\]