% 2009 年计算机学科专业基础综合全国联考卷
% 2009年 计算机 考研 全国卷

\subsection{一、单项选择题}
第1~40 小题,每小题2 分,共80 分.下列每题给出的四个选项中,只有一个选项最符合试题要求.

1. 为解决计算机主机与打印机之间速度不匹配问题,通常设置一个打印数据缓冲区,主机将要输出的数据依次写入该缓冲区,而打印机则依次从该缓冲区中取出数据.该缓冲区的逻辑结构应该是. \\
A.栈 $\qquad$ B.队列 $\qquad$ C.树 $\qquad$ D.图

2. 设栈S和队列Q的初始状态均为空,元素a,b,c,d,e,f,g依次进入栈S.若每个元素出栈后立即进入队列Q,且7个元素出队的顺序是b,d,c,f,e,a,g,则栈S的容量至少是. \\
A.1 $\qquad$ B.2 $\qquad$ C.3 $\qquad$ D.4

3. 给定二叉树如图所示.设N 代表二叉树的根,L 代表根结点的左子树,R 代表根结点的右子树.若遍历后的结点序列是3,1,7,5,6,2,4,则其遍历方式是. \\
\begin{figure}[ht]
\centering
\includegraphics[width=5cm]{./figures/CSN09_1.png}
\caption{第3题图} \label{CSN09_fig1}
\end{figure}
A.LRN $\qquad$ B.NRL $\qquad$ C.RLN $\qquad$ D.RNL

4. 下列二叉排序树中,满足平衡二叉树定义的是______.\\
\begin{figure}[ht]
\centering
\includegraphics[width=14.25cm]{./figures/CSN09_2.png}
\caption{第4题图} \label{CSN09_fig2}
\end{figure}

5. 已知一棵完全二叉树的第6 层(设根为第1 层)有8 个叶结点,则该完全二叉树的结点个数最多是______. \\
A. 39 $\qquad$ B.52 $\qquad$ C.111 $\qquad$ D.119

6. 将森林转换为对应的二叉树,若在二叉树中,结点u是结点v的父结点的父结点,则在原来的森林中,u和v可能具有的关系是______. \\
Ⅰ.父子关系 $\qquad$ Ⅱ.兄弟关系 \\
Ⅲ.u的父结点与v的父结点是兄弟关系 \\
A. 只有Ⅱ $\qquad$ B.Ⅰ和Ⅱ $\qquad$ C.Ⅰ和Ⅲ $\qquad$ D.Ⅰ、Ⅱ和Ⅲ

7. 下列关于无向连通图特性的叙述中,正确的是______. \\
I. 所有顶点的度之和为偶数 \\
II. 边数大于顶点个数减1 \\
III. 至少有一个顶点的度为1 \\
A. 只有Ⅰ $\qquad$ B.只有Ⅱ $\qquad$ C.Ⅰ和Ⅱ $\qquad$ D.Ⅰ和Ⅲ

8. 下列叙述中,\textbf{不}符合m阶B树定义要求的是______. \\
A.根节点最多有m棵子树 $\qquad$ B.所有叶结点都在同一层上 \\
C.各结点内关键字均升序或降序排列 $\qquad$ D.叶结点之间通过指针链接

9. 已知关键字序列5,8,12,19,28,20,15,22 是小根堆(最小堆),插入关键字3,调整后得到的小根堆是______. \\
A.3,5,12,8,28,20,15,22,19 \\
B.3,5,12,19,20,15,22,8,28 \\
C.3,8,12,5,20,15,22,28,19 \\
D.3,12,5,8,28,20,15,22,19 \\

10. 若数据元素序列11,12,13,7,8,9,23,4,5 是采用下列排序方法之一得到的第二趟排序后的结果,
则该排序算法只能是______. \\
A.起泡排序 $\qquad$ B.插入排序 $\qquad$ C.选择排序 $\qquad$ D.二路归并排序

11. 冯•诺依曼计算机中指令和数据均以二进制形式存放在存储器中,CPU区分它们的依据是. \\
A.指令操作码的译码结果 $\qquad$ B.指令和数据的寻址方式 \\
C.指令周期的不同阶段 $\qquad$ D.指令和数据所在的存储单元

12. 一个C语言程序在一台32位机器上运行.程序中定义了三个变量x、y和z,其中x和z为int型,y为short型.当x=127,y=-9时,执行赋值语句z=x+y 后,x、y和z的值分别是. \\
A.x=0000007FH,y=FFF9H,z=00000076H \\
B. x=0000007FH,y=FFF9H,z=FFFF0076H \\
C. x=0000007FH,y=FFF7H,z=FFFF0076H \\
D. x=0000007FH,y=FFF7H,z=00000076H

13. 浮点数加、减运算过程一般包括对阶、尾数运算、规格化、舍入和判溢出等步骤.设浮点数的阶码和尾数均采用补码表示,且位数分别为5位和7位(均含2位符号位).若有两个数X=27×29/32,Y=25×5/8,则用浮点加法计算X+Y 的最终结果是. \\
A.00111 1100010 $\qquad$ B.00111 0100010 \\
C.01000 0010001 $\qquad$ D.发生溢出

14. 某计算机的Cache共有16块,采用2路组相联映射方式(即每组2块).每个主存块大小为32字节,按字节编址.主存129 号单元所在主存块应装入到的Cache组号是. \\
A.0 $\qquad$ B.1 $\qquad$ C.4 $\qquad$ D.6

