% 导数的性质与构造(高中)
% keys 导数|性质|构造|恒等
% license Usr
% type Tutor

\begin{issues}
\issueDraft
\end{issues}

\pentry{导数\nref{nod_HsDerv},函数的性质\nref{nod_HsFunC}}{nod_139b}

导数是分析函数变化规律的关键工具,它揭示了函数随变量变化时的趋势和行为。可以把导函数看作原函数的一张“素描画”,虽然简化了某些信息,但保留了足够的信息来描述函数的几乎所有性质。例如,导数可以用来判断函数在某一区间内是上升还是下降,还可以分析图像的弯曲方向是“开口向上”还是“开口向下”,甚至帮助找到最高点和最低点的位置。这种功能类似于在地图上标注道路的坡度,通过这些标记可以快速了解路段的起伏情况。同样,导函数为函数的变化贴上了清晰的“标签”,使得人们能够一目了然地把握它在不同区间的行为。

在解题过程中,导数相关的典型问题主要包括以下三类:

\begin{itemize}
\item \textbf{最值问题}:通过确定导数为零的位置,分析函数的变化趋势,找到函数的最高点或最低点。
\item \textbf{零点问题}:利用导数得到单调区间,结合区间两端函数值的符号情况,利用\aref{零点存在定理}{the_HsFunC_1},判断函数在某区间内的零点情况。
\item \textbf{恒成立问题}:通过导数符号的变化,分析函数在某区间内的单调性和极值的符号,判断函数是否满足某些恒成立条件。
\end{itemize}

解题的核心是找出导数为零的点,并分析导数在不同区间内的正负变化,从而勾勒出函数的变化趋势和形态。在此过程中,构造适当的函数并进行求导,以及合理选取区间和相异的函数值,是解决这类问题的两个主要难点。本文会对此提供一些常见的方法。

\subsection{近似代替}

在导数的\aref{几何含义}{sub_HsDerv_1}中就提到过“以直代曲”。

\begin{equation}
f(x_0+\Delta x)\approx f(x_0)+f'(x_0)\Delta x~.
\end{equation}

导函数的奇偶性通常与原函数相反,而它们的周期性却保持一致。

\subsection{单调性和极值点}
\subsubsection{单调性}
在介绍\aref{导函数}{sub_HsDerv_2}时,提及区间的中函数的增减与导函数的符号相关。
\begin{theorem}{单调性与导数的关系}
导函数  $f'(x)$  代表了原函数  f(x)  图像在每一点的切线斜率。
\begin{itemize}
\item 在$f'(x)>0$的区间上,原函数的图像单调递增。
\item 在$f'(x)<0$的区间上,原函数的图像单调递减。
\item 在$f'(x)=0$的区间上,原函数的图像是水平的。
\end{itemize}
\end{theorem}


\subsubsection{极值点}

\begin{definition}{驻点}
对于函数$y=f(x)$,如果某点$x_0$满足$f'(x_0)=0$,即$x_0$是$f'(x)$的零点,则称$x_0$为$f(x)$的\textbf{驻点(stationary point)}。
\end{definition}

\begin{definition}{极值点}
对于函数 $y=f(x)$,若对于一点 $x_0$ ,存在的某一邻域 $\left( x_0-\delta,x_0+\delta \right)$ $\left( \delta>0 \right)$ ,使对于此邻域中的任意点 $x$ ,都有 $f(x) \leq f\left(x_0\right)$ ,称 $x_0$ 是的极大值点;若都有 $f(x) \geq f\left(x_0\right)$ ,则称 $x_0$ 为极小值点。极大值点与极小值点统称为极值点。
\end{definition}

明细极值点

若  是 的极值点,那么  只可能是  的零点或 的不可导点。


$f'(x) = 0$的点表示函数的输出值停止增加或减少的点,被称为\textbf{驻点}。在该点是水平的,可能是极值点。

严格的来说,在点x0某一邻域内,f(x0)>=f(x)或f(x0)<=f(x),x0才是极值点。
f’(x)=0的点并不一定是极值点,在该点附近f’(x)必须要变号,如y=x^3,在x=0处一阶导数为0,但两侧不变号,就不是极值点
极值点与该点的一阶导数是否存在无关,只要两侧的f’(x)变号,就是极值点,如y=|x|,在x=0处就是极值点,但在x=0处一阶导数不存在。

\subsection{高阶导数}

导函数作为原函数,则又可以求得它的导函数,这也被称为高阶导数。

凹凸性

连续曲线的凹弧与凸弧的交界点。(在该点的二阶导并不一定有定义)
f"(x)=0,且该点 两侧 二阶导数变号,那么该点就是极值点。
当然,可能在一点x0处,二阶导数并不存在,在x0左侧的二阶导数趋于正无穷,右侧的二阶导数趋于负无穷,该点也是拐点。

\subsection{常用构造}

\pentry{恒等式与不等式恒成立\nref{nod_HsIden}}{nod_b069}

\subsubsection{逆向使用求导法则}

逆用积法则:

$x^n f'(x) + f(x) \geq 0$,构造 $F(x) = x^n f(x)$,$[x^n f(x)] = x^n f'(x) + nx^{n-1} f(x) = x^{n-1} [x f'(x) + nf(x)]$。特别地,当$n=1$时有$x f'(x) + f(x) \geq 0$,构造 $F(x) = x f(x)$,$[x f(x)]' = x f'(x) + f(x)$

$f'(x) + k f(x) \geq 0$,构造 $F(x) = \E^{kx} f(x)$,$[e^{kx} f(x)]' = e^{kx} [f'(x) + kf(x)]$。特别地,当$k=1$时有$f'(x) + f(x) \geq 0$,构造 $F(x) = \E^x f(x)$,$[e^x f(x)]' = e^x [f'(x) + f(x)]$

逆用商法则:

$xf'(x) - f(x) \geq 0$,构造 $F(x) = \frac{f(x)}{x}$,  
    $\therefore \left[\frac{f(x)}{x}\right]' = \frac{f'(x) \cdot x - f(x)}{x^2}$

$f'(x) - f(x) \geq 0$,构造 $F(x) = \frac{f(x)}{e^x}$,  
    $\therefore \left[\frac{f(x)}{e^x}\right]' = \frac{e^x \cdot f'(x) - e^x \cdot f(x)}{e^{2x}} = \frac{f'(x) - f(x)}{e^x}$

$x^n f'(x) - n f(x) \geq 0$,构造 $F(x) = \frac{f(x)}{x^n}$,  
    $\therefore \left[\frac{f(x)}{x^n}\right]' = \frac{x^n \cdot f'(x) - n x^{n-1} \cdot f(x)}{x^{2n}} = \frac{f'(x) - n f(x)}{x^{n+1}}$

$f'(x) - k f(x) \geq 0$,构造 $F(x) = \frac{f(x)}{e^{kx}}$,  
    $\therefore \left[\frac{f(x)}{e^{kx}}\right]' = \frac{e^{kx} \cdot f'(x) - k e^{kx} \cdot f(x)}{e^{2kx}} = \frac{f'(x) - k f(x)}{e^{kx}}$
