% 中科院2012年普物
% 中科院|2012|普通物理
\subsection{选择题}
1. 如\autoref{CAS12_fig1} 所示, 两个固定小球质量分别是 $m_{1}$ 和 $m_{2}$, 在它们的连线上总可以找到一
点 $p$, 使得质量为 $m$ 的质点在该点所受到的万有引力的合力为零, 则质点在 $p$
点的万有引力势能\\
(A) 与无穷远处的万有引力势能相等;\\
(B) 与该质点在两小球连线间其它各点处相比势能最大;\\
(C) 与该质点在两小球连线间其它各点处相比势能最小;\\
(D) 无法判定.
\begin{figure}[ht]
\centering
\includegraphics[width=3.5cm]{./figures/CAS12_1.pdf}
\caption{选择题1图示} \label{CAS12_fig1}
\end{figure}
2.两个全同的均质小球 $\mathrm{A}$ 和 $\mathrm{B}$ 都放置在光滑水平面上, 球 $\mathrm{A}$ 静止.在某一时刻 球 $\mathrm{B}$ 与 $\mathrm{A}$ 发生完全弹性斜碰撞(即碰撞时 $\mathrm{A}$、 $\mathrm{~B}$ 的质心连线方向与球 $\mathrm{B}$ 的速 度方向不同),则碰撞后两球的速度方向\\
(A) 相同;$\quad$
(B) 夹角为锐角;$\quad$
(C)相垂直;$\quad$
(D) 夹角为钝角.

3. 一质点同时参与相互垂直的两个谐振动, 且振动的频率相等.下列说法错误 的是\\
(A) 若两振动的初相位相同, 则质点轨迹为直线段;\\
(B) 若两振动的初相位相差 $\pi / 4$, 且振幅相等, 则质点轨迹为椭圆;\\
(C) 若两振动的初相位相差 $\pi / 2$, 且振幅不相等, 则质点轨迹为䧎圆;\\
(D) 若两振动的初相位相差 $\pi$, 且振幅相等, 则质点轨迹为圆.

4. 一个电量为 $q$ 、质量为 $m$ 的带电粒子在匀强磁场中作半径为 $r$ 的圆周运动.如果运动的频率是 $f$, 则磁感应强度大小为\\
(A) $\frac{4 \pi m f}{q}$;\\
(B) $\frac{3 \pi m f}{q}$;\\
(C) $\frac{2 \pi m f}{q}$;\\
(D) $\frac{\pi m f r}{q}$ .

5. 无限长直导线均匀带电, 电荷线密度为 $\lambda$ .距直导线距离 $r$ 处的电场强度大小为\\
(A) $\frac{\lambda}{4 \pi \varepsilon_{0} r^{2}}$;\\
(B) $\frac{\lambda}{4 \pi \varepsilon_{0} r}$;\\
(C) $\frac{\lambda}{2 \pi \varepsilon_{0} r^{2}}$;\\
(D) $\frac{\lambda}{2 \pi \varepsilon_{0} r}$ .

6. 在国际单位制中, 磁通量的量纲为\\
(A) $M L^{2} T^{-2} I^{-1}$;$\quad$
(B) $M L^{2} T I^{-1}$;$\quad$
(C) $M L^{2} T^{-2} I$;$\quad$
(D) $M L T^{-2} I^{-1}$ .

7. 关于平衡态下理想气体,以下哪个说法是错误的?\\
(A) 分子大小比分子间的平均距离小得多, 分子的大小可以忽略不计;\\
(B) 除碰撞瞬间外, 分子之间以及分子与容器壁之间都没有相互作用力;\\
(C) 各个分子的速度大小相同;\\
(D) 分子向各个方向运动的几率均等.

8. 动能相同的电子与质子的德布罗意波长哪个较长?\\
(A) 电子;$\quad$
(B) 质子;$\quad$
(C) 一样长;$\quad$
(D) 不能确定

\subsection{简答题}

1. 荡秋千时,为什么人可以越荡越高, 而固定在秋千上的物体却越荡越低?试 分析其原因并简述之.

2. 试写出真空中麦克斯韦方程组的积分形式, 并简述位移电流的含义.

3. 什么是牛顿环?它的特点是什么?

三、 两根相同的均质杆 $A B$ 和 $B C$, 质量均为 $m$, 长均为 $l, A$ 端被光 滑铰链到一个固定点, 两杆始终在竖直平面内运动. $C$ 点有外力使得两杆保
持静止, $A $、$ C$ 在同一水平线上, $\angle A B C=90^{\circ}$ .某时刻撤去该力,\\
(1) 若两杆在 $\mathrm{B}$ 点固结在一起, 求初始瞬间两杆的角加速率;\\
(2) 若两杆在 $\mathrm{B}$ 点光滑铰接在一起, 求初始瞬间两杆的角加速率.