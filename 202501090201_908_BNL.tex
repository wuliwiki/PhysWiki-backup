% 丹尼尔·伯努利(综述)
% license CCBYSA3
% type Wiki

本文根据 CC-BY-SA 协议转载翻译自维基百科\href{https://en.wikipedia.org/wiki/Michael_Faraday}{相关文章}。

\begin{figure}[ht]
\centering
\includegraphics[width=6cm]{./figures/12ffafd992474bb6.png}
\caption{丹尼尔·伯努利的肖像,约1720-1725年} \label{fig_BNL_1}
\end{figure}
丹尼尔·伯努利 FRS(/bɜːrˈnuːli/ 伯-努-利;瑞士标准德语:[ˈdaːni̯eːl bɛrˈnʊli];1700年2月8日[公历1月29日] – 1782年3月27日)是一位瑞士数学家和物理学家,并且是来自巴塞尔的伯努利家族中的众多杰出数学家之一。他特别以其将数学应用于力学,尤其是流体力学,以及在概率和统计学领域的开创性工作而闻名。[3] 他的名字在伯努利原理中得以纪念,这一原理是能量守恒的一个具体例子,描述了支撑20世纪两项重要技术——化油器和飞机机翼——运作原理的数学机制。[4][5]
\subsection{早年生活}
\begin{figure}[ht]
\centering
\includegraphics[width=6cm]{./figures/5f34cc0a7fd312e1.png}
\caption{《流体动力学》(1738年)封面} \label{fig_BNL_2}
\end{figure}
丹尼尔·伯努利出生于荷兰格罗宁根,来自一个著名的数学家家庭。[6] 伯努利家族最初来自安特卫普,当时属于西属荷兰,但为了逃避西班牙对新教徒的迫害,家族移民至瑞士。伯努利家族在短暂居住法兰克福后,最终定居巴塞尔。

丹尼尔是约翰·伯努利(微积分的早期发展者之一)之子,也是雅各布·伯努利(概率论的早期研究者及数学常数e的发现者)之侄。[6] 他有两个兄弟,尼克劳斯和约翰二世。丹尼尔·伯努利被W. W. 罗丝·鲍尔描述为“年轻伯努利家族中最有才华的成员”。[7]

据说,他与父亲的关系很差。他们都参加了巴黎大学的一场科学竞赛,并且并列第一。约翰因无法忍受被认为与丹尼尔平起平坐的“耻辱”,将丹尼尔逐出了家门。约翰据称曾在他的《水力学》一书中抄袭了丹尼尔的《流体动力学》一书中的关键思想,并将这些想法的时间回溯到《流体动力学》出版之前。[citation needed] 丹尼尔曾尝试与父亲和解,但都未能成功。[8]

在上学时,约翰鼓励丹尼尔学习商业,理由是数学家们的收入非常微薄。丹尼尔最初拒绝了,但后来在父亲的要求下,决定一边学习商业一边学习医学,前提是父亲会私下教授他数学。[8] 丹尼尔在巴塞尔、海德堡和斯特拉斯堡学习医学,并于1721年获得解剖学和植物学博士学位。[9]

他是莱昂哈德·欧拉的同代人和亲密朋友。[10] 1724年,他作为数学教授前往圣彼得堡,但在那里非常不愉快。一次短期的生病[8]、俄国东正教会的审查[11]以及与薪水的争执让他找到了离开圣彼得堡的借口,并于1733年离开。[12] 他回到巴塞尔大学,先后担任医学、形而上学和自然哲学的教授直至去世。[13]

1750年5月,他被选为皇家学会会士。[14]



他最早的数学著作是《Exercitationes》(数学练习),该书于1724年在戈尔德巴赫的帮助下出版。两年后,他首次指出,常常需要将复合运动分解为平移运动和旋转运动。他的主要著作是《Hydrodynamica》(流体动力学),于1738年出版。这本书与约瑟夫·路易·拉格朗日的《解析力学》相似,按单一原理——能量守恒原理来安排,其中的所有结果都可以视为该原理的推论。随后,他发表了关于潮汐理论的论文,与欧拉和科林·麦克劳林的论文一起,获得了法国科学院的奖项:这三篇论文包含了从艾萨克·牛顿的《自然哲学的数学原理》出版到皮埃尔-西蒙·拉普拉斯的研究期间,关于潮汐问题的所有成果。伯努利还撰写了大量关于各种力学问题的论文,特别是与振动弦相关的问题,并给出了布鲁克·泰勒和让·勒龙·达朗贝尔的解答。[7]