% 帕斯夸尔·约尔丹(综述)
% license CCBYSA3
% type Wiki

本文根据 CC-BY-SA 协议转载翻译自维基百科\href{https://en.wikipedia.org/wiki/Pascual_Jordan}{相关文章}。

\begin{figure}[ht]
\centering
\includegraphics[width=6cm]{./figures/1d93dfb61493db05.png}
\caption{帕斯库尔·约当在1920年代} \label{fig_YRD_1}
\end{figure}
恩斯特·帕斯库尔·约当(Ernst Pascual Jordan,德语发音:[ˈɛʁnst pasˈku̯al ˈjɔʁdaːn];1902年10月18日-1980年7月31日)是一位德国理论与数学物理学家,对量子力学和量子场论做出了重要贡献。他为矩阵力学的数学形式奠定了基础,并发展了费米子的规范反对易关系。他还引入了**约当代数**,试图形式化量子场论;此后,这些代数在数学中得到了广泛应用。[1]

约当于1933年加入了纳粹党,但并未追随当时拒绝由阿尔伯特·爱因斯坦及其他犹太物理学家发展出的量子物理的德意志物理运动。二战后,他加入了保守派政党基督教民主联盟(CDU),并于1957年至1961年担任议会议员。
\subsection{家庭背景与教育}
约当出生于恩斯特·帕斯夸尔·约当(1858–1924)和艾娃·费舍尔(Eva Fischer)的家庭。恩斯特·约当是一位以肖像画和风景画闻名的画家,并在汉诺威工业大学担任艺术副教授。家族姓氏原为“Jorda”,起源于西班牙,家族中长子均取名为“Pasqual”或其变体“Pascual”。1815年滑铁卢战役后,家族定居于汉诺威,并在某个时期将姓氏改为“Jordan”(德语发音:[ˈjɔʁdaːn])。恩斯特·约当于1892年与艾娃·费舍尔结婚。

约当的祖先帕斯夸尔·约当(Pascual Jordan)是一位西班牙贵族及骑兵军官,在拿破仑战争期间与战后为英国效力,最终定居汉诺威。那时,汉诺威王室统治着英国。家族传统规定,每一代的长子必须以“Pascual”命名。[3] 约当自幼接受传统宗教教育,12岁时尝试将《圣经》的字面解读与达尔文进化论调和。他的宗教老师说服他科学与宗教并不矛盾(约当在其一生中撰写了许多关于两者关系的文章)。[3]

1921年,约当进入汉诺威工业大学,学习动物学、数学和物理学。按照当时德国大学生的常规,他在获得学位前转学至另一所大学。1923年,他来到当时在数学和物理科学领域处于巅峰的哥廷根大学,由数学家大卫·希尔伯特(David Hilbert)和物理学家阿诺德·索末菲(Arnold Sommerfeld)指导。在哥廷根期间,他曾短暂担任数学家理查德·柯朗(Richard Courant)的助手,随后在马克斯·玻恩(Max Born)的指导下研究物理学,并在遗传学家兼种族科学家阿尔弗雷德·库恩(Alfred Kühn)的指导下攻读博士学位。[4]

约当一生中受口吃困扰,在即兴讲话时常常严重结巴。[5] 1926年,尼尔斯·玻尔曾主动提出支付治疗费用。在威廉·伦茨(Wilhelm Lenz)的建议下,约当前往维也纳的阿尔弗雷德·阿德勒诊所寻求治疗。[6][7]