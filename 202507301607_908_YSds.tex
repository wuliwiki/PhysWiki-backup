% 域上的代数(综述)
% license CCBYSA3
% type Wiki

本文根据 CC-BY-SA 协议转载翻译自维基百科\href{https://en.wikipedia.org/wiki/Algebra_over_a_field}{相关文章}。

在数学中,一个域上的代数(通常简称为代数,algebra)是一个带有双线性乘法的向量空间。因此,代数是一种代数结构,由一个集合以及加法、乘法和域元素的数量乘法构成,并且满足“向量空间”和“双线性”所蕴含的公理。[1]

代数中的乘法运算可以是结合的,也可以不是,这对应于结合代数(假设乘法具有结合性)和非结合代数(不假设乘法具有结合性,但也不排除结合性)。例如,给定一个整数 $n$,实数域上的 $n$ 阶实方阵环,在矩阵加法和矩阵乘法下,是一个结合代数,因为矩阵乘法是结合的。另一方面,三维欧几里得空间在向量叉积运算下是一个非结合代数,因为向量叉积不满足结合律,但满足雅可比恒等式。

如果代数在乘法下具有单位元,则称其为有单位代数。例如,$n$ 阶实方阵环是一个有单位代数,因为 $n$ 阶单位矩阵是矩阵乘法的单位元。它是一个有单位结合代数的例子,即既是(有单位的)环,又是一个向量空间。

许多作者使用“代数”一词时,实际上是指结合代数,或有单位结合代数,或者在某些学科(如代数几何)中,指有单位结合交换代数。

将标量的域替换为交换环,就得到了更一般的环上的代数的概念。需要注意的是,代数不同于带双线性型的向量空间(如内积空间),因为对于这样的空间,乘积的结果不在空间本身,而是在系数域中。
\subsection{定义与动机}
\subsubsection{动机性示例}
\begin{figure}[ht]
\centering
\includegraphics[width=14.25cm]{./figures/e88003d816a27498.png}
\caption{} \label{fig_YSds_1}
\end{figure}