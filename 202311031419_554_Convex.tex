% 凸函数(补)
% keys 凸集合;凸函数
% license Usr
% type Tutor

\begin{definition}{凸集}
对于一个集合$C\subseteq\mathbb{R}^n,\forall\boldsymbol{x}_1,\boldsymbol{x}_2,\theta\in[0,1],$满足
\begin{equation}
\theta\boldsymbol{x}_1+(1-\theta)\boldsymbol{x}_2\in C~
\end{equation}
称$C$是凸集.
\end{definition}
\begin{definition}{凸集}
对于一个集合$C\subseteq\mathbb{R}^n,\forall\boldsymbol{x}_1,\cdot,\boldsymbol{x}_n,\theta_k\in[0,1],k=1,2,\cdots,n$满足
\begin{equation}
\theta_1\boldsymbol{x}_1+\cdots+\theta_n\boldsymbol{x}_n\in C~
\end{equation}
称$C$是凸集.
\end{definition}
\begin{definition}{凸集}
对于一个集合$C\subseteq\mathbb{R}^n,$存在映射$p:\mathbb{R}^n\to\mathbb{R},p(\boldsymbol{x}),\int_Cp(\boldsymbol{x})\mathrm{d}\boldsymbol{x}=1,\forall\boldsymbol{x}\in C,$满足
\begin{equation}
\int_C\boldsymbol{x}p(\boldsymbol{x})\mathrm{d}\boldsymbol{x}\in C~
\end{equation}
称$C$是凸集.
\end{definition}
\begin{definition}{凸函数}
存在映射$f:\mathbb{R}^n\to\mathbb{R},\mathrm{dom} f$是凸集合,如果$\forall\boldsymbol{x_1},\boldsymbol{x_2}\in\mathrm{dom} f, \theta\in[0,1],$满足
\begin{equation}
f(\theta\boldsymbol{x}_1+(1-\theta)\boldsymbol{x}_2)\leqslant\theta f(\boldsymbol{x}_1) + (1-\theta)f(\boldsymbol{x}_2)~
\end{equation}
\end{definition}
称映射$f$是定义在$\mathrm{dom} f$上的凸函数.如果满足
\begin{equation}
f(\theta\boldsymbol{x}_1+(1-\theta)\boldsymbol{x}_2)<\theta f(\boldsymbol{x}_1) + (1-\theta)f(\boldsymbol{x}_2)~
\end{equation}
称映射$f$是定义在$\mathrm{dom} f$上的强凸函数.
\begin{definition}{凹函数}
存在映射$f:\mathbb{R}^n\to\mathbb{R},\mathrm{dom} f$是凸集合,如果$\forall\boldsymbol{x_1},\boldsymbol{x_2}\in\mathrm{dom} f, \theta\in[0,1],$满足
\begin{equation}
f(\theta\boldsymbol{x}_1+(1-\theta)\boldsymbol{x}_2)\geqslant\theta f(\boldsymbol{x}_1) + (1-\theta)f(\boldsymbol{x}_2)~
\end{equation}
\end{definition}
称映射$f$是定义在$\mathrm{dom} f$上的凹函数.如果满足
\begin{equation}
f(\theta\boldsymbol{x}_1+(1-\theta)\boldsymbol{x}_2)>\theta f(\boldsymbol{x}_1) + (1-\theta)f(\boldsymbol{x}_2)~
\end{equation}
称映射$f$是定义在$\mathrm{dom} f$上的强凹函数.
\begin{theorem}{凸函数判定}
$f:\mathbb{R}^n\to\mathbb{R}$是凸函数,当且仅当$g:\mathbb{R}\to\mathbb{R},g(t)=f(\boldsymbol{x}+t\boldsymbol{v}),t=\{t|\boldsymbol{x}+t\boldsymbol{v}\in\mathrm{dom} f,\boldsymbol{v}\in\mathbb{R}^n\}$是凸函数.
\end{theorem}
\textbf{证明}:(1)必要性

$f(\boldsymbol{x})$是凸函数,令$g(t)=f(\boldsymbol{x}+t\boldsymbol{v}),$那么对$\forall t_1,t_2,$有以下关系:
$$
\begin{aligned}
g\left( \theta t_1+\left( 1-\theta \right) t_2 \right) &=f\left( \boldsymbol{x}+\left( \theta t_1+\left( 1-\theta \right) t_2 \right) \boldsymbol{v} \right)\\
&=f\left( \boldsymbol{x}+\theta t_1\boldsymbol{v}+\left( 1-\theta \right) t_2\boldsymbol{v} \right)\\
&=f\left( \theta \left( \boldsymbol{x}+t_1\boldsymbol{v} \right) +\left( 1-\theta \right) \left( \boldsymbol{x}+t_2\boldsymbol{v} \right) \right) \\
&\leqslant \theta f\left( \boldsymbol{x}+t_1\boldsymbol{v} \right) +\left( 1-\theta \right) f\left( \boldsymbol{x}+t_2\boldsymbol{v} \right) =\theta g\left( t_1 \right) +\left( 1-\theta \right) g\left( t_2 \right)
\end{aligned}~
$$
由此可知,$g(t)$是凸函数.

(2)充分性

$g(t)=f(\boldsymbol{x}+t\boldsymbol{v})$是凸函数,因此,\forall \theta \in \left[ 0,1 \right] 
$$
\begin{aligned}
g\left( \theta t_1+\left( 1-\theta \right) t_2 \right) &=f\left( \boldsymbol{x}+\left( \theta t_1+\left( 1-\theta \right) t_2 \right) \boldsymbol{v} \right) \\
&=f\left( \boldsymbol{x}+\theta t_1\boldsymbol{v}+\left( 1-\theta \right) t_2\boldsymbol{v} \right) =f\left( \theta \left( \boldsymbol{x}+t_1\boldsymbol{v} \right) +\left( 1-\theta \right) \left( \boldsymbol{x}+t_2\boldsymbol{v} \right) \right) \\
&\leqslant \theta g\left( t_1 \right) +\left( 1-\theta \right) g\left( t_2 \right)\\
&=\theta f\left( \boldsymbol{x}+t_1\boldsymbol{v} \right) +\left( 1-\theta \right) f\left( \boldsymbol{x}+t_2\boldsymbol{v} \right) 
\end{aligned}~
$$
令$\boldsymbol{z}_1=\boldsymbol{x}+t_1\boldsymbol{v},\boldsymbol{z}_2=\boldsymbol{x}+t_2\boldsymbol{v},$上述不等式等价为
$$
f\left( \theta \boldsymbol{z}_1+\left( 1-\theta \right) \boldsymbol{z}_2 \right) \leqslant \theta f\left( \boldsymbol{z}_1 \right) +\left( 1-\theta \right) f\left( \boldsymbol{z}_2 \right)~
$$
由此可知,$f(\boldsymbol{x})$是凸函数.
\begin{theorem}{一阶条件}
假设$f:\mathbb{R}^n\to\mathbb{R}$一阶可微,$f:\mathbb{R}^n\to\mathbb{R}$是凸函数,当且仅当
\begin{equation}
f(\boldsymbol{y})\geqslant f(\boldsymbol{x})+\nabla^\top f(\boldsymbol{x})(\boldsymbol{y}-\boldsymbol{x})~
\end{equation}
对$\forall\boldsymbol{x},\boldsymbol{y}\in\mathrm{dom} f$成立.
\end{theorem}
\textbf{证明}:(1)必要性

令$\boldsymbol{z}=\theta \boldsymbol{y}+\left( 1-\theta \right) \boldsymbol{x}=\boldsymbol{x}+\theta \left( \boldsymbol{y}-\boldsymbol{x} \right),$由凸函数定位可知,
$$
f\left( \theta \boldsymbol{y}+\left( 1-\theta \right) \boldsymbol{x} \right) =f\left( \boldsymbol{x}+\theta \left( \boldsymbol{y}-\boldsymbol{x} \right) \right) \leqslant \theta f\left( \boldsymbol{y} \right) +\left( 1-\theta \right) f\left( \boldsymbol{x} \right)~
$$
整理得,
$$
f\left( \boldsymbol{y} \right) \geqslant \frac{f\left( \boldsymbol{x}+\theta \left( \boldsymbol{y}-\boldsymbol{x} \right) \right) -\left( 1-\theta \right) f\left( \boldsymbol{x} \right)}{\theta}=f\left( \boldsymbol{x} \right) +\frac{f\left( \boldsymbol{x}+\theta \left( \boldsymbol{y}-\boldsymbol{x} \right) \right) -f\left( \boldsymbol{x} \right)}{\theta}~
$$
让$\theta\to 0$可得
$$
f\left( \boldsymbol{y} \right) \geqslant f\left( \boldsymbol{x} \right) +\lim_{\theta \rightarrow 0} \frac{f\left( \boldsymbol{x}+\theta \left( \boldsymbol{y}-\boldsymbol{x} \right) \right) -f\left( \boldsymbol{x} \right)}{\theta}=f\left( \boldsymbol{x} \right) +\nabla ^{\top}f\left( \boldsymbol{x} \right) \left( \boldsymbol{y}-\boldsymbol{x} \right)~
$$

(2)充分性

令$z=\theta\boldsymbol{x}+(1-\theta)\boldsymbol{y},$于是可以得到
$$
f\left( \boldsymbol{x} \right) \geqslant f\left( \boldsymbol{z} \right) +\nabla ^{\top}f\left( \boldsymbol{z} \right) \left( \boldsymbol{x}-\boldsymbol{z} \right)~
$$
$$
f\left( \boldsymbol{y} \right) \geqslant f\left( \boldsymbol{z} \right) +\nabla ^{\top}f\left( \boldsymbol{z} \right) \left( \boldsymbol{y}-\boldsymbol{z} \right)~
$$
也即是
$$
\theta f\left( \boldsymbol{x} \right) +\left( 1-\theta \right) f\left( \boldsymbol{y} \right) \geqslant \theta \left( f\left( \boldsymbol{z} \right) +\nabla ^{\top}f\left( \boldsymbol{z} \right) \left( \boldsymbol{x}-\boldsymbol{z} \right) \right) +\left( 1-\theta \right) \left( f\left( \boldsymbol{z} \right) +\nabla ^{\top}f\left( \boldsymbol{z} \right) \left( \boldsymbol{y}-\boldsymbol{z} \right) \right)~
$$
展开得
$$
\theta f\left( \boldsymbol{x} \right) +\left( 1-\theta \right) f\left( \boldsymbol{y} \right) \geqslant \theta f\left( \boldsymbol{z} \right) +\theta \nabla ^{\top}f\left( \boldsymbol{z} \right) \left( \boldsymbol{x}-\boldsymbol{z} \right) +\left( 1-\theta \right) f\left( \boldsymbol{z} \right) +\left( 1-\theta \right) \nabla ^{\top}f\left( \boldsymbol{z} \right) \left( \boldsymbol{y}-\boldsymbol{z} \right)~
$$
进一步得到
$$
\theta f\left( \boldsymbol{x} \right) +\left( 1-\theta \right) f\left( \boldsymbol{y} \right) \geqslant \nabla ^{\top}f\left( \boldsymbol{z} \right) \left[ \theta \left( \boldsymbol{x}-\boldsymbol{z} \right) +\left( 1-\theta \right) \left( \boldsymbol{y}-\boldsymbol{z} \right) \right] +f\left( \boldsymbol{z} \right)~
$$
也即是
$$
f\left( \boldsymbol{z} \right) \leqslant \theta f\left( \boldsymbol{x} \right) +\left( 1-\theta \right) f\left( \boldsymbol{y} \right)~
$$
即
$$
f\left( \theta \boldsymbol{x}+\left( 1-\theta \right) \boldsymbol{y} \right) \leqslant \theta f\left( \boldsymbol{x} \right) +\left( 1-\theta \right) f\left( \boldsymbol{y} \right)~
$$
因此,$f(\boldsymbol{x})$是凸函数.

\begin{theorem}{二阶条件}
假设$f:\mathbb{R}^n\to\mathbb{R}$二阶可微,$f:\mathbb{R}^n\to\mathbb{R}$是凸函数,当且仅当
\begin{equation}
\nabla^2 f(\boldsymbol{x})\succeq 0~
\end{equation}
对$\forall\boldsymbol{x}\in\mathrm{dom} f$成立.
\end{theorem}
\textbf{证明}:(1)必要性

由函数的泰勒展开,
$$
f\left( \boldsymbol{y} \right) =f\left( \boldsymbol{x} \right) +\nabla ^{\top}f\left( \boldsymbol{x} \right) \left( \boldsymbol{y}-\boldsymbol{x} \right) +\frac{1}{2}\left( \boldsymbol{y}-\boldsymbol{x} \right) ^{\top}\nabla ^2f\left( \boldsymbol{x} \right) \left( \boldsymbol{y}-\boldsymbol{x} \right) +o\left( \left\| \boldsymbol{y}-\boldsymbol{x} \right\| ^2 \right)~
$$
再由定理2可知,
$$
f\left( \boldsymbol{y} \right) \geqslant f\left( \boldsymbol{x} \right) +\nabla ^{\top}f\left( \boldsymbol{x} \right) \left( \boldsymbol{y}-\boldsymbol{x} \right)~
$$
于是可得,
$$
f\left( \boldsymbol{x} \right) +\nabla ^{\top}f\left( \boldsymbol{x} \right) \left( \boldsymbol{y}-\boldsymbol{x} \right) +\frac{1}{2}\left( \boldsymbol{y}-\boldsymbol{x} \right) ^{\top}\nabla ^2f\left( \boldsymbol{x} \right) \left( \boldsymbol{y}-\boldsymbol{x} \right) +o\left( \left\| \boldsymbol{y}-\boldsymbol{x} \right\| ^2 \right) \geqslant f\left( \boldsymbol{x} \right) +\nabla ^{\top}f\left( \boldsymbol{x} \right) \left( \boldsymbol{y}-\boldsymbol{x} \right)~
$$
因此,
$$
\frac{1}{2}\left( \boldsymbol{y}-\boldsymbol{x} \right) ^{\top}\nabla ^2f\left( \boldsymbol{x} \right) \left( \boldsymbol{y}-\boldsymbol{x} \right) +o\left( \left\| \boldsymbol{y}-\boldsymbol{x} \right\| ^2 \right) \geqslant 0~
$$
也即
$$
\left( \boldsymbol{y}-\boldsymbol{x} \right) ^{\top}\nabla ^2f\left( \boldsymbol{x} \right) \left( \boldsymbol{y}-\boldsymbol{x} \right) \geqslant 0~
$$
由此可知
$$
\nabla ^2f\left( \boldsymbol{x} \right) \succeq 0~
$$

(2)充分性

由函数的泰勒展开,
$$
f\left( \boldsymbol{y} \right) =f\left( \boldsymbol{x} \right) +\nabla ^{\top}f\left( \boldsymbol{x} \right) \left( \boldsymbol{y}-\boldsymbol{x} \right) +\frac{1}{2}\left( \boldsymbol{y}-\boldsymbol{x} \right) ^{\top}\nabla ^2f\left( \boldsymbol{x} \right) \left( \boldsymbol{y}-\boldsymbol{x} \right) +o\left( \left\| \boldsymbol{y}-\boldsymbol{x}\right\| ^2 \right)~
$$
由$\nabla ^2f\left( \boldsymbol{x} \right) \succeq$,因此
$$
\left( \boldsymbol{y}-\boldsymbol{x} \right) ^{\top}\nabla ^2f\left( \boldsymbol{x} \right) \left( \boldsymbol{y}-\boldsymbol{x} \right) \geqslant 0~
$$
也即
$$
f\left( \boldsymbol{y} \right) =f\left( \boldsymbol{x} \right) +\nabla ^{\top}f\left( \boldsymbol{x} \right) \left( \boldsymbol{y}-\boldsymbol{x} \right) +\frac{1}{2}\left( \boldsymbol{y}-\boldsymbol{x} \right) ^{\top}\nabla ^2f\left( \boldsymbol{x} \right) \left( \boldsymbol{y}-\boldsymbol{x} \right) +o\left( \left\| \boldsymbol{y}-\boldsymbol{x} \right\| ^2 \right) \geqslant f\left( \boldsymbol{x} \right) +\nabla ^{\top}f\left( \boldsymbol{x} \right) \left( \boldsymbol{y}-\boldsymbol{x} \right)~
$$
由定理2可知,$f(\boldsymbol{x})$是凸函数.
\begin{exercise}{判断下列函数的凹凸性}
\begin{enumerate}[(1)]
\item $f(\boldsymbol{x})=\max\{x_1,\cdots,x_n\}$
\item $f(\boldsymbol{x})=\min\{x_1,\cdots,x_n\}$
\item $f\left( \boldsymbol{x} \right) =\frac{1}{2}\boldsymbol{x}^{\top}P\boldsymbol{x}+\boldsymbol{q}^{\top}\boldsymbol{x}+r,P\succeq 0$
\item $f\left( \boldsymbol{x} \right) =\left\| A\boldsymbol{x}-b \right\| _{2}^{2},A\in \mathbb{R} ^{m\times n}$
\item $f\left( \boldsymbol{x} \right) =\log \sum_{k=1}^n{\exp \,x_k}$
\item $f\left( \boldsymbol{x} \right) =\prod_{k=1}^n{\left( x_k \right) ^{\frac{1}{n}}}$
\end{enumerate}
\end{exercise}
\begin{exercise}
证明: 存在映射$f:\mathbb{R}^n\to\mathbb{R}$,另有映射$g:\mathbb{R}^m\to\mathbb{R}$,且$g(\boldsymbol{x})=f(A\boldsymbol{x}+\boldsymbol{b})$,其中$A\in\mathbb{R}^{n\times m},\boldsymbol{b}$,$f$he
\end{exercise}