% 反常积分
% 积分|定积分|微积分
\begin{issues}
\issueTODO
\issueDraft
\end{issues}

\pentry{定积分\upref{DefInt}}
\subsection{积分限为无穷的反常积分}

\subsubsection{引入}
在许多物理、数学问题中我们会遇到一些积分到无穷的积分。例如 势能(简介)\upref{POTENT} 中,我们定义势能是
$$E_P=\int ^\infty_a \bvec F_{in} \cdot \dd \bvec r$$
咋看起来,积分到无限远处会导致结果发散?结果真的是这样吗?

我们先看一个简单的例子:$\int^a_1 \frac{1}{x^2} \dd x$

\begin{table}[ht]
\centering
\caption{请输入表格标题}\label{impro_tab1}
\begin{tabular}{|c|c|}
\hline
* & * \\
\hline
* & * \\
\hline
* & * \\
\hline
* & * \\
\hline
\end{tabular}
\end{table}

\subsubsection{定义}
\begin{definition}{反常积分}
设函数 $f(x)$ 在 $[a, +\infty)$ 上连续(或分段连续),对于任意 $t>a$,积分 $\displaystyle \int^t_af(x)\mathrm{d} x$ 存在,则定义 $\displaystyle \int ^{+\infty}_a f(x)\mathrm{d} x=\lim_{t\rightarrow+\infty }\int _a^{+\infty}f(x)\mathrm{d} x$,并称 $\displaystyle \int ^{+\infty}_a f(x)\mathrm{d} x $ 为 $f(x)$ 在 $[a, +\infty)$ 的反常积分。

如果 $\displaystyle \lim_{t\rightarrow+\infty }\int _a^{+\infty}f(x)\mathrm{d} x$ 存在,则称反常积分 $\displaystyle \int ^{+\infty}_a f(x)\mathrm{d} x$ \textbf{收敛},且该极限值称为反常积分的值;若此极限不存在,则称反常积分 $\displaystyle \lim_{t\rightarrow+\infty }\int _a^{+\infty}f(x)\mathrm{d} x$ \textbf{发散}。
\end{definition}

同样地,可以定义在 $(-\infty,b]$ 上的连续函数 $f(x)$ 的反常积分为 $\displaystyle \int ^b _{-\infty}f(x)\mathrm{d} x=\lim_{t\rightarrow-\infty }\int ^b _{-\infty}f(x)\mathrm{d} x$

对于定义在 $(-\infty,+\infty )$ 上的连续函数 $f(x)$ 的反常积分 $\displaystyle \int ^{+\infty}_{-\infty}f(x)\mathrm{d} x$ 作如下定义
$$\displaystyle \int ^{+\infty}_{-\infty}f(x)\mathrm{d} x=\displaystyle \int ^{+\infty}_c f(x)\mathrm{d} x+\displaystyle \int ^c _{-\infty}f(x)\mathrm{d} x$$
其中 $c$ 是任意实数。当且仅当等式右边的两个积分同时收敛时,称反常积分 $\displaystyle \int ^{+\infty}_{-\infty}f(x)\mathrm{d} x$ 收敛,且右端两个积分值的和称为反常积分 $\displaystyle \int ^{+\infty}_{-\infty}f(x)\mathrm{d} x$ 的值;否则称 $\displaystyle \int ^{+\infty}_{-\infty}f(x)\mathrm{d} x$ 发散。
