% 球谐函数
% keys 球坐标系|拉普拉斯方程|球谐函数|正交归一性
% license Xiao
% type Tutor

\pentry{球坐标的拉普拉斯方程\nref{nod_SphLap}, 连带勒让德函数\nref{nod_AsLgdr}}{nod_bf42}

\footnote{参考 Wikipedia \href{https://en.wikipedia.org/wiki/Spherical_harmonics}{相关页面}。}当\enref{球坐标中的拉普拉斯方程}{SphLap}分离变量后, 关于极角 $\theta$ 的函数为连带勒让德函数 $P_l^m(\cos\theta)$, 方向角函数为 $\E^{\I m\phi}$。 我们定义\textbf{球谐函数(spherical harmonics)}为\footnote{有些教材也将球谐函数记为 $Y_l^m$}\footnote{特殊地, $Y_{l,0}(\uvec r) = \sqrt{(2l + 1)/(4\pi)}P_l(\cos\theta)$, $P_l$ 是\enref{勒让德多项式}{Legen}。}
\begin{equation}\label{eq_SphHar_1}
Y_{l, m} (\uvec r) = Y_{l, m}(\theta, \phi) = A_{l,m} P_l^m(\cos \theta) \E^{\I m\phi}~,
\end{equation}
其中 $\uvec r$ 表示一个\enref{单位矢量}{GVec}, 指向球坐标中的点 $(1, \theta, \phi)$。 $l, m$ 为整数, $l \geqslant 0$, $-l \leqslant m \leqslant l$。 $A_{l,m}$ 是归一化系数, 使得 $\abs{Y_{l, m} (\uvec r)}^2$ 在单位球面上的面积分等于 1 (\autoref{eq_SphHar_12})。
\begin{equation}\label{eq_SphHar_2}
A_{l,m} =  \sqrt{\frac{2l + 1}{4\pi }\frac{(l - m)!}{(l + m)!}}~.
\end{equation}
球谐函数可以看作是将单位球面上的每一点(或者三维空间中的每个方向)映射到一个复数函数值。 常见的球谐函数见 “\enref{球谐函数列表}{YlmTab}”, 函数图常用\autoref{fig_SphHar_1} 的方式绘制。

\begin{figure}[ht]
\centering
\includegraphics[width=13cm]{./figures/8d9ea7de9e8215b0.pdf}
\caption{$\abs{Y_{l,m}(\uvec r)}$ 在 $\phi=0$ 平面的\enref{极坐标曲线}{PolCrd}。红色代表 $Y_{l,m}(\uvec r) > 0$, 蓝色代表 $Y_{l,m}(\uvec r) < 0$。 第 1 行到第 4 行分别为 $l = 0$ 到 $3$, 每行从左到右分别为 $m = -l$ 到 $l$。 图中的右上角标明了 $r$ 的单位长度。} \label{fig_SphHar_1}
\end{figure}

当 $m = 0$ 时球谐函数可以用勒让德多项式表示为关于 $z$ 轴对称的函数
\begin{equation}\label{eq_SphHar_11}
Y_{l,0}(\uvec r) = \sqrt{\frac{2l+1}{4\pi}} P_l(\cos\theta)~,
\end{equation}
另外根据\autoref{eq_SphHar_1} 和\autoref{eq_AsLgdr_2} 可得
\begin{equation}
Y_{l,m}(\uvec z) = \sqrt{\frac{2l+1}{4\pi}} \delta_{m,0}~.
\end{equation}

\subsection{偏微分方程}
球谐函数同时满足偏微分方程
\begin{align}
\laplacian_\Omega Y_{l,m} &= -l(l+1)Y_{l,m}~,\\
\pdv{\phi} Y_{l,m} &= \I m Y_{l,m}~,
\end{align}
其中
\begin{equation}
\laplacian_\Omega = \frac{1}{\sin\theta}\pdv{\theta} \qty(\sin \theta \pdv{\theta}) + \frac{1}{\sin^2 \theta} \qty(\pdv[2]{\phi})~
\end{equation}
是球坐标系拉普拉斯算子中与 $r$ 无关的部分(\autoref{eq_SphNab_3})。

\subsection{归一化系数}
由球谐函数的归一化条件, 对单位球面积分(\autoref{ex_CrIntN_1})得
\begin{equation}\label{eq_SphHar_12}\ali{
1 &= \int \abs{Y_{l, m}(\uvec r)}^2 \dd{\Omega} = \int_0^\pi  \int_0^{2\pi}  \abs{Y_{l, m}(\theta, \phi)}^2 \sin\theta\dd{\theta}\dd{\phi} \\
&= \abs{A_{l,m}}^2 \int_{-1}^1  \abs{P_l^m(\cos\theta)}^2 \dd{(\cos \theta)} \int_0^{2\pi } \abs{\E^{\I m\phi}}^2  \dd{\phi}\\
&= \frac{2\pi}{\abs{A_{l,m}'}^2} \abs{A_{l,m}}^2~,
}\end{equation}
其中 $A_{l,m}'$ 是 $P_l^m(x)$ 的归一化系数(见\autoref{eq_AsLgdr_3}), 代入后可得 $A_{l,m}$。

\subsection{Condon–Shortley 相位}\label{sub_SphHar_1}
与连带勒让德函数相同, 在定义球谐函数时我们也可以选择是否包含 \textbf{Condon–Shortley 相位} $(-1)^m$。 物理中一般选择包含, 小时百科中也统一包含。 如果包含, 我们可以选择将其包含在连带勒让德函数中(如\autoref{eq_AsLgdr_1}), 或者包含在球谐函数的定义中。 如果不包含, 该相位在两个定义中都不出现。

\subsection{正交归一性}
由勒让德函数的正交归一性(\autoref{eq_AsLgdr_4})以及 $\E^{\I m \phi}$  的正交归一性(\autoref{eq_FSExp_4}), 不难证明球谐函数的正交归一性
\begin{equation}
\int Y_{l', m'}^*(\uvec r) Y_{l, m} (\uvec r) \dd{\Omega} = \delta_{l,l'}\delta_{m,m'}~,
\end{equation}
其中 $\delta_{i,j}$ 是\enref{克罗内克 delta 函数}{Kronec}。 与\enref{多元函数的傅里叶级数}{NdFuri}类似, 球谐函数可以展开所有性质良好的二元函数:
\begin{equation}\label{eq_SphHar_14}
f(\uvec r) = f(\theta, \phi) = \sum_{l = 0}^\infty\sum_{m=-l}^l C_{l,m}Y_{l,m}(\theta, \phi) \quad (\theta \in [0, \pi], \phi \in [2, 2\pi])~,
\end{equation}
\begin{equation}
C_{l,m} = \int Y^*_{l,m}(\uvec r) f(\uvec r) \dd{\Omega}~.
\end{equation}
若 $f(\uvec r)$ 是实函数, 那么(证明:令\autoref{eq_SphHar_14} 等于自己的复共轭再使用\autoref{eq_SphHar_6} 即可)
\begin{equation}\label{eq_SphHar_13}
C_{l,m} = (-1)^m C_{l,-m}^*~.
\end{equation}

性质良好的三维函数 $f(\bvec r)$ 也可以展开为径向函数 $f_{l,m}(r)$ 和球谐函数的乘积
\begin{equation}\label{eq_SphHar_9}
f(\bvec r) = f(r, \theta, \phi) = \sum_{l = 0}^\infty\sum_{m=-l}^l f_{l,m}(r) Y_{l,m}(\theta, \phi)~,
\end{equation}
\begin{equation}\label{eq_SphHar_7}
f_{l,m}(r) = \int Y^*_{l,m}(\uvec r) f(\bvec r) \dd{\Omega}~.
\end{equation}
若 $f(\bvec r)$ 是实函数, 那么
\begin{equation}
f_{l,-m}(r) = (-1)^m f_{l,m}(r)~.
\end{equation}

\subsection{旋转变换}
\begin{equation}
Y_{l,m}(\uvec r') = \sum_{m'=-l}^l D_{m,m'}^{(l)} (\mathcal R)^* Y_{l,m'}(\uvec r)~,
\end{equation}
其中 $\mat D^{(l)}$ 是\enref{Wigner D 矩阵}{WigDmt}。 从量子力学的角度来说, 总角动量是与方向无关的, 只有角动量在某方向的分量有关。

\subsection{中心对称}
$l$ 为偶数时, 球谐函数是中心对称的(\enref{偶宇称}{IntPry}), 否则是反对称的(奇宇称)。
\begin{equation}\label{eq_SphHar_8}
Y_{l,m}(-\uvec r) = (-1)^l Y_{l,m}(\uvec r)~,
\end{equation}
可以用\autoref{fig_SphHar_1} 验证。

\subsection{求导和积分}
以下公式在量子力学中计算\enref{氢原子跃迁矩阵元}{SelRul}以及\enref{数值解 TDSE}{HyTDSE} 时有应用。

三个球谐函数之积的积分可以表示成两个 \enref{CG 系数}{SphCup}或 \enref{3j 符号}{ThreeJ}相乘\footnote{见 Bransden 附录 A4, 以及 Wikipedia 的 3j/CG coefficients 页面}
\begin{equation}\label{eq_SphHar_3}
% 已反复验证
\ali{
&\quad \int Y_{l_1 m_1} (\uvec r) Y_{l_2 m_2} (\uvec r) Y_{l_3 m_3}(\uvec r) \dd{\Omega}\\
&= (-1)^{m_3} \sqrt{\frac{(2l_1+1)(2l_2+1)}{4\pi(2l_3+1)}} \bmat{l_1& l_2& l_3\\ 0 & 0 & 0}\bmat{l_1 & l_2 & l_3\\  m_1 & m_2 & -m_3}\\
&= \sqrt{\frac{(2l_1+1)(2l_2+1)(2l_3+1)}{4\pi}}  \pmat{l_1& l_2& l_3\\ 0 & 0 & 0}\pmat{l_1 & l_2 & l_3\\  m_1 & m_2 & m_3}~,
}\end{equation}
使用\autoref{eq_SphHar_6} 把第一个球谐函数变为复共轭得
\begin{equation}\label{eq_SphHar_5}
\begin{aligned}
&\quad \int Y_{l_1 m_1}^*(\uvec r) Y_{l_2 m_2} (\uvec r) Y_{l_3 m_3}(\uvec r) \dd{\Omega}\\
& = (-1)^{m_1}\sqrt{\frac{(2l_1+1)(2l_2+1)(2l_3+1)}{4\pi}} \pmat{l_1& l_2& l_3\\ 0 & 0 & 0}\pmat{l_1 & l_2 & l_3\\  -m_1 & m_2 & m_3}~.
\end{aligned}
\end{equation}
注意 $\cos\theta$ 是一个厄米(对称)算符。

特殊地,考虑到 $\cos\theta = \sqrt{\frac{4\pi}{3}}Y_{1,0}$, 有
\begin{equation} \label{eq_SphHar_20}
\begin{aligned} % Mathematica 已验证多次
\int Y_{l,m}^* (\uvec r) \cos\theta (\uvec r) Y_{l+1, m}(\uvec r)
&= \int Y_{l+1, m}^* (\uvec r) \cos\theta (\uvec r) Y_{l,m}(\uvec r)\\
&=  \sqrt{\frac{(l+1)^2-m^2}{(2l+1)(2l+3)}}~.
\end{aligned}
\end{equation}
为了方便起见, 下文把该式记为 $\mathcal C_{l,m}$。

根据 $P_l^m(x)$ 的求导公式(\autoref{eq_AsLgdr_8}), 有
\begin{equation}\label{eq_SphHar_18} % Mathematica 已验证
\pdv{\theta}Y_{l,m}(\uvec r) = \frac{1}{\sin\theta}\qty[l\ \mathcal C_{l,m}Y_{l+1, m}(\uvec r)  -(l+1)\mathcal C_{l-1,m}Y_{l-1, m}(\uvec r)]~.
\end{equation}
故
\begin{equation} % Mathematica 已验证多次
\int Y^*_{l+1, m}(\uvec r)\sin\theta\pdv{\theta}Y_{l,m}(\uvec r)\dd{\Omega} = l\ \mathcal C_{l,m}~,
\end{equation}
\begin{equation} % Mathematica 已验证多次
\int Y^*_{l, m}(\uvec r)\sin\theta\pdv{\theta}Y_{l+1,m}(\uvec r)\dd{\Omega} = -(l+2)\mathcal C_{l,m}~.
\end{equation}
注意 $\sin\theta\ \pdv*{\theta}$ 不是一个反对称算符, 但若加上 $\cos\theta$ 就是。 反对称算符乘以 $\I$ 就是厄米算符。

\subsection{矢量相加定理}
这里的\textbf{矢量相加定理(vector addition theorem)}是指把勒让德多项式展开为两个球谐函数的乘积
\begin{equation}\label{eq_SphHar_10}
P_l(\cos\theta) = \frac{4\pi}{2l+1} \sum_{m=-l}^l Y_{l,m}^*(\uvec u) Y_{l,m}(\uvec v)~,
\end{equation}
其中 $\theta$ 是两个单位矢量 $\uvec u, \uvec v$ 的夹角。
% \addTODO{为什么叫做矢量相加? 以及证明}

\subsection{其他性质}
\begin{equation}\label{eq_SphHar_6}
Y_{l,m}(\uvec r) = (-1)^m Y_{l,-m}^*(\uvec r)~.
\end{equation}
这里的 $(-1)^m$ 是由连带勒让德函数的性质(\autoref{eq_AsLgdr_9})而来, 而复共轭由 $\exp(\I m\phi)$ 因子而来。

根据 “\enref{球坐标与直角坐标的转换}{SphCar}” 中定义的直角坐标系以及\enref{球谐函数表}{YlmTab}可得
\begin{equation}\label{eq_SphHar_4}
x = \sqrt{\frac{2\pi}{3}} r (Y_{1,-1} - Y_{1,1})~, \qquad
y = \I \sqrt{\frac{2\pi}{3}} r (Y_{1,-1}+Y_{1,1})~, \qquad
z =2\sqrt{\frac{\pi}{3}} rY_{1,0}~.
\end{equation}

\subsection{应用}
\enref{平面波的球谐展开}{Pl2Ylm}, \enref{电多极子展开}{EMulPo}, \enref{球坐标系中的拉普拉斯方程}{SphLap}, \enref{类氢原子的定态波函数}{HWF}等。
