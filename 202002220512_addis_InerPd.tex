% 内积

\pentry{矢量空间\upref{LSpace}}

在矢量空间中, 我们可以另外定义任意两个矢量的\textbf{内积}运算, 运算的结果是一个实数或复数. 内积运算不是矢量空间所必须的, 但物理中的矢量空间几乎都定义了内积运算. 我们把定义了内积运算的空间称为\textbf{内积空间}\footnote{满足一定收敛条件的内积空间也叫\textbf{希尔伯特空间(Hilbert space)}.}
% 链接未完成

两个矢量内积的定义必须满足(其中 $a, b$ 为复数)
\begin{enumerate}
\item $\braket{u}{v} = \braket{v}{u}^*$ (\textbf{共轭对称})
\item $\braket{w}{a u + b v} = a\braket{w}{u} + b\braket{w}{v}$ (\textbf{线性})
\item $\braket{v}{v} > 0 \quad (v \ne 0)$  (\textbf{正定})
\end{enumerate}

说明: 当矢量空间为实数矢量空间\footnote{标量积只允许实数}时, $a, b$ 也必须是实数, 第 1 点中的共轭可以略去. 第 3 点中, 一个非零矢量和自身的内积必定是大于零的实数.

推论: 根据第 2 点, 零矢量和任何矢量的内积都必定是 0. 结合 1,2 两点, 可以得到 $\braket{a u + b v}{w} = a^*\braket{w}{u} + b^*\braket{w}{v}$.

内积一个重要性质就是满足柯西不等式\upref{CSNeq}
\begin{equation}
\abs{\braket{u}{v}}^2 \leqslant \braket{u}{u} \cdot \braket{v}{v}
\end{equation}
即两个矢量内积绝对值的平方小于它们分别和自身内积再相乘. 由柯西不等式可以证明内积空间必然可以定义范数(证明见下文)
\begin{equation}
\norm{v} = \braket{v}{v}
\end{equation}

\subsection{勾股定理}
内积空间的\textbf{勾股定理(Pythagorean theorem)}: 对任意两个正交的矢量, 有
\begin{equation}\label{InerPd_eq1}
\braket{u + v}{u + v} = \braket{u}{u} + \braket{v}{v}
\end{equation}
证明:
\begin{equation}
\braket{u + v}{u + v} = \braket{u}{u} + \braket{v}{v} + \braket{u}{v} + \braket{v}{u}
\end{equation}
根据正交的定义, $\braket{u}{v} = 0$. 证毕.

\subsection{证明内积必定是范数}
要证 $\braket{x}{x}$ 满足范数的要求, 最关键是证明性质
\begin{equation}
\norm{x+y}^2 \leqslant (\norm{x} + \norm{y})^2 = \norm{x}^2 + \norm{y}^2 + 2\norm{x}\norm{y}
\end{equation}
即证
\begin{equation}
\braket{x+y}{x+y} - \braket{x}{x} - \braket{y}{y} \leqslant 2\norm{x}\norm{y}
\end{equation}
其中
\begin{equation}
\braket{x+y}{x+y} = \braket{x}{x} + \braket{y}{y} + 2\Re[{\braket{x}{y}}]
\end{equation}
带入, 即证
\begin{equation}
\Re[{\braket{x}{y}}]^2 \leqslant \norm{x}^2\norm{y}^2 = \braket{x}{x}\braket{y}{y}
\end{equation}
由柯西不等式
\begin{equation}
\Re[\braket{x}{y}]^2 + \Im[\braket{x}{y}]^2 = \abs{\braket{x}{y}}^2 \leqslant \braket{x}{x}\braket{y}{y}
\end{equation}
而 $\Im[\braket{x}{y}]^2 > 0$. 证毕.
