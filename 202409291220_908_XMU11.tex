% 厦门大学 2011 年 考研 量子力学
% license Usr
% type Note

\textbf{声明}:“该内容来源于网络公开资料,不保证真实性,如有侵权请联系管理员”

\subsection{(25分)简述题(每小题5分)}

(1) 波函数有没有物理含义?它的物理含义体现在哪里?物理上对波函数有哪些要求?

(2) 什么是态矢?在什么情况下,可以用态矢来描述量子体系的特征?定念具有哪些性质?

(3) 利用测不准关系,估算一维无限深势阱 $V(x) = \begin{cases} 
0, & |x| < a \\\\
\infty, & |x| \geq a
\end{cases}$ 中质点为基态的粒子的基态能量。

(4) 在希尔伯特 (Hilbert) 空间中,正规算符的完整集相互独立(orthogonal);试分析常见的哈密顿算符与角动量算符的关系,讨论这些算符是否正交。

(5) 假设有一束在无界平面上传播的粒子,垂直中心势场 $V(r)$ 弹性散射后,在 $r=0$ 处的波函数近似为
$$ \psi = e^{ikz} + f(\\theta) \frac{e^{ikr}}{r},~$$
试解释表达式中第一项和第二项的物理意义,并说明 $f(\theta)$ 的含义。
