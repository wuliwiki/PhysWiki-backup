% 量子力学科普视频脚本
% license Usr
% type Note

\begin{issues}
\issueDraft
\end{issues}

\pentry{量子力学的基本原理(科普)\nref{nod_QM0}}{nod_5f89}

解说:现代的量子力学如何描述微观粒子的运动? 为了简单我们先看直线运动。 经典力学中,根据牛顿第一定律,不受力的小球保持静止或匀速直线运动。

动画:画出一个数轴(动画拉长方式出现,有刻度线,配 click 音效),小球静止在原点(渐变出现), 一个火柴人出现,踢一脚(画出受力箭头突然变长然后变短,速度箭头突然变长然后固定不变),匀速向右运动, 镜头跟随。

画面暂停: 解说:可以看出,经典力学中小球在每个时刻的状态由位置和速度完全决定。

解说:在量子力学中,微观粒子在某时刻的状态完全由波函数描述,波函数 $\psi(x)$ 把每个坐标 $x$ 对应到一个复数的函数值。

动画:渐变成开始的坐标系。
