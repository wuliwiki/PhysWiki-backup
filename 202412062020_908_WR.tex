% 赫尔曼·外尔(综述)
% license CCBYSA3
% type Wiki

本文根据 CC-BY-SA 协议转载翻译自维基百科\href{https://en.wikipedia.org/wiki/Hermann_Weyl#Weyl_equation}{相关文章}。

\begin{figure}[ht]
\centering
\includegraphics[width=6cm]{./figures/b6bfe45a42d757ab.png}
\caption{赫尔曼·克劳斯·雨果·外尔  1885年11月9日  德国帝国,埃尔姆斯霍恩} \label{fig_WR_1}
\end{figure}
赫尔曼·克劳斯·雨果·外尔(Hermann Klaus Hugo Weyl,ForMemRS,德语发音:[vaɪl];1885年11月9日-1955年12月8日)是一位德国数学家、理论物理学家、逻辑学家和哲学家。尽管他的大部分职业生涯是在瑞士苏黎世和美国新泽西州普林斯顿度过的,他依然被认为是哥廷根大学数学传统的一部分,该传统由卡尔·弗里德里希·高斯、大卫·希尔伯特和赫尔曼·闵可夫斯基代表。

外尔的研究在理论物理学和纯数学领域(例如数论)具有重要意义。他是20世纪最具影响力的数学家之一,也是早期普林斯顿高等研究院的重要成员。[4][5]

外尔在空间、时间、物质、哲学、逻辑、对称性以及数学史等领域作出了非凡的贡献。他是最早设想将广义相对论与电磁学定律结合起来的人之一。弗里曼·戴森(Freeman Dyson)曾写道,外尔是唯一一个可以与19世纪“最后的伟大通才数学家”亨利·庞加莱和大卫·希尔伯特相提并论的人。迈克尔·阿蒂亚(Michael Atiyah)特别指出,每当他研究一个数学主题时,总会发现外尔早已涉足其中。[7]

\subsection{传记}
赫尔曼·外尔出生于德国汉堡附近的小镇埃尔姆斯霍恩,曾就读于阿尔托纳的克里斯蒂亚内乌姆文理中学(Gymnasium Christianeum)。[8] 他的父亲路德维希·外尔(Ludwig Weyl)是一名银行家,母亲安娜·外尔(Anna Weyl,娘家姓Dieck)则来自一个富裕家庭。[9]

1904年至1908年间,外尔在哥廷根大学和慕尼黑大学学习数学和物理。他在哥廷根大学获得博士学位,导师是他非常敬仰的大卫·希尔伯特。

1913年9月,外尔在哥廷根与弗里德里克·贝尔塔·海伦·约瑟夫(Friederike Bertha Helene Joseph,1893年3月30日-1948年9月5日)结婚,她的昵称是“海拉”(Hella)。海伦是布鲁诺·约瑟夫博士(Dr. Bruno Joseph,1861年12月13日-1934年6月10日)的女儿,后者是一名医生,在德国里布尼茨-达姆加滕(Ribnitz-Damgarten)担任卫生官职(Sanitätsrat)。海伦是一位哲学家(现象学家埃德蒙·胡塞尔的弟子),也是西班牙文学作品的翻译家,尤其将西班牙哲学家何塞·奥尔特加·伊·加塞特的作品译成德文和英文。[12] 正是通过海伦与胡塞尔的密切联系,赫尔曼得以熟悉并深受胡塞尔思想的影响。

赫尔曼与海伦有两个儿子:弗里茨·约阿希姆·外尔(Fritz Joachim Weyl,1915年2月19日-1977年7月20日)和迈克尔·外尔(Michael Weyl,1917年9月15日-2011年3月19日),两人均出生于瑞士苏黎世。[13] 海伦于1948年9月5日在新泽西州普林斯顿去世。同年9月9日,普林斯顿为她举行了纪念仪式,致辞者包括她的儿子弗里茨·约阿希姆·外尔,以及数学家奥斯瓦尔德·维布伦(Oswald Veblen)和理查德·柯朗特(Richard Courant)。[14]

1950年,赫尔曼与雕塑家埃伦·贝尔(Ellen Bär,娘家姓Lohnstein,1902年4月17日-1988年7月14日)结婚。埃伦是苏黎世教授理查德·约瑟夫·贝尔(Richard Josef Bär,1892年9月11日-1940年12月15日)的遗孀。[15][16]