% FFmpeg 笔记
% license Xiao
% type Note

\begin{issues}
\issueDraft
\end{issues}

\subsection{获取信息}
\begin{itemize}
\item 查看视频全部流的编码信息用 \verb|ffprobe -i 视频文件 -show_streams| 或者 \verb|ffprobe -show_streams 视频文件|。
\item 如果只查看视频流或音频流, 用 \verb|-select_streams v| 或 \verb|a|, 可以用 \verb|a:0| 查看第一个音频流。 要查看字幕流,用 \verb|s|, 数据流用 \verb|d|,附件用 \verb|t|
\item 3d 视频通常有两个视频流,音频流可以用于不同语言的配音,注意每个音频流都可以支持双声道而不是用不同流区分。
\item 数据流可以用于储存多媒体信息, 视频分节信息, 等
\item 在 \verb|-show_streams| 的输出中, \verb|codec| 是编码格式, MP4 最常用的是 H264, 高清视频也常用 H265。 音频最常用 ACC
\item \verb|bit_rate| (码率)是每个流每秒钟的平均比特数。 一个 4GB 的两小时视频大概是 \verb|4.4Mbit/s|, 也就是 \verb|550kB/s|, iPhone Xs max 录像大约是 \verb|1MB/s|, 录屏(886x1920px)大约 \verb|2MB/s|(注意 MB 是 MByte)。
\item 微信传视频如果不选原视频,会压缩到 \verb|440kbit/s|。
\item 油管 1080p 30fps 视频是 \verb`1MB/s` 到 \verb`1.2MB/s`。 MP4 电影大约 \verb`0.6MB/s`
\end{itemize}

\subsection{格式转换}
\begin{itemize}
\item 要从视频中提取音频并保持音频编码不变, 用 \verb|ffmpeg -i input_video.mp4 -vn -acodec copy output_audio.mp3|
\item 要导出一段时间内的一些列的帧到图片,用 \verb|ffmpeg -i input_video.mp4 -ss start_time -t duration -vf fps=1 output_%03d.jpg|
\item 不同的视频格式(mp4, mov, mkv,ts)只是外面的 container 不同, 编码可能是相同的(例如最常用的 H264)。 所以转换格式的时候只需要让 ffmpeg 把容器换壳即可: \verb|ffmpeg -y -i 输入视频 -c copy 输出视频|。 这样的转换是完全无损的。
\end{itemize}

\subsection{视频压缩}
\begin{itemize}
\item \verb|ffmpeg -i 输入视频 -c:v libx264 [-crf 23 或 -b:v 1M] -c:a copy 输出视频|。
\item 其中 \verb|-c:v libx264| 使用 H264 编码。 \verb|-crf| (Constant Rate Factor)用于控制转换质量, 数值越小质量越高。 \verb|-b:v 1M| 用于指定码率(1 Mbps)(用 \verb|500k| 表示 500 kbps), 使用该选项会让程序无视 \verb|-crf| 选项。 \verb|-c:a copy| 直接复制音频不做改动。
\item 加上 \verb|-preset slow| 可以提高压缩率, \verb|-tune| 可以进一步调节。
\item \verb|-profile:v baseline或main或high| 也会有影响, 默认为 \verb|high|
\item 如果还要继续压缩, 就上 2 pass。 也就是变码率的压缩:
\item \verb|ffmpeg -y -i input.mp4 -c:v libx264 -b:v 1M -pass 1 -an -f null /dev/null|
\item \verb|ffmpeg -i input.mp4 -c:v libx264 -b:v 1M -pass 2 -c:a copy output.mp4|
\item 该命令的视频压缩对 H264 编码默认使用所有 cpu 核, 要指定线程数可以用 \verb|-threads| 选项。
\end{itemize}

\subsection{字幕}
\begin{itemize}
\item 要给视频添加 \verb|srt| 字幕(可以开启关闭的), 用 \verb|ffmpeg -i input.mp4 -scodec mov_text -i input.srt -c:v copy -c:a copy output.mp4|
\item 如果要把字母直接刻在视频中, 用 \verb|ffmpeg -i input.mp4 -vf "subtitles=input.srt" output.mp4|
\item 如果要不想让视频被字幕破坏, 可以在视频下面拓宽一条空间, 刻上字母: \verb|ffmpeg -i input.mp4 -vf "pad=width=iw:height=ih+120:color=black,subtitles=input.srt:force_style='MarginV=100,Alignment=10'" output.mp4|
\end{itemize}
