% IDA-star 算法
% 搜索|算法

\pentry{迭代加深\upref{ID} A-star 算法\upref{Astar}}

IDA* 算法是加了估价函数的迭代加深。

A* 算法是在优先队列 BFS 上加了估价函数,估价函数也当然可以和 DFS 结合,但如果只是和最普通的 DFS 结合在一起很容易出现搜索深度很深,但答案深度很浅的情况,所以可以将迭代加深 DFS 和估价函数结合在一起,就形成了 IDA-star 算法。IDA* 算法的估价函数和 A* 非常类似,都是表示当前状态到目标状态的估计距离,IDA* 当然也要满足估计距离不大于真实距离这个前提。IDA* 使用迭代的框架,如果当前深度加估计距离大于深度限制,则直接回溯。

\textbf{IDA* 的框架:}
\begin{lstlisting}[language=cpp]
int f();   // 估价函数

// depth 当前搜索层数,max_depth 为迭代加深的搜索深度限制
bool dfs(int depth, int max_depth)  
{
    // 如果当前层数 + 估价函数 > 深度限制,则直接回溯
    if (depth + f() > max_depth) return false; 
    if (!f()) return true;  // 一般估价函数为 0 说明找到了答案
    
    /*
    以下为 dfs 内容
    */
    
    return false;   // 找不到答案就回溯
}

int main()
{
    int depth = 0;
    while (!dfs(0, depth)) depth ++ ;   // 迭代加深
    
    return 0;
}
\end{lstlisting}

\textbf{\href{http://poj.org/problem?id=3460}{例题 $1$:排书}}

题目大意:每次可以将打乱的图书的一段取出放到其他位置的后面,问最少需要多少次可以将打乱的图书按照递增的顺序排列。

首先确定搜索顺序:枚举序列中每一段的图书摆到哪些位置,对于一段图书摆到一个位置的前面或者一个位置的后面是等价的,所以只需枚举摆到后面,摆的位置从选的那一段图书的右端点的位置加一开始枚举。这样就可以不遗漏的枚举出每种状态。

设计估价函数:设定一个正确/错误后继的概念,若一个图书 $i$,如果按照递增的顺序排列,它的下一个位置应该是 $i + 1$,我们称 $i + 1$ 是 $i$ 的正确后继,错误后继显然就是图书 $i$ 后面不是 $i + 1$ 的情况。

想要将一个打乱的图书变为递增的序列,显然序列中错误的后继数为 $0$,所以我们统计一个图书序列中错误的后继数,记为 $\tt cnt$,可以发现每次操作最多更改三本书的后继,若每次操作是最理想的,将一个乱序变为递增的序列最少也需要 $\left\lceil\dfrac{\tt cnt}{3}\right\rceil$ 次操作,满足估价函数的前提。故可将估价函数设计为 $f(\tt state) = \left\lceil\dfrac{\tt cnt}{3}\right\rceil$。

所以深度从 $1 \sim 4$ 增加,每次枚举序列中的每一段数放在哪些位置的后面,若当前深度加估价函数大于深度限制,则直接回溯,回溯时记得要恢复现场。

\textbf{时间复杂度:} $\mathcal{O}(\dfrac{(n - 1) \times n \times (n + 1)}{6}) = 560^4$。

对于整个图书中每次选择的一段图书,假设要选择的图书长度为 $i$,则一共有 $n - i + 1$ 种选法,这样还剩 $n - i$ 本书,还有 $n - i + 1$ 个位置可以放,所以一共有 $(n - i + 1) \times (n - i)$ 种选择,前面提到了,将一段书放在一个的位置的前面或后面没有区别,所以需要除二。

$i$ 从 $1 \sim 15$ 开始枚举,因此总共的选法有:

$\dfrac{15 \times 14 + 14 \times 13 + \cdots + 2 \times 1}{2}$ 种。

因为 $n \times (n - 1) + (n - 1) \times (n - 2) + \cdots + 2 \times 1 = \dfrac{(n - 1) \times n \times (n + 1)}{3}$。

所以对于一个长度为 $n$ 的序列,总的选法为 $\dfrac{(n - 1) \times n \times (n + 1)}{6}$。

虽然理论的时间复杂度为 $\mathcal{O}(560^4)$,但 IDA* 的实际效率很高。

还有一个细节就是如何将一段书放到一个位置的后面呢?可以使用一个数组 $W$,详细的可以看下图和代码。

\begin{figure}[ht]
\centering
\includegraphics[width=14cm]{./figures/IDAs_1.png}
\caption{示意图} \label{IDAs_fig1}
\end{figure}

\textbf{C++ 代码:}

\begin{lstlisting}[language=cpp]
const int N = 20;
int n, q[N], w[5][N];

// 估价函数,为当前序列中错误的后继的个数
int f()
{
    int cnt = 0;
    for (int i = 0; i + 1 < n; i ++ )
        if (q[i + 1] != q[i] + 1)
            cnt ++ ;
    return (cnt + 2) / 3;
}

bool dfs(int u, int depth)
{
    // 迭代加深
    if (u + f() > depth) return false;
    if (!f()) return true;  // 错误的后继为 0,说明序列单调递增

    for (int len = 1; len <= n; len ++ )  // 枚举序列长度
        for (int l = 0; l + len - 1 < n; l ++ ) // 枚举左端点
        {
            int r = l + len - 1;    // 右端点
            for (int k = r + 1; k < n; k ++ )   // 枚举能插入的位置
            {
                memcpy(w[u], q, sizeof q);  // 备份一下,为后面的回溯做准备

                // 将一段数 [l, r] 插到从 k 的后面
                int y = l;  
                for (int x = r + 1; x <= k; x ++ , y ++ ) q[y] = w[u][x];
                for (int x = l; x <= r; x ++ , y ++ ) q[y] = w[u][x];

                if (dfs(u + 1, depth)) return true;
                memcpy(q, w[u], sizeof q);  // 回溯时要恢复现场
            }
        }

    return false;
}
\end{lstlisting}

\textbf{\href{http://poj.org/problem?id=2286}{例题 $2$:回转游戏}}

首先确定一下搜索顺序,对于每个状态,每次枚举当下的八种操作,可以发现一个剪枝优化,每个操作对应着有个逆操作,例如先操作了 $A$ 操作,下次就不要再操作 $F$ 了,所以每次操作前可以判断一下当前操作是不是上一个操作的逆操作。

搜索树的深度很深,但答案应该不会太深,所以可以使用迭代加深,加一个估价函数,就变成了 IDA*。为了要使井形棋盘最中间的 $8$ 个格子里的数字相同,如果中间$8$ 个格子里的数字出现次数最多为 $k$,还有其他的 $8 - k$ 个数字,则最少还需 $8 - k$ 次操作,所以可以设计估价函数为 $f(\texttt{state}) = 8 - k$。

棋盘中的操作比较复杂,所以可以将棋盘中每个数字给予一个编号,