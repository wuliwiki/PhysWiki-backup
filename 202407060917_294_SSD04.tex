% 首都师范大学 2004 年硕士入学考试试卷
% keys 首都师范大学|考研|物理|2004
% license Copy
% type Tutor

\begin{enumerate}
\item 竖直弹簧振子悬吊在电梯的天花板上,在竖直方向做微幅振动。振子的质量为m,轻弹簧的弹性系数为k:电梯以匀加速度a上升。求振子的振动周期。
\item 半径为R、质量为m的圆柱体,放置在摩擦系数为$\mu$的水平地面上。一水平拉力F作用在圆柱体的中心轴上。为保证圆柱体只滚动不滑动,水平拉力F的数值不能大于多少?
\item 一质量为M、长度为2L的均匀细杆,静止地放置在光滑、水平桌面上,一质量为 m(m<<M)的质点,沿水平桌面、以垂直杆的方向运动,并与杆的一端发生完全非弹性碰撞。设碰撞前质点的速度为$V_0$,求撞击后杆的运动情况。
\item 质量为m,电量为q的粒子在电场强度为$\vec E$的均匀电场中运动。设t=0时刻粒子的位置矢径为$\vec r_0$,速度为$\vec v_0$。试求该粒子任意t时刻的位置矢径与时间的函数关系。
\item 半径为R的金属球外包着一层厚为R的介质球壳,壳的介电常数为$\varepsilon_r=2$,壳外是真空。如图1。现将电荷量为Q的自由电荷均匀分布在介质壳体内,在金属球的电势为0的情况下,试求介质壳的外表面上的电势。
\begin{figure}[ht]
\centering
\includegraphics[width=6cm]{./figures/7e377c3a2951ba20.png}
\caption{} \label{fig_SSD04_1}
\end{figure}
\item 已知氘核的质量比质子大一倍,电荷量与质子相同:$\alpha$粒子的质量是质子质量的四倍,电荷量是质子的两倍。(1)试问静止的质子、氘核和$\alpha$粒子经过相同的电压加速后,它们的动能之比是多少?(2)当它们经过这样加速后进入同一均匀磁场,它们都作圆周运动,测得质子圆轨道的半径为10cm,试问核和“粒子轨道的半径各是多少?
\end{enumerate}