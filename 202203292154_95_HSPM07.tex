% 功和机械能
% keys 功|机械能|势能|动能|能量守恒定律

\begin{issues}
\issueDraft
\issueTODO
\end{issues}

\subsection{功}

如果一个力$\bvec F$作用在物体上,且物体在这个力的方向上发生了位移$\bvec s$,就说这个力对物体做了\textbf{机械功}(简称\textbf{功}).功常用字母$W$表示.做功的过程是能量变化的过程,功是能量变化的量度.

\subsubsection{恒力$\bvec F$对物体做的功}

当力的方向与物体位移的方向相同时,功的大小等于力的大小与位移大小的乘积,即:
\begin{equation}\label{HSPM07_eq1}
W=Fs
\end{equation}

力和位移都是矢量,从定义上看,功是力和位移的内积,因此功是一个标量.当物体在多个力的作用下发生了一段位移,它们对物体所做的总功等于各个力对物体所做的功的代数和.

在国际单位中,功的单位是\textbf{焦耳},简称\textbf{焦},符号为$\mathrm{J}$,$1\mathrm{J}=1\mathrm{N \cdot m}$.

当力的方向与物体位移的方向的夹角为$\theta$时,对$\bvec F$进行正交分解可得$\bvec F$沿位移方向的分力$F_1=F\cos \theta$,垂直于位移方向的分力$F_2=F\sin \theta$.易知物体在分力$F_1$的方向上发生了位移$\bvec s$,在$F_2$的方向上没有位移,$\bvec F$对物体做功大小为:
\begin{equation}\label{HSPM07_eq2}
W=Fs\cos \theta
\end{equation}

\begin{itemize}
\item 当$\theta = 0$时,\autoref{HSPM07_eq2} 可化简为\autoref{HSPM07_eq1} 的形式.
\item 当$0\leq \theta < \pi/2$时,$W>0$,力$\bvec F$为动力,推动物体的运动,对物体做正功.
\item 当$\pi/2< \theta \leq \pi$时,$W<0$,力$\bvec F$为阻力,阻碍物体的运动,对物体做负功.
\item 当$\theta = \pi/2$时,$W=0$,力$\bvec F$对物体不做功.
\end{itemize}

要注意的是,对于功的计算式,位移必须是在力在作用过程中发生的.

\subsubsection{变力对物体做的功}

对于变力做功,需要根据实际情况,选择不同的方法,此处列举几个常用的方法:

\addTODO{以下内容待完善后添加引用}

\begin{itemize}
\item 分解成多个恒力做功的阶段,分别计算再求代数和.
\item 图像法:已知力—位移图像(力和位移在同一直线上)时,曲线与$x$轴上方围成的面积为正功,与$x$轴下方围成的面积为负功,高中阶段适用于规则几何图形及割补法可计算的情况.
\item 求平均力:如果力的方向不变,其大小随位移均匀变化,可求出物体所受的平均力,进而用\autoref{HSPM07_eq2} 求解,这种情况也可从用图像法计算三角形面积的情况推出.
\item 利用动能定理列式求解
\item 微元法:如一个物体在一个大小不变的拉力$\bvec F$下做半径为$r$的匀速圆周运动,力的方向和运动方向始终一致,总功等于在无数段极小的位移上恒力$\bvec F$做功的和,所有极小位移的大小之和为$2\pi r$,则$W=F\Delta s_1+F\Delta s_2+\dots+F\Delta s_n=2\pi rF$
\item 已知恒定功率或平均功率以及做功时间,用式3求解.
\end{itemize}
