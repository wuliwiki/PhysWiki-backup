% 多元函数的极值
% keys 极值|充要条件|二次型

\pentry{二元函数的极值(简明微积分)\upref{F2Exm}}
\subsection{极值}
设函数 $u=f(x_1,\cdots,x_n)$ 定义于区域 $\mathcal{D}$ 中,且 $(x_1^0,\cdots,x_n^0)$ 是这区域的内点.

\begin{definition}{极值}
若点 $(x_1^0,\cdots,x_n^0)$ 有这样一个领域
\[(x_1^0-\delta,x_1^0+\delta;\cdots;x_n^0-\delta,x_n^0+\delta)\]
使对于其中一切点都能成立不等式
\begin{equation}
\begin{aligned}
f(x_1,\cdots,x_n)&\leq f(x_1^0,\cdots,x_n^0)\\
&(\geq)
\end{aligned}
\end{equation}
就说,函数 $f(x_1,\cdots,x_n)$ 在点 $(x_1^0,\cdots,x_n^0)$ 处有\textbf{极大值}(\textbf{极小值}).

若在除去点 $(x_1^0,\cdots,x_n^0)$ 本身以外区域中的每一点都能成立严格不等式
\begin{equation}
\begin{aligned}
f(x_1,\cdots,x_n)&< f(x_1^0,\cdots,x_n^0)\\
&(>)
\end{aligned}
\end{equation}
就说,函数 $f(x_1,\cdots,x_n)$ 在点 $(x_1^0,\cdots,x_n^0)$ 处有\textbf{真正的}极大值(极小值);否则,极大值(极小值)就称为\textbf{广义的}.

极大值和极小值总称为\textbf{极值}.
\end{definition}