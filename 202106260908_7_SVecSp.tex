% 超线性空间
% super vector space|超线性空间|超代数|super algebra|超向量空间|超对称性|super symmetry|分次空间|graded space

\pentry{矢量空间\upref{LSpace},环的理想和商环\upref{Ideal}}

超线性空间是一种在现代理论物理中应用的代数结构,用于描述\textbf{超对称性}的各种代数性质.它是普通线性空间的一个简单推广,附加了对其中元素奇偶性的判断.

\subsection{基本概念}

\begin{definition}{超线性空间}
域$\mathbb{F}$上的\textbf{超线性空间(super vector space)}$V$是一个\textbf{分次(graded)线性空间},它是两个齐次子空间$V_0$和$V_1$的直和:$V=V_0\oplus V_1$,其中$0, 1\in \mathbb{Z}_2$.
\end{definition}


$V_i$中的元素称为\textbf{齐次的(homogeneous)}.$V_i$的子空间被称为$V$的\textbf{齐次子空间(homogeneous subspace)},比如$V_i$本身.

齐次元素的\textbf{奇偶性(parity)}由所属子空间决定:对于$v\in V_i$,定义$\abs{v}=i\in\mathbb{Z}_2$,其中$\abs{v}=0$时称$v$是\textbf{偶(even)}的,否则是\textbf{奇(odd)}的.在理论物理中,偶元素也被称为\textbf{玻色元素(Bose element)},奇元素则被称为\textbf{费米元素(Fermi element)}.

由于超线性空间是两个齐次空间的直和,因此超线性空间中也存在非齐次的元素.

\begin{definition}{维度}
令$V=V_0\oplus V_1$是超线性空间,$V_i$是它的齐次子空间.如果$\opn{dim}V_0=n$,$\opn{dim}V_1=m$,那么称$V$的维度是$n|m$.$V$的坐标空间是$\mathbb{F}^{n+m}$上赋予分次结构的结果,记为$\mathbb{F}^{n|m}$.

约定$\mathbb{F}^{n+m}$的前$n$个坐标表示$V_0$中的坐标,后$m$个表示$V_1$中的.
\end{definition}

\begin{definition}{反演算子}
对于上述超线性空间$V$,我们定义$\Pi V=(\Pi V)_0\oplus(\Pi V)_1$为另一个超线性空间,其中$(\Pi V)_0=V_1$,$(\Pi V)_1=V_0$.
\end{definition}


\subsection{线性变换}

超线性空间空间之间的\textbf{同态(homomorphism)}\footnote{这种同态是超线性空间集合上的一种\textbf{射(morphism)},从而构成了\textbf{超线性空间范畴}.范畴的概念请参见\textbf{范畴论}\upref{Cat}.},定义为一个\textbf{保次线性映射},即如果有两个超线性空间$V=V_0\oplus V_1$和$W=W_0\oplus W_1$,那么一个线性映射$f:V\to W$是保次的,当且仅当$f(V_i)\subseteq W_i, i=0, 1$.

换句话说,超线性空间同态不改变元素的奇偶性.




































