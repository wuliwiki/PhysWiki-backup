% 朱利安·施温格
% license CCBYSA3
% type Wiki

(本文根据 CC-BY-SA 协议转载自原搜狗科学百科对英文维基百科的翻译)

\begin{figure}[ht]
\centering
\includegraphics[width=6cm]{./figures/2545751ad57d154c.png}
\caption{朱利安施温格,1965年诺贝尔物理学奖获得者。原始描述:“他的实验室是他的圆珠笔。”} \label{fig_Julian_2}
\end{figure}
\textbf{朱利安·西摩·施温格} (/ˈʃwɪŋər/;1918年2月12日-1994年7月16日)是一个赢得诺贝尔奖的美国理论物理学家。他最出名的是量子电动力学理论 (QED),特别是用于发展相对不变量微扰理论,并将量子电动力学理论重新规格化为一个循环顺序。施温格是几所大学的物理学教授。

施温格被公认为二十世纪最伟大的物理学家之一,他对现代量子场论有很大贡献,包括变分方法和量子场的运动方程。他开发了第一个弱电模型和第一个1+1维约束的例子。施格温还负责多重中微子理论、施温格项和自旋3/2场理论。

\subsection{传记}
朱利安·西摩·施温格出生于纽约市,父母是犹太人,来自波兰,贝尔(内·罗森菲尔德)和本杰明·施温格,经营着服装制造商,[1] 曾移民到美国。 他的父亲和母亲的父母都是富裕的服装制造商,但是这个家族企业在1929年华尔街崩盘后就衰落了 。施温格一家人跟随了正统犹太人传统。施温格先在汤森·哈里斯上高中,然后在纽约城市学院读本科,之后转到哥伦比亚大学,并1936年获得学士学位,在1939年21岁时博士学位(由 伊多·艾萨克·拉比)19 加州学伯克利分校 (在J罗伯特·奥本海默),后领导来工作被任命在普渡大学就任。

\subsubsection{1.1 职业}
在与奥本海默共事后,施温格的第一次定期学术任命是1941年在普渡大学。在普渡休假期间,第二次世界大战期间他在麻省理工学院的辐射实验室工作 而不是在洛斯阿拉莫斯国家实验室 。他为雷达的发展提供了理论支持。战后,施温格离开普渡前往哈佛大学,从1945年到1974年在那里教书。1966年,他成为哈佛大学尤金·希金斯物理学教授。

施温格从他的雷达工作中发展出格林函数,并用这些方法以相对论不变的方式,用局域格林函数来建立量子场论。这使他能够精确不含糊地量子电动力学中电子磁矩的第一个修正。早期的非协变工作得出了无穷大的答案,但是他的方法中额外对称性允许施温格分离出正确的有限修正。

施温格发展了重正化,明确地将量子电动力学公式化为一个环序。

在同一时代,他通过计算电场中隧道效应产生电子-正电子对的速度将非微扰方法引入了量子场论,这一过程现在被称为“施温格效应”。在微扰理论中,这种效应在任何有限阶上都是看不到的。

施温格在量子场论方面的基础工作构建了场相关函数及其运动方程的现代框架。他的方法从一个量子作用开始,用格拉斯曼积分的微分形式使玻色子和费米子第一次被同等对待。他为自旋统计定理和CPT定理提供了完美的证据,并指出 短距离奇点场,代数导致异了各种经典恒等式中的异常施温格术语。这些是场论的基础结果,有助于正确理解反常。

在其他著名的早期作品中,拉力塔和施温格阐述了抽象泡利和Fierz理论具体形式的自旋3/2场理论,作为狄拉克自旋的矢量。为了使自旋-3/2场一致地相互作用,某种形式的超对称性是必需的,施温格后来后悔他没有对这项工作进行足够深入的跟踪以发现超对称性。

施温格发现中微子有多种,一种是电子,另一种是μ介子。现在已知有三个轻中微子;第三个是τ轻子。

20世纪60年代,施温格制定并分析了现在被称为施温格模型的理论,在一个空间和一个时间维度的量子电动力学,这是一个例子的第一个例子 限制理论。他也是第一个提出电弱规范理论的人,一个SU(2)规范群在长距离内自发地被电磁U(1)打破。他的学生谢尔登·格拉秀将这一观点扩展到了公认的弱电统一模式。他试图用点磁单极子来阐述量子电动力学理论,因为当电荷量很小时,单极子有强烈的相互作用,所以这个项目的成功率有限。

施温格曾指导了73篇博士论文 ,[2] 被认为是物理学领域最多产的研究生导师之一。他的四名学生获得了诺贝尔奖:Roy Glauber,本杰明·罗伊·莫特森,谢尔登·格拉秀和沃尔特·科恩 (化学)。

施温格和他的同事关系复杂,因为他总是追求独立的研究,不同于主流时尚。特别是施温格发展了源理论,[3] 是基本粒子物理学的现象学理论,是现代物理学有效场论的前身。它将量子场视为长距离现象,并使用类似经典场论中电流的辅助“源”。源理论是一个数学上一致的场论,具有清晰的现象学结果。他的哈佛同事的批评导致施温格于1972年离开学院进入加州大学洛杉矶分校。 这是一个广为人知的故事,史蒂芬·温伯继承了施温格在莱曼实验室小组,在那里发现了一双旧鞋,上面隐含着这样的信息:“你认为你能穿上吗?”。在加州大学洛杉矶分校,在他职业生涯的剩余时间里,施温格继续发展源理论及其各种应用。

1989年以后,施温格对冷聚变的非主流研究产生了浓厚的兴趣。他就此写了八篇理论论文。在他们拒绝发表他的论文之后,他辞去了美国物理学会的工作。[4] 他觉得冷聚变研究受到了压制,并且学术自由受到了侵犯。他写道:“从众的压力是巨大的。我曾经历过编辑基于对匿名推荐人恶毒的批评而拒绝提交论文的情况。审查制度取代公正审查将是科学的死亡。"

施温格在他最近的出版物中提出了一个声致发光理论,它是一种长距离量子辐射现象,与原子无关,而是与坍缩气泡中存在介电常数不连续的快速运动的表面有关。目前实验支持的声致发光机理主要聚焦于气泡内的过热气体作为光源。[5]

施温格与理查德·费曼和shin ' ichirtomonaga一起因为他们在量子电动力学(QED)方面的工作被授予诺贝尔物理学奖。施温格的奖项和荣誉在他获得诺贝尔奖之前就已经数不胜数了。包括第一个阿尔伯特·爱因斯坦奖(1951年),美国国家科学奖章(1964年)、普渡大学荣誉博士学位(1961年)和哈佛大学荣誉博士学位(1962年)以及美国光的本质奖国家科学院(1949年)。

\subsubsection{1.2 施温格和费曼}
作为著名的物理学家,施温格经常被人们与那一代的另一位传奇物理学家理查德·费曼进行比较。施温格更倾向于量子场论的符号工作。他在研究局部场算符时,发现了它们之间的关系,他觉得物理学家应该理解局部场的代数,不管它有多矛盾。相比之下,费曼更直观,他相信物理学可以完全从费曼图给出的粒子图像提取出来。施温格对费曼图的评论如下:

像最近几年的硅芯片一样,费曼图把计算带到了大众面前。 [6][7]

施温格不喜欢费曼图,因为他觉得它让学生把注意力集中在粒子上,而忽略了局部场,在他看来,这些抑制了学生们的理解。他甚至完全禁止他们上他的课,尽管他完全理解他们。然而真正的区别更深,施温格在下面的文章中表达了这一点,

最后,这些想法导致了量子力学的拉格朗日或作用公式,以两种不同但相关的形式出现,我将其区分为微分和积分。以费曼为首的后一种观点已经得到了所有的新闻报道,但我仍然相信存在差异的观点更为普遍、更完美、更有用。[8]

尽管分享了诺贝尔奖,施温格和费曼对量子电动力学和一般量子场论有不同的看法。费曼使用了一个调节器,而施温格能够在没有明确调节器的情况下正式地 重整为一个环路。施温格相信局部场的形式主义,而费曼相信粒子路径。他们密切关注彼此的工作,互相尊重。费曼死后,施温格形容他为

一个诚实的人,我们这个时代杰出的直观主义者,对于任何胆敢追随不同鼓声的人来说,这是一个很好的例子。[9]

\subsubsection{1.3 死亡}
施温格死于胰腺癌。他被安葬在 奥本山公墓:α/2π刻在他墓碑上的名字上面。这些符号指的是他对电子磁矩的校正(“异常”)。
\begin{figure}[ht]
\centering
\includegraphics[width=6cm]{./figures/3d2209a416d83cc5.png}
\caption{马萨诸塞州剑桥奥本山公墓朱利安施温格的墓碑。} \label{fig_Julian_1}
\end{figure}

\subsection{精选出版物}
\begin{itemize}
\item Schwinger, J (1948). "On Quantum-Electrodynamics and the Magnetic Moment of the Electron". Phys. Rev. \textbf{73} (4): 416–417. Bibcode:1948PhRv...73..416S. doi:10.1103/PhysRev.73.416.
\item Schwinger, J (1948). "Quantum Electrodynamics. I. A Covariant Formulation". Phys. Rev. \textbf{74} (10): 1439–1461. Bibcode:1948PhRv...74.1439S. doi:10.1103/PhysRev.74.1439.
\item Schwinger, J (1949). "Quantum Electrodynamics. II. Vacuum Polarization and Self-Energy". Phys. Rev. 75 (4): 651–679. Bibcode:1949PhRv...75..651S. doi:10.1103/PhysRev.75.651.
\item Schwinger, J (1949). "Quantum Electrodynamics. III. The Electromagnetic Properties of the Electron Radiative Corrections to Scattering". Phys. Rev. \textbf{76} (6): 790–817. Bibcode:1949PhRv...76..790S. doi:10.1103/PhysRev.76.790.
\item 费什巴赫、h、施温格、j和J. A .哈勒。 "Effect of Tensor Range in Nuclear Two-Body Problems",计算实验室 哈佛大学, 美国能源部 (通过前身机构 原子能委员会)(1949年11月)。
\item Schwinger, J (1951). "On Gauge Invariance and Vacuum Polarization". Phys. Rev. \textbf{82} (5): 664–679. Bibcode:1951PhRv...82..664S. doi:10.1103/PhysRev.82.664.
\item 施温格 "On Angular Momentum", 哈佛大学核发展协会, 美国能源部 (通过前身机构 原子能委员会)(1952年1月26日)。
\item 施温格 "The Theory of Quantized Fields. II", 哈佛大学, 美国能源部 (通过前身机构 原子能委员会)(1951年)。
\item 施温格 "The Theory of Quantizied Fields. Part 3", 哈佛大学, 美国能源部 (通过前身机构 原子能委员会)(1953年5月)。
施温格 爱因斯坦的遗产 (1986年)。科学美国人图书馆。
\end{itemize}

\subsection{参考文献}
[1]
^Mehra, Jagdish (2000). Climbing the mountain: the scientific biography of Julian Schwinger. Oxford University Press. pp. 1–5..

[2]
^"Julian Schwinger Foundation" (PDF). nus.edu.sg. Archived (PDF) from the original on 26 March 2016. Retrieved 1 May 2018..

[3]
^Schwinger, J.S. Particles, Sources, and Fields. Vol. 1 (1970) ISBN 9780738200538, Vol. 2 (1973) ISBN 9780738200545, Reading, MA: Addison-Wesley.

[4]
^Jagdish Mehra, K. A. Milton, Julian Seymour Schwinger (2000), Oxford University Press, ed., Climbing the Mountain: The Scientific Biography of Julian Schwinger (illustrated ed.), New York: Oxford University Press, p. 550, ISBN 978-0-19-850658-4CS1 maint: Multiple names: authors list (link), Also Close 1993,第197–198页.

[5]
^Brenner, M. P.; Hilgenfeldt, S.; Lohse, D. (2002). "Single-bubble sonoluminescence". Reviews of Modern Physics. 74 (2): 425–484. Bibcode:2002RvMP...74..425B. doi:10.1103/RevModPhys.74.425..