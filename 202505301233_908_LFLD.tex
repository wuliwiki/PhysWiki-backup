% 列夫·朗道(综述)
% license CCBYSA3
% type Wiki

本文根据 CC-BY-SA 协议转载翻译自维基百科 \href{https://en.wikipedia.org/wiki/Lev_Landau}{相关文章}。

列夫·达维多维奇·朗道(俄语:Лев Дави́дович Ланда́у,1908年1月22日-1968年4月1日)是一位苏联物理学家,在理论物理的诸多领域作出了基础性的贡献。\(^\text{[1][2][3]}\)他被认为是最后一批在物理学各个分支都造诣深厚并做出开创性贡献的科学家之一。\(^\text{[4]}\)他被誉为20世纪凝聚态物理学的奠基人,\(^\text{[5]}\)同时也被广泛认为是苏联最杰出的理论物理学家。\(^\text{[6]}\)
\subsection{生平}
\subsubsection{早年时期}
\begin{figure}[ht]
\centering
\includegraphics[width=6cm]{./figures/ce294ae7da64364f.png}
\caption{朗道一家,1910年} \label{fig_LFLD_1}
\end{figure}
朗道于1908年1月22日出生在俄罗斯帝国的巴库(今属阿塞拜疆),父母是犹太人[11][12][13][14]。他父亲达维德·列沃维奇·朗道是一位从事当地石油工业的工程师,母亲柳博芙·维尼亚米诺芙娜·加尔卡维-朗道是一名医生。两人都来自莫吉廖夫,并毕业于当地的文理中学[15][16]。朗道12岁学习微分学,13岁学习积分学,并在1920年13岁时从中学毕业。由于父母认为他年龄太小,不适合直接升入大学,他先在巴库经济技术学校学习了一年。
1922年,年仅14岁的朗道进入巴库国立大学,同时注册了两个系:物理与数学系以及化学系。后来他中止了化学的学习,但终其一生对化学始终保有兴趣。
\subsubsection{列宁格勒与欧洲时期}
\begin{figure}[ht]
\centering
\includegraphics[width=6cm]{./figures/5c374c1bae8ad643.png}
\caption{1914年的少年朗道} \label{fig_LFLD_2}
\end{figure}
1924年,朗道前往当时苏联物理学的主要中心——列宁格勒国立大学物理系,专注于理论物理的学习,并于1927年毕业。此后,他进入列宁格勒物理技术研究所攻读研究生,并最终于1934年获得物理-数学科学博士学位。[17]1929年至1931年,朗道首次获得出国机会,依靠苏联政府(教育人民委员部)提供的出国奖学金,同时也得到了洛克菲勒基金会的资助。在这段时间里,他已能流利地使用德语和法语,并能以英语交流。[18] 后来,他进一步提高了英语水平,并学习了丹麦语。[19]

朗道曾短暂访问哥廷根和莱比锡,随后于1930年4月8日前往哥本哈根,在尼尔斯·玻尔理论物理研究所工作,直到同年5月3日离开。这次访问之后,朗道始终视自己为尼尔斯·玻尔的学生,而他的物理研究方法也受到玻尔深刻的影响。离开哥本哈根后,朗道于1930年中期访问剑桥,与保罗·狄拉克合作研究,[20] 同年9月至11月他再次回到哥本哈根,[21] 接着于1930年12月至1931年1月在苏黎世与沃尔夫冈·泡利共事。[20] 从苏黎世出发后,他第三次前往哥本哈根,[22] 并于1931年2月25日至3月19日再次在那里停留,然后于同年返回列宁格勒。[23]
\subsubsection{乌克兰哈尔科夫:国家科学中心哈尔科夫物理技术研究所}
1932年至1937年间,朗道担任国家科学中心哈尔科夫物理技术研究所理论物理系主任,并在哈尔科夫大学和哈尔科夫理工学院讲授课程。除理论研究外,朗道还是乌克兰哈尔科夫理论物理学传统的主要奠基人,这一学派有时被称为“朗道学派”。在哈尔科夫,他与朋友兼前学生叶甫根尼·利夫希茨开始撰写著名的《理论物理教程》,这一套涵盖理论物理全部领域的十卷巨著,至今仍被广泛用作研究生阶段的物理教材。在大清洗期间,朗道因涉及哈尔科夫的“UPTI 案”受到调查,但他设法离开哈尔科夫,前往莫斯科接受新职。[3]

朗道制定了一项著名的综合考试,被称为“理论最低限”,学生只有通过这项考试后才能正式进入他的学派学习。该考试涵盖理论物理的各个方面。从1934年到1961年,仅有43人通过,但这些通过者后来都成为了非常杰出的理论物理学家。[24][25]

1932年,朗道计算出了“钱德拉塞卡极限”;[26] 然而,他当时并未将其应用于白矮星。[27]
\subsubsection{莫斯科物理问题研究所}
\begin{figure}[ht]
\centering
\includegraphics[width=8cm]{./figures/cfc5b79c45f1b664.png}
\caption{} \label{fig_LFLD_3}
\end{figure}
自1937年至1962年,朗道担任莫斯科物理问题研究所理论部主任。[28]

1938年4月27日,朗道因持有一份传单而被捕,该传单将斯大林主义与德国纳粹主义和意大利法西斯主义相提并论。[3][29] 他被关押在内务人民委员部卢比扬卡监狱,直到1939年4月29日才获释。这次获释是由于彼得·卡皮察(著名的低温实验物理学家、该研究所创始人及所长)和尼尔斯·玻尔向约瑟夫·斯大林写信为他求情。[30][31] 卡皮察亲自担保朗道的品行,并威胁若不释放朗道将辞职离所。[32]获释后,朗道提出了解释卡皮察发现的超流现象的理论,使用声子(声波激发)与一种新的激发形式——旋子。[3]

朗道还带领一支数学家团队,支持苏联原子弹与氢弹的研制。他曾计算出首枚苏联热核武器的动力学过程,并预测了其当量。由于这项工作,朗道在1949年与1953年两度获得“斯大林奖”,并于1954年被授予“社会主义劳动英雄”称号。[3]

朗道的学生包括:列夫·皮塔耶夫斯基、阿列克谢·阿布里科索夫、亚历山大·阿希耶泽尔、伊戈尔·贾洛辛斯基、叶甫根尼·利夫希茨、列夫·戈尔科夫、伊萨克·哈拉托尼科夫、罗阿尔德·萨格杰耶夫和伊萨克·波梅朗丘克。
