% 度量空间的连续映射和等距
% keys 连续|度量空间|等距|同胚
% license Usr
% type Tutor

\pentry{度量空间\nref{nod_Metric}}{nod_9cd3}
连续映射在分析学和拓扑学中都有定义。从逻辑上来说,度量空间是拓扑空间的特殊情形,初等分析学中的函数(数)空间是度量空间的特殊情形。因此,\textbf{更好}的学习方式是从底层理论开始学的。例如,“连续映射”的概念应当以拓扑学中的概念为最基本的概念,其它的情形的定义只不过是这一基本概念的特殊情形。然而,从现实来说,底层框架仅仅给出了构造世界的最基本构件,而生命(意识)的诞生需要在基本框架上附加更复杂的结构。因此从生命(意识)的认识来说,一开始接触到框架本身就是嵌套了额外的复杂结构,而这对于生命(意识)来说则更加感性具体。正如连续映射,从生命(意识)的认识来说,分析学中的函数的连续性则更加感性具体。这也解释为什么探寻真理的过程需要不断的进行抽象,寻找出其中更普遍的结构。因此,于我们而言,\textbf{更合理}的学习方式是从具有复杂结构的理论开始学习,再循序渐进的到更基本的理论中去。

因此,对连续映射的认识不是从拓扑空间中开始,而是以初等分析学中函数的连续性作为起点,再到达度量空间,最后才是拓扑空间。本节将从度量空间的视角去认识连续性的概念。

\subsection{连续映射}
函数的连续概念通过数与数之差的绝对值来定义,这一绝对值度量了两数之间的距离。在度量空间中,有距离的概念,因此,可以类似的建立起度量空间中连续映射的概念。
\begin{definition}{连续映射}\label{def_ConIso_1}
设 $(X_1,d_1),(X_2,d_2)$ 是两度量空间,$f:X_1\rightarrow X_2$ 是 $X_1$ 到 $X_2$ 上的映射,$x_0\in X_1$。若对任意的 $\epsilon>0$,存在 $\delta>0$ 满足
\begin{equation}
d_1(x,x_0)<\delta \Rightarrow d_2(f(x),f(x_0))<\epsilon,~
\end{equation}
则称 $f$ 在 $x_0$ 处\textbf{连续}。若 $f$ 在 $X_1$ 的每一点都连续,则称 $f$ 在 $X_1$ 上\textbf{连续}。
\end{definition}

\begin{example}{连续函数}
若 $X_1,X_2\in\mathbb R$,则\autoref{def_ConIso_1} 的连续性定义便是连续函数的定义。
\end{example}
类似的,可以定义多元映射的连续性。
\begin{definition}{}
设 $(X,d),(X_i,d_i),i=1,\ldots,n$ 是度量空间,$f:X_1\times\cdots\times X_n\rightarrow X$ 是映射,$x=(x_1,\ldots,x_n)\in X_1\times\cdots\times X_n$。若对任意的 $\epsilon>0$,存在 $\delta>0$ 满足
\begin{equation}
d_i(y_i,x_i)<\delta,i=1\cdots,n\quad \Rightarrow\quad d(f(y),f(x))<\epsilon,~
\end{equation}
则称 $f$ 在 $x$ 处\textbf{连续},其中 $y=(y_1,\cdots,y_n)$。若 $f$ 在 $X_1\times\cdots\times X_n$ 的每一点都连续,则称 $f$ 在 $X_1\times\cdots\times X_n$ 上\textbf{连续}。
\end{definition}
\subsection{距离是连续的}
\begin{lemma}{有关距离的一个不等式}\label{lem_ConIso_1}
设 $(X,d)$ 是度量空间,则
\begin{equation}\label{eq_ConIso_1}
\abs{d(x,y)-d(x_0,y_0)}\leq d(x_0,x)+d(y_0,y).~
\end{equation}
\end{lemma}

\textbf{证明:}
由三角不等式(\autoref{def_Metric_2}),有
\begin{equation}
\begin{aligned}
\abs{d(x,y)-d(x_0,y_0)}&\leq\abs{d(x,y_0)+d(y_0,y)-d(x_0,y_0)}\\
&\leq\abs{d(x,x_0)+d(x_0,y_0)+d(y_0,y)-d(x_0,y_0)}\\
&\xlongequal{d\text{的正定性和对称性}}d(x_0,x)+d(y_0,y).
\end{aligned}~
\end{equation}
\textbf{证毕!}

\begin{theorem}{距离的连续性}\label{the_ConIso_1}
度量空间的距离是连续的。即若 $(X,d)$ 是度量空间,则 $d$ 在 $X$ 上是连续的。
\end{theorem}

\textbf{证明:}
由\autoref{lem_ConIso_1} ,对任意 $\epsilon>0$,选取 $\delta=\epsilon/2$,则对 $d(x,x_0)<\delta,d(y,y_0)<\delta$,成立
\begin{equation}
\abs{d(x,y)-d(x_0,y_0)}\leq d(x,x_0)+d(y,y_0)<\epsilon.~
\end{equation}
\textbf{证毕!}
\subsection{等距}
\begin{definition}{同胚}
度量空间 $X$ 到 $Y$ 的连续映射 $f:X\rightarrow Y$ 称为\textbf{同胚},若 $f$ 是双射。可以建立同胚的空间 $X,Y$ 称为相互\textbf{同胚的}。
\end{definition}

\begin{example}{}
\begin{equation}
y=\frac{2}{\pi}\arctan x~
\end{equation}
是 $(-\infty,\infty)$ 到 $(-1,1)$ 上的同胚映射。 
\end{example}

\begin{definition}{等距}\label{def_ConIso_2}
设 $(X_1,d_1),(X_2,d_2)$ 是两度量空间,$f:X\rightarrow Y$ 是双射。若对 $\forall x,y\in X_1$,成立
\begin{equation}
d_2(f(x),f(y))=d_1(x,y),~
\end{equation}
则称 $f$ 为 $(X_1,d_1)$ 到 $(X_2,d_2)$ 的\textbf{等距}(isometry)。能够建立等距映射的两度量空间称为\textbf{等距的}(isometric)。
\end{definition}












