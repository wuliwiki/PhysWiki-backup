% Julia 项目的创建与引入
% 项目 创建 引入

本文授权转载自郝林的 《Julia 编程基础》. 原文链接:\href{https://github.com/hyper0x/JuliaBasics/blob/master/book/ch02.md}{第2章:编程环境}.


\subsubsection{2.3 项目的创建与引入}

到目前为止,我们已经对 Julia 的项目环境有了一定的了解,并讨论了仓库目录、程序包存储目录和环境配置这几个重要的概念.

项目的环境配置一般由项目文件\verb|Project.toml|和清单文件\verb|Manifest.toml|体现.全局环境配置针对的是 REPL 环境中的代码和独立的 Julia 程序.该环境配置的文件通常会在\verb|~/.julia/environments|目录的特性版本子目录(如\verb|v1.3|)中.

我们已经学会了在全局环境中怎样安装和导入程序包.这对于编写和部署一些脚本程序来说已经足够了.不过,我们在正式开发 Julia 项目的时候往往还需要专属的环境配置文件.这样才能与全局环境区别开.这么做有 3 个好处:

\begin{enumerate}
\item 一旦有了专属的环境配置文件,我们的项目就可以独立地管理依赖包了.
\item 针对某个 Julia 项目的程序包管理操作不会影响到全局的环境配置.反之亦然.
\item 拥有环境配置文件的 Julia 项目可以为项目的分发(以供他人使用)做好准备.
\end{enumerate}

换句话说,我们的 Julia 项目可以因此成为独立的、可重用的以及对分发友好的项目.

\textbf{2.3.1 项目的创建}

创建一个 Julia 项目很容易,在 REPL 环境中就可以办到.我们先在命令行中通过输入\verb|julia|命令进入到 REPL 环境.然后,我们在它的 shell 模式中进入某个专用的目录(比如\verb|~/Projects|),就像这样:

\begin{lstlisting}[language=julia]
shell> cd ~/Projects/
/Users/haolin/Projects

julia> 
\end{lstlisting}

这时,我们可以再确认一下当前的目录:

\begin{lstlisting}[language=julia]
julia> pwd()
"/Users/haolin/Projects"

julia> 
\end{lstlisting}

这里的\verb|pwd|是 Print Working Directory 的缩写.所以,\verb|pwd|函数的含义就是打印当前的工作目录.这与在命令行中输入\verb|pwd|命令的作用是类似的.只不过调用表达式\verb|pwd()|的求值结果是一个字符串.

在确认了工作目录之后,我们就可以切换到 REPL 环境的 pkg 模式,然后输入\verb|generate|命令,并后跟一个空格和项目的名称\verb|Programs|(你也可以用别的名字):

\begin{lstlisting}[language=julia]
(v1.3) pkg> generate Programs
Generating project Programs:
    Programs/Project.toml
    Programs/src/Programs.jl

(v1.3) pkg> 
\end{lstlisting}

注意,我在\verb|generate|命令后面追加的参数是我们要创建的 Julia 项目的名称.随后,这个命令创建了一个名为\verb|Programs|的目录,并在该目录下生成了两个文件.一个是项目文件\verb|Project.toml|,另一个是\verb|src|目录(即源码目录)下的源码文件\verb|Programs.jl|.

我们先来看项目文件,它的内容如下:

\begin{lstlisting}% [language=markdown]
name = "Programs"
uuid = "e525bb1a-bb1e-11e9-07f5-1125a61c95e2"
authors = ["robert.hao <hypermind@outlook.com>"]
version = "0.1.0"
\end{lstlisting}

这里有 4 个条目,分别代表项目的名称、UUID、作者信息和初始版本号.其中的 UUID 是 Julia 的程序包管理器自动生成的.而项目作者信息是从当前操作系统中的 Git 配置信息复制过来的.

我们再来看源码文件\verb|Programs.jl|的内容:

\begin{lstlisting}[language=julia]
module Programs

greet() = print("Hello World!")

end # module
\end{lstlisting}

其中只定义了一个名为\verb|Programs|的模块.并且,该模块仅包含了一个可以向计算机的标准输出打印\verb|Hello World!|的函数\verb|greet|.这显然只是一个简单的程序模板.不过,它为我们后续的编码开了个头.

注意,这个源码文件是有重要意义的:

1. 该文件可以被称为\verb|Programs|项目的源码入口.或者说,它是这个项目的主源码文件.这是由于该文件的主文件名与项目的(主)名称是一致的.
2. 该文件中定义的(最外层的)模块\verb|Programs|将会是其所属项目的主模块(或者说默认模块).这是由于该模块的名称与项目的(主)名称是一致的.

正因为有了这样的一个源码文件,使得\verb|Programs|项目可以被 Julia 视为一个程序包.更明确地讲,如果存在一个名为\verb|X|或\verb|X.jl|的 Julia 项目,只要该项目包含一个相对路径为\verb|src/X.jl|的源码文件,并且在该文件中定义的最外层模块名为\verb|X|,那么它就是一个有效的程序包.

最后,一个可选的操作是,我们可以把这个项目的名称变更为\verb|Programs.jl|.如此可以让它更具 Julia 项目的特色.由前述内容可知,这样做并不会妨碍此项目成为一个有效的程序包.注意,项目\verb|Programs.jl|所代表的程序包的名称依然是\verb|Programs|,同时它的主模块的名称也依然是\verb|Programs|.

\textbf{2.3.2 程序包的引入}

既然\verb|Programs.jl|项目已经是一个有效的程序包了,那么我们就可以在代码中对它进行引入(更明确地说,是引入它的主模块\verb|Programs|).具体怎么做呢?

当我们试图在全局环境中导入该程序包的时候,Julia 会提示找不到这个程序包:

\begin{lstlisting}[language=julia]
julia> import Programs
ERROR: ArgumentError: Package Programs not found in current path:
- Run `import Pkg; Pkg.add("Programs")` to install the Programs package.

Stacktrace:
 [1] require(::Module, ::Symbol) at ./loading.jl:887

julia> 
\end{lstlisting}

为了解决这个问题,我们可以先在 REPL 环境下进入到\verb|Programs.jl|项目所在的目录,然后切换到 pkg 模式,并输入命令\verb|activate .|.注意,这里的输入是\verb|activate|加一个空格" ",再加一个英文点号"\verb|.|".示例如下:

\begin{lstlisting}[language=julia]
shell> cd ~/Projects/Programs.jl
/Users/haolin/Projects/Programs.jl

(v1.3) pkg> activate .

(Programs) pkg> 
\end{lstlisting}

我们可以看到,在使用\verb|activate|命令之后,REPL 环境的提示符再次改变了,变成了当前程序包的名称\verb|Programs|,也就是在当前目录下的\verb|Project.toml|文件中记录的那个名称.命令\verb|activate .|的作用正是把程序包管理器的操作目录切换到当前项目所在的目录,即:\verb|~/Projects/Programs.jl|.还记得吗?它原先的(或者说默认的)操作目录是\verb|~/.julia/environments/v1.3|,对应于 Julia 的\verb|v1.3|版本的全局环境.顺便说一下,如果你想切换回全局环境,那么只需要再次输入命令\verb|activate|(不加任何参数)就可以了.

在这之后,我们再在当前的 REPL 环境中导入\verb|Programs|就不会有问题了:

\begin{lstlisting}[language=julia]
julia> import Programs
[ Info: Precompiling Programs [e525bb1a-bb1e-11e9-07f5-1125a61c95e2]

julia> Programs.greet()
Hello World!
\end{lstlisting}

如果我们确实需要在全局环境中引入\verb|Programs|,那么可以先把这个项目上传到一个代码托管仓库(比如 GitHub)中,然后再使用 Julia 的程序包管理器把它安装到本地的仓库目录. 

比如,我们的这个\verb|Programs.jl|项目已经在 GitHub 上了,它的 git 地址是\verb|git@github.com:hyper0x/Programs.jl.git|.所以,我们现在就可以直接在 REPL 环境中进行如下操作:

\begin{lstlisting}[language=julia]
(Programs) pkg> activate

(v1.3) pkg> add git@github.com:hyper0x/Programs.jl.git
  Updating registry at `~/.julia/registries/General`
  Updating git-repo `https://github.com/JuliaRegistries/General.git`
   Cloning git-repo `git@github.com:hyper0x/Programs.jl.git`
  Updating git-repo `git@github.com:hyper0x/Programs.jl.git`
 Resolving package versions...
  Updating `~/.julia/environments/v1.3/Project.toml`
  [d2b7efac] + Programs v0.1.0 #master(git@github.com:hyper0x/Programs.jl.git)
  Updating `~/.julia/environments/v1.3/Manifest.toml`
  [d2b7efac] + Programs v0.1.0 #master(git@github.com:hyper0x/Programs.jl.git)

(v1.3) pkg> 
\end{lstlisting}

一旦\verb|Programs|程序包被记录在了全局环境的项目文件中,我们在该环境下引入它也就不会有问题了.