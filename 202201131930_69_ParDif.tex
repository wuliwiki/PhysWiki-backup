% 偏导数(微分学)
% keys 多元微积分|导数|偏导数|混合偏导|数学分析

\pentry{导数(数学分析)\upref{Der2},偏导数(简明微积分)\upref{ParDer}}
\subsection{从导数到偏导数}
导数的几何意义是一元函数在某一点处的斜率,而我们可以将这个概念推广到多元函数.$n$ 元实函数是指从 $\mathbb{R} ^n$ 的一个子集 $U$ 到 $\mathbb{R}$ 的映射:
\begin{equation}
\begin{aligned}
f:U\subset \mathbb{R} ^n&\rightarrow \mathbb{R}\\
(x_1,x_2,\cdots,x_n)&\mapsto f(x_1,x_2,\cdots,x_n)
\end{aligned}
\end{equation}
我们定义 $f$ 对 $x_i$ 的\textbf{偏导数}为
\begin{equation}
\lim\limits_{x'_i\rightarrow x_{i}}\frac{f(x_1,\cdots,x'_i,\cdots,x_n)-f(x_1,\cdots,x_i,\cdots,x_n)}{x'_i-x_i}
\end{equation}
如果该极限存在,那么函数在 $\bvec x_0=(x_1,\cdots,x_n)$ 处对 $x_i$ 的偏导数存在,记为
\begin{equation}
\frac{\partial f(x_1,\cdots,x_n)}{\partial x_i}=\left.\frac{\partial f}{\partial x_i}\right|_{(x_1,\cdots,x_n)}=f'_{x_i}(x_1,\cdots,x_n)
\end{equation}

在讨论多元函数时,我们约定用粗体字(例如 $\bvec x$)来表示 $\mathbb{R}^n$ 中的一个向量.对于给定的 $\bvec x=(x_1,\cdots,x_n)$,如果构造一元函数 $g(x)=f(x_1,\cdots,x_{i-1},x,x_{i+1},\cdots,x_n)$,那么根据一元函数导数的定义,容易发现
\begin{equation}
\left.\frac{\partial f}{\partial x_i}\right|_{\bvec x}=\left.\frac{\dd g}{\dd x}\right|_{x_i}
\end{equation}
因此我们可以使用一元函数的求导公式来进行偏导数的计算.固定其他 $x_j(j\neq i)$ 不动,对 $x_i$ 求导,得到的结果就是我们要求的偏导数.

如果多元函数 $f(\bvec x)$ 在开集 $U$ 上有定义,且在 $U$ 上每一点处都有对 $x_i$ 的偏导数,那么偏导数就是一个新的 $n$ 元函数:
\begin{equation}
\begin{aligned}
\frac{\partial f}{\partial x_i}: U&\rightarrow \mathbb{R}\\
\bvec x&\mapsto \frac{\partial f(\bvec x)}{\partial x_i}
\end{aligned}
\end{equation}
因此我们可以定义二阶偏导数(如果存在的话),甚至更高阶的偏导数.例如,如果再对 $\partial f/\partial x_i$ 求对 $x_j$ 的偏导数,则可以记为
\begin{equation}
\frac{\partial}{\partial x_j} \frac{\partial f(\bvec x)}{\partial x_i}=\left.\frac{\partial^2 f}{\partial x_j\partial x_i }\right|_{\bvec x}
\end{equation}

有时对于性质较好的函数 $f(\bvec x)$(例如 $f$ 是 $n$ 元连续函数),可以将它想象成欧几里德空间 $\mathbb{R}^{n+1}$ 中的一个曲面,曲面上的点 $(x_1,\cdots,x_n,x_{n+1})$ 意味着 $x_{n+1}=f(x_1,\cdots,x_n)$.那么偏导数的几何意义就是曲面在某一个方向上的斜率 $\dd x_{n+1}/\dd x_i$.$n$ 个偏导数 $\partial f/\partial x_1,\cdots,\partial f/\partial x_n$ 代表了曲面在 $n$ 个平行于坐标轴的方向上的斜率.

\subsection{方向导数}
偏导数的几何意义是在 $n$ 个平行于坐标轴的方向上的斜率,可以将它推广至任意方向上的斜率.