% 李代数(综述)
% license CCBYSA3
% type Wiki

本文根据 CC-BY-SA 协议转载翻译自维基百科\href{https://en.wikipedia.org/wiki/Lie_algebra}{相关文章}。

在数学中,李代数(发音为 /liː/,LEE)是一个向量空间 \( \mathfrak{g} \),配有一个名为李括号的运算,它是一个交替双线性映射 \(\mathfrak{g}\times \mathfrak{g}\to \mathfrak{g}\),满足雅可比恒等式。换句话说,李代数是一个定义在域上的代数,其中的乘法运算(称为李括号)是交替的,并且满足雅可比恒等式。两个向量 \(x\)和 \(y\) 的李括号记作 \(xy\)。李代数通常是一个非结合代数。然而,每个结合代数都可以生成一个李代数,该李代数由相同的向量空间构成,且使用交换子李括号,即 \([x,y]=xy-yx\)。

李代数与李群密切相关,李群是既是群又是光滑流形的群:每个李群都会生成一个李代数,该李代数是单位元处的切空间。(在这种情况下,李括号衡量了李群不满足交换律的程度。)反过来,任何定义在实数或复数上的有限维李代数,都有一个对应的连通李群,且唯一性仅限于覆盖空间(李的第三定理)。这种对应关系使得我们能够通过李代数这一线性代数的简化对象来研究李群的结构和分类。

更详细地说:对于任何李群,单位元1附近的乘法操作在一阶近似下是交换的。换句话说,每个李群 \( G \) 在一阶近似下大致是一个实向量空间,即 \( G \) 在单位元处的切空间 \( \mathfrak{g} \)。在二阶近似下,群操作可能是非交换的,描述 \( G \) 在单位元附近不交换性的二阶项赋予了 \( \mathfrak{g} \) 李代数的结构。一个显著的事实是,这些二阶项(李代数)完全决定了 \( G \) 在单位元附近的群结构。它们甚至决定了 \( G \) 的全局结构,直到覆盖空间为止。

在物理学中,李群作为物理系统的对称群出现,它们的李代数(单位元附近的切向量)可以被看作是无穷小的对称运动。因此,李代数及其表示在物理学中被广泛使用,特别是在量子力学和粒子物理学中。

一个基础的例子(并非直接来自一个结合代数)是三维空间 \(\mathfrak{g} = \mathbb{R}^3\),其李括号由叉积定义:\([x, y] = x \times y.\)这是反对称的,因为\(x \times y = -y \times x\),并且它不满足结合性,而是满足雅可比恒等式:
\[x \times (y \times z) + y \times (z \times x) + z \times (x \times y) = 0.~\]
这是空间旋转群的李代数,每个向量 \(v \in \mathbb{R}^3\) 可以被视为绕轴 \(v\) 的无穷小旋转,角速度等于 \(v\) 的大小。李括号是衡量两个旋转之间不交换性的工具。由于旋转与自身交换,因此具有交替性质:\([x, x] = x \times x = 0\).
\subsection{历史}  
李代数是由索福斯·李(Sophus Lie)在1870年代提出的,用于研究无穷小变换的概念,[1] 并在1880年代由威廉·基林(Wilhelm Killing)独立发现。[2] "李代数" 这一名称由赫尔曼·外尔(Hermann Weyl)在1930年代提出;在早期的文献中,常用“无穷小群”(infinitesimal group)这一术语。
\subsection{李代数的定义}  
李代数是一个在域 \( F \) 上的向量空间 \( \mathfrak{g} \),并且具有一个二元运算叫做李括号 \( [\cdot, \cdot]: \mathfrak{g} \times \mathfrak{g} \to \mathfrak{g} \),满足以下公理:

\begin{itemize}
\item 双线性:  
   \[
   [ax + by, z] = a[x, z] + b[y, z], \quad [z, ax + by] = a[z, x] + b[z, y]~
   \]
   对于所有标量 \( a, b \in F \) 和所有元素 \( x, y, z \in \mathfrak{g} \)。
\item 交替性质:
   \[
   [x, x] = 0 \quad \text{对于所有} \ x \in \mathfrak{g}~
   \]
\item 雅可比恒等式:
   \[
   [x, [y, z]] + [y, [z, x]] + [z, [x, y]] = 0~
   \]
\end{itemize}
   对于所有 \( x, y, z \in \mathfrak{g} \)。

   给定一个李群,其李代数的雅可比恒等式可以从群运算的结合性得到。

使用双线性展开李括号 \( [x + y, x + y] \) 并利用交替性质可以得到:\([x, y] + [y, x] = 0\)
对于所有 \( x, y \in \mathfrak{g} \)。因此,双线性和交替性质共同意味着:
\begin{itemize}
\item 反交换性:
   \[
   [x, y] = -[y, x]~
   \]
对于所有 \( x, y \in \mathfrak{g} \)。如果域的特征不为2,则反交换性意味着交替性质,因为它意味着:\([x, x] = -[x, x]\)
\end{itemize}
   
通常,李代数用小写的花体字母表示,例如 \( \mathfrak{g}, \mathfrak{h}, \mathfrak{b}, \mathfrak{n} \)。如果一个李代数与一个李群相关联,则该代数用群名称的花体字母表示:例如,SU(n) 的李代数为 \( \mathfrak{su}(n) \)。
\subsubsection{生成元与维度}  
李代数的维度指的是它作为一个向量空间的维度。在物理学中,李群 \( G \) 的李代数的一个向量空间基可以称为 \( G \) 的生成元集合(可以说它们是 \( G \) 的“无穷小生成元”)。在数学中,李代数 \( \mathfrak{g} \) 的一组生成元集合 \( S \) 指的是 \( \mathfrak{g} \) 的一个子集,使得任何包含 \( S \) 的李子代数(如下定义)必须是整个 \( \mathfrak{g} \)。等价地,\( \mathfrak{g} \) 是由 \( S \) 中所有元素的迭代括号生成的(作为一个向量空间)。
\subsection{基本例子}  
\subsubsection{阿贝尔李代数}  
任何赋予零李括号的向量空间 \( V \) 都是一个李代数。这样的李代数称为阿贝尔李代数。由于李括号的交替性,任何一维李代数都是阿贝尔李代数。
\subsubsection{李代数的矩阵}  
\begin{itemize}
\item 在一个代数体 \( A \) 上,如果定义乘法为 \( xy \),则可以通过交换子来定义李括号:  \([x, y] = xy - yx\)有了这个括号,\( A \) 就是一个李代数。(李代数的雅可比恒等式来源于 \( A \) 上乘法的结合性。)  
\item 一个带有上述李括号的 \( F \)-向量空间 \( V \) 的自同构环记作 \( \mathfrak{gl}(V) \)。  
\item 对于一个域 \( F \) 和正整数 \( n \),\( F \) 上的 \( n \times n \) 矩阵空间,记作 \( \mathfrak{gl}(n,F) \) 或 \( \mathfrak{gl}_n(F) \),它是一个李代数,括号由矩阵的交换子给出:\([X, Y] = XY - YX\)这是前面例子的一个特例;它是李代数的一个关键例子,称为一般线性李代数。  
当 \( F \) 为实数时,\( \mathfrak{gl}(n, \mathbb{R}) \) 是一般线性群 \( \mathrm{GL}(n, \mathbb{R}) \) 的李代数,\( \mathrm{GL}(n, \mathbb{R}) \) 是可逆的 \( n \times n \) 实矩阵(或等价地,行列式非零的矩阵)组成的群,其中群运算是矩阵乘法。类似地,\( \mathfrak{gl}(n, \mathbb{C}) \) 是复李群 \( \mathrm{GL}(n, \mathbb{C}) \) 的李代数。  
\( \mathfrak{gl}(n, \mathbb{R}) \) 上的李括号描述了矩阵乘法的不交换性,或等价地,线性映射组合的不交换性。对于任意域 \( F \),\( \mathfrak{gl}(n, F) \) 可以看作代数群 \( \mathrm{GL}(n) \) 在 \( F \) 上的李代数。
\end{itemize}
\subsection{定义}  
\subsubsection{子代数、理想和同态} 
李括号不要求具有结合性,这意味着\([[x,y],z] \neq [x,[y,z]]\)然而,许多与结合代数环(和群)相关的术语,对于李代数也有类似的概念。一个李子代数是一个线性子空间 \( \mathfrak{h} \subseteq \mathfrak{g} \),它对李括号是封闭的。一个理想 \( \mathfrak{i} \subseteq \mathfrak{g} \) 是一个线性子空间,满足更强的条件:  
\[
[\mathfrak{g}, \mathfrak{i}] \subseteq \mathfrak{i}~
\]  
在李群与李代数之间的对应关系中,子群对应于李子代数,正规子群对应于理想。  

一个李代数同态是一个与相应李括号兼容的线性映射:  
\[
\phi \colon \mathfrak{g} \to \mathfrak{h}, \quad \phi([x,y]) = [\phi(x), \phi(y)] \quad \text{对于所有} \, x, y \in \mathfrak{g}.~
\]  
一个李代数的同构是一个双射同态。

与群中的正规子群类似,李代数中的理想恰好是同态的核。给定一个李代数 \( \mathfrak{g} \) 和它的一个理想 \( \mathfrak{i} \),商李代数 \( \mathfrak{g}/\mathfrak{i} \) 被定义,并且存在一个李代数的满同态映射 \( \mathfrak{g} \to \mathfrak{g}/\mathfrak{i} \)。李代数的第一同构定理成立:对于任何李代数同态 \( \phi \colon \mathfrak{g} \to \mathfrak{h} \),\( \phi \) 的像是 \( \mathfrak{h} \) 中一个李子代数,并且与 \( \mathfrak{g}/\text{ker}(\phi) \) 同构。

对于李群的李代数,李括号是一种无穷小的交换子。因此,对于任何李代数,两个元素 \( x, y \in \mathfrak{g} \) 被称为交换的,如果它们的李括号为零:\([x,y] = 0\)

一个子集 \( S \subset \mathfrak{g} \) 的中心化子代数是与 \( S \) 中所有元素交换的元素集合,即\(\mathfrak{z}_{\mathfrak{g}}(S) = \{ x \in \mathfrak{g} : [x,s] = 0 \ \text{对所有} \ s \in S \}\)李代数 \( \mathfrak{g} \) 本身的中心是其中心 \( \mathfrak{z}(\mathfrak{g}) \)。类似地,对于一个子空间 \( S \),\( S \) 的正规化子代数是\(\mathfrak{n}_{\mathfrak{g}}(S) = \{ x \in \mathfrak{g} : [x,s] \in S \ \text{对所有} \ s \in S \}\)如果 \( S \) 是一个李子代数,\( \mathfrak{n}_{\mathfrak{g}}(S) \) 是最大的子代数,使得 \( S \) 是 \( \mathfrak{n}_{\mathfrak{g}}(S) \) 的理想。
\subsubsection{例子}  
李代数 \( \mathfrak{t}_n \) 中的对角矩阵子空间是 \( \mathfrak{gl}(n,F) \) 的一个交换李子代数。(它是 \( \mathfrak{gl}(n) \) 的一个卡尔坦子代数,类似于紧李群理论中的最大托罗斯。)在这里,当 \( n \geq 2 \) 时, \( \mathfrak{t}_n \) 不是 \( \mathfrak{gl}(n) \) 的理想。例如,当 \( n = 2 \) 时,这可以通过以下计算得出:
\[
\left[ \begin{bmatrix} a & b \\ c & d \end{bmatrix}, \begin{bmatrix} x & 0 \\ 0 & y \end{bmatrix} \right] = \begin{bmatrix} ax & by \\ cx & dy \end{bmatrix} - \begin{bmatrix} ax & bx \\ cy & dy \end{bmatrix} = \begin{bmatrix} 0 & b(y - x) \\ c(x - y) & 0 \end{bmatrix}~
\]
(这不一定总是在 \( \mathfrak{t}_2 \) 中)。  

每个李代数 \( \mathfrak{g} \) 的一维线性子空间都是一个交换李子代数,但它不一定是理想。
\subsubsection{积和半直积}  
对于两个李代数 \( \mathfrak{g} \) 和 \( \mathfrak{g'} \),积李代数是向量空间 \( \mathfrak{g} \times \mathfrak{g'} \),由所有有序对 \( (x, x') \),其中 \( x \in \mathfrak{g} \) 且 \( x' \in \mathfrak{g'} \) 组成,李括号定义为:
\[
[(x,x'),(y,y')] = ([x,y], [x',y']).~
\]
这是李代数范畴中的积。注意,\( \mathfrak{g} \) 和 \( \mathfrak{g'} \) 在 \( \mathfrak{g} \times \mathfrak{g'} \) 中的副本彼此交换:\([(x,0),(0,x')] =0\)

设 \( \mathfrak{g} \) 是一个李代数,\( \mathfrak{i} \) 是 \( \mathfrak{g} \) 的一个理想。如果标准映射 \( \mathfrak{g} \to \mathfrak{g}/\mathfrak{i} \) 是可分裂的(即存在一个李代数同态的截面 \( \mathfrak{g}/\mathfrak{i} \to \mathfrak{g} \)),则称 \( \mathfrak{g} \) 是 \( \mathfrak{i} \) 和 \( \mathfrak{g}/\mathfrak{i} \) 的半直积,记作:\(\mathfrak{g} = \mathfrak{g}/\mathfrak{i} \ltimes \mathfrak{i}.\)
另见李代数的半直和。
\subsubsection{导数算子}  
对于一个在域 \( F \) 上的代数 \( A \),\( A \) 关于 \( F \) 的导数算子是一个线性映射 \( D \colon A \to A \),它满足莱布尼茨法则:
\[
D(xy) = D(x)y + xD(y)~
\]
对于所有 \( x, y \in A \)。这个定义对于可能是非结合的代数也是有意义的。给定两个导数算子 \( D_1 \) 和 \( D_2 \),它们的对易子定义为:\([D_1, D_2] := D_1D_2 - D_2D_1\)这个对易子仍然是一个导数算子。这个操作使得 \( A \) 上的所有导数算子构成的空间 \( \text{Der}_k(A) \) 成为一个李代数。

非正式地讲,代数 \( A \) 的导数算子空间是 \( A \) 自同构群的李代数。(当自同构群是李群时,这在字面上是成立的,例如当 \( F \) 是实数且 \( A \) 作为向量空间的维数有限时。)因此,导数算子空间是构造李代数的自然方式:它们是 \( A \) 的“无穷小自同构”。事实上,写出如下条件:
\[
(1 + \epsilon D)(xy) \equiv (1 + \epsilon D)(x) \cdot (1 + \epsilon D)(y) \pmod{\epsilon^2}~
\]
(其中 1 表示 \( A \) 上的恒等映射)正好给出了 \( D \) 是导数算子的定义。

\textbf{例子:向量场的李代数}设 \( A \) 是光滑流形 \( X \) 上的光滑函数环 \( C^{\infty}(X) \)。那么,\( A \) 关于 \( \mathbb{R} \) 的导数算子等价于 \( X \) 上的向量场。(一个向量场 \( v \) 通过在 \( v \) 方向上对函数进行微分,给出光滑函数空间的导数算子。)这使得向量场空间 \( \text{Vect}(X) \) 成为一个李代数(见向量场的李括号)。非正式地讲,\( \text{Vect}(X) \) 是 \( X \) 上微分同胚群的李代数。因此,向量场的李括号描述了微分同胚群的非交换性。一个李群 \( G \) 在流形 \( X \) 上的作用决定了一个李代数的同态:\(\mathfrak{g} \to \text{Vect}(X)\)
(下面将举例说明。)

一个李代数可以看作是一个非结合代数,因此,每个在域 \( F \) 上的李代数 \( \mathfrak{g} \) 决定了它的导数李代数 \( \text{Der}_F(\mathfrak{g}) \)。也就是说,\( \mathfrak{g} \) 的一个导数算子是一个线性映射:\(D \colon \mathfrak{g} \to \mathfrak{g}\)满足:
\[
D([x,y]) = [D(x),y] + [x,D(y)]~
\]
对于所有 \( x, y \in \mathfrak{g} \)。与任意 \( x \in \mathfrak{g} \) 相关的内导数算子是伴随映射 \( \mathrm{ad}_x \),定义为:\(\mathrm{ad}_x(y) := [x, y]\)(这是导数算子,因为它是雅可比恒等式的结果。)这给出了一个李代数的同态:\(\mathrm{ad} \colon \mathfrak{g} \to \text{Der}_F(\mathfrak{g})\)
其像 \( \text{Inn}_F(\mathfrak{g}) \) 是 \( \text{Der}_F(\mathfrak{g}) \) 中的一个理想,而外导数李代数定义为商李代数:\(\text{Out}_F(\mathfrak{g}) = \text{Der}_F(\mathfrak{g}) / \text{Inn}_F(\mathfrak{g})\)(这与群的外自同构群完全类似。)对于一个特征为零的半单李代数(见下文定义),每个导数算子都是内导数算子。这与半单李群的外自同构群是有限的定理有关。

相比之下,阿贝尔李代数有许多外导数算子。即,对于一个李括号为零的向量空间 \( V \),李代数 \( \text{Out}_F(V) \) 可以与 \( \mathfrak{gl}(V) \) 同构。
\subsection{例子}
\subsubsection{矩阵李代数}
一个矩阵群是一个包含可逆矩阵的李群,记作 \( G \subset \mathrm{GL}(n, \mathbb{R}) \),其中 \( G \) 的群运算是矩阵乘法。相应的李代数 \( \mathfrak{g} \) 是与 \( G \) 相切的矩阵空间,位于线性空间 \( M_n(\mathbb{R}) \) 中:它由李群 \( G \) 中平滑曲线在单位矩阵 \( I \) 处的导数组成:
\[
\mathfrak{g} = \{ X = c'(0) \in M_n(\mathbb{R}) : \text{平滑曲线 } c : \mathbb{R} \to G, \ c(0) = I \}.~
\]
李代数 \( \mathfrak{g} \) 的李括号由矩阵的交换子给出:\([X, Y] = XY - YX\).
给定李代数 \( \mathfrak{g} \subset \mathfrak{gl}(n, \mathbb{R}) \),可以通过矩阵指数运算从 \( \mathfrak{g} \) 中的元素恢复李群。具体地说,若 \( G \) 是连通的,那么这个过程给出了 \( G \) 的单位成分。这里的指数映射\(\exp : M_n(\mathbb{R}) \to M_n(\mathbb{R})\)定义为\(\exp(X) = I + X + \frac{1}{2!}X^2 + \frac{1}{3!}X^3 + \cdots\),这个级数对每个矩阵 \( X \) 都是收敛的。

相同的评论也适用于复数李子群 \( \mathrm{GL}(n, \mathbb{C}) \) 和复矩阵指数:\(\exp : M_n(\mathbb{C}) \to M_n(\mathbb{C})\),其定义与实数情况相同。

这里是一些矩阵李群及其李代数的例子。[14]
\begin{itemize}
\item 对于一个正整数 \( n \),特殊线性群 \( \mathrm{SL}(n, \mathbb{R}) \) 包含所有行列式为 1 的实 \( n \times n \) 矩阵。这个群是从 \( \mathbb{R}^n \) 到自身的线性映射群,它们保持体积和方向性。更抽象地,\( \mathrm{SL}(n, \mathbb{R}) \) 是一般线性群 \( \mathrm{GL}(n, \mathbb{R}) \) 的交换子群。它的李代数 \( \mathfrak{sl}(n, \mathbb{R}) \) 包含所有迹为 0 的实 \( n \times n \) 矩阵。类似地,可以定义对应的复数李群 \( \mathrm{SL}(n, \mathbb{C}) \) 及其李代数 \( \mathfrak{sl}(n, \mathbb{C}) \)。
\item 正交群 \( \mathrm{O}(n) \) 在几何中起着基本作用:它是从 \( \mathbb{R}^n \) 到自身的线性映射群,这些映射保持向量的长度。例如,旋转和反射属于 \( \mathrm{O}(n) \)。等价地,这是所有 \( n \times n \) 的正交矩阵的群,意味着 \( A^{\mathrm{T}} = A^{-1} \),其中 \( A^{\mathrm{T}} \) 表示矩阵的转置。正交群有两个连通分支;其中单位分支叫做特殊正交群 \( \mathrm{SO}(n) \),它由行列式为 1 的正交矩阵组成。两个群具有相同的李代数 \( \mathfrak{so}(n) \),即在 \( \mathfrak{gl}(n, \mathbb{R}) \) 中的反对称矩阵子空间(即 \( X^{\mathrm{T}} = -X \))。参见反对称矩阵的微分旋转。
复数正交群 \( \mathrm{O}(n, \mathbb{C}) \)、它的单位分支 \( \mathrm{SO}(n, \mathbb{C}) \),以及李代数 \( \mathfrak{so}(n, \mathbb{C}) \) 由应用于 \( n \times n \) 复矩阵的相同公式给出。等价地,\( \mathrm{O}(n, \mathbb{C}) \) 是 \( \mathrm{GL}(n, \mathbb{C}) \) 的一个子群,保持 \( \mathbb{C}^n \) 上的标准对称双线性形式。
\item 酉群 \( \mathrm{U}(n) \) 是复数线性群 \( \mathrm{GL}(n, \mathbb{C}) \) 的一个子群,保持 \( \mathbb{C}^n \) 中向量的长度(相对于标准的厄米内积)。等价地,这是所有 \( n \times n \) 酉矩阵的群(满足 \( A^{*} = A^{-1} \),其中 \( A^{*} \) 表示矩阵的共轭转置)。它的李代数 \( \mathfrak{u}(n) \) 由 \( \mathfrak{gl}(n, \mathbb{C}) \) 中的反厄米矩阵组成(即 \( X^{*} = -X \))。这是一个实数上的李代数,而不是复数上的李代数。(事实上,反厄米矩阵乘以虚数单位 \( i \) 得到的是厄米矩阵,而不是反厄米矩阵。)同样,酉群 \( \mathrm{U}(n) \) 是复数李群 \( \mathrm{GL}(n, \mathbb{C}) \) 的一个实李子群。例如,\( \mathrm{U}(1) \) 是圆群,它的李代数(从这个角度来看)是 \( i\mathbb{R} \subset \mathbb{C} = \mathfrak{gl}(1, \mathbb{C}) \)。
\item 特殊酉群 \( \mathrm{SU}(n) \) 是酉群 \( \mathrm{U}(n) \) 中行列式为 1 的子群。它的李代数 \( \mathfrak{su}(n) \) 由行列式为零的反厄米矩阵组成。
\item 辛群 \( \mathrm{Sp}(2n, \mathbb{R}) \) 是复数线性群 \( \mathrm{GL}(2n, \mathbb{R}) \) 的子群,它保持 \( \mathbb{R}^{2n} \) 上的标准交替双线性型。它的李代数是辛李代数 \( \mathfrak{sp}(2n, \mathbb{R}) \)。
\item 经典李代数包括上述李代数及其在任意域上的变体。
\end{itemize}
\subsubsection{二维}  
这里描述了一些低维李代数的例子。有关更多示例,请参见低维实李代数的分类。

\begin{itemize}
\item 对于任何域 \( F \),存在唯一的二维非阿贝尔李代数 \( \mathfrak{g} \),同构不唯一。[15] 其中,李代数 \( \mathfrak{g} \) 有一个基 \( X, Y \),其李括号由以下关系给出:\([X, Y] = Y\).

(这完全确定了李括号,因为公理意味着 \( [X, X] = 0 \) 和 \( [Y, Y] = 0 \)。)在实数域上,李代数 \( \mathfrak{g} \) 可以视为李群 \( G = \mathrm{Aff}(1, \mathbb{R}) \) 的李代数,后者是实数线的仿射变换群,变换形式为 \( x \mapsto ax + b \)。

仿射群 \( G \) 可以与矩阵群
\[
\left( \begin{array}{cc} a & b \\ 0 & 1 \end{array} \right)~
\]
在矩阵乘法下同构,其中 \( a, b \in \mathbb{R} \),且 \( a \neq 0 \)。它的李代数是 \( \mathfrak{g} \),即 \( \mathfrak{gl}(2, \mathbb{R}) \) 中的子代数,由所有矩阵
\[
\left( \begin{array}{cc} c & d \\ 0 & 0 \end{array} \right)~
\]
组成。在这些术语中,李代数 \( \mathfrak{g} \) 的基是由矩阵
\[
X = \left( \begin{array}{cc} 1 & 0 \\ 0 & 0 \end{array} \right), \quad Y = \left( \begin{array}{cc} 0 & 1 \\ 0 & 0 \end{array} \right)~
\]
给出的。

对于任何域 \( F \),一维子空间 \( F \cdot Y \) 是二维李代数 \( \mathfrak{g} \) 的一个理想,满足以下公式:\([X, Y] = Y \in F \cdot Y\).

李代数 \( F \cdot Y \) 和 \( \mathfrak{g} / (F \cdot Y) \) 都是阿贝尔的(因为是一维的)。从这个角度来看,李代数 \( \mathfrak{g} \) 可以分解为阿贝尔的“部分”,这意味着它是可解的(但不是幂零的),用下面的术语来说。
\end{itemize}
\subsubsection{三维}  
\begin{itemize}
\item 海森堡代数 \( \mathfrak{h}_3(F) \) 是定义在域 \( F \) 上的三维李代数,具有基 \( X, Y, Z \),其李括号满足以下关系:[16]
\[
[X, Y] = Z, \quad [X, Z] = 0, \quad [Y, Z] = 0.~
\]
它可以被看作是3×3严格上三角矩阵的李代数,李括号为交换子,基为:
\[
X = \left( \begin{array}{ccc} 0 & 1 & 0 \\ 0 & 0 & 0 \\ 0 & 0 & 0 \end{array} \right), \quad Y = \left( \begin{array}{ccc} 0 & 0 & 0 \\ 0 & 0 & 1 \\ 0 & 0 & 0 \end{array} \right), \quad Z = \left( \begin{array}{ccc} 0 & 0 & 1 \\ 0 & 0 & 0 \\ 0 & 0 & 0 \end{array} \right).~
\]
在实数域 \( \mathbb{R} \) 上,\( \mathfrak{h}_3(\mathbb{R}) \) 是海森堡群 \( H_3(\mathbb{R}) \) 的李代数,也就是说,它是矩阵群
\[
\left( \begin{array}{ccc} 1 & a & c \\ 0 & 1 & b \\ 0 & 0 & 1 \end{array} \right)~
\]
在矩阵乘法下的李代数。

对于任何域 \( F \),李代数 \( \mathfrak{h}_3(F) \) 的中心是一维理想 \( F \cdot Z \),而商代数 \( \mathfrak{h}_3(F) / (F \cdot Z) \) 是阿贝尔的,同构于 \( F^2 \)。根据下面的术语,可以得出 \( \mathfrak{h}_3(F) \) 是幂零的(但不是阿贝尔的)。
\item 李代数 \( \mathfrak{so}(3) \) 是旋转群 \( \mathrm{SO}(3) \) 的李代数,它是实数域 \( \mathbb{R} \) 上的反对称 3×3 矩阵空间。其基由以下三个矩阵给出:[17]
\[
F_1 = \left( \begin{array}{ccc} 0 & 0 & 0 \\ 0 & 0 & -1 \\ 0 & 1 & 0 \end{array} \right), \quad F_2 = \left( \begin{array}{ccc} 0 & 0 & 1 \\ 0 & 0 & 0 \\ -1 & 0 & 0 \end{array} \right), \quad F_3 = \left( \begin{array}{ccc} 0 & -1 & 0 \\ 1 & 0 & 0 \\ 0 & 0 & 0 \end{array} \right).~
\]
这些生成元之间的交换关系为:
\[
[F_1, F_2] = F_3, \quad [F_2, F_3] = F_1, \quad [F_3, F_1] = F_2.~
\]
在 \( \mathbb{R}^3 \) 中,向量的叉乘可以通过标准基表示同样的公式来给出;因此,这个李代数同构于 \( \mathfrak{so}(3) \)。此外,李代数 \( \mathfrak{so}(3) \) 等价于量子力学中自旋-1 粒子的自旋角动量分量算符。[18]

李代数 \( \mathfrak{so}(3) \) 无法像前面的例子那样被分解成几个部分:它是简单的,这意味着它不是阿贝尔的,并且它唯一的理想是零和整个 \( \mathfrak{so}(3) \) 本身。
\item 另一个简单的李代数,维数为 3,这里是复数域 \( \mathbb{C} \) 上的李代数 \( \mathfrak{sl}(2, \mathbb{C}) \),它是所有迹为零的 2×2 矩阵的空间。其基由以下三个矩阵给出:
\[
H = \left( \begin{array}{cc} 1 & 0 \\ 0 & -1 \end{array} \right), \quad E = \left( \begin{array}{cc} 0 & 1 \\ 0 & 0 \end{array} \right), \quad F = \left( \begin{array}{cc} 0 & 0 \\ 1 & 0 \end{array} \right).~
\]
\end{itemize}