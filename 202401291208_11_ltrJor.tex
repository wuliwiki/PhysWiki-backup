% 一般线性变换的 Jordan(若尔当)标准形
% license Usr
% type Tutor


\begin{issues}
\issueDraft
\issueTODO Jordan块阶数和个数与线性变换的联系
\end{issues}
利用幂零变换的结论(\autoref{the_Jordan_1}~\upref{Jordan}),本节进一步证明,任意线性变换都能化为类似的Jordan形,并且Jordan形是唯一的。
\subsection{线性变换的Jordan形}
由\autoref{the_nullpl_1}~\upref{nullpl}可知,任意线性空间都可以分解为任意线性变换的不变子空间之直和,例如对于线性变换$A$,我们有:
\begin{equation}
V=\opn{ker}f(A)=\opn{ker}f_1(A)\oplus\opn{ker}f_2(A)...\oplus\opn{ker}f_m(A)~,
\end{equation}
其中$f(A)$是$A$的特征多项式,$f_i(A)$是特征值相关的互素项:$f_i(A)=(A-\lambda_i)^{k_i}$。为方便计,设$\opn{ker}f_i(A)=W_i,f_i(A)=B_i$,则$A=A|_{W_1}\oplus A|_{W_2}\oplus...\oplus A|_{W_n}=\bigoplus^n_{i=1}(B_i+\lambda_iI)$

由\autoref{the_Jordan_1}~\upref{Jordan}可知,幂零变换意味着在$W_i$的某组基上,$B_i$都是Jordan形矩阵,称对应的$A|_{W_i}$为一般线性变换的Jordan形。归纳此段讨论为下述定理:
\begin{theorem}{}
设$A$为域$\mathbb F$上$n$维线性空间$V$内的一个线性变换,其特征多项式的所有根$\lambda\in\mathbb F$,则在$V$内存在一组基,使线性变换$A$在这组基下可以表示为如下块对角矩阵:
\begin{equation}
J=\left(\begin{array}{cccc}
J_1 & & & 0 \\
& J_2 & & \\
& & \ddots & \\
0 & & & J_s
\end{array}\right), \quad J_i=\left(\begin{array}{cccc}
\lambda_i & 1 & & 0 \\
& \lambda_i & \ddots & \\
& & \ddots & 1 \\
0 & & & \lambda_i
\end{array}\right)~.
\end{equation}
\end{theorem}
$A|_{W_i}$由若干个Jordan块直和而成,每块对应一个特征向量,不同块的特征数量线性无关,因此块数就是几何重数。实数域上不能保证根都在域内,因而线性变换未必有Jordan形,而复数域上的线性变换总能表示为该形式。
\subsubsection{另一种证明思路}
对于具有特征值$\lambda$的线性变换$A$而言,我们有如下基本结论:
\begin{equation}
\begin{aligned}
&\opn{ker}B\subseteq\opn{ker}B^2\subseteq...\subseteq\opn{ker}B^k\\
&\opn{Im}B\supseteq\opn{Im}B^2\supseteq...\supseteq\opn{Im}B^k~,
\end{aligned}
\end{equation}
其中$B=A-\lambda A$。在这两个序列中,一旦有两个子空间相等,后续序列中的子空间都相等。
\begin{lemma}{}
设$\opn{ker}B^k=\opn{ker}B^{k+1}$,对于任意$m>k$,有\begin{equation}
\opn{ker}B^{k+m}=\opn{ker}B^{k++m+1},\quad\opn{Im}B^{k+m}=\opn{Im}B^{k+m+1}~.
\end{equation}
\end{lemma}

\subsection{Jordan形的唯一性}

\subsection{Jordan矩阵的计算方法}