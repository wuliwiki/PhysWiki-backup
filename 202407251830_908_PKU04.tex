% 北京大学 2004 年 考研 量子力学
% license Usr
% type Note

\textbf{声明}:“该内容来源于网络公开资料,不保证真实性,如有侵权请联系管理员”



\subsection{(15分)}

(a) 解释态叠加原理,给出性原理和态的统计解释。

(b) 写出由薛定谔方程一阶近似得到的二级能量修正公式。

(c) 在中心立场中,径向波函数 $R_{10}(r)$, $R_{20}(r)$, $R_{12}(r)$ 各有几个零点。

(d) 什么叫定态?有哪些量的特征态的线性组合加的态是否是定态?

(e) 简述并解释费曼规则。

(f) 解释(正常)塞曼效应及其多普勒效应。

\subsection{10分}

(a) 氢原子和谐振子的某套态波函数集是否是完备的?

(b) 在外电磁场的下,求电子在其中的哈密顿量。

(c) 两个自旋为$\frac{1}{2}$的全同粒子处于一维无限深势阱中,试求两粒子处于基态的总自旋波函数。

(d) $\hat{\sigma}_x = \hat{i} \hat{\sigma}_x, \hat{\sigma}_z = \hat{i} \hat{\sigma}_z$ 的定义。

(e) $\hat{L} = \hat{L}_x + i\hat{L}_y$, 求 ${\hat{L}_+, \hat{L}_-, \hat{L}_x, \hat{L}_y, \hat{L}_z, \hat{L} }$。

(f) 在中心立场中,基态的轨道角动量为何值?并做简要解释。


\subsection{16分}
在 $\langle S^2 \rangle$ 表象中,

(a) 求 $\langle S_z \rangle$ 的共同本征态及其对应的本征值。

(b) $\hat{S}^2, \hat{S}_z$ 在中所有态的平均值。

\subsection{11分}
已知薛定谔定常方程:$i\hbar \frac{\partial \psi}{\partial t} = \left[ -\frac{\hbar^2}{2m} \nabla^2 + V(\mathbf{r}) \right] \Psi$,试求动量表象中的薛定谔方程。

\subsection{16分}
已知两个电子均处于单量子态 $\mathbf{a}, \mathbf{b}$ 的方空间任意态 $\mathbf{a} \cdot \mathbf{b}$,求在上述单态中的平均值。

\subsection{11分}
已知 $\hat{V}(\mathbf{x}) = -\frac{e^2}{4 \pi \epsilon_0} \frac{1}{|\mathbf{x} - \mathbf{x}_0|}$, 且有 $\mathbf{x} \rightarrow \infty$, 试求势能 $V(x)$ 的具体表示。

\subsection{11分}
已知介质为均匀的材料,质量为 $m$ 的全同粒子处于一个半径为 $R$ 的球体内的一个圆周上,并且处在这个圆周上的速度为 $v$,试求介质中的动能转动动能的转化为动能。

