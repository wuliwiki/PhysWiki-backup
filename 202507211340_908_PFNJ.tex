% 帕夫努季·切比雪夫(综述)
% license CCBYSA3
% type Wiki

本文根据 CC-BY-SA 协议转载翻译自维基百科\href{https://en.wikipedia.org/wiki/Pafnuty_Chebyshev}{相关文章}。

\begin{figure}[ht]
\centering
\includegraphics[width=6cm]{./figures/97a93cba399441cc.png}
\caption{} \label{fig_PFNJ_1}
\end{figure}
帕夫努季·利沃维奇·切比雪夫(俄语:Пафну́тий Льво́вич Чебышёв,发音:[pɐfˈnutʲɪj ˈlʲvovʲɪtɕ tɕɪbɨˈʂof],1821年5月16日[俄历5月4日]—1894年12月8日[俄历11月26日])是一位俄罗斯数学家,被认为是俄罗斯数学的奠基人。

切比雪夫以其在概率论、统计学、力学以及数论领域的基础性贡献而著称。许多重要的数学概念以他的名字命名,包括切比雪夫不等式(可用于证明大数定律的弱形式)、伯特兰-切比雪夫定理、切比雪夫多项式、切比雪夫连杆机构以及切比雪夫偏差。
