% 狄拉克方程的非相对论近似
% keys 狄拉克方程|泡利方程

\pentry{狄拉克方程\upref{qed4}}

狄拉克方程\upref{qed4}是描述相对论性自由电子的方程\autoref{qed4_eq9}~\upref{qed4}:(这里我们没有采用自然但位置,所以需要带上 $\hbar,c$)
\begin{equation}
\begin{aligned}
&i\hbar \pdv{t} \psi = H\psi\\
&H=c\bvec \alpha\cdot \bvec p+mc^2\beta
\end{aligned}
\end{equation}
其中 $\bvec p=-i\hbar\nabla$;$\bvec \alpha,\beta$ 是四维矩阵代数中的元素,满足一定的反对易关系.更常见地,上式也可以写成\autoref{qed4_eq22}~\upref{qed4} 的形式:
\begin{equation}
i\qty(\gamma^\mu \partial_\mu-\frac{mc}{\hbar})\psi(x)=0
\end{equation}
其中 $\partial_\mu=\qty(\pdv{(ct)},\pdv{x},\pdv{y},\pdv{z})$.

\subsection{自由电子狄拉克方程的非相对论近似}

下面我们将从狄拉克方程出发,得到它的非相对论近似.设平面波解
\begin{equation}
\psi=\pmat{\varphi\\\chi}\exp(-imc^2t/\hbar)
\end{equation}
将它代入狄拉克方程.这里不妨采用 $\bvec \alpha,\beta$ 的标准表示\autoref{qed4_eq6}~\upref{qed4},则有
\begin{equation}\label{DiracB_eq1}
\begin{aligned}
&i\hbar\pdv{t} \varphi=c\bvec \sigma\cdot \bvec p \chi\\
&i\hbar\pdv{t} \chi = c\bvec \sigma\cdot \bvec p \varphi -2mc^2 \chi 
\end{aligned}
\end{equation}
注意上面的第二行式子中,由于在非相对论极限下 $\pdv{}{t} \chi$ 相比等式右边带 $c$ 的分量可以略去,所以有近似等式
\begin{equation}
\chi\approx \frac{1}{2mc} \bvec \sigma\cdot \bvec p \varphi
\end{equation}
将它代入\autoref{DiracB_eq1} 的第一行,可以得到
\begin{equation}
i\hbar\pdv{t} \varphi=\frac{1}{2m}(\bvec \sigma\cdot \bvec p)^2 \varphi
\end{equation}
可以将此式与\autoref{scheq2_eq3}~\upref{scheq2} 进行对比,可以发现两种方式推导出的非相对论性自旋 1/2 粒子的波函数是一致的.
