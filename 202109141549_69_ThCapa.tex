% 热容量
% 热容|比热容|等压热容|等体热容|理想气体

\begin{issues}
\issueTODO
\end{issues}
\pentry{热力学第一定律\upref{Th1Law}}
\subsection{热容}
一个系统在一定条件下的\textbf{热容量(heat capacity)}定义为\footnote{这个定义可以类比电容量\upref{Cpctor}}
\begin{equation}
C = \lim\limits_{\Delta T\rightarrow 0}\frac{\Delta Q}{\Delta T}
\end{equation}
热容可能跟温度压强等有关.

定义\textbf{比热容(specific heat capacity)}为热容除以质量
\begin{equation}
c = \frac{C}{m}
\end{equation}

定义摩尔热容为 $1 \rm{mol}$ 物质的热容
\begin{equation}
C_m=\frac{C}{n}
\end{equation}

\begin{example}{}
两份水初始温度分别为 $300\rm{K}$ 和 $360\rm{K}$,体积分别为 $1\rm{L}$ 或 $2 \rm{L}$.将它们放入绝热容器种混合均匀,求末温度.(注:水的比热容 $c$ 随温度的变化不大,可以近似看成一个常数)

设末温度为 $T$,那么第一份水吸收的热量为 $c m_1(T-300\rm{K})$,第二份水放出的热量为 $cm_2(360{\rm{K}}-T)$.由于在绝热容器中混合,且 $m_2=2m_1$,可以解得 $T=340\rm{K}$
\end{example}

我们可以定义等体热容为系统在等体过程\upref{EqVol}中的热容.根据热力学第一定律\upref{Th1Law} ,$\dd U=\delta Q-p\dd V$,可知等体过程中 $\delta Q=\dd U$(这里定义了态函数\textbf{焓} $H=U+pV$),所以
\begin{equation}\label{ThCapa_eq1}
C_V=\left(\frac{\partial U}{\partial T}\right)_V
\end{equation}

类似地,定义等压热容为系统在等压过程\upref{EqPre}中的热容.在等压过程中 $\delta Q=\dd U+p\dd V=\dd (U+pV)=\dd H$,所以
\begin{equation}
C_p=\left(\frac{\partial H}{\partial T}\right)_p
\end{equation}

类似可以定义摩尔等体热容 $C_{V,m}=C_V/n$,摩尔等压热容 $C_{p,m}=C_P/n$.

\subsection{理想气体的等压热热容与等体热容}

根据理想气体的状态方程\upref{PVnRT},对于一定物质的量($n \rm{mol}$)的理想气体,内能 $U$ 只和温度有关,所以 \autoref{ThCapa_eq1} 的偏导数可以写为导数,即
\begin{equation}
C_V=\frac{\dd U}{\dd T}
\end{equation}

于是理想气体内能也可以写成积分表达式
\begin{equation}
U=\int C_V\dd T + U_0
\end{equation}

我们还可以求得 $C_p$ 和 $C_V$ 的关系:
\begin{align}
&C_p-C_V=\frac{\dd (U+pV)}{\dd T}-\frac{\dd U}{\dd T}=\frac{\dd(pV)}{\dd T}=\frac{\dd (nRT)}{\dd T}=nR\\
&C_{p,m}-C_{V,m}=R
\end{align}

$U$ 是 $T$ 的函数,因此 $C_V$ 和 $C_p$ 都是 $T$ 的函数.现在,用 $\gamma$ 表示 $C_p/C_V$,$\gamma$ 也是 $T$ 的函数.那么有

\begin{equation}
C_V=\frac{nR}{\gamma-1},C_p=\gamma\frac{nR}{\gamma-1}
\end{equation}

设 $i$ 为气体分子自由度数,例如单原子气体分子自由度为 $3$,而双原子分子自由度为 $5$($3$ 个平动自由度和 $2$ 个转动自由度,我们先不考虑振动).那么\textbf{通常情况}下,$C_V$ 约为 $inR/2$(这可以用能量均分定理\upref{EqEng} 来解释),于是 $C_p=(i+2)nR/2$,$\gamma=(i+2)/i$.从这一公式可知,单原子分子的 $\gamma=1.667$,双原子分子的 $\gamma=1.40$.然而在实验中观察到,双原子分子气体的 $\gamma$ 随温度的变化有明显的变化,而且更为合理的假设应该是 $i=7$(算上两个原子作简谐振动的自由度).在实验的低温情况下,气体分子的这些自由度似乎被“冻结”了.这些是经典理论无法解释的.
\addTODO{待进一步探索}
\begin{example}{$\gamma$ 与理想气体绝热过程}

理想气体在准静态绝热过程\upref{Adiab}中满足
\begin{equation}
pV^{\gamma}=\text{常量}
\end{equation}

这是因为在绝热过程中,$\delta{Q}=0$,$\dd{U}=\delta{W}$ 意味着 $C_V\dd{T}=-p\dd{V}$,所以
\begin{equation}
\dd(pV)=nR\dd{T}=C_V(\gamma-1)\dd{T}=-(\gamma-1)p\dd{V}
\end{equation}

解得 $V\dd{p}+\gamma p\dd{V} = 0$,所以 $\dd{(pV^\gamma)}=0$,即 $pV^{\gamma}$ 为常量.
\end{example}
