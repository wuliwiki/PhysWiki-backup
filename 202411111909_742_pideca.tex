% pi 介子衰变
% keys 介子|衰变|弱作用
% license Usr
% type Wiki



\pentry{狄拉克场\nref{nod_Dirac}}{nod_3634}

\subsection*{$\pi$介子衰变} 
这一节我们来讨论$\pi$介子衰变。我们首先讨论带电$\pi$的衰变 
\begin{equation}
\pi^- \rightarrow l^- + \bar \nu_l~.
\end{equation}
其中$l$是$\mu$子或电子。

我们不知道$W$粒子是如何耦合到$\pi$介子的。但是我们知道$W$粒子是如何耦合到轻子的,这个过程的散射振幅可以写为
\begin{equation}
\mathcal M = \frac{g_w^2}{8(M_W c)^2} [\bar u (3) \gamma_\mu (1-\gamma^5)v(2)] F^\mu~.
\end{equation}

其中$F^\mu$是描写$\pi$到$W$的形状因子。它是某个标量乘上$p^\mu$.

\begin{equation}
F^\mu = f_\pi p^\mu~. 
\end{equation}

$f_\pi$被称为$\pi$衰变常数。

对出射自旋求和,我们可以得到
\begin{equation}
\begin{aligned}
\langle |\mathcal M|^2 \rangle &= \bigg[ \frac{f_\pi}{8} \bigg( \frac{g_w}{M_W c} \bigg)^2  \bigg]^2 p_\mu p_\nu {\rm Tr} [\gamma^\mu (1-\gamma^5) p\!\!\!/_2\gamma^\nu (1-\gamma^5) (p\!\!\!/_3 + m_l c)   ] 
\\
& = \frac{1}{8} \bigg[ f_\pi \bigg( \frac{g_w}{M_W c} \bigg)^2 \bigg]^2 [2(p\cdot p_2)(p\cdot p_3) - p^2 (p_2\cdot p_3) (p\cdot p_3) - p^2 (p_2 \cdot p_3) ] ~.
\end{aligned}
\end{equation}

化简得
\begin{equation}
\langle |\mathcal M|^2 \rangle = \bigg( \frac{g_w}{2 M_W} \bigg)^4 f_\pi^2 m_l^2 (m_\pi^2 - m_l^2) ~.
\end{equation}

由散射振幅的模平方可以计算衰变率 
\begin{equation}
\Gamma = \frac{|\mathbf p_2|}{8\pi \hbar m_\pi^2 c} \langle | \mathcal M | \rangle ~.
\end{equation}

经计算
\begin{equation}\label{eq_pideca_1}
\Gamma = \frac{f_\pi^2}{\pi \hbar m_\pi^3} \bigg( \frac{g_w}{4 M_W} \bigg)^4 m_l^2 (m_\pi^2 - m_l^2)^2 ~.
\end{equation}

电子和$\mu$子衰变率的比值
\begin{equation}
\frac{\Gamma(\pi^- \rightarrow e^- +\bar \nu_e)}{\Gamma(\pi^- \rightarrow \mu^- +\bar \nu_\mu)} = \frac{m_e^2(m_\pi^2-m_e^2)^2}{m_\mu^2(m_\pi^2-m_\mu^2)^2} = 1.283 \times 10^{-4}~.
\end{equation}

$\pi$喜欢$\mu$子道。

这个结果非常令人吃惊,因为电子要比$\mu$子轻很多。相空间喜好衰变到那些质量降低越大越好的道,除非某种守恒定律介入。一般来说,最轻的末态是最常见的,但是$\pi$衰变是著名的例外。

\autoref{eq_pideca_1} 如果电子是无质量的,$\pi^- $