% 为什么月球没有被太阳吸走
% license Usr
% type Tutor

为什么我们在分析月亮和卫星的运动时,可以不考虑太阳的引力? 事实上,太阳对月亮的引力比地球对月亮的引力要大!

你有兴趣的话,可以自行查找太阳、地球、月球的质量,以及他们的平均距离,然后用牛顿的万有引力公式计算:
\begin{equation}
F = \frac{Gm_1m_2}{r^2}~.
\end{equation}

如果你还记得高中的知识,也可以把地球中心作为坐标系的原点,假设太阳不存在,计算验证月球绕地球旋转的周期和轨道。 事实上你会发现这和真实月球真实的运动所差无几。

那么问题来了,为什么在计算月球绕地球旋转时不需要考虑太阳的引力?若硬要加上引力反而会得到 “月亮将被太阳吸走” 这样奇怪的结论。

在回答这个问题以前,我们先提出另一个问题。 这个问题是: 若把地球中心作为坐标系的原点,为什么地球没有被太阳吸走?难道换个参考系地球就不受太阳引力了吗?

这似乎有点循环论证的意思,我们知道运动是相对于观察者所在参考系的。 因为我们定义了地球中心是坐标原点,所以地球在这个坐标系中自然是不动的。 但为什么受了力的物体会没有加速度呢? 难道牛顿第二定律 $F = ma$ 在这里失效了? 如果太阳对地球的引力在这个参考系中对地球不起作用,你为什么会认为它对月球要起作用呢?

事实上许多人在学习牛顿定律时忽略了一个前提:\textbf{使用的参考系必须是惯性系}!

事实上牛顿第一定律的作用就是定义惯性系: 在某个参考系中,如果任何不受力(或者所有外力抵消)的物体都做匀速直线运动或静止不动,那么它就是一个惯性系。

所以在以上分析中,若要直接使用原汁原味的牛顿三定律和万有引力公式计算地球和月球的运动,必须要以太阳为参考系。当然了,严格来说太阳也在绕银河系中心做圆周运动,也并不完全是惯性系……但为了不把事情弄得太复杂,我们姑且人为假设太阳参考系是一个惯性系。

难道牛顿定律还有不那么原汁原味的版本?是的,因为我们经常用一些数学小技巧来让牛顿定律在非惯性参考系中也变得可以使用。 我们先看一个更简单的例子。

假设地面参考系是惯性系, 地面上有一个以恒定加速度 $a$ 直线行驶的长方体车, 若它里面有一个质量为 $m$ 的小球在车的尾部,那么根据牛顿第二定律, 车的后壁对小球必须施加一个向前的压力 $N = ma$。

\begin{figure}[ht]
\centering
\includegraphics[width=8cm]{./figures/bb8ffcd88a31ba31.pdf}
\caption{} \label{fig_EquivI_1}
\end{figure}

但如果我们以车作为参考系, 就会发现虽然小球受的合力 $F = N$ 不为零, 但小球却并没有动! 显然牛顿第二定律在车的参考系中失效了。 那么如何才能拯救 $F = ma$ 呢? 在车中 $a$ 为零, 所以合力 $F$ 也应该为零。 那么我们不妨假设小球额外受到一个假想的力,称为\textbf{惯性力},与 $N$ 相同但方向相反,不就可以把合力凑成 0 了吗? 我们把这个惯性力记为
\begin{equation}\label{eq_EquivI_1}
F_\text{惯} = -ma~.
\end{equation}
这个惯性力的假设能彻底在车的参考系中拯救牛顿第二定律吗?我们可以考虑一般情况:

首先我们要知道加速度和速度一样是可以叠加的。 小球在车中向前的加速度为 $a_1$, 那么小球相对于地面的加速度就是 $a + a_1$ (想想为什么)。所以在地面参考系中, 小球的真实受力
\begin{equation}\label{eq_EquivI_2}
F_\text{真} = m(a + a_1) = ma + ma_1~.
\end{equation}
注意真实的力是不随参考系变化的。 但在车的参考系中,我们假设了额外的惯性力 $F_\text{惯}$, 所以在车中列出牛顿第二定律是
\begin{equation}
F_\text{真} + F_\text{惯} = ma_1~.
\end{equation}
把\autoref{eq_EquivI_1} 带入得
\begin{equation}
F_\text{真} -ma = ma_1~.
\end{equation}
但这移项以后就是\autoref{eq_EquivI_2}!

这就说明加上惯性力以后,无论小球在车中做什么运动,牛顿第二定律都成立了。也就是说车内的任何物体, 无论他们做什么运动, 在列出牛顿第二定律时只要假设他们受到一个额外的惯性力,就可以让牛顿第二定律重获新生。

注意惯性力\autoref{eq_EquivI_1} 必须正比于物体的质量,这不禁让我们联想到重力(也就是万有引力)。 这就好比说在一个加速的车中,我们可以假设有一个指向车尾部的重力场,所有的惯性力都是这个重力场产生的, 而这个重力场对应的重力加速度就是 $-a$, 使得质量为 $m$ 的物体受到额外的 “重力” $-ma$。

现在我们用惯性力来解释一下失重的概念。 假设地面为惯性系,一个做垂直自由落体的电梯中有一个实验者。 若忽略空气,假设电梯以重力加速度 $g$ 下落, 那么实验者会观察到完全的失重。 此时他将可以漂浮在电梯中相对电梯不动, 因为在地面参考系中,他受大小为 $F_\text{真}=mg$ 的向下的重力, 但也同时在向下以加速度 $a = g$ 加速。 所以牛顿第二定律 $F_\text{真}=ma$ 成立。

\begin{figure}[ht]
\centering
\includegraphics[width=6cm]{./figures/2740dbb58c35f300.pdf}
\caption{} \label{fig_EquivI_2}
\end{figure}
而在电梯参考系中, 我们要假设实验者受到向上的惯性力 $F_\text{惯} = -ma = -mg$ (这里向上为负),这刚好和重力等大反向。 所以他的合力恰好为零: $F_\text{惯} + F_\text{真} = 0$。 这就解释了他在电梯参考系中为什么感受不到重力。

事实上如果实验者无法观察到电梯外的事物,他将不可能通过任何实验检测他此时处于一个没有万有引力的惯性系还是处于做自由落体的电梯中!如果他抛出一个小球,不考虑空气阻力小球将做匀速直线运动。事实上这正是广义相对论的基础,称为\textbf{等效原理}。

现在如果实验者用一根绳子把一个地球模型和一个月球模型连接,并让它们转起来,那么它们将一直转下去,绳子给月球模型提供的拉力也精确满足它做圆周运动所需的向心力……完全不需要考虑这两个模型还同时受到了重力!


