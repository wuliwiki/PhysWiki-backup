% 复变函数的导数 柯西—黎曼条件
% 柯西|黎曼|导数

\begin{issues}
\issueTODO
\end{issues}

\pentry{复变函数\upref{Cplx}, 全微分\upref{TDiff}}

\begin{definition}{}
类似实函数的导数\upref{Der}, 定义复变函数 $w = f(z)$ ($f:\mathbb C\to \mathbb C$)的导数为
\begin{equation}\label{CauRie_eq4}
f'(z) = \lim_{h\to 0} \frac{f(z + h) - f(z)}{h}
\end{equation}
其中 $h$ 也是一个复数. 极限 $h \to 0$ 在这里是指 $h$ 可以在复平面上以任意方式趋近于 $0$ 都得到同一个极限值, 否则极限不存在.
\end{definition}

对于实变量的函数, 只有正负两个方向趋于零, 所以复变函数可导的条件更复杂.

\begin{theorem}{柯西—黎曼条件}
令 $z = x + y\I$ 且
\begin{equation}
f(z) = u(x, y) + \I v(x, y)
\end{equation}
那么 $f(z)$ 在该点可导的充分必要条件是: 实函数 $u,v$ 在复平面的某点 $z$ 可微, 且
\begin{equation}\label{CauRie_eq1}
\pdv{u}{x} = \pdv{v}{y} \qquad
\pdv{u}{y} = - \pdv{v}{x}
\end{equation}
该式被称为\textbf{柯西—黎曼条件(Cauchy-Riemann condition)}.
\end{theorem}
可见若柯西—黎曼条件成立, 函数 $u,v$ 的四个偏导数中只有两个是独立的.

在复平面的一个开集(链接未完成) $D$ 上, 如果函数 $f(z)$ 处处复可微, 那么它就是一个\textbf{全纯函数(holomorphic function)}也叫做\textbf{解析函数(analytical function)}; 如果除了一些孤立点外处处复可微, 就叫\textbf{亚纯函数(meromorphic function)}.

\subsubsection{推导}
根据全微分\upref{TDiff}
\begin{equation}\label{CauRie_eq2}
\dd{w} = \dd{u} + \I \dd{v} = \qty(\pdv{u}{x} \dd{x} + \pdv{u}{y} \dd{y}) + \qty(\pdv{v}{x} \dd{x} + \pdv{v}{y} \dd{y})\I
\end{equation}
如果直接由此计算 $\dv*{w}{z} = (\dd{u} + \I \dd{v})/(\dd{x} + \I \dd{y})$ 会发现结果和 $\dd{y}/\dd{x}$ 有关, 即与\autoref{CauRie_eq4} 中 $h$ 趋近于零点的方向有关. 所以我们换一种思路, 先假设导数存在, 直接令导数为
\begin{equation}
f'(z) = a(z) + b(z)\I
\end{equation}
写成微分形式
\begin{equation}
\dd{w} = f'(z)\dd{z}
\end{equation}
即
\begin{equation}
\dd{u} + \I \dd{v} = (a + b\I)(\dd{x} + \I\dd{y}) = (a \dd{x} - b\dd{y}) + \I (b \dd{x} + a\dd{y})
\end{equation}
该式对比\autoref{CauRie_eq2} 得
\begin{equation}
a = \pdv{u}{x} = \pdv{v}{y} \qquad
b = -\pdv{u}{y} = \pdv{v}{x}
\end{equation}
这样, 不仅得到了柯西—黎曼条件, 也得到了导数的表达式
\begin{equation}
f'(z) = \pdv{u}{x} -\pdv{u}{y} \I = \pdv{v}{y} + \pdv{v}{x}\I
\end{equation}
可见我们仅需要 $u, v$ 中的一个就可以求出导数, 因为它们并不是独立的.

\addTODO{举例子, 例如 $\exp$, $z^a$ 等都满足柯西—黎曼}
