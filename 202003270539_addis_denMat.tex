% 密度矩阵(量子力学)

\pentry{矩阵的迹\upref{trace}}

\footnote{参考 Shankar, Principles of Quantum Mechanics 2ed}若一个系综中的 $N$ 个系统中, 有 $n_i$ ($i = 1,2,\dots,k$) 个在状态 $\ket{i}$ (这里假设 $\ket{i}$ 是正交归一的). 那么这个系综可以用\textbf{密度矩阵(density matrix)}(或算符)描述
\begin{equation}
\rho = \sum_i p_i\ket{i}\bra{i}   
\end{equation}
其中 $p_i = n_i/N$ 是随机选一个系统, 处于状态 $\ket{i}$ 的概率. 若所有系统都处于同一个 $\ket{i}$, 那么这个系综就是\textbf{纯的(pure)}, 否则就是\textbf{混合的(mixed)}.

对于某个物理量对应的算符 $\Omega$, 它的\textbf{系综平均值(ensemble average)}为
\begin{equation}
\ev{\bar\Omega} = \sum_i p_i \mel{i}{\Omega}{i}
\end{equation}
这个平均值既包含了每个 $\ket{i}$ 的平均, 又包含了对每个系统的平均.

系综平均也可以用迹表示为 $\opn{tr}(\Omega\rho)$. 根据迹的定义,
\begin{equation}
\opn{tr}(\Omega\rho) = \sum_j \mel{j}{\Omega\rho}{j} = \sum_{i,j} p_i\mel{j}{\Omega}{i} \braket{i}{j} = \sum_{i} p_i\mel{i}{\Omega}{i} = \ev{\bar\Omega}
\end{equation}
证毕.

对于纯态, 
