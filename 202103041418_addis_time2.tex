% 时间的计量 2

\pentry{时间的计量\upref{TimeCa}}}

通过遥远天体的观测等方法决定一个参考系(International Celestial Reference Frame, ICRF), 决定地球在 ICRF 中的转角\textbf{地球旋转角(Earth Rotation Angle, ERA)} $\theta$, 并以此定义 \textbf{UT1 儒略日}为
\begin{equation}
\theta = 2\pi(0.7790572732640 + 1.00273781191135448 T_u)
\end{equation}
其中 $T_u = \text{UT1 儒略日} - 2451545.0$, 且 $2.7378\times 10^{-3}$ 约等于 $1/365.25$ 即一回归年的太阳日个数.

把 UT1 儒略日的小数部分除以 86400 ($24\times 60\times 60$), 就可以得到 UT1 时间.


 若用 $T_0$ (国际单位)表示地球相对 ICRF 旋转一周, 那么 $T_u$ (国际单位)就是
\begin{equation}
T_u = \frac{T_0}{1.00273781191135448}
\end{equation}

\subsection{两种太阳时}

平太阳时, 
均时差
