% 浮力的计算(散度公式)
% license Usr
% type Tutor


\pentry{散度\upref{Divgnc},牛顿—莱布尼兹公式的高维拓展\upref{NLext}}

现在我们用面积分的方法表示浮力。 令 $z$ 轴竖直向上, 且水面处 $z = 0$, 则水面下压强为
\begin{equation}
P = -\rho_0 g z~.
\end{equation}
现在把上述的闭合曲面划分为许多个微面元, 第 $i$ 个面元用矢量 $\Delta \bvec s_i$, 表示, 其中模长为面元的面积, 方向为从内向外的法向。 这个面元受到外界液体的压力为
\begin{equation}
\Delta \bvec F_i = -P\Delta \bvec s_i = \rho_0 g z \Delta \bvec s_i~.
\end{equation}
现在把所有面元所受的压力求和, 并用曲面积分\upref{SurInt}表示为
\begin{equation}
\bvec F = \oint \rho_0 g z \dd{\bvec s}~,
\end{equation}
这就是物体所受的浮力。 使用\autoref{eq_NLext_3}~\upref{NLext} 得
\begin{equation}
\bvec F = \int \grad(\rho_0 g z) \dd{V} = \rho_0 g V_0 \uvec z~,
\end{equation}
可见该结论与“等效法”中得出的一致。
