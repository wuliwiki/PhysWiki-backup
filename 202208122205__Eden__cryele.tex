% 晶体中电子在电场和磁场中的运动
% 晶体|电子|能带|有效质量

\pentry{近自由电子模型\upref{egasmd}}
在\textbf{单电子近似}下,利用近自由电子模型\upref{egasmd}或者紧束缚模型\upref{tbappx},可以有效地分析晶体中电子的行为.和自由电子模型相比,描述电子状态的波函数从平面波过渡到 Bloch (布洛赫)波,准连续的 $\epsilon(k)$ 也分裂为\textbf{能带}.电子在周期性势场中运动的本征态,对应的能量本征值 $\epsilon$,和其 Bloch 波函数的波数 $\bvec k$($\hbar \bvec k$也被称为准动量)有一个对应关系,被称为色散关系,画在 $\epsilon$-${\bvec k}$ 图上就呈现出一个曲面,由于周期性势场的影响这些曲线画在简约布里渊区中呈现出若干个能带.

可以说,能带反映了晶体中电子的几乎一切行为,可以用来解释金属、半导体、绝缘体的导电行为.因此,近自由电子模型或紧束缚模型虽然对晶体模型作了许多简化,但在解释晶体的性质方面取得了很大成功.利用能带,我们不仅可以知道晶体中电子的行为,还可以知道对晶体加外场(例如电场或磁场)时电子的变化,从而计算得到晶体对外场的响应.

如何分析加外场后电子的响应呢?一种方法是求解外加场的情况下薛定谔方程的能量本征解,即
\begin{equation}
\qty[\frac{\hbar^2}{2m}\nabla^2+V(\bvec r)+U]\psi = E\psi
\end{equation}
另一种方法是将电子的运动近似当作经典粒子来处理.下面我们介绍的就是这种分析计算方法.
\subsection{电子运动的准经典模型}

