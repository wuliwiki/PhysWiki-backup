% 微分中值定理
% keys 拉格朗日中值定理|柯西中值定理|罗尔中值定理|mean value theorem

\pentry{导数\upref{Der}}

\begin{definition}{函数极值}
考虑实函数$f(x)$.如果存在一个实数轴上的开集$O$,且有$x_0\in O$,使得对于任意的$x\in O$,都有$f(x_0)\geq f(x)$,则称$f(x_0)$是$f$在$O$上的一个\textbf{极大值(maximum)};如果$x_0$满足的条件改为对于任意的$x\in O$,都有$f(x_0)\leq f(x)$,则称$f(x_0)$是$f$在$O$上的一个\textbf{极小值(minimum)}.

极大值和极小值统称为\textbf{极值(extremum)}.

如果$f(x_0)$是一个极大值,那么称$x_0$是一个\textbf{极大值点(maximum point)};相应地,极小值对应的自变量$x_0$是一个\textbf{极小值点(minimum point)}.极大值点和极小值点统称\textbf{极值点(extremum point)}.
\end{definition}

\begin{theorem}{Fermat定理}
考虑实函数$f(x)$.如果$x_0$是$f$的一个极值点,且$f(x)$在$x_0$处可导,那么$f'(x_0)=0$.
\end{theorem}

\textbf{证明}:

假设$f(x)$在$x_0$处可导且取极大值,

\textbf{证毕}.











