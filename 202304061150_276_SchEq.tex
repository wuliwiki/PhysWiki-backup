% 定态薛定谔方程(单粒子一维)
% 量子力学|波函数|薛定谔方程|哈密顿|束缚态

\begin{issues}
\issueTODO
\end{issues}

\pentry{量子力学的算符和本征问题\upref{QM1}, 矢量算符\upref{VecOp}}

\subsection{定态薛定谔方程}
\footnote{参考 Wikipedia \href{https://en.wikipedia.org/wiki/Finite_potential_well}{相关页面}。}定态薛定谔方程就是能量算符即哈密顿算符的本征方程, 本征值就是能量 $E$。
\begin{equation}
H \psi(x) = E \psi(x)
\end{equation}
由于哈密顿算符 $H$ 是线性的, 单从方程来看, 把 $\psi$ 乘一个任意复常数仍然是解。 但由于 $\psi$ 需要满足一定的归一化条件, 所以 $\psi$ 乘以相位因子 $\E^{\I\theta}$ 仍然是解。 事实上任何测量量算符的本征函数都是如此。 下面会看到 $H$ 是一个实算符\footnote{实算符这里指其把实值函数映射到实值函数, 虚值函数映射到虚值函数。}, 所以它的解 $\psi(x)$ 也总可以取实值函数。

哈密顿算符对应粒子的总能量, 总能量算符可以表示为动能算符和势能算符之和
\begin{equation}
H = T + V~.
\end{equation}

\subsection{一维定态薛定谔方程}
\pentry{施图姆—刘维尔定理\upref{SLthrm}}
一维运动的单个质点, 波函数是坐标 $x$ 的函数 $\Psi(x)$
\begin{equation}
T = -\frac{\hbar^2}{2m} \dv[2]{x} \qquad V = V(x)
\end{equation}
所以定态薛定谔方程为
\begin{equation}\label{eq_SchEq_1}
-\frac{\hbar^2}{2m} \dv[2]{\Psi}{x} + V(x)\Psi = E \Psi
\end{equation}

\subsubsection{束缚态}
这是一个二阶线性常微分方程。 数学上来看无论 $E$ 是多少, 必有两个线性无关的解(连接未完成), 它们的线性组合也是解(二维解空间)。 然而物理上可能存在的波函数必须要\textbf{可归一化}, 即满足波函数和自身的内积 $\braket{\psi}{\psi} = 1$。 对一维波函数来说, 归一化条件为
\begin{equation}
\int_{-\infty}^{+\infty} \abs{\psi}^2 \dd{x} = 1
\end{equation}
一种判断归一化的方法是, 波函数有界($\abs{\psi}$ 有最大值), 且随着 $\abs{x} \to \infty$,  $\abs{\psi}^2$ 在极限 $\abs{x}\to\infty$ 的过程中下降得比 $1/x$ 要快\footnote{这是因为积分 $\int_a^\infty 1/x^r \dd{x}$ ($a > 0$)收敛当且仅当 $r > 1$。}。

可以证明只有对于某些离散的 $E$ 我们才能解出这些波函数。 我们把每个 $E_n$ 叫做\textbf{能级(energy level)}, 对应的波函数叫做\textbf{束缚态(bound state)}。 可以证明所有的能级都处于区间
\begin{equation}\label{eq_SchEq_2}
\min V(x) < E_i < \min V(\pm\infty)
\end{equation}
其中 $\min V(x)$ 是函数 $V(x)$ 的最小值(可以是负无穷), $\min V(\pm\infty)$ 表示 $V(x)$ 的正无穷极限和负无穷极限中较小的一个。 这是因为当 $E$ 太小时, 波函数必定会在无穷远处爆炸(趋于无穷大), 而当 $E$ 太大时, 波函数虽然不会爆炸但也不会趋于零。 \autoref{eq_SchEq_2} 要求势能函数 $V(x)$ 中存在某种形状的凹陷, 称为\textbf{势阱(potential well)}。

初学者常用的势阱有: 无限深势阱\upref{ISW},有限深势阱\upref{FSW}, 简谐振子(升降算符)\upref{QSHOop}。 读者可以通过这些具体的例子巩固以上结论。

\begin{theorem}{束缚态存在性}
对一维或二维的有限深势阱, 如果各方向无穷远处势能为零, 那么都至少有一个束缚态\footnote{一维证明参考 \cite{Teschl} 的第九章。}。 若不满足则未必存在束缚态。 三维或以上的有限深势阱也未必存在束缚态\footnote{详见 \cite{Landau} 45 节。}。
\end{theorem}
一个不存在束缚态的一维势阱的例子见 “有限深不对称方势阱\upref{AMW}”, 它不符合定理中 “各方向无穷远处势能为零” 的条件。

\begin{theorem}{束缚态的正交归一条件}
一维势阱的束缚态必定是正交的。 正交归一条件可记为
\begin{equation}
\int_{-\infty}^{+\infty} \psi^*_m(x) \psi_n(x)\dd{x} = \delta_{m,n}
\end{equation}
其中 $\delta_{m,n}$ 是克罗内克 $\delta$ 函数\upref{Kronec}。
\end{theorem}
证明: 直接对一维薛定谔方程应用施图姆—刘维尔定理\upref{SLthrm}即可。 证毕。

\subsubsection{简并}
束缚态能级的\textbf{简并(degenerate)}是指一个能量本征值 $E$ 对应多个线性无关解的束缚态。 一个能量最多有几个线性无关的束缚态, 那就说该能级有多少\textbf{重}简并\textbf{(degeneracy)}。 这和线性代数的厄米矩阵本征值问题\upref{HerEig}中的 “简并” 用法类似。

由施图姆—刘维尔定理\upref{SLthrm}可知\textbf{一维束缚态不存在简并}。
\addTODO{放到哪个词条? 有时候虽然不存在严格的解,但存在很长 life time 解, 例如类氢原子的 Stark 效应(抛物线坐标系)\upref{HStrk2}。}

\subsubsection{散射态}
当 $E > \min V(\pm\infty)$ 时, 虽然波函数必定不满足归一化, 但它们仍然有重要的应用。 我们把它们叫做\textbf{散射态(scattering states)}。 计算散射态时, 通常我们要求两个极限 $V(\pm \infty)$ 都存在。 这样在满足无穷远处 $E > V$ 的方向, 波函数在无穷远处将会接近于平面波。

由于散射态不要求像束缚态那样归一化, \textbf{一维散射态总是存在双重简并}, 因为对同一个 $E$ 定态薛定谔方程作为二阶微分方程总有两个线性无关的解(链接未完成)。

散射态最简单的例子是自由粒子($V\equiv 0$)的散射态, 即平面波 $\exp(\I kx)$。 对于同一个 $E$, $k$ 取一对相反数 $k = \pm\sqrt{2mE}/\hbar$ 就是两个不同的散射态。

由于散射态不能用 $\braket{\psi}{\psi} = 1$ 的方式归一化, 它们并不是物理的, 即粒子不能处于某个散射态。 然而散射态的线性组合却可能满足归一化。 这里指的是无穷多个散射态的线性组合, 求和变为积分:
\begin{equation}
\psi_{wp}(x) = \int C_1(E) \psi_{s,1}(E, x)\dd{E} + \int C_2(E) \psi_{s,2}(E, x)\dd{E}
\end{equation}
其中 $\psi_{wp}$ 代表可以被归一化的波包, $\psi_{s,1}$ 和 $\psi_{s,2}$ 代表能量 $E$ 的两个线性无关的散射态。

容易看出对于自由粒子, 该积分就是反傅里叶变换, 用换元积分法把积分变量改为 $k$, 用第一个积分表示 $k\ge 0$, 第二个表示 $k<0$。 例如自由粒子($V \equiv 0$)的高斯波包\upref{GausPk} 就可以看作由平面波线性叠加而来, 即反傅里叶变换。

某种意义上, 给定一个势能函数, \textbf{所有的束缚态和散射态可以构成一组完备的正交归一的函数基底}, 可以展开所有可归一化的波函数。 展开系数的计算方法和傅里叶变换类似。 散射态有特殊的正交归一化条件, 这需要用到较为复杂的数学工具, 我们留到 “一维散射态的正交归一化\upref{ScaNrm}” 介绍。

\subsection{波函数的对称性}
\pentry{宇称算符\upref{Parity}, 对易厄米矩阵与共同本征矢\upref{Commut}}
对任意维薛定谔方程, 若哈密顿算符和宇称算符 $\Pi$ 对易, 则它们具有一组共同的本征波函数\upref{Commut}, 其中每个都具有奇宇称或者偶宇称。

\subsubsection{一维束缚态}
若一维定态薛定谔方程中势能函数是偶函数, 则束缚态必定是奇函数或偶函数, 这是因为, 一维势能的束缚态都是非简并的, 唯一的一组能量本征矢必定就是共同本征矢。 根据施图姆—刘维尔定理\upref{SLthrm}, 随着能级递增, 波函数的零点数也递增, 所以奇函数和偶函数束缚态会交错出现。 这是因为奇函数必有奇数个零点而偶函数必有偶数个零点。

\subsubsection{一维散射态}
若一维定态薛定谔方程中势能函数是偶函数, 则束缚态必定是奇函数或偶函数, 散射态可以取奇函数或偶函数。 射态都具有二重简并, 对于给定的 $E$, 两个线性无关解的任意线性组合也都是解。

下面说明散射态波函数\textbf{总可以取}奇函数和偶函数,即
\begin{equation}\label{eq_SchEq_3}
\psi(-x) = \pm \psi(x)
\end{equation}
这是因为,对于一个给定的 $E$ 如果 $\psi(x)$ 满足一维定态薛定谔方程, 且既不是奇函数又不是偶函数, 那么由于 $V(x)$ 是偶函数,$\psi(-x)$ 也一定满足方程。这样一来,它们的线性组合 $\psi(x)\pm\psi(-x)$ 也满足。$\psi(x)\pm \psi(-x)$ 是奇函数或偶函数,我们可以将它们设作二重简并子空间的两个正交基底, 其他 $\psi(x)$ 总是可以由它们展开。
