% 泛函分析笔记
% 泛函分析|数学分析|空间|Banach 空间|希尔伯特空间

参考书: Applied Functional Analysis application to mathematical physics (Zeidler)
% 另参考 OneNote 和 iMessage 中的笔记

\subsection{Banach 空间}

\begin{itemize}
\item $\mathbb R$ 和 $\mathbb C$ 分别表示实数域和复数域复数域, $\mathbb K$ 表示二者中的一个

\item $\mathbb K^N$ 表示 $N$ 元 \textbf{tuple} $(\xi_1, \xi_2, \dots, \xi_N)$

\item 某区间上的连续函数 $u:[a, b] \to \mathbb{R}$ 可以表示为 $C[a, b]$

\item $\mathbb{K}$ \textbf{上(over $\mathbb K$)}的矢量空间(linear space 就是 vector space) 表示只能以 $\mathbb K$ 的元乘以某个矢量

\item $C[a, b]$ 是无穷维矢量空间

\item 用 $\norm{u} > 0$ 表示\textbf{范数(norm)}

\item 定义了范数的空间就叫\textbf{赋范空间(normed space)}, 满足 (1) $\norm{u} \geqslant 0$, (2) $\norm{u} = 0$ iff $u = 0$, (3) $\norm{\alpha u} = \abs{\alpha} \norm{u}$, (4) $\norm{u + v} \leqslant \norm{u} + \norm{v}$

\item 定义两个矢量之间的 \textbf{distance} 为 $\norm{u - v}$

\item 模可以用于定义极限 $\lim_{n\to\infty} u_n = u$ 为 $\lim_{n\to\infty} \norm{u_n - u} = 0$, 即 $u_n$ 收敛到 $u$

\item 上一条中, (1) $u$ 是唯一的, (2) $u_n$ 是有界的(bounded), (3) $\norm{u_n} \to \norm{u}$, (4) $u_n + v_n \to u + v$, (5) $\alpha_n u_n \to \alpha u$

\item 柯西序列(Cauchy sequence): 对任意 $\varepsilon > 0$, 存在 $N$, 当 $n, m \geqslant N$ 就有 $\norm{u_n - u_m} < \varepsilon$

\item 在赋范空间中, 每个收敛序列都是柯西序列

\item 赋范空间 $X$ 是 Banach 空间当且仅当每个柯西数列都收敛

\item 在 Banach 空间中, 收敛序列都是柯西序列

\item 空间 $X := C[a, b]$ 是实数 Banach 空间, 定义模长为 $\norm{u} := \max_{a \leqslant x \leqslant b} \abs{u(x)}$. $u_n \to u$ 意味着 $\norm{u_n - u} = \max_{a \leqslant x \leqslant b}  \abs{u_n(x) - u(x)} \to 0$. 也就是 $u_n(x)$ 一致收敛到 $u$

\item 如果柯西序列 $u_n$ 的子序列 $u_{n'} \to u$, 那么 $u_n \to u$

\item 若 $\sum_{j=1}^\infty \norm{u_{j+1} - u_j} < \infty$, 那么 $u_n$ 是柯西序列

\item 集合 $U_\varepsilon (u_0) := \{u \in X: \norm{u - u_0} < \varepsilon\}$ 叫做 $u_0$ 的 $\varepsilon$-\textbf{邻域 (neighborhood)}

\item $X$ 的子集 $M$ 是\textbf{开集} 当且仅当对任意 $u \in M$ 都存在属于 $M$ 的邻域

\item $X$ 的子集 $M$ 是\textbf{闭集} 当且仅当每个序列的极限都属于 $M$

\item $X$ 的子集 $M$ 是 closed 当且仅当 $X - M$ 是开的

\item $M$ 和 $Y$ 是集合, $u \in M, v \in Y$ 算符 $A: M \to Y$ 代表映射 $v = Au$, 其中 $M$ 是\textbf{定义域(domain of definition)}, 也记为 $D(A)$. 值域(range) 是 $A(M) := \{v \in Y: v = Au, u \in M\}$, 也记为 $R(A)$

\item $A$ 叫 \textbf{满射(surjective)} 当且仅当 $A(M) = Y$, 叫做 \textbf{单射(injective)} 当且仅当 $Au = Av$ 意味着 $u = v$, \textbf{双射(bijective)} 如果前两者都符合

\item 如果 $A$ 是 bijective, 存在逆算符 $A^{-1}: Y \to M$, 定义为 $A^{-1} v = u$ 当且仅当 $Au = v$


\item 算符也叫函数

\item $A: M \subseteq X \to Y$ 表示 $A: M \to Y$ 且 $M \subseteq X$, 当 $Y = \mathbb K$, 就把 $A$ 叫做\textbf{泛函(functional)}
\end{itemize}

\subsection{1.11 Compactness}
\begin{itemize}
\item 如果赋范空间的集合 $M$ 满足每个序列都有收敛的子序列, 那么 $M$ 就是 \textbf{相对紧的(relatively compact)}

\item 如果赋范空间的集合 $M$ 满足每个序列都有收敛的子序列且收敛到 $M$ 中, 那么 $M$ 就是 \textbf{紧的(compact)}

\item 如果存在 $r \geqslant 0$ 使任意 $u \in M$ 都有 $\norm{u} \leqslant r$, 那么 $M$ 就是有界的

\item $M$ 是紧的当且仅当它是相对紧的且闭合

\item 每个紧集都是有界的

\item $\mathbb K^N$ 上的子集若使用 $\norm{u} := \abs{u}_\infty$, 那么它是相对紧的当且仅当它是有界的

\item \textbf{Arzela-Ascoli theorem}: 令 $X := C[a, b]$, 且 $\norm{u} := \max_{a\leqslant x\leqslant b}\abs{u(x)}$. 那么若 $M \subseteq X$ 有界且一致连续, 那么 $M$ 就是相对紧的

\item Weierstrass 定理: 令 $f: M\to \mathbb R$ 为赋范空间中非空紧子集 $M$ 上的连续函数, 那么 $f$ 在 $M$ 上存在一个最大值和最小值

\item 令 $X, Y$ 为 $\mathbb K$ 上的赋范空间, 令 $A: M \subseteq X \to Y$ 中 $M$ 为非空紧集, $A$ 为连续算符. 那么 $A$ 是一致连续的.

\item \textbf{finite $\epsilon$ net} 对任意 $\epsilon > 0$, 存在有限多个点 $v_1, \dots, v_J \in M$ 

\item $A: M \subseteq X \to Y$ 称为\textbf{紧的(compact)} 当且仅当 $A$ 是连续的且 $A$ 将有界集变换到相对紧集.

\item 算符 $(Au)(x) := \int_a^b F(x, y, u(y)) \dd{y}$ 是紧的, 其中 $-\infty < a < b < \infty$, $u(x)$ 有界.
\end{itemize}

\subsection{1.13 The Minkowski Functional and Homeomorphisms}
\begin{itemize}
\item 赋范空间 $X$ 中的两个范数叫做 \textbf{equivalent} 当且仅当存在正数 $\alpha$ 和 $\beta$ 使得 $\alpha \norm{u} \leqslant \norm{u}_1 \leqslant \beta \norm{u}$ 对所有的 $u \in X$ 成立

\item 有限维空间的任意两个范数都是 equivalent 的

\item 每个有限维赋范空间都是 Banach space.

\item 如果 $u_0, \dots, u_N$  满足 $u_1 - u_0, u_2 - u_0,\dots, u_N - u_0$, 那么就说它们在 \textbf{一般位置(general position)}. 这个性质与 $u_0, \dots, u_N$ 的顺序无关.

\item $N$-单纯形($N$-simplex) 是 $\mathcal S := \opn{co}\qty{u_0, \dots, u_N}$ ($\opn{co}$ is convex hull), 其中 $u_0, \dots, u_N$ 在一般位置. $0$-单纯形是一个点.

\item 单纯形的 \textbf{barycenter} 是 $b := \sum_{j=0}^N u_j / (N+1)$

\item $\mathcal S$ 的 $k$-face 是单纯形中 $k+1$ 个不同 vertices 的 convex hull

\item $\opn{diam} M := \sup_{u, v\in M} \norm{u - v}$ is the \textbf{diameter} of $M$, and $\opn{dist}(u, M) := \inf_{w \in M} \norm{u - w}$ is the \textbf{distance} of the point $u$ from the point $M$.
\end{itemize}

\subsection{1.14 The Brouwer Fixed-Point Theorem}

\begin{itemize}
\item 当 $M$ 是有限维赋范空间中紧的, 凸的, 非空的子集, 连续算符 $A: M \to N$ 拥有 \textbf{fixed point}
\end{itemize}

\subsection{1.15 The Schauder Fixed-Point Theorem}
\begin{itemize}
\item 紧算符 $A: M \to M$ 有一个 fixed point 如果 $M$ 是 Banach 空间的一个有界的, 闭得, 突的, 非空的子集.
\end{itemize}

\subsection{1.20 Linear Operators}
\begin{itemize}
\item \textbf{线性算符} $A: L \subseteq X\to Y$ 是线性的当且仅当 $A (\alpha u + \beta v) = \alpha Au + \beta Av$

\item 用 $L(X, Y)$ 表示线性连续算符 $A: X \to Y$, $X$ 是 $\mathbb K$ 上的赋范空间, $Y$ 是 $\mathbb K$ 上的 Banach 空间. $L(X, Y)$ 是 $\mathbb K$ 上的 Banach 空间, 范数就是算符的范数

\item 有限维矢量空间中的线性算符可以表示为矩阵, 所有这些算符都是连续的

\item 算符的\textbf{零空间(null space)} 为 $N(A) := \qty{u \in X: Au = 0}$

\item 线性算符 $A$ 是连续的, 当且仅当存在 $c > 0$ 使 $\norm{Au} \leqslant c\norm{u}$ 对所有 $u$ 都成立

\item 线性算符是单射的当且仅当 $N(A) = {0}$

\item 定义线性连续算符 $A: X \to Y$ 的\textbf{算符范数(operator norm)} 为 $\norm{A} := \sup_{\norm{v} \leqslant 1} \norm{A v}$

\item 当 $X \ne {0}$, 有 $\norm{A} := \sup_{\norm{v} = 1} \norm{A v}$
\end{itemize}

\subsection{1.21 The Dual Space}

\begin{itemize}
\item 令 $X$ 为 $\mathbb K$ 上的一个赋范空间, 一个线性的连续算符 $f: X \to \mathbb K$ 称为\textbf{线性连续泛函(linear continuous functional)}

\item 所有 $X$ 上的线性连续泛函叫做 $X$ 的\textbf{对偶空间(dual space)} $X^*$, $X^* = L(X, \mathbb K)$

\item $f \in X^*$ 作用在 $u \in X$ 上可以记为 $\ev{f, u} := f(u)$

\item $f\in X^*$ 的范数为 $\norm{f} := \sup_{\norm{v} \leqslant 1} \abs{f(v)}$, 所以 $\abs{f(u)} \leqslant \norm{f}\norm{u}$

\item 令 $X$ 为 $\mathbb K$ 上的赋范空间, 那么对偶空间 $X^*$ 使用上述范数就是 $\mathbb K$ 上的 Banach 空间.
\end{itemize}

\subsection{1.23 Banach Algebras and Operator Functions}
\begin{itemize}
\item By a \textbf{Banach algebra} $\mathcal B$ over $\mathbb K$ we understand a Banach space over $\mathbb K$, where an additional multiplication $AB$ is defined such that $AB \in \mathcal B$ for all $A, B \in \mathcal B$. More over, for $A, B, C \in \mathcal B$ and $\alpha \in \mathbb K$, $(AB)C = A(BC)$, $A(B+C) = AB + AC$, $(B+C)A = BA + CA$, $\alpha(AB) = (\alpha A)B = A(\alpha B)$, $\norm{AB} \leqslant \norm{A}\norm{B}$. Exist $E \in \mathcal B$ such that $AE = EA$ for all $A \in \mathcal B$ and $\norm{E} = 1$

\item define \textbf{operator function} through $F(A) := \sum_{j=0}^\infty a_j A^j$, and $F(z) = \sum_{j=0}^\infty a_j z^j, z\in \mathbb K$

\item Let $X$ be a Banach space over $\mathbb K$. For each $A \in L(X, X)$ with $\norm{A} < r$, $F(A) \in L(X, X)$
\end{itemize}

\subsection{1.25 Application to the Spectrum}
\begin{itemize}
\item 考虑 $Au = \lambda u, u \in X, \lambda \in \mathbb C$

\item 令 $A\in L(X,X)$, $X$ 是非空的复 Banach 空间.  $\lambda$ 是\textbf{本征值(eigen value)} 当 $u\ne 0$.

\item \textbf{预解集(resolvent set)} $\rho(A)$: 使得 $(A-\lambda I)^{-1}: X \to X$ 存在且 $\in L(X,X)$. $(A-\lambda I)$ 叫做 resolvent.

\item 预解集中的 $\lambda$ 带入本征方程只能解得 0 矢量.

\item $\sigma(A) := \mathbb C - \rho(A)$ 叫做\textbf{谱 (spectrum)}

\item 谱是 $\mathbb C$ 中的紧子集且 $\abs{\lambda} \leqslant \norm{A}$ 对所有 $\lambda\in\sigma(A)$ 成立

\item 每个本征值都属于谱

\item 预解集 $\rho(A)$ 是一个开集

\item Banach 空间 $X$ 上的一个算符 $B: X\to X$ 如果值域 $R(B)$ 是闭的且零空间是有限维的, 那么他就是 \textbf{semi-Fredholm} 的

\item \textbf{本质谱(essential spectrum)} $\sigma_e(A)$ 包括所有使得 $(A - \lambda I)$ 不是 semi-Fredholm 的 $\lambda$
\item $\sigma_e(A) \subseteq \sigma(A)$
\item $\sigma_e(A)$ 就是所有具有无穷\textbf{简并(degeneracy)}的本征值的集合
\item 如果 $X$ 是有限维的, 那么 $A$ 的本质谱是空的 
\end{itemize}

\subsection{1.26 Density and Approximation}
\begin{itemize}
\item $M \subseteq X$ 被称为在 $X$ 中\textbf{稠密的(dense)} 当且仅当 $\bar M = X$, 其中 $\bar M$ 是 $M$ 的闭包.

\item \textbf{可数的(countable)}, \textbf{至多可数(at most countable)}

\item $X$ 叫\textbf{可分的(separable)} 的当且仅当它存在至多可数的稠密子集 $M \subseteq X$

\item \textbf{Weierstrass 近似理论}: $X := C[a, b]$, $-\infty < a < b < \infty$. 所有实系数多项式的集合在 $X$ 内稠密

\item $C[a, b]$ 是可分的.

\item 任何有限维赋范空间都是可分的.

\item 令 $X$ 为可分的赋范空间. 存在一个序列 $\qty{X_n}$ ($X_n$ 是 $X$ 的有限维线性子空间), 使得 $X_1 \subseteq X_2 \subseteq \dots \subseteq X$ 以及 $\bigcup_{n=1}^\infty X_n = X$

\end{itemize}

\subsection{2.1 Hilbert Spaces}
\begin{itemize}
\item 内积\upref{InerPd} 记为 $(u|v)$, $(u|v) \in K$

\item \textbf{pre-Hilbert 空间}  就是定义了内积的线性空间

\item \textbf{柯西—施瓦兹不等式}\upref{CSNeq} 是 (pre-) Hilbert 空间中最重要的性质.

\item pre-Hilbert 空间都是赋范空间, 范数为 $\norm{u} := \sqrt{(u|u)}$

\item 希尔伯特空间定义: 1. 是一个内积空间, 2. 是一个 Banach 空间(或者任意柯西序列的极限都属于它本身)

\end{itemize}

\subsection{2.2 Standard Examples}

\begin{itemize}
\item 空间 $X := \mathbb K^N$ 是一个希尔伯特空间, 内积为 $(x|y) := \sum_j \bar \xi_j \eta_j$, 范数为 $\norm{x} = (x|x)^{1/2}$

\item 令 $-\infty \leqslant a < b \leqslant \infty$, 令 $L_2(a, b)$ 为所有 measurable 函数 $u :]a, b[ \to \mathbb R$(其中 $]a, b[$ 表示开区间) $\qty{x \in R : a < x < b}$, 满足 $\int_a^b \abs{u}^2 \dd{x} < \infty$. 那么 $L_2(a, b)$ 是无穷维的实希尔伯特空间, 内积为 $(u, v) := \int_a^b uv \dd{x}$.

\item $L_2(a, b)$ 的 identification principle: 两个函数 $u$ 和 $v$ 是同一个元素当且仅当 $u(x) = v(x)$ 对几乎所有 $x \in ]a, b[$ 成立.

\item 令 $G$ 表示 $\mathbb R^N$ ($N \geqslant 1$)中的可测度非空子集, $L_2^{\mathbb K}(G)$ 表示可测度函数 $u: G \to \mathbb K$ 的集合, 满足 $\int_G \abs{u}^2 \dd{x} < \infty$. 那么 $L_2^{\mathbb K}(G)$ 是一个希尔伯特空间, 内积定义为 $(u|v) := \int_G \bar u v \dd{x}$

\item 请写出 $L_2^{\mathbb K}(G)$ 中的\textbf{施瓦兹(Schwarz)不等式}

\item 令 $G$ 为 $\mathbb R^N$ 中的非空开子集($N > 1$). 那么 $C^k(G)$ 表示 $k$ 阶连续可偏导的函数 $u: G \to R$ 的集合.

\item $C^k(\bar G)$ 包含 $C^k$ 中所有满足各阶偏导数能拓展到 $G$ 的闭包 $\bar G$ 上的函数.

\item 如果 $u \in C^k(G)$ 对所有的 $k = 0, 1, \dots$ 都成立, 那么我们记 $u \in C^\infty(G)$. 同理可以定义 $C^\infty(\bar G)$

\item $C_0^\infty (G)$ 是所有 $C^\infty(G)$ 中的函数, 满足在 $G$ 的紧子集 $C$ 恒外为零.

\item 令 $G$ 为 $\mathbb R^N$ 中的一个非空开集, $N \geqslant 1$. 那么 (i) $C_0^\infty(G)$ 和 $C(\bar G)$ 在 $L_2(G)$ 中稠密.

\item $C_0^\infty(G)_{\mathbb C}$ 和 $C(\bar G)_{\mathbb C}$  在 $L_2^{\mathbb C}(G)$ 中稠密
\end{itemize}

\subsection{2.5 The Functional Analytic Justification of the Dirichlet Principle}
\begin{itemize}
\item 
\end{itemize}


\subsection{5.1 Extensions and Embeddings}
\begin{itemize}
\item 线性空间 $X$ 到 $Y$ 的算符 $A$ 和 $B$ 记为 $B \subseteq A$ 当且仅当定义域 $D(B) \subseteq D(A)$ 且它们在 $D(B)$ 上是同一算符. 这时 $A$ 是 $B$ 的 \textbf{extension}

\item $A = B$ 当且仅当 $A\subseteq B$ 且 $B\subseteq A$

\item \textbf{embedding} $X \subseteq Y$ 是\textbf{连续的}当且仅当存他们之间存在线性,单射,连续的算符

\item \textbf{embedding} $X \subseteq Y$ 是\textbf{紧的}当且仅当存他们之间存在线性,单射, 紧的算符
\end{itemize}


\subsection{5.2 Self-Adjoint Operators}
\begin{itemize}
\item 令线性算符 $A: D(A) \subseteq X \to X$ 的定义域 $D(A)$ 在希尔伯特空间 $X$ 上稠密. 定义 $v \in D(A^*)$ 当且仅当存在 $w\in X$ 使 $(v|Au) = (w|u)$ 对任意 $u\in D(A)$ 都成立. 令 $A^*v := w$, 就得到了\textbf{伴随(adjoint)}算符 $A^*: D(A^*) \subseteq X \to X$. 所以 $D(A^*)$ 是满足定义的最大集合.

\item 自伴算符: (1) 是线性的 (2) $(\alpha A)^* = \bar \alpha A^*$ (3) $A \subseteq B$ 意味着 $B^* \subseteq A^*$

\item 如果 $D(A^*)$ 在 $X$ 上连续, 那么存在 $(A^*)^*$, 记为 $A^{**}$.

\item 希尔伯特空间 $X$ 上的线性算符 $A$ 是\textbf{对称的(symmetric)} 当且仅当 $A \subseteq A^*$, 即 $(Au|v)=(u|Av)$ 对所有 $u, v\in D(A)$ 成立

\item 希尔伯特空间 $X$ 上的线性算符 $A$ 是\textbf{自伴的(self-adjoint)} 当且仅当 $A = A^*$, 注意 $D(A) = D(A^*)$. 自伴算符都是对称算符

\item 希尔伯特空间 $X$ 上的线性算符 $A$ 是 \textbf{skew-symmetric} 当且仅当 $A \subseteq -A^*$

\item 希尔伯特空间 $X$ 上的线性算符 $A$ 是 \textbf{skew-adjoint} 当且仅当 $A = -A^*$

\item 希尔伯特空间 $X$ 上的线性连续算符 $A$ 的伴随算符也是线性连续的, 且 $\norm{A} = \norm{A^*}$ 以及 $A^{**} = A$

\item 令 $f:[a, b]\times[a,b]\to\mathbb R$ 为连续函数且 $-\infty< a < b < \infty$. 定义算符 $A$ 为 $(Au)(x) := \int_a^b f(x, y)u(y)\dd{y}$, 令 $X := L_2(a, b)$. 那么 (1) $A: X\to X$ 是线性紧算符, (2) $(A^*u)(x) = \int_a^b f(y, x) u(y)\dd{y}$, $A^*$ 也是线性紧算符. (3) 如果 $f(x, y) = f(y, x)$ 对任意 $x, y\in[a, b]$ 成立, 那么算符 $A$ 是自伴算符.

\item 希尔伯特空间 $X$ 上的任意的线性自伴算符 $A$ 都是 \textbf{maximally symmetric}. 也就是说, 如果 $S$ 是对称算符且 $A \subseteq S$, 那么 $A = S$

\item 令 $X := L_2^{\mathbb C}(\mathbb R)$, 且 $(Au)(x) := u'(x)$ ($x \in \mathbb R$), $D(A) := {u\in X: u'\in X}$, $u'$ 为广义导数, 那么 (1) 算符 $A$ 是 skew-adjoint 的, (2) $\I A$ 是自伴算符.

\item 令 $B$ 为上一条中的 $A$ 改成 “非广义” 的求导算符, 且定义域为 $X \bigcap C^1(\mathbb R)$, 那么 $B$ 是 skew-symmetric 算符且 $\I B$ 是对称算符
\end{itemize}
