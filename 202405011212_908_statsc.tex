% 统计物理·微观与宏观之间的桥梁
% keys 统计物理|科普|微观
% license Xiao
% type Tutor
\addTODO{本文的定位是一篇简介+综述…… 我打算介绍热力学、统计物理的基本学科逻辑,以及它和其他各学科之间的关系。}
\subsection{微观世界与宏观世界}
物理学史上,人们对大自然的探索主要有两条路线。其中之一,是对物质的结构不断“细分”,通过对\textbf{微观世界}的研究来探索最根本的自然法则。这个思想源自于古希腊时期德谟克里特的“原子论”,随后被发扬光大,对物理学、化学等学科都产生了深远的影响。而在二十世纪\textbf{相对论}与\textbf{量子力学}\footnote{这一时期诞生了\enref{狭义相对论}{SpeRel}、广义相对论,以及量子力学。在这之后量子场论的蓬勃发展极大地推动了人们对粒子物理的认识。}的革命之后,伴随着加速器技术的发展,人们发现了重子、介子、轻子等形形色色的粒子,在许许多多物理学家的共同努力下,“粒子物理标准模型”诞生,人们对微观世界的认识达到了一个崭新的高度。这一条探索路线上发生过无数次物理学的“革命”,带来了各种各样的惊喜。

另一条路线则与之不同,人们不是仅仅着眼于那些自然法则,而是去挑战\textbf{复杂的系统}所特有的物理学性质,从我们所处的宇宙,到我们身边的空气、河流……这纷繁的世界之间似乎有一种遥远的相似性,似乎存在着许许多多重要的关于复杂的系统的规律,独立于那些最根本的自然法则。具有代表性的成果是\textbf{热力学和统计物理}\footnote{可以参考热力学与统计力学导航\upref{StatMe}。}。在卡诺、焦耳、克劳修斯等科学家的努力下,人们建立起一套完备的自洽的热力学体系,这是以热力学第零定律到热力学第三定律\footnote{事实上还包括其他的一些假设,例如广延量和强度量的约定等等。}为基本假设建立的一套关于多粒子系统\footnote{或者说,是“大量微观粒子组成的宏观客体”。这其实暗示了,在微观物理定律与宏观物质性质之间有大量的物理可以挖掘。}的物理理论。而基于这个物理理论,人们得到了许许多多美妙的结果。我们能够运用流体力学分析大气、海洋,能够运用连续介质力学研究地壳的运动,运用等离子体物理研究太阳风、地球磁层,运用固体物理研究金属、半导体等材料的性质…… 对宏观世界中多粒子系统的研究与科学技术的发展、与人类的生产生活方式息息相关。而正因为 "\textbf{More is different}"\footnote{出自 Anderson 的著名论文 More is different,可以看作是凝聚态物理的《独立宣言》。},关于复杂系统有数不清的无限多的难题等着科学家们去解决。

微观与宏观、基本定律与复杂系统之间的研究是相辅相成的。宇宙学标准模型的建立离不开广义相对论和热力学的发展,凝聚态系统中的准粒子(其实质是场的激发)也启发着人们对量子场论的理解。微观物理和宏观物理的联系是如此微妙又神奇。当玻尔兹曼\footnote{玻尔兹曼是现代物理学奠基人,对统计物理的发展有巨大的贡献。}写下 $S=k\ln \Omega$\footnote{玻尔兹曼墓碑上的公式,阐释了热力学熵的统计涵义。}时,他一定不会想到统计力学在物理学史上具有如此重要的地位——它掀起了近代物理的革命,使人们对事物的认识达到了一个崭新的层次。
\subsection{统计物理的思想}
19世纪末,以麦克斯韦、玻尔兹曼为代表的物理学家们开始对大量粒子系统的微观与宏观之间的联系产生兴趣。人们开始思考并尝试解构一些经典的热力学概念,例如温度、内能、热传导…… 得益于对原子、分子的认识,我们可以将宏观物质放大再放大,并尝试通过微观的粒子的运动、它们之间的相互作用来“还原”出宏观的一些物理量。人们对自然界的认识常常是从一般到特殊,所以平衡态的热力学系统自然而然地成为了重要的研究对象。
\subsubsection{经典气体}
想象你有一瓶空气,这瓶空气中有许许多多的空气分子,可以是氮气分子、水分子、氧气分子等等。这里为了简单起见,我们考虑一瓶水蒸气,即只有水分子。一小瓶水蒸气所包含的分子的数量是巨大的,人们常用阿伏伽德罗常数($1 {\rm N_A}\approx 6.02\times 10^{23}$)为单位来刻画粒子的数量,$1 {\rm N_A}$ 的粒子记为 $1 {\rm mol}$。分子的大小非常小,以至于相邻气体分子之间的距离达到分子半径的数十倍。在这个距离上,气体分子之间的相互作用如此之弱,以至于我们暂时可以忽略它\footnote{如果考虑范德瓦尔斯力,则得到的热力学性质将与理想气体不同,参考范德瓦尔斯气体\upref{Vand}文章。}。

\begin{figure}[ht]
\centering
\includegraphics[width=5cm]{./figures/d8eaa32f4159aeb8.png}
\caption{一小瓶气体分子示意图,图中的小球是气体分子,可以有更复杂的内部结构。} \label{fig_statsc_1}
\end{figure}
当然,气体分子无时无刻都在运动。英国植物学家布朗发现悬浮在水中的微粒在做不规则运动,这被称为“布朗运动”。同样地,如果我们标记瓶子中的某一个气体分子,称它为“小明”,那么他的运动轨迹也将是无规则的,这是由于小明常常与其他的气体分子发生碰撞。人们用 $\lambda$ 来衡量小明在相邻两次碰撞间走过的平均的距离,这被称为分子的平均自由程。为了方便讨论,我们先假定\footnote{事实上物理学家们常常通过约定一些假设来简化问题;而最后当物理学家们对该理论有更深刻了解时,再回过头来看这些假设,对它们进行修改、简化,或者揭示这些假设的本质。所谓的非弹性碰撞实际上意味着气体分子的内部自由度:自旋角动量、轨道角动量、振动等因素也被考虑到碰撞过程中。而在系综理论中,我们能够充分地考虑这些自由度,从而更细致地描写系统。}分子间的碰撞全是弹性碰撞,并忽视气体分子的内部的复杂结构,将它们看作是刚性的小球——这被称为\textbf{理想气体}\footnote{对理想气体的更细致的定义参考理想气体\upref{Igas}文章。}。模型的简化给我们带来了很多的方便,例如我们暂时地撇去了分子内部结构因素,从而可以将该模型运用到对各种各样气体的讨论。理想气体模型等价于极稀薄的温度趋近于 $0 \rm K$ 的系统,或者说 $p\rightarrow 0,V\rightarrow \infty$,实验上这样的气体系统服从简单的理想气体状态方程\footnote{参考理想气体状态方程\upref{PVnRT}文章。}。
我们还得到了重要的守恒定律:能量守恒。对于理想气体而言,这个能量是指微观系统的一切分子的动能与在外场\footnote{可以是重力场、电场或是其他形式的场。}下势能的和\footnote{由于我们忽略了分子内部自由度,所以分子的振动、转动可以不被算在这个能量中。}。

热力学中也有一条守恒定律,即热力学第一定律:$\dd E=\delta Q+\delta W$。于是微观守恒定律和宏观守恒定律被\textbf{统一}了起来。当我们分析理想气体时,我们可以初步地将内能 $E$ 定义为所有分子的动能之和。在无外场的情况下,根据麦克斯韦速度分布律\footnote{由理想气体的状态方程,以及各向同性、方向独立等性质,可以推出麦克斯韦—玻尔兹曼分布\upref{MxwBzm}。},理想气体系统中,速度空间 $\dd v_x\dd v_y\dd v_z$ 单位体积元内的分子数与 $e^{-\beta E_k}$ 成正比。其中的 $\beta$ 取决于气体分子在单个自由度上的平均能量,根据理想气体状态方程或能量均分定理,$\beta$ 与温度成反比:$\beta=1/kT$。于是我们由麦克斯韦速度分布率知道理想气体的(几乎)一切微观信息,由这些速度分布率公式和 $\dd E=T\dd S-p\dd V$ 等热力学方程,可以推导出压强、熵、自由能等一切宏观热力学量。
\subsubsection{统计物理基本假设和系综理论}
在物理学的发展过程中,人们常常先总结出实验现象的规律,再考虑去提出合理的模型和假设去定性和定量地描述这一现象;另一方面,不同尺度、不同类型的研究对象之间总是有着千丝万缕的联系。这导致一个理论的形成过程中不仅仅只有一条逻辑链,而是许许多多条合理的逻辑链交织在一起。例如,我们可以构建一个模型,从 $A$ 出发,提出一定假设,提出 $B$,再经过一系列推导得到结论 $C$;我们也可以通过将 $C$ 设立为公理,提出一定假设,从 $C$ 出发得到 $B$,最后推出 $A$。不同逻辑往往都有一定的合理性,而且如果只局限于其中一条逻辑,会大大地增加理解难度并且违背物理学发展的规律,所以一般教科书上喜欢将一系列结论呈现出来,让读者自己思考和理解并建立自己的知识体系。

尽管从不同的假设出发都能够建立知识体系,但某些假设是重要的而极具推广价值的,而另一些假设则无法推广到一般的情形。例如前面讨论理想气体时,以理想气体状态方程和能量均分定理出发讨论是不具推广价值的,当考虑分子相互作用时、分子内部自由度,这些讨论则变得异常复杂,甚至不再成立。对于实际气体,麦克斯韦速度分布律只是一个近似;而对于液体、固体、外场中的气体等等,速度分布率则需要重新考虑。 我们如何能够利用已知的分子性质,通过系统的微观状态来分析它的宏观性质呢?我们很难从纯力学上继续对系统的宏观物理量进行任何具体的计算,尽管近年来计算机的发展使得我们能够利用分子动力学的数值方法来模拟自由度很大的系统,但这无法帮助我们洞察这背后的一些重要规律。玻尔兹曼开创性地引入统计物理的假设,即\textbf{各态历经}假设——在足够长的时间内,系统的代表点将会在系统的相空间能量曲面上的各个区域停留相同的时间。在这样的假设下,我们可以将足够长时间内所有可能的系统微观状态放在一起考虑,它们被称为\textbf{系综},并假设在这个系综内每个微观状态出现的概率都是相等的,这被称为\textbf{等概率原理}\footnote{这相当于在相空间能量曲面上均匀地撒点。值得一提的是,如果令这些点一同随着时间的演化,那么根据刘维尔定理,相空间可以视为不可压缩流体,这些点仍然是均匀分布的,即仍然服从等概率原理。};可以假设可观测的宏观热力学量的值就是物理量的系综平均值,这被称为\textbf{吉布斯假设}。玻尔兹曼和吉布斯的假设完全抛开了统计物理中形而上(机械的)因素,隐去了对于纯力学的不必要的依赖。尽管能够从数学上严格证明各态历经的系统大多都是那些极其简化的系统,但人们可以得到十分丰富的物理结果,这些结果可以被大量的物理实验所直接证实。

有了统计物理的基本假设,我们可以重新思考系统中粒子的分布。对于经典理想气体,麦克斯韦速度分布率实际上对应于系综中的\textbf{最概然分布}。不同的系统粒子速度分布(满足能量和为定值)在系综中出现的概率是不同的,其中概率最大的分布就是最概然分布,而利用组合数学、拉格朗日乘数法的技巧可以证明最概然分布就是麦克斯韦速度分布律。统计物理的假设甚至对量子系统也是成立的,当我们将上面的讨论拓展到其他一切近独立体系\footnote{忽略粒子间的力学相互作用,不同粒子间可以看作是近独立的。同时,粒子间可以有交换相互作用,考虑到量子系统中玻色子和费米子的性质\upref{depsys}。},系综中的系统微观状态则需要服从玻色统计或费米统计。两个费米子不能够占据同一个量子态,而多个玻色子可以占据同一个量子态,经过一系列的计算将得到不同于麦克斯韦速度分布律的公式。实际上,根据吉布斯假设,麦克斯韦玻尔兹曼分布、玻色分布、费米分布也可以由系综平均值得出,即可以计算某个量子态上的粒子占据数的数学期望,可以证明当粒子数趋于无穷时它等同于最概然分布得到的结果。系综中的系统微观状态可以偏离最概然分布,系统的物理量也可能会围绕着系综平均值上下波动,这被称为\textbf{涨落},可以证明当粒子数趋于无穷时,涨落趋于零。

在上面的分析中,我们采取了这样一套分析方法:假定系统的能量 $E$ 固定\footnote{准确地来说,是固定在一个区间 $E$ 到 $E+\dd E$,最后计算的结果应除去 $\dd E$。},体积 $V$ 、粒子数 $N$ 固定,这一多体系统所有可能的量子态组成一个系综,然后计算物理量的系综平均值。固定 $E$ 和 $V$ 的系综被称为\textbf{微正则系综}。由于我们固定了所有粒子的能量的总和,所以这个系综平均值实际上很难计算,我们转而采取了最概然分布作近似计算。为了通过微正则系综的分析重建系统状态方程 $E=E(S,V,N)$,我们实际上需要对这里出现的热力学熵 $S=S(E,V,N)$ 进行微观的重新定义。这实际上就是著名的\textbf{玻尔兹曼熵公式}:$S=k\ln \Omega(E,V,N)$,其中 $\Omega(E,V,N)$ 是体积为 $V$、粒子数 $N$、能量为 $E$ 的系统微观状态数密度。这样我们就从统计物理基本假设出发得到了完备的热力学体系。通过状态方程 $E=E(S,V,N)$,可以通过微分关系、热力学势的公式推出温度、压强、自由能等一切物理量。

为了更方便地计算系综平均,人们开始考虑另一套方法:取消能量 $E$ 固定的性质,而是假定温度 $T$ 固定,能量 $E$ 可以有涨落。这实际上对应这样一个物理模型:将系统和无穷大的温度为 $T$ 的热源接触,系统和热源间无粒子数交换,但可以有热量交换。此时系综中粒子的能量总和不再受限,被称为\textbf{正则系综}。如果将系统以及热源作为一个整体看作微正则系综,则系统和热源作为一个整体的微观状态服从等概率原理,基于此可以推出:系统的微观量子态 $S$ 出现的概率正比于 $e^{-E_S/kT}$。而我们考虑的热力学方程应当从熵表象切换到温度表象,即 $F=F(T,V,N)$\footnote{在热力学中,自由能 $F$ 可以通过勒让德变换与 $E$ 联系:$F=E-TS$,那么 $\dd F=-S\dd T-p\dd V$。},并且有公式 $F=-kT\log Z$,其中 $Z$ 是正则系综的配分函数。再之后,人们发现固定粒子数 $N$ 也是不方便的,于是便取消对粒子数 $N$ 的限制,转而约定与它共轭的化学势 $\mu$ 恒定。仅固定 $T,V,\mu$ 的系综被称为\textbf{巨正则系综},相当于将固定体积的系统与一个大热源(温度为 $T$)、大粒子源(化学势为 $\mu$)接触。在巨正则系综中,系统的微观量子态 $S$ 出现的概率正比于 $e^{-(E_S-\mu N)/kT}$。巨正则系综的意义在于不固定 $E$ 和 $N$,从而对近独立体系计算巨配分函数 $\sum_S e^{-(E_S-\mu N)/kT}$ 的时候可以将它拆分成不同量子态的巨配分函数的乘积,这极大地简化了计算。当粒子间存在相互作用时,巨配分函数的计算仍然十分复杂,但幸运的是,巨配分函数可以表达为 ${\rm tr} (e^{-\beta(\hat H-\mu N)})$,人们能够运用\textbf{虚时路径积分}\footnote{这体现了场论方法在多体系统中的精彩应用,参考有限温度量子场论。}的方法对它进行计算。
