% 南京理工大学 2006 量子真题
% license Usr
% type Note

\textbf{声明}:“该内容来源于网络公开资料,不保证真实性,如有侵权请联系管理员”

\subsection{填空题(每题6分}

1.德布罗意关系为_________。

2.写出量子力学五个基本假设中的两个_________。

3.波函数的标准条件为_________。

4.能量算符表达式为_________;栋梁算符表达式为_________。

5.坐标和动量的对易关系为 $[x, p_x] = \underline{\quad\quad}$;
          测不准关系是 $(\Delta x)^2 \cdot (\Delta p_x)^2 \geq \underline{\quad\quad}$。

6.对氢原子,不考虑电子的自旋,能级的简并度_________;考虑自旋但不考虑自旋与轨道角动量的耦合时,能级的简并度为_________。

7.费米子和玻色子所组成的全同粒子体系的波函数分别具有性_________和_________性。

8.原子跃迁的选择定则中角量子数应满足$\Delta l = \underline{\quad\quad}$。磁量子数应满足 $\Delta m = \underline{\quad\quad}$。

9.考虑自旋后,波函数在自旋空间表示为 $\psi_s = \begin{pmatrix} \psi_(1)(x, y, z) \\\\ \psi_2(x, y, z) \end{pmatrix}$(已归一化),则在态 $\psi$下,自旋算符_________。对自旋的平均可表示为_________;对坐标和自旋同时求平均的结果可表示为_________。