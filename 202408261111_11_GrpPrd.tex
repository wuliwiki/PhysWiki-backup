% 直积和半直积(群)
% keys 群|直积|笛卡尔积|直和|半直和
% license Xiao
% type Tutor

\pentry{正规子群\nref{nod_NormSG}}{nod_28ef}

\addTODO{群的半直积尚未分为“内”(子群间的运算)和“外”(两个群间的运算)}

\subsection{直积}

群的直积,是在群作为集合的\enref{笛卡尔积}{CartPr}上,由群运算自然导出的一个群。

\begin{definition}{两个群的直积}
给定群 $G$ 和 $H$,群运算的符号省略。在集合 $G\times H$ 上定义运算:对于任意 $(g_i, h_i)\in G\times H$,有 $(g_1, h_1)(g_2, h_2)=(g_1g_2, h_1h_2)$。集合 $G\times H$ 配合以上定义的运算,构成一个群,称为群 $G$ 和 $H$ 的\textbf{直积(direct product)}。
\end{definition}

容易看出,两个群直积的单位元是 $(e, e)$——注意这里的两个 $e$ 分属不同的群,通常是不同的元素。

这个定义分割开了参与直积的不同群的运算,因此可以很方便地直接推广到任意多个群的直积:

\begin{definition}{任意多个群的直积}
给定任意多个群 $\{H_i\}$,在这些群作为集合的笛卡尔积上,各分量运算分别进行运算,且遵循各自所属群的运算规则。该笛卡尔积配合该运算规则构成一个群,称为这些群的\textbf{直积(direct product)},记为 $\bigotimes_iH_i$ 或 $\prod_iH_i$。
\end{definition}

% 在群论中还有一个和直积很类似的概念,常使人混淆,这就是群的\textbf{直和}。直和实际上是直积的一个特例:任意给定群,都可以使用这些群来构造直积,但是直和指的是已经给定了一个群,使用它的特定子群来生成它。% 没有听说过这种说法,更常见的说法应该是,内直积和外直积

\begin{definition}{群的直和}
当 $G$ 和 $H$ 都是交换群时,我们也称 $G\times H$ 为这两个群的\textbf{直和(direct sum)},此时也可以把它表示为 $G+H$。另外,任意个交换群的直积 $\bigotimes_iH_i$ 也可以表示为 $\bigoplus_iH_i$ 或 $\sum_iH_i$。
\end{definition}

直和这一术语的来源不难理解:交换群的群运算通常被称为“加法”。

\begin{theorem}{商群的直积}\label{the_GrpPrd_1}
任意给定群 $G$ 和 $K$,并分别给出它们的一个正规子群 $H\triangleleft G$ 和 $J\triangleleft K$。那么 $(G\times K)/(H\times J)\cong G/H\times K/J$。
\end{theorem}

\autoref{the_GrpPrd_1} 的证明用一个示意图即可直观地展示出来。

\begin{figure}[ht]
\centering
\includegraphics[width=14cm]{./figures/7576f29d8bc51df2.pdf}
\caption{\autoref{the_GrpPrd_1} 的示意图。群 $G$ 和 $K$ 被表示为两条相互垂直的虚线,$G\times K$ 就是它们生成的平面。$H$ 和 $J$ 被表示为两条线段,$H\times J$ 就是作图中间那个方形区域。右图用垂直的虚线把 $G$ 区分成若干区域,就是 $G/H$ 的元素;同理,水平虚线划分出了 $K/J$ 的元素。从图示可以很容易看出,$(G\times K)/(H\times J)$ 的元素就是虚线划分出来的矩形区域们,而这些矩形区域正好是线段的笛卡尔积,也就是 $G/H\times K/J$。} \label{fig_GrpPrd_1}
\end{figure}
\subsection{内直积与外直积}
在引入内直积的概念之前,我们先看一个例子。

设$G=G_1\times G_2$,对于任意$g_i\in G_1,h_i\in G_2$,可以验证$N_1=(g_i,e_2)\vartriangleleft G,N_2=(e_1,h_i)\vartriangleleft G$,其中$e_1,e_2$分别是$G_1,G_2$的单位元。易见$N_1\cap N_2=\{e\}$,且从运算角度上看$G=N_1N_2$,因此称$N_1N_2$是$G$的内直积,所选取的元素为$G$的“分解”表示。

又比如三维线性空间$V$,任意向量都可以分解为基矢组的线性表示,以$\{\bvec i,\bvec j,\bvec k\}$为基矢组,则三个正规子群分别为$(x,0,0),(0,y,0)$和$(0,0,z)$。
\begin{definition}{}
设$N_i(i=1,2...k)$是群$G$的\textbf{正规子群},且满足
\begin{itemize}
\item $G=N_1N_2...N_k$;
\item $N_i\cap N_1N_2...N_{i-1}N_{i+1}..N_k=\{e\}$对一切$i=1,2...k$都成立;
\end{itemize}
则称$G$是$N_i(i=1,2...k)$的内直积。
\end{definition}
在满足题设和条件一的前提下,\textbf{条件二}分别等价于下面两种情况:
\begin{enumerate}
\item \textbf{$G$的单位元有唯一分解表示},即若存在$g_i\in N_i$且$g_1g_2...g_k=e$,则$g_i=e_i$。
\item \textbf{$G$的任意群元都有唯一表示},即若$g_1g_2...g_k=h_1h_2...h_k$则$g_i=h_i$对任意$g_i,h_i\in N_i$恒成立。
\end{enumerate}
我们来尝试证明这一点。

首先证明条件二能推出情况一。
由于对于$i\neq j$有$N_i\cap N_i\subset N_{i-1}\cap N_1N_2...N_{i+1}N_j...N_k=\{e\}$,则任意两个不同的正规子群其交集只有单位元,且有$g_ig_j=g_jg_i$\footnote{根据消去律只需要证明$g_ig_jg_i^{-1}g_j^{-1}=e$,而左边元素属于$N_i\cap N_j$,得证。}。因此若$g_1g_2...g_k=e$,我们有
\begin{equation}
g_1=g_k^{-1}g_{k-1}^{-1}...g_2^{-1}=N_1\cap N_2N_3...N_k=e~,
\end{equation}
,其余元素同理可证。在这个证明里,我们可以看到关键利用的是元素可交换性。

然后证明情况一能推出情况二。设$i\neq j,g\in N_i\cap N_j=g_i=g_j$,其中$g_i\in N_i,g_j\in N_j$,则$g_ig_j^{-1}=e$,填充该等式后我们有$e...g_i...g_j^{-1}...e=e$,则由单位元的分解唯一性得$g_i=g_j=e$,即$N_i\cap N_j=e$,于是$g_ig_j=g_jg_i$。


若$g_1g_2...g_k=h_1h_2...h_k$,我们有
\begin{equation}
(g_1h_1^{-1})(h_2g_2^{-1})(h_3g_3^{-1})...h_kg_k^{-1}=e~,
\end{equation}
由分解唯一性得$g_i=h_i$。

然后我们来证明情况二能推出条件二。设$g=N_i\cap N_1N_2...N_{i-1}N_{i+1}...N_k$,则设存在一系列群元使得$g=g_i=g_1g_2...g_{i-1}g_{i+1}...g_k$。左右填充单位元后,利用表示的唯一性可得:任意群元都为单位元,于是条件二得证。


从上述推导过程可知,内直积表示实际上是寻求\textbf{若干特殊的正规子群,彼此相交元素只有单位元}。

内直积和外直积并非泾渭分明的关系。回顾最初的例子,对于外直积,我们可以构建若干正规子群,使之同构于这些子群的内直积。

\begin{theorem}{}
$G_1,G_2...G_n$是一系列群,设$G=G_1\times G_2...\times G_n$。定义
\begin{equation}
N_i=\left\{\left(e_1, \cdots, e_{i-1}, a_i, e_{i+1}, \cdots, e_n\right) \mid a_i \in G_i\right\}~,
\end{equation}
则有
\begin{enumerate}
\item $N_i\vartriangleleft G,N_i\simeq G_i$。
\item 对于任意$g\in G$,都有$G=N_1N_2...N_k$,且该分解结果是唯一的。
\end{enumerate}
\end{theorem}
\textbf{证明:}第一点略,我们主要证明第二点。(待补充)

由结论一可知,$G$总可以分解为这些zhengygui


反过来,这些特殊正规子群的内直积也可以写为外直积形式。
\begin{theorem}{}
若$N_i\triangleleft G(i=1,2...k)$,且$G$是$N_i$的内直积,则$G\cong N_1\times N_2...\times N_k$。
\end{theorem}
\textbf{证明:}
依旧设$g_i\in N_i$,由于$G=N_1N_2...N_k$,我们可以构建映射使得$f(g_1g_2...g_k)=(g_1,g_2,...,g_k)$,易见这是一个满射。由内直积的定义可知,对于$i\neq j$,有$g_ig_j=g_jg_i$。因而,$(g_1g_2...g_k)(h_1h_2...h_k)=(g_1h_1)(g_2h_2)...(g_nh_n)$,所以
\begin{equation}
\begin{aligned}
f\qty((g_1g_2...g_k)(h_1h_2...h_k))&=f((g_1h_1)(g_2h_2)...(g_nh_n))\\
&=(g_1h_1,g_2h_2,...,g_nh_n)\\
&=f(g_1g_2...g_k)f(h_1h_2...h_k)~.
\end{aligned}
\end{equation}
,因此$f$确实是同态映射。由于$\opn{ker}f=\{e\}$,因此$f$是单射。综上所述,$f$是同构映射,定理得证。


\subsection{半直积}

群的直积可以推广为以下概念:

\begin{definition}{半直积}
给定群 $G$,如果有 $G$ 的一个\textbf{正规子群}$N$ 和一个\textbf{子群}$H$,使得 $G = N H$\footnote{就是说,集合 $G= \{n h \mid n \in N, h\in H\}$。},并且 $N \cap H = \{e\}$,那么我们称 $G$ 是 $N$ 和 $H$ 的\textbf{半直积(semi-direct product)},记为 $G = N \rtimes H$。
\end{definition}

可以注意到,直积是半直积的一种,只要把 $\{(g, e)\}$ 和 $\{g\}$ 等同、把 $\{(e, h)\}$ 和 $\{h\}$ 等同即可。这样,尽管本节中直积是用“运算的笛卡尔积”来定义的,而半直积是用“已有的群运算”来定义的,这两个在特定情况下是等价的。

半直积不一定是内直积,这是因为定义中我们只要求参与运算的两个群中的一个为正规子群,而如果两个群 $G$ 和 $H$ 进行直积,那么容易证明它们俩都是群 $G\times H$ 的正规子群。从这也可以看出来为什么此处半直积的定义要先给出 $G$,而不是像直积的定义一样直接用两个群的乘积得到 $G\times H$。

事实上,我们也可以用以上定义半直积的语言来描述直积:给定群 $G$,如果有 $G$ 的两个正规子群 $H$ 和 $N$,满足 $H\cap N=\{e\}$,并且 $G=NH$,那么称 $G$ 是 $N$ 和 $H$ 的直积,记为 $G=N\times H$。

半直积的定义也可以不依赖于预先给定的 $G$,只是这样会稍显复杂一些:

\begin{definition}{半直积}
给定群 $N$ 和 $H$,并且有同态:$f:H\rightarrow \opn{Aut}N$。在笛卡尔积集合 $N\times H$ 上定义运算:对于 $n_i\in N, h_i\in H$,有 $(n_1, h_1)\cdot(n_2, h_2)=(n_1\cdot f_{h_1}(n_2), h_1\cdot h_2)$。集合 $N\times H$ 配合这个运算可以得到一个群,称为\textbf{群}$N$\textbf{和群}$H$\textbf{关于同态}$f$\textbf{的半直积},记为 $N\rtimes_fH$。
\end{definition}

我们梳理一下这个定义。为了定义半直积,我们要用到三个部分,即两个群 $N$ 和 $H$,再加上一个同态 $f$。要注意,$f$ 不是 $H$ 到 $N$ 的同态,而是到 $\opn{Aut} N$ 的,后者是 $N$ 的\textbf{自同构群}。也就是说,每个 $f(h)$ 表示一个 $N\rightarrow N$ 的自同构映射,为了方便,简写为 $f_h$。

\addTODO{如何理解两个 半直积 之间的一致性?}












