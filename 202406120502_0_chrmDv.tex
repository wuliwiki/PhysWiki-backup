% Chrome 开发工具笔记
% license Usr
% type Note


\begin{issues}
\issueDraft
\end{issues}

\begin{itemize}
\item \verb`F12` 打开开发笔记,许多分界线都可以拖动,可以 \verb`Ctrl` + \verb`+` 控制字体大小。
\item 主要面板有 \verb`Elements, Console, Sources, Network, ...`
\item \verb`Elements` 就是当前 DOM 树,其中很多元素可能是 js 运行后的结果,所以跟直接看页面的 source 可能是不一样的。
\item 打开开发截面以后,在页面任何地方右键 Inspect 都可以直接跳到 \verb`Elements` 的对应元素中。
\item \verb`Sources` 里面会显示加载的所有源文件,按照 URL 中的层级来分类(云朵图标),如果是方框图标就说明是浏览器插件或者 Chrome 自带的东西。
\end{itemize}

\subsubsection{Network 面板}
在 \verb`Network` 面板中选中一个文件
\begin{itemize}
\item 在 \verb`General` 里面可以看到 url,ip,端口, http 方法,状态等
\item \verb`Header` 子面板里面可以看到 http 的请求头和返回头(包括使用的 http 版本,所有的 header 包括 cache 管理等)。
\item \verb`Payload` 里面有请求中的 \verb`query`, 就是 \verb`?` 后面的东西。
\item \verb`Initiator` 可以看到这个请求是什么文件发起的(因为最开始在浏览器输入的就是一个 URL,所有其他的文件都是可以追溯来源的)。
\item \verb`Timing` 可以看到详细的用时,包括什么时候开始请求,等待服务器响应多久,下载多久。
\item 点漏斗图标可以展开和隐藏 Filter 面板
\begin{figure}[ht]
\centering
\includegraphics[width=10cm]{./figures/559c8eb603a4edcb.png}
\caption{Filter 面板} \label{fig_chrmDv_1}
\end{figure}
\item Filter 中的 \verb`fetch/XHR` 是 javascript 向服务器请求资源的两个常用函数。 前者比较新,支持 \verb`async, promise`。 \verb`XHR` 是 \verb`XMLHttpRequest` 的缩写。
\item \verb`fetch` 有更好的 \textbf{Cross-Origin Resource Sharing (CORS)}
\end{itemize}

\subsubsection{CORS}
\begin{itemize}
\item 当浏览器加载一个网站的页面,请求获取另一个网站的资源时,浏览器对另一个网站发出的请求中会包含名为 \verb`Origin` 的 http 头。 如果第二个网站返回的 http 头中有 \verb`Access-Control-Allow-` 开头的字段, 那么浏览器就会用其判断该请求是否时第二个网站所允许的。 如果不允许则请求失败。
\item CORS 有什么好处呢? 比如说我网站上有一些自用的开源库例如 MathJax,如果没有 CORS,别的网站就可以蹭这些资源,那么流量费就要算到我头上了。如果我在网站设置了 CORS,就可以只允许我网站的页面使用这些资源。
\item 一个例子是,如果把小时百科一篇文章的 html 下载到桌面然后双击打开(或者用本地的 web server 通过 localhost 打开),那么里面许多图标将无法加载。
\end{itemize}

\subsection{简单 Web Server}
\begin{itemize}
\item \verb`http://localhost` 等效于 \verb`http://127.0.0.1`
\item 也可以在 url 后面用 \verb`:端口号` 指定端口。
\end{itemize}
