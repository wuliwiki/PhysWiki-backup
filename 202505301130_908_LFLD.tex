% 列夫·朗道(综述)
% license CCBYSA3
% type Wiki

本文根据 CC-BY-SA 协议转载翻译自维基百科 \href{https://en.wikipedia.org/wiki/Lev_Landau}{相关文章}。

列夫·达维多维奇·朗道(俄语:Лев Дави́дович Ланда́у,1908年1月22日-1968年4月1日)是一位苏联物理学家,在理论物理的诸多领域作出了基础性的贡献。\(^\text{[1][2][3]}\)他被认为是最后一批在物理学各个分支都造诣深厚并做出开创性贡献的科学家之一。\(^\text{[4]}\)他被誉为20世纪凝聚态物理学的奠基人,\(^\text{[5]}\)同时也被广泛认为是苏联最杰出的理论物理学家。\(^\text{[6]}\)
\subsection{生平}
\subsubsection{早年时期}
朗道于1908年1月22日出生在俄罗斯帝国的巴库(今属阿塞拜疆),父母是犹太人[11][12][13][14]。他父亲达维德·列沃维奇·朗道是一位从事当地石油工业的工程师,母亲柳博芙·维尼亚米诺芙娜·加尔卡维-朗道是一名医生。两人都来自莫吉廖夫,并毕业于当地的文理中学[15][16]。朗道12岁学习微分学,13岁学习积分学,并在1920年13岁时从中学毕业。由于父母认为他年龄太小,不适合直接升入大学,他先在巴库经济技术学校学习了一年。
1922年,年仅14岁的朗道进入巴库国立大学,同时注册了两个系:物理与数学系以及化学系。后来他中止了化学的学习,但终其一生对化学始终保有兴趣。
