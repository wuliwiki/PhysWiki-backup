% 施瓦西度规下时空的测地线
% keys 广义相对论|施瓦西度规
% license Usr
% type Tutor

\pentry{测地线\nref{nod_geodes},世界线与固有时\nref{nod_wdline},Christoffel 符号\nref{nod_CrstfS}}{nod_6191}
\subsection{施瓦西度规下时空的测地线}
回顾施瓦西度规
\begin{equation}
\dd  s^2 = -\left(c^2 - \frac{2GM}{r}\right) \dd t^2+ \left(1-\frac{2GM}{r c^2}\right)^{-1}\dd r^2 + r^2(\dd \theta^2 + \sin^2 \theta \dd \varphi^2) ~,
\end{equation}
这指出在大天体 $M$ 附近的时空的情况。

而考虑对于一条类时的测地线 $\gamma(\tau)$,其中 $\tau$ 是固有时,若要求分量的参数表达式 $x^\mu(\tau)$ 则是要解测地线方程
\begin{equation}
\dv[2]{x^\mu}{\tau} + \Gamma^{\mu}_{\nu \lambda} \dv{x^\nu}{\tau} \dv{x^\lambda}{\tau} = 0~.
\end{equation}
其中 $\Gamma^\mu_{\nu \lambda}$ 是 Christoffel 符号。

而对于施瓦西度规下的时空,由于对称性,总可以选取某个坐标系使得 $\gamma(\tau)$ 的 ¥
