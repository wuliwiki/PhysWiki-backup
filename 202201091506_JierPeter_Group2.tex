% 群的同态与同构
% 同态|同构|商群|群同态基本定理

\pentry{正规子群\upref{Group1}}

\subsection{同构}

让我们来观察两个群 $(\mathbb{Z}, +)$ 和 $(2\mathbb{Z},+)$.如果我们把 $2\mathbb{Z}$ 中的 $2$ 都看成 $1$,$4$ 都看成 $2$,以此类推,将 $2k$ 都看成 $k$,那么两个群的运算规则是一模一样的.比如说,$2\mathbb{Z}$ 中有 $2+4=6$,对应的是 $\mathbb{Z}$ 中 $1+2=3$ 的等式.

我们研究集合和群的时候,元素叫什么名字并不重要,重要的是元素之间是否相同以及运算规则是怎样的.那么,如果我们真的将 $2\mathbb{Z}$ 中的元素 $2k$ 都重命名为 $k$,它就和 $\mathbb{Z}$ 没什么区别了.所以在群的意义上,如果不考虑子群关系,单独把 $\mathbb{Z}$ 和 $2\mathbb{Z}$ 拿出来的时候,我们就认为它们是不可区分的,完全相同的两个群.

如果我们建立一个映射 $f:\mathbb{Z}\rightarrow2\mathbb{Z}$,定义为 $f(k)=2k$,那么这个 $f$ 就是一个双射,它在两个群的元素之间一一对应地建立了联系.这样,对于任意整数 $m, n$,有 $f(m)+f(n)=f(m+n)$,也就是说“先运算再映射”和“先映射再运算”结果是相同的.

类似地,对于任意的两个群 $G$ 和 $K$,如果存在一个\textbf{双射} $f:G\rightarrow K$,使得对于任意的 $x, y\in G$ 都满足 $f(x)f(y)=f(xy)$,那么这两个群的运算结构就是一模一样的.这时我们说这两个群是\textbf{同构(isomorphic)}的,而这个使得它们同构的双射就被称为 $G$ 和 $K$ 之间的\textbf{同构映射(isomorphic mapping)},也可以简称\textbf{同构(isomorphism)}这里加粗的两个“同构”,前者是形容词,后者是名词.

\begin{definition}{自同构}
称群到自身的同构为一个\textbf{自同构(automorphism)}\footnote{这个词是用词根“auto(自身的)”和单词“isomorphism(同构)”组合而成的.}.


群 $G$ 的全体自同构配合映射的复合,又构成一个群,称为 $G$ 的\textbf{自同构群},记为 $\opn{Aut}(G)$.


\end{definition}

由于同构使得两个群各方面表现一模一样,研究同构其实没有太大意义,我们甚至直接把同构的两个群看成同一个群,不管元素具体怎么命名的.有意思的结构,是以下定义的“同态映射”.

\subsection{同态}

同构映射是一个双射.如果把这个要求拿掉,我们就得到同态的概念:

\begin{definition}{同态映射}\label{Group2_def1}
对于两个群 $G$ 和 $K$,如果映射\textbf{(不一定是双射)}$f:G\rightarrow K$ 使得 $\forall x, y\in G, f(x)f(y)=f(xy)$,那么称 $G$ 和 $K$ 是\textbf{同态(homomorphic)}的,称 $f$ 是\textbf{同态映射(homomorphic mapping)}或\textbf{同态(homomorphism)}.
\end{definition}

\begin{definition}{像和核}
沿用\autoref{Group2_def1} 的设定.$K$ 中被映射到的元素构成的集合,称为 $f$ 的\textbf{像(image)},记作 $f(G)$.$G$ 中映射到 $K$ 的单位元 $e_K$ 的元素构成的集合,称为 $f$ 的\textbf{核(kernal)},记为 $\ker(f)$.
\end{definition}

注意,$f(G)\subset K$,$\ker(f)\subset G$.

同态的两个群,运算结构很相似但又不完全一样.在以上定义的例子中,$K$ 的行为就像是一个弱化版的 $G$,可能会丢失一些细节,但保留的方面和 $G$ 是一模一样的.这么说可能不够具体,我们用\autoref{Group2_exe2} 和\autoref{Group2_exe1} 来理解同态的“似而不同”.


\begin{exercise}{}\label{Group2_exe2}
设两个群 $G$ 和 $K$,$f:G\rightarrow K$ 是一个同态,$x\in G$,求证 $f(x^{-1})=(f(x))^{-1}$.
\end{exercise}


\begin{exercise}{群同态基本定理}\label{Group2_exe1}
设两个群 $G$ 和 $K$,$f:G\rightarrow K$ 是一个同态.求证:
\begin{enumerate}
\item $\ker(f)$ 是 $G$ 的一个正规子群\footnote{这保证了群 $G/\ker(f)$ 存在.}.
\item 对于 $x, y\in G$,如果 $x_1$ 和 $y_1$ 分别和 $x, y$ 同余,或者换句话来说,$x_1^{-1}x\in H$ 和 $y_1^{-1}y\in H$,那么 $f(x)=f(x_1)$,$f(y)=f(y_1)$.
\item 由前两条的结论,证明可以用 $f$ 来导出一个映射 $f': G/\ker(f)\rightarrow K$ ,它是一个同构.

\end{enumerate}
\end{exercise}

由\autoref{Group2_exe1},同态的实质就是商群 $G/\ker(f)$ 和 $K$ 之间的同构.$G/\ker(f)$ 继承了 $G$ 的运算,但是由于把同余的元素全都当作同一个了,也就丢失了一部分细节.因此我们说同态的两个群也是“似而不同”的.



\subsection{内自同构和外自同构}



回顾线性代数中的知识:给定线性空间的基以后,线性变换和矩阵就一一对应(我们称之为给定基下用矩阵表示线性变换),而改变基以后同一个线性变换的基也会变.因此,群 $(\{\text{线性变换}\}, \text{映射的复合})$ 与群 $(\{\text{矩阵}\}, \text{矩阵乘法})$ 之间可以建立同构.这样的同构不是唯一的,而是依赖于基的选择.

不同的矩阵可以看成同一个线性变换在不同基下的表示,也可以看成两个不同的线性变换在同一个基下的表示.因此,我们可以用一个基来将矩阵对应到线性变换上,再用另一个基将线性变换对应到另一个矩阵上,由此就得到了矩阵之间的对应,这个对应就是矩阵乘法群到自身的同构.同一个线性变换在不同基之间的矩阵表示的关系是\textbf{相似},具体参见\textbf{过渡矩阵}\upref{TransM}小节.

由上段论述可知,给定可逆矩阵 $\bvec{Q}$,则矩阵到自身的映射 $f$ 是一个自同构,其中 $f(\bvec{M})=\bvec{Q}^{-1}\bvec{MQ}$.这提示我们一种构建群自同构的方法.

\begin{definition}{内自同构}
给定群 $G$.取 $g\in G$,定义映射 $\opn{Ad}_g:G\to G$ 如下:对于任意 $x\in G$,都有 $\opn{Ad}_g(x)=gxg^{-1}$.

称 $\opn{Ad}_g$ 是 $G$ 上的\textbf{内自同构(inner automorphism)},或者\textbf{共轭自同构(cogredient automorphism)}.


群 $G$ 的全体内自同构构成 $\opn{Aut}(G)$ 的一个子群,称为 $G$ 的\textbf{内自同构群}, 记为 $\opn{Inn}(G)$.

\end{definition}

\begin{theorem}{}\label{Group2_the1}
内自同构群是自同构群的正规子群.
\end{theorem}

\textbf{证明}:

给定群 $G$,设 $f\in \opn{Aut}(G)$.任取 $g, x\in G$,则由\autoref{Group2_exe2} 可得,
\begin{equation}\label{Group2_eq1}
\begin{aligned}
f^{-1}\qty(gf(x)g^{-1})&=f^{-1}\qty(g)f^{-1}\qty(f(x))f^{-1}\qty(g^{-1})\\
&=f^{-1}\qty(g)xf^{-1}\qty(g^{-1})\\
&=f^{-1}\qty(g)x\qty(f^{-1}\qty(g))^{-1}
\end{aligned}
\end{equation}

由 $x$ 的任意性,\autoref{Group2_eq1} 意味着 $f\circ \opn{Ad}_g\circ f^{-1}=\opn{Ad}_{f^{-1}(g)}$.因此 $f\circ\opn{Inn}(G)\circ f^{-1}\subseteq\opn{Inn}(G)$,故 $\opn{Inn}(G)$ 是 $\opn{Aut}(G)$ 的正规子群.

\textbf{证毕}.

由\autoref{Group2_the1} ,我们可以计算商群 $\opn{Aut}(G)/\opn{Inn}(G)$.

\begin{definition}{外自同构}
给定群 $G$,称 $\opn{Aut}(G)/\opn{Inn}(G)$ 为 $G$ 的\textbf{外自同构群(outer automorphism)},记为 $\opn{Out} (G)$.
\end{definition}

\begin{example}{不是内自同构的自同构}
取复数乘法群 $(\mathbb{C}, \times)$,则取共轭映射 $f(z)=\bar{z}$ 是其上一个自同构,但它显然不是内自同构.由于复数乘法的交换性,复数乘法群上的内自同构只有恒等映射一种.
\end{example}














