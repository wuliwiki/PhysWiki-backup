% 素数定理的证明
% keys 素数定理
% license Usr
% type Tutor

\pentry{素数定理\nref{nod_prithy},数论函数 theta 与 psi 的阶\nref{nod_tpont},渐进估计与阶\nref{nod_asymeq}}{nod_6a5d}

\begin{theorem}{素数个数函数的阶}
\begin{equation}
\pi(x) \sim \frac{\vartheta(x)}{\ln x} \sim \frac{\psi(x)}{\ln x} ~.
\end{equation}

\end{theorem}
\textbf{证明}:我们利用 $\vartheta(x)$ 与 $\psi(x)$ 的阶的估计可以导出素数定理的证明。首先指出
\begin{equation}
\vartheta(x) = \sum_{p \le x} \ln p \le \ln x \sum_{p \le x} 1 = \pi(x) \ln x ~,
\end{equation}
从而直接得到了
\begin{equation}\label{eq_prthpr_1}
\pi(x) \ge \vartheta(x)/\ln x > Ax / \ln x ~.
\end{equation}
而若 $0 < \delta < 1$,
\begin{equation}
\begin{aligned}
\vartheta(x) &\ge \sum_{x^{1-\delta} < p \le x} \ln p ~\\
&\ge (1-\delta) \ln x \sum_{x^{1-\delta} < p \le x} 1 ~\\
&= (1-\delta) \ln x \left( \pi(x) - \pi(x^{1-\delta}) \right) ~\\
&\ge (1-\delta) \ln x \left( \pi(x) - x^{1-\delta} \right) ~.
\end{aligned}
\end{equation}

这就是说,
\begin{equation}\label{eq_prthpr_2}
\pi(x) \le x^{1-\delta} + \frac{\vartheta(x)}{(1-\delta) \ln x} < Bx/\ln x ~.
\end{equation}

而利用\autoref{the_tpont_1} 与\autoref{lem_tpont_1},只需要证明 $\pi(x) \sim \vartheta(x)/\ln x$ 即可。利用\autoref{eq_prthpr_1} 与\autoref{eq_prthpr_2} 就直接有
\begin{equation}
1 \le \frac{\pi(x) \ln x}{\vartheta(x)} \le \frac{x^{1-\delta} \ln x}{\vartheta(x)} + \frac{1}{1-\delta} ~.
\end{equation}

对于任意的 $\varepsilon > 0$,可以选取 $\delta = \delta(\varepsilon) > 0$ 使得 $1/(1-\delta) < 1+ (\varepsilon/2)$,而选取 $N = N(\delta, \varepsilon) = N(\varepsilon)$ 使得对于所有的 $x > N$ 均有
\begin{equation}
\frac{x^{1-\delta}\ln x}{\vartheta(x)} < \frac{C \ln x}{x^\delta} < \varepsilon/2 ~.
\end{equation}
从而对于所有 $x>N$ 都有
\begin{equation}
1 \le \frac{\pi(x) \ln x}{\vartheta(x)} < 1+ \varepsilon ~.
\end{equation}
而 $\varepsilon$ 可以任意小,就证明了第一部分。也就完成了这个定理的证明!

而我们已经证明了,$\vartheta(x)$ 与 $\psi(x)$ 的阶都是 $x$,故 $\pi(x)$ 的阶是 $x/\ln x$ 也就被成功证明!

\begin{corollary}{素数定理的推论}
\begin{equation}
p_n \asymp n \ln n ~.
\end{equation}
\end{corollary}
\textbf{证明}:我们用 $A$ 表示待定常数,但特别的,$A$ 之间可以互不相等,而仅用 $A$ 来估计阶。

利用素数定理,
\begin{equation}
n = \pi(p_n) < \frac{A p_n}{\ln p_n}, ~ p_n > An\ln p_n > An \ln n ~.
\end{equation}
而
\begin{equation}
n = \pi(p_n) > \frac{Ap_n}{\ln p_n} ~,
\end{equation}
从而
\begin{equation}
\sqrt{p_n} < \frac{A p_n}{\ln p_n} < A n, ~ p_n < A n^2 ~,
\end{equation}
这指出,
\begin{equation}
p_n < A n \ln p_n < An \ln n ~.
\end{equation}
就完成证明。


