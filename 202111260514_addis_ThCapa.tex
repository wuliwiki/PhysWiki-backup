% 热容量
% 热容|比热容|等压热容|等体热容|理想气体

\begin{issues}
\issueTODO
\end{issues}
\pentry{热力学第一定律\upref{Th1Law}}
\subsection{热容}
一个系统在一定条件下的\textbf{热容量(heat capacity)}定义为\footnote{这个定义可以类比电容量\upref{Cpctor}}
\begin{equation}
C = \dv{Q}{T}
\end{equation}
热容可能跟温度压强等有关. 定义\textbf{比热容(specific heat capacity)}为热容除以质量
\begin{equation}
c = \frac{C}{m}
\end{equation}
定义\textbf{摩尔热容}为 $1 \rm{mol}$ 物质的热容
\begin{equation}
C_m=\frac{C}{n}
\end{equation}

\begin{example}{}
两份水初始温度分别为 $300\rm{K}$ 和 $360\rm{K}$,体积分别为 $1\rm{L}$ 或 $2 \rm{L}$.将它们放入绝热容器种混合均匀,求末温度.(注:水的比热容 $c$ 随温度的变化不大,可以近似看成一个常数)

设末温度为 $T$,那么第一份水吸收的热量为 $c m_1(T-300\rm{K})$,第二份水放出的热量为 $cm_2(360{\rm{K}}-T)$.由于在绝热容器中混合,且 $m_2=2m_1$,可以解得 $T=340\rm{K}$

我们也可以从\textbf{能量守恒}的角度考虑这个问题,可以得到更清晰的认识.第一份水的内能为 $cm_1T_1+m_1u_0$,第二份水的内能为 $cm_2T_2+m_2u_0$,将它们混合后,总质量为 $m=m_1+m_2$,其内能为 $cmT+mu_0$.水的体积近似不变,所以忽略对外做功;列能量守恒可以得到 $m_1T_1+m_2T_2=mT$,从而可以计算出 $T=340K$.
\end{example}

我们可以定义等体热容为系统在等体过程\upref{EqVol}中的热容.根据热力学第一定律\upref{Th1Law} ,$\dd U=\dd Q-P\dd V$,可知等体过程中 $\dd Q=\dd U$(这里定义了态函数\textbf{焓} $H=U+pV$),所以
\begin{equation}\label{ThCapa_eq1}
C_V=\left(\frac{\partial U}{\partial T}\right)_V
\end{equation}

类似地,定义等压热容为系统在等压过程\upref{EqPre}中的热容.在等压过程中 $\dd Q=\dd U+P\dd V=\dd (U+pV)=\dd H$,所以
\begin{equation}\label{ThCapa_eq2}
C_P=\left(\frac{\partial H}{\partial T}\right)_P
\end{equation}

类似可以定义摩尔等体热容 $C_{V,m}=C_V/n$,摩尔等压热容 $C_{P,m}=C_P/n$.

\subsection{理想气体的等压热热容与等体热容}

根据理想气体的状态方程\upref{PVnRT},对于一定物质的量($n \rm{mol}$)的理想气体,内能 $U$ 只和温度有关,所以 \autoref{ThCapa_eq1} 的偏导数可以写为导数,即
\begin{equation}
C_V=\frac{\dd U}{\dd T}
\end{equation}

于是理想气体内能也可以写成积分表达式
\begin{equation}
U=\int C_V\dd T + U_0
\end{equation}

我们还可以求得 $C_P$ 和 $C_V$ 的关系:
\begin{equation}
\begin{aligned}
&C_P-C_V=\frac{\dd (U+pV)}{\dd T}-\frac{\dd U}{\dd T}=\frac{\dd(pV)}{\dd T}=\frac{\dd (nRT)}{\dd T}=nR\\
&C_{P,m}-C_{V,m}=R
\end{aligned}
\end{equation}

$U$ 是 $T$ 的函数,因此 $C_V$ 和 $C_P$ 都是 $T$ 的函数.现在,用 $\gamma$ 表示 $C_P/C_V$,$\gamma$ 也是 $T$ 的函数.那么有

\begin{equation}\label{ThCapa_eq4}
C_V=\frac{nR}{\gamma-1},C_P=\gamma\frac{nR}{\gamma-1}
\end{equation}

设 $i$ 为气体分子自由度数,例如单原子气体分子自由度为 $3$,而双原子分子自由度为 $5$($3$ 个平动自由度和 $2$ 个转动自由度,我们先不考虑振动).那么\textbf{通常情况}下,$C_V$ 约为 $inR/2$(这可以用能量均分定理\upref{EqEng} 来解释),于是 $C_P=(i+2)nR/2$,$\gamma=(i+2)/i$.从这一公式可知,单原子分子的 $\gamma=1.667$,双原子分子的 $\gamma=1.40$.然而在实验中观察到,双原子分子气体的 $\gamma$ 随温度的变化有明显的变化,而且更为合理的假设应该是 $i=7$(算上两个原子作简谐振动的自由度).在实验的低温情况下,气体分子的这些自由度似乎被“冻结”了.这些是经典理论无法解释的.
\addTODO{待进一步探索}
\begin{example}{$\gamma$ 与理想气体绝热过程}

理想气体在准静态绝热过程\upref{Adiab}中满足
\begin{equation}
pV^{\gamma}=\text{常量}
\end{equation}

这是因为在绝热过程中,$\dd{Q}=0$,$\dd{U}=\dd{W}$ 意味着 $C_V\dd{T}=-P\dd{V}$,所以
\begin{equation}
\dd(pV)=nR\dd{T}=C_V(\gamma-1)\dd{T}=-(\gamma-1)P\dd{V}
\end{equation}

解得 $V\dd{P}+\gamma P\dd{V} = 0$,所以 $\dd{(pV^\gamma)}=0$,即 $pV^{\gamma}$ 为常量.
\end{example}

\subsection{任意气体的热容}
将 $U$ 看成 $T,V$ 的函数,那么有

\begin{equation}
\dd U=\left(\frac{\partial U}{\partial T}\right)_V\dd T+\left(\frac{\partial U}{\partial V}\right)_T\dd V
\end{equation}

将热力学第一定律方程代入,可以得到

\begin{equation}\label{ThCapa_eq3}
\dd Q=\left(\frac{\partial U}{\partial T}\right)_V\dd T+\left(P+\left(\frac{\partial U}{\partial V}\right)_T\right)\dd V
=C_V \dd T+L_V \dd V 
\end{equation}

可以看出,当固定体积升高温度时,吸收热量为 $C_V\Delta T$,当固定温度改变体积时,吸收热量为 $L_V\Delta V$,其中 $L_V=P+(\partial U/\partial V)_T$.对于理想气体来说, $(\partial U/\partial V)_T=0$,所以固定温度时系统内能不变,系统吸收的热量等于对外做功 $P\dd V$.对于一般气体,$(\partial U/\partial V)_T$ 不为零,这来自于范德瓦尔斯力(分子间有林纳德琼斯势).例如,当分子间存在吸引势,体积增加时会导致分子平均动能减小,为了维持温度,除了要吸收 $P\dd V$ 的热量抵销对外做的功,必须要从外界吸收额外热量.此时 $(\partial U/\partial V)_T>0$.一般称 $(\partial U/\partial V)_T$ 为内压.

将 \autoref{ThCapa_eq3} 中的 $\dd V$ 换成 $\dd V=(\partial V/\partial T)_P\dd T+(\partial V/\partial P)_T\dd P$,可以得到

\begin{equation}
\dd Q=\left(C_V+L_V\left(\frac{\partial V}{\partial T}\right)_P\right)\dd T+L_V\left(\frac{\partial V}{\partial P}\right)_T\dd P
=C_V \dd T+L_V \dd V 
\end{equation}

所以在等压条件下的热容 $C_P$ 就满足关系式

\begin{equation}
C_P=C_V+L_V\left(\frac{\partial V}{\partial T}\right)_P
\end{equation}

对于理想气体,$L_V=P$,$L_V(\partial V/\partial T)_P=(\partial (pV)/\partial T)_P=(\partial (nRT)/\partial T)_P=nR$.所以有简单的关系式 $C_P=C_V+nR$.

在实验上我们无法之间测量系统的内能.我们能测的物理量有 $P,V,T,C_P,C_V$.为了能更好地检验热力学第一定律,消去式中的 $U$,我们可以利用热容的以下关系式

\begin{align}
\left(\frac{\partial U}{\partial P}\right)_V=C_V\left(\frac{\partial T}{\partial P}\right)_V,\\ 
\left(\frac{\partial U}{\partial V}\right)_P=C_P\left(\frac{\partial T}{\partial V}\right)_P-P
\end{align}
再利用 $\partial^2 U/\partial P\partial V=\partial^2 U/\partial V\partial P$,由以上两式可得
\begin{equation}
(C_P-C_V)\frac{\partial^2T}{\partial P\partial V}+\left(\frac{\partial C_P}{\partial P}\right)_P\left(\frac{\partial T}{\partial V}\right)_P-\left(\frac{\partial C_V}{\partial V}\right)_V\left(\frac{\partial T}{\partial P}\right)_V=1
\end{equation}
只要在实验上测量各个物理量,对任意处于平衡态的气体系统,都满足以上关系式,那么就成功检验了热力学第一定律.
