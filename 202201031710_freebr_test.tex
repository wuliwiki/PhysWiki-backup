% test
% test|测试|编辑器

\subsubsection{一些符号}
参考 \href{https://docs.julialang.org/en/v1/manual/unicode-input/}{Julia 符号表}和 \href{https://oeis.org/wiki/List_of_LaTeX_mathematical_symbols}{LaTeX 符号表}, 以及 \href{http://www.onemathematicalcat.org/MathJaxDocumentation/TeXSyntax.htm}{MathJax 符号表}.

\begin{equation}
\cap, \bigcap, \cup, \bigcup, \vee, \wedge, \int, \iint, \iiint, \oint
\end{equation}
\begin{equation}
\diamond, \ominus, \triangleleft, \triangleright, \Longleftarrow, \Longrightarrow, \iff, \leftrightarrow, \updownarrow, \cdots
\end{equation}
\begin{equation}
\ddots, \top, \bot, \measuredangle
\end{equation}

\begin{figure}[ht]
\centering
\includegraphics[width=5cm]{./figures/test_1.png}
\caption{小时百科} \label{test_fig1}
\end{figure}

\subsubsection{化学式}
编辑器预览的 MathJax 3 开始支持化学式了. 但是网站用的 2 还不支持. 也不确定 LaTeX 是否支持.
\begin{equation}
\ce{SO4^2- + Ba^2+ -> BaSO4 v}
\end{equation}

\subsubsection{公式中的链接}
公式里面居然可以用 \verb|\href|, 这个功能很强大, 但不知道 texlive 是否支持.
\begin{equation}
\href{https://wuli.wiki/online}{a}^2 + b^2 = c^2
\end{equation}

\subsubsection{付费内容}
我们要学习的公式为
\begin{equation}\label{test_eq1}
a = 1
\end{equation}

\begin{example}{}
请问 $a$ 为多少?
\pay

实际答案就是
\begin{equation}\label{test_eq2}
a + 1 = 2
\end{equation}
\paid
\end{example}

\begin{equation}\label{test_eq3}
b = 3
\end{equation}

相对于数字货币,各国政府发行的货币称为法币. 法币的出现最早在 1694 年的英国, 在此之前的漫长岁月中, 人类基本在使用一些天然的物质充当货币, 如早期的贝壳, 直到黄金. 直到今天, 黄金仍然承担着货币的角色, 各个国家的国库中依然存有大量的黄金作为法币的信用保证.

黄金能作为等价物是大家的共识, 但注意它没有任何政府背书或发行, 且总量基本固定. 那么它作为货币时, 价值体现在哪里呢? 简单来说就体现在它的记账功能. 如果直接以商品交换商品, 效率非常低, 因为很难配对到合适的人. 黄金或者一般等价物的使用, 相当于产生了不可篡改账本, 安全地记录每个人所拥有的资源.

黄金使交易更便捷, 从而对古代经济发展起到了重大的助推作用.为什么不能用纸和笔记账? 因为不安全——陌生人难以辨认真伪, 很可能造假, 也可能被大量增发. 可见要让大家都信任一个中立的记账系统, 它必须具有以下特点
\begin{enumerate}
\item \textbf{安全性}: 不能伪造, 不能被篡改, 不能被发行者大量增发, 不会变质. 黄金在这点上的优势无疑是巨大的, 黄金几乎无法人工合成(只有核反应才能合成极少量), 金矿勘探和开采的难度非常大, 在自然条件下可以稳定存放. 但有一个缺点就是鉴别真伪成本较高, 需要彻底熔化或者使用现代仪器.
\item \textbf{便携}: 由于黄金总量相对较少, 所以用作货币单价很高, 只需携带少量即可满足日常需求. 但这比起当代电子设备还是不够方便.
\item \textbf{转账方便}: 黄金交易在古代的确是算非常方便, 但在比起比起电子转账同样差很多.
\end{enumerate}

综上,黄金的价值就在于它利用其出色的物理特性, 在历史上相当长的时间里充当账本的角色, 极大地促进了人类社会的早期发展. 需要指出的是这个 “账本” 只记录每人的余额而不记录转账历史, 这是又一缺点.

\begin{equation}
1+1=2
\end{equation}

相对于数字货币,各国政府发行的货币称为法币. 法币的出现最早在 1694 年的英国, 在此之前的漫长岁月中, 人类基本在使用一些天然的物质充当货币, 如早期的贝壳, 直到黄金. 直到今天, 黄金仍然承担着货币的角色, 各个国家的国库中依然存有大量的黄金作为法币的信用保证.

黄金能作为等价物是大家的共识, 但注意它没有任何政府背书或发行, 且总量基本固定. 那么它作为货币时, 价值体现在哪里呢? 简单来说就体现在它的记账功能. 如果直接以商品交换商品, 效率非常低, 因为很难配对到合适的人. 黄金或者一般等价物的使用, 相当于产生了不可篡改账本, 安全地记录每个人所拥有的资源.

黄金使交易更便捷, 从而对古代经济发展起到了重大的助推作用.为什么不能用纸和笔记账? 因为不安全——陌生人难以辨认真伪, 很可能造假, 也可能被大量增发. 可见要让大家都信任一个中立的记账系统, 它必须具有以下特点
\begin{enumerate}
\item \textbf{安全性}: 不能伪造, 不能被篡改, 不能被发行者大量增发, 不会变质. 黄金在这点上的优势无疑是巨大的, 黄金几乎无法人工合成(只有核反应才能合成极少量), 金矿勘探和开采的难度非常大, 在自然条件下可以稳定存放. 但有一个缺点就是鉴别真伪成本较高, 需要彻底熔化或者使用现代仪器.
\item \textbf{便携}: 由于黄金总量相对较少, 所以用作货币单价很高, 只需携带少量即可满足日常需求. 但这比起当代电子设备还是不够方便.
\item \textbf{转账方便}: 黄金交易在古代的确是算非常方便, 但在比起比起电子转账同样差很多.
\end{enumerate}

综上,黄金的价值就在于它利用其出色的物理特性, 在历史上相当长的时间里充当账本的角色, 极大地促进了人类社会的早期发展. 需要指出的是这个 “账本” 只记录每人的余额而不记录转账历史, 这是又一缺点.

\begin{equation}
3\E>2\pi
\end{equation}


引用\autoref{test_eq1}, 以及隐藏的\autoref{test_eq2}. 注意\autoref{test_eq3} 的序号在隐藏后不会改变.

引用外部公式 \autoref{AU_eq6}~\upref{AU}
