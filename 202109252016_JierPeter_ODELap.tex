% 拉普拉斯变换与常系数线性微分方程
% Laplace Transformation|常微分方程|ODE|非齐次方程

\pentry{常系数线性微分方程\upref{ODEb2},拉普拉斯变换的性质\upref{ProLap}}

\subsection{拉普拉斯变换}

\subsubsection{导数定理}

拉普拉斯变换的导数定理(见\textbf{拉普拉斯变换}\upref{LapTra})使得我们可以把常系数线性微分方程化为代数方程,不论是否齐次.在\textbf{常系数线性微分方程}\upref{ODEb2}中,我们通过把方程写为算子$\frac{\dd }{\dd t}$的代数方程,实现了化微分方程为代数方程,但这种操作对非齐次方程没有用.拉普拉斯变换是另一种化为代数方程的思路,而且不拘泥于齐次与否.

引用\textbf{拉普拉斯变换的性质}\upref{ProLap},将导数定理誊写如下:
\begin{theorem}{拉普拉斯变换的导数定理}\label{ODELap_the1}
设$[0, +\infty]$上有$k$次可导函数$f(t)$,其拉普拉斯变换为
\begin{equation}
\mathscr{L}(f(t))=F(s)=\int^{+\infty}_0f(t)\E^{-st}\dd t
\end{equation}

则有
\begin{equation}
\mathscr{L}(\frac{\dd}{\dd t}f(t))=sF(s)-f(0)
\end{equation}
\end{theorem}

由\autoref{ODELap_the1} ,容易计算得
\begin{equation}
\begin{aligned}
\mathscr{L}(\frac{\mathrm{d}^2}{\dd t^2}f(t))&=s\mathscr{L}(\frac{\dd}{\dd t}f(t))-f'(0)\\
&=s^2F(s)-sf(0)-f'(0)
\end{aligned}
\end{equation}

以此类推,可得:

\begin{corollary}{}\label{ODELap_cor1}
\begin{equation}
\mathscr{L}(\frac{\mathrm{d}^k}{\dd t^k}f(t))=s^nF(s)-\sum_{\substack{i, j>0\\   i+j=n-1}}s^if^{(j)}(0)
\end{equation}
其中$f^{(j)}$表示$f$的$j$次导函数.



\end{corollary}




\subsubsection{函数乘以一个指数函数后的拉普拉斯变换}




\begin{theorem}{}\label{ODELap_the2}
对于$f(t)$,若设其拉普拉斯变换为$F(s)$,则
\begin{equation}
\mathscr{L}(\E^{at}f(t))=F(s-a)
\end{equation}
\end{theorem}









\subsubsection{常见函数的拉普拉斯变换}

\begin{example}{幂函数}
\begin{equation}
\mathscr{L}(t^n)=\frac{n!}{s^{n+1}}
\end{equation}

这一事实可以通过应用分部积分得证:
\begin{equation}\label{ODELap_eq2}
\begin{aligned}
\mathscr{L}(t^n)&=\int_0^{+\infty}t^n\E^{-st}\dd t\\
&=-\frac{1}{s}t^n\E^{-st}|^{+\infty}_0+\frac{n}{s}\int_0^{+\infty}t^{n-1}\E^{-st}\dd t\\
&=\frac{n}{s}\int_0^{+\infty}t^{n-1}\E^{-st}\dd t\\
&=\frac{n}{s}\mathscr{L}(t^{n-1})
\end{aligned}
\end{equation}
\end{example}

\begin{example}{指数函数}
\begin{equation}\label{ODELap_eq1}
\mathscr{L}(\E^{at})=\frac{1}{s-a}
\end{equation}

\end{example}

\begin{example}{三角函数}
利用$\E^{\I x}=\cos x+\I\sin x$,将$a=\omega\I$代入\autoref{ODELap_eq1} ,分开实部和虚部,可得三角函数的拉普拉斯变换:
\begin{equation}\label{ODELap_eq3}
\leftgroup{
    \mathscr{L}(\cos \omega t)&=\frac{s}{s^2+\omega^2}\\
    \mathscr{L}(\sin \omega t)&=\frac{\omega}{s^2+\omega^2}
}
\end{equation}
\end{example}


有了\autoref{ODELap_eq2} 、\autoref{ODELap_eq1} 和\autoref{ODELap_eq3} ,再结合\autoref{ODELap_the2} ,我们就可以计算出多数情况下的拉普拉斯变换了.



\subsection{微分方程}

拉普拉斯变换的一大实用之处,就是可以用来解\textbf{常系数线性微分方程}.我们先举一个例子来说明其解法.

\begin{example}{拉普拉斯变换解非齐次方程}
考虑方程
\begin{equation}\label{ODELap_eq4}
\qty(\frac{\mathrm{d}^2}{\dd t}+\frac{\dd}{\dd t}+2)f(t)=\frac{t^2}{2}
\end{equation}

设$\mathscr{L}(f(t))=F(s)$,那么根据\autoref{ODELap_cor1}  ,给\autoref{ODELap_eq4} 左右同时进行拉普拉斯变换,得
\begin{equation}
s^2F(s)-s
\end{equation}



\end{example}

















