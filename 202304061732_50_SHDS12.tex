% 2012年沈阳航空航天大学818/数据结构专业综合考研真题
% 2012年沈阳航空航天大学818/数据结构专业综合考研真题


一、选择题(每题2分,共30分)
1从逻辑上可以把数据结构分为( )两大类。
人动态结构、静态结构
B. 顺序结构、链式结构
C.线性结构、非线性结构
D.初等结构、构造型结构
2.在下面程序段中,对x的赋值语句的频度为()
for(k=l;k<n;k++)
for(jrk; j<en; j+)
x=x+4;
L0(2n)X2
B.0(n)
C.0ln)
D.00og田
3算法的时间复杂度与() 有关。
上问题规模
B.计算机硬件的运行速度
C.源程序的长度
D.编译后可执行程序的质量
4. 在下列关于线性表的叙述中,正确的是()
上线性表的逻辑顺序和物理顺序总是一致的,
B. 线性表的顺序存储结构优于链式存储结构
C就线性表的查找效率而言,链式存储结构比顺序存储结构高。
D. 就线性表的插入效率而言,链式存储结构比顺序存储结构高。
5. 已如循环链表的最后一个结点由ρ指针指向,若要在该结点后插入s指针指
向的新结点,则应执行下列( ) 操作。
L p->nextns; s->next=NULL;
B. p->next"s; s->next-p .
C. s->next*p->next; p->next-s;
D. s->nextep->next; p=s;
6.一个核的入找序列是a,b. c, d. e,则栈的不可能的输出序列是69 )。
A edcbe
B. decba
C. dceab
D. abede
1.若用一个大小为6的数组来实现循环队列, 且当前rear和front的值分别为
10和3, 当从队列中删除一个元素, 再加入两个元素后,rear和front的值
分别为多少? ()
上1和S
B.2和4
C.4和2
D.5和」
