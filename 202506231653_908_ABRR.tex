% 阿贝尔-鲁菲尼定理(综述)
% license CCBYSA3
% type Wiki

本文根据 CC-BY-SA 协议转载翻译自维基百科 \href{https://en.wikipedia.org/wiki/Abel\%E2\%80\%93Ruffini_theorem}{相关文章}。

在数学中,阿贝尔–鲁芬尼定理(也称为阿贝尔不可能性定理)指出:对于一般的五次及更高次数的多项式方程,无法通过根式(即有限次加减乘除和开方)求出其解。这里的“一般”是指将方程的系数看作不定元,并对其进行运算。

该定理以保罗·鲁芬尼和尼尔斯·亨里克·阿贝尔的名字命名。鲁芬尼于1799年给出了一个不完整的证明(该证明在1813年被完善,并被柯西接受),而阿贝尔在1824年提供了完整的证明。

“阿贝尔–鲁芬尼定理”也常指一个稍强的命题:存在某些五次或更高次数的方程无法用根式求解。这个结论虽不直接出现在阿贝尔的定理陈述中,但可以从他的证明推导而来——他的证明基于这样一个事实:某些由方程系数组成的多项式不是零多项式。这个更强的结论也可以直接从伽罗瓦理论中的“一个不可解的五次方程例子”中得出。

伽罗瓦理论还暗示:
$$
x^5 - x - 1 = 0~
$$
是最简单的不能用根式求解的方程。而且,几乎所有的五次及以上次数的多项式都无法用根式求解。

这种在五次及以上多项式中求解的“不可能性”,与低次数的情况形成鲜明对比:二次方程有求根公式,三次方程有三次公式,四次方程有四次公式,但五次及以上则无通解公式。
\subsection{背景}
二次多项式方程可以通过求根公式求解,这种方法自古代起就已为人所知。同样,三次方程的求解公式(三次公式)和四次方程的求解公式(四次公式)也分别在16世纪被发现。自那时起,一个根本性的问题便是:更高次数的方程是否也可以用类似的方法求解。

在 17 世纪,数学家们提出了一个重要的断言:每一个正次数的多项式方程都有解(即使该解可能是复数),但直到 19 世纪初这一断言才被彻底证明。这就是著名的代数学基本定理。不过,该定理并不提供计算这些解的具体方法,虽然我们现在知道很多数值近似解法,可以以任意精度逼近所有解。

从 16 世纪到 19 世纪初,代数学的核心问题一直是:寻找一个通用的公式,用于解五次及更高次数的多项式方程。这也就是为何“代数学基本定理”曾一度被理解为“寻找求解公式”的意思。所谓“公式”,指的是用根式表达解,即仅使用方程系数,以及加、减、乘、除和开n次根等代数运算构成的表达式。

阿贝尔–鲁菲尼定理(Abel–Ruffini 定理)证明了通解形式的根式求解是不可能的。然而,这个不可能性并不意味着所有高次方程都不能用根式求解。恰恰相反,对于任意次数,确实存在可以用根式解出的特定方程。例如,方程$x^n - 1 = 0$对于任意的 $n$,都可以通过根式求解;再如由圆分多项式定义的方程,其所有解都可以用根式表达。

阿贝尔在其定理的证明中并没有明确指出某个特定方程无法用根式解出。事实上,他定理的表述并不排除如下的可能性:“每一个具体的五次方程也许都可以用某种特别的公式解出”。\(^\text{[5]}\)也就是说,阿贝尔定理的陈述本身并不等同于我们现在熟知的“存在不可根式求解的具体方程”这个结论。

不过,从阿贝尔的证明过程来看,似乎可以推导出这样的结论:确实存在某些不能用根式解的具体方程。这是因为阿贝尔的证明利用了一个事实:某些系数的多项式不恒为零,而有限多个非零多项式在某些变量取值下可以同时不为零,从而得到无法根式解的实例。

阿贝尔发表其证明后不久,埃瓦里斯特·伽罗瓦引入了现在被称为伽罗瓦理论的理论框架,它可以判断任意一个具体多项式方程是否可以用根式求解。在电子计算机普及之前,这仍是一个纯理论的问题。如今,借助现代计算机和软件,人们可以判断次数超过 100 的多项式是否可以用根式解出。\(^\text{[6]}\)不过,即便一个方程可以用根式求解,其求解过程也可能极其复杂。哪怕是五次方程,其根式表达式往往过于庞大,几乎没有实际意义。
\subsection{证明}
阿贝尔–鲁菲尼定理的证明早于伽罗瓦理论的建立。然而,伽罗瓦理论使这一问题更加清晰易懂,因此现代的定理证明通常基于伽罗瓦理论,而阿贝尔和鲁菲尼的原始证明如今主要作为历史文献被呈现。\(^\text{[1][7][8][9]}\)

基于伽罗瓦理论的证明通常包含以下四个主要步骤:
1. 用域论描述可解方程:首先要从域扩张的角度刻画一个多项式是否可以用根式求解。
2. 利用伽罗瓦对应:通过伽罗瓦群与其对应域扩张之间的对应关系,将“可根式求解”的条件转化为伽罗瓦群是可解群。
3. 证明对称群在五次及以上不可解:证明当次数为五或更高时,对应的对称群 $S_n$ 是不可解群,因此不能表示为一系列阿贝尔群扩张。
4. 构造具有对称伽罗瓦群的多项式:指出确实存在一些五次及以上的多项式,其伽罗瓦群就是对称群 $S_n$,因此这些多项式无法用根式求解。
\subsubsection{代数解与域论}
一个多项式方程的代数解,是指使用四则运算(加、减、乘、除)和开方运算构成的表达式。可以将这样的表达式看作是一个计算过程的描述:从方程的系数出发,依次计算出一系列数值,直到得到方程的解。

在这个计算过程中,每一步都可以考虑包含目前为止所有计算结果的最小域。只有在进行n 次开方运算的步骤时,这个域才会发生变化。

因此,一个代数解对应于如下形式的一系列域的扩张:
$$
F_0 \subseteq F_1 \subseteq \cdots \subseteq F_k~
$$
并且存在元素 $x_i \in F_i$,使得:
$$
F_i = F_{i-1}(x_i) \quad \text{且} \quad x_i^{n_i} \in F_{i-1} \quad \text{其中} \quad n_i > 1~
$$
也就是说,每一步都是在前一域中添加一个 $n_i$ 次根,从而得到新的域。如果存在这样一个域链,使得 $F_k$ 包含了多项式方程的一个解,那么就说这个方程有一个代数解。

为了确保这些扩张是正规扩张,这在伽罗瓦理论中是基本前提,必须对上述域链做进一步精炼:如果某一步中的 $F_{i-1}$ 不包含所有 $n_i$ 次单位根(即复数单位根),那么我们需要引入一个扩张域 $K_i$,它是通过向 $F_{i-1}$ 添加一个原始单位根得到的,并重新定义:$F_i = K_i(x_i)$

综上所述:如果一个方程能通过根式表示其解,那么它对应着一条逐步增加的域扩张链;在这条链中,每一扩张都是一个可正规化的扩张,且其伽罗瓦群是循环群;反过来,如果能构造出这样的域链,则说明该多项式可以用根式求解。

因此,证明一个方程是否可以用根式解的关键,就是判断是否存在这样一条域链,而这正是伽罗瓦理论所提供的工具。
\subsubsection{伽罗瓦对换关系}
伽罗瓦对换关系确立了正规扩域 $F/E$ 的子扩域与其伽罗瓦群的子群之间的一一对应关系。

这一对应关系如下:给定一个域 $K$,满足 $E \subseteq K \subseteq F$,它对应的子群是$\operatorname{Gal}(F/K)$即保持 $K$ 不变的 $F$ 上自同构的集合。反过来,给定 $\operatorname{Gal}(F/E)$ 的一个子群 $H$,它对应的子域是$F^H$即所有在 $H$ 的作用下保持不变的 $F$ 中的元素所构成的子域。

结合前一节的结论:一个多项式方程可以用根式解,当且仅当其分裂域(包含该方程所有根的最小扩域)的伽罗瓦群是可解群。一个群 $G$ 被称为“可解的”,如果它有一个子群链:$G = G_0 \triangleright G_1 \triangleright \cdots \triangleright G_k = \{1\}$其中每一个子群 $G_{i+1}$ 在 $G_i$ 中是正规子群,且对应的商群 $G_i / G_{i+1}$ 是循环群(或更常见地,阿贝尔群,两者在有限群的情况下是等价的——由有限阿贝尔群基本结构定理保证)。

因此,为了证明 Abel–Ruffini 定理,只需完成两个关键步骤:1. 证明对称群 $S_5$ 不是可解群;2. 证明确实存在伽罗瓦群为 $S_5$ 的多项式。

这两点结合在一起,说明了某些五次及以上的多项式方程,其解无法用根式表示,从而完成 Abel–Ruffini 不可解性定理的证明。
