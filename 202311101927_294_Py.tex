% Python 基础
% keys 基础
% license Xiao
% type Tutor

\subsection{标识符}

\begin{enumerate}
\item 标识符是指在使用Python的过程中定义名字时的一种规范,主要包括以下几种:
\begin{itemize}
\item 开头不可以是数字
\item 必须由数字,字母,下划线组成
\item 区分大小写
\item 不能使用内置关键词(比如:print,int,float等等)
\end{itemize}
\item 在Python中,有一些常见的命名规则和约定,这些规则有助于编写可读性强且一致的代码。以下是一些常见的Python命名规则:
\begin{itemize}
\item
1.变量名和函数名:\\
2.使用小写字母。\\
3.使用下划线来分隔单词,这被称为蛇形命名法。my_test)。\\
4.选择具有描述性的名称,以便代码的含义清晰可见。(my_name)

\item 常量名:\\
1.使用大写字母。\\
2.如果需要,可以使用下划线来分隔单词。\\
3.常量通常在模块级别定义。(PI, MAX_VALUE)\\
\item 类名:\\
1.使用驼峰命名法。(CamelCase)\\
2.每个单词的首字母都大写,不使用下划线。\\
3.例如:MyClass, PersonInfo, HttpRequest.
\end{itemize}
\end{enumerate}