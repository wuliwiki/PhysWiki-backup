% Möbius函数(数论)
% 莫比乌斯函数|初等数论|莫比乌斯反演公式|Möbius反演公式

\pentry{数论函数\upref{NumFun}}

德国数学家August Ferdinand Möbius于1832年提出Möbius函数的概念,这是数论中一个重要的积性函数.Möbius函数在初等数论和解析数论中随处可见,多以Möbius反演的形式出现.


任何正整数都可以唯一地分解为其质因数的幂的乘积.比如说,$24=2^3\times 3$,$300=2^3\times 3\times 5^2$.一般地,我们把整数的质因数分解记为$\prod_{k=1}^r p_k^{f_k}$,其中各$p_k$是互不相等的素数,各$f_k$都是正整数.Möbius的概念正是建立在正整数质因数分解上的:


\begin{definition}{Möbius函数}
对于任意正整数$n=\prod_{k=1}^r p_k^{f_k}$,其Möbius函数$\mu:\mathbb{N}\to\{-1, 0, 1\}$定义为:
\begin{equation}
\mu(n)=\mu(\prod_{k=1}^r p_k^{f_k})=
\leftgroup{
    1, \quad\text{如果}n=1\\
    (-1)^r, \quad\text{如果}f_k=1\text{恒成立}\\
    0, \quad\text{如果有一个}f_k>1
}
\end{equation}



\end{definition}
























