% 高能物理
% license CCBYSA3
% type Wiki

(本文根据 CC-BY-SA 协议转载自原搜狗科学百科对英文维基百科的翻译)

粒子物理(也称为高能物理学)是物理学的一个分支,主要研究构成物质和辐射的粒子的性质。虽然“粒子”一词可以指代各种类型的微小物体(例如质子、气体粒子,甚至家里的灰尘),但粒子物理通常研究不可再分割的可探测的最小粒子(“基本粒子”)及其相互作用。根据目前的理解,这些“基本粒子”都是量子场的激发,它们之间的相互作用也受到量子场的支配。而解释这些“基本粒子”,量子场及其动力学的主流理论叫做标准模型。现代粒子物理通常研究标准模型及其各种可能的扩展,这些扩展通常会涉及到最新发现的希格斯玻色子,甚至众所皆知的力场——引力。[1][2]

\subsection{亚原子粒子}
\begin{figure}[ht]
\centering
\includegraphics[width=10cm]{./figures/2cba813a245660be.png}
\caption{物理学标准模型的粒子含量} \label{fig_GNWL_1}
\end{figure}

现代粒子物理研究的重点是亚原子粒子,包括原子的组分如电子、质子和中子(质子和中子是复合粒子,称为重子,由夸克组成)等。这些亚原子粒子(如光子、中微子、μ子,以及各种奇异粒子)一般通过辐射,放射或散射过程产生。粒子的动力学受到量子力学的支配,具有波粒二象性,例如它们在某些实验条件下表现出类似于粒子的行为,而在另一些实验条件下则表现出类似于波的行为。用更专业的术语来说,这些粒子的状态是用希尔伯特空间中的量子态向量来描写的,当然在量子场论中也是如此。按照粒子物理学家们的惯例,“基本粒子”一词仅适用于那些就目前所知不可再分割的,不是由其它粒子构成的粒子[3]。

\begin{table}[ht]
\centering
\caption{\textbf{基本粒子}}\label{tab_GNWL}
\begin{tabular}{|c|c|c|c|c|c}
\hline
 &\textbf{类型}&\textbf{世代}&\textbf{反粒子}&\textbf{颜色}& \textbf{总数} \\
\hline
\textbf{夸克}& 2 & 3 & 一对 & 3 & 36 \\
\hline
\textbf{轻子}& 2 & 3 & 一对 & 无 & 12 \\
\hline
\textbf{胶子}& 1 & 1 & 自身 & 8 & 8 \\
\hline
\textbf{光子}& 1 & 1 & 自身 & 无 & 1 \\
\hline
\textbf{Z玻色子}& 1 & 1 & 自身 & 无 & 1 \\
\hline
\textbf{W玻色子}& 1 & 1 & 一对 & 无 & 2 \\
\hline
\textbf{希格斯}& 1 & 1 & 自身 & 无 & 1 \\
\hline
\textbf{(已知)基本粒子的总数}: & & & & &61\\
\hline
\end{tabular}
\end{table}
迄今为止,几乎所有发现的粒子及其相互作用都可用被称为标准模型的量子场论来描述。[4]在当前的标准模型中,有61个“基本粒子”。[3]这些“基本粒子”可以结合形成复合粒子,这解释了自20世纪60年代以来发现的数百种复合粒子。

标准模型的预言几乎与迄今为止所进行的实验都一致。然而,大多数粒子物理学家们认为它并非一个完整描写自然的理论,应该还存在一个更基本的理论(万物理论)有待发现。而近年来中微子质量的实验测量值也首次表现出与标准模型预言的偏差。

\subsection{历史}
一切物质均由“基本粒子”组成的观点至少可追溯到公元前6世纪。[5]在19世纪,约翰·道尔顿(John Dalton)通过对化学计量的研究得出结论,自然界的每一种元素都是由一种独特的粒子组成的。[6]原子(atom)一词来源于希腊单词atomos,为“不可再分割”的意思,在当时表示一种化学元素的最小粒子。然而不久,物理学家们发现,原子实际上并不是自然界的基本粒子,它是由更小粒子(例如,电子等)构成的集合。20世纪初对核物理和量子物理的探索促使莉泽·迈特纳( Lise Meitner)在1939年证明了核裂变(基于奥托·哈恩(Otto Hahn)的实验),以及同年汉斯·贝特(Hans Bethe)证明了核聚变。这两项发现直接催生了核武器的发展。20世纪50年代至60年代,在能量不断增加的粒子对撞实验中发现了各种令人困惑的粒子,它们被非正式地称为“粒子动物园”。这个称呼在20世纪70年代标准模型提出来以后被摒弃。因为按照标准模型,之前所发现的大量粒子实际是由相对少量而更基本粒子组成的复合粒子。

\subsection{标准模型}
目前所发现的所有“基本粒子”都可用标准模型进行分类。历史上,20世纪70年代中期,在实验上证实了夸克的存在之后,标准模型才开始得到广泛的接受。在标准模型中,利用中间规范玻色子来描写强、弱和电磁相互作用。规范玻色子有:八个胶子、W±和Z中间玻色子以及光子。[4]此外,标准模型还包含24个基本费米子(考虑对应的反粒子),它们构成了所有的物质。[7]最后,标准模型也预言了一种被称为“希格斯玻色子”的玻色子。2012年7月4日清晨,欧洲核子研究中心(CERN)大型强子对撞机(LHC)的物理学家们宣布,他们发现了一种新粒子,其行为与希格斯玻色子相似。[8]

\subsection{实验室}
世界上主要的粒子物理实验室有:
\begin{itemize}
\item 布鲁克海文国家实验室(BNL,美国长岛)。其主要利用相对论重离子对撞机(RHIC)对撞金离子等重离子和极化质子。它是世界上第一台重离子对撞机,也是世界上唯一的极化质子对撞机。[9]
\item 布德克核物理研究所(BINP,俄罗斯新西伯利亚)。它现在的主要设备是自2006年开始运行的VEPP-2000[10]正负电子对撞机和1994年开始运行的VEPP-4对撞机。[11]其早期的设备包括:1964年至1968年运行的第一台电子束对撞机VEP-1,1965年至1974年运行的VEPP-2正负电子对撞机及1974年至2000年运行的VEPP-2M[12]。[13]
\item 欧洲核子研究中心(CERN,法国-瑞士边境,日内瓦附近)。它现在的主要设备是自2008年9月10日进行第一次束流循环的大型强子对撞机(LHC),这也是目前世界上能量最高的质子对撞机。在它开始对撞铅离子之后,也成为了能量最高的重离子对撞机。该中心早期的设施包括:大型正负电子对撞机(LEP),该对撞机于2000年11月2日停止运行,随后被拆除,为LHC让路;以及超级质子同步加速器,它被重新用作LHC和固定目标实验的预加速器。[14]
\item 德国电子同步加速器(DESY,德国汉堡)。它的主要设备是强子电子环形加速器(HERA),它将电子和正电子与质子碰撞。[15]加速器复合体现在的重点是与 PETRA III、FLASH和欧洲XFEL进行同步辐射的生产。
\item 费米国家加速器实验室 (Fermilab,美国伊利诺州巴达维亚)。直到2011年,它的主要设备是Tevatron,对撞质子和反质子,曾经是地球上能量最高的粒子对撞机,直到2009年11月29日LHC超过它。[16]
\item 高能物理研究所(IHEP,中国北京)。它管理着中国的一些主要粒子物理设备,包括北京正负电子对撞机二号(BEPC II)、北京谱仪(BES)、北京同步辐射设施(BSRF)、西藏羊八井国际宇宙射线天文台、大亚湾反应堆中微子实验、中国散裂中子源、硬X光调制望远镜(HXMT)、加速器驱动的亚临界系统(ADS) 以及江门地下中微子观测站(JUNO)。[17]
\item KEK(日本筑波)。这里进行着多项粒子物理实验,例如K2K实验、中微子振荡实验和Belle实验。其中,Belle实验是测量B介子系统CP破坏的实验。[18]
\item SLAC国家加速器实验室(美国门洛帕克)。它2英里长的线性加速器于1962年开始运行。也是从那时起,线性加速器被用于直线加速器相干光源X光激光器及先进的加速器设计研究。直到2008年,SLAC线性粒子加速器实验仍是一个基于诸多电子和正电子对撞的实验。将来,SLAC的工作人员会继续参与开发和建造世界各地的粒子探测器。[19]
\end{itemize}
当然,还有一些其他的粒子加速器存在。

现代实验粒子物理需要的技术非常多样和复杂,已经成为了一个与该领域理论方面几乎完全不同的分支。

\subsection{理论}
理论粒子物理试图通过发展模型、理论框架和数学工具来解释当前的实验并对未来的实验做出预测(见理论物理学)。目前理论粒子物理主要致力于以下几个相互关联的方面。

第一,更好地理解标准模型及对其进行更精准的检验。例如,通过从不确定性较小的实验中提取标准模型的参数,探索标准模型的局限性,进而加深我们对自然的理解。然而,在量子色动力学中,由于计算方面的困难,给进一步的研究带来了不小挑战。从事这一领域工作的理论家被称为现象学家(唯象学家),他们需要以量子场论和有效场论作为研究工具。其他一些利用格点场论进行研究的理论家称为格点理论家。

第二,在更高能标或更小尺度,发展超越标准模型的新物理。这项工作通常以“层级问题”作为出发点,并利用现有的实验数据对模型进行限制。研究中可能涉及到超对称、希格斯机制的替代物、额外维(如Randall-Sundrum模型)、Preon理论以及这些理论的组合等。

第三,弦理论。弦理论学家试图通过建立基于小弦和膜而不是粒子的理论来构建量子力学和广义相对论的统一描述。如果这个理论成功了,它可能被认为是“万物理论”。

除了这几方面,与理论粒子物理相关的还有,粒子宇宙学和圈量子引力等。

粒子物理的这种分工反映在预印本arXiv上的类别名称为:[20]hep-th(理论),hep-ph(现象学或唯象学),hep-ex(实验),hep-lat(格点规范理论)。

\subsection{实际应用}
原则上,所有的物理学(以及由此发展起来的实际应用)都可来源于“基本粒子”的研究。实际上,即使“粒子物理”仅指“高能原子破碎机”,在这些开创性的研究中也发展了很多技术,这些技术后来在社会中得到了广泛应用。粒子加速器可产生用于研究和治疗的医用同位素(例如,用于正电子发射断层成像的同位素),或者直接用于外放射治疗;超导体在粒子物理中的应用推动了超导体的发展;万维网和触摸屏技术最初是在欧洲核子研究中心开发的。除此之外,粒子物理在医学、国家安全、工业、计算、科学和劳动力发展等越来越多的领域也有实际应用。[21]

\subsection{未来发展}
现有的标准模型不能解释天文观测发现的暗物质和实验测得的中微子小质量。这些超出标准模型的物理现象表明可能存在新的物理学。根据相关理论,这种新物理学应该就在可触及的能标。

寻找这种新物理学的大部分努力都集中在新的对撞机实验上。于2008年建成的大型强子对撞机(LHC),便是用于寻找希格斯玻色子、超对称粒子和其他新物理。此外,筹备建造中的国际直线对撞机(ILC)也打也算对新发现粒子的性质进行更精确的测量,进而作为对LHC实验的补充。2004年8月,人们确定了建造ILC的相关技术,然而它的具体建造地点仍有待商定。

除了对撞机实验外,非对撞机实验对发现和理解标准模型之外的物理也具有重要意义。其中,一项很重要的非对撞机实验是确定中微子质量,这些质量可能来自中微子与非常重的粒子的混合。此外,尽管没有对撞机实验可能无法确定暗物质的确切性质,但宇宙学观测仍可对暗物质提供很多有用的限制。最后,通过非对撞机实验测量质子长寿命的下限,也可对大统一理论进行限制,而相关的能标远高于现有的对撞机实验。

2014年5月,粒子物理优先项目专门组发布了关于未来十年美国粒子物理优先资助事项的报告。这份报告强调了美国继续参与LHC和ILC,扩大地下中微子实验,以及其它方面的建议。

\subsection{高能物理与低能物理的比较}
高能物理这个术语需要特别说明。直觉上,将“高能”与微观粒子(如亚原子粒子)相关的物理学联系起来似乎是不正确的。拿一个宏观系统的例子来说明,例如一克氢,其质量约为一个质子质量的$6\times10^{23}$倍。[22]而即使对LHC循环的整个质子束(约$3.23\times10^{14}$个质子,[23]每个质子具有$6.5\times10^{12}$ eV的能量)来说,其总束能量约为$2.1\times10^{27}$ eV或约336.4 MJ,仍比一克氢的质量(能量)低约$2.7\times10^{5}$倍。然而,前者对应的是“低能物理”,而后者对应的是“高能物理”。

事实上,在其他物理和科学领域,所研究的相互作用能量相对很低。例如,可见光的光子能量约为1.8 eV~3.1 eV。碳-碳键的键离解能约为3.6 eV,其它的化学反应通常也具有类似大小的能量。即便是能量高得多的光子,如放射性衰变产生的伽马射线(大部分光子能量在$10^{5} $eV~$10^{7}$eV 之间),其光子能量仍然比单个质子的质量低两个数量级。事实上,放射性衰变伽马射线属于是核物理的一部分,而非高能物理。

质子的质量约为$9.4\times10^{8}$ eV。对于其他的“基本粒子”和强子,也具有很高的质量(能量)。由于单个粒子的能量非常高,因此粒子物理属于高能物理。

\subsection{参考文献}
[1]
^"The Higgs Boson". CERN..

[2]
^"The BEH-Mechanism, Interactions with Short Range Forces and Scalar Particles" (PDF). 8 October 2013..

[3]
^Braibant, S.; Giacomelli, G.; Spurio, M. (2009). Particles and Fundamental Interactions: An Introduction to Particle Physics. Springer. pp. 313–314. ISBN 978-94-007-2463-1..

[4]
^"Particle Physics and Astrophysics Research". The Henryk Niewodniczanski Institute of Nuclear Physics. Retrieved 31 May 2012..

[5]
^"Fundamentals of Physics and Nuclear Physics" (PDF). Archived from the original (PDF) on 2 October 2012. Retrieved 21 July 2012..

[6]
^"Scientific Explorer: Quasiparticles". Sciexplorer.blogspot.com. 22 May 2012. Retrieved 21 July 2012..

[7]
^Nakamura, K (1 July 2010). "Review of Particle Physics". Journal of Physics G: Nuclear and Particle Physics. 37 (7A): 075021. Bibcode:2010JPhG...37g5021N. doi:10.1088/0954-3899/37/7A/075021..

[8]
^Mann, Adam (28 March 2013). "Newly Discovered Particle Appears to Be Long-Awaited Higgs Boson". Wired Science. Retrieved 6 February 2014..

[9]
^"Brookhaven National Laboratory – A Passion for Discovery". Bnl.gov. 
Retrieved 23 June 2012..

[10]
^"index". Vepp2k.inp.nsk.su. Retrieved 21 July 2012..

[11]
^"The VEPP-4 accelerating-storage complex". V4.inp.nsk.su. Retrieved 21 July 2012..

[12]
^"VEPP-2M collider complex" (in 俄语). Inp.nsk.su. Retrieved 21 July 2012..

[13]
^"The Budker Institute Of Nuclear Physics". English Russia. 21 January 2012. Retrieved 23 June 2012..

[14]
^"Welcome to". Info.cern.ch. Retrieved 23 June 2012..

[15]
^"Germany's largest accelerator centre". Deutsches Elektronen-Synchrotron DESY. Retrieved 23 June 2012..

[16]
^"Fermilab | Home". Fnal.gov. Retrieved 23 June 2012..

[17]
^"IHEP | Home". ihep.ac.cn. Archived from the original on 1 February 2016. Retrieved 29 November 2015..

[18]
^"Kek | High Energy Accelerator Research Organization". Legacy.kek.jp. Archived from the original on 21 June 2012. Retrieved 23 June 2012..

[19]
^"SLAC National Accelerator Laboratory Home Page". Retrieved 19 February 2015..

[20]
^"arXiv.org e-Print archive"..

[21]
^"Fermilab | Science at Fermilab | Benefits to Society". Fnal.gov. Retrieved 23 June 2012..

[22]
^“CODATA值:阿伏伽德罗常数”。常数、单位和不确定性的NIST参考。美国国家标准和技术研究所。2015年6月。检索于2016年12月10日。.

[23]
^"Beam Requirements and Fundamental Choices" (PDF). CERN Engineering & Equipment Data Management Service (EDMS). Retrieved 10 December 2016..