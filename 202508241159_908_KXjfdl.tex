% 柯西积分定理(综述)
% license CCBYSA3
% type Wiki

本文根据 CC-BY-SA 协议转载翻译自维基百科\href{https://en.wikipedia.org/wiki/Cauchy\%27s_integral_theorem}{相关文章}。

在数学中,柯西积分定理(也称为柯西–古尔萨定理)是复分析中的一个重要结论,以 奥古斯丁–路易·柯西(和爱德华·古尔萨的名字命名。该定理描述了复平面上全纯函数的路径积分性质。其核心内容是:如果函数$f(z)$在一个单连通域$\Omega$ 内是全纯的,那么对于 $\Omega$ 内的任何闭合路径$C$,沿着该路径的积分都为零:
$$
\int_{C} f(z)\, dz = 0.~
$$
\subsection{命题}
\subsubsection{复线积分的基本定理}
如果函数 $f(z)$ 在某个开区域 $U$ 上是全纯函数,且曲线 $\gamma$ 位于该区域内,从点 $z_0$ 延伸到点 $z_1$,则有:
$$
\int_{\gamma} f'(z)\,dz = f(z_1) - f(z_0).~
$$
此外,如果 $f(z)$ 在开区域 $U$ 内存在一个单值原函数,那么在该区域内,路径积分$\int_{\gamma} f(z)\,dz$对于所有路径来说都是路径无关的。
