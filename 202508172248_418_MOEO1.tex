% 分子轨道理论
% keys 分子轨道理论
% license Usr
% type Wiki

分子轨道是指,由于原子核足够接近后,原子轨道重合,各原子的轨道线性组合而成的轨道。成键遵循以下三大原则:
\begin{enumerate}
\item 对称性匹配:原子轨道(s,p,d,f...)关于点、线、面等的对称性在形成分子轨道的时候(尽量)不被破坏;
\item 能量接近:能量相近的原子轨道更容易形成分子轨道;
\item 最大重叠:轨道重叠的程度越大,最终形成的分子轨道能量越低。
\end{enumerate}

形成的线性组合轨道有以下三种:
\begin{itemize}
\item 总能量降低,使得分子更稳定的轨道,称为\textbf{成键轨道}(例如二原子分子中对应 $\psi = \psi_1 + \psi_2$);
\item 总能量升高,使得分子更不稳定,有排斥的交换作用,称为\textbf{反键轨道}(二原子分子中对应 $\psi = \psi_1 - \psi_2$);
\item 能量不变,称为\textbf{非键轨道}。
\end{itemize}



