% 稀疏矩阵
% keys 系数矩阵|数据结构|计算物理|数值计算

\textbf{稀疏矩阵(Sparse Matrix)}有不同的储存方式(数据结构), 这里介绍几种常见的\footnote{参考 Wikipedia \href{https://en.wikipedia.org/wiki/Sparse_matrix}{相关页面}.}. 本文的数组索引从 0 开始.

\subsection{Banded}
Banded 矩阵只储存矩阵主对角线上下的若干条对角线, 上带宽和下带宽分别指定主对角线上面和下面有几条对角线, 例如三对角矩阵的上带宽和下带宽都是 1. 带内即使有矩阵元为零也必须储存. 这样就可以按照 row major 或者 column major 来储存.

\subsection{Coordinate List (COO)}
COO 格式列出非零矩阵元和对应的行标列标. 通常将它们储存为三个数组 \verb|a|, \verb|ia|, \verb|ja|, 顺序任意. 除此之外, 有时还需要储存三个数组的长度 \verb|nnz| (none zero) 以及矩阵的尺寸.

\subsection{Compressed Sparse Row (CSR)}
也叫 Compressed Row Storage (CRS), 这种格式做矩阵与矢量相乘较快.

CRS 格式储存为三个一维数组 \verb|a|, \verb|ia|, \verb|ja|, 想象 COO 格式的矩阵元按照行主序排列后储存, 那么行标将会出现类似 \verb|0,0,1,1,1,1,2,2,3,3,3| 这样的重复. 所以为了提高效率可以把行标矩阵 \verb|ia| 的信息压缩, 令 \verb|ia[i]| 表示第 \verb|i| 行上方所有行的矩阵元个数, \verb|ia| 的长度是矩阵行数加一. 所以 \verb|ia[0]| 恒为零, 且最后一个元素就是非零元的个数 \verb|Nnz|. 第 \verb|i| 行矩阵元从 \verb|a[ia[i]]| 一直到 \verb|a[ia[i+1]-1]|. 注意如果矩阵的前几行都是空的, 那么 \verb|ia| 的前几个都是零. \verb|ja| 是对应矩阵元的列标

\subsection{Compressed Sparse Column (CSC)}
也叫 Compressed Column Storage (CCS), 与 CRS 一样, 只是改为 column major.
