% 直线和平面的交点

\pentry{高中解析立体几何, 几何矢量\upref{GVec}}

若平面上任意一点为 $\bvec p = (p_x, p_y, p_z)$, 法向量为 $\bvec n = (n_x, n_y, n_z)$. 直线上一点为 $\bvec s = (s_x, s_y, s_z)$, 方向为 $\bvec v = (v_x, v_y, v_z)$, 求射线与平面的交点. 注意 $\bvec n$ 和 $\bvec v$ 不必是单位矢量.

平面方程为
\begin{equation}\label{LPint_eq2}
(\bvec r - \bvec p) \vdot \bvec n = 0
\end{equation}
直线的参数方程为
\begin{equation}\label{LPint_eq1}
\bvec r = \lambda \bvec v + \bvec s
\end{equation}
\autoref{LPint_eq1} 带入\autoref{LPint_eq2} 解得
\begin{equation}
\lambda = \frac{(\bvec p - \bvec s)\vdot \bvec n}{\bvec v \vdot \bvec n}
\end{equation}
再带入\autoref{LPint_eq1} 得交点为
\begin{equation}\label{LPint_eq3}
\bvec r = \frac{(\bvec p - \bvec s)\vdot \bvec n}{\bvec v \vdot \bvec n} \bvec v + \bvec s
\end{equation}

(未完成: 给出 Matlab 代码)
