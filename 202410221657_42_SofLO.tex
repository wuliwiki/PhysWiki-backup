% 线性算子的谱
% keys 谱|线性算子
% license Usr
% type Tutor

\pentry{拓扑线性空间中的线性算子\nref{nod_TLinO}}{nod_085d}
\cite{Ke1}在算子理论中,谱的概念或许是最重要的概念。在提到谱的地方,我们都假定算子作用域复空间。

回顾有限维空间的情形。设 $A$ 是 $n$ 维空间 $\mathbb C^n$ 的线性算子,如果方程
\begin{equation}
Ax=\lambda x~
\end{equation}
有非零解,则数 $\lambda$ 称为算子 $A$ 的\enref{\enref{本征值}{LiOper}}{eign}。所有本征值的总体称为算子的\textbf{谱}。而其它的 $\lambda$ 值称为\textbf{正则点}。即 $\lambda$ 是正则点,当且仅当算子 $(A-\lambda I)$ 可逆。此时 $(A-\lambda I)^{-1}$ 定义在整个 $\mathbb C^n$ 上,并且作为有限为空间的任意算子,它是有界的。于是,在有限维空间中有两种可能:
\begin{enumerate}
\item 方程 $Ax=\lambda x$ 有非零解,$\lambda$ 是 $A$ 的本征值,这时算子 $(A-\lambda I)^{-1}$ 不存在;
\item 存在定义在整个空间上的算子 $(A-\lambda I)^{-1}$,即 $\lambda$ 时正则点。
\end{enumerate}

但是在无限维空间中,有第三种可能:3)算子 $(A-\lambda I)^{-1}$ 存在,但不是定义在整个空间上的,且可能无界。
\subsection{基本定义}
无穷维空间里,本征值、谱和正则点的基本概念几乎是一样的。为正式起见,这里给出基本的定义。一般线性拓扑空间中第三种可能的存在将导致一些本质性区别。
\begin{definition}{本征值,谱,正则点,预解式}
设 $A$ 是任意线性拓扑空间中的线性算子(定义空间和象空间一样),则称 $(A-\lambda I)^{-1}$ 为算子 $A$ 的\textbf{预解式}。若 $A$ 的预解式 $(A-\lambda I)^{-1}$ 存在且有界、并定义在整个空间上,则称 $\lambda$ 对 $A$ 是\textbf{正则的}。所有其余的值 $\lambda$ 称为 $A$ 的\textbf{谱}。使 $Ax=\lambda x$ 有非零矢量解的数 $\lambda$ 称为算子 $A$ 的\textbf{本征值}。
\end{definition}

\begin{theorem}{}
本征值属于谱。
\end{theorem}

\textbf{证明:}若 $\lambda$ 是本征值,则预解式不存在,即 $\lambda$ 不对线性算子正则,因而属于谱。

\textbf{证毕!}

该定理引入了下面的概念。
\begin{definition}{点谱}
本征值的总体称为\textbf{点谱}。
\end{definition}

对完备赋范空间,由\autoref{the_InvOp_1},定义在全空间上的预解式是有界的。


由定义,对连续谱中的 $\lambda$ ,$(A-\lambda I)^{-1}$ 存在,但不是定义在整个空间上的,并且可能无界。这是无限维和有限维空间的本质区别。

