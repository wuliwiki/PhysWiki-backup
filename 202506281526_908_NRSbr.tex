% 尼尔斯·玻尔(综述)
% license CCBYSA3
% type Wiki

本文根据 CC-BY-SA 协议转载翻译自维基百科\href{https://en.wikipedia.org/wiki/Niels_Bohr}{相关文章}。

\begin{figure}[ht]
\centering
\includegraphics[width=6cm]{./figures/2f702d956c83f882.png}
\caption{1922 年的玻尔} \label{fig_NRSbr_3}
\end{figure}
尼尔斯·亨里克·戴维·玻尔(Niels Henrik David Bohr,美国发音:/boʊr/,英国发音:/bɔː/,\(^\text{[2]}\)丹麦语:[ˈne̝ls ˈpoɐˀ];1885年10月7日-1962年11月18日)是一位丹麦理论物理学家,他在原子结构和量子理论的理解方面作出了奠基性贡献,并因此于1922年获得诺贝尔物理学奖。玻尔同时也是一位哲学家和科学研究的推动者。

玻尔提出了著名的玻尔原子模型,他提出电子的能级是离散的,电子围绕原子核在稳定轨道上运行,但可以从一个能级(或轨道)跃迁到另一个能级(或轨道)。尽管玻尔模型已被其他模型取代,但其基本原理仍然有效。他提出了互补性原理:事物可以从相互矛盾的属性中被分别分析,例如表现为波动或粒子流的行为。这一互补性概念贯穿了玻尔在科学和哲学领域的思考。

玻尔在哥本哈根大学创立了理论物理研究所(现称尼尔斯·玻尔研究所),该研究所于1920年开放。玻尔指导并与多位物理学家合作,包括汉斯·克拉默斯、奥斯卡·克莱因、乔治·德·赫维希和沃尔夫冈·海森堡。他预测了一种类似锆的新元素的特性,该元素因在哥本哈根被发现而以哥本哈根的拉丁名称命名为“铪”。后来,合成元素“𬬻”因玻尔在原子结构领域的开创性工作而以他的名字命名。

在20世纪30年代,玻尔帮助了逃离纳粹主义的难民。丹麦被德国占领后,他会见了已成为德国核武器项目负责人的海森堡。1943年9月,玻尔得知德国人即将逮捕他,于是他逃往瑞典。从那里,他被空运到英国,加入了英国的“合金管”核武器项目,并作为英国代表团成员参与了曼哈顿计划。战争结束后,玻尔呼吁在核能领域开展国际合作。他参与了欧洲核子研究中心(CERN)和丹麦原子能委员会下属的里瑟研究机构的建立,并于1957年成为北欧理论物理研究所的首任主席。
\subsection{早年生活}
尼尔斯·亨里克·戴维·玻尔于1885年10月7日出生在丹麦哥本哈根,是克里斯蒂安·玻尔和妻子埃伦(娘家姓阿德勒,Ellen née Adler)的三个孩子中的老二。\(^\text{[3][4]}\)其父克里斯蒂安是哥本哈根大学的生理学教授,母亲埃伦出身于一个富裕的犹太银行世家。\(^\text{[5]}\)他有一个姐姐珍妮和一个弟弟哈拉尔。\(^\text{[3]}\)珍妮后来成为教师,\(^\text{[4]}\)而哈拉尔成为数学家和足球运动员,曾代表丹麦国家队参加1908年在伦敦举行的夏季奥运会。尼尔斯本人也是一名热情的足球运动员,两兄弟曾一起为位于哥本哈根的学术足球俱乐部效力,尼尔斯担任守门员。\(^\text{[6]}\)

玻尔七岁时进入加梅尔霍姆拉丁学校就读。\(^\text{[7]}\)1903年,玻尔进入哥本哈根大学本科就读,主修物理学,师从当时该校唯一的物理学教授克里斯蒂安·克里斯蒂安森。此外,他还在托瓦尔·蒂勒教授指导下学习天文学和数学,并在其父的朋友哈拉尔·霍夫丁教授指导下学习哲学。\(^\text{[8][9]}\)
\begin{figure}[ht]
\centering
\includegraphics[width=6cm]{./figures/3736fdffe67b90d2.png}
\caption{年轻时的玻尔} \label{fig_NRSbr_1}
\end{figure}
1905 年,丹麦皇家科学院举办了一项金质奖章竞赛,题目是研究测量液体表面张力的方法,该方法最初由瑞利勋爵于 1879 年提出。这项研究需要测量水射流半径振动的频率。玻尔在大学里利用他父亲的实验室进行了一系列实验;当时大学本身并没有物理实验室。为了完成实验,他不得不自己制作玻璃器皿,吹制出具有所需椭圆形横截面的试管。他不仅完成了原先的任务,还在瑞利的理论和方法上进行了改进,他考虑了水的黏滞性,并使用有限振幅而非仅限于无穷小振幅进行实验。他在最后一刻提交的论文赢得了这项奖章。他随后将改进后的论文提交给伦敦皇家学会,在《皇家学会哲学汇刊》上发表。\(^\text{[10][11][9][12]}\)

哈拉尔德是玻尔兄弟中第一个获得硕士学位的人,他于 1909 年 4 月获得数学硕士学位。尼尔斯又花了九个月时间,于同年完成了关于金属电子理论的硕士论文,这一课题是由他的导师克里斯琴森布置的。随后,玻尔将硕士论文扩展成了篇幅更大的博士论文。他调研了该领域的文献,最终选择了保罗·德鲁德提出并由亨德里克·洛伦兹完善的模型,该模型认为金属中的电子表现得像气体一样。玻尔在洛伦兹模型的基础上进行了扩展,但仍无法解释霍尔效应等现象,最终他得出结论:电子理论无法完全解释金属的磁性特性。论文于 1911 年 4 月被接收,\(^\text{[13]}\)玻尔于 5 月 13 日进行了正式答辩。哈拉尔德在前一年已获得博士学位。\(^\text{[14]}\)

玻尔的论文具有开创性,但由于当时哥本哈根大学要求论文必须用丹麦语撰写,因此在斯堪的纳维亚以外地区鲜有人关注。1921 年,荷兰物理学家亨德里卡·约翰娜·范·李文独立推导出了玻尔论文中的一个定理,今天被称为玻尔–范·李文定理。\(^\text{[15]}\)
\begin{figure}[ht]
\centering
\includegraphics[width=6cm]{./figures/3f69c50087df3760.png}
\caption{玻尔与玛格丽特·讷鲁恩于 1910 年订婚时合影} \label{fig_NRSbr_2}
\end{figure}
1910 年,玻尔结识了数学家尼尔斯·埃里克·讷鲁恩的妹妹玛格丽特·讷鲁恩。\(^\text{[16]}\)玻尔于 1912 年 4 月 16 日退出丹麦国教会,并于同年 8 月 1 日在斯莱厄瑟市政厅与玛格丽特举行了民事婚礼。多年后,他的弟弟哈拉尔德在结婚前也同样退出了教会。\(^\text{[17]}\)

玻尔和玛格丽特育有六个儿子。\(^\text{[18]}\)长子克里斯蒂安于 1934 年在一次划船事故中去世,\(^\text{[19]}\)另一位儿子哈拉尔德有严重智力障碍,在四岁时被送到离家较远的机构安置,六年后因儿童脑膜炎去世。\(^\text{[20][18]}\)阿格·玻尔成为了一名成功的物理学家,并于 1975 年获得了与父亲相同的诺贝尔物理学奖。阿格的儿子维尔赫姆·A·玻尔是一位科学家,供职于哥本哈根大学\(^\text{[21]}\) 和美国国家衰老研究所。\(^\text{[22]}\)

汉斯(Hans [da])成为医生;埃里克(Erik [da])成为化学工程师;欧内斯特成为律师。\(^\text{[23]}\)与他的叔叔哈拉尔德一样,欧内斯特·玻尔也成为奥运运动员,曾代表丹麦参加 1948 年伦敦夏季奥运会曲棍球比赛。\(^\text{[24]}\)
\subsection{物理学}
\subsubsection{玻尔模型}
1911 年 9 月,玻尔在卡尔斯伯基金会的奖学金资助下前往英国,当时关于原子和分子结构的大部分理论工作都在英国进行。\(^\text{[25]}\)他拜访了剑桥大学三一学院和卡文迪许实验室的 J.J. 汤姆孙,听取了詹姆斯·金斯和约瑟夫·拉默关于电磁学的讲座,并做了一些阴极射线的研究,但未能给汤姆孙留下深刻印象。[26][27] 他在与年轻物理学家,如澳大利亚的威廉·劳伦斯·布拉格\(^\text{[28]}\)和新西兰的欧内斯特·卢瑟福的交流中取得了更大的收获。卢瑟福在 1911 年提出的原子小而集中的原子核模型对汤姆孙 1904 年提出的“葡萄干布丁模型”提出了挑战。\(^\text{[29]}\)卢瑟福邀请玻尔到曼彻斯特维多利亚大学进行博士后研究,\(^\text{[30]}\)在那里玻尔结识了乔治·德·赫维希和查尔斯·高尔顿·达尔文(玻尔称其为“真正达尔文的孙子”)。\(^\text{[31]}\)

玻尔于 1912 年 7 月回到丹麦准备婚礼,并在英格兰和苏格兰度蜜月。返回后,他成为哥本哈根大学的私人讲师,讲授热力学课程。马丁·克努森提名玻尔担任讲师一职,该提名于 1913 年 7 月获得批准,玻尔随后开始为医学生授课。\(^\text{[32]}\)他那三篇后来被称为“玻尔三部曲”的论文\(^\text{[30]}\),于当年 7 月、9 月和 11 月发表于《哲学杂志》上。\(^\text{[33][34][35][36]}\)他将卢瑟福的核结构与马克斯·普朗克的量子理论结合起来,创立了著名的玻尔原子模型。\(^\text{[34]}\)

行星式原子模型并非玻尔首创,但玻尔的处理方式具有创新性。\(^\text{[37]}\)他以 1912 年达尔文关于电子在 $\alpha$粒子与原子核相互作用中作用的论文为起点,\(^\text{[38][39]}\)提出了电子围绕原子核在量子化“稳定状态”轨道中运行以维持原子稳定的理论,但直到 1921 年的论文中他才展示出各元素的化学性质在很大程度上取决于其原子外层轨道电子数量。\(^\text{[40][41][42][43]}\)他提出电子可以从高能轨道跃迁到低能轨道,并在此过程中发射出一个离散能量量子的观点。这一观点成为后来被称为“旧量子论”的基础。\(^\text{[44]}\)
