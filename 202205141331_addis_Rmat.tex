% R-矩阵法(量子力学)

\subsection{一维薛定谔方程}
一个算符是否为厄米算符与边界条件有关. 例如
\begin{equation}
H = -\frac{1}{2m}\dv[2]{x} + V(r)
\end{equation}
要证明厄米性, 用分部积分法得
\begin{equation}\label{Rmat_eq1}
\int_{-\infty}^{+\infty} uHv\dd{x} - \int_{-\infty}^{+\infty} vHu\dd{x}
= \eval{-\frac{1}{2m}[uv' - u'v]}_{-\infty}^{+\infty}
\end{equation}
由于我们假设波函数在无穷远处消失, 则该式为零, 说明 $H$ 是厄米的. 但如果在有限区间 $[0, a]$ 中, 则波函数的边界条件必须满足 $\eval{[uv' - u'v]}_{0}^{a} = 0$ 才能保证厄米性.

但若边界条件不符合该要求, 为了在 $[0,a]$ 内构造一组离散的正交归一基底, 我们可以拼凑一个厄米算符. 把
