% 集合的基本运算(高中)
% keys 集合|基本运算|运算
% license Usr
% type Tutor
\pentry{集合\nref{nod_HsSet}}{nod_120e}

\begin{issues}
\issueDraft
\end{issues}
相信此刻的你,已经基本了解了集合的基础知识。就像小学接触数字那样,认识了十以内的数字,就要开始接触他们的加减乘除了,“加减乘除”被称为\textbf{运算}(Operations),本文介绍的就是集合的基本运算。

\subsection{三种基础运算}

\subsubsection{交集}

生活中,当我们感慨与一个人没有联系,彼此渐行渐远,互不来往,逐渐消失在对方的生活里时,常常会说,“我与他没有交集了”。而当心里有很多感情同时涌现出来的时候,会说自己“百感交集”。尽管这些词创立的时候与下面要讲的这个运算没有直接的关系,但他们在语义上表达的内容差不多,前者是指两者没有任何相同的内容共存,后者是指同时有很多感情共存。下面给出交集的定义,可以参照前面给出的现实中的例子去体会。

\begin{definition}{交集}
一般地,由既属于集合 $A$ 又属于集合 $B$ 的所有元素组成的集合叫做 $A$ 与 $B$ 的\textbf{交集}(intersection),记作 $A \cap B$,读作“$A$交$B$”,或“$A$与$B$的交集”,即
\begin{equation}
A\cap B = \begin{Bmatrix} x|x\in A\land x\in B \end{Bmatrix}~.
\end{equation}
其中,$\land$表示逻辑上的“且”。
\end{definition}

简单来讲,集合$A$和集合$B$的公共元素组成的集合就是集合$A$与$B$的交集。

\subsubsection{并集}

\begin{definition}{并集}
由属于集合 $A$ 或属于集合 $B$ 的所有元素组成的集合,叫作 $A$ 与 $B$ 的\textbf{并集}(union),记作 $A\cup B$,读作“A并B”,或“$A$与$B$的并集”,即
\begin{equation}
A\cup B = \begin{Bmatrix}x|x\in A \lor x\in B\end{Bmatrix}~.
\end{equation}
其中,$\lor$表示逻辑上的“或”。
\end{definition}

简单来讲,集合$A$和集合$B$的全部元素组成的集合就是集合$A$与$B$的并集。

\subsubsection{补集}
\begin{definition}{补集}
设 $U$ 是全集,$A$ 是 $U$ 的一个子集(即$A\subseteq U$),则由 $U$ 中所有不属于 $A$ 的元素组成的集合,叫作 $U$ 中子集 $A$ 的\textbf{补集}\footnote{也称为\textbf{余集}}(complementary set),记作$\complement_UA$,即
\begin{equation}
\complement_UA = \begin{Bmatrix}x|x\in U \wedge \notin A\end{Bmatrix}~.
\end{equation}
\end{definition}

由补集定义可得,
\begin{equation}
\begin{aligned}
&A\cup (\complement_UA) = U~, \\
&A\cap (\complement_UA) = \varnothing~.
\end{aligned}
\end{equation}

\subsection{运算性质与运算律}

\subsubsection{交集与并集}

下面的性质都是可以从定义中直接推知的,因此不加证明地给出。由于交集和并集是\textbf{对偶}(duality)的,因此他们在性质上互相类似。在下表中会并行给出,方便观察规律:
\begin{table}[ht]
\centering
\caption{交集与并集的性质}\label{tab_HsSeOp1}
\begin{tabular}{|c|c|c|c|}
\hline
 & 交集$\cap$ & 并集$\cup$ & 备注 \\
\hline
1 & $A\cap B = B\cap A$ & $A\cup B = B\cup A$ & 交换律 \\
\hline
2 & $ A \cap (B \cap C) = (A \cap B) \cap C$  &$ A \cup (B \cup C) = (A \cup B) \cup C$ & 结合律 \\
\hline
3 & $ A \cap (B \cup C) = (A \cap B) \cup (A \cap C) $  & $ A \cup (B \cap C) = (A \cup B) \cap (A \cup C) $ & 分配律 \\
\hline
4 & $ A \cup (A \cap B) = A $  &$ A \cap (A \cup B) = A $ & 吸收律 \\
\hline
5 & $A\cap A = A$ & $A\cup A = A$ & 幂等性 \\
\hline
6 & $(A\cap B) \subseteq A$<br> $(A\cap B) \subseteq B$& $(A\cup B) \supseteq A$<br>$(A\cup B) \supseteq B$ & * \\
\hline
7 & $A\cap \varnothing = \varnothing$ & $A\cup \varnothing = A$ & 与空集的关系 \\
\hline
8 & $ A \cap U = A $  &$ A \cup U = U $ & 与全集的关系 \\
\hline

\end{tabular}
\end{table}

\subsubsection{补集}

$ A \cup A^c = U $
$A \cap A^c = \varnothing$

结合律(Associative Law)、分配律(Distributive Law)、吸收律(Absorption Law):


同一律(Identity Law):
$A \cup \varnothing = A$
$ A \cap U = A $
 
\subsubsection{*德摩根定律}

\textbf{德摩根定律}(De Morgan’s Laws)在当前的高中教材中已经移除,但由于其功能强大,且在未来的学习中会经常使用,因此在此处予以介绍。

$ (A \cup B)^c = A^c \cap B^c $
$ (A \cap B)^c = A^c \cup B^c $

\begin{equation}
\complement_U(\bigcup_{i\in I} A_i)=\bigcap_{i\in I} \complement_UA_i.~
\end{equation}

\begin{equation}
\complement_U(\bigcap_{i\in I} A_i)=\bigcup_{i\in I} {\complement_UA_i}.~
\end{equation}


\subsection{*集合运算与逻辑运算的关系}


\addTODO{下面的内容移动至本科内容中}

\subsection{*差集}
\subsection{势}
\subsection{幂集}