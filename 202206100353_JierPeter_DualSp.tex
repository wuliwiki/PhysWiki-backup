% 对偶空间
% 线性映射|矢量空间|对偶空间|对偶基|列向量|行向量|线性空间|向量空间|狄拉克符号|Dirac 符号


\pentry{张成空间\upref{VecSpn}}% 未完成

\subsection{线性函数}

中学数学中一个贯穿始终的研究对象就是\textbf{函数(function)}.函数是\textbf{映射(mapping)}的一种,在现代数学的习惯中通常把从任意集合到一个数字集合\footnote{数字集合就是我们通常叫做数字的元素的集合,比如实数集 $\mathbb{R}$,有理数集 $\mathbb{R}$,整数集 $\mathbb{R}$,复数集 $\mathbb{C}$ 等,甚至 $\{0, 1\}$ 这样的集合都叫数字集合.}的映射称为一个函数.特别地,我们把定义线性空间时使用的域 $\mathbb{K}$ 都当作数字集合,多数情况下这些域都是实数域、复数域或者它们的某种等价类划分\footnote{如整数\upref{intger}中提到的模运算,其元素也被称为数字,只不过不是通常意义上的整数.}.

如果给定一个域 $\mathbb{K}$ 上的线性空间 $V$,把 $V$ 中的任意向量 $\bvec{v}$ 映射到 $\mathbb{K}$ 中的一个元素 $f(\bvec{v})$ 上,那么映射 $f:V\rightarrow\mathbb{K}$ 就是线性空间 $V$ 上的一个函数.你可以任意指定函数的对应规则,但我们只关心所有可能的函数中最容易研究的一类,线性函数.

线性函数由于具有可加性,通俗来说就是“函数图像是平直的”,因此非常容易讨论.在微分几何中,尽管我们研究的不一定是平直的函数,但是总能在局部取平直的线性函数来考察任一点附近的函数性质,这其实就是微分的思想.换句话说,微分实际上是一种“线性近似”,在给定点把可微函数近似为一个线性函数,方便讨论.

\begin{definition}{线性函数}\label{DualSp_def3}
给定一个域 $\mathbb{K}$ 上的线性空间 $V$,如果映射 $f:V\rightarrow\mathbb{K}$ 满足线性性,即对于任意 $\bvec{v}, \bvec{u}\in V$ 以及任意的 $a, b\in\mathbb{K}$,都有 $f(a\bvec{v}+b\bvec{u})=af(\bvec{v})+bf(\bvec{u})$,那么我们称 $f$ 是 $V$ 上的一个\textbf{线性函数(linear function)}.
\end{definition}

线性性质的好处就在于,线性函数的性质只依赖于基向量的函数值.就是说,当我们取定了一组基 $\{\hat{\bvec{e}}_i\}$,那么只要知道了线性函数 $f$ 对它们的取值 $f(\hat{\bvec{e}}_i)$,我们就可以利用线性性算出任意向量 $\bvec{v}\in V$ 的函数值 $f(\bvec{v})$,因为每一个 $\bvec{v}$ 都是基向量的某个线性组合.

如果要定义 $V$ 上一个任意的函数,我们的自由度很高,可以对每一个向量都指定一个函数值,各向量的函数值并不互相影响;但是对于线性函数,自由度就只有 $\opn{dim} V$ 个,一旦选定了基向量的函数值,所有向量的函数值就都确定了.不过基向量的函数值仍然可以任意指定.

函数之间可以进行加减法和乘法运算,这使得函数自然构成一个环\upref{Ring},线性函数自然也不例外.目前我们只关心线性函数之间的加减法.

\begin{definition}{线性函数的加减法}\label{DualSp_def1}
给定一个域 $\mathbb{K}$ 上的线性空间 $V$,对于两个线性函数 $f, g: V\rightarrow\mathbb{K}$,定义函数的和 $(f+g):V\rightarrow\mathbb{K}$ 如下:对于任意 $\bvec{v}\in V$,有 $(f+g)(\bvec{v})=f(\bvec{v})+g(\bvec{v})$.
\end{definition}

这样一来,我们就可以对函数也进行加减法,就像对数字所做的那样.同样地,我们可以定义函数的数乘,即用域中的数字和函数相乘来得到新的函数.对于线性函数,数乘定义如下:

\begin{definition}{线性函数的数乘}\label{DualSp_def2}
给定一个域 $\mathbb{K}$ 上的线性空间 $V$,标量 $a\in\mathbb{K}$ 以及线性函数 $f: V\rightarrow\mathbb{K}$,定义函数的数乘 $af:V\rightarrow\mathbb{K}$ 如下:对于任何 $\bvec{v}\in V$,有 $(af)(\bvec{v})=a\cdot f(\bvec{v})$.
\end{definition}

线性函数的数乘对加法具有分配律:

\begin{theorem}{线性函数的数乘对加法的分配律}\label{DualSp_the1}
给定一个域 $\mathbb{K}$ 上的线性空间 $V$,对于两个线性函数 $f, g: V\rightarrow\mathbb{K}$ 和标量 $a\in\mathbb{K}$,有:$a(f+g)=af+ag$.
\end{theorem}

这一定理的证明是非常直接的,应用域 $\mathbb{K}$ 上的乘法分配律就可以.

\subsection{对偶空间}

从\autoref{DualSp_def1} 和\autoref{DualSp_def2} ,以及\autoref{DualSp_the1} 我们可以看出,对于一个域 $\mathbb{K}$ 上的线性空间 $V$,$V$ 上的全体线性函数也满足向量空间的定义,构成一个向量空间.我们把这个空间称作 $V$ 的\textbf{对偶空间}.

\begin{definition}{对偶空间}
给定一个域 $\mathbb{K}$ 上的线性空间 $V$,记 $V^*$ 为集合 $\{V\text{上全体线性函数}\}$.在 $V^*$ 上定义加法和数乘如\autoref{DualSp_def1} 和\autoref{DualSp_def2},那么 $V^*$ 构成一个线性空间,称作 $V$ 的\textbf{对偶空间}.
\end{definition}

看起来,$V^*$ 称作线性函数空间更直接,为什么要特地取“对偶”一名字呢?初看对偶空间的定义,你可能会觉得 $V$ 和 $V^*$ 的元素之间是不对等的,$V^*$ 中的函数是用来对 $V$ 中的向量进行映射的.可是反过来看的话,$V$ 中向量也可以看成是 $V^*$ 中向量的映射.这就是说,$f\in V^*$ 可以看成是一个 $V\rightarrow\mathbb{K}$ 的映射,而 $\bvec{v}\in V$ 也可以看成是 $V^*\rightarrow\mathbb{K}$ 的映射.

\begin{theorem}{$V$ 是 $V^*$ 的对偶空间}

考虑一个域 $\mathbb{K}$ 上的线性空间 $V$ 及其对偶空间 $V^*$,那么对于任意 $\bvec{v}\in V$,$\bvec{v}$ 是 $V^*$ 上的线性函数.

\end{theorem}

如果我们形式上进行一下简化,把 $f$ 记为向量形式 $\bvec{f}$,然后把 $f(\bvec{v})$ 记为 $\bvec{f}\bvec{v}$,那么可以更清晰地看出它们互为彼此的线性函数:给定一个域 $\mathbb{K}$ 上的线性空间 $V$ 及其对偶空间 $V^*$,取 $a_i\in\mathbb{K}$、$\bvec{f}_i\in V^*$ 和 $\bvec{v}_i\in V$,那么从 $\bvec{f}_0(a_1\bvec{v}_1+a_2\bvec{v}_2)=a_1\bvec{f}_0\bvec{v}_1+a_2\bvec{f}_0\bvec{v}_2$ 可知,$\bvec{f}$ 是 $V$ 上的线性函数;从 $(a_1\bvec{f}_1+a_2\bvec{f}_2)\bvec{v}_0=a_1\bvec{f}_1\bvec{v}_0+a_2\bvec{f}_2\bvec{v}_0$ 可知,$\bvec{v}$ 是 $V^*$ 上的线性函数.

我们可以这么理解:$\bvec{f}\bvec{v}$ 实际上就是“取 $V$ 和 $V^*$ 中各一个元素,把它们俩对应到一个标量上”的过程.两个空间形式上平等,互为对偶空间,甚至可以写成 $V^{**}=V^*$.由此,我们可以更对称地定义对偶空间的概念:

\begin{definition}{对偶空间}
给定域 $\mathbb{K}$ 上两个 $n$ 维线性空间 $V$ 和 $W$.如果指定了一个双线性映射 $m:V\times W\rightarrow\mathbb{K}$\footnote{即对于 $a_i, b_i\in\mathbb{K}$、$\bvec{v}_i\in V$ 和 $\bvec{w}_j\in W$,有 $m(\sum\limits_{i=1}^na_i\bvec{v}_i,\sum\limits_{j=1}^nb_j\bvec{w}_j )=\sum\limits_{i, j=1}^na_ib_jm(\bvec{v}_i)m(\bvec{w}_j)$},那么称 $V$ 和 $W$ 是对偶的.如果把 $\bvec{v}\in V$ 视为线性函数 $f_{\bvec{v}}:W\rightarrow\mathbb{K}$,那么对于任意 $\bvec{w}\in W$ 有 $f_{\bvec{v}}(\bvec{w})=f(\bvec{v}, \bvec{w})$.
\end{definition}


\subsection{对偶空间的表示}\label{DualSp_sub1}%对偶基;

给定线性空间 $V$ 作为原空间,再给定 $V$ 的一组基,我们就可以把 $V$ 中的向量表示为列矩阵.对应地,$V^*$ 中的向量表示为行矩阵,这个行矩阵是该向量在对偶基下的坐标.这样一来,向量和对偶向量互相乘起来得到标量的过程,就是这两个矩阵相乘的过程,其中对偶向量的行矩阵放在左边,向量的列矩阵放在右边.


\subsubsection{对偶基}

给定了 $V$ 的基 $\{\bvec{e}_i\}$ 后,为了方便计算,我们通常规定 $V^*$ 的基 $\{\bvec{f}_j\}$ 满足关系:$\bvec{e}_i\bvec{f}_j=\delta_{ij}$.这样我们就由 $V$ 的基\textbf{唯一}地决定了 $V^*$ 的基,称这样的 $V^*$ 的基为 $V$ 的\textbf{对偶基(dual space)}.

事实上,以上定义是对偶基的一个特例.一般地,如果在给定了 $V$ 的基 $\{\bvec{e}_i\}$ 后,有一个对称矩阵 $g_{ij}$,它被定义为 $V$ 的度量在给定基下的表示\footnote{如果两个向量的坐标分别是 $x_i$ 和 $y_j$,那么它们的内积定义为 $x_iy_jg_{ij}$.},那么对偶基的计算应该满足 $\bvec{e}_i\bvec{f}_j=g_{ij}$.注意,这里并没有要求 $g_{ij}$ 是正定的矩阵.具体的度量矩阵的定义要由问题情景来决定,比如说,狭义相对论中的闵可夫斯基度量就是:
\begin{equation}
g_{ij}=\pmat{1&0&0&0\\0&-1&0&0\\0&0&-1&0\\0&0&0&-1}
\end{equation}

在闵可夫斯基度量下,如果一个向量在某基下的坐标是 $(a_1, a_2, a_3, a_4)\Tr$,一个对偶向量在\textbf{对偶基}下的坐标是 $(b_1, b_2, b_3, b_4)$,那么它们相乘之后的结果就是 $a_1b_1-a_2b_2-a_3b_3-a_4b_4$.



\subsubsection{狄拉克符号}
%狄拉克符号

狄拉克符号是一种用来描述量子态的表示方法.量子力学中认为每个量子态都是一个向量,定义在复数域上的线性空间中,因此狄拉克符号表示的都是复向量.

狄拉克符号分为左矢和右矢,其中左矢相当于行矩阵,右矢相当于列矩阵;左矢和右矢在两个不同的空间中,称为左矢空间和右矢空间,这两个空间彼此对偶.每个量子态向量都有双重表示,一个左矢表示,一个右矢表示.同一个量子态的左矢和右矢是对偶向量,故满足以下条件:在同一个基\footnote{量子力学中所说的基和所谓的“表象”相关,比如在空间表象下的基就是空间的特征态,也就是在空间坐标下表示的波函数为 $\delta$ 函数的那些态;动量表象下的基就是动量的特征态,这些基向量都是在全空间概率均匀分布的波函数,不过用动量坐标来表示的话,这些波函数又是 $\delta$ 函数了.同一个波函数在两种坐标之间的表示由傅里叶变换计算出,具体方式详见量子力学.}下,如果同一个量子态向量 $s$ 的左矢表达是 $\langle s|$,右矢表达是 $|s\rangle$,那么有 $\langle s|\cdot|s\rangle=1$.为了简便,我们也把这个表达写成 $\langle s|\cdot|s\rangle=\langle s|s\rangle$.

一般来说,量子态空间都是无穷维希尔伯特空间,并不能把向量坐标简单表示为矩阵.但是我们依然可以用矩阵来做类比,在这样的类比下,如果一个表象下的一个量子态的右矢中第 $i$ 个坐标是复数 $s_i$,那么其左矢中第 $i$ 个坐标就是 $s_i^*$,也就是对应坐标值取其复数共轭,这样就能满足条件 $\langle s|s\rangle=1$.

由于量子态的另一种理解方式是波函数,所以我们其实可以把左矢、右矢的乘积理解为波函数的乘积积分,就像最常见的函数向量一样.具体来说,如果有两个量子态 $s$ 和 $t$,分别对应波函数 $f_s$ 和 $f_t$,以及右矢 $|s\rangle$ 和 $|t\rangle$,那么我们有对应关系:
\begin{equation}\langle s|t\rangle=\int\limits_{\text{全空间}}f_s^*f_tdx\end{equation}
注意左矢对应的 $f_s$ 取了复共轭.









