% 晶核的长大

\pentry{经典形核理论\upref{NCLT}}
\footnote{本文参考了刘智恩的《材料科学基础》与Callister的 Material Science and Engineering An Introduction} 在形核理论 \upref{NCLT} 中我们已经探讨了形核的必要条件。在本文中我们将处理晶核最终长大为晶粒的问题。

\subsection{长大动力学}
\subsubsection{过冷度与长大速率}
\begin{figure}[ht]
\centering
\includegraphics[width=8 cm]{./figures/dc67abf0ad7d0ec1.pdf}
\caption{长大速率示意图} \label{fig_GGRW_1}
\end{figure}

晶核的长大是原子从液相扩散至已形成的固态晶核的过程。晶核整体的长大速率也受两方面因素的制约:
\begin{itemize}
\item 首先,晶核长大的前提是要有晶核,所以长大速率与形核速率有关
\item 其次,长大过程是原子从液相扩散到固相的过程,因此长大速率还与原子的扩散速率有关。
\end{itemize}
我们已经在形核理论 \upref{NCLT} 中探讨了形核率与过冷度的联系:随过冷度增高,形核率会先升高再降低;与此同时,低温将会抑制原子扩散的能力,因此随过冷度增高,扩散速率会降低。

\textsl{矛盾双方对立统一}的结果是,随过冷度上升,长大速率会上升再减小。在简单的情况下,在长大速率呈现下降前液体就已经凝固完成,因此有时可以简单地认为随过冷度升高,长大速度将升高。

\subsubsection{过冷度与晶粒尺寸}
\begin{figure}[ht]
\centering
\includegraphics[width=8 cm]{./figures/815e75cba9e05772.pdf}
\caption{过冷度与晶粒尺寸} \label{fig_GGRW_fig2}
\end{figure}
从以上的讨论中我们还可以推断出过冷度与晶粒尺寸的关系。在较低的过冷度(凝固时的温度接近熔点)下,形核率较低而扩散速率较高,主导的机制是扩散,最终形成的晶粒也将更大也更少(很显然,晶粒的数量和大小成反比,因为可供结晶的物质的总量是一定的)。反过来说,在较高的过冷度下,由于形核率较高而扩散速率较低,主导的机制是形核,因此最终形成的晶粒将更多也更小。形象地说,快速冷却时晶粒来不及长大就被凝固了。

这个结论启发我们,通过调整冷却速率以控制过冷度,就能调节材料的晶粒尺寸。这在实际生产中十分有用。

\begin{example}{并非简单降温}
看到这里,你可能会觉得控制材料结构原来如此简单?事实上,实际问题更加复杂,单纯地快速冷却往往会导致额外的问题。

比如说热胀冷缩。当冷却速率足够高时,材料外层已经充分降温,而材料内部由于传热速率的限制仍然处于高温状态。我们知道“热胀冷缩”,内外的温差将导致材料内外变形程度不一致。这种不一致就可能在材料内部引入大量的内应力、空洞,甚至直接使材料破裂。

同时,对于铁等存在固态相变的金属 \upref{PHS} ,晶体结构与冷却速率的关系非常复杂,是一个非平衡热力学问题。工业上常用CCT(连续冷却)与TTT(等温冷却)相图描述。
\end{example}

\begin{example}{材料科学与地质学}
\footnote{本节参考了Lutgens, et al. 的 Essentials of Geology.}除了在工厂中,我们还在自然界的许多场合见到凝固过程,比如...\textsl{火山喷发}。

当喷发出的高温岩浆逐渐遇冷凝固时,它也要遇到过冷度与晶粒尺寸的问题。比如,暴露在空气中的岩浆遇到较大的过冷度,因此快速冷却、凝固,常形成小晶粒的矿物;而处于底层的岩浆较难散热,过冷度低,晶粒有充分的时间长大,因此常形成大晶粒的矿物。

在极端情况下,岩浆可能极速冷却,此时岩浆甚至没有时间形成晶体结构,从而产生非晶(amorphous,又称无定性、玻璃态)物质,这类物质的典型代表是黑曜石。\textsl{怪不得在Minecraft中岩浆源遇水会生成黑曜石!}
\end{example}
