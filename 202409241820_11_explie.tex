% 李群的指数映射
% license Usr
% type Tutor

%预备知识需要添加.
\pentry{李群的李代数\nref{nod_LieGA},流\nref{nod_flow}}{nod_016b}
(本文默认左不变切场都是切空间的切向量经左平移映射延拓得到)
\begin{definition}{}
设$\mathfrak g$是李群$G$的李代数。定义指数映射$\exp :\mathfrak g\to G$,对于任意$X_e\in \mathfrak g$ 有$\exp (X_e)=c_X(1)$。其中$c_X$正是左不变切场$X$的积分流。
\end{definition}
在物理上,我们经常要用到矩阵李群。可以证明,对于矩阵李群,指数映射恰为矩阵的指数函数,可说是名副其实了。

\begin{theorem}{指数映射的性质}
\begin{enumerate}
\item 对于任意$X_e\in \mathfrak g$,$X$的开始于$e$的积分曲线为$c_X(t)\equiv\theta_t(e)=\exp (tX_e)$。
\item 对于任意$X_e\in \mathfrak g$,左不变切场$X$开始于$g$的积分曲线为$g\exp (tX_e)$。
\item 对于任意$x,t\in\mathbb R,X_e\mathfrak g$,指数映射是$\mathbb R\to G$的群同态,满足$\exp ((s+t)X_e)=(\exp sX_e)(\exp tX_e)$。

\item 指数映射是光滑的。
\item 指数映射在$t=0$处的切映射是单位映射。
\item 对于一般线性群$GL(n,\mathbb R)$,有
\begin{equation}
\exp  A=\sum_{k=0}^\infty\frac{A^k}{k!},\quad \forall A\in\mathfrak{gl}(n,\mathbb{R})~.
\end{equation}
\end{enumerate}
\end{theorem}

\textbf{证明:}

(1)设任意$t\in \mathbb R$,$s$是可变参量,$c_X(s)$为$X$的开始于$e$的积分曲线$\theta_t(e)$。定义$\tau(s)=c_X(ts)$。则
\begin{equation}
\dv{c_X(ts)}{s}=\tau'(s)=tc'_X(ts)=tX_{\tau (s)}~.
\end{equation}
因为$\tau(0)=e$并且每一处的切向量场都是$tX$,由极大积分曲线的唯一性得$\tau(s)=c_{tX}(s)=c_X(ts)$。代入$s=1$得$c_{tX}(1)=\exp(tX_e)=c_X(t)$。

(2)由(1)知,我们需要证明对于任意$X_e\in \mathfrak g$,左不变切场$X$开始于$g$的积分曲线为$gc(t)$。
因为
\begin{equation}
\begin{aligned}
\dv{(gc(t))}{t}&=\dv{L_gc(t)}{t}\\
&=L_{g*}\dv{c(t)}{t}\\
&=X_{gc(t)}~.
\end{aligned}
\end{equation}

因此$gc(t)$确实是$X$开始于$g$的积分曲线。


(3)由题意知,我们需要证明$c(s+t)=c(s)c(t)$。依然设$s$为变化参量,$t$为任意固定的常数,$c(t)$为$X$开始于$e$的积分曲线,问题转化为需证等式两端是相同的积分曲线。
因为
\begin{equation}
\dv{c(s+t)}{s}=X_{c(s+t)}~,
\end{equation}
所以$c(s+t)$为$X$开始于$c(s)$的积分曲线。由上题知,$c(s)c(t)$亦是如此。由极大积分曲线的唯一性知这是同一条积分曲线,所以得证。

(4)

(5)因为$\exp_*(tX_e)|_{t=0}=c'(0)=X_e$,因此得证。

(6)只要证明$c(t)=\sum^{\infty}_{k=0}(tA)^k/(k!)$确实是左不变切场$\widetilde A$开始于$e$的积分曲线即可。易见$c(0)=I=e$,又因为
\begin{equation}
\begin{aligned}
c'(t)&=\sum_{k=1}^\infty\frac{kA^kt^{k-1}}{k!}\\
&=\sum_{k=1}^\infty\frac{A^{k-1}t^{k-1}A}{(k-1)!}\\
&=c(t)A=L_{c(t)}A=\widetilde A_{c(t)}~,
\end{aligned}
\end{equation}
因此得证。

\begin{theorem}{}
设$f:H\to G$为李群同态映射,对于任意$X\in T_e H$有
\begin{equation}\label{eq_explie_1}
\exp(f_*X)=f(\exp X)~.
\end{equation}
\end{theorem}
\textbf{证明:}
首先我们来证明一个很有用的引理。题设不变,并设任意$h\in H$,我们有
\begin{equation}
f\circ L_{h}=L_{f(h)}\circ f~.
\end{equation}
两边对任意$h_1\in H$作用,由同态性质即可得$f\circ L_h(h_1)=f(hh_1)=L_{f(h)}\circ f(h_1)$。更进一步,结合前推映射的结合性得:
\begin{equation}
f_*\circ L_{h*}=L_{f(h)*}\circ f_*~.
\end{equation}

为了证明\autoref{eq_explie_1} ,我们证明对于$t\in \mathbb R$都有$\exp (tf_*X)=f(\exp tX)$,最后令$t=1$即可。

那么要验证的式子实际上为
\begin{equation}
f\circ c(t)=c_{f_*(X)}(t)~.
\end{equation}
因为
\begin{equation}
\begin{aligned}
(f\circ c)'(t)=f_*[c'(t)]&=f_*\circ L_{c(t)*}(X)\\
&=L_{f\circ c(t)*}(f_*X)=\widetilde X~.
\end{aligned}
\end{equation}

