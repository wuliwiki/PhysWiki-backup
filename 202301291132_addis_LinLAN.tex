% 搭建 Linux 局域网

\begin{issues}
\issueDraft
\end{issues}

\pentry{Linux 网络笔记\upref{LinWeb}}

\subsection{wifi 路由器局域网}
\begin{itemize}
\item wifi 路由器同样也支持有线连接,可以充当 switch, 还有路由器, 和 DHCH 服务器。
\item 路由器是否会改变机器的 ip? (如果机器太多应该会的)
\item 实测学校的 wifi 几乎不会改变办公室主机的 ip(笔记本就不确定了)。
\end{itemize}

\subsection{有线局域网}
\begin{itemize}
\item 8 口交换机, 支持 cat 6 网线
\item 若干 cat6 网线
\item 一个台式机(用 wifi 接入校园网), 网线接入局域网。
\item 若干台笔记本电脑, 网线接入局域网, wifi 关闭。
\item 全部电脑安装 Ubuntu 22.04 桌面版。
\end{itemize}

\subsubsection{笔记本合上盖子不作为}
\begin{itemize}
\item \verb|sudo vim /etc/systemd/logind.conf|, 找到所有包含 \verb|LidSwitch| 的选项, 取消注释, 值改成 \verb|ignore|
\item 然后运行 \verb|systemctl restart systemd-logind.service| 生效。 注意系统会重新 login
\end{itemize}


\subsection{用 Windows 共享网络}
\begin{itemize}
\item 台式机也可以用 Windows 电脑来提供网络, 例如通过 wifi 接互联网, 就再控制面板中找到 view network status and tasks, 点 Wi-Fi 2 (Wifi 名称), 点最下面的 properties, 点 sharing 面板, 勾选 Allow other network users to connect through this computer's Internet connection, 然后在下面选中要分享的 Ethernet 网卡。 下面那个勾也可以打上(... control or disable the shared Internet connection)。 这样就成功了, 链接 switch 的所有电脑都可以上网了。
\item 如果重启 Windows 以后, Linux 连不上, 就把上面的 share 关闭, 连 Linux, 再开一次。
\item Switch 的 DHCP 功能可以关掉, 因为这个设置下 DHCP 和 NAT 都由 Windows 提供。 NAT 根据 DHCP 分配的 ip 工作, 所以如果 Linux 机器自行设置 ip 地址, 那么局域网中该 ip 可以用, 但是这个机器就上不了外网了(因为 Windows 的 NAT 转发失败)。 所以最好还是不碰, Linux 被分配的 ip 貌似也不会改变。
\item Windows 上可以用 cmd 里面的 \verb|ipconfig| 命令来查看 ip。 如果 Win 上装了 WSL, 那么 WSL 里面看到的 ip 和 cmd 里面是一样的。
\end{itemize}

\subsection{网络配置}
在 Ubuntu 台式机上:
\begin{lstlisting}[caption=/etc/netplan/*.yaml]
# Let NetworkManager manage all devices on this system
network:
  version: 2
  renderer: NetworkManager
  ethernets:
    enp3s0:
      dhcp4: false
      addresses: [192.168.137.1/24]
      nameservers:
        addresses: [10.130.30.52, 10.130.30.53]
\end{lstlisting}
在笔记本上:
\begin{lstlisting}[caption=/etc/netplan/*.yaml]
# Let NetworkManager manage all devices on this system
network:
  version: 2
  renderer: NetworkManager
  ethernets:
    enp3s0:
      dhcp4: false
      addresses: [192.168.137.11/24]
      routes: [{to: default, via: 192.168.137.1}]
      nameservers:
        addresses: [10.130.30.52, 10.130.30.53]
\end{lstlisting}


\subsection{ssh 互通}
\begin{itemize}
\item 安装 \verb|openssh-server|, 设置见\upref{SSH}。 注意每个用户还是单独生成自己的密钥比较好, 否则可能连接会被拒绝。
\end{itemize}

\subsection{共享文件夹}
Linux 详见 “Linux 创建网络文件夹(sshfs 和 NFS)\upref{NFS}”。 要把 windows 上的目录挂载到 linux, 建议用 sshfs, 好处是 windows 上无需安装任何第三方的 server, 只要 linux 能 ssh 到 windows 的 WSL1 就行。 但是 windows 重启以后可能就会需要重新连一次。

Linux 之间共享目录还是用 NFS 比较好。 但是 Windows 上面访问 Linux 创建的 NFS 似乎有点复杂, 别弄了, 要和 Windows 共享就用上面的 sshfs 就好。

\subsection{远程桌面}
\begin{itemize}
\item 远程桌面对 wayland 支持一般欠佳。 较新的 ubuntu 如果用的显示服务器是 wayland, 可以退出登录, 在登录的时候选择 xorg。
\item 向日葵用不了(即使换成 Xorg 也不行), teamviewer + Xorg 可以。
\end{itemize}

\subsection{安装 DHCP 服务}
\begin{itemize}
\item 局域网都用静态 ip 的话就没必要
\item 参考\href{https://www.linuxtechi.com/how-to-configure-dhcp-server-on-ubuntu/}{这里}。
\end{itemize}
