% 含时微扰理论
% 微扰理论|量子力学|薛定谔方程|近似
% 未完成:考虑把第一节的内容开新词条

\pentry{薛定谔方程\upref{TDSE}}

\subsection{薛定谔方程的矩阵形式}
\footnote{本文参考 \cite{GriffQ}.}在讲微扰理论之前,我们先来看如何把含时薛定谔方程写为矩阵的形式.含时薛定谔方程的一般形式为
\begin{equation}\label{TDPT_eq1}
H \ket{\psi(t)} = \I\hbar \pdv{t} \ket{\psi(t)}
\end{equation}
我们把哈密顿算符分为不含时部分 $H_0$ 和含时部分 $H'(t)$
\begin{equation}
H = H_0 + H'(t)
\end{equation}
我们已经知道 % 链接未完成
$H'(t) = 0$ 的情况下含时薛定谔方程的通解:先解出对应的定态薛定谔方程的波函数 $\ket{\psi_n}$ 和能级 $E_n$,通解可表示为(为简单起见我们暂时假设 $H_0$ 只有离散的束缚, 例如简谐振子\upref{QSHOop})
\begin{equation}
\ket{\psi_0(t)} = \sum_n c_n \ket{\psi_n} \E^{-\I E_n t/\hbar}
\end{equation}
注意其中 $c_n$ 为常数,由初始波函数决定.我们可以定义一组\textbf{含时}基底
\begin{equation}
\ket{\phi_n(t)} = \ket{\psi_n} \E^{-\I E_n t/\hbar} \qquad (n = 1, 2, \dots)
\end{equation}
用于展开\autoref{TDPT_eq1} 中的含时波函数.注意任何时刻这组基底都正交归一.
\begin{equation}
\ket{\psi(t)} = \sum_n c_n(t) \ket{\phi_n(t)} = \sum_n c_n(t) \ket{\psi_n} \E^{-\I E_n t/\hbar}
\end{equation}
由于基底并不是总哈密顿算符 $H$ 的本征矢,系数需由常数拓展为时间的函数 $c_n(t)$.

选择了基底后,就可以把薛定谔方程表示为矩阵的形式. % 链接未完成
把上式代入薛定谔方程(\autoref{TDPT_eq1})得
\begin{equation}
\begin{aligned}
&\sum_n c_n(t) H_0 \ket{\phi_n(t)} + \sum_n c_n(t) H'(t) \ket{\phi_n(t)} \\ 
={} &\I\hbar \sum_n \dv{t} c_n(t) \ket{\phi_n(t)}
+ \I\hbar \sum_n c_n(t) \dv{t} \ket{\phi_n(t)}
\end{aligned}
\end{equation}
考虑到
\begin{equation}
H_0\ket{\phi_n(t)} = \I\hbar \dv{t} \ket{\phi_n(t)}
\end{equation}
% 引用公式未完成
可化简为
\begin{equation}
\sum_n c_n(t) H'(t) \ket{\phi_n(t)}
= \I\hbar \sum_n \dv{t} c_n(t) \ket{\phi_n(t)}
\end{equation}
两边左乘 $\bra{\phi_m(t)} = \E^{\I E_m t/\hbar}\bra{\psi_m}$ (即要求每个分量相等), 且令
\begin{equation}
\omega_{mn} = \frac{E_m-E_n}{\hbar}
\end{equation}
\begin{equation}
H'_{mn}(t) = \bra{\psi_m} H'(t)\ket{\psi_n}
\end{equation}
\begin{equation}\label{TDPT_eq7}
\tilde H_{mn}(t) = \bra{\phi_m(t)} H'(t) \ket{\phi_n(t)} = H'_{mn} \E^{\I\omega_{mn}t/\hbar}
\end{equation}
得
\begin{equation}\label{TDPT_eq2}
\sum_n \tilde H_{mn}(t) c_n(t)
= \I\hbar \dv{t} c_m(t)
\end{equation}
写成矩阵形式为(对矢量求导即对每个分量分别求导\upref{DerV})
\begin{equation}\label{TDPT_eq3}
\tilde{\mat {H}} \bvec c = \I\hbar \dv{t} \bvec c
\end{equation}
到此为止我们还没有做任何近似,该式和\autoref{TDPT_eq1} 完全等效.

\subsection{含时微扰理论}
若哈密顿算符中的势能包含时间,只有极少数情况下存在解析解.这时我们可以用\textbf{含时微扰理论}来近似求解.类比不含时微扰理论,我们引入一个常数 $\lambda$ 来分离不同阶数的近似,最后只需令 $\lambda = 1$ 即可.理论上当阶数足够高时,近似解将会收敛到精确解.% 未完成:真的吗??

令哈密顿算符,系数矢量分别为
\begin{equation}
H = H_0 + \lambda H'(t)
\end{equation}
\begin{equation}
\bvec c(t) = \bvec c^{(0)}(t) + \lambda\bvec c^{(1)}(t) + \lambda^2\bvec c^{(2)}(t) \dots
\end{equation}
代入\autoref{TDPT_eq3},根据 $\lambda = 1$ 的阶数分离方程,得
\begin{align}
&\dv{t} \bvec c^{(0)}(t) = \bvec 0 &&\text{(0 阶近似)}\label{TDPT_eq4}\\
&\dv{t} \bvec c^{(1)}(t) = \frac{1}{\I\hbar} \tilde{\mat H}(t) \bvec c^{(0)}(t) &&\text{(1 阶近似)} \label{TDPT_eq5}\\
%\vdots\notag \\
&\dv{t} \bvec c^{(n)}(t) = \frac{1}{\I\hbar} \tilde{\mat H}(t) \bvec c^{(n-1)}(t) &&\text{($n$ 阶近似)} \label{TDPT_eq6}
\end{align}

为了求解各阶近似,我们假设 $t=0$ 时只有 0 阶系数 $\bvec c^{(0)}(0)$ 不为零.若给出初始波函数 $\ket{\psi(0)}$,可用 $\ket{\phi_n(0)} = \ket{\psi_n}$ 展开得到 $\bvec c^{(0)}(0)$.\autoref{TDPT_eq4} 说明零阶系数矢量为常数,所以零阶近似解就是 $\bvec c^{(0)}(t) = \bvec c^{(0)}(0)$.继续把 $\bvec c^{(0)}(t)$ 代入\autoref{TDPT_eq5},两边对时间从 0 到 $t$ 定积分(矢量的积分即对每个分量分别积分)得
\begin{equation}
\bvec c^{(1)}(t) - \bvec c^{(1)}(0) = \frac{1}{\I\hbar} \int_{0}^{t}  \tilde{\mat H}(t) \bvec c^{(0)}(t) \dd{t}
\end{equation}
代入 $\bvec c^{(1)}(0) = 0$,$\bvec c^{(0)}(t) = \bvec c^{(0)}(0)$,得一阶近似解为
\begin{equation}\label{TDPT_eq9}
\bvec c^{(1)}(t) = \frac{1}{\I\hbar} \int_{0}^{t}  \tilde{\mat H}(t)\bvec c^{(0)}(0) \dd{t}
\end{equation}
类似地,对\autoref{TDPT_eq6} 积分,若已知 $\bvec c^{(n-1)}(t)$, 有
\begin{equation}\label{TDPT_eq8}
\bvec c^{(n)}(t) = \frac{1}{\I\hbar} \int_{0}^{t}  \tilde{\mat H}(t) \bvec c^{(n-1)}(t) \dd{t}
\end{equation}
所以要想得到 $n$ 阶近似解,积分 $n$ 次即可. 为了明确起见,\autoref{TDPT_eq8} 的分量表达式为
\begin{equation}\label{TDPT_eq10}
c_i^{(n)}(t) = \frac{1}{\I\hbar} \int_{0}^{t}  \sum_j H'_{ij}(t) \E^{\I\omega_{ij}t} c_j^{(n-1)}(t) \dd{t}
\end{equation}

\subsection{简单的一阶微扰}
大多数情况下我们只使用一阶近似,一种简单且常见的情况是,若初态为 $H_0$ 的某个本征态 $\ket{\psi_j}$
\begin{equation}
c_i(t) = \frac{1}{\I\hbar} \int_0^t \mel{i}{H'(t)}{j} \E^{\I\omega_{ij} t} \dd{t}
\end{equation}
我们把 $c_i(t)$ 叫做\textbf{跃迁幅(transition amplitude)}. 所以在一阶近似中,波函数在 $t$ 时刻出现在 $\ket{\psi_i}$ 的概率约为
\begin{equation}
P_{ij}(t) = \abs{c_i(t)}^2
\end{equation}
注意一阶微扰仅当 $P_{ij} \ll 1$ 时有效.

在此基础上, 一种更简单的情况是: 如果 $H'(t)$ 中的时间函数可以分离出来
\begin{equation}
H'(t) = W f(t)
\end{equation}
其中 $W$ 是一个不含时的算符. 令 $W_{ij} = \mel{\psi_i}{W}{\psi_j}$, 有
\begin{equation}
H'_{ij}(t) = \mel{\psi_i}{H'}{\psi_j} = W_{ij}f(t)
\end{equation}
此时一阶微扰公式(\autoref{TDPT_eq9}) 变为 $f(t)$ 的反傅里叶变换\upref{FTExp}. 末态 $\ket{\psi_i}$ ($i \neq j$)的系数为
\begin{equation}
c_i(t) = \frac{W_{ij}}{\I\hbar} \int_0^t f(t)\E^{\I\omega_{ij} t} \dd{t}
\end{equation}
当 $f(t)$ 是一个波包的时候, 该积分就是 $f(t)$ 的(反)傅里叶变换 $\tilde f(\omega_{ij})$.
