% 堆(综述)
% license CCBYSA3
% type Wiki

(本文根据 CC-BY-SA 协议转载自原搜狗科学百科对英文维基百科的翻译)

在计算机科学中,堆是一种特殊的基于树的数据结构,本质上是一棵几乎完全的[1]树, 它满足堆属性:在最大堆中,对于任意给定的节点C,如果P是C的父节点,那么P的键(值)大于或等于C的键。在最小堆中,P的键小于或等于C的键。堆“顶部”的节点(没有父节点)称为根节点。

堆是称为优先队列的抽象数据类型的最有效实现,事实上优先队列通常被称为“堆”,不管它们是如何实现的。在堆中,最高(或最低)优先级的元素总是存储在根节点。然而,堆不是排序结构,它可以被认为是部分有序的。当需要重复移除优先级最高(或最低)的对象时,堆是一种有用的数据结构。

堆的一个常见实现是二叉堆,其中树是二叉树(见图)。堆数据结构,特别是二叉堆,是由J. W. J. Williams在1964年引入的,作为堆排序算法的数据结构。[2] 堆在一些高效的图算法中也很重要,比如Dijkstra算法。当堆是一棵完全二叉树时,它有一个最小的可能高度——一个有N个节点的堆,对于每个节点,一个分支总是有loga N 的高度。

注意,如图所示,兄弟和堂兄弟之间没有隐含的排序,也没有有序遍历的隐含序列(例如,在二叉搜索树中)。上面提到的堆关系仅适用于节点和它们的父亲、祖父等之间。每个节点可以拥有的最大子节点数取决于堆的类型。

\subsection{操作}
涉及堆的常见操作有:

基础
\begin{itemize}
\item 查找最大值(或查找最小值):分别查找最大堆的最大项或最小堆的最小项(也称为peek)
插入:向堆中添加一个新键(也可称push[3])
\item 提取最大值(或提取最小值):从堆中移除节点后,返回最大堆的最大值[或最小堆的最小值](也称为pop)[4])
\item 删除-最大(或删除-最小):分别删除最大堆(或最小堆)的根节点
\item 替换:弹出根节点并压入新键。比先进行pop,后进行push更有效,因为只需要平衡一次,而不是两次,并且适合固定大小的堆。[5]
\end{itemize}

创建
\begin{itemize}
\item 创建堆:创建一个空堆
\item 堆化:用给定的元素数组创建一个堆
\item 合并(merge、union):连接两个堆,形成一个包含两个堆的所有元素的有效新堆,保留原始堆。
\item 合并(meld):连接两个堆,形成一个包含两个堆的所有元素的有效新堆,销毁原始堆。
\end{itemize}

检查
\begin{itemize}
\item 
\end{itemize}
