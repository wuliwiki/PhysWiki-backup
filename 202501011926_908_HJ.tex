% 斯蒂芬·霍金(综述)
% license CCBYSA3
% type Wiki

本文根据 CC-BY-SA 协议转载翻译自维基百科\href{https://en.wikipedia.org/wiki/Stephen_Hawking}{相关文章}。

\begin{figure}[ht]
\centering
\includegraphics[width=6cm]{./figures/d175407efe80fdaf.png}
\caption{霍金,大约1980年} \label{fig_HJ_1}
\end{figure}
斯蒂芬·威廉·霍金(Stephen William Hawking,1942年1月8日—2018年3月14日)是英国的理论物理学家、宇宙学家和作家,曾担任剑桥大学理论宇宙学研究中心的研究主任。[6][17][18] 从1979年到2009年,他是剑桥大学的卢卡斯数学教授,这一职位被广泛认为是世界上最具声望的学术职务之一。[19]

霍金出生于牛津,来自一个医学世家。1959年10月,17岁的他开始在牛津大学大学学院学习,并获得了物理学一等荣誉学位。1962年10月,他开始在剑桥大学三一学院攻读研究生,并于1966年3月获得应用数学和理论物理学博士学位,专业方向为广义相对论和宇宙学。1963年,霍金在21岁时被诊断为一种早期发病、进展缓慢的运动神经元病,这种病症在几十年中逐渐使他瘫痪。[20][21] 失去语言能力后,他通过语音生成设备进行交流,最初使用手持开关,后来通过单个面部肌肉来控制设备。[22]

霍金的科学成就包括与罗杰·彭罗斯(Roger Penrose)合作研究广义相对论框架下的引力奇点定理,以及理论预测黑洞会发射辐射,这一现象通常被称为霍金辐射。最初,霍金辐射的预测颇具争议。但到了1970年代末,随着进一步研究的发表,这一发现被广泛接受,成为理论物理学中的重大突破。霍金是第一个提出将广义相对论与量子力学结合来解释宇宙学的理论的人。他是多世界解释的积极支持者。[23][24] 他还提出了微型黑洞的概念。[25]

霍金通过几部畅销的科普作品取得了商业成功,他在书中讨论了自己的理论和宇宙学问题。他的著作《时间简史》曾连续237周登上《星期日泰晤士报》畅销书榜,创下纪录。霍金是英国皇家学会会员、教宗科学院终身会员,并获得了美国总统自由勋章,这是美国的最高平民荣誉奖。2002年,霍金在BBC的“100位最伟大的英国人”评选中排名第25位。霍金于2018年去世,享年76岁,诊断为运动神经元病后,他活过了50多年。
\subsection{早年生活} 
\subsubsection{家庭}  
霍金于1942年1月8日出生在牛津,父母是弗兰克·霍金和伊莎贝尔·艾琳·霍金(娘家姓沃克)。霍金的母亲出生在苏格兰格拉斯哥的一个医生家庭。霍金的父亲,来自约克郡的富有曾祖父,在20世纪初的大农业萧条中因购买农田过度投资而破产。霍金的曾祖母通过在家中开设学校,成功将家庭从经济困境中解救出来。尽管家庭经济条件拮据,霍金的父母都进入了牛津大学,弗兰克学习医学,伊莎贝尔学习哲学、政治学和经济学。伊莎贝尔曾在一家医学研究所担任秘书,而弗兰克是一名医学研究员。霍金有两个妹妹,菲利帕和玛丽,以及一个收养的哥哥,爱德华·弗兰克·大卫(1955年–2003年)。

1950年,当霍金的父亲成为国家医学研究所寄生虫学部的负责人时,霍金一家搬到了赫特福德郡的圣奥尔本斯。在圣奥尔本斯,霍金一家被认为是非常聪明且有些古怪;全家人常常在一起吃饭时各自安静地看书。他们生活朴素,住在一栋大而凌乱、维护不善的房子里,出行则开着一辆改装过的伦敦出租车。在霍金的父亲因工作常常前往非洲时,霍金一家曾在马约卡岛度过四个月,探访霍金母亲的朋友贝里尔和她的丈夫,诗人罗伯特·格雷夫斯。
\subsubsection{小学和中学时期}  
霍金在伦敦的拜伦之家学校开始了他的学业。他后来指责该校的“进步教育方法”使他在学校时未能学会阅读。在圣奥尔本斯,年仅8岁的霍金曾在圣奥尔本斯女子中学就读了几个月。当时,年轻男孩可以进入某些楼宇上课。

霍金就读了两所私立学校,首先是拉德莱特学校,然后从1952年9月起,进入了圣奥尔本斯学校。霍金早早通过了11+考试,并提前一年进入圣奥尔本斯学校。霍金一家对教育非常重视。霍金的父亲希望他能进入威斯敏斯特学校,但13岁的霍金在考试当天生病,因此未能参加奖学金考试。由于没有奖学金的经济资助,霍金的家庭负担不起学费,因此他继续留在圣奥尔本斯学校。这个决定有一个积极的后果——霍金得以与一群喜欢玩桌游、制造烟花、制作模型飞机和船只的朋友们保持密切联系,还一起长时间讨论基督教和超感知知觉等话题。从1958年开始,在数学老师迪克兰·塔塔的帮助下,他们用时钟零件、旧电话交换机和其他回收零件一起造了一台计算机。

虽然在学校里他被称为“爱因斯坦”,但霍金最初并不是学术上非常成功。随着时间的推移,他开始展现出相当强的科学天赋,并在塔塔的启发下,决定在大学学习数学。霍金的父亲建议他学习医学,担心数学专业毕业生的工作机会较少。他还希望儿子能进入自己母校——牛津大学。由于当时牛津大学无法开设数学专业,霍金决定学习物理和化学。尽管校长建议他等到明年再考,霍金还是在1959年3月参加了考试并获得了奖学金。
\subsubsection{本科时期}
霍金于1959年10月,在17岁时开始了在牛津大学大学学院的大学教育。在前18个月里,他感到无聊且孤单——他认为学术工作“轻而易举”。他的物理学导师罗伯特·伯曼后来表示,“他只需要知道某件事是可以做到的,他就能做到,而不需要看别人是怎么做的。”在他的第二和第三学年里,情况发生了变化。根据伯曼的说法,霍金开始更加努力“融入集体”,并逐渐发展成一个受欢迎、活泼且机智的大学成员,喜欢古典音乐和科幻小说。这一转变的一部分原因是他决定加入大学的划船俱乐部——大学学院划船俱乐部,并成为一名舵手。在当时的划船教练看来,霍金培养了一个“冒险者”的形象,他经常让划船队在充满风险的航道上划行,导致船只损坏。霍金估计,在牛津大学的三年里,他大约学习了1,000个小时。这些不太令人印象深刻的学习习惯使得他参加期末考试时面临挑战,他决定只回答那些需要理论物理知识的问题,而不回答那些需要记忆事实的问题。获得一等学位是他计划继续在剑桥大学攻读宇宙学研究生的条件。

考试前一天晚上,霍金焦虑不安,导致睡眠不好,考试结果处于一等和二等荣誉学位的边界,这使得他需要接受牛津大学考官的口试(viva)。霍金担心自己被认为是一个懒散且难以相处的学生。因此,在口试中当被问到自己的计划时,他回答道:“如果你给我一等学位,我就去剑桥。如果我得到二等学位,我就留在牛津,所以我希望你们会给我一等学位。”他比自己认为的更受人尊敬;正如伯曼所评论的,考官“足够聪明,意识到他们正在与一个比大多数人都聪明的人交谈”。在获得物理学一等学位后,并与朋友一起完成了去伊朗的旅行,霍金于1962年10月开始了在剑桥大学三一学院的研究生学习。
\subsubsection{研究生时期}
霍金的第一年博士生涯非常困难。最初,他感到失望,因为他被分配给了现代宇宙学的奠基人之一丹尼斯·威廉·赛阿玛作为导师,而不是著名的天文学家弗雷德·霍伊尔。霍金还发现,他在数学方面的训练不足以应对广义相对论和宇宙学的研究工作。在被诊断出患有运动神经元病后,霍金陷入了抑郁症——尽管医生建议他继续学业,但他觉得这样做没有什么意义。他的病情进展比医生预期的要慢。尽管霍金无法独立行走,且他的言语几乎无法理解,但最初医生预测他只剩下两年寿命的诊断并未成真。在赛阿玛的鼓励下,霍金重新投入到他的研究中。

霍金在学术界开始以聪明和大胆著称,当他在1964年6月的一次讲座上公开挑战霍伊尔及其学生贾扬特·纳尔利卡的工作时,他的声誉得到了进一步的提升。

当霍金开始他的博士研究时,物理学界关于宇宙创生的主流理论——大爆炸理论与稳态理论——存在着激烈的争论。在罗杰·彭罗斯关于黑洞中心时空奇点定理的启发下,霍金将这种思维方式应用到整个宇宙上;并在1965年完成了以此为主题的博士论文。霍金的论文于1966年获得批准。此外,还有其他积极的发展:霍金获得了剑桥大学冈维尔与凯厄斯学院的研究奖学金;他于1966年3月获得应用数学与理论物理博士学位,专业为广义相对论和宇宙学;他的论文《奇点与时空几何》与彭罗斯的论文并列获得当年著名的亚当斯奖。