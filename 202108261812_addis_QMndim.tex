% 多维空间中的量子力学
% 量子力学|梯度|张量积空间|本征值

\pentry{张量积空间\upref{DirPro}, 梯度\upref{Grad}}

\subsection{单个粒子在多维空间中的波函数}
本词条中的“多维” 指的是二维和三维. 与牛顿力学一样, 在学习完粒子的直线运动(一维)后, 我们希望能了解粒子在平面上的运动(二维), 或者空间中的运动(三维). 在多于一维的情况下, 波函数变为位置矢量 $\bvec r$ 以及时间 $t$ 的函数
\begin{equation}
\Psi(\bvec r, t)
\end{equation}
例如在二维直角坐标系中, $\Psi(\bvec r, t) = \Psi(x, y, t)$, 又例如在三维的球坐标系中, $\Psi(\bvec r, t) = \Psi(r, \theta, \phi, t)$
% 未完成: 概率

\subsection{矢量算符}
在多维空间中, 位置和动量分别从标量拓展为矢量, 所以对应地, 位置算符和动量算符也分别拓展为\textbf{矢量算符}(我们暂时把这两个算符的定义看作是量子力学的基本假设)
\begin{equation}
\Q{\bvec r} = \bvec r = x\uvec x + y\uvec y + z\uvec z
\end{equation}
\begin{equation}\ali{
&\quad \Q{\bvec p} = -\I\hbar \grad = \Q p_x \uvec x + \Q p_y \uvec y + \Q p_z \uvec z\\
&= \qty(-\I\hbar \pdv{x}) \uvec x + \qty(-\I\hbar \pdv{y}) \uvec y + \qty(-\I\hbar \pdv{z}) \uvec z
}\end{equation}
当矢量算符作用在波函数上后, 得到的函数的自变量仍然是 $(\bvec r, t)$, 而函数值却变为一个复数矢量(矢量的三个分量都是复数). 把位置算符作用在波函数上, 就是把波函数分别乘以 $x, y, z$ 的坐标, 并作为函数值的三个分量. 而把动量算符作用在波函数上, 就先把各个方向的动量算符分别作用, 并作为函数值的三个分量. 所以在矢量算符的本征方程
\begin{equation}
\Q{\bvec Q} \Psi(\bvec r) = \bvec \lambda \Psi(\bvec r)
\end{equation}
中, 我们只有使用矢量本征值才能保证等号两边都是矢量. 矢量算符的本征方程也可以写成三个分量的形式
\begin{equation}
\Q Q_x \Psi(\bvec r) = \lambda_x \Psi(\bvec r) \qquad
\Q Q_y \Psi(\bvec r) = \lambda_y \Psi(\bvec r) \qquad
\Q Q_z \Psi(\bvec r) = \lambda_z \Psi(\bvec r)
\end{equation}

不难验证, 位置的本征函数就是三维 $\delta$ 函数
\begin{equation}
\delta(\bvec r - \bvec r_0) = \delta(x - x_0) \delta(y - y_0) \delta(z - z_0)
\end{equation}
对应的本征值为 $\bvec r_0$. 动量的本征函数就是三维平面波
\begin{equation}
\exp(\I \bvec k \vdot \bvec r) = \exp(\I k_x x) \exp(\I k_y y) \exp(\I k_z z)
\end{equation}
对应的本征值为 $\bvec p = \hbar\bvec k$.

\subsection{多粒子波函数}
我认为题主可能没有分清两个波包和两个粒子的区别. 根据你所说描述的两个波包(例如发生碰撞,干涉,等)并不代表两个粒子. 举个例子:

\begin{example}{两个粒子的库仑作用}
考虑一维波函数, $\Psi_1(x,t)$ 表示一个从左向右移动的波包, $\Psi_2(x,t)$ 表示从右到左移动的波包, 那么把它们叠加起来之后的波函数 $\Psi_1 + \Psi_2$ 代表什么? 是两个粒子的碰撞吗? 不是的,是同一个粒子和自身的干涉. 又如电子双缝干涉实验中,两个缝隙发出的波或波包互相干涉,是指两个不同的电子的干涉吗? 不是的, 是同一个电子. 双缝干涉实验强调一次只发射一个电子.既然是同一个电子,那么即使它的波函数由两个波包组成, 也不会存在库仑力, 只能互相发生干涉而已, 不存在一个波包把另一个推开的可能性.

如果要考虑两个直线运动的带电粒子(假设质量相同)在库仑力作用下发生速度的变化, 就要考虑双粒子波函数 $\Psi(x_1, x_2, t)$. 注意这是一个二维波函数, 而不是两个一维波函数相加. 相互作用(库仑力)体现在薛定谔方程的势能项中
\begin{equation}
-\frac{1}{2}\laplacian \Psi + \qty(V(x_1) + V(x_2) + \frac{q_1 q_2}{\abs{x_2 - x_1}}) \Psi = \I \pdv{t}\Psi
\end{equation}
例如考虑一维简谐振子势能中的两个粒子, 那么 $V(x_i) = ax_i^2/2$, 如果除了相互作用没有外部势能, 那么 $V(x_i) = 0$.

现在考虑两个直线运动的可区分(非全同\upref{IdPar})带电粒子($V(x_i) = 0$)以相同的初速度靠近并在库仑力作用下反弹, 可以假设初始的二维波函数 $\Psi(x_1, x_2, t)$ 是一个二维波包, 波包中心延着轨迹 $x_2 = -x_1$ 从左上方向坐标原点方向几乎匀速地运动. 注意这并不是两个一维波包的运动, 而是一个二维波包的运动. 从另一个角度来说,这个问题完全等效于二维平面 $x$-$y$ 上单个粒子的波包在二维势能 $V(x,y) = q_1q_2/\abs{y-x}$ 中的运动. 画出该势能的等势线会发现它们是和初始波包的运动轨迹 $y-x$ 垂直的, 且越靠近原点势能越大, 也就是初始波包一直在走上坡路. 那么初始波包在该势能的作用下会减速,并延着原来的路径反弹. 这就体现了第一个粒子 $x_1$ 先从负无穷靠近 $0$ 再反弹回负无穷, 第二个粒子 $x_2$ 则从正无穷靠近 $0$ 再反弹回正无穷, 也就是两个粒子在相遇以前就已经被库仑力反弹了.
\end{example}
