% 原子结构和波粒二象性(高中)
% 必修三能量子|普朗克黑体辐射|光电效应|原子核式结构|玻尔模型|粒子波动性|量子力学初步

\begin{issues}
\issueTODO
\end{issues}

%\pentry{功和机械能\upref{HSPM07},分子动力学\upref{thermo} }

\subsection{普朗克黑体辐射理论}
\subsubsection{黑体辐射}
在了解什么是黑体辐射之前,必须先明确黑体的定理。\textbf{黑体}指的是一种可以完全吸收入射的各种波长的电磁波而不将其反射的物体。我们常认为一个带有小孔的空腔可以作为一个黑体,因为如果有电磁波从小孔入射,在其中不管发生了多少次的反射和折射,都很难从空腔再度射出,因此满足对于黑体的定义。

一个常见的误区是,黑体虽然不能\textsl{反射},却仍然可以向外\textsl{辐射}电磁波,这是由于黑体在吸收能量的时候具有一定温度。这种辐射称之为\textbf{黑体辐射}。更进一步的,对于一般材料的热辐射,不仅仅和所选材料的温度有关,还和材料的种类以及表面状态相关,但是在黑体的情况下,黑体辐射的电磁波强度按照波长的分布仅仅和黑体的温度有关。

\subsubsection{黑体辐射的实验规律}
利用现代化设备,可以测出黑体辐射电磁波的强度按波长分布情况。实验表明,随着温度的升高,各个波长的辐射强度都有增加,这是符合我们直观认识的;与此同时,辐射强度的极大值会向着波长较短的方向进行移动。

这种现象该如何从微观上去进行解释呢?我们知道,物体存在着不停运动的带电微粒,每个带电微粒的振动都会产生变换的电磁场,这是物体电磁辐射的来源,而微粒的运动又和热密切相关,因此即可把温度和电磁辐射相互关联起来。德国物理学家维恩和英国物理学家瑞利分别提出了辐射强度按波长分布的理论公式。维恩公式在duanbo
% 补充图片 黑体辐射的实验规律 P68
\subsection{光电效应}
\subsection{原子核式结构}
\subsection{波尔模型}
\subsection{粒子波动性}
\subsection{量子力学初步}

%% 画图时间
% 
%% 错别字纠正
% 