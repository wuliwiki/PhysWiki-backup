% 氢原子波函数分析
% keys 氢原子|波函数|动量谱|电离|光电子

\pentry{库仑波函数\upref{CulmWf}}

\subsection{光电子动量谱}

在计算类氢原子光电离时, 当外场消失后, 每个能量本征态(散射态)的概率就固定不变了. 然而动量的本征态系数还是会变(除非时间无穷大). 要得到时间无穷大时电子的三维动量分布, 我们可以直接将波函数投影到库仑波函数(渐进平面波)上. 事实上这样的动量谱通常被称为 angular resolved energy spectrum (毕竟是能量的本征态), 为了方便我们还是直接叫做动量谱.
\begin{equation}
P(\bvec k) = \abs{f(\bvec k)}^2 = \abs{\braket{\Psi^{(-)}_{\bvec k}}{\Psi(\bvec r)}}^2
\end{equation}
归一化为
\begin{equation}
\int P(\bvec k) \dd[3]{k} = \iint P(\bvec k) \dd{\Omega} \cdot k^2\dd{k} = 1
\end{equation}

那么被电离的光电子(PE 或 photo-electron) 的能量谱如何计算呢? 先看动量大小的分布, 归一化为
\begin{equation}\label{Hanaly_eq2}
\int_0^\infty P(k) \dd{k} = 1
\end{equation}
能量分布归一化为
\begin{equation}\label{Hanaly_eq1}
\int P(E) \dd{E} = \int P(E) k\dd{k} = 1
\end{equation}
二者的关系为 $P(k) = kP(E)$.

我们只需要把总波函数投影到动量绝对值为 $k$ 的子空间上即可. 对比以上几式得
\begin{equation}
P(k) = k^2 \int P(\bvec k) \dd{\Omega}
\end{equation}

但是类氢原子的波函数一般在球坐标中进行, 我们试图直接用(径向)库仑函数来计算. 将波函数投影到归一化得库仑球面波\autoref{CulmWf_eq1}~\upref{CulmWf}得
\begin{equation}
P(l, m, k) = \abs{f_{l,m}(k)}^2 
\end{equation}
其中
\begin{equation}
f_{l,m}(k) = \braket{C_{l,m}(k)}{\Psi(\bvec r)} = \sqrt{\frac{2}{\pi}} \int F_l(k, r) \psi_{l,m}(r) \dd{r}
\end{equation}
其中 $\psi_{l,m}(r)$ 是 scaled 的径向波函数\autoref{RYTDSE_eq1}~\upref{RYTDSE}.

归一化为\footnote{为什么对 $k$ 得积分没有 $k^2$ 项? 这取决有库伦球面波得归一化\autoref{CulmWf_eq2}~\upref{CulmWf}.}
\begin{equation}
\sum_{l,m} \int P(l, m, k) \dd{k} = 0
\end{equation}
对比\autoref{Hanaly_eq2} 和\autoref{Hanaly_eq1} 得
\begin{equation}
P(k) = \sum_{l,m} P(l, m, k) = \frac{2}{\pi} \sum_{l,m} \abs{\int F_l(k, r) \psi_{l,m}(r) \dd{r}}^2
\end{equation}
虽然这个公式看起来只包括了径向动能的分布, 但实际上也有角向的动能, 体现在 $l$ 量子数里面\footnote{想一下库仑函数的微分方程中 $l$ 是如何决定角向动能的? 注意与 $m$ 无关.}.

\subsection{平均能量}
球坐标中的额外势能如果表示为\autoref{HyTDSE_eq6}~\upref{HyTDSE}
\begin{equation}
V'(\bvec r) = \sum_{l,m} V'_{l,m}(r) Y_{l,m}(\uvec r)
\end{equation}
那么对应的能量为
\begin{equation}
E' = \mel{\Psi}{V'(\bvec r)}{\Psi} = \sum_{l_1,m_1}\sum_{l_2,m_2}\sum_{l,m} \int \psi_{l,m}^*(r) V'_{l,m}(r) \psi_{l,m}(r) \dd{\Omega}\dd{r}
\end{equation}
