% 四川大学 2013 年硕士物理考试试题(933)
% license Usr
% type Note

\textbf{声明}:“该内容来源于网络公开资料,不保证真实性,如有侵权请联系管理员”

\subsection{(本题8分)}
如一,3题图所示,一理想气体分别经①、②、③过程出$A$状态到达 $B,C,D$状态,其中②)是绝热过程。试分析这三个过程中哪些是吸热过程,哪些是放热过程。
\begin{figure}[ht]
\centering
\includegraphics[width=6cm]{./figures/800d56d6ce15aa0d.png}
\caption{} \label{fig_CD13_1}
\end{figure}
\subsection{(本题6分)}
有直径为$16cm$及$10cm$ 的非常薄的两个铜制球壳,同心放置时,内球的电势为 $2700V$,外球带有电荷量为$8.0\times10^{-9}C$。现把内球和外球接触,两球的电势各变化多少$(\varepsilon_0=8.85\times10^{-12} C^2/(N\cdot m^2))$
\subsection{(本题6分)}
所示的三个闭合回路1、2、3,分别写出磁感应强度$B$沿它们的环流值。设$I_1=I_2=I$。并讨论:
\begin{enumerate}
\item 在每个闭合回路上各点的$B$值是否相等?
\item 在回路3上各点的$B$是否均等于零?
\end{enumerate}
\subsection{(本题8分)}
判断下列各一.6 题图中的导线段 AC.或者导线框内的感应电动势的方向。
\begin{figure}[ht]
\centering
\includegraphics[width=6cm]{./figures/2e68364f6c75a0b1.png}
\caption{} \label{fig_CD13_2}
\end{figure}