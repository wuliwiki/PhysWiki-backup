% 阶乘(高中)
% keys 整数|gamma 函数|pi 符号
% license Xiao
% type Tutor

\begin{issues}
\issueDraft
\end{issues}

%\pentry{\enref{求积符号(累乘)}{ProdSy}}{nod_fe64}
%933:不需要预备知识

在排列组合的学习中,经常会出现多个连续的整数相乘的情形。

\begin{example}{求:从7名学生中,选出3名学生前后站成一列的方法数}\label{ex_factor_1}
由于三人前后站成一列,所以对确定的三个人A、B、C,“ABC”和“CBA”是两种站法。因此,分别对三个位置进行考虑,第一个位置可以任选一个人,第二个位置可以从剩下的学生里再选一个人,第三个位置再从剩下的学生里选一个人。由乘法原理,共有$7\times6\times5=210$种方法。
\end{example}

数字少一点还好说,如果数字过多,要么要写一堆的数字和乘号,要么每次都要写$\cdots$。鉴于这个连续相乘的使用次数太多,数学家们发明了一种用来表示连乘的运算:阶乘。

对自然数 $n$\footnote{拓展知识:阶乘的定义本身只限于自然数。但随着研究深入,偶尔会出现使用分数等数值的阶乘的场景,为此,欧拉推导了\enref{Gamma 函数}{Gamma}$\Gamma(x)$实现了对阶乘的解析延拓,即:1.保证它的函数值与阶乘对应$\Gamma(n)=(n-1)!$;2.保证自变量取其他实数值(如:部分负数、分数、和无理数等)时也可以有结果;3.函数的性质满足特定条件。于是,数学中也经常会用“Gamma函数”来代替需要表达“阶乘”概念的场合。}, \textbf{阶乘}(factorial)定义为所有小于等于 $n$ 的正整数的乘积,即
\begin{equation}
n! := 1 \times 2 \times 3 \times \cdots \times(n - 2)\times(n - 1)\times n=\prod_{i = 1}^n i ~.
\end{equation}
其中,$\prod_{i = 1}^n i$是使用\enref{求积符号}{ProdSy}来简写乘积,它和阶乘一样是一个简记符号,如果觉得复杂先忽略也不影响。特殊地,约定
\begin{equation}
0! := 1~.
\end{equation}

\begin{example}{求5的阶乘}
小于等于5的正整数有:1,2,3,4,5,因此:
$$5! = 1\times 2\times 3\times 4\times 5 = 120~.$$
\end{example}

使用阶乘不仅可以表达从1开始的连乘,还可以利用除法,可以让两个阶乘做除法来约掉某些项。下面的两道例题是比较常遇到的情景。

\begin{example}{用阶乘表示"从某个自然数m到另一个自然数n的连续整数的乘积"}
\begin{align*}
m\times(m+1)\times\cdots \times(n-1)\times n&={1\times\cdots \times(m-1)\times m\times(m+1)\cdots \times(n-1)\times n\over 1\times\cdots \times(m-1)}\\
&={n!\over (m-1)!}~.
\end{align*}
\end{example}

\begin{example}{用阶乘表示"从自然数n开始依次递减的m个数的乘积"}
\begin{align*}
n\times(n-1)\times\cdots \times[n-(m-2)]\times [n-(m-1)]&={n\times\cdots \times [n-(m-1)]\times(n-m)\times\cdots\times1\over (n-m)\times\cdots \times1}\\
&={n!\over (n-m)!}~.
\end{align*}
\end{example}

因此,\autoref{ex_factor_1} 表示的就是“从自然数7开始依次递减的3个数的乘积”就可以简单写为:
$${7!\over (7-3)!}={7!\over 4!}=210.~$$

常用的阶乘数值如下\begin{table}[ht]
\centering
\caption{常用阶乘值}\label{tab_factor1}
\begin{tabular}{\textwidth}{|X|X|X|X|X|X|X|X|X|}
\hline
n & 0 & 1 & 2 & 3 & 4 & 5 & 6 & 7\\
\hline
n! & 1 & 1 & 2 & 6 & 24 & 120 & 720 & 5040\\
\hline
\end{tabular}
\end{table}
