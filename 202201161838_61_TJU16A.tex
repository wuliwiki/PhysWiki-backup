% 天津大学 2016 年考研量子力学答案
% keys 考研|天津大学|量子力学|2016|答案

\begin{issues}
\issueDraft
\issueTODO
\end{issues}


\subsection{ }
\begin{enumerate}
\item 
\item 不计自旋,守恒量有$\hat H,\hat {L^2},\hat {L_z} $.
\item $\because \quad A^{\dagger} = A,B^{\dagger} = B,(AB)^{\dagger} \neq AB $ \\
$\therefore (1)\quad (AB)^{\dagger} = B^*A^* = BA $,不是厄米算符. \\
$\quad (2)\quad (AB+BA)^{\dagger} = B^*A^*+A^*+B^* = AB+BA $,是厄米算符. \\
$\quad (3)\quad (AB-BA)^{\dagger} = B^*A^*-A^*B^* = -(AB-BA) $,不是厄米算符. \\
$\quad (4)\quad (ABA)^{\dagger} = (ABA)^* = ABA $,是厄米算符. \\
$\quad (5)\quad [i(AB-BA)]^{\dagger} = -i(BA-AB) = i(AB-BA) $,是厄密算符.
\end{enumerate}
\subsection{ }
\begin{enumerate}
\item 由$\psi (x,0) = \frac{1}{\sqrt{2}}[\varphi_{0}(x) + \varphi_{1}(x)] $可得任意时刻的波函数为:\\
$\psi(x,t) = \frac{1}{\sqrt{2}}\left[\varphi_{0}(x)e^{-\frac{iE_{0}t}{\hbar}} + \varphi_{1}(x)e^{-\frac{iE_{1}t}{\hbar}} \right] $
\item 由谐振子的基本性质,有:\\
$x\psi_{n} = \frac{1}{\alpha} \left[\sqrt{\frac{n}{2}\psi_{n-1}(x)} + \sqrt{\frac{n+1}{2}\psi_{n+1}(x)} \right] $ \\
所以任意时刻坐标的平均值为:\\
\begin{equation}
\begin{aligned}
\bar{x} =& \int^{\infty}_{-\infty} \psi^{*}(x)\psi \dd{x} \\
=& \frac{1}{2\alpha} \int^{\infty}_{-\infty} \left[\varphi^{*}_{0}(x)+\varphi^{*}_{1}(x)\right](x)\left[\varphi_{0}(x)+\varphi_{1}(x)\right] \dd{x} \\
=& \frac{1}{2\alpha} \left[\sqrt{\frac{1}{2}}\delta_{00}+\sqrt{\frac{1}{2}}\delta_{11} \right] \\
=& \frac{\sqrt{2}}{2} \alpha
\end{aligned}
\end{equation}
其中$\alpha = \sqrt{\frac{m\omega}{\hbar}} $.
\end{enumerate}
\subsection{ }
由一维无限深势阱基本结论得:\\

$E^{0}_{n} = \frac{n^{2} \pi^{2} \hbar^{2}}{2ma^{2}},\quad n = 1,2,3 $ \\

$\psi^{0}_{n} = \sqrt{\frac{2}{a}} \sin{\frac{n \pi x}{a}} $
