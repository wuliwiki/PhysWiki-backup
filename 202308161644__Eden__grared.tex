% 引力红移
% keys 引力红移
% license Xiao
% type Tutor

\pentry{万有引力、引力势能\upref{Gravty},牛顿运动定律、惯性系\upref{New3}}
\addTODO{增加爱因斯坦等效原理的词条}

爱因斯坦的弱等效原理指出,引力质量 $m_g$ 等于惯性质量 $m_i$,所以根据万有引力定律和牛顿运动定律 $\bvec F=-m_g \grad \Phi = m_i \bvec g$ 可以得到 
\begin{equation}
\bvec g = -\nabla\Phi~.
\end{equation}
其中 $\Phi$ 为引力势能,根据万有引力公式可以表达为
\begin{equation}
\Phi(\bvec x)=-\sum_{i} \frac{G m^{(i)}}{|\bvec x^{(i)}-\bvec x|}~.
\end{equation}
爱因斯坦等效原理说的另一件事是,在一个足够小的区域内(引力几乎是均匀的),无法通过任何局域的实验探测出引力的存在。例如站在地面上的人受到地面向上的支持力,它等效于一个以加速度 $-\bvec g$ 加速上升的电梯。