% 非结合代数(综述)
% license CCBYSA3
% type Wiki

本文根据 CC-BY-SA 协议转载翻译自维基百科\href{https://en.wikipedia.org/wiki/Non-associative_algebra}{相关文章}。

非结合代数[1](或称分配代数)是指一种域上的代数,其二元乘法运算不假定具有结合性。也就是说,一个代数结构 $A$ 若是域 $K$ 上的非结合代数,则它是一个 $K$-向量空间,并配备了一个 $K$-双线性的二元乘法运算 $A \times A \to A$,该运算可以是结合的,也可以不是。例子包括李代数、约当代数、八元数,以及带有叉乘运算的三维欧几里得空间。由于不假定乘法是结合的,因此必须使用括号来表示运算顺序。例如,$(ab)(cd)$、$(a(bc))d$ 和 $a(b(cd))$ 的结果可能完全不同。

这里的“非结合”意味着不要求结合律成立,但并不意味着结合律被禁止。换句话说,“非结合”就是“未必结合”的意思,正如“非交换”环中的“非交换”并不是绝对禁止交换律,而是指“未必交换”。

一个代数是幺代数(unital 或 unitary),如果它存在一个单位元 $e$,满足对代数中所有元素 $x$ 都有 $ex = x = xe$。例如,八元数是幺代数,但李代数从来都不是。

对 $A$ 的非结合代数结构,可以通过将其关联到其他结合代数来研究,这些结合代数是$A$ 作为 $K$-向量空间时其全体 $K$-自同态代数的子代数。其中有两个重要的例子:导子代数和包络代数(后者在某种意义上是“包含 $A$ 的最小结合代数”)。

更一般地,有些作者把非结合代数的概念扩展到交换环 $R$ 上:即一个带有 $R$-双线性二元乘法运算的 $R$-模[2]。如果一个结构满足除了结合律以外的所有环公理(例如任何 $R$-代数),那么它自然就是一个 $\mathbb{Z}$-代数,因此有些作者称非结合的 $\mathbb{Z}$-代数为非结合环。
\subsection{满足恒等式的代数}
具有两个二元运算、但没有其他限制的类环结构是一类非常广泛的对象,过于笼统而难以研究。出于这个原因,最为人熟知的非结合代数类型往往满足某些恒等式或性质,从而在一定程度上简化了乘法。这些性质包括以下几类。
