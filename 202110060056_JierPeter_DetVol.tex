% 行列式与体积
% 行列式|体积|二阶|三阶|平行四边形|平行六面体|立方体|行变换

\begin{issues}
\issueDraft
\end{issues}

\pentry{行列式的性质\upref{DetPro}}

\subsection{向量组的体积}

在 “行列式\upref{Deter}” 中我们看到了二阶和三阶行列式从几何上分别对应平行四边形的面积(即二维体积)和平行六面体的体积. 我们现在来证明 $N > 0$ 维空间的情况.

\begin{theorem}{行列式就是向量组的体积}
在标准正交基下,所有向量都可以用列矩阵来表示.在$N$维空间中,选出$N$个向量,将它们的列矩阵排成一行,得到一个$N\times N$方阵$M$.那么以这组向量为边长的平行多面体的“体积”,就是$M$的行列式$\abs{M}$.
\end{theorem}


我们先来看最简单的例子: 一个对角线\footnote{行列式或矩阵的对角线特指所有行标和列标相同的元素}元素都为 1, 其他元素为零的行列式
\begin{equation}
\begin{vmatrix}
1 & & &\\
  & 1 & &\\
  &  & \ddots &\\
  & & & 1
\end{vmatrix} = 1
\end{equation}
这代表 $N$ 维空间中边长都是 1 的立方体的体积.

进一步,
\begin{equation}
\begin{vmatrix}
a_1 & & &\\
  & a_2 & &\\
  &  & \ddots &\\
  & & & a_N
\end{vmatrix} = a_1a_2\cdots a_N
\end{equation}
是$N$维空间中各边长为$$

% $N$ 维空间中的任意平行体都可以由该立方体经过两个操作(包括任意次序多次操作)得到. 一个是将某条边长乘以常数 $\lambda$ (可以为负), 第二个是将某条边所在的矢量乘以常数 $\lambda$ 然后加到另一条边上. 前者使体积乘以 $\abs{\lambda}$, 后者保持体积不变.

如果我们把行列式的每一行(或每一列, 下同)对应到平行体的每条边, 那么根据\autoref{DetPro_the3}~\upref{DetPro} 和\autoref{DetPro_the5}~\upref{DetPro}, 前者同样使行列式的值乘以常数 $\lambda$, 后者同样保持行列式的值不变. 从立方体开始所以经过一系列变换后, 行列式的绝对值同样表示对应平行体的体积.
