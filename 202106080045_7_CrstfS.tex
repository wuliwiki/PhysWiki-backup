% Christoffel符号
% 克里斯托费尔|克氏符|克氏|Christoffel|测地线|geodesic|广义相对论|relativity

\pentry{仿射联络(流形)\upref{affcon}, 爱因斯坦求和约定\upref{EinSum}}

仿射联络的定义是高度抽象的,并不涉及具体的运算.在物理学中,计算出某个参考系中各切向量的坐标分量是必须的,因为这些坐标分量才是观察者能看到、测量到的直接现象.Christoffel符号就是方便进行运算的一个概念.

本词条中默认$(M, \nabla)$为一个带仿射联络的实流形.

\subsection{联络形式}

对于任意$p\in M$,取$p$的一个邻域$U\subseteq M$,使得存在一组光滑向量场$\{\uvec{e}_i\}$构成$C^{\infty}(U)$上的一组基.这就是说,$U$上的每个光滑向量场都可以表示为$f^i\uvec{e}_i$的形式,其中各$f^i$是$U$上的光滑函数.

对于任意$X\in\mathfrak{X}(U)$,我们知道$\nabla_X\uvec{e}_i$也是一个光滑向量场,因此存在一组\textbf{光滑函数}$W^j_i(X)$,使得$\nabla_x\u$

















