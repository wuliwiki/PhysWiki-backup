% 华中师范大学 2012 年 考研 量子力学
% license Usr
% type Note

\textbf{声明}:“该内容来源于网络公开资料,不保证真实性,如有侵权请联系管理员”

\subsection{选择题(共 18分,每小题3分)}

1.在给定的状态 $\psi(x,t)$ 中 “测量” 粒子的坐标$\underline{\hspace{2cm}}$\\
    A、测量” 使得 $\psi(x,t)$ 不再按照薛定谔方程演化\\
    B、测量” 使得微观粒子不在任何位置\\
    C、“测量” 使得微观粒子的坐标越精确动量就越精确\\
    D、“测量” 使得 $\psi(x,t)$ 突然和不连续的坍塌\\\\
2.坐标对时间导数的算符是$\underline{\hspace{2cm}}$\\
    A、坐标算符 \\
    B、动量算符 \\
    C、速度算符 \\
    D、角动量算符\\\\
3.设质量为 $m$ 粒子的两个本征函数分别是 $\psi_1(x) = c_1e^{-ax^2/2}$,$\psi_2(x) = c_2(x^2+b)e^{-ax^2}$,则粒子这两状态的能级间隔为$\underline{\hspace{2cm}}$\\
    A、$-\hbar^2/mb$\\
    B、$-\hbar^2/(mb)^2$\\
    C、$-\hbar/mb$\\
    D、$\hbar^2/(mb)^2$\\\\
4.对于任意的 $\mathbf{a}$,若 $(\mathbf{a}|\mathbf{b}) = (\mathbf{a}|\mathbf{c})$,则$\underline{\hspace{2cm}}$\\
    A、$\mathbf{b} \ne \mathbf{c}$\\
    b、$\mathbf{b} = \mathbf{c}$\\
    c、$(\mathbf{a}| \mathbf{b}) = (\mathbf{a}| \mathbf{c})$\\
    D、$(\mathbf{a}| = \mathbf{b})$\\\\
5.在一维情况下,若 $U(x) \text{连续}, U(\pm \infty) = 0$ 且 $U(x) < 0$,则该体系$\underline{\hspace{2cm}}$\\\\
    A、两个束缚态\\
    B、无束缚态\\
    C、一个束缚态\\
    D、至少存在一个束缚态\\\\
6.微观体系存在任意态 $\psi(x)$ 中,能量的平均值 $\bar{E}\underline{\hspace{2cm}}$\\
    A、$\leq$体系的基态能量\\
    B、没有确定值\\
    C、$\geq$ 体系的基态能量 \\
    D、= 体系的基态能量\\
\subsection{问答题(共 32 分,每小题 8 分)}
\begin{enumerate}
\item 试论述量子力学的基本假设。
\item 量子力学在描述体系运动时,可以选择不同的绘景和不同的表象,试分别论
述薛定谔绘景和海森伯绘景及其特点。
\item $A$ 是一个守恒量,但并不一定意味着它在任意状态中都取确定的值,这是为
什么?
\item 如何理解微观粒子的波粒二象性?
\end{enumerate}
\subsection{计算题(共 50 分,每小题 25 分)}
\begin{enumerate}
\item 一维无限深势阱中的粒子受到$H' = bx$作用(其中$b$为常数),求能级和波函
数的一级修正。
\item 设体系的哈密顿量为 
\[
\hat{H}(t) = \hat{H}_0 + H'(t)~
\]
其中
\[
\hat{H}_0 \psi_k^{(0)}(x) = E_k^{(0)} \psi_k^{(0)}(x)~
\]

$k=1.2..$,求微扰
\[
\hat{H}'(t) = 
\begin{cases} 
0, & t < 0 \\
\hat{A} \cos(\omega t), & t \geq 0 
\end{cases}~
\]
作用下,体系的跃迁概率。

\end{enumerate}
\subsection{证明题(共 50 分,每小题 25 分)}
\begin{enumerate}
\item 能量为 $E$ 的粒子沿 $x$ 轴正向射入到一维势场 
\[
U(x) = 
\begin{cases} 
U_0, & 0 < x < a \\
0, & x < 0 , x > a 
\end{cases}~
\]
其中 $U_0 > 0$。

证明在 $E > U_0$ 的情况下,粒子的反射系数 $R$ 和透射系数 $D$ 满足:$D + R = 1$。

\end{enumerate}

