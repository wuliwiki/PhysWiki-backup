% 连分数

\footnote{参考 Wikipedia \href{https://en.wikipedia.org/wiki/Continued_fraction}{相关页面}。}\textbf{连分数(continued fraction)}形如
\begin{equation}
a_0 + \frac{1}{\displaystyle a_1 + \frac{1}{\displaystyle a_2 + \frac{1}{\displaystyle \ddots + \frac{1}{a_n}}}}
\end{equation}
其中要求所有 $a_i$ 都是整数, 且除 $a_0$ 外, 其他 $a_i$ 都大于零。 若有无穷多项即 $n\to\infty$, 则将其称为\textbf{无穷连分数}。 如果不要求 $a_i$ 为整数, 也不要求分子都为 1, 那么称其为\textbf{广义连分数}, 见下文。 为了方便书写, 也可以记为
\begin{equation}
a_0 + \frac{1}{a_1 + \dots}\frac{1}{a_2 + \dots}\dots \frac{1}{a_n}
\end{equation}
或者更简洁地, 记为
\begin{equation}
[a_0;\ a_1,\ a_2,\ \dots\ ,\ a_n]
\end{equation}

每个实数都可以被表示为一个唯一的连分数: 对实数 $x$, 取其整数部分为 $a_0$, 把剩余部分取倒数, 令 $x_1 = 1/(x-a_0)$, 那么它的整数部分就是 $a_1$, 再把剩余部分取导数, 令 $x_2 = 1/(x_1 - a_1)$, 它的整数部分就是 $a_2$, 以此类推。

\subsection{广义连分数}
广义连分数的形式为
\begin{equation}
b_0 + \frac{a_1}{\displaystyle b_1 + \frac{a_2}{\displaystyle b_2 + \frac{a_3}{\displaystyle b_3 + \dots}}}
\end{equation}
其中 $a_i, b_i$ 不要求是整数, 甚至可以是函数。
