% 微分拓扑(综述)
% license CCBYSA3
% type Wiki

本文根据 CC-BY-SA 协议转载翻译自维基百科\href{https://en.wikipedia.org/wiki/Differential_topology}{相关文章}

在数学中,微分拓扑是研究光滑流形的拓扑性质和光滑性质\(^\text{[a]}\)的领域。从这个意义上说,微分拓扑不同于密切相关的微分几何。微分几何关注的是光滑流形的几何性质,包括大小、距离和刚性形状等概念;而微分拓扑关注的是更为粗略的性质,例如流形中有多少个“洞”、它的同伦型或其微分同胚群的结构。由于这些粗略性质中的许多可以通过代数手段来刻画,微分拓扑与代数拓扑有着紧密的联系\(^\text{[1]}\)。
\begin{figure}[ht]
\centering
\includegraphics[width=6cm]{./figures/6d7860d55520d137.png}
\caption{} \label{fig_WFTP_1}
\end{figure}
微分拓扑领域的核心目标是对所有光滑流形按微分同胚(diffeomorphism)进行分类**。由于维数在微分同胚意义下是光滑流形的不变量,这一分类工作通常通过分别研究各个维度中的(连通)流形来进行:

\begin{itemize}
\item 一维(dimension 1)**
  在一维中,按微分同胚分类后,唯一的光滑流形包括:圆(circle)、实数轴(real number line),以及带边界的半闭区间 $[0,1)$ 和闭区间 $[0,1]$\[2]。

\item 二维(dimension 2)**
  在二维中,每一个闭曲面(closed surface)都可以通过其亏格(genus,洞的数量,或者等价地说其欧拉示性数)以及是否可定向(orientable)来按微分同胚进行分类。这就是著名的**闭曲面分类定理**\[3]\[4]。然而,即便在二维情况下,非紧曲面(non-compact surfaces)的分类也变得困难,例如由于存在“Jacob's ladder”这类特殊空间,使问题更加复杂。

\item 三维(dimension 3)**
  在三维中,威廉·瑟斯顿(William Thurston)提出的**几何化猜想(geometrization conjecture)**,由格里高利·佩雷尔曼(Grigori Perelman)证明,给出了紧三维流形的部分分类。其中包括著名的**庞加莱猜想(Poincaré conjecture)**,它指出任何闭的、单连通的三维流形都与三维球面(3-sphere)同胚(实际上也是微分同胚的)。

\end{itemize}