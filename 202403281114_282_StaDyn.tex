% 静态和动态编程语言
% license Usr
% type Tutor

C 语言的每一个函数(如果没有被 inline)都对应二进制文件中的一个地址。 二进制文件是写好的,不能运行时修改的。内存中专门有一块区间用于存放程序指令。 CPU 有一个 program counter 就是用于记录当前执行到程序的哪个位置。程序指令在运行时是不可以改动的,写死在程序中的数据(literal)也不可以改动。

函数中的每个本地变量都对应 stack 上的一块固定大小的,相对位置也确定的内存。这些也是编译时候定死的。

运行时可以改变的东西一个是 stack 上的具体数据, 另一个就是 heap 中的动态内存分配以及上面的数据。

静态语言和动态语言的最本质区别就是,静态语言的函数和变量都是对应到程序文件中和 stack 中的具体位置的。动态语言一切都是可以改变的,变量并不具有固定地址而是类似于指针,可以指向任何地方。 函数也不对应

判断语句和函数调用都可以根据运行时的数据让 program counter 跳到想要的地方。

静态语言的类型信息在编译后就完全丢失了,除非手动以变量的形式保存。 那么静态语言如何实现 poly 
