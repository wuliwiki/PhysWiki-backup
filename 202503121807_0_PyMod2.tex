% Python 创建模块笔记
% license Xiao
% type Note

\begin{issues}
\issueDraft
\end{issues}

\subsection{区分脚本和模块}
有时候我们希望同一个 \verb`.py` 文件既可以作为脚本直接执行也可以作为模块被导入, 并且希望该文件可以根据这两种方式自动选择执行什么。 这时可以用 \verb`__name__` 来判断:
\begin{lstlisting}[language=python, caption=my\_module.py]
#!/usr/bin/python3
# 执行一些命令(无论作为脚本还是模块都会被运行)
# 被赋值的变量会作为模块的全局变量, 定义的函数会作为模板中的函数

print('欢迎使用 my_module.py')

def plus1(num):
    return num + 1

num = 3.5
num2 = plus1(num)

if __name__ == '__main__':
    # 脚本模式下运行的命令
    print('正在被作为脚本执行,  __name__ 的值为 __main__')
else:
    # 作为模块导入时运行的命令
    print('正在被作为模块导入, __name__ 的值为', __name__)
\end{lstlisting}
若运行 \verb`./my_module.py` 或者 \verb`python3 my_module.py` 或者 \verb`python3 -m my_module`, 就会得到(\verb`-m` 选项把模块作为脚本运行)
\begin{lstlisting}[language=none]
欢迎使用 my_module.py
正在被作为脚本执行,  __name__ 的值为 __main__
\end{lstlisting}
若在同目录下进入 python3 运行 \verb`import my_module` (或者 \verb`import my_module as 其他名字`), 就会得到
\begin{lstlisting}[language=none]
欢迎使用 my_module.py
正在被作为模块导入, __name__ 的值为 my_module
\end{lstlisting}

再来写另一个模块调用上面的模块
\begin{lstlisting}[language=python,caption=my\_module2.py]
#!/usr/bin/python3

print('欢迎使用 my_module2.py')

def plus2(num):
    return num + 2

if __name__ == '__main__':
    # 脚本模式下运行的命令
    print('正在被作为脚本执行,  __name__ 的值为 __main__')
else:
    # 作为模块导入时运行的命令
    print('正在被作为模块导入, __name__ 的值为', __name__)

import my_module
\end{lstlisting}

在同目录打开命令行, 运行
\begin{lstlisting}[language=python]
>>> import my_module2
欢迎使用 my_module2.py
正在被作为模块导入, __name__ 的值为 my_module2
欢迎使用 my_module.py
正在被作为模块导入, __name__ 的值为 my_module
>>> my_module2.plus2(2)
4
>>> my_module2.my_module.plus1(2)
3
>>> my_module2.my_module.plus1.__module__
'my_module'
>>> my_module2.plus2.__module__
'my_module2'
# fun.__globals__['__file__'] 可以查看一个函数是在哪个文件中定义的(不能是 builtin 函数)。
>>> my_module2.plus2.__globals__['__file__']
'/mnt/c/Users/addis/Desktop/my_module2.py'
>>> my_module2.my_module.plus1.__globals__['__file__']
'/mnt/c/Users/addis/Desktop/my_module.py'
\end{lstlisting}

\subsection{搜索顺序}
\begin{itemize}
\item 当 \verb`import` 一个包时, Python 以什么顺序在哪些路径搜索所需的包呢?参考\href{https://docs.python.org/3/tutorial/modules.html#the-module-search-path}{官方文档}。
\end{itemize}
