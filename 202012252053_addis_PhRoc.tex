% 光子火箭

\begin{issues}
\issueDraft
\end{issues}

一个光子的能量
\begin{equation}
\hbar\omega = mc^2
\end{equation}
一个光子的动量
\begin{equation}
p = mc = \frac{\hbar\omega}{c}
\end{equation}
功率和推力的关系
\begin{equation}
F = p \dv{N}{t} = \frac{\hbar\omega}{c}\dv{N}{t} = \frac{P}{c}
\end{equation}
所以 $1N$ 的推力需要 $3\times 10^8\Si{W}$ 的功率! 这与光子的能量无关.

如果使用光子火箭, 用反物质湮灭产生的光子推动, 有
\begin{equation}
F = \frac{P}{c} = c\dv{m}{t}
\end{equation}
产生 100 吨的推力只需要每秒消耗 3.3 克反物质, 而功率却是惊人的 $3\times 10^{14}\Si{W}$, 可以每秒钟汽化约 $10^5$ 立方米的水(从常温常压到水蒸气).

\subsection{一般情况}
若使用经典力学, 用于推进的介质喷出速度越大, 同样的推力功率就越大.

功率
\begin{equation}
P = \frac{v^2}{2}\dv{m}{t}
\end{equation}
或者
\begin{equation}\label{PhRoc_eq1}
\dv{m}{t} = \frac{2P}{v^2}
\end{equation}
推力
\begin{equation}\label{PhRoc_eq2}
F = v\dv{m}{t} = \frac{2P}{v}
\end{equation}
所以在设计推进器时, \autoref{PhRoc_eq1} \autoref{PhRoc_eq2} 会互相制约, 在同样的功率下, 要想推力翻倍, 工质需要消耗原来的 4 倍.

\subsection{火箭加速的完整计算}
在没有引力的环境. 火箭以以相对速度 $u$ 和工质消耗 $\alpha = \dv*{m}{t}$  喷出燃料.
\begin{equation}
F = u\alpha
\end{equation}
火箭质量
\begin{equation}
m(t) = M - \alpha t
\end{equation}
加速度
\begin{equation}
a(t) = \frac{F}{m(t)} = \frac{u}{M/\alpha - t}
\end{equation}
速度
\begin{equation}
v(t) = \int_0^t a(t') \dd{t'} = -u \ln(1 - \alpha t / M)
\end{equation}
若火箭质量减去燃料为 $M_0$, 那么时间最大取 $t = (M - M_0)/\alpha$, 最大速度为
\begin{equation}
v_f = u \ln(M / M_0)
\end{equation}
或者
\begin{equation}
M = M_0\E^{v_f/u}
\end{equation}
可见燃料随莫速度呈指数增长. 注意这与 $\alpha$ 无关.
