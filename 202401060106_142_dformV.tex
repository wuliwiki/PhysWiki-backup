% 微分形式和体积形式(线性代数)
% license Usr
% type Wiki

\pentry{对称/反对称多线性映射\upref{SASmap},对偶空间\upref{DualSp}}

\subsection{微分形式}

\begin{theorem}{}
全体反对称 $k$-形式构成的向量空间与 ${\large \wedge}^k V^*$ 自然同构(不依赖基的选取)。
\end{theorem}
\textbf{证明:}全体反对称 $k$-形式构成的向量空间自然同构于 $({\large \wedge}^k V)^*$,它自然同构于 ${\large \wedge}^k V^*$。

我们把向量空间 $V$ 上的反对称 $k$-形式称为\textbf{微分形式},因为我们可以定义微分算子(外微分/外导数)。

\begin{definition}{微分算子}
向量空间上的一个微分算子是一个线性映射 $\dd{} : {\large \wedge}^k V^* \to {\large \wedge}^{k + 1} V^*$\footnote{为了方便,当 $k < 0$ 时我们记 ${\large \wedge}^k V^* = 0$},满足
\begin{itemize}
\item 对于 $c \in \mathbb{F} = {\large \wedge}^0 V^*$,$\dd c = 0$,
\item $\dd(\alpha \wedge \beta) = \dd\alpha \wedge \beta + (-1)^p \alpha \wedge \dd\beta$,其中$\alpha \in {\large \wedge}^p V^*$。
\end{itemize}


\end{definition}
