% 渐近线
% keys 渐近线|垂直渐近线|斜渐近线|水平渐近线|
% license Usr
% type Tutor

\begin{issues}
\issueDraft
\end{issues}

无论是在学习反比例函数还是指数函数时,都会被告知到它们存在\textbf{渐近线(Asymptotes)}。对着这些函数图像时,朴素的直觉会告诉我们,当自变量趋近于某些极端值时,函数的图像会无限接近某条直线,但不与它相交。换句话说,渐近线就像函数在某个方向上的“边界线”或“吸引线”,随着自变量的变化,函数与这条直线之间的距离在无穷远处趋于零。

尽管存在上面的朴素认知,但往往没有深入探讨渐近线的具体含义和性质。本文将专门讨论渐近线的概念。研究渐近线的定义和性质,能够对函数在极端情况下的表现产生深入的理解,尤其是在解析函数的极限、收敛,研究函数的长期行为或分析函数的增长与衰减行为时,渐近线能够提供重要的几何直观。

\subsection{函数意义下的渐近线}

通常,研究渐近线是从函数的背景下开始的,本文也将沿用这一思路。渐近线是描述函数图像在特定条件下的“极限行为”的一种重要工具。可以简单地理解为,当自变量 $x$ 或函数值 $y$ 趋向无穷大或无穷小时,函数的图像会逐渐靠近某条固定的直线。正如在反比例函数和指数函数中已经看到的那样,渐近线揭示了函数在远离中心区域时的变化趋势。

渐近线可以分为多种类型,最常见的包括水平渐近线、垂直渐近线和斜渐近线,它们分别对应不同的极端情况,反映函数在远端的不同表现。通过对渐近线的研究,我们能够更加深入地理解函数的性质,从而更有效地分析其在现实问题中的应用。

\subsubsection{垂直渐近线}

\begin{definition}{垂直渐近线}
若函数$f(x)$满足$\displaystyle \lim_{x\to x_0}f(x)=\infty$,则称直线$x=x_0$是$f(x)$的\textbf{垂直渐近线(Vertical Asymptote,也称铅直渐近线)}。
\end{definition}

垂直渐近线代表了函数在某些点上发生了不连续性,比如在这些点上,函数值趋于无穷大。所以垂直渐近线所在的位置一定是函数的间断点。一般此时,函数分母趋于零或是分子的高阶无穷小。

\subsubsection{水平渐近线}

\begin{definition}{水平渐近线}
若函数$f(x)$满足$\displaystyle \lim_{x\to +\infty}f(x)=y_0$或$\displaystyle \lim_{x\to -\infty}f(x)=y_0$,则称直线$y=y_0$是$f(x)$的\textbf{水平渐近线(Horizontal Asymptote)}。
\end{definition}

偶尔可能会见到类似$\displaystyle \lim_{x\to \infty}f(x)=y_0$的表述,但这种写法暗含了两侧极限存在且相等的意味,也即要求渐近线只能有一条。而对于如$\displaystyle f(x)={1+|x|\over x}$等函数,事实上存在两条水平渐近线。此处选择了稍嫌麻烦但更严谨的定义。

\subsubsection{斜渐近线}

\begin{definition}{斜渐近线}
若函数$f(x)$满足$\displaystyle \lim_{x\to +\infty}f(x)-kx-b=0$或$\displaystyle \lim_{x\to -\infty}f(x)-kx-b=0$,则称直线$y=kx+b$是$f(x)$的\textbf{斜渐近线(Oblique Asymptote)}。
\end{definition}

这里的情况与水平渐近线相同。其实可以看出,水平渐近线是$k=0$时的特例。

关于“渐近线与函数不相交”这件事,其实是受到了常见情况的误导。

\subsubsection{求函数的渐近线}

在画函数图形时,为了更准确,一般需要体现出函数的渐近线。

由于函数的性质限定,在同一个方向上,斜渐近线和水平渐近线不共存,却可以和垂直渐近线共存(如:$\displaystyle f(x)=x+{1\over x}$ )。同样,水平渐近线也可以和垂直渐近线共存(如:$\displaystyle f(x)={1\over x}$ )。而垂直渐近线只存在于间断点的特性也使得它与另两类判断不太相同。因此,一般寻找渐近线的步骤是:先找垂直渐近线,再找水平渐近线,最后找斜渐近线。

垂直渐近线:垂直渐近线只可能在函数不连续的点处出现。这是为什么?因为从连续函数的性质知道,闭区间的连续函数有界,所以如果是连续的话,它的每一点的极限都是有限的(我们可以选一个很小的包含这点的连续区间)。
找到不连续的点后,再在这点求极限。如果左右极限有一个趋于无穷大,那么这点处就有垂直渐近线。
水平渐近线:确定垂直渐近线后,就开始寻找水平渐近线。分别令$x$趋近于正、负无穷大,如果极限存在(不包括无穷大,无穷大是极限不存在的一种),那么就有水平渐近线;
斜渐近线:如果一个方向有水平渐近线,就不会有斜渐近线。也就是说,一个方向有水平渐近线,就不用找斜渐近线了(为什么?)。 如果没有水平渐近线,就来确定有没有斜渐近线。



\subsection{解析几何背景下的渐近线}

在解析几何的背景下,渐近线的概念可以被统一为一条曲线无限接近但不相交于另一曲线的直线。无论是水平渐近线、垂直渐近线还是斜渐近线,渐近线都代表了一个曲线在趋于无穷远或某特殊点时的行为。这种统一的定义可以不局限于函数,而是更广泛地适用于曲线的情况。

采用解析几何的视角,可以尝试着将渐近线定义为:

\begin{definition}{渐近线}
设以极坐标表示的平面曲线$C:r=r(\theta)$。如果存在一个方向  $\theta_0$  和一条直线  $L$ ,使得当  $\theta$  趋于  $\theta_0$  时,满足径向距离趋于无穷大且曲线到直线的距离趋于零,即;
\begin{equation}
\begin{cases}
\displaystyle\lim_{\theta \to \theta_0} r(\theta) = \infty\\
\displaystyle\lim_{\theta \to \theta_0} d(C, L) = 0~.
\end{cases}
\end{equation}
那么,称直线  $L$  是曲线  $C$  在方向  $\theta_0$  上的渐近线。
\end{definition}

前面提到的水平渐近线和垂直渐近线分别是$\theta_0=k\pi,(k\in\mathbb{Z})$以及$\displaystyle\theta_0=(2k+1){\pi\over2},(k\in\mathbb{Z})$时的特例。

$\displaystyle y={\sin x\over x}$,$y=0$
$\displaystyle y=\sin {1\over x}$,$y=0$

由上面的定义可以得知,任何闭合曲线都不可能有渐近线。