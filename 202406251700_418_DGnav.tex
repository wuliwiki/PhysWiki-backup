% 【导航】微分几何
% license Xiao
% type Map

\begin{issues}
\issueAi
\end{issues}

\subsection{微分几何学}

微分几何学是数学的一个迷人分支,它运用微积分和数学分析方法研究光滑曲线、曲面和高维流形,即在局部上类似于欧几里得空间的几何对象,的几何性质。它研究流形的曲率 (curvature)、测地线 (geodesics) 和度量 (metrics) 等概念。微分几何学在物理学、工程学、计算机图形学和广义相对论中都有应用,对于理解时空的曲率以及质量巨大的天体周围粒子和光的行为具有重要意义。

\subsubsection{微分几何学的主要组成部分}

以下列表将探讨微分几何学的主要子学科,每个子学科都有助于深入理解不同背景下的几何结构:

\begin{enumerate}
\item 曲线与曲面(经典微分几何学):这是子流形理论中的一个特殊主题,专注于三维空间中的曲线和曲面的几何学。它涉及到研究曲率 (curvature)、挠率 (torsion)、测地线 (geodesics) 和极小曲面 (minimal surfaces),代表了微分几何学的经典基础。

\item 黎曼几何学 (Riemannian Geometry):黎曼几何学研究配备有黎曼度量张量的流形,这允许测量长度和角度。它在理解曲面和高维流形的内在几何学中发挥着重要作用。

\item 辛几何学 (Symplectic Geometry) 和切触几何学 (Contact Geometry):辛几何学研究配备有闭合非退化2-形式的偶数维流形,而切触几何学则是其奇数维的对应,涉及配备有联系形式的联系流形。这两个领域在经典力学、哈密顿动力学和保守系统 (conservative systems) 的研究中至关重要。

\item 微分拓扑学 (Differential Topology):微分拓扑学探索流形的拓扑性质,关注光滑映射、嵌入和不可微结构的分类。

\item 同伦,同调与上同调:研究流形上的“洞”。

\item 几何分析学 (Geometric Analysis):几何分析学结合微分几何、偏微分方程和泛函分析的方法,研究流形的几何性质。它涉及到曲率 (curvature)、本征值 (eigenvalues)、热方程 (heat equations) 等问题。

\item 李群与李代数 (Lie Groups and Lie Algebras):这个子学科涉及李群的研究,李群是配备有群结构的可微流形,并涉及李代数的研究,李代数是配备有李括号的向量空间。它在理解流形上的对称性和变换方面扮演着重要角色。

\item 复几何学 (Complex Geometry):复几何学涉及复流形和全纯映射的研究。它探索复曲线和曲面的性质,以及它们与复代数簇 (Complex Varieties) 的关系。对于(近)复流形,其上有(近)复结构。此外还经常研究凯勒流形等等。

\item 芬斯勒几何学 (Finsler Geometry):芬斯勒几何学是微分几何学的一个分支,它研究芬斯勒流形,是黎曼流形的推广,这些流形是配备了更加灵活的度量张量的光滑流形。芬斯勒几何学在建模非欧几里得空间和优化路径等方面有应用。它是黎曼几何学的有益补充,并在理论物理、最优输运和数学生物学等领域具有重要意义。
\end{enumerate}


\subsubsection{其他主题}

\begin{enumerate}
\item 子流形理论 (Submanifold Theory):子流形理论探索嵌入在高维流形中的低维流形(子流形)。它研究这些子流形的内在性质以及与周围空间的关系。曲线与曲面(经典微分几何学)是子流形理论的一个特例,专注于三维空间中的曲线和曲面的几何学。

\item 全局微分几何学 (Global Differential Geometry):全局微分几何学研究与局部坐标无关的流形的性质和不变量。它探索高斯-波内定理和泊松-霍普定理等全局现象,这些对于理解流形的全局拓扑和曲率非常重要。
\end{enumerate}


此外,还有其他一些高级主题,这个主题涵盖微分几何学的各种高级领域,包括\footnote{从Utrecht Geometry Centre复制而来}:

\begin{itemize}
\item 叶分离理论 (Foliation Theory)
\item 广义几何学 (Generalised Geometry)
\item 偏微分方程的几何学 (Geometry of PDEs)
\item h-原理 (h-Principle)
\item 指数理论 (Index Theory)
% \item 李理论,包括李代数和李群 (Lie Theory, including Lie Algebroids and Groupoids)
% int256: Lie 上面已经提到过了
\item 非线性椭圆偏微分方程及其在微分拓扑学中的应用 (Non-linear Elliptic PDEs and Their Applications in Differential Topology)
\item 泊松几何学 (Poisson Geometry)
\item 亚黎曼几何学 (Subriemannian Geometry)
\item 卡拉比-丘流形 (Calabi-Yau Manifold),是复几何中第一示性类为 $0$ 的紧凯勒流形
\item 纤维丛理论
\item 以及更多
\end{itemize}

以上子学科代表了微分几何学的专门领域,为我们深入理解构成数学和物理世界的复杂结构提供了宝贵的洞察。

 