% 字符编码

这里要讨论的一个基本的问题是, 计算机如何保存字符? 例如我们打开一个 txt 文件, 在里面写一篇文章, 保存的时候这些数据以什么形式储存在硬盘上? 答案是:某个整数. 例如在著名的 ascii 编码中, 每个字符被储存为一个 0 到 127 的整数, 这些包括大写和小写字母, 数字, 常见标点, 以及一些格式上的符号如空格, 换行符, 制表符等. ascii 编码支持通常的英语写作, 但不支持其他语言如中文.

这里我们使用 VScode 编辑器为例进行介绍. 

当我们在在中版的 Windows 操作系统中新建一个 txt 并输入一些中文的时候, 默认编码是 GB2312,

\subsection{CR 和 LF}
Windows 默认使用 CRLF, MacOS 默认用 CR, Unix 系统一般用 LF. 高级一些的编辑器会自动检测

\subsection{Unicode}
Unicode 支持许多语言.
