% 量子力学导航

\begin{issues}
\issueDraft
\end{issues}

\subsection{量子力学的数学基础:线性代数}
严格来说, 量子力学的数学基础是泛函分析而不是线性代数. 因为线性代数只研究有限维空间中的问题, 而泛函分析将研究对象拓展到了无限维空间. 量子力学中的波函数一般是无限维希尔伯特空间中的向量. 但是, 对于绝大部分物理系学生来说, 泛函分析都不是必修内容, 在物理系的量子力学的教学中, 绝大部分也只是采用 “有限维空间类比到无穷维空间” 的模糊说法, 并通过一些实例来讲解离散本征值和连续本征值的区别. 考虑到这仍然是无法轻易改变的现状, 我们在初学量子力学时同样不涉及泛函分析.

在笔者看来脱离了线性代数和矢量空间来讲解量子力学, 就好比脱离了微积分讲解牛顿力学, 是不切实际的. 而泛函分析相对于线性代数的关系, 就好比数学分析之于微积分. 虽然泛函分析能让数学严格化, 但对物理意义的启发并不太大.

要用线性代数理解量子力学, 矢量空间\upref{LSpace}的概念尤为重要. 事实上国外的线性代数课程大部分都是首先讲解线性空间而不是例如行列式的计算方法. 在量子力学中常用狄拉克符号\upref{braket}表示矢量. 我们需要明白什么是子空间\upref{SubSpc}, 为什么矩阵可以表示有限维矢量空间之间的线性映射\upref{LinMap}, 也就是\textbf{算符}. 厄米矩阵的本征问题\upref{HerEig} 是非常重要的概念, 因为量子力学中的测量量都有各自的厄米算符. 矩阵对易与共同本征矢\upref{OpComu} 也非常重要, 因为两个算符是否对易决定了它们是是否可以同时测量, 也就是是否存在不确定性原理.

需要明白波函数为什么可以看作矢量空间(希尔伯特空间)中的矢量. 需要明白什么是把矢量投影到子空间\upref{projOp}. 因为量子力学中的测量就是一个投影操作.

要学习量子力学中单个粒子的自旋问题, 首先需要明白张量积空间\upref{DirPro}, 空间波函数和自旋态的相乘就是一个张量积, 粒子的总态矢存在于张量积空间中.

多粒子的量子力学更加需要使用张量积空间\upref{DirPro}, 多粒子的波函数就是处于这样一个矢量空间中. 全同粒子\upref{IdPar}假设使波函数只能存在于张量积空间中的对称子空间或者反对称子空间.

\subsection{量子力学部分的讲解思路}

在量子力学建立以前, 玻尔原子模型\upref{BohrMd}. 这并不是真正的量子力学, 而是量子力学早期的一个半经典模型. 甚至可以说它算出的氢原子能级能符合实验是一个巧合. 它除了能给出正确的能级外, 其他方面几乎没有正确之处.

要学习真正的量子力学, 首先我们通过一篇量子力学科普\upref{QM0} 简单概括量子力学的基本原理, 然后为了避免直接解薛定谔方程, 我们先看 “量子力学与矩阵\upref{QMmat}”. 然后通过 “算符和本征问题\upref{QM1}” 进一步明确量子力学的基本假设. 但这并不是所有的基本假设, 其中并没有讲解如何处理连续本征值和散射态, 没有涉及自旋和全同粒子假设.

接下来学习定态薛定谔方程\upref{SchEq} 以及含时薛定谔方程\upref{TDSE}. 那么由薛定谔方程就可以解出任意波函数在已知势能中的演化.

在知道波函数如何根据薛定谔方程演化后, 我们开始研究测量量: 要知道如何定义一个物理量的平均值\upref{QMavg} 以及什么是守恒量\upref{QMcons}.

在讲解过程中会伴随一些基础的势能例子, 一维势能如无限深势阱\upref{ISW}, 有限深方势阱\upref{FSW}, 方势垒的定态波函数\upref{SqrPot}, 简谐振子\upref{SHO}, 一维自由粒子(量子)\upref{FreeP1}以及高斯波包\upref{GausPk}. 多维势能如无限深圆形势阱\upref{CirISW}, 三维量子简谐振子(球坐标系)\upref{SHOSph}.

接下来学习最简单的原子模型:类氢原子的定态波函数\upref{HWF}. 氢原子是唯一具有解析解的原子, 其重要性可见一斑. 氢原子一般需要在球坐标中求解, 这就涉及球坐标中的用分离变量法分离变量法解偏微分方程\upref{SepVar}, 该方法典型的产物就是球谐函数\upref{SphHar}.

在经典的量子力学中, 电磁场通常是不进行量子化的, 他们仍然是电磁学中的连续场. 例如一个恒电场在薛定谔方程中体现为在势能项中加上 $-q\bvec E \vdot \bvec r$. 所以在经典的量子力学的方程中不会明确出现 “光子” 的概念. 光子作为一种没有质量的玻色子, 不能像电子一样直接用普通的薛定谔方程来描述, 而是需要\textbf{量子电动力学}.
