% 泡利不相容原理
% 泡利不相容|全同粒子|玻色子|费米子|能量本征态

\pentry{全同粒子\upref{IdPar}}
\footnote{参考 Wikipedia \href{https://en.wikipedia.org/wiki/Pauli_exclusion_principle}{相关词条} 和 \cite{GriffQ}.}在量子力学中, \textbf{泡利不相容原理(Pauli exclusion principle)}指的是, 在含有若干个全同费米子(即自旋为 1/2 的粒子)的系统中, 任意两个费米子不能处于同一个态.

事实上费米子对态矢的反对称性要求已经包含了泡利不相容原理, 所以该原理只是费米子性质的一个推论. 若假设系统只有两个全同费米子, 用矢量空间的语言来解释泡利不相容原理就是: 双粒子态表示为张量积 $\ket{\psi}\ket{\psi}$ 时是一个交换对称的态矢, 而费米子的态矢必须是反对称的, 所以两个费米子不可能处于 $\ket{\psi}\ket{\psi}$ 形式的态.

泡利不相容原理常用于解释原子中的电子排布.

\begin{example}{原子壳层}
在原子壳层理论中, 如果忽略电子之间的相互作用, 那么单个电子具有一系列正交归一的能量本征态, 其中能量最低的态叫做基态, 具有唯一的波函数 $\psi_g(\bvec r)$. 如果两个电子都处于基态, 就意味着双电子波函数为
\begin{equation}
\psi(\bvec r_1, \bvec r_2) = \psi_g(\bvec r_1)\psi_g(\bvec r_2)
\end{equation}
这个波函数显然是交换对称的.

但电子是费米子, 即总态矢必须是反对称的, 所以总自旋态必须是反对称的(见\autoref{IdPar_eq4}~\upref{IdPar}的说明), 所以如果双电子都处于基态, 那么唯一可能的状态就是
\begin{equation}
\Psi(\bvec r_1, \bvec r_2) \chi_{1,2} = \frac{1}{\sqrt{2}}\psi_g(\bvec r_1)\psi_g(\bvec r_2)(\uparrow\downarrow - \downarrow\uparrow)
\end{equation}

根据泡利不相容原理, $\psi_g(\bvec r_1)\psi_g(\bvec r_2)\uparrow\uparrow$ 和 $\psi_g(\bvec r_1)\psi_g(\bvec r_2)\downarrow\downarrow$ 态不可能存在是因为它们表示 “两个电子处于同一个状态”.
\end{example}
