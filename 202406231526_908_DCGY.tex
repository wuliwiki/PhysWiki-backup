% 电磁感应
% license CCBYSA3
% type Wiki

(本文根据 CC-BY-SA 协议转载自原搜狗科学百科对英文维基百科的翻译)
\begin{figure}[ht]
\centering
\includegraphics[width=8cm]{./figures/c17b904609835bc2.png}
\caption{法拉第的实验显示了线圈之间的感应:液体电池(右)提供流过小线圈的电流(一)创造一个磁场。当线圈静止时,不会感应出电流。但是当小线圈移入或移出大线圈时(二),通过大线圈的磁通量变化,感应出由检流计检测到的电流(G)。[1]} \label{fig_DCGY_1}
\end{figure}

\textbf{电磁感应}是在变化的磁场中跨越电导体产生的电动势(即电压)。

麦可·法拉第通常被认为是1831年感应的发现,和詹姆斯·克拉克·麦克斯韦数学上将其描述为法拉第感应定律。楞次定律描述感应场的方向。法拉第定律后来被推广为麦克斯韦-法拉第方程,这四个方程之一麦克斯韦方程在他的理论中电磁。

电磁感应有许多应用,包括电气元件如电感器和变压器,以及设备如电动机和发电机。
\subsection{基本概念}
1831年,一位叫迈克尔·法拉第的科学家发现了磁与电之间的相互联系和转化关系。只要穿过闭合电路的磁通量发生变化,闭合电路中就会产生感应电流。这种利用磁场产生电流的现象称为电磁感应(Electromagnetic induction),产生的电流叫做感应电流。

电磁感应现象的产生条件有两点(缺一不可)。
\begin{itemize}
\item 闭合电路。

\item 穿过闭合电路的磁通量发生变化。
\end{itemize}

让磁通量发生变化的方法有两种。一种方法是让闭合电路中的导体在磁场中做切割磁感线的运动;另一种方法是让磁场在导体内运动。
\subsubsection{1.1 磁通量}
设在匀强磁场中有一个与磁场方向垂直的平面,磁场的磁感应强度为 $ B $,平面的面积为 $S $。

(1) 定义:在匀强磁场中,磁感应强度 $ B $) 与垂直磁场方向的面积 $ S $ 的乘积,叫做穿过这个面的磁通量,简称磁通。

(2) 定义式:
$\varphi = B \cdot S$

当平面与磁场方向不垂直时
$\varphi = B \cdot S = BS \cos \theta$

(( $\theta$) 为平面的垂线与磁场方向的夹角)

(3) 物理意义

垂直穿过某个面的磁感线条数表示穿过这个面的磁通量。

(4) 单位:在国际单位制中,磁通量的单位是韦伯,简称韦,符号是 Wb。

1 Wb = 1 T·$m{^2}$ = 1 V·s。

(5) 标量性:磁通量是标量,但是有正负之分。
