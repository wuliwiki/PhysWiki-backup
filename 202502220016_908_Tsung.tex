% 李政道(综述)
% license CCBYSA3
% type Wiki

本文根据 CC-BY-SA 协议转载翻译自维基百科\href{https://en.wikipedia.org/wiki/Tsung-Dao_Lee}{相关文章}。

\begin{figure}[ht]
\centering
\includegraphics[width=6cm]{./figures/cc118c4172086978.png}
\caption{李氏,1956年} \label{fig_Tsung_1}
\end{figure}
李政道(Tsung-Dao Lee,中文:李政道;拼音:Lǐ Zhèngdào;1926年11月24日-2024年8月4日)是一位中美物理学家,以其在宇称不守恒、李–杨定理、粒子物理学、相对论重离子碰撞(RHIC)物理学、非拓扑孤立子和孤立子星方面的工作而闻名。他是纽约哥伦比亚大学的名誉教授,曾在该校教授直到2012年退休。

1957年,李政道在30岁时与杨振宁共同获得诺贝尔物理学奖,因他们在弱相互作用中宇称不守恒的研究工作,该工作由吴健雄在1956至1957年间通过著名的吴实验实验性地证明。

李政道仍然是第二次世界大战后最年轻的诺贝尔科学奖得主。他是历史上第三年轻的诺贝尔科学奖得主,仅次于威廉·L·布拉格(William L. Bragg,1915年与父亲威廉·H·布拉格一起获奖,年仅25岁)和维尔纳·海森堡(Werner Heisenberg,1932年获奖时同样30岁)。李政道和杨振宁是第一批获奖的中国人。由于他于1962年成为美国公民,李政道也是迄今为止最年轻的美国诺贝尔奖得主。
\subsection{传记}  
\subsubsection{家庭}  
李政道出生于中国上海,祖籍邻近的苏州。他的父亲李駿康(Lǐ Jùn-kāng),是南京大学的首批毕业生之一,曾是一位化学工业家和商人,参与了中国早期现代合成肥料的开发。李政道的祖父李仲覃(Lǐ Zhòng-tán)是苏州圣约翰堂(蘇州聖約翰堂)首位华人卫理公会高级牧师。[4][5]

李政道有四个兄弟和一个妹妹。教育家Robert C. T. Lee是李政道的兄弟之一。[6] 李政道的母亲张和他的兄弟Robert C. T. 在1950年代移居台湾。
\subsubsection{早年生活}  
李政道在上海接受中学教育(东吴大学附属中学)和江西(江西联合中学)。由于第二次中日战争,李政道的高中教育中断,因此他未能获得中学毕业文凭。然而,在1943年,李政道直接申请并被录取到国立浙江大学(当时称国立浙江大学)。最初,李政道注册为化学工程系的学生。很快,李政道的才华被发现,他对物理学的兴趣迅速增长。包括舒星北和王淦昌在内的几位物理学教授给予了李政道很大的指导,他很快转入了国立浙江大学物理系,并在那里学习了1943年至1944年。[5][需要更多引用]

然而,随着日本进一步的侵略,李政道于1945年继续在昆明的国立西南联合大学学习,在那里他跟随吴大猷教授学习物理。[5]
