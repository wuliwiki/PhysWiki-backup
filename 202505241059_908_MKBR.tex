% 麦克斯韦-玻尔兹曼分布(综述)
% license CCBYSA3
% type Wiki

本文根据 CC-BY-SA 协议转载翻译自维基百科\href{https://en.wikipedia.org/wiki/Maxwell\%E2\%80\%93Boltzmann_distribution}{相关文章}。

在物理学中(特别是在统计力学中),麦克斯韦–玻尔兹曼分布(Maxwell–Boltzmann distribution,或称麦克斯韦分布)是一种特定的概率分布,以詹姆斯·克拉克·麦克斯韦和路德维希·玻尔兹曼的名字命名。

该分布最初被定义并用于描述理想气体中粒子的速度分布,在这种理想化模型中,粒子在静止容器内自由运动,彼此之间没有相互作用,除了极短暂的碰撞,在这些碰撞中粒子与其他粒子或热环境交换能量和动量。在此语境下,“粒子”仅指气体粒子(即原子或分子),并假设该粒子系统已达到热力学平衡状态。[1]这类粒子的能量遵循麦克斯韦–玻尔兹曼统计,其速度的统计分布可通过将粒子能量与动能等同来推导得出。

从数学上讲,麦克斯韦–玻尔兹曼分布是具有三个自由度的卡方分布(对应于欧几里得空间中速度矢量的三个分量),其尺度参数以与 $T/m$(温度与粒子质量之比)的平方根成正比的单位来度量速度。[2]

麦克斯韦–玻尔兹曼分布是气体动理论的结果,气体动理论为许多基本的气体性质(包括压强和扩散)提供了简化的解释。[3] 麦克斯韦–玻尔兹曼分布本质上适用于三维空间中粒子的速度分布,但结果仅依赖于速率(即速度的大小),而与方向无关。粒子速率的概率分布表示哪些速率更为常见:一个随机选取的粒子,其速率将以该分布为依据进行抽样,因此落在某一速率范围内的可能性会高于另一速率范围。气体动理论适用于经典理想气体,这是一种对真实气体的理想化模型。对于真实气体,存在多种效应(如范德华力、涡流流动、相对论速度极限以及量子交换相互作用),可能使其速率分布偏离麦克斯韦–玻尔兹曼形式。然而,在常温下的稀薄气体表现得非常接近理想气体,其速率分布可被麦克斯韦分布很好地近似描述。这一点对于理想等离子体也成立,即在足够低密度下的电离气体。[4]

该分布最早由麦克斯韦于1860年以启发式方法推导得到。[5][6] 随后在19世纪70年代,玻尔兹曼对该分布的物理起源进行了深入研究。该分布还可以通过最大化系统熵的原理推导出来。以下是一些推导方法的列表:
\begin{enumerate}
\item 在相空间中,通过最大熵原理得到的最大熵概率分布,其约束条件是系统的平均能量守恒:$\langle H \rangle = E$
\item 正则系综方法。
\end{enumerate}
