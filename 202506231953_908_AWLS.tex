% 埃瓦里斯特·伽罗瓦(综述)
% license CCBYSA3
% type Wiki

本文根据 CC-BY-SA 协议转载翻译自维基百科 \href{https://en.wikipedia.org/wiki/\%C3\%89variste_Galois}{相关文章}。

埃瓦里斯特·伽罗瓦(Évariste Galois,/ɡælˈwɑː/;法语发音:[evaʁist ɡalwa];1811年10月25日-1832年5月31日)是一位法国数学家和政治活动家。他在少年时期就成功找出了一个多项式是否可用根式求解的充要条件,从而解决了一个困扰数学界达350年的难题。他的工作奠定了伽罗瓦理论和群论的基础——这两个领域后来成为抽象代数的主要分支。

伽罗瓦是一位坚定的共和主义者,积极参与了围绕1830年法国革命的政治动荡。由于他的政治活动,他多次被捕,并服刑数月。出狱不久,他因某些至今仍不明的原因参加了一场决斗,并因伤重身亡。
\subsection{生平}
\subsubsection{早年生活}
\begin{figure}[ht]
\centering
\includegraphics[width=6cm]{./figures/709d53b4b96fb0d8.png}
\caption{伽罗瓦少年时期就读的路易大帝中学校内的“荣誉庭院”。} \label{fig_AWLS_1}
\end{figure}
伽罗瓦于1811年10月25日出生于尼古拉-加布里埃尔·伽罗瓦和阿德莱德-玛丽(Adélaïde-Marie,娘家姓德芒特,Demante)夫妇之家。\(^\text{[2][4]}\) 他的父亲是一位共和主义者,是布尔拉雷讷自由党派的领袖。1814年路易十八复辟后,他的父亲成为该村的市长。\(^\text{[2]}\)他的母亲是一位法学家的女儿,精通拉丁语和古典文学,并负责伽罗瓦前十二年的教育。

1823年10月,伽罗瓦进入路易大帝中学,他的老师路易·保罗·埃米尔·理查德识别出了他的非凡才华。\(^\text{[5]}\)14岁时,他开始对数学产生浓厚兴趣。\(^\text{[5]}\)

伽罗瓦找到了一本阿德里安-玛丽·勒让德的《几何原本》,据说他“像读小说一样”阅读这本书,并在第一次阅读时就掌握了其内容。15岁时,他已经在阅读约瑟夫-路易·拉格朗日的原始论文,如《代数方程解法的思考》,这很可能激发了他后来在方程理论方面的研究,\(^\text{[6]}\)以及《函数运算讲义》——这是专为专业数学家撰写的著作。然而,他在课堂上的表现并不出色,老师还指责他摆出一副天才的架势。\(^\text{[4]}\)
\subsubsection{初露锋芒的数学家}
1828年,伽罗瓦参加了法国当时最负盛名的数学学府——综合理工学院的入学考试,但由于缺乏数学方面的常规准备,在口试时未能充分说明自己的解题过程,因此落榜。同年,他进入了当时数学研究水平较低的正规师范学院(École Normale,当时称为“预备学院”l’École préparatoire),在那里他遇到了一些对他持同情态度的教授。[7]
\begin{figure}[ht]
\centering
\includegraphics[width=6cm]{./figures/ef1f974ae15f836d.png}
\caption{奥古斯丁-路易·柯西审阅了伽罗瓦早期的数学论文。} \label{fig_AWLS_2}
\end{figure}
次年,伽罗瓦的第一篇论文《关于简单连分数》发表了。\(^\text{[8]}\)大约在同一时期,他开始在多项式方程理论方面取得根本性发现。他将两篇关于这一主题的论文提交给了科学院。著名数学家奥古斯丁-路易·柯西对这些论文进行了评审,但出于至今仍不明确的原因,拒绝将其发表。

尽管有许多相反的说法,广泛观点认为柯西确实认识到了伽罗瓦工作的重大意义,他只是建议将这两篇论文合并为一篇,以便参加科学院设立的数学大奖评选。柯西当时是享有盛誉的数学家,尽管在政治立场上与伽罗瓦截然相反,但他认为伽罗瓦的成果极有可能获奖。\(^\text{[9]}\)

1829年7月28日,伽罗瓦的父亲在与当地神父的一场激烈政治争执后自杀身亡。\(^\text{[10]}\)几天后,伽罗瓦第二次也是最后一次尝试报考巴黎综合理工学院(École Polytechnique),但再次落榜。\(^\text{[10]}\)毋庸置疑,伽罗瓦的能力远远超过入学标准;关于他落榜的原因,众说纷纭。较为可信的说法是,伽罗瓦在答题时逻辑跳跃过大,令水平不足的主考官无法理解,从而激怒了伽罗瓦。此外,父亲的突然去世可能也影响了他的情绪与表现。\(^\text{[4]}\)

在被综合理工学院拒之门外后,伽罗瓦改为参加学士考试,以进入巴黎师范学院(当时称为École Normale)。\(^\text{[10]}\)他顺利通过考试,并于1829年12月29日取得学位。\(^\text{[10]}\)他的数学考官评价道:“这名学生在表达自己的思想时有时显得晦涩,但他很聪明,展现出卓越的研究精神。”

伽罗瓦多次提交关于方程理论的论文稿,但在他生前从未获得发表。尽管他的第一次尝试被柯西拒绝了,但在1830年2月,他按照柯西的建议将论文递交给科学院秘书约瑟夫·傅里叶,\(^\text{[10]}\) 希望角逐科学院的大奖。不幸的是,傅里叶不久后去世,\(^\text{[10]}\)而那篇论文稿也随之遗失。\(^\text{[10]}\)当年的大奖最终追授给了尼尔斯·亨利克·阿贝尔,并与卡尔·雅可比共享。

尽管主要论文丢失,伽罗瓦当年还是发表了三篇论文。其中一篇奠定了伽罗瓦理论的基础;\(^\text{[11]}\)第二篇探讨了方程的数值解法(即现代所谓的“求根”问题);\(^\text{[12]}\)第三篇则是数论方面的重要成果,首次明确提出了有限域的概念。\(^\text{[13]}\)
\subsubsection{政治激进分子}
\begin{figure}[ht]
\centering
\includegraphics[width=6cm]{./figures/df88489679bda119.png}
\caption{让-维克多·施内茨的《市政厅之战》。加洛瓦作为坚定的共和主义者,本希望参加1830年七月革命,但被师范学校的校长阻止了。} \label{fig_AWLS_3}
\end{figure}
加洛瓦生活在法国政治动荡的时期。1824年,查理十世继路易十八登上王位,但在1827年,他的政党在选举中遭遇重大挫败,到1830年,自由党成为多数党。面对议会的政治反对,查理发动政变,发布臭名昭著的《七月法令》,引发了七月革命\(^\text{[10]}\),最终以路易-菲利普即位为王告终。

当时,理工学院的学生正在街头书写历史,而加洛瓦却被锁在了师范学校里,学校校长不让他外出。对此愤怒不已的加洛瓦写了一封措辞激烈的信,严厉批评校长,并将信件投稿给《学校公报》,并署上了全名。虽然编辑在刊登时删去了署名,但加洛瓦还是因此被学校开除。\(^\text{[14]}\)

尽管他的开除正式生效日期是1831年1月4日,加洛瓦还是立即退学,并加入了国民自卫队中坚定的共和派炮兵部队。他在数学研究和政治活动之间分配时间。然而,这支部队引起的争议很快导致它的解散——1830年12月31日,也就是加洛瓦刚刚加入不久,当局出于担心他们会动摇政府的稳定,解散了国民自卫队的炮兵部队。大约在同一时期,加洛瓦所在的原部队的19名军官被逮捕,并被指控密谋推翻政府。

1831年4月,这些军官被全部无罪释放。1831年5月9日,一场为他们举行的庆功宴上,许多名人出席,其中包括大仲马。宴会过程逐渐变得喧闹不堪。加洛瓦在某个时刻站起身来举杯祝酒,说道:“敬路易-菲利普”,同时高举一把匕首在杯子上方。在场的共和派宾客将此举解读为对国王生命的威胁,并热烈欢呼。第二天他就在母亲家中被捕,被关押在圣佩拉吉监狱,直到1831年6月15日接受审判。\(^\text{[9]}\)

加洛瓦的辩护律师巧妙地辩称,加洛瓦实际说的是:“敬路易-菲利普,如果他背叛人民”,但最后一句话被欢呼声淹没了。检察官随后提出了一些问题,也许是考虑到加洛瓦的年轻,陪审团当日就宣布他无罪。\(^\text{[9][10][14][15]}\)

在接下来的巴士底日(1831年7月14日),加洛瓦站在抗议队伍的最前方,身穿已被解散的炮兵部队的制服,全副武装,携带了几把手枪、一支上膛的步枪和一把匕首。他再次被逮捕。\(^\text{[10]}\)

在狱中,加洛瓦第一次在狱友的怂恿下饮酒。其中一位狱友弗朗索瓦-文森特·拉斯帕伊在1831年7月25日写信记录了加洛瓦酒醉时所说的话,信中摘录如下:\(^\text{[9]}\)

“我告诉你,我会因为一个低级风尘女子的缘故死于决斗。为什么?因为她会要我为她的名誉复仇,那是被另一个人玷污的。你知道我缺什么吗,我的朋友?我只能向你倾诉:我缺一个我能爱并只在精神上爱的人。我失去了父亲,从来没有人能代替他,你明白吗……?”

拉斯帕伊继续写道,加洛瓦在仍处于酒醉迷狂之中时曾试图自杀,如果不是狱友强行阻止,他很可能已经成功。\(^\text{[9]}\)几个月后,加洛瓦于1831年10月23日出庭受审,因非法穿戴制服被判入狱六个月。\(^\text{[10][16][17]}\)在狱中,他继续发展自己的数学思想。他于1832年4月29日获释。
\subsubsection{最后的日子}
\begin{figure}[ht]
\centering
\includegraphics[width=6cm]{./figures/9ead57925ef6e5ae.png}
\caption{西梅翁·德尼·泊松审阅了伽罗瓦关于方程理论的论文,并称其为“难以理解”。} \label{fig_AWLS_4}
\end{figure}
在被巴黎高等师范学校开除后,伽罗瓦重新投身数学研究,尽管他仍然花时间参与政治活动。1831年1月,他被正式开除后,尝试开设一个私人高等代数课程,最初引起了一些兴趣,但很快就冷却了,因为人们感觉他更重视政治活动。\(^\text{[4][9]}\)西缅·德尼·泊松曾要求他提交关于方程理论的工作,伽罗瓦于1831年1月17日提交了论文。大约在1831年7月4日,泊松认为伽罗瓦的论文“难以理解”,并表示:“[伽罗瓦的]论证既不够清晰,也不够完整,无法判断其严谨性”;不过,这份拒绝报告在最后给出了鼓励性的建议:“因此我们建议作者应该将其全部工作发表,以便得出明确的判断。”\(^\text{[18]}\)虽然这份报告是在伽罗瓦7月14日被捕前写的,但直到10月才送到他在狱中的手中。考虑到他的性格和当时的境况,伽罗瓦对这封拒稿信做出了激烈反应,决定不再通过科学院发表论文,而是转由他的朋友奥古斯特·谢瓦利耶私人出版。然而显然他并未完全忽视泊松的建议,因为他在狱中开始整理所有数学手稿,并在狱中继续打磨自己的想法,直到1832年4月29日获释。\(^\text{[14]}\)不久之后,不知为何他被说服去参加了一场决斗。\(^\text{[10]}\)

伽罗瓦的致命决斗发生在1832年5月30日。\(^\text{[19]}\)决斗背后的真实动机仍不清楚,外界对此有诸多猜测。已知的是,在他去世前五天,他写信给谢瓦利耶,信中明显提到了一个破裂的恋情。\(^\text{[9]}\)

对原始信件的档案调查表明,这位与他有感情纠葛的女性可能是斯蒂芬妮-费利西·波特兰·杜·莫特尔,她是伽罗瓦晚年寄宿处的一位医生的女儿。\(^\text{[20]}\) 一些由伽罗瓦亲自誊写、部分内容(如她的名字)被抹去或刻意省略的信件片段仍然存世。\(^\text{[21]}\)这些信件暗示波特兰·杜·莫特尔曾向伽罗瓦倾诉她的烦恼,这可能促使伽罗瓦为了她的名誉主动挑起了决斗。这一推测也得到了伽罗瓦在去世前一晚写给朋友们的其他信件的支持。伽罗瓦的表弟加布里埃尔·德曼特在被问及是否知道决斗的原因时提到,伽罗瓦“当时面对一位所谓的叔叔和一位所谓的未婚夫,这两人都激怒了他并挑起了决斗。”伽罗瓦本人曾愤怒地喊道:“我是一个可耻的妖艳女人及其两个受骗者的牺牲品。”\(^\text{[14]}\)

关于伽罗瓦决斗对手的身份,亚历山大·仲马曾提到是佩谢·德埃班维尔,\(^\text{[15]}\)此人实际上是那十九位炮兵军官之一,他们的无罪释放曾在一场宴会上被庆祝,而正是那场宴会导致了伽罗瓦首次被捕。\(^\text{[22]}\)然而,这一说法仅见于仲马一人之口,而且即便他是正确的,也不清楚德埃班维尔为何会卷入其中。有猜测认为他当时是波特兰·杜·莫特尔所谓的“未婚夫”(她最终嫁给了别人),但目前没有确凿证据支持这一推测。另一方面,决斗发生后数日的一些报纸剪报中对伽罗瓦对手的描述(以“L.D.”为缩写)似乎更符合伽罗瓦的一位共和党朋友的特征——最可能是欧内斯特·迪沙特莱,他曾因相同指控与伽罗瓦一同入狱。\(^\text{[23]}\)鉴于现有资料相互矛盾,伽罗瓦真正的决斗对手是谁,或许已经永远湮没在历史之中。

不论这场决斗的真正原因如何,伽罗瓦坚信自己将命不久矣,因此在前一晚彻夜未眠,写下了致共和党友人的信件,并写下了他著名的数学遗书——致奥古斯特·谢瓦利耶的一封信,其中概述了他的一系列数学构想,并附上了三份手稿。\(^\text{[24]}\)数学家赫尔曼·外尔曾评价这封遗书说:“就其所包含的思想的新颖性与深刻性而言,这封信或许是人类文学中最重要的一篇。” 然而,关于伽罗瓦在决斗前夜倾尽数学心血于纸上的传说可能被夸大了。\(^\text{[9]}\)在这些最终文稿中,他粗略勾勒了自己在分析方面的一些工作思路,并在提交科学院的手稿和其他论文上做了注释。
\begin{figure}[ht]
\centering
\includegraphics[width=8cm]{./figures/7d35c96dc5d9656e.png}
\caption{布尔拉雷讷公墓中的伽罗瓦纪念碑。埃瓦里斯特·伽罗瓦被葬于一座无名公墓中,具体位置已不可考。} \label{fig_AWLS_5}
\end{figure}
1832年5月30日清晨,伽罗瓦腹部中枪,\(^\text{[19]}\)被对手和自己的助手遗弃,最终被一位路过的农夫发现。他于次日上午十点在科尚医院去世,\(^\text{[19]}\)死因很可能是腹膜炎。他临终时拒绝接受神父的临终仪式。他的葬礼最终演变成一场骚乱。\(^\text{[19]}\)当时本有计划在葬礼期间发动起义,但由于共和派领袖们在同一时间得知让·马克西米利安·拉马克将军去世的消息,起义计划被推迟,最终直到6月5日才爆发起义。在伽罗瓦去世前,只有他的弟弟被告知相关情况。\(^\text{[25]}\)当时伽罗瓦年仅20岁。他对弟弟阿尔弗雷德留下的最后一句话是:

“Ne pleure pas, Alfred ! J'ai besoin de tout mon courage pour mourir à vingt ans !”
(“别哭,阿尔弗雷德!我需要拿出我所有的勇气,才能在二十岁时死去!”)

1832年6月2日,埃瓦里斯特·伽罗瓦被埋葬在蒙帕纳斯公墓的一座无名公墓中,\(^\text{[19][17]}\)其确切位置至今不明。在他家乡布尔拉雷讷的公墓里,他的亲属墓地旁立有一座纪念碑,以示悼念。\(^\text{[26]}\)

埃瓦里斯特·伽罗瓦于1832年去世。约瑟夫·李乌维尔在1842年开始研究伽罗瓦的未发表手稿,并在1843年承认了这些手稿的价值。1832年至1842年这十年间发生了什么,以及究竟是什么促使李乌维尔开始阅读伽罗瓦的手稿,目前尚不清楚。Jesper Lützen 在其关于李乌维尔的著作中第十四章《伽罗瓦理论》中对此问题进行了详细探讨,但并未得出确切结论。\(^\text{[27]}\)

有一种可能性是,数学家们(包括李乌维尔)并不想公开伽罗瓦的论文,因为他是一名共和派政治激进分子,在巴黎共和党人于1832年发起的反君主制六月起义前五天身亡,而这场起义最终失败。在伽罗瓦的讣告中,他的朋友奥古斯特·谢瓦利耶几乎指责综合理工学院的学者们“害死”了伽罗瓦,认为如果他们当初没有拒绝伽罗瓦的论文,他就会成为一名数学家,而不会投身于后来的政治运动,也就不会被卷入导致其死亡的事件。考虑到当时法国仍生活在“恐怖统治”和拿破仑时代的阴影之下,李乌维尔可能是等到六月起义失败后政治局势稍微平静,才开始关注伽罗瓦的手稿。\(^\text{[27]}\)

李乌维尔最终在1846年10月至11月的《纯粹与应用数学杂志》上发表了伽罗瓦的手稿。\(^\text{[28][29]}\)伽罗瓦最著名的贡献是他用一种全新的方式证明了五次及更高次数方程一般无法用根式求解——也就是说,不存在通解的“五次公式”。虽然尼尔斯·亨利克·阿贝尔早在1824年就已经证明了五次方程根式无解的不可能性,保罗·鲁菲尼在1799年也曾发表过一个后来被证明有瑕疵的解法,但伽罗瓦的方法引领了对后来被称为“伽罗瓦理论”的深入研究,该理论可以用来判断任意一个多项式方程是否可以用根式求解。
\subsection{对数学的贡献}
\begin{figure}[ht]
\centering
\includegraphics[width=6cm]{./figures/f2b89010f872a527.png}
\caption{} \label{fig_AWLS_6}
\end{figure}
在1832年5月29日,也就是加洛瓦去世前两天,他致友人奥古斯特·舍瓦利耶信件的结尾处写道:\(^\text{[24]}\)

Tu prieras publiquement Jacobi ou Gauss de donner leur avis, non sur la vérité, mais sur l'importance des théorèmes.

Après cela, il y aura, j'espère, des gens qui trouveront leur profit à déchiffrer tout ce gâchis.

(请你公开请求雅可比或高斯发表他们的看法——不是关于这些定理是否正确,而是它们是否重要。此后,我希望会有人从中获益,去解读这一团乱麻。)

在加洛瓦约六十页的论文遗稿中,蕴含着许多极其重要的思想,这些思想对几乎所有数学分支都产生了深远影响。\(^\text{[30][31]}\)他的工作常被拿来与尼尔斯·亨里克·阿贝尔(Niels Henrik Abel, 1802–1829)作比较,阿贝尔是他那一时期的另一位早逝的数学天才,两人的研究在内容上也有显著重叠。
\subsubsection{代数学}
虽然在加洛瓦之前,许多数学家已经考虑过现在被称为“群”的概念,但加洛瓦是第一个使用“群”(法语 groupe)一词,并赋予它接近今天技术定义含义的人。因此,他被视为代数学分支——群论的奠基人之一。他将一个群分解为左陪集和右陪集,并称“左陪集与右陪集相同”的分解为“恰当分解”,这一概念引出了今天所谓的正规子群的定义。\(^\text{[24]}\)他还引入了有限域(为纪念他也称为加洛瓦域)的概念,形式上几乎与今天我们所理解的一致。\(^\text{[13]}\)

在他致舍瓦利耶的最后一封信\(^\text{[24]}\)及附带的三份手稿中(第二份手稿中),加洛瓦对有限域上的线性群进行了基本研究:
\begin{itemize}
\item 他构造了素域上的一般线性群 GL(ν, p),并计算了它的阶(即元素个数),以研究一般次数为 $p^\nu$ 的代数方程的加洛瓦群。\(^\text{[32]}\)
\item 他构造了投影特殊线性群 PSL(2, p)。加洛瓦将这些群构造为分式线性变换,并观察到这些群在 $p = 2$ 或 $p = 3$ 的情况下例外地不是单群,其余情况都是单群。\(^\text{[33]}\)这是仅次于交错群的第二个有限单群族类。\(^\text{[34]}\)
\item 他还注意到一个特殊事实:PSL(2, p) 是单群并且能作用在 p 个点上,当且仅当 $p = 5, 7$ 或 $11$。\(^\text{[35][36]}\)
\end{itemize}
\subsubsection{加洛瓦理论}
加洛瓦对数学最重要的贡献是他发展出的加洛瓦理论。他认识到,一个多项式方程的代数解与该多项式根的置换群结构有关,这个置换群就是该多项式的加洛瓦群。他发现,如果能够找到这个加洛瓦群的一系列子群,其中每一个都是其下一个的正规子群,且商群是阿贝尔群,那么这个方程就可以用根式解。这一发现成为一个极其有成果的思路,后来被其他数学家推广到方程理论以外的多个数学领域。\(^\text{[30]}\)
\subsubsection{分析学}
加洛瓦还对阿贝尔积分和连分数理论作出了一些贡献。

正如他在最后一封信中所写的那样,\(^\text{[24]}\)加洛瓦从对椭圆函数的研究出发,转向了对最一般的代数微分的积分(今天称为阿贝尔积分)的研究。他将这些积分分为三类。
\subsubsection{连分数}
在加洛瓦1828年的第一篇论文中,\(^\text{[8]}\)他证明了:一个表示二次无理数 ζ 的正则连分数是纯周期的,当且仅当 ζ 是一个最简根式,即满足$\zeta > 1$,且其共轭$\eta$满足 $-1 < \eta < 0$
