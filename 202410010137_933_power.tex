% 幂运算与幂函数(高中)
% keys 幂函数|幂运算
% license Usr
% type Tutor

\pentry{函数\nref{nod_functi},函数的性质\nref{nod_HsFunC}}{nod_9bbd}


在初中阶段,已经学过的几类典型的函数形式:正比例函数$f(x) = ax$、反比例函数$\displaystyle f(x) = \frac{k}{x}$和二次函数$f(x) = ax^2+bx+c$,其实都是幂函数组合得到的特例。可以这么说,幂函数是初中唯一接触过的函数。在高中阶段,将扩展这些知识,进一步研究幂函数的更一般形式。

\subsection{实数域下的幂运算}

幂函数的定义是基于乘方运算的。幂运算是继加法、减法、乘法和除法之后的另一种重要的数学运算。它最开始的意思是表示一个数自乘若干次,这也是初中学过的含义。但随着各自定义的扩张,幂运算包含了一些其他内容。

\begin{definition}{幂运算}\label{def_power_1}
形如
\begin{equation}
a^b=c~.
\end{equation}
的运算:

在研究c与a的关系时称为\textbf{幂运算(power operation,或乘方运算)},其中a称为\textbf{底数(base)},b称为\textbf{指数(powe,指数运算中一般称Exponent)},c称为\textbf{幂值(Exponential Value)},要求$a,b$不同时为$0$。$b=2$时称为\textbf{平方},$b=3$时称为\textbf{立方},$\displaystyle b=\frac{1}{2}$时称为\textbf{开方},$b=-1$时称为\textbf{倒数}。

在研究c与b的关系时称为\textbf{指数运算(exponentiation)},此时c称为\textbf{真数(argument)},要求$a>0,a\neq1$。

尽管有两个名字,但他们是同一种运算,只是关注点不同。
\end{definition}

为了逻辑一致性,幂运算在实数内的合法范围要求如下:
\begin{itemize}
\item $0$当分母的运算是非法的,因此,幂运算要求$0^{-k},k\in \mathbb{N}$时不存在。
\item 在实数域上负数的偶数次方根均无意义,因此,幂运算要求$\displaystyle(-x)^\frac{1}{2k}$,在$k\in \mathbb{N},x\in\mathbb{R}^+$时不存在。
\item 在实数域上负数的无理数次幂不存在,因此,幂运算要求$\displaystyle(-x)^q$,在$q\text{为无理数},x\in\mathbb{R}^+$时不存在\footnote{这里会在\autoref{sub_power_2} 给出说明}。
\end{itemize}

下面是一些幂运算的法则,它是对初中已有的整数幂次的扩展。扩展后指数部分可以是任意实数。如果尚不了解原理或感到陌生,直接默认其成立熟练使用即可,下面的运算均要求合法的前提下进行:

\begin{theorem}{幂运算法则}\label{the_power_1}
\begin{itemize}
\item 分数次幂:$a^{\frac{1}{n}} = \sqrt[n]{a} \quad a \geq 0$
\item 负次幂:$\displaystyle a^{-1}=\frac{1}{a},a\neq0$
\item 零次幂:$a^0=1,a\neq0$
\item 一次幂:$a^1=a$
\item 积的幂:$(a \cdot b)^n = a^n \cdot b^n$
\item 拆分指数的和:$a^{m+n}=a^m \cdot a^n$
\item 拆分指数的积:$a^{m \cdot n}= (a^m)^n =(a^n)^m $
\end{itemize}
\end{theorem}

这里独立说一下正实数的无理数次幂,它的定义是由极限得到的。如果有两列数字分别为$p_i,q_i$,满足对任意的$i$都有$p_i<b<q_i$,如果随着$i$增加$a^{p_i},a^{q_i}$之间的区别越来越小,直至二者几乎相等,那么就认为$a^b$就是那个确定的实数。这里采用了比较模糊的表述,在高中教材中也是如此描述。它实际上反映了极限运算的本质。此处参照\autoref{ex_power_1} 有一个感性认识就可以。

\begin{example}{求$2^\pi$}\label{ex_power_1}
已知$3.1<\pi<3.2$,$3.14<\pi<3.15$,$3.141<\pi<3.142$,$3.1415<\pi<3.1416$……随着精度的增加,两边的数值都更接近$\pi$。从而,$2^{3.1}<2^\pi<2^{3.2}$,$2^{3.14}<2^\pi<2^{3.15}$,$2^{3.141}<2^\pi<2^{3.142}$,$2^{3.1415}<2^\pi<2^{3.1416}$……两侧的数值也逐渐相等,于是那个最终相等的实数就是$2^\pi$。
\end{example}

\subsection{幂函数}

将指数作为参数,底数作为自变量的函数就称为幂函数,幂函数的名称指的就是自变量的幂次是函数值,注意不要与\enref{指数函数}{HsExpF}相混淆。

\begin{definition}{幂函数}
形如
\begin{equation}
f(x) = x^a~.
\end{equation}
的函数称作\textbf{幂函数(Power function)},其中 $a\in\mathbb R$。
\end{definition}

$a$会影响函数的性质的各个方面,在学习时需要时刻注意。

由于幂函数的样式很多,直接研究每个具体的形式会很混乱。但对$y=x^a$,在$x>0$时,总是有定义的,而且此时$y$一定为正。因此,函数一定会在第一象限有图象。本文会先讨论定义域以及奇偶性,奠定好研究基础,然后再专注研究其在第一象限的规律。这也反映了一般分析函数的过程。

\subsection{定义域与奇偶性}

下面的讨论,若$a\in\mathbb{Q}$则以$\displaystyle a=\frac{n}{m}$($n,m$互质,$m>0$)的形式进行。

\subsubsection{定义域}

通常来讲,幂函数的定义域是$x\in\mathbb{R}$,根据前面提到的幂运算非法情况,定义域需要调整:

\begin{itemize}
\item $n\leq0$时,幂函数的定义域为$({-\infty},0)\cup(0,{+\infty})$。
\item ${m}$为偶数时,幂函数的定义域为$[0,{+\infty})$。
\item $a$为无理数时,幂函数的定义域为$[0,{+\infty})$。
\end{itemize}

在正式的研究开始之前,还需要先提及一个特例,当 $a = 0$ 时,幂函数会退化为常值函数 $f(x) = 1(x\neq0)$,该函数在所有非$0$处值恒为 $1$,图像是一个在$0$处无定义的水平直线。它是在定义域上恒为正的偶函数。

这里有必要讨论一下$0^0$。严格来说,$0^0$在高中阶段是未定义的,但在大部分场合(例如幂级数展开)会默认其值为$1$。事实上,如果从幂运算的最基础定义来讲,一般会先定义$0^0=1$来保证归纳得到的其他幂次不会出现多余的参数。这里也可以理解为,幂运算是比指数运算更底层的运算,因此要优先保证幂运算$x^0=1$成立,而非$0^x=0$成立。

\subsubsection{奇偶性}

根据定义域的讨论,只有$m$为奇数时,定义域才是关于$0$对称的,因此,针对奇偶性的讨论将会在$m$是奇数的前提下进行。

由幂运算的定义始终有$(-x)^2=x^2,(-x)^1=-(x)^1$,即$a=2$时是偶函数,$a=1$时是奇函数。下面以此为基础讨论其他形式的幂次的奇偶性:

\begin{itemize}
\item 若$a$为正整数,则取$a=2k+b$,$k$为任意自然数,$b=0$时$a$为偶数,$b=1$时$a$为奇数。从而$(-x)^{2k+b}=(-1)^b(x)^{2k+b}$,结果取决于$b$的取值。代入可知,$a$为偶数时是偶函数,$a$为奇数时是奇函数。
\item 若$a$为负整数,则取$t=x^{-1}$,有$x^a=t^{|a|}$,结论与上一条相同。
\item 若$a$为分数,则取$t=x^\frac{1}{m}$,有$x^\frac{n}{m}=t^{n}$,结论与之前两条相同。
\end{itemize}

整理一下,幂函数的奇偶性分为三种情况:

\begin{itemize}
\item 偶函数:$n$是偶数,$m$是奇数。
\item 奇函数:$n$是奇数,$m$是奇数。
\item 非奇非偶:$m$是偶数或$a$为无理数。
\end{itemize}

事实上,这也是奇偶性名称最直观的体现,它与$n$的奇偶性相同。$n=0$时也符合这里的条件。

总结本章的内容可知,幂函数的形态与$n,m$是否同号以及奇偶关系有关,因此在遇到幂函数时,首先要关注的就是这两个数字的性质。

\subsection{在第一象限的幂函数}

在第一象限时,$x>0$。后面的讨论是在这个基础上的。根据$x^a$运算的性质,在$a>0$时满足:
\begin{enumerate}
\item $x>1\implies x^a>1$
\item $0<x<1\implies 0<x^a<1$
\end{enumerate}

\subsubsection{单调性}
对于$f(x)=x^a$,任取$x_1>x_2$,则平均变化率为
\begin{equation}\label{eq_power_1}
\frac{f(x_1)-f(x_2)}{x_1-x_2}=\frac{(x_2)^a\left(\left(\frac{x_1}{x_2}\right)^a-1\right)}{(x_1-x_2)}~.
\end{equation}

由于\autoref{eq_power_1} 中的分母和$(x_2)^a$为正,因此讨论$\left(\frac{x_1}{x_2}\right)^a$与$1$的关系。设$\displaystyle p=\frac{x_1}{x_2}$,由于$x_1>x_2$,可知$p>1$。从而,$a>0$时,$p^a>1$,$a<0$时,$\displaystyle p^a=\left(\frac{1}{p}\right)^{|a|}<1$。

综上,$a>0$时,\autoref{eq_power_1} 的值大于$0$,函数在第一象限是递增的;反之$a<0$时,函数则是递减的。

\subsubsection{不同的$a$的函数图象的关系}

当$a_1>a_2$时,有$a_1-a_2>0$。此时,对$x>1$,有$x^{a_1-a_2}>1$,即$\displaystyle \frac{x^{a_1}}{x^{a_2}}>1$,由于在第一象限$x^{a_2}>0$,从而有$x^{a_1}>x^{a_2}$;同理,对$0<x<1$,$x^{a_1}<x^{a_2}$。

形象地来说,在第一象限内,如果在$x=1$的右侧从下到上画一条垂线,则会先穿过$a$值较小的函数图象,后穿过$a$值较大的函数图象;在$x=1$的左侧从下到上画一条垂线,则会先穿过$a$值较大的函数图象,后穿过$a$值较小的函数图象。

\subsubsection{其他性质}

将定义域限定在$(0,+\infty)$时,由于$\displaystyle(x^a)^{\frac{1}{a}}=x$,所以$\displaystyle x^{\frac{1}{a}}$与$x^a$ 互为反函数。也即二者的图象关于$y=x$对称,而这条对称轴本身也是$a=1$的情况,即它与自身互为反函数。

顺便一提,有两个特殊的局部:$x$在$0$附近时,如果$a>0$,则$x^a$趋于$0$,如果$a<0$,则$x^a$趋于无穷。当$x$趋向于无穷时,如果$a>0$,则$x^a$趋于无穷,如果$a<0$,则$x^a$趋于$0$。这是根据幂运算的性质得到的。关于“趋于”、“无穷”、“附近”这三个词,现在只需要有一个感性的理解就可以了,它是符合几何直观的。关于这三个词的具体内涵,会在大学阶段学习。

\subsubsection{第一象限图像总结}

把前面的分析组合起来考虑,如果将第一象限以$x=1,y=x,y=1$三条线分成六个区域。这三条线的交点是$(1,1)$,由于不论$a$取何值,$1^a=1$,因而幂函数一定会过点$(1,1)$。如果将$x=1,y=x$之间大于$1$的部分记作区域$\rm{I}$,则可顺时针得到$\rm{I, II, III, IV, V, VI}$一共六个区域:
\begin{itemize}
\item *\footnote{这里的两个与无穷相关的内容,给出只是作为参考,事实上这时已经不在函数的范畴内了。}$a\to+\infty$时,幂函数像一条折线,在$(0,1)$上为$0$,在$1$处突然变为$x=1$,之后都没有定义了(因为是无穷)。
\item $a>1$时,幂函数通过区域$\rm{I,IV}$
\item $a=1$时,幂函数为$y=x$
\item $0<a<1$时,幂函数通过区域$\rm{II,V}$
\item $a=0$时,幂函数为$y=1,(x\neq0)$
\item $a<0$时,幂函数通过区域$\rm{III,VI}$
\item *$a\to-\infty$时,幂函数像一条折线,在$1$处为$x=1$,在$(1,+\infty)$上为$0$,之前都没有定义了(因为是无穷)。
\end{itemize}
函数是光滑的,并且$|a|$越大,越会靠近直线$x=1$,$|a|$越小,越会靠近直线$y=1$。全部画在一起时如\autoref{fig_power_1} 展示了幂函数 $f(x) = x^a$ 第一象限的函数图象。

\begin{figure}[ht]
\centering
\includegraphics[width=8cm]{./figures/86604297d1436480.pdf}
\caption{实参数的幂函数(相同颜色的函数互为反函数)}\label{fig_power_1}
\end{figure}

之前说过,幂函数的图像有规律,但很复杂。现在终于将所有的规律都探索完毕了。还有最后一步就是针对一个具体的表达式,在画出第一象限、分析定义域和奇偶性的基础上,补全其他象限的图。没有必要一次全都列出,因此以一道习题为例进行演示,你也可以自己试试。

\begin{exercise}{描述$f(x)=x^{-\frac{4}{5}}$的图像}
解:

根据前面的分析,$n=-4,m=5,a<0$,所以函数的定义域是$({-\infty},0)\cup(0,{+\infty})$,是偶函数,函数图像关于$y$轴对称。在第一象限的区域$\rm{III,VI}$中,即在$(0,1)$区间内的图像在$y=1$上方,在$(1,+\infty)$区间内的图像在$y=1$下方。在$(0,{+\infty})$上递减,在$({-\infty},0)$上递增。是两条分别过点$(-1,1),(1,1)$的光滑曲线。(*在$0$两侧函数趋于无穷,在无穷处函数值趋于$0$。)
\end{exercise}

\subsection{*复数域扩展}\label{sub_power_2}

\pentry{复数\nref{nod_CplxNo},三角函数\nref{nod_HsTrFu}}{nod_b678}

在实数域上,无理数次幂和偶次方根只定义在正数上。因此,函数只在第一象限有图象,但是根据欧拉公式,引入复数后将函数拓展到复数域中可以有一些新知,下面略窥这些新想法。请注意,下面的内容在高中阶段完全不需要理解,甚至在本科基础阶段都不会涉及。此处给出只是为了扩展视野,如果想要具体了解需要学习\enref{复变函数}{Cplx}。

首先,不加证明地给出\textbf{欧拉公式}\footnote{如果代入$x=\pi$,则会得到恒等式$e^{i\pi}+1=0$,这被很多人称为数学中最优美的等式之一。}:

\begin{theorem}{欧拉公式}
\begin{equation}
\forall x\in\mathbb{R},\cos(x) + i\sin(x)=e^{ix} ~.
\end{equation}
\end{theorem}

注意到,等号左侧是一个模为$1$的复数。因此,任意复数可以表示为 $z = r (\cos \theta + i \sin \theta)$,称$r$ 是复数$z$的\textbf{模},$\theta$ 是复数与实轴的夹角,称为\textbf{辐角},由于正弦与余弦都是周期函数,所以$\theta+2k\pi,k\in\mathbb{Z}$都是辐角,称呼$\theta\in[0,2\pi)$为\textbf{辐角主值}。于是,通过欧拉公式有\textbf{复数的指数形式}:

\begin{equation}
z = r e^{i(\theta+2k\pi)},k\in\mathbb{Z}~.
\end{equation}

由于 $-x$的辐角为 $\pi$,模为$|x|$,将 $-x$ 写成指数形式为:
\begin{equation}
-x = |x| e^{i\pi(1+2k)},k\in\mathbb{Z}~.
\end{equation}

\subsubsection{负数偶次方根的情况}

假设要计算负负实数的偶次方根 $(-x)^{\frac{1}{4}},x>0$,使用幂运算公式:

\begin{equation}
\displaystyle
(-x)^{\frac{1}{4}} = \left( |x| e^{i\pi(1+2k)} \right)^{\frac{1}{4}} =|x|^{\frac{1}{4}} \cdot e^{i\pi\frac{1+2k}{4}}=|x|^{\frac{\pi}{4}}\cos(\frac{1+2k}{4}\pi) + i|x|^{\frac{\pi}{4}}\sin(\frac{1+2k}{4}\pi),k\in\mathbb{Z}~.
\end{equation}

由于正、余弦函数的周期性,与无理数部分不同,这只会产生四个不同的解,分别为:

\begin{itemize}
\item $\displaystyle k = 0,\quad |x|^{\frac{1}{4}} \left( \cos\frac{\pi}{4} + i\sin\frac{\pi}{4} \right) = |x|^{\frac{1}{4}} \cdot \frac{\sqrt{2}}{2} \left( 1 + i \right)$
\item $\displaystyle k = 1,\quad |x|^{\frac{1}{4}} \left( \cos\frac{3\pi}{4} + i\sin\frac{3\pi}{4} \right) = |x|^{\frac{1}{4}} \cdot \frac{\sqrt{2}}{2} \left( -1 +i \right)$
\item $\displaystyle k = 2,\quad |x|^{\frac{1}{4}} \left( \cos\frac{5\pi}{4} + i\sin\frac{5\pi}{4} \right) = |x|^{\frac{1}{4}} \cdot \frac{\sqrt{2}}{2} \left( -1 - i \right)$
\item $\displaystyle k = 3,\quad |x|^{\frac{1}{4}} \left( \cos\frac{7\pi}{4} + i\sin\frac{7\pi}{4} \right) = |x|^{\frac{1}{4}} \cdot \frac{\sqrt{2}}{2} \left( 1 - i \right)$
\end{itemize}

其实,有理数次幂函数 $x^{n/m}$ ($x\in \mathbb R$, $n$ 为整数, $m$ 为正整数) 总是有 $m$ 个可能的值
\begin{equation}
x^{n/m} = \leftgroup{
&\abs{x^n}^{1/m}\E^{\I 2\pi k/m} & (x^n > 0)\\
&\abs{x^n}^{1/m}\E^{\I 2\pi (k+1/2)/m} & (x^n < 0)
}\qquad (k = 0,1,\dots, m-1)~.
\end{equation}

\subsubsection{无理数次幂}

假设要计算负实数的无理数($\pi$)次幂 $(-x)^\pi,(x>0)$,使用幂运算公式:

\begin{equation}
(-x)^\pi = \left( |x| e^{i\pi(1+2k)} \right)^\pi =|x|^\pi \cdot e^{i\pi^2(1+2k)}=|x|^\pi\cos(\pi^2(1+2k)) + i|x|^\pi\sin(\pi^2(1+2k)),k\in\mathbb{Z}~.
\end{equation}

可以看出,负实数的$\pi$次幂是复数,且是由于$\pi^2(1+2k)$与正、余弦函数的周期$2\pi$没有最小公倍数,使得得到的复数有无穷多个。其他无理数次幂也依此计算。事实上,在这样的观点下,正实数的无理数次幂,除了幅角主值对应的是一个实数之外,其余的都是复数。

上面的内容带你对复变函数初窥门径,希望能够让你感受到数学的魅力。