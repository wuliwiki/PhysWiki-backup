% 海伦公式

\footnote{参考 Wikipedia \href{https://en.wikipedia.org/wiki/Heron's_formula}{相关页面}.}若已知三角形的边长, 其面积可以用海伦公式计算
\begin{equation}
A = \sqrt{s(s-a)(s-b)(s-c)}
\end{equation}
其中 $s = (a+b+c)/2$.
\addTODO{证明}
\subsubsection{推导}
\begin{equation}
\begin{aligned}
A&=\frac{1}{2}\text{底}\times\text{高}\\
&=\frac{1}{2}ab\sin\theta\\
&=\frac{1}{2}ab\sqrt{1-\cos^2\theta}\\
&=\frac{1}{2}ab\sqrt{1-\frac{1}{4a^2b^2}(a^2+b^2-c^2)^2}\\
&=\frac{1}{4}\sqrt{4a^2b^2-(a^2+b^2-c^2)^2}\\
&=\frac{1}{4}\sqrt{(2ab+a^2+b^2-c^2)(2ab-a^2-b^2+c^2)}\\
&=\frac{1}{4}\sqrt{[(a+b)^2-c^2][c^2-(a-b)^2]}\\
&=\frac{1}{4}\sqrt{(a+b+c)(a+b-c)(a-b+c)(-a+b+c)}\\
&=\sqrt{\frac{(a+b+c)}{2}\frac{(a+b-c)}{2}\frac{(a-b+c)}{2}\frac{(-a+b+c)}{2}}\\
&=\sqrt{\frac{(a+b+c)}{2}\frac{(a+b-c)}{2}\frac{(a-b+c)}{2}\frac{(-a+b+c)}{2}}
\end{aligned}
\end{equation}
