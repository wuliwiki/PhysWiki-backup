% 量子力学的基本原理(量子力学)
% keys 量子力学|态矢

\pentry{线性代数,微积分,经典力学}


在介绍量子力学的基本原理之前,我们要对“什么是物理理论”做一个简单分析.这是因为量子力学常因“反直觉”而让初学者迷惘,我希望以下讨论能帮助初学者理清概念,从而自然地接受量子力学的语言.这些讨论归入\autoref{QMPrcp_sub1} ,读者可以跳过.

\subsection{从牛顿理论到量子理论}\label{QMPrcp_sub1}

从现代科学哲学的视角看,一个物理理论是一个数学模型,在这个模型中有一些概念是有实验对应的.这就是说,一个物理理论首先是一个数学理论,而使它区分于数学、成为物理的因素即是“实验”,可以直观理解为“有能在仪器上看到、用感官观测到”的量,通常称之为“可观测量”.

以牛顿力学为例.牛顿力学可以认为是四维空间中的几何学,其中“点的坐标”这一概念就是可观测量,它可以显示为尺子上的数值.更准确地说,考虑到牛顿力学中时间的绝对性,该理论应该是一维空间上处处沾了一片三维空间的“纤维丛”上的几何学,不同的观察者眼中会有不同的三维空间坐标,但是时间坐标不变.

光是几何学,那就不是物理理论了,因此牛顿力学还规定了质点运动的三大定律,相当于限定哪些几何轨迹是“合法”的.这三大定律定义了一个概念,“力”.力本身不是可观测量,但我们可以借助此概念来描述物体运动的规律.质量为$m$的物体被劲度系数为$k$、原长为$l$的弹簧拉着,做角速度为$\omega$的匀速圆周运动,则规律预言,弹簧的伸长量是$\frac{l\omega^2m}{k-m\omega^2}$.伸长量是可观测量,所以我们可以做实验,看看测出来的伸长量是否是这个值,以此来判断牛顿理论的准确性.
\addTODO{弹簧的长度是心算的,可能有误.核算后再删除此“未完成”.}

牛顿力学怎么定义质点的状态?时间坐标、空间坐标以及空间坐标对时间的导数等.这种定义方式很直观,但我们要跳出直觉,理解牛顿力学的“数学结构”,才能平滑地过渡到量子理论的数学结构.

量子理论则用了截然不同的数学模型,它可以被认为是希尔伯特空间中的线性代数理论.它讨论系统的量子态,并将量子态定义为一个希尔伯特空间中的矢量,这些矢量可以表示为波函数.和力一样,态矢量、波函数等概念都是不可观测的.和牛顿三定律一样,态矢量也不是任意变化的,约束态矢量变化的就是著名的薛定谔方程.因此,许多课本中会告诉你,薛定谔方程在量子力学中的地位,就和三定律在牛顿力学中的地位一样.

牛顿力学中,质点的坐标是可观测量;量子力学中,\textbf{厄米算符}的\textbf{特征值}是可观测量.
























