% 能量法解谐振动问题

\pentry{简谐振子\upref{SHO}}

由于谐振动系统的总能量是常量,反过来,我们可以用机械能守恒定律来导出谐振动的运动方程.仍以弹簧振子为例,在任一时刻,振子作谐振动的总机械能等于势能加上动能,即
\begin{equation}
E=E_k+E_p=\frac{1}{2} m v^{2}+\frac{1}{2} k x^{2}
\end{equation}
由于总机械能$E$是常量,不随时间而改变,上式对时间$t$求导得
\begin{equation}
m v \frac{\mathrm{d} v}{\mathrm{d} t}+k x \frac{\mathrm{d} x}{\mathrm{d} t}=0
\end{equation}
由速度和坐标间的关系$v = \dfrac{\mathrm d x}{\mathrm dt}$,代入整理后即得谐振动的运动方程
\begin{equation}
\frac{\mathrm{d}^{2} x}{\mathrm{d} t^{2}}+\frac{k}{m} x=0
\end{equation}

下面来看一个例题.

\begin{example}{考虑弹簧质量的弹簧振子}
当考虑弹簧的质量时,试用能量法求解弹簧振子的周期,设弹簧的质量为$m_0$,且小于振动物体的质量$m$,弹簧的劲度系数为$k$.

\begin{figure}[ht]
\centering
\includegraphics[width=10cm]{./figures/EnerVi_1.png}
\caption{考虑弹簧质量的弹簧振子} \label{EnerVi_fig1}
\end{figure}
\end{example}

