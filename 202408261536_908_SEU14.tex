% 东南大学 2014 年 考研 量子力学
% license Usr
% type Note

\textbf{声明}:“该内容来源于网络公开资料,不保证真实性,如有侵权请联系管理员”

\textbf{1.(15 分)}下列叙述是否正确:
(1) 电子自旋态空间是 2 维的\\
(2) 全同费米子系统的波函数具有交换反对称性\\
(3) 氢原子的基态是简并的\\
(4) 宇称不是可观测量\\
(5) 每一个可观测量均可以用一个幺正算符表示。\\

\textbf{2.(15 分)}质量为$m$的粒子作一运动,儿率守恒定理为$\partial \rho/\partial t + \partial j/\partial x = 0$,其中:$\rho(x, t) = |\psi|^2,j(x, t) = -i\hbar/2m \left( \psi^* \partial \psi/\partial x - \psi \partial \psi^*/\partial x \right).$
\begin{enumerate}
\item 若粒子处于定志$\psi = \phi(x) \exp(-iEt/\hbar)$, 试证:$j=c$(与$x,t$无关的常数),.
\item 若自由拉子处于动星本征态$\psi(x, t) = \exp(ipx/\hbar - iEt/\hbar)$, 试证: $j = p/m.$
\end{enumerate}

\textbf{3.(15 分)}试在坐标表象中写出:
\begin{enumerate}
        \item 位置算符 $\hat{x}$ 的本征函数;
        \item 动量算符 $\hat{p}_x$ 的本征函数;
        \item $\hat{x}, \hat{p}_x]$ 的共同本征函数。
    \end{enumerate}

