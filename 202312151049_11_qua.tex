% 二次型(线性代数)
% license Xiao
% type Tutor

\begin{issues}
\issueTODO 
\begin{itemize}
\item 本篇放在线性代数第五章 线性方程组,面向初学者。用矩阵语言介绍二次型,区分高代中张量形式的二次型(虽然它们当然是一回事)。
\item 不知道初等变换在哪个词条,需要加入到预备知识里
\item 未完成:定义惯性指数,二次型的秩,退化, 惯性定理的证明。
\item  本篇最好加上一些应用实例,譬如几何上通过二次型简化方程,以及物理上的惯性张量(?)等等。
\end{itemize}
\end{issues}
\pentry{集合的基数\upref{CardiN}}
\begin{definition}{}
二次型是关于变量的二次齐次多项式。即满足$q(k\bvec{v})=k^2 q(\bvec{v})$,对于任意$\bvec{v}=(v_1,v_2...v_n),k\in \mathbb F$。比如:
\begin{equation}
q(\bvec{v})=av_1x_2+bv_2v_1+cv_1^2+dv_2^2~.
\end{equation}
容易验证,任意二次型都可以写为如下形式:
\begin{equation}
q(\bvec{v})=\bvec{v}^TQ\bvec{v}~.
\end{equation}
一般称矩阵$Q$为二次型$q$的方阵形式。
\end{definition}

二次型相应的指标表达式为$q(\bvec v)=\sum\limits_{i,j}Q^i_j v^j v^i$。矩阵元素$Q^i_j$为结果中$v^j v^i$的系数。由于$v^j v^i=v^i v^j$,因此$Q^i_j+Q^j_i$为实际相应项的系数,满足该条件的矩阵可以有很多个。
\begin{equation}
\begin{aligned}
& \left(\begin{array}{ll}
x & y
\end{array}\right)\left(\begin{array}{cc}
1 & 2 \\
0 & -1
\end{array}\right)\left(\begin{array}{l}
x \\
y
\end{array}\right) \\
= & x^2+2 x y+0 y x-y^2 \\
= & x^2+x y+y x-y^2 \\
= & \left(\begin{array}{ll}
x & y
\end{array}\right)\left(\begin{array}{cc}
1 & 1 \\
1 & -1
\end{array}\right)\left(\begin{array}{l}
x \\
y
\end{array}\right)
\end{aligned}~.
\end{equation}
但二次型总能对应唯一一个对称矩阵\footnote{前提为:域的特征不为2。因为特征为2的域有:-1=1}。
\subsection{二次型的坐标变换}
当我们用上述定义表示二次型时,如果基向量组变换,二次型的形式亦有所不同。具体而言,如果利用过渡矩阵$B$改变基向量组,由相似变换的知识可知,在新基下向量$\bvec{v'}=B\bvec{v}$,那么新的二次型形式为:
\begin{equation}
\bvec{v}^T Q\bvec{v}=(B^{-1}\bvec{v})^T Q(B^{-1}\bvec{v})=\bvec{v}^T (B^{-1})^{T}QB^{-1}\bvec{v}~.
\end{equation}
因此,新的二次型形式对应矩阵$Q'=(B^{-1})^{T}QB^{-1}$,这就是常说的合同变换。合同变换的结果是同一二次型在不同基下的表示\footnote{$f(\bvec{v}):V\rightarrow \mathbb F$没有发生改变,虽然坐标不同,但还是同一个向量嘛}。
\begin{definition}{合同}
如果存在可逆方阵$C$使得
\begin{equation}
B=C^T AC~
\end{equation}
则称矩阵$A,B$合同。
\end{definition}
可以证明合同关系是一种\textbf{等价划分}。即满足反身性、传递性与对称性,等价划分实际上是在划分不同的二次型,等价类内二次型有不同的矩阵形式而已。

\begin{definition}{二次型的等价性}
给定线性空间的二次型$q_i$,如果$q_1,q_2$在某\textbf{两个}基下矩阵形式相同,则称这两个二次型等价。
\end{definition}

通过坐标变换,二次型可以简化为最简单的一种形式:对角矩阵。在对角矩阵下,二次型形式只有平方项,这就是所说的\textbf{标准二次型}。如果二次型 $q$
等价于标准二次型 $p$,那么称 $p$ 是 $q$ 的\textbf{标准形}。

由于实对称矩阵总能通过合同变换化为对角矩阵。因此实数域上的二次型总有标准形。
\begin{theorem}{}
给定实数域上的二次型 $V^T QV$,那么它总有标准形。
\end{theorem}
由于合同变换的结果总为对称矩阵,因此证明过程相对简洁,只需要利用对角元,通过初等变换把上三角的非对角元部分化为0即可\footnote{回顾初等变换,左乘可逆矩阵是行变换,右乘是列变换}。
\subsection{二次型的正定性}
二次型正定意味着$f(\bvec{v})>0$,负定则意味着$f(\bvec{v})< 0$。为了方便,我们可以进一步把实对称矩阵化为对角元为$\pm 1$,非对角元为$0$的形式,比如想要把第$n$个对角化化为$\pm 1$,则合同变换为第$n$列乘以$k$和第$n$行乘以$k$,在\textbf{实数域}上$k^2>0$,因此二次型正定意味着对角元都为$1$。

上述形式还意味着基向量组是“标准正交”的。比如\textbf{“正定”}即对角矩阵为$E$,则二次型$\bvec{v}^{T}Q\bvec{v}=\bvec{v}^{T}\bvec{v}$。

因此,二次型实际上是同向量内积的推广。

可以证明,二次型的“正负号”数量不会随基的改变而改变。
\begin{theorem}{惯性定理}
给定\textbf{实数域}$\mathbb R$上的\textbf{有限维}线性空间$V$和其上一个二次型$f$。令$E=\{\bvec e_i\}$和$F=\{\bvec \theta_i\}$是$V$上的两组关于$f$的\textbf{标准正交基},$Q,P$分别为$f$在$\{\bvec e_i\}$和$\{\bvec \theta_i\}$上的矩阵形式。定义
\begin{equation}
\begin{aligned}
E^{+} & =\left\{\mathbf{e}_i \in E \mid Q^i_i>0\right\} \\
E^{-} & =\left\{\mathbf{e}_i \in E \mid Q^i_i<0\right\} \\
E^0 & =\left\{\mathbf{e}_i \in E \mid Q^i_i=0\right\}
\end{aligned}~.
\end{equation}
以及
\begin{equation}
\begin{aligned}
F^{+} & =\left\{\mathbf{e}_i \in E \mid P^i_i>0\right\} \\
F^{-} & =\left\{\mathbf{e}_i \in E \mid P^i_i<0\right\} \\
F^0 & =\left\{\mathbf{e}_i \in E \mid P^i_i=0\right\}
\end{aligned}~.
\end{equation}
则必有
\begin{equation}
\left\{
\begin{aligned}
\mid E^{+}\mid&=\mid F^{+}\mid\\
\mid E^{-}\mid&=\mid F^{-}\mid\\
\opn{Span}\, E^{0}&=\opn{Span}\, F^{0}\\
\end{aligned}\right.~.
\end{equation}
上述的符号$\mid \quad\mid$表示集合内元素的数量,$\opn{Span}$表示集合元素张成的向量空间.
\end{theorem}
Proof\footnote{引自Jie Peter的《代数学基础》}.
为了证明方便,这里拓展二次型的概念到广义内积,即对称双线性函数,这是为了符合内积的对称性$(\bvec v,\bvec w)=(\bvec w,\bvec v)$。对称矩阵总能保证该性质成立。设$q(\bvec v)=v^T Q v$,拓展至广义内积
\begin{equation}
f(\bvec v,\bvec w)=v^T Q w~.
\end{equation}

首先验证第三条,设$V_1$为$\{\bvec u\mid f(\bvec{v},\bvec u)=0,\forall \bvec v\in V\}$,证明思路为$\opn{Span}E_0=V_1=\opn{Span}F_0$。

容易验证,$\opn{Span}E_0\subseteq V_1$,现在通过反证法证明$ V_1\subseteq \opn{Span}E_0$。假设$\bvec x\in V_1,\bvec x\notin \opn{Span}E_0$,则$\bvec v$必至少含有一个$E^0$以外的基向量,设为$\bvec e_i$,因此$f(\bvec v,\bvec e_i)=0$,则与假设矛盾。因此$\opn{Span}E_0=V_1$。由于内积结果和具体的基无关,同理有$V_1=\opn{Span}F_0$,第三条得证。

现在证明第一条,只需要否定$\mid E^{+}\mid<\mid F^{+}\mid$和$\mid E^{+}\mid>\mid F^{+}\mid$的情况即可。现在假设$\mid E^{+}\mid<\mid F^{+}\mid$这意味着$E^0\cup E^{-1} \cup F^{+}$必然线性相关,不然整体维度会大于空间维度。


