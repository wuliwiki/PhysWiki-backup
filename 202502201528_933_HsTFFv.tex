% 函数视角下的三角函数(高中)
% keys 函数|三角函数|周期|性质
% license Usr
% type Tutor

\begin{issues}
\issueDraft
\end{issues}

\pentry{三角函数\nref{nod_HsTrFu},函数\nref{nod_functi},函数的性质\nref{nod_HsFunC},导数\nref{nod_HsDerv}}{nod_5a43}

在前面已经学习过三角函数的定义,并基于这些定义推导出了诱导公司和同教的三角函数兼关系下面将会在这个这些已知内容的基础上来讨论在函数视角下的三角函数。或者说从函数的视角来看一看三角函数具有哪些性质。

\subsection{三角函数的性质}

\subsubsection{定义域}
易得,正弦函数和余弦函数的\textbf{定义域为全体实数},正切函数的定义域为 $\begin{Bmatrix}\alpha|\alpha \neq \frac{\pi}{2}+k\pi,k\in Z\end{Bmatrix}~.$

\subsubsection{周期性}

正弦函数、余弦函数、正切函数的都是周期函数,根据定义易得,正弦函数和余弦函数,周期为 $2k\pi(k\in Z,k\neq0)$,正切函数的周期为 $k\pi(k\in Z,k\neq0)$.


\subsubsection{图像}
根据前面的推导,可以得到基本三角函数的图像如下图。
\begin{figure}[ht]
\centering
\includegraphics[width=14.25cm]{./figures/14fd66d8d1e6e0b5.png}
\caption{正弦函数和余弦函数} \label{fig_HsTFFv_1}
\end{figure}
可以看出正弦函数和余弦函数是定义域为 $R$ 值域为 $[-1,1]$ 最小正周期 $T = 2\pi$ 的周期函数。

\begin{figure}[ht]
\centering
\includegraphics[width=14.25cm]{./figures/6f97182187b36e36.png}
\caption{正切函数与余切函数} \label{fig_HsTFFv_3}
\end{figure}

作为扩展,下面也给出正割函数与余割函数的函数图像,他们的性质均可通过与正弦和余弦的关系分析得到,此处不予赘述。

\begin{figure}[ht]
\centering
\includegraphics[width=14.25cm]{./figures/56f93ee1a7fb0faa.png}
\caption{正割函数与余割函数} \label{fig_HsTFFv_2}
\end{figure}

\subsection{导数}