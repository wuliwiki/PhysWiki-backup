% 正则变换
% 哈密顿量|分析力学|拉格朗日|哈密顿原理|哈密顿正则方程|正则变换


\pentry{哈密顿正则方程\upref{HamCan}}
简单来说,正则变换就是在相空间中选择另一组变量 $P,Q$ 作为坐标,使得和原来的变量 $p,q$ 之间一一对应,并且要求新坐标 $P,Q$ 仍满足哈密顿正则方程。本词条阐明正则变换的意义,条件,生成函数及生成函数的四种主要形式。

为了适应现代语言的书写习惯,本词条将指标q的坐标分量指标写在上面,而动量p的指标写在下面。
\subsection{为什么要作正则变换}
若存在循环坐标 $q^s$, 即拉氏量 $L$ 不显含 $q^s$,则哈密顿量
\begin{equation}
H=\sum_i p_i\dot q^i-L~,
\end{equation}
显然也不含循环坐标 $q^s$。由哈密顿正则方程\autoref{HamCan_eq2}~\upref{HamCan}
\begin{equation}\label{CanTra_eq7}
\dot p_s=-\pdv{H}{q^s}=0~,
\end{equation}
即 $p_s$ 是体系的循环积分(守恒量)。这就是说,哈密顿正则方程具有容易获得系统循环积分的优点。

之所以要在相空间中选择新的变量作为坐标,是因为人们总希望尽可能多找到系统运动方程的积分(守恒量),也就是要使尽可能多的坐标成为循环坐标。然而,能否出现循环坐标及循环坐标的数量是同坐标系的选择直接相关的,这可以从下面的例子看出。
\begin{example}{}
已知在二维平面上,质量为 $m$ 的质点受到来自 $O$ 点的引力场的作用(引力势能为 $-\mu\frac{m}{r}$)。那么在极坐标和平面直角坐标系下,其拉氏量和动量分别为
\begin{equation}
\begin{aligned}
L&=\frac{m}{2}\qty(\dot r^2+r^2\dot\theta^2)+\mu\frac{m}{r}~,\\
L&=\frac{m}{2}\qty(\dot x^2+\dot y^2)+\mu\frac{m}{\sqrt{x^2+y^2}}~,\\
p_\theta&=\pdv{L}{\dot \theta}=mr^2\dot\theta~,\\
p_r&=\pdv{L}{\dot r}=m\dot r~,\\
p_x&=\pdv{L}{\dot x}=m\dot x~,\\
p_y&=\pdv{L}{\dot y}=m\dot y~.
\end{aligned}
\end{equation}

而哈密顿量分别为(注意,在拉氏量 $L=T-V$ 时,$H=T+V$)
\begin{equation}
\begin{aligned}
H&=\frac{m}{2}\qty(\dot r^2+r^2\dot\theta^2)-\mu\frac{m}{r}\\
&=\frac{1}{2m}\qty(p_r^2+\frac{p_\theta^2}{r^2})-\mu\frac{m}{r}~,\\
H&=\frac{m}{2}(\dot x^2+\dot y^2)-\mu\frac{m}{\sqrt{x^2+y^2}}\\
&=\frac{1}{2m}\qty(p_x^2+p_y^2)-\mu\frac{m}{\sqrt{x^2+y^2}}~.
\end{aligned}
\end{equation}

从哈密顿量看到,在极坐标系中 $\theta$ 是循环坐标,从而就有一循环积分;而在直角坐标系中,并无循环坐标存在。 
\end{example}

考虑到通过正则方程获得循环积分的便利性,和循环坐标与坐标系选择的相关性,人们一方面要在进行坐标变换时,使新坐标描述系统时仍能满足正则方程;一方面要设法找到成为循环坐标的变量。说白了,我们所需要的和真正感兴趣的坐标变换,绝非随意的一种变换,而是指在限定意义下的坐标变换,即以新坐标描述的体系的哈密顿量仍能满足正则方程为前提条件,这样的坐标变换就称为\textbf{正则变换(canonical transformation)}。

\subsection{正则变换的条件}
在相空间中,设旧坐标 $(p,q)$ 和新坐标 $(Q,P)$ 的转换关系为
\begin{equation}
Q^i=Q^i(p,q,t),\quad P_i=P_i(p,q,t)~.
\end{equation}

设新坐标描述的哈密顿量为 $H'=H'(P,Q)$,要使运动方程具有正则形式,即新变量要满足下面的正则方程
\begin{equation}
\dot Q^i=\pdv{H'}{P_i},\quad \dot P_i=-\pdv{H'}{Q^i}~.
\end{equation}
该方程可以通过下面变分得到
\begin{equation}\label{CanTra_eq1}
\delta\int\qty(\sum_iP_i\dd Q^i-H'\dd t)=0~,
\end{equation}
而旧变量的哈密顿正则方程是通过变分
\begin{equation}\label{CanTra_eq2}
\delta\int\qty(\sum_ip_i\dd q^i-H\dd t)=0
\end{equation}
得到的。要使 \autoref{CanTra_eq1} ,\autoref{CanTra_eq2} 描述的是同一物理系统,那么它们的被积函数只能相差某个关于相空间坐标的函数 $F$ 的全微分,因为这是在变分时不起作用的常数。因此
\begin{equation}
\sum_ip_i\dd q^i-H\dd t=\sum_iP_i\dd Q^i-H'\dd t+\dd F~.
\end{equation}
显然,若坐标变换满足该条件,该变换便是\textbf{正则变换}。

\begin{definition}{正则变换}
在相空间 $(p,q)$ 中,若以新变量 $(P,Q)$ 为新坐标,且新旧坐标的变换
\begin{equation}\label{CanTra_eq3}
Q^i=Q^i(p,q,t),\quad P_i=P_i(p,q,t)
\end{equation}
满足关系
\begin{equation}\label{CanTra_eq4}
\sum_ip_i\dd q^i-H\dd t=\sum_iP_i\dd Q^i-H'\dd t+\dd F~,
\end{equation}
其中 $H,H'$ 分别是以旧坐标 $(p,q)$ 和新坐标 $(P,Q)$ 描述的哈密顿量。则称坐标变换\autoref{CanTra_eq3} 为对应物理系统的\textbf{正则变换},函数 $F$ 称为该正则变换的\textbf{生成函数}或\textbf{母函数}。
\end{definition}

\subsection{生成函数的四种形式}
既然生成函数只要是关于相空间坐标的函数,那么我们就可以期望它是变量 $p,q,P,Q$ 的函数,在各种可能的生成函数中,主要的生成函数是自变量为一新一旧变量和时间 $t$ 的函数。这有四种可能:$F_1(p,P,t),F_2(p,Q,t),F_3(q,P,t),F_4(q,Q,t)$。
\subsubsection{生成函数 $F_4(q,Q,t)$}
改写 \autoref{CanTra_eq4} 为:
\begin{equation}\label{CanTra_eq5}
\dd F=\sum_ip_i\dd q^i-\sum_iP_i\dd Q^i+(H'-H)\dd t~,
\end{equation}
则得到的生成函数是关于新旧变量 $q,Q$ (和时间 $t$)的函数 $F=F(q,Q,t)$。此时,有
\begin{equation}
p_i=\pdv{F}{q^i},\quad P_i=-\pdv{F}{Q^i},\quad H'=H+\pdv{F}{t}~.
\end{equation}
显然,$F_4=F$。
\subsubsection{生成函数 $F_3(q,P,t)$}
若在\autoref{CanTra_eq5} 中让 $F$ 由 $F+\sum_iP_i Q^i$ 代替,那么
\begin{equation}
\dd (F+\sum_iP_i\dd Q^i)=\sum_i p_i\dd q^i+\sum_i Q^i\dd P_i+(H'-H)\dd t~.
\end{equation}
于是就得到关于 $q,P$ 的母函数 $F_3=F+\sum_iP_i Q^i$。此时
\begin{equation}\label{CanTra_eq8}
p_i=\pdv{F_3}{q^i},\quad Q^i=\pdv{F_3}{P_i},\quad H'=H+\pdv{F_3}{t}
\end{equation}

\subsubsection{生成函数 $F_2(p,Q,t)$}
同理,若用 $F-\sum_ip_iq^i$ 代替 $F$,就有
\begin{equation}\label{CanTra_eq6}
\dd (F-\sum_ip_i q^i)=-\sum_i q^i\dd p_i-\sum_i P_i\dd Q^i+(H'-H)\dd t~,
\end{equation}
于是就得到关于 $p,Q$ 的母函数 $F_2=F-\sum_ip_i q^i$。此时
\begin{equation}
q^i=-\pdv{F_2}{p_i},\quad P_i=-\pdv{F_2}{Q^i},\quad H'=H+\pdv{F_2}{t}
\end{equation}
\subsubsection{生成函数 $F_1(p,P,t)$}
用 $F_2+\sum_iP_iQ^i$ 代替\autoref{CanTra_eq6} 中的 $F_2$,就有
\begin{equation}
\dd(F-\sum_ip_i q^i+\sum_i P_i Q^i)=-\sum_i q^i\dd p_i+\sum_i Q^i\dd P_i+(H'-H)\dd t
\end{equation}
于是就得到关于 $p,P$ 的母函数 $F_1=F-\sum_ip_i q^i+\sum_i P_i Q^i$。此时
\begin{equation}
q^i=-\pdv{F_1}{p_i},\quad Q^i=\pdv{F_1}{P_i},\quad H'=H+\pdv{F_1}{t}
\end{equation}

对于这四种母函数,实际上只需要记住关于 $q,Q$ 的母函数 $F_4=F$,其满足
\begin{equation}
\dd F=\sum_ip_i\dd q^i-\sum_iP_i\dd Q^i+(H'-H)\dd t
\end{equation}
\subsubsection{小记}
其它几种可以这样方便记忆:若母函数自变量相对 $F$ 只改变关于新变量 $P,Q$ 的函数,则只要母函数 $F$ 加上 $\sum_iP_iQ^i$;若只改变成关于旧变量 $p,q$ 的函数,则 $F$ 减 $-\sum_ip_iq^i$;若同时改变,则 $F$ 既加上 $\sum_iP_iQ^i$ 又减 $-\sum_ip_iq^i$。

值得注意的是,母函数关于时间的偏导数总具有相同的形式 $H'-H$。

正则变换的广泛性,使得原来的广义坐标和广义动量 $q,p$ 的概念丧失其原始含义,因为 $P,Q$ 都和 $p,q$ 联系在一起,$Q$ 不再是纯粹的空间坐标,$P$ 也不再是纯粹的动量。$P,Q$ 的区别仅仅在于名称的不同,例如变换 $Q^i=p_i,P_i=-q^i$ 并不影响方程的正则形式。由于这个原因,变量 $p,q$ 经常被称为\textbf{正则共轭变量}。

使用何种母函数,要视方便程度而定。
\begin{theorem}{}
泊松括号想对正则变换是不变的,即
\begin{equation}
\{f,g\}_{p,q}=\{P,Q\}_{P,Q}
\end{equation}
其中下指标是指微分运算是对哪两个变量进行的。
\end{theorem}
\footnote{参考 朗道《力学》}为避免繁琐的证明,可这样论证:假定 $g$ 是某个系统的哈密顿量,那么 $\{f,g\}_{p,q}=\dv{f}{t}$。然而 $\dv{f}{t}$ 只依赖于该系统的运动性质,而不以变量的特定选择有关。因此,从一组正则变量变到另一组正则变量时,泊松括号并不改变。
\begin{example}{一个重要的例子}
设 $p_t,q_t$ 是正则变量 $p,q$ 在 $t$ 时刻的值,而 $p_{t+\tau},q_{t+\tau}$ 是在另一时刻 $t+\tau$ 的值。后者是前者的某一函数
\begin{equation}
\begin{aligned}
p_{t+\tau}&=p(p_t,q_t,t,\tau)\\
q_{t+\tau}&=q(p_t,q_t,t,\tau)
\end{aligned}
\end{equation}
若将这一变换视作 $p_t,q_t$ 到 $p_{t+\tau},q_{t+\tau}$ 的坐标变换,则这即是正则变换,因为由\autoref{EPAct_eq1}~\upref{EPAct}
\begin{equation}
\dd S=\sum \qty(p_{t+\tau}\dd {q_{t+\tau}}-p_{t}\dd {q_{t}})-(H_{t+\tau}-H_t)\dd t
\end{equation}
于\autoref{CanTra_eq5} 对比,可知 $-S$ 是变换的母函数。 

\end{example}

%\footnote{参考 \cite{Goldstein}。}对于同一系统, 我们可以采用不同的广义坐标, 广义动量和哈密顿量。 那么如何在这些不同选择中做变换呢? 我们把一组广义坐标和广义动量记为 $Q^1,\dots, Q^N, p_1, \dots, p_N$, 哈密顿量为 $H(q, p, t)$。 另一组记为 $Q^1,\dots, Q^N, P_1, \dots, P_N$, 哈密顿量为 $K(Q, P, t)$。 那么他们满足什么关系,或者在二者之间进行变换呢?

%首先, 根据勒让德变换\autoref{HamCan_eq1}~\upref{HamCan}, 拉格朗日量可以表示为
%\begin{equation}
%L = \sum_i \dot Q^i p_i - H(q, p, t)
%\end{equation}

%如果对于不同的变量 $Q, P$, 以及哈密顿量 $K$, 那么

%代入哈密顿原理

%* 生成函数并不能生成所有的正则变换, 另一种方法是什么矩阵
