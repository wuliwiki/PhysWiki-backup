% 一阶常微分方程解法:常数变易法
% 微分方程|differential equation|考研

\pentry{常微分方程解法:变量可分离方程\upref{ODEa1}}

观察以下方程:

\begin{equation}\label{ODEa2_eq1}
\frac{\dd y}{\dd x}=P(x)y+Q(x)
\end{equation}
其中$P$和$Q$都是所考虑区间上的连续函数.

这样的方程被称作“一阶线性方程”,其中如果$Q(x)=0$,则还可以称之为“齐次”的方程,否则便是“非齐次”的.

容易看出,一阶齐次线性方程$\frac{\dd y}{\dd x}=P(x)y$就是变量可分离方程,我们已经在预备知识\textbf{一阶常微分方程解法:变量可分离方程}\upref{ODEa1}中详细讨论过了.

\begin{exercise}{}\label{ODEa2_exe1}
证明:$\frac{\dd y}{\dd x}=P(x)y$的通解是$y=C\E^{\int P(x)\dd x}$.
\end{exercise}

\autoref{ODEa2_exe1} 给出了齐次方程的解,其中含一个待定常数$C$.非齐次方程是齐次方程的拓展,可以用常数变易法来解.

\textbf{常数变易法}就是将齐次方程解中的$C$拓展为一个待定函数$C(x)$,再代回非齐次方程,看看能不能解出这个$C(x)$;如果能,那么非齐次方程也就得解了.

\begin{example}{}
考虑方程\autoref{ODEa2_eq1} .根据齐次方程的解\autoref{ODEa2_exe1} ,假设\autoref{ODEa2_eq1} 的解是$y=C(x)\E^{\int P(x)\dd x}$.

则
\begin{equation}
\begin{aligned}
\frac{\dd y}{\dd x}=C'(x)\E^{\int P(x)\dd x}+C(x)P(x)\E^{\int P(x)\dd x}
\end{aligned}
\end{equation}

比较\autoref{ODEa2_eq1} 和\autoref{ODEa2_eq1} ,发现我们需要令
\end{example}















