% 投影和表示 (转载)
% license Usr
% type Art

(本文根据 CC-BY 协议转载自季燕江的《量子序曲》, 进行了重新排版和少量修改)

\subsection{太阳比喻}

在晴朗的日子里出去走一走,我们看不见自己,但却能看见自己的影子。这是很有意思的事情。

如果太阳不是很晒的话,我们可以站在一个空旷平坦的地上观察我们的影子,它和阳光射来的方向相对,在地上留下一个阴影,如果时间早的话,太阳升的不是很“高”,光线会斜斜地在地上投下一个较长的阴影,随着时间的流逝,太阳会沿着自己的轨道在天空中划出一个圆弧,随着太阳的升“高”,阴影会越来越短,当太阳升到最“高”的时候,阴影也最短。

但说高并不精确,我们可以把眼睛眯起,朝太阳的方向看,所谓“高”就是我们要仰起脖子才能“追踪”到太阳,我们仰起脖子的角度越大、太阳越高,我们可以把这个仰角定义为“太阳-观察者”连接线与地面的夹角$\theta$。当这个角度为$90^o$的时候,太阳在天顶,光线垂直地射下来,此时我们在地上的影子会“消失”\footnote{阴影之内没有光线是暗的,而阴影之外会被阳光照亮,光在这里更多地体现出“粒子性”,它以直线传播,绝对不会绕过障碍物。光从$\theta$方向照射到物体上,在地面上留下一个影子,假设物体的高度是$H$,影子的长度将是$H \cdot \frac{\cos \theta}{\sin \theta } = H \cdot \cot \theta$。}。

\begin{figure}[ht]
\centering
\includegraphics[width=6cm]{./figures/2f63172810481ba4.png}
\caption{太阳光⼊射,与竖直⽅向成 α 角。} \label{fig_QMPre3_4}
\end{figure}

有时我们也以竖直的方向为基准,定义太阳光与竖直方向的夹角为$\alpha$($\alpha = \frac{\pi}{2} - \theta $),当$\alpha = 0$时,阳光笔直地照射在地面上,这时照射到单位面积上太阳光的能量最大,当角度$\alpha$逐渐增大时,照射到单位面积上太阳光的能量会变小,变小的比例正比于$\cos \alpha$。
人类走出非洲后,一路向北,先来到中近东,然后扩散到欧洲、亚洲等其它地方。中近东、欧洲、亚洲比非洲的纬度高,太阳会以一个更大的角度$\alpha$照射下来,随着$\alpha$的增大,单位表面积上地球吸收到的能量会减少,气温会随之降低,尤其是夜晚温度会更低。

我们现在都是住在屋子里的,但在远古人类甚至连制造房屋的技术都没有发明,冷了只能去山洞。但山洞里已经有其他动物占领了,比如曾广泛分布于欧洲和中近东各地的洞熊(cave bear)。洞熊的体型庞大,雄性洞熊的体重可高达1吨,可以想象与洞熊争夺山洞的战役是人类走出非洲后碰到的一大挑战。在这个过程中,火的使用是决定性的,因为在各种动物中只有人类不怕火,甚至还学会了使用火,发明了保存火种的方法,甚至制造火种的技术\footnote{维特鲁威在《建筑十书》中说:“远古时候,人类生来就像出没森林、洞穴和丛林中的野兽一样,茹毛饮血,辛苦度日。那时有一个地方,生长着密集繁茂的森林,狂风袭来,树木剧烈摇晃,树枝相互摩擦而起火。住在附近的人们被火焰吓坏了,逃之夭夭。但后来他们凑近时发现,火的热量对人体有极大的好处,他们将原木投入火中,将火种保存下来。”}。

可以想象人类曾长期生活在生有篝火的洞穴里,而这样的一个原始记忆也被用于比喻说理中,比如柏拉图在《理想国》中借用“洞穴”比喻了城邦和知识。

那么我们的洞穴经验是什么样的呢?

首先需要一个封闭的空间,比如在伸手不见五指的夜晚,任何一个房屋都可以是个洞穴,山洞无非也是个封闭的空间。

漆黑的夜晚,我们呆在山洞或封闭的房子里。我们什么都看不见,我们看不见自己,也看不见他人和物体。我们点燃一个火把或蜡烛。人是喜欢光亮的,于是都凑过去,此时我们在墙壁上看到影子,因为火把的光比较弱,反射一次后就基本没亮光了,洞穴中的影子会比阳光下的更显著和夸张,光和影在一起给我们的视觉极大的刺激。

阳光下我们不能清晰地看到物体的轮廓,但在洞穴经验中,阴影和光亮是截然分开的,我们甚至可以想象一个人去描摹阴影的轮廓。

用简单的线条去对象化一个物体是认识活动的开始,比如自我是不可见的,俗话说我们是在别人的眼睛(其实就是镜子)里认识自己的,但在洞穴经验里,我们在墙壁上能直接看见自己的阴影,比如我们可以面对着墙壁,背对着火把,伸出一只手,举过头顶……然后,我看见我面对的那个阴影会同步地作出这种种动作。

这就从视觉经验上把自我对象化了,同时我还能看见别人的阴影和其他物体的阴影……

火把的好处是可以随意移动,要想看清楚什么东西我们只需要把火把拿过来照一照就可以了。这意味着我们可以控制光线行进的方向,我可以让光向上方射,只需要我们把火把放在物体的下方,我们也可以让光向左射,只需要把火把放在物体的右边……

在洞穴中,我们举着火把从各个方向照物体,为的是要看清某物,光从某个方向射过来,我们看到的是光照亮的那个“面”,物体其他面的形象对我们是隐藏的,我们必须移动火把,使光从另外的方向射向物体,这个动作其实就是选择,我们选择从另一个角度“照亮”物体,刚才对我们显现的将隐藏在黑暗里,但新的面,新的形象会对我们显现。

\begin{figure}[ht]
\centering
\includegraphics[width=6cm]{./figures/d272e38bdd758e77.png}
\caption{三视图就是往三个⽅向做投影。} \label{fig_QMPre3_5}
\end{figure}

同时照亮所有的面则需要很多火把,比如我们可以从两个、三个,甚至六个方向上照亮物体。假设物体是三维的,并且假设物体是“透明”的,我们需要至少从三个互相垂直的方向上照亮物体,才能获得对物体的整体认识。这个其实就是工程里的三视图,上视、侧视和前视\footnote{假如物体不是透明的那就很复杂,因为还涉及物体内部构造的问题,即便不考虑内部构造,我们也得假设物体必须是“凸起”的,才能通过六视图获得物体的整体概念。}。

光源(太阳)、物体、阴影也构成一个常见的“认识论比喻”,这就是柏拉图的“太阳喻”。我们能“看”,是因为有光,而光是源自太阳的;光照射在物体上,我们像洞穴中背对着光源的原始人一样只能看到物体的阴影,即物体本身是不对我们显现的,对我们显现的只是物体的阴影。

这里我们的兴趣并不是介绍哲学上的“太阳喻”,我们只是借助这一图像建立量子力学中的“表示概念”。

在量子力学中没有物体,只有量子态,使用狄拉克记号,记作$\left| \alpha \right\rangle$,量子态本身是无法直接被“看”到的。我们需要对量子态建立一个表示,所谓表示就是选择一个观看的方式。

以观看物体为例,就是我们拿着火把以什么样的方式把物体仔细打量一番?比如我们可以选择从$x$,$y$,$z$三个方向上照亮物体,从三个方向照亮物体其实就是把物体对这三个方向做投影。

我们把矢量$V$看做是最简单的物体,往三个方向做投影就是:

\begin{equation}
\text{$x$方向,方向是$e_x$,投影是$V_x$ = $e_x$ $\cdot V$}~
\text{$y$方向,方向是$e_y$,投影是$V_y = e_y \cdot V$}~
\text{$z$方向,方向是$e_z$,投影是$V_z = e_z \cdot V$}~
\end{equation}

我们把量子态想象成一个矢量(态矢量,state vector),它可能有很多“方向”,每个方向都有一个单位矢量,称作基矢,记为$\left| n \right\rangle$。

在量子力学中,态矢量$\left| \alpha \right\rangle $并不直接对应观测值,在这个意义下我们也说我们是“看不见”量子态的。但我们能“看见”态矢量的投影$\left\langle n | \alpha \right\rangle $,根据玻恩的统计解释,一个量子态处在$\left| n \right\rangle$态的几率正比于$\left|  \left\langle n | \alpha \right\rangle  \right|^2 $,我们管$\left\langle n | \alpha \right\rangle$叫几率幅\footnote{我们一般只讨论已经归一化了的量子态,所以就是等于了,即量子态$\left| \alpha \right\rangle$处在$\left| n \right\rangle$态的几率等于$\left\langle n | \alpha \right\rangle^2 $。}。

我们把投影算符$P_n$定义为:$\left| n \right\rangle \left\langle n \right|$,对量子态$\left| \alpha \right\rangle$投影的效果就是获得投影$\left| n \right\rangle \left\langle n | \alpha \right\rangle$。

\begin{figure}[ht]
\centering
\includegraphics[width=6cm]{./figures/07c4087d68d4bb07.png}
\caption{ ⼀个量⼦⼒学版的“太阳比喻”} \label{fig_QMPre3_6}
\end{figure}

这里基矢$\{ \left| n \right\rangle \}$的选取是关键,它决定了观看方式,我们一般是通过构造一组和哈密顿$H$两两相互都对易的算符集$\{  H, A, B, ...  \}$来构造$\{ \left| n \right\rangle \}$的。

这样几率$\left| \left\langle n | \alpha \right\rangle \right|^2 $就有了明确的物理意义:

\begin{equation}
\text{现在我们把$\{ \left| n \right\rangle \}$改写成$\{ \left| E_n , a, b, ... \right\rangle  \}$,几率$\left|  \left\langle E_n, a, b | \alpha \right\rangle  \right|^2 $就是量子态$\left| \alpha \right\rangle $坍缩到态$\left| E_n, a, b \right\rangle$上的几率。}~
\end{equation}

\begin{figure}[ht]
\centering
\includegraphics[width=6cm]{./figures/9900ea0c38088112.png}
\caption{请添加图片标题} \label{fig_QMPre3_7}
\end{figure}

\subsection{狄拉克记号}

下面我们将结合自旋1/2这个例子讨论量子力学中的狄拉克记号(Dirac Notation)。在理论物理学中记号法很重要,合适的记号法会使数学推导清晰简洁并且不容易出错。

狄拉克记号就是括号,括号在英文里是:

\begin{equation}
bracket~
\end{equation}

我们把它表示为:

\begin{equation}
\left\langle  {bra}  \right|  {c} \left|   {ket}  \right\rangle~
\end{equation}

左边括号的“尖头”是向左的,我们称$\left\langle  {  }  \right|$是左矢空间(bra space)中的一个矢量(简称“左矢”),右边的括号的“尖头”是向右的,我们称$\left|  { }  \right\rangle $是右矢空间(ket space)中的一个矢量(简称“右矢”)。左矢和右矢中间夹着一个“内容”(content),这个内容是算符(operator),我们一般用大写字母$A$表示算符。

\begin{figure}[ht]
\centering
\includegraphics[width=6cm]{./figures/0209b0d5b21f90af.png}
\caption{⼀个卡通化的“bra”和“cat”,即“胸罩”和“咪咪”。这个图对应的是“内积”。} \label{fig_QMPre3_8}
\end{figure}

\subsubsection{向量、左矢和右矢}

针对自旋1/2的量子态,我们建立如下映射/表示关系:
\begin{equation}
|+\rangle \equiv \begin{pmatrix}1 \\ 0 
\end{pmatrix}\qquad \\\\ 
|-\rangle \equiv \begin{pmatrix}0\\1
\end{pmatrix}~
\end{equation}
\begin{equation}
\end{equation}

这里我们把$\left| z \pm \right\rangle$简记为$\left| \pm \right\rangle$,$\left| \pm \right\rangle$是互相排斥的同时完备的两个分类“标准”,任意的一个二维列向量$\left| \alpha \right\rangle$可以表示为它们的叠加:

\begin{equation}
|\alpha\rangle = a|+\rangle + b|-\rangle = \begin{pmatrix}
a \\ b \end{pmatrix}~
\end{equation}

这里的叠加系数 $a, b$ 是复数 (complex number)。在量子力学中我们用态矢量 $|\alpha\rangle$ 表示系统的一个态。复数 $c$ 是所调对易数,它可以随便出现在列向量 (态) 的左边或者右边,或者我们把它和 $\alpha$ 写在一起放在括号里:

\begin{equation}
c|\alpha\rangle = |\alpha\rangle c = |c\alpha\rangle~
\end{equation}

假设有两个向量$\left|  \alpha \right\rangle$、$\left| \beta \right\rangle$,我们想知道这两个向量的相似程度。如果在笛卡尔空间,我们让$\left|  \alpha \right\rangle$向$\left|  \beta \right\rangle$投影,记做:

\begin{equation}
\left|  \beta \right\rangle \left\langle \beta | \alpha  \right\rangle~
\end{equation}

投影之后,向量在$\left| \beta \right\rangle$方向上,同时大小变成$\left\langle \beta | \alpha  \right\rangle$,即$\left| \alpha \right\rangle$在$\left| \beta \right\rangle$方向上的投影是$\left\langle \beta | \alpha  \right\rangle$倍的$\left| \beta \right\rangle$。

$\left\langle \beta | \alpha  \right\rangle$是个数,假如:

\begin{equation}
\left| \alpha \right\rangle = \left( \begin{array}{ccc} a \\ b \end{array} \right)~ \\
\left| \beta \right\rangle = \left( \begin{array}{ccc} c \\ d \end{array} \right)~
\end{equation}

$\left\langle \beta | \alpha  \right\rangle$被定义为:

\begin{equation}
\left\langle \beta | \alpha  \right\rangle = \left( c^* , d^* \right) \left( \begin{array}{ccc}  a \\ b  \end{array}  \right) = c^* a + d^* b~
\end{equation}

我们称$\left\langle \beta | \alpha  \right\rangle$为内积(inner product),这样定义内积的好处是会使向量$\left| \alpha \right\rangle$,自己和自己的内积——$ \left\langle \alpha | \alpha  \right\rangle $——是非负的。

\begin{equation}
\left\langle \alpha | \alpha  \right\rangle = \left( a^* , b^* \right) \left( \begin{array}{ccc}  a \\ b  \end{array}  \right) = a^*a + b^* b \ge 0~
\end{equation}

如果向量$\left| \alpha \right\rangle$本身不是0向量,那么我们总可以把它归一化(normalized),归一化的因子是:

\begin{equation}
\frac{1}{\sqrt { a^* a + b^* b }}~
\end{equation}

为了叙述的方便,下面我们引入几个术语(terminology):

\begin{enumerate}
\item 

$\left| \alpha \right\rangle$叫狄拉克“右矢”(ket),对每一个狄拉克右矢而言,都有一个狄拉克“左矢”(bra)与其对应,记作:$\left\langle \alpha \right|$。

\item

以二维复系数线性空间为例,一个狄拉克右矢可以表示为$\left| \alpha \right\rangle = \left( \begin{array}{ccc} a \\ b \end{array}  \right)$,与之对应的左矢$ \left\langle \alpha \right| $可表示为:$\left( a^* , b^*  \right)$。

即转置(transpose)再加复共轭(complex conjugate)。

转置的定义是:

\begin{equation}
\left( a_{ij}  \right)^T = \left( a_{ji} \right)~
\end{equation}

即矩阵中“行变成列,列变成行”。比如:第一行第一列的元素,变到第一行第一列;第二行第一列的元素,变到第一行第二列。

复共轭的定义是:

\begin{equation}
\left( a_{ij}  \right)^* = \left( a_{ij}^* \right)~
\end{equation}

即矩阵中每一个元素都取复共轭。

转置再加复共轭整体叫做厄米共轭(Hermite conjugate),表示为:

\begin{equation}
\left( a_{ij}  \right)^{\dagger} = \left( a_{ji}^* \right)~
\end{equation}

\item

我们管$\left| \alpha \right\rangle  \leftrightarrow  \left\langle \alpha \right|$这样的关系叫“对偶”(dual correspondence),即对一个右矢而言存在一个左矢与之对应,相反对一个左矢而言也存在一个右矢与之对应。

\end{enumerate}

关于对偶,我们可以总结如下性质:

\begin{enumerate}
\item 

两向量相加的对偶:

\begin{equation}
\left| \alpha \right\rangle + \left| \beta \right\rangle  \leftrightarrow \left\langle \alpha \right| + \left\langle \beta \right|~
\end{equation}

\item

复数乘一个向量的对偶:

\begin{equation}
c \left| \alpha \right\rangle  \leftrightarrow c^* \left\langle \alpha \right|~
\end{equation}

\item

内积的对偶:

\begin{equation}
\left\langle \beta | \alpha \right\rangle = \left\langle \alpha | \beta \right\rangle^*~
\end{equation}

\end{enumerate}

\subsubsection{算符}

算符(operator)就是操作,我们把算符定义为对矢量的操作,它把一个矢量映射为另一个矢量,比如它把一个右矢映射为另一个右矢:

\begin{equation}
A \left| \gamma \right\rangle = \left| \alpha \right\rangle \left\langle \beta |  \gamma \right\rangle =  \left\langle \beta |  \gamma \right\rangle  \left| \alpha \right\rangle = \left| \delta \right\rangle~
\end{equation}

这里$ \left| \alpha \right\rangle \left\langle \beta \right| $,我们称之为两个向量的外积(outer product),就是算符$A$,它把右矢$\left|  \gamma \right\rangle$映射为$\left| \delta \right\rangle$。

\begin{equation}
A \left| \gamma \right\rangle = \left| \delta \right\rangle~
\end{equation}

$\left| \delta \right\rangle$的对偶是$\left\langle  \delta \right|$,那么$A \left| \gamma \right\rangle$的对偶是什么呢?

我们把$A \left| \gamma \right\rangle$的对偶记作:

\begin{equation}
\left\langle \gamma \right| A^\dagger = \left\langle \delta \right|~
\end{equation}

为了避免运算中出错,这里算符$A^\dagger$看作是向右——即对右矢——起作用的。

我们称$A^\dagger $是$A$的厄米共轭,并且$A$与$A^\dagger $互为厄米共轭。

下面我们来做几个练习:

\begin{enumerate}
\item 

$A = \left| \alpha \right\rangle \left\langle \beta \right|$,那么$A^\dagger = ?$

解:

把$A$作用于任意右矢$\left| \gamma \right\rangle$,然后求它的对偶:

\begin{equation}
\left| \alpha \right\rangle \left\langle \beta | \gamma \right\rangle \leftrightarrow \left\langle \beta | \gamma \right\rangle^* \left\langle \alpha \right| = \left\langle \gamma | \beta \right\rangle \left\langle \alpha \right|~
\end{equation}

同时我们又有厄米共轭的定义:

\begin{equation}
A \left| \gamma \right\rangle \leftrightarrow \left\langle \gamma \right|   A^\dagger~
\end{equation}

因此:

\begin{equation}
\left(  \left| \alpha \right\rangle \left\langle \beta \right|  \right)^\dagger = \left| \beta \right\rangle \left\langle \alpha \right|~
\end{equation}


\item

定义两算符的相乘$AB $为

\begin{equation}
AB \left| \alpha \right\rangle = A \left( B \left| \alpha \right\rangle  \right)~
\end{equation}

那么$AB$整体的厄米共轭$(AB)^\dagger = ?$

解:

先考虑$B$对$\left| \alpha \right\rangle$的作用$\left| \beta \right\rangle =  B \left| \alpha \right\rangle$,它的对偶是:

\begin{equation}
\left| \beta \right\rangle = B \left| \alpha \right\rangle \leftrightarrow \left\langle \beta \right| = \left\langle \alpha \right| B^\dagger~
\end{equation}

现在考虑$AB \left| \alpha \right\rangle$的对偶:

\begin{equation}
A B \left| \alpha \right\rangle = A \left| \beta \right\rangle \leftrightarrow \left\langle \beta \right| A^\dagger = \left\langle \alpha \right| B^\dagger A^\dagger ~
\end{equation}

同时我们又知道:

\begin{equation}
A B \left| \alpha \right\rangle  \leftrightarrow \left\langle \alpha \right|  \left( AB \right)^\dagger~
\end{equation}

因此:

\begin{equation}
\left(  AB \right)^\dagger = B^\dagger A^\dagger~ 
\end{equation}

\item

定义厄米算符为:

\begin{equation}
A^\dagger = A~
\end{equation}

即:如果一个算符$A$的厄米共轭$A^\dagger $就是它自己的话,我们就说它是厄米的(Hermitian)。

证明如下等式对厄米算符成立:

\begin{equation}
\left\langle \beta \right| A \left| \alpha \right\rangle =  \left\langle \alpha \right| A \left| \beta \right\rangle^*~
\end{equation}

证:

\begin{equation}
\left\langle \beta \right| A \left| \alpha \right\rangle = \left\langle \beta \right|  \left( A \left| \alpha \right\rangle \right)  =   \{ \left(  \left\langle \alpha \right| A^\dagger \right) \left| \beta \right\rangle \}^* ~
\end{equation}

考虑到$A$是厄米的,$A^\dagger = A$:

\begin{equation}
\left\langle \beta \right| A \left| \alpha \right\rangle = \left\langle \alpha \right| A \left| \beta \right\rangle^*~
\end{equation}

假如$\left| \beta \right\rangle = \left| \alpha \right\rangle$的话,

\begin{equation}
\left\langle \alpha \right| A \left| \alpha \right\rangle = \left\langle \alpha \right| A \left| \alpha \right\rangle^*~
\end{equation}

一个数的复共轭就是这个数自己,这说明$\left\langle \alpha \right| A \left| \alpha \right\rangle$是个实数(real number)。

由于物理可观测量的数值必须是实数(而不能是复数),在量子态$\left| \alpha \right\rangle$下,$\left\langle \alpha \right| A \left| \alpha \right\rangle$就可以用来表示物理量A的期望值(expectation value),前提是我们必须用一个厄米算符$A$来表示物理量A。

\end{enumerate}

\subsubsection{本征值问题}

考虑:

\begin{equation}
A \left| a' \right\rangle = a' \left| a' \right\rangle~
\end{equation}

即:算符作用于右矢上,其后果是这个算符自身再乘上一个复数,于是这个复数就可用来标记(命名)这个右矢。

$a'$叫做本征值(eigen value),$\left| a' \right\rangle$叫做本征矢(eigen vector),$\left| a' \right\rangle$是与$a'$对应的本征矢。把所有符合$A \left| a' \right\rangle = a' \left| a' \right\rangle$条件的$a'$及其对应的$\left| a' \right\rangle $都求出来的挑战叫做“求解本征值问题”。

我们可以证明,假如$A$是厄米算符的话,其本征值$a'$一定是实数。并且如果有两个本征值$a'$和$a''$,假如$a' \neq a''$的话,它们对应的本征矢$\left\langle a' | a'' \right\rangle = 0$,即$\left| a' \right\rangle$和$\left| a'' \right\rangle$一点都不像。

证:

由本征值问题:

\begin{equation}
A \left| a' \right\rangle = a' \left| a' \right\rangle~
\end{equation}

出发。其对偶形式是:

\begin{equation}
\left\langle a'' \right| A^\dagger =  \left\langle a'' \right| A = a''^* \left\langle a'' \right|~
\end{equation}

再把上式作用于右矢$\left| a' \right\rangle$

\begin{equation}
\left\langle a'' \right| A \left| a' \right\rangle = a' \left\langle a'' | a' \right\rangle = a''^* \left\langle a'' | a' \right\rangle~
\end{equation}

整理一下:

\begin{equation}
\left( a' - a''^* \right) \left\langle a'' | a' \right\rangle = 0~
\end{equation}

假如$a'$就是$a''$的话,只要$\left| a' \right\rangle$不是0矢量,$\left\langle a' | a' \right\rangle > 0 $,因此:

\begin{equation}
a' = a'^*~
\end{equation}

一个数的复共轭就等于这个数自己,这意味着这个数的虚部为0,或这个数是实数。

现在假设$a' \neq a''$,同时利用刚刚证明出的结果$a'$和$a''$都是实数,考虑:

\begin{equation}
\left( a' - a'' \right) \left\langle a''  | a'  \right\rangle = 0~
\end{equation}

因为$a' - a'' \neq 0$,内积$\left\langle a''  | a'  \right\rangle$就必须为0。

~

假设$\left| a' \right\rangle$是$A$的本征矢,任意乘一个复数,$c \left| a' \right\rangle$也是$A$的本征矢,一般我们提到的本征矢$\left| a' \right\rangle$都是归一化的本征矢。

根据以上证明,算符$A$的所有本征矢$\left| a' \right\rangle$构成一个集合$\{ \left| a' \right\rangle \}$,并且满足正交归一(orthonormal)的关系。

\begin{equation}
\left\langle a' | a'' \right\rangle = \delta_{a', a''}~
\end{equation}

这里$\delta_{a', a''}$是克罗尼克记号,当$a' = a''$时,$\left| a' \right\rangle$是归一的,即$\delta_{a', a''} = 1$;当$a' \neq a''$时,$\left| a' \right\rangle$和$\left| a'' \right\rangle$是正交的,即$\delta_{a', a''} = 0$。

这里我们忽略了一种情况,即一个本征值$a'$,它有可能对应多个不同的本征矢量,比如$\left| a'_1 \right\rangle $、$\left| a'_2 \right\rangle $……$\left| a'_f \right\rangle$。一个本征值和$f$个不同的本征矢量对应,我们称这种情况为简并,简并度为$f$。

对于存在简并的本征值问题,原则上我们也能得到一个正交归一的本征矢的集合$\{ \left| a' \right\rangle \}$,难的是给它们起名字,或做标记。

因为一个$a'$有可能对应好几个不同的$\left| a'_1 \right\rangle $、$\left| a'_2 \right\rangle $、……,仅仅靠$a'$我们是无法分辨它们的。

这就要求我们考虑新的本征值问题。比如,考虑算符$B$的本征值问题,为了使$B$及其对应的分类命名体系在物理上有意义,我们要求$B$也是个厄米算符。

假设$A$和$B$是对易的,即

\begin{equation}
AB = BA~
\end{equation}

$A$的本征值问题存在简并,即存在某个本征值$a'$,有$\left| a'_1 \right\rangle$,$\left| a'_2 \right\rangle$,……,$\left| a'_f \right\rangle$,$f$个不同的本征矢与$a'$对应,现在的关键是找到给它们标记的方案,光靠$A$的本征值$a'$是不够的。

首先,对$i = 1, 2,... f$,我们有:

\begin{equation}
A \left| a'_i \right\rangle = a' \left| a'_i \right\rangle ~
\end{equation}

我们定义这$f$个不同的本征矢的线性叠加$\left| \alpha \right\rangle $:

\begin{equation}
\left| \alpha \right\rangle = \sum\limits_i c_i \left| a'_i \right\rangle~
\end{equation}

对$\left| \alpha \right\rangle $,我们依然有:

\begin{equation}
A \left| \alpha \right\rangle  = a' \left| \alpha \right\rangle~
\end{equation}

考虑到$A$,$B$对易:

\begin{equation}
AB \left| \alpha \right\rangle = B A \left| \alpha \right\rangle = a' B \left| \alpha \right\rangle~
\end{equation}

上式中最左边的一项和最右边的一项说明$B \left| \alpha \right\rangle $依然是$A$的本征矢。

假设算符$B$满足关系:$B \left| \alpha \right\rangle = b' \left| \alpha \right\rangle$,求解本征值问题:

\begin{equation}
B \left| \alpha \right\rangle = b' \left| \alpha \right\rangle~
\end{equation}

$\left| \alpha \right\rangle$可以表示为一个$f$维的列向量。

\begin{equation}
\left| \alpha \right\rangle \dot = \left( \begin{array}{ccc} c_1 \\ c_2 \\ ... \\ c_f  \end{array} \right)~
\end{equation}

我们需要求解的线性代数问题是:

\begin{equation}
\left( B - b' I \right) \left( \begin{array}{ccc} c_1 \\ c_2 \\ ... \\ c_f  \end{array} \right)  = 0~
\end{equation}

如果我们要求关于$\left( \begin{array}{ccc} c_1 \\ c_2 \\ ... \\ c_f  \end{array} \right)$的非零解,我们就得要求如下行列式为0:

\begin{equation}
\det \left( B - b' I \right) = 0~
\end{equation}

这是一个关于$b'$的一元$f$次方程。

幸运的话,我们能求出$f$个不同的$b'$的取值,$b'_1, b'_2, ... b'_f$。把$b'$值代回方程我们能得到不同的本征矢,记作$\left| a', b'_i \right\rangle$,$i = 1, 2, ...f$。


~

我们一般会笼统地把$A$,$B$的共同本征态记作$\left| a', b' \right\rangle$,

%但也可能求不出$f$个不同的$b'$,我们可以笼统地把$A$,$B$的共同本征态记作$\left| a', b' \right\rangle$,

%存在$A$,$B$的共同本征态$\left| a', b' \right\rangle$使得:

\begin{equation}
A \left| a', b' \right\rangle = a' \left| a', b' \right\rangle \qquad\\
B \left| a', b' \right\rangle = b' \left| a', b' \right\rangle~
\end{equation}

在引入新的指标$b'$后,$a', b'$在一起就有可能把简并消除掉,即不同的本征矢$\{  \left| a', b' \right\rangle  \}$可以和指标$\{ a', b' \}$构成“一一对应”。

如果还存在简并的话,原则上我们可以引入更多两两都对易的厄米算符的集合;由于在量子力学中我们用厄米算符表示物理量,我们也说找到一个“两两都对易的力学量(或物理量)的集合”,$\left( H, A, B, ...  \right)$。

这里总会包括哈密顿算符$H$,因为哈密顿算符对应能量,在物理里我们最关心能量,比如对求解氢原子的量子力学问题,我们会取这样一个两两都对易的力学量的集合$\left( H, L^2, L_z  \right)$。

$H$对应的是主量子数$n$,对应的本征值问题是:

\begin{equation}
H \left| n \right\rangle = E_n \left| n \right\rangle~
\end{equation}

$L^2$是轨道角动量算符的平方,$L_z$是轨道角动量算符在$z$方向上的分量,对应的本征值问题是:

\begin{equation}
L^2 \left| l, m \right\rangle  =  l (l + 1 )\hbar^2 \left| l, m \right\rangle \qquad
L_z \left| l, m \right\rangle  =  m \hbar \left| l, m \right\rangle~
\end{equation}

$n, l, m$一起构成了对氢原子量子态$\left| n, l, m \right\rangle$进行描述的恰当语言。(所谓恰当就是分类的合理,每个不同的对象都有命名,而命名和命名之间有绝不重复)

所有的$\left| n, l, m \right\rangle$,两两都是正交归一的:

\begin{equation}
\left\langle n', l', m' | n, l, m \right\rangle = \delta_{nn'} \delta_{ll'} \delta_{mm'}~
\end{equation}

它们的集合$\{ \left| n, l, m \right\rangle \}$构成了基矢(basisi),任意用$H$描述的物理系统的量子态可以表示为$\{ \left| n, l, m \right\rangle \}$的线性叠加。

\subsubsection{投影算符}

假设$\{  \left| n \right\rangle \}$构成基矢,基矢就相当于笛卡尔坐标系里的$\{ i, j, k \}$

笛卡尔坐标系里的任意向量$V$可以表示为向$i$($x$方向的单位矢量),$j$($y$方向的单位矢量),$k$($z$方向的单位矢量)三个方向的投影之和。

\begin{equation}
V = i V_x + j V_y + k V_z~
\end{equation}

这里$V_x$是$V$向$i$的投影,表示为$V \cdot i$,我们也可以把它们按照狄拉克记号的形式进行改写:

\begin{equation}
\left| V \right\rangle =\left| i \right\rangle \left\langle i | V \right\rangle + \left| j \right\rangle \left\langle j | V \right\rangle + \left| k \right\rangle \left\langle k | V \right\rangle~
\end{equation}

这里的$\left| i \right\rangle \left\langle i \right|$,$\left| j \right\rangle \left\langle j \right|$和$\left| k \right\rangle \left\langle k \right|$就是投影算符,分别表示把向量$\left| V \right\rangle $向$\left| i \right\rangle $,$\left| j \right\rangle $或$\left| k \right\rangle $投影的动作。

向所有方向投影再加起来就是单位算符$I$

\begin{equation}
I = \left| i \right\rangle \left\langle i \right| + \left| j \right\rangle \left\langle j \right| + \left| k \right\rangle \left\langle k \right|~
\end{equation}

对量子力学而言,我们的基矢是$\left| n \right\rangle $,投影算符$P_n$是:

\begin{equation}
P_n = \left| n \right\rangle \left\langle n \right|~
\end{equation}

表示向第$n$个基矢$\left| n \right\rangle$的投影动作。

单位算符是:

\begin{equation}
I = \sum\limits_n \left| n \right\rangle \left\langle n \right|~
\end{equation}

这个表达式是量子力学中最重要的公式,因为我们可以选择不同的分类体系(其实就是表象),这样我们就可以让物理陈述在不同的表象间变换。

\subsubsection{幺正算符}

考虑一个已经被归一化了的量子态$\alpha$,

\begin{equation}
\left\langle \alpha | \alpha \right\rangle = 1~
\end{equation}

假设态可以经历一个变化,这个变化用算符$U$表示,它使得:

\begin{equation}
\left| \alpha \right\rangle  \to  U \left| \alpha \right\rangle \\ \qquad
\left\langle \alpha \right|  \to  \left\langle \alpha \right| U^\dagger~
\end{equation}

假设我们考虑的物理过程$U$不会导致粒子的产生和湮灭,即:

\begin{equation}
\left\langle \alpha | \alpha \right\rangle \to  \left\langle \alpha \right| U^\dagger U \left| \alpha \right\rangle = 1~
\end{equation}

这意味着$U$满足:

\begin{equation}
U^\dagger U = I~
\end{equation}

这里$I$表示单位算符,但我们往往也把单位算符就简单地写作1,实际上在进行推导的时候,必要的简略是不可避免的,甚至如何简略公式的表达本来就是物理思维的一部分。

满足关系$U^\dagger U = 1$的算符$U$是幺正算符(Unitary operator),一般来说幺正算符不是厄米算符。

%并不仅仅是厄米算符才在量子力学中有用。

\subsection{自旋的泡利矩阵表示}

回到自旋1/2的例子,我们定义自旋算符$S$为

\begin{equation}
S = i S_x + j S_y + k S_z~
\end{equation}

这里的$S_z$是自旋算符在$z$方向上的分量,它可定义为:

\begin{equation}
S_z = \frac{\hbar}{2} \left( \left| + \right\rangle \left\langle + \right| - \left| - \right\rangle \left\langle - \right|  \right)~
\end{equation}

它使得:

\begin{equation}
S_z \left| + \right\rangle = \frac{\hbar}{2} \left| + \right\rangle \\ \qquad
S_z \left| - \right\rangle = - \frac{\hbar}{2} \left| - \right\rangle~
\end{equation}

这对应$z$方向的非均匀磁场(SGz)可以把“非极化”的Ag原子束分裂为相等percentage的上下两束。

我们还可以写出$S_z$的矩阵表示:

\begin{equation}
S_z =  \frac{\hbar}{2} \left( \begin{array}{ccc}  1 & 0 \\ 0 & 1   \end{array}  \right)~
\end{equation}

我们把$S_x$定义为:

\begin{equation}
S_x = \frac{\hbar}{2} \left( \left| x+ \right\rangle \left\langle x+ \right| - \left| x- \right\rangle \left\langle x- \right| \right)~
\end{equation}

这意味着$x$方向的非均匀磁场(SGx)可以把“非极化”的Ag原子束分裂为相等percentage的“左右”两束。

我们一般取$\left| z+ \right\rangle = \left| + \right\rangle = \left( \begin{array}{ccc} 1 \\ 0 \end{array} \right)  $,$\left| z- \right\rangle = \left| + \right\rangle = \left( \begin{array}{ccc} 0 \\ 1 \end{array} \right)$,这种对基矢的取法叫$S_z$表象。这意味着$S_x$的矩阵表示就不能是对角的了。

我们可以这么来计算$S_x$的矩阵表示:

\begin{equation}
S_x \dot = \left( \begin{array}{ccc}  \left\langle + \right| S_x \left| + \right\rangle  &  \left\langle + \right| S_x \left| - \right\rangle \\   \left\langle - \right| S_x \left| + \right\rangle  &   \left\langle - \right| S_x \left| - \right\rangle  \end{array} \right) ~
\end{equation}

利用:

\begin{equation}
\left| x+ \right\rangle = \frac{1}{\sqrt{2}} \left(  \begin{array}{ccc} 1 \\ 1  \end{array} \right) \qquad
\left| x- \right\rangle = \frac{1}{\sqrt{2}} \left(  \begin{array}{ccc} -1 \\ 1  \end{array} \right)~
\end{equation}

我们可以计算比如矩阵中的第一项:

\begin{equation}
\left\langle + \right| S_x \left| + \right\rangle = \frac{\hbar}{2} \left( \left\langle + | x+ \right\rangle \left\langle x+ | + \right\rangle - \left\langle + | x- \right\rangle \left\langle x- | + \right\rangle   \right) \\
{} = \frac{\hbar}{2} \left( \frac{1}{2} - \frac{1}{2} \right) = 0~
\end{equation}

最终计算出$S_x$的矩阵表示是:

\begin{equation}
S_x  = \frac{\hbar}{2} \left( \begin{array}{ccc} 0 &  1 \\ 1 & 0 \end{array} \right)~
\end{equation}

我们把$S_y$定义为:

\begin{equation}
S_y = \frac{\hbar}{2} \left( \left| y+ \right\rangle \left\langle y+ \right| - \left| y- \right\rangle \left\langle y- \right| \right)~
\end{equation}

其矩阵表示为:

\begin{equation}
S_y \dot = \left( \begin{array}{ccc}  \left\langle + \right| S_y \left| + \right\rangle  &  \left\langle + \right| S_y \left| - \right\rangle \\   \left\langle - \right| S_y \left| + \right\rangle  &   \left\langle - \right| S_y \left| - \right\rangle  \end{array} \right) = \frac{\hbar}{2} \left( \begin{array}{ccc} 0 &  -i \\ i & 0 \end{array} \right) ~
\end{equation}

小结一下:

\begin{equation}
S_x = \frac{\hbar}{2} \sigma_x =  \frac{\hbar}{2} \left( \begin{array}{ccc} 0 &  1 \\ 1 & 0 \end{array} \right)~ \qquad
S_y = \frac{\hbar}{2} \sigma_y = \frac{\hbar}{2} \left( \begin{array}{ccc} 0 &  -i \\ i & 0 \end{array} \right)~  \qquad
S_z = \frac{\hbar}{2} \sigma_z = \frac{\hbar}{2} \left( \begin{array}{ccc} 1 &  0 \\ 0 & -1 \end{array} \right)~
\end{equation}

这里$\sigma_x = \left( \begin{array}{ccc} 0 &  1 \\ 1 & 0 \end{array} \right) $,$\sigma_y = \left( \begin{array}{ccc} 0 &  -i \\ i & 0 \end{array} \right) $,$\sigma_z = \left( \begin{array}{ccc} 1 &  0 \\ 0 & -1 \end{array} \right) $。再加上单位矩阵$I =  \left( \begin{array}{ccc} 1 &  0 \\ 0 & 1 \end{array} \right)$,总共四个合称泡利矩阵(Pauli matrices)。

%任意一个$2 \times 2$的厄米矩阵都可以表示为以上四个泡利矩阵的线性叠加。

%这是因为一个$2 \times 2$的厄米矩阵可表示为:

%\begin{equation}
%\left( \begin{array}{ccc}  a  & c+ id  \\  c- i d & b  \end{array} %\right)
%\end{equation}

%这里$a, b, c, d$都是实数,正好对应四个独立的变量。

我们可以验证自旋$S$确实符合角动量的性质:

\begin{equation}
\left[ S_x , S_y  \right]   =  i \hbar S_z \qquad
\left[ S_y , S_z  \right]   =  i \hbar S_x \qquad
\left[ S_z , S_x  \right]   =  i \hbar S_y \qquad
\left[S_x , S^2 \right]  =  \left[S_y , S^2 \right] = \left[S_z , S^2 \right] = 0~
\end{equation}

升降算符$S^+$和$S^-$可以大大简化涉及自旋算符的量子力学计算:

%我们定义升降算符$S^+$和$S^-$为:

\begin{equation}
S^+ = \hbar  \left| + \right\rangle \left\langle - \right| \qquad
S^- = \hbar  \left| - \right\rangle \left\langle + \right|~
\end{equation}

$S^+$的作用是使$\left| - \right\rangle$ 变成 $\left| + \right\rangle$,

\begin{equation}
S^+ \left| - \right\rangle = \hbar  \left| + \right\rangle \left\langle -  | -   \right\rangle = \hbar \left| + \right\rangle~
\end{equation}

$S^-$的作用是使$\left| + \right\rangle$ 变成 $\left| - \right\rangle$,

\begin{equation}
S^- \left| + \right\rangle = \hbar  \left| - \right\rangle \left\langle +  | +   \right\rangle = \hbar \left| - \right\rangle~
\end{equation}

升降算符是极形象的命名。

我们可以把“升降算符”表示为一个$2 \times 2$的方阵,它使得一个“$2 \times 2$的方阵”乘以一个“二维列向量”等于一个“二维列向量”,这就是算符的“矩阵表示”。

比如$S^+$的矩阵表示是:

\begin{equation}
S^+ \dot = \left(  \begin{array}{ccc}  S^+_{++} &  S^+_{+-} \\  S^+_{-+} & S^+_{--}  \end{array} \right)~
\end{equation}

这里的矩阵元分别是:

\begin{equation}
S^+_{++} = \left\langle + \right| S^+ \left| + \right\rangle = \hbar \left\langle + | +  \right\rangle  \left\langle  - | + \right\rangle = 0 \qquad
S^+_{+-} = \left\langle + \right| S^+ \left| - \right\rangle = \hbar \left\langle + | + \right\rangle  \left\langle - | - \right\rangle = 1 \qquad
S^+_{-+} = \left\langle - \right| S^+ \left| + \right\rangle = \hbar \left\langle - | + \right\rangle \left\langle - | + \right\rangle = 0 \qquad
S^+_{--} = \left\langle - \right| S^+ \left| - \right\rangle = \hbar \left\langle - | + \right\rangle \left\langle - | - \right\rangle = 0~
\end{equation}

这里我们利用了:

\begin{equation}
\left\langle + | + \right\rangle  =  \left\langle - | - \right\rangle = 1 \qquad
\left\langle + | - \right\rangle  =  \left\langle - | + \right\rangle = 0~
\end{equation}

因此,自旋1/2的升算符$S^+$可表示为:

\begin{equation}
S^+ \dot = \hbar \left(  \begin{array}{ccc}  0 &  1 \\  0 & 0  \end{array} \right)~
\end{equation}

类似地,可求出降算符$S^-$的矩阵表示(或直接利用性质:$\left( S^+ \right)^\dagger = S^-$):

\begin{equation}
S^- \dot = \hbar \left(  \begin{array}{ccc}  0 &  0 \\  1 & 0  \end{array} \right)~
\end{equation}

最后,我们可以验证:

\begin{equation}
S^{\pm} = S_x \pm i S_y~
\end{equation}

\subsection{角动量的代数解法}

我们现在来求解$(L^2, L_z)$的共同本征值问题,通常来说这需要我们求解一个关于极角$\theta$和方位角$\phi$的偏微分方程,这是极啰嗦的,但我们也可由对易式出发,仅凭代数关系就把$(L^2, L_z)$的共同本征值问题求出来,而且我们发现在这个代数解法中,$l = 1/2$这样的解不再是不允许的了\footnote{由于代数解法中根本就不出现$\theta$和$\phi$,所以$l= 1/2$这样的解就是允许的。},这实际上给出了一个关于轨道角动量(量子数取整数)和自旋角动量(量子数取1/2)的共同描述。

假设$(L^2, L_z)$的共同本征态是$\left| \lambda, \mu  \right\rangle$,

\begin{equation}
L^2 \left| \lambda, \mu \right\rangle = \lambda \hbar^2 \left| \lambda, \mu  \right\rangle \qquad
L_z \left| \lambda, \mu  \right\rangle = \mu \hbar \left| \lambda, \mu  \right\rangle~
\end{equation}


首先我们模仿自旋1/2的情形定义升降算符$L^{\pm}$:

\begin{equation}
L^+ = L_x + i L_y \qquad
L^- = L_x - i L_y ~
\end{equation}

我们可以证明对易式:

\begin{equation}
[ L_z, L^+ ] = [L_z, L_x + i L_y] = L_z L^+ - L^+ L_z = \hbar L^+~
\end{equation}

利用:

\begin{equation}
( L_z L^+ - L^+ L_z ) \left| \lambda, \mu  \right\rangle = \hbar L^+ \left| \lambda, \mu \right\rangle~
\end{equation}

考虑到:$L^+ L_z \left| \lambda, \mu \right\rangle = \mu \hbar L^+ \left| \lambda, \mu \right\rangle $,

\begin{equation}
L_z L^+ \left| \lambda, \mu \right\rangle = (\mu + 1) \hbar L^+ \left| \lambda, \mu \right\rangle~
\end{equation}

这里我们把$L^+ \left| \lambda, \mu \right\rangle$看成是一个新的态矢量,它也是$L_z$的本征矢,本征值为$( \mu + 1 ) \hbar$,因此$L^+ \left| \lambda, \mu \right\rangle$可表示为:

\begin{equation}
L^+ \left| \lambda, \mu \right\rangle = C \left| \lambda, \mu+1 \right\rangle~
\end{equation}

这里$C$是待定的复数因子。

即$L^+$可以使量子数$\mu+1$,我们形象地称之为升算符。类似地,我们可证明$L^- \left| \lambda, \mu \right\rangle $也是$L_z$的本征矢,对应本征值为$(\mu-1) \hbar$。

因此,$L^- \left| \lambda, \mu \right\rangle$可表示为:

\begin{equation}
L^- \left| \lambda, \mu \right\rangle = C' \left| \lambda, \mu-1 \right\rangle~
\end{equation}

这里$C'$是待定的复数因子。

即$L^-$可以使量子数$\mu-1$,我们形象地称之为降算符。

现在的问题是这种“升、降”操作是否可无限制地进行下去?

如果可以的话,就意味着$L_z$的本征值的大小就没有限制了,而这是不可能的。因为$L_z$仅仅是$L = i L_x + j L_y + k L_z$算符的一个分量,而我们假设$L^2 = L_x^2 + L_y^2 + L_z^2$的本征值是$\lambda \hbar^2$,这样“升、降”操作就不能无限制地进行下去。

通俗地说就是我们假设角动量量子数是$l$,那么$\mu$的取值最大将只是$l$,再大是不可能的。

我们把这句话翻译成数学的语言,对态$\left| \lambda, l \right\rangle$,必须有:

\begin{equation}
L^+ \left| \lambda, l \right\rangle = 0~
\end{equation}

对零矢量前面再乘一个算符$L^-$并不会改变计算结果,即:

\begin{equation}
L^- L^+ \left| \lambda, l \right\rangle = 0 ~
\end{equation}

我们把$L^- L^+$展开:

\begin{equation}
L^- L^+ = ( L_x - i L_y ) (L_x + i L_y) = L_x^2 + L_y^2 + i [ L_x, L_y ] 
{} = L^2 - L_z^2 - \hbar L_z~
\end{equation}

即:

\begin{equation}
L^2 \left| \lambda, l \right\rangle = (L_z^2 + \hbar L_z) \left| \lambda, l \right\rangle = l (l + 1)\hbar^2 \left| \lambda, l \right\rangle~
\end{equation}

现在我们重新把$\lambda$写作$l (l +1)$,$\mu$重新写作$m$,$\left|  \lambda, \mu \right\rangle$写作$\left| l, m \right\rangle$,得到:

\begin{equation}
L^2 \left| l,m \right\rangle = l(l+1)\hbar^2 \left| l,m \right\rangle \qquad
L_z \left| l,m \right\rangle = m \hbar \left| l,m \right\rangle~
\end{equation}

这里$m$的取值是从$l$开始逐渐“减1”,“减1”,最后得到对称的$-l$,或者我们由$-l$开始逐渐“加1”,“加1”,最后得到$l$。要得到这种结构,$l$取整数$0, 1, 2 ...$是可以的,取半整数$1/2, 3/2, ...$也是可以的,但取其他情况就不可以了。

小结一下:

\begin{equation}
l = 0, 1/2, 1, 3/2, 2, ... \qquad
m = l, l-1, ..., -l +1 , -l~
\end{equation}

对轨道角动量而言$l$只能取整数,是我们这里讨论的一种特殊情况。

下面我们继续证明一个有用的性质,对升降算符$L^{\pm}$,有:

\begin{equation}
L^{\pm} \left| l,m \right\rangle = \hbar \sqrt{(l \mp m)(l \pm m +1)
} \left|l, m \pm 1 \right\rangle~
\end{equation}

定义: $\left| \gamma \right\rangle = L^+ \left| lm
\right\rangle$, 对应到左矢空间, $\left\langle \gamma \right| =
\left\langle lm  \right| L^-$,

因此:

\begin{equation}
\left\langle \gamma | \gamma \right\rangle = \left\langle lm \right|
L^- L^+ \left| lm \right\rangle~
\end{equation}

上式中的$L^- L^+$展开是:

\begin{equation}
(L_x-iL_y) (L_x + i L_y) = L_x^2 + L_y^2 +i [L_x, L_y] = L^2 - L_z^2
- \hbar L_z~
\end{equation}

因此:

\begin{equation}
\left\langle \gamma | \gamma \right\rangle = \left\langle L^2 -
L_z^2 - \hbar L_z \right\rangle = l(l+1)\hbar^2 - m^2\hbar^2 -m
\hbar^2~
\end{equation}

上式化简可得:

\begin{equation}
\left\langle \gamma | \gamma \right\rangle =  (l - m)(l+m+1)\hbar^2~
\end{equation}

即:

\begin{equation}
L^+  \left| l, m \right\rangle = \hbar \sqrt{(l - m)(l + m +1) } \left|l, m + 1 \right\rangle~
\end{equation}

类似地,我们还可对降算符$L^-$证明:

\begin{equation}
L^-  \left| l, m \right\rangle = \hbar \sqrt{(l + m)(l - m +1) } \left|l, m - 1 \right\rangle~
\end{equation}

\subsection{自旋单态和自旋三重态}

假设有两个自旋,自旋甲和自旋乙,它们是独立的,所谓独立就是说甲的态不会影响乙,反过来乙的态也不会影响甲。但我们现在考虑的系统是“甲和乙”。对这样“甲和乙”的自旋的态我们应如何描述。

但考虑自旋甲的话,自旋甲可以是随便一个态$\left| \alpha \right\rangle$,如果我们使用$S_z$表象的话,我们取自旋甲在$z$方向向上和向下的态$\{ \left| + \right\rangle, \left| - \right\rangle,   \}$为基矢。如此我们“张”成一个二维的线性空间,所谓“张”是个很形象的动作,好比我们在雨季撑开一把伞,$\left| + \right\rangle$和$\left| - \right\rangle$是两个互相正交归一的态,即:

\begin{equation}
\left\langle + | + \right\rangle = \left\langle - | - \right\rangle  = 1 \qquad
\left\langle + | - \right\rangle = \left\langle - | + \right\rangle = 0~
\end{equation}

这两个互相正交归一的态矢量“像”笛卡尔坐标系中的$\vec e_x$和$\vec e_y$一样成为表示任意矢量的“一套单位向量。”

但我们要注意以上所有陈述都是针对自旋甲而言的,我们应该在记号里予以明确的说明或规定,比如我们在右下角标个指标“1”,像这样:

\begin{equation}
\left| \alpha \right\rangle_1 = \left( \begin{array}{ccc} a \\ b \end{array} \right)_1 = a \left| + \right\rangle_1 + b \left| - \right\rangle_1~
\end{equation}

我们还有自旋乙,用完全类似的方案把它表示出来。单考虑自旋乙,任意态$\left| \beta \right\rangle_2$——这里用指标“2”表示自旋乙——可表示为:

\begin{equation}
\left| \beta \right\rangle_2 = \left( \begin{array}{ccc} c \\ d \end{array} \right)_2 = c \left| + \right\rangle_2 + d \left| - \right\rangle_2~
\end{equation}

“甲和乙”一起考虑,我们首先要找到描述“甲和乙”的基矢,既然甲和乙是独立的,一种自然的选择是:

\begin{equation}
\left| ++ \right\rangle  =  \left| + \right\rangle_1 \left| + \right\rangle_2\qquad
\left| +- \right\rangle  =  \left| + \right\rangle_1 \left| - \right\rangle_2\qquad
\left| -+ \right\rangle  =  \left| - \right\rangle_1 \left| + \right\rangle_2\qquad
\left| -- \right\rangle  =  \left| - \right\rangle_1 \left| - \right\rangle_2 ~
\end{equation}

这里的规则是把甲、乙两个自旋的态并列,我们规定左边的那个是甲,右边的那个是乙。总共得到四个基矢$\{ \left| ++ \right\rangle, \left| +- \right\rangle, \left| -+ \right\rangle, \left| - - \right\rangle \}$,我们可以验证这四个基矢也是两两正交归一的,比如:

\begin{equation}
\left\langle ++ | ++ \right\rangle = \left\langle -- | -- \right\rangle = 1 \qquad
\left\langle ++ | -- \right\rangle = \left\langle ++ | +- \right\rangle = 0 \qquad
{}  ...  {}~
\end{equation}

这样我们就得到了直接乘积(direct product)表示,即对“甲和乙”这个系统里的任意态矢量$\left| \alpha \right\rangle$,我们可以把它表示为:

\begin{equation}
\left| \alpha \right\rangle = \left( \left| {++} \right\rangle \left\langle {++}  \right|  + \left| {+-} \right\rangle \left\langle {+-}  \right| + \left| {-+} \right\rangle \left\langle {-+} \right| + \left| {--} \right\rangle \left\langle {--}  \right|  \right) \left| \alpha \right\rangle
{} = a \left| {++} \right\rangle + b \left| {+-} \right\rangle + c \left| {-+} \right\rangle + d \left| {--} \right\rangle~
\end{equation}

这里$I = \left( \left| {++} \right\rangle \left\langle {++}  \right|  + \left| {+-} \right\rangle \left\langle {+-}  \right| + \left| {-+} \right\rangle \left\langle {-+}  \right| + \left| {--} \right\rangle \left\langle {--}  \right|  \right)$是单位算符,它的意思是向所有方向做投影。$a, b, c, d$是投影(或几率幅):

\begin{align}
a &= \langle ++|\alpha \rangle \\\\
b &= \langle +-| \alpha \rangle \\\\
c &= \langle -+|\alpha \rangle \\\\
d &= \langle --|\alpha \rangle~
\end{align}

态矢量 $|\alpha \rangle$ 所在的空间叫直接乘积空间 (direct product space) $2^5$。

这样的描述有什么“物理意义”吗?或者我们可以把什么样的物理量和 ${ \lvert ++ \rangle , \lvert + - \rangle , \lvert - + \rangle , \lvert -- \rangle }$ 相对应?

对第一个态 $\lvert ++ \rangle$ 我们可以立刻猜测它描述的是总自旋为 1 和总自旋的 $z$ 分量为 1 的态。类似地,对第四个态 $\lvert -- \rangle$ 我们猜测它描述的是总自旋为 -1 和总自旋的 $z$ 分量为 1 的态。

当然,我们需要先定义总自旋。

我们现在来描述什么是直积空间 $C = A \otimes B$:

$A$ 是一个希尔伯特空间,$A = {a_1, a_2, a_3, \ldots, a_i, \ldots}$,$a_i$ 是 $A$ 中的任一元素;同样 $B$ 是另一个希尔伯特空间,$B = \{b_1, b_2, b_3, \ldots, b_j, \ldots\}$,$b_j$ 是 $B$ 中的任一元素,我们把 $a_i, b_j$ 并列地放在一起,$a_i b_j$ 构成了希尔伯特空间 $C$ 中的任一元素,$C = {a_1 b_1, a_1 b_2, a_1 b_3, \ldots, a_i b_j, \ldots}$。

容易证明:

1. 如果 $A$ 中有 $n$ 个元素,$B$ 中有 $m$ 个元素,则 $C$ 中有 $m \times n$ 个元素。

2. 如果 $A$ 是一个 $m$ 维的希尔伯特空间,基矢是 ${a_i}$,$i=1,2,\ldots,m$,$B$ 是 $n$ 维的希尔伯特空间,基矢是 ${b_j}$,$j=1,2,\ldots,n$,则 $C$ 是 $m \times n$ 维的希尔伯特空间,基矢是 ${a_i b_j}$,$i=1,2,\ldots,m; j=1,2,\ldots,n$。

