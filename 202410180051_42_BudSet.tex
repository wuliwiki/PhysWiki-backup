% 有界集
% keys 有界集
% license Usr
% type Tutor

\pentry{拓扑向量空间\nref{nod_tvs}}{nod_84c5}
有界集的概念在拓扑线性空间起着重要的作用,它不仅是能够用来判定一个集合、空间是否有界,还是定义有界算子的基础。

\begin{definition}{有界集}
在\enref{拓扑线性空间}{tvs}中的集合 $M$ 称为\textbf{有界的}(bounded),是指对每一个零邻域(零矢量的邻域)$U$,存在 $n>0$,使得对所有 $\abs{\lambda}\geq n,M\subset \lambda U$。 
\end{definition}
有界集的概念和我们在赋范空间中按范数有界(即可把集置于某一球 $\norm{x}\leq R$ 内部)的理解是一致的。

\begin{definition}{局部有界}
拓扑线性空间 $E$ 称为\textbf{局部有界}的,若 $E$ 中至少存在一个非空有界开集。
\end{definition}

对于有界性有下面的定理成立。
\begin{theorem}{}
设 $E$ 是拓扑线性空间,则成立:
\begin{enumerate}
\item 集 $M\subset E$ 是有界的,当且将当对任何序列 $\{x_n\}\subset M$ 及任何趋于零的正数列 $\{\epsilon_n\}$,序列 $\epsilon_n x_n$ 趋于零;
\item 若 $\{x_n\}_{n=1}^\infty\subset E$ 且 $x_n\rightarrow x$,则 $\{x_n\}$ 是有界集;
\item 如果 $E$ 局部有界,则在 $E$ 中第一可数性公理成立。 
\end{enumerate}

\end{theorem}

\textbf{证明:}1. 要证 $\epsilon_n x_n\rightarrow0$,就是要证对任一零邻域 $U$,存在正整数 $N$,使得 $n\geq N$,就有 $\epsilon_n x_n\subset U$。其证明如下:

由数乘的连续性,,任一零邻域 $U$,存在 $m>0$,只要 $\abs{\lambda}\geq m$,就有 $M\subset\lambda U$。由拓扑线性空间数乘的连续性,存在 $\delta>0$,和 $u_$




\textbf{证毕!}











