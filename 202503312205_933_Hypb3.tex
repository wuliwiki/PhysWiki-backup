% 双曲线(高中)
% keys 极坐标系|直角坐标系|圆锥曲线|双曲线|渐近线
% license Xiao
% type Tutor

\begin{issues}
\issueDraft
\end{issues}

\pentry{解析几何\nref{nod_JXJH},圆\nref{nod_HsCirc},椭圆\nref{nod_Elips3},复数\nref{nod_CplxNo}}{nod_fd80}

在发电厂里,有一种巨大的冷却装置,高高耸立,外形像一个中间收紧、两端张开的沙漏。这种特殊的形状不是为了美观,而是工程设计上的巧思。它能有效抵抗强风,减少材料使用,还能提高散热效率。建筑师们发现,如果把某种曲线绕着轴旋转,就能得到这种优雅又实用的形状。这种曲线在数学中早已有详细的研究,它由两条彼此对称的互相分离的分支组成,尽管共用一个坐标系的中心吗,各自却延伸向无穷远处,永不相交,他就被被称为双曲线。

相信,听到双曲线这个名字,读者一定会想起第一次听到双曲线这个词,最有可能的就是在初中学习反比例函数的时候了。相信大多数人在学习时都听过这句“反比例函数的图像是双曲线。”至于什么是双曲线?在当时朴素的概念里,应该就是说它是“两根”“曲线”,至于再细的,老师不会说,学生自然也不会问。

既然反比例函数就是双曲线,那么相信读者也一定记得,作为双曲线非常重要的一个特性,渐近线在高中阶段很少被提及,双曲线永远靠近渐近线,但永不触碰,就像永恒的“靠近而不达”。本文也会对此进行一些分析,以扩充视野。

举个例子:你站在一个房间的两个扬声器之间,发现无论你怎么移动,总能找到一个位置,那里两个扬声器传来的声音大小之差是固定的。连起这些点,就是一条双曲线。

\subsection{“反比例函数是双曲线”}

如前面所说,下面就对照前面推导的结果,来仔细研究一下反比例函数是双曲线这件事。

\begin{example}{求反比例函数$\displaystyle y={1\over x}$的图像逆时针旋转$45^\circ$后的表达式。}
根据\aref{图像旋转的规律}{sub_FunTra_3},将$\displaystyle y={1\over x}$逆时针旋转$45^\circ$,即$\displaystyle\theta=-{\pi\over4}$。
\begin{equation}\label{eq_Hypb3_2}
\begin{cases}
\displaystyle
X_0&=X_1 \cos \left(-{\displaystyle\frac{\pi}{4}}\right) + Y_1 \sin  \left(-{\displaystyle\frac{\pi}{4}}\right)\\
Y_0&=Y_1 \cos  \left(-{\displaystyle\frac{\pi}{4}}\right) - X_1 \sin  \left(-{\displaystyle\frac{\pi}{4}}\right)\\
\end{cases}\implies
\begin{cases}
X_0&={\displaystyle\frac{1}{\sqrt{2}}}(X_1 - Y_1)\\
Y_0&={\displaystyle\frac{1}{\sqrt{2}}}(Y_1+ X_1) \\
\end{cases}~.
\end{equation}
将\autoref{eq_Hypb3_2} 代入$xy=1$有:
\begin{equation}
\displaystyle
{1\over\sqrt{2}}(X_1 - Y_1)\cdot{1\over\sqrt{2}}(Y_1+ X_1)=1\implies {X_1^2\over2}- {Y_1^2\over2}=1~.
\end{equation}
即旋转后的方程是:
\begin{equation}
{x^2\over2}- {y^2\over2}=1~.
\end{equation}
\end{example}

这个表达式的形式看着很眼熟,是的,他和半径为$\sqrt{2}$圆相比只差了一个$y^2$前面的符号。这不禁让人联想起椭圆来,椭圆和标准的圆也只是相差了系数。可双曲线和圆、椭圆,外形看上去相差了那么多,hui you shen e m
由于旋转变换不会改变形状,因此,原本曾经说过反比例函数有两条渐近线是$x,y$轴,经过旋转之后会变成$y=\pm x$。也就是说旋转后的图形,尽管向两方延伸,却不会超过$y=\pm x$的范围。

他的图像是中心位于原点,顶点分别在 $x = \pm \sqrt{2}$ 处,渐近线为$y = \pm x$的离心率为$\sqrt{2}$的双曲线。因此原本的反比例函数$\displaystyle y={1\over x}$的图像是双曲线,且渐近线为$x$轴和$y$轴。至此,初中阶段就接触到的“反比例函数图像是双曲线”得到了证明。
\subsection{双曲线的几何定义}

在研究椭圆时,曾经提过,如果从圆的定义出发,探索其推广的可能性,就要修改定义的表达方式。如果想要修改$|O_1P|=|O_2P|=r$,打开第二个等号是垂直平分线,打开第一个等号,将$|O_1P|=r,|O_2P|=r$作为一个初始条件,而不要求二者一直相等。如果要求$|O_1P|+|O_2P|=2r$,也就是说这两个距离始终保持他们的和是一个定值$m$,则得到了椭圆。那么,如果认为$|O_1P|-|O_2P|=0$呢?似乎也是一个比较容易的做法。但如果一直是$0$那相当于并没有打开第一个等号,如果像椭圆那样修改,认为这两个距离始终保持他们的差是一个定值$m$,那么将会得到什么呢?


\begin{example}{对两定点 $F_1(-c, 0)$ 和 $F_2(c, 0),(c>0)$,若点$P$满足$|PF_1| - |PF_2| = m,(0<m <2c)$,求$P$方程。}
解:

设椭圆上的任意点为 $P(x, y)$,根据题意有:
\begin{equation}
\sqrt{(x + c)^2 + y^2} - \sqrt{(x - c)^2 + y^2} = m~.
\end{equation}
移项后,两边平方有:
\begin{equation}
(x + c)^2 + y^2 = m^2 + 2m\sqrt{(x - c)^2 + y^2} + (x - c)^2 + y^2~.
\end{equation}
打开整理有:
\begin{equation}
2m\sqrt{(x - c)^2 + y^2}= 4cx - m^2~.
\end{equation}
两边平方,打开有:
\begin{equation}
4m^2(x^2 - 2cx+c^2) + 4m^2y^2= m^4-4m^2\cdot2cx+16c^2x^2~.
\end{equation}
整理后得到:
\begin{equation}
4(m^2 -4c^2)x^2 + 4m^2y^2= m^2(m^2-4c^2)~.
\end{equation}
两侧同时除以$(m^2-4c^2)m^2$后得到:
\begin{equation}\label{eq_Hypb3_3}
\frac{x^2}{\left(\displaystyle\frac{m}{2}\right)^2} -\frac{y^2}{\displaystyle c^2-\left(\frac{m}{2}\right)^2}=1~.
\end{equation}
\end{example}

由于$2c>m>0$,也就是$\displaystyle c^2-\left(\frac{m}{2}\right)^2>0$。这里有几点需要注意的:
\begin{itemize}
\item 尽管题目选择的不变量由椭圆的和变成了差,但对比\aref{椭圆的推导过程}{ex_Elips3_1}可以看出,推导的过程与结果其实完全相同,唯一的区别是从原本的$m>2c$,变成了$m<2c$,结果为了保证分母为正,使得\autoref{eq_Hypb3_3} 与\autoref{eq_Elips3_4} 形式上产生了差异。
\item 推导的条件中隐含了一个条件就是$|PF_1| >|PF_2|$,但更换关系之后,得到的结果与这里给出的结果也一样。或者说例题中的准确结果应该添加一个限定范围,即$x>0$,将其画出之后,恰巧是在右侧的一半。而当把条件改为$||PF_1| - |PF_2|| = m,(0<m <2c)$时,才完整对应了\autoref{eq_Hypb3_3} 的表达式,此时两侧都有图像。
\item 易知,如果两个定点分别为 $F_1(0,-c)$ 和 $F_2(0,c),(c>0)$,则相当于更换$x,y$的结果位置,得到的表达式是:
\begin{equation}
\frac{y^2}{\left(\displaystyle\frac{m}{2}\right)^2}-\frac{x^2}{\displaystyle\left(\frac{m}{2}\right)^2-c^2} =1~.
\end{equation}
与椭圆不同的是,这里调换的分母同时还带着符号。
\item 对于\autoref{eq_Hypb3_3} ,带入$y=0$时,求得的$x$轴交点为$\displaystyle x=\pm\frac{m}{2}$。带入$x=0$时,在实数范围内无解,但在复数域有解,解为$\displaystyle y=\pm\I\sqrt{c^2-\left(\frac{m}{2}\right)^2}$。这一点是与椭圆非常不同的。标准椭圆与坐标轴有四个交点,而这条曲线则只有两个交点。
\end{itemize}



就称其为\textbf{双曲线(ellipse)}。



大多数情况都会认为双曲线是开放的或者说不封闭的,不过转换一下视角,想象一下,当视角来到无穷远处,如果把它们两支相互对称的地方连起来,似乎就形成了一个封闭的环,只不过这个环太大了。不知道你看到双曲线时是否会有这样的想象,他们要是能连起来就太好了。其实,在射影几何的视角下,它们正是在无穷远处相交。而这也带来了非常重要的视角。
 

\subsection{双曲线的方程}

与椭圆不同的是,双曲线的实轴与虚轴并不因为$a,b$值的大小而影响,而是完全取决于符号。
\begin{theorem}{双曲线的标准方程}
\begin{itemize}
\item 实轴在$x$轴上,虚轴在$y$轴上的双曲线方程为:
\begin{equation}\label{eq_Hypb3_4}
\frac{x^2}{a^2} - \frac{y^2}{b^2} = 1~.
\end{equation}
\item 实轴在$y$轴上,虚轴在$x$轴上的双曲线方程为:
\begin{equation}
\frac{y^2}{b^2} -\frac{x^2}{a^2}  = 1~.
\end{equation}
\end{itemize}
\end{theorem}

\subsection{渐近线}
\begin{figure}[ht]
\centering
\includegraphics[width=4.8cm]{./figures/eb70b650d9fa932b.pdf}
\caption{双曲线的渐近线} \label{fig_Hypb3_1}
\end{figure}

当 $x,y$ 都无穷大时, \autoref{eq_Hypb3_4} 中的 $1$ 可以忽略不计,有 $y/x = \pm b/a$,\enref{渐近线}{Asmpto}与 $x$ 轴夹角为
\begin{equation}\label{eq_Hypb3_1}
\theta_0 = \arctan(b/a)~.
\end{equation}
两条渐近线到两个焦点的距离都为
\begin{equation}\label{eq_Hypb3_11}
c\sin\theta_0 = c\cdot b/c = b~.
\end{equation}


事实上这么推导渐近线并不严谨, 在学习了高数的相关内容(见“\enref{泰勒展开}{Taylor}”)后,由\autoref{eq_Hypb3_4} 得
\begin{equation}
y = \frac{bx}{a} \sqrt{1-\frac{a^2}{x^2}}~.
\end{equation}
把根号部分关于 $a^2/x^2$ 进行泰勒展开, 有
\begin{equation}\label{eq_Hypb3_13}
y = \frac ba x - \frac{ab}{2x} + \order{\frac{1}{x^3}}~.
\end{equation}
所以当 $x\to\infty$ 时, 就有渐进线 $y = bx/a$。 之所以要这样做, 是为了防止\autoref{eq_Hypb3_13} 右边出现常数项。 如果存在常数项 $\lambda$, 那么双曲线的渐近线就是 $y = bx/a + \lambda$ 了。

\subsection{双曲线的性质}
面积









