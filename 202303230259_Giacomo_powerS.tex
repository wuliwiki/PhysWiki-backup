% 幂级数(简明微积分)
% 幂级数|收敛|发散|收敛半径|比值判别法

\pentry{级数(简明微积分)\upref{SerCal}}

我们把形如
\begin{equation}\label{powerS_eq1}
f(x) = c_0 + c_1 (x - x_0)^1 + \dots = \sum_{n=0}^\infty c_n (x-x_0)^n
\end{equation}
的函数叫做\textbf{幂级数(power series)}。 其中 $(x - x_0)^0$ 始终视为 $1$, 即使取 $x = x_0$。

幂级数是无穷级数的一种, 是一个极限。 如果我们把前有限项的求和记为
\begin{equation}
f_m(x) = \sum_{n=0}^m c_n (x-x_0)^n
\end{equation}
那么幂级数\autoref{powerS_eq1} 就是极限
\begin{equation}
f(x) = \lim_{m\to\infty} f_m(x)
\end{equation}
的简写。对给定的 $x$,当极限存在时,我们就说级数\textbf{收敛(converge)},反之就说级数\textbf{不收敛}或\textbf{发散(diverge)}。幂级数可以定义在复数域上,即 $c_n, x, x_0$ 都可以取复数\upref{CplxNo}值。

\subsection{收敛半径}
一种极端的情况是幂级数\autoref{powerS_eq1} 只在 $x = x_0$ 一点处收敛(例如 $c_n = n^n$)。 除此之外, 必定存在一个\textbf{收敛半径(radius of convergence)} $0 \leq r \leq +\infty$ , 使得当 $\abs{x-x_0} < r$ 时幂级数总是收敛, 当 $\abs{x-x_0} > r$ 时幂级数总是发散(不收敛),当 $\abs{x - x_0} = r$ 时幂级数可能收敛也可能不收敛;当 $r = 0$ 时,幂级数只在 $x_0$ 处收敛,当 $r = + \infty$ ,幂级数在整个 $\mathbb{R}$ (或者 $\mathbb{C}$) 上收敛。 当 $x$ 是复数时, 复平面上收敛的区域就是以 $x_0$ 为圆心的一个圆盘。

计算收敛半径的一种简单方法是使用比值判别法
\begin{equation}
r = \lim_{n \to \infty} \abs{\frac{c_n}{c_{n+1}}}
\end{equation}
但前提是该极限存在, 普适的方法见柯西—阿达玛公式\upref{CHF}。

\subsection{例子}
\begin{example}{}
\begin{itemize}
\item $f(x) = e^x = \sum {1 \over n!} x^n$,收敛半径为正无穷,它定义在整个复平面上;
\item $f(x) = {1 \over 1 - x} = \sum x^n$,收敛半径为 $1$;
\item $f(x) = \sum n^n x^n$,收敛半径为 $0$;
\end{itemize}
\end{example}
% Giacomo:可以再举例一些常见的例子,我懒得举了