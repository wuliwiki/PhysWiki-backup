% 哥德尔不完备定理(综述)
% license CCBYSA3
% type Wiki

本文根据 CC-BY-SA 协议转载翻译自维基百科\href{https://en.wikipedia.org/wiki/G\%C3\%B6del\%27s_incompleteness_theorems}{相关文章}。

哥德尔的不完全性定理是数学逻辑中的两个定理,涉及形式公理化理论中可证明性的极限。这些结果由库尔特·哥德尔在1931年发布,在数学逻辑和数学哲学中都具有重要意义。这些定理被广泛地,但并非普遍地解释为,证明了希尔伯特寻找一个完整且一致的公理集合来描述所有数学的计划是不可能实现的。

第一个不完全性定理声明,任何一个一致的公理系统,只要其定理可以通过有效程序(即算法)列出,都无法证明关于自然数算术的所有真理。对于任何这样的形式系统,总会存在一些关于自然数的陈述,这些陈述是正确的,但在该系统内无法证明。

第二个不完全性定理,是第一个定理的扩展,表明该系统无法证明自身的一致性。

通过使用对角线论证,哥德尔的不完全性定理是首批关于形式系统局限性的紧密相关定理之一。随后,塔尔斯基提出了真理的形式不可定义性定理,丘奇证明了希尔伯特的判定问题是不可解的,图灵的定理表明不存在可以解决停机问题的算法。
\subsection{形式系统:完整性、一致性和有效公理化} 
不完全性定理适用于那些足够复杂的形式系统,这些系统能够表达自然数的基本算术,并且是一致的且具有有效的公理化。特别是在一阶逻辑的背景下,形式系统也被称为形式理论。一般来说,形式系统是一个推理工具,由一组特定的公理以及符号操作规则(或推理规则)组成,这些规则允许从公理推导出新的定理。一个这样的系统的例子是一阶皮亚诺算术系统,这是一个所有变量都指代自然数的系统。在其他系统中,如集合论,只有一些形式系统中的句子表达关于自然数的陈述。不完全性定理涉及的是这些系统内的形式可证明性,而不是“非正式意义上的可证明性”。

形式系统可能具有几个属性,包括完整性、一致性和有效公理化的存在。不完全性定理表明,包含足够算术内容的系统无法同时具备这三种属性。
\subsubsection{有效公理化}  
如果一个形式系统的定理集合是递归可枚举的,则该系统被称为有效公理化(也称为有效生成)。这意味着存在一个计算机程序,理论上可以枚举该系统的所有定理,而不会列出任何非定理的陈述。有效生成的理论的例子包括皮亚诺算术和泽梅洛–弗兰克尔集合论(ZFC)。\(^\text{[1]}\)

被称为真算术的理论包括在皮亚诺算术语言中关于标准整数的所有真陈述。该理论是一致且完整的,并包含足够的算术内容。然而,它没有递归可枚举的公理集合,因此不满足不完全性定理的假设。
\subsubsection{完整性}  
一组公理是(语法上或否定上)完整的,如果对于公理语言中的任何陈述,该陈述或其否定可以从这些公理中证明出来。\(^\text{[2]}\)这是与哥德尔的第一个不完全性定理相关的概念。它不应与语义完整性混淆,语义完整性意味着该公理集合能证明给定语言中的所有语义重言式。在他的完整性定理中(不应与这里描述的不完全性定理混淆),哥德尔证明了一阶逻辑在语义上是完整的。但它不是语法完整的,因为在一阶逻辑的语言中,有些句子既不能从逻辑的公理中证明,也不能反驳。

在一个数学系统中,像希尔伯特这样的思想家认为,找到这样一种公理化方法,使得可以通过证明其否定来证明或反驳每个数学公式,迟早只是时间问题。

一个形式系统可能是有意设计为语法不完整的,正如逻辑通常那样。或者它可能是不完整的,仅仅是因为并没有发现或包含所有必要的公理。例如,没有平行公设的欧几里得几何是不完整的,因为语言中的一些陈述(例如平行公设本身)不能从其余的公理中证明出来。同样,稠密线性序的理论是不完整的,但在添加一个额外的公理,即在序列中没有端点的公理后,它变得完整。连续统假设是ZFC语言中的一个陈述,在ZFC中无法证明,因此ZFC不是完整的。在这种情况下,没有显而易见的候选公理能够解决这个问题。

一阶皮亚诺算术的理论似乎是一致的。假设这确实是正确的,请注意它有一个无限但递归可枚举的公理集合,并且可以编码足够的算术内容以符合不完全性定理的假设。因此,根据第一个不完全性定理,皮亚诺算术不是完整的。该定理给出了一个算术陈述的明确示例,该陈述在皮亚诺算术中既不能被证明,也不能被反驳。此外,这个陈述在通常模型中为真。此外,没有有效公理化的一致扩展皮亚诺算术能够是完整的。
\subsubsection{一致性}  
一组公理是(简单地说)一致的,如果没有任何陈述既能从公理中证明,又能证明其否定,否则该公理集合是不一致的。也就是说,一个一致的公理系统是没有矛盾的。

皮亚诺算术可以从ZFC中证明其一致性,但不能从自身内部证明。类似地,ZFC不能从自身内部证明其一致性,但ZFC + “存在一个不可接近的基数”证明了ZFC是一致的,因为如果 \( \kappa \) 是最小的这样的基数,那么 \( V_\kappa \) 位于冯·诺伊曼宇宙内部是ZFC的一个模型,而一个理论是一致的,当且仅当它有一个模型。

如果将皮亚诺算术语言中的所有陈述都作为公理,那么这个理论是完整的,具有递归可枚举的公理集合,并且可以描述加法和乘法。然而,它不是一致的。

不一致理论的其他例子来自于集合论中,当假设无限制的理解公理 schema 时所产生的悖论。