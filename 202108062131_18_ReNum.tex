% 实数

\subsection{从有理数到实数}

我们知道, 有理数集$\mathbb{Q}$是对四则运算封闭的最小的数系. 从正整数开始, 为了使得任意两个整数都能相减, 我们引入了零和负整数, 从而得到了整数集$\mathbb{Z}$; 而为了使得任意两个整数的除法都有意义 (当然, 要剔去除数为零的情形), 我们又引入了形如$m/n$的有理数, 从而得到了有理数集$\mathbb{Q}$. 小学算术已经告诉我们, 有理数的和, 差, 积, 商都是有理数(仍然需要假定除数不等于零), 而且对于任何有理数$r$都有$r+0=r$, $r\cdot1=r$. 用近代代数学的语言, 这表示有理数集构成了一个\textbf{域 (field)}.

但是我们也知道, 并非所有来自实际问题的度量对象都能用有理数来表示. 例如, 假若承认勾股定理 (在古希腊, 发现并证明它的是毕达哥拉斯), 那么直角边长为1的等腰直角三角形的斜边长$c$满足$c^2=2$. 毕达哥拉斯的门徒西帕索斯发现, 这个奇特的shu