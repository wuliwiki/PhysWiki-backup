% LorenzAttractor
\subsection{洛伦兹方程}

20世纪60年代,蓬勃发展的计算机技术开始得到广泛应用,其中包括长期天气预报. 大气与液体同属流体,太阳照射使地面升温, 靠近地面的气体受到加热, 而高层大气还是冷的, 于是上、下层气体之间将会出现对流, 产生类似于贝纳德实验中的对流现象. 在美国气象局工作的数学家洛伦兹(E.N. Lorenz)将大气对流与贝纳德液体对流联系起来,企图用数值方法进行长期天气预报. 从贝纳德对流出发, 利用流体力学中的纳维叶-斯托克斯(Navier-Stokes)方程、热传导方程和连续性方程,洛伦兹推导出了描述大气对流的微分方程
\begin{align}
&\frac{dx}{dt}=-\sigma (x-y),\\
&\frac{dy}{dt}=rx-y-xz,\\
&\frac{dz}{dt}=-bz+xy,
\end{align}
式子中,$r$是对流的翻动速率; $y$ 正比于上流与下流液体之间的温差;$z$是垂直方向的温度梯度; $\sigma$为无量纲因子,称为Prandtl数; $b$ 为反映速度阻尼的常数. 其中,$xz$与$xy$是非线性项.
