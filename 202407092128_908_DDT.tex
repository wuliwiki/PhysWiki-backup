% 电导体
% license CCBYSA3
% type Wiki

(本文根据 CC-BY-SA 协议转载自原搜狗科学百科对英文维基百科的翻译)

在物理和电气工程中,电导体是一种允许电荷(电流)在一个或多个方向流动的物体或材料。金属材料是常见的电导体。在某些情况下,电流是由带负电的电子流、带正电的空穴和正负离子产生的。

为了使电流流动,一个带电粒子不必从产生电流的机器运动到消耗电流的机器。相反,带电粒子只需以一个有限的推理轻推它相邻的粒子,相邻粒子间就会不停地轻推,直到一个粒子被轻推进入消费者体内,从而给机器供电。本质上,正在发生的是移动电荷载流子之间的一长串动量转移;德鲁伊传导模型更严格地描述了这个过程。这种动量转移模型使金属成为导体的理想选择,因为金属的特点是拥有离域的电子海洋,这使电子有足够的移动性碰撞,从而实现动量转移。

如上所述,电子是金属的主要动力;然而,诸如电池的阳离子电解质、或燃料电池质子导体的可移动质子的其他装置,都依赖于正电荷载体。绝缘体是不导电的材料,只有很少的移动电荷,只支持微小的电流。

\subsection{ 电阻和电导}
\begin{figure}[ht]
\centering
\includegraphics[width=6cm]{./figures/942e5c1d6c90c017.png}
\caption{两端都有电触点的一块电阻材料。} \label{fig_DDT_1}
\end{figure}
给定导体的电阻取决于它的材料和尺寸。对于给定的材料,电阻与横截面积成反比[1]。 例如,粗铜线的电阻比相同的细铜线低。此外,对于给定的材料,电阻与长度成正比;例如,长铜线比其他相同的短铜线具有更高的电阻。因此,等截面导体的电阻R和电导G可以计算为[1]
\begin{equation}
\begin{aligned}
R &= \rho \frac{\ell}{A}, \\\\
G &= \sigma \frac{A}{\ell}~.
\end{aligned}
\end{equation}
其中$\ell$是导体的长度,以米(m)为单位,A是导体的横截面积,以平方米($m^2$)为单位,$\sigma$是以西门子每米($s.m^{-1}$)为单位测量的电导率,$\rho$是材料的电阻率(也称为电阻率),以欧姆米($\omega.m$)为单位测量。电阻率和电导率是比例常数,因此仅取决于导线的材料,而不是导线的几何形状。电阻率和电导率是倒数:$\rho = \frac{1}{\sigma}$。电阻率是材料抵抗电流能力的量度。

这个公式并不精确:它假设导体中的电流密度是完全均匀的,这在实际情况中并不总是正确的。然而,这个公式仍然为长细导线(如电线)提供了一个很好的近似值。

这个公式不精确的另一种情况是交流电(AC),因为趋肤效应会抑制导体中心附近的电流流动。然后,几何横截面不同于电流实际流动的有效横截面,因此电阻高于预期。类似地,如果两个导体携带交流电流彼此靠近,它们的电阻会由于邻近效应而增加。在商用电源频率下,这些效应对于承载大电流的大导体很重要,例如变电站的母线[2] 或承载数百安培以上电流的大电力电缆。

除了导线的几何形状之外,温度对导体的效能也有显著影响。温度在两个主要方面影响导体,第一是材料在受热时会膨胀。材料膨胀的量由特定材料的热膨胀系数决定。这种膨胀(或收缩)将改变导体的几何形状,从而改变其特征电阻。然而,这种影响通常很小,约为10-6。温度的升高也会增加材料中产生的声子数量。声子本质上是晶格振动,或者更确切地说,是材料原子的一个小的谐波运动。很像弹球机的震动,声子用来扰乱电子的路径,导致它们散射。这种电子散射将减少电子碰撞的次数,因此将减少传输的电流总量。

\subsection{导体材料}
\begin{figure}[ht]
\centering
\includegraphics[width=14.25cm]{./figures/79fdcb3fb2913de0.png}
\caption\label{fig_DDT_2}
\end{figure}

导电材料包括金属、电解质、超导体、半导体、等离子体和一些非金属导体,如石墨和导电聚合物。

铜具有高导电性。退火铜是所有其他电导体比较的国际标准;国际退火铜标准电导率为58 MS/m,尽管超纯铜可能略高于101\% IACS。电气应用(如建筑电线、电机绕组、电缆和母线)所用的铜的主要等级是耐电解沥青(ETP)铜(CW004A或美国材料试验学会(ASTM)指定C100140)。如果必须焊接或铜焊或在还原气氛中使用高导电率铜,则可以使用无氧高导电率铜(CW008A或ASTM指定C10100)[3]。 因为铜易于焊接或夹紧连接,所以铜仍然是大多数轻型电线最常见的选择。

银的导电率比铜高6\%,但是由于成本的原因,在大多数情况下银是不实用的。然而,它被用于特殊设备,如卫星,并作为薄镀层,以减轻高频趋肤效应损失。众所周知,第二次世界大战期间,由于战时铜的短缺,从财政部借来的14,700短吨(13,300吨)银被用于制造卡鲁顿磁铁。

铝线是电力输配中最常见的金属。虽然铜的导电率只有横截面积的61\%,但其密度较低,导电率是其质量的两倍。由于铝的重量约为铜的三分之一,因此当需要大导体时,其经济优势相当可观。

铝布线的缺点在于其机械和化学性能。它很容易形成绝缘氧化物,使连接变热。它的热膨胀系数比用于连接器的黄铜材料大,导致连接松动。铝也可能“蠕变”,在负载下缓慢变形,这也会导致连接断开。这些影响可以通过适当设计的连接器和安装中的额外小心来减轻,但是它们使得铝建筑布线在服务中断后不受欢迎。

有机化合物,如辛烷,有8个碳原子和18个氢原子,不能导电。石油是碳氢化合物,因为碳具有四椭圆的性质,并与其他元素如氢形成共价键,它不会失去或获得电子,因此不会形成离子。共价键只是电子的共享。因此,当电流通过时,离子不会分离。所以液体(石油或任何有机化合物)不能导电。

虽然纯水不是电导体,但即使是一小部分离子杂质,如盐,也能迅速将其转化为导体。