% 奇异值分解(SVD)
% keys SVD|singular value decomposition|奇异值分解|正交矩阵|正交变换|线性代数|对角矩阵
% license Xiao
% type Tutor

\begin{issues}
\issueTODO 引用酉矩阵对角化定理。
\end{issues}



本节默认在\textbf{复数域及酉空间}下讨论。
\subsection{定理陈述}
\begin{theorem}{SVD}
给定复数域$\mathbb{C}$上的$n$行$m$列矩阵$\bvec{M}$,则存在$n$行$m$列对角矩阵$\bvec{D}$和酉矩阵$\bvec{P}$、$\bvec{Q}$,使得
\begin{equation}
\bvec{M} = \bvec{P^{\dagger}DQ}~. 
\end{equation}
称$\bvec{D}$的对角元为$\bvec{M}$的\textbf{奇异值(singular value)},$\bvec{P^{\dagger}DQ}$为$\bvec{M}$的\textbf{奇异值分解(singular value decomposition)},简称为\textbf{SVD}。
\end{theorem}


\subsection{证明过程}
\addTODO{作图表示该定理}
我们先从\textbf{可逆方阵}证起,然后再过渡到不要求可逆的\textbf{任意复矩阵}。
\begin{theorem}{}\label{the_SVD_1}
对于$n$阶\textbf{可逆方阵}$M$,存在\textbf{酉方阵}$P,Q$和\textbf{可逆对角矩阵}$D$使得
\begin{equation}\label{eq_SVD_1}
M=P^{\dagger}DQ~.
\end{equation}

\end{theorem}
\textbf{证明:}因为$M^{\dagger}M$是半正定厄米矩阵,必存在一组标准正交基使之对角化,设为$A$,对应的过渡矩阵为酉方阵$Q$,则该过程表示为$QM^{\dagger}MQ^{\dagger}=A$。左右取行列式可证$A$是可逆矩阵。在这组基下,对对角元计算后,我们可以得到$A^{\frac{1}{2}},A^{-\frac{1}{2}}$,显然互逆。

我们可以把$M$表示为$MQ^{\dagger}A^{-\frac{1}{2}}A^{\frac{1}{2}}Q$,和\autoref{eq_SVD_1} 比较,我们需要验证$MQ^{\dagger}A^{-\frac{1}{2}}$是酉方阵。

设$P^{\dagger}=MQ^{\dagger}A^{-\frac{1}{2}}$,则$P=A^{-\frac{1}{2}}QM^{\dagger}$,则
\begin{equation}
\begin{aligned}
P^{\dagger}P&=MQ^{\dagger}A^{-\frac{1}{2}}A^{-\frac{1}{2}}QM^{\dagger}\\
&=MQ^{\dagger}(Q^{\dagger})^{-1}(M)^{-1}(M^{\dagger})^{-1}Q^{-1}QM^{\dagger}\\
&=E~,
\end{aligned}
~.
\end{equation}
定理得证。

在\autoref{the_SVD_1} 的陈述里$D$也是可逆矩阵,这并非偶然。考虑到$P^{-1}=P^{\dagger}$,该定理实际上是$M$的广义相似变换。如下图所示,线性映射$f:V\rightarrow W$在不同基下有不同表示。设$\{\bvec e_i\},\{\bvec \theta_i\}$是$V,W$的旧基,且$f$在该基下的表示为$M$。$P,Q$是对应$W,V$的从旧基到新基的过渡矩阵,则$D$就自然是$f$在新基下的表示了。作为$f$的不同表示,$M$与$N$的行列数相同,秩也相同。

如此一来,我们似乎可以拓展该定理至$M$不可逆的情况。

\begin{lemma}{}\label{lem_SVD_1}
对于任意矩阵$M$,存在酉矩阵$Q,S$和\textbf{满秩}矩阵$N$,使得
\begin{equation}\label{eq_SVD_2}
M=S^{\dagger}\left(\begin{array}{cc}
N & 0 \\
0 & 0
\end{array}\right) Q~.
\end{equation}
\end{lemma}

\addTODO{引用定理“欧几里得空间总存在标准正交基”,“非退化情况下,子空间的标准正交基可拓展到全空间”}

换一个方式表述,\autoref{lem_SVD_1} 实际上是在说对于酉空间的任意线性映射$f$,原像和像空间存在一组标准正交基,使得$f$在该基下是\autoref{eq_SVD_2} 所示的块对角形式。

考虑到$\opn{dim}V=\opn{dim}\opn{ker}f+\opn{dim}\opn{Im}f$,我们可以在$\opn{ker}f$内选一组标准正交基,然后扩展到全空间。设$k=\opn{dim}\opn{ker}f,s=\opn{dim}\opn{Im}f$。则在该组基下,$f$的后$k$列必为$0$。

同理,$\opn{Im}f$作为$W$的子空间,也可以找到一组标准正交基,并拓展到其正交补。因而除去前$s$行,其他行都为$0$。如此我们就证明了$f$具有\autoref{eq_SVD_2} 所示的块对角形式,其中$N$必为满秩的$s\times s$\textbf{方阵},即其可逆。如若不然,$\opn{Im}f\cap\opn{ker}f\neq\{\bvec 0\}$。

现在我们来证明奇异值分解定理。
对于\textbf{任意复矩阵}$M$,我们可以分解为\autoref{lem_SVD_1} 的形式。那么对于\textbf{满秩}$N$矩阵,自然而然的,我们可以利用\autoref{the_SVD_1} 进行进一步的分解。设该分解表示为$N=P^{\}D$











