% Liapunov 函数(稳定性直接法)
% keys Liapunov 函数|V函数
% license Usr
% type Tutor

\pentry{Liapunov 稳定性(常微分方程)\nref{nod_ODELia}}{nod_fcab}
Liapunov 对于非线性的问题又提出了“V 函数法”,又称 Liapunov 函数法、Liapunov 直接法。这个方法借助一个辅助函数直接从微分方程的动力性质研究,所以若知道这辅助函数之后研究稳定性将比较简单,只不过对于给定系统构造一个这样的函数是比较难的。



\subsection{非线性定常系统}
考虑一个非线性定常系统 
\begin{equation}
\dv{t} \bvec x = \bvec f(\bvec x) ~.
\end{equation}
约定 $\bvec f(x)$ 在包含坐标原点的某区域 $G \subseteq \mathbb R^n$ 内有连续的一阶偏导数,且 $\bvec f(0) = 0$。

\begin{definition}{}
对于一个函数 $ V(\bvec x)$ 在原点 $O$ 的某一邻域内有连续的一阶偏导数,同时 $ V(\bvec 0 ) = 0$,定义:
\begin{enumerate}
\item 若存在 $h>0$,当 $\Vert \bvec x \Vert < h$ 时,$V(\bvec x) \ge 0$($\le 0$),则称 $V$ 是常正(常负)函数,统称常号函数;
\item 若存在 $h > 0$,当 $0 < \Vert \bvec x \Vert < h$ 时,$V(\bvec x)$
\end{enumerate}

\end{definition}