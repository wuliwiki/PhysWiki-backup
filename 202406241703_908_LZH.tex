% 量子化
% license CCBYSA3
% type Wiki

(本文根据 CC-BY-SA 协议转载自原搜狗科学百科对英文维基百科的翻译)

在物理学中,\textbf{量子化}是从对物理现象的经典理解过渡到被称为量子力学的新理解的过程,这是一个从经典场论开始构建量子场论的过程,从经典力学建立量子力学的过程的概括。同样相关的是\textbf{场量子化},如在“电磁场的量子化”中,将光子称为场“量子”(例如光量子)。这个过程是粒子物理、核物理、凝聚态物理和量子光学理论的基础。

\subsection{量化方法}
量子化将经典场转换成作用于场论量子态的算符。最低能量状态称为真空状态。发展量子化理论是为了通过计算量子振幅来推断材料、物体或粒子的各种复杂性质。在做这种计算时必须要解决好“重整化”的问题,如果忽视这些细微的差异,往往会导致无意义的结果,例如出现各种振幅的不定式。量化过程的完整规范需要执行重正化的方法。

场理论量子化的第一种方法是规范量子化。虽然这在足够简单的理论上非常容易实现,但在许多情况下,其他量化方法会产生更有效的计算量子振幅的过程。然而,规范量子化的使用在量子场论的语言和解释上留下了印记。

\subsubsection{1.1 规范量子化}
场论的规范量子化类似于从经典力学中构造量子力学。经典场被视为一个叫做规范坐标的动力学变量,它的时间导数就是规范动量。人们引入了它们之间的换向关系,这与量子力学中粒子位置和动量之间的换向关系完全相同。从技术上讲,人们通过创造和湮灭算子的组合,将场转化为算子。场算符作用于理论的量子态。最低能量状态称为真空状态。该过程也称为二次量化。

这个过程可以应用于任何场论的量子化:无论是费米子还是玻色子,以及任何内部对称性。然而,它导致真空状态的相当简单的图像,并且不容易适用于某些量子场论,例如量子色动力学,已知其具有以许多不同冷凝物为特征的复杂真空。

\subsubsection{1.2 量子化方案}
即使在规范量化的设置中,也难以量化经典相空间上的任意可观测值。这就是排序的模糊性:经典的位置和动量变量x和p交换,但是它们的量子力学对应物没有。已经提出了各种量化方案来解决这种模糊性,[1] 其中最流行的是Weyl量化方案。然而,格罗内沃尔德-范霍夫定理表明不存在完美的量化方案。特别地,如果x 和 p的量化被认为是通常的位置和动量算符,那么没有任何量化方案能够完美地再现经典可观测值之间的泊松括号关系。[2]

\subsubsection{1.3 协变规范量子化}
有一种方法可以实现规范量子化,而不必求助于叶化时空和选择哈密顿量的非协变方法。这种方法基于一个经典的动作,但不同于函数积分法。

该方法不适用于所有可能的操作(例如,具有非因果结构的操作或具有计量“流量”的操作)。它从配置空间上所有(光滑)活跃的经典代数开始。这个代数是由欧拉-拉格朗日方程产生的理想商数。然后,这个商代数通过引入一个可以从这个动作中导出的泊松括号(称为佩尔斯括号)被转换成泊松代数。然后,该泊松代数以与规范化相同的方式进行 $\bar{h}$ 变形。


还有一种方法可以通过测量“流量”来量化动作。它涉及巴塔林-维尔科维奇形式主义,是BRST形式主义的延伸。

\subsubsection{1.4 变形量子化}

\subsubsection{1.5 几何量子化}

在数学物理中,几何量子化是定义对应于给定经典理论的量子理论的数学方法。它试图进行量子化,一般来说没有确切的方法,以这种方式经典理论和量子理论之间的某些类比仍然显而易见。例如,量子力学海森堡图像中的海森堡方程和经典物理中的哈密顿方程之间的相似性应该被建立起来。

自然量子化的最早尝试之一是1927年赫尔曼·韦勒提出的韦勒量化。这里,我们试图将量子力学观测值(希尔伯特空间上的自伴算子)与经典相空间上的实值函数联系起来。这个相空间中的位置和动量被映射到海森堡群的发生器上,希尔伯特空间表现为海森堡群的群表示。1946年,格罗内沃尔德[3] 考虑了一对这样的可观测值的乘积,并问在经典相空间上相应的函数是什么。这使他发现了一对函数的相空间星积。更一般地,这种技术导致变形量字化,其中★-积被认为是辛流形或泊松流形上函数代数的变形。然而,作为一种自然量化方案(函子),韦勒映射并不令人满意。例如,经典角动量平方的韦勒映射不仅仅是量子角动量平方算符,它还包含一个常数项3ħ2/2.(这个额外的项实际上是有重大物理意义的,因为它解释了氢原子中基态玻尔轨道的非负角动量。)[4] 虽然它仅仅是一种表象变换,但韦勒映射实际上是传统量子力学交替相空间公式的基础。

伯特伦·科斯塔特和让-玛丽·苏里奥在20世纪70年代提出了一种更为几何化的量子化方法,其中经典相空间可以是一个一般的辛流形。该方法分两个阶段进行。首先,一旦在相空间上构造了一个由平方可积函数(或者更确切地说,线束的部分)组成的“前量子希尔伯特空间”。这里,我们可以构造满足恰好对应于经典泊松括号关系的换向关系的算子。另一方面,这个前量子希尔伯特空间太大,没有物理意义。然后,根据相空间上一半的变量,限制函数(或部分),产生量子希尔伯特空间。

\subsubsection{1.6 循环量子化}

\subsubsection{1.7 路径积分量子化}

一个经典的力学理论是由一个动作给出的,允许的构形是相对于动作的功能变化而言是极值的构形。经典系统的量子力学描述也可以通过路径积分公式由系统的作用来构造。

\subsubsection{1.8 量子统计力学方法}

\subsubsection{1.9 施温格变分方法}

\subsection{参考文献}
[1]
^Hall 2013 Chapter 13.

[2]
^Hall 2013 Theorem 13.13.

[3]
^Groenewold, H.J. (1946). "On the principles of elementary quantum mechanics". Physica. 12 (7): 405–460. doi:10.1016/S0031-8914(46)80059-4. ISSN 0031-8914..

[4]
^Dahl, Jens Peder; Schleich, Wolfgang P. (2002). "Concepts of radial and angular kinetic energies". Physical Review A. 65 (2). arXiv:quant-ph/0110134. doi:10.1103/PhysRevA.65.022109. ISSN 1050-2947..